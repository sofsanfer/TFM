%
\begin{isabellebody}%
\setisabellecontext{Resumen}%
%
\isadelimtheory
%
\endisadelimtheory
%
\isatagtheory
%
\endisatagtheory
{\isafoldtheory}%
%
\isadelimtheory
%
\endisadelimtheory
%
\begin{isamarkuptext}%
El objetivo de la Lógica es la formalización del conocimiento y su
razonamiento. Este trabajo constituye una continuación del Trabajo de Fin de
Grado \isa{Elementos\ de\ lógica\ formalizados\ en\ Isabelle{\isacharslash}HOL} \isa{{\isacharbrackleft}{\isadigit{4}}{\isacharbrackright}}, donde se estudiaron
la sintaxis, semántica y \isa{Lema\ de\ Hintikka} de la Lógica Proposicional
desde la perspectiva teórica de \isa{First−Order\ Logic\ and\ Automated\ Theorem\ Proving} 
\isa{{\isacharbrackleft}{\isadigit{5}}{\isacharbrackright}} de Melvin Fitting. Manteniendo dicha perspectiva, nos centraremos en la 
demostración del \isa{Teorema\ de\ Existencia\ de\ Modelo}, concluyendo con 
el \isa{Teorema\ de\ Compacidad} como consecuencia del mismo. Siguiendo la 
inspiración de \isa{Propositional\ Proof\ Systems} \isa{{\isacharbrackleft}{\isadigit{1}}{\isadigit{0}}{\isacharbrackright}} por Julius Michaelis y Tobias 
Nipkow, los resultados expuestos serán formalizados mediante Isabelle: un demostrador 
interactivo que incluye herramientas de razonamiento automático para guiar al usuario 
en el proceso de formalización, verificación y automatización de resultados. 
Concretamente, Isabelle/HOL es una especialización de Isabelle para la lógica de orden 
superior. Las demostraciones de los resultados en Isabelle/HOL se elaborarán siguiendo 
dos tácticas distintas a lo largo del trabajo. En primer lugar, cada lema será probado 
de manera detallada prescindiendo de toda herramienta de razonamiento automático, como 
resultado de una búsqueda inversa en cada paso de la prueba. En contraposición, 
elaboraremos una demostración automática alternativa de cada resultado que utilice todas 
las herramientas de razonamiento automático que proporciona el demostrador. De este modo, 
se evidenciará la capacidad de razonamiento automático de Isabelle.%
\end{isamarkuptext}\isamarkuptrue%
%
\begin{isamarkuptext}%
Logic’s purpose is about knowledge’s formalisation and its 
reasoning. This project is a continuation of \isa{Elementos\ de\ lógica\ formalizados\ en\ Isabelle{\isacharslash}HOL} \isa{{\isacharbrackleft}{\isadigit{4}}{\isacharbrackright}} in which we studied Syntax, Semantics and
a propositional version of Hintikka's lemma from the theoretical perspective 
of \isa{First−Order\ Logic\ and\ Automated\ Theorem\ Proving} \isa{{\isacharbrackleft}{\isadigit{5}}{\isacharbrackright}} by Melvin Fitting. 
Following the same perspective, we will focus on the demonstration of 
\isa{Propositional\ Model\ Existence} theorem, concluding with the \isa{Propositional\ Compactness} theorem as a consecuence. Inspired by \isa{Propositional\ Proof\ Systems} \isa{{\isacharbrackleft}{\isadigit{1}}{\isadigit{0}}{\isacharbrackright}} by 
Julius Michaelis and Tobias Nipkow, these results will be formalised using 
Isabelle: a proof assistant including automatic reasoning tools to guide the 
user on formalising, verifying and automating results. In particular, 
Isabelle/HOL is the specialization of Isabelle for High-Order Logic. The 
processing of the results formalised in Isabelle/HOL follows two directions. 
In the first place, each lemma will be proved on detail without any automation, 
as the result of an inverse research on every step of the demonstration until 
it is only completed with deductions based on elementary rules and definitions. 
Conversely, we will alternatively prove the results using all the 
automatic reasoning tools that are provide by the proof assistant. In 
this way, Isabelle’s power of automatic reasoning will be shown as the 
contrast between these two different proving tactics.%
\end{isamarkuptext}\isamarkuptrue%
%
\isadelimtheory
%
\endisadelimtheory
%
\isatagtheory
%
\endisatagtheory
{\isafoldtheory}%
%
\isadelimtheory
%
\endisadelimtheory
%
\end{isabellebody}%
\endinput
%:%file=~/TFM/TFM/Resumen.thy%:%
%:%19=8%:%
%:%20=9%:%
%:%21=10%:%
%:%22=11%:%
%:%23=12%:%
%:%24=13%:%
%:%25=14%:%
%:%26=15%:%
%:%27=16%:%
%:%28=17%:%
%:%29=18%:%
%:%30=19%:%
%:%31=20%:%
%:%32=21%:%
%:%33=22%:%
%:%34=23%:%
%:%35=24%:%
%:%36=25%:%
%:%37=26%:%
%:%38=27%:%
%:%42=31%:%
%:%43=32%:%
%:%43=33%:%
%:%44=34%:%
%:%45=35%:%
%:%46=36%:%
%:%47=37%:%
%:%47=38%:%
%:%48=39%:%
%:%49=40%:%
%:%50=41%:%
%:%51=42%:%
%:%52=43%:%
%:%53=44%:%
%:%54=45%:%
%:%55=46%:%
%:%56=47%:%
%:%57=48%:%
%:%58=49%:%
%:%59=50%:%