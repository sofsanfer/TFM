%
\begin{isabellebody}%
\setisabellecontext{Notunif}%
%
\isadelimtheory
%
\endisadelimtheory
%
\isatagtheory
%
\endisatagtheory
{\isafoldtheory}%
%
\isadelimtheory
%
\endisadelimtheory
%
\begin{isamarkuptext}%
\comentario{Localización de sello.png.}%
\end{isamarkuptext}\isamarkuptrue%
%
\begin{isamarkuptext}%
En este capítulo introduciremos la notación uniforme inicialmente 
  desarrollada por \isa{R{\isachardot}\ M{\isachardot}\ Smullyan} (añadir referencia bibliográfica). La finalidad
  de dicha notación es reducir el número de casos a considerar sobre la estructura de 
  las fórmulas al clasificar éstas en dos categorías, facilitando las demostraciones
  y métodos empleados en adelante.

  \comentario{Añadir referencia bibliográfica.}

  De este modo, las fórmulas proposicionales pueden ser de dos tipos:
  las de tipo conjuntivo (las fórmulas \isa{{\isasymalpha}}) y las de tipo disyuntivo (las fórmulas \isa{{\isasymbeta}}). 
  Cada fórmula de tipo \isa{{\isasymalpha}}, o \isa{{\isasymbeta}} respectivamente, tiene asociada sus  
  dos componentes \isa{{\isasymalpha}\isactrlsub {\isadigit{1}}} y \isa{{\isasymalpha}\isactrlsub {\isadigit{2}}}, o \isa{{\isasymbeta}\isactrlsub {\isadigit{1}}} y \isa{{\isasymbeta}\isactrlsub {\isadigit{2}}} respectivamente. Para justificar dicha clasificación,
  introduzcamos inicialmente la definición de fórmulas semánticamente equivalentes.

  \begin{definicion}
    Dos fórmulas son \isa{semánticamente\ equivalentes} si tienen el mismo valor para toda 
    interpretación.
  \end{definicion}

  En Isabelle podemos formalizar la definición de la siguiente manera.%
\end{isamarkuptext}\isamarkuptrue%
\isacommand{definition}\isamarkupfalse%
\ {\isachardoublequoteopen}semanticEq\ F\ G\ {\isasymequiv}\ {\isasymforall}{\isasymA}{\isachardot}\ {\isacharparenleft}{\isasymA}\ {\isasymTurnstile}\ F{\isacharparenright}\ {\isasymlongleftrightarrow}\ {\isacharparenleft}{\isasymA}\ {\isasymTurnstile}\ G{\isacharparenright}{\isachardoublequoteclose}%
\begin{isamarkuptext}%
Según la definición del valor de verdad de una fórmula proposicional en una 
  interpretación dada, podemos ver los siguientes ejemplos de fórmulas semánticamente equivalentes.%
\end{isamarkuptext}\isamarkuptrue%
\isacommand{lemma}\isamarkupfalse%
\ {\isachardoublequoteopen}semanticEq\ {\isacharparenleft}Atom\ p{\isacharparenright}\ {\isacharparenleft}{\isacharparenleft}Atom\ p{\isacharparenright}\ \isactrlbold {\isasymor}\ {\isacharparenleft}Atom\ p{\isacharparenright}{\isacharparenright}{\isachardoublequoteclose}\ \isanewline
%
\isadelimproof
\ \ %
\endisadelimproof
%
\isatagproof
\isacommand{by}\isamarkupfalse%
\ {\isacharparenleft}simp\ add{\isacharcolon}\ semanticEq{\isacharunderscore}def{\isacharparenright}%
\endisatagproof
{\isafoldproof}%
%
\isadelimproof
\isanewline
%
\endisadelimproof
\isanewline
\isacommand{lemma}\isamarkupfalse%
\ {\isachardoublequoteopen}semanticEq\ {\isacharparenleft}Atom\ p{\isacharparenright}\ {\isacharparenleft}{\isacharparenleft}Atom\ p{\isacharparenright}\ \isactrlbold {\isasymand}\ {\isacharparenleft}Atom\ p{\isacharparenright}{\isacharparenright}{\isachardoublequoteclose}\ \isanewline
%
\isadelimproof
\ \ %
\endisadelimproof
%
\isatagproof
\isacommand{by}\isamarkupfalse%
\ {\isacharparenleft}simp\ add{\isacharcolon}\ semanticEq{\isacharunderscore}def{\isacharparenright}%
\endisatagproof
{\isafoldproof}%
%
\isadelimproof
\isanewline
%
\endisadelimproof
\isanewline
\isacommand{lemma}\isamarkupfalse%
\ {\isachardoublequoteopen}semanticEq\ {\isasymbottom}\ {\isacharparenleft}{\isasymbottom}\ \isactrlbold {\isasymand}\ {\isasymbottom}{\isacharparenright}{\isachardoublequoteclose}\ \isanewline
%
\isadelimproof
\ \ %
\endisadelimproof
%
\isatagproof
\isacommand{by}\isamarkupfalse%
\ {\isacharparenleft}simp\ add{\isacharcolon}\ semanticEq{\isacharunderscore}def{\isacharparenright}%
\endisatagproof
{\isafoldproof}%
%
\isadelimproof
\isanewline
%
\endisadelimproof
\isanewline
\isacommand{lemma}\isamarkupfalse%
\ {\isachardoublequoteopen}semanticEq\ {\isasymbottom}\ {\isacharparenleft}{\isasymbottom}\ \isactrlbold {\isasymor}\ {\isasymbottom}{\isacharparenright}{\isachardoublequoteclose}\ \isanewline
%
\isadelimproof
\ \ %
\endisadelimproof
%
\isatagproof
\isacommand{by}\isamarkupfalse%
\ {\isacharparenleft}simp\ add{\isacharcolon}\ semanticEq{\isacharunderscore}def{\isacharparenright}%
\endisatagproof
{\isafoldproof}%
%
\isadelimproof
\isanewline
%
\endisadelimproof
\isanewline
\isacommand{lemma}\isamarkupfalse%
\ {\isachardoublequoteopen}semanticEq\ {\isasymbottom}\ {\isacharparenleft}\isactrlbold {\isasymnot}\ {\isasymtop}{\isacharparenright}{\isachardoublequoteclose}\isanewline
%
\isadelimproof
\ \ %
\endisadelimproof
%
\isatagproof
\isacommand{by}\isamarkupfalse%
\ {\isacharparenleft}simp\ add{\isacharcolon}\ semanticEq{\isacharunderscore}def\ top{\isacharunderscore}semantics{\isacharparenright}%
\endisatagproof
{\isafoldproof}%
%
\isadelimproof
\isanewline
%
\endisadelimproof
\isanewline
\isacommand{lemma}\isamarkupfalse%
\ {\isachardoublequoteopen}semanticEq\ F\ {\isacharparenleft}\isactrlbold {\isasymnot}{\isacharparenleft}\isactrlbold {\isasymnot}\ F{\isacharparenright}{\isacharparenright}{\isachardoublequoteclose}\isanewline
%
\isadelimproof
\ \ %
\endisadelimproof
%
\isatagproof
\isacommand{by}\isamarkupfalse%
\ {\isacharparenleft}simp\ add{\isacharcolon}\ semanticEq{\isacharunderscore}def{\isacharparenright}%
\endisatagproof
{\isafoldproof}%
%
\isadelimproof
\isanewline
%
\endisadelimproof
\isanewline
\isacommand{lemma}\isamarkupfalse%
\ {\isachardoublequoteopen}semanticEq\ {\isacharparenleft}\isactrlbold {\isasymnot}{\isacharparenleft}\isactrlbold {\isasymnot}\ F{\isacharparenright}{\isacharparenright}\ {\isacharparenleft}F\ \isactrlbold {\isasymor}\ F{\isacharparenright}{\isachardoublequoteclose}\isanewline
%
\isadelimproof
\ \ %
\endisadelimproof
%
\isatagproof
\isacommand{by}\isamarkupfalse%
\ {\isacharparenleft}simp\ add{\isacharcolon}\ semanticEq{\isacharunderscore}def{\isacharparenright}%
\endisatagproof
{\isafoldproof}%
%
\isadelimproof
\isanewline
%
\endisadelimproof
\isanewline
\isacommand{lemma}\isamarkupfalse%
\ {\isachardoublequoteopen}semanticEq\ {\isacharparenleft}\isactrlbold {\isasymnot}{\isacharparenleft}\isactrlbold {\isasymnot}\ F{\isacharparenright}{\isacharparenright}\ {\isacharparenleft}F\ \isactrlbold {\isasymand}\ F{\isacharparenright}{\isachardoublequoteclose}\isanewline
%
\isadelimproof
\ \ %
\endisadelimproof
%
\isatagproof
\isacommand{by}\isamarkupfalse%
\ {\isacharparenleft}simp\ add{\isacharcolon}\ semanticEq{\isacharunderscore}def{\isacharparenright}%
\endisatagproof
{\isafoldproof}%
%
\isadelimproof
\isanewline
%
\endisadelimproof
\isanewline
\isacommand{lemma}\isamarkupfalse%
\ {\isachardoublequoteopen}semanticEq\ {\isacharparenleft}\isactrlbold {\isasymnot}\ F\ \isactrlbold {\isasymand}\ \isactrlbold {\isasymnot}\ G{\isacharparenright}\ {\isacharparenleft}\isactrlbold {\isasymnot}{\isacharparenleft}F\ \isactrlbold {\isasymor}\ G{\isacharparenright}{\isacharparenright}{\isachardoublequoteclose}\isanewline
%
\isadelimproof
\ \ %
\endisadelimproof
%
\isatagproof
\isacommand{by}\isamarkupfalse%
\ {\isacharparenleft}simp\ add{\isacharcolon}\ semanticEq{\isacharunderscore}def{\isacharparenright}%
\endisatagproof
{\isafoldproof}%
%
\isadelimproof
\isanewline
%
\endisadelimproof
\isanewline
\isacommand{lemma}\isamarkupfalse%
\ {\isachardoublequoteopen}semanticEq\ {\isacharparenleft}F\ \isactrlbold {\isasymrightarrow}\ G{\isacharparenright}\ {\isacharparenleft}\isactrlbold {\isasymnot}\ F\ \isactrlbold {\isasymor}\ G{\isacharparenright}{\isachardoublequoteclose}\isanewline
%
\isadelimproof
\ \ %
\endisadelimproof
%
\isatagproof
\isacommand{by}\isamarkupfalse%
\ {\isacharparenleft}simp\ add{\isacharcolon}\ semanticEq{\isacharunderscore}def{\isacharparenright}%
\endisatagproof
{\isafoldproof}%
%
\isadelimproof
%
\endisadelimproof
%
\begin{isamarkuptext}%
En contraposición, también podemos dar ejemplos de fórmulas que no son semánticamente 
  equivalentes.%
\end{isamarkuptext}\isamarkuptrue%
\isacommand{lemma}\isamarkupfalse%
\ {\isachardoublequoteopen}{\isasymnot}\ semanticEq\ {\isacharparenleft}Atom\ p{\isacharparenright}\ {\isacharparenleft}\isactrlbold {\isasymnot}{\isacharparenleft}Atom\ p{\isacharparenright}{\isacharparenright}{\isachardoublequoteclose}\isanewline
%
\isadelimproof
\ \ %
\endisadelimproof
%
\isatagproof
\isacommand{by}\isamarkupfalse%
\ {\isacharparenleft}simp\ add{\isacharcolon}\ semanticEq{\isacharunderscore}def{\isacharparenright}%
\endisatagproof
{\isafoldproof}%
%
\isadelimproof
\isanewline
%
\endisadelimproof
\isanewline
\isacommand{lemma}\isamarkupfalse%
\ {\isachardoublequoteopen}{\isasymnot}\ semanticEq\ {\isasymbottom}\ {\isasymtop}{\isachardoublequoteclose}\isanewline
%
\isadelimproof
\ \ %
\endisadelimproof
%
\isatagproof
\isacommand{by}\isamarkupfalse%
\ {\isacharparenleft}simp\ add{\isacharcolon}\ semanticEq{\isacharunderscore}def\ top{\isacharunderscore}semantics{\isacharparenright}%
\endisatagproof
{\isafoldproof}%
%
\isadelimproof
%
\endisadelimproof
%
\begin{isamarkuptext}%
Por tanto, diremos intuitivamente que una fórmula es de tipo \isa{{\isasymalpha}} con componentes \isa{{\isasymalpha}\isactrlsub {\isadigit{1}}} y \isa{{\isasymalpha}\isactrlsub {\isadigit{2}}}
  si es semánticamente equivalente a la fórmula \isa{{\isasymalpha}\isactrlsub {\isadigit{1}}\ {\isasymand}\ {\isasymalpha}\isactrlsub {\isadigit{2}}}. Del mismo modo, una fórmula será de tipo
  \isa{{\isasymbeta}} con componentes \isa{{\isasymbeta}\isactrlsub {\isadigit{1}}} y \isa{{\isasymbeta}\isactrlsub {\isadigit{2}}} si es semánticamente equivalente a la fórmula \isa{{\isasymbeta}\isactrlsub {\isadigit{1}}\ {\isasymor}\ {\isasymbeta}\isactrlsub {\isadigit{2}}}.

  \begin{definicion}
    Las fórmulas de tipo \isa{{\isasymalpha}} (\isa{fórmulas\ conjuntivas}) y sus correspondientes componentes
    \isa{{\isasymalpha}\isactrlsub {\isadigit{1}}} y \isa{{\isasymalpha}\isactrlsub {\isadigit{2}}} se definen como sigue: dadas \isa{F} y \isa{G} fórmulas cualesquiera,
    \begin{enumerate}
      \item \isa{F\ {\isasymand}\ G} es una fórmula de tipo \isa{{\isasymalpha}} cuyas componentes son \isa{F} y \isa{G}.
      \item \isa{{\isasymnot}{\isacharparenleft}F\ {\isasymor}\ G{\isacharparenright}} es una fórmula de tipo \isa{{\isasymalpha}} cuyas componentes son \isa{{\isasymnot}\ F} y \isa{{\isasymnot}\ G}.
      \item \isa{{\isasymnot}{\isacharparenleft}F\ {\isasymlongrightarrow}\ G{\isacharparenright}} es una fórmula de tipo \isa{{\isasymalpha}} cuyas componentes son \isa{F} y \isa{{\isasymnot}\ G}.
    \end{enumerate} 
  \end{definicion}

  De este modo, de los ejemplos anteriores podemos deducir que las fórmulas atómicas son de tipo \isa{{\isasymalpha}}
  y sus componentes \isa{{\isasymalpha}\isactrlsub {\isadigit{1}}} y \isa{{\isasymalpha}\isactrlsub {\isadigit{2}}} son la propia fórmula. Del mismo modo, la constante \isa{{\isasymbottom}} también es 
  una fórmula conjuntiva cuyas componentes son ella misma. Por último, podemos observar que dada
  una fórmula cualquiera \isa{F}, su doble negación \isa{{\isasymnot}{\isacharparenleft}{\isasymnot}\ F{\isacharparenright}} es una fórmula de tipo \isa{{\isasymalpha}} y componentes
  \isa{F} y \isa{F}.

  Formalizaremos en Isabelle el conjunto de fórmulas \isa{{\isasymalpha}} como un predicato inductivo. De este modo,
  las reglas anteriores que construyen el conjunto de fórmulas de tipo \isa{{\isasymalpha}} se formalizan en Isabelle 
  como reglas de introducción. Además, añadiremos explícitamente una cuarta regla que introduce la 
  doble negación de una fórmula como fórmula de tipo \isa{{\isasymalpha}}. De este modo, facilitaremos la prueba de 
  resultados posteriores relacionados con la definición de conjunto de Hintikka, que constituyen una
  base para la demostración del \isa{teorema\ de\ existencia\ de\ modelo}.%
\end{isamarkuptext}\isamarkuptrue%
\isacommand{inductive}\isamarkupfalse%
\ Con\ {\isacharcolon}{\isacharcolon}\ {\isachardoublequoteopen}{\isacharprime}a\ formula\ {\isacharequal}{\isachargreater}\ {\isacharprime}a\ formula\ {\isacharequal}{\isachargreater}\ {\isacharprime}a\ formula\ {\isacharequal}{\isachargreater}\ bool{\isachardoublequoteclose}\ \isakeyword{where}\isanewline
{\isachardoublequoteopen}Con\ {\isacharparenleft}And\ F\ G{\isacharparenright}\ F\ G{\isachardoublequoteclose}\ {\isacharbar}\isanewline
{\isachardoublequoteopen}Con\ {\isacharparenleft}Not\ {\isacharparenleft}Or\ F\ G{\isacharparenright}{\isacharparenright}\ {\isacharparenleft}Not\ F{\isacharparenright}\ {\isacharparenleft}Not\ G{\isacharparenright}{\isachardoublequoteclose}\ {\isacharbar}\isanewline
{\isachardoublequoteopen}Con\ {\isacharparenleft}Not\ {\isacharparenleft}Imp\ F\ G{\isacharparenright}{\isacharparenright}\ F\ {\isacharparenleft}Not\ G{\isacharparenright}{\isachardoublequoteclose}\ {\isacharbar}\isanewline
{\isachardoublequoteopen}Con\ {\isacharparenleft}Not\ {\isacharparenleft}Not\ F{\isacharparenright}{\isacharparenright}\ F\ F{\isachardoublequoteclose}%
\begin{isamarkuptext}%
Las reglas de introducción que proporciona la definición anterior son
  las siguientes.

  \begin{itemize}
    \item[] \isa{Con\ {\isacharparenleft}F\ \isactrlbold {\isasymand}\ G{\isacharparenright}\ F\ G\isasep\isanewline%
Con\ {\isacharparenleft}\isactrlbold {\isasymnot}\ {\isacharparenleft}F\ \isactrlbold {\isasymor}\ G{\isacharparenright}{\isacharparenright}\ {\isacharparenleft}\isactrlbold {\isasymnot}\ F{\isacharparenright}\ {\isacharparenleft}\isactrlbold {\isasymnot}\ G{\isacharparenright}\isasep\isanewline%
Con\ {\isacharparenleft}\isactrlbold {\isasymnot}\ {\isacharparenleft}F\ \isactrlbold {\isasymrightarrow}\ G{\isacharparenright}{\isacharparenright}\ F\ {\isacharparenleft}\isactrlbold {\isasymnot}\ G{\isacharparenright}\isasep\isanewline%
Con\ {\isacharparenleft}\isactrlbold {\isasymnot}\ {\isacharparenleft}\isactrlbold {\isasymnot}\ F{\isacharparenright}{\isacharparenright}\ F\ F} 
      \hfill (\isa{Con{\isachardot}intros})
  \end{itemize}
  
  Por otro lado, definamos las fórmulas disyuntivas.

  \begin{definicion}
    Las fórmulas de tipo \isa{{\isasymbeta}} (\isa{fórmulas\ disyuntivas}) y sus correspondientes componentes
    \isa{{\isasymbeta}\isactrlsub {\isadigit{1}}} y \isa{{\isasymbeta}\isactrlsub {\isadigit{2}}} se definen como sigue: dadas \isa{F} y \isa{G} fórmulas cualesquiera,
    \begin{enumerate}
      \item \isa{F\ {\isasymor}\ G} es una fórmula de tipo \isa{{\isasymbeta}} cuyas componentes son \isa{F} y \isa{G}.
      \item \isa{F\ {\isasymlongrightarrow}\ G} es una fórmula de tipo \isa{{\isasymbeta}} cuyas componentes son \isa{{\isasymnot}\ F} y \isa{G}.
      \item \isa{{\isasymnot}{\isacharparenleft}F\ {\isasymand}\ G{\isacharparenright}} es una fórmula de tipo \isa{{\isasymbeta}} cuyas componentes son \isa{{\isasymnot}\ F} y \isa{{\isasymnot}\ G}.
    \end{enumerate} 
  \end{definicion}

  De los ejemplos dados anteriormente, podemos deducir análogamente que las fórmulas atómicas, la
  constante \isa{{\isasymbottom}} y la doble negación sob también fórmulas disyuntivas con las mismas componentes que
  las dadas para el tipo conjuntivo.

  Del mismo modo, su formalización se realiza como un predicado inductivo, de manera que las reglas 
  que definen el conjunto de fórmulas de tipo \isa{{\isasymbeta}} se formalizan en Isabelle como reglas de 
  introducción. Análogamente, introduciremos de manera explícita una regla que señala que la doble 
  negación de una fórmula es una fórmula de tipo disyuntivo.%
\end{isamarkuptext}\isamarkuptrue%
\isacommand{inductive}\isamarkupfalse%
\ Dis\ {\isacharcolon}{\isacharcolon}\ {\isachardoublequoteopen}{\isacharprime}a\ formula\ {\isacharequal}{\isachargreater}\ {\isacharprime}a\ formula\ {\isacharequal}{\isachargreater}\ {\isacharprime}a\ formula\ {\isacharequal}{\isachargreater}\ bool{\isachardoublequoteclose}\ \isakeyword{where}\isanewline
{\isachardoublequoteopen}Dis\ {\isacharparenleft}Or\ F\ G{\isacharparenright}\ F\ G{\isachardoublequoteclose}\ {\isacharbar}\isanewline
{\isachardoublequoteopen}Dis\ {\isacharparenleft}Imp\ F\ G{\isacharparenright}\ {\isacharparenleft}Not\ F{\isacharparenright}\ G{\isachardoublequoteclose}\ {\isacharbar}\isanewline
{\isachardoublequoteopen}Dis\ {\isacharparenleft}Not\ {\isacharparenleft}And\ F\ G{\isacharparenright}{\isacharparenright}\ {\isacharparenleft}Not\ F{\isacharparenright}\ {\isacharparenleft}Not\ G{\isacharparenright}{\isachardoublequoteclose}\ {\isacharbar}\isanewline
{\isachardoublequoteopen}Dis\ {\isacharparenleft}Not\ {\isacharparenleft}Not\ F{\isacharparenright}{\isacharparenright}\ F\ F{\isachardoublequoteclose}%
\begin{isamarkuptext}%
Las reglas de introducción que proporciona esta formalización se muestran a continuación.

  \begin{itemize}
    \item[] \isa{Dis\ {\isacharparenleft}F\ \isactrlbold {\isasymor}\ G{\isacharparenright}\ F\ G\isasep\isanewline%
Dis\ {\isacharparenleft}F\ \isactrlbold {\isasymrightarrow}\ G{\isacharparenright}\ {\isacharparenleft}\isactrlbold {\isasymnot}\ F{\isacharparenright}\ G\isasep\isanewline%
Dis\ {\isacharparenleft}\isactrlbold {\isasymnot}\ {\isacharparenleft}F\ \isactrlbold {\isasymand}\ G{\isacharparenright}{\isacharparenright}\ {\isacharparenleft}\isactrlbold {\isasymnot}\ F{\isacharparenright}\ {\isacharparenleft}\isactrlbold {\isasymnot}\ G{\isacharparenright}\isasep\isanewline%
Dis\ {\isacharparenleft}\isactrlbold {\isasymnot}\ {\isacharparenleft}\isactrlbold {\isasymnot}\ F{\isacharparenright}{\isacharparenright}\ F\ F} 
      \hfill (\isa{Dis{\isachardot}intros})
  \end{itemize}

  Cabe observar que las formalizaciones de la definiciones de fórmulas de tipo \isa{{\isasymalpha}} y \isa{{\isasymbeta}} son 
  definiciones sintácticas, pues construyen los correspondientes conjuntos de fórmulas a partir de 
  una reglas sintácticas concretas. Se trata de una simplificación de la intuición original de la 
  clasificación de las fórmulas mediante notación uniforme, ya que se prescinde de la noción de 
  equivalencia semántica que permite clasificar la totalidad de las fórmulas proposicionales. 

  Veamos la clasificación de casos concretos de fórmulas. Por ejemplo, según hemos definido la 
  fórmula \isa{{\isasymtop}}, es sencillo comprobar que se trata de una fórmula disyuntiva.%
\end{isamarkuptext}\isamarkuptrue%
\isacommand{lemma}\isamarkupfalse%
\ {\isachardoublequoteopen}Dis\ {\isasymtop}\ {\isacharparenleft}\isactrlbold {\isasymnot}\ {\isasymbottom}{\isacharparenright}\ {\isasymbottom}{\isachardoublequoteclose}\ \isanewline
%
\isadelimproof
\ \ %
\endisadelimproof
%
\isatagproof
\isacommand{unfolding}\isamarkupfalse%
\ Top{\isacharunderscore}def\ \isacommand{by}\isamarkupfalse%
\ {\isacharparenleft}simp\ only{\isacharcolon}\ Dis{\isachardot}intros{\isacharparenleft}{\isadigit{2}}{\isacharparenright}{\isacharparenright}%
\endisatagproof
{\isafoldproof}%
%
\isadelimproof
%
\endisadelimproof
%
\begin{isamarkuptext}%
Por otro lado, se observa a partir de las correspondientes definiciones que la conjunción
  generalizada de una lista de fórmulas es una fórmula de tipo \isa{{\isasymalpha}} y la disyunción generalizada de
  una lista de fórmulas es una fórmula de tipo \isa{{\isasymbeta}}.%
\end{isamarkuptext}\isamarkuptrue%
\isacommand{lemma}\isamarkupfalse%
\ {\isachardoublequoteopen}Con\ {\isacharparenleft}\isactrlbold {\isasymAnd}{\isacharparenleft}F{\isacharhash}Fs{\isacharparenright}{\isacharparenright}\ F\ {\isacharparenleft}\isactrlbold {\isasymAnd}Fs{\isacharparenright}{\isachardoublequoteclose}\isanewline
%
\isadelimproof
\ \ %
\endisadelimproof
%
\isatagproof
\isacommand{by}\isamarkupfalse%
\ {\isacharparenleft}simp\ only{\isacharcolon}\ BigAnd{\isachardot}simps\ Con{\isachardot}intros{\isacharparenleft}{\isadigit{1}}{\isacharparenright}{\isacharparenright}%
\endisatagproof
{\isafoldproof}%
%
\isadelimproof
\isanewline
%
\endisadelimproof
\isanewline
\isacommand{lemma}\isamarkupfalse%
\ {\isachardoublequoteopen}Dis\ {\isacharparenleft}\isactrlbold {\isasymOr}{\isacharparenleft}F{\isacharhash}Fs{\isacharparenright}{\isacharparenright}\ F\ {\isacharparenleft}\isactrlbold {\isasymOr}Fs{\isacharparenright}{\isachardoublequoteclose}\isanewline
%
\isadelimproof
\ \ %
\endisadelimproof
%
\isatagproof
\isacommand{by}\isamarkupfalse%
\ {\isacharparenleft}simp\ only{\isacharcolon}\ BigOr{\isachardot}simps\ Dis{\isachardot}intros{\isacharparenleft}{\isadigit{1}}{\isacharparenright}{\isacharparenright}%
\endisatagproof
{\isafoldproof}%
%
\isadelimproof
%
\endisadelimproof
%
\begin{isamarkuptext}%
Finalmente, de las reglas que definen las fórmulas conjuntivas y disyuntivas se deduce que
  la doble negación de una fórmula es una fórmula perteneciente a ambos tipos.%
\end{isamarkuptext}\isamarkuptrue%
\isacommand{lemma}\isamarkupfalse%
\ notDisCon{\isacharcolon}\ {\isachardoublequoteopen}Con\ {\isacharparenleft}Not\ {\isacharparenleft}Not\ F{\isacharparenright}{\isacharparenright}\ F\ F{\isachardoublequoteclose}\ {\isachardoublequoteopen}Dis\ {\isacharparenleft}Not\ {\isacharparenleft}Not\ F{\isacharparenright}{\isacharparenright}\ F\ F{\isachardoublequoteclose}\ \isanewline
%
\isadelimproof
\ \ %
\endisadelimproof
%
\isatagproof
\isacommand{by}\isamarkupfalse%
\ {\isacharparenleft}simp\ only{\isacharcolon}\ Con{\isachardot}intros{\isacharparenleft}{\isadigit{4}}{\isacharparenright}\ Dis{\isachardot}intros{\isacharparenleft}{\isadigit{4}}{\isacharparenright}{\isacharparenright}{\isacharplus}%
\endisatagproof
{\isafoldproof}%
%
\isadelimproof
%
\endisadelimproof
%
\begin{isamarkuptext}%
A continuación vamos a introducir el siguiente lema que caracteriza las fórmulas de tipo \isa{{\isasymalpha}} 
  y \isa{{\isasymbeta}}, facilitando el uso de la notación uniforme en Isabelle.%
\end{isamarkuptext}\isamarkuptrue%
\isacommand{lemma}\isamarkupfalse%
\ con{\isacharunderscore}dis{\isacharunderscore}simps{\isacharcolon}\isanewline
\ \ {\isachardoublequoteopen}Con\ a{\isadigit{1}}\ a{\isadigit{2}}\ a{\isadigit{3}}\ {\isacharequal}\ {\isacharparenleft}a{\isadigit{1}}\ {\isacharequal}\ a{\isadigit{2}}\ \isactrlbold {\isasymand}\ a{\isadigit{3}}\ {\isasymor}\ \isanewline
\ \ \ \ {\isacharparenleft}{\isasymexists}F\ G{\isachardot}\ a{\isadigit{1}}\ {\isacharequal}\ \isactrlbold {\isasymnot}\ {\isacharparenleft}F\ \isactrlbold {\isasymor}\ G{\isacharparenright}\ {\isasymand}\ a{\isadigit{2}}\ {\isacharequal}\ \isactrlbold {\isasymnot}\ F\ {\isasymand}\ a{\isadigit{3}}\ {\isacharequal}\ \isactrlbold {\isasymnot}\ G{\isacharparenright}\ {\isasymor}\ \isanewline
\ \ \ \ {\isacharparenleft}{\isasymexists}G{\isachardot}\ a{\isadigit{1}}\ {\isacharequal}\ \isactrlbold {\isasymnot}\ {\isacharparenleft}a{\isadigit{2}}\ \isactrlbold {\isasymrightarrow}\ G{\isacharparenright}\ {\isasymand}\ a{\isadigit{3}}\ {\isacharequal}\ \isactrlbold {\isasymnot}\ G{\isacharparenright}\ {\isasymor}\ \isanewline
\ \ \ \ a{\isadigit{1}}\ {\isacharequal}\ \isactrlbold {\isasymnot}\ {\isacharparenleft}\isactrlbold {\isasymnot}\ a{\isadigit{2}}{\isacharparenright}\ {\isasymand}\ a{\isadigit{3}}\ {\isacharequal}\ a{\isadigit{2}}{\isacharparenright}{\isachardoublequoteclose}\isanewline
\ \ {\isachardoublequoteopen}Dis\ a{\isadigit{1}}\ a{\isadigit{2}}\ a{\isadigit{3}}\ {\isacharequal}\ {\isacharparenleft}a{\isadigit{1}}\ {\isacharequal}\ a{\isadigit{2}}\ \isactrlbold {\isasymor}\ a{\isadigit{3}}\ {\isasymor}\ \isanewline
\ \ \ \ {\isacharparenleft}{\isasymexists}F\ G{\isachardot}\ a{\isadigit{1}}\ {\isacharequal}\ F\ \isactrlbold {\isasymrightarrow}\ G\ {\isasymand}\ a{\isadigit{2}}\ {\isacharequal}\ \isactrlbold {\isasymnot}\ F\ {\isasymand}\ a{\isadigit{3}}\ {\isacharequal}\ G{\isacharparenright}\ {\isasymor}\ \isanewline
\ \ \ \ {\isacharparenleft}{\isasymexists}F\ G{\isachardot}\ a{\isadigit{1}}\ {\isacharequal}\ \isactrlbold {\isasymnot}\ {\isacharparenleft}F\ \isactrlbold {\isasymand}\ G{\isacharparenright}\ {\isasymand}\ a{\isadigit{2}}\ {\isacharequal}\ \isactrlbold {\isasymnot}\ F\ {\isasymand}\ a{\isadigit{3}}\ {\isacharequal}\ \isactrlbold {\isasymnot}\ G{\isacharparenright}\ {\isasymor}\ \isanewline
\ \ \ \ a{\isadigit{1}}\ {\isacharequal}\ \isactrlbold {\isasymnot}\ {\isacharparenleft}\isactrlbold {\isasymnot}\ a{\isadigit{2}}{\isacharparenright}\ {\isasymand}\ a{\isadigit{3}}\ {\isacharequal}\ a{\isadigit{2}}{\isacharparenright}{\isachardoublequoteclose}\ \isanewline
%
\isadelimproof
\ \ %
\endisadelimproof
%
\isatagproof
\isacommand{by}\isamarkupfalse%
\ {\isacharparenleft}simp{\isacharunderscore}all\ add{\isacharcolon}\ Con{\isachardot}simps\ Dis{\isachardot}simps{\isacharparenright}%
\endisatagproof
{\isafoldproof}%
%
\isadelimproof
%
\endisadelimproof
%
\begin{isamarkuptext}%
Por último, introduzcamos un resultado que permite caracterizar los conjuntos de Hintikka 
  empleando la notación uniforme.

  \begin{lema}[Caracterización de los conjuntos de Hintikka mediante la notación uniforme]
    Dado un conjunto de fórmulas proposicionales \isa{S}, son equivalentes:
    \begin{enumerate}
      \item \isa{S} es un conjunto de Hintikka.
      \item Se verifican las condiciones siguientes:
      \begin{itemize}
        \item \isa{{\isasymbottom}} no pertenece a \isa{S}.
        \item Dada \isa{p} una fórmula atómica cualquiera, no se tiene 
        simultáneamente que\\ \isa{p\ {\isasymin}\ S} y \isa{{\isasymnot}\ p\ {\isasymin}\ S}.
        \item Para toda fórmula de tipo \isa{{\isasymalpha}} con componentes \isa{{\isasymalpha}\isactrlsub {\isadigit{1}}} y \isa{{\isasymalpha}\isactrlsub {\isadigit{2}}} se verifica 
        que si la fórmula pertenece a \isa{S}, entonces \isa{{\isasymalpha}\isactrlsub {\isadigit{1}}} y \isa{{\isasymalpha}\isactrlsub {\isadigit{2}}} también.
        \item Para toda fórmula de tipo \isa{{\isasymbeta}} con componentes \isa{{\isasymbeta}\isactrlsub {\isadigit{1}}} y \isa{{\isasymbeta}\isactrlsub {\isadigit{2}}} se verifica 
        que si la fórmula pertenece a \isa{S}, entonces o bien \isa{{\isasymbeta}\isactrlsub {\isadigit{1}}} pertenece
        a \isa{S} o bien \isa{{\isasymbeta}\isactrlsub {\isadigit{2}}} pertenece a \isa{S}.
      \end{itemize} 
    \end{enumerate}
  \end{lema}

  En Isabelle/HOL se formaliza del siguiente modo.%
\end{isamarkuptext}\isamarkuptrue%
\isacommand{lemma}\isamarkupfalse%
\ {\isachardoublequoteopen}Hintikka\ S\ {\isacharequal}\ {\isacharparenleft}{\isasymbottom}\ {\isasymnotin}\ S\isanewline
{\isasymand}\ {\isacharparenleft}{\isasymforall}k{\isachardot}\ Atom\ k\ {\isasymin}\ S\ {\isasymlongrightarrow}\ \isactrlbold {\isasymnot}\ {\isacharparenleft}Atom\ k{\isacharparenright}\ {\isasymin}\ S\ {\isasymlongrightarrow}\ False{\isacharparenright}\isanewline
{\isasymand}\ {\isacharparenleft}{\isasymforall}F\ G\ H{\isachardot}\ Con\ F\ G\ H\ {\isasymlongrightarrow}\ F\ {\isasymin}\ S\ {\isasymlongrightarrow}\ G\ {\isasymin}\ S\ {\isasymand}\ H\ {\isasymin}\ S{\isacharparenright}\isanewline
{\isasymand}\ {\isacharparenleft}{\isasymforall}F\ G\ H{\isachardot}\ Dis\ F\ G\ H\ {\isasymlongrightarrow}\ F\ {\isasymin}\ S\ {\isasymlongrightarrow}\ G\ {\isasymin}\ S\ {\isasymor}\ H\ {\isasymin}\ S{\isacharparenright}{\isacharparenright}{\isachardoublequoteclose}\ \isanewline
%
\isadelimproof
\ \ %
\endisadelimproof
%
\isatagproof
\isacommand{oops}\isamarkupfalse%
%
\endisatagproof
{\isafoldproof}%
%
\isadelimproof
%
\endisadelimproof
%
\begin{isamarkuptext}%
Procedamos a la demostración del resultado.

\begin{demostracion}
  Para probar la equivalencia, veamos cada una de las implicaciones por separado.

\textbf{\isa{{\isadigit{1}}{\isacharparenright}\ {\isasymLongrightarrow}\ {\isadigit{2}}{\isacharparenright}}}

  Supongamos que \isa{S} es un conjunto de Hintikka. Vamos a probar que, en efecto, se 
  verifican las condiciones del enunciado del lema.

  Por definición de conjunto de Hintikka, \isa{S} verifica las siguientes condiciones:
  \begin{enumerate}
    \item \isa{{\isasymbottom}\ {\isasymnotin}\ S}.
    \item Dada \isa{p} una fórmula atómica cualquiera, no se tiene 
      simultáneamente que\\ \isa{p\ {\isasymin}\ S} y \isa{{\isasymnot}\ p\ {\isasymin}\ S}.
    \item Si \isa{G\ {\isasymand}\ H\ {\isasymin}\ S}, entonces \isa{G\ {\isasymin}\ S} y \isa{H\ {\isasymin}\ S}.
    \item Si \isa{G\ {\isasymor}\ H\ {\isasymin}\ S}, entonces \isa{G\ {\isasymin}\ S} o \isa{H\ {\isasymin}\ S}.
    \item Si \isa{G\ {\isasymrightarrow}\ H\ {\isasymin}\ S}, entonces \isa{{\isasymnot}\ G\ {\isasymin}\ S} o \isa{H\ {\isasymin}\ S}.
    \item Si \isa{{\isasymnot}{\isacharparenleft}{\isasymnot}\ G{\isacharparenright}\ {\isasymin}\ S}, entonces \isa{G\ {\isasymin}\ S}.
    \item Si \isa{{\isasymnot}{\isacharparenleft}G\ {\isasymand}\ H{\isacharparenright}\ {\isasymin}\ S}, entonces \isa{{\isasymnot}\ G\ {\isasymin}\ S} o \isa{{\isasymnot}\ H\ {\isasymin}\ S}.
    \item Si \isa{{\isasymnot}{\isacharparenleft}G\ {\isasymor}\ H{\isacharparenright}\ {\isasymin}\ S}, entonces \isa{{\isasymnot}\ G\ {\isasymin}\ S} y \isa{{\isasymnot}\ H\ {\isasymin}\ S}. 
    \item Si \isa{{\isasymnot}{\isacharparenleft}G\ {\isasymrightarrow}\ H{\isacharparenright}\ {\isasymin}\ S}, entonces \isa{G\ {\isasymin}\ S} y \isa{{\isasymnot}\ H\ {\isasymin}\ S}. 
  \end{enumerate}  
  De este modo, el conjunto \isa{S} cumple la primera y la segunda condición del
  enunciado del lema, que se corresponden con las dos primeras condiciones
  de la definición de conjunto de Hintikka. Veamos que, además, verifica las
  dos últimas condiciones del resultado.

  En primer lugar, probemos que para toda fórmula de tipo \isa{{\isasymalpha}} con 
  componentes \isa{{\isasymalpha}\isactrlsub {\isadigit{1}}} y \isa{{\isasymalpha}\isactrlsub {\isadigit{2}}} se verifica que si la fórmula pertenece al conjunto 
  \isa{S}, entonces \isa{{\isasymalpha}\isactrlsub {\isadigit{1}}} y \isa{{\isasymalpha}\isactrlsub {\isadigit{2}}} también. Para ello, supongamos que una fórmula 
  cualquiera de tipo \isa{{\isasymalpha}} pertence a \isa{S}. Por definición de este tipo de
  fórmulas, tenemos que \isa{{\isasymalpha}} puede ser de la forma \isa{G\ {\isasymand}\ H}, \isa{{\isasymnot}{\isacharparenleft}{\isasymnot}\ G{\isacharparenright}},\\ \isa{{\isasymnot}{\isacharparenleft}G\ {\isasymor}\ H{\isacharparenright}} 
  o \isa{{\isasymnot}{\isacharparenleft}G\ {\isasymlongrightarrow}\ H{\isacharparenright}} para fórmulas \isa{G} y \isa{H} cualesquiera. Probemos que, para cada
  tipo de fórmula \isa{{\isasymalpha}} perteneciente a \isa{S}, sus componentes \isa{{\isasymalpha}\isactrlsub {\isadigit{1}}} y \isa{{\isasymalpha}\isactrlsub {\isadigit{2}}} están en
  \isa{S}.

  \isa{{\isasymsqdot}\ Fórmula\ del\ tipo\ G\ {\isasymand}\ H}: Sus componentes conjuntivas son \isa{G} y \isa{H}. 
  Por la tercera condición de la definición de conjunto de Hintikka, obtenemos 
  que si \isa{G\ {\isasymand}\ H} pertenece a \isa{S}, entonces \isa{G} y \isa{H} están ambas en el conjunto,
  lo que prueba este caso.
    
  \isa{{\isasymsqdot}\ Fórmula\ del\ tipo\ {\isasymnot}{\isacharparenleft}{\isasymnot}\ G{\isacharparenright}}: Sus componentes conjuntivas son ambas \isa{G}.
  Por la sexta condición de la definición de conjunto de Hintikka, obtenemos que
  si \isa{{\isasymnot}{\isacharparenleft}{\isasymnot}\ G{\isacharparenright}} pertenece a \isa{S}, entonces \isa{G} pertenece al conjunto, lo que prueba
  este caso.

  \isa{{\isasymsqdot}\ Fórmula\ del\ tipo\ {\isasymnot}{\isacharparenleft}G\ {\isasymor}\ H{\isacharparenright}}: Sus componentes conjuntivas son \isa{{\isasymnot}\ G} y \isa{{\isasymnot}\ H}. 
  Por la octava condición de la definición de conjunto de Hintikka, obtenemos 
  que si \isa{{\isasymnot}{\isacharparenleft}G\ {\isasymor}\ H{\isacharparenright}} pertenece a \isa{S}, entonces \isa{{\isasymnot}\ G} y \isa{{\isasymnot}\ H} están ambas en el conjunto,
  lo que prueba este caso.

  \isa{{\isasymsqdot}\ Fórmula\ del\ tipo\ {\isasymnot}{\isacharparenleft}G\ {\isasymlongrightarrow}\ H{\isacharparenright}}: Sus componentes conjuntivas son \isa{G} y \isa{{\isasymnot}\ H}. 
  Por la novena condición de la definición de conjunto de Hintikka, obtenemos 
  que si\\ \isa{{\isasymnot}{\isacharparenleft}G\ {\isasymlongrightarrow}\ H{\isacharparenright}} pertenece a \isa{S}, entonces \isa{G} y \isa{{\isasymnot}\ H} están ambas en el conjunto,
  lo que prueba este caso.

  Finalmente, probemos que para toda fórmula de tipo \isa{{\isasymbeta}} con componentes \isa{{\isasymbeta}\isactrlsub {\isadigit{1}}} y 
  \isa{{\isasymbeta}\isactrlsub {\isadigit{2}}} se verifica que si la fórmula pertenece al conjunto \isa{S}, entonces o bien \isa{{\isasymbeta}\isactrlsub {\isadigit{1}}} 
  pertenece al conjunto o bien \isa{{\isasymbeta}\isactrlsub {\isadigit{2}}} pertenece a conjunto. Para ello, supongamos que 
  una fórmula cualquiera de tipo \isa{{\isasymbeta}} pertence a \isa{S}. Por definición de este tipo de
  fórmulas, tenemos que \isa{{\isasymbeta}} puede ser de la forma \isa{G\ {\isasymor}\ H}, \isa{G\ {\isasymlongrightarrow}\ H}, \isa{{\isasymnot}{\isacharparenleft}{\isasymnot}\ G{\isacharparenright}} 
  o \isa{{\isasymnot}{\isacharparenleft}G\ {\isasymand}\ H{\isacharparenright}} para fórmulas \isa{G} y \isa{H} cualesquiera. Probemos que, para cada
  tipo de fórmula \isa{{\isasymbeta}} perteneciente a \isa{S}, o bien su componente \isa{{\isasymbeta}\isactrlsub {\isadigit{1}}} pertenece a \isa{S} 
  o bien su componente \isa{{\isasymbeta}\isactrlsub {\isadigit{2}}} pertenece a \isa{S}.

  \isa{{\isasymsqdot}\ Fórmula\ del\ tipo\ G\ {\isasymor}\ H}: Sus componentes disyuntivas son \isa{G} y \isa{H}. 
    Por la cuarta condición de la definición de conjunto de Hintikka, obtenemos 
    que si \isa{G\ {\isasymor}\ H} pertenece a \isa{S}, entonces o bien \isa{G} está en \isa{S} o bien \isa{H} está
    en \isa{S}, lo que prueba este caso.

  \isa{{\isasymsqdot}\ Fórmula\ del\ tipo\ G\ {\isasymlongrightarrow}\ H}: Sus componentes disyuntivas son \isa{{\isasymnot}\ G} y \isa{H}.
    Por la quinta condición de la definición de conjunto de Hintikka, obtenemos que
    si \isa{G\ {\isasymlongrightarrow}\ H} pertenece a \isa{S}, entonces o bien \isa{{\isasymnot}\ G} pertenece al conjunto o bien
    \isa{H} pertenece al conjunto, lo que prueba este caso.

  \isa{{\isasymsqdot}\ Fórmula\ del\ tipo\ {\isasymnot}{\isacharparenleft}{\isasymnot}\ G{\isacharparenright}}: Sus componentes conjuntivas son ambas \isa{G}.
    Por la sexta condición de la definición de conjunto de Hintikka, obtenemos 
    que si \isa{{\isasymnot}{\isacharparenleft}{\isasymnot}\ G{\isacharparenright}} pertenece a \isa{S}, entonces \isa{G} pertenece al conjunto. De este modo,
    por la regla de introducción a la disyunción, se prueba que o bien una de las 
    componentes está en el conjunto o bien lo está la otra pues, en este caso,
    coinciden.

  \isa{{\isasymsqdot}\ Fórmula\ del\ tipo\ {\isasymnot}{\isacharparenleft}G\ {\isasymand}\ H{\isacharparenright}}: Sus componentes conjuntivas son \isa{{\isasymnot}\ G} y \isa{{\isasymnot}\ H}. 
    Por la séptima condición de la definición de conjunto de Hintikka, obtenemos 
    que si\\ \isa{{\isasymnot}{\isacharparenleft}G\ {\isasymand}\ H{\isacharparenright}} pertenece a \isa{S}, entonces o bien \isa{{\isasymnot}\ G} pertenece al conjunto
    o bien \isa{{\isasymnot}\ H} pertenece al conjunto, lo que prueba este caso.

\textbf{\isa{{\isadigit{2}}{\isacharparenright}\ {\isasymLongrightarrow}\ {\isadigit{1}}{\isacharparenright}}}

  Supongamos que se verifican las condiciones del enunciado del lema:

  \begin{itemize}
    \item \isa{{\isasymbottom}} no pertenece a \isa{S}.
    \item Dada \isa{p} una fórmula atómica cualquiera, no se tiene 
    simultáneamente que\\ \isa{p\ {\isasymin}\ S} y \isa{{\isasymnot}\ p\ {\isasymin}\ S}.
    \item Para toda fórmula de tipo \isa{{\isasymalpha}} con componentes \isa{{\isasymalpha}\isactrlsub {\isadigit{1}}} y \isa{{\isasymalpha}\isactrlsub {\isadigit{2}}} se verifica 
    que si la fórmula pertenece a \isa{S}, entonces \isa{{\isasymalpha}\isactrlsub {\isadigit{1}}} y \isa{{\isasymalpha}\isactrlsub {\isadigit{2}}} también.
    \item Para toda fórmula de tipo \isa{{\isasymbeta}} con componentes \isa{{\isasymbeta}\isactrlsub {\isadigit{1}}} y \isa{{\isasymbeta}\isactrlsub {\isadigit{2}}} se verifica 
    que si la fórmula pertenece a \isa{S}, entonces o bien \isa{{\isasymbeta}\isactrlsub {\isadigit{1}}} pertenece
    a \isa{S} o bien \isa{{\isasymbeta}\isactrlsub {\isadigit{2}}} pertenece a \isa{S}.
  \end{itemize}  

  Vamos a probar que \isa{S} es un conjunto de Hintikka.

  Por la definición de conjunto de Hintikka, es suficiente probar las siguientes
  condiciones:

  \begin{enumerate}
    \item \isa{{\isasymbottom}\ {\isasymnotin}\ S}.
    \item Dada \isa{p} una fórmula atómica cualquiera, no se tiene 
      simultáneamente que\\ \isa{p\ {\isasymin}\ S} y \isa{{\isasymnot}\ p\ {\isasymin}\ S}.
    \item Si \isa{G\ {\isasymand}\ H\ {\isasymin}\ S}, entonces \isa{G\ {\isasymin}\ S} y \isa{H\ {\isasymin}\ S}.
    \item Si \isa{G\ {\isasymor}\ H\ {\isasymin}\ S}, entonces \isa{G\ {\isasymin}\ S} o \isa{H\ {\isasymin}\ S}.
    \item Si \isa{G\ {\isasymrightarrow}\ H\ {\isasymin}\ S}, entonces \isa{{\isasymnot}\ G\ {\isasymin}\ S} o \isa{H\ {\isasymin}\ S}.
    \item Si \isa{{\isasymnot}{\isacharparenleft}{\isasymnot}\ G{\isacharparenright}\ {\isasymin}\ S}, entonces \isa{G\ {\isasymin}\ S}.
    \item Si \isa{{\isasymnot}{\isacharparenleft}G\ {\isasymand}\ H{\isacharparenright}\ {\isasymin}\ S}, entonces \isa{{\isasymnot}\ G\ {\isasymin}\ S} o \isa{{\isasymnot}\ H\ {\isasymin}\ S}.
    \item Si \isa{{\isasymnot}{\isacharparenleft}G\ {\isasymor}\ H{\isacharparenright}\ {\isasymin}\ S}, entonces \isa{{\isasymnot}\ G\ {\isasymin}\ S} y \isa{{\isasymnot}\ H\ {\isasymin}\ S}. 
    \item Si \isa{{\isasymnot}{\isacharparenleft}G\ {\isasymrightarrow}\ H{\isacharparenright}\ {\isasymin}\ S}, entonces \isa{G\ {\isasymin}\ S} y \isa{{\isasymnot}\ H\ {\isasymin}\ S}. 
  \end{enumerate} 

  En primer lugar se observa que, por hipótesis, se verifican las dos primeras
  condiciones de la definición de conjunto de Hintikka. Veamos que, en efecto, se
  cumplen las demás.

  \begin{enumerate}
    \item[\isa{{\isadigit{3}}{\isacharparenright}}] Supongamos que \isa{G\ {\isasymand}\ H} está en \isa{S} para fórmulas \isa{G} y \isa{H} cualesquiera.
    Por definición, \isa{G\ {\isasymand}\ H} es una fórmula de tipo \isa{{\isasymalpha}} con componentes \isa{G} y \isa{H}. 
    Por lo tanto, por hipótesis se cumple que \isa{G} y \isa{H} están en \isa{S}.
    \item[\isa{{\isadigit{4}}{\isacharparenright}}] Supongamos que \isa{G\ {\isasymor}\ H} está en \isa{S} para fórmulas \isa{G} y \isa{H} cualesquiera.
    Por definición, \isa{G\ {\isasymor}\ H} es una fórmula de tipo \isa{{\isasymbeta}} con componentes \isa{G} y \isa{H}. 
    Por lo tanto, por hipótesis se cumple que o bien \isa{G} está en \isa{S} o bien \isa{H} está 
    en \isa{S}.
    \item[\isa{{\isadigit{5}}{\isacharparenright}}] Supongamos que \isa{G\ {\isasymlongrightarrow}\ H} está en \isa{S} para fórmulas \isa{G} y \isa{H} cualesquiera.
    Por definición, \isa{G\ {\isasymlongrightarrow}\ H} es una fórmula de tipo \isa{{\isasymbeta}} con componentes \isa{{\isasymnot}\ G} y \isa{H}. 
    Por lo tanto, por hipótesis se cumple que o bien \isa{{\isasymnot}\ G} está en \isa{S} o bien \isa{H} está 
    en \isa{S}.
    \item[\isa{{\isadigit{6}}{\isacharparenright}}] Supongamos que \isa{{\isasymnot}{\isacharparenleft}{\isasymnot}\ G{\isacharparenright}} está en \isa{S} para una fórmula \isa{G} cualquiera.
    Por definición, \isa{{\isasymnot}{\isacharparenleft}{\isasymnot}\ G{\isacharparenright}} es una fórmula de tipo \isa{{\isasymalpha}} cuyas componentes son ambas \isa{G}. 
    Por lo tanto, por hipótesis se cumple que \isa{G} está en \isa{S}.
    \item[\isa{{\isadigit{7}}{\isacharparenright}}] Supongamos que \isa{{\isasymnot}{\isacharparenleft}G\ {\isasymand}\ H{\isacharparenright}} está en \isa{S} para fórmulas \isa{G} y \isa{H} cualesquiera.
    Por definición, \isa{{\isasymnot}{\isacharparenleft}G\ {\isasymand}\ H{\isacharparenright}} es una fórmula de tipo \isa{{\isasymbeta}} con componentes \isa{{\isasymnot}\ G} y \isa{{\isasymnot}\ H}. 
    Por lo tanto, por hipótesis se cumple que o bien \isa{{\isasymnot}\ G} está en \isa{S} o bien \isa{{\isasymnot}\ H} está 
    en \isa{S}.
    \item[\isa{{\isadigit{8}}{\isacharparenright}}] Supongamos que \isa{{\isasymnot}{\isacharparenleft}G\ {\isasymor}\ H{\isacharparenright}} está en \isa{S} para fórmulas \isa{G} y \isa{H} cualesquiera.
    Por definición, \isa{{\isasymnot}{\isacharparenleft}G\ {\isasymor}\ H{\isacharparenright}} es una fórmula de tipo \isa{{\isasymalpha}} con componentes \isa{{\isasymnot}\ G} y \isa{{\isasymnot}\ H}. 
    Por lo tanto, por hipótesis se cumple que \isa{{\isasymnot}\ G} y \isa{{\isasymnot}\ H} están en \isa{S}.
    \item[\isa{{\isadigit{9}}{\isacharparenright}}] Supongamos que \isa{{\isasymnot}{\isacharparenleft}G\ {\isasymlongrightarrow}\ H{\isacharparenright}} está en \isa{S} para fórmulas \isa{G} y \isa{H} cualesquiera. 
    Por definición, \isa{{\isasymnot}{\isacharparenleft}G\ {\isasymlongrightarrow}\ H{\isacharparenright}} es una fórmula de tipo \isa{{\isasymalpha}} con componentes \isa{G} y \isa{{\isasymnot}\ H}.
    Por lo tanto, por hipótesis se cumple que \isa{G} y \isa{{\isasymnot}\ H} están en \isa{S}.
  \end{enumerate}

  Por tanto, queda probado el resultado.
\end{demostracion}

  Para probar de manera detallada el lema en Isabelle vamos a demostrar
  cada una de las implicaciones de la equivalencia por separado. 

  La primera implicación del lema se basa en dos lemas auxiliares. El primero de ellos 
  prueba que la tercera, sexta, octava y novena condición de la definición de conjunto de 
  Hintikka son suficientes para probar que para toda fórmula de tipo \isa{{\isasymalpha}} con componentes 
  \isa{{\isasymalpha}\isactrlsub {\isadigit{1}}} y \isa{{\isasymalpha}\isactrlsub {\isadigit{2}}} se verifica que si la fórmula pertenece al conjunto \isa{S}, entonces \isa{{\isasymalpha}\isactrlsub {\isadigit{1}}} y 
  \isa{{\isasymalpha}\isactrlsub {\isadigit{2}}} también. Su demostración detallada en Isabelle se muestra a continuación.%
\end{isamarkuptext}\isamarkuptrue%
\isacommand{lemma}\isamarkupfalse%
\ Hintikka{\isacharunderscore}alt{\isadigit{1}}Con{\isacharcolon}\isanewline
\ \ \isakeyword{assumes}\ {\isachardoublequoteopen}{\isacharparenleft}{\isasymforall}G\ H{\isachardot}\ G\ \isactrlbold {\isasymand}\ H\ {\isasymin}\ S\ {\isasymlongrightarrow}\ G\ {\isasymin}\ S\ {\isasymand}\ H\ {\isasymin}\ S{\isacharparenright}\isanewline
\ \ {\isasymand}\ {\isacharparenleft}{\isasymforall}G{\isachardot}\ \isactrlbold {\isasymnot}\ {\isacharparenleft}\isactrlbold {\isasymnot}\ G{\isacharparenright}\ {\isasymin}\ S\ {\isasymlongrightarrow}\ G\ {\isasymin}\ S{\isacharparenright}\isanewline
\ \ {\isasymand}\ {\isacharparenleft}{\isasymforall}G\ H{\isachardot}\ \isactrlbold {\isasymnot}{\isacharparenleft}G\ \isactrlbold {\isasymor}\ H{\isacharparenright}\ {\isasymin}\ S\ {\isasymlongrightarrow}\ \isactrlbold {\isasymnot}\ G\ {\isasymin}\ S\ {\isasymand}\ \isactrlbold {\isasymnot}\ H\ {\isasymin}\ S{\isacharparenright}\isanewline
\ \ {\isasymand}\ {\isacharparenleft}{\isasymforall}G\ H{\isachardot}\ \isactrlbold {\isasymnot}{\isacharparenleft}G\ \isactrlbold {\isasymrightarrow}\ H{\isacharparenright}\ {\isasymin}\ S\ {\isasymlongrightarrow}\ G\ {\isasymin}\ S\ {\isasymand}\ \isactrlbold {\isasymnot}\ H\ {\isasymin}\ S{\isacharparenright}{\isachardoublequoteclose}\isanewline
\ \ \isakeyword{shows}\ {\isachardoublequoteopen}Con\ F\ G\ H\ {\isasymlongrightarrow}\ F\ {\isasymin}\ S\ {\isasymlongrightarrow}\ G\ {\isasymin}\ S\ {\isasymand}\ H\ {\isasymin}\ S{\isachardoublequoteclose}\isanewline
%
\isadelimproof
%
\endisadelimproof
%
\isatagproof
\isacommand{proof}\isamarkupfalse%
\ {\isacharparenleft}rule\ impI{\isacharparenright}\isanewline
\ \ \isacommand{assume}\isamarkupfalse%
\ {\isachardoublequoteopen}Con\ F\ G\ H{\isachardoublequoteclose}\isanewline
\ \ \isacommand{then}\isamarkupfalse%
\ \isacommand{have}\isamarkupfalse%
\ {\isachardoublequoteopen}F\ {\isacharequal}\ G\ \isactrlbold {\isasymand}\ H\ {\isasymor}\ \isanewline
\ \ \ \ {\isacharparenleft}{\isacharparenleft}{\isasymexists}G{\isadigit{1}}\ H{\isadigit{1}}{\isachardot}\ F\ {\isacharequal}\ \isactrlbold {\isasymnot}\ {\isacharparenleft}G{\isadigit{1}}\ \isactrlbold {\isasymor}\ H{\isadigit{1}}{\isacharparenright}\ {\isasymand}\ G\ {\isacharequal}\ \isactrlbold {\isasymnot}\ G{\isadigit{1}}\ {\isasymand}\ H\ {\isacharequal}\ \isactrlbold {\isasymnot}\ H{\isadigit{1}}{\isacharparenright}\ {\isasymor}\ \isanewline
\ \ \ \ {\isacharparenleft}{\isasymexists}H{\isadigit{2}}{\isachardot}\ F\ {\isacharequal}\ \isactrlbold {\isasymnot}\ {\isacharparenleft}G\ \isactrlbold {\isasymrightarrow}\ H{\isadigit{2}}{\isacharparenright}\ {\isasymand}\ H\ {\isacharequal}\ \isactrlbold {\isasymnot}\ H{\isadigit{2}}{\isacharparenright}\ {\isasymor}\ \isanewline
\ \ \ \ F\ {\isacharequal}\ \isactrlbold {\isasymnot}\ {\isacharparenleft}\isactrlbold {\isasymnot}\ G{\isacharparenright}\ {\isasymand}\ H\ {\isacharequal}\ G{\isacharparenright}{\isachardoublequoteclose}\isanewline
\ \ \ \ \isacommand{by}\isamarkupfalse%
\ {\isacharparenleft}simp\ only{\isacharcolon}\ con{\isacharunderscore}dis{\isacharunderscore}simps{\isacharparenleft}{\isadigit{1}}{\isacharparenright}{\isacharparenright}\isanewline
\ \ \isacommand{thus}\isamarkupfalse%
\ {\isachardoublequoteopen}F\ {\isasymin}\ S\ {\isasymlongrightarrow}\ G\ {\isasymin}\ S\ {\isasymand}\ H\ {\isasymin}\ S{\isachardoublequoteclose}\isanewline
\ \ \isacommand{proof}\isamarkupfalse%
\ {\isacharparenleft}rule\ disjE{\isacharparenright}\isanewline
\ \ \ \ \isacommand{assume}\isamarkupfalse%
\ {\isachardoublequoteopen}F\ {\isacharequal}\ G\ \isactrlbold {\isasymand}\ H{\isachardoublequoteclose}\isanewline
\ \ \ \ \isacommand{have}\isamarkupfalse%
\ {\isachardoublequoteopen}{\isasymforall}G\ H{\isachardot}\ G\ \isactrlbold {\isasymand}\ H\ {\isasymin}\ S\ {\isasymlongrightarrow}\ G\ {\isasymin}\ S\ {\isasymand}\ H\ {\isasymin}\ S{\isachardoublequoteclose}\isanewline
\ \ \ \ \ \ \isacommand{using}\isamarkupfalse%
\ assms\ \isacommand{by}\isamarkupfalse%
\ {\isacharparenleft}rule\ conjunct{\isadigit{1}}{\isacharparenright}\isanewline
\ \ \ \ \isacommand{thus}\isamarkupfalse%
\ {\isachardoublequoteopen}F\ {\isasymin}\ S\ {\isasymlongrightarrow}\ G\ {\isasymin}\ S\ {\isasymand}\ H\ {\isasymin}\ S{\isachardoublequoteclose}\isanewline
\ \ \ \ \ \ \isacommand{using}\isamarkupfalse%
\ {\isacartoucheopen}F\ {\isacharequal}\ G\ \isactrlbold {\isasymand}\ H{\isacartoucheclose}\ \isacommand{by}\isamarkupfalse%
\ {\isacharparenleft}iprover\ elim{\isacharcolon}\ allE{\isacharparenright}\isanewline
\ \ \isacommand{next}\isamarkupfalse%
\ \isanewline
\ \ \ \ \isacommand{assume}\isamarkupfalse%
\ {\isachardoublequoteopen}{\isacharparenleft}{\isasymexists}G{\isadigit{1}}\ H{\isadigit{1}}{\isachardot}\ F\ {\isacharequal}\ \isactrlbold {\isasymnot}\ {\isacharparenleft}G{\isadigit{1}}\ \isactrlbold {\isasymor}\ H{\isadigit{1}}{\isacharparenright}\ {\isasymand}\ G\ {\isacharequal}\ \isactrlbold {\isasymnot}\ G{\isadigit{1}}\ {\isasymand}\ H\ {\isacharequal}\ \isactrlbold {\isasymnot}\ H{\isadigit{1}}{\isacharparenright}\ {\isasymor}\ \isanewline
\ \ \ \ {\isacharparenleft}{\isacharparenleft}{\isasymexists}H{\isadigit{2}}{\isachardot}\ F\ {\isacharequal}\ \isactrlbold {\isasymnot}\ {\isacharparenleft}G\ \isactrlbold {\isasymrightarrow}\ H{\isadigit{2}}{\isacharparenright}\ {\isasymand}\ H\ {\isacharequal}\ \isactrlbold {\isasymnot}\ H{\isadigit{2}}{\isacharparenright}\ {\isasymor}\ \isanewline
\ \ \ \ F\ {\isacharequal}\ \isactrlbold {\isasymnot}\ {\isacharparenleft}\isactrlbold {\isasymnot}\ G{\isacharparenright}\ {\isasymand}\ H\ {\isacharequal}\ G{\isacharparenright}{\isachardoublequoteclose}\isanewline
\ \ \ \ \isacommand{thus}\isamarkupfalse%
\ {\isachardoublequoteopen}F\ {\isasymin}\ S\ {\isasymlongrightarrow}\ G\ {\isasymin}\ S\ {\isasymand}\ H\ {\isasymin}\ S{\isachardoublequoteclose}\ \isanewline
\ \ \ \ \isacommand{proof}\isamarkupfalse%
\ {\isacharparenleft}rule\ disjE{\isacharparenright}\isanewline
\ \ \ \ \ \ \isacommand{assume}\isamarkupfalse%
\ E{\isadigit{1}}{\isacharcolon}{\isachardoublequoteopen}{\isasymexists}G{\isadigit{1}}\ H{\isadigit{1}}{\isachardot}\ F\ {\isacharequal}\ \isactrlbold {\isasymnot}\ {\isacharparenleft}G{\isadigit{1}}\ \isactrlbold {\isasymor}\ H{\isadigit{1}}{\isacharparenright}\ {\isasymand}\ G\ {\isacharequal}\ \isactrlbold {\isasymnot}\ G{\isadigit{1}}\ {\isasymand}\ H\ {\isacharequal}\ \isactrlbold {\isasymnot}\ H{\isadigit{1}}{\isachardoublequoteclose}\isanewline
\ \ \ \ \ \ \isacommand{obtain}\isamarkupfalse%
\ G{\isadigit{1}}\ H{\isadigit{1}}\ \isakeyword{where}\ A{\isadigit{1}}{\isacharcolon}{\isachardoublequoteopen}F\ {\isacharequal}\ \isactrlbold {\isasymnot}\ {\isacharparenleft}G{\isadigit{1}}\ \isactrlbold {\isasymor}\ H{\isadigit{1}}{\isacharparenright}\ {\isasymand}\ G\ {\isacharequal}\ \isactrlbold {\isasymnot}\ G{\isadigit{1}}\ {\isasymand}\ H\ {\isacharequal}\ \isactrlbold {\isasymnot}\ H{\isadigit{1}}{\isachardoublequoteclose}\isanewline
\ \ \ \ \ \ \ \ \isacommand{using}\isamarkupfalse%
\ E{\isadigit{1}}\ \isacommand{by}\isamarkupfalse%
\ {\isacharparenleft}iprover\ elim{\isacharcolon}\ exE{\isacharparenright}\isanewline
\ \ \ \ \ \ \isacommand{then}\isamarkupfalse%
\ \isacommand{have}\isamarkupfalse%
\ {\isachardoublequoteopen}F\ {\isacharequal}\ \isactrlbold {\isasymnot}\ {\isacharparenleft}G{\isadigit{1}}\ \isactrlbold {\isasymor}\ H{\isadigit{1}}{\isacharparenright}{\isachardoublequoteclose}\isanewline
\ \ \ \ \ \ \ \ \isacommand{by}\isamarkupfalse%
\ {\isacharparenleft}rule\ conjunct{\isadigit{1}}{\isacharparenright}\isanewline
\ \ \ \ \ \ \isacommand{have}\isamarkupfalse%
\ {\isachardoublequoteopen}G\ {\isacharequal}\ \isactrlbold {\isasymnot}\ G{\isadigit{1}}{\isachardoublequoteclose}\isanewline
\ \ \ \ \ \ \ \ \isacommand{using}\isamarkupfalse%
\ A{\isadigit{1}}\ \isacommand{by}\isamarkupfalse%
\ {\isacharparenleft}iprover\ elim{\isacharcolon}\ conjunct{\isadigit{1}}{\isacharparenright}\isanewline
\ \ \ \ \ \ \isacommand{have}\isamarkupfalse%
\ {\isachardoublequoteopen}H\ {\isacharequal}\ \isactrlbold {\isasymnot}\ H{\isadigit{1}}{\isachardoublequoteclose}\isanewline
\ \ \ \ \ \ \ \ \isacommand{using}\isamarkupfalse%
\ A{\isadigit{1}}\ \isacommand{by}\isamarkupfalse%
\ {\isacharparenleft}iprover\ elim{\isacharcolon}\ conjunct{\isadigit{1}}{\isacharparenright}\isanewline
\ \ \ \ \ \ \isacommand{have}\isamarkupfalse%
\ {\isachardoublequoteopen}{\isasymforall}G\ H{\isachardot}\ \isactrlbold {\isasymnot}{\isacharparenleft}G\ \isactrlbold {\isasymor}\ H{\isacharparenright}\ {\isasymin}\ S\ {\isasymlongrightarrow}\ \isactrlbold {\isasymnot}\ G\ {\isasymin}\ S\ {\isasymand}\ \isactrlbold {\isasymnot}\ H\ {\isasymin}\ S{\isachardoublequoteclose}\isanewline
\ \ \ \ \ \ \ \ \isacommand{using}\isamarkupfalse%
\ assms\ \isacommand{by}\isamarkupfalse%
\ {\isacharparenleft}iprover\ elim{\isacharcolon}\ conjunct{\isadigit{2}}\ conjunct{\isadigit{1}}{\isacharparenright}\isanewline
\ \ \ \ \ \ \isacommand{thus}\isamarkupfalse%
\ {\isachardoublequoteopen}F\ {\isasymin}\ S\ {\isasymlongrightarrow}\ G\ {\isasymin}\ S\ {\isasymand}\ H\ {\isasymin}\ S{\isachardoublequoteclose}\isanewline
\ \ \ \ \ \ \ \ \isacommand{using}\isamarkupfalse%
\ {\isacartoucheopen}F\ {\isacharequal}\ \isactrlbold {\isasymnot}\ {\isacharparenleft}G{\isadigit{1}}\ \isactrlbold {\isasymor}\ H{\isadigit{1}}{\isacharparenright}{\isacartoucheclose}\ {\isacartoucheopen}G\ {\isacharequal}\ \isactrlbold {\isasymnot}\ G{\isadigit{1}}{\isacartoucheclose}\ {\isacartoucheopen}H\ {\isacharequal}\ \isactrlbold {\isasymnot}\ H{\isadigit{1}}{\isacartoucheclose}\ \isacommand{by}\isamarkupfalse%
\ {\isacharparenleft}iprover\ elim{\isacharcolon}\ allE{\isacharparenright}\isanewline
\ \ \ \ \isacommand{next}\isamarkupfalse%
\isanewline
\ \ \ \ \ \ \isacommand{assume}\isamarkupfalse%
\ {\isachardoublequoteopen}{\isacharparenleft}{\isasymexists}H{\isadigit{2}}{\isachardot}\ F\ {\isacharequal}\ \isactrlbold {\isasymnot}\ {\isacharparenleft}G\ \isactrlbold {\isasymrightarrow}\ H{\isadigit{2}}{\isacharparenright}\ {\isasymand}\ H\ {\isacharequal}\ \isactrlbold {\isasymnot}\ H{\isadigit{2}}{\isacharparenright}\ {\isasymor}\ \isanewline
\ \ \ \ \ \ F\ {\isacharequal}\ \isactrlbold {\isasymnot}\ {\isacharparenleft}\isactrlbold {\isasymnot}\ G{\isacharparenright}\ {\isasymand}\ H\ {\isacharequal}\ G{\isachardoublequoteclose}\isanewline
\ \ \ \ \ \ \isacommand{thus}\isamarkupfalse%
\ {\isachardoublequoteopen}F\ {\isasymin}\ S\ {\isasymlongrightarrow}\ G\ {\isasymin}\ S\ {\isasymand}\ H\ {\isasymin}\ S{\isachardoublequoteclose}\ \isanewline
\ \ \ \ \ \ \isacommand{proof}\isamarkupfalse%
\ {\isacharparenleft}rule\ disjE{\isacharparenright}\isanewline
\ \ \ \ \ \ \ \ \isacommand{assume}\isamarkupfalse%
\ E{\isadigit{2}}{\isacharcolon}{\isachardoublequoteopen}{\isasymexists}H{\isadigit{2}}{\isachardot}\ F\ {\isacharequal}\ \isactrlbold {\isasymnot}\ {\isacharparenleft}G\ \isactrlbold {\isasymrightarrow}\ H{\isadigit{2}}{\isacharparenright}\ {\isasymand}\ H\ {\isacharequal}\ \isactrlbold {\isasymnot}\ H{\isadigit{2}}{\isachardoublequoteclose}\isanewline
\ \ \ \ \ \ \ \ \isacommand{obtain}\isamarkupfalse%
\ H{\isadigit{2}}\ \isakeyword{where}\ A{\isadigit{2}}{\isacharcolon}{\isachardoublequoteopen}F\ {\isacharequal}\ \isactrlbold {\isasymnot}\ {\isacharparenleft}G\ \isactrlbold {\isasymrightarrow}\ H{\isadigit{2}}{\isacharparenright}\ {\isasymand}\ H\ {\isacharequal}\ \isactrlbold {\isasymnot}\ H{\isadigit{2}}{\isachardoublequoteclose}\isanewline
\ \ \ \ \ \ \ \ \ \ \isacommand{using}\isamarkupfalse%
\ E{\isadigit{2}}\ \isacommand{by}\isamarkupfalse%
\ {\isacharparenleft}rule\ exE{\isacharparenright}\isanewline
\ \ \ \ \ \ \ \ \isacommand{have}\isamarkupfalse%
\ {\isachardoublequoteopen}F\ {\isacharequal}\ \isactrlbold {\isasymnot}\ {\isacharparenleft}G\ \isactrlbold {\isasymrightarrow}\ H{\isadigit{2}}{\isacharparenright}{\isachardoublequoteclose}\isanewline
\ \ \ \ \ \ \ \ \ \ \isacommand{using}\isamarkupfalse%
\ A{\isadigit{2}}\ \isacommand{by}\isamarkupfalse%
\ {\isacharparenleft}rule\ conjunct{\isadigit{1}}{\isacharparenright}\isanewline
\ \ \ \ \ \ \ \ \isacommand{have}\isamarkupfalse%
\ {\isachardoublequoteopen}H\ {\isacharequal}\ \isactrlbold {\isasymnot}\ H{\isadigit{2}}{\isachardoublequoteclose}\isanewline
\ \ \ \ \ \ \ \ \ \ \isacommand{using}\isamarkupfalse%
\ A{\isadigit{2}}\ \isacommand{by}\isamarkupfalse%
\ {\isacharparenleft}rule\ conjunct{\isadigit{2}}{\isacharparenright}\isanewline
\ \ \ \ \ \ \ \ \isacommand{have}\isamarkupfalse%
\ {\isachardoublequoteopen}{\isasymforall}G\ H{\isachardot}\ \isactrlbold {\isasymnot}{\isacharparenleft}G\ \isactrlbold {\isasymrightarrow}\ H{\isacharparenright}\ {\isasymin}\ S\ {\isasymlongrightarrow}\ G\ {\isasymin}\ S\ {\isasymand}\ \isactrlbold {\isasymnot}\ H\ {\isasymin}\ S{\isachardoublequoteclose}\isanewline
\ \ \ \ \ \ \ \ \ \ \isacommand{using}\isamarkupfalse%
\ assms\ \isacommand{by}\isamarkupfalse%
\ {\isacharparenleft}iprover\ elim{\isacharcolon}\ conjunct{\isadigit{2}}\ conjunct{\isadigit{1}}{\isacharparenright}\isanewline
\ \ \ \ \ \ \ \ \isacommand{thus}\isamarkupfalse%
\ {\isachardoublequoteopen}F\ {\isasymin}\ S\ {\isasymlongrightarrow}\ G\ {\isasymin}\ S\ {\isasymand}\ H\ {\isasymin}\ S{\isachardoublequoteclose}\isanewline
\ \ \ \ \ \ \ \ \ \ \isacommand{using}\isamarkupfalse%
\ {\isacartoucheopen}F\ {\isacharequal}\ \isactrlbold {\isasymnot}\ {\isacharparenleft}G\ \isactrlbold {\isasymrightarrow}\ H{\isadigit{2}}{\isacharparenright}{\isacartoucheclose}\ {\isacartoucheopen}H\ {\isacharequal}\ \isactrlbold {\isasymnot}\ H{\isadigit{2}}{\isacartoucheclose}\ \isacommand{by}\isamarkupfalse%
\ {\isacharparenleft}iprover\ elim{\isacharcolon}\ allE{\isacharparenright}\isanewline
\ \ \ \ \ \ \isacommand{next}\isamarkupfalse%
\ \isanewline
\ \ \ \ \ \ \ \ \isacommand{assume}\isamarkupfalse%
\ {\isachardoublequoteopen}F\ {\isacharequal}\ \isactrlbold {\isasymnot}\ {\isacharparenleft}\isactrlbold {\isasymnot}\ G{\isacharparenright}\ {\isasymand}\ H\ {\isacharequal}\ G{\isachardoublequoteclose}\isanewline
\ \ \ \ \ \ \ \ \isacommand{then}\isamarkupfalse%
\ \isacommand{have}\isamarkupfalse%
\ {\isachardoublequoteopen}F\ {\isacharequal}\ \isactrlbold {\isasymnot}\ {\isacharparenleft}\isactrlbold {\isasymnot}\ G{\isacharparenright}{\isachardoublequoteclose}\isanewline
\ \ \ \ \ \ \ \ \ \ \isacommand{by}\isamarkupfalse%
\ {\isacharparenleft}rule\ conjunct{\isadigit{1}}{\isacharparenright}\isanewline
\ \ \ \ \ \ \ \ \isacommand{have}\isamarkupfalse%
\ {\isachardoublequoteopen}H\ {\isacharequal}\ G{\isachardoublequoteclose}\isanewline
\ \ \ \ \ \ \ \ \ \ \isacommand{using}\isamarkupfalse%
\ {\isacartoucheopen}F\ {\isacharequal}\ \isactrlbold {\isasymnot}\ {\isacharparenleft}\isactrlbold {\isasymnot}\ G{\isacharparenright}\ {\isasymand}\ H\ {\isacharequal}\ G{\isacartoucheclose}\ \isacommand{by}\isamarkupfalse%
\ {\isacharparenleft}rule\ conjunct{\isadigit{2}}{\isacharparenright}\isanewline
\ \ \ \ \ \ \ \ \isacommand{have}\isamarkupfalse%
\ {\isachardoublequoteopen}{\isasymforall}G{\isachardot}\ \isactrlbold {\isasymnot}\ {\isacharparenleft}\isactrlbold {\isasymnot}\ G{\isacharparenright}\ {\isasymin}\ S\ {\isasymlongrightarrow}\ G\ {\isasymin}\ S{\isachardoublequoteclose}\isanewline
\ \ \ \ \ \ \ \ \ \ \isacommand{using}\isamarkupfalse%
\ assms\ \isacommand{by}\isamarkupfalse%
\ {\isacharparenleft}iprover\ elim{\isacharcolon}\ conjunct{\isadigit{2}}\ conjunct{\isadigit{1}}{\isacharparenright}\isanewline
\ \ \ \ \ \ \ \ \isacommand{then}\isamarkupfalse%
\ \isacommand{have}\isamarkupfalse%
\ {\isachardoublequoteopen}\isactrlbold {\isasymnot}\ {\isacharparenleft}\isactrlbold {\isasymnot}\ G{\isacharparenright}\ {\isasymin}\ S\ {\isasymlongrightarrow}\ G\ {\isasymin}\ S{\isachardoublequoteclose}\isanewline
\ \ \ \ \ \ \ \ \ \ \isacommand{by}\isamarkupfalse%
\ {\isacharparenleft}rule\ allE{\isacharparenright}\isanewline
\ \ \ \ \ \ \ \ \isacommand{then}\isamarkupfalse%
\ \isacommand{have}\isamarkupfalse%
\ {\isachardoublequoteopen}F\ {\isasymin}\ S\ {\isasymlongrightarrow}\ G\ {\isasymin}\ S{\isachardoublequoteclose}\isanewline
\ \ \ \ \ \ \ \ \ \ \isacommand{by}\isamarkupfalse%
\ {\isacharparenleft}simp\ only{\isacharcolon}\ {\isacartoucheopen}F\ {\isacharequal}\ \isactrlbold {\isasymnot}\ {\isacharparenleft}\isactrlbold {\isasymnot}\ G{\isacharparenright}{\isacartoucheclose}{\isacharparenright}\ \isanewline
\ \ \ \ \ \ \ \ \isacommand{then}\isamarkupfalse%
\ \isacommand{have}\isamarkupfalse%
\ {\isachardoublequoteopen}F\ {\isasymin}\ S\ {\isasymlongrightarrow}\ G\ {\isasymin}\ S\ {\isasymand}\ G\ {\isasymin}\ S{\isachardoublequoteclose}\isanewline
\ \ \ \ \ \ \ \ \ \ \isacommand{by}\isamarkupfalse%
\ {\isacharparenleft}simp\ only{\isacharcolon}\ conj{\isacharunderscore}absorb{\isacharparenright}\isanewline
\ \ \ \ \ \ \ \ \isacommand{thus}\isamarkupfalse%
\ {\isachardoublequoteopen}F\ {\isasymin}\ S\ {\isasymlongrightarrow}\ G\ {\isasymin}\ S\ {\isasymand}\ H\ {\isasymin}\ S{\isachardoublequoteclose}\isanewline
\ \ \ \ \ \ \ \ \ \ \isacommand{by}\isamarkupfalse%
\ {\isacharparenleft}simp\ only{\isacharcolon}\ {\isacartoucheopen}H{\isacharequal}G{\isacartoucheclose}{\isacharparenright}\isanewline
\ \ \ \ \ \ \isacommand{qed}\isamarkupfalse%
\isanewline
\ \ \ \ \isacommand{qed}\isamarkupfalse%
\isanewline
\ \ \isacommand{qed}\isamarkupfalse%
\isanewline
\isacommand{qed}\isamarkupfalse%
%
\endisatagproof
{\isafoldproof}%
%
\isadelimproof
%
\endisadelimproof
%
\begin{isamarkuptext}%
Por otro lado, el segundo lema auxiliar prueba que la cuarta, quinta, sexta
  y séptima condición de la definición de conjunto de Hintikka son suficientes para
  probar que para toda fórmula de tipo \isa{{\isasymbeta}} con componentes \isa{{\isasymbeta}\isactrlsub {\isadigit{1}}} y \isa{{\isasymbeta}\isactrlsub {\isadigit{2}}} se verifica 
  que si la fórmula pertenece al conjunto \isa{S}, entonces o bien \isa{{\isasymbeta}\isactrlsub {\isadigit{1}}} pertenece al
  conjunto o bien \isa{{\isasymbeta}\isactrlsub {\isadigit{2}}} pertenece al conjunto. Veamos su prueba detallada en 
  Isabelle/HOL.%
\end{isamarkuptext}\isamarkuptrue%
\isacommand{lemma}\isamarkupfalse%
\ Hintikka{\isacharunderscore}alt{\isadigit{1}}Dis{\isacharcolon}\isanewline
\ \ \isakeyword{assumes}\ \ {\isachardoublequoteopen}{\isacharparenleft}{\isasymforall}\ G\ H{\isachardot}\ G\ \isactrlbold {\isasymor}\ H\ {\isasymin}\ S\ {\isasymlongrightarrow}\ G\ {\isasymin}\ S\ {\isasymor}\ H\ {\isasymin}\ S{\isacharparenright}\isanewline
\ \ {\isasymand}\ {\isacharparenleft}{\isasymforall}\ G\ H{\isachardot}\ G\ \isactrlbold {\isasymrightarrow}\ H\ {\isasymin}\ S\ {\isasymlongrightarrow}\ \isactrlbold {\isasymnot}\ G\ {\isasymin}\ S\ {\isasymor}\ H\ {\isasymin}\ S{\isacharparenright}\isanewline
\ \ {\isasymand}\ {\isacharparenleft}{\isasymforall}\ G{\isachardot}\ \isactrlbold {\isasymnot}\ {\isacharparenleft}\isactrlbold {\isasymnot}\ G{\isacharparenright}\ {\isasymin}\ S\ {\isasymlongrightarrow}\ G\ {\isasymin}\ S{\isacharparenright}\isanewline
\ \ {\isasymand}\ {\isacharparenleft}{\isasymforall}\ G\ H{\isachardot}\ \isactrlbold {\isasymnot}{\isacharparenleft}G\ \isactrlbold {\isasymand}\ H{\isacharparenright}\ {\isasymin}\ S\ {\isasymlongrightarrow}\ \isactrlbold {\isasymnot}\ G\ {\isasymin}\ S\ {\isasymor}\ \isactrlbold {\isasymnot}\ H\ {\isasymin}\ S{\isacharparenright}{\isachardoublequoteclose}\isanewline
\ \ \isakeyword{shows}\ {\isachardoublequoteopen}Dis\ F\ G\ H\ {\isasymlongrightarrow}\ F\ {\isasymin}\ S\ {\isasymlongrightarrow}\ G\ {\isasymin}\ S\ {\isasymor}\ H\ {\isasymin}\ S{\isachardoublequoteclose}\isanewline
%
\isadelimproof
%
\endisadelimproof
%
\isatagproof
\isacommand{proof}\isamarkupfalse%
\ {\isacharparenleft}rule\ impI{\isacharparenright}\isanewline
\ \ \isacommand{assume}\isamarkupfalse%
\ {\isachardoublequoteopen}Dis\ F\ G\ H{\isachardoublequoteclose}\isanewline
\ \ \isacommand{then}\isamarkupfalse%
\ \isacommand{have}\isamarkupfalse%
\ {\isachardoublequoteopen}F\ {\isacharequal}\ G\ \isactrlbold {\isasymor}\ H\ {\isasymor}\ \isanewline
\ \ \ \ {\isacharparenleft}{\isasymexists}G{\isadigit{1}}\ H{\isadigit{1}}{\isachardot}\ F\ {\isacharequal}\ G{\isadigit{1}}\ \isactrlbold {\isasymrightarrow}\ H{\isadigit{1}}\ {\isasymand}\ G\ {\isacharequal}\ \isactrlbold {\isasymnot}\ G{\isadigit{1}}\ {\isasymand}\ H\ {\isacharequal}\ H{\isadigit{1}}{\isacharparenright}\ {\isasymor}\ \isanewline
\ \ \ \ {\isacharparenleft}{\isasymexists}G{\isadigit{2}}\ H{\isadigit{2}}{\isachardot}\ F\ {\isacharequal}\ \isactrlbold {\isasymnot}\ {\isacharparenleft}G{\isadigit{2}}\ \isactrlbold {\isasymand}\ H{\isadigit{2}}{\isacharparenright}\ {\isasymand}\ G\ {\isacharequal}\ \isactrlbold {\isasymnot}\ G{\isadigit{2}}\ {\isasymand}\ H\ {\isacharequal}\ \isactrlbold {\isasymnot}\ H{\isadigit{2}}{\isacharparenright}\ {\isasymor}\ \isanewline
\ \ \ \ F\ {\isacharequal}\ \isactrlbold {\isasymnot}\ {\isacharparenleft}\isactrlbold {\isasymnot}\ G{\isacharparenright}\ {\isasymand}\ H\ {\isacharequal}\ G{\isachardoublequoteclose}\ \isanewline
\ \ \ \ \isacommand{by}\isamarkupfalse%
\ {\isacharparenleft}simp\ only{\isacharcolon}\ con{\isacharunderscore}dis{\isacharunderscore}simps{\isacharparenleft}{\isadigit{2}}{\isacharparenright}{\isacharparenright}\isanewline
\ \ \isacommand{thus}\isamarkupfalse%
\ {\isachardoublequoteopen}F\ {\isasymin}\ S\ {\isasymlongrightarrow}\ G\ {\isasymin}\ S\ {\isasymor}\ H\ {\isasymin}\ S{\isachardoublequoteclose}\ \isanewline
\ \ \isacommand{proof}\isamarkupfalse%
\ {\isacharparenleft}rule\ disjE{\isacharparenright}\isanewline
\ \ \ \ \isacommand{assume}\isamarkupfalse%
\ {\isachardoublequoteopen}F\ {\isacharequal}\ G\ \isactrlbold {\isasymor}\ H{\isachardoublequoteclose}\isanewline
\ \ \ \ \isacommand{have}\isamarkupfalse%
\ {\isachardoublequoteopen}{\isasymforall}G\ H{\isachardot}\ G\ \isactrlbold {\isasymor}\ H\ {\isasymin}\ S\ {\isasymlongrightarrow}\ G\ {\isasymin}\ S\ {\isasymor}\ H\ {\isasymin}\ S{\isachardoublequoteclose}\isanewline
\ \ \ \ \ \ \isacommand{using}\isamarkupfalse%
\ assms\ \isacommand{by}\isamarkupfalse%
\ {\isacharparenleft}rule\ conjunct{\isadigit{1}}{\isacharparenright}\isanewline
\ \ \ \ \isacommand{thus}\isamarkupfalse%
\ {\isachardoublequoteopen}F\ {\isasymin}\ S\ {\isasymlongrightarrow}\ G\ {\isasymin}\ S\ {\isasymor}\ H\ {\isasymin}\ S{\isachardoublequoteclose}\ \isanewline
\ \ \ \ \ \ \isacommand{using}\isamarkupfalse%
\ {\isacartoucheopen}F\ {\isacharequal}\ G\ \isactrlbold {\isasymor}\ H{\isacartoucheclose}\ \isacommand{by}\isamarkupfalse%
\ {\isacharparenleft}iprover\ elim{\isacharcolon}\ allE{\isacharparenright}\isanewline
\ \ \isacommand{next}\isamarkupfalse%
\isanewline
\ \ \ \ \isacommand{assume}\isamarkupfalse%
\ {\isachardoublequoteopen}{\isacharparenleft}{\isasymexists}G{\isadigit{1}}\ H{\isadigit{1}}{\isachardot}\ F\ {\isacharequal}\ G{\isadigit{1}}\ \isactrlbold {\isasymrightarrow}\ H{\isadigit{1}}\ {\isasymand}\ G\ {\isacharequal}\ \isactrlbold {\isasymnot}\ G{\isadigit{1}}\ {\isasymand}\ H\ {\isacharequal}\ H{\isadigit{1}}{\isacharparenright}\ {\isasymor}\ \isanewline
\ \ \ \ {\isacharparenleft}{\isasymexists}G{\isadigit{2}}\ H{\isadigit{2}}{\isachardot}\ F\ {\isacharequal}\ \isactrlbold {\isasymnot}\ {\isacharparenleft}G{\isadigit{2}}\ \isactrlbold {\isasymand}\ H{\isadigit{2}}{\isacharparenright}\ {\isasymand}\ G\ {\isacharequal}\ \isactrlbold {\isasymnot}\ G{\isadigit{2}}\ {\isasymand}\ H\ {\isacharequal}\ \isactrlbold {\isasymnot}\ H{\isadigit{2}}{\isacharparenright}\ {\isasymor}\ \isanewline
\ \ \ \ F\ {\isacharequal}\ \isactrlbold {\isasymnot}\ {\isacharparenleft}\isactrlbold {\isasymnot}\ G{\isacharparenright}\ {\isasymand}\ H\ {\isacharequal}\ G{\isachardoublequoteclose}\isanewline
\ \ \ \ \isacommand{thus}\isamarkupfalse%
\ {\isachardoublequoteopen}F\ {\isasymin}\ S\ {\isasymlongrightarrow}\ G\ {\isasymin}\ S\ {\isasymor}\ H\ {\isasymin}\ S{\isachardoublequoteclose}\isanewline
\ \ \ \ \isacommand{proof}\isamarkupfalse%
\ {\isacharparenleft}rule\ disjE{\isacharparenright}\isanewline
\ \ \ \ \ \ \isacommand{assume}\isamarkupfalse%
\ E{\isadigit{1}}{\isacharcolon}{\isachardoublequoteopen}{\isasymexists}G{\isadigit{1}}\ H{\isadigit{1}}{\isachardot}\ F\ {\isacharequal}\ G{\isadigit{1}}\ \isactrlbold {\isasymrightarrow}\ H{\isadigit{1}}\ {\isasymand}\ G\ {\isacharequal}\ \isactrlbold {\isasymnot}\ G{\isadigit{1}}\ {\isasymand}\ H\ {\isacharequal}\ H{\isadigit{1}}{\isachardoublequoteclose}\isanewline
\ \ \ \ \ \ \isacommand{obtain}\isamarkupfalse%
\ G{\isadigit{1}}\ H{\isadigit{1}}\ \isakeyword{where}\ A{\isadigit{1}}{\isacharcolon}{\isachardoublequoteopen}F\ {\isacharequal}\ G{\isadigit{1}}\ \isactrlbold {\isasymrightarrow}\ H{\isadigit{1}}\ {\isasymand}\ G\ {\isacharequal}\ \isactrlbold {\isasymnot}\ G{\isadigit{1}}\ {\isasymand}\ H\ {\isacharequal}\ H{\isadigit{1}}{\isachardoublequoteclose}\isanewline
\ \ \ \ \ \ \ \ \isacommand{using}\isamarkupfalse%
\ E{\isadigit{1}}\ \isacommand{by}\isamarkupfalse%
\ {\isacharparenleft}iprover\ elim{\isacharcolon}\ exE{\isacharparenright}\isanewline
\ \ \ \ \ \ \isacommand{have}\isamarkupfalse%
\ {\isachardoublequoteopen}F\ {\isacharequal}\ G{\isadigit{1}}\ \isactrlbold {\isasymrightarrow}\ H{\isadigit{1}}{\isachardoublequoteclose}\isanewline
\ \ \ \ \ \ \ \ \isacommand{using}\isamarkupfalse%
\ A{\isadigit{1}}\ \isacommand{by}\isamarkupfalse%
\ {\isacharparenleft}rule\ conjunct{\isadigit{1}}{\isacharparenright}\isanewline
\ \ \ \ \ \ \isacommand{have}\isamarkupfalse%
\ {\isachardoublequoteopen}G\ {\isacharequal}\ \isactrlbold {\isasymnot}\ G{\isadigit{1}}{\isachardoublequoteclose}\isanewline
\ \ \ \ \ \ \ \ \isacommand{using}\isamarkupfalse%
\ A{\isadigit{1}}\ \isacommand{by}\isamarkupfalse%
\ {\isacharparenleft}iprover\ elim{\isacharcolon}\ conjunct{\isadigit{1}}{\isacharparenright}\isanewline
\ \ \ \ \ \ \isacommand{have}\isamarkupfalse%
\ {\isachardoublequoteopen}H\ {\isacharequal}\ H{\isadigit{1}}{\isachardoublequoteclose}\isanewline
\ \ \ \ \ \ \ \ \isacommand{using}\isamarkupfalse%
\ A{\isadigit{1}}\ \isacommand{by}\isamarkupfalse%
\ {\isacharparenleft}iprover\ elim{\isacharcolon}\ conjunct{\isadigit{2}}\ conjunct{\isadigit{1}}{\isacharparenright}\isanewline
\ \ \ \ \ \ \isacommand{have}\isamarkupfalse%
\ {\isachardoublequoteopen}{\isasymforall}G\ H{\isachardot}\ G\ \isactrlbold {\isasymrightarrow}\ H\ {\isasymin}\ S\ {\isasymlongrightarrow}\ \isactrlbold {\isasymnot}\ G\ {\isasymin}\ S\ {\isasymor}\ H\ {\isasymin}\ S{\isachardoublequoteclose}\isanewline
\ \ \ \ \ \ \ \ \isacommand{using}\isamarkupfalse%
\ assms\ \isacommand{by}\isamarkupfalse%
\ {\isacharparenleft}iprover\ elim{\isacharcolon}\ conjunct{\isadigit{2}}\ conjunct{\isadigit{1}}{\isacharparenright}\isanewline
\ \ \ \ \ \ \isacommand{thus}\isamarkupfalse%
\ {\isachardoublequoteopen}F\ {\isasymin}\ S\ {\isasymlongrightarrow}\ G\ {\isasymin}\ S\ {\isasymor}\ H\ {\isasymin}\ S{\isachardoublequoteclose}\isanewline
\ \ \ \ \ \ \ \ \isacommand{using}\isamarkupfalse%
\ {\isacartoucheopen}F\ {\isacharequal}\ G{\isadigit{1}}\ \isactrlbold {\isasymrightarrow}\ H{\isadigit{1}}{\isacartoucheclose}\ {\isacartoucheopen}G\ {\isacharequal}\ \isactrlbold {\isasymnot}\ G{\isadigit{1}}{\isacartoucheclose}\ {\isacartoucheopen}H\ {\isacharequal}\ H{\isadigit{1}}{\isacartoucheclose}\ \isacommand{by}\isamarkupfalse%
\ {\isacharparenleft}iprover\ elim{\isacharcolon}\ allE{\isacharparenright}\isanewline
\ \ \ \ \isacommand{next}\isamarkupfalse%
\isanewline
\ \ \ \ \ \ \isacommand{assume}\isamarkupfalse%
\ {\isachardoublequoteopen}{\isacharparenleft}{\isasymexists}G{\isadigit{2}}\ H{\isadigit{2}}{\isachardot}\ F\ {\isacharequal}\ \isactrlbold {\isasymnot}\ {\isacharparenleft}G{\isadigit{2}}\ \isactrlbold {\isasymand}\ H{\isadigit{2}}{\isacharparenright}\ {\isasymand}\ G\ {\isacharequal}\ \isactrlbold {\isasymnot}\ G{\isadigit{2}}\ {\isasymand}\ H\ {\isacharequal}\ \isactrlbold {\isasymnot}\ H{\isadigit{2}}{\isacharparenright}\ {\isasymor}\ \isanewline
\ \ \ \ \ \ F\ {\isacharequal}\ \isactrlbold {\isasymnot}\ {\isacharparenleft}\isactrlbold {\isasymnot}\ G{\isacharparenright}\ {\isasymand}\ H\ {\isacharequal}\ G{\isachardoublequoteclose}\isanewline
\ \ \ \ \ \ \isacommand{thus}\isamarkupfalse%
\ {\isachardoublequoteopen}F\ {\isasymin}\ S\ {\isasymlongrightarrow}\ G\ {\isasymin}\ S\ {\isasymor}\ H\ {\isasymin}\ S{\isachardoublequoteclose}\isanewline
\ \ \ \ \ \ \isacommand{proof}\isamarkupfalse%
\ {\isacharparenleft}rule\ disjE{\isacharparenright}\isanewline
\ \ \ \ \ \ \ \ \isacommand{assume}\isamarkupfalse%
\ E{\isadigit{2}}{\isacharcolon}{\isachardoublequoteopen}{\isasymexists}G{\isadigit{2}}\ H{\isadigit{2}}{\isachardot}\ F\ {\isacharequal}\ \isactrlbold {\isasymnot}\ {\isacharparenleft}G{\isadigit{2}}\ \isactrlbold {\isasymand}\ H{\isadigit{2}}{\isacharparenright}\ {\isasymand}\ G\ {\isacharequal}\ \isactrlbold {\isasymnot}\ G{\isadigit{2}}\ {\isasymand}\ H\ {\isacharequal}\ \isactrlbold {\isasymnot}\ H{\isadigit{2}}{\isachardoublequoteclose}\isanewline
\ \ \ \ \ \ \ \ \isacommand{obtain}\isamarkupfalse%
\ G{\isadigit{2}}\ H{\isadigit{2}}\ \isakeyword{where}\ A{\isadigit{2}}{\isacharcolon}{\isachardoublequoteopen}F\ {\isacharequal}\ \isactrlbold {\isasymnot}\ {\isacharparenleft}G{\isadigit{2}}\ \isactrlbold {\isasymand}\ H{\isadigit{2}}{\isacharparenright}\ {\isasymand}\ G\ {\isacharequal}\ \isactrlbold {\isasymnot}\ G{\isadigit{2}}\ {\isasymand}\ H\ {\isacharequal}\ \isactrlbold {\isasymnot}\ H{\isadigit{2}}{\isachardoublequoteclose}\ \isanewline
\ \ \ \ \ \ \ \ \ \ \isacommand{using}\isamarkupfalse%
\ E{\isadigit{2}}\ \isacommand{by}\isamarkupfalse%
\ {\isacharparenleft}iprover\ elim{\isacharcolon}\ exE{\isacharparenright}\isanewline
\ \ \ \ \ \ \ \ \isacommand{have}\isamarkupfalse%
\ {\isachardoublequoteopen}F\ {\isacharequal}\ \isactrlbold {\isasymnot}\ {\isacharparenleft}G{\isadigit{2}}\ \isactrlbold {\isasymand}\ H{\isadigit{2}}{\isacharparenright}{\isachardoublequoteclose}\ \isanewline
\ \ \ \ \ \ \ \ \ \ \isacommand{using}\isamarkupfalse%
\ A{\isadigit{2}}\ \isacommand{by}\isamarkupfalse%
\ {\isacharparenleft}rule\ conjunct{\isadigit{1}}{\isacharparenright}\isanewline
\ \ \ \ \ \ \ \ \isacommand{have}\isamarkupfalse%
\ {\isachardoublequoteopen}G\ {\isacharequal}\ \isactrlbold {\isasymnot}\ G{\isadigit{2}}{\isachardoublequoteclose}\isanewline
\ \ \ \ \ \ \ \ \ \ \isacommand{using}\isamarkupfalse%
\ A{\isadigit{2}}\ \isacommand{by}\isamarkupfalse%
\ {\isacharparenleft}iprover\ elim{\isacharcolon}\ conjunct{\isadigit{2}}\ conjunct{\isadigit{1}}{\isacharparenright}\isanewline
\ \ \ \ \ \ \ \ \isacommand{have}\isamarkupfalse%
\ {\isachardoublequoteopen}H\ {\isacharequal}\ \isactrlbold {\isasymnot}\ H{\isadigit{2}}{\isachardoublequoteclose}\isanewline
\ \ \ \ \ \ \ \ \ \ \isacommand{using}\isamarkupfalse%
\ A{\isadigit{2}}\ \isacommand{by}\isamarkupfalse%
\ {\isacharparenleft}iprover\ elim{\isacharcolon}\ conjunct{\isadigit{1}}{\isacharparenright}\isanewline
\ \ \ \ \ \ \ \ \isacommand{have}\isamarkupfalse%
\ {\isachardoublequoteopen}{\isasymforall}\ G\ H{\isachardot}\ \isactrlbold {\isasymnot}{\isacharparenleft}G\ \isactrlbold {\isasymand}\ H{\isacharparenright}\ {\isasymin}\ S\ {\isasymlongrightarrow}\ \isactrlbold {\isasymnot}\ G\ {\isasymin}\ S\ {\isasymor}\ \isactrlbold {\isasymnot}\ H\ {\isasymin}\ S{\isachardoublequoteclose}\isanewline
\ \ \ \ \ \ \ \ \ \ \isacommand{using}\isamarkupfalse%
\ assms\ \isacommand{by}\isamarkupfalse%
\ {\isacharparenleft}iprover\ elim{\isacharcolon}\ conjunct{\isadigit{2}}\ conjunct{\isadigit{1}}{\isacharparenright}\isanewline
\ \ \ \ \ \ \ \ \isacommand{thus}\isamarkupfalse%
\ {\isachardoublequoteopen}F\ {\isasymin}\ S\ {\isasymlongrightarrow}\ G\ {\isasymin}\ S\ {\isasymor}\ H\ {\isasymin}\ S{\isachardoublequoteclose}\isanewline
\ \ \ \ \ \ \ \ \ \ \isacommand{using}\isamarkupfalse%
\ {\isacartoucheopen}F\ {\isacharequal}\ \isactrlbold {\isasymnot}{\isacharparenleft}G{\isadigit{2}}\ \isactrlbold {\isasymand}\ H{\isadigit{2}}{\isacharparenright}{\isacartoucheclose}\ {\isacartoucheopen}G\ {\isacharequal}\ \isactrlbold {\isasymnot}\ G{\isadigit{2}}{\isacartoucheclose}\ {\isacartoucheopen}H\ {\isacharequal}\ \isactrlbold {\isasymnot}\ H{\isadigit{2}}{\isacartoucheclose}\ \isacommand{by}\isamarkupfalse%
\ {\isacharparenleft}iprover\ elim{\isacharcolon}\ allE{\isacharparenright}\isanewline
\ \ \ \ \ \ \isacommand{next}\isamarkupfalse%
\isanewline
\ \ \ \ \ \ \ \ \isacommand{assume}\isamarkupfalse%
\ {\isachardoublequoteopen}F\ {\isacharequal}\ \isactrlbold {\isasymnot}\ {\isacharparenleft}\isactrlbold {\isasymnot}\ G{\isacharparenright}\ {\isasymand}\ H\ {\isacharequal}\ G{\isachardoublequoteclose}\isanewline
\ \ \ \ \ \ \ \ \isacommand{then}\isamarkupfalse%
\ \isacommand{have}\isamarkupfalse%
\ {\isachardoublequoteopen}F\ {\isacharequal}\ \isactrlbold {\isasymnot}\ {\isacharparenleft}\isactrlbold {\isasymnot}\ G{\isacharparenright}{\isachardoublequoteclose}\ \isanewline
\ \ \ \ \ \ \ \ \ \ \isacommand{by}\isamarkupfalse%
\ {\isacharparenleft}rule\ conjunct{\isadigit{1}}{\isacharparenright}\isanewline
\ \ \ \ \ \ \ \ \isacommand{have}\isamarkupfalse%
\ {\isachardoublequoteopen}H\ {\isacharequal}\ G{\isachardoublequoteclose}\isanewline
\ \ \ \ \ \ \ \ \ \ \isacommand{using}\isamarkupfalse%
\ {\isacartoucheopen}F\ {\isacharequal}\ \isactrlbold {\isasymnot}\ {\isacharparenleft}\isactrlbold {\isasymnot}\ G{\isacharparenright}\ {\isasymand}\ H\ {\isacharequal}\ G{\isacartoucheclose}\ \isacommand{by}\isamarkupfalse%
\ {\isacharparenleft}rule\ conjunct{\isadigit{2}}{\isacharparenright}\isanewline
\ \ \ \ \ \ \ \ \isacommand{have}\isamarkupfalse%
\ {\isachardoublequoteopen}{\isasymforall}\ G{\isachardot}\ \isactrlbold {\isasymnot}\ {\isacharparenleft}\isactrlbold {\isasymnot}\ G{\isacharparenright}\ {\isasymin}\ S\ {\isasymlongrightarrow}\ G\ {\isasymin}\ S{\isachardoublequoteclose}\isanewline
\ \ \ \ \ \ \ \ \ \ \isacommand{using}\isamarkupfalse%
\ assms\ \isacommand{by}\isamarkupfalse%
\ {\isacharparenleft}iprover\ elim{\isacharcolon}\ conjunct{\isadigit{2}}\ conjunct{\isadigit{1}}{\isacharparenright}\isanewline
\ \ \ \ \ \ \ \ \isacommand{then}\isamarkupfalse%
\ \isacommand{have}\isamarkupfalse%
\ {\isachardoublequoteopen}\isactrlbold {\isasymnot}\ {\isacharparenleft}\isactrlbold {\isasymnot}\ G{\isacharparenright}\ {\isasymin}\ S\ {\isasymlongrightarrow}\ G\ {\isasymin}\ S{\isachardoublequoteclose}\isanewline
\ \ \ \ \ \ \ \ \ \ \isacommand{by}\isamarkupfalse%
\ {\isacharparenleft}rule\ allE{\isacharparenright}\isanewline
\ \ \ \ \ \ \ \ \isacommand{then}\isamarkupfalse%
\ \isacommand{have}\isamarkupfalse%
\ {\isachardoublequoteopen}F\ {\isasymin}\ S\ {\isasymlongrightarrow}\ G\ {\isasymin}\ S{\isachardoublequoteclose}\isanewline
\ \ \ \ \ \ \ \ \ \ \isacommand{by}\isamarkupfalse%
\ {\isacharparenleft}simp\ only{\isacharcolon}\ {\isacartoucheopen}F\ {\isacharequal}\ \isactrlbold {\isasymnot}\ {\isacharparenleft}\isactrlbold {\isasymnot}\ G{\isacharparenright}{\isacartoucheclose}{\isacharparenright}\isanewline
\ \ \ \ \ \ \ \ \isacommand{then}\isamarkupfalse%
\ \isacommand{have}\isamarkupfalse%
\ {\isachardoublequoteopen}F\ {\isasymin}\ S\ {\isasymlongrightarrow}\ G\ {\isasymin}\ S\ {\isasymor}\ G\ {\isasymin}\ S{\isachardoublequoteclose}\isanewline
\ \ \ \ \ \ \ \ \ \ \isacommand{by}\isamarkupfalse%
\ {\isacharparenleft}simp\ only{\isacharcolon}\ disj{\isacharunderscore}absorb{\isacharparenright}\isanewline
\ \ \ \ \ \ \ \ \isacommand{thus}\isamarkupfalse%
\ {\isachardoublequoteopen}F\ {\isasymin}\ S\ {\isasymlongrightarrow}\ G\ {\isasymin}\ S\ {\isasymor}\ H\ {\isasymin}\ S{\isachardoublequoteclose}\isanewline
\ \ \ \ \ \ \ \ \isacommand{by}\isamarkupfalse%
\ {\isacharparenleft}simp\ only{\isacharcolon}\ {\isacartoucheopen}H\ {\isacharequal}\ G{\isacartoucheclose}{\isacharparenright}\isanewline
\ \ \ \ \ \ \isacommand{qed}\isamarkupfalse%
\isanewline
\ \ \ \ \isacommand{qed}\isamarkupfalse%
\isanewline
\ \ \isacommand{qed}\isamarkupfalse%
\isanewline
\isacommand{qed}\isamarkupfalse%
%
\endisatagproof
{\isafoldproof}%
%
\isadelimproof
%
\endisadelimproof
%
\begin{isamarkuptext}%
Finalmente, podemos demostrar detalladamente esta primera implicación de la
  equivalencia del lema en Isabelle.%
\end{isamarkuptext}\isamarkuptrue%
\isacommand{lemma}\isamarkupfalse%
\ Hintikka{\isacharunderscore}alt{\isadigit{1}}{\isacharcolon}\isanewline
\ \ \isakeyword{assumes}\ {\isachardoublequoteopen}Hintikka\ S{\isachardoublequoteclose}\isanewline
\ \ \isakeyword{shows}\ {\isachardoublequoteopen}{\isasymbottom}\ {\isasymnotin}\ S\isanewline
{\isasymand}\ {\isacharparenleft}{\isasymforall}k{\isachardot}\ Atom\ k\ {\isasymin}\ S\ {\isasymlongrightarrow}\ \isactrlbold {\isasymnot}\ {\isacharparenleft}Atom\ k{\isacharparenright}\ {\isasymin}\ S\ {\isasymlongrightarrow}\ False{\isacharparenright}\isanewline
{\isasymand}\ {\isacharparenleft}{\isasymforall}F\ G\ H{\isachardot}\ Con\ F\ G\ H\ {\isasymlongrightarrow}\ F\ {\isasymin}\ S\ {\isasymlongrightarrow}\ G\ {\isasymin}\ S\ {\isasymand}\ H\ {\isasymin}\ S{\isacharparenright}\isanewline
{\isasymand}\ {\isacharparenleft}{\isasymforall}F\ G\ H{\isachardot}\ Dis\ F\ G\ H\ {\isasymlongrightarrow}\ F\ {\isasymin}\ S\ {\isasymlongrightarrow}\ G\ {\isasymin}\ S\ {\isasymor}\ H\ {\isasymin}\ S{\isacharparenright}{\isachardoublequoteclose}\isanewline
%
\isadelimproof
%
\endisadelimproof
%
\isatagproof
\isacommand{proof}\isamarkupfalse%
\ {\isacharminus}\isanewline
\ \ \isacommand{have}\isamarkupfalse%
\ Hk{\isacharcolon}{\isachardoublequoteopen}{\isacharparenleft}{\isasymbottom}\ {\isasymnotin}\ S\isanewline
\ \ {\isasymand}\ {\isacharparenleft}{\isasymforall}k{\isachardot}\ Atom\ k\ {\isasymin}\ S\ {\isasymlongrightarrow}\ \isactrlbold {\isasymnot}\ {\isacharparenleft}Atom\ k{\isacharparenright}\ {\isasymin}\ S\ {\isasymlongrightarrow}\ False{\isacharparenright}\isanewline
\ \ {\isasymand}\ {\isacharparenleft}{\isasymforall}G\ H{\isachardot}\ G\ \isactrlbold {\isasymand}\ H\ {\isasymin}\ S\ {\isasymlongrightarrow}\ G\ {\isasymin}\ S\ {\isasymand}\ H\ {\isasymin}\ S{\isacharparenright}\isanewline
\ \ {\isasymand}\ {\isacharparenleft}{\isasymforall}G\ H{\isachardot}\ G\ \isactrlbold {\isasymor}\ H\ {\isasymin}\ S\ {\isasymlongrightarrow}\ G\ {\isasymin}\ S\ {\isasymor}\ H\ {\isasymin}\ S{\isacharparenright}\isanewline
\ \ {\isasymand}\ {\isacharparenleft}{\isasymforall}G\ H{\isachardot}\ G\ \isactrlbold {\isasymrightarrow}\ H\ {\isasymin}\ S\ {\isasymlongrightarrow}\ \isactrlbold {\isasymnot}G\ {\isasymin}\ S\ {\isasymor}\ H\ {\isasymin}\ S{\isacharparenright}\isanewline
\ \ {\isasymand}\ {\isacharparenleft}{\isasymforall}G{\isachardot}\ \isactrlbold {\isasymnot}\ {\isacharparenleft}\isactrlbold {\isasymnot}G{\isacharparenright}\ {\isasymin}\ S\ {\isasymlongrightarrow}\ G\ {\isasymin}\ S{\isacharparenright}\isanewline
\ \ {\isasymand}\ {\isacharparenleft}{\isasymforall}G\ H{\isachardot}\ \isactrlbold {\isasymnot}{\isacharparenleft}G\ \isactrlbold {\isasymand}\ H{\isacharparenright}\ {\isasymin}\ S\ {\isasymlongrightarrow}\ \isactrlbold {\isasymnot}\ G\ {\isasymin}\ S\ {\isasymor}\ \isactrlbold {\isasymnot}\ H\ {\isasymin}\ S{\isacharparenright}\isanewline
\ \ {\isasymand}\ {\isacharparenleft}{\isasymforall}G\ H{\isachardot}\ \isactrlbold {\isasymnot}{\isacharparenleft}G\ \isactrlbold {\isasymor}\ H{\isacharparenright}\ {\isasymin}\ S\ {\isasymlongrightarrow}\ \isactrlbold {\isasymnot}\ G\ {\isasymin}\ S\ {\isasymand}\ \isactrlbold {\isasymnot}\ H\ {\isasymin}\ S{\isacharparenright}\isanewline
\ \ {\isasymand}\ {\isacharparenleft}{\isasymforall}G\ H{\isachardot}\ \isactrlbold {\isasymnot}{\isacharparenleft}G\ \isactrlbold {\isasymrightarrow}\ H{\isacharparenright}\ {\isasymin}\ S\ {\isasymlongrightarrow}\ G\ {\isasymin}\ S\ {\isasymand}\ \isactrlbold {\isasymnot}\ H\ {\isasymin}\ S{\isacharparenright}{\isacharparenright}{\isachardoublequoteclose}\isanewline
\ \ \ \ \isacommand{using}\isamarkupfalse%
\ assms\ \isacommand{by}\isamarkupfalse%
\ {\isacharparenleft}rule\ auxEq{\isacharparenright}\isanewline
\ \ \isacommand{then}\isamarkupfalse%
\ \isacommand{have}\isamarkupfalse%
\ C{\isadigit{1}}{\isacharcolon}\ {\isachardoublequoteopen}{\isasymbottom}\ {\isasymnotin}\ S{\isachardoublequoteclose}\isanewline
\ \ \ \ \isacommand{by}\isamarkupfalse%
\ {\isacharparenleft}rule\ conjunct{\isadigit{1}}{\isacharparenright}\isanewline
\ \ \isacommand{have}\isamarkupfalse%
\ C{\isadigit{2}}{\isacharcolon}\ {\isachardoublequoteopen}{\isasymforall}k{\isachardot}\ Atom\ k\ {\isasymin}\ S\ {\isasymlongrightarrow}\ \isactrlbold {\isasymnot}\ {\isacharparenleft}Atom\ k{\isacharparenright}\ {\isasymin}\ S\ {\isasymlongrightarrow}\ False{\isachardoublequoteclose}\isanewline
\ \ \ \ \isacommand{using}\isamarkupfalse%
\ Hk\ \isacommand{by}\isamarkupfalse%
\ {\isacharparenleft}iprover\ elim{\isacharcolon}\ conjunct{\isadigit{2}}\ conjunct{\isadigit{1}}{\isacharparenright}\isanewline
\ \ \isacommand{have}\isamarkupfalse%
\ C{\isadigit{3}}{\isacharcolon}\ {\isachardoublequoteopen}{\isasymforall}F\ G\ H{\isachardot}\ Con\ F\ G\ H\ {\isasymlongrightarrow}\ F\ {\isasymin}\ S\ {\isasymlongrightarrow}\ G\ {\isasymin}\ S\ {\isasymand}\ H\ {\isasymin}\ S{\isachardoublequoteclose}\isanewline
\ \ \isacommand{proof}\isamarkupfalse%
\ {\isacharparenleft}rule\ allI{\isacharparenright}{\isacharplus}\isanewline
\ \ \ \ \isacommand{fix}\isamarkupfalse%
\ F\ G\ H\isanewline
\ \ \ \ \isacommand{have}\isamarkupfalse%
\ C{\isadigit{3}}{\isadigit{1}}{\isacharcolon}{\isachardoublequoteopen}{\isasymforall}G\ H{\isachardot}\ G\ \isactrlbold {\isasymand}\ H\ {\isasymin}\ S\ {\isasymlongrightarrow}\ G\ {\isasymin}\ S\ {\isasymand}\ H\ {\isasymin}\ S{\isachardoublequoteclose}\isanewline
\ \ \ \ \ \ \isacommand{using}\isamarkupfalse%
\ Hk\ \isacommand{by}\isamarkupfalse%
\ {\isacharparenleft}iprover\ elim{\isacharcolon}\ conjunct{\isadigit{2}}\ conjunct{\isadigit{1}}{\isacharparenright}\isanewline
\ \ \ \ \isacommand{have}\isamarkupfalse%
\ C{\isadigit{3}}{\isadigit{2}}{\isacharcolon}{\isachardoublequoteopen}{\isasymforall}G{\isachardot}\ \isactrlbold {\isasymnot}\ {\isacharparenleft}\isactrlbold {\isasymnot}\ G{\isacharparenright}\ {\isasymin}\ S\ {\isasymlongrightarrow}\ G\ {\isasymin}\ S{\isachardoublequoteclose}\isanewline
\ \ \ \ \ \ \isacommand{using}\isamarkupfalse%
\ Hk\ \isacommand{by}\isamarkupfalse%
\ {\isacharparenleft}iprover\ elim{\isacharcolon}\ conjunct{\isadigit{2}}\ conjunct{\isadigit{1}}{\isacharparenright}\isanewline
\ \ \ \ \isacommand{have}\isamarkupfalse%
\ C{\isadigit{3}}{\isadigit{3}}{\isacharcolon}{\isachardoublequoteopen}{\isasymforall}G\ H{\isachardot}\ \isactrlbold {\isasymnot}{\isacharparenleft}G\ \isactrlbold {\isasymor}\ H{\isacharparenright}\ {\isasymin}\ S\ {\isasymlongrightarrow}\ \isactrlbold {\isasymnot}\ G\ {\isasymin}\ S\ {\isasymand}\ \isactrlbold {\isasymnot}\ H\ {\isasymin}\ S{\isachardoublequoteclose}\isanewline
\ \ \ \ \ \ \isacommand{using}\isamarkupfalse%
\ Hk\ \isacommand{by}\isamarkupfalse%
\ {\isacharparenleft}iprover\ elim{\isacharcolon}\ conjunct{\isadigit{2}}\ conjunct{\isadigit{1}}{\isacharparenright}\isanewline
\ \ \ \ \isacommand{have}\isamarkupfalse%
\ C{\isadigit{3}}{\isadigit{4}}{\isacharcolon}{\isachardoublequoteopen}{\isasymforall}G\ H{\isachardot}\ \isactrlbold {\isasymnot}{\isacharparenleft}G\ \isactrlbold {\isasymrightarrow}\ H{\isacharparenright}\ {\isasymin}\ S\ {\isasymlongrightarrow}\ G\ {\isasymin}\ S\ {\isasymand}\ \isactrlbold {\isasymnot}\ H\ {\isasymin}\ S{\isachardoublequoteclose}\isanewline
\ \ \ \ \ \ \isacommand{using}\isamarkupfalse%
\ Hk\ \isacommand{by}\isamarkupfalse%
\ {\isacharparenleft}iprover\ elim{\isacharcolon}\ conjunct{\isadigit{2}}\ conjunct{\isadigit{1}}{\isacharparenright}\isanewline
\ \ \ \ \isacommand{have}\isamarkupfalse%
\ {\isachardoublequoteopen}{\isacharparenleft}{\isasymforall}G\ H{\isachardot}\ G\ \isactrlbold {\isasymand}\ H\ {\isasymin}\ S\ {\isasymlongrightarrow}\ G\ {\isasymin}\ S\ {\isasymand}\ H\ {\isasymin}\ S{\isacharparenright}\isanewline
\ \ \ \ \ \ \ \ \ \ {\isasymand}\ {\isacharparenleft}{\isasymforall}G{\isachardot}\ \isactrlbold {\isasymnot}\ {\isacharparenleft}\isactrlbold {\isasymnot}\ G{\isacharparenright}\ {\isasymin}\ S\ {\isasymlongrightarrow}\ G\ {\isasymin}\ S{\isacharparenright}\isanewline
\ \ \ \ \ \ \ \ \ \ {\isasymand}\ {\isacharparenleft}{\isasymforall}G\ H{\isachardot}\ \isactrlbold {\isasymnot}{\isacharparenleft}G\ \isactrlbold {\isasymor}\ H{\isacharparenright}\ {\isasymin}\ S\ {\isasymlongrightarrow}\ \isactrlbold {\isasymnot}\ G\ {\isasymin}\ S\ {\isasymand}\ \isactrlbold {\isasymnot}\ H\ {\isasymin}\ S{\isacharparenright}\isanewline
\ \ \ \ \ \ \ \ \ \ {\isasymand}\ {\isacharparenleft}{\isasymforall}G\ H{\isachardot}\ \isactrlbold {\isasymnot}{\isacharparenleft}G\ \isactrlbold {\isasymrightarrow}\ H{\isacharparenright}\ {\isasymin}\ S\ {\isasymlongrightarrow}\ G\ {\isasymin}\ S\ {\isasymand}\ \isactrlbold {\isasymnot}\ H\ {\isasymin}\ S{\isacharparenright}{\isachardoublequoteclose}\ \isanewline
\ \ \ \ \ \ \isacommand{using}\isamarkupfalse%
\ C{\isadigit{3}}{\isadigit{1}}\ C{\isadigit{3}}{\isadigit{2}}\ C{\isadigit{3}}{\isadigit{3}}\ C{\isadigit{3}}{\isadigit{4}}\ \isacommand{by}\isamarkupfalse%
\ {\isacharparenleft}iprover\ intro{\isacharcolon}\ conjI{\isacharparenright}\isanewline
\ \ \ \ \isacommand{thus}\isamarkupfalse%
\ {\isachardoublequoteopen}Con\ F\ G\ H\ {\isasymlongrightarrow}\ F\ {\isasymin}\ S\ {\isasymlongrightarrow}\ G\ {\isasymin}\ S\ {\isasymand}\ H\ {\isasymin}\ S{\isachardoublequoteclose}\isanewline
\ \ \ \ \ \ \isacommand{by}\isamarkupfalse%
\ {\isacharparenleft}rule\ Hintikka{\isacharunderscore}alt{\isadigit{1}}Con{\isacharparenright}\isanewline
\ \ \isacommand{qed}\isamarkupfalse%
\isanewline
\ \ \isacommand{have}\isamarkupfalse%
\ C{\isadigit{4}}{\isacharcolon}{\isachardoublequoteopen}{\isasymforall}F\ G\ H{\isachardot}\ Dis\ F\ G\ H\ {\isasymlongrightarrow}\ F\ {\isasymin}\ S\ {\isasymlongrightarrow}\ G\ {\isasymin}\ S\ {\isasymor}\ H\ {\isasymin}\ S{\isachardoublequoteclose}\isanewline
\ \ \isacommand{proof}\isamarkupfalse%
\ {\isacharparenleft}rule\ allI{\isacharparenright}{\isacharplus}\isanewline
\ \ \ \ \isacommand{fix}\isamarkupfalse%
\ F\ G\ H\isanewline
\ \ \ \ \isacommand{have}\isamarkupfalse%
\ C{\isadigit{4}}{\isadigit{1}}{\isacharcolon}{\isachardoublequoteopen}{\isasymforall}G\ H{\isachardot}\ G\ \isactrlbold {\isasymor}\ H\ {\isasymin}\ S\ {\isasymlongrightarrow}\ G\ {\isasymin}\ S\ {\isasymor}\ H\ {\isasymin}\ S{\isachardoublequoteclose}\isanewline
\ \ \ \ \ \ \isacommand{using}\isamarkupfalse%
\ Hk\ \isacommand{by}\isamarkupfalse%
\ {\isacharparenleft}iprover\ elim{\isacharcolon}\ conjunct{\isadigit{2}}\ conjunct{\isadigit{1}}{\isacharparenright}\isanewline
\ \ \ \ \isacommand{have}\isamarkupfalse%
\ C{\isadigit{4}}{\isadigit{2}}{\isacharcolon}{\isachardoublequoteopen}{\isasymforall}G\ H{\isachardot}\ G\ \isactrlbold {\isasymrightarrow}\ H\ {\isasymin}\ S\ {\isasymlongrightarrow}\ \isactrlbold {\isasymnot}\ G\ {\isasymin}\ S\ {\isasymor}\ H\ {\isasymin}\ S{\isachardoublequoteclose}\isanewline
\ \ \ \ \ \ \isacommand{using}\isamarkupfalse%
\ Hk\ \isacommand{by}\isamarkupfalse%
\ {\isacharparenleft}iprover\ elim{\isacharcolon}\ conjunct{\isadigit{2}}\ conjunct{\isadigit{1}}{\isacharparenright}\isanewline
\ \ \ \ \isacommand{have}\isamarkupfalse%
\ C{\isadigit{4}}{\isadigit{3}}{\isacharcolon}{\isachardoublequoteopen}{\isasymforall}G{\isachardot}\ \isactrlbold {\isasymnot}\ {\isacharparenleft}\isactrlbold {\isasymnot}\ G{\isacharparenright}\ {\isasymin}\ S\ {\isasymlongrightarrow}\ G\ {\isasymin}\ S{\isachardoublequoteclose}\isanewline
\ \ \ \ \ \ \isacommand{using}\isamarkupfalse%
\ Hk\ \isacommand{by}\isamarkupfalse%
\ {\isacharparenleft}iprover\ elim{\isacharcolon}\ conjunct{\isadigit{2}}\ conjunct{\isadigit{1}}{\isacharparenright}\isanewline
\ \ \ \ \isacommand{have}\isamarkupfalse%
\ C{\isadigit{4}}{\isadigit{4}}{\isacharcolon}{\isachardoublequoteopen}{\isasymforall}G\ H{\isachardot}\ \isactrlbold {\isasymnot}{\isacharparenleft}G\ \isactrlbold {\isasymand}\ H{\isacharparenright}\ {\isasymin}\ S\ {\isasymlongrightarrow}\ \isactrlbold {\isasymnot}\ G\ {\isasymin}\ S\ {\isasymor}\ \isactrlbold {\isasymnot}\ H\ {\isasymin}\ S{\isachardoublequoteclose}\isanewline
\ \ \ \ \ \ \isacommand{using}\isamarkupfalse%
\ Hk\ \isacommand{by}\isamarkupfalse%
\ {\isacharparenleft}iprover\ elim{\isacharcolon}\ conjunct{\isadigit{2}}\ conjunct{\isadigit{1}}{\isacharparenright}\isanewline
\ \ \ \ \isacommand{have}\isamarkupfalse%
\ {\isachardoublequoteopen}{\isacharparenleft}{\isasymforall}G\ H{\isachardot}\ G\ \isactrlbold {\isasymor}\ H\ {\isasymin}\ S\ {\isasymlongrightarrow}\ G\ {\isasymin}\ S\ {\isasymor}\ H\ {\isasymin}\ S{\isacharparenright}\isanewline
\ \ \ \ \ \ \ \ \ \ {\isasymand}\ {\isacharparenleft}{\isasymforall}G\ H{\isachardot}\ G\ \isactrlbold {\isasymrightarrow}\ H\ {\isasymin}\ S\ {\isasymlongrightarrow}\ \isactrlbold {\isasymnot}\ G\ {\isasymin}\ S\ {\isasymor}\ H\ {\isasymin}\ S{\isacharparenright}\isanewline
\ \ \ \ \ \ \ \ \ \ {\isasymand}\ {\isacharparenleft}{\isasymforall}G{\isachardot}\ \isactrlbold {\isasymnot}\ {\isacharparenleft}\isactrlbold {\isasymnot}\ G{\isacharparenright}\ {\isasymin}\ S\ {\isasymlongrightarrow}\ G\ {\isasymin}\ S{\isacharparenright}\isanewline
\ \ \ \ \ \ \ \ \ \ {\isasymand}\ {\isacharparenleft}{\isasymforall}G\ H{\isachardot}\ \isactrlbold {\isasymnot}{\isacharparenleft}G\ \isactrlbold {\isasymand}\ H{\isacharparenright}\ {\isasymin}\ S\ {\isasymlongrightarrow}\ \isactrlbold {\isasymnot}\ G\ {\isasymin}\ S\ {\isasymor}\ \isactrlbold {\isasymnot}\ H\ {\isasymin}\ S{\isacharparenright}{\isachardoublequoteclose}\isanewline
\ \ \ \ \ \ \isacommand{using}\isamarkupfalse%
\ C{\isadigit{4}}{\isadigit{1}}\ C{\isadigit{4}}{\isadigit{2}}\ C{\isadigit{4}}{\isadigit{3}}\ C{\isadigit{4}}{\isadigit{4}}\ \isacommand{by}\isamarkupfalse%
\ {\isacharparenleft}iprover\ intro{\isacharcolon}\ conjI{\isacharparenright}\isanewline
\ \ \ \ \isacommand{thus}\isamarkupfalse%
\ {\isachardoublequoteopen}Dis\ F\ G\ H\ {\isasymlongrightarrow}\ F\ {\isasymin}\ S\ {\isasymlongrightarrow}\ G\ {\isasymin}\ S\ {\isasymor}\ H\ {\isasymin}\ S{\isachardoublequoteclose}\isanewline
\ \ \ \ \ \ \isacommand{by}\isamarkupfalse%
\ {\isacharparenleft}rule\ Hintikka{\isacharunderscore}alt{\isadigit{1}}Dis{\isacharparenright}\isanewline
\ \ \isacommand{qed}\isamarkupfalse%
\isanewline
\ \ \isacommand{show}\isamarkupfalse%
\ {\isachardoublequoteopen}{\isasymbottom}\ {\isasymnotin}\ S\isanewline
\ \ {\isasymand}\ {\isacharparenleft}{\isasymforall}k{\isachardot}\ Atom\ k\ {\isasymin}\ S\ {\isasymlongrightarrow}\ \isactrlbold {\isasymnot}\ {\isacharparenleft}Atom\ k{\isacharparenright}\ {\isasymin}\ S\ {\isasymlongrightarrow}\ False{\isacharparenright}\isanewline
\ \ {\isasymand}\ {\isacharparenleft}{\isasymforall}F\ G\ H{\isachardot}\ Con\ F\ G\ H\ {\isasymlongrightarrow}\ F\ {\isasymin}\ S\ {\isasymlongrightarrow}\ G\ {\isasymin}\ S\ {\isasymand}\ H\ {\isasymin}\ S{\isacharparenright}\isanewline
\ \ {\isasymand}\ {\isacharparenleft}{\isasymforall}F\ G\ H{\isachardot}\ Dis\ F\ G\ H\ {\isasymlongrightarrow}\ F\ {\isasymin}\ S\ {\isasymlongrightarrow}\ G\ {\isasymin}\ S\ {\isasymor}\ H\ {\isasymin}\ S{\isacharparenright}{\isachardoublequoteclose}\isanewline
\ \ \ \ \isacommand{using}\isamarkupfalse%
\ C{\isadigit{1}}\ C{\isadigit{2}}\ C{\isadigit{3}}\ C{\isadigit{4}}\ \isacommand{by}\isamarkupfalse%
\ {\isacharparenleft}iprover\ intro{\isacharcolon}\ conjI{\isacharparenright}\isanewline
\isacommand{qed}\isamarkupfalse%
%
\endisatagproof
{\isafoldproof}%
%
\isadelimproof
%
\endisadelimproof
%
\begin{isamarkuptext}%
Por último, probamos la implicación recíproca de forma detallada en Isabelle mediante
  el siguiente lema.%
\end{isamarkuptext}\isamarkuptrue%
\isacommand{lemma}\isamarkupfalse%
\ Hintikka{\isacharunderscore}alt{\isadigit{2}}{\isacharcolon}\isanewline
\ \ \isakeyword{assumes}\ {\isachardoublequoteopen}{\isasymbottom}\ {\isasymnotin}\ S\isanewline
{\isasymand}\ {\isacharparenleft}{\isasymforall}k{\isachardot}\ Atom\ k\ {\isasymin}\ S\ {\isasymlongrightarrow}\ \isactrlbold {\isasymnot}\ {\isacharparenleft}Atom\ k{\isacharparenright}\ {\isasymin}\ S\ {\isasymlongrightarrow}\ False{\isacharparenright}\isanewline
{\isasymand}\ {\isacharparenleft}{\isasymforall}F\ G\ H{\isachardot}\ Con\ F\ G\ H\ {\isasymlongrightarrow}\ F\ {\isasymin}\ S\ {\isasymlongrightarrow}\ G\ {\isasymin}\ S\ {\isasymand}\ H\ {\isasymin}\ S{\isacharparenright}\ \isanewline
{\isasymand}\ {\isacharparenleft}{\isasymforall}F\ G\ H{\isachardot}\ Dis\ F\ G\ H\ {\isasymlongrightarrow}\ F\ {\isasymin}\ S\ {\isasymlongrightarrow}\ G\ {\isasymin}\ S\ {\isasymor}\ H\ {\isasymin}\ S{\isacharparenright}{\isachardoublequoteclose}\ \ \isanewline
\ \ \isakeyword{shows}\ {\isachardoublequoteopen}Hintikka\ S{\isachardoublequoteclose}\isanewline
%
\isadelimproof
%
\endisadelimproof
%
\isatagproof
\isacommand{proof}\isamarkupfalse%
\ {\isacharminus}\isanewline
\ \ \isacommand{have}\isamarkupfalse%
\ Con{\isacharcolon}{\isachardoublequoteopen}{\isasymforall}F\ G\ H{\isachardot}\ Con\ F\ G\ H\ {\isasymlongrightarrow}\ F\ {\isasymin}\ S\ {\isasymlongrightarrow}\ G\ {\isasymin}\ S\ {\isasymand}\ H\ {\isasymin}\ S{\isachardoublequoteclose}\isanewline
\ \ \ \ \isacommand{using}\isamarkupfalse%
\ assms\ \isacommand{by}\isamarkupfalse%
\ {\isacharparenleft}iprover\ elim{\isacharcolon}\ conjunct{\isadigit{2}}\ conjunct{\isadigit{1}}{\isacharparenright}\isanewline
\ \ \isacommand{have}\isamarkupfalse%
\ Dis{\isacharcolon}{\isachardoublequoteopen}{\isasymforall}F\ G\ H{\isachardot}\ Dis\ F\ G\ H\ {\isasymlongrightarrow}\ F\ {\isasymin}\ S\ {\isasymlongrightarrow}\ G\ {\isasymin}\ S\ {\isasymor}\ H\ {\isasymin}\ S{\isachardoublequoteclose}\isanewline
\ \ \ \ \isacommand{using}\isamarkupfalse%
\ assms\ \isacommand{by}\isamarkupfalse%
\ {\isacharparenleft}iprover\ elim{\isacharcolon}\ conjunct{\isadigit{2}}\ conjunct{\isadigit{1}}{\isacharparenright}\isanewline
\ \ \isacommand{have}\isamarkupfalse%
\ {\isachardoublequoteopen}{\isasymbottom}\ {\isasymnotin}\ S\isanewline
\ \ {\isasymand}\ {\isacharparenleft}{\isasymforall}k{\isachardot}\ Atom\ k\ {\isasymin}\ S\ {\isasymlongrightarrow}\ \isactrlbold {\isasymnot}\ {\isacharparenleft}Atom\ k{\isacharparenright}\ {\isasymin}\ S\ {\isasymlongrightarrow}\ False{\isacharparenright}\isanewline
\ \ {\isasymand}\ {\isacharparenleft}{\isasymforall}G\ H{\isachardot}\ G\ \isactrlbold {\isasymand}\ H\ {\isasymin}\ S\ {\isasymlongrightarrow}\ G\ {\isasymin}\ S\ {\isasymand}\ H\ {\isasymin}\ S{\isacharparenright}\isanewline
\ \ {\isasymand}\ {\isacharparenleft}{\isasymforall}G\ H{\isachardot}\ G\ \isactrlbold {\isasymor}\ H\ {\isasymin}\ S\ {\isasymlongrightarrow}\ G\ {\isasymin}\ S\ {\isasymor}\ H\ {\isasymin}\ S{\isacharparenright}\isanewline
\ \ {\isasymand}\ {\isacharparenleft}{\isasymforall}G\ H{\isachardot}\ G\ \isactrlbold {\isasymrightarrow}\ H\ {\isasymin}\ S\ {\isasymlongrightarrow}\ \isactrlbold {\isasymnot}G\ {\isasymin}\ S\ {\isasymor}\ H\ {\isasymin}\ S{\isacharparenright}\isanewline
\ \ {\isasymand}\ {\isacharparenleft}{\isasymforall}G{\isachardot}\ \isactrlbold {\isasymnot}\ {\isacharparenleft}\isactrlbold {\isasymnot}G{\isacharparenright}\ {\isasymin}\ S\ {\isasymlongrightarrow}\ G\ {\isasymin}\ S{\isacharparenright}\isanewline
\ \ {\isasymand}\ {\isacharparenleft}{\isasymforall}G\ H{\isachardot}\ \isactrlbold {\isasymnot}{\isacharparenleft}G\ \isactrlbold {\isasymand}\ H{\isacharparenright}\ {\isasymin}\ S\ {\isasymlongrightarrow}\ \isactrlbold {\isasymnot}\ G\ {\isasymin}\ S\ {\isasymor}\ \isactrlbold {\isasymnot}\ H\ {\isasymin}\ S{\isacharparenright}\isanewline
\ \ {\isasymand}\ {\isacharparenleft}{\isasymforall}G\ H{\isachardot}\ \isactrlbold {\isasymnot}{\isacharparenleft}G\ \isactrlbold {\isasymor}\ H{\isacharparenright}\ {\isasymin}\ S\ {\isasymlongrightarrow}\ \isactrlbold {\isasymnot}\ G\ {\isasymin}\ S\ {\isasymand}\ \isactrlbold {\isasymnot}\ H\ {\isasymin}\ S{\isacharparenright}\isanewline
\ \ {\isasymand}\ {\isacharparenleft}{\isasymforall}G\ H{\isachardot}\ \isactrlbold {\isasymnot}{\isacharparenleft}G\ \isactrlbold {\isasymrightarrow}\ H{\isacharparenright}\ {\isasymin}\ S\ {\isasymlongrightarrow}\ G\ {\isasymin}\ S\ {\isasymand}\ \isactrlbold {\isasymnot}\ H\ {\isasymin}\ S{\isacharparenright}{\isachardoublequoteclose}\isanewline
\ \ \isacommand{proof}\isamarkupfalse%
\ {\isacharminus}\isanewline
\ \ \ \ \isacommand{have}\isamarkupfalse%
\ C{\isadigit{1}}{\isacharcolon}{\isachardoublequoteopen}{\isasymbottom}\ {\isasymnotin}\ S{\isachardoublequoteclose}\isanewline
\ \ \ \ \ \ \isacommand{using}\isamarkupfalse%
\ assms\ \isacommand{by}\isamarkupfalse%
\ {\isacharparenleft}rule\ conjunct{\isadigit{1}}{\isacharparenright}\isanewline
\ \ \ \ \isacommand{have}\isamarkupfalse%
\ C{\isadigit{2}}{\isacharcolon}{\isachardoublequoteopen}{\isasymforall}k{\isachardot}\ Atom\ k\ {\isasymin}\ S\ {\isasymlongrightarrow}\ \isactrlbold {\isasymnot}\ {\isacharparenleft}Atom\ k{\isacharparenright}\ {\isasymin}\ S\ {\isasymlongrightarrow}\ False{\isachardoublequoteclose}\isanewline
\ \ \ \ \ \ \isacommand{using}\isamarkupfalse%
\ assms\ \isacommand{by}\isamarkupfalse%
\ {\isacharparenleft}iprover\ elim{\isacharcolon}\ conjunct{\isadigit{2}}\ conjunct{\isadigit{1}}{\isacharparenright}\isanewline
\ \ \ \ \isacommand{have}\isamarkupfalse%
\ C{\isadigit{3}}{\isacharcolon}{\isachardoublequoteopen}{\isasymforall}G\ H{\isachardot}\ G\ \isactrlbold {\isasymand}\ H\ {\isasymin}\ S\ {\isasymlongrightarrow}\ G\ {\isasymin}\ S\ {\isasymand}\ H\ {\isasymin}\ S{\isachardoublequoteclose}\isanewline
\ \ \ \ \isacommand{proof}\isamarkupfalse%
\ {\isacharparenleft}rule\ allI{\isacharparenright}{\isacharplus}\isanewline
\ \ \ \ \ \ \isacommand{fix}\isamarkupfalse%
\ G\ H\isanewline
\ \ \ \ \ \ \isacommand{show}\isamarkupfalse%
\ {\isachardoublequoteopen}G\ \isactrlbold {\isasymand}\ H\ {\isasymin}\ S\ {\isasymlongrightarrow}\ G\ {\isasymin}\ S\ {\isasymand}\ H\ {\isasymin}\ S{\isachardoublequoteclose}\isanewline
\ \ \ \ \ \ \isacommand{proof}\isamarkupfalse%
\ {\isacharparenleft}rule\ impI{\isacharparenright}\isanewline
\ \ \ \ \ \ \ \ \isacommand{assume}\isamarkupfalse%
\ {\isachardoublequoteopen}G\ \isactrlbold {\isasymand}\ H\ {\isasymin}\ S{\isachardoublequoteclose}\isanewline
\ \ \ \ \ \ \ \ \isacommand{have}\isamarkupfalse%
\ {\isachardoublequoteopen}Con\ {\isacharparenleft}G\ \isactrlbold {\isasymand}\ H{\isacharparenright}\ G\ H{\isachardoublequoteclose}\isanewline
\ \ \ \ \ \ \ \ \ \ \isacommand{by}\isamarkupfalse%
\ {\isacharparenleft}simp\ only{\isacharcolon}\ Con{\isachardot}intros{\isacharparenleft}{\isadigit{1}}{\isacharparenright}{\isacharparenright}\isanewline
\ \ \ \ \ \ \ \ \isacommand{have}\isamarkupfalse%
\ {\isachardoublequoteopen}Con\ {\isacharparenleft}G\ \isactrlbold {\isasymand}\ H{\isacharparenright}\ G\ H\ {\isasymlongrightarrow}\ G\ \isactrlbold {\isasymand}\ H\ {\isasymin}\ S\ {\isasymlongrightarrow}\ G\ {\isasymin}\ S\ {\isasymand}\ H\ {\isasymin}\ S{\isachardoublequoteclose}\isanewline
\ \ \ \ \ \ \ \ \ \ \isacommand{using}\isamarkupfalse%
\ Con\ \isacommand{by}\isamarkupfalse%
\ {\isacharparenleft}iprover\ elim{\isacharcolon}\ allE{\isacharparenright}\isanewline
\ \ \ \ \ \ \ \ \isacommand{then}\isamarkupfalse%
\ \isacommand{have}\isamarkupfalse%
\ {\isachardoublequoteopen}G\ \isactrlbold {\isasymand}\ H\ {\isasymin}\ S\ {\isasymlongrightarrow}\ G\ {\isasymin}\ S\ {\isasymand}\ H\ {\isasymin}\ S{\isachardoublequoteclose}\isanewline
\ \ \ \ \ \ \ \ \ \ \isacommand{using}\isamarkupfalse%
\ {\isacartoucheopen}Con\ {\isacharparenleft}G\ \isactrlbold {\isasymand}\ H{\isacharparenright}\ G\ H{\isacartoucheclose}\ \isacommand{by}\isamarkupfalse%
\ {\isacharparenleft}rule\ mp{\isacharparenright}\isanewline
\ \ \ \ \ \ \ \ \isacommand{thus}\isamarkupfalse%
\ {\isachardoublequoteopen}G\ {\isasymin}\ S\ {\isasymand}\ H\ {\isasymin}\ S{\isachardoublequoteclose}\isanewline
\ \ \ \ \ \ \ \ \ \ \isacommand{using}\isamarkupfalse%
\ {\isacartoucheopen}G\ \isactrlbold {\isasymand}\ H\ {\isasymin}\ S{\isacartoucheclose}\ \isacommand{by}\isamarkupfalse%
\ {\isacharparenleft}rule\ mp{\isacharparenright}\isanewline
\ \ \ \ \ \ \isacommand{qed}\isamarkupfalse%
\isanewline
\ \ \ \ \isacommand{qed}\isamarkupfalse%
\isanewline
\ \ \ \ \isacommand{have}\isamarkupfalse%
\ C{\isadigit{4}}{\isacharcolon}{\isachardoublequoteopen}{\isasymforall}G\ H{\isachardot}\ G\ \isactrlbold {\isasymor}\ H\ {\isasymin}\ S\ {\isasymlongrightarrow}\ G\ {\isasymin}\ S\ {\isasymor}\ H\ {\isasymin}\ S{\isachardoublequoteclose}\isanewline
\ \ \ \ \isacommand{proof}\isamarkupfalse%
\ {\isacharparenleft}rule\ allI{\isacharparenright}{\isacharplus}\isanewline
\ \ \ \ \ \ \isacommand{fix}\isamarkupfalse%
\ G\ H\isanewline
\ \ \ \ \ \ \isacommand{show}\isamarkupfalse%
\ {\isachardoublequoteopen}G\ \isactrlbold {\isasymor}\ H\ {\isasymin}\ S\ {\isasymlongrightarrow}\ G\ {\isasymin}\ S\ {\isasymor}\ H\ {\isasymin}\ S{\isachardoublequoteclose}\isanewline
\ \ \ \ \ \ \isacommand{proof}\isamarkupfalse%
\ {\isacharparenleft}rule\ impI{\isacharparenright}\isanewline
\ \ \ \ \ \ \ \ \isacommand{assume}\isamarkupfalse%
\ {\isachardoublequoteopen}G\ \isactrlbold {\isasymor}\ H\ {\isasymin}\ S{\isachardoublequoteclose}\isanewline
\ \ \ \ \ \ \ \ \isacommand{have}\isamarkupfalse%
\ {\isachardoublequoteopen}Dis\ {\isacharparenleft}G\ \isactrlbold {\isasymor}\ H{\isacharparenright}\ G\ H{\isachardoublequoteclose}\isanewline
\ \ \ \ \ \ \ \ \ \ \isacommand{by}\isamarkupfalse%
\ {\isacharparenleft}simp\ only{\isacharcolon}\ Dis{\isachardot}intros{\isacharparenleft}{\isadigit{1}}{\isacharparenright}{\isacharparenright}\isanewline
\ \ \ \ \ \ \ \ \isacommand{have}\isamarkupfalse%
\ {\isachardoublequoteopen}Dis\ {\isacharparenleft}G\ \isactrlbold {\isasymor}\ H{\isacharparenright}\ G\ H\ {\isasymlongrightarrow}\ G\ \isactrlbold {\isasymor}\ H\ {\isasymin}\ S\ {\isasymlongrightarrow}\ G\ {\isasymin}\ S\ {\isasymor}\ H\ {\isasymin}\ S{\isachardoublequoteclose}\isanewline
\ \ \ \ \ \ \ \ \ \ \isacommand{using}\isamarkupfalse%
\ Dis\ \isacommand{by}\isamarkupfalse%
\ {\isacharparenleft}iprover\ elim{\isacharcolon}\ allE{\isacharparenright}\isanewline
\ \ \ \ \ \ \ \ \isacommand{then}\isamarkupfalse%
\ \isacommand{have}\isamarkupfalse%
\ {\isachardoublequoteopen}G\ \isactrlbold {\isasymor}\ H\ {\isasymin}\ S\ {\isasymlongrightarrow}\ G\ {\isasymin}\ S\ {\isasymor}\ H\ {\isasymin}\ S{\isachardoublequoteclose}\isanewline
\ \ \ \ \ \ \ \ \ \ \isacommand{using}\isamarkupfalse%
\ {\isacartoucheopen}Dis\ {\isacharparenleft}G\ \isactrlbold {\isasymor}\ H{\isacharparenright}\ G\ H{\isacartoucheclose}\ \isacommand{by}\isamarkupfalse%
\ {\isacharparenleft}rule\ mp{\isacharparenright}\isanewline
\ \ \ \ \ \ \ \ \isacommand{thus}\isamarkupfalse%
\ {\isachardoublequoteopen}G\ {\isasymin}\ S\ {\isasymor}\ H\ {\isasymin}\ S{\isachardoublequoteclose}\isanewline
\ \ \ \ \ \ \ \ \ \ \isacommand{using}\isamarkupfalse%
\ {\isacartoucheopen}G\ \isactrlbold {\isasymor}\ H\ {\isasymin}\ S{\isacartoucheclose}\ \isacommand{by}\isamarkupfalse%
\ {\isacharparenleft}rule\ mp{\isacharparenright}\isanewline
\ \ \ \ \ \ \isacommand{qed}\isamarkupfalse%
\isanewline
\ \ \ \ \isacommand{qed}\isamarkupfalse%
\isanewline
\ \ \ \ \isacommand{have}\isamarkupfalse%
\ C{\isadigit{5}}{\isacharcolon}{\isachardoublequoteopen}{\isasymforall}G\ H{\isachardot}\ G\ \isactrlbold {\isasymrightarrow}\ H\ {\isasymin}\ S\ {\isasymlongrightarrow}\ \isactrlbold {\isasymnot}\ G\ {\isasymin}\ S\ {\isasymor}\ H\ {\isasymin}\ S{\isachardoublequoteclose}\isanewline
\ \ \ \ \isacommand{proof}\isamarkupfalse%
\ {\isacharparenleft}rule\ allI{\isacharparenright}{\isacharplus}\isanewline
\ \ \ \ \ \ \isacommand{fix}\isamarkupfalse%
\ G\ H\isanewline
\ \ \ \ \ \ \isacommand{show}\isamarkupfalse%
\ {\isachardoublequoteopen}G\ \isactrlbold {\isasymrightarrow}\ H\ {\isasymin}\ S\ {\isasymlongrightarrow}\ \isactrlbold {\isasymnot}\ G\ {\isasymin}\ S\ {\isasymor}\ H\ {\isasymin}\ S{\isachardoublequoteclose}\isanewline
\ \ \ \ \ \ \isacommand{proof}\isamarkupfalse%
\ {\isacharparenleft}rule\ impI{\isacharparenright}\isanewline
\ \ \ \ \ \ \ \ \isacommand{assume}\isamarkupfalse%
\ {\isachardoublequoteopen}G\ \isactrlbold {\isasymrightarrow}\ H\ {\isasymin}\ S{\isachardoublequoteclose}\ \isanewline
\ \ \ \ \ \ \ \ \isacommand{have}\isamarkupfalse%
\ {\isachardoublequoteopen}Dis\ {\isacharparenleft}G\ \isactrlbold {\isasymrightarrow}\ H{\isacharparenright}\ {\isacharparenleft}\isactrlbold {\isasymnot}\ G{\isacharparenright}\ H{\isachardoublequoteclose}\isanewline
\ \ \ \ \ \ \ \ \ \ \isacommand{by}\isamarkupfalse%
\ {\isacharparenleft}simp\ only{\isacharcolon}\ Dis{\isachardot}intros{\isacharparenleft}{\isadigit{2}}{\isacharparenright}{\isacharparenright}\isanewline
\ \ \ \ \ \ \ \ \isacommand{have}\isamarkupfalse%
\ {\isachardoublequoteopen}Dis\ {\isacharparenleft}G\ \isactrlbold {\isasymrightarrow}\ H{\isacharparenright}\ {\isacharparenleft}\isactrlbold {\isasymnot}\ G{\isacharparenright}\ H\ {\isasymlongrightarrow}\ G\ \isactrlbold {\isasymrightarrow}\ H\ {\isasymin}\ S\ {\isasymlongrightarrow}\ \isactrlbold {\isasymnot}\ G\ {\isasymin}\ S\ {\isasymor}\ H\ {\isasymin}\ S{\isachardoublequoteclose}\isanewline
\ \ \ \ \ \ \ \ \ \ \isacommand{using}\isamarkupfalse%
\ Dis\ \isacommand{by}\isamarkupfalse%
\ {\isacharparenleft}iprover\ elim{\isacharcolon}\ allE{\isacharparenright}\isanewline
\ \ \ \ \ \ \ \ \isacommand{then}\isamarkupfalse%
\ \isacommand{have}\isamarkupfalse%
\ {\isachardoublequoteopen}G\ \isactrlbold {\isasymrightarrow}\ H\ {\isasymin}\ S\ {\isasymlongrightarrow}\ \isactrlbold {\isasymnot}\ G\ {\isasymin}\ S\ {\isasymor}\ H\ {\isasymin}\ S{\isachardoublequoteclose}\ \isanewline
\ \ \ \ \ \ \ \ \ \ \isacommand{using}\isamarkupfalse%
\ {\isacartoucheopen}Dis\ {\isacharparenleft}G\ \isactrlbold {\isasymrightarrow}\ H{\isacharparenright}\ {\isacharparenleft}\isactrlbold {\isasymnot}\ G{\isacharparenright}\ H{\isacartoucheclose}\ \isacommand{by}\isamarkupfalse%
\ {\isacharparenleft}rule\ mp{\isacharparenright}\isanewline
\ \ \ \ \ \ \ \ \isacommand{thus}\isamarkupfalse%
\ {\isachardoublequoteopen}\isactrlbold {\isasymnot}\ G\ {\isasymin}\ S\ {\isasymor}\ H\ {\isasymin}\ S{\isachardoublequoteclose}\isanewline
\ \ \ \ \ \ \ \ \ \ \isacommand{using}\isamarkupfalse%
\ {\isacartoucheopen}G\ \isactrlbold {\isasymrightarrow}\ H\ {\isasymin}\ S{\isacartoucheclose}\ \isacommand{by}\isamarkupfalse%
\ {\isacharparenleft}rule\ mp{\isacharparenright}\isanewline
\ \ \ \ \ \ \isacommand{qed}\isamarkupfalse%
\isanewline
\ \ \ \ \isacommand{qed}\isamarkupfalse%
\isanewline
\ \ \ \ \isacommand{have}\isamarkupfalse%
\ C{\isadigit{6}}{\isacharcolon}{\isachardoublequoteopen}{\isasymforall}G{\isachardot}\ \isactrlbold {\isasymnot}{\isacharparenleft}\isactrlbold {\isasymnot}\ G{\isacharparenright}\ {\isasymin}\ S\ {\isasymlongrightarrow}\ G\ {\isasymin}\ S{\isachardoublequoteclose}\isanewline
\ \ \ \ \isacommand{proof}\isamarkupfalse%
\ {\isacharparenleft}rule\ allI{\isacharparenright}\isanewline
\ \ \ \ \ \ \isacommand{fix}\isamarkupfalse%
\ G\isanewline
\ \ \ \ \ \ \isacommand{show}\isamarkupfalse%
\ {\isachardoublequoteopen}\isactrlbold {\isasymnot}{\isacharparenleft}\isactrlbold {\isasymnot}\ G{\isacharparenright}\ {\isasymin}\ S\ {\isasymlongrightarrow}\ G\ {\isasymin}\ S{\isachardoublequoteclose}\isanewline
\ \ \ \ \ \ \isacommand{proof}\isamarkupfalse%
\ {\isacharparenleft}rule\ impI{\isacharparenright}\isanewline
\ \ \ \ \ \ \ \ \isacommand{assume}\isamarkupfalse%
\ {\isachardoublequoteopen}\isactrlbold {\isasymnot}\ {\isacharparenleft}\isactrlbold {\isasymnot}\ G{\isacharparenright}\ {\isasymin}\ S{\isachardoublequoteclose}\ \isanewline
\ \ \ \ \ \ \ \ \isacommand{have}\isamarkupfalse%
\ {\isachardoublequoteopen}Con\ {\isacharparenleft}\isactrlbold {\isasymnot}\ {\isacharparenleft}\isactrlbold {\isasymnot}\ G{\isacharparenright}{\isacharparenright}\ G\ G{\isachardoublequoteclose}\isanewline
\ \ \ \ \ \ \ \ \ \ \isacommand{by}\isamarkupfalse%
\ {\isacharparenleft}simp\ only{\isacharcolon}\ Con{\isachardot}intros{\isacharparenleft}{\isadigit{4}}{\isacharparenright}{\isacharparenright}\isanewline
\ \ \ \ \ \ \ \ \isacommand{have}\isamarkupfalse%
\ {\isachardoublequoteopen}Con\ {\isacharparenleft}\isactrlbold {\isasymnot}{\isacharparenleft}\isactrlbold {\isasymnot}\ G{\isacharparenright}{\isacharparenright}\ G\ G\ {\isasymlongrightarrow}\ {\isacharparenleft}\isactrlbold {\isasymnot}{\isacharparenleft}\isactrlbold {\isasymnot}\ G{\isacharparenright}{\isacharparenright}\ {\isasymin}\ S\ {\isasymlongrightarrow}\ G\ {\isasymin}\ S\ {\isasymand}\ G\ {\isasymin}\ S{\isachardoublequoteclose}\isanewline
\ \ \ \ \ \ \ \ \ \ \isacommand{using}\isamarkupfalse%
\ Con\ \isacommand{by}\isamarkupfalse%
\ {\isacharparenleft}iprover\ elim{\isacharcolon}\ allE{\isacharparenright}\isanewline
\ \ \ \ \ \ \ \ \isacommand{then}\isamarkupfalse%
\ \isacommand{have}\isamarkupfalse%
\ {\isachardoublequoteopen}{\isacharparenleft}\isactrlbold {\isasymnot}{\isacharparenleft}\isactrlbold {\isasymnot}\ G{\isacharparenright}{\isacharparenright}\ {\isasymin}\ S\ {\isasymlongrightarrow}\ G\ {\isasymin}\ S\ {\isasymand}\ G\ {\isasymin}\ S{\isachardoublequoteclose}\isanewline
\ \ \ \ \ \ \ \ \ \ \isacommand{using}\isamarkupfalse%
\ {\isacartoucheopen}Con\ {\isacharparenleft}\isactrlbold {\isasymnot}\ {\isacharparenleft}\isactrlbold {\isasymnot}\ G{\isacharparenright}{\isacharparenright}\ G\ G{\isacartoucheclose}\ \isacommand{by}\isamarkupfalse%
\ {\isacharparenleft}rule\ mp{\isacharparenright}\isanewline
\ \ \ \ \ \ \ \ \isacommand{then}\isamarkupfalse%
\ \isacommand{have}\isamarkupfalse%
\ {\isachardoublequoteopen}G\ {\isasymin}\ S\ {\isasymand}\ G\ {\isasymin}\ S{\isachardoublequoteclose}\isanewline
\ \ \ \ \ \ \ \ \ \ \isacommand{using}\isamarkupfalse%
\ {\isacartoucheopen}\isactrlbold {\isasymnot}\ {\isacharparenleft}\isactrlbold {\isasymnot}\ G{\isacharparenright}\ {\isasymin}\ S{\isacartoucheclose}\ \isacommand{by}\isamarkupfalse%
\ {\isacharparenleft}rule\ mp{\isacharparenright}\isanewline
\ \ \ \ \ \ \ \ \isacommand{thus}\isamarkupfalse%
\ {\isachardoublequoteopen}G\ {\isasymin}\ S{\isachardoublequoteclose}\isanewline
\ \ \ \ \ \ \ \ \ \ \isacommand{by}\isamarkupfalse%
\ {\isacharparenleft}simp\ only{\isacharcolon}\ conj{\isacharunderscore}absorb{\isacharparenright}\isanewline
\ \ \ \ \ \ \isacommand{qed}\isamarkupfalse%
\isanewline
\ \ \ \ \isacommand{qed}\isamarkupfalse%
\isanewline
\ \ \ \ \isacommand{have}\isamarkupfalse%
\ C{\isadigit{7}}{\isacharcolon}{\isachardoublequoteopen}{\isasymforall}G\ H{\isachardot}\ \isactrlbold {\isasymnot}{\isacharparenleft}G\ \isactrlbold {\isasymand}\ H{\isacharparenright}\ {\isasymin}\ S\ {\isasymlongrightarrow}\ \isactrlbold {\isasymnot}\ G\ {\isasymin}\ S\ {\isasymor}\ \isactrlbold {\isasymnot}\ H\ {\isasymin}\ S{\isachardoublequoteclose}\isanewline
\ \ \ \ \isacommand{proof}\isamarkupfalse%
\ {\isacharparenleft}rule\ allI{\isacharparenright}{\isacharplus}\isanewline
\ \ \ \ \ \ \isacommand{fix}\isamarkupfalse%
\ G\ H\isanewline
\ \ \ \ \ \ \isacommand{show}\isamarkupfalse%
\ {\isachardoublequoteopen}\isactrlbold {\isasymnot}{\isacharparenleft}G\ \isactrlbold {\isasymand}\ H{\isacharparenright}\ {\isasymin}\ S\ {\isasymlongrightarrow}\ \isactrlbold {\isasymnot}\ G\ {\isasymin}\ S\ {\isasymor}\ \isactrlbold {\isasymnot}\ H\ {\isasymin}\ S{\isachardoublequoteclose}\isanewline
\ \ \ \ \ \ \isacommand{proof}\isamarkupfalse%
\ {\isacharparenleft}rule\ impI{\isacharparenright}\isanewline
\ \ \ \ \ \ \ \ \isacommand{assume}\isamarkupfalse%
\ {\isachardoublequoteopen}\isactrlbold {\isasymnot}{\isacharparenleft}G\ \isactrlbold {\isasymand}\ H{\isacharparenright}\ {\isasymin}\ S{\isachardoublequoteclose}\isanewline
\ \ \ \ \ \ \ \ \isacommand{have}\isamarkupfalse%
\ {\isachardoublequoteopen}Dis\ {\isacharparenleft}\isactrlbold {\isasymnot}{\isacharparenleft}G\ \isactrlbold {\isasymand}\ H{\isacharparenright}{\isacharparenright}\ {\isacharparenleft}\isactrlbold {\isasymnot}\ G{\isacharparenright}\ {\isacharparenleft}\isactrlbold {\isasymnot}\ H{\isacharparenright}{\isachardoublequoteclose}\isanewline
\ \ \ \ \ \ \ \ \ \ \isacommand{by}\isamarkupfalse%
\ {\isacharparenleft}simp\ only{\isacharcolon}\ Dis{\isachardot}intros{\isacharparenleft}{\isadigit{3}}{\isacharparenright}{\isacharparenright}\isanewline
\ \ \ \ \ \ \ \ \isacommand{have}\isamarkupfalse%
\ {\isachardoublequoteopen}Dis\ {\isacharparenleft}\isactrlbold {\isasymnot}{\isacharparenleft}G\ \isactrlbold {\isasymand}\ H{\isacharparenright}{\isacharparenright}\ {\isacharparenleft}\isactrlbold {\isasymnot}\ G{\isacharparenright}\ {\isacharparenleft}\isactrlbold {\isasymnot}\ H{\isacharparenright}\ {\isasymlongrightarrow}\ \isactrlbold {\isasymnot}{\isacharparenleft}G\ \isactrlbold {\isasymand}\ H{\isacharparenright}\ {\isasymin}\ S\ {\isasymlongrightarrow}\ \isactrlbold {\isasymnot}\ G\ {\isasymin}\ S\ {\isasymor}\ \isactrlbold {\isasymnot}\ H\ {\isasymin}\ S{\isachardoublequoteclose}\isanewline
\ \ \ \ \ \ \ \ \ \ \isacommand{using}\isamarkupfalse%
\ Dis\ \isacommand{by}\isamarkupfalse%
\ {\isacharparenleft}iprover\ elim{\isacharcolon}\ allE{\isacharparenright}\isanewline
\ \ \ \ \ \ \ \ \isacommand{then}\isamarkupfalse%
\ \isacommand{have}\isamarkupfalse%
\ {\isachardoublequoteopen}\isactrlbold {\isasymnot}{\isacharparenleft}G\ \isactrlbold {\isasymand}\ H{\isacharparenright}\ {\isasymin}\ S\ {\isasymlongrightarrow}\ \isactrlbold {\isasymnot}\ G\ {\isasymin}\ S\ {\isasymor}\ \isactrlbold {\isasymnot}\ H\ {\isasymin}\ S{\isachardoublequoteclose}\isanewline
\ \ \ \ \ \ \ \ \ \ \isacommand{using}\isamarkupfalse%
\ {\isacartoucheopen}Dis\ {\isacharparenleft}\isactrlbold {\isasymnot}{\isacharparenleft}G\ \isactrlbold {\isasymand}\ H{\isacharparenright}{\isacharparenright}\ {\isacharparenleft}\isactrlbold {\isasymnot}\ G{\isacharparenright}\ {\isacharparenleft}\isactrlbold {\isasymnot}\ H{\isacharparenright}{\isacartoucheclose}\ \isacommand{by}\isamarkupfalse%
\ {\isacharparenleft}rule\ mp{\isacharparenright}\isanewline
\ \ \ \ \ \ \ \ \isacommand{thus}\isamarkupfalse%
\ {\isachardoublequoteopen}\isactrlbold {\isasymnot}\ G\ {\isasymin}\ S\ {\isasymor}\ \isactrlbold {\isasymnot}\ H\ {\isasymin}\ S{\isachardoublequoteclose}\isanewline
\ \ \ \ \ \ \ \ \ \ \isacommand{using}\isamarkupfalse%
\ {\isacartoucheopen}\isactrlbold {\isasymnot}{\isacharparenleft}G\ \isactrlbold {\isasymand}\ H{\isacharparenright}\ {\isasymin}\ S{\isacartoucheclose}\ \isacommand{by}\isamarkupfalse%
\ {\isacharparenleft}rule\ mp{\isacharparenright}\isanewline
\ \ \ \ \ \ \isacommand{qed}\isamarkupfalse%
\isanewline
\ \ \ \ \isacommand{qed}\isamarkupfalse%
\isanewline
\ \ \ \ \isacommand{have}\isamarkupfalse%
\ C{\isadigit{8}}{\isacharcolon}{\isachardoublequoteopen}{\isasymforall}G\ H{\isachardot}\ \isactrlbold {\isasymnot}{\isacharparenleft}G\ \isactrlbold {\isasymor}\ H{\isacharparenright}\ {\isasymin}\ S\ {\isasymlongrightarrow}\ \isactrlbold {\isasymnot}\ G\ {\isasymin}\ S\ {\isasymand}\ \isactrlbold {\isasymnot}\ H\ {\isasymin}\ S{\isachardoublequoteclose}\isanewline
\ \ \ \ \isacommand{proof}\isamarkupfalse%
\ {\isacharparenleft}rule\ allI{\isacharparenright}{\isacharplus}\isanewline
\ \ \ \ \ \ \isacommand{fix}\isamarkupfalse%
\ G\ H\isanewline
\ \ \ \ \ \ \isacommand{show}\isamarkupfalse%
\ {\isachardoublequoteopen}\isactrlbold {\isasymnot}{\isacharparenleft}G\ \isactrlbold {\isasymor}\ H{\isacharparenright}\ {\isasymin}\ S\ {\isasymlongrightarrow}\ \isactrlbold {\isasymnot}\ G\ {\isasymin}\ S\ {\isasymand}\ \isactrlbold {\isasymnot}\ H\ {\isasymin}\ S{\isachardoublequoteclose}\isanewline
\ \ \ \ \ \ \isacommand{proof}\isamarkupfalse%
\ {\isacharparenleft}rule\ impI{\isacharparenright}\isanewline
\ \ \ \ \ \ \ \ \isacommand{assume}\isamarkupfalse%
\ {\isachardoublequoteopen}\isactrlbold {\isasymnot}{\isacharparenleft}G\ \isactrlbold {\isasymor}\ H{\isacharparenright}\ {\isasymin}\ S{\isachardoublequoteclose}\isanewline
\ \ \ \ \ \ \ \ \isacommand{have}\isamarkupfalse%
\ {\isachardoublequoteopen}Con\ {\isacharparenleft}\isactrlbold {\isasymnot}{\isacharparenleft}G\ \isactrlbold {\isasymor}\ H{\isacharparenright}{\isacharparenright}\ {\isacharparenleft}\isactrlbold {\isasymnot}\ G{\isacharparenright}\ {\isacharparenleft}\isactrlbold {\isasymnot}\ H{\isacharparenright}{\isachardoublequoteclose}\isanewline
\ \ \ \ \ \ \ \ \ \ \isacommand{by}\isamarkupfalse%
\ {\isacharparenleft}simp\ only{\isacharcolon}\ Con{\isachardot}intros{\isacharparenleft}{\isadigit{2}}{\isacharparenright}{\isacharparenright}\isanewline
\ \ \ \ \ \ \ \ \isacommand{have}\isamarkupfalse%
\ {\isachardoublequoteopen}Con\ {\isacharparenleft}\isactrlbold {\isasymnot}{\isacharparenleft}G\ \isactrlbold {\isasymor}\ H{\isacharparenright}{\isacharparenright}\ {\isacharparenleft}\isactrlbold {\isasymnot}\ G{\isacharparenright}\ {\isacharparenleft}\isactrlbold {\isasymnot}\ H{\isacharparenright}\ {\isasymlongrightarrow}\ \isactrlbold {\isasymnot}{\isacharparenleft}G\ \isactrlbold {\isasymor}\ H{\isacharparenright}\ {\isasymin}\ S\ {\isasymlongrightarrow}\ \isactrlbold {\isasymnot}\ G\ {\isasymin}\ S\ {\isasymand}\ \isactrlbold {\isasymnot}\ H\ {\isasymin}\ S{\isachardoublequoteclose}\isanewline
\ \ \ \ \ \ \ \ \ \ \isacommand{using}\isamarkupfalse%
\ Con\ \isacommand{by}\isamarkupfalse%
\ {\isacharparenleft}iprover\ elim{\isacharcolon}\ allE{\isacharparenright}\isanewline
\ \ \ \ \ \ \ \ \isacommand{then}\isamarkupfalse%
\ \isacommand{have}\isamarkupfalse%
\ {\isachardoublequoteopen}\isactrlbold {\isasymnot}{\isacharparenleft}G\ \isactrlbold {\isasymor}\ H{\isacharparenright}\ {\isasymin}\ S\ {\isasymlongrightarrow}\ \isactrlbold {\isasymnot}\ G\ {\isasymin}\ S\ {\isasymand}\ \isactrlbold {\isasymnot}\ H\ {\isasymin}\ S{\isachardoublequoteclose}\isanewline
\ \ \ \ \ \ \ \ \ \ \isacommand{using}\isamarkupfalse%
\ {\isacartoucheopen}Con\ {\isacharparenleft}\isactrlbold {\isasymnot}{\isacharparenleft}G\ \isactrlbold {\isasymor}\ H{\isacharparenright}{\isacharparenright}\ {\isacharparenleft}\isactrlbold {\isasymnot}\ G{\isacharparenright}\ {\isacharparenleft}\isactrlbold {\isasymnot}\ H{\isacharparenright}{\isacartoucheclose}\ \isacommand{by}\isamarkupfalse%
\ {\isacharparenleft}rule\ mp{\isacharparenright}\isanewline
\ \ \ \ \ \ \ \ \isacommand{thus}\isamarkupfalse%
\ {\isachardoublequoteopen}\isactrlbold {\isasymnot}\ G\ {\isasymin}\ S\ {\isasymand}\ \isactrlbold {\isasymnot}\ H\ {\isasymin}\ S{\isachardoublequoteclose}\isanewline
\ \ \ \ \ \ \ \ \ \ \isacommand{using}\isamarkupfalse%
\ {\isacartoucheopen}\isactrlbold {\isasymnot}{\isacharparenleft}G\ \isactrlbold {\isasymor}\ H{\isacharparenright}\ {\isasymin}\ S{\isacartoucheclose}\ \isacommand{by}\isamarkupfalse%
\ {\isacharparenleft}rule\ mp{\isacharparenright}\isanewline
\ \ \ \ \ \ \isacommand{qed}\isamarkupfalse%
\isanewline
\ \ \ \ \isacommand{qed}\isamarkupfalse%
\isanewline
\ \ \ \ \isacommand{have}\isamarkupfalse%
\ C{\isadigit{9}}{\isacharcolon}{\isachardoublequoteopen}{\isasymforall}G\ H{\isachardot}\ \isactrlbold {\isasymnot}{\isacharparenleft}G\ \isactrlbold {\isasymrightarrow}\ H{\isacharparenright}\ {\isasymin}\ S\ {\isasymlongrightarrow}\ G\ {\isasymin}\ S\ {\isasymand}\ \isactrlbold {\isasymnot}\ H\ {\isasymin}\ S{\isachardoublequoteclose}\isanewline
\ \ \ \ \isacommand{proof}\isamarkupfalse%
\ {\isacharparenleft}rule\ allI{\isacharparenright}{\isacharplus}\isanewline
\ \ \ \ \ \ \isacommand{fix}\isamarkupfalse%
\ G\ H\isanewline
\ \ \ \ \ \ \isacommand{show}\isamarkupfalse%
\ {\isachardoublequoteopen}\isactrlbold {\isasymnot}{\isacharparenleft}G\ \isactrlbold {\isasymrightarrow}\ H{\isacharparenright}\ {\isasymin}\ S\ {\isasymlongrightarrow}\ G\ {\isasymin}\ S\ {\isasymand}\ \isactrlbold {\isasymnot}\ H\ {\isasymin}\ S{\isachardoublequoteclose}\isanewline
\ \ \ \ \ \ \isacommand{proof}\isamarkupfalse%
\ {\isacharparenleft}rule\ impI{\isacharparenright}\isanewline
\ \ \ \ \ \ \ \ \isacommand{assume}\isamarkupfalse%
\ {\isachardoublequoteopen}\isactrlbold {\isasymnot}{\isacharparenleft}G\ \isactrlbold {\isasymrightarrow}\ H{\isacharparenright}\ {\isasymin}\ S{\isachardoublequoteclose}\isanewline
\ \ \ \ \ \ \ \ \isacommand{have}\isamarkupfalse%
\ {\isachardoublequoteopen}Con\ {\isacharparenleft}\isactrlbold {\isasymnot}{\isacharparenleft}G\ \isactrlbold {\isasymrightarrow}\ H{\isacharparenright}{\isacharparenright}\ G\ {\isacharparenleft}\isactrlbold {\isasymnot}\ H{\isacharparenright}{\isachardoublequoteclose}\isanewline
\ \ \ \ \ \ \ \ \ \ \isacommand{by}\isamarkupfalse%
\ {\isacharparenleft}simp\ only{\isacharcolon}\ Con{\isachardot}intros{\isacharparenleft}{\isadigit{3}}{\isacharparenright}{\isacharparenright}\isanewline
\ \ \ \ \ \ \ \ \isacommand{have}\isamarkupfalse%
\ {\isachardoublequoteopen}Con\ {\isacharparenleft}\isactrlbold {\isasymnot}{\isacharparenleft}G\ \isactrlbold {\isasymrightarrow}\ H{\isacharparenright}{\isacharparenright}\ G\ {\isacharparenleft}\isactrlbold {\isasymnot}\ H{\isacharparenright}\ {\isasymlongrightarrow}\ \isactrlbold {\isasymnot}{\isacharparenleft}G\ \isactrlbold {\isasymrightarrow}\ H{\isacharparenright}\ {\isasymin}\ S\ {\isasymlongrightarrow}\ G\ {\isasymin}\ S\ {\isasymand}\ \isactrlbold {\isasymnot}\ H\ {\isasymin}\ S{\isachardoublequoteclose}\isanewline
\ \ \ \ \ \ \ \ \ \ \isacommand{using}\isamarkupfalse%
\ Con\ \isacommand{by}\isamarkupfalse%
\ {\isacharparenleft}iprover\ elim{\isacharcolon}\ allE{\isacharparenright}\isanewline
\ \ \ \ \ \ \ \ \isacommand{then}\isamarkupfalse%
\ \isacommand{have}\isamarkupfalse%
\ {\isachardoublequoteopen}\isactrlbold {\isasymnot}{\isacharparenleft}G\ \isactrlbold {\isasymrightarrow}\ H{\isacharparenright}\ {\isasymin}\ S\ {\isasymlongrightarrow}\ G\ {\isasymin}\ S\ {\isasymand}\ \isactrlbold {\isasymnot}\ H\ {\isasymin}\ S{\isachardoublequoteclose}\isanewline
\ \ \ \ \ \ \ \ \ \ \isacommand{using}\isamarkupfalse%
\ {\isacartoucheopen}Con\ {\isacharparenleft}\isactrlbold {\isasymnot}{\isacharparenleft}G\ \isactrlbold {\isasymrightarrow}\ H{\isacharparenright}{\isacharparenright}\ G\ {\isacharparenleft}\isactrlbold {\isasymnot}\ H{\isacharparenright}{\isacartoucheclose}\ \isacommand{by}\isamarkupfalse%
\ {\isacharparenleft}rule\ mp{\isacharparenright}\isanewline
\ \ \ \ \ \ \ \ \isacommand{thus}\isamarkupfalse%
\ {\isachardoublequoteopen}G\ {\isasymin}\ S\ {\isasymand}\ \isactrlbold {\isasymnot}\ H\ {\isasymin}\ S{\isachardoublequoteclose}\isanewline
\ \ \ \ \ \ \ \ \ \ \isacommand{using}\isamarkupfalse%
\ {\isacartoucheopen}\isactrlbold {\isasymnot}{\isacharparenleft}G\ \isactrlbold {\isasymrightarrow}\ H{\isacharparenright}\ {\isasymin}\ S{\isacartoucheclose}\ \isacommand{by}\isamarkupfalse%
\ {\isacharparenleft}rule\ mp{\isacharparenright}\isanewline
\ \ \ \ \ \ \isacommand{qed}\isamarkupfalse%
\isanewline
\ \ \ \ \isacommand{qed}\isamarkupfalse%
\isanewline
\ \ \ \ \isacommand{have}\isamarkupfalse%
\ A{\isacharcolon}{\isachardoublequoteopen}{\isasymbottom}\ {\isasymnotin}\ S\isanewline
\ \ \ \ {\isasymand}\ {\isacharparenleft}{\isasymforall}k{\isachardot}\ Atom\ k\ {\isasymin}\ S\ {\isasymlongrightarrow}\ \isactrlbold {\isasymnot}\ {\isacharparenleft}Atom\ k{\isacharparenright}\ {\isasymin}\ S\ {\isasymlongrightarrow}\ False{\isacharparenright}\isanewline
\ \ \ \ {\isasymand}\ {\isacharparenleft}{\isasymforall}G\ H{\isachardot}\ G\ \isactrlbold {\isasymand}\ H\ {\isasymin}\ S\ {\isasymlongrightarrow}\ G\ {\isasymin}\ S\ {\isasymand}\ H\ {\isasymin}\ S{\isacharparenright}\isanewline
\ \ \ \ {\isasymand}\ {\isacharparenleft}{\isasymforall}G\ H{\isachardot}\ G\ \isactrlbold {\isasymor}\ H\ {\isasymin}\ S\ {\isasymlongrightarrow}\ G\ {\isasymin}\ S\ {\isasymor}\ H\ {\isasymin}\ S{\isacharparenright}\isanewline
\ \ \ \ {\isasymand}\ {\isacharparenleft}{\isasymforall}G\ H{\isachardot}\ G\ \isactrlbold {\isasymrightarrow}\ H\ {\isasymin}\ S\ {\isasymlongrightarrow}\ \isactrlbold {\isasymnot}G\ {\isasymin}\ S\ {\isasymor}\ H\ {\isasymin}\ S{\isacharparenright}{\isachardoublequoteclose}\isanewline
\ \ \ \ \ \ \isacommand{using}\isamarkupfalse%
\ C{\isadigit{1}}\ C{\isadigit{2}}\ C{\isadigit{3}}\ C{\isadigit{4}}\ C{\isadigit{5}}\ \isacommand{by}\isamarkupfalse%
\ {\isacharparenleft}iprover\ intro{\isacharcolon}\ conjI{\isacharparenright}\isanewline
\ \ \ \ \isacommand{have}\isamarkupfalse%
\ B{\isacharcolon}{\isachardoublequoteopen}{\isacharparenleft}{\isasymforall}G{\isachardot}\ \isactrlbold {\isasymnot}\ {\isacharparenleft}\isactrlbold {\isasymnot}G{\isacharparenright}\ {\isasymin}\ S\ {\isasymlongrightarrow}\ G\ {\isasymin}\ S{\isacharparenright}\isanewline
\ \ \ \ {\isasymand}\ {\isacharparenleft}{\isasymforall}G\ H{\isachardot}\ \isactrlbold {\isasymnot}{\isacharparenleft}G\ \isactrlbold {\isasymand}\ H{\isacharparenright}\ {\isasymin}\ S\ {\isasymlongrightarrow}\ \isactrlbold {\isasymnot}\ G\ {\isasymin}\ S\ {\isasymor}\ \isactrlbold {\isasymnot}\ H\ {\isasymin}\ S{\isacharparenright}\isanewline
\ \ \ \ {\isasymand}\ {\isacharparenleft}{\isasymforall}G\ H{\isachardot}\ \isactrlbold {\isasymnot}{\isacharparenleft}G\ \isactrlbold {\isasymor}\ H{\isacharparenright}\ {\isasymin}\ S\ {\isasymlongrightarrow}\ \isactrlbold {\isasymnot}\ G\ {\isasymin}\ S\ {\isasymand}\ \isactrlbold {\isasymnot}\ H\ {\isasymin}\ S{\isacharparenright}\isanewline
\ \ \ \ {\isasymand}\ {\isacharparenleft}{\isasymforall}G\ H{\isachardot}\ \isactrlbold {\isasymnot}{\isacharparenleft}G\ \isactrlbold {\isasymrightarrow}\ H{\isacharparenright}\ {\isasymin}\ S\ {\isasymlongrightarrow}\ G\ {\isasymin}\ S\ {\isasymand}\ \isactrlbold {\isasymnot}\ H\ {\isasymin}\ S{\isacharparenright}{\isachardoublequoteclose}\isanewline
\ \ \ \ \ \ \isacommand{using}\isamarkupfalse%
\ C{\isadigit{6}}\ C{\isadigit{7}}\ C{\isadigit{8}}\ C{\isadigit{9}}\ \isacommand{by}\isamarkupfalse%
\ {\isacharparenleft}iprover\ intro{\isacharcolon}\ conjI{\isacharparenright}\isanewline
\ \ \ \ \isacommand{have}\isamarkupfalse%
\ {\isachardoublequoteopen}{\isacharparenleft}{\isasymbottom}\ {\isasymnotin}\ S\isanewline
\ \ \ \ {\isasymand}\ {\isacharparenleft}{\isasymforall}k{\isachardot}\ Atom\ k\ {\isasymin}\ S\ {\isasymlongrightarrow}\ \isactrlbold {\isasymnot}\ {\isacharparenleft}Atom\ k{\isacharparenright}\ {\isasymin}\ S\ {\isasymlongrightarrow}\ False{\isacharparenright}\isanewline
\ \ \ \ {\isasymand}\ {\isacharparenleft}{\isasymforall}G\ H{\isachardot}\ G\ \isactrlbold {\isasymand}\ H\ {\isasymin}\ S\ {\isasymlongrightarrow}\ G\ {\isasymin}\ S\ {\isasymand}\ H\ {\isasymin}\ S{\isacharparenright}\isanewline
\ \ \ \ {\isasymand}\ {\isacharparenleft}{\isasymforall}G\ H{\isachardot}\ G\ \isactrlbold {\isasymor}\ H\ {\isasymin}\ S\ {\isasymlongrightarrow}\ G\ {\isasymin}\ S\ {\isasymor}\ H\ {\isasymin}\ S{\isacharparenright}\isanewline
\ \ \ \ {\isasymand}\ {\isacharparenleft}{\isasymforall}G\ H{\isachardot}\ G\ \isactrlbold {\isasymrightarrow}\ H\ {\isasymin}\ S\ {\isasymlongrightarrow}\ \isactrlbold {\isasymnot}G\ {\isasymin}\ S\ {\isasymor}\ H\ {\isasymin}\ S{\isacharparenright}{\isacharparenright}\isanewline
\ \ \ \ {\isasymand}\ {\isacharparenleft}{\isacharparenleft}{\isasymforall}G{\isachardot}\ \isactrlbold {\isasymnot}\ {\isacharparenleft}\isactrlbold {\isasymnot}G{\isacharparenright}\ {\isasymin}\ S\ {\isasymlongrightarrow}\ G\ {\isasymin}\ S{\isacharparenright}\isanewline
\ \ \ \ {\isasymand}\ {\isacharparenleft}{\isasymforall}G\ H{\isachardot}\ \isactrlbold {\isasymnot}{\isacharparenleft}G\ \isactrlbold {\isasymand}\ H{\isacharparenright}\ {\isasymin}\ S\ {\isasymlongrightarrow}\ \isactrlbold {\isasymnot}\ G\ {\isasymin}\ S\ {\isasymor}\ \isactrlbold {\isasymnot}\ H\ {\isasymin}\ S{\isacharparenright}\isanewline
\ \ \ \ {\isasymand}\ {\isacharparenleft}{\isasymforall}G\ H{\isachardot}\ \isactrlbold {\isasymnot}{\isacharparenleft}G\ \isactrlbold {\isasymor}\ H{\isacharparenright}\ {\isasymin}\ S\ {\isasymlongrightarrow}\ \isactrlbold {\isasymnot}\ G\ {\isasymin}\ S\ {\isasymand}\ \isactrlbold {\isasymnot}\ H\ {\isasymin}\ S{\isacharparenright}\isanewline
\ \ \ \ {\isasymand}\ {\isacharparenleft}{\isasymforall}G\ H{\isachardot}\ \isactrlbold {\isasymnot}{\isacharparenleft}G\ \isactrlbold {\isasymrightarrow}\ H{\isacharparenright}\ {\isasymin}\ S\ {\isasymlongrightarrow}\ G\ {\isasymin}\ S\ {\isasymand}\ \isactrlbold {\isasymnot}\ H\ {\isasymin}\ S{\isacharparenright}{\isacharparenright}{\isachardoublequoteclose}\isanewline
\ \ \ \ \ \ \isacommand{using}\isamarkupfalse%
\ A\ B\ \isacommand{by}\isamarkupfalse%
\ {\isacharparenleft}rule\ conjI{\isacharparenright}\isanewline
\ \ \ \ \isacommand{thus}\isamarkupfalse%
\ {\isachardoublequoteopen}{\isasymbottom}\ {\isasymnotin}\ S\isanewline
\ \ \ \ {\isasymand}\ {\isacharparenleft}{\isasymforall}k{\isachardot}\ Atom\ k\ {\isasymin}\ S\ {\isasymlongrightarrow}\ \isactrlbold {\isasymnot}\ {\isacharparenleft}Atom\ k{\isacharparenright}\ {\isasymin}\ S\ {\isasymlongrightarrow}\ False{\isacharparenright}\isanewline
\ \ \ \ {\isasymand}\ {\isacharparenleft}{\isasymforall}G\ H{\isachardot}\ G\ \isactrlbold {\isasymand}\ H\ {\isasymin}\ S\ {\isasymlongrightarrow}\ G\ {\isasymin}\ S\ {\isasymand}\ H\ {\isasymin}\ S{\isacharparenright}\isanewline
\ \ \ \ {\isasymand}\ {\isacharparenleft}{\isasymforall}G\ H{\isachardot}\ G\ \isactrlbold {\isasymor}\ H\ {\isasymin}\ S\ {\isasymlongrightarrow}\ G\ {\isasymin}\ S\ {\isasymor}\ H\ {\isasymin}\ S{\isacharparenright}\isanewline
\ \ \ \ {\isasymand}\ {\isacharparenleft}{\isasymforall}G\ H{\isachardot}\ G\ \isactrlbold {\isasymrightarrow}\ H\ {\isasymin}\ S\ {\isasymlongrightarrow}\ \isactrlbold {\isasymnot}G\ {\isasymin}\ S\ {\isasymor}\ H\ {\isasymin}\ S{\isacharparenright}\isanewline
\ \ \ \ {\isasymand}\ {\isacharparenleft}{\isasymforall}G{\isachardot}\ \isactrlbold {\isasymnot}\ {\isacharparenleft}\isactrlbold {\isasymnot}G{\isacharparenright}\ {\isasymin}\ S\ {\isasymlongrightarrow}\ G\ {\isasymin}\ S{\isacharparenright}\isanewline
\ \ \ \ {\isasymand}\ {\isacharparenleft}{\isasymforall}G\ H{\isachardot}\ \isactrlbold {\isasymnot}{\isacharparenleft}G\ \isactrlbold {\isasymand}\ H{\isacharparenright}\ {\isasymin}\ S\ {\isasymlongrightarrow}\ \isactrlbold {\isasymnot}\ G\ {\isasymin}\ S\ {\isasymor}\ \isactrlbold {\isasymnot}\ H\ {\isasymin}\ S{\isacharparenright}\isanewline
\ \ \ \ {\isasymand}\ {\isacharparenleft}{\isasymforall}G\ H{\isachardot}\ \isactrlbold {\isasymnot}{\isacharparenleft}G\ \isactrlbold {\isasymor}\ H{\isacharparenright}\ {\isasymin}\ S\ {\isasymlongrightarrow}\ \isactrlbold {\isasymnot}\ G\ {\isasymin}\ S\ {\isasymand}\ \isactrlbold {\isasymnot}\ H\ {\isasymin}\ S{\isacharparenright}\isanewline
\ \ \ \ {\isasymand}\ {\isacharparenleft}{\isasymforall}G\ H{\isachardot}\ \isactrlbold {\isasymnot}{\isacharparenleft}G\ \isactrlbold {\isasymrightarrow}\ H{\isacharparenright}\ {\isasymin}\ S\ {\isasymlongrightarrow}\ G\ {\isasymin}\ S\ {\isasymand}\ \isactrlbold {\isasymnot}\ H\ {\isasymin}\ S{\isacharparenright}{\isachardoublequoteclose}\ \isanewline
\ \ \ \ \ \ \isacommand{by}\isamarkupfalse%
\ {\isacharparenleft}iprover\ intro{\isacharcolon}\ conj{\isacharunderscore}assoc{\isacharparenright}\isanewline
\ \ \isacommand{qed}\isamarkupfalse%
\isanewline
\ \ \isacommand{thus}\isamarkupfalse%
\ {\isachardoublequoteopen}Hintikka\ S{\isachardoublequoteclose}\isanewline
\ \ \ \ \isacommand{unfolding}\isamarkupfalse%
\ Hintikka{\isacharunderscore}def\ \isacommand{by}\isamarkupfalse%
\ this\isanewline
\isacommand{qed}\isamarkupfalse%
%
\endisatagproof
{\isafoldproof}%
%
\isadelimproof
%
\endisadelimproof
%
\begin{isamarkuptext}%
En conclusión, el lema completo se demuestra detalladamente en Isabelle/HOL como sigue.%
\end{isamarkuptext}\isamarkuptrue%
\isacommand{lemma}\isamarkupfalse%
\ {\isachardoublequoteopen}Hintikka\ S\ {\isacharequal}\ {\isacharparenleft}{\isasymbottom}\ {\isasymnotin}\ S\isanewline
{\isasymand}\ {\isacharparenleft}{\isasymforall}k{\isachardot}\ Atom\ k\ {\isasymin}\ S\ {\isasymlongrightarrow}\ \isactrlbold {\isasymnot}\ {\isacharparenleft}Atom\ k{\isacharparenright}\ {\isasymin}\ S\ {\isasymlongrightarrow}\ False{\isacharparenright}\isanewline
{\isasymand}\ {\isacharparenleft}{\isasymforall}F\ G\ H{\isachardot}\ Con\ F\ G\ H\ {\isasymlongrightarrow}\ F\ {\isasymin}\ S\ {\isasymlongrightarrow}\ G\ {\isasymin}\ S\ {\isasymand}\ H\ {\isasymin}\ S{\isacharparenright}\isanewline
{\isasymand}\ {\isacharparenleft}{\isasymforall}F\ G\ H{\isachardot}\ Dis\ F\ G\ H\ {\isasymlongrightarrow}\ F\ {\isasymin}\ S\ {\isasymlongrightarrow}\ G\ {\isasymin}\ S\ {\isasymor}\ H\ {\isasymin}\ S{\isacharparenright}{\isacharparenright}{\isachardoublequoteclose}\ \ \isanewline
%
\isadelimproof
%
\endisadelimproof
%
\isatagproof
\isacommand{proof}\isamarkupfalse%
\ {\isacharparenleft}rule\ iffI{\isacharparenright}\isanewline
\ \ \isacommand{assume}\isamarkupfalse%
\ {\isachardoublequoteopen}Hintikka\ S{\isachardoublequoteclose}\isanewline
\ \ \isacommand{thus}\isamarkupfalse%
\ {\isachardoublequoteopen}{\isacharparenleft}{\isasymbottom}\ {\isasymnotin}\ S\isanewline
\ \ {\isasymand}\ {\isacharparenleft}{\isasymforall}k{\isachardot}\ Atom\ k\ {\isasymin}\ S\ {\isasymlongrightarrow}\ \isactrlbold {\isasymnot}\ {\isacharparenleft}Atom\ k{\isacharparenright}\ {\isasymin}\ S\ {\isasymlongrightarrow}\ False{\isacharparenright}\isanewline
\ \ {\isasymand}\ {\isacharparenleft}{\isasymforall}F\ G\ H{\isachardot}\ Con\ F\ G\ H\ {\isasymlongrightarrow}\ F\ {\isasymin}\ S\ {\isasymlongrightarrow}\ G\ {\isasymin}\ S\ {\isasymand}\ H\ {\isasymin}\ S{\isacharparenright}\isanewline
\ \ {\isasymand}\ {\isacharparenleft}{\isasymforall}F\ G\ H{\isachardot}\ Dis\ F\ G\ H\ {\isasymlongrightarrow}\ F\ {\isasymin}\ S\ {\isasymlongrightarrow}\ G\ {\isasymin}\ S\ {\isasymor}\ H\ {\isasymin}\ S{\isacharparenright}{\isacharparenright}{\isachardoublequoteclose}\isanewline
\ \ \ \ \isacommand{by}\isamarkupfalse%
\ {\isacharparenleft}rule\ Hintikka{\isacharunderscore}alt{\isadigit{1}}{\isacharparenright}\isanewline
\isacommand{next}\isamarkupfalse%
\isanewline
\ \ \isacommand{assume}\isamarkupfalse%
\ {\isachardoublequoteopen}{\isacharparenleft}{\isasymbottom}\ {\isasymnotin}\ S\isanewline
\ \ {\isasymand}\ {\isacharparenleft}{\isasymforall}k{\isachardot}\ Atom\ k\ {\isasymin}\ S\ {\isasymlongrightarrow}\ \isactrlbold {\isasymnot}\ {\isacharparenleft}Atom\ k{\isacharparenright}\ {\isasymin}\ S\ {\isasymlongrightarrow}\ False{\isacharparenright}\isanewline
\ \ {\isasymand}\ {\isacharparenleft}{\isasymforall}F\ G\ H{\isachardot}\ Con\ F\ G\ H\ {\isasymlongrightarrow}\ F\ {\isasymin}\ S\ {\isasymlongrightarrow}\ G\ {\isasymin}\ S\ {\isasymand}\ H\ {\isasymin}\ S{\isacharparenright}\isanewline
\ \ {\isasymand}\ {\isacharparenleft}{\isasymforall}F\ G\ H{\isachardot}\ Dis\ F\ G\ H\ {\isasymlongrightarrow}\ F\ {\isasymin}\ S\ {\isasymlongrightarrow}\ G\ {\isasymin}\ S\ {\isasymor}\ H\ {\isasymin}\ S{\isacharparenright}{\isacharparenright}{\isachardoublequoteclose}\isanewline
\ \ \isacommand{thus}\isamarkupfalse%
\ {\isachardoublequoteopen}Hintikka\ S{\isachardoublequoteclose}\isanewline
\ \ \ \ \isacommand{by}\isamarkupfalse%
\ {\isacharparenleft}rule\ Hintikka{\isacharunderscore}alt{\isadigit{2}}{\isacharparenright}\isanewline
\isacommand{qed}\isamarkupfalse%
%
\endisatagproof
{\isafoldproof}%
%
\isadelimproof
%
\endisadelimproof
%
\begin{isamarkuptext}%
Por último, veamos su demostración automática.%
\end{isamarkuptext}\isamarkuptrue%
\isacommand{lemma}\isamarkupfalse%
\ Hintikka{\isacharunderscore}alt{\isacharcolon}\ {\isachardoublequoteopen}Hintikka\ S\ {\isacharequal}\ {\isacharparenleft}{\isasymbottom}\ {\isasymnotin}\ S\isanewline
{\isasymand}\ {\isacharparenleft}{\isasymforall}k{\isachardot}\ Atom\ k\ {\isasymin}\ S\ {\isasymlongrightarrow}\ \isactrlbold {\isasymnot}\ {\isacharparenleft}Atom\ k{\isacharparenright}\ {\isasymin}\ S\ {\isasymlongrightarrow}\ False{\isacharparenright}\isanewline
{\isasymand}\ {\isacharparenleft}{\isasymforall}F\ G\ H{\isachardot}\ Con\ F\ G\ H\ {\isasymlongrightarrow}\ F\ {\isasymin}\ S\ {\isasymlongrightarrow}\ G\ {\isasymin}\ S\ {\isasymand}\ H\ {\isasymin}\ S{\isacharparenright}\isanewline
{\isasymand}\ {\isacharparenleft}{\isasymforall}F\ G\ H{\isachardot}\ Dis\ F\ G\ H\ {\isasymlongrightarrow}\ F\ {\isasymin}\ S\ {\isasymlongrightarrow}\ G\ {\isasymin}\ S\ {\isasymor}\ H\ {\isasymin}\ S{\isacharparenright}{\isacharparenright}{\isachardoublequoteclose}\ \ \isanewline
%
\isadelimproof
\ \ %
\endisadelimproof
%
\isatagproof
\isacommand{apply}\isamarkupfalse%
{\isacharparenleft}simp\ add{\isacharcolon}\ Hintikka{\isacharunderscore}def\ con{\isacharunderscore}dis{\isacharunderscore}simps{\isacharparenright}\isanewline
\ \ \isacommand{apply}\isamarkupfalse%
{\isacharparenleft}rule\ iffI{\isacharparenright}\isanewline
\ \ \ \isacommand{subgoal}\isamarkupfalse%
\ \isacommand{by}\isamarkupfalse%
\ blast\isanewline
\ \ \isacommand{subgoal}\isamarkupfalse%
\ \isacommand{by}\isamarkupfalse%
\ safe\ metis{\isacharplus}\isanewline
\ \ \isacommand{done}\isamarkupfalse%
\isanewline
%
\endisatagproof
{\isafoldproof}%
%
\isadelimproof
%
\endisadelimproof
%
\isadelimtheory
%
\endisadelimtheory
%
\isatagtheory
%
\endisatagtheory
{\isafoldtheory}%
%
\isadelimtheory
%
\endisadelimtheory
%
\end{isabellebody}%
\endinput
%:%file=~/TFM/TFM/Notunif.thy%:%
%:%19=11%:%
%:%23=15%:%
%:%24=16%:%
%:%25=17%:%
%:%26=18%:%
%:%27=19%:%
%:%28=20%:%
%:%29=21%:%
%:%30=22%:%
%:%31=23%:%
%:%32=24%:%
%:%33=25%:%
%:%34=26%:%
%:%35=27%:%
%:%36=28%:%
%:%37=29%:%
%:%38=30%:%
%:%39=31%:%
%:%40=32%:%
%:%41=33%:%
%:%42=34%:%
%:%44=36%:%
%:%45=36%:%
%:%47=38%:%
%:%48=39%:%
%:%50=41%:%
%:%51=41%:%
%:%54=42%:%
%:%58=42%:%
%:%59=42%:%
%:%64=42%:%
%:%67=43%:%
%:%68=44%:%
%:%69=44%:%
%:%72=45%:%
%:%76=45%:%
%:%77=45%:%
%:%82=45%:%
%:%85=46%:%
%:%86=47%:%
%:%87=47%:%
%:%90=48%:%
%:%94=48%:%
%:%95=48%:%
%:%100=48%:%
%:%103=49%:%
%:%104=50%:%
%:%105=50%:%
%:%108=51%:%
%:%112=51%:%
%:%113=51%:%
%:%118=51%:%
%:%121=52%:%
%:%122=53%:%
%:%123=53%:%
%:%126=54%:%
%:%130=54%:%
%:%131=54%:%
%:%136=54%:%
%:%139=55%:%
%:%140=56%:%
%:%141=56%:%
%:%144=57%:%
%:%148=57%:%
%:%149=57%:%
%:%154=57%:%
%:%157=58%:%
%:%158=59%:%
%:%159=59%:%
%:%162=60%:%
%:%166=60%:%
%:%167=60%:%
%:%172=60%:%
%:%175=61%:%
%:%176=62%:%
%:%177=62%:%
%:%180=63%:%
%:%184=63%:%
%:%185=63%:%
%:%190=63%:%
%:%193=64%:%
%:%194=65%:%
%:%195=65%:%
%:%198=66%:%
%:%202=66%:%
%:%203=66%:%
%:%208=66%:%
%:%211=67%:%
%:%212=68%:%
%:%213=68%:%
%:%216=69%:%
%:%220=69%:%
%:%221=69%:%
%:%230=71%:%
%:%231=72%:%
%:%233=74%:%
%:%234=74%:%
%:%237=75%:%
%:%241=75%:%
%:%242=75%:%
%:%247=75%:%
%:%250=76%:%
%:%251=77%:%
%:%252=77%:%
%:%255=78%:%
%:%259=78%:%
%:%260=78%:%
%:%269=80%:%
%:%270=81%:%
%:%271=82%:%
%:%272=83%:%
%:%273=84%:%
%:%274=85%:%
%:%275=86%:%
%:%276=87%:%
%:%277=88%:%
%:%278=89%:%
%:%279=90%:%
%:%280=91%:%
%:%281=92%:%
%:%282=93%:%
%:%283=94%:%
%:%284=95%:%
%:%285=96%:%
%:%286=97%:%
%:%287=98%:%
%:%288=99%:%
%:%289=100%:%
%:%290=101%:%
%:%291=102%:%
%:%292=103%:%
%:%293=104%:%
%:%294=105%:%
%:%296=107%:%
%:%297=107%:%
%:%298=108%:%
%:%299=109%:%
%:%300=110%:%
%:%301=111%:%
%:%303=113%:%
%:%304=114%:%
%:%305=115%:%
%:%306=116%:%
%:%307=117%:%
%:%310=117%:%
%:%311=118%:%
%:%312=119%:%
%:%313=120%:%
%:%314=121%:%
%:%315=122%:%
%:%316=123%:%
%:%317=124%:%
%:%318=125%:%
%:%319=126%:%
%:%320=127%:%
%:%321=128%:%
%:%322=129%:%
%:%323=130%:%
%:%324=131%:%
%:%325=132%:%
%:%326=133%:%
%:%327=134%:%
%:%328=135%:%
%:%329=136%:%
%:%330=137%:%
%:%331=138%:%
%:%332=139%:%
%:%333=140%:%
%:%335=142%:%
%:%336=142%:%
%:%337=143%:%
%:%338=144%:%
%:%339=145%:%
%:%340=146%:%
%:%342=148%:%
%:%343=149%:%
%:%344=150%:%
%:%345=151%:%
%:%348=151%:%
%:%349=152%:%
%:%350=153%:%
%:%351=154%:%
%:%352=155%:%
%:%353=156%:%
%:%354=157%:%
%:%355=158%:%
%:%356=159%:%
%:%357=160%:%
%:%358=161%:%
%:%359=162%:%
%:%361=164%:%
%:%362=164%:%
%:%365=165%:%
%:%369=165%:%
%:%370=165%:%
%:%371=165%:%
%:%380=167%:%
%:%381=168%:%
%:%382=169%:%
%:%384=171%:%
%:%385=171%:%
%:%388=172%:%
%:%392=172%:%
%:%393=172%:%
%:%398=172%:%
%:%401=173%:%
%:%402=174%:%
%:%403=174%:%
%:%406=175%:%
%:%410=175%:%
%:%411=175%:%
%:%420=177%:%
%:%421=178%:%
%:%423=180%:%
%:%424=180%:%
%:%427=181%:%
%:%431=181%:%
%:%432=181%:%
%:%441=183%:%
%:%442=184%:%
%:%444=186%:%
%:%445=186%:%
%:%446=187%:%
%:%449=190%:%
%:%450=191%:%
%:%453=194%:%
%:%456=195%:%
%:%460=195%:%
%:%461=195%:%
%:%470=197%:%
%:%471=198%:%
%:%472=199%:%
%:%473=200%:%
%:%474=201%:%
%:%475=202%:%
%:%476=203%:%
%:%477=204%:%
%:%478=205%:%
%:%479=206%:%
%:%480=207%:%
%:%481=208%:%
%:%482=209%:%
%:%483=210%:%
%:%484=211%:%
%:%485=212%:%
%:%486=213%:%
%:%487=214%:%
%:%488=215%:%
%:%489=216%:%
%:%490=217%:%
%:%491=218%:%
%:%493=220%:%
%:%494=220%:%
%:%497=223%:%
%:%500=224%:%
%:%504=224%:%
%:%514=226%:%
%:%515=227%:%
%:%516=228%:%
%:%517=229%:%
%:%518=230%:%
%:%519=231%:%
%:%520=232%:%
%:%521=233%:%
%:%522=234%:%
%:%523=235%:%
%:%524=236%:%
%:%525=237%:%
%:%526=238%:%
%:%527=239%:%
%:%528=240%:%
%:%529=241%:%
%:%530=242%:%
%:%531=243%:%
%:%532=244%:%
%:%533=245%:%
%:%534=246%:%
%:%535=247%:%
%:%536=248%:%
%:%537=249%:%
%:%538=250%:%
%:%539=251%:%
%:%540=252%:%
%:%541=253%:%
%:%542=254%:%
%:%543=255%:%
%:%544=256%:%
%:%545=257%:%
%:%546=258%:%
%:%547=259%:%
%:%548=260%:%
%:%549=261%:%
%:%550=262%:%
%:%551=263%:%
%:%552=264%:%
%:%553=265%:%
%:%554=266%:%
%:%555=267%:%
%:%556=268%:%
%:%557=269%:%
%:%558=270%:%
%:%559=271%:%
%:%560=272%:%
%:%561=273%:%
%:%562=274%:%
%:%563=275%:%
%:%564=276%:%
%:%565=277%:%
%:%566=278%:%
%:%567=279%:%
%:%568=280%:%
%:%569=281%:%
%:%570=282%:%
%:%571=283%:%
%:%572=284%:%
%:%573=285%:%
%:%574=286%:%
%:%575=287%:%
%:%576=288%:%
%:%577=289%:%
%:%578=290%:%
%:%579=291%:%
%:%580=292%:%
%:%581=293%:%
%:%582=294%:%
%:%583=295%:%
%:%584=296%:%
%:%585=297%:%
%:%586=298%:%
%:%587=299%:%
%:%588=300%:%
%:%589=301%:%
%:%590=302%:%
%:%591=303%:%
%:%592=304%:%
%:%593=305%:%
%:%594=306%:%
%:%595=307%:%
%:%596=308%:%
%:%597=309%:%
%:%598=310%:%
%:%599=311%:%
%:%600=312%:%
%:%601=313%:%
%:%602=314%:%
%:%603=315%:%
%:%604=316%:%
%:%605=317%:%
%:%606=318%:%
%:%607=319%:%
%:%608=320%:%
%:%609=321%:%
%:%610=322%:%
%:%611=323%:%
%:%612=324%:%
%:%613=325%:%
%:%614=326%:%
%:%615=327%:%
%:%616=328%:%
%:%617=329%:%
%:%618=330%:%
%:%619=331%:%
%:%620=332%:%
%:%621=333%:%
%:%622=334%:%
%:%623=335%:%
%:%624=336%:%
%:%625=337%:%
%:%626=338%:%
%:%627=339%:%
%:%628=340%:%
%:%629=341%:%
%:%630=342%:%
%:%631=343%:%
%:%632=344%:%
%:%633=345%:%
%:%634=346%:%
%:%635=347%:%
%:%636=348%:%
%:%637=349%:%
%:%638=350%:%
%:%639=351%:%
%:%640=352%:%
%:%641=353%:%
%:%642=354%:%
%:%643=355%:%
%:%644=356%:%
%:%645=357%:%
%:%646=358%:%
%:%647=359%:%
%:%648=360%:%
%:%649=361%:%
%:%650=362%:%
%:%651=363%:%
%:%652=364%:%
%:%653=365%:%
%:%654=366%:%
%:%655=367%:%
%:%656=368%:%
%:%657=369%:%
%:%658=370%:%
%:%659=371%:%
%:%660=372%:%
%:%661=373%:%
%:%662=374%:%
%:%663=375%:%
%:%664=376%:%
%:%665=377%:%
%:%666=378%:%
%:%667=379%:%
%:%668=380%:%
%:%669=381%:%
%:%670=382%:%
%:%671=383%:%
%:%672=384%:%
%:%673=385%:%
%:%674=386%:%
%:%675=387%:%
%:%676=388%:%
%:%678=390%:%
%:%679=390%:%
%:%680=391%:%
%:%683=394%:%
%:%684=395%:%
%:%691=396%:%
%:%692=396%:%
%:%693=397%:%
%:%694=397%:%
%:%695=398%:%
%:%696=398%:%
%:%697=398%:%
%:%700=401%:%
%:%701=402%:%
%:%702=402%:%
%:%703=403%:%
%:%704=403%:%
%:%705=404%:%
%:%706=404%:%
%:%707=405%:%
%:%708=405%:%
%:%709=406%:%
%:%710=406%:%
%:%711=407%:%
%:%712=407%:%
%:%713=407%:%
%:%714=408%:%
%:%715=408%:%
%:%716=409%:%
%:%717=409%:%
%:%718=409%:%
%:%719=410%:%
%:%720=410%:%
%:%721=411%:%
%:%722=411%:%
%:%724=413%:%
%:%725=414%:%
%:%726=414%:%
%:%727=415%:%
%:%728=415%:%
%:%729=416%:%
%:%730=416%:%
%:%731=417%:%
%:%732=417%:%
%:%733=418%:%
%:%734=418%:%
%:%735=418%:%
%:%736=419%:%
%:%737=419%:%
%:%738=419%:%
%:%739=420%:%
%:%740=420%:%
%:%741=421%:%
%:%742=421%:%
%:%743=422%:%
%:%744=422%:%
%:%745=422%:%
%:%746=423%:%
%:%747=423%:%
%:%748=424%:%
%:%749=424%:%
%:%750=424%:%
%:%751=425%:%
%:%752=425%:%
%:%753=426%:%
%:%754=426%:%
%:%755=426%:%
%:%756=427%:%
%:%757=427%:%
%:%758=428%:%
%:%759=428%:%
%:%760=428%:%
%:%761=429%:%
%:%762=429%:%
%:%763=430%:%
%:%764=430%:%
%:%765=431%:%
%:%766=432%:%
%:%767=432%:%
%:%768=433%:%
%:%769=433%:%
%:%770=434%:%
%:%771=434%:%
%:%772=435%:%
%:%773=435%:%
%:%774=436%:%
%:%775=436%:%
%:%776=436%:%
%:%777=437%:%
%:%778=437%:%
%:%779=438%:%
%:%780=438%:%
%:%781=438%:%
%:%782=439%:%
%:%783=439%:%
%:%784=440%:%
%:%785=440%:%
%:%786=440%:%
%:%787=441%:%
%:%788=441%:%
%:%789=442%:%
%:%790=442%:%
%:%791=442%:%
%:%792=443%:%
%:%793=443%:%
%:%794=444%:%
%:%795=444%:%
%:%796=444%:%
%:%797=445%:%
%:%798=445%:%
%:%799=446%:%
%:%800=446%:%
%:%801=447%:%
%:%802=447%:%
%:%803=447%:%
%:%804=448%:%
%:%805=448%:%
%:%806=449%:%
%:%807=449%:%
%:%808=450%:%
%:%809=450%:%
%:%810=450%:%
%:%811=451%:%
%:%812=451%:%
%:%813=452%:%
%:%814=452%:%
%:%815=452%:%
%:%816=453%:%
%:%817=453%:%
%:%818=453%:%
%:%819=454%:%
%:%820=454%:%
%:%821=455%:%
%:%822=455%:%
%:%823=455%:%
%:%824=456%:%
%:%825=456%:%
%:%826=457%:%
%:%827=457%:%
%:%828=457%:%
%:%829=458%:%
%:%830=458%:%
%:%831=459%:%
%:%832=459%:%
%:%833=460%:%
%:%834=460%:%
%:%835=461%:%
%:%836=461%:%
%:%837=462%:%
%:%838=462%:%
%:%839=463%:%
%:%840=463%:%
%:%841=464%:%
%:%851=466%:%
%:%852=467%:%
%:%853=468%:%
%:%854=469%:%
%:%855=470%:%
%:%856=471%:%
%:%858=473%:%
%:%859=473%:%
%:%860=474%:%
%:%863=477%:%
%:%864=478%:%
%:%871=479%:%
%:%872=479%:%
%:%873=480%:%
%:%874=480%:%
%:%875=481%:%
%:%876=481%:%
%:%877=481%:%
%:%880=484%:%
%:%881=485%:%
%:%882=485%:%
%:%883=486%:%
%:%884=486%:%
%:%885=487%:%
%:%886=487%:%
%:%887=488%:%
%:%888=488%:%
%:%889=489%:%
%:%890=489%:%
%:%891=490%:%
%:%892=490%:%
%:%893=490%:%
%:%894=491%:%
%:%895=491%:%
%:%896=492%:%
%:%897=492%:%
%:%898=492%:%
%:%899=493%:%
%:%900=493%:%
%:%901=494%:%
%:%902=494%:%
%:%904=496%:%
%:%905=497%:%
%:%906=497%:%
%:%907=498%:%
%:%908=498%:%
%:%909=499%:%
%:%910=499%:%
%:%911=500%:%
%:%912=500%:%
%:%913=501%:%
%:%914=501%:%
%:%915=501%:%
%:%916=502%:%
%:%917=502%:%
%:%918=503%:%
%:%919=503%:%
%:%920=503%:%
%:%921=504%:%
%:%922=504%:%
%:%923=505%:%
%:%924=505%:%
%:%925=505%:%
%:%926=506%:%
%:%927=506%:%
%:%928=507%:%
%:%929=507%:%
%:%930=507%:%
%:%931=508%:%
%:%932=508%:%
%:%933=509%:%
%:%934=509%:%
%:%935=509%:%
%:%936=510%:%
%:%937=510%:%
%:%938=511%:%
%:%939=511%:%
%:%940=511%:%
%:%941=512%:%
%:%942=512%:%
%:%943=513%:%
%:%944=513%:%
%:%945=514%:%
%:%946=515%:%
%:%947=515%:%
%:%948=516%:%
%:%949=516%:%
%:%950=517%:%
%:%951=517%:%
%:%952=518%:%
%:%953=518%:%
%:%954=519%:%
%:%955=519%:%
%:%956=519%:%
%:%957=520%:%
%:%958=520%:%
%:%959=521%:%
%:%960=521%:%
%:%961=521%:%
%:%962=522%:%
%:%963=522%:%
%:%964=523%:%
%:%965=523%:%
%:%966=523%:%
%:%967=524%:%
%:%968=524%:%
%:%969=525%:%
%:%970=525%:%
%:%971=525%:%
%:%972=526%:%
%:%973=526%:%
%:%974=527%:%
%:%975=527%:%
%:%976=527%:%
%:%977=528%:%
%:%978=528%:%
%:%979=529%:%
%:%980=529%:%
%:%981=529%:%
%:%982=530%:%
%:%983=530%:%
%:%984=531%:%
%:%985=531%:%
%:%986=532%:%
%:%987=532%:%
%:%988=532%:%
%:%989=533%:%
%:%990=533%:%
%:%991=534%:%
%:%992=534%:%
%:%993=535%:%
%:%994=535%:%
%:%995=535%:%
%:%996=536%:%
%:%997=536%:%
%:%998=537%:%
%:%999=537%:%
%:%1000=537%:%
%:%1001=538%:%
%:%1002=538%:%
%:%1003=538%:%
%:%1004=539%:%
%:%1005=539%:%
%:%1006=540%:%
%:%1007=540%:%
%:%1008=540%:%
%:%1009=541%:%
%:%1010=541%:%
%:%1011=542%:%
%:%1012=542%:%
%:%1013=542%:%
%:%1014=543%:%
%:%1015=543%:%
%:%1016=544%:%
%:%1017=544%:%
%:%1018=545%:%
%:%1019=545%:%
%:%1020=546%:%
%:%1021=546%:%
%:%1022=547%:%
%:%1023=547%:%
%:%1024=548%:%
%:%1025=548%:%
%:%1026=549%:%
%:%1036=551%:%
%:%1037=552%:%
%:%1039=554%:%
%:%1040=554%:%
%:%1041=555%:%
%:%1042=556%:%
%:%1045=559%:%
%:%1052=560%:%
%:%1053=560%:%
%:%1054=561%:%
%:%1055=561%:%
%:%1063=569%:%
%:%1064=570%:%
%:%1065=570%:%
%:%1066=570%:%
%:%1067=571%:%
%:%1068=571%:%
%:%1069=571%:%
%:%1070=572%:%
%:%1071=572%:%
%:%1072=573%:%
%:%1073=573%:%
%:%1074=574%:%
%:%1075=574%:%
%:%1076=574%:%
%:%1077=575%:%
%:%1078=575%:%
%:%1079=576%:%
%:%1080=576%:%
%:%1081=577%:%
%:%1082=577%:%
%:%1083=578%:%
%:%1084=578%:%
%:%1085=579%:%
%:%1086=579%:%
%:%1087=579%:%
%:%1088=580%:%
%:%1089=580%:%
%:%1090=581%:%
%:%1091=581%:%
%:%1092=581%:%
%:%1093=582%:%
%:%1094=582%:%
%:%1095=583%:%
%:%1096=583%:%
%:%1097=583%:%
%:%1098=584%:%
%:%1099=584%:%
%:%1100=585%:%
%:%1101=585%:%
%:%1102=585%:%
%:%1103=586%:%
%:%1104=586%:%
%:%1107=589%:%
%:%1108=590%:%
%:%1109=590%:%
%:%1110=590%:%
%:%1111=591%:%
%:%1112=591%:%
%:%1113=592%:%
%:%1114=592%:%
%:%1115=593%:%
%:%1116=593%:%
%:%1117=594%:%
%:%1118=594%:%
%:%1119=595%:%
%:%1120=595%:%
%:%1121=596%:%
%:%1122=596%:%
%:%1123=597%:%
%:%1124=597%:%
%:%1125=598%:%
%:%1126=598%:%
%:%1127=598%:%
%:%1128=599%:%
%:%1129=599%:%
%:%1130=600%:%
%:%1131=600%:%
%:%1132=600%:%
%:%1133=601%:%
%:%1134=601%:%
%:%1135=602%:%
%:%1136=602%:%
%:%1137=602%:%
%:%1138=603%:%
%:%1139=603%:%
%:%1140=604%:%
%:%1141=604%:%
%:%1142=604%:%
%:%1143=605%:%
%:%1144=605%:%
%:%1147=608%:%
%:%1148=609%:%
%:%1149=609%:%
%:%1150=609%:%
%:%1151=610%:%
%:%1152=610%:%
%:%1153=611%:%
%:%1154=611%:%
%:%1155=612%:%
%:%1156=612%:%
%:%1157=613%:%
%:%1158=613%:%
%:%1161=616%:%
%:%1162=617%:%
%:%1163=617%:%
%:%1164=617%:%
%:%1165=618%:%
%:%1175=620%:%
%:%1176=621%:%
%:%1178=623%:%
%:%1179=623%:%
%:%1180=624%:%
%:%1183=627%:%
%:%1184=628%:%
%:%1191=629%:%
%:%1192=629%:%
%:%1193=630%:%
%:%1194=630%:%
%:%1195=631%:%
%:%1196=631%:%
%:%1197=631%:%
%:%1198=632%:%
%:%1199=632%:%
%:%1200=633%:%
%:%1201=633%:%
%:%1202=633%:%
%:%1203=634%:%
%:%1204=634%:%
%:%1212=642%:%
%:%1213=643%:%
%:%1214=643%:%
%:%1215=644%:%
%:%1216=644%:%
%:%1217=645%:%
%:%1218=645%:%
%:%1219=645%:%
%:%1220=646%:%
%:%1221=646%:%
%:%1222=647%:%
%:%1223=647%:%
%:%1224=647%:%
%:%1225=648%:%
%:%1226=648%:%
%:%1227=649%:%
%:%1228=649%:%
%:%1229=650%:%
%:%1230=650%:%
%:%1231=651%:%
%:%1232=651%:%
%:%1233=652%:%
%:%1234=652%:%
%:%1235=653%:%
%:%1236=653%:%
%:%1237=654%:%
%:%1238=654%:%
%:%1239=655%:%
%:%1240=655%:%
%:%1241=656%:%
%:%1242=656%:%
%:%1243=657%:%
%:%1244=657%:%
%:%1245=657%:%
%:%1246=658%:%
%:%1247=658%:%
%:%1248=658%:%
%:%1249=659%:%
%:%1250=659%:%
%:%1251=659%:%
%:%1252=660%:%
%:%1253=660%:%
%:%1254=661%:%
%:%1255=661%:%
%:%1256=661%:%
%:%1257=662%:%
%:%1258=662%:%
%:%1259=663%:%
%:%1260=663%:%
%:%1261=664%:%
%:%1262=664%:%
%:%1263=665%:%
%:%1264=665%:%
%:%1265=666%:%
%:%1266=666%:%
%:%1267=667%:%
%:%1268=667%:%
%:%1269=668%:%
%:%1270=668%:%
%:%1271=669%:%
%:%1272=669%:%
%:%1273=670%:%
%:%1274=670%:%
%:%1275=671%:%
%:%1276=671%:%
%:%1277=672%:%
%:%1278=672%:%
%:%1279=673%:%
%:%1280=673%:%
%:%1281=673%:%
%:%1282=674%:%
%:%1283=674%:%
%:%1284=674%:%
%:%1285=675%:%
%:%1286=675%:%
%:%1287=675%:%
%:%1288=676%:%
%:%1289=676%:%
%:%1290=677%:%
%:%1291=677%:%
%:%1292=677%:%
%:%1293=678%:%
%:%1294=678%:%
%:%1295=679%:%
%:%1296=679%:%
%:%1297=680%:%
%:%1298=680%:%
%:%1299=681%:%
%:%1300=681%:%
%:%1301=682%:%
%:%1302=682%:%
%:%1303=683%:%
%:%1304=683%:%
%:%1305=684%:%
%:%1306=684%:%
%:%1307=685%:%
%:%1308=685%:%
%:%1309=686%:%
%:%1310=686%:%
%:%1311=687%:%
%:%1312=687%:%
%:%1313=688%:%
%:%1314=688%:%
%:%1315=689%:%
%:%1316=689%:%
%:%1317=689%:%
%:%1318=690%:%
%:%1319=690%:%
%:%1320=690%:%
%:%1321=691%:%
%:%1322=691%:%
%:%1323=691%:%
%:%1324=692%:%
%:%1325=692%:%
%:%1326=693%:%
%:%1327=693%:%
%:%1328=693%:%
%:%1329=694%:%
%:%1330=694%:%
%:%1331=695%:%
%:%1332=695%:%
%:%1333=696%:%
%:%1334=696%:%
%:%1335=697%:%
%:%1336=697%:%
%:%1337=698%:%
%:%1338=698%:%
%:%1339=699%:%
%:%1340=699%:%
%:%1341=700%:%
%:%1342=700%:%
%:%1343=701%:%
%:%1344=701%:%
%:%1345=702%:%
%:%1346=702%:%
%:%1347=703%:%
%:%1348=703%:%
%:%1349=704%:%
%:%1350=704%:%
%:%1351=705%:%
%:%1352=705%:%
%:%1353=705%:%
%:%1354=706%:%
%:%1355=706%:%
%:%1356=706%:%
%:%1357=707%:%
%:%1358=707%:%
%:%1359=707%:%
%:%1360=708%:%
%:%1361=708%:%
%:%1362=708%:%
%:%1363=709%:%
%:%1364=709%:%
%:%1365=709%:%
%:%1366=710%:%
%:%1367=710%:%
%:%1368=711%:%
%:%1369=711%:%
%:%1370=712%:%
%:%1371=712%:%
%:%1372=713%:%
%:%1373=713%:%
%:%1374=714%:%
%:%1375=714%:%
%:%1376=715%:%
%:%1377=715%:%
%:%1378=716%:%
%:%1379=716%:%
%:%1380=717%:%
%:%1381=717%:%
%:%1382=718%:%
%:%1383=718%:%
%:%1384=719%:%
%:%1385=719%:%
%:%1386=720%:%
%:%1387=720%:%
%:%1388=721%:%
%:%1389=721%:%
%:%1390=722%:%
%:%1391=722%:%
%:%1392=723%:%
%:%1393=723%:%
%:%1394=723%:%
%:%1395=724%:%
%:%1396=724%:%
%:%1397=724%:%
%:%1398=725%:%
%:%1399=725%:%
%:%1400=725%:%
%:%1401=726%:%
%:%1402=726%:%
%:%1403=727%:%
%:%1404=727%:%
%:%1405=727%:%
%:%1406=728%:%
%:%1407=728%:%
%:%1408=729%:%
%:%1409=729%:%
%:%1410=730%:%
%:%1411=730%:%
%:%1412=731%:%
%:%1413=731%:%
%:%1414=732%:%
%:%1415=732%:%
%:%1416=733%:%
%:%1417=733%:%
%:%1418=734%:%
%:%1419=734%:%
%:%1420=735%:%
%:%1421=735%:%
%:%1422=736%:%
%:%1423=736%:%
%:%1424=737%:%
%:%1425=737%:%
%:%1426=738%:%
%:%1427=738%:%
%:%1428=739%:%
%:%1429=739%:%
%:%1430=739%:%
%:%1431=740%:%
%:%1432=740%:%
%:%1433=740%:%
%:%1434=741%:%
%:%1435=741%:%
%:%1436=741%:%
%:%1437=742%:%
%:%1438=742%:%
%:%1439=743%:%
%:%1440=743%:%
%:%1441=743%:%
%:%1442=744%:%
%:%1443=744%:%
%:%1444=745%:%
%:%1445=745%:%
%:%1446=746%:%
%:%1447=746%:%
%:%1448=747%:%
%:%1449=747%:%
%:%1450=748%:%
%:%1451=748%:%
%:%1452=749%:%
%:%1453=749%:%
%:%1454=750%:%
%:%1455=750%:%
%:%1456=751%:%
%:%1457=751%:%
%:%1458=752%:%
%:%1459=752%:%
%:%1460=753%:%
%:%1461=753%:%
%:%1462=754%:%
%:%1463=754%:%
%:%1464=755%:%
%:%1465=755%:%
%:%1466=755%:%
%:%1467=756%:%
%:%1468=756%:%
%:%1469=756%:%
%:%1470=757%:%
%:%1471=757%:%
%:%1472=757%:%
%:%1473=758%:%
%:%1474=758%:%
%:%1475=759%:%
%:%1476=759%:%
%:%1477=759%:%
%:%1478=760%:%
%:%1479=760%:%
%:%1480=761%:%
%:%1481=761%:%
%:%1482=762%:%
%:%1483=762%:%
%:%1487=766%:%
%:%1488=767%:%
%:%1489=767%:%
%:%1490=767%:%
%:%1491=768%:%
%:%1492=768%:%
%:%1495=771%:%
%:%1496=772%:%
%:%1497=772%:%
%:%1498=772%:%
%:%1499=773%:%
%:%1500=773%:%
%:%1508=781%:%
%:%1509=782%:%
%:%1510=782%:%
%:%1511=782%:%
%:%1512=783%:%
%:%1513=783%:%
%:%1521=791%:%
%:%1522=792%:%
%:%1523=792%:%
%:%1524=793%:%
%:%1525=793%:%
%:%1526=794%:%
%:%1527=794%:%
%:%1528=795%:%
%:%1529=795%:%
%:%1530=795%:%
%:%1531=796%:%
%:%1541=798%:%
%:%1543=800%:%
%:%1544=800%:%
%:%1547=803%:%
%:%1554=804%:%
%:%1555=804%:%
%:%1556=805%:%
%:%1557=805%:%
%:%1558=806%:%
%:%1559=806%:%
%:%1562=809%:%
%:%1563=810%:%
%:%1564=810%:%
%:%1565=811%:%
%:%1566=811%:%
%:%1567=812%:%
%:%1568=812%:%
%:%1571=815%:%
%:%1572=816%:%
%:%1573=816%:%
%:%1574=817%:%
%:%1575=817%:%
%:%1576=818%:%
%:%1586=820%:%
%:%1588=822%:%
%:%1589=822%:%
%:%1592=825%:%
%:%1595=826%:%
%:%1599=826%:%
%:%1600=826%:%
%:%1601=827%:%
%:%1602=827%:%
%:%1603=828%:%
%:%1604=828%:%
%:%1605=828%:%
%:%1606=829%:%
%:%1607=829%:%
%:%1608=829%:%
%:%1609=830%:%
%:%1610=830%:%