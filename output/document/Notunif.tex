%
\begin{isabellebody}%
\setisabellecontext{Notunif}%
%
\isadelimtheory
%
\endisadelimtheory
%
\isatagtheory
%
\endisatagtheory
{\isafoldtheory}%
%
\isadelimtheory
%
\endisadelimtheory
%
\begin{isamarkuptext}%
\comentario{Localización de sello.png.}
\comentario{Cambiar los directores}
\comentario{Introducción. Mirar fitting p. 53 y 54}%
\end{isamarkuptext}\isamarkuptrue%
%
\begin{isamarkuptext}%
En esta sección introduciremos la notación uniforme inicialmente 
  desarrollada por \isa{R{\isachardot}\ M{\isachardot}\ Smullyan} (añadir referencia bibliográfica). La finalidad
  de dicha notación es reducir el número de casos a considerar sobre la estructura de 
  las fórmulas al clasificar éstas en dos categorías, facilitando las demostraciones
  y métodos empleados en adelante.

  \comentario{Añadir referencia bibliográfica.}

  De este modo, las fórmulas proposicionales pueden ser de dos tipos: aquellas que 
  de tipo conjuntivo (las fórmulas \isa{{\isasymalpha}}) y las de tipo disyuntivo (las fórmulas \isa{{\isasymbeta}}). 
  Cada fórmula de tipo \isa{{\isasymalpha}}, o \isa{{\isasymbeta}} respectivamente, tiene asociada sus  
  dos componentes \isa{{\isasymalpha}\isactrlsub {\isadigit{1}}} y \isa{{\isasymalpha}\isactrlsub {\isadigit{2}}}, o \isa{{\isasymbeta}\isactrlsub {\isadigit{1}}} y \isa{{\isasymbeta}\isactrlsub {\isadigit{2}}} respectivamente. Para justificar dicha clasificación,
  introduzcamos inicialmente la definición de fórmulas semánticamente equivalentes.

  \begin{definicion}
    Dos fórmulas son \isa{semánticamente\ equivalentes} si tienen el mismo valor para toda 
    interpretación.
  \end{definicion}

  En Isabelle podemos formalizar la definición de la siguiente manera.%
\end{isamarkuptext}\isamarkuptrue%
\isacommand{definition}\isamarkupfalse%
\ {\isachardoublequoteopen}semanticEq\ F\ G\ {\isasymequiv}\ {\isasymforall}{\isasymA}{\isachardot}\ {\isacharparenleft}{\isasymA}\ {\isasymTurnstile}\ F{\isacharparenright}\ {\isasymlongleftrightarrow}\ {\isacharparenleft}{\isasymA}\ {\isasymTurnstile}\ G{\isacharparenright}{\isachardoublequoteclose}%
\begin{isamarkuptext}%
De este modo, según la definición del valor de verdad de una fórmula proposicional en una 
  interpretación dada, podemos ver los siguientes ejemplos de fórmulas semánticamente equivalentes.%
\end{isamarkuptext}\isamarkuptrue%
\isacommand{lemma}\isamarkupfalse%
\ {\isachardoublequoteopen}semanticEq\ {\isacharparenleft}Atom\ p{\isacharparenright}\ {\isacharparenleft}{\isacharparenleft}Atom\ p{\isacharparenright}\ \isactrlbold {\isasymor}\ {\isacharparenleft}Atom\ p{\isacharparenright}{\isacharparenright}{\isachardoublequoteclose}\ \isanewline
%
\isadelimproof
\ \ %
\endisadelimproof
%
\isatagproof
\isacommand{by}\isamarkupfalse%
\ {\isacharparenleft}simp\ add{\isacharcolon}\ semanticEq{\isacharunderscore}def{\isacharparenright}%
\endisatagproof
{\isafoldproof}%
%
\isadelimproof
\isanewline
%
\endisadelimproof
\isanewline
\isacommand{lemma}\isamarkupfalse%
\ {\isachardoublequoteopen}semanticEq\ {\isacharparenleft}Atom\ p{\isacharparenright}\ {\isacharparenleft}{\isacharparenleft}Atom\ p{\isacharparenright}\ \isactrlbold {\isasymand}\ {\isacharparenleft}Atom\ p{\isacharparenright}{\isacharparenright}{\isachardoublequoteclose}\ \isanewline
%
\isadelimproof
\ \ %
\endisadelimproof
%
\isatagproof
\isacommand{by}\isamarkupfalse%
\ {\isacharparenleft}simp\ add{\isacharcolon}\ semanticEq{\isacharunderscore}def{\isacharparenright}%
\endisatagproof
{\isafoldproof}%
%
\isadelimproof
\isanewline
%
\endisadelimproof
\isanewline
\isacommand{lemma}\isamarkupfalse%
\ {\isachardoublequoteopen}semanticEq\ {\isasymbottom}\ {\isacharparenleft}{\isasymbottom}\ \isactrlbold {\isasymand}\ {\isasymbottom}{\isacharparenright}{\isachardoublequoteclose}\ \isanewline
%
\isadelimproof
\ \ %
\endisadelimproof
%
\isatagproof
\isacommand{by}\isamarkupfalse%
\ {\isacharparenleft}simp\ add{\isacharcolon}\ semanticEq{\isacharunderscore}def{\isacharparenright}%
\endisatagproof
{\isafoldproof}%
%
\isadelimproof
\isanewline
%
\endisadelimproof
\isanewline
\isacommand{lemma}\isamarkupfalse%
\ {\isachardoublequoteopen}semanticEq\ {\isasymbottom}\ {\isacharparenleft}{\isasymbottom}\ \isactrlbold {\isasymor}\ {\isasymbottom}{\isacharparenright}{\isachardoublequoteclose}\ \isanewline
%
\isadelimproof
\ \ %
\endisadelimproof
%
\isatagproof
\isacommand{by}\isamarkupfalse%
\ {\isacharparenleft}simp\ add{\isacharcolon}\ semanticEq{\isacharunderscore}def{\isacharparenright}%
\endisatagproof
{\isafoldproof}%
%
\isadelimproof
\isanewline
%
\endisadelimproof
\isanewline
\isacommand{lemma}\isamarkupfalse%
\ {\isachardoublequoteopen}semanticEq\ {\isasymbottom}\ {\isacharparenleft}\isactrlbold {\isasymnot}\ {\isasymtop}{\isacharparenright}{\isachardoublequoteclose}\isanewline
%
\isadelimproof
\ \ %
\endisadelimproof
%
\isatagproof
\isacommand{by}\isamarkupfalse%
\ {\isacharparenleft}simp\ add{\isacharcolon}\ semanticEq{\isacharunderscore}def\ top{\isacharunderscore}semantics{\isacharparenright}%
\endisatagproof
{\isafoldproof}%
%
\isadelimproof
\isanewline
%
\endisadelimproof
\isanewline
\isacommand{lemma}\isamarkupfalse%
\ {\isachardoublequoteopen}semanticEq\ F\ {\isacharparenleft}\isactrlbold {\isasymnot}{\isacharparenleft}\isactrlbold {\isasymnot}\ F{\isacharparenright}{\isacharparenright}{\isachardoublequoteclose}\isanewline
%
\isadelimproof
\ \ %
\endisadelimproof
%
\isatagproof
\isacommand{by}\isamarkupfalse%
\ {\isacharparenleft}simp\ add{\isacharcolon}\ semanticEq{\isacharunderscore}def{\isacharparenright}%
\endisatagproof
{\isafoldproof}%
%
\isadelimproof
\isanewline
%
\endisadelimproof
\isanewline
\isacommand{lemma}\isamarkupfalse%
\ {\isachardoublequoteopen}semanticEq\ {\isacharparenleft}\isactrlbold {\isasymnot}{\isacharparenleft}\isactrlbold {\isasymnot}\ F{\isacharparenright}{\isacharparenright}\ {\isacharparenleft}F\ \isactrlbold {\isasymor}\ F{\isacharparenright}{\isachardoublequoteclose}\isanewline
%
\isadelimproof
\ \ %
\endisadelimproof
%
\isatagproof
\isacommand{by}\isamarkupfalse%
\ {\isacharparenleft}simp\ add{\isacharcolon}\ semanticEq{\isacharunderscore}def{\isacharparenright}%
\endisatagproof
{\isafoldproof}%
%
\isadelimproof
\isanewline
%
\endisadelimproof
\isanewline
\isacommand{lemma}\isamarkupfalse%
\ {\isachardoublequoteopen}semanticEq\ {\isacharparenleft}\isactrlbold {\isasymnot}{\isacharparenleft}\isactrlbold {\isasymnot}\ F{\isacharparenright}{\isacharparenright}\ {\isacharparenleft}F\ \isactrlbold {\isasymand}\ F{\isacharparenright}{\isachardoublequoteclose}\isanewline
%
\isadelimproof
\ \ %
\endisadelimproof
%
\isatagproof
\isacommand{by}\isamarkupfalse%
\ {\isacharparenleft}simp\ add{\isacharcolon}\ semanticEq{\isacharunderscore}def{\isacharparenright}%
\endisatagproof
{\isafoldproof}%
%
\isadelimproof
\isanewline
%
\endisadelimproof
\isanewline
\isacommand{lemma}\isamarkupfalse%
\ {\isachardoublequoteopen}semanticEq\ {\isacharparenleft}\isactrlbold {\isasymnot}\ F\ \isactrlbold {\isasymand}\ \isactrlbold {\isasymnot}\ G{\isacharparenright}\ {\isacharparenleft}\isactrlbold {\isasymnot}{\isacharparenleft}F\ \isactrlbold {\isasymor}\ G{\isacharparenright}{\isacharparenright}{\isachardoublequoteclose}\isanewline
%
\isadelimproof
\ \ %
\endisadelimproof
%
\isatagproof
\isacommand{by}\isamarkupfalse%
\ {\isacharparenleft}simp\ add{\isacharcolon}\ semanticEq{\isacharunderscore}def{\isacharparenright}%
\endisatagproof
{\isafoldproof}%
%
\isadelimproof
\isanewline
%
\endisadelimproof
\isanewline
\isacommand{lemma}\isamarkupfalse%
\ {\isachardoublequoteopen}semanticEq\ {\isacharparenleft}F\ \isactrlbold {\isasymrightarrow}\ G{\isacharparenright}\ {\isacharparenleft}\isactrlbold {\isasymnot}\ F\ \isactrlbold {\isasymor}\ G{\isacharparenright}{\isachardoublequoteclose}\isanewline
%
\isadelimproof
\ \ %
\endisadelimproof
%
\isatagproof
\isacommand{by}\isamarkupfalse%
\ {\isacharparenleft}simp\ add{\isacharcolon}\ semanticEq{\isacharunderscore}def{\isacharparenright}%
\endisatagproof
{\isafoldproof}%
%
\isadelimproof
%
\endisadelimproof
%
\begin{isamarkuptext}%
En contraposición, también podemos dar ejemplos de fórmulas que no son semánticamente 
  equivalentes.%
\end{isamarkuptext}\isamarkuptrue%
\isacommand{lemma}\isamarkupfalse%
\ {\isachardoublequoteopen}{\isasymnot}\ semanticEq\ {\isacharparenleft}Atom\ p{\isacharparenright}\ {\isacharparenleft}\isactrlbold {\isasymnot}{\isacharparenleft}Atom\ p{\isacharparenright}{\isacharparenright}{\isachardoublequoteclose}\isanewline
%
\isadelimproof
\ \ %
\endisadelimproof
%
\isatagproof
\isacommand{by}\isamarkupfalse%
\ {\isacharparenleft}simp\ add{\isacharcolon}\ semanticEq{\isacharunderscore}def{\isacharparenright}%
\endisatagproof
{\isafoldproof}%
%
\isadelimproof
\isanewline
%
\endisadelimproof
\isanewline
\isacommand{lemma}\isamarkupfalse%
\ {\isachardoublequoteopen}{\isasymnot}\ semanticEq\ {\isasymbottom}\ {\isasymtop}{\isachardoublequoteclose}\isanewline
%
\isadelimproof
\ \ %
\endisadelimproof
%
\isatagproof
\isacommand{by}\isamarkupfalse%
\ {\isacharparenleft}simp\ add{\isacharcolon}\ semanticEq{\isacharunderscore}def\ top{\isacharunderscore}semantics{\isacharparenright}%
\endisatagproof
{\isafoldproof}%
%
\isadelimproof
%
\endisadelimproof
%
\begin{isamarkuptext}%
Por tanto, diremos intuitivamente que una fórmula es de tipo \isa{{\isasymalpha}} con componentes \isa{{\isasymalpha}\isactrlsub {\isadigit{1}}} y \isa{{\isasymalpha}\isactrlsub {\isadigit{2}}}
  si es semánticamente equivalente a la fórmula \isa{{\isasymalpha}\isactrlsub {\isadigit{1}}\ {\isasymand}\ {\isasymalpha}\isactrlsub {\isadigit{2}}}. Del mismo modo, una fórmula será de tipo
  \isa{{\isasymbeta}} con componentes \isa{{\isasymbeta}\isactrlsub {\isadigit{1}}} y \isa{{\isasymbeta}\isactrlsub {\isadigit{2}}} si es semánticamente equivalente a la fórmula \isa{{\isasymbeta}\isactrlsub {\isadigit{1}}\ {\isasymor}\ {\isasymbeta}\isactrlsub {\isadigit{2}}}.

  \begin{definicion}
    Las fórmulas de tipo \isa{{\isasymalpha}} (\isa{fórmulas\ conjuntivas}) y sus correspondientes componentes
    \isa{{\isasymalpha}\isactrlsub {\isadigit{1}}} y \isa{{\isasymalpha}\isactrlsub {\isadigit{2}}} se definen como sigue: dadas \isa{F} y \isa{G} fórmulas cualesquiera,
    \begin{enumerate}
      \item \isa{F\ {\isasymand}\ G} es una fórmula de tipo \isa{{\isasymalpha}} cuyas componentes son \isa{F} y \isa{G}.
      \item \isa{{\isasymnot}{\isacharparenleft}F\ {\isasymor}\ G{\isacharparenright}} es una fórmula de tipo \isa{{\isasymalpha}} cuyas componentes son \isa{{\isasymnot}\ F} y \isa{{\isasymnot}\ G}.
      \item \isa{{\isasymnot}{\isacharparenleft}F\ {\isasymlongrightarrow}\ G{\isacharparenright}} es una fórmula de tipo \isa{{\isasymalpha}} cuyas componentes son \isa{F} y \isa{{\isasymnot}\ G}.
    \end{enumerate} 
  \end{definicion}

  De este modo, de los ejemplos anteriores podemos deducir que las fórmulas atómicas son de tipo \isa{{\isasymalpha}}
  y sus componentes \isa{{\isasymalpha}\isactrlsub {\isadigit{1}}} y \isa{{\isasymalpha}\isactrlsub {\isadigit{2}}} son la propia fórmula. Del mismo modo, la constante \isa{{\isasymbottom}} también es 
  una fórmula conjuntiva cuyas componentes son ella misma. Por último, podemos observar que dada
  una fórmula cualquiera \isa{F}, su doble negación \isa{{\isasymnot}{\isacharparenleft}{\isasymnot}\ F{\isacharparenright}} es una fórmula de tipo \isa{{\isasymalpha}} y componentes
  \isa{F} y \isa{F}.

  Formalizaremos en Isabelle el conjunto de fórmulas \isa{{\isasymalpha}} como un predicato inductivo. De este modo,
  las reglas anteriores que construyen el conjunto de fórmulas de tipo \isa{{\isasymalpha}} se formalizan en Isabelle 
  como reglas de introducción. Además, añadiremos explícitamente una cuarta regla que introduce la 
  doble negación de una fórmula como fórmula de tipo \isa{{\isasymalpha}}. De este modo, facilitaremos la prueba de 
  resultados posteriores relacionados con la definición de conjunto de Hintikka, que constituyen una
  base para la demostración del \isa{teorema\ de\ existencia\ de\ modelo}.%
\end{isamarkuptext}\isamarkuptrue%
\isacommand{inductive}\isamarkupfalse%
\ Con\ {\isacharcolon}{\isacharcolon}\ {\isachardoublequoteopen}{\isacharprime}a\ formula\ {\isacharequal}{\isachargreater}\ {\isacharprime}a\ formula\ {\isacharequal}{\isachargreater}\ {\isacharprime}a\ formula\ {\isacharequal}{\isachargreater}\ bool{\isachardoublequoteclose}\ \isakeyword{where}\isanewline
{\isachardoublequoteopen}Con\ {\isacharparenleft}And\ F\ G{\isacharparenright}\ F\ G{\isachardoublequoteclose}\ {\isacharbar}\isanewline
{\isachardoublequoteopen}Con\ {\isacharparenleft}Not\ {\isacharparenleft}Or\ F\ G{\isacharparenright}{\isacharparenright}\ {\isacharparenleft}Not\ F{\isacharparenright}\ {\isacharparenleft}Not\ G{\isacharparenright}{\isachardoublequoteclose}\ {\isacharbar}\isanewline
{\isachardoublequoteopen}Con\ {\isacharparenleft}Not\ {\isacharparenleft}Imp\ F\ G{\isacharparenright}{\isacharparenright}\ F\ {\isacharparenleft}Not\ G{\isacharparenright}{\isachardoublequoteclose}\ {\isacharbar}\isanewline
{\isachardoublequoteopen}Con\ {\isacharparenleft}Not\ {\isacharparenleft}Not\ F{\isacharparenright}{\isacharparenright}\ F\ F{\isachardoublequoteclose}%
\begin{isamarkuptext}%
Las reglas de introducción que proporciona la definición anterior son
  las siguientes.

  \begin{itemize}
    \item[] \isa{Con\ {\isacharparenleft}F\ \isactrlbold {\isasymand}\ G{\isacharparenright}\ F\ G\isasep\isanewline%
Con\ {\isacharparenleft}\isactrlbold {\isasymnot}\ {\isacharparenleft}F\ \isactrlbold {\isasymor}\ G{\isacharparenright}{\isacharparenright}\ {\isacharparenleft}\isactrlbold {\isasymnot}\ F{\isacharparenright}\ {\isacharparenleft}\isactrlbold {\isasymnot}\ G{\isacharparenright}\isasep\isanewline%
Con\ {\isacharparenleft}\isactrlbold {\isasymnot}\ {\isacharparenleft}F\ \isactrlbold {\isasymrightarrow}\ G{\isacharparenright}{\isacharparenright}\ F\ {\isacharparenleft}\isactrlbold {\isasymnot}\ G{\isacharparenright}\isasep\isanewline%
Con\ {\isacharparenleft}\isactrlbold {\isasymnot}\ {\isacharparenleft}\isactrlbold {\isasymnot}\ F{\isacharparenright}{\isacharparenright}\ F\ F} 
      \hfill (\isa{Con{\isachardot}intros})
  \end{itemize}
  
  Por otro lado, definamos las fórmulas disyuntivas.

  \begin{definicion}
    Las fórmulas de tipo \isa{{\isasymbeta}} (\isa{fórmulas\ disyuntivas}) y sus correspondientes componentes
    \isa{{\isasymbeta}\isactrlsub {\isadigit{1}}} y \isa{{\isasymbeta}\isactrlsub {\isadigit{2}}} se definen como sigue: dadas \isa{F} y \isa{G} fórmulas cualesquiera,
    \begin{enumerate}
      \item \isa{F\ {\isasymor}\ G} es una fórmula de tipo \isa{{\isasymbeta}} cuyas componentes son \isa{F} y \isa{G}.
      \item \isa{F\ {\isasymlongrightarrow}\ G} es una fórmula de tipo \isa{{\isasymbeta}} cuyas componentes son \isa{{\isasymnot}\ F} y \isa{G}.
      \item \isa{{\isasymnot}{\isacharparenleft}F\ {\isasymand}\ G{\isacharparenright}} es una fórmula de tipo \isa{{\isasymbeta}} cuyas componentes son \isa{{\isasymnot}\ F} y \isa{{\isasymnot}\ G}.
    \end{enumerate} 
  \end{definicion}

  De los ejemplos dados anteriormente, podemos deducir análogamente que las fórmulas atómicas, la
  constante \isa{{\isasymbottom}} y la doble negación sob también fórmulas disyuntivas con las mismas componentes que
  las dadas para el tipo conjuntivo.

  Del mismo modo, su formalización se realiza como un predicado inductivo, de manera que las reglas 
  que definen el conjunto de fórmulas de tipo \isa{{\isasymbeta}} se formalizan en Isabelle como reglas de 
  introducción. Análogamente, introduciremos de manera explícita una regla que señala que la doble 
  negación de una fórmula es una fórmula de tipo disyuntivo.%
\end{isamarkuptext}\isamarkuptrue%
\isacommand{inductive}\isamarkupfalse%
\ Dis\ {\isacharcolon}{\isacharcolon}\ {\isachardoublequoteopen}{\isacharprime}a\ formula\ {\isacharequal}{\isachargreater}\ {\isacharprime}a\ formula\ {\isacharequal}{\isachargreater}\ {\isacharprime}a\ formula\ {\isacharequal}{\isachargreater}\ bool{\isachardoublequoteclose}\ \isakeyword{where}\isanewline
{\isachardoublequoteopen}Dis\ {\isacharparenleft}Or\ F\ G{\isacharparenright}\ F\ G{\isachardoublequoteclose}\ {\isacharbar}\isanewline
{\isachardoublequoteopen}Dis\ {\isacharparenleft}Imp\ F\ G{\isacharparenright}\ {\isacharparenleft}Not\ F{\isacharparenright}\ G{\isachardoublequoteclose}\ {\isacharbar}\isanewline
{\isachardoublequoteopen}Dis\ {\isacharparenleft}Not\ {\isacharparenleft}And\ F\ G{\isacharparenright}{\isacharparenright}\ {\isacharparenleft}Not\ F{\isacharparenright}\ {\isacharparenleft}Not\ G{\isacharparenright}{\isachardoublequoteclose}\ {\isacharbar}\isanewline
{\isachardoublequoteopen}Dis\ {\isacharparenleft}Not\ {\isacharparenleft}Not\ F{\isacharparenright}{\isacharparenright}\ F\ F{\isachardoublequoteclose}%
\begin{isamarkuptext}%
Del mismo modo, las reglas de introducción que proporciona esta formalización se muestran a 
  continuación.

  \begin{itemize}
    \item[] \isa{Dis\ {\isacharparenleft}F\ \isactrlbold {\isasymor}\ G{\isacharparenright}\ F\ G\isasep\isanewline%
Dis\ {\isacharparenleft}F\ \isactrlbold {\isasymrightarrow}\ G{\isacharparenright}\ {\isacharparenleft}\isactrlbold {\isasymnot}\ F{\isacharparenright}\ G\isasep\isanewline%
Dis\ {\isacharparenleft}\isactrlbold {\isasymnot}\ {\isacharparenleft}F\ \isactrlbold {\isasymand}\ G{\isacharparenright}{\isacharparenright}\ {\isacharparenleft}\isactrlbold {\isasymnot}\ F{\isacharparenright}\ {\isacharparenleft}\isactrlbold {\isasymnot}\ G{\isacharparenright}\isasep\isanewline%
Dis\ {\isacharparenleft}\isactrlbold {\isasymnot}\ {\isacharparenleft}\isactrlbold {\isasymnot}\ F{\isacharparenright}{\isacharparenright}\ F\ F} 
      \hfill (\isa{Dis{\isachardot}intros})
  \end{itemize}

  Cabe observar que las formalizaciones de la definiciones de fórmulas de tipo \isa{{\isasymalpha}} y \isa{{\isasymbeta}} son 
  definiciones sintácticas, pues construyen los correspondientes conjuntos de fórmulas a partir de 
  una reglas sintácticas concretas. Se trata de una simplificación de la intuición original de la 
  clasificación de las fórmulas mediante notación uniforme, ya que se prescinde de la noción de 
  equivalencia semántica que permite clasificar la totalidad de las fórmulas proposicionales. 

  Veamos la clasificación de casos concretos de fórmulas. Por ejemplo, según hemos definido la 
  fórmula \isa{{\isasymtop}}, es sencillo comprobar que se trata de una fórmula disyuntiva.%
\end{isamarkuptext}\isamarkuptrue%
\isacommand{lemma}\isamarkupfalse%
\ {\isachardoublequoteopen}Dis\ {\isasymtop}\ {\isacharparenleft}\isactrlbold {\isasymnot}\ {\isasymbottom}{\isacharparenright}\ {\isasymbottom}{\isachardoublequoteclose}\ \isanewline
%
\isadelimproof
\ \ %
\endisadelimproof
%
\isatagproof
\isacommand{unfolding}\isamarkupfalse%
\ Top{\isacharunderscore}def\ \isacommand{by}\isamarkupfalse%
\ {\isacharparenleft}simp\ only{\isacharcolon}\ Dis{\isachardot}intros{\isacharparenleft}{\isadigit{2}}{\isacharparenright}{\isacharparenright}%
\endisatagproof
{\isafoldproof}%
%
\isadelimproof
%
\endisadelimproof
%
\begin{isamarkuptext}%
Por otro lado, se observa a partir de las correspondientes definiciones que la conjunción
  generalizada de una lista de fórmulas es una fórmula de tipo \isa{{\isasymalpha}} y la disyunción generalizada de
  una lista de fórmulas es una fórmula de tipo \isa{{\isasymbeta}}.%
\end{isamarkuptext}\isamarkuptrue%
\isacommand{lemma}\isamarkupfalse%
\ {\isachardoublequoteopen}Con\ {\isacharparenleft}\isactrlbold {\isasymAnd}{\isacharparenleft}F{\isacharhash}Fs{\isacharparenright}{\isacharparenright}\ F\ {\isacharparenleft}\isactrlbold {\isasymAnd}Fs{\isacharparenright}{\isachardoublequoteclose}\isanewline
%
\isadelimproof
\ \ %
\endisadelimproof
%
\isatagproof
\isacommand{by}\isamarkupfalse%
\ {\isacharparenleft}simp\ only{\isacharcolon}\ BigAnd{\isachardot}simps\ Con{\isachardot}intros{\isacharparenleft}{\isadigit{1}}{\isacharparenright}{\isacharparenright}%
\endisatagproof
{\isafoldproof}%
%
\isadelimproof
\isanewline
%
\endisadelimproof
\isanewline
\isacommand{lemma}\isamarkupfalse%
\ {\isachardoublequoteopen}Dis\ {\isacharparenleft}\isactrlbold {\isasymOr}{\isacharparenleft}F{\isacharhash}Fs{\isacharparenright}{\isacharparenright}\ F\ {\isacharparenleft}\isactrlbold {\isasymOr}Fs{\isacharparenright}{\isachardoublequoteclose}\isanewline
%
\isadelimproof
\ \ %
\endisadelimproof
%
\isatagproof
\isacommand{by}\isamarkupfalse%
\ {\isacharparenleft}simp\ only{\isacharcolon}\ BigOr{\isachardot}simps\ Dis{\isachardot}intros{\isacharparenleft}{\isadigit{1}}{\isacharparenright}{\isacharparenright}%
\endisatagproof
{\isafoldproof}%
%
\isadelimproof
%
\endisadelimproof
%
\begin{isamarkuptext}%
Finalmente, de las reglas que definen las fórmulas conjuntivas y disyuntivas se deduce que
  la doble negación de una fórmula es una fórmula perteneciente a ambos tipos.%
\end{isamarkuptext}\isamarkuptrue%
\isacommand{lemma}\isamarkupfalse%
\ notDisCon{\isacharcolon}\ {\isachardoublequoteopen}Con\ {\isacharparenleft}Not\ {\isacharparenleft}Not\ F{\isacharparenright}{\isacharparenright}\ F\ F{\isachardoublequoteclose}\ {\isachardoublequoteopen}Dis\ {\isacharparenleft}Not\ {\isacharparenleft}Not\ F{\isacharparenright}{\isacharparenright}\ F\ F{\isachardoublequoteclose}\ \isanewline
%
\isadelimproof
\ \ %
\endisadelimproof
%
\isatagproof
\isacommand{by}\isamarkupfalse%
\ {\isacharparenleft}simp\ only{\isacharcolon}\ Con{\isachardot}intros{\isacharparenleft}{\isadigit{4}}{\isacharparenright}\ Dis{\isachardot}intros{\isacharparenleft}{\isadigit{4}}{\isacharparenright}{\isacharparenright}{\isacharplus}%
\endisatagproof
{\isafoldproof}%
%
\isadelimproof
%
\endisadelimproof
%
\begin{isamarkuptext}%
A continuación vamos a introducir el siguiente lema que caracteriza las fórmulas de tipo \isa{{\isasymalpha}} 
  y \isa{{\isasymbeta}}, facilitando el uso de la notación uniforme en Isabelle.%
\end{isamarkuptext}\isamarkuptrue%
\isacommand{lemma}\isamarkupfalse%
\ con{\isacharunderscore}dis{\isacharunderscore}simps{\isacharcolon}\isanewline
\ \ {\isachardoublequoteopen}Con\ a{\isadigit{1}}\ a{\isadigit{2}}\ a{\isadigit{3}}\ {\isacharequal}\ {\isacharparenleft}a{\isadigit{1}}\ {\isacharequal}\ a{\isadigit{2}}\ \isactrlbold {\isasymand}\ a{\isadigit{3}}\ {\isasymor}\ \isanewline
\ \ \ \ {\isacharparenleft}{\isasymexists}F\ G{\isachardot}\ a{\isadigit{1}}\ {\isacharequal}\ \isactrlbold {\isasymnot}\ {\isacharparenleft}F\ \isactrlbold {\isasymor}\ G{\isacharparenright}\ {\isasymand}\ a{\isadigit{2}}\ {\isacharequal}\ \isactrlbold {\isasymnot}\ F\ {\isasymand}\ a{\isadigit{3}}\ {\isacharequal}\ \isactrlbold {\isasymnot}\ G{\isacharparenright}\ {\isasymor}\ \isanewline
\ \ \ \ {\isacharparenleft}{\isasymexists}G{\isachardot}\ a{\isadigit{1}}\ {\isacharequal}\ \isactrlbold {\isasymnot}\ {\isacharparenleft}a{\isadigit{2}}\ \isactrlbold {\isasymrightarrow}\ G{\isacharparenright}\ {\isasymand}\ a{\isadigit{3}}\ {\isacharequal}\ \isactrlbold {\isasymnot}\ G{\isacharparenright}\ {\isasymor}\ \isanewline
\ \ \ \ a{\isadigit{1}}\ {\isacharequal}\ \isactrlbold {\isasymnot}\ {\isacharparenleft}\isactrlbold {\isasymnot}\ a{\isadigit{2}}{\isacharparenright}\ {\isasymand}\ a{\isadigit{3}}\ {\isacharequal}\ a{\isadigit{2}}{\isacharparenright}{\isachardoublequoteclose}\isanewline
\ \ {\isachardoublequoteopen}Dis\ a{\isadigit{1}}\ a{\isadigit{2}}\ a{\isadigit{3}}\ {\isacharequal}\ {\isacharparenleft}a{\isadigit{1}}\ {\isacharequal}\ a{\isadigit{2}}\ \isactrlbold {\isasymor}\ a{\isadigit{3}}\ {\isasymor}\ \isanewline
\ \ \ \ {\isacharparenleft}{\isasymexists}F\ G{\isachardot}\ a{\isadigit{1}}\ {\isacharequal}\ F\ \isactrlbold {\isasymrightarrow}\ G\ {\isasymand}\ a{\isadigit{2}}\ {\isacharequal}\ \isactrlbold {\isasymnot}\ F\ {\isasymand}\ a{\isadigit{3}}\ {\isacharequal}\ G{\isacharparenright}\ {\isasymor}\ \isanewline
\ \ \ \ {\isacharparenleft}{\isasymexists}F\ G{\isachardot}\ a{\isadigit{1}}\ {\isacharequal}\ \isactrlbold {\isasymnot}\ {\isacharparenleft}F\ \isactrlbold {\isasymand}\ G{\isacharparenright}\ {\isasymand}\ a{\isadigit{2}}\ {\isacharequal}\ \isactrlbold {\isasymnot}\ F\ {\isasymand}\ a{\isadigit{3}}\ {\isacharequal}\ \isactrlbold {\isasymnot}\ G{\isacharparenright}\ {\isasymor}\ \isanewline
\ \ \ \ a{\isadigit{1}}\ {\isacharequal}\ \isactrlbold {\isasymnot}\ {\isacharparenleft}\isactrlbold {\isasymnot}\ a{\isadigit{2}}{\isacharparenright}\ {\isasymand}\ a{\isadigit{3}}\ {\isacharequal}\ a{\isadigit{2}}{\isacharparenright}{\isachardoublequoteclose}\ \isanewline
%
\isadelimproof
\ \ %
\endisadelimproof
%
\isatagproof
\isacommand{by}\isamarkupfalse%
\ {\isacharparenleft}simp{\isacharunderscore}all\ add{\isacharcolon}\ Con{\isachardot}simps\ Dis{\isachardot}simps{\isacharparenright}%
\endisatagproof
{\isafoldproof}%
%
\isadelimproof
%
\endisadelimproof
%
\begin{isamarkuptext}%
Por último, introduzcamos resultados que permiten caracterizar los conjuntos de Hintikka y la 
  propiedad de consistencia proposicional empleando la notación uniforme.

  \begin{lema}[Caracterización de los conjuntos de Hintikka mediante la notación uniforme]
    Dado un conjunto de fórmulas proposicionales \isa{S}, son equivalentes:
    \begin{enumerate}
      \item \isa{S} es un conjunto de Hintikka.
      \item Se verifican las condiciones siguientes:
      \begin{itemize}
        \item \isa{{\isasymbottom}} no pertenece a \isa{S}.
        \item Dada \isa{p} una fórmula atómica cualquiera, no se tiene 
        simultáneamente que\\ \isa{p\ {\isasymin}\ S} y \isa{{\isasymnot}\ p\ {\isasymin}\ S}.
        \item Para toda fórmula de tipo \isa{{\isasymalpha}} con componentes \isa{{\isasymalpha}\isactrlsub {\isadigit{1}}} y \isa{{\isasymalpha}\isactrlsub {\isadigit{2}}} se verifica 
        que si la fórmula pertenece a \isa{S}, entonces \isa{{\isasymalpha}\isactrlsub {\isadigit{1}}} y \isa{{\isasymalpha}\isactrlsub {\isadigit{2}}} también.
        \item Para toda fórmula de tipo \isa{{\isasymbeta}} con componentes \isa{{\isasymbeta}\isactrlsub {\isadigit{1}}} y \isa{{\isasymbeta}\isactrlsub {\isadigit{2}}} se verifica 
        que si la fórmula pertenece a \isa{S}, entonces o bien \isa{{\isasymbeta}\isactrlsub {\isadigit{1}}} pertenece
        a \isa{S} o bien \isa{{\isasymbeta}\isactrlsub {\isadigit{2}}} pertenece a \isa{S}.
      \end{itemize} 
    \end{enumerate}
  \end{lema}

  En Isabelle/HOL se formaliza del siguiente modo.%
\end{isamarkuptext}\isamarkuptrue%
\isacommand{lemma}\isamarkupfalse%
\ {\isachardoublequoteopen}Hintikka\ S\ {\isacharequal}\ {\isacharparenleft}{\isasymbottom}\ {\isasymnotin}\ S\isanewline
{\isasymand}\ {\isacharparenleft}{\isasymforall}k{\isachardot}\ Atom\ k\ {\isasymin}\ S\ {\isasymlongrightarrow}\ \isactrlbold {\isasymnot}\ {\isacharparenleft}Atom\ k{\isacharparenright}\ {\isasymin}\ S\ {\isasymlongrightarrow}\ False{\isacharparenright}\isanewline
{\isasymand}\ {\isacharparenleft}{\isasymforall}F\ G\ H{\isachardot}\ Con\ F\ G\ H\ {\isasymlongrightarrow}\ F\ {\isasymin}\ S\ {\isasymlongrightarrow}\ G\ {\isasymin}\ S\ {\isasymand}\ H\ {\isasymin}\ S{\isacharparenright}\isanewline
{\isasymand}\ {\isacharparenleft}{\isasymforall}F\ G\ H{\isachardot}\ Dis\ F\ G\ H\ {\isasymlongrightarrow}\ F\ {\isasymin}\ S\ {\isasymlongrightarrow}\ G\ {\isasymin}\ S\ {\isasymor}\ H\ {\isasymin}\ S{\isacharparenright}{\isacharparenright}{\isachardoublequoteclose}\ \isanewline
%
\isadelimproof
\ \ %
\endisadelimproof
%
\isatagproof
\isacommand{oops}\isamarkupfalse%
%
\endisatagproof
{\isafoldproof}%
%
\isadelimproof
%
\endisadelimproof
%
\begin{isamarkuptext}%
Procedamos a la demostración del resultado.

\begin{demostracion}
  Para probar la equivalencia, veamos cada una de las implicaciones por separado.

\textbf{\isa{{\isadigit{1}}{\isacharparenright}\ {\isasymLongrightarrow}\ {\isadigit{2}}{\isacharparenright}}}

  Supongamos que \isa{S} es un conjunto de Hintikka. Vamos a probar que, en efecto, se 
  verifican las condiciones del enunciado del lema.

  Por definición de conjunto de Hintikka, \isa{S} verifica las siguientes condiciones:
  \begin{enumerate}
    \item \isa{{\isasymbottom}\ {\isasymnotin}\ S}.
    \item Dada \isa{p} una fórmula atómica cualquiera, no se tiene 
      simultáneamente que\\ \isa{p\ {\isasymin}\ S} y \isa{{\isasymnot}\ p\ {\isasymin}\ S}.
    \item Si \isa{G\ {\isasymand}\ H\ {\isasymin}\ S}, entonces \isa{G\ {\isasymin}\ S} y \isa{H\ {\isasymin}\ S}.
    \item Si \isa{G\ {\isasymor}\ H\ {\isasymin}\ S}, entonces \isa{G\ {\isasymin}\ S} o \isa{H\ {\isasymin}\ S}.
    \item Si \isa{G\ {\isasymrightarrow}\ H\ {\isasymin}\ S}, entonces \isa{{\isasymnot}\ G\ {\isasymin}\ S} o \isa{H\ {\isasymin}\ S}.
    \item Si \isa{{\isasymnot}{\isacharparenleft}{\isasymnot}\ G{\isacharparenright}\ {\isasymin}\ S}, entonces \isa{G\ {\isasymin}\ S}.
    \item Si \isa{{\isasymnot}{\isacharparenleft}G\ {\isasymand}\ H{\isacharparenright}\ {\isasymin}\ S}, entonces \isa{{\isasymnot}\ G\ {\isasymin}\ S} o \isa{{\isasymnot}\ H\ {\isasymin}\ S}.
    \item Si \isa{{\isasymnot}{\isacharparenleft}G\ {\isasymor}\ H{\isacharparenright}\ {\isasymin}\ S}, entonces \isa{{\isasymnot}\ G\ {\isasymin}\ S} y \isa{{\isasymnot}\ H\ {\isasymin}\ S}. 
    \item Si \isa{{\isasymnot}{\isacharparenleft}G\ {\isasymrightarrow}\ H{\isacharparenright}\ {\isasymin}\ S}, entonces \isa{G\ {\isasymin}\ S} y \isa{{\isasymnot}\ H\ {\isasymin}\ S}. 
  \end{enumerate}  
  De este modo, el conjunto \isa{S} cumple la primera y la segunda condición del
  enunciado del lema, que se corresponden con las dos primeras condiciones
  de la definición de conjunto de Hintikka. Veamos que, además, verifica las
  dos últimas condiciones del resultado.

  En primer lugar, probemos que para toda fórmula de tipo \isa{{\isasymalpha}} con 
  componentes \isa{{\isasymalpha}\isactrlsub {\isadigit{1}}} y \isa{{\isasymalpha}\isactrlsub {\isadigit{2}}} se verifica que si la fórmula pertenece al conjunto 
  \isa{S}, entonces \isa{{\isasymalpha}\isactrlsub {\isadigit{1}}} y \isa{{\isasymalpha}\isactrlsub {\isadigit{2}}} también. Para ello, supongamos que una fórmula 
  cualquiera de tipo \isa{{\isasymalpha}} pertence a \isa{S}. Por definición de este tipo de
  fórmulas, tenemos que \isa{{\isasymalpha}} puede ser de la forma \isa{G\ {\isasymand}\ H}, \isa{{\isasymnot}{\isacharparenleft}{\isasymnot}\ G{\isacharparenright}},\\ \isa{{\isasymnot}{\isacharparenleft}G\ {\isasymor}\ H{\isacharparenright}} 
  o \isa{{\isasymnot}{\isacharparenleft}G\ {\isasymlongrightarrow}\ H{\isacharparenright}} para fórmulas \isa{G} y \isa{H} cualesquiera. Probemos que, para cada
  tipo de fórmula \isa{{\isasymalpha}} perteneciente a \isa{S}, sus componentes \isa{{\isasymalpha}\isactrlsub {\isadigit{1}}} y \isa{{\isasymalpha}\isactrlsub {\isadigit{2}}} están en
  \isa{S}.

  \isa{{\isasymsqdot}\ Fórmula\ del\ tipo\ G\ {\isasymand}\ H}: Sus componentes conjuntivas son \isa{G} y \isa{H}. 
  Por la tercera condición de la definición de conjunto de Hintikka, obtenemos 
  que si \isa{G\ {\isasymand}\ H} pertenece a \isa{S}, entonces \isa{G} y \isa{H} están ambas en el conjunto,
  lo que prueba este caso.
    
  \isa{{\isasymsqdot}\ Fórmula\ del\ tipo\ {\isasymnot}{\isacharparenleft}{\isasymnot}\ G{\isacharparenright}}: Sus componentes conjuntivas son ambas \isa{G}.
  Por la sexta condición de la definición de conjunto de Hintikka, obtenemos que
  si \isa{{\isasymnot}{\isacharparenleft}{\isasymnot}\ G{\isacharparenright}} pertenece a \isa{S}, entonces \isa{G} pertenece al conjunto, lo que prueba
  este caso.

  \isa{{\isasymsqdot}\ Fórmula\ del\ tipo\ {\isasymnot}{\isacharparenleft}G\ {\isasymor}\ H{\isacharparenright}}: Sus componentes conjuntivas son \isa{{\isasymnot}\ G} y \isa{{\isasymnot}\ H}. 
  Por la octava condición de la definición de conjunto de Hintikka, obtenemos 
  que si \isa{{\isasymnot}{\isacharparenleft}G\ {\isasymor}\ H{\isacharparenright}} pertenece a \isa{S}, entonces \isa{{\isasymnot}\ G} y \isa{{\isasymnot}\ H} están ambas en el conjunto,
  lo que prueba este caso.

  \isa{{\isasymsqdot}\ Fórmula\ del\ tipo\ {\isasymnot}{\isacharparenleft}G\ {\isasymlongrightarrow}\ H{\isacharparenright}}: Sus componentes conjuntivas son \isa{G} y \isa{{\isasymnot}\ H}. 
  Por la novena condición de la definición de conjunto de Hintikka, obtenemos 
  que si\\ \isa{{\isasymnot}{\isacharparenleft}G\ {\isasymlongrightarrow}\ H{\isacharparenright}} pertenece a \isa{S}, entonces \isa{G} y \isa{{\isasymnot}\ H} están ambas en el conjunto,
  lo que prueba este caso.

  Finalmente, probemos que para toda fórmula de tipo \isa{{\isasymbeta}} con componentes \isa{{\isasymbeta}\isactrlsub {\isadigit{1}}} y 
  \isa{{\isasymbeta}\isactrlsub {\isadigit{2}}} se verifica que si la fórmula pertenece al conjunto \isa{S}, entonces o bien \isa{{\isasymbeta}\isactrlsub {\isadigit{1}}} 
  pertenece al conjunto o bien \isa{{\isasymbeta}\isactrlsub {\isadigit{2}}} pertenece a conjunto. Para ello, supongamos que 
  una fórmula cualquiera de tipo \isa{{\isasymbeta}} pertence a \isa{S}. Por definición de este tipo de
  fórmulas, tenemos que \isa{{\isasymbeta}} puede ser de la forma \isa{G\ {\isasymor}\ H}, \isa{G\ {\isasymlongrightarrow}\ H}, \isa{{\isasymnot}{\isacharparenleft}{\isasymnot}\ G{\isacharparenright}} 
  o \isa{{\isasymnot}{\isacharparenleft}G\ {\isasymand}\ H{\isacharparenright}} para fórmulas \isa{G} y \isa{H} cualesquiera. Probemos que, para cada
  tipo de fórmula \isa{{\isasymbeta}} perteneciente a \isa{S}, o bien su componente \isa{{\isasymbeta}\isactrlsub {\isadigit{1}}} pertenece a \isa{S} 
  o bien su componente \isa{{\isasymbeta}\isactrlsub {\isadigit{2}}} pertenece a \isa{S}.

  \isa{{\isasymsqdot}\ Fórmula\ del\ tipo\ G\ {\isasymor}\ H}: Sus componentes disyuntivas son \isa{G} y \isa{H}. 
    Por la cuarta condición de la definición de conjunto de Hintikka, obtenemos 
    que si \isa{G\ {\isasymor}\ H} pertenece a \isa{S}, entonces o bien \isa{G} está en \isa{S} o bien \isa{H} está
    en \isa{S}, lo que prueba este caso.

  \isa{{\isasymsqdot}\ Fórmula\ del\ tipo\ G\ {\isasymlongrightarrow}\ H}: Sus componentes disyuntivas son \isa{{\isasymnot}\ G} y \isa{H}.
    Por la quinta condición de la definición de conjunto de Hintikka, obtenemos que
    si \isa{G\ {\isasymlongrightarrow}\ H} pertenece a \isa{S}, entonces o bien \isa{{\isasymnot}\ G} pertenece al conjunto o bien
    \isa{H} pertenece al conjunto, lo que prueba este caso.

  \isa{{\isasymsqdot}\ Fórmula\ del\ tipo\ {\isasymnot}{\isacharparenleft}{\isasymnot}\ G{\isacharparenright}}: Sus componentes conjuntivas son ambas \isa{G}.
    Por la sexta condición de la definición de conjunto de Hintikka, obtenemos 
    que si \isa{{\isasymnot}{\isacharparenleft}{\isasymnot}\ G{\isacharparenright}} pertenece a \isa{S}, entonces \isa{G} pertenece al conjunto. De este modo,
    por la regla de introducción a la disyunción, se prueba que o bien una de las 
    componentes está en el conjunto o bien lo está la otra pues, en este caso,
    coinciden.

  \isa{{\isasymsqdot}\ Fórmula\ del\ tipo\ {\isasymnot}{\isacharparenleft}G\ {\isasymand}\ H{\isacharparenright}}: Sus componentes conjuntivas son \isa{{\isasymnot}\ G} y \isa{{\isasymnot}\ H}. 
    Por la séptima condición de la definición de conjunto de Hintikka, obtenemos 
    que si\\ \isa{{\isasymnot}{\isacharparenleft}G\ {\isasymand}\ H{\isacharparenright}} pertenece a \isa{S}, entonces o bien \isa{{\isasymnot}\ G} pertenece al conjunto
    o bien \isa{{\isasymnot}\ H} pertenece al conjunto, lo que prueba este caso.

\textbf{\isa{{\isadigit{2}}{\isacharparenright}\ {\isasymLongrightarrow}\ {\isadigit{1}}{\isacharparenright}}}

  Supongamos que se verifican las condiciones del enunciado del lema:

  \begin{itemize}
    \item \isa{{\isasymbottom}} no pertenece a \isa{S}.
    \item Dada \isa{p} una fórmula atómica cualquiera, no se tiene 
    simultáneamente que\\ \isa{p\ {\isasymin}\ S} y \isa{{\isasymnot}\ p\ {\isasymin}\ S}.
    \item Para toda fórmula de tipo \isa{{\isasymalpha}} con componentes \isa{{\isasymalpha}\isactrlsub {\isadigit{1}}} y \isa{{\isasymalpha}\isactrlsub {\isadigit{2}}} se verifica 
    que si la fórmula pertenece a \isa{S}, entonces \isa{{\isasymalpha}\isactrlsub {\isadigit{1}}} y \isa{{\isasymalpha}\isactrlsub {\isadigit{2}}} también.
    \item Para toda fórmula de tipo \isa{{\isasymbeta}} con componentes \isa{{\isasymbeta}\isactrlsub {\isadigit{1}}} y \isa{{\isasymbeta}\isactrlsub {\isadigit{2}}} se verifica 
    que si la fórmula pertenece a \isa{S}, entonces o bien \isa{{\isasymbeta}\isactrlsub {\isadigit{1}}} pertenece
    a \isa{S} o bien \isa{{\isasymbeta}\isactrlsub {\isadigit{2}}} pertenece a \isa{S}.
  \end{itemize}  

  Vamos a probar que \isa{S} es un conjunto de Hintikka.

  Por la definición de conjunto de Hintikka, es suficiente probar las siguientes
  condiciones:

  \begin{enumerate}
    \item \isa{{\isasymbottom}\ {\isasymnotin}\ S}.
    \item Dada \isa{p} una fórmula atómica cualquiera, no se tiene 
      simultáneamente que\\ \isa{p\ {\isasymin}\ S} y \isa{{\isasymnot}\ p\ {\isasymin}\ S}.
    \item Si \isa{G\ {\isasymand}\ H\ {\isasymin}\ S}, entonces \isa{G\ {\isasymin}\ S} y \isa{H\ {\isasymin}\ S}.
    \item Si \isa{G\ {\isasymor}\ H\ {\isasymin}\ S}, entonces \isa{G\ {\isasymin}\ S} o \isa{H\ {\isasymin}\ S}.
    \item Si \isa{G\ {\isasymrightarrow}\ H\ {\isasymin}\ S}, entonces \isa{{\isasymnot}\ G\ {\isasymin}\ S} o \isa{H\ {\isasymin}\ S}.
    \item Si \isa{{\isasymnot}{\isacharparenleft}{\isasymnot}\ G{\isacharparenright}\ {\isasymin}\ S}, entonces \isa{G\ {\isasymin}\ S}.
    \item Si \isa{{\isasymnot}{\isacharparenleft}G\ {\isasymand}\ H{\isacharparenright}\ {\isasymin}\ S}, entonces \isa{{\isasymnot}\ G\ {\isasymin}\ S} o \isa{{\isasymnot}\ H\ {\isasymin}\ S}.
    \item Si \isa{{\isasymnot}{\isacharparenleft}G\ {\isasymor}\ H{\isacharparenright}\ {\isasymin}\ S}, entonces \isa{{\isasymnot}\ G\ {\isasymin}\ S} y \isa{{\isasymnot}\ H\ {\isasymin}\ S}. 
    \item Si \isa{{\isasymnot}{\isacharparenleft}G\ {\isasymrightarrow}\ H{\isacharparenright}\ {\isasymin}\ S}, entonces \isa{G\ {\isasymin}\ S} y \isa{{\isasymnot}\ H\ {\isasymin}\ S}. 
  \end{enumerate} 

  En primer lugar se observa que, por hipótesis, se verifican las dos primeras
  condiciones de la definición de conjunto de Hintikka. Veamos que, en efecto, se
  cumplen las demás.

  \begin{enumerate}
    \item[\isa{{\isadigit{3}}{\isacharparenright}}] Supongamos que \isa{G\ {\isasymand}\ H} está en \isa{S} para fórmulas \isa{G} y \isa{H} cualesquiera.
    Por definición, \isa{G\ {\isasymand}\ H} es una fórmula de tipo \isa{{\isasymalpha}} con componentes \isa{G} y \isa{H}. 
    Por lo tanto, por hipótesis se cumple que \isa{G} y \isa{H} están en \isa{S}.
    \item[\isa{{\isadigit{4}}{\isacharparenright}}] Supongamos que \isa{G\ {\isasymor}\ H} está en \isa{S} para fórmulas \isa{G} y \isa{H} cualesquiera.
    Por definición, \isa{G\ {\isasymor}\ H} es una fórmula de tipo \isa{{\isasymbeta}} con componentes \isa{G} y \isa{H}. 
    Por lo tanto, por hipótesis se cumple que o bien \isa{G} está en \isa{S} o bien \isa{H} está 
    en \isa{S}.
    \item[\isa{{\isadigit{5}}{\isacharparenright}}] Supongamos que \isa{G\ {\isasymlongrightarrow}\ H} está en \isa{S} para fórmulas \isa{G} y \isa{H} cualesquiera.
    Por definición, \isa{G\ {\isasymlongrightarrow}\ H} es una fórmula de tipo \isa{{\isasymbeta}} con componentes \isa{{\isasymnot}\ G} y \isa{H}. 
    Por lo tanto, por hipótesis se cumple que o bien \isa{{\isasymnot}\ G} está en \isa{S} o bien \isa{H} está 
    en \isa{S}.
    \item[\isa{{\isadigit{6}}{\isacharparenright}}] Supongamos que \isa{{\isasymnot}{\isacharparenleft}{\isasymnot}\ G{\isacharparenright}} está en \isa{S} para una fórmula \isa{G} cualquiera.
    Por definición, \isa{{\isasymnot}{\isacharparenleft}{\isasymnot}\ G{\isacharparenright}} es una fórmula de tipo \isa{{\isasymalpha}} cuyas componentes son ambas \isa{G}. 
    Por lo tanto, por hipótesis se cumple que \isa{G} está en \isa{S}.
    \item[\isa{{\isadigit{7}}{\isacharparenright}}] Supongamos que \isa{{\isasymnot}{\isacharparenleft}G\ {\isasymand}\ H{\isacharparenright}} está en \isa{S} para fórmulas \isa{G} y \isa{H} cualesquiera.
    Por definición, \isa{{\isasymnot}{\isacharparenleft}G\ {\isasymand}\ H{\isacharparenright}} es una fórmula de tipo \isa{{\isasymbeta}} con componentes \isa{{\isasymnot}\ G} y \isa{{\isasymnot}\ H}. 
    Por lo tanto, por hipótesis se cumple que o bien \isa{{\isasymnot}\ G} está en \isa{S} o bien \isa{{\isasymnot}\ H} está 
    en \isa{S}.
    \item[\isa{{\isadigit{8}}{\isacharparenright}}] Supongamos que \isa{{\isasymnot}{\isacharparenleft}G\ {\isasymor}\ H{\isacharparenright}} está en \isa{S} para fórmulas \isa{G} y \isa{H} cualesquiera.
    Por definición, \isa{{\isasymnot}{\isacharparenleft}G\ {\isasymor}\ H{\isacharparenright}} es una fórmula de tipo \isa{{\isasymalpha}} con componentes \isa{{\isasymnot}\ G} y \isa{{\isasymnot}\ H}. 
    Por lo tanto, por hipótesis se cumple que \isa{{\isasymnot}\ G} y \isa{{\isasymnot}\ H} están en \isa{S}.
    \item[\isa{{\isadigit{9}}{\isacharparenright}}] Supongamos que \isa{{\isasymnot}{\isacharparenleft}G\ {\isasymlongrightarrow}\ H{\isacharparenright}} está en \isa{S} para fórmulas \isa{G} y \isa{H} cualesquiera. 
    Por definición, \isa{{\isasymnot}{\isacharparenleft}G\ {\isasymlongrightarrow}\ H{\isacharparenright}} es una fórmula de tipo \isa{{\isasymalpha}} con componentes \isa{G} y \isa{{\isasymnot}\ H}.
    Por lo tanto, por hipótesis se cumple que \isa{G} y \isa{{\isasymnot}\ H} están en \isa{S}.
  \end{enumerate}

  Por tanto, queda probado el resultado.
\end{demostracion}

  Para probar de manera detallada el lema en Isabelle vamos a demostrar
  cada una de las implicaciones de la equivalencia por separado. 

  La primera implicación del lema se basa en dos lemas auxiliares. El primero de ellos 
  prueba que la tercera, sexta, octava y novena condición de la definición de conjunto de 
  Hintikka son suficientes para probar que para toda fórmula de tipo \isa{{\isasymalpha}} con componentes 
  \isa{{\isasymalpha}\isactrlsub {\isadigit{1}}} y \isa{{\isasymalpha}\isactrlsub {\isadigit{2}}} se verifica que si la fórmula pertenece al conjunto \isa{S}, entonces \isa{{\isasymalpha}\isactrlsub {\isadigit{1}}} y 
  \isa{{\isasymalpha}\isactrlsub {\isadigit{2}}} también. Su demostración detallada en Isabelle se muestra a continuación.%
\end{isamarkuptext}\isamarkuptrue%
\isacommand{lemma}\isamarkupfalse%
\ Hintikka{\isacharunderscore}alt{\isadigit{1}}Con{\isacharcolon}\isanewline
\ \ \isakeyword{assumes}\ {\isachardoublequoteopen}{\isacharparenleft}{\isasymforall}G\ H{\isachardot}\ G\ \isactrlbold {\isasymand}\ H\ {\isasymin}\ S\ {\isasymlongrightarrow}\ G\ {\isasymin}\ S\ {\isasymand}\ H\ {\isasymin}\ S{\isacharparenright}\isanewline
\ \ {\isasymand}\ {\isacharparenleft}{\isasymforall}G{\isachardot}\ \isactrlbold {\isasymnot}\ {\isacharparenleft}\isactrlbold {\isasymnot}\ G{\isacharparenright}\ {\isasymin}\ S\ {\isasymlongrightarrow}\ G\ {\isasymin}\ S{\isacharparenright}\isanewline
\ \ {\isasymand}\ {\isacharparenleft}{\isasymforall}G\ H{\isachardot}\ \isactrlbold {\isasymnot}{\isacharparenleft}G\ \isactrlbold {\isasymor}\ H{\isacharparenright}\ {\isasymin}\ S\ {\isasymlongrightarrow}\ \isactrlbold {\isasymnot}\ G\ {\isasymin}\ S\ {\isasymand}\ \isactrlbold {\isasymnot}\ H\ {\isasymin}\ S{\isacharparenright}\isanewline
\ \ {\isasymand}\ {\isacharparenleft}{\isasymforall}G\ H{\isachardot}\ \isactrlbold {\isasymnot}{\isacharparenleft}G\ \isactrlbold {\isasymrightarrow}\ H{\isacharparenright}\ {\isasymin}\ S\ {\isasymlongrightarrow}\ G\ {\isasymin}\ S\ {\isasymand}\ \isactrlbold {\isasymnot}\ H\ {\isasymin}\ S{\isacharparenright}{\isachardoublequoteclose}\isanewline
\ \ \isakeyword{shows}\ {\isachardoublequoteopen}Con\ F\ G\ H\ {\isasymlongrightarrow}\ F\ {\isasymin}\ S\ {\isasymlongrightarrow}\ G\ {\isasymin}\ S\ {\isasymand}\ H\ {\isasymin}\ S{\isachardoublequoteclose}\isanewline
%
\isadelimproof
%
\endisadelimproof
%
\isatagproof
\isacommand{proof}\isamarkupfalse%
\ {\isacharparenleft}rule\ impI{\isacharparenright}\isanewline
\ \ \isacommand{assume}\isamarkupfalse%
\ {\isachardoublequoteopen}Con\ F\ G\ H{\isachardoublequoteclose}\isanewline
\ \ \isacommand{then}\isamarkupfalse%
\ \isacommand{have}\isamarkupfalse%
\ {\isachardoublequoteopen}F\ {\isacharequal}\ G\ \isactrlbold {\isasymand}\ H\ {\isasymor}\ \isanewline
\ \ \ \ {\isacharparenleft}{\isacharparenleft}{\isasymexists}G{\isadigit{1}}\ H{\isadigit{1}}{\isachardot}\ F\ {\isacharequal}\ \isactrlbold {\isasymnot}\ {\isacharparenleft}G{\isadigit{1}}\ \isactrlbold {\isasymor}\ H{\isadigit{1}}{\isacharparenright}\ {\isasymand}\ G\ {\isacharequal}\ \isactrlbold {\isasymnot}\ G{\isadigit{1}}\ {\isasymand}\ H\ {\isacharequal}\ \isactrlbold {\isasymnot}\ H{\isadigit{1}}{\isacharparenright}\ {\isasymor}\ \isanewline
\ \ \ \ {\isacharparenleft}{\isasymexists}H{\isadigit{2}}{\isachardot}\ F\ {\isacharequal}\ \isactrlbold {\isasymnot}\ {\isacharparenleft}G\ \isactrlbold {\isasymrightarrow}\ H{\isadigit{2}}{\isacharparenright}\ {\isasymand}\ H\ {\isacharequal}\ \isactrlbold {\isasymnot}\ H{\isadigit{2}}{\isacharparenright}\ {\isasymor}\ \isanewline
\ \ \ \ F\ {\isacharequal}\ \isactrlbold {\isasymnot}\ {\isacharparenleft}\isactrlbold {\isasymnot}\ G{\isacharparenright}\ {\isasymand}\ H\ {\isacharequal}\ G{\isacharparenright}{\isachardoublequoteclose}\isanewline
\ \ \ \ \isacommand{by}\isamarkupfalse%
\ {\isacharparenleft}simp\ only{\isacharcolon}\ con{\isacharunderscore}dis{\isacharunderscore}simps{\isacharparenleft}{\isadigit{1}}{\isacharparenright}{\isacharparenright}\isanewline
\ \ \isacommand{thus}\isamarkupfalse%
\ {\isachardoublequoteopen}F\ {\isasymin}\ S\ {\isasymlongrightarrow}\ G\ {\isasymin}\ S\ {\isasymand}\ H\ {\isasymin}\ S{\isachardoublequoteclose}\isanewline
\ \ \isacommand{proof}\isamarkupfalse%
\ {\isacharparenleft}rule\ disjE{\isacharparenright}\isanewline
\ \ \ \ \isacommand{assume}\isamarkupfalse%
\ {\isachardoublequoteopen}F\ {\isacharequal}\ G\ \isactrlbold {\isasymand}\ H{\isachardoublequoteclose}\isanewline
\ \ \ \ \isacommand{have}\isamarkupfalse%
\ {\isachardoublequoteopen}{\isasymforall}G\ H{\isachardot}\ G\ \isactrlbold {\isasymand}\ H\ {\isasymin}\ S\ {\isasymlongrightarrow}\ G\ {\isasymin}\ S\ {\isasymand}\ H\ {\isasymin}\ S{\isachardoublequoteclose}\isanewline
\ \ \ \ \ \ \isacommand{using}\isamarkupfalse%
\ assms\ \isacommand{by}\isamarkupfalse%
\ {\isacharparenleft}rule\ conjunct{\isadigit{1}}{\isacharparenright}\isanewline
\ \ \ \ \isacommand{thus}\isamarkupfalse%
\ {\isachardoublequoteopen}F\ {\isasymin}\ S\ {\isasymlongrightarrow}\ G\ {\isasymin}\ S\ {\isasymand}\ H\ {\isasymin}\ S{\isachardoublequoteclose}\isanewline
\ \ \ \ \ \ \isacommand{using}\isamarkupfalse%
\ {\isacartoucheopen}F\ {\isacharequal}\ G\ \isactrlbold {\isasymand}\ H{\isacartoucheclose}\ \isacommand{by}\isamarkupfalse%
\ {\isacharparenleft}iprover\ elim{\isacharcolon}\ allE{\isacharparenright}\isanewline
\ \ \isacommand{next}\isamarkupfalse%
\ \isanewline
\ \ \ \ \isacommand{assume}\isamarkupfalse%
\ {\isachardoublequoteopen}{\isacharparenleft}{\isasymexists}G{\isadigit{1}}\ H{\isadigit{1}}{\isachardot}\ F\ {\isacharequal}\ \isactrlbold {\isasymnot}\ {\isacharparenleft}G{\isadigit{1}}\ \isactrlbold {\isasymor}\ H{\isadigit{1}}{\isacharparenright}\ {\isasymand}\ G\ {\isacharequal}\ \isactrlbold {\isasymnot}\ G{\isadigit{1}}\ {\isasymand}\ H\ {\isacharequal}\ \isactrlbold {\isasymnot}\ H{\isadigit{1}}{\isacharparenright}\ {\isasymor}\ \isanewline
\ \ \ \ {\isacharparenleft}{\isacharparenleft}{\isasymexists}H{\isadigit{2}}{\isachardot}\ F\ {\isacharequal}\ \isactrlbold {\isasymnot}\ {\isacharparenleft}G\ \isactrlbold {\isasymrightarrow}\ H{\isadigit{2}}{\isacharparenright}\ {\isasymand}\ H\ {\isacharequal}\ \isactrlbold {\isasymnot}\ H{\isadigit{2}}{\isacharparenright}\ {\isasymor}\ \isanewline
\ \ \ \ F\ {\isacharequal}\ \isactrlbold {\isasymnot}\ {\isacharparenleft}\isactrlbold {\isasymnot}\ G{\isacharparenright}\ {\isasymand}\ H\ {\isacharequal}\ G{\isacharparenright}{\isachardoublequoteclose}\isanewline
\ \ \ \ \isacommand{thus}\isamarkupfalse%
\ {\isachardoublequoteopen}F\ {\isasymin}\ S\ {\isasymlongrightarrow}\ G\ {\isasymin}\ S\ {\isasymand}\ H\ {\isasymin}\ S{\isachardoublequoteclose}\ \isanewline
\ \ \ \ \isacommand{proof}\isamarkupfalse%
\ {\isacharparenleft}rule\ disjE{\isacharparenright}\isanewline
\ \ \ \ \ \ \isacommand{assume}\isamarkupfalse%
\ E{\isadigit{1}}{\isacharcolon}{\isachardoublequoteopen}{\isasymexists}G{\isadigit{1}}\ H{\isadigit{1}}{\isachardot}\ F\ {\isacharequal}\ \isactrlbold {\isasymnot}\ {\isacharparenleft}G{\isadigit{1}}\ \isactrlbold {\isasymor}\ H{\isadigit{1}}{\isacharparenright}\ {\isasymand}\ G\ {\isacharequal}\ \isactrlbold {\isasymnot}\ G{\isadigit{1}}\ {\isasymand}\ H\ {\isacharequal}\ \isactrlbold {\isasymnot}\ H{\isadigit{1}}{\isachardoublequoteclose}\isanewline
\ \ \ \ \ \ \isacommand{obtain}\isamarkupfalse%
\ G{\isadigit{1}}\ H{\isadigit{1}}\ \isakeyword{where}\ A{\isadigit{1}}{\isacharcolon}{\isachardoublequoteopen}F\ {\isacharequal}\ \isactrlbold {\isasymnot}\ {\isacharparenleft}G{\isadigit{1}}\ \isactrlbold {\isasymor}\ H{\isadigit{1}}{\isacharparenright}\ {\isasymand}\ G\ {\isacharequal}\ \isactrlbold {\isasymnot}\ G{\isadigit{1}}\ {\isasymand}\ H\ {\isacharequal}\ \isactrlbold {\isasymnot}\ H{\isadigit{1}}{\isachardoublequoteclose}\isanewline
\ \ \ \ \ \ \ \ \isacommand{using}\isamarkupfalse%
\ E{\isadigit{1}}\ \isacommand{by}\isamarkupfalse%
\ {\isacharparenleft}iprover\ elim{\isacharcolon}\ exE{\isacharparenright}\isanewline
\ \ \ \ \ \ \isacommand{then}\isamarkupfalse%
\ \isacommand{have}\isamarkupfalse%
\ {\isachardoublequoteopen}F\ {\isacharequal}\ \isactrlbold {\isasymnot}\ {\isacharparenleft}G{\isadigit{1}}\ \isactrlbold {\isasymor}\ H{\isadigit{1}}{\isacharparenright}{\isachardoublequoteclose}\isanewline
\ \ \ \ \ \ \ \ \isacommand{by}\isamarkupfalse%
\ {\isacharparenleft}rule\ conjunct{\isadigit{1}}{\isacharparenright}\isanewline
\ \ \ \ \ \ \isacommand{have}\isamarkupfalse%
\ {\isachardoublequoteopen}G\ {\isacharequal}\ \isactrlbold {\isasymnot}\ G{\isadigit{1}}{\isachardoublequoteclose}\isanewline
\ \ \ \ \ \ \ \ \isacommand{using}\isamarkupfalse%
\ A{\isadigit{1}}\ \isacommand{by}\isamarkupfalse%
\ {\isacharparenleft}iprover\ elim{\isacharcolon}\ conjunct{\isadigit{1}}{\isacharparenright}\isanewline
\ \ \ \ \ \ \isacommand{have}\isamarkupfalse%
\ {\isachardoublequoteopen}H\ {\isacharequal}\ \isactrlbold {\isasymnot}\ H{\isadigit{1}}{\isachardoublequoteclose}\isanewline
\ \ \ \ \ \ \ \ \isacommand{using}\isamarkupfalse%
\ A{\isadigit{1}}\ \isacommand{by}\isamarkupfalse%
\ {\isacharparenleft}iprover\ elim{\isacharcolon}\ conjunct{\isadigit{1}}{\isacharparenright}\isanewline
\ \ \ \ \ \ \isacommand{have}\isamarkupfalse%
\ {\isachardoublequoteopen}{\isasymforall}G\ H{\isachardot}\ \isactrlbold {\isasymnot}{\isacharparenleft}G\ \isactrlbold {\isasymor}\ H{\isacharparenright}\ {\isasymin}\ S\ {\isasymlongrightarrow}\ \isactrlbold {\isasymnot}\ G\ {\isasymin}\ S\ {\isasymand}\ \isactrlbold {\isasymnot}\ H\ {\isasymin}\ S{\isachardoublequoteclose}\isanewline
\ \ \ \ \ \ \ \ \isacommand{using}\isamarkupfalse%
\ assms\ \isacommand{by}\isamarkupfalse%
\ {\isacharparenleft}iprover\ elim{\isacharcolon}\ conjunct{\isadigit{2}}\ conjunct{\isadigit{1}}{\isacharparenright}\isanewline
\ \ \ \ \ \ \isacommand{thus}\isamarkupfalse%
\ {\isachardoublequoteopen}F\ {\isasymin}\ S\ {\isasymlongrightarrow}\ G\ {\isasymin}\ S\ {\isasymand}\ H\ {\isasymin}\ S{\isachardoublequoteclose}\isanewline
\ \ \ \ \ \ \ \ \isacommand{using}\isamarkupfalse%
\ {\isacartoucheopen}F\ {\isacharequal}\ \isactrlbold {\isasymnot}\ {\isacharparenleft}G{\isadigit{1}}\ \isactrlbold {\isasymor}\ H{\isadigit{1}}{\isacharparenright}{\isacartoucheclose}\ {\isacartoucheopen}G\ {\isacharequal}\ \isactrlbold {\isasymnot}\ G{\isadigit{1}}{\isacartoucheclose}\ {\isacartoucheopen}H\ {\isacharequal}\ \isactrlbold {\isasymnot}\ H{\isadigit{1}}{\isacartoucheclose}\ \isacommand{by}\isamarkupfalse%
\ {\isacharparenleft}iprover\ elim{\isacharcolon}\ allE{\isacharparenright}\isanewline
\ \ \ \ \isacommand{next}\isamarkupfalse%
\isanewline
\ \ \ \ \ \ \isacommand{assume}\isamarkupfalse%
\ {\isachardoublequoteopen}{\isacharparenleft}{\isasymexists}H{\isadigit{2}}{\isachardot}\ F\ {\isacharequal}\ \isactrlbold {\isasymnot}\ {\isacharparenleft}G\ \isactrlbold {\isasymrightarrow}\ H{\isadigit{2}}{\isacharparenright}\ {\isasymand}\ H\ {\isacharequal}\ \isactrlbold {\isasymnot}\ H{\isadigit{2}}{\isacharparenright}\ {\isasymor}\ \isanewline
\ \ \ \ \ \ F\ {\isacharequal}\ \isactrlbold {\isasymnot}\ {\isacharparenleft}\isactrlbold {\isasymnot}\ G{\isacharparenright}\ {\isasymand}\ H\ {\isacharequal}\ G{\isachardoublequoteclose}\isanewline
\ \ \ \ \ \ \isacommand{thus}\isamarkupfalse%
\ {\isachardoublequoteopen}F\ {\isasymin}\ S\ {\isasymlongrightarrow}\ G\ {\isasymin}\ S\ {\isasymand}\ H\ {\isasymin}\ S{\isachardoublequoteclose}\ \isanewline
\ \ \ \ \ \ \isacommand{proof}\isamarkupfalse%
\ {\isacharparenleft}rule\ disjE{\isacharparenright}\isanewline
\ \ \ \ \ \ \ \ \isacommand{assume}\isamarkupfalse%
\ E{\isadigit{2}}{\isacharcolon}{\isachardoublequoteopen}{\isasymexists}H{\isadigit{2}}{\isachardot}\ F\ {\isacharequal}\ \isactrlbold {\isasymnot}\ {\isacharparenleft}G\ \isactrlbold {\isasymrightarrow}\ H{\isadigit{2}}{\isacharparenright}\ {\isasymand}\ H\ {\isacharequal}\ \isactrlbold {\isasymnot}\ H{\isadigit{2}}{\isachardoublequoteclose}\isanewline
\ \ \ \ \ \ \ \ \isacommand{obtain}\isamarkupfalse%
\ H{\isadigit{2}}\ \isakeyword{where}\ A{\isadigit{2}}{\isacharcolon}{\isachardoublequoteopen}F\ {\isacharequal}\ \isactrlbold {\isasymnot}\ {\isacharparenleft}G\ \isactrlbold {\isasymrightarrow}\ H{\isadigit{2}}{\isacharparenright}\ {\isasymand}\ H\ {\isacharequal}\ \isactrlbold {\isasymnot}\ H{\isadigit{2}}{\isachardoublequoteclose}\isanewline
\ \ \ \ \ \ \ \ \ \ \isacommand{using}\isamarkupfalse%
\ E{\isadigit{2}}\ \isacommand{by}\isamarkupfalse%
\ {\isacharparenleft}rule\ exE{\isacharparenright}\isanewline
\ \ \ \ \ \ \ \ \isacommand{have}\isamarkupfalse%
\ {\isachardoublequoteopen}F\ {\isacharequal}\ \isactrlbold {\isasymnot}\ {\isacharparenleft}G\ \isactrlbold {\isasymrightarrow}\ H{\isadigit{2}}{\isacharparenright}{\isachardoublequoteclose}\isanewline
\ \ \ \ \ \ \ \ \ \ \isacommand{using}\isamarkupfalse%
\ A{\isadigit{2}}\ \isacommand{by}\isamarkupfalse%
\ {\isacharparenleft}rule\ conjunct{\isadigit{1}}{\isacharparenright}\isanewline
\ \ \ \ \ \ \ \ \isacommand{have}\isamarkupfalse%
\ {\isachardoublequoteopen}H\ {\isacharequal}\ \isactrlbold {\isasymnot}\ H{\isadigit{2}}{\isachardoublequoteclose}\isanewline
\ \ \ \ \ \ \ \ \ \ \isacommand{using}\isamarkupfalse%
\ A{\isadigit{2}}\ \isacommand{by}\isamarkupfalse%
\ {\isacharparenleft}rule\ conjunct{\isadigit{2}}{\isacharparenright}\isanewline
\ \ \ \ \ \ \ \ \isacommand{have}\isamarkupfalse%
\ {\isachardoublequoteopen}{\isasymforall}G\ H{\isachardot}\ \isactrlbold {\isasymnot}{\isacharparenleft}G\ \isactrlbold {\isasymrightarrow}\ H{\isacharparenright}\ {\isasymin}\ S\ {\isasymlongrightarrow}\ G\ {\isasymin}\ S\ {\isasymand}\ \isactrlbold {\isasymnot}\ H\ {\isasymin}\ S{\isachardoublequoteclose}\isanewline
\ \ \ \ \ \ \ \ \ \ \isacommand{using}\isamarkupfalse%
\ assms\ \isacommand{by}\isamarkupfalse%
\ {\isacharparenleft}iprover\ elim{\isacharcolon}\ conjunct{\isadigit{2}}\ conjunct{\isadigit{1}}{\isacharparenright}\isanewline
\ \ \ \ \ \ \ \ \isacommand{thus}\isamarkupfalse%
\ {\isachardoublequoteopen}F\ {\isasymin}\ S\ {\isasymlongrightarrow}\ G\ {\isasymin}\ S\ {\isasymand}\ H\ {\isasymin}\ S{\isachardoublequoteclose}\isanewline
\ \ \ \ \ \ \ \ \ \ \isacommand{using}\isamarkupfalse%
\ {\isacartoucheopen}F\ {\isacharequal}\ \isactrlbold {\isasymnot}\ {\isacharparenleft}G\ \isactrlbold {\isasymrightarrow}\ H{\isadigit{2}}{\isacharparenright}{\isacartoucheclose}\ {\isacartoucheopen}H\ {\isacharequal}\ \isactrlbold {\isasymnot}\ H{\isadigit{2}}{\isacartoucheclose}\ \isacommand{by}\isamarkupfalse%
\ {\isacharparenleft}iprover\ elim{\isacharcolon}\ allE{\isacharparenright}\isanewline
\ \ \ \ \ \ \isacommand{next}\isamarkupfalse%
\ \isanewline
\ \ \ \ \ \ \ \ \isacommand{assume}\isamarkupfalse%
\ {\isachardoublequoteopen}F\ {\isacharequal}\ \isactrlbold {\isasymnot}\ {\isacharparenleft}\isactrlbold {\isasymnot}\ G{\isacharparenright}\ {\isasymand}\ H\ {\isacharequal}\ G{\isachardoublequoteclose}\isanewline
\ \ \ \ \ \ \ \ \isacommand{then}\isamarkupfalse%
\ \isacommand{have}\isamarkupfalse%
\ {\isachardoublequoteopen}F\ {\isacharequal}\ \isactrlbold {\isasymnot}\ {\isacharparenleft}\isactrlbold {\isasymnot}\ G{\isacharparenright}{\isachardoublequoteclose}\isanewline
\ \ \ \ \ \ \ \ \ \ \isacommand{by}\isamarkupfalse%
\ {\isacharparenleft}rule\ conjunct{\isadigit{1}}{\isacharparenright}\isanewline
\ \ \ \ \ \ \ \ \isacommand{have}\isamarkupfalse%
\ {\isachardoublequoteopen}H\ {\isacharequal}\ G{\isachardoublequoteclose}\isanewline
\ \ \ \ \ \ \ \ \ \ \isacommand{using}\isamarkupfalse%
\ {\isacartoucheopen}F\ {\isacharequal}\ \isactrlbold {\isasymnot}\ {\isacharparenleft}\isactrlbold {\isasymnot}\ G{\isacharparenright}\ {\isasymand}\ H\ {\isacharequal}\ G{\isacartoucheclose}\ \isacommand{by}\isamarkupfalse%
\ {\isacharparenleft}rule\ conjunct{\isadigit{2}}{\isacharparenright}\isanewline
\ \ \ \ \ \ \ \ \isacommand{have}\isamarkupfalse%
\ {\isachardoublequoteopen}{\isasymforall}G{\isachardot}\ \isactrlbold {\isasymnot}\ {\isacharparenleft}\isactrlbold {\isasymnot}\ G{\isacharparenright}\ {\isasymin}\ S\ {\isasymlongrightarrow}\ G\ {\isasymin}\ S{\isachardoublequoteclose}\isanewline
\ \ \ \ \ \ \ \ \ \ \isacommand{using}\isamarkupfalse%
\ assms\ \isacommand{by}\isamarkupfalse%
\ {\isacharparenleft}iprover\ elim{\isacharcolon}\ conjunct{\isadigit{2}}\ conjunct{\isadigit{1}}{\isacharparenright}\isanewline
\ \ \ \ \ \ \ \ \isacommand{then}\isamarkupfalse%
\ \isacommand{have}\isamarkupfalse%
\ {\isachardoublequoteopen}\isactrlbold {\isasymnot}\ {\isacharparenleft}\isactrlbold {\isasymnot}\ G{\isacharparenright}\ {\isasymin}\ S\ {\isasymlongrightarrow}\ G\ {\isasymin}\ S{\isachardoublequoteclose}\isanewline
\ \ \ \ \ \ \ \ \ \ \isacommand{by}\isamarkupfalse%
\ {\isacharparenleft}rule\ allE{\isacharparenright}\isanewline
\ \ \ \ \ \ \ \ \isacommand{then}\isamarkupfalse%
\ \isacommand{have}\isamarkupfalse%
\ {\isachardoublequoteopen}F\ {\isasymin}\ S\ {\isasymlongrightarrow}\ G\ {\isasymin}\ S{\isachardoublequoteclose}\isanewline
\ \ \ \ \ \ \ \ \ \ \isacommand{by}\isamarkupfalse%
\ {\isacharparenleft}simp\ only{\isacharcolon}\ {\isacartoucheopen}F\ {\isacharequal}\ \isactrlbold {\isasymnot}\ {\isacharparenleft}\isactrlbold {\isasymnot}\ G{\isacharparenright}{\isacartoucheclose}{\isacharparenright}\ \isanewline
\ \ \ \ \ \ \ \ \isacommand{then}\isamarkupfalse%
\ \isacommand{have}\isamarkupfalse%
\ {\isachardoublequoteopen}F\ {\isasymin}\ S\ {\isasymlongrightarrow}\ G\ {\isasymin}\ S\ {\isasymand}\ G\ {\isasymin}\ S{\isachardoublequoteclose}\isanewline
\ \ \ \ \ \ \ \ \ \ \isacommand{by}\isamarkupfalse%
\ {\isacharparenleft}simp\ only{\isacharcolon}\ conj{\isacharunderscore}absorb{\isacharparenright}\isanewline
\ \ \ \ \ \ \ \ \isacommand{thus}\isamarkupfalse%
\ {\isachardoublequoteopen}F\ {\isasymin}\ S\ {\isasymlongrightarrow}\ G\ {\isasymin}\ S\ {\isasymand}\ H\ {\isasymin}\ S{\isachardoublequoteclose}\isanewline
\ \ \ \ \ \ \ \ \ \ \isacommand{by}\isamarkupfalse%
\ {\isacharparenleft}simp\ only{\isacharcolon}\ {\isacartoucheopen}H{\isacharequal}G{\isacartoucheclose}{\isacharparenright}\isanewline
\ \ \ \ \ \ \isacommand{qed}\isamarkupfalse%
\isanewline
\ \ \ \ \isacommand{qed}\isamarkupfalse%
\isanewline
\ \ \isacommand{qed}\isamarkupfalse%
\isanewline
\isacommand{qed}\isamarkupfalse%
%
\endisatagproof
{\isafoldproof}%
%
\isadelimproof
%
\endisadelimproof
%
\begin{isamarkuptext}%
Por otro lado, el segundo lema auxiliar prueba que la cuarta, quinta, sexta
  y séptima condición de la definición de conjunto de Hintikka son suficientes para
  probar que para toda fórmula de tipo \isa{{\isasymbeta}} con componentes \isa{{\isasymbeta}\isactrlsub {\isadigit{1}}} y \isa{{\isasymbeta}\isactrlsub {\isadigit{2}}} se verifica 
  que si la fórmula pertenece al conjunto \isa{S}, entonces o bien \isa{{\isasymbeta}\isactrlsub {\isadigit{1}}} pertenece al
  conjunto o bien \isa{{\isasymbeta}\isactrlsub {\isadigit{2}}} pertenece al conjunto. Veamos su prueba detallada en 
  Isabelle/HOL.%
\end{isamarkuptext}\isamarkuptrue%
\isacommand{lemma}\isamarkupfalse%
\ Hintikka{\isacharunderscore}alt{\isadigit{1}}Dis{\isacharcolon}\isanewline
\ \ \isakeyword{assumes}\ \ {\isachardoublequoteopen}{\isacharparenleft}{\isasymforall}\ G\ H{\isachardot}\ G\ \isactrlbold {\isasymor}\ H\ {\isasymin}\ S\ {\isasymlongrightarrow}\ G\ {\isasymin}\ S\ {\isasymor}\ H\ {\isasymin}\ S{\isacharparenright}\isanewline
\ \ {\isasymand}\ {\isacharparenleft}{\isasymforall}\ G\ H{\isachardot}\ G\ \isactrlbold {\isasymrightarrow}\ H\ {\isasymin}\ S\ {\isasymlongrightarrow}\ \isactrlbold {\isasymnot}\ G\ {\isasymin}\ S\ {\isasymor}\ H\ {\isasymin}\ S{\isacharparenright}\isanewline
\ \ {\isasymand}\ {\isacharparenleft}{\isasymforall}\ G{\isachardot}\ \isactrlbold {\isasymnot}\ {\isacharparenleft}\isactrlbold {\isasymnot}\ G{\isacharparenright}\ {\isasymin}\ S\ {\isasymlongrightarrow}\ G\ {\isasymin}\ S{\isacharparenright}\isanewline
\ \ {\isasymand}\ {\isacharparenleft}{\isasymforall}\ G\ H{\isachardot}\ \isactrlbold {\isasymnot}{\isacharparenleft}G\ \isactrlbold {\isasymand}\ H{\isacharparenright}\ {\isasymin}\ S\ {\isasymlongrightarrow}\ \isactrlbold {\isasymnot}\ G\ {\isasymin}\ S\ {\isasymor}\ \isactrlbold {\isasymnot}\ H\ {\isasymin}\ S{\isacharparenright}{\isachardoublequoteclose}\isanewline
\ \ \isakeyword{shows}\ {\isachardoublequoteopen}Dis\ F\ G\ H\ {\isasymlongrightarrow}\ F\ {\isasymin}\ S\ {\isasymlongrightarrow}\ G\ {\isasymin}\ S\ {\isasymor}\ H\ {\isasymin}\ S{\isachardoublequoteclose}\isanewline
%
\isadelimproof
%
\endisadelimproof
%
\isatagproof
\isacommand{proof}\isamarkupfalse%
\ {\isacharparenleft}rule\ impI{\isacharparenright}\isanewline
\ \ \isacommand{assume}\isamarkupfalse%
\ {\isachardoublequoteopen}Dis\ F\ G\ H{\isachardoublequoteclose}\isanewline
\ \ \isacommand{then}\isamarkupfalse%
\ \isacommand{have}\isamarkupfalse%
\ {\isachardoublequoteopen}F\ {\isacharequal}\ G\ \isactrlbold {\isasymor}\ H\ {\isasymor}\ \isanewline
\ \ \ \ {\isacharparenleft}{\isasymexists}G{\isadigit{1}}\ H{\isadigit{1}}{\isachardot}\ F\ {\isacharequal}\ G{\isadigit{1}}\ \isactrlbold {\isasymrightarrow}\ H{\isadigit{1}}\ {\isasymand}\ G\ {\isacharequal}\ \isactrlbold {\isasymnot}\ G{\isadigit{1}}\ {\isasymand}\ H\ {\isacharequal}\ H{\isadigit{1}}{\isacharparenright}\ {\isasymor}\ \isanewline
\ \ \ \ {\isacharparenleft}{\isasymexists}G{\isadigit{2}}\ H{\isadigit{2}}{\isachardot}\ F\ {\isacharequal}\ \isactrlbold {\isasymnot}\ {\isacharparenleft}G{\isadigit{2}}\ \isactrlbold {\isasymand}\ H{\isadigit{2}}{\isacharparenright}\ {\isasymand}\ G\ {\isacharequal}\ \isactrlbold {\isasymnot}\ G{\isadigit{2}}\ {\isasymand}\ H\ {\isacharequal}\ \isactrlbold {\isasymnot}\ H{\isadigit{2}}{\isacharparenright}\ {\isasymor}\ \isanewline
\ \ \ \ F\ {\isacharequal}\ \isactrlbold {\isasymnot}\ {\isacharparenleft}\isactrlbold {\isasymnot}\ G{\isacharparenright}\ {\isasymand}\ H\ {\isacharequal}\ G{\isachardoublequoteclose}\ \isanewline
\ \ \ \ \isacommand{by}\isamarkupfalse%
\ {\isacharparenleft}simp\ only{\isacharcolon}\ con{\isacharunderscore}dis{\isacharunderscore}simps{\isacharparenleft}{\isadigit{2}}{\isacharparenright}{\isacharparenright}\isanewline
\ \ \isacommand{thus}\isamarkupfalse%
\ {\isachardoublequoteopen}F\ {\isasymin}\ S\ {\isasymlongrightarrow}\ G\ {\isasymin}\ S\ {\isasymor}\ H\ {\isasymin}\ S{\isachardoublequoteclose}\ \isanewline
\ \ \isacommand{proof}\isamarkupfalse%
\ {\isacharparenleft}rule\ disjE{\isacharparenright}\isanewline
\ \ \ \ \isacommand{assume}\isamarkupfalse%
\ {\isachardoublequoteopen}F\ {\isacharequal}\ G\ \isactrlbold {\isasymor}\ H{\isachardoublequoteclose}\isanewline
\ \ \ \ \isacommand{have}\isamarkupfalse%
\ {\isachardoublequoteopen}{\isasymforall}G\ H{\isachardot}\ G\ \isactrlbold {\isasymor}\ H\ {\isasymin}\ S\ {\isasymlongrightarrow}\ G\ {\isasymin}\ S\ {\isasymor}\ H\ {\isasymin}\ S{\isachardoublequoteclose}\isanewline
\ \ \ \ \ \ \isacommand{using}\isamarkupfalse%
\ assms\ \isacommand{by}\isamarkupfalse%
\ {\isacharparenleft}rule\ conjunct{\isadigit{1}}{\isacharparenright}\isanewline
\ \ \ \ \isacommand{thus}\isamarkupfalse%
\ {\isachardoublequoteopen}F\ {\isasymin}\ S\ {\isasymlongrightarrow}\ G\ {\isasymin}\ S\ {\isasymor}\ H\ {\isasymin}\ S{\isachardoublequoteclose}\ \isanewline
\ \ \ \ \ \ \isacommand{using}\isamarkupfalse%
\ {\isacartoucheopen}F\ {\isacharequal}\ G\ \isactrlbold {\isasymor}\ H{\isacartoucheclose}\ \isacommand{by}\isamarkupfalse%
\ {\isacharparenleft}iprover\ elim{\isacharcolon}\ allE{\isacharparenright}\isanewline
\ \ \isacommand{next}\isamarkupfalse%
\isanewline
\ \ \ \ \isacommand{assume}\isamarkupfalse%
\ {\isachardoublequoteopen}{\isacharparenleft}{\isasymexists}G{\isadigit{1}}\ H{\isadigit{1}}{\isachardot}\ F\ {\isacharequal}\ G{\isadigit{1}}\ \isactrlbold {\isasymrightarrow}\ H{\isadigit{1}}\ {\isasymand}\ G\ {\isacharequal}\ \isactrlbold {\isasymnot}\ G{\isadigit{1}}\ {\isasymand}\ H\ {\isacharequal}\ H{\isadigit{1}}{\isacharparenright}\ {\isasymor}\ \isanewline
\ \ \ \ {\isacharparenleft}{\isasymexists}G{\isadigit{2}}\ H{\isadigit{2}}{\isachardot}\ F\ {\isacharequal}\ \isactrlbold {\isasymnot}\ {\isacharparenleft}G{\isadigit{2}}\ \isactrlbold {\isasymand}\ H{\isadigit{2}}{\isacharparenright}\ {\isasymand}\ G\ {\isacharequal}\ \isactrlbold {\isasymnot}\ G{\isadigit{2}}\ {\isasymand}\ H\ {\isacharequal}\ \isactrlbold {\isasymnot}\ H{\isadigit{2}}{\isacharparenright}\ {\isasymor}\ \isanewline
\ \ \ \ F\ {\isacharequal}\ \isactrlbold {\isasymnot}\ {\isacharparenleft}\isactrlbold {\isasymnot}\ G{\isacharparenright}\ {\isasymand}\ H\ {\isacharequal}\ G{\isachardoublequoteclose}\isanewline
\ \ \ \ \isacommand{thus}\isamarkupfalse%
\ {\isachardoublequoteopen}F\ {\isasymin}\ S\ {\isasymlongrightarrow}\ G\ {\isasymin}\ S\ {\isasymor}\ H\ {\isasymin}\ S{\isachardoublequoteclose}\isanewline
\ \ \ \ \isacommand{proof}\isamarkupfalse%
\ {\isacharparenleft}rule\ disjE{\isacharparenright}\isanewline
\ \ \ \ \ \ \isacommand{assume}\isamarkupfalse%
\ E{\isadigit{1}}{\isacharcolon}{\isachardoublequoteopen}{\isasymexists}G{\isadigit{1}}\ H{\isadigit{1}}{\isachardot}\ F\ {\isacharequal}\ G{\isadigit{1}}\ \isactrlbold {\isasymrightarrow}\ H{\isadigit{1}}\ {\isasymand}\ G\ {\isacharequal}\ \isactrlbold {\isasymnot}\ G{\isadigit{1}}\ {\isasymand}\ H\ {\isacharequal}\ H{\isadigit{1}}{\isachardoublequoteclose}\isanewline
\ \ \ \ \ \ \isacommand{obtain}\isamarkupfalse%
\ G{\isadigit{1}}\ H{\isadigit{1}}\ \isakeyword{where}\ A{\isadigit{1}}{\isacharcolon}{\isachardoublequoteopen}F\ {\isacharequal}\ G{\isadigit{1}}\ \isactrlbold {\isasymrightarrow}\ H{\isadigit{1}}\ {\isasymand}\ G\ {\isacharequal}\ \isactrlbold {\isasymnot}\ G{\isadigit{1}}\ {\isasymand}\ H\ {\isacharequal}\ H{\isadigit{1}}{\isachardoublequoteclose}\isanewline
\ \ \ \ \ \ \ \ \isacommand{using}\isamarkupfalse%
\ E{\isadigit{1}}\ \isacommand{by}\isamarkupfalse%
\ {\isacharparenleft}iprover\ elim{\isacharcolon}\ exE{\isacharparenright}\isanewline
\ \ \ \ \ \ \isacommand{have}\isamarkupfalse%
\ {\isachardoublequoteopen}F\ {\isacharequal}\ G{\isadigit{1}}\ \isactrlbold {\isasymrightarrow}\ H{\isadigit{1}}{\isachardoublequoteclose}\isanewline
\ \ \ \ \ \ \ \ \isacommand{using}\isamarkupfalse%
\ A{\isadigit{1}}\ \isacommand{by}\isamarkupfalse%
\ {\isacharparenleft}rule\ conjunct{\isadigit{1}}{\isacharparenright}\isanewline
\ \ \ \ \ \ \isacommand{have}\isamarkupfalse%
\ {\isachardoublequoteopen}G\ {\isacharequal}\ \isactrlbold {\isasymnot}\ G{\isadigit{1}}{\isachardoublequoteclose}\isanewline
\ \ \ \ \ \ \ \ \isacommand{using}\isamarkupfalse%
\ A{\isadigit{1}}\ \isacommand{by}\isamarkupfalse%
\ {\isacharparenleft}iprover\ elim{\isacharcolon}\ conjunct{\isadigit{1}}{\isacharparenright}\isanewline
\ \ \ \ \ \ \isacommand{have}\isamarkupfalse%
\ {\isachardoublequoteopen}H\ {\isacharequal}\ H{\isadigit{1}}{\isachardoublequoteclose}\isanewline
\ \ \ \ \ \ \ \ \isacommand{using}\isamarkupfalse%
\ A{\isadigit{1}}\ \isacommand{by}\isamarkupfalse%
\ {\isacharparenleft}iprover\ elim{\isacharcolon}\ conjunct{\isadigit{2}}\ conjunct{\isadigit{1}}{\isacharparenright}\isanewline
\ \ \ \ \ \ \isacommand{have}\isamarkupfalse%
\ {\isachardoublequoteopen}{\isasymforall}G\ H{\isachardot}\ G\ \isactrlbold {\isasymrightarrow}\ H\ {\isasymin}\ S\ {\isasymlongrightarrow}\ \isactrlbold {\isasymnot}\ G\ {\isasymin}\ S\ {\isasymor}\ H\ {\isasymin}\ S{\isachardoublequoteclose}\isanewline
\ \ \ \ \ \ \ \ \isacommand{using}\isamarkupfalse%
\ assms\ \isacommand{by}\isamarkupfalse%
\ {\isacharparenleft}iprover\ elim{\isacharcolon}\ conjunct{\isadigit{2}}\ conjunct{\isadigit{1}}{\isacharparenright}\isanewline
\ \ \ \ \ \ \isacommand{thus}\isamarkupfalse%
\ {\isachardoublequoteopen}F\ {\isasymin}\ S\ {\isasymlongrightarrow}\ G\ {\isasymin}\ S\ {\isasymor}\ H\ {\isasymin}\ S{\isachardoublequoteclose}\isanewline
\ \ \ \ \ \ \ \ \isacommand{using}\isamarkupfalse%
\ {\isacartoucheopen}F\ {\isacharequal}\ G{\isadigit{1}}\ \isactrlbold {\isasymrightarrow}\ H{\isadigit{1}}{\isacartoucheclose}\ {\isacartoucheopen}G\ {\isacharequal}\ \isactrlbold {\isasymnot}\ G{\isadigit{1}}{\isacartoucheclose}\ {\isacartoucheopen}H\ {\isacharequal}\ H{\isadigit{1}}{\isacartoucheclose}\ \isacommand{by}\isamarkupfalse%
\ {\isacharparenleft}iprover\ elim{\isacharcolon}\ allE{\isacharparenright}\isanewline
\ \ \ \ \isacommand{next}\isamarkupfalse%
\isanewline
\ \ \ \ \ \ \isacommand{assume}\isamarkupfalse%
\ {\isachardoublequoteopen}{\isacharparenleft}{\isasymexists}G{\isadigit{2}}\ H{\isadigit{2}}{\isachardot}\ F\ {\isacharequal}\ \isactrlbold {\isasymnot}\ {\isacharparenleft}G{\isadigit{2}}\ \isactrlbold {\isasymand}\ H{\isadigit{2}}{\isacharparenright}\ {\isasymand}\ G\ {\isacharequal}\ \isactrlbold {\isasymnot}\ G{\isadigit{2}}\ {\isasymand}\ H\ {\isacharequal}\ \isactrlbold {\isasymnot}\ H{\isadigit{2}}{\isacharparenright}\ {\isasymor}\ \isanewline
\ \ \ \ \ \ F\ {\isacharequal}\ \isactrlbold {\isasymnot}\ {\isacharparenleft}\isactrlbold {\isasymnot}\ G{\isacharparenright}\ {\isasymand}\ H\ {\isacharequal}\ G{\isachardoublequoteclose}\isanewline
\ \ \ \ \ \ \isacommand{thus}\isamarkupfalse%
\ {\isachardoublequoteopen}F\ {\isasymin}\ S\ {\isasymlongrightarrow}\ G\ {\isasymin}\ S\ {\isasymor}\ H\ {\isasymin}\ S{\isachardoublequoteclose}\isanewline
\ \ \ \ \ \ \isacommand{proof}\isamarkupfalse%
\ {\isacharparenleft}rule\ disjE{\isacharparenright}\isanewline
\ \ \ \ \ \ \ \ \isacommand{assume}\isamarkupfalse%
\ E{\isadigit{2}}{\isacharcolon}{\isachardoublequoteopen}{\isasymexists}G{\isadigit{2}}\ H{\isadigit{2}}{\isachardot}\ F\ {\isacharequal}\ \isactrlbold {\isasymnot}\ {\isacharparenleft}G{\isadigit{2}}\ \isactrlbold {\isasymand}\ H{\isadigit{2}}{\isacharparenright}\ {\isasymand}\ G\ {\isacharequal}\ \isactrlbold {\isasymnot}\ G{\isadigit{2}}\ {\isasymand}\ H\ {\isacharequal}\ \isactrlbold {\isasymnot}\ H{\isadigit{2}}{\isachardoublequoteclose}\isanewline
\ \ \ \ \ \ \ \ \isacommand{obtain}\isamarkupfalse%
\ G{\isadigit{2}}\ H{\isadigit{2}}\ \isakeyword{where}\ A{\isadigit{2}}{\isacharcolon}{\isachardoublequoteopen}F\ {\isacharequal}\ \isactrlbold {\isasymnot}\ {\isacharparenleft}G{\isadigit{2}}\ \isactrlbold {\isasymand}\ H{\isadigit{2}}{\isacharparenright}\ {\isasymand}\ G\ {\isacharequal}\ \isactrlbold {\isasymnot}\ G{\isadigit{2}}\ {\isasymand}\ H\ {\isacharequal}\ \isactrlbold {\isasymnot}\ H{\isadigit{2}}{\isachardoublequoteclose}\ \isanewline
\ \ \ \ \ \ \ \ \ \ \isacommand{using}\isamarkupfalse%
\ E{\isadigit{2}}\ \isacommand{by}\isamarkupfalse%
\ {\isacharparenleft}iprover\ elim{\isacharcolon}\ exE{\isacharparenright}\isanewline
\ \ \ \ \ \ \ \ \isacommand{have}\isamarkupfalse%
\ {\isachardoublequoteopen}F\ {\isacharequal}\ \isactrlbold {\isasymnot}\ {\isacharparenleft}G{\isadigit{2}}\ \isactrlbold {\isasymand}\ H{\isadigit{2}}{\isacharparenright}{\isachardoublequoteclose}\ \isanewline
\ \ \ \ \ \ \ \ \ \ \isacommand{using}\isamarkupfalse%
\ A{\isadigit{2}}\ \isacommand{by}\isamarkupfalse%
\ {\isacharparenleft}rule\ conjunct{\isadigit{1}}{\isacharparenright}\isanewline
\ \ \ \ \ \ \ \ \isacommand{have}\isamarkupfalse%
\ {\isachardoublequoteopen}G\ {\isacharequal}\ \isactrlbold {\isasymnot}\ G{\isadigit{2}}{\isachardoublequoteclose}\isanewline
\ \ \ \ \ \ \ \ \ \ \isacommand{using}\isamarkupfalse%
\ A{\isadigit{2}}\ \isacommand{by}\isamarkupfalse%
\ {\isacharparenleft}iprover\ elim{\isacharcolon}\ conjunct{\isadigit{2}}\ conjunct{\isadigit{1}}{\isacharparenright}\isanewline
\ \ \ \ \ \ \ \ \isacommand{have}\isamarkupfalse%
\ {\isachardoublequoteopen}H\ {\isacharequal}\ \isactrlbold {\isasymnot}\ H{\isadigit{2}}{\isachardoublequoteclose}\isanewline
\ \ \ \ \ \ \ \ \ \ \isacommand{using}\isamarkupfalse%
\ A{\isadigit{2}}\ \isacommand{by}\isamarkupfalse%
\ {\isacharparenleft}iprover\ elim{\isacharcolon}\ conjunct{\isadigit{1}}{\isacharparenright}\isanewline
\ \ \ \ \ \ \ \ \isacommand{have}\isamarkupfalse%
\ {\isachardoublequoteopen}{\isasymforall}\ G\ H{\isachardot}\ \isactrlbold {\isasymnot}{\isacharparenleft}G\ \isactrlbold {\isasymand}\ H{\isacharparenright}\ {\isasymin}\ S\ {\isasymlongrightarrow}\ \isactrlbold {\isasymnot}\ G\ {\isasymin}\ S\ {\isasymor}\ \isactrlbold {\isasymnot}\ H\ {\isasymin}\ S{\isachardoublequoteclose}\isanewline
\ \ \ \ \ \ \ \ \ \ \isacommand{using}\isamarkupfalse%
\ assms\ \isacommand{by}\isamarkupfalse%
\ {\isacharparenleft}iprover\ elim{\isacharcolon}\ conjunct{\isadigit{2}}\ conjunct{\isadigit{1}}{\isacharparenright}\isanewline
\ \ \ \ \ \ \ \ \isacommand{thus}\isamarkupfalse%
\ {\isachardoublequoteopen}F\ {\isasymin}\ S\ {\isasymlongrightarrow}\ G\ {\isasymin}\ S\ {\isasymor}\ H\ {\isasymin}\ S{\isachardoublequoteclose}\isanewline
\ \ \ \ \ \ \ \ \ \ \isacommand{using}\isamarkupfalse%
\ {\isacartoucheopen}F\ {\isacharequal}\ \isactrlbold {\isasymnot}{\isacharparenleft}G{\isadigit{2}}\ \isactrlbold {\isasymand}\ H{\isadigit{2}}{\isacharparenright}{\isacartoucheclose}\ {\isacartoucheopen}G\ {\isacharequal}\ \isactrlbold {\isasymnot}\ G{\isadigit{2}}{\isacartoucheclose}\ {\isacartoucheopen}H\ {\isacharequal}\ \isactrlbold {\isasymnot}\ H{\isadigit{2}}{\isacartoucheclose}\ \isacommand{by}\isamarkupfalse%
\ {\isacharparenleft}iprover\ elim{\isacharcolon}\ allE{\isacharparenright}\isanewline
\ \ \ \ \ \ \isacommand{next}\isamarkupfalse%
\isanewline
\ \ \ \ \ \ \ \ \isacommand{assume}\isamarkupfalse%
\ {\isachardoublequoteopen}F\ {\isacharequal}\ \isactrlbold {\isasymnot}\ {\isacharparenleft}\isactrlbold {\isasymnot}\ G{\isacharparenright}\ {\isasymand}\ H\ {\isacharequal}\ G{\isachardoublequoteclose}\isanewline
\ \ \ \ \ \ \ \ \isacommand{then}\isamarkupfalse%
\ \isacommand{have}\isamarkupfalse%
\ {\isachardoublequoteopen}F\ {\isacharequal}\ \isactrlbold {\isasymnot}\ {\isacharparenleft}\isactrlbold {\isasymnot}\ G{\isacharparenright}{\isachardoublequoteclose}\ \isanewline
\ \ \ \ \ \ \ \ \ \ \isacommand{by}\isamarkupfalse%
\ {\isacharparenleft}rule\ conjunct{\isadigit{1}}{\isacharparenright}\isanewline
\ \ \ \ \ \ \ \ \isacommand{have}\isamarkupfalse%
\ {\isachardoublequoteopen}H\ {\isacharequal}\ G{\isachardoublequoteclose}\isanewline
\ \ \ \ \ \ \ \ \ \ \isacommand{using}\isamarkupfalse%
\ {\isacartoucheopen}F\ {\isacharequal}\ \isactrlbold {\isasymnot}\ {\isacharparenleft}\isactrlbold {\isasymnot}\ G{\isacharparenright}\ {\isasymand}\ H\ {\isacharequal}\ G{\isacartoucheclose}\ \isacommand{by}\isamarkupfalse%
\ {\isacharparenleft}rule\ conjunct{\isadigit{2}}{\isacharparenright}\isanewline
\ \ \ \ \ \ \ \ \isacommand{have}\isamarkupfalse%
\ {\isachardoublequoteopen}{\isasymforall}\ G{\isachardot}\ \isactrlbold {\isasymnot}\ {\isacharparenleft}\isactrlbold {\isasymnot}\ G{\isacharparenright}\ {\isasymin}\ S\ {\isasymlongrightarrow}\ G\ {\isasymin}\ S{\isachardoublequoteclose}\isanewline
\ \ \ \ \ \ \ \ \ \ \isacommand{using}\isamarkupfalse%
\ assms\ \isacommand{by}\isamarkupfalse%
\ {\isacharparenleft}iprover\ elim{\isacharcolon}\ conjunct{\isadigit{2}}\ conjunct{\isadigit{1}}{\isacharparenright}\isanewline
\ \ \ \ \ \ \ \ \isacommand{then}\isamarkupfalse%
\ \isacommand{have}\isamarkupfalse%
\ {\isachardoublequoteopen}\isactrlbold {\isasymnot}\ {\isacharparenleft}\isactrlbold {\isasymnot}\ G{\isacharparenright}\ {\isasymin}\ S\ {\isasymlongrightarrow}\ G\ {\isasymin}\ S{\isachardoublequoteclose}\isanewline
\ \ \ \ \ \ \ \ \ \ \isacommand{by}\isamarkupfalse%
\ {\isacharparenleft}rule\ allE{\isacharparenright}\isanewline
\ \ \ \ \ \ \ \ \isacommand{then}\isamarkupfalse%
\ \isacommand{have}\isamarkupfalse%
\ {\isachardoublequoteopen}F\ {\isasymin}\ S\ {\isasymlongrightarrow}\ G\ {\isasymin}\ S{\isachardoublequoteclose}\isanewline
\ \ \ \ \ \ \ \ \ \ \isacommand{by}\isamarkupfalse%
\ {\isacharparenleft}simp\ only{\isacharcolon}\ {\isacartoucheopen}F\ {\isacharequal}\ \isactrlbold {\isasymnot}\ {\isacharparenleft}\isactrlbold {\isasymnot}\ G{\isacharparenright}{\isacartoucheclose}{\isacharparenright}\isanewline
\ \ \ \ \ \ \ \ \isacommand{then}\isamarkupfalse%
\ \isacommand{have}\isamarkupfalse%
\ {\isachardoublequoteopen}F\ {\isasymin}\ S\ {\isasymlongrightarrow}\ G\ {\isasymin}\ S\ {\isasymor}\ G\ {\isasymin}\ S{\isachardoublequoteclose}\isanewline
\ \ \ \ \ \ \ \ \ \ \isacommand{by}\isamarkupfalse%
\ {\isacharparenleft}simp\ only{\isacharcolon}\ disj{\isacharunderscore}absorb{\isacharparenright}\isanewline
\ \ \ \ \ \ \ \ \isacommand{thus}\isamarkupfalse%
\ {\isachardoublequoteopen}F\ {\isasymin}\ S\ {\isasymlongrightarrow}\ G\ {\isasymin}\ S\ {\isasymor}\ H\ {\isasymin}\ S{\isachardoublequoteclose}\isanewline
\ \ \ \ \ \ \ \ \isacommand{by}\isamarkupfalse%
\ {\isacharparenleft}simp\ only{\isacharcolon}\ {\isacartoucheopen}H\ {\isacharequal}\ G{\isacartoucheclose}{\isacharparenright}\isanewline
\ \ \ \ \ \ \isacommand{qed}\isamarkupfalse%
\isanewline
\ \ \ \ \isacommand{qed}\isamarkupfalse%
\isanewline
\ \ \isacommand{qed}\isamarkupfalse%
\isanewline
\isacommand{qed}\isamarkupfalse%
%
\endisatagproof
{\isafoldproof}%
%
\isadelimproof
%
\endisadelimproof
%
\begin{isamarkuptext}%
Finalmente, podemos demostrar detalladamente esta primera implicación de la
  equivalencia del lema en Isabelle.%
\end{isamarkuptext}\isamarkuptrue%
\isacommand{lemma}\isamarkupfalse%
\ Hintikka{\isacharunderscore}alt{\isadigit{1}}{\isacharcolon}\isanewline
\ \ \isakeyword{assumes}\ {\isachardoublequoteopen}Hintikka\ S{\isachardoublequoteclose}\isanewline
\ \ \isakeyword{shows}\ {\isachardoublequoteopen}{\isasymbottom}\ {\isasymnotin}\ S\isanewline
{\isasymand}\ {\isacharparenleft}{\isasymforall}k{\isachardot}\ Atom\ k\ {\isasymin}\ S\ {\isasymlongrightarrow}\ \isactrlbold {\isasymnot}\ {\isacharparenleft}Atom\ k{\isacharparenright}\ {\isasymin}\ S\ {\isasymlongrightarrow}\ False{\isacharparenright}\isanewline
{\isasymand}\ {\isacharparenleft}{\isasymforall}F\ G\ H{\isachardot}\ Con\ F\ G\ H\ {\isasymlongrightarrow}\ F\ {\isasymin}\ S\ {\isasymlongrightarrow}\ G\ {\isasymin}\ S\ {\isasymand}\ H\ {\isasymin}\ S{\isacharparenright}\isanewline
{\isasymand}\ {\isacharparenleft}{\isasymforall}F\ G\ H{\isachardot}\ Dis\ F\ G\ H\ {\isasymlongrightarrow}\ F\ {\isasymin}\ S\ {\isasymlongrightarrow}\ G\ {\isasymin}\ S\ {\isasymor}\ H\ {\isasymin}\ S{\isacharparenright}{\isachardoublequoteclose}\isanewline
%
\isadelimproof
%
\endisadelimproof
%
\isatagproof
\isacommand{proof}\isamarkupfalse%
\ {\isacharminus}\isanewline
\ \ \isacommand{have}\isamarkupfalse%
\ Hk{\isacharcolon}{\isachardoublequoteopen}{\isacharparenleft}{\isasymbottom}\ {\isasymnotin}\ S\isanewline
\ \ {\isasymand}\ {\isacharparenleft}{\isasymforall}k{\isachardot}\ Atom\ k\ {\isasymin}\ S\ {\isasymlongrightarrow}\ \isactrlbold {\isasymnot}\ {\isacharparenleft}Atom\ k{\isacharparenright}\ {\isasymin}\ S\ {\isasymlongrightarrow}\ False{\isacharparenright}\isanewline
\ \ {\isasymand}\ {\isacharparenleft}{\isasymforall}G\ H{\isachardot}\ G\ \isactrlbold {\isasymand}\ H\ {\isasymin}\ S\ {\isasymlongrightarrow}\ G\ {\isasymin}\ S\ {\isasymand}\ H\ {\isasymin}\ S{\isacharparenright}\isanewline
\ \ {\isasymand}\ {\isacharparenleft}{\isasymforall}G\ H{\isachardot}\ G\ \isactrlbold {\isasymor}\ H\ {\isasymin}\ S\ {\isasymlongrightarrow}\ G\ {\isasymin}\ S\ {\isasymor}\ H\ {\isasymin}\ S{\isacharparenright}\isanewline
\ \ {\isasymand}\ {\isacharparenleft}{\isasymforall}G\ H{\isachardot}\ G\ \isactrlbold {\isasymrightarrow}\ H\ {\isasymin}\ S\ {\isasymlongrightarrow}\ \isactrlbold {\isasymnot}G\ {\isasymin}\ S\ {\isasymor}\ H\ {\isasymin}\ S{\isacharparenright}\isanewline
\ \ {\isasymand}\ {\isacharparenleft}{\isasymforall}G{\isachardot}\ \isactrlbold {\isasymnot}\ {\isacharparenleft}\isactrlbold {\isasymnot}G{\isacharparenright}\ {\isasymin}\ S\ {\isasymlongrightarrow}\ G\ {\isasymin}\ S{\isacharparenright}\isanewline
\ \ {\isasymand}\ {\isacharparenleft}{\isasymforall}G\ H{\isachardot}\ \isactrlbold {\isasymnot}{\isacharparenleft}G\ \isactrlbold {\isasymand}\ H{\isacharparenright}\ {\isasymin}\ S\ {\isasymlongrightarrow}\ \isactrlbold {\isasymnot}\ G\ {\isasymin}\ S\ {\isasymor}\ \isactrlbold {\isasymnot}\ H\ {\isasymin}\ S{\isacharparenright}\isanewline
\ \ {\isasymand}\ {\isacharparenleft}{\isasymforall}G\ H{\isachardot}\ \isactrlbold {\isasymnot}{\isacharparenleft}G\ \isactrlbold {\isasymor}\ H{\isacharparenright}\ {\isasymin}\ S\ {\isasymlongrightarrow}\ \isactrlbold {\isasymnot}\ G\ {\isasymin}\ S\ {\isasymand}\ \isactrlbold {\isasymnot}\ H\ {\isasymin}\ S{\isacharparenright}\isanewline
\ \ {\isasymand}\ {\isacharparenleft}{\isasymforall}G\ H{\isachardot}\ \isactrlbold {\isasymnot}{\isacharparenleft}G\ \isactrlbold {\isasymrightarrow}\ H{\isacharparenright}\ {\isasymin}\ S\ {\isasymlongrightarrow}\ G\ {\isasymin}\ S\ {\isasymand}\ \isactrlbold {\isasymnot}\ H\ {\isasymin}\ S{\isacharparenright}{\isacharparenright}{\isachardoublequoteclose}\isanewline
\ \ \ \ \isacommand{using}\isamarkupfalse%
\ assms\ \isacommand{by}\isamarkupfalse%
\ {\isacharparenleft}rule\ auxEq{\isacharparenright}\isanewline
\ \ \isacommand{then}\isamarkupfalse%
\ \isacommand{have}\isamarkupfalse%
\ C{\isadigit{1}}{\isacharcolon}\ {\isachardoublequoteopen}{\isasymbottom}\ {\isasymnotin}\ S{\isachardoublequoteclose}\isanewline
\ \ \ \ \isacommand{by}\isamarkupfalse%
\ {\isacharparenleft}rule\ conjunct{\isadigit{1}}{\isacharparenright}\isanewline
\ \ \isacommand{have}\isamarkupfalse%
\ C{\isadigit{2}}{\isacharcolon}\ {\isachardoublequoteopen}{\isasymforall}k{\isachardot}\ Atom\ k\ {\isasymin}\ S\ {\isasymlongrightarrow}\ \isactrlbold {\isasymnot}\ {\isacharparenleft}Atom\ k{\isacharparenright}\ {\isasymin}\ S\ {\isasymlongrightarrow}\ False{\isachardoublequoteclose}\isanewline
\ \ \ \ \isacommand{using}\isamarkupfalse%
\ Hk\ \isacommand{by}\isamarkupfalse%
\ {\isacharparenleft}iprover\ elim{\isacharcolon}\ conjunct{\isadigit{2}}\ conjunct{\isadigit{1}}{\isacharparenright}\isanewline
\ \ \isacommand{have}\isamarkupfalse%
\ C{\isadigit{3}}{\isacharcolon}\ {\isachardoublequoteopen}{\isasymforall}F\ G\ H{\isachardot}\ Con\ F\ G\ H\ {\isasymlongrightarrow}\ F\ {\isasymin}\ S\ {\isasymlongrightarrow}\ G\ {\isasymin}\ S\ {\isasymand}\ H\ {\isasymin}\ S{\isachardoublequoteclose}\isanewline
\ \ \isacommand{proof}\isamarkupfalse%
\ {\isacharparenleft}rule\ allI{\isacharparenright}{\isacharplus}\isanewline
\ \ \ \ \isacommand{fix}\isamarkupfalse%
\ F\ G\ H\isanewline
\ \ \ \ \isacommand{have}\isamarkupfalse%
\ C{\isadigit{3}}{\isadigit{1}}{\isacharcolon}{\isachardoublequoteopen}{\isasymforall}G\ H{\isachardot}\ G\ \isactrlbold {\isasymand}\ H\ {\isasymin}\ S\ {\isasymlongrightarrow}\ G\ {\isasymin}\ S\ {\isasymand}\ H\ {\isasymin}\ S{\isachardoublequoteclose}\isanewline
\ \ \ \ \ \ \isacommand{using}\isamarkupfalse%
\ Hk\ \isacommand{by}\isamarkupfalse%
\ {\isacharparenleft}iprover\ elim{\isacharcolon}\ conjunct{\isadigit{2}}\ conjunct{\isadigit{1}}{\isacharparenright}\isanewline
\ \ \ \ \isacommand{have}\isamarkupfalse%
\ C{\isadigit{3}}{\isadigit{2}}{\isacharcolon}{\isachardoublequoteopen}{\isasymforall}G{\isachardot}\ \isactrlbold {\isasymnot}\ {\isacharparenleft}\isactrlbold {\isasymnot}\ G{\isacharparenright}\ {\isasymin}\ S\ {\isasymlongrightarrow}\ G\ {\isasymin}\ S{\isachardoublequoteclose}\isanewline
\ \ \ \ \ \ \isacommand{using}\isamarkupfalse%
\ Hk\ \isacommand{by}\isamarkupfalse%
\ {\isacharparenleft}iprover\ elim{\isacharcolon}\ conjunct{\isadigit{2}}\ conjunct{\isadigit{1}}{\isacharparenright}\isanewline
\ \ \ \ \isacommand{have}\isamarkupfalse%
\ C{\isadigit{3}}{\isadigit{3}}{\isacharcolon}{\isachardoublequoteopen}{\isasymforall}G\ H{\isachardot}\ \isactrlbold {\isasymnot}{\isacharparenleft}G\ \isactrlbold {\isasymor}\ H{\isacharparenright}\ {\isasymin}\ S\ {\isasymlongrightarrow}\ \isactrlbold {\isasymnot}\ G\ {\isasymin}\ S\ {\isasymand}\ \isactrlbold {\isasymnot}\ H\ {\isasymin}\ S{\isachardoublequoteclose}\isanewline
\ \ \ \ \ \ \isacommand{using}\isamarkupfalse%
\ Hk\ \isacommand{by}\isamarkupfalse%
\ {\isacharparenleft}iprover\ elim{\isacharcolon}\ conjunct{\isadigit{2}}\ conjunct{\isadigit{1}}{\isacharparenright}\isanewline
\ \ \ \ \isacommand{have}\isamarkupfalse%
\ C{\isadigit{3}}{\isadigit{4}}{\isacharcolon}{\isachardoublequoteopen}{\isasymforall}G\ H{\isachardot}\ \isactrlbold {\isasymnot}{\isacharparenleft}G\ \isactrlbold {\isasymrightarrow}\ H{\isacharparenright}\ {\isasymin}\ S\ {\isasymlongrightarrow}\ G\ {\isasymin}\ S\ {\isasymand}\ \isactrlbold {\isasymnot}\ H\ {\isasymin}\ S{\isachardoublequoteclose}\isanewline
\ \ \ \ \ \ \isacommand{using}\isamarkupfalse%
\ Hk\ \isacommand{by}\isamarkupfalse%
\ {\isacharparenleft}iprover\ elim{\isacharcolon}\ conjunct{\isadigit{2}}\ conjunct{\isadigit{1}}{\isacharparenright}\isanewline
\ \ \ \ \isacommand{have}\isamarkupfalse%
\ {\isachardoublequoteopen}{\isacharparenleft}{\isasymforall}G\ H{\isachardot}\ G\ \isactrlbold {\isasymand}\ H\ {\isasymin}\ S\ {\isasymlongrightarrow}\ G\ {\isasymin}\ S\ {\isasymand}\ H\ {\isasymin}\ S{\isacharparenright}\isanewline
\ \ \ \ \ \ \ \ \ \ {\isasymand}\ {\isacharparenleft}{\isasymforall}G{\isachardot}\ \isactrlbold {\isasymnot}\ {\isacharparenleft}\isactrlbold {\isasymnot}\ G{\isacharparenright}\ {\isasymin}\ S\ {\isasymlongrightarrow}\ G\ {\isasymin}\ S{\isacharparenright}\isanewline
\ \ \ \ \ \ \ \ \ \ {\isasymand}\ {\isacharparenleft}{\isasymforall}G\ H{\isachardot}\ \isactrlbold {\isasymnot}{\isacharparenleft}G\ \isactrlbold {\isasymor}\ H{\isacharparenright}\ {\isasymin}\ S\ {\isasymlongrightarrow}\ \isactrlbold {\isasymnot}\ G\ {\isasymin}\ S\ {\isasymand}\ \isactrlbold {\isasymnot}\ H\ {\isasymin}\ S{\isacharparenright}\isanewline
\ \ \ \ \ \ \ \ \ \ {\isasymand}\ {\isacharparenleft}{\isasymforall}G\ H{\isachardot}\ \isactrlbold {\isasymnot}{\isacharparenleft}G\ \isactrlbold {\isasymrightarrow}\ H{\isacharparenright}\ {\isasymin}\ S\ {\isasymlongrightarrow}\ G\ {\isasymin}\ S\ {\isasymand}\ \isactrlbold {\isasymnot}\ H\ {\isasymin}\ S{\isacharparenright}{\isachardoublequoteclose}\ \isanewline
\ \ \ \ \ \ \isacommand{using}\isamarkupfalse%
\ C{\isadigit{3}}{\isadigit{1}}\ C{\isadigit{3}}{\isadigit{2}}\ C{\isadigit{3}}{\isadigit{3}}\ C{\isadigit{3}}{\isadigit{4}}\ \isacommand{by}\isamarkupfalse%
\ {\isacharparenleft}iprover\ intro{\isacharcolon}\ conjI{\isacharparenright}\isanewline
\ \ \ \ \isacommand{thus}\isamarkupfalse%
\ {\isachardoublequoteopen}Con\ F\ G\ H\ {\isasymlongrightarrow}\ F\ {\isasymin}\ S\ {\isasymlongrightarrow}\ G\ {\isasymin}\ S\ {\isasymand}\ H\ {\isasymin}\ S{\isachardoublequoteclose}\isanewline
\ \ \ \ \ \ \isacommand{by}\isamarkupfalse%
\ {\isacharparenleft}rule\ Hintikka{\isacharunderscore}alt{\isadigit{1}}Con{\isacharparenright}\isanewline
\ \ \isacommand{qed}\isamarkupfalse%
\isanewline
\ \ \isacommand{have}\isamarkupfalse%
\ C{\isadigit{4}}{\isacharcolon}{\isachardoublequoteopen}{\isasymforall}F\ G\ H{\isachardot}\ Dis\ F\ G\ H\ {\isasymlongrightarrow}\ F\ {\isasymin}\ S\ {\isasymlongrightarrow}\ G\ {\isasymin}\ S\ {\isasymor}\ H\ {\isasymin}\ S{\isachardoublequoteclose}\isanewline
\ \ \isacommand{proof}\isamarkupfalse%
\ {\isacharparenleft}rule\ allI{\isacharparenright}{\isacharplus}\isanewline
\ \ \ \ \isacommand{fix}\isamarkupfalse%
\ F\ G\ H\isanewline
\ \ \ \ \isacommand{have}\isamarkupfalse%
\ C{\isadigit{4}}{\isadigit{1}}{\isacharcolon}{\isachardoublequoteopen}{\isasymforall}G\ H{\isachardot}\ G\ \isactrlbold {\isasymor}\ H\ {\isasymin}\ S\ {\isasymlongrightarrow}\ G\ {\isasymin}\ S\ {\isasymor}\ H\ {\isasymin}\ S{\isachardoublequoteclose}\isanewline
\ \ \ \ \ \ \isacommand{using}\isamarkupfalse%
\ Hk\ \isacommand{by}\isamarkupfalse%
\ {\isacharparenleft}iprover\ elim{\isacharcolon}\ conjunct{\isadigit{2}}\ conjunct{\isadigit{1}}{\isacharparenright}\isanewline
\ \ \ \ \isacommand{have}\isamarkupfalse%
\ C{\isadigit{4}}{\isadigit{2}}{\isacharcolon}{\isachardoublequoteopen}{\isasymforall}G\ H{\isachardot}\ G\ \isactrlbold {\isasymrightarrow}\ H\ {\isasymin}\ S\ {\isasymlongrightarrow}\ \isactrlbold {\isasymnot}\ G\ {\isasymin}\ S\ {\isasymor}\ H\ {\isasymin}\ S{\isachardoublequoteclose}\isanewline
\ \ \ \ \ \ \isacommand{using}\isamarkupfalse%
\ Hk\ \isacommand{by}\isamarkupfalse%
\ {\isacharparenleft}iprover\ elim{\isacharcolon}\ conjunct{\isadigit{2}}\ conjunct{\isadigit{1}}{\isacharparenright}\isanewline
\ \ \ \ \isacommand{have}\isamarkupfalse%
\ C{\isadigit{4}}{\isadigit{3}}{\isacharcolon}{\isachardoublequoteopen}{\isasymforall}G{\isachardot}\ \isactrlbold {\isasymnot}\ {\isacharparenleft}\isactrlbold {\isasymnot}\ G{\isacharparenright}\ {\isasymin}\ S\ {\isasymlongrightarrow}\ G\ {\isasymin}\ S{\isachardoublequoteclose}\isanewline
\ \ \ \ \ \ \isacommand{using}\isamarkupfalse%
\ Hk\ \isacommand{by}\isamarkupfalse%
\ {\isacharparenleft}iprover\ elim{\isacharcolon}\ conjunct{\isadigit{2}}\ conjunct{\isadigit{1}}{\isacharparenright}\isanewline
\ \ \ \ \isacommand{have}\isamarkupfalse%
\ C{\isadigit{4}}{\isadigit{4}}{\isacharcolon}{\isachardoublequoteopen}{\isasymforall}G\ H{\isachardot}\ \isactrlbold {\isasymnot}{\isacharparenleft}G\ \isactrlbold {\isasymand}\ H{\isacharparenright}\ {\isasymin}\ S\ {\isasymlongrightarrow}\ \isactrlbold {\isasymnot}\ G\ {\isasymin}\ S\ {\isasymor}\ \isactrlbold {\isasymnot}\ H\ {\isasymin}\ S{\isachardoublequoteclose}\isanewline
\ \ \ \ \ \ \isacommand{using}\isamarkupfalse%
\ Hk\ \isacommand{by}\isamarkupfalse%
\ {\isacharparenleft}iprover\ elim{\isacharcolon}\ conjunct{\isadigit{2}}\ conjunct{\isadigit{1}}{\isacharparenright}\isanewline
\ \ \ \ \isacommand{have}\isamarkupfalse%
\ {\isachardoublequoteopen}{\isacharparenleft}{\isasymforall}G\ H{\isachardot}\ G\ \isactrlbold {\isasymor}\ H\ {\isasymin}\ S\ {\isasymlongrightarrow}\ G\ {\isasymin}\ S\ {\isasymor}\ H\ {\isasymin}\ S{\isacharparenright}\isanewline
\ \ \ \ \ \ \ \ \ \ {\isasymand}\ {\isacharparenleft}{\isasymforall}G\ H{\isachardot}\ G\ \isactrlbold {\isasymrightarrow}\ H\ {\isasymin}\ S\ {\isasymlongrightarrow}\ \isactrlbold {\isasymnot}\ G\ {\isasymin}\ S\ {\isasymor}\ H\ {\isasymin}\ S{\isacharparenright}\isanewline
\ \ \ \ \ \ \ \ \ \ {\isasymand}\ {\isacharparenleft}{\isasymforall}G{\isachardot}\ \isactrlbold {\isasymnot}\ {\isacharparenleft}\isactrlbold {\isasymnot}\ G{\isacharparenright}\ {\isasymin}\ S\ {\isasymlongrightarrow}\ G\ {\isasymin}\ S{\isacharparenright}\isanewline
\ \ \ \ \ \ \ \ \ \ {\isasymand}\ {\isacharparenleft}{\isasymforall}G\ H{\isachardot}\ \isactrlbold {\isasymnot}{\isacharparenleft}G\ \isactrlbold {\isasymand}\ H{\isacharparenright}\ {\isasymin}\ S\ {\isasymlongrightarrow}\ \isactrlbold {\isasymnot}\ G\ {\isasymin}\ S\ {\isasymor}\ \isactrlbold {\isasymnot}\ H\ {\isasymin}\ S{\isacharparenright}{\isachardoublequoteclose}\isanewline
\ \ \ \ \ \ \isacommand{using}\isamarkupfalse%
\ C{\isadigit{4}}{\isadigit{1}}\ C{\isadigit{4}}{\isadigit{2}}\ C{\isadigit{4}}{\isadigit{3}}\ C{\isadigit{4}}{\isadigit{4}}\ \isacommand{by}\isamarkupfalse%
\ {\isacharparenleft}iprover\ intro{\isacharcolon}\ conjI{\isacharparenright}\isanewline
\ \ \ \ \isacommand{thus}\isamarkupfalse%
\ {\isachardoublequoteopen}Dis\ F\ G\ H\ {\isasymlongrightarrow}\ F\ {\isasymin}\ S\ {\isasymlongrightarrow}\ G\ {\isasymin}\ S\ {\isasymor}\ H\ {\isasymin}\ S{\isachardoublequoteclose}\isanewline
\ \ \ \ \ \ \isacommand{by}\isamarkupfalse%
\ {\isacharparenleft}rule\ Hintikka{\isacharunderscore}alt{\isadigit{1}}Dis{\isacharparenright}\isanewline
\ \ \isacommand{qed}\isamarkupfalse%
\isanewline
\ \ \isacommand{show}\isamarkupfalse%
\ {\isachardoublequoteopen}{\isasymbottom}\ {\isasymnotin}\ S\isanewline
\ \ {\isasymand}\ {\isacharparenleft}{\isasymforall}k{\isachardot}\ Atom\ k\ {\isasymin}\ S\ {\isasymlongrightarrow}\ \isactrlbold {\isasymnot}\ {\isacharparenleft}Atom\ k{\isacharparenright}\ {\isasymin}\ S\ {\isasymlongrightarrow}\ False{\isacharparenright}\isanewline
\ \ {\isasymand}\ {\isacharparenleft}{\isasymforall}F\ G\ H{\isachardot}\ Con\ F\ G\ H\ {\isasymlongrightarrow}\ F\ {\isasymin}\ S\ {\isasymlongrightarrow}\ G\ {\isasymin}\ S\ {\isasymand}\ H\ {\isasymin}\ S{\isacharparenright}\isanewline
\ \ {\isasymand}\ {\isacharparenleft}{\isasymforall}F\ G\ H{\isachardot}\ Dis\ F\ G\ H\ {\isasymlongrightarrow}\ F\ {\isasymin}\ S\ {\isasymlongrightarrow}\ G\ {\isasymin}\ S\ {\isasymor}\ H\ {\isasymin}\ S{\isacharparenright}{\isachardoublequoteclose}\isanewline
\ \ \ \ \isacommand{using}\isamarkupfalse%
\ C{\isadigit{1}}\ C{\isadigit{2}}\ C{\isadigit{3}}\ C{\isadigit{4}}\ \isacommand{by}\isamarkupfalse%
\ {\isacharparenleft}iprover\ intro{\isacharcolon}\ conjI{\isacharparenright}\isanewline
\isacommand{qed}\isamarkupfalse%
%
\endisatagproof
{\isafoldproof}%
%
\isadelimproof
%
\endisadelimproof
%
\begin{isamarkuptext}%
Por último, probamos la implicación recíproca de forma detallada en Isabelle mediante
  el siguiente lema.%
\end{isamarkuptext}\isamarkuptrue%
\isacommand{lemma}\isamarkupfalse%
\ Hintikka{\isacharunderscore}alt{\isadigit{2}}{\isacharcolon}\isanewline
\ \ \isakeyword{assumes}\ {\isachardoublequoteopen}{\isasymbottom}\ {\isasymnotin}\ S\isanewline
{\isasymand}\ {\isacharparenleft}{\isasymforall}k{\isachardot}\ Atom\ k\ {\isasymin}\ S\ {\isasymlongrightarrow}\ \isactrlbold {\isasymnot}\ {\isacharparenleft}Atom\ k{\isacharparenright}\ {\isasymin}\ S\ {\isasymlongrightarrow}\ False{\isacharparenright}\isanewline
{\isasymand}\ {\isacharparenleft}{\isasymforall}F\ G\ H{\isachardot}\ Con\ F\ G\ H\ {\isasymlongrightarrow}\ F\ {\isasymin}\ S\ {\isasymlongrightarrow}\ G\ {\isasymin}\ S\ {\isasymand}\ H\ {\isasymin}\ S{\isacharparenright}\ \isanewline
{\isasymand}\ {\isacharparenleft}{\isasymforall}F\ G\ H{\isachardot}\ Dis\ F\ G\ H\ {\isasymlongrightarrow}\ F\ {\isasymin}\ S\ {\isasymlongrightarrow}\ G\ {\isasymin}\ S\ {\isasymor}\ H\ {\isasymin}\ S{\isacharparenright}{\isachardoublequoteclose}\ \ \isanewline
\ \ \isakeyword{shows}\ {\isachardoublequoteopen}Hintikka\ S{\isachardoublequoteclose}\isanewline
%
\isadelimproof
%
\endisadelimproof
%
\isatagproof
\isacommand{proof}\isamarkupfalse%
\ {\isacharminus}\isanewline
\ \ \isacommand{have}\isamarkupfalse%
\ Con{\isacharcolon}{\isachardoublequoteopen}{\isasymforall}F\ G\ H{\isachardot}\ Con\ F\ G\ H\ {\isasymlongrightarrow}\ F\ {\isasymin}\ S\ {\isasymlongrightarrow}\ G\ {\isasymin}\ S\ {\isasymand}\ H\ {\isasymin}\ S{\isachardoublequoteclose}\isanewline
\ \ \ \ \isacommand{using}\isamarkupfalse%
\ assms\ \isacommand{by}\isamarkupfalse%
\ {\isacharparenleft}iprover\ elim{\isacharcolon}\ conjunct{\isadigit{2}}\ conjunct{\isadigit{1}}{\isacharparenright}\isanewline
\ \ \isacommand{have}\isamarkupfalse%
\ Dis{\isacharcolon}{\isachardoublequoteopen}{\isasymforall}F\ G\ H{\isachardot}\ Dis\ F\ G\ H\ {\isasymlongrightarrow}\ F\ {\isasymin}\ S\ {\isasymlongrightarrow}\ G\ {\isasymin}\ S\ {\isasymor}\ H\ {\isasymin}\ S{\isachardoublequoteclose}\isanewline
\ \ \ \ \isacommand{using}\isamarkupfalse%
\ assms\ \isacommand{by}\isamarkupfalse%
\ {\isacharparenleft}iprover\ elim{\isacharcolon}\ conjunct{\isadigit{2}}\ conjunct{\isadigit{1}}{\isacharparenright}\isanewline
\ \ \isacommand{have}\isamarkupfalse%
\ {\isachardoublequoteopen}{\isasymbottom}\ {\isasymnotin}\ S\isanewline
\ \ {\isasymand}\ {\isacharparenleft}{\isasymforall}k{\isachardot}\ Atom\ k\ {\isasymin}\ S\ {\isasymlongrightarrow}\ \isactrlbold {\isasymnot}\ {\isacharparenleft}Atom\ k{\isacharparenright}\ {\isasymin}\ S\ {\isasymlongrightarrow}\ False{\isacharparenright}\isanewline
\ \ {\isasymand}\ {\isacharparenleft}{\isasymforall}G\ H{\isachardot}\ G\ \isactrlbold {\isasymand}\ H\ {\isasymin}\ S\ {\isasymlongrightarrow}\ G\ {\isasymin}\ S\ {\isasymand}\ H\ {\isasymin}\ S{\isacharparenright}\isanewline
\ \ {\isasymand}\ {\isacharparenleft}{\isasymforall}G\ H{\isachardot}\ G\ \isactrlbold {\isasymor}\ H\ {\isasymin}\ S\ {\isasymlongrightarrow}\ G\ {\isasymin}\ S\ {\isasymor}\ H\ {\isasymin}\ S{\isacharparenright}\isanewline
\ \ {\isasymand}\ {\isacharparenleft}{\isasymforall}G\ H{\isachardot}\ G\ \isactrlbold {\isasymrightarrow}\ H\ {\isasymin}\ S\ {\isasymlongrightarrow}\ \isactrlbold {\isasymnot}G\ {\isasymin}\ S\ {\isasymor}\ H\ {\isasymin}\ S{\isacharparenright}\isanewline
\ \ {\isasymand}\ {\isacharparenleft}{\isasymforall}G{\isachardot}\ \isactrlbold {\isasymnot}\ {\isacharparenleft}\isactrlbold {\isasymnot}G{\isacharparenright}\ {\isasymin}\ S\ {\isasymlongrightarrow}\ G\ {\isasymin}\ S{\isacharparenright}\isanewline
\ \ {\isasymand}\ {\isacharparenleft}{\isasymforall}G\ H{\isachardot}\ \isactrlbold {\isasymnot}{\isacharparenleft}G\ \isactrlbold {\isasymand}\ H{\isacharparenright}\ {\isasymin}\ S\ {\isasymlongrightarrow}\ \isactrlbold {\isasymnot}\ G\ {\isasymin}\ S\ {\isasymor}\ \isactrlbold {\isasymnot}\ H\ {\isasymin}\ S{\isacharparenright}\isanewline
\ \ {\isasymand}\ {\isacharparenleft}{\isasymforall}G\ H{\isachardot}\ \isactrlbold {\isasymnot}{\isacharparenleft}G\ \isactrlbold {\isasymor}\ H{\isacharparenright}\ {\isasymin}\ S\ {\isasymlongrightarrow}\ \isactrlbold {\isasymnot}\ G\ {\isasymin}\ S\ {\isasymand}\ \isactrlbold {\isasymnot}\ H\ {\isasymin}\ S{\isacharparenright}\isanewline
\ \ {\isasymand}\ {\isacharparenleft}{\isasymforall}G\ H{\isachardot}\ \isactrlbold {\isasymnot}{\isacharparenleft}G\ \isactrlbold {\isasymrightarrow}\ H{\isacharparenright}\ {\isasymin}\ S\ {\isasymlongrightarrow}\ G\ {\isasymin}\ S\ {\isasymand}\ \isactrlbold {\isasymnot}\ H\ {\isasymin}\ S{\isacharparenright}{\isachardoublequoteclose}\isanewline
\ \ \isacommand{proof}\isamarkupfalse%
\ {\isacharminus}\isanewline
\ \ \ \ \isacommand{have}\isamarkupfalse%
\ C{\isadigit{1}}{\isacharcolon}{\isachardoublequoteopen}{\isasymbottom}\ {\isasymnotin}\ S{\isachardoublequoteclose}\isanewline
\ \ \ \ \ \ \isacommand{using}\isamarkupfalse%
\ assms\ \isacommand{by}\isamarkupfalse%
\ {\isacharparenleft}rule\ conjunct{\isadigit{1}}{\isacharparenright}\isanewline
\ \ \ \ \isacommand{have}\isamarkupfalse%
\ C{\isadigit{2}}{\isacharcolon}{\isachardoublequoteopen}{\isasymforall}k{\isachardot}\ Atom\ k\ {\isasymin}\ S\ {\isasymlongrightarrow}\ \isactrlbold {\isasymnot}\ {\isacharparenleft}Atom\ k{\isacharparenright}\ {\isasymin}\ S\ {\isasymlongrightarrow}\ False{\isachardoublequoteclose}\isanewline
\ \ \ \ \ \ \isacommand{using}\isamarkupfalse%
\ assms\ \isacommand{by}\isamarkupfalse%
\ {\isacharparenleft}iprover\ elim{\isacharcolon}\ conjunct{\isadigit{2}}\ conjunct{\isadigit{1}}{\isacharparenright}\isanewline
\ \ \ \ \isacommand{have}\isamarkupfalse%
\ C{\isadigit{3}}{\isacharcolon}{\isachardoublequoteopen}{\isasymforall}G\ H{\isachardot}\ G\ \isactrlbold {\isasymand}\ H\ {\isasymin}\ S\ {\isasymlongrightarrow}\ G\ {\isasymin}\ S\ {\isasymand}\ H\ {\isasymin}\ S{\isachardoublequoteclose}\isanewline
\ \ \ \ \isacommand{proof}\isamarkupfalse%
\ {\isacharparenleft}rule\ allI{\isacharparenright}{\isacharplus}\isanewline
\ \ \ \ \ \ \isacommand{fix}\isamarkupfalse%
\ G\ H\isanewline
\ \ \ \ \ \ \isacommand{show}\isamarkupfalse%
\ {\isachardoublequoteopen}G\ \isactrlbold {\isasymand}\ H\ {\isasymin}\ S\ {\isasymlongrightarrow}\ G\ {\isasymin}\ S\ {\isasymand}\ H\ {\isasymin}\ S{\isachardoublequoteclose}\isanewline
\ \ \ \ \ \ \isacommand{proof}\isamarkupfalse%
\ {\isacharparenleft}rule\ impI{\isacharparenright}\isanewline
\ \ \ \ \ \ \ \ \isacommand{assume}\isamarkupfalse%
\ {\isachardoublequoteopen}G\ \isactrlbold {\isasymand}\ H\ {\isasymin}\ S{\isachardoublequoteclose}\isanewline
\ \ \ \ \ \ \ \ \isacommand{have}\isamarkupfalse%
\ {\isachardoublequoteopen}Con\ {\isacharparenleft}G\ \isactrlbold {\isasymand}\ H{\isacharparenright}\ G\ H{\isachardoublequoteclose}\isanewline
\ \ \ \ \ \ \ \ \ \ \isacommand{by}\isamarkupfalse%
\ {\isacharparenleft}simp\ only{\isacharcolon}\ Con{\isachardot}intros{\isacharparenleft}{\isadigit{1}}{\isacharparenright}{\isacharparenright}\isanewline
\ \ \ \ \ \ \ \ \isacommand{have}\isamarkupfalse%
\ {\isachardoublequoteopen}Con\ {\isacharparenleft}G\ \isactrlbold {\isasymand}\ H{\isacharparenright}\ G\ H\ {\isasymlongrightarrow}\ G\ \isactrlbold {\isasymand}\ H\ {\isasymin}\ S\ {\isasymlongrightarrow}\ G\ {\isasymin}\ S\ {\isasymand}\ H\ {\isasymin}\ S{\isachardoublequoteclose}\isanewline
\ \ \ \ \ \ \ \ \ \ \isacommand{using}\isamarkupfalse%
\ Con\ \isacommand{by}\isamarkupfalse%
\ {\isacharparenleft}iprover\ elim{\isacharcolon}\ allE{\isacharparenright}\isanewline
\ \ \ \ \ \ \ \ \isacommand{then}\isamarkupfalse%
\ \isacommand{have}\isamarkupfalse%
\ {\isachardoublequoteopen}G\ \isactrlbold {\isasymand}\ H\ {\isasymin}\ S\ {\isasymlongrightarrow}\ G\ {\isasymin}\ S\ {\isasymand}\ H\ {\isasymin}\ S{\isachardoublequoteclose}\isanewline
\ \ \ \ \ \ \ \ \ \ \isacommand{using}\isamarkupfalse%
\ {\isacartoucheopen}Con\ {\isacharparenleft}G\ \isactrlbold {\isasymand}\ H{\isacharparenright}\ G\ H{\isacartoucheclose}\ \isacommand{by}\isamarkupfalse%
\ {\isacharparenleft}rule\ mp{\isacharparenright}\isanewline
\ \ \ \ \ \ \ \ \isacommand{thus}\isamarkupfalse%
\ {\isachardoublequoteopen}G\ {\isasymin}\ S\ {\isasymand}\ H\ {\isasymin}\ S{\isachardoublequoteclose}\isanewline
\ \ \ \ \ \ \ \ \ \ \isacommand{using}\isamarkupfalse%
\ {\isacartoucheopen}G\ \isactrlbold {\isasymand}\ H\ {\isasymin}\ S{\isacartoucheclose}\ \isacommand{by}\isamarkupfalse%
\ {\isacharparenleft}rule\ mp{\isacharparenright}\isanewline
\ \ \ \ \ \ \isacommand{qed}\isamarkupfalse%
\isanewline
\ \ \ \ \isacommand{qed}\isamarkupfalse%
\isanewline
\ \ \ \ \isacommand{have}\isamarkupfalse%
\ C{\isadigit{4}}{\isacharcolon}{\isachardoublequoteopen}{\isasymforall}G\ H{\isachardot}\ G\ \isactrlbold {\isasymor}\ H\ {\isasymin}\ S\ {\isasymlongrightarrow}\ G\ {\isasymin}\ S\ {\isasymor}\ H\ {\isasymin}\ S{\isachardoublequoteclose}\isanewline
\ \ \ \ \isacommand{proof}\isamarkupfalse%
\ {\isacharparenleft}rule\ allI{\isacharparenright}{\isacharplus}\isanewline
\ \ \ \ \ \ \isacommand{fix}\isamarkupfalse%
\ G\ H\isanewline
\ \ \ \ \ \ \isacommand{show}\isamarkupfalse%
\ {\isachardoublequoteopen}G\ \isactrlbold {\isasymor}\ H\ {\isasymin}\ S\ {\isasymlongrightarrow}\ G\ {\isasymin}\ S\ {\isasymor}\ H\ {\isasymin}\ S{\isachardoublequoteclose}\isanewline
\ \ \ \ \ \ \isacommand{proof}\isamarkupfalse%
\ {\isacharparenleft}rule\ impI{\isacharparenright}\isanewline
\ \ \ \ \ \ \ \ \isacommand{assume}\isamarkupfalse%
\ {\isachardoublequoteopen}G\ \isactrlbold {\isasymor}\ H\ {\isasymin}\ S{\isachardoublequoteclose}\isanewline
\ \ \ \ \ \ \ \ \isacommand{have}\isamarkupfalse%
\ {\isachardoublequoteopen}Dis\ {\isacharparenleft}G\ \isactrlbold {\isasymor}\ H{\isacharparenright}\ G\ H{\isachardoublequoteclose}\isanewline
\ \ \ \ \ \ \ \ \ \ \isacommand{by}\isamarkupfalse%
\ {\isacharparenleft}simp\ only{\isacharcolon}\ Dis{\isachardot}intros{\isacharparenleft}{\isadigit{1}}{\isacharparenright}{\isacharparenright}\isanewline
\ \ \ \ \ \ \ \ \isacommand{have}\isamarkupfalse%
\ {\isachardoublequoteopen}Dis\ {\isacharparenleft}G\ \isactrlbold {\isasymor}\ H{\isacharparenright}\ G\ H\ {\isasymlongrightarrow}\ G\ \isactrlbold {\isasymor}\ H\ {\isasymin}\ S\ {\isasymlongrightarrow}\ G\ {\isasymin}\ S\ {\isasymor}\ H\ {\isasymin}\ S{\isachardoublequoteclose}\isanewline
\ \ \ \ \ \ \ \ \ \ \isacommand{using}\isamarkupfalse%
\ Dis\ \isacommand{by}\isamarkupfalse%
\ {\isacharparenleft}iprover\ elim{\isacharcolon}\ allE{\isacharparenright}\isanewline
\ \ \ \ \ \ \ \ \isacommand{then}\isamarkupfalse%
\ \isacommand{have}\isamarkupfalse%
\ {\isachardoublequoteopen}G\ \isactrlbold {\isasymor}\ H\ {\isasymin}\ S\ {\isasymlongrightarrow}\ G\ {\isasymin}\ S\ {\isasymor}\ H\ {\isasymin}\ S{\isachardoublequoteclose}\isanewline
\ \ \ \ \ \ \ \ \ \ \isacommand{using}\isamarkupfalse%
\ {\isacartoucheopen}Dis\ {\isacharparenleft}G\ \isactrlbold {\isasymor}\ H{\isacharparenright}\ G\ H{\isacartoucheclose}\ \isacommand{by}\isamarkupfalse%
\ {\isacharparenleft}rule\ mp{\isacharparenright}\isanewline
\ \ \ \ \ \ \ \ \isacommand{thus}\isamarkupfalse%
\ {\isachardoublequoteopen}G\ {\isasymin}\ S\ {\isasymor}\ H\ {\isasymin}\ S{\isachardoublequoteclose}\isanewline
\ \ \ \ \ \ \ \ \ \ \isacommand{using}\isamarkupfalse%
\ {\isacartoucheopen}G\ \isactrlbold {\isasymor}\ H\ {\isasymin}\ S{\isacartoucheclose}\ \isacommand{by}\isamarkupfalse%
\ {\isacharparenleft}rule\ mp{\isacharparenright}\isanewline
\ \ \ \ \ \ \isacommand{qed}\isamarkupfalse%
\isanewline
\ \ \ \ \isacommand{qed}\isamarkupfalse%
\isanewline
\ \ \ \ \isacommand{have}\isamarkupfalse%
\ C{\isadigit{5}}{\isacharcolon}{\isachardoublequoteopen}{\isasymforall}G\ H{\isachardot}\ G\ \isactrlbold {\isasymrightarrow}\ H\ {\isasymin}\ S\ {\isasymlongrightarrow}\ \isactrlbold {\isasymnot}\ G\ {\isasymin}\ S\ {\isasymor}\ H\ {\isasymin}\ S{\isachardoublequoteclose}\isanewline
\ \ \ \ \isacommand{proof}\isamarkupfalse%
\ {\isacharparenleft}rule\ allI{\isacharparenright}{\isacharplus}\isanewline
\ \ \ \ \ \ \isacommand{fix}\isamarkupfalse%
\ G\ H\isanewline
\ \ \ \ \ \ \isacommand{show}\isamarkupfalse%
\ {\isachardoublequoteopen}G\ \isactrlbold {\isasymrightarrow}\ H\ {\isasymin}\ S\ {\isasymlongrightarrow}\ \isactrlbold {\isasymnot}\ G\ {\isasymin}\ S\ {\isasymor}\ H\ {\isasymin}\ S{\isachardoublequoteclose}\isanewline
\ \ \ \ \ \ \isacommand{proof}\isamarkupfalse%
\ {\isacharparenleft}rule\ impI{\isacharparenright}\isanewline
\ \ \ \ \ \ \ \ \isacommand{assume}\isamarkupfalse%
\ {\isachardoublequoteopen}G\ \isactrlbold {\isasymrightarrow}\ H\ {\isasymin}\ S{\isachardoublequoteclose}\ \isanewline
\ \ \ \ \ \ \ \ \isacommand{have}\isamarkupfalse%
\ {\isachardoublequoteopen}Dis\ {\isacharparenleft}G\ \isactrlbold {\isasymrightarrow}\ H{\isacharparenright}\ {\isacharparenleft}\isactrlbold {\isasymnot}\ G{\isacharparenright}\ H{\isachardoublequoteclose}\isanewline
\ \ \ \ \ \ \ \ \ \ \isacommand{by}\isamarkupfalse%
\ {\isacharparenleft}simp\ only{\isacharcolon}\ Dis{\isachardot}intros{\isacharparenleft}{\isadigit{2}}{\isacharparenright}{\isacharparenright}\isanewline
\ \ \ \ \ \ \ \ \isacommand{have}\isamarkupfalse%
\ {\isachardoublequoteopen}Dis\ {\isacharparenleft}G\ \isactrlbold {\isasymrightarrow}\ H{\isacharparenright}\ {\isacharparenleft}\isactrlbold {\isasymnot}\ G{\isacharparenright}\ H\ {\isasymlongrightarrow}\ G\ \isactrlbold {\isasymrightarrow}\ H\ {\isasymin}\ S\ {\isasymlongrightarrow}\ \isactrlbold {\isasymnot}\ G\ {\isasymin}\ S\ {\isasymor}\ H\ {\isasymin}\ S{\isachardoublequoteclose}\isanewline
\ \ \ \ \ \ \ \ \ \ \isacommand{using}\isamarkupfalse%
\ Dis\ \isacommand{by}\isamarkupfalse%
\ {\isacharparenleft}iprover\ elim{\isacharcolon}\ allE{\isacharparenright}\isanewline
\ \ \ \ \ \ \ \ \isacommand{then}\isamarkupfalse%
\ \isacommand{have}\isamarkupfalse%
\ {\isachardoublequoteopen}G\ \isactrlbold {\isasymrightarrow}\ H\ {\isasymin}\ S\ {\isasymlongrightarrow}\ \isactrlbold {\isasymnot}\ G\ {\isasymin}\ S\ {\isasymor}\ H\ {\isasymin}\ S{\isachardoublequoteclose}\ \isanewline
\ \ \ \ \ \ \ \ \ \ \isacommand{using}\isamarkupfalse%
\ {\isacartoucheopen}Dis\ {\isacharparenleft}G\ \isactrlbold {\isasymrightarrow}\ H{\isacharparenright}\ {\isacharparenleft}\isactrlbold {\isasymnot}\ G{\isacharparenright}\ H{\isacartoucheclose}\ \isacommand{by}\isamarkupfalse%
\ {\isacharparenleft}rule\ mp{\isacharparenright}\isanewline
\ \ \ \ \ \ \ \ \isacommand{thus}\isamarkupfalse%
\ {\isachardoublequoteopen}\isactrlbold {\isasymnot}\ G\ {\isasymin}\ S\ {\isasymor}\ H\ {\isasymin}\ S{\isachardoublequoteclose}\isanewline
\ \ \ \ \ \ \ \ \ \ \isacommand{using}\isamarkupfalse%
\ {\isacartoucheopen}G\ \isactrlbold {\isasymrightarrow}\ H\ {\isasymin}\ S{\isacartoucheclose}\ \isacommand{by}\isamarkupfalse%
\ {\isacharparenleft}rule\ mp{\isacharparenright}\isanewline
\ \ \ \ \ \ \isacommand{qed}\isamarkupfalse%
\isanewline
\ \ \ \ \isacommand{qed}\isamarkupfalse%
\isanewline
\ \ \ \ \isacommand{have}\isamarkupfalse%
\ C{\isadigit{6}}{\isacharcolon}{\isachardoublequoteopen}{\isasymforall}G{\isachardot}\ \isactrlbold {\isasymnot}{\isacharparenleft}\isactrlbold {\isasymnot}\ G{\isacharparenright}\ {\isasymin}\ S\ {\isasymlongrightarrow}\ G\ {\isasymin}\ S{\isachardoublequoteclose}\isanewline
\ \ \ \ \isacommand{proof}\isamarkupfalse%
\ {\isacharparenleft}rule\ allI{\isacharparenright}\isanewline
\ \ \ \ \ \ \isacommand{fix}\isamarkupfalse%
\ G\isanewline
\ \ \ \ \ \ \isacommand{show}\isamarkupfalse%
\ {\isachardoublequoteopen}\isactrlbold {\isasymnot}{\isacharparenleft}\isactrlbold {\isasymnot}\ G{\isacharparenright}\ {\isasymin}\ S\ {\isasymlongrightarrow}\ G\ {\isasymin}\ S{\isachardoublequoteclose}\isanewline
\ \ \ \ \ \ \isacommand{proof}\isamarkupfalse%
\ {\isacharparenleft}rule\ impI{\isacharparenright}\isanewline
\ \ \ \ \ \ \ \ \isacommand{assume}\isamarkupfalse%
\ {\isachardoublequoteopen}\isactrlbold {\isasymnot}\ {\isacharparenleft}\isactrlbold {\isasymnot}\ G{\isacharparenright}\ {\isasymin}\ S{\isachardoublequoteclose}\ \isanewline
\ \ \ \ \ \ \ \ \isacommand{have}\isamarkupfalse%
\ {\isachardoublequoteopen}Con\ {\isacharparenleft}\isactrlbold {\isasymnot}\ {\isacharparenleft}\isactrlbold {\isasymnot}\ G{\isacharparenright}{\isacharparenright}\ G\ G{\isachardoublequoteclose}\isanewline
\ \ \ \ \ \ \ \ \ \ \isacommand{by}\isamarkupfalse%
\ {\isacharparenleft}simp\ only{\isacharcolon}\ Con{\isachardot}intros{\isacharparenleft}{\isadigit{4}}{\isacharparenright}{\isacharparenright}\isanewline
\ \ \ \ \ \ \ \ \isacommand{have}\isamarkupfalse%
\ {\isachardoublequoteopen}Con\ {\isacharparenleft}\isactrlbold {\isasymnot}{\isacharparenleft}\isactrlbold {\isasymnot}\ G{\isacharparenright}{\isacharparenright}\ G\ G\ {\isasymlongrightarrow}\ {\isacharparenleft}\isactrlbold {\isasymnot}{\isacharparenleft}\isactrlbold {\isasymnot}\ G{\isacharparenright}{\isacharparenright}\ {\isasymin}\ S\ {\isasymlongrightarrow}\ G\ {\isasymin}\ S\ {\isasymand}\ G\ {\isasymin}\ S{\isachardoublequoteclose}\isanewline
\ \ \ \ \ \ \ \ \ \ \isacommand{using}\isamarkupfalse%
\ Con\ \isacommand{by}\isamarkupfalse%
\ {\isacharparenleft}iprover\ elim{\isacharcolon}\ allE{\isacharparenright}\isanewline
\ \ \ \ \ \ \ \ \isacommand{then}\isamarkupfalse%
\ \isacommand{have}\isamarkupfalse%
\ {\isachardoublequoteopen}{\isacharparenleft}\isactrlbold {\isasymnot}{\isacharparenleft}\isactrlbold {\isasymnot}\ G{\isacharparenright}{\isacharparenright}\ {\isasymin}\ S\ {\isasymlongrightarrow}\ G\ {\isasymin}\ S\ {\isasymand}\ G\ {\isasymin}\ S{\isachardoublequoteclose}\isanewline
\ \ \ \ \ \ \ \ \ \ \isacommand{using}\isamarkupfalse%
\ {\isacartoucheopen}Con\ {\isacharparenleft}\isactrlbold {\isasymnot}\ {\isacharparenleft}\isactrlbold {\isasymnot}\ G{\isacharparenright}{\isacharparenright}\ G\ G{\isacartoucheclose}\ \isacommand{by}\isamarkupfalse%
\ {\isacharparenleft}rule\ mp{\isacharparenright}\isanewline
\ \ \ \ \ \ \ \ \isacommand{then}\isamarkupfalse%
\ \isacommand{have}\isamarkupfalse%
\ {\isachardoublequoteopen}G\ {\isasymin}\ S\ {\isasymand}\ G\ {\isasymin}\ S{\isachardoublequoteclose}\isanewline
\ \ \ \ \ \ \ \ \ \ \isacommand{using}\isamarkupfalse%
\ {\isacartoucheopen}\isactrlbold {\isasymnot}\ {\isacharparenleft}\isactrlbold {\isasymnot}\ G{\isacharparenright}\ {\isasymin}\ S{\isacartoucheclose}\ \isacommand{by}\isamarkupfalse%
\ {\isacharparenleft}rule\ mp{\isacharparenright}\isanewline
\ \ \ \ \ \ \ \ \isacommand{thus}\isamarkupfalse%
\ {\isachardoublequoteopen}G\ {\isasymin}\ S{\isachardoublequoteclose}\isanewline
\ \ \ \ \ \ \ \ \ \ \isacommand{by}\isamarkupfalse%
\ {\isacharparenleft}simp\ only{\isacharcolon}\ conj{\isacharunderscore}absorb{\isacharparenright}\isanewline
\ \ \ \ \ \ \isacommand{qed}\isamarkupfalse%
\isanewline
\ \ \ \ \isacommand{qed}\isamarkupfalse%
\isanewline
\ \ \ \ \isacommand{have}\isamarkupfalse%
\ C{\isadigit{7}}{\isacharcolon}{\isachardoublequoteopen}{\isasymforall}G\ H{\isachardot}\ \isactrlbold {\isasymnot}{\isacharparenleft}G\ \isactrlbold {\isasymand}\ H{\isacharparenright}\ {\isasymin}\ S\ {\isasymlongrightarrow}\ \isactrlbold {\isasymnot}\ G\ {\isasymin}\ S\ {\isasymor}\ \isactrlbold {\isasymnot}\ H\ {\isasymin}\ S{\isachardoublequoteclose}\isanewline
\ \ \ \ \isacommand{proof}\isamarkupfalse%
\ {\isacharparenleft}rule\ allI{\isacharparenright}{\isacharplus}\isanewline
\ \ \ \ \ \ \isacommand{fix}\isamarkupfalse%
\ G\ H\isanewline
\ \ \ \ \ \ \isacommand{show}\isamarkupfalse%
\ {\isachardoublequoteopen}\isactrlbold {\isasymnot}{\isacharparenleft}G\ \isactrlbold {\isasymand}\ H{\isacharparenright}\ {\isasymin}\ S\ {\isasymlongrightarrow}\ \isactrlbold {\isasymnot}\ G\ {\isasymin}\ S\ {\isasymor}\ \isactrlbold {\isasymnot}\ H\ {\isasymin}\ S{\isachardoublequoteclose}\isanewline
\ \ \ \ \ \ \isacommand{proof}\isamarkupfalse%
\ {\isacharparenleft}rule\ impI{\isacharparenright}\isanewline
\ \ \ \ \ \ \ \ \isacommand{assume}\isamarkupfalse%
\ {\isachardoublequoteopen}\isactrlbold {\isasymnot}{\isacharparenleft}G\ \isactrlbold {\isasymand}\ H{\isacharparenright}\ {\isasymin}\ S{\isachardoublequoteclose}\isanewline
\ \ \ \ \ \ \ \ \isacommand{have}\isamarkupfalse%
\ {\isachardoublequoteopen}Dis\ {\isacharparenleft}\isactrlbold {\isasymnot}{\isacharparenleft}G\ \isactrlbold {\isasymand}\ H{\isacharparenright}{\isacharparenright}\ {\isacharparenleft}\isactrlbold {\isasymnot}\ G{\isacharparenright}\ {\isacharparenleft}\isactrlbold {\isasymnot}\ H{\isacharparenright}{\isachardoublequoteclose}\isanewline
\ \ \ \ \ \ \ \ \ \ \isacommand{by}\isamarkupfalse%
\ {\isacharparenleft}simp\ only{\isacharcolon}\ Dis{\isachardot}intros{\isacharparenleft}{\isadigit{3}}{\isacharparenright}{\isacharparenright}\isanewline
\ \ \ \ \ \ \ \ \isacommand{have}\isamarkupfalse%
\ {\isachardoublequoteopen}Dis\ {\isacharparenleft}\isactrlbold {\isasymnot}{\isacharparenleft}G\ \isactrlbold {\isasymand}\ H{\isacharparenright}{\isacharparenright}\ {\isacharparenleft}\isactrlbold {\isasymnot}\ G{\isacharparenright}\ {\isacharparenleft}\isactrlbold {\isasymnot}\ H{\isacharparenright}\ {\isasymlongrightarrow}\ \isactrlbold {\isasymnot}{\isacharparenleft}G\ \isactrlbold {\isasymand}\ H{\isacharparenright}\ {\isasymin}\ S\ {\isasymlongrightarrow}\ \isactrlbold {\isasymnot}\ G\ {\isasymin}\ S\ {\isasymor}\ \isactrlbold {\isasymnot}\ H\ {\isasymin}\ S{\isachardoublequoteclose}\isanewline
\ \ \ \ \ \ \ \ \ \ \isacommand{using}\isamarkupfalse%
\ Dis\ \isacommand{by}\isamarkupfalse%
\ {\isacharparenleft}iprover\ elim{\isacharcolon}\ allE{\isacharparenright}\isanewline
\ \ \ \ \ \ \ \ \isacommand{then}\isamarkupfalse%
\ \isacommand{have}\isamarkupfalse%
\ {\isachardoublequoteopen}\isactrlbold {\isasymnot}{\isacharparenleft}G\ \isactrlbold {\isasymand}\ H{\isacharparenright}\ {\isasymin}\ S\ {\isasymlongrightarrow}\ \isactrlbold {\isasymnot}\ G\ {\isasymin}\ S\ {\isasymor}\ \isactrlbold {\isasymnot}\ H\ {\isasymin}\ S{\isachardoublequoteclose}\isanewline
\ \ \ \ \ \ \ \ \ \ \isacommand{using}\isamarkupfalse%
\ {\isacartoucheopen}Dis\ {\isacharparenleft}\isactrlbold {\isasymnot}{\isacharparenleft}G\ \isactrlbold {\isasymand}\ H{\isacharparenright}{\isacharparenright}\ {\isacharparenleft}\isactrlbold {\isasymnot}\ G{\isacharparenright}\ {\isacharparenleft}\isactrlbold {\isasymnot}\ H{\isacharparenright}{\isacartoucheclose}\ \isacommand{by}\isamarkupfalse%
\ {\isacharparenleft}rule\ mp{\isacharparenright}\isanewline
\ \ \ \ \ \ \ \ \isacommand{thus}\isamarkupfalse%
\ {\isachardoublequoteopen}\isactrlbold {\isasymnot}\ G\ {\isasymin}\ S\ {\isasymor}\ \isactrlbold {\isasymnot}\ H\ {\isasymin}\ S{\isachardoublequoteclose}\isanewline
\ \ \ \ \ \ \ \ \ \ \isacommand{using}\isamarkupfalse%
\ {\isacartoucheopen}\isactrlbold {\isasymnot}{\isacharparenleft}G\ \isactrlbold {\isasymand}\ H{\isacharparenright}\ {\isasymin}\ S{\isacartoucheclose}\ \isacommand{by}\isamarkupfalse%
\ {\isacharparenleft}rule\ mp{\isacharparenright}\isanewline
\ \ \ \ \ \ \isacommand{qed}\isamarkupfalse%
\isanewline
\ \ \ \ \isacommand{qed}\isamarkupfalse%
\isanewline
\ \ \ \ \isacommand{have}\isamarkupfalse%
\ C{\isadigit{8}}{\isacharcolon}{\isachardoublequoteopen}{\isasymforall}G\ H{\isachardot}\ \isactrlbold {\isasymnot}{\isacharparenleft}G\ \isactrlbold {\isasymor}\ H{\isacharparenright}\ {\isasymin}\ S\ {\isasymlongrightarrow}\ \isactrlbold {\isasymnot}\ G\ {\isasymin}\ S\ {\isasymand}\ \isactrlbold {\isasymnot}\ H\ {\isasymin}\ S{\isachardoublequoteclose}\isanewline
\ \ \ \ \isacommand{proof}\isamarkupfalse%
\ {\isacharparenleft}rule\ allI{\isacharparenright}{\isacharplus}\isanewline
\ \ \ \ \ \ \isacommand{fix}\isamarkupfalse%
\ G\ H\isanewline
\ \ \ \ \ \ \isacommand{show}\isamarkupfalse%
\ {\isachardoublequoteopen}\isactrlbold {\isasymnot}{\isacharparenleft}G\ \isactrlbold {\isasymor}\ H{\isacharparenright}\ {\isasymin}\ S\ {\isasymlongrightarrow}\ \isactrlbold {\isasymnot}\ G\ {\isasymin}\ S\ {\isasymand}\ \isactrlbold {\isasymnot}\ H\ {\isasymin}\ S{\isachardoublequoteclose}\isanewline
\ \ \ \ \ \ \isacommand{proof}\isamarkupfalse%
\ {\isacharparenleft}rule\ impI{\isacharparenright}\isanewline
\ \ \ \ \ \ \ \ \isacommand{assume}\isamarkupfalse%
\ {\isachardoublequoteopen}\isactrlbold {\isasymnot}{\isacharparenleft}G\ \isactrlbold {\isasymor}\ H{\isacharparenright}\ {\isasymin}\ S{\isachardoublequoteclose}\isanewline
\ \ \ \ \ \ \ \ \isacommand{have}\isamarkupfalse%
\ {\isachardoublequoteopen}Con\ {\isacharparenleft}\isactrlbold {\isasymnot}{\isacharparenleft}G\ \isactrlbold {\isasymor}\ H{\isacharparenright}{\isacharparenright}\ {\isacharparenleft}\isactrlbold {\isasymnot}\ G{\isacharparenright}\ {\isacharparenleft}\isactrlbold {\isasymnot}\ H{\isacharparenright}{\isachardoublequoteclose}\isanewline
\ \ \ \ \ \ \ \ \ \ \isacommand{by}\isamarkupfalse%
\ {\isacharparenleft}simp\ only{\isacharcolon}\ Con{\isachardot}intros{\isacharparenleft}{\isadigit{2}}{\isacharparenright}{\isacharparenright}\isanewline
\ \ \ \ \ \ \ \ \isacommand{have}\isamarkupfalse%
\ {\isachardoublequoteopen}Con\ {\isacharparenleft}\isactrlbold {\isasymnot}{\isacharparenleft}G\ \isactrlbold {\isasymor}\ H{\isacharparenright}{\isacharparenright}\ {\isacharparenleft}\isactrlbold {\isasymnot}\ G{\isacharparenright}\ {\isacharparenleft}\isactrlbold {\isasymnot}\ H{\isacharparenright}\ {\isasymlongrightarrow}\ \isactrlbold {\isasymnot}{\isacharparenleft}G\ \isactrlbold {\isasymor}\ H{\isacharparenright}\ {\isasymin}\ S\ {\isasymlongrightarrow}\ \isactrlbold {\isasymnot}\ G\ {\isasymin}\ S\ {\isasymand}\ \isactrlbold {\isasymnot}\ H\ {\isasymin}\ S{\isachardoublequoteclose}\isanewline
\ \ \ \ \ \ \ \ \ \ \isacommand{using}\isamarkupfalse%
\ Con\ \isacommand{by}\isamarkupfalse%
\ {\isacharparenleft}iprover\ elim{\isacharcolon}\ allE{\isacharparenright}\isanewline
\ \ \ \ \ \ \ \ \isacommand{then}\isamarkupfalse%
\ \isacommand{have}\isamarkupfalse%
\ {\isachardoublequoteopen}\isactrlbold {\isasymnot}{\isacharparenleft}G\ \isactrlbold {\isasymor}\ H{\isacharparenright}\ {\isasymin}\ S\ {\isasymlongrightarrow}\ \isactrlbold {\isasymnot}\ G\ {\isasymin}\ S\ {\isasymand}\ \isactrlbold {\isasymnot}\ H\ {\isasymin}\ S{\isachardoublequoteclose}\isanewline
\ \ \ \ \ \ \ \ \ \ \isacommand{using}\isamarkupfalse%
\ {\isacartoucheopen}Con\ {\isacharparenleft}\isactrlbold {\isasymnot}{\isacharparenleft}G\ \isactrlbold {\isasymor}\ H{\isacharparenright}{\isacharparenright}\ {\isacharparenleft}\isactrlbold {\isasymnot}\ G{\isacharparenright}\ {\isacharparenleft}\isactrlbold {\isasymnot}\ H{\isacharparenright}{\isacartoucheclose}\ \isacommand{by}\isamarkupfalse%
\ {\isacharparenleft}rule\ mp{\isacharparenright}\isanewline
\ \ \ \ \ \ \ \ \isacommand{thus}\isamarkupfalse%
\ {\isachardoublequoteopen}\isactrlbold {\isasymnot}\ G\ {\isasymin}\ S\ {\isasymand}\ \isactrlbold {\isasymnot}\ H\ {\isasymin}\ S{\isachardoublequoteclose}\isanewline
\ \ \ \ \ \ \ \ \ \ \isacommand{using}\isamarkupfalse%
\ {\isacartoucheopen}\isactrlbold {\isasymnot}{\isacharparenleft}G\ \isactrlbold {\isasymor}\ H{\isacharparenright}\ {\isasymin}\ S{\isacartoucheclose}\ \isacommand{by}\isamarkupfalse%
\ {\isacharparenleft}rule\ mp{\isacharparenright}\isanewline
\ \ \ \ \ \ \isacommand{qed}\isamarkupfalse%
\isanewline
\ \ \ \ \isacommand{qed}\isamarkupfalse%
\isanewline
\ \ \ \ \isacommand{have}\isamarkupfalse%
\ C{\isadigit{9}}{\isacharcolon}{\isachardoublequoteopen}{\isasymforall}G\ H{\isachardot}\ \isactrlbold {\isasymnot}{\isacharparenleft}G\ \isactrlbold {\isasymrightarrow}\ H{\isacharparenright}\ {\isasymin}\ S\ {\isasymlongrightarrow}\ G\ {\isasymin}\ S\ {\isasymand}\ \isactrlbold {\isasymnot}\ H\ {\isasymin}\ S{\isachardoublequoteclose}\isanewline
\ \ \ \ \isacommand{proof}\isamarkupfalse%
\ {\isacharparenleft}rule\ allI{\isacharparenright}{\isacharplus}\isanewline
\ \ \ \ \ \ \isacommand{fix}\isamarkupfalse%
\ G\ H\isanewline
\ \ \ \ \ \ \isacommand{show}\isamarkupfalse%
\ {\isachardoublequoteopen}\isactrlbold {\isasymnot}{\isacharparenleft}G\ \isactrlbold {\isasymrightarrow}\ H{\isacharparenright}\ {\isasymin}\ S\ {\isasymlongrightarrow}\ G\ {\isasymin}\ S\ {\isasymand}\ \isactrlbold {\isasymnot}\ H\ {\isasymin}\ S{\isachardoublequoteclose}\isanewline
\ \ \ \ \ \ \isacommand{proof}\isamarkupfalse%
\ {\isacharparenleft}rule\ impI{\isacharparenright}\isanewline
\ \ \ \ \ \ \ \ \isacommand{assume}\isamarkupfalse%
\ {\isachardoublequoteopen}\isactrlbold {\isasymnot}{\isacharparenleft}G\ \isactrlbold {\isasymrightarrow}\ H{\isacharparenright}\ {\isasymin}\ S{\isachardoublequoteclose}\isanewline
\ \ \ \ \ \ \ \ \isacommand{have}\isamarkupfalse%
\ {\isachardoublequoteopen}Con\ {\isacharparenleft}\isactrlbold {\isasymnot}{\isacharparenleft}G\ \isactrlbold {\isasymrightarrow}\ H{\isacharparenright}{\isacharparenright}\ G\ {\isacharparenleft}\isactrlbold {\isasymnot}\ H{\isacharparenright}{\isachardoublequoteclose}\isanewline
\ \ \ \ \ \ \ \ \ \ \isacommand{by}\isamarkupfalse%
\ {\isacharparenleft}simp\ only{\isacharcolon}\ Con{\isachardot}intros{\isacharparenleft}{\isadigit{3}}{\isacharparenright}{\isacharparenright}\isanewline
\ \ \ \ \ \ \ \ \isacommand{have}\isamarkupfalse%
\ {\isachardoublequoteopen}Con\ {\isacharparenleft}\isactrlbold {\isasymnot}{\isacharparenleft}G\ \isactrlbold {\isasymrightarrow}\ H{\isacharparenright}{\isacharparenright}\ G\ {\isacharparenleft}\isactrlbold {\isasymnot}\ H{\isacharparenright}\ {\isasymlongrightarrow}\ \isactrlbold {\isasymnot}{\isacharparenleft}G\ \isactrlbold {\isasymrightarrow}\ H{\isacharparenright}\ {\isasymin}\ S\ {\isasymlongrightarrow}\ G\ {\isasymin}\ S\ {\isasymand}\ \isactrlbold {\isasymnot}\ H\ {\isasymin}\ S{\isachardoublequoteclose}\isanewline
\ \ \ \ \ \ \ \ \ \ \isacommand{using}\isamarkupfalse%
\ Con\ \isacommand{by}\isamarkupfalse%
\ {\isacharparenleft}iprover\ elim{\isacharcolon}\ allE{\isacharparenright}\isanewline
\ \ \ \ \ \ \ \ \isacommand{then}\isamarkupfalse%
\ \isacommand{have}\isamarkupfalse%
\ {\isachardoublequoteopen}\isactrlbold {\isasymnot}{\isacharparenleft}G\ \isactrlbold {\isasymrightarrow}\ H{\isacharparenright}\ {\isasymin}\ S\ {\isasymlongrightarrow}\ G\ {\isasymin}\ S\ {\isasymand}\ \isactrlbold {\isasymnot}\ H\ {\isasymin}\ S{\isachardoublequoteclose}\isanewline
\ \ \ \ \ \ \ \ \ \ \isacommand{using}\isamarkupfalse%
\ {\isacartoucheopen}Con\ {\isacharparenleft}\isactrlbold {\isasymnot}{\isacharparenleft}G\ \isactrlbold {\isasymrightarrow}\ H{\isacharparenright}{\isacharparenright}\ G\ {\isacharparenleft}\isactrlbold {\isasymnot}\ H{\isacharparenright}{\isacartoucheclose}\ \isacommand{by}\isamarkupfalse%
\ {\isacharparenleft}rule\ mp{\isacharparenright}\isanewline
\ \ \ \ \ \ \ \ \isacommand{thus}\isamarkupfalse%
\ {\isachardoublequoteopen}G\ {\isasymin}\ S\ {\isasymand}\ \isactrlbold {\isasymnot}\ H\ {\isasymin}\ S{\isachardoublequoteclose}\isanewline
\ \ \ \ \ \ \ \ \ \ \isacommand{using}\isamarkupfalse%
\ {\isacartoucheopen}\isactrlbold {\isasymnot}{\isacharparenleft}G\ \isactrlbold {\isasymrightarrow}\ H{\isacharparenright}\ {\isasymin}\ S{\isacartoucheclose}\ \isacommand{by}\isamarkupfalse%
\ {\isacharparenleft}rule\ mp{\isacharparenright}\isanewline
\ \ \ \ \ \ \isacommand{qed}\isamarkupfalse%
\isanewline
\ \ \ \ \isacommand{qed}\isamarkupfalse%
\isanewline
\ \ \ \ \isacommand{have}\isamarkupfalse%
\ A{\isacharcolon}{\isachardoublequoteopen}{\isasymbottom}\ {\isasymnotin}\ S\isanewline
\ \ \ \ {\isasymand}\ {\isacharparenleft}{\isasymforall}k{\isachardot}\ Atom\ k\ {\isasymin}\ S\ {\isasymlongrightarrow}\ \isactrlbold {\isasymnot}\ {\isacharparenleft}Atom\ k{\isacharparenright}\ {\isasymin}\ S\ {\isasymlongrightarrow}\ False{\isacharparenright}\isanewline
\ \ \ \ {\isasymand}\ {\isacharparenleft}{\isasymforall}G\ H{\isachardot}\ G\ \isactrlbold {\isasymand}\ H\ {\isasymin}\ S\ {\isasymlongrightarrow}\ G\ {\isasymin}\ S\ {\isasymand}\ H\ {\isasymin}\ S{\isacharparenright}\isanewline
\ \ \ \ {\isasymand}\ {\isacharparenleft}{\isasymforall}G\ H{\isachardot}\ G\ \isactrlbold {\isasymor}\ H\ {\isasymin}\ S\ {\isasymlongrightarrow}\ G\ {\isasymin}\ S\ {\isasymor}\ H\ {\isasymin}\ S{\isacharparenright}\isanewline
\ \ \ \ {\isasymand}\ {\isacharparenleft}{\isasymforall}G\ H{\isachardot}\ G\ \isactrlbold {\isasymrightarrow}\ H\ {\isasymin}\ S\ {\isasymlongrightarrow}\ \isactrlbold {\isasymnot}G\ {\isasymin}\ S\ {\isasymor}\ H\ {\isasymin}\ S{\isacharparenright}{\isachardoublequoteclose}\isanewline
\ \ \ \ \ \ \isacommand{using}\isamarkupfalse%
\ C{\isadigit{1}}\ C{\isadigit{2}}\ C{\isadigit{3}}\ C{\isadigit{4}}\ C{\isadigit{5}}\ \isacommand{by}\isamarkupfalse%
\ {\isacharparenleft}iprover\ intro{\isacharcolon}\ conjI{\isacharparenright}\isanewline
\ \ \ \ \isacommand{have}\isamarkupfalse%
\ B{\isacharcolon}{\isachardoublequoteopen}{\isacharparenleft}{\isasymforall}G{\isachardot}\ \isactrlbold {\isasymnot}\ {\isacharparenleft}\isactrlbold {\isasymnot}G{\isacharparenright}\ {\isasymin}\ S\ {\isasymlongrightarrow}\ G\ {\isasymin}\ S{\isacharparenright}\isanewline
\ \ \ \ {\isasymand}\ {\isacharparenleft}{\isasymforall}G\ H{\isachardot}\ \isactrlbold {\isasymnot}{\isacharparenleft}G\ \isactrlbold {\isasymand}\ H{\isacharparenright}\ {\isasymin}\ S\ {\isasymlongrightarrow}\ \isactrlbold {\isasymnot}\ G\ {\isasymin}\ S\ {\isasymor}\ \isactrlbold {\isasymnot}\ H\ {\isasymin}\ S{\isacharparenright}\isanewline
\ \ \ \ {\isasymand}\ {\isacharparenleft}{\isasymforall}G\ H{\isachardot}\ \isactrlbold {\isasymnot}{\isacharparenleft}G\ \isactrlbold {\isasymor}\ H{\isacharparenright}\ {\isasymin}\ S\ {\isasymlongrightarrow}\ \isactrlbold {\isasymnot}\ G\ {\isasymin}\ S\ {\isasymand}\ \isactrlbold {\isasymnot}\ H\ {\isasymin}\ S{\isacharparenright}\isanewline
\ \ \ \ {\isasymand}\ {\isacharparenleft}{\isasymforall}G\ H{\isachardot}\ \isactrlbold {\isasymnot}{\isacharparenleft}G\ \isactrlbold {\isasymrightarrow}\ H{\isacharparenright}\ {\isasymin}\ S\ {\isasymlongrightarrow}\ G\ {\isasymin}\ S\ {\isasymand}\ \isactrlbold {\isasymnot}\ H\ {\isasymin}\ S{\isacharparenright}{\isachardoublequoteclose}\isanewline
\ \ \ \ \ \ \isacommand{using}\isamarkupfalse%
\ C{\isadigit{6}}\ C{\isadigit{7}}\ C{\isadigit{8}}\ C{\isadigit{9}}\ \isacommand{by}\isamarkupfalse%
\ {\isacharparenleft}iprover\ intro{\isacharcolon}\ conjI{\isacharparenright}\isanewline
\ \ \ \ \isacommand{have}\isamarkupfalse%
\ {\isachardoublequoteopen}{\isacharparenleft}{\isasymbottom}\ {\isasymnotin}\ S\isanewline
\ \ \ \ {\isasymand}\ {\isacharparenleft}{\isasymforall}k{\isachardot}\ Atom\ k\ {\isasymin}\ S\ {\isasymlongrightarrow}\ \isactrlbold {\isasymnot}\ {\isacharparenleft}Atom\ k{\isacharparenright}\ {\isasymin}\ S\ {\isasymlongrightarrow}\ False{\isacharparenright}\isanewline
\ \ \ \ {\isasymand}\ {\isacharparenleft}{\isasymforall}G\ H{\isachardot}\ G\ \isactrlbold {\isasymand}\ H\ {\isasymin}\ S\ {\isasymlongrightarrow}\ G\ {\isasymin}\ S\ {\isasymand}\ H\ {\isasymin}\ S{\isacharparenright}\isanewline
\ \ \ \ {\isasymand}\ {\isacharparenleft}{\isasymforall}G\ H{\isachardot}\ G\ \isactrlbold {\isasymor}\ H\ {\isasymin}\ S\ {\isasymlongrightarrow}\ G\ {\isasymin}\ S\ {\isasymor}\ H\ {\isasymin}\ S{\isacharparenright}\isanewline
\ \ \ \ {\isasymand}\ {\isacharparenleft}{\isasymforall}G\ H{\isachardot}\ G\ \isactrlbold {\isasymrightarrow}\ H\ {\isasymin}\ S\ {\isasymlongrightarrow}\ \isactrlbold {\isasymnot}G\ {\isasymin}\ S\ {\isasymor}\ H\ {\isasymin}\ S{\isacharparenright}{\isacharparenright}\isanewline
\ \ \ \ {\isasymand}\ {\isacharparenleft}{\isacharparenleft}{\isasymforall}G{\isachardot}\ \isactrlbold {\isasymnot}\ {\isacharparenleft}\isactrlbold {\isasymnot}G{\isacharparenright}\ {\isasymin}\ S\ {\isasymlongrightarrow}\ G\ {\isasymin}\ S{\isacharparenright}\isanewline
\ \ \ \ {\isasymand}\ {\isacharparenleft}{\isasymforall}G\ H{\isachardot}\ \isactrlbold {\isasymnot}{\isacharparenleft}G\ \isactrlbold {\isasymand}\ H{\isacharparenright}\ {\isasymin}\ S\ {\isasymlongrightarrow}\ \isactrlbold {\isasymnot}\ G\ {\isasymin}\ S\ {\isasymor}\ \isactrlbold {\isasymnot}\ H\ {\isasymin}\ S{\isacharparenright}\isanewline
\ \ \ \ {\isasymand}\ {\isacharparenleft}{\isasymforall}G\ H{\isachardot}\ \isactrlbold {\isasymnot}{\isacharparenleft}G\ \isactrlbold {\isasymor}\ H{\isacharparenright}\ {\isasymin}\ S\ {\isasymlongrightarrow}\ \isactrlbold {\isasymnot}\ G\ {\isasymin}\ S\ {\isasymand}\ \isactrlbold {\isasymnot}\ H\ {\isasymin}\ S{\isacharparenright}\isanewline
\ \ \ \ {\isasymand}\ {\isacharparenleft}{\isasymforall}G\ H{\isachardot}\ \isactrlbold {\isasymnot}{\isacharparenleft}G\ \isactrlbold {\isasymrightarrow}\ H{\isacharparenright}\ {\isasymin}\ S\ {\isasymlongrightarrow}\ G\ {\isasymin}\ S\ {\isasymand}\ \isactrlbold {\isasymnot}\ H\ {\isasymin}\ S{\isacharparenright}{\isacharparenright}{\isachardoublequoteclose}\isanewline
\ \ \ \ \ \ \isacommand{using}\isamarkupfalse%
\ A\ B\ \isacommand{by}\isamarkupfalse%
\ {\isacharparenleft}rule\ conjI{\isacharparenright}\isanewline
\ \ \ \ \isacommand{thus}\isamarkupfalse%
\ {\isachardoublequoteopen}{\isasymbottom}\ {\isasymnotin}\ S\isanewline
\ \ \ \ {\isasymand}\ {\isacharparenleft}{\isasymforall}k{\isachardot}\ Atom\ k\ {\isasymin}\ S\ {\isasymlongrightarrow}\ \isactrlbold {\isasymnot}\ {\isacharparenleft}Atom\ k{\isacharparenright}\ {\isasymin}\ S\ {\isasymlongrightarrow}\ False{\isacharparenright}\isanewline
\ \ \ \ {\isasymand}\ {\isacharparenleft}{\isasymforall}G\ H{\isachardot}\ G\ \isactrlbold {\isasymand}\ H\ {\isasymin}\ S\ {\isasymlongrightarrow}\ G\ {\isasymin}\ S\ {\isasymand}\ H\ {\isasymin}\ S{\isacharparenright}\isanewline
\ \ \ \ {\isasymand}\ {\isacharparenleft}{\isasymforall}G\ H{\isachardot}\ G\ \isactrlbold {\isasymor}\ H\ {\isasymin}\ S\ {\isasymlongrightarrow}\ G\ {\isasymin}\ S\ {\isasymor}\ H\ {\isasymin}\ S{\isacharparenright}\isanewline
\ \ \ \ {\isasymand}\ {\isacharparenleft}{\isasymforall}G\ H{\isachardot}\ G\ \isactrlbold {\isasymrightarrow}\ H\ {\isasymin}\ S\ {\isasymlongrightarrow}\ \isactrlbold {\isasymnot}G\ {\isasymin}\ S\ {\isasymor}\ H\ {\isasymin}\ S{\isacharparenright}\isanewline
\ \ \ \ {\isasymand}\ {\isacharparenleft}{\isasymforall}G{\isachardot}\ \isactrlbold {\isasymnot}\ {\isacharparenleft}\isactrlbold {\isasymnot}G{\isacharparenright}\ {\isasymin}\ S\ {\isasymlongrightarrow}\ G\ {\isasymin}\ S{\isacharparenright}\isanewline
\ \ \ \ {\isasymand}\ {\isacharparenleft}{\isasymforall}G\ H{\isachardot}\ \isactrlbold {\isasymnot}{\isacharparenleft}G\ \isactrlbold {\isasymand}\ H{\isacharparenright}\ {\isasymin}\ S\ {\isasymlongrightarrow}\ \isactrlbold {\isasymnot}\ G\ {\isasymin}\ S\ {\isasymor}\ \isactrlbold {\isasymnot}\ H\ {\isasymin}\ S{\isacharparenright}\isanewline
\ \ \ \ {\isasymand}\ {\isacharparenleft}{\isasymforall}G\ H{\isachardot}\ \isactrlbold {\isasymnot}{\isacharparenleft}G\ \isactrlbold {\isasymor}\ H{\isacharparenright}\ {\isasymin}\ S\ {\isasymlongrightarrow}\ \isactrlbold {\isasymnot}\ G\ {\isasymin}\ S\ {\isasymand}\ \isactrlbold {\isasymnot}\ H\ {\isasymin}\ S{\isacharparenright}\isanewline
\ \ \ \ {\isasymand}\ {\isacharparenleft}{\isasymforall}G\ H{\isachardot}\ \isactrlbold {\isasymnot}{\isacharparenleft}G\ \isactrlbold {\isasymrightarrow}\ H{\isacharparenright}\ {\isasymin}\ S\ {\isasymlongrightarrow}\ G\ {\isasymin}\ S\ {\isasymand}\ \isactrlbold {\isasymnot}\ H\ {\isasymin}\ S{\isacharparenright}{\isachardoublequoteclose}\ \isanewline
\ \ \ \ \ \ \isacommand{by}\isamarkupfalse%
\ {\isacharparenleft}iprover\ intro{\isacharcolon}\ conj{\isacharunderscore}assoc{\isacharparenright}\isanewline
\ \ \isacommand{qed}\isamarkupfalse%
\isanewline
\ \ \isacommand{thus}\isamarkupfalse%
\ {\isachardoublequoteopen}Hintikka\ S{\isachardoublequoteclose}\isanewline
\ \ \ \ \isacommand{unfolding}\isamarkupfalse%
\ Hintikka{\isacharunderscore}def\ \isacommand{by}\isamarkupfalse%
\ this\isanewline
\isacommand{qed}\isamarkupfalse%
%
\endisatagproof
{\isafoldproof}%
%
\isadelimproof
%
\endisadelimproof
%
\begin{isamarkuptext}%
En conclusión, el lema completo se demuestra detalladamente en Isabelle/HOL como sigue.%
\end{isamarkuptext}\isamarkuptrue%
\isacommand{lemma}\isamarkupfalse%
\ {\isachardoublequoteopen}Hintikka\ S\ {\isacharequal}\ {\isacharparenleft}{\isasymbottom}\ {\isasymnotin}\ S\isanewline
{\isasymand}\ {\isacharparenleft}{\isasymforall}k{\isachardot}\ Atom\ k\ {\isasymin}\ S\ {\isasymlongrightarrow}\ \isactrlbold {\isasymnot}\ {\isacharparenleft}Atom\ k{\isacharparenright}\ {\isasymin}\ S\ {\isasymlongrightarrow}\ False{\isacharparenright}\isanewline
{\isasymand}\ {\isacharparenleft}{\isasymforall}F\ G\ H{\isachardot}\ Con\ F\ G\ H\ {\isasymlongrightarrow}\ F\ {\isasymin}\ S\ {\isasymlongrightarrow}\ G\ {\isasymin}\ S\ {\isasymand}\ H\ {\isasymin}\ S{\isacharparenright}\isanewline
{\isasymand}\ {\isacharparenleft}{\isasymforall}F\ G\ H{\isachardot}\ Dis\ F\ G\ H\ {\isasymlongrightarrow}\ F\ {\isasymin}\ S\ {\isasymlongrightarrow}\ G\ {\isasymin}\ S\ {\isasymor}\ H\ {\isasymin}\ S{\isacharparenright}{\isacharparenright}{\isachardoublequoteclose}\ \ \isanewline
%
\isadelimproof
%
\endisadelimproof
%
\isatagproof
\isacommand{proof}\isamarkupfalse%
\ {\isacharparenleft}rule\ iffI{\isacharparenright}\isanewline
\ \ \isacommand{assume}\isamarkupfalse%
\ {\isachardoublequoteopen}Hintikka\ S{\isachardoublequoteclose}\isanewline
\ \ \isacommand{thus}\isamarkupfalse%
\ {\isachardoublequoteopen}{\isacharparenleft}{\isasymbottom}\ {\isasymnotin}\ S\isanewline
\ \ {\isasymand}\ {\isacharparenleft}{\isasymforall}k{\isachardot}\ Atom\ k\ {\isasymin}\ S\ {\isasymlongrightarrow}\ \isactrlbold {\isasymnot}\ {\isacharparenleft}Atom\ k{\isacharparenright}\ {\isasymin}\ S\ {\isasymlongrightarrow}\ False{\isacharparenright}\isanewline
\ \ {\isasymand}\ {\isacharparenleft}{\isasymforall}F\ G\ H{\isachardot}\ Con\ F\ G\ H\ {\isasymlongrightarrow}\ F\ {\isasymin}\ S\ {\isasymlongrightarrow}\ G\ {\isasymin}\ S\ {\isasymand}\ H\ {\isasymin}\ S{\isacharparenright}\isanewline
\ \ {\isasymand}\ {\isacharparenleft}{\isasymforall}F\ G\ H{\isachardot}\ Dis\ F\ G\ H\ {\isasymlongrightarrow}\ F\ {\isasymin}\ S\ {\isasymlongrightarrow}\ G\ {\isasymin}\ S\ {\isasymor}\ H\ {\isasymin}\ S{\isacharparenright}{\isacharparenright}{\isachardoublequoteclose}\isanewline
\ \ \ \ \isacommand{by}\isamarkupfalse%
\ {\isacharparenleft}rule\ Hintikka{\isacharunderscore}alt{\isadigit{1}}{\isacharparenright}\isanewline
\isacommand{next}\isamarkupfalse%
\isanewline
\ \ \isacommand{assume}\isamarkupfalse%
\ {\isachardoublequoteopen}{\isacharparenleft}{\isasymbottom}\ {\isasymnotin}\ S\isanewline
\ \ {\isasymand}\ {\isacharparenleft}{\isasymforall}k{\isachardot}\ Atom\ k\ {\isasymin}\ S\ {\isasymlongrightarrow}\ \isactrlbold {\isasymnot}\ {\isacharparenleft}Atom\ k{\isacharparenright}\ {\isasymin}\ S\ {\isasymlongrightarrow}\ False{\isacharparenright}\isanewline
\ \ {\isasymand}\ {\isacharparenleft}{\isasymforall}F\ G\ H{\isachardot}\ Con\ F\ G\ H\ {\isasymlongrightarrow}\ F\ {\isasymin}\ S\ {\isasymlongrightarrow}\ G\ {\isasymin}\ S\ {\isasymand}\ H\ {\isasymin}\ S{\isacharparenright}\isanewline
\ \ {\isasymand}\ {\isacharparenleft}{\isasymforall}F\ G\ H{\isachardot}\ Dis\ F\ G\ H\ {\isasymlongrightarrow}\ F\ {\isasymin}\ S\ {\isasymlongrightarrow}\ G\ {\isasymin}\ S\ {\isasymor}\ H\ {\isasymin}\ S{\isacharparenright}{\isacharparenright}{\isachardoublequoteclose}\isanewline
\ \ \isacommand{thus}\isamarkupfalse%
\ {\isachardoublequoteopen}Hintikka\ S{\isachardoublequoteclose}\isanewline
\ \ \ \ \isacommand{by}\isamarkupfalse%
\ {\isacharparenleft}rule\ Hintikka{\isacharunderscore}alt{\isadigit{2}}{\isacharparenright}\isanewline
\isacommand{qed}\isamarkupfalse%
%
\endisatagproof
{\isafoldproof}%
%
\isadelimproof
%
\endisadelimproof
%
\begin{isamarkuptext}%
Por último, veamos su demostración automática.%
\end{isamarkuptext}\isamarkuptrue%
\isacommand{lemma}\isamarkupfalse%
\ Hintikka{\isacharunderscore}alt{\isacharcolon}\ {\isachardoublequoteopen}Hintikka\ S\ {\isacharequal}\ {\isacharparenleft}{\isasymbottom}\ {\isasymnotin}\ S\isanewline
{\isasymand}\ {\isacharparenleft}{\isasymforall}k{\isachardot}\ Atom\ k\ {\isasymin}\ S\ {\isasymlongrightarrow}\ \isactrlbold {\isasymnot}\ {\isacharparenleft}Atom\ k{\isacharparenright}\ {\isasymin}\ S\ {\isasymlongrightarrow}\ False{\isacharparenright}\isanewline
{\isasymand}\ {\isacharparenleft}{\isasymforall}F\ G\ H{\isachardot}\ Con\ F\ G\ H\ {\isasymlongrightarrow}\ F\ {\isasymin}\ S\ {\isasymlongrightarrow}\ G\ {\isasymin}\ S\ {\isasymand}\ H\ {\isasymin}\ S{\isacharparenright}\isanewline
{\isasymand}\ {\isacharparenleft}{\isasymforall}F\ G\ H{\isachardot}\ Dis\ F\ G\ H\ {\isasymlongrightarrow}\ F\ {\isasymin}\ S\ {\isasymlongrightarrow}\ G\ {\isasymin}\ S\ {\isasymor}\ H\ {\isasymin}\ S{\isacharparenright}{\isacharparenright}{\isachardoublequoteclose}\ \ \isanewline
%
\isadelimproof
\ \ %
\endisadelimproof
%
\isatagproof
\isacommand{apply}\isamarkupfalse%
{\isacharparenleft}simp\ add{\isacharcolon}\ Hintikka{\isacharunderscore}def\ con{\isacharunderscore}dis{\isacharunderscore}simps{\isacharparenright}\isanewline
\ \ \isacommand{apply}\isamarkupfalse%
{\isacharparenleft}rule\ iffI{\isacharparenright}\isanewline
\ \ \ \isacommand{subgoal}\isamarkupfalse%
\ \isacommand{by}\isamarkupfalse%
\ blast\isanewline
\ \ \isacommand{subgoal}\isamarkupfalse%
\ \isacommand{by}\isamarkupfalse%
\ safe\ metis{\isacharplus}\isanewline
\ \ \isacommand{done}\isamarkupfalse%
\isanewline
%
\endisatagproof
{\isafoldproof}%
%
\isadelimproof
%
\endisadelimproof
%
\isadelimtheory
%
\endisadelimtheory
%
\isatagtheory
%
\endisatagtheory
{\isafoldtheory}%
%
\isadelimtheory
%
\endisadelimtheory
%
\end{isabellebody}%
\endinput
%:%file=~/TFM/TFM/Notunif.thy%:%
%:%19=11%:%
%:%20=12%:%
%:%21=13%:%
%:%25=17%:%
%:%26=18%:%
%:%27=19%:%
%:%28=20%:%
%:%29=21%:%
%:%30=22%:%
%:%31=23%:%
%:%32=24%:%
%:%33=25%:%
%:%34=26%:%
%:%35=27%:%
%:%36=28%:%
%:%37=29%:%
%:%38=30%:%
%:%39=31%:%
%:%40=32%:%
%:%41=33%:%
%:%42=34%:%
%:%43=35%:%
%:%44=36%:%
%:%46=38%:%
%:%47=38%:%
%:%49=40%:%
%:%50=41%:%
%:%52=43%:%
%:%53=43%:%
%:%56=44%:%
%:%60=44%:%
%:%61=44%:%
%:%66=44%:%
%:%69=45%:%
%:%70=46%:%
%:%71=46%:%
%:%74=47%:%
%:%78=47%:%
%:%79=47%:%
%:%84=47%:%
%:%87=48%:%
%:%88=49%:%
%:%89=49%:%
%:%92=50%:%
%:%96=50%:%
%:%97=50%:%
%:%102=50%:%
%:%105=51%:%
%:%106=52%:%
%:%107=52%:%
%:%110=53%:%
%:%114=53%:%
%:%115=53%:%
%:%120=53%:%
%:%123=54%:%
%:%124=55%:%
%:%125=55%:%
%:%128=56%:%
%:%132=56%:%
%:%133=56%:%
%:%138=56%:%
%:%141=57%:%
%:%142=58%:%
%:%143=58%:%
%:%146=59%:%
%:%150=59%:%
%:%151=59%:%
%:%156=59%:%
%:%159=60%:%
%:%160=61%:%
%:%161=61%:%
%:%164=62%:%
%:%168=62%:%
%:%169=62%:%
%:%174=62%:%
%:%177=63%:%
%:%178=64%:%
%:%179=64%:%
%:%182=65%:%
%:%186=65%:%
%:%187=65%:%
%:%192=65%:%
%:%195=66%:%
%:%196=67%:%
%:%197=67%:%
%:%200=68%:%
%:%204=68%:%
%:%205=68%:%
%:%210=68%:%
%:%213=69%:%
%:%214=70%:%
%:%215=70%:%
%:%218=71%:%
%:%222=71%:%
%:%223=71%:%
%:%232=73%:%
%:%233=74%:%
%:%235=76%:%
%:%236=76%:%
%:%239=77%:%
%:%243=77%:%
%:%244=77%:%
%:%249=77%:%
%:%252=78%:%
%:%253=79%:%
%:%254=79%:%
%:%257=80%:%
%:%261=80%:%
%:%262=80%:%
%:%271=82%:%
%:%272=83%:%
%:%273=84%:%
%:%274=85%:%
%:%275=86%:%
%:%276=87%:%
%:%277=88%:%
%:%278=89%:%
%:%279=90%:%
%:%280=91%:%
%:%281=92%:%
%:%282=93%:%
%:%283=94%:%
%:%284=95%:%
%:%285=96%:%
%:%286=97%:%
%:%287=98%:%
%:%288=99%:%
%:%289=100%:%
%:%290=101%:%
%:%291=102%:%
%:%292=103%:%
%:%293=104%:%
%:%294=105%:%
%:%295=106%:%
%:%296=107%:%
%:%298=109%:%
%:%299=109%:%
%:%300=110%:%
%:%301=111%:%
%:%302=112%:%
%:%303=113%:%
%:%305=115%:%
%:%306=116%:%
%:%307=117%:%
%:%308=118%:%
%:%309=119%:%
%:%312=119%:%
%:%313=120%:%
%:%314=121%:%
%:%315=122%:%
%:%316=123%:%
%:%317=124%:%
%:%318=125%:%
%:%319=126%:%
%:%320=127%:%
%:%321=128%:%
%:%322=129%:%
%:%323=130%:%
%:%324=131%:%
%:%325=132%:%
%:%326=133%:%
%:%327=134%:%
%:%328=135%:%
%:%329=136%:%
%:%330=137%:%
%:%331=138%:%
%:%332=139%:%
%:%333=140%:%
%:%334=141%:%
%:%335=142%:%
%:%337=144%:%
%:%338=144%:%
%:%339=145%:%
%:%340=146%:%
%:%341=147%:%
%:%342=148%:%
%:%344=150%:%
%:%345=151%:%
%:%346=152%:%
%:%347=153%:%
%:%348=154%:%
%:%351=154%:%
%:%352=155%:%
%:%353=156%:%
%:%354=157%:%
%:%355=158%:%
%:%356=159%:%
%:%357=160%:%
%:%358=161%:%
%:%359=162%:%
%:%360=163%:%
%:%361=164%:%
%:%362=165%:%
%:%364=167%:%
%:%365=167%:%
%:%368=168%:%
%:%372=168%:%
%:%373=168%:%
%:%374=168%:%
%:%383=170%:%
%:%384=171%:%
%:%385=172%:%
%:%387=174%:%
%:%388=174%:%
%:%391=175%:%
%:%395=175%:%
%:%396=175%:%
%:%401=175%:%
%:%404=176%:%
%:%405=177%:%
%:%406=177%:%
%:%409=178%:%
%:%413=178%:%
%:%414=178%:%
%:%423=180%:%
%:%424=181%:%
%:%426=183%:%
%:%427=183%:%
%:%430=184%:%
%:%434=184%:%
%:%435=184%:%
%:%444=186%:%
%:%445=187%:%
%:%447=189%:%
%:%448=189%:%
%:%449=190%:%
%:%452=193%:%
%:%453=194%:%
%:%456=197%:%
%:%459=198%:%
%:%463=198%:%
%:%464=198%:%
%:%473=200%:%
%:%474=201%:%
%:%475=202%:%
%:%476=203%:%
%:%477=204%:%
%:%478=205%:%
%:%479=206%:%
%:%480=207%:%
%:%481=208%:%
%:%482=209%:%
%:%483=210%:%
%:%484=211%:%
%:%485=212%:%
%:%486=213%:%
%:%487=214%:%
%:%488=215%:%
%:%489=216%:%
%:%490=217%:%
%:%491=218%:%
%:%492=219%:%
%:%493=220%:%
%:%494=221%:%
%:%496=223%:%
%:%497=223%:%
%:%500=226%:%
%:%503=227%:%
%:%507=227%:%
%:%517=229%:%
%:%518=230%:%
%:%519=231%:%
%:%520=232%:%
%:%521=233%:%
%:%522=234%:%
%:%523=235%:%
%:%524=236%:%
%:%525=237%:%
%:%526=238%:%
%:%527=239%:%
%:%528=240%:%
%:%529=241%:%
%:%530=242%:%
%:%531=243%:%
%:%532=244%:%
%:%533=245%:%
%:%534=246%:%
%:%535=247%:%
%:%536=248%:%
%:%537=249%:%
%:%538=250%:%
%:%539=251%:%
%:%540=252%:%
%:%541=253%:%
%:%542=254%:%
%:%543=255%:%
%:%544=256%:%
%:%545=257%:%
%:%546=258%:%
%:%547=259%:%
%:%548=260%:%
%:%549=261%:%
%:%550=262%:%
%:%551=263%:%
%:%552=264%:%
%:%553=265%:%
%:%554=266%:%
%:%555=267%:%
%:%556=268%:%
%:%557=269%:%
%:%558=270%:%
%:%559=271%:%
%:%560=272%:%
%:%561=273%:%
%:%562=274%:%
%:%563=275%:%
%:%564=276%:%
%:%565=277%:%
%:%566=278%:%
%:%567=279%:%
%:%568=280%:%
%:%569=281%:%
%:%570=282%:%
%:%571=283%:%
%:%572=284%:%
%:%573=285%:%
%:%574=286%:%
%:%575=287%:%
%:%576=288%:%
%:%577=289%:%
%:%578=290%:%
%:%579=291%:%
%:%580=292%:%
%:%581=293%:%
%:%582=294%:%
%:%583=295%:%
%:%584=296%:%
%:%585=297%:%
%:%586=298%:%
%:%587=299%:%
%:%588=300%:%
%:%589=301%:%
%:%590=302%:%
%:%591=303%:%
%:%592=304%:%
%:%593=305%:%
%:%594=306%:%
%:%595=307%:%
%:%596=308%:%
%:%597=309%:%
%:%598=310%:%
%:%599=311%:%
%:%600=312%:%
%:%601=313%:%
%:%602=314%:%
%:%603=315%:%
%:%604=316%:%
%:%605=317%:%
%:%606=318%:%
%:%607=319%:%
%:%608=320%:%
%:%609=321%:%
%:%610=322%:%
%:%611=323%:%
%:%612=324%:%
%:%613=325%:%
%:%614=326%:%
%:%615=327%:%
%:%616=328%:%
%:%617=329%:%
%:%618=330%:%
%:%619=331%:%
%:%620=332%:%
%:%621=333%:%
%:%622=334%:%
%:%623=335%:%
%:%624=336%:%
%:%625=337%:%
%:%626=338%:%
%:%627=339%:%
%:%628=340%:%
%:%629=341%:%
%:%630=342%:%
%:%631=343%:%
%:%632=344%:%
%:%633=345%:%
%:%634=346%:%
%:%635=347%:%
%:%636=348%:%
%:%637=349%:%
%:%638=350%:%
%:%639=351%:%
%:%640=352%:%
%:%641=353%:%
%:%642=354%:%
%:%643=355%:%
%:%644=356%:%
%:%645=357%:%
%:%646=358%:%
%:%647=359%:%
%:%648=360%:%
%:%649=361%:%
%:%650=362%:%
%:%651=363%:%
%:%652=364%:%
%:%653=365%:%
%:%654=366%:%
%:%655=367%:%
%:%656=368%:%
%:%657=369%:%
%:%658=370%:%
%:%659=371%:%
%:%660=372%:%
%:%661=373%:%
%:%662=374%:%
%:%663=375%:%
%:%664=376%:%
%:%665=377%:%
%:%666=378%:%
%:%667=379%:%
%:%668=380%:%
%:%669=381%:%
%:%670=382%:%
%:%671=383%:%
%:%672=384%:%
%:%673=385%:%
%:%674=386%:%
%:%675=387%:%
%:%676=388%:%
%:%677=389%:%
%:%678=390%:%
%:%679=391%:%
%:%681=393%:%
%:%682=393%:%
%:%683=394%:%
%:%686=397%:%
%:%687=398%:%
%:%694=399%:%
%:%695=399%:%
%:%696=400%:%
%:%697=400%:%
%:%698=401%:%
%:%699=401%:%
%:%700=401%:%
%:%703=404%:%
%:%704=405%:%
%:%705=405%:%
%:%706=406%:%
%:%707=406%:%
%:%708=407%:%
%:%709=407%:%
%:%710=408%:%
%:%711=408%:%
%:%712=409%:%
%:%713=409%:%
%:%714=410%:%
%:%715=410%:%
%:%716=410%:%
%:%717=411%:%
%:%718=411%:%
%:%719=412%:%
%:%720=412%:%
%:%721=412%:%
%:%722=413%:%
%:%723=413%:%
%:%724=414%:%
%:%725=414%:%
%:%727=416%:%
%:%728=417%:%
%:%729=417%:%
%:%730=418%:%
%:%731=418%:%
%:%732=419%:%
%:%733=419%:%
%:%734=420%:%
%:%735=420%:%
%:%736=421%:%
%:%737=421%:%
%:%738=421%:%
%:%739=422%:%
%:%740=422%:%
%:%741=422%:%
%:%742=423%:%
%:%743=423%:%
%:%744=424%:%
%:%745=424%:%
%:%746=425%:%
%:%747=425%:%
%:%748=425%:%
%:%749=426%:%
%:%750=426%:%
%:%751=427%:%
%:%752=427%:%
%:%753=427%:%
%:%754=428%:%
%:%755=428%:%
%:%756=429%:%
%:%757=429%:%
%:%758=429%:%
%:%759=430%:%
%:%760=430%:%
%:%761=431%:%
%:%762=431%:%
%:%763=431%:%
%:%764=432%:%
%:%765=432%:%
%:%766=433%:%
%:%767=433%:%
%:%768=434%:%
%:%769=435%:%
%:%770=435%:%
%:%771=436%:%
%:%772=436%:%
%:%773=437%:%
%:%774=437%:%
%:%775=438%:%
%:%776=438%:%
%:%777=439%:%
%:%778=439%:%
%:%779=439%:%
%:%780=440%:%
%:%781=440%:%
%:%782=441%:%
%:%783=441%:%
%:%784=441%:%
%:%785=442%:%
%:%786=442%:%
%:%787=443%:%
%:%788=443%:%
%:%789=443%:%
%:%790=444%:%
%:%791=444%:%
%:%792=445%:%
%:%793=445%:%
%:%794=445%:%
%:%795=446%:%
%:%796=446%:%
%:%797=447%:%
%:%798=447%:%
%:%799=447%:%
%:%800=448%:%
%:%801=448%:%
%:%802=449%:%
%:%803=449%:%
%:%804=450%:%
%:%805=450%:%
%:%806=450%:%
%:%807=451%:%
%:%808=451%:%
%:%809=452%:%
%:%810=452%:%
%:%811=453%:%
%:%812=453%:%
%:%813=453%:%
%:%814=454%:%
%:%815=454%:%
%:%816=455%:%
%:%817=455%:%
%:%818=455%:%
%:%819=456%:%
%:%820=456%:%
%:%821=456%:%
%:%822=457%:%
%:%823=457%:%
%:%824=458%:%
%:%825=458%:%
%:%826=458%:%
%:%827=459%:%
%:%828=459%:%
%:%829=460%:%
%:%830=460%:%
%:%831=460%:%
%:%832=461%:%
%:%833=461%:%
%:%834=462%:%
%:%835=462%:%
%:%836=463%:%
%:%837=463%:%
%:%838=464%:%
%:%839=464%:%
%:%840=465%:%
%:%841=465%:%
%:%842=466%:%
%:%843=466%:%
%:%844=467%:%
%:%854=469%:%
%:%855=470%:%
%:%856=471%:%
%:%857=472%:%
%:%858=473%:%
%:%859=474%:%
%:%861=476%:%
%:%862=476%:%
%:%863=477%:%
%:%866=480%:%
%:%867=481%:%
%:%874=482%:%
%:%875=482%:%
%:%876=483%:%
%:%877=483%:%
%:%878=484%:%
%:%879=484%:%
%:%880=484%:%
%:%883=487%:%
%:%884=488%:%
%:%885=488%:%
%:%886=489%:%
%:%887=489%:%
%:%888=490%:%
%:%889=490%:%
%:%890=491%:%
%:%891=491%:%
%:%892=492%:%
%:%893=492%:%
%:%894=493%:%
%:%895=493%:%
%:%896=493%:%
%:%897=494%:%
%:%898=494%:%
%:%899=495%:%
%:%900=495%:%
%:%901=495%:%
%:%902=496%:%
%:%903=496%:%
%:%904=497%:%
%:%905=497%:%
%:%907=499%:%
%:%908=500%:%
%:%909=500%:%
%:%910=501%:%
%:%911=501%:%
%:%912=502%:%
%:%913=502%:%
%:%914=503%:%
%:%915=503%:%
%:%916=504%:%
%:%917=504%:%
%:%918=504%:%
%:%919=505%:%
%:%920=505%:%
%:%921=506%:%
%:%922=506%:%
%:%923=506%:%
%:%924=507%:%
%:%925=507%:%
%:%926=508%:%
%:%927=508%:%
%:%928=508%:%
%:%929=509%:%
%:%930=509%:%
%:%931=510%:%
%:%932=510%:%
%:%933=510%:%
%:%934=511%:%
%:%935=511%:%
%:%936=512%:%
%:%937=512%:%
%:%938=512%:%
%:%939=513%:%
%:%940=513%:%
%:%941=514%:%
%:%942=514%:%
%:%943=514%:%
%:%944=515%:%
%:%945=515%:%
%:%946=516%:%
%:%947=516%:%
%:%948=517%:%
%:%949=518%:%
%:%950=518%:%
%:%951=519%:%
%:%952=519%:%
%:%953=520%:%
%:%954=520%:%
%:%955=521%:%
%:%956=521%:%
%:%957=522%:%
%:%958=522%:%
%:%959=522%:%
%:%960=523%:%
%:%961=523%:%
%:%962=524%:%
%:%963=524%:%
%:%964=524%:%
%:%965=525%:%
%:%966=525%:%
%:%967=526%:%
%:%968=526%:%
%:%969=526%:%
%:%970=527%:%
%:%971=527%:%
%:%972=528%:%
%:%973=528%:%
%:%974=528%:%
%:%975=529%:%
%:%976=529%:%
%:%977=530%:%
%:%978=530%:%
%:%979=530%:%
%:%980=531%:%
%:%981=531%:%
%:%982=532%:%
%:%983=532%:%
%:%984=532%:%
%:%985=533%:%
%:%986=533%:%
%:%987=534%:%
%:%988=534%:%
%:%989=535%:%
%:%990=535%:%
%:%991=535%:%
%:%992=536%:%
%:%993=536%:%
%:%994=537%:%
%:%995=537%:%
%:%996=538%:%
%:%997=538%:%
%:%998=538%:%
%:%999=539%:%
%:%1000=539%:%
%:%1001=540%:%
%:%1002=540%:%
%:%1003=540%:%
%:%1004=541%:%
%:%1005=541%:%
%:%1006=541%:%
%:%1007=542%:%
%:%1008=542%:%
%:%1009=543%:%
%:%1010=543%:%
%:%1011=543%:%
%:%1012=544%:%
%:%1013=544%:%
%:%1014=545%:%
%:%1015=545%:%
%:%1016=545%:%
%:%1017=546%:%
%:%1018=546%:%
%:%1019=547%:%
%:%1020=547%:%
%:%1021=548%:%
%:%1022=548%:%
%:%1023=549%:%
%:%1024=549%:%
%:%1025=550%:%
%:%1026=550%:%
%:%1027=551%:%
%:%1028=551%:%
%:%1029=552%:%
%:%1039=554%:%
%:%1040=555%:%
%:%1042=557%:%
%:%1043=557%:%
%:%1044=558%:%
%:%1045=559%:%
%:%1048=562%:%
%:%1055=563%:%
%:%1056=563%:%
%:%1057=564%:%
%:%1058=564%:%
%:%1066=572%:%
%:%1067=573%:%
%:%1068=573%:%
%:%1069=573%:%
%:%1070=574%:%
%:%1071=574%:%
%:%1072=574%:%
%:%1073=575%:%
%:%1074=575%:%
%:%1075=576%:%
%:%1076=576%:%
%:%1077=577%:%
%:%1078=577%:%
%:%1079=577%:%
%:%1080=578%:%
%:%1081=578%:%
%:%1082=579%:%
%:%1083=579%:%
%:%1084=580%:%
%:%1085=580%:%
%:%1086=581%:%
%:%1087=581%:%
%:%1088=582%:%
%:%1089=582%:%
%:%1090=582%:%
%:%1091=583%:%
%:%1092=583%:%
%:%1093=584%:%
%:%1094=584%:%
%:%1095=584%:%
%:%1096=585%:%
%:%1097=585%:%
%:%1098=586%:%
%:%1099=586%:%
%:%1100=586%:%
%:%1101=587%:%
%:%1102=587%:%
%:%1103=588%:%
%:%1104=588%:%
%:%1105=588%:%
%:%1106=589%:%
%:%1107=589%:%
%:%1110=592%:%
%:%1111=593%:%
%:%1112=593%:%
%:%1113=593%:%
%:%1114=594%:%
%:%1115=594%:%
%:%1116=595%:%
%:%1117=595%:%
%:%1118=596%:%
%:%1119=596%:%
%:%1120=597%:%
%:%1121=597%:%
%:%1122=598%:%
%:%1123=598%:%
%:%1124=599%:%
%:%1125=599%:%
%:%1126=600%:%
%:%1127=600%:%
%:%1128=601%:%
%:%1129=601%:%
%:%1130=601%:%
%:%1131=602%:%
%:%1132=602%:%
%:%1133=603%:%
%:%1134=603%:%
%:%1135=603%:%
%:%1136=604%:%
%:%1137=604%:%
%:%1138=605%:%
%:%1139=605%:%
%:%1140=605%:%
%:%1141=606%:%
%:%1142=606%:%
%:%1143=607%:%
%:%1144=607%:%
%:%1145=607%:%
%:%1146=608%:%
%:%1147=608%:%
%:%1150=611%:%
%:%1151=612%:%
%:%1152=612%:%
%:%1153=612%:%
%:%1154=613%:%
%:%1155=613%:%
%:%1156=614%:%
%:%1157=614%:%
%:%1158=615%:%
%:%1159=615%:%
%:%1160=616%:%
%:%1161=616%:%
%:%1164=619%:%
%:%1165=620%:%
%:%1166=620%:%
%:%1167=620%:%
%:%1168=621%:%
%:%1178=623%:%
%:%1179=624%:%
%:%1181=626%:%
%:%1182=626%:%
%:%1183=627%:%
%:%1186=630%:%
%:%1187=631%:%
%:%1194=632%:%
%:%1195=632%:%
%:%1196=633%:%
%:%1197=633%:%
%:%1198=634%:%
%:%1199=634%:%
%:%1200=634%:%
%:%1201=635%:%
%:%1202=635%:%
%:%1203=636%:%
%:%1204=636%:%
%:%1205=636%:%
%:%1206=637%:%
%:%1207=637%:%
%:%1215=645%:%
%:%1216=646%:%
%:%1217=646%:%
%:%1218=647%:%
%:%1219=647%:%
%:%1220=648%:%
%:%1221=648%:%
%:%1222=648%:%
%:%1223=649%:%
%:%1224=649%:%
%:%1225=650%:%
%:%1226=650%:%
%:%1227=650%:%
%:%1228=651%:%
%:%1229=651%:%
%:%1230=652%:%
%:%1231=652%:%
%:%1232=653%:%
%:%1233=653%:%
%:%1234=654%:%
%:%1235=654%:%
%:%1236=655%:%
%:%1237=655%:%
%:%1238=656%:%
%:%1239=656%:%
%:%1240=657%:%
%:%1241=657%:%
%:%1242=658%:%
%:%1243=658%:%
%:%1244=659%:%
%:%1245=659%:%
%:%1246=660%:%
%:%1247=660%:%
%:%1248=660%:%
%:%1249=661%:%
%:%1250=661%:%
%:%1251=661%:%
%:%1252=662%:%
%:%1253=662%:%
%:%1254=662%:%
%:%1255=663%:%
%:%1256=663%:%
%:%1257=664%:%
%:%1258=664%:%
%:%1259=664%:%
%:%1260=665%:%
%:%1261=665%:%
%:%1262=666%:%
%:%1263=666%:%
%:%1264=667%:%
%:%1265=667%:%
%:%1266=668%:%
%:%1267=668%:%
%:%1268=669%:%
%:%1269=669%:%
%:%1270=670%:%
%:%1271=670%:%
%:%1272=671%:%
%:%1273=671%:%
%:%1274=672%:%
%:%1275=672%:%
%:%1276=673%:%
%:%1277=673%:%
%:%1278=674%:%
%:%1279=674%:%
%:%1280=675%:%
%:%1281=675%:%
%:%1282=676%:%
%:%1283=676%:%
%:%1284=676%:%
%:%1285=677%:%
%:%1286=677%:%
%:%1287=677%:%
%:%1288=678%:%
%:%1289=678%:%
%:%1290=678%:%
%:%1291=679%:%
%:%1292=679%:%
%:%1293=680%:%
%:%1294=680%:%
%:%1295=680%:%
%:%1296=681%:%
%:%1297=681%:%
%:%1298=682%:%
%:%1299=682%:%
%:%1300=683%:%
%:%1301=683%:%
%:%1302=684%:%
%:%1303=684%:%
%:%1304=685%:%
%:%1305=685%:%
%:%1306=686%:%
%:%1307=686%:%
%:%1308=687%:%
%:%1309=687%:%
%:%1310=688%:%
%:%1311=688%:%
%:%1312=689%:%
%:%1313=689%:%
%:%1314=690%:%
%:%1315=690%:%
%:%1316=691%:%
%:%1317=691%:%
%:%1318=692%:%
%:%1319=692%:%
%:%1320=692%:%
%:%1321=693%:%
%:%1322=693%:%
%:%1323=693%:%
%:%1324=694%:%
%:%1325=694%:%
%:%1326=694%:%
%:%1327=695%:%
%:%1328=695%:%
%:%1329=696%:%
%:%1330=696%:%
%:%1331=696%:%
%:%1332=697%:%
%:%1333=697%:%
%:%1334=698%:%
%:%1335=698%:%
%:%1336=699%:%
%:%1337=699%:%
%:%1338=700%:%
%:%1339=700%:%
%:%1340=701%:%
%:%1341=701%:%
%:%1342=702%:%
%:%1343=702%:%
%:%1344=703%:%
%:%1345=703%:%
%:%1346=704%:%
%:%1347=704%:%
%:%1348=705%:%
%:%1349=705%:%
%:%1350=706%:%
%:%1351=706%:%
%:%1352=707%:%
%:%1353=707%:%
%:%1354=708%:%
%:%1355=708%:%
%:%1356=708%:%
%:%1357=709%:%
%:%1358=709%:%
%:%1359=709%:%
%:%1360=710%:%
%:%1361=710%:%
%:%1362=710%:%
%:%1363=711%:%
%:%1364=711%:%
%:%1365=711%:%
%:%1366=712%:%
%:%1367=712%:%
%:%1368=712%:%
%:%1369=713%:%
%:%1370=713%:%
%:%1371=714%:%
%:%1372=714%:%
%:%1373=715%:%
%:%1374=715%:%
%:%1375=716%:%
%:%1376=716%:%
%:%1377=717%:%
%:%1378=717%:%
%:%1379=718%:%
%:%1380=718%:%
%:%1381=719%:%
%:%1382=719%:%
%:%1383=720%:%
%:%1384=720%:%
%:%1385=721%:%
%:%1386=721%:%
%:%1387=722%:%
%:%1388=722%:%
%:%1389=723%:%
%:%1390=723%:%
%:%1391=724%:%
%:%1392=724%:%
%:%1393=725%:%
%:%1394=725%:%
%:%1395=726%:%
%:%1396=726%:%
%:%1397=726%:%
%:%1398=727%:%
%:%1399=727%:%
%:%1400=727%:%
%:%1401=728%:%
%:%1402=728%:%
%:%1403=728%:%
%:%1404=729%:%
%:%1405=729%:%
%:%1406=730%:%
%:%1407=730%:%
%:%1408=730%:%
%:%1409=731%:%
%:%1410=731%:%
%:%1411=732%:%
%:%1412=732%:%
%:%1413=733%:%
%:%1414=733%:%
%:%1415=734%:%
%:%1416=734%:%
%:%1417=735%:%
%:%1418=735%:%
%:%1419=736%:%
%:%1420=736%:%
%:%1421=737%:%
%:%1422=737%:%
%:%1423=738%:%
%:%1424=738%:%
%:%1425=739%:%
%:%1426=739%:%
%:%1427=740%:%
%:%1428=740%:%
%:%1429=741%:%
%:%1430=741%:%
%:%1431=742%:%
%:%1432=742%:%
%:%1433=742%:%
%:%1434=743%:%
%:%1435=743%:%
%:%1436=743%:%
%:%1437=744%:%
%:%1438=744%:%
%:%1439=744%:%
%:%1440=745%:%
%:%1441=745%:%
%:%1442=746%:%
%:%1443=746%:%
%:%1444=746%:%
%:%1445=747%:%
%:%1446=747%:%
%:%1447=748%:%
%:%1448=748%:%
%:%1449=749%:%
%:%1450=749%:%
%:%1451=750%:%
%:%1452=750%:%
%:%1453=751%:%
%:%1454=751%:%
%:%1455=752%:%
%:%1456=752%:%
%:%1457=753%:%
%:%1458=753%:%
%:%1459=754%:%
%:%1460=754%:%
%:%1461=755%:%
%:%1462=755%:%
%:%1463=756%:%
%:%1464=756%:%
%:%1465=757%:%
%:%1466=757%:%
%:%1467=758%:%
%:%1468=758%:%
%:%1469=758%:%
%:%1470=759%:%
%:%1471=759%:%
%:%1472=759%:%
%:%1473=760%:%
%:%1474=760%:%
%:%1475=760%:%
%:%1476=761%:%
%:%1477=761%:%
%:%1478=762%:%
%:%1479=762%:%
%:%1480=762%:%
%:%1481=763%:%
%:%1482=763%:%
%:%1483=764%:%
%:%1484=764%:%
%:%1485=765%:%
%:%1486=765%:%
%:%1490=769%:%
%:%1491=770%:%
%:%1492=770%:%
%:%1493=770%:%
%:%1494=771%:%
%:%1495=771%:%
%:%1498=774%:%
%:%1499=775%:%
%:%1500=775%:%
%:%1501=775%:%
%:%1502=776%:%
%:%1503=776%:%
%:%1511=784%:%
%:%1512=785%:%
%:%1513=785%:%
%:%1514=785%:%
%:%1515=786%:%
%:%1516=786%:%
%:%1524=794%:%
%:%1525=795%:%
%:%1526=795%:%
%:%1527=796%:%
%:%1528=796%:%
%:%1529=797%:%
%:%1530=797%:%
%:%1531=798%:%
%:%1532=798%:%
%:%1533=798%:%
%:%1534=799%:%
%:%1544=801%:%
%:%1546=803%:%
%:%1547=803%:%
%:%1550=806%:%
%:%1557=807%:%
%:%1558=807%:%
%:%1559=808%:%
%:%1560=808%:%
%:%1561=809%:%
%:%1562=809%:%
%:%1565=812%:%
%:%1566=813%:%
%:%1567=813%:%
%:%1568=814%:%
%:%1569=814%:%
%:%1570=815%:%
%:%1571=815%:%
%:%1574=818%:%
%:%1575=819%:%
%:%1576=819%:%
%:%1577=820%:%
%:%1578=820%:%
%:%1579=821%:%
%:%1589=823%:%
%:%1591=825%:%
%:%1592=825%:%
%:%1595=828%:%
%:%1598=829%:%
%:%1602=829%:%
%:%1603=829%:%
%:%1604=830%:%
%:%1605=830%:%
%:%1606=831%:%
%:%1607=831%:%
%:%1608=831%:%
%:%1609=832%:%
%:%1610=832%:%
%:%1611=832%:%
%:%1612=833%:%
%:%1613=833%:%