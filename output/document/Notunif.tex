%
\begin{isabellebody}%
\setisabellecontext{Notunif}%
%
\isadelimtheory
%
\endisadelimtheory
%
\isatagtheory
%
\endisatagtheory
{\isafoldtheory}%
%
\isadelimtheory
%
\endisadelimtheory
%
\begin{isamarkuptext}%
En este capítulo introduciremos la notación uniforme inicialmente 
  desarrollada por R. M. Smullyan \isa{{\isacharbrackleft}{\isadigit{1}}{\isadigit{2}}{\isacharbrackright}}. La finalidad de dicha notación es 
  reducir el número de casos a considerar sobre la estructura de las fórmulas 
  al clasificar estas en dos categorías, facilitando las demostraciones
  y métodos empleados en adelante.

  De este modo, las fórmulas proposicionales pueden ser de dos tipos:
  las de tipo conjuntivo (las fórmulas \isa{{\isasymalpha}}) y las de tipo disyuntivo (las fórmulas \isa{{\isasymbeta}}). 
  Cada fórmula de tipo \isa{{\isasymalpha}}, o \isa{{\isasymbeta}} respectivamente, tiene asociada sus  
  dos componentes \isa{{\isasymalpha}\isactrlsub {\isadigit{1}}} y \isa{{\isasymalpha}\isactrlsub {\isadigit{2}}}, o \isa{{\isasymbeta}\isactrlsub {\isadigit{1}}} y \isa{{\isasymbeta}\isactrlsub {\isadigit{2}}} respectivamente. Para justificar dicha clasificación,
  introduzcamos inicialmente la definición de fórmulas semánticamente equivalentes.

  \begin{definicion}
    Dos fórmulas son \isa{semánticamente\ equivalentes} si tienen el mismo valor para toda 
    interpretación.
  \end{definicion}

  En Isabelle podemos formalizar la definición de la siguiente manera.%
\end{isamarkuptext}\isamarkuptrue%
\isacommand{definition}\isamarkupfalse%
\ {\isachardoublequoteopen}semanticEq\ F\ G\ {\isasymequiv}\ {\isasymforall}{\isasymA}{\isachardot}\ {\isacharparenleft}{\isasymA}\ {\isasymTurnstile}\ F{\isacharparenright}\ {\isasymlongleftrightarrow}\ {\isacharparenleft}{\isasymA}\ {\isasymTurnstile}\ G{\isacharparenright}{\isachardoublequoteclose}%
\begin{isamarkuptext}%
Según la definición del valor de verdad de una fórmula proposicional en una 
  interpretación dada, podemos ver los siguientes ejemplos de fórmulas semánticamente equivalentes.%
\end{isamarkuptext}\isamarkuptrue%
\isacommand{lemma}\isamarkupfalse%
\ {\isachardoublequoteopen}semanticEq\ {\isacharparenleft}Atom\ p{\isacharparenright}\ {\isacharparenleft}{\isacharparenleft}Atom\ p{\isacharparenright}\ \isactrlbold {\isasymor}\ {\isacharparenleft}Atom\ p{\isacharparenright}{\isacharparenright}{\isachardoublequoteclose}\ \isanewline
%
\isadelimproof
\ \ %
\endisadelimproof
%
\isatagproof
\isacommand{by}\isamarkupfalse%
\ {\isacharparenleft}simp\ add{\isacharcolon}\ semanticEq{\isacharunderscore}def{\isacharparenright}%
\endisatagproof
{\isafoldproof}%
%
\isadelimproof
\isanewline
%
\endisadelimproof
\isanewline
\isacommand{lemma}\isamarkupfalse%
\ {\isachardoublequoteopen}semanticEq\ {\isacharparenleft}Atom\ p{\isacharparenright}\ {\isacharparenleft}{\isacharparenleft}Atom\ p{\isacharparenright}\ \isactrlbold {\isasymand}\ {\isacharparenleft}Atom\ p{\isacharparenright}{\isacharparenright}{\isachardoublequoteclose}\ \isanewline
%
\isadelimproof
\ \ %
\endisadelimproof
%
\isatagproof
\isacommand{by}\isamarkupfalse%
\ {\isacharparenleft}simp\ add{\isacharcolon}\ semanticEq{\isacharunderscore}def{\isacharparenright}%
\endisatagproof
{\isafoldproof}%
%
\isadelimproof
\isanewline
%
\endisadelimproof
\isanewline
\isacommand{lemma}\isamarkupfalse%
\ {\isachardoublequoteopen}semanticEq\ {\isasymbottom}\ {\isacharparenleft}{\isasymbottom}\ \isactrlbold {\isasymand}\ {\isasymbottom}{\isacharparenright}{\isachardoublequoteclose}\ \isanewline
%
\isadelimproof
\ \ %
\endisadelimproof
%
\isatagproof
\isacommand{by}\isamarkupfalse%
\ {\isacharparenleft}simp\ add{\isacharcolon}\ semanticEq{\isacharunderscore}def{\isacharparenright}%
\endisatagproof
{\isafoldproof}%
%
\isadelimproof
\isanewline
%
\endisadelimproof
\isanewline
\isacommand{lemma}\isamarkupfalse%
\ {\isachardoublequoteopen}semanticEq\ {\isasymbottom}\ {\isacharparenleft}{\isasymbottom}\ \isactrlbold {\isasymor}\ {\isasymbottom}{\isacharparenright}{\isachardoublequoteclose}\ \isanewline
%
\isadelimproof
\ \ %
\endisadelimproof
%
\isatagproof
\isacommand{by}\isamarkupfalse%
\ {\isacharparenleft}simp\ add{\isacharcolon}\ semanticEq{\isacharunderscore}def{\isacharparenright}%
\endisatagproof
{\isafoldproof}%
%
\isadelimproof
\isanewline
%
\endisadelimproof
\isanewline
\isacommand{lemma}\isamarkupfalse%
\ {\isachardoublequoteopen}semanticEq\ {\isasymbottom}\ {\isacharparenleft}\isactrlbold {\isasymnot}\ {\isasymtop}{\isacharparenright}{\isachardoublequoteclose}\isanewline
%
\isadelimproof
\ \ %
\endisadelimproof
%
\isatagproof
\isacommand{by}\isamarkupfalse%
\ {\isacharparenleft}simp\ add{\isacharcolon}\ semanticEq{\isacharunderscore}def\ top{\isacharunderscore}semantics{\isacharparenright}%
\endisatagproof
{\isafoldproof}%
%
\isadelimproof
\isanewline
%
\endisadelimproof
\isanewline
\isacommand{lemma}\isamarkupfalse%
\ {\isachardoublequoteopen}semanticEq\ F\ {\isacharparenleft}\isactrlbold {\isasymnot}{\isacharparenleft}\isactrlbold {\isasymnot}\ F{\isacharparenright}{\isacharparenright}{\isachardoublequoteclose}\isanewline
%
\isadelimproof
\ \ %
\endisadelimproof
%
\isatagproof
\isacommand{by}\isamarkupfalse%
\ {\isacharparenleft}simp\ add{\isacharcolon}\ semanticEq{\isacharunderscore}def{\isacharparenright}%
\endisatagproof
{\isafoldproof}%
%
\isadelimproof
\isanewline
%
\endisadelimproof
\isanewline
\isacommand{lemma}\isamarkupfalse%
\ {\isachardoublequoteopen}semanticEq\ {\isacharparenleft}\isactrlbold {\isasymnot}{\isacharparenleft}\isactrlbold {\isasymnot}\ F{\isacharparenright}{\isacharparenright}\ {\isacharparenleft}F\ \isactrlbold {\isasymor}\ F{\isacharparenright}{\isachardoublequoteclose}\isanewline
%
\isadelimproof
\ \ %
\endisadelimproof
%
\isatagproof
\isacommand{by}\isamarkupfalse%
\ {\isacharparenleft}simp\ add{\isacharcolon}\ semanticEq{\isacharunderscore}def{\isacharparenright}%
\endisatagproof
{\isafoldproof}%
%
\isadelimproof
\isanewline
%
\endisadelimproof
\isanewline
\isacommand{lemma}\isamarkupfalse%
\ {\isachardoublequoteopen}semanticEq\ {\isacharparenleft}\isactrlbold {\isasymnot}{\isacharparenleft}\isactrlbold {\isasymnot}\ F{\isacharparenright}{\isacharparenright}\ {\isacharparenleft}F\ \isactrlbold {\isasymand}\ F{\isacharparenright}{\isachardoublequoteclose}\isanewline
%
\isadelimproof
\ \ %
\endisadelimproof
%
\isatagproof
\isacommand{by}\isamarkupfalse%
\ {\isacharparenleft}simp\ add{\isacharcolon}\ semanticEq{\isacharunderscore}def{\isacharparenright}%
\endisatagproof
{\isafoldproof}%
%
\isadelimproof
\isanewline
%
\endisadelimproof
\isanewline
\isacommand{lemma}\isamarkupfalse%
\ {\isachardoublequoteopen}semanticEq\ {\isacharparenleft}\isactrlbold {\isasymnot}\ F\ \isactrlbold {\isasymand}\ \isactrlbold {\isasymnot}\ G{\isacharparenright}\ {\isacharparenleft}\isactrlbold {\isasymnot}{\isacharparenleft}F\ \isactrlbold {\isasymor}\ G{\isacharparenright}{\isacharparenright}{\isachardoublequoteclose}\isanewline
%
\isadelimproof
\ \ %
\endisadelimproof
%
\isatagproof
\isacommand{by}\isamarkupfalse%
\ {\isacharparenleft}simp\ add{\isacharcolon}\ semanticEq{\isacharunderscore}def{\isacharparenright}%
\endisatagproof
{\isafoldproof}%
%
\isadelimproof
\isanewline
%
\endisadelimproof
\isanewline
\isacommand{lemma}\isamarkupfalse%
\ {\isachardoublequoteopen}semanticEq\ {\isacharparenleft}F\ \isactrlbold {\isasymrightarrow}\ G{\isacharparenright}\ {\isacharparenleft}\isactrlbold {\isasymnot}\ F\ \isactrlbold {\isasymor}\ G{\isacharparenright}{\isachardoublequoteclose}\isanewline
%
\isadelimproof
\ \ %
\endisadelimproof
%
\isatagproof
\isacommand{by}\isamarkupfalse%
\ {\isacharparenleft}simp\ add{\isacharcolon}\ semanticEq{\isacharunderscore}def{\isacharparenright}%
\endisatagproof
{\isafoldproof}%
%
\isadelimproof
%
\endisadelimproof
%
\begin{isamarkuptext}%
En contraposición, también podemos dar ejemplos de fórmulas que no son semánticamente 
  equivalentes.%
\end{isamarkuptext}\isamarkuptrue%
\isacommand{lemma}\isamarkupfalse%
\ {\isachardoublequoteopen}{\isasymnot}\ semanticEq\ {\isacharparenleft}Atom\ p{\isacharparenright}\ {\isacharparenleft}\isactrlbold {\isasymnot}{\isacharparenleft}Atom\ p{\isacharparenright}{\isacharparenright}{\isachardoublequoteclose}\isanewline
%
\isadelimproof
\ \ %
\endisadelimproof
%
\isatagproof
\isacommand{by}\isamarkupfalse%
\ {\isacharparenleft}simp\ add{\isacharcolon}\ semanticEq{\isacharunderscore}def{\isacharparenright}%
\endisatagproof
{\isafoldproof}%
%
\isadelimproof
\isanewline
%
\endisadelimproof
\isanewline
\isacommand{lemma}\isamarkupfalse%
\ {\isachardoublequoteopen}{\isasymnot}\ semanticEq\ {\isasymbottom}\ {\isasymtop}{\isachardoublequoteclose}\isanewline
%
\isadelimproof
\ \ %
\endisadelimproof
%
\isatagproof
\isacommand{by}\isamarkupfalse%
\ {\isacharparenleft}simp\ add{\isacharcolon}\ semanticEq{\isacharunderscore}def\ top{\isacharunderscore}semantics{\isacharparenright}%
\endisatagproof
{\isafoldproof}%
%
\isadelimproof
%
\endisadelimproof
%
\begin{isamarkuptext}%
Por tanto, diremos intuitivamente que una fórmula es de tipo \isa{{\isasymalpha}} con componentes \isa{{\isasymalpha}\isactrlsub {\isadigit{1}}} y \isa{{\isasymalpha}\isactrlsub {\isadigit{2}}}
  si es semánticamente equivalente a la fórmula \isa{{\isasymalpha}\isactrlsub {\isadigit{1}}\ {\isasymand}\ {\isasymalpha}\isactrlsub {\isadigit{2}}}. Del mismo modo, una fórmula será de tipo
  \isa{{\isasymbeta}} con componentes \isa{{\isasymbeta}\isactrlsub {\isadigit{1}}} y \isa{{\isasymbeta}\isactrlsub {\isadigit{2}}} si es semánticamente equivalente a la fórmula \isa{{\isasymbeta}\isactrlsub {\isadigit{1}}\ {\isasymor}\ {\isasymbeta}\isactrlsub {\isadigit{2}}}.

  \begin{definicion}
    Las fórmulas de tipo \isa{{\isasymalpha}} (\isa{fórmulas\ conjuntivas}) y sus correspondientes componentes
    \isa{{\isasymalpha}\isactrlsub {\isadigit{1}}} y \isa{{\isasymalpha}\isactrlsub {\isadigit{2}}} se definen como sigue: dadas \isa{F} y \isa{G} fórmulas cualesquiera,
    \begin{enumerate}
      \item \isa{F\ {\isasymand}\ G} es una fórmula de tipo \isa{{\isasymalpha}} cuyas componentes son \isa{F} y \isa{G}.
      \item \isa{{\isasymnot}{\isacharparenleft}F\ {\isasymor}\ G{\isacharparenright}} es una fórmula de tipo \isa{{\isasymalpha}} cuyas componentes son \isa{{\isasymnot}\ F} y \isa{{\isasymnot}\ G}.
      \item \isa{{\isasymnot}{\isacharparenleft}F\ {\isasymlongrightarrow}\ G{\isacharparenright}} es una fórmula de tipo \isa{{\isasymalpha}} cuyas componentes son \isa{F} y \isa{{\isasymnot}\ G}.
    \end{enumerate} 
  \end{definicion}

  De este modo, de los ejemplos anteriores podemos deducir que las fórmulas atómicas son de tipo \isa{{\isasymalpha}}
  y sus componentes \isa{{\isasymalpha}\isactrlsub {\isadigit{1}}} y \isa{{\isasymalpha}\isactrlsub {\isadigit{2}}} son la propia fórmula. Del mismo modo, la constante \isa{{\isasymbottom}} también es 
  una fórmula conjuntiva cuyas componentes son ella misma. Por último, podemos observar que dada
  una fórmula cualquiera \isa{F}, su doble negación \isa{{\isasymnot}{\isacharparenleft}{\isasymnot}\ F{\isacharparenright}} es una fórmula de tipo \isa{{\isasymalpha}} y componentes
  \isa{F} y \isa{F}.

  Formalizaremos en Isabelle el conjunto de fórmulas \isa{{\isasymalpha}} como un predicato inductivo. De este modo,
  las reglas anteriores que construyen el conjunto de fórmulas de tipo \isa{{\isasymalpha}} se formalizan en Isabelle 
  como reglas de introducción. Además, añadiremos explícitamente una cuarta regla que introduce la 
  doble negación de una fórmula como fórmula de tipo \isa{{\isasymalpha}}. De este modo, facilitaremos la prueba de 
  resultados posteriores relacionados con la definición de conjunto de Hintikka, que constituyen una
  base para la demostración del \isa{Teorema\ de\ Existencia\ de\ Modelo}.%
\end{isamarkuptext}\isamarkuptrue%
\isacommand{inductive}\isamarkupfalse%
\ Con\ {\isacharcolon}{\isacharcolon}\ {\isachardoublequoteopen}{\isacharprime}a\ formula\ {\isacharequal}{\isachargreater}\ {\isacharprime}a\ formula\ {\isacharequal}{\isachargreater}\ {\isacharprime}a\ formula\ {\isacharequal}{\isachargreater}\ bool{\isachardoublequoteclose}\ \isakeyword{where}\isanewline
{\isachardoublequoteopen}Con\ {\isacharparenleft}And\ F\ G{\isacharparenright}\ F\ G{\isachardoublequoteclose}\ {\isacharbar}\isanewline
{\isachardoublequoteopen}Con\ {\isacharparenleft}Not\ {\isacharparenleft}Or\ F\ G{\isacharparenright}{\isacharparenright}\ {\isacharparenleft}Not\ F{\isacharparenright}\ {\isacharparenleft}Not\ G{\isacharparenright}{\isachardoublequoteclose}\ {\isacharbar}\isanewline
{\isachardoublequoteopen}Con\ {\isacharparenleft}Not\ {\isacharparenleft}Imp\ F\ G{\isacharparenright}{\isacharparenright}\ F\ {\isacharparenleft}Not\ G{\isacharparenright}{\isachardoublequoteclose}\ {\isacharbar}\isanewline
{\isachardoublequoteopen}Con\ {\isacharparenleft}Not\ {\isacharparenleft}Not\ F{\isacharparenright}{\isacharparenright}\ F\ F{\isachardoublequoteclose}%
\begin{isamarkuptext}%
Las reglas de introducción que proporciona la definición anterior son
  las siguientes.

  \begin{itemize}
    \item[] \isa{Con\ {\isacharparenleft}F\ \isactrlbold {\isasymand}\ G{\isacharparenright}\ F\ G\isasep\isanewline%
Con\ {\isacharparenleft}\isactrlbold {\isasymnot}\ {\isacharparenleft}F\ \isactrlbold {\isasymor}\ G{\isacharparenright}{\isacharparenright}\ {\isacharparenleft}\isactrlbold {\isasymnot}\ F{\isacharparenright}\ {\isacharparenleft}\isactrlbold {\isasymnot}\ G{\isacharparenright}\isasep\isanewline%
Con\ {\isacharparenleft}\isactrlbold {\isasymnot}\ {\isacharparenleft}F\ \isactrlbold {\isasymrightarrow}\ G{\isacharparenright}{\isacharparenright}\ F\ {\isacharparenleft}\isactrlbold {\isasymnot}\ G{\isacharparenright}\isasep\isanewline%
Con\ {\isacharparenleft}\isactrlbold {\isasymnot}\ {\isacharparenleft}\isactrlbold {\isasymnot}\ F{\isacharparenright}{\isacharparenright}\ F\ F} 
      \hfill (\isa{Con{\isachardot}intros})
  \end{itemize}
  
  Por otro lado, definamos las fórmulas disyuntivas.

  \begin{definicion}
    Las fórmulas de tipo \isa{{\isasymbeta}} (\isa{fórmulas\ disyuntivas}) y sus correspondientes componentes
    \isa{{\isasymbeta}\isactrlsub {\isadigit{1}}} y \isa{{\isasymbeta}\isactrlsub {\isadigit{2}}} se definen como sigue: dadas \isa{F} y \isa{G} fórmulas cualesquiera,
    \begin{enumerate}
      \item \isa{F\ {\isasymor}\ G} es una fórmula de tipo \isa{{\isasymbeta}} cuyas componentes son \isa{F} y \isa{G}.
      \item \isa{F\ {\isasymlongrightarrow}\ G} es una fórmula de tipo \isa{{\isasymbeta}} cuyas componentes son \isa{{\isasymnot}\ F} y \isa{G}.
      \item \isa{{\isasymnot}{\isacharparenleft}F\ {\isasymand}\ G{\isacharparenright}} es una fórmula de tipo \isa{{\isasymbeta}} cuyas componentes son \isa{{\isasymnot}\ F} y \isa{{\isasymnot}\ G}.
    \end{enumerate} 
  \end{definicion}

  De los ejemplos dados anteriormente, podemos deducir análogamente que las fórmulas atómicas, la
  constante \isa{{\isasymbottom}} y la doble negación sob también fórmulas disyuntivas con las mismas componentes que
  las dadas para el tipo conjuntivo.

  Del mismo modo, su formalización se realiza como un predicado inductivo, de manera que las reglas 
  que definen el conjunto de fórmulas de tipo \isa{{\isasymbeta}} se formalizan en Isabelle como reglas de 
  introducción. Análogamente, introduciremos de manera explícita una regla que señala que la doble 
  negación de una fórmula es una fórmula de tipo disyuntivo.%
\end{isamarkuptext}\isamarkuptrue%
\isacommand{inductive}\isamarkupfalse%
\ Dis\ {\isacharcolon}{\isacharcolon}\ {\isachardoublequoteopen}{\isacharprime}a\ formula\ {\isacharequal}{\isachargreater}\ {\isacharprime}a\ formula\ {\isacharequal}{\isachargreater}\ {\isacharprime}a\ formula\ {\isacharequal}{\isachargreater}\ bool{\isachardoublequoteclose}\ \isakeyword{where}\isanewline
{\isachardoublequoteopen}Dis\ {\isacharparenleft}Or\ F\ G{\isacharparenright}\ F\ G{\isachardoublequoteclose}\ {\isacharbar}\isanewline
{\isachardoublequoteopen}Dis\ {\isacharparenleft}Imp\ F\ G{\isacharparenright}\ {\isacharparenleft}Not\ F{\isacharparenright}\ G{\isachardoublequoteclose}\ {\isacharbar}\isanewline
{\isachardoublequoteopen}Dis\ {\isacharparenleft}Not\ {\isacharparenleft}And\ F\ G{\isacharparenright}{\isacharparenright}\ {\isacharparenleft}Not\ F{\isacharparenright}\ {\isacharparenleft}Not\ G{\isacharparenright}{\isachardoublequoteclose}\ {\isacharbar}\isanewline
{\isachardoublequoteopen}Dis\ {\isacharparenleft}Not\ {\isacharparenleft}Not\ F{\isacharparenright}{\isacharparenright}\ F\ F{\isachardoublequoteclose}%
\begin{isamarkuptext}%
Las reglas de introducción que proporciona esta formalización se muestran a continuación.

  \begin{itemize}
    \item[] \isa{Dis\ {\isacharparenleft}F\ \isactrlbold {\isasymor}\ G{\isacharparenright}\ F\ G\isasep\isanewline%
Dis\ {\isacharparenleft}F\ \isactrlbold {\isasymrightarrow}\ G{\isacharparenright}\ {\isacharparenleft}\isactrlbold {\isasymnot}\ F{\isacharparenright}\ G\isasep\isanewline%
Dis\ {\isacharparenleft}\isactrlbold {\isasymnot}\ {\isacharparenleft}F\ \isactrlbold {\isasymand}\ G{\isacharparenright}{\isacharparenright}\ {\isacharparenleft}\isactrlbold {\isasymnot}\ F{\isacharparenright}\ {\isacharparenleft}\isactrlbold {\isasymnot}\ G{\isacharparenright}\isasep\isanewline%
Dis\ {\isacharparenleft}\isactrlbold {\isasymnot}\ {\isacharparenleft}\isactrlbold {\isasymnot}\ F{\isacharparenright}{\isacharparenright}\ F\ F} 
      \hfill (\isa{Dis{\isachardot}intros})
  \end{itemize}

  Cabe observar que las formalizaciones de la definiciones de fórmulas de tipo \isa{{\isasymalpha}} y \isa{{\isasymbeta}} son 
  definiciones sintácticas, pues construyen los correspondientes conjuntos de fórmulas a partir de 
  una reglas sintácticas concretas. Se trata de una simplificación de la intuición original de la 
  clasificación de las fórmulas mediante notación uniforme, ya que se prescinde de la noción de 
  equivalencia semántica que permite clasificar la totalidad de las fórmulas proposicionales. 

  Veamos la clasificación de casos concretos de fórmulas. Por ejemplo, según hemos definido la 
  fórmula \isa{{\isasymtop}}, es sencillo comprobar que se trata de una fórmula disyuntiva.%
\end{isamarkuptext}\isamarkuptrue%
\isacommand{lemma}\isamarkupfalse%
\ {\isachardoublequoteopen}Dis\ {\isasymtop}\ {\isacharparenleft}\isactrlbold {\isasymnot}\ {\isasymbottom}{\isacharparenright}\ {\isasymbottom}{\isachardoublequoteclose}\ \isanewline
%
\isadelimproof
\ \ %
\endisadelimproof
%
\isatagproof
\isacommand{unfolding}\isamarkupfalse%
\ Top{\isacharunderscore}def\ \isacommand{by}\isamarkupfalse%
\ {\isacharparenleft}simp\ only{\isacharcolon}\ Dis{\isachardot}intros{\isacharparenleft}{\isadigit{2}}{\isacharparenright}{\isacharparenright}%
\endisatagproof
{\isafoldproof}%
%
\isadelimproof
%
\endisadelimproof
%
\begin{isamarkuptext}%
Por otro lado, se observa a partir de las correspondientes definiciones que la conjunción
  generalizada de una lista de fórmulas es una fórmula de tipo \isa{{\isasymalpha}} y la disyunción generalizada de
  una lista de fórmulas es una fórmula de tipo \isa{{\isasymbeta}}.%
\end{isamarkuptext}\isamarkuptrue%
\isacommand{lemma}\isamarkupfalse%
\ {\isachardoublequoteopen}Con\ {\isacharparenleft}\isactrlbold {\isasymAnd}{\isacharparenleft}F{\isacharhash}Fs{\isacharparenright}{\isacharparenright}\ F\ {\isacharparenleft}\isactrlbold {\isasymAnd}Fs{\isacharparenright}{\isachardoublequoteclose}\isanewline
%
\isadelimproof
\ \ %
\endisadelimproof
%
\isatagproof
\isacommand{by}\isamarkupfalse%
\ {\isacharparenleft}simp\ only{\isacharcolon}\ BigAnd{\isachardot}simps\ Con{\isachardot}intros{\isacharparenleft}{\isadigit{1}}{\isacharparenright}{\isacharparenright}%
\endisatagproof
{\isafoldproof}%
%
\isadelimproof
\isanewline
%
\endisadelimproof
\isanewline
\isacommand{lemma}\isamarkupfalse%
\ {\isachardoublequoteopen}Dis\ {\isacharparenleft}\isactrlbold {\isasymOr}{\isacharparenleft}F{\isacharhash}Fs{\isacharparenright}{\isacharparenright}\ F\ {\isacharparenleft}\isactrlbold {\isasymOr}Fs{\isacharparenright}{\isachardoublequoteclose}\isanewline
%
\isadelimproof
\ \ %
\endisadelimproof
%
\isatagproof
\isacommand{by}\isamarkupfalse%
\ {\isacharparenleft}simp\ only{\isacharcolon}\ BigOr{\isachardot}simps\ Dis{\isachardot}intros{\isacharparenleft}{\isadigit{1}}{\isacharparenright}{\isacharparenright}%
\endisatagproof
{\isafoldproof}%
%
\isadelimproof
%
\endisadelimproof
%
\begin{isamarkuptext}%
Finalmente, de las reglas que definen las fórmulas conjuntivas y disyuntivas se deduce que
  la doble negación de una fórmula es una fórmula perteneciente a ambos tipos.%
\end{isamarkuptext}\isamarkuptrue%
\isacommand{lemma}\isamarkupfalse%
\ notDisCon{\isacharcolon}\ {\isachardoublequoteopen}Con\ {\isacharparenleft}Not\ {\isacharparenleft}Not\ F{\isacharparenright}{\isacharparenright}\ F\ F{\isachardoublequoteclose}\ {\isachardoublequoteopen}Dis\ {\isacharparenleft}Not\ {\isacharparenleft}Not\ F{\isacharparenright}{\isacharparenright}\ F\ F{\isachardoublequoteclose}\ \isanewline
%
\isadelimproof
\ \ %
\endisadelimproof
%
\isatagproof
\isacommand{by}\isamarkupfalse%
\ {\isacharparenleft}simp\ only{\isacharcolon}\ Con{\isachardot}intros{\isacharparenleft}{\isadigit{4}}{\isacharparenright}\ Dis{\isachardot}intros{\isacharparenleft}{\isadigit{4}}{\isacharparenright}{\isacharparenright}{\isacharplus}%
\endisatagproof
{\isafoldproof}%
%
\isadelimproof
%
\endisadelimproof
%
\begin{isamarkuptext}%
A continuación vamos a introducir el siguiente lema que caracteriza las fórmulas de tipo \isa{{\isasymalpha}} 
  y \isa{{\isasymbeta}}, facilitando el uso de la notación uniforme en Isabelle.%
\end{isamarkuptext}\isamarkuptrue%
\isacommand{lemma}\isamarkupfalse%
\ con{\isacharunderscore}dis{\isacharunderscore}simps{\isacharcolon}\isanewline
\ \ {\isachardoublequoteopen}Con\ a{\isadigit{1}}\ a{\isadigit{2}}\ a{\isadigit{3}}\ {\isacharequal}\ {\isacharparenleft}a{\isadigit{1}}\ {\isacharequal}\ a{\isadigit{2}}\ \isactrlbold {\isasymand}\ a{\isadigit{3}}\ {\isasymor}\ \isanewline
\ \ \ \ {\isacharparenleft}{\isasymexists}F\ G{\isachardot}\ a{\isadigit{1}}\ {\isacharequal}\ \isactrlbold {\isasymnot}\ {\isacharparenleft}F\ \isactrlbold {\isasymor}\ G{\isacharparenright}\ {\isasymand}\ a{\isadigit{2}}\ {\isacharequal}\ \isactrlbold {\isasymnot}\ F\ {\isasymand}\ a{\isadigit{3}}\ {\isacharequal}\ \isactrlbold {\isasymnot}\ G{\isacharparenright}\ {\isasymor}\ \isanewline
\ \ \ \ {\isacharparenleft}{\isasymexists}G{\isachardot}\ a{\isadigit{1}}\ {\isacharequal}\ \isactrlbold {\isasymnot}\ {\isacharparenleft}a{\isadigit{2}}\ \isactrlbold {\isasymrightarrow}\ G{\isacharparenright}\ {\isasymand}\ a{\isadigit{3}}\ {\isacharequal}\ \isactrlbold {\isasymnot}\ G{\isacharparenright}\ {\isasymor}\ \isanewline
\ \ \ \ a{\isadigit{1}}\ {\isacharequal}\ \isactrlbold {\isasymnot}\ {\isacharparenleft}\isactrlbold {\isasymnot}\ a{\isadigit{2}}{\isacharparenright}\ {\isasymand}\ a{\isadigit{3}}\ {\isacharequal}\ a{\isadigit{2}}{\isacharparenright}{\isachardoublequoteclose}\isanewline
\ \ {\isachardoublequoteopen}Dis\ a{\isadigit{1}}\ a{\isadigit{2}}\ a{\isadigit{3}}\ {\isacharequal}\ {\isacharparenleft}a{\isadigit{1}}\ {\isacharequal}\ a{\isadigit{2}}\ \isactrlbold {\isasymor}\ a{\isadigit{3}}\ {\isasymor}\ \isanewline
\ \ \ \ {\isacharparenleft}{\isasymexists}F\ G{\isachardot}\ a{\isadigit{1}}\ {\isacharequal}\ F\ \isactrlbold {\isasymrightarrow}\ G\ {\isasymand}\ a{\isadigit{2}}\ {\isacharequal}\ \isactrlbold {\isasymnot}\ F\ {\isasymand}\ a{\isadigit{3}}\ {\isacharequal}\ G{\isacharparenright}\ {\isasymor}\ \isanewline
\ \ \ \ {\isacharparenleft}{\isasymexists}F\ G{\isachardot}\ a{\isadigit{1}}\ {\isacharequal}\ \isactrlbold {\isasymnot}\ {\isacharparenleft}F\ \isactrlbold {\isasymand}\ G{\isacharparenright}\ {\isasymand}\ a{\isadigit{2}}\ {\isacharequal}\ \isactrlbold {\isasymnot}\ F\ {\isasymand}\ a{\isadigit{3}}\ {\isacharequal}\ \isactrlbold {\isasymnot}\ G{\isacharparenright}\ {\isasymor}\ \isanewline
\ \ \ \ a{\isadigit{1}}\ {\isacharequal}\ \isactrlbold {\isasymnot}\ {\isacharparenleft}\isactrlbold {\isasymnot}\ a{\isadigit{2}}{\isacharparenright}\ {\isasymand}\ a{\isadigit{3}}\ {\isacharequal}\ a{\isadigit{2}}{\isacharparenright}{\isachardoublequoteclose}\ \isanewline
%
\isadelimproof
\ \ %
\endisadelimproof
%
\isatagproof
\isacommand{by}\isamarkupfalse%
\ {\isacharparenleft}simp{\isacharunderscore}all\ add{\isacharcolon}\ Con{\isachardot}simps\ Dis{\isachardot}simps{\isacharparenright}%
\endisatagproof
{\isafoldproof}%
%
\isadelimproof
%
\endisadelimproof
%
\begin{isamarkuptext}%
Por último, introduzcamos un resultado que permite caracterizar los conjuntos de Hintikka 
  empleando la notación uniforme.

  \begin{lema}[Caracterización de los conjuntos de Hintikka mediante la notación uniforme]
    Dado un conjunto de fórmulas proposicionales \isa{S}, son equivalentes:
    \begin{enumerate}
      \item \isa{S} es un conjunto de Hintikka.
      \item Se verifican las condiciones siguientes:
      \begin{itemize}
        \item \isa{{\isasymbottom}} no pertenece a \isa{S}.
        \item Dada \isa{p} una fórmula atómica cualquiera, no se tiene 
        simultáneamente que\\ \isa{p\ {\isasymin}\ S} y \isa{{\isasymnot}\ p\ {\isasymin}\ S}.
        \item Para toda fórmula de tipo \isa{{\isasymalpha}} con componentes \isa{{\isasymalpha}\isactrlsub {\isadigit{1}}} y \isa{{\isasymalpha}\isactrlsub {\isadigit{2}}} se verifica 
        que si la fórmula pertenece a \isa{S}, entonces \isa{{\isasymalpha}\isactrlsub {\isadigit{1}}} y \isa{{\isasymalpha}\isactrlsub {\isadigit{2}}} también.
        \item Para toda fórmula de tipo \isa{{\isasymbeta}} con componentes \isa{{\isasymbeta}\isactrlsub {\isadigit{1}}} y \isa{{\isasymbeta}\isactrlsub {\isadigit{2}}} se verifica 
        que si la fórmula pertenece a \isa{S}, entonces o bien \isa{{\isasymbeta}\isactrlsub {\isadigit{1}}} pertenece
        a \isa{S} o bien \isa{{\isasymbeta}\isactrlsub {\isadigit{2}}} pertenece a \isa{S}.
      \end{itemize} 
    \end{enumerate}
  \end{lema}

  En Isabelle/HOL se formaliza del siguiente modo.%
\end{isamarkuptext}\isamarkuptrue%
\isacommand{lemma}\isamarkupfalse%
\ {\isachardoublequoteopen}Hintikka\ S\ {\isacharequal}\ {\isacharparenleft}{\isasymbottom}\ {\isasymnotin}\ S\isanewline
{\isasymand}\ {\isacharparenleft}{\isasymforall}k{\isachardot}\ Atom\ k\ {\isasymin}\ S\ {\isasymlongrightarrow}\ \isactrlbold {\isasymnot}\ {\isacharparenleft}Atom\ k{\isacharparenright}\ {\isasymin}\ S\ {\isasymlongrightarrow}\ False{\isacharparenright}\isanewline
{\isasymand}\ {\isacharparenleft}{\isasymforall}F\ G\ H{\isachardot}\ Con\ F\ G\ H\ {\isasymlongrightarrow}\ F\ {\isasymin}\ S\ {\isasymlongrightarrow}\ G\ {\isasymin}\ S\ {\isasymand}\ H\ {\isasymin}\ S{\isacharparenright}\isanewline
{\isasymand}\ {\isacharparenleft}{\isasymforall}F\ G\ H{\isachardot}\ Dis\ F\ G\ H\ {\isasymlongrightarrow}\ F\ {\isasymin}\ S\ {\isasymlongrightarrow}\ G\ {\isasymin}\ S\ {\isasymor}\ H\ {\isasymin}\ S{\isacharparenright}{\isacharparenright}{\isachardoublequoteclose}\ \isanewline
%
\isadelimproof
\ \ %
\endisadelimproof
%
\isatagproof
\isacommand{oops}\isamarkupfalse%
%
\endisatagproof
{\isafoldproof}%
%
\isadelimproof
%
\endisadelimproof
%
\begin{isamarkuptext}%
Procedamos a la demostración del resultado.

\begin{demostracion}
  Para probar la equivalencia, veamos cada una de las implicaciones por separado.

\textbf{\isa{{\isadigit{1}}{\isacharparenright}\ {\isasymLongrightarrow}\ {\isadigit{2}}{\isacharparenright}}}

  Supongamos que \isa{S} es un conjunto de Hintikka. Vamos a probar que, en efecto, se 
  verifican las condiciones del enunciado del lema.

  Por definición de conjunto de Hintikka, \isa{S} verifica las siguientes condiciones:
  \begin{enumerate}
    \item \isa{{\isasymbottom}\ {\isasymnotin}\ S}.
    \item Dada \isa{p} una fórmula atómica cualquiera, no se tiene 
      simultáneamente que\\ \isa{p\ {\isasymin}\ S} y \isa{{\isasymnot}\ p\ {\isasymin}\ S}.
    \item Si \isa{G\ {\isasymand}\ H\ {\isasymin}\ S}, entonces \isa{G\ {\isasymin}\ S} y \isa{H\ {\isasymin}\ S}.
    \item Si \isa{G\ {\isasymor}\ H\ {\isasymin}\ S}, entonces \isa{G\ {\isasymin}\ S} o \isa{H\ {\isasymin}\ S}.
    \item Si \isa{G\ {\isasymrightarrow}\ H\ {\isasymin}\ S}, entonces \isa{{\isasymnot}\ G\ {\isasymin}\ S} o \isa{H\ {\isasymin}\ S}.
    \item Si \isa{{\isasymnot}{\isacharparenleft}{\isasymnot}\ G{\isacharparenright}\ {\isasymin}\ S}, entonces \isa{G\ {\isasymin}\ S}.
    \item Si \isa{{\isasymnot}{\isacharparenleft}G\ {\isasymand}\ H{\isacharparenright}\ {\isasymin}\ S}, entonces \isa{{\isasymnot}\ G\ {\isasymin}\ S} o \isa{{\isasymnot}\ H\ {\isasymin}\ S}.
    \item Si \isa{{\isasymnot}{\isacharparenleft}G\ {\isasymor}\ H{\isacharparenright}\ {\isasymin}\ S}, entonces \isa{{\isasymnot}\ G\ {\isasymin}\ S} y \isa{{\isasymnot}\ H\ {\isasymin}\ S}. 
    \item Si \isa{{\isasymnot}{\isacharparenleft}G\ {\isasymrightarrow}\ H{\isacharparenright}\ {\isasymin}\ S}, entonces \isa{G\ {\isasymin}\ S} y \isa{{\isasymnot}\ H\ {\isasymin}\ S}. 
  \end{enumerate}  
  De este modo, el conjunto \isa{S} cumple la primera y la segunda condición del
  enunciado del lema, que se corresponden con las dos primeras condiciones
  de la definición de conjunto de Hintikka. Veamos que, además, verifica las
  dos últimas condiciones del resultado.

  En primer lugar, probemos que para toda fórmula de tipo \isa{{\isasymalpha}} con 
  componentes \isa{{\isasymalpha}\isactrlsub {\isadigit{1}}} y \isa{{\isasymalpha}\isactrlsub {\isadigit{2}}} se verifica que si la fórmula pertenece al conjunto 
  \isa{S}, entonces \isa{{\isasymalpha}\isactrlsub {\isadigit{1}}} y \isa{{\isasymalpha}\isactrlsub {\isadigit{2}}} también. Para ello, supongamos que una fórmula 
  cualquiera de tipo \isa{{\isasymalpha}} pertence a \isa{S}. Por definición de este tipo de
  fórmulas, tenemos que \isa{{\isasymalpha}} puede ser de la forma \isa{G\ {\isasymand}\ H}, \isa{{\isasymnot}{\isacharparenleft}{\isasymnot}\ G{\isacharparenright}},\\ \isa{{\isasymnot}{\isacharparenleft}G\ {\isasymor}\ H{\isacharparenright}} 
  o \isa{{\isasymnot}{\isacharparenleft}G\ {\isasymlongrightarrow}\ H{\isacharparenright}} para fórmulas \isa{G} y \isa{H} cualesquiera. Probemos que, para cada
  tipo de fórmula \isa{{\isasymalpha}} perteneciente a \isa{S}, sus componentes \isa{{\isasymalpha}\isactrlsub {\isadigit{1}}} y \isa{{\isasymalpha}\isactrlsub {\isadigit{2}}} están en
  \isa{S}.

  \isa{{\isasymsqdot}\ Fórmula\ del\ tipo\ G\ {\isasymand}\ H}: Sus componentes conjuntivas son \isa{G} y \isa{H}. 
  Por la tercera condición de la definición de conjunto de Hintikka, obtenemos 
  que si \isa{G\ {\isasymand}\ H} pertenece a \isa{S}, entonces \isa{G} y \isa{H} están ambas en el conjunto,
  lo que prueba este caso.
    
  \isa{{\isasymsqdot}\ Fórmula\ del\ tipo\ {\isasymnot}{\isacharparenleft}{\isasymnot}\ G{\isacharparenright}}: Sus componentes conjuntivas son ambas \isa{G}.
  Por la sexta condición de la definición de conjunto de Hintikka, obtenemos que
  si \isa{{\isasymnot}{\isacharparenleft}{\isasymnot}\ G{\isacharparenright}} pertenece a \isa{S}, entonces \isa{G} pertenece al conjunto, lo que prueba
  este caso.

  \isa{{\isasymsqdot}\ Fórmula\ del\ tipo\ {\isasymnot}{\isacharparenleft}G\ {\isasymor}\ H{\isacharparenright}}: Sus componentes conjuntivas son \isa{{\isasymnot}\ G} y \isa{{\isasymnot}\ H}. 
  Por la octava condición de la definición de conjunto de Hintikka, obtenemos 
  que si \isa{{\isasymnot}{\isacharparenleft}G\ {\isasymor}\ H{\isacharparenright}} pertenece a \isa{S}, entonces \isa{{\isasymnot}\ G} y \isa{{\isasymnot}\ H} están ambas en el conjunto,
  lo que prueba este caso.

  \isa{{\isasymsqdot}\ Fórmula\ del\ tipo\ {\isasymnot}{\isacharparenleft}G\ {\isasymlongrightarrow}\ H{\isacharparenright}}: Sus componentes conjuntivas son \isa{G} y \isa{{\isasymnot}\ H}. 
  Por la novena condición de la definición de conjunto de Hintikka, obtenemos 
  que si\\ \isa{{\isasymnot}{\isacharparenleft}G\ {\isasymlongrightarrow}\ H{\isacharparenright}} pertenece a \isa{S}, entonces \isa{G} y \isa{{\isasymnot}\ H} están ambas en el conjunto,
  lo que prueba este caso.

  Finalmente, probemos que para toda fórmula de tipo \isa{{\isasymbeta}} con componentes \isa{{\isasymbeta}\isactrlsub {\isadigit{1}}} y 
  \isa{{\isasymbeta}\isactrlsub {\isadigit{2}}} se verifica que si la fórmula pertenece al conjunto \isa{S}, entonces o bien \isa{{\isasymbeta}\isactrlsub {\isadigit{1}}} 
  pertenece al conjunto o bien \isa{{\isasymbeta}\isactrlsub {\isadigit{2}}} pertenece a conjunto. Para ello, supongamos que 
  una fórmula cualquiera de tipo \isa{{\isasymbeta}} pertence a \isa{S}. Por definición de este tipo de
  fórmulas, tenemos que \isa{{\isasymbeta}} puede ser de la forma \isa{G\ {\isasymor}\ H}, \isa{G\ {\isasymlongrightarrow}\ H}, \isa{{\isasymnot}{\isacharparenleft}{\isasymnot}\ G{\isacharparenright}} 
  o \isa{{\isasymnot}{\isacharparenleft}G\ {\isasymand}\ H{\isacharparenright}} para fórmulas \isa{G} y \isa{H} cualesquiera. Probemos que, para cada
  tipo de fórmula \isa{{\isasymbeta}} perteneciente a \isa{S}, o bien su componente \isa{{\isasymbeta}\isactrlsub {\isadigit{1}}} pertenece a \isa{S} 
  o bien su componente \isa{{\isasymbeta}\isactrlsub {\isadigit{2}}} pertenece a \isa{S}.

  \isa{{\isasymsqdot}\ Fórmula\ del\ tipo\ G\ {\isasymor}\ H}: Sus componentes disyuntivas son \isa{G} y \isa{H}. 
    Por la cuarta condición de la definición de conjunto de Hintikka, obtenemos 
    que si \isa{G\ {\isasymor}\ H} pertenece a \isa{S}, entonces o bien \isa{G} está en \isa{S} o bien \isa{H} está
    en \isa{S}, lo que prueba este caso.

  \isa{{\isasymsqdot}\ Fórmula\ del\ tipo\ G\ {\isasymlongrightarrow}\ H}: Sus componentes disyuntivas son \isa{{\isasymnot}\ G} y \isa{H}.
    Por la quinta condición de la definición de conjunto de Hintikka, obtenemos que
    si \isa{G\ {\isasymlongrightarrow}\ H} pertenece a \isa{S}, entonces o bien \isa{{\isasymnot}\ G} pertenece al conjunto o bien
    \isa{H} pertenece al conjunto, lo que prueba este caso.

  \isa{{\isasymsqdot}\ Fórmula\ del\ tipo\ {\isasymnot}{\isacharparenleft}{\isasymnot}\ G{\isacharparenright}}: Sus componentes conjuntivas son ambas \isa{G}.
    Por la sexta condición de la definición de conjunto de Hintikka, obtenemos 
    que si \isa{{\isasymnot}{\isacharparenleft}{\isasymnot}\ G{\isacharparenright}} pertenece a \isa{S}, entonces \isa{G} pertenece al conjunto. De este modo,
    por la regla de introducción a la disyunción, se prueba que o bien una de las 
    componentes está en el conjunto o bien lo está la otra pues, en este caso,
    coinciden.

  \isa{{\isasymsqdot}\ Fórmula\ del\ tipo\ {\isasymnot}{\isacharparenleft}G\ {\isasymand}\ H{\isacharparenright}}: Sus componentes conjuntivas son \isa{{\isasymnot}\ G} y \isa{{\isasymnot}\ H}. 
    Por la séptima condición de la definición de conjunto de Hintikka, obtenemos 
    que si\\ \isa{{\isasymnot}{\isacharparenleft}G\ {\isasymand}\ H{\isacharparenright}} pertenece a \isa{S}, entonces o bien \isa{{\isasymnot}\ G} pertenece al conjunto
    o bien \isa{{\isasymnot}\ H} pertenece al conjunto, lo que prueba este caso.

\textbf{\isa{{\isadigit{2}}{\isacharparenright}\ {\isasymLongrightarrow}\ {\isadigit{1}}{\isacharparenright}}}

  Supongamos que se verifican las condiciones del enunciado del lema:

  \begin{itemize}
    \item \isa{{\isasymbottom}} no pertenece a \isa{S}.
    \item Dada \isa{p} una fórmula atómica cualquiera, no se tiene 
    simultáneamente que\\ \isa{p\ {\isasymin}\ S} y \isa{{\isasymnot}\ p\ {\isasymin}\ S}.
    \item Para toda fórmula de tipo \isa{{\isasymalpha}} con componentes \isa{{\isasymalpha}\isactrlsub {\isadigit{1}}} y \isa{{\isasymalpha}\isactrlsub {\isadigit{2}}} se verifica 
    que si la fórmula pertenece a \isa{S}, entonces \isa{{\isasymalpha}\isactrlsub {\isadigit{1}}} y \isa{{\isasymalpha}\isactrlsub {\isadigit{2}}} también.
    \item Para toda fórmula de tipo \isa{{\isasymbeta}} con componentes \isa{{\isasymbeta}\isactrlsub {\isadigit{1}}} y \isa{{\isasymbeta}\isactrlsub {\isadigit{2}}} se verifica 
    que si la fórmula pertenece a \isa{S}, entonces o bien \isa{{\isasymbeta}\isactrlsub {\isadigit{1}}} pertenece
    a \isa{S} o bien \isa{{\isasymbeta}\isactrlsub {\isadigit{2}}} pertenece a \isa{S}.
  \end{itemize}  

  Vamos a probar que \isa{S} es un conjunto de Hintikka.

  Por la definición de conjunto de Hintikka, es suficiente probar las siguientes
  condiciones:

  \begin{enumerate}
    \item \isa{{\isasymbottom}\ {\isasymnotin}\ S}.
    \item Dada \isa{p} una fórmula atómica cualquiera, no se tiene 
      simultáneamente que\\ \isa{p\ {\isasymin}\ S} y \isa{{\isasymnot}\ p\ {\isasymin}\ S}.
    \item Si \isa{G\ {\isasymand}\ H\ {\isasymin}\ S}, entonces \isa{G\ {\isasymin}\ S} y \isa{H\ {\isasymin}\ S}.
    \item Si \isa{G\ {\isasymor}\ H\ {\isasymin}\ S}, entonces \isa{G\ {\isasymin}\ S} o \isa{H\ {\isasymin}\ S}.
    \item Si \isa{G\ {\isasymrightarrow}\ H\ {\isasymin}\ S}, entonces \isa{{\isasymnot}\ G\ {\isasymin}\ S} o \isa{H\ {\isasymin}\ S}.
    \item Si \isa{{\isasymnot}{\isacharparenleft}{\isasymnot}\ G{\isacharparenright}\ {\isasymin}\ S}, entonces \isa{G\ {\isasymin}\ S}.
    \item Si \isa{{\isasymnot}{\isacharparenleft}G\ {\isasymand}\ H{\isacharparenright}\ {\isasymin}\ S}, entonces \isa{{\isasymnot}\ G\ {\isasymin}\ S} o \isa{{\isasymnot}\ H\ {\isasymin}\ S}.
    \item Si \isa{{\isasymnot}{\isacharparenleft}G\ {\isasymor}\ H{\isacharparenright}\ {\isasymin}\ S}, entonces \isa{{\isasymnot}\ G\ {\isasymin}\ S} y \isa{{\isasymnot}\ H\ {\isasymin}\ S}. 
    \item Si \isa{{\isasymnot}{\isacharparenleft}G\ {\isasymrightarrow}\ H{\isacharparenright}\ {\isasymin}\ S}, entonces \isa{G\ {\isasymin}\ S} y \isa{{\isasymnot}\ H\ {\isasymin}\ S}. 
  \end{enumerate} 

  En primer lugar se observa que, por hipótesis, se verifican las dos primeras
  condiciones de la definición de conjunto de Hintikka. Veamos que, en efecto, se
  cumplen las demás.

  \begin{enumerate}
    \item[\isa{{\isadigit{3}}{\isacharparenright}}] Supongamos que \isa{G\ {\isasymand}\ H} está en \isa{S} para fórmulas \isa{G} y \isa{H} cualesquiera.
    Por definición, \isa{G\ {\isasymand}\ H} es una fórmula de tipo \isa{{\isasymalpha}} con componentes \isa{G} y \isa{H}. 
    Por lo tanto, por hipótesis se cumple que \isa{G} y \isa{H} están en \isa{S}.
    \item[\isa{{\isadigit{4}}{\isacharparenright}}] Supongamos que \isa{G\ {\isasymor}\ H} está en \isa{S} para fórmulas \isa{G} y \isa{H} cualesquiera.
    Por definición, \isa{G\ {\isasymor}\ H} es una fórmula de tipo \isa{{\isasymbeta}} con componentes \isa{G} y \isa{H}. 
    Por lo tanto, por hipótesis se cumple que o bien \isa{G} está en \isa{S} o bien \isa{H} está 
    en \isa{S}.
    \item[\isa{{\isadigit{5}}{\isacharparenright}}] Supongamos que \isa{G\ {\isasymlongrightarrow}\ H} está en \isa{S} para fórmulas \isa{G} y \isa{H} cualesquiera.
    Por definición, \isa{G\ {\isasymlongrightarrow}\ H} es una fórmula de tipo \isa{{\isasymbeta}} con componentes \isa{{\isasymnot}\ G} y \isa{H}. 
    Por lo tanto, por hipótesis se cumple que o bien \isa{{\isasymnot}\ G} está en \isa{S} o bien \isa{H} está 
    en \isa{S}.
    \item[\isa{{\isadigit{6}}{\isacharparenright}}] Supongamos que \isa{{\isasymnot}{\isacharparenleft}{\isasymnot}\ G{\isacharparenright}} está en \isa{S} para una fórmula \isa{G} cualquiera.
    Por definición, \isa{{\isasymnot}{\isacharparenleft}{\isasymnot}\ G{\isacharparenright}} es una fórmula de tipo \isa{{\isasymalpha}} cuyas componentes son ambas \isa{G}. 
    Por lo tanto, por hipótesis se cumple que \isa{G} está en \isa{S}.
    \item[\isa{{\isadigit{7}}{\isacharparenright}}] Supongamos que \isa{{\isasymnot}{\isacharparenleft}G\ {\isasymand}\ H{\isacharparenright}} está en \isa{S} para fórmulas \isa{G} y \isa{H} cualesquiera.
    Por definición, \isa{{\isasymnot}{\isacharparenleft}G\ {\isasymand}\ H{\isacharparenright}} es una fórmula de tipo \isa{{\isasymbeta}} con componentes \isa{{\isasymnot}\ G} y \isa{{\isasymnot}\ H}. 
    Por lo tanto, por hipótesis se cumple que o bien \isa{{\isasymnot}\ G} está en \isa{S} o bien \isa{{\isasymnot}\ H} está 
    en \isa{S}.
    \item[\isa{{\isadigit{8}}{\isacharparenright}}] Supongamos que \isa{{\isasymnot}{\isacharparenleft}G\ {\isasymor}\ H{\isacharparenright}} está en \isa{S} para fórmulas \isa{G} y \isa{H} cualesquiera.
    Por definición, \isa{{\isasymnot}{\isacharparenleft}G\ {\isasymor}\ H{\isacharparenright}} es una fórmula de tipo \isa{{\isasymalpha}} con componentes \isa{{\isasymnot}\ G} y \isa{{\isasymnot}\ H}. 
    Por lo tanto, por hipótesis se cumple que \isa{{\isasymnot}\ G} y \isa{{\isasymnot}\ H} están en \isa{S}.
    \item[\isa{{\isadigit{9}}{\isacharparenright}}] Supongamos que \isa{{\isasymnot}{\isacharparenleft}G\ {\isasymlongrightarrow}\ H{\isacharparenright}} está en \isa{S} para fórmulas \isa{G} y \isa{H} cualesquiera. 
    Por definición, \isa{{\isasymnot}{\isacharparenleft}G\ {\isasymlongrightarrow}\ H{\isacharparenright}} es una fórmula de tipo \isa{{\isasymalpha}} con componentes \isa{G} y \isa{{\isasymnot}\ H}.
    Por lo tanto, por hipótesis se cumple que \isa{G} y \isa{{\isasymnot}\ H} están en \isa{S}.
  \end{enumerate}

  Por tanto, queda probado el resultado.
\end{demostracion}

  Para probar de manera detallada el lema en Isabelle vamos a demostrar
  cada una de las implicaciones de la equivalencia por separado. 

  La primera implicación del lema se basa en dos lemas auxiliares. El primero de ellos 
  prueba que la tercera, sexta, octava y novena condición de la definición de conjunto de 
  Hintikka son suficientes para probar que para toda fórmula de tipo \isa{{\isasymalpha}} con componentes 
  \isa{{\isasymalpha}\isactrlsub {\isadigit{1}}} y \isa{{\isasymalpha}\isactrlsub {\isadigit{2}}} se verifica que si la fórmula pertenece al conjunto \isa{S}, entonces \isa{{\isasymalpha}\isactrlsub {\isadigit{1}}} y 
  \isa{{\isasymalpha}\isactrlsub {\isadigit{2}}} también. Su demostración detallada en Isabelle se muestra a continuación.%
\end{isamarkuptext}\isamarkuptrue%
\isacommand{lemma}\isamarkupfalse%
\ Hintikka{\isacharunderscore}alt{\isadigit{1}}Con{\isacharcolon}\isanewline
\ \ \isakeyword{assumes}\ {\isachardoublequoteopen}{\isacharparenleft}{\isasymforall}G\ H{\isachardot}\ G\ \isactrlbold {\isasymand}\ H\ {\isasymin}\ S\ {\isasymlongrightarrow}\ G\ {\isasymin}\ S\ {\isasymand}\ H\ {\isasymin}\ S{\isacharparenright}\isanewline
\ \ {\isasymand}\ {\isacharparenleft}{\isasymforall}G{\isachardot}\ \isactrlbold {\isasymnot}\ {\isacharparenleft}\isactrlbold {\isasymnot}\ G{\isacharparenright}\ {\isasymin}\ S\ {\isasymlongrightarrow}\ G\ {\isasymin}\ S{\isacharparenright}\isanewline
\ \ {\isasymand}\ {\isacharparenleft}{\isasymforall}G\ H{\isachardot}\ \isactrlbold {\isasymnot}{\isacharparenleft}G\ \isactrlbold {\isasymor}\ H{\isacharparenright}\ {\isasymin}\ S\ {\isasymlongrightarrow}\ \isactrlbold {\isasymnot}\ G\ {\isasymin}\ S\ {\isasymand}\ \isactrlbold {\isasymnot}\ H\ {\isasymin}\ S{\isacharparenright}\isanewline
\ \ {\isasymand}\ {\isacharparenleft}{\isasymforall}G\ H{\isachardot}\ \isactrlbold {\isasymnot}{\isacharparenleft}G\ \isactrlbold {\isasymrightarrow}\ H{\isacharparenright}\ {\isasymin}\ S\ {\isasymlongrightarrow}\ G\ {\isasymin}\ S\ {\isasymand}\ \isactrlbold {\isasymnot}\ H\ {\isasymin}\ S{\isacharparenright}{\isachardoublequoteclose}\isanewline
\ \ \isakeyword{shows}\ {\isachardoublequoteopen}Con\ F\ G\ H\ {\isasymlongrightarrow}\ F\ {\isasymin}\ S\ {\isasymlongrightarrow}\ G\ {\isasymin}\ S\ {\isasymand}\ H\ {\isasymin}\ S{\isachardoublequoteclose}\isanewline
%
\isadelimproof
%
\endisadelimproof
%
\isatagproof
\isacommand{proof}\isamarkupfalse%
\ {\isacharparenleft}rule\ impI{\isacharparenright}\isanewline
\ \ \isacommand{assume}\isamarkupfalse%
\ {\isachardoublequoteopen}Con\ F\ G\ H{\isachardoublequoteclose}\isanewline
\ \ \isacommand{then}\isamarkupfalse%
\ \isacommand{have}\isamarkupfalse%
\ {\isachardoublequoteopen}F\ {\isacharequal}\ G\ \isactrlbold {\isasymand}\ H\ {\isasymor}\ \isanewline
\ \ \ \ {\isacharparenleft}{\isacharparenleft}{\isasymexists}G{\isadigit{1}}\ H{\isadigit{1}}{\isachardot}\ F\ {\isacharequal}\ \isactrlbold {\isasymnot}\ {\isacharparenleft}G{\isadigit{1}}\ \isactrlbold {\isasymor}\ H{\isadigit{1}}{\isacharparenright}\ {\isasymand}\ G\ {\isacharequal}\ \isactrlbold {\isasymnot}\ G{\isadigit{1}}\ {\isasymand}\ H\ {\isacharequal}\ \isactrlbold {\isasymnot}\ H{\isadigit{1}}{\isacharparenright}\ {\isasymor}\ \isanewline
\ \ \ \ {\isacharparenleft}{\isasymexists}H{\isadigit{2}}{\isachardot}\ F\ {\isacharequal}\ \isactrlbold {\isasymnot}\ {\isacharparenleft}G\ \isactrlbold {\isasymrightarrow}\ H{\isadigit{2}}{\isacharparenright}\ {\isasymand}\ H\ {\isacharequal}\ \isactrlbold {\isasymnot}\ H{\isadigit{2}}{\isacharparenright}\ {\isasymor}\ \isanewline
\ \ \ \ F\ {\isacharequal}\ \isactrlbold {\isasymnot}\ {\isacharparenleft}\isactrlbold {\isasymnot}\ G{\isacharparenright}\ {\isasymand}\ H\ {\isacharequal}\ G{\isacharparenright}{\isachardoublequoteclose}\isanewline
\ \ \ \ \isacommand{by}\isamarkupfalse%
\ {\isacharparenleft}simp\ only{\isacharcolon}\ con{\isacharunderscore}dis{\isacharunderscore}simps{\isacharparenleft}{\isadigit{1}}{\isacharparenright}{\isacharparenright}\isanewline
\ \ \isacommand{thus}\isamarkupfalse%
\ {\isachardoublequoteopen}F\ {\isasymin}\ S\ {\isasymlongrightarrow}\ G\ {\isasymin}\ S\ {\isasymand}\ H\ {\isasymin}\ S{\isachardoublequoteclose}\isanewline
\ \ \isacommand{proof}\isamarkupfalse%
\ {\isacharparenleft}rule\ disjE{\isacharparenright}\isanewline
\ \ \ \ \isacommand{assume}\isamarkupfalse%
\ {\isachardoublequoteopen}F\ {\isacharequal}\ G\ \isactrlbold {\isasymand}\ H{\isachardoublequoteclose}\isanewline
\ \ \ \ \isacommand{have}\isamarkupfalse%
\ {\isachardoublequoteopen}{\isasymforall}G\ H{\isachardot}\ G\ \isactrlbold {\isasymand}\ H\ {\isasymin}\ S\ {\isasymlongrightarrow}\ G\ {\isasymin}\ S\ {\isasymand}\ H\ {\isasymin}\ S{\isachardoublequoteclose}\isanewline
\ \ \ \ \ \ \isacommand{using}\isamarkupfalse%
\ assms\ \isacommand{by}\isamarkupfalse%
\ {\isacharparenleft}rule\ conjunct{\isadigit{1}}{\isacharparenright}\isanewline
\ \ \ \ \isacommand{thus}\isamarkupfalse%
\ {\isachardoublequoteopen}F\ {\isasymin}\ S\ {\isasymlongrightarrow}\ G\ {\isasymin}\ S\ {\isasymand}\ H\ {\isasymin}\ S{\isachardoublequoteclose}\isanewline
\ \ \ \ \ \ \isacommand{using}\isamarkupfalse%
\ {\isacartoucheopen}F\ {\isacharequal}\ G\ \isactrlbold {\isasymand}\ H{\isacartoucheclose}\ \isacommand{by}\isamarkupfalse%
\ {\isacharparenleft}iprover\ elim{\isacharcolon}\ allE{\isacharparenright}\isanewline
\ \ \isacommand{next}\isamarkupfalse%
\ \isanewline
\ \ \ \ \isacommand{assume}\isamarkupfalse%
\ {\isachardoublequoteopen}{\isacharparenleft}{\isasymexists}G{\isadigit{1}}\ H{\isadigit{1}}{\isachardot}\ F\ {\isacharequal}\ \isactrlbold {\isasymnot}\ {\isacharparenleft}G{\isadigit{1}}\ \isactrlbold {\isasymor}\ H{\isadigit{1}}{\isacharparenright}\ {\isasymand}\ G\ {\isacharequal}\ \isactrlbold {\isasymnot}\ G{\isadigit{1}}\ {\isasymand}\ H\ {\isacharequal}\ \isactrlbold {\isasymnot}\ H{\isadigit{1}}{\isacharparenright}\ {\isasymor}\ \isanewline
\ \ \ \ {\isacharparenleft}{\isacharparenleft}{\isasymexists}H{\isadigit{2}}{\isachardot}\ F\ {\isacharequal}\ \isactrlbold {\isasymnot}\ {\isacharparenleft}G\ \isactrlbold {\isasymrightarrow}\ H{\isadigit{2}}{\isacharparenright}\ {\isasymand}\ H\ {\isacharequal}\ \isactrlbold {\isasymnot}\ H{\isadigit{2}}{\isacharparenright}\ {\isasymor}\ \isanewline
\ \ \ \ F\ {\isacharequal}\ \isactrlbold {\isasymnot}\ {\isacharparenleft}\isactrlbold {\isasymnot}\ G{\isacharparenright}\ {\isasymand}\ H\ {\isacharequal}\ G{\isacharparenright}{\isachardoublequoteclose}\isanewline
\ \ \ \ \isacommand{thus}\isamarkupfalse%
\ {\isachardoublequoteopen}F\ {\isasymin}\ S\ {\isasymlongrightarrow}\ G\ {\isasymin}\ S\ {\isasymand}\ H\ {\isasymin}\ S{\isachardoublequoteclose}\ \isanewline
\ \ \ \ \isacommand{proof}\isamarkupfalse%
\ {\isacharparenleft}rule\ disjE{\isacharparenright}\isanewline
\ \ \ \ \ \ \isacommand{assume}\isamarkupfalse%
\ E{\isadigit{1}}{\isacharcolon}{\isachardoublequoteopen}{\isasymexists}G{\isadigit{1}}\ H{\isadigit{1}}{\isachardot}\ F\ {\isacharequal}\ \isactrlbold {\isasymnot}\ {\isacharparenleft}G{\isadigit{1}}\ \isactrlbold {\isasymor}\ H{\isadigit{1}}{\isacharparenright}\ {\isasymand}\ G\ {\isacharequal}\ \isactrlbold {\isasymnot}\ G{\isadigit{1}}\ {\isasymand}\ H\ {\isacharequal}\ \isactrlbold {\isasymnot}\ H{\isadigit{1}}{\isachardoublequoteclose}\isanewline
\ \ \ \ \ \ \isacommand{obtain}\isamarkupfalse%
\ G{\isadigit{1}}\ H{\isadigit{1}}\ \isakeyword{where}\ A{\isadigit{1}}{\isacharcolon}{\isachardoublequoteopen}F\ {\isacharequal}\ \isactrlbold {\isasymnot}\ {\isacharparenleft}G{\isadigit{1}}\ \isactrlbold {\isasymor}\ H{\isadigit{1}}{\isacharparenright}\ {\isasymand}\ G\ {\isacharequal}\ \isactrlbold {\isasymnot}\ G{\isadigit{1}}\ {\isasymand}\ H\ {\isacharequal}\ \isactrlbold {\isasymnot}\ H{\isadigit{1}}{\isachardoublequoteclose}\isanewline
\ \ \ \ \ \ \ \ \isacommand{using}\isamarkupfalse%
\ E{\isadigit{1}}\ \isacommand{by}\isamarkupfalse%
\ {\isacharparenleft}iprover\ elim{\isacharcolon}\ exE{\isacharparenright}\isanewline
\ \ \ \ \ \ \isacommand{then}\isamarkupfalse%
\ \isacommand{have}\isamarkupfalse%
\ {\isachardoublequoteopen}F\ {\isacharequal}\ \isactrlbold {\isasymnot}\ {\isacharparenleft}G{\isadigit{1}}\ \isactrlbold {\isasymor}\ H{\isadigit{1}}{\isacharparenright}{\isachardoublequoteclose}\isanewline
\ \ \ \ \ \ \ \ \isacommand{by}\isamarkupfalse%
\ {\isacharparenleft}rule\ conjunct{\isadigit{1}}{\isacharparenright}\isanewline
\ \ \ \ \ \ \isacommand{have}\isamarkupfalse%
\ {\isachardoublequoteopen}G\ {\isacharequal}\ \isactrlbold {\isasymnot}\ G{\isadigit{1}}{\isachardoublequoteclose}\isanewline
\ \ \ \ \ \ \ \ \isacommand{using}\isamarkupfalse%
\ A{\isadigit{1}}\ \isacommand{by}\isamarkupfalse%
\ {\isacharparenleft}iprover\ elim{\isacharcolon}\ conjunct{\isadigit{1}}{\isacharparenright}\isanewline
\ \ \ \ \ \ \isacommand{have}\isamarkupfalse%
\ {\isachardoublequoteopen}H\ {\isacharequal}\ \isactrlbold {\isasymnot}\ H{\isadigit{1}}{\isachardoublequoteclose}\isanewline
\ \ \ \ \ \ \ \ \isacommand{using}\isamarkupfalse%
\ A{\isadigit{1}}\ \isacommand{by}\isamarkupfalse%
\ {\isacharparenleft}iprover\ elim{\isacharcolon}\ conjunct{\isadigit{1}}{\isacharparenright}\isanewline
\ \ \ \ \ \ \isacommand{have}\isamarkupfalse%
\ {\isachardoublequoteopen}{\isasymforall}G\ H{\isachardot}\ \isactrlbold {\isasymnot}{\isacharparenleft}G\ \isactrlbold {\isasymor}\ H{\isacharparenright}\ {\isasymin}\ S\ {\isasymlongrightarrow}\ \isactrlbold {\isasymnot}\ G\ {\isasymin}\ S\ {\isasymand}\ \isactrlbold {\isasymnot}\ H\ {\isasymin}\ S{\isachardoublequoteclose}\isanewline
\ \ \ \ \ \ \ \ \isacommand{using}\isamarkupfalse%
\ assms\ \isacommand{by}\isamarkupfalse%
\ {\isacharparenleft}iprover\ elim{\isacharcolon}\ conjunct{\isadigit{2}}\ conjunct{\isadigit{1}}{\isacharparenright}\isanewline
\ \ \ \ \ \ \isacommand{thus}\isamarkupfalse%
\ {\isachardoublequoteopen}F\ {\isasymin}\ S\ {\isasymlongrightarrow}\ G\ {\isasymin}\ S\ {\isasymand}\ H\ {\isasymin}\ S{\isachardoublequoteclose}\isanewline
\ \ \ \ \ \ \ \ \isacommand{using}\isamarkupfalse%
\ {\isacartoucheopen}F\ {\isacharequal}\ \isactrlbold {\isasymnot}\ {\isacharparenleft}G{\isadigit{1}}\ \isactrlbold {\isasymor}\ H{\isadigit{1}}{\isacharparenright}{\isacartoucheclose}\ {\isacartoucheopen}G\ {\isacharequal}\ \isactrlbold {\isasymnot}\ G{\isadigit{1}}{\isacartoucheclose}\ {\isacartoucheopen}H\ {\isacharequal}\ \isactrlbold {\isasymnot}\ H{\isadigit{1}}{\isacartoucheclose}\ \isacommand{by}\isamarkupfalse%
\ {\isacharparenleft}iprover\ elim{\isacharcolon}\ allE{\isacharparenright}\isanewline
\ \ \ \ \isacommand{next}\isamarkupfalse%
\isanewline
\ \ \ \ \ \ \isacommand{assume}\isamarkupfalse%
\ {\isachardoublequoteopen}{\isacharparenleft}{\isasymexists}H{\isadigit{2}}{\isachardot}\ F\ {\isacharequal}\ \isactrlbold {\isasymnot}\ {\isacharparenleft}G\ \isactrlbold {\isasymrightarrow}\ H{\isadigit{2}}{\isacharparenright}\ {\isasymand}\ H\ {\isacharequal}\ \isactrlbold {\isasymnot}\ H{\isadigit{2}}{\isacharparenright}\ {\isasymor}\ \isanewline
\ \ \ \ \ \ F\ {\isacharequal}\ \isactrlbold {\isasymnot}\ {\isacharparenleft}\isactrlbold {\isasymnot}\ G{\isacharparenright}\ {\isasymand}\ H\ {\isacharequal}\ G{\isachardoublequoteclose}\isanewline
\ \ \ \ \ \ \isacommand{thus}\isamarkupfalse%
\ {\isachardoublequoteopen}F\ {\isasymin}\ S\ {\isasymlongrightarrow}\ G\ {\isasymin}\ S\ {\isasymand}\ H\ {\isasymin}\ S{\isachardoublequoteclose}\ \isanewline
\ \ \ \ \ \ \isacommand{proof}\isamarkupfalse%
\ {\isacharparenleft}rule\ disjE{\isacharparenright}\isanewline
\ \ \ \ \ \ \ \ \isacommand{assume}\isamarkupfalse%
\ E{\isadigit{2}}{\isacharcolon}{\isachardoublequoteopen}{\isasymexists}H{\isadigit{2}}{\isachardot}\ F\ {\isacharequal}\ \isactrlbold {\isasymnot}\ {\isacharparenleft}G\ \isactrlbold {\isasymrightarrow}\ H{\isadigit{2}}{\isacharparenright}\ {\isasymand}\ H\ {\isacharequal}\ \isactrlbold {\isasymnot}\ H{\isadigit{2}}{\isachardoublequoteclose}\isanewline
\ \ \ \ \ \ \ \ \isacommand{obtain}\isamarkupfalse%
\ H{\isadigit{2}}\ \isakeyword{where}\ A{\isadigit{2}}{\isacharcolon}{\isachardoublequoteopen}F\ {\isacharequal}\ \isactrlbold {\isasymnot}\ {\isacharparenleft}G\ \isactrlbold {\isasymrightarrow}\ H{\isadigit{2}}{\isacharparenright}\ {\isasymand}\ H\ {\isacharequal}\ \isactrlbold {\isasymnot}\ H{\isadigit{2}}{\isachardoublequoteclose}\isanewline
\ \ \ \ \ \ \ \ \ \ \isacommand{using}\isamarkupfalse%
\ E{\isadigit{2}}\ \isacommand{by}\isamarkupfalse%
\ {\isacharparenleft}rule\ exE{\isacharparenright}\isanewline
\ \ \ \ \ \ \ \ \isacommand{have}\isamarkupfalse%
\ {\isachardoublequoteopen}F\ {\isacharequal}\ \isactrlbold {\isasymnot}\ {\isacharparenleft}G\ \isactrlbold {\isasymrightarrow}\ H{\isadigit{2}}{\isacharparenright}{\isachardoublequoteclose}\isanewline
\ \ \ \ \ \ \ \ \ \ \isacommand{using}\isamarkupfalse%
\ A{\isadigit{2}}\ \isacommand{by}\isamarkupfalse%
\ {\isacharparenleft}rule\ conjunct{\isadigit{1}}{\isacharparenright}\isanewline
\ \ \ \ \ \ \ \ \isacommand{have}\isamarkupfalse%
\ {\isachardoublequoteopen}H\ {\isacharequal}\ \isactrlbold {\isasymnot}\ H{\isadigit{2}}{\isachardoublequoteclose}\isanewline
\ \ \ \ \ \ \ \ \ \ \isacommand{using}\isamarkupfalse%
\ A{\isadigit{2}}\ \isacommand{by}\isamarkupfalse%
\ {\isacharparenleft}rule\ conjunct{\isadigit{2}}{\isacharparenright}\isanewline
\ \ \ \ \ \ \ \ \isacommand{have}\isamarkupfalse%
\ {\isachardoublequoteopen}{\isasymforall}G\ H{\isachardot}\ \isactrlbold {\isasymnot}{\isacharparenleft}G\ \isactrlbold {\isasymrightarrow}\ H{\isacharparenright}\ {\isasymin}\ S\ {\isasymlongrightarrow}\ G\ {\isasymin}\ S\ {\isasymand}\ \isactrlbold {\isasymnot}\ H\ {\isasymin}\ S{\isachardoublequoteclose}\isanewline
\ \ \ \ \ \ \ \ \ \ \isacommand{using}\isamarkupfalse%
\ assms\ \isacommand{by}\isamarkupfalse%
\ {\isacharparenleft}iprover\ elim{\isacharcolon}\ conjunct{\isadigit{2}}\ conjunct{\isadigit{1}}{\isacharparenright}\isanewline
\ \ \ \ \ \ \ \ \isacommand{thus}\isamarkupfalse%
\ {\isachardoublequoteopen}F\ {\isasymin}\ S\ {\isasymlongrightarrow}\ G\ {\isasymin}\ S\ {\isasymand}\ H\ {\isasymin}\ S{\isachardoublequoteclose}\isanewline
\ \ \ \ \ \ \ \ \ \ \isacommand{using}\isamarkupfalse%
\ {\isacartoucheopen}F\ {\isacharequal}\ \isactrlbold {\isasymnot}\ {\isacharparenleft}G\ \isactrlbold {\isasymrightarrow}\ H{\isadigit{2}}{\isacharparenright}{\isacartoucheclose}\ {\isacartoucheopen}H\ {\isacharequal}\ \isactrlbold {\isasymnot}\ H{\isadigit{2}}{\isacartoucheclose}\ \isacommand{by}\isamarkupfalse%
\ {\isacharparenleft}iprover\ elim{\isacharcolon}\ allE{\isacharparenright}\isanewline
\ \ \ \ \ \ \isacommand{next}\isamarkupfalse%
\ \isanewline
\ \ \ \ \ \ \ \ \isacommand{assume}\isamarkupfalse%
\ {\isachardoublequoteopen}F\ {\isacharequal}\ \isactrlbold {\isasymnot}\ {\isacharparenleft}\isactrlbold {\isasymnot}\ G{\isacharparenright}\ {\isasymand}\ H\ {\isacharequal}\ G{\isachardoublequoteclose}\isanewline
\ \ \ \ \ \ \ \ \isacommand{then}\isamarkupfalse%
\ \isacommand{have}\isamarkupfalse%
\ {\isachardoublequoteopen}F\ {\isacharequal}\ \isactrlbold {\isasymnot}\ {\isacharparenleft}\isactrlbold {\isasymnot}\ G{\isacharparenright}{\isachardoublequoteclose}\isanewline
\ \ \ \ \ \ \ \ \ \ \isacommand{by}\isamarkupfalse%
\ {\isacharparenleft}rule\ conjunct{\isadigit{1}}{\isacharparenright}\isanewline
\ \ \ \ \ \ \ \ \isacommand{have}\isamarkupfalse%
\ {\isachardoublequoteopen}H\ {\isacharequal}\ G{\isachardoublequoteclose}\isanewline
\ \ \ \ \ \ \ \ \ \ \isacommand{using}\isamarkupfalse%
\ {\isacartoucheopen}F\ {\isacharequal}\ \isactrlbold {\isasymnot}\ {\isacharparenleft}\isactrlbold {\isasymnot}\ G{\isacharparenright}\ {\isasymand}\ H\ {\isacharequal}\ G{\isacartoucheclose}\ \isacommand{by}\isamarkupfalse%
\ {\isacharparenleft}rule\ conjunct{\isadigit{2}}{\isacharparenright}\isanewline
\ \ \ \ \ \ \ \ \isacommand{have}\isamarkupfalse%
\ {\isachardoublequoteopen}{\isasymforall}G{\isachardot}\ \isactrlbold {\isasymnot}\ {\isacharparenleft}\isactrlbold {\isasymnot}\ G{\isacharparenright}\ {\isasymin}\ S\ {\isasymlongrightarrow}\ G\ {\isasymin}\ S{\isachardoublequoteclose}\isanewline
\ \ \ \ \ \ \ \ \ \ \isacommand{using}\isamarkupfalse%
\ assms\ \isacommand{by}\isamarkupfalse%
\ {\isacharparenleft}iprover\ elim{\isacharcolon}\ conjunct{\isadigit{2}}\ conjunct{\isadigit{1}}{\isacharparenright}\isanewline
\ \ \ \ \ \ \ \ \isacommand{then}\isamarkupfalse%
\ \isacommand{have}\isamarkupfalse%
\ {\isachardoublequoteopen}\isactrlbold {\isasymnot}\ {\isacharparenleft}\isactrlbold {\isasymnot}\ G{\isacharparenright}\ {\isasymin}\ S\ {\isasymlongrightarrow}\ G\ {\isasymin}\ S{\isachardoublequoteclose}\isanewline
\ \ \ \ \ \ \ \ \ \ \isacommand{by}\isamarkupfalse%
\ {\isacharparenleft}rule\ allE{\isacharparenright}\isanewline
\ \ \ \ \ \ \ \ \isacommand{then}\isamarkupfalse%
\ \isacommand{have}\isamarkupfalse%
\ {\isachardoublequoteopen}F\ {\isasymin}\ S\ {\isasymlongrightarrow}\ G\ {\isasymin}\ S{\isachardoublequoteclose}\isanewline
\ \ \ \ \ \ \ \ \ \ \isacommand{by}\isamarkupfalse%
\ {\isacharparenleft}simp\ only{\isacharcolon}\ {\isacartoucheopen}F\ {\isacharequal}\ \isactrlbold {\isasymnot}\ {\isacharparenleft}\isactrlbold {\isasymnot}\ G{\isacharparenright}{\isacartoucheclose}{\isacharparenright}\ \isanewline
\ \ \ \ \ \ \ \ \isacommand{then}\isamarkupfalse%
\ \isacommand{have}\isamarkupfalse%
\ {\isachardoublequoteopen}F\ {\isasymin}\ S\ {\isasymlongrightarrow}\ G\ {\isasymin}\ S\ {\isasymand}\ G\ {\isasymin}\ S{\isachardoublequoteclose}\isanewline
\ \ \ \ \ \ \ \ \ \ \isacommand{by}\isamarkupfalse%
\ {\isacharparenleft}simp\ only{\isacharcolon}\ conj{\isacharunderscore}absorb{\isacharparenright}\isanewline
\ \ \ \ \ \ \ \ \isacommand{thus}\isamarkupfalse%
\ {\isachardoublequoteopen}F\ {\isasymin}\ S\ {\isasymlongrightarrow}\ G\ {\isasymin}\ S\ {\isasymand}\ H\ {\isasymin}\ S{\isachardoublequoteclose}\isanewline
\ \ \ \ \ \ \ \ \ \ \isacommand{by}\isamarkupfalse%
\ {\isacharparenleft}simp\ only{\isacharcolon}\ {\isacartoucheopen}H{\isacharequal}G{\isacartoucheclose}{\isacharparenright}\isanewline
\ \ \ \ \ \ \isacommand{qed}\isamarkupfalse%
\isanewline
\ \ \ \ \isacommand{qed}\isamarkupfalse%
\isanewline
\ \ \isacommand{qed}\isamarkupfalse%
\isanewline
\isacommand{qed}\isamarkupfalse%
%
\endisatagproof
{\isafoldproof}%
%
\isadelimproof
%
\endisadelimproof
%
\begin{isamarkuptext}%
Por otro lado, el segundo lema auxiliar prueba que la cuarta, quinta, sexta
  y séptima condición de la definición de conjunto de Hintikka son suficientes para
  probar que para toda fórmula de tipo \isa{{\isasymbeta}} con componentes \isa{{\isasymbeta}\isactrlsub {\isadigit{1}}} y \isa{{\isasymbeta}\isactrlsub {\isadigit{2}}} se verifica 
  que si la fórmula pertenece al conjunto \isa{S}, entonces o bien \isa{{\isasymbeta}\isactrlsub {\isadigit{1}}} pertenece al
  conjunto o bien \isa{{\isasymbeta}\isactrlsub {\isadigit{2}}} pertenece al conjunto. Veamos su prueba detallada en 
  Isabelle/HOL.%
\end{isamarkuptext}\isamarkuptrue%
\isacommand{lemma}\isamarkupfalse%
\ Hintikka{\isacharunderscore}alt{\isadigit{1}}Dis{\isacharcolon}\isanewline
\ \ \isakeyword{assumes}\ \ {\isachardoublequoteopen}{\isacharparenleft}{\isasymforall}\ G\ H{\isachardot}\ G\ \isactrlbold {\isasymor}\ H\ {\isasymin}\ S\ {\isasymlongrightarrow}\ G\ {\isasymin}\ S\ {\isasymor}\ H\ {\isasymin}\ S{\isacharparenright}\isanewline
\ \ {\isasymand}\ {\isacharparenleft}{\isasymforall}\ G\ H{\isachardot}\ G\ \isactrlbold {\isasymrightarrow}\ H\ {\isasymin}\ S\ {\isasymlongrightarrow}\ \isactrlbold {\isasymnot}\ G\ {\isasymin}\ S\ {\isasymor}\ H\ {\isasymin}\ S{\isacharparenright}\isanewline
\ \ {\isasymand}\ {\isacharparenleft}{\isasymforall}\ G{\isachardot}\ \isactrlbold {\isasymnot}\ {\isacharparenleft}\isactrlbold {\isasymnot}\ G{\isacharparenright}\ {\isasymin}\ S\ {\isasymlongrightarrow}\ G\ {\isasymin}\ S{\isacharparenright}\isanewline
\ \ {\isasymand}\ {\isacharparenleft}{\isasymforall}\ G\ H{\isachardot}\ \isactrlbold {\isasymnot}{\isacharparenleft}G\ \isactrlbold {\isasymand}\ H{\isacharparenright}\ {\isasymin}\ S\ {\isasymlongrightarrow}\ \isactrlbold {\isasymnot}\ G\ {\isasymin}\ S\ {\isasymor}\ \isactrlbold {\isasymnot}\ H\ {\isasymin}\ S{\isacharparenright}{\isachardoublequoteclose}\isanewline
\ \ \isakeyword{shows}\ {\isachardoublequoteopen}Dis\ F\ G\ H\ {\isasymlongrightarrow}\ F\ {\isasymin}\ S\ {\isasymlongrightarrow}\ G\ {\isasymin}\ S\ {\isasymor}\ H\ {\isasymin}\ S{\isachardoublequoteclose}\isanewline
%
\isadelimproof
%
\endisadelimproof
%
\isatagproof
\isacommand{proof}\isamarkupfalse%
\ {\isacharparenleft}rule\ impI{\isacharparenright}\isanewline
\ \ \isacommand{assume}\isamarkupfalse%
\ {\isachardoublequoteopen}Dis\ F\ G\ H{\isachardoublequoteclose}\isanewline
\ \ \isacommand{then}\isamarkupfalse%
\ \isacommand{have}\isamarkupfalse%
\ {\isachardoublequoteopen}F\ {\isacharequal}\ G\ \isactrlbold {\isasymor}\ H\ {\isasymor}\ \isanewline
\ \ \ \ {\isacharparenleft}{\isasymexists}G{\isadigit{1}}\ H{\isadigit{1}}{\isachardot}\ F\ {\isacharequal}\ G{\isadigit{1}}\ \isactrlbold {\isasymrightarrow}\ H{\isadigit{1}}\ {\isasymand}\ G\ {\isacharequal}\ \isactrlbold {\isasymnot}\ G{\isadigit{1}}\ {\isasymand}\ H\ {\isacharequal}\ H{\isadigit{1}}{\isacharparenright}\ {\isasymor}\ \isanewline
\ \ \ \ {\isacharparenleft}{\isasymexists}G{\isadigit{2}}\ H{\isadigit{2}}{\isachardot}\ F\ {\isacharequal}\ \isactrlbold {\isasymnot}\ {\isacharparenleft}G{\isadigit{2}}\ \isactrlbold {\isasymand}\ H{\isadigit{2}}{\isacharparenright}\ {\isasymand}\ G\ {\isacharequal}\ \isactrlbold {\isasymnot}\ G{\isadigit{2}}\ {\isasymand}\ H\ {\isacharequal}\ \isactrlbold {\isasymnot}\ H{\isadigit{2}}{\isacharparenright}\ {\isasymor}\ \isanewline
\ \ \ \ F\ {\isacharequal}\ \isactrlbold {\isasymnot}\ {\isacharparenleft}\isactrlbold {\isasymnot}\ G{\isacharparenright}\ {\isasymand}\ H\ {\isacharequal}\ G{\isachardoublequoteclose}\ \isanewline
\ \ \ \ \isacommand{by}\isamarkupfalse%
\ {\isacharparenleft}simp\ only{\isacharcolon}\ con{\isacharunderscore}dis{\isacharunderscore}simps{\isacharparenleft}{\isadigit{2}}{\isacharparenright}{\isacharparenright}\isanewline
\ \ \isacommand{thus}\isamarkupfalse%
\ {\isachardoublequoteopen}F\ {\isasymin}\ S\ {\isasymlongrightarrow}\ G\ {\isasymin}\ S\ {\isasymor}\ H\ {\isasymin}\ S{\isachardoublequoteclose}\ \isanewline
\ \ \isacommand{proof}\isamarkupfalse%
\ {\isacharparenleft}rule\ disjE{\isacharparenright}\isanewline
\ \ \ \ \isacommand{assume}\isamarkupfalse%
\ {\isachardoublequoteopen}F\ {\isacharequal}\ G\ \isactrlbold {\isasymor}\ H{\isachardoublequoteclose}\isanewline
\ \ \ \ \isacommand{have}\isamarkupfalse%
\ {\isachardoublequoteopen}{\isasymforall}G\ H{\isachardot}\ G\ \isactrlbold {\isasymor}\ H\ {\isasymin}\ S\ {\isasymlongrightarrow}\ G\ {\isasymin}\ S\ {\isasymor}\ H\ {\isasymin}\ S{\isachardoublequoteclose}\isanewline
\ \ \ \ \ \ \isacommand{using}\isamarkupfalse%
\ assms\ \isacommand{by}\isamarkupfalse%
\ {\isacharparenleft}rule\ conjunct{\isadigit{1}}{\isacharparenright}\isanewline
\ \ \ \ \isacommand{thus}\isamarkupfalse%
\ {\isachardoublequoteopen}F\ {\isasymin}\ S\ {\isasymlongrightarrow}\ G\ {\isasymin}\ S\ {\isasymor}\ H\ {\isasymin}\ S{\isachardoublequoteclose}\ \isanewline
\ \ \ \ \ \ \isacommand{using}\isamarkupfalse%
\ {\isacartoucheopen}F\ {\isacharequal}\ G\ \isactrlbold {\isasymor}\ H{\isacartoucheclose}\ \isacommand{by}\isamarkupfalse%
\ {\isacharparenleft}iprover\ elim{\isacharcolon}\ allE{\isacharparenright}\isanewline
\ \ \isacommand{next}\isamarkupfalse%
\isanewline
\ \ \ \ \isacommand{assume}\isamarkupfalse%
\ {\isachardoublequoteopen}{\isacharparenleft}{\isasymexists}G{\isadigit{1}}\ H{\isadigit{1}}{\isachardot}\ F\ {\isacharequal}\ G{\isadigit{1}}\ \isactrlbold {\isasymrightarrow}\ H{\isadigit{1}}\ {\isasymand}\ G\ {\isacharequal}\ \isactrlbold {\isasymnot}\ G{\isadigit{1}}\ {\isasymand}\ H\ {\isacharequal}\ H{\isadigit{1}}{\isacharparenright}\ {\isasymor}\ \isanewline
\ \ \ \ {\isacharparenleft}{\isasymexists}G{\isadigit{2}}\ H{\isadigit{2}}{\isachardot}\ F\ {\isacharequal}\ \isactrlbold {\isasymnot}\ {\isacharparenleft}G{\isadigit{2}}\ \isactrlbold {\isasymand}\ H{\isadigit{2}}{\isacharparenright}\ {\isasymand}\ G\ {\isacharequal}\ \isactrlbold {\isasymnot}\ G{\isadigit{2}}\ {\isasymand}\ H\ {\isacharequal}\ \isactrlbold {\isasymnot}\ H{\isadigit{2}}{\isacharparenright}\ {\isasymor}\ \isanewline
\ \ \ \ F\ {\isacharequal}\ \isactrlbold {\isasymnot}\ {\isacharparenleft}\isactrlbold {\isasymnot}\ G{\isacharparenright}\ {\isasymand}\ H\ {\isacharequal}\ G{\isachardoublequoteclose}\isanewline
\ \ \ \ \isacommand{thus}\isamarkupfalse%
\ {\isachardoublequoteopen}F\ {\isasymin}\ S\ {\isasymlongrightarrow}\ G\ {\isasymin}\ S\ {\isasymor}\ H\ {\isasymin}\ S{\isachardoublequoteclose}\isanewline
\ \ \ \ \isacommand{proof}\isamarkupfalse%
\ {\isacharparenleft}rule\ disjE{\isacharparenright}\isanewline
\ \ \ \ \ \ \isacommand{assume}\isamarkupfalse%
\ E{\isadigit{1}}{\isacharcolon}{\isachardoublequoteopen}{\isasymexists}G{\isadigit{1}}\ H{\isadigit{1}}{\isachardot}\ F\ {\isacharequal}\ G{\isadigit{1}}\ \isactrlbold {\isasymrightarrow}\ H{\isadigit{1}}\ {\isasymand}\ G\ {\isacharequal}\ \isactrlbold {\isasymnot}\ G{\isadigit{1}}\ {\isasymand}\ H\ {\isacharequal}\ H{\isadigit{1}}{\isachardoublequoteclose}\isanewline
\ \ \ \ \ \ \isacommand{obtain}\isamarkupfalse%
\ G{\isadigit{1}}\ H{\isadigit{1}}\ \isakeyword{where}\ A{\isadigit{1}}{\isacharcolon}{\isachardoublequoteopen}F\ {\isacharequal}\ G{\isadigit{1}}\ \isactrlbold {\isasymrightarrow}\ H{\isadigit{1}}\ {\isasymand}\ G\ {\isacharequal}\ \isactrlbold {\isasymnot}\ G{\isadigit{1}}\ {\isasymand}\ H\ {\isacharequal}\ H{\isadigit{1}}{\isachardoublequoteclose}\isanewline
\ \ \ \ \ \ \ \ \isacommand{using}\isamarkupfalse%
\ E{\isadigit{1}}\ \isacommand{by}\isamarkupfalse%
\ {\isacharparenleft}iprover\ elim{\isacharcolon}\ exE{\isacharparenright}\isanewline
\ \ \ \ \ \ \isacommand{have}\isamarkupfalse%
\ {\isachardoublequoteopen}F\ {\isacharequal}\ G{\isadigit{1}}\ \isactrlbold {\isasymrightarrow}\ H{\isadigit{1}}{\isachardoublequoteclose}\isanewline
\ \ \ \ \ \ \ \ \isacommand{using}\isamarkupfalse%
\ A{\isadigit{1}}\ \isacommand{by}\isamarkupfalse%
\ {\isacharparenleft}rule\ conjunct{\isadigit{1}}{\isacharparenright}\isanewline
\ \ \ \ \ \ \isacommand{have}\isamarkupfalse%
\ {\isachardoublequoteopen}G\ {\isacharequal}\ \isactrlbold {\isasymnot}\ G{\isadigit{1}}{\isachardoublequoteclose}\isanewline
\ \ \ \ \ \ \ \ \isacommand{using}\isamarkupfalse%
\ A{\isadigit{1}}\ \isacommand{by}\isamarkupfalse%
\ {\isacharparenleft}iprover\ elim{\isacharcolon}\ conjunct{\isadigit{1}}{\isacharparenright}\isanewline
\ \ \ \ \ \ \isacommand{have}\isamarkupfalse%
\ {\isachardoublequoteopen}H\ {\isacharequal}\ H{\isadigit{1}}{\isachardoublequoteclose}\isanewline
\ \ \ \ \ \ \ \ \isacommand{using}\isamarkupfalse%
\ A{\isadigit{1}}\ \isacommand{by}\isamarkupfalse%
\ {\isacharparenleft}iprover\ elim{\isacharcolon}\ conjunct{\isadigit{2}}\ conjunct{\isadigit{1}}{\isacharparenright}\isanewline
\ \ \ \ \ \ \isacommand{have}\isamarkupfalse%
\ {\isachardoublequoteopen}{\isasymforall}G\ H{\isachardot}\ G\ \isactrlbold {\isasymrightarrow}\ H\ {\isasymin}\ S\ {\isasymlongrightarrow}\ \isactrlbold {\isasymnot}\ G\ {\isasymin}\ S\ {\isasymor}\ H\ {\isasymin}\ S{\isachardoublequoteclose}\isanewline
\ \ \ \ \ \ \ \ \isacommand{using}\isamarkupfalse%
\ assms\ \isacommand{by}\isamarkupfalse%
\ {\isacharparenleft}iprover\ elim{\isacharcolon}\ conjunct{\isadigit{2}}\ conjunct{\isadigit{1}}{\isacharparenright}\isanewline
\ \ \ \ \ \ \isacommand{thus}\isamarkupfalse%
\ {\isachardoublequoteopen}F\ {\isasymin}\ S\ {\isasymlongrightarrow}\ G\ {\isasymin}\ S\ {\isasymor}\ H\ {\isasymin}\ S{\isachardoublequoteclose}\isanewline
\ \ \ \ \ \ \ \ \isacommand{using}\isamarkupfalse%
\ {\isacartoucheopen}F\ {\isacharequal}\ G{\isadigit{1}}\ \isactrlbold {\isasymrightarrow}\ H{\isadigit{1}}{\isacartoucheclose}\ {\isacartoucheopen}G\ {\isacharequal}\ \isactrlbold {\isasymnot}\ G{\isadigit{1}}{\isacartoucheclose}\ {\isacartoucheopen}H\ {\isacharequal}\ H{\isadigit{1}}{\isacartoucheclose}\ \isacommand{by}\isamarkupfalse%
\ {\isacharparenleft}iprover\ elim{\isacharcolon}\ allE{\isacharparenright}\isanewline
\ \ \ \ \isacommand{next}\isamarkupfalse%
\isanewline
\ \ \ \ \ \ \isacommand{assume}\isamarkupfalse%
\ {\isachardoublequoteopen}{\isacharparenleft}{\isasymexists}G{\isadigit{2}}\ H{\isadigit{2}}{\isachardot}\ F\ {\isacharequal}\ \isactrlbold {\isasymnot}\ {\isacharparenleft}G{\isadigit{2}}\ \isactrlbold {\isasymand}\ H{\isadigit{2}}{\isacharparenright}\ {\isasymand}\ G\ {\isacharequal}\ \isactrlbold {\isasymnot}\ G{\isadigit{2}}\ {\isasymand}\ H\ {\isacharequal}\ \isactrlbold {\isasymnot}\ H{\isadigit{2}}{\isacharparenright}\ {\isasymor}\ \isanewline
\ \ \ \ \ \ F\ {\isacharequal}\ \isactrlbold {\isasymnot}\ {\isacharparenleft}\isactrlbold {\isasymnot}\ G{\isacharparenright}\ {\isasymand}\ H\ {\isacharequal}\ G{\isachardoublequoteclose}\isanewline
\ \ \ \ \ \ \isacommand{thus}\isamarkupfalse%
\ {\isachardoublequoteopen}F\ {\isasymin}\ S\ {\isasymlongrightarrow}\ G\ {\isasymin}\ S\ {\isasymor}\ H\ {\isasymin}\ S{\isachardoublequoteclose}\isanewline
\ \ \ \ \ \ \isacommand{proof}\isamarkupfalse%
\ {\isacharparenleft}rule\ disjE{\isacharparenright}\isanewline
\ \ \ \ \ \ \ \ \isacommand{assume}\isamarkupfalse%
\ E{\isadigit{2}}{\isacharcolon}{\isachardoublequoteopen}{\isasymexists}G{\isadigit{2}}\ H{\isadigit{2}}{\isachardot}\ F\ {\isacharequal}\ \isactrlbold {\isasymnot}\ {\isacharparenleft}G{\isadigit{2}}\ \isactrlbold {\isasymand}\ H{\isadigit{2}}{\isacharparenright}\ {\isasymand}\ G\ {\isacharequal}\ \isactrlbold {\isasymnot}\ G{\isadigit{2}}\ {\isasymand}\ H\ {\isacharequal}\ \isactrlbold {\isasymnot}\ H{\isadigit{2}}{\isachardoublequoteclose}\isanewline
\ \ \ \ \ \ \ \ \isacommand{obtain}\isamarkupfalse%
\ G{\isadigit{2}}\ H{\isadigit{2}}\ \isakeyword{where}\ A{\isadigit{2}}{\isacharcolon}{\isachardoublequoteopen}F\ {\isacharequal}\ \isactrlbold {\isasymnot}\ {\isacharparenleft}G{\isadigit{2}}\ \isactrlbold {\isasymand}\ H{\isadigit{2}}{\isacharparenright}\ {\isasymand}\ G\ {\isacharequal}\ \isactrlbold {\isasymnot}\ G{\isadigit{2}}\ {\isasymand}\ H\ {\isacharequal}\ \isactrlbold {\isasymnot}\ H{\isadigit{2}}{\isachardoublequoteclose}\ \isanewline
\ \ \ \ \ \ \ \ \ \ \isacommand{using}\isamarkupfalse%
\ E{\isadigit{2}}\ \isacommand{by}\isamarkupfalse%
\ {\isacharparenleft}iprover\ elim{\isacharcolon}\ exE{\isacharparenright}\isanewline
\ \ \ \ \ \ \ \ \isacommand{have}\isamarkupfalse%
\ {\isachardoublequoteopen}F\ {\isacharequal}\ \isactrlbold {\isasymnot}\ {\isacharparenleft}G{\isadigit{2}}\ \isactrlbold {\isasymand}\ H{\isadigit{2}}{\isacharparenright}{\isachardoublequoteclose}\ \isanewline
\ \ \ \ \ \ \ \ \ \ \isacommand{using}\isamarkupfalse%
\ A{\isadigit{2}}\ \isacommand{by}\isamarkupfalse%
\ {\isacharparenleft}rule\ conjunct{\isadigit{1}}{\isacharparenright}\isanewline
\ \ \ \ \ \ \ \ \isacommand{have}\isamarkupfalse%
\ {\isachardoublequoteopen}G\ {\isacharequal}\ \isactrlbold {\isasymnot}\ G{\isadigit{2}}{\isachardoublequoteclose}\isanewline
\ \ \ \ \ \ \ \ \ \ \isacommand{using}\isamarkupfalse%
\ A{\isadigit{2}}\ \isacommand{by}\isamarkupfalse%
\ {\isacharparenleft}iprover\ elim{\isacharcolon}\ conjunct{\isadigit{2}}\ conjunct{\isadigit{1}}{\isacharparenright}\isanewline
\ \ \ \ \ \ \ \ \isacommand{have}\isamarkupfalse%
\ {\isachardoublequoteopen}H\ {\isacharequal}\ \isactrlbold {\isasymnot}\ H{\isadigit{2}}{\isachardoublequoteclose}\isanewline
\ \ \ \ \ \ \ \ \ \ \isacommand{using}\isamarkupfalse%
\ A{\isadigit{2}}\ \isacommand{by}\isamarkupfalse%
\ {\isacharparenleft}iprover\ elim{\isacharcolon}\ conjunct{\isadigit{1}}{\isacharparenright}\isanewline
\ \ \ \ \ \ \ \ \isacommand{have}\isamarkupfalse%
\ {\isachardoublequoteopen}{\isasymforall}\ G\ H{\isachardot}\ \isactrlbold {\isasymnot}{\isacharparenleft}G\ \isactrlbold {\isasymand}\ H{\isacharparenright}\ {\isasymin}\ S\ {\isasymlongrightarrow}\ \isactrlbold {\isasymnot}\ G\ {\isasymin}\ S\ {\isasymor}\ \isactrlbold {\isasymnot}\ H\ {\isasymin}\ S{\isachardoublequoteclose}\isanewline
\ \ \ \ \ \ \ \ \ \ \isacommand{using}\isamarkupfalse%
\ assms\ \isacommand{by}\isamarkupfalse%
\ {\isacharparenleft}iprover\ elim{\isacharcolon}\ conjunct{\isadigit{2}}\ conjunct{\isadigit{1}}{\isacharparenright}\isanewline
\ \ \ \ \ \ \ \ \isacommand{thus}\isamarkupfalse%
\ {\isachardoublequoteopen}F\ {\isasymin}\ S\ {\isasymlongrightarrow}\ G\ {\isasymin}\ S\ {\isasymor}\ H\ {\isasymin}\ S{\isachardoublequoteclose}\isanewline
\ \ \ \ \ \ \ \ \ \ \isacommand{using}\isamarkupfalse%
\ {\isacartoucheopen}F\ {\isacharequal}\ \isactrlbold {\isasymnot}{\isacharparenleft}G{\isadigit{2}}\ \isactrlbold {\isasymand}\ H{\isadigit{2}}{\isacharparenright}{\isacartoucheclose}\ {\isacartoucheopen}G\ {\isacharequal}\ \isactrlbold {\isasymnot}\ G{\isadigit{2}}{\isacartoucheclose}\ {\isacartoucheopen}H\ {\isacharequal}\ \isactrlbold {\isasymnot}\ H{\isadigit{2}}{\isacartoucheclose}\ \isacommand{by}\isamarkupfalse%
\ {\isacharparenleft}iprover\ elim{\isacharcolon}\ allE{\isacharparenright}\isanewline
\ \ \ \ \ \ \isacommand{next}\isamarkupfalse%
\isanewline
\ \ \ \ \ \ \ \ \isacommand{assume}\isamarkupfalse%
\ {\isachardoublequoteopen}F\ {\isacharequal}\ \isactrlbold {\isasymnot}\ {\isacharparenleft}\isactrlbold {\isasymnot}\ G{\isacharparenright}\ {\isasymand}\ H\ {\isacharequal}\ G{\isachardoublequoteclose}\isanewline
\ \ \ \ \ \ \ \ \isacommand{then}\isamarkupfalse%
\ \isacommand{have}\isamarkupfalse%
\ {\isachardoublequoteopen}F\ {\isacharequal}\ \isactrlbold {\isasymnot}\ {\isacharparenleft}\isactrlbold {\isasymnot}\ G{\isacharparenright}{\isachardoublequoteclose}\ \isanewline
\ \ \ \ \ \ \ \ \ \ \isacommand{by}\isamarkupfalse%
\ {\isacharparenleft}rule\ conjunct{\isadigit{1}}{\isacharparenright}\isanewline
\ \ \ \ \ \ \ \ \isacommand{have}\isamarkupfalse%
\ {\isachardoublequoteopen}H\ {\isacharequal}\ G{\isachardoublequoteclose}\isanewline
\ \ \ \ \ \ \ \ \ \ \isacommand{using}\isamarkupfalse%
\ {\isacartoucheopen}F\ {\isacharequal}\ \isactrlbold {\isasymnot}\ {\isacharparenleft}\isactrlbold {\isasymnot}\ G{\isacharparenright}\ {\isasymand}\ H\ {\isacharequal}\ G{\isacartoucheclose}\ \isacommand{by}\isamarkupfalse%
\ {\isacharparenleft}rule\ conjunct{\isadigit{2}}{\isacharparenright}\isanewline
\ \ \ \ \ \ \ \ \isacommand{have}\isamarkupfalse%
\ {\isachardoublequoteopen}{\isasymforall}\ G{\isachardot}\ \isactrlbold {\isasymnot}\ {\isacharparenleft}\isactrlbold {\isasymnot}\ G{\isacharparenright}\ {\isasymin}\ S\ {\isasymlongrightarrow}\ G\ {\isasymin}\ S{\isachardoublequoteclose}\isanewline
\ \ \ \ \ \ \ \ \ \ \isacommand{using}\isamarkupfalse%
\ assms\ \isacommand{by}\isamarkupfalse%
\ {\isacharparenleft}iprover\ elim{\isacharcolon}\ conjunct{\isadigit{2}}\ conjunct{\isadigit{1}}{\isacharparenright}\isanewline
\ \ \ \ \ \ \ \ \isacommand{then}\isamarkupfalse%
\ \isacommand{have}\isamarkupfalse%
\ {\isachardoublequoteopen}\isactrlbold {\isasymnot}\ {\isacharparenleft}\isactrlbold {\isasymnot}\ G{\isacharparenright}\ {\isasymin}\ S\ {\isasymlongrightarrow}\ G\ {\isasymin}\ S{\isachardoublequoteclose}\isanewline
\ \ \ \ \ \ \ \ \ \ \isacommand{by}\isamarkupfalse%
\ {\isacharparenleft}rule\ allE{\isacharparenright}\isanewline
\ \ \ \ \ \ \ \ \isacommand{then}\isamarkupfalse%
\ \isacommand{have}\isamarkupfalse%
\ {\isachardoublequoteopen}F\ {\isasymin}\ S\ {\isasymlongrightarrow}\ G\ {\isasymin}\ S{\isachardoublequoteclose}\isanewline
\ \ \ \ \ \ \ \ \ \ \isacommand{by}\isamarkupfalse%
\ {\isacharparenleft}simp\ only{\isacharcolon}\ {\isacartoucheopen}F\ {\isacharequal}\ \isactrlbold {\isasymnot}\ {\isacharparenleft}\isactrlbold {\isasymnot}\ G{\isacharparenright}{\isacartoucheclose}{\isacharparenright}\isanewline
\ \ \ \ \ \ \ \ \isacommand{then}\isamarkupfalse%
\ \isacommand{have}\isamarkupfalse%
\ {\isachardoublequoteopen}F\ {\isasymin}\ S\ {\isasymlongrightarrow}\ G\ {\isasymin}\ S\ {\isasymor}\ G\ {\isasymin}\ S{\isachardoublequoteclose}\isanewline
\ \ \ \ \ \ \ \ \ \ \isacommand{by}\isamarkupfalse%
\ {\isacharparenleft}simp\ only{\isacharcolon}\ disj{\isacharunderscore}absorb{\isacharparenright}\isanewline
\ \ \ \ \ \ \ \ \isacommand{thus}\isamarkupfalse%
\ {\isachardoublequoteopen}F\ {\isasymin}\ S\ {\isasymlongrightarrow}\ G\ {\isasymin}\ S\ {\isasymor}\ H\ {\isasymin}\ S{\isachardoublequoteclose}\isanewline
\ \ \ \ \ \ \ \ \isacommand{by}\isamarkupfalse%
\ {\isacharparenleft}simp\ only{\isacharcolon}\ {\isacartoucheopen}H\ {\isacharequal}\ G{\isacartoucheclose}{\isacharparenright}\isanewline
\ \ \ \ \ \ \isacommand{qed}\isamarkupfalse%
\isanewline
\ \ \ \ \isacommand{qed}\isamarkupfalse%
\isanewline
\ \ \isacommand{qed}\isamarkupfalse%
\isanewline
\isacommand{qed}\isamarkupfalse%
%
\endisatagproof
{\isafoldproof}%
%
\isadelimproof
%
\endisadelimproof
%
\begin{isamarkuptext}%
Finalmente, podemos demostrar detalladamente esta primera implicación de la
  equivalencia del lema en Isabelle.%
\end{isamarkuptext}\isamarkuptrue%
\isacommand{lemma}\isamarkupfalse%
\ Hintikka{\isacharunderscore}alt{\isadigit{1}}{\isacharcolon}\isanewline
\ \ \isakeyword{assumes}\ {\isachardoublequoteopen}Hintikka\ S{\isachardoublequoteclose}\isanewline
\ \ \isakeyword{shows}\ {\isachardoublequoteopen}{\isasymbottom}\ {\isasymnotin}\ S\isanewline
{\isasymand}\ {\isacharparenleft}{\isasymforall}k{\isachardot}\ Atom\ k\ {\isasymin}\ S\ {\isasymlongrightarrow}\ \isactrlbold {\isasymnot}\ {\isacharparenleft}Atom\ k{\isacharparenright}\ {\isasymin}\ S\ {\isasymlongrightarrow}\ False{\isacharparenright}\isanewline
{\isasymand}\ {\isacharparenleft}{\isasymforall}F\ G\ H{\isachardot}\ Con\ F\ G\ H\ {\isasymlongrightarrow}\ F\ {\isasymin}\ S\ {\isasymlongrightarrow}\ G\ {\isasymin}\ S\ {\isasymand}\ H\ {\isasymin}\ S{\isacharparenright}\isanewline
{\isasymand}\ {\isacharparenleft}{\isasymforall}F\ G\ H{\isachardot}\ Dis\ F\ G\ H\ {\isasymlongrightarrow}\ F\ {\isasymin}\ S\ {\isasymlongrightarrow}\ G\ {\isasymin}\ S\ {\isasymor}\ H\ {\isasymin}\ S{\isacharparenright}{\isachardoublequoteclose}\isanewline
%
\isadelimproof
%
\endisadelimproof
%
\isatagproof
\isacommand{proof}\isamarkupfalse%
\ {\isacharminus}\isanewline
\ \ \isacommand{have}\isamarkupfalse%
\ Hk{\isacharcolon}{\isachardoublequoteopen}{\isacharparenleft}{\isasymbottom}\ {\isasymnotin}\ S\isanewline
\ \ {\isasymand}\ {\isacharparenleft}{\isasymforall}k{\isachardot}\ Atom\ k\ {\isasymin}\ S\ {\isasymlongrightarrow}\ \isactrlbold {\isasymnot}\ {\isacharparenleft}Atom\ k{\isacharparenright}\ {\isasymin}\ S\ {\isasymlongrightarrow}\ False{\isacharparenright}\isanewline
\ \ {\isasymand}\ {\isacharparenleft}{\isasymforall}G\ H{\isachardot}\ G\ \isactrlbold {\isasymand}\ H\ {\isasymin}\ S\ {\isasymlongrightarrow}\ G\ {\isasymin}\ S\ {\isasymand}\ H\ {\isasymin}\ S{\isacharparenright}\isanewline
\ \ {\isasymand}\ {\isacharparenleft}{\isasymforall}G\ H{\isachardot}\ G\ \isactrlbold {\isasymor}\ H\ {\isasymin}\ S\ {\isasymlongrightarrow}\ G\ {\isasymin}\ S\ {\isasymor}\ H\ {\isasymin}\ S{\isacharparenright}\isanewline
\ \ {\isasymand}\ {\isacharparenleft}{\isasymforall}G\ H{\isachardot}\ G\ \isactrlbold {\isasymrightarrow}\ H\ {\isasymin}\ S\ {\isasymlongrightarrow}\ \isactrlbold {\isasymnot}G\ {\isasymin}\ S\ {\isasymor}\ H\ {\isasymin}\ S{\isacharparenright}\isanewline
\ \ {\isasymand}\ {\isacharparenleft}{\isasymforall}G{\isachardot}\ \isactrlbold {\isasymnot}\ {\isacharparenleft}\isactrlbold {\isasymnot}G{\isacharparenright}\ {\isasymin}\ S\ {\isasymlongrightarrow}\ G\ {\isasymin}\ S{\isacharparenright}\isanewline
\ \ {\isasymand}\ {\isacharparenleft}{\isasymforall}G\ H{\isachardot}\ \isactrlbold {\isasymnot}{\isacharparenleft}G\ \isactrlbold {\isasymand}\ H{\isacharparenright}\ {\isasymin}\ S\ {\isasymlongrightarrow}\ \isactrlbold {\isasymnot}\ G\ {\isasymin}\ S\ {\isasymor}\ \isactrlbold {\isasymnot}\ H\ {\isasymin}\ S{\isacharparenright}\isanewline
\ \ {\isasymand}\ {\isacharparenleft}{\isasymforall}G\ H{\isachardot}\ \isactrlbold {\isasymnot}{\isacharparenleft}G\ \isactrlbold {\isasymor}\ H{\isacharparenright}\ {\isasymin}\ S\ {\isasymlongrightarrow}\ \isactrlbold {\isasymnot}\ G\ {\isasymin}\ S\ {\isasymand}\ \isactrlbold {\isasymnot}\ H\ {\isasymin}\ S{\isacharparenright}\isanewline
\ \ {\isasymand}\ {\isacharparenleft}{\isasymforall}G\ H{\isachardot}\ \isactrlbold {\isasymnot}{\isacharparenleft}G\ \isactrlbold {\isasymrightarrow}\ H{\isacharparenright}\ {\isasymin}\ S\ {\isasymlongrightarrow}\ G\ {\isasymin}\ S\ {\isasymand}\ \isactrlbold {\isasymnot}\ H\ {\isasymin}\ S{\isacharparenright}{\isacharparenright}{\isachardoublequoteclose}\isanewline
\ \ \ \ \isacommand{using}\isamarkupfalse%
\ assms\ \isacommand{by}\isamarkupfalse%
\ {\isacharparenleft}rule\ auxEq{\isacharparenright}\isanewline
\ \ \isacommand{then}\isamarkupfalse%
\ \isacommand{have}\isamarkupfalse%
\ C{\isadigit{1}}{\isacharcolon}\ {\isachardoublequoteopen}{\isasymbottom}\ {\isasymnotin}\ S{\isachardoublequoteclose}\isanewline
\ \ \ \ \isacommand{by}\isamarkupfalse%
\ {\isacharparenleft}rule\ conjunct{\isadigit{1}}{\isacharparenright}\isanewline
\ \ \isacommand{have}\isamarkupfalse%
\ C{\isadigit{2}}{\isacharcolon}\ {\isachardoublequoteopen}{\isasymforall}k{\isachardot}\ Atom\ k\ {\isasymin}\ S\ {\isasymlongrightarrow}\ \isactrlbold {\isasymnot}\ {\isacharparenleft}Atom\ k{\isacharparenright}\ {\isasymin}\ S\ {\isasymlongrightarrow}\ False{\isachardoublequoteclose}\isanewline
\ \ \ \ \isacommand{using}\isamarkupfalse%
\ Hk\ \isacommand{by}\isamarkupfalse%
\ {\isacharparenleft}iprover\ elim{\isacharcolon}\ conjunct{\isadigit{2}}\ conjunct{\isadigit{1}}{\isacharparenright}\isanewline
\ \ \isacommand{have}\isamarkupfalse%
\ C{\isadigit{3}}{\isacharcolon}\ {\isachardoublequoteopen}{\isasymforall}F\ G\ H{\isachardot}\ Con\ F\ G\ H\ {\isasymlongrightarrow}\ F\ {\isasymin}\ S\ {\isasymlongrightarrow}\ G\ {\isasymin}\ S\ {\isasymand}\ H\ {\isasymin}\ S{\isachardoublequoteclose}\isanewline
\ \ \isacommand{proof}\isamarkupfalse%
\ {\isacharparenleft}rule\ allI{\isacharparenright}{\isacharplus}\isanewline
\ \ \ \ \isacommand{fix}\isamarkupfalse%
\ F\ G\ H\isanewline
\ \ \ \ \isacommand{have}\isamarkupfalse%
\ C{\isadigit{3}}{\isadigit{1}}{\isacharcolon}{\isachardoublequoteopen}{\isasymforall}G\ H{\isachardot}\ G\ \isactrlbold {\isasymand}\ H\ {\isasymin}\ S\ {\isasymlongrightarrow}\ G\ {\isasymin}\ S\ {\isasymand}\ H\ {\isasymin}\ S{\isachardoublequoteclose}\isanewline
\ \ \ \ \ \ \isacommand{using}\isamarkupfalse%
\ Hk\ \isacommand{by}\isamarkupfalse%
\ {\isacharparenleft}iprover\ elim{\isacharcolon}\ conjunct{\isadigit{2}}\ conjunct{\isadigit{1}}{\isacharparenright}\isanewline
\ \ \ \ \isacommand{have}\isamarkupfalse%
\ C{\isadigit{3}}{\isadigit{2}}{\isacharcolon}{\isachardoublequoteopen}{\isasymforall}G{\isachardot}\ \isactrlbold {\isasymnot}\ {\isacharparenleft}\isactrlbold {\isasymnot}\ G{\isacharparenright}\ {\isasymin}\ S\ {\isasymlongrightarrow}\ G\ {\isasymin}\ S{\isachardoublequoteclose}\isanewline
\ \ \ \ \ \ \isacommand{using}\isamarkupfalse%
\ Hk\ \isacommand{by}\isamarkupfalse%
\ {\isacharparenleft}iprover\ elim{\isacharcolon}\ conjunct{\isadigit{2}}\ conjunct{\isadigit{1}}{\isacharparenright}\isanewline
\ \ \ \ \isacommand{have}\isamarkupfalse%
\ C{\isadigit{3}}{\isadigit{3}}{\isacharcolon}{\isachardoublequoteopen}{\isasymforall}G\ H{\isachardot}\ \isactrlbold {\isasymnot}{\isacharparenleft}G\ \isactrlbold {\isasymor}\ H{\isacharparenright}\ {\isasymin}\ S\ {\isasymlongrightarrow}\ \isactrlbold {\isasymnot}\ G\ {\isasymin}\ S\ {\isasymand}\ \isactrlbold {\isasymnot}\ H\ {\isasymin}\ S{\isachardoublequoteclose}\isanewline
\ \ \ \ \ \ \isacommand{using}\isamarkupfalse%
\ Hk\ \isacommand{by}\isamarkupfalse%
\ {\isacharparenleft}iprover\ elim{\isacharcolon}\ conjunct{\isadigit{2}}\ conjunct{\isadigit{1}}{\isacharparenright}\isanewline
\ \ \ \ \isacommand{have}\isamarkupfalse%
\ C{\isadigit{3}}{\isadigit{4}}{\isacharcolon}{\isachardoublequoteopen}{\isasymforall}G\ H{\isachardot}\ \isactrlbold {\isasymnot}{\isacharparenleft}G\ \isactrlbold {\isasymrightarrow}\ H{\isacharparenright}\ {\isasymin}\ S\ {\isasymlongrightarrow}\ G\ {\isasymin}\ S\ {\isasymand}\ \isactrlbold {\isasymnot}\ H\ {\isasymin}\ S{\isachardoublequoteclose}\isanewline
\ \ \ \ \ \ \isacommand{using}\isamarkupfalse%
\ Hk\ \isacommand{by}\isamarkupfalse%
\ {\isacharparenleft}iprover\ elim{\isacharcolon}\ conjunct{\isadigit{2}}\ conjunct{\isadigit{1}}{\isacharparenright}\isanewline
\ \ \ \ \isacommand{have}\isamarkupfalse%
\ {\isachardoublequoteopen}{\isacharparenleft}{\isasymforall}G\ H{\isachardot}\ G\ \isactrlbold {\isasymand}\ H\ {\isasymin}\ S\ {\isasymlongrightarrow}\ G\ {\isasymin}\ S\ {\isasymand}\ H\ {\isasymin}\ S{\isacharparenright}\isanewline
\ \ \ \ \ \ \ \ \ \ {\isasymand}\ {\isacharparenleft}{\isasymforall}G{\isachardot}\ \isactrlbold {\isasymnot}\ {\isacharparenleft}\isactrlbold {\isasymnot}\ G{\isacharparenright}\ {\isasymin}\ S\ {\isasymlongrightarrow}\ G\ {\isasymin}\ S{\isacharparenright}\isanewline
\ \ \ \ \ \ \ \ \ \ {\isasymand}\ {\isacharparenleft}{\isasymforall}G\ H{\isachardot}\ \isactrlbold {\isasymnot}{\isacharparenleft}G\ \isactrlbold {\isasymor}\ H{\isacharparenright}\ {\isasymin}\ S\ {\isasymlongrightarrow}\ \isactrlbold {\isasymnot}\ G\ {\isasymin}\ S\ {\isasymand}\ \isactrlbold {\isasymnot}\ H\ {\isasymin}\ S{\isacharparenright}\isanewline
\ \ \ \ \ \ \ \ \ \ {\isasymand}\ {\isacharparenleft}{\isasymforall}G\ H{\isachardot}\ \isactrlbold {\isasymnot}{\isacharparenleft}G\ \isactrlbold {\isasymrightarrow}\ H{\isacharparenright}\ {\isasymin}\ S\ {\isasymlongrightarrow}\ G\ {\isasymin}\ S\ {\isasymand}\ \isactrlbold {\isasymnot}\ H\ {\isasymin}\ S{\isacharparenright}{\isachardoublequoteclose}\ \isanewline
\ \ \ \ \ \ \isacommand{using}\isamarkupfalse%
\ C{\isadigit{3}}{\isadigit{1}}\ C{\isadigit{3}}{\isadigit{2}}\ C{\isadigit{3}}{\isadigit{3}}\ C{\isadigit{3}}{\isadigit{4}}\ \isacommand{by}\isamarkupfalse%
\ {\isacharparenleft}iprover\ intro{\isacharcolon}\ conjI{\isacharparenright}\isanewline
\ \ \ \ \isacommand{thus}\isamarkupfalse%
\ {\isachardoublequoteopen}Con\ F\ G\ H\ {\isasymlongrightarrow}\ F\ {\isasymin}\ S\ {\isasymlongrightarrow}\ G\ {\isasymin}\ S\ {\isasymand}\ H\ {\isasymin}\ S{\isachardoublequoteclose}\isanewline
\ \ \ \ \ \ \isacommand{by}\isamarkupfalse%
\ {\isacharparenleft}rule\ Hintikka{\isacharunderscore}alt{\isadigit{1}}Con{\isacharparenright}\isanewline
\ \ \isacommand{qed}\isamarkupfalse%
\isanewline
\ \ \isacommand{have}\isamarkupfalse%
\ C{\isadigit{4}}{\isacharcolon}{\isachardoublequoteopen}{\isasymforall}F\ G\ H{\isachardot}\ Dis\ F\ G\ H\ {\isasymlongrightarrow}\ F\ {\isasymin}\ S\ {\isasymlongrightarrow}\ G\ {\isasymin}\ S\ {\isasymor}\ H\ {\isasymin}\ S{\isachardoublequoteclose}\isanewline
\ \ \isacommand{proof}\isamarkupfalse%
\ {\isacharparenleft}rule\ allI{\isacharparenright}{\isacharplus}\isanewline
\ \ \ \ \isacommand{fix}\isamarkupfalse%
\ F\ G\ H\isanewline
\ \ \ \ \isacommand{have}\isamarkupfalse%
\ C{\isadigit{4}}{\isadigit{1}}{\isacharcolon}{\isachardoublequoteopen}{\isasymforall}G\ H{\isachardot}\ G\ \isactrlbold {\isasymor}\ H\ {\isasymin}\ S\ {\isasymlongrightarrow}\ G\ {\isasymin}\ S\ {\isasymor}\ H\ {\isasymin}\ S{\isachardoublequoteclose}\isanewline
\ \ \ \ \ \ \isacommand{using}\isamarkupfalse%
\ Hk\ \isacommand{by}\isamarkupfalse%
\ {\isacharparenleft}iprover\ elim{\isacharcolon}\ conjunct{\isadigit{2}}\ conjunct{\isadigit{1}}{\isacharparenright}\isanewline
\ \ \ \ \isacommand{have}\isamarkupfalse%
\ C{\isadigit{4}}{\isadigit{2}}{\isacharcolon}{\isachardoublequoteopen}{\isasymforall}G\ H{\isachardot}\ G\ \isactrlbold {\isasymrightarrow}\ H\ {\isasymin}\ S\ {\isasymlongrightarrow}\ \isactrlbold {\isasymnot}\ G\ {\isasymin}\ S\ {\isasymor}\ H\ {\isasymin}\ S{\isachardoublequoteclose}\isanewline
\ \ \ \ \ \ \isacommand{using}\isamarkupfalse%
\ Hk\ \isacommand{by}\isamarkupfalse%
\ {\isacharparenleft}iprover\ elim{\isacharcolon}\ conjunct{\isadigit{2}}\ conjunct{\isadigit{1}}{\isacharparenright}\isanewline
\ \ \ \ \isacommand{have}\isamarkupfalse%
\ C{\isadigit{4}}{\isadigit{3}}{\isacharcolon}{\isachardoublequoteopen}{\isasymforall}G{\isachardot}\ \isactrlbold {\isasymnot}\ {\isacharparenleft}\isactrlbold {\isasymnot}\ G{\isacharparenright}\ {\isasymin}\ S\ {\isasymlongrightarrow}\ G\ {\isasymin}\ S{\isachardoublequoteclose}\isanewline
\ \ \ \ \ \ \isacommand{using}\isamarkupfalse%
\ Hk\ \isacommand{by}\isamarkupfalse%
\ {\isacharparenleft}iprover\ elim{\isacharcolon}\ conjunct{\isadigit{2}}\ conjunct{\isadigit{1}}{\isacharparenright}\isanewline
\ \ \ \ \isacommand{have}\isamarkupfalse%
\ C{\isadigit{4}}{\isadigit{4}}{\isacharcolon}{\isachardoublequoteopen}{\isasymforall}G\ H{\isachardot}\ \isactrlbold {\isasymnot}{\isacharparenleft}G\ \isactrlbold {\isasymand}\ H{\isacharparenright}\ {\isasymin}\ S\ {\isasymlongrightarrow}\ \isactrlbold {\isasymnot}\ G\ {\isasymin}\ S\ {\isasymor}\ \isactrlbold {\isasymnot}\ H\ {\isasymin}\ S{\isachardoublequoteclose}\isanewline
\ \ \ \ \ \ \isacommand{using}\isamarkupfalse%
\ Hk\ \isacommand{by}\isamarkupfalse%
\ {\isacharparenleft}iprover\ elim{\isacharcolon}\ conjunct{\isadigit{2}}\ conjunct{\isadigit{1}}{\isacharparenright}\isanewline
\ \ \ \ \isacommand{have}\isamarkupfalse%
\ {\isachardoublequoteopen}{\isacharparenleft}{\isasymforall}G\ H{\isachardot}\ G\ \isactrlbold {\isasymor}\ H\ {\isasymin}\ S\ {\isasymlongrightarrow}\ G\ {\isasymin}\ S\ {\isasymor}\ H\ {\isasymin}\ S{\isacharparenright}\isanewline
\ \ \ \ \ \ \ \ \ \ {\isasymand}\ {\isacharparenleft}{\isasymforall}G\ H{\isachardot}\ G\ \isactrlbold {\isasymrightarrow}\ H\ {\isasymin}\ S\ {\isasymlongrightarrow}\ \isactrlbold {\isasymnot}\ G\ {\isasymin}\ S\ {\isasymor}\ H\ {\isasymin}\ S{\isacharparenright}\isanewline
\ \ \ \ \ \ \ \ \ \ {\isasymand}\ {\isacharparenleft}{\isasymforall}G{\isachardot}\ \isactrlbold {\isasymnot}\ {\isacharparenleft}\isactrlbold {\isasymnot}\ G{\isacharparenright}\ {\isasymin}\ S\ {\isasymlongrightarrow}\ G\ {\isasymin}\ S{\isacharparenright}\isanewline
\ \ \ \ \ \ \ \ \ \ {\isasymand}\ {\isacharparenleft}{\isasymforall}G\ H{\isachardot}\ \isactrlbold {\isasymnot}{\isacharparenleft}G\ \isactrlbold {\isasymand}\ H{\isacharparenright}\ {\isasymin}\ S\ {\isasymlongrightarrow}\ \isactrlbold {\isasymnot}\ G\ {\isasymin}\ S\ {\isasymor}\ \isactrlbold {\isasymnot}\ H\ {\isasymin}\ S{\isacharparenright}{\isachardoublequoteclose}\isanewline
\ \ \ \ \ \ \isacommand{using}\isamarkupfalse%
\ C{\isadigit{4}}{\isadigit{1}}\ C{\isadigit{4}}{\isadigit{2}}\ C{\isadigit{4}}{\isadigit{3}}\ C{\isadigit{4}}{\isadigit{4}}\ \isacommand{by}\isamarkupfalse%
\ {\isacharparenleft}iprover\ intro{\isacharcolon}\ conjI{\isacharparenright}\isanewline
\ \ \ \ \isacommand{thus}\isamarkupfalse%
\ {\isachardoublequoteopen}Dis\ F\ G\ H\ {\isasymlongrightarrow}\ F\ {\isasymin}\ S\ {\isasymlongrightarrow}\ G\ {\isasymin}\ S\ {\isasymor}\ H\ {\isasymin}\ S{\isachardoublequoteclose}\isanewline
\ \ \ \ \ \ \isacommand{by}\isamarkupfalse%
\ {\isacharparenleft}rule\ Hintikka{\isacharunderscore}alt{\isadigit{1}}Dis{\isacharparenright}\isanewline
\ \ \isacommand{qed}\isamarkupfalse%
\isanewline
\ \ \isacommand{show}\isamarkupfalse%
\ {\isachardoublequoteopen}{\isasymbottom}\ {\isasymnotin}\ S\isanewline
\ \ {\isasymand}\ {\isacharparenleft}{\isasymforall}k{\isachardot}\ Atom\ k\ {\isasymin}\ S\ {\isasymlongrightarrow}\ \isactrlbold {\isasymnot}\ {\isacharparenleft}Atom\ k{\isacharparenright}\ {\isasymin}\ S\ {\isasymlongrightarrow}\ False{\isacharparenright}\isanewline
\ \ {\isasymand}\ {\isacharparenleft}{\isasymforall}F\ G\ H{\isachardot}\ Con\ F\ G\ H\ {\isasymlongrightarrow}\ F\ {\isasymin}\ S\ {\isasymlongrightarrow}\ G\ {\isasymin}\ S\ {\isasymand}\ H\ {\isasymin}\ S{\isacharparenright}\isanewline
\ \ {\isasymand}\ {\isacharparenleft}{\isasymforall}F\ G\ H{\isachardot}\ Dis\ F\ G\ H\ {\isasymlongrightarrow}\ F\ {\isasymin}\ S\ {\isasymlongrightarrow}\ G\ {\isasymin}\ S\ {\isasymor}\ H\ {\isasymin}\ S{\isacharparenright}{\isachardoublequoteclose}\isanewline
\ \ \ \ \isacommand{using}\isamarkupfalse%
\ C{\isadigit{1}}\ C{\isadigit{2}}\ C{\isadigit{3}}\ C{\isadigit{4}}\ \isacommand{by}\isamarkupfalse%
\ {\isacharparenleft}iprover\ intro{\isacharcolon}\ conjI{\isacharparenright}\isanewline
\isacommand{qed}\isamarkupfalse%
%
\endisatagproof
{\isafoldproof}%
%
\isadelimproof
%
\endisadelimproof
%
\begin{isamarkuptext}%
Por último, probamos la implicación recíproca de forma detallada en Isabelle mediante
  el siguiente lema.%
\end{isamarkuptext}\isamarkuptrue%
\isacommand{lemma}\isamarkupfalse%
\ Hintikka{\isacharunderscore}alt{\isadigit{2}}{\isacharcolon}\isanewline
\ \ \isakeyword{assumes}\ {\isachardoublequoteopen}{\isasymbottom}\ {\isasymnotin}\ S\isanewline
{\isasymand}\ {\isacharparenleft}{\isasymforall}k{\isachardot}\ Atom\ k\ {\isasymin}\ S\ {\isasymlongrightarrow}\ \isactrlbold {\isasymnot}\ {\isacharparenleft}Atom\ k{\isacharparenright}\ {\isasymin}\ S\ {\isasymlongrightarrow}\ False{\isacharparenright}\isanewline
{\isasymand}\ {\isacharparenleft}{\isasymforall}F\ G\ H{\isachardot}\ Con\ F\ G\ H\ {\isasymlongrightarrow}\ F\ {\isasymin}\ S\ {\isasymlongrightarrow}\ G\ {\isasymin}\ S\ {\isasymand}\ H\ {\isasymin}\ S{\isacharparenright}\ \isanewline
{\isasymand}\ {\isacharparenleft}{\isasymforall}F\ G\ H{\isachardot}\ Dis\ F\ G\ H\ {\isasymlongrightarrow}\ F\ {\isasymin}\ S\ {\isasymlongrightarrow}\ G\ {\isasymin}\ S\ {\isasymor}\ H\ {\isasymin}\ S{\isacharparenright}{\isachardoublequoteclose}\ \ \isanewline
\ \ \isakeyword{shows}\ {\isachardoublequoteopen}Hintikka\ S{\isachardoublequoteclose}\isanewline
%
\isadelimproof
%
\endisadelimproof
%
\isatagproof
\isacommand{proof}\isamarkupfalse%
\ {\isacharminus}\isanewline
\ \ \isacommand{have}\isamarkupfalse%
\ Con{\isacharcolon}{\isachardoublequoteopen}{\isasymforall}F\ G\ H{\isachardot}\ Con\ F\ G\ H\ {\isasymlongrightarrow}\ F\ {\isasymin}\ S\ {\isasymlongrightarrow}\ G\ {\isasymin}\ S\ {\isasymand}\ H\ {\isasymin}\ S{\isachardoublequoteclose}\isanewline
\ \ \ \ \isacommand{using}\isamarkupfalse%
\ assms\ \isacommand{by}\isamarkupfalse%
\ {\isacharparenleft}iprover\ elim{\isacharcolon}\ conjunct{\isadigit{2}}\ conjunct{\isadigit{1}}{\isacharparenright}\isanewline
\ \ \isacommand{have}\isamarkupfalse%
\ Dis{\isacharcolon}{\isachardoublequoteopen}{\isasymforall}F\ G\ H{\isachardot}\ Dis\ F\ G\ H\ {\isasymlongrightarrow}\ F\ {\isasymin}\ S\ {\isasymlongrightarrow}\ G\ {\isasymin}\ S\ {\isasymor}\ H\ {\isasymin}\ S{\isachardoublequoteclose}\isanewline
\ \ \ \ \isacommand{using}\isamarkupfalse%
\ assms\ \isacommand{by}\isamarkupfalse%
\ {\isacharparenleft}iprover\ elim{\isacharcolon}\ conjunct{\isadigit{2}}\ conjunct{\isadigit{1}}{\isacharparenright}\isanewline
\ \ \isacommand{have}\isamarkupfalse%
\ {\isachardoublequoteopen}{\isasymbottom}\ {\isasymnotin}\ S\isanewline
\ \ {\isasymand}\ {\isacharparenleft}{\isasymforall}k{\isachardot}\ Atom\ k\ {\isasymin}\ S\ {\isasymlongrightarrow}\ \isactrlbold {\isasymnot}\ {\isacharparenleft}Atom\ k{\isacharparenright}\ {\isasymin}\ S\ {\isasymlongrightarrow}\ False{\isacharparenright}\isanewline
\ \ {\isasymand}\ {\isacharparenleft}{\isasymforall}G\ H{\isachardot}\ G\ \isactrlbold {\isasymand}\ H\ {\isasymin}\ S\ {\isasymlongrightarrow}\ G\ {\isasymin}\ S\ {\isasymand}\ H\ {\isasymin}\ S{\isacharparenright}\isanewline
\ \ {\isasymand}\ {\isacharparenleft}{\isasymforall}G\ H{\isachardot}\ G\ \isactrlbold {\isasymor}\ H\ {\isasymin}\ S\ {\isasymlongrightarrow}\ G\ {\isasymin}\ S\ {\isasymor}\ H\ {\isasymin}\ S{\isacharparenright}\isanewline
\ \ {\isasymand}\ {\isacharparenleft}{\isasymforall}G\ H{\isachardot}\ G\ \isactrlbold {\isasymrightarrow}\ H\ {\isasymin}\ S\ {\isasymlongrightarrow}\ \isactrlbold {\isasymnot}G\ {\isasymin}\ S\ {\isasymor}\ H\ {\isasymin}\ S{\isacharparenright}\isanewline
\ \ {\isasymand}\ {\isacharparenleft}{\isasymforall}G{\isachardot}\ \isactrlbold {\isasymnot}\ {\isacharparenleft}\isactrlbold {\isasymnot}G{\isacharparenright}\ {\isasymin}\ S\ {\isasymlongrightarrow}\ G\ {\isasymin}\ S{\isacharparenright}\isanewline
\ \ {\isasymand}\ {\isacharparenleft}{\isasymforall}G\ H{\isachardot}\ \isactrlbold {\isasymnot}{\isacharparenleft}G\ \isactrlbold {\isasymand}\ H{\isacharparenright}\ {\isasymin}\ S\ {\isasymlongrightarrow}\ \isactrlbold {\isasymnot}\ G\ {\isasymin}\ S\ {\isasymor}\ \isactrlbold {\isasymnot}\ H\ {\isasymin}\ S{\isacharparenright}\isanewline
\ \ {\isasymand}\ {\isacharparenleft}{\isasymforall}G\ H{\isachardot}\ \isactrlbold {\isasymnot}{\isacharparenleft}G\ \isactrlbold {\isasymor}\ H{\isacharparenright}\ {\isasymin}\ S\ {\isasymlongrightarrow}\ \isactrlbold {\isasymnot}\ G\ {\isasymin}\ S\ {\isasymand}\ \isactrlbold {\isasymnot}\ H\ {\isasymin}\ S{\isacharparenright}\isanewline
\ \ {\isasymand}\ {\isacharparenleft}{\isasymforall}G\ H{\isachardot}\ \isactrlbold {\isasymnot}{\isacharparenleft}G\ \isactrlbold {\isasymrightarrow}\ H{\isacharparenright}\ {\isasymin}\ S\ {\isasymlongrightarrow}\ G\ {\isasymin}\ S\ {\isasymand}\ \isactrlbold {\isasymnot}\ H\ {\isasymin}\ S{\isacharparenright}{\isachardoublequoteclose}\isanewline
\ \ \isacommand{proof}\isamarkupfalse%
\ {\isacharminus}\isanewline
\ \ \ \ \isacommand{have}\isamarkupfalse%
\ C{\isadigit{1}}{\isacharcolon}{\isachardoublequoteopen}{\isasymbottom}\ {\isasymnotin}\ S{\isachardoublequoteclose}\isanewline
\ \ \ \ \ \ \isacommand{using}\isamarkupfalse%
\ assms\ \isacommand{by}\isamarkupfalse%
\ {\isacharparenleft}rule\ conjunct{\isadigit{1}}{\isacharparenright}\isanewline
\ \ \ \ \isacommand{have}\isamarkupfalse%
\ C{\isadigit{2}}{\isacharcolon}{\isachardoublequoteopen}{\isasymforall}k{\isachardot}\ Atom\ k\ {\isasymin}\ S\ {\isasymlongrightarrow}\ \isactrlbold {\isasymnot}\ {\isacharparenleft}Atom\ k{\isacharparenright}\ {\isasymin}\ S\ {\isasymlongrightarrow}\ False{\isachardoublequoteclose}\isanewline
\ \ \ \ \ \ \isacommand{using}\isamarkupfalse%
\ assms\ \isacommand{by}\isamarkupfalse%
\ {\isacharparenleft}iprover\ elim{\isacharcolon}\ conjunct{\isadigit{2}}\ conjunct{\isadigit{1}}{\isacharparenright}\isanewline
\ \ \ \ \isacommand{have}\isamarkupfalse%
\ C{\isadigit{3}}{\isacharcolon}{\isachardoublequoteopen}{\isasymforall}G\ H{\isachardot}\ G\ \isactrlbold {\isasymand}\ H\ {\isasymin}\ S\ {\isasymlongrightarrow}\ G\ {\isasymin}\ S\ {\isasymand}\ H\ {\isasymin}\ S{\isachardoublequoteclose}\isanewline
\ \ \ \ \isacommand{proof}\isamarkupfalse%
\ {\isacharparenleft}rule\ allI{\isacharparenright}{\isacharplus}\isanewline
\ \ \ \ \ \ \isacommand{fix}\isamarkupfalse%
\ G\ H\isanewline
\ \ \ \ \ \ \isacommand{show}\isamarkupfalse%
\ {\isachardoublequoteopen}G\ \isactrlbold {\isasymand}\ H\ {\isasymin}\ S\ {\isasymlongrightarrow}\ G\ {\isasymin}\ S\ {\isasymand}\ H\ {\isasymin}\ S{\isachardoublequoteclose}\isanewline
\ \ \ \ \ \ \isacommand{proof}\isamarkupfalse%
\ {\isacharparenleft}rule\ impI{\isacharparenright}\isanewline
\ \ \ \ \ \ \ \ \isacommand{assume}\isamarkupfalse%
\ {\isachardoublequoteopen}G\ \isactrlbold {\isasymand}\ H\ {\isasymin}\ S{\isachardoublequoteclose}\isanewline
\ \ \ \ \ \ \ \ \isacommand{have}\isamarkupfalse%
\ {\isachardoublequoteopen}Con\ {\isacharparenleft}G\ \isactrlbold {\isasymand}\ H{\isacharparenright}\ G\ H{\isachardoublequoteclose}\isanewline
\ \ \ \ \ \ \ \ \ \ \isacommand{by}\isamarkupfalse%
\ {\isacharparenleft}simp\ only{\isacharcolon}\ Con{\isachardot}intros{\isacharparenleft}{\isadigit{1}}{\isacharparenright}{\isacharparenright}\isanewline
\ \ \ \ \ \ \ \ \isacommand{have}\isamarkupfalse%
\ {\isachardoublequoteopen}Con\ {\isacharparenleft}G\ \isactrlbold {\isasymand}\ H{\isacharparenright}\ G\ H\ {\isasymlongrightarrow}\ G\ \isactrlbold {\isasymand}\ H\ {\isasymin}\ S\ {\isasymlongrightarrow}\ G\ {\isasymin}\ S\ {\isasymand}\ H\ {\isasymin}\ S{\isachardoublequoteclose}\isanewline
\ \ \ \ \ \ \ \ \ \ \isacommand{using}\isamarkupfalse%
\ Con\ \isacommand{by}\isamarkupfalse%
\ {\isacharparenleft}iprover\ elim{\isacharcolon}\ allE{\isacharparenright}\isanewline
\ \ \ \ \ \ \ \ \isacommand{then}\isamarkupfalse%
\ \isacommand{have}\isamarkupfalse%
\ {\isachardoublequoteopen}G\ \isactrlbold {\isasymand}\ H\ {\isasymin}\ S\ {\isasymlongrightarrow}\ G\ {\isasymin}\ S\ {\isasymand}\ H\ {\isasymin}\ S{\isachardoublequoteclose}\isanewline
\ \ \ \ \ \ \ \ \ \ \isacommand{using}\isamarkupfalse%
\ {\isacartoucheopen}Con\ {\isacharparenleft}G\ \isactrlbold {\isasymand}\ H{\isacharparenright}\ G\ H{\isacartoucheclose}\ \isacommand{by}\isamarkupfalse%
\ {\isacharparenleft}rule\ mp{\isacharparenright}\isanewline
\ \ \ \ \ \ \ \ \isacommand{thus}\isamarkupfalse%
\ {\isachardoublequoteopen}G\ {\isasymin}\ S\ {\isasymand}\ H\ {\isasymin}\ S{\isachardoublequoteclose}\isanewline
\ \ \ \ \ \ \ \ \ \ \isacommand{using}\isamarkupfalse%
\ {\isacartoucheopen}G\ \isactrlbold {\isasymand}\ H\ {\isasymin}\ S{\isacartoucheclose}\ \isacommand{by}\isamarkupfalse%
\ {\isacharparenleft}rule\ mp{\isacharparenright}\isanewline
\ \ \ \ \ \ \isacommand{qed}\isamarkupfalse%
\isanewline
\ \ \ \ \isacommand{qed}\isamarkupfalse%
\isanewline
\ \ \ \ \isacommand{have}\isamarkupfalse%
\ C{\isadigit{4}}{\isacharcolon}{\isachardoublequoteopen}{\isasymforall}G\ H{\isachardot}\ G\ \isactrlbold {\isasymor}\ H\ {\isasymin}\ S\ {\isasymlongrightarrow}\ G\ {\isasymin}\ S\ {\isasymor}\ H\ {\isasymin}\ S{\isachardoublequoteclose}\isanewline
\ \ \ \ \isacommand{proof}\isamarkupfalse%
\ {\isacharparenleft}rule\ allI{\isacharparenright}{\isacharplus}\isanewline
\ \ \ \ \ \ \isacommand{fix}\isamarkupfalse%
\ G\ H\isanewline
\ \ \ \ \ \ \isacommand{show}\isamarkupfalse%
\ {\isachardoublequoteopen}G\ \isactrlbold {\isasymor}\ H\ {\isasymin}\ S\ {\isasymlongrightarrow}\ G\ {\isasymin}\ S\ {\isasymor}\ H\ {\isasymin}\ S{\isachardoublequoteclose}\isanewline
\ \ \ \ \ \ \isacommand{proof}\isamarkupfalse%
\ {\isacharparenleft}rule\ impI{\isacharparenright}\isanewline
\ \ \ \ \ \ \ \ \isacommand{assume}\isamarkupfalse%
\ {\isachardoublequoteopen}G\ \isactrlbold {\isasymor}\ H\ {\isasymin}\ S{\isachardoublequoteclose}\isanewline
\ \ \ \ \ \ \ \ \isacommand{have}\isamarkupfalse%
\ {\isachardoublequoteopen}Dis\ {\isacharparenleft}G\ \isactrlbold {\isasymor}\ H{\isacharparenright}\ G\ H{\isachardoublequoteclose}\isanewline
\ \ \ \ \ \ \ \ \ \ \isacommand{by}\isamarkupfalse%
\ {\isacharparenleft}simp\ only{\isacharcolon}\ Dis{\isachardot}intros{\isacharparenleft}{\isadigit{1}}{\isacharparenright}{\isacharparenright}\isanewline
\ \ \ \ \ \ \ \ \isacommand{have}\isamarkupfalse%
\ {\isachardoublequoteopen}Dis\ {\isacharparenleft}G\ \isactrlbold {\isasymor}\ H{\isacharparenright}\ G\ H\ {\isasymlongrightarrow}\ G\ \isactrlbold {\isasymor}\ H\ {\isasymin}\ S\ {\isasymlongrightarrow}\ G\ {\isasymin}\ S\ {\isasymor}\ H\ {\isasymin}\ S{\isachardoublequoteclose}\isanewline
\ \ \ \ \ \ \ \ \ \ \isacommand{using}\isamarkupfalse%
\ Dis\ \isacommand{by}\isamarkupfalse%
\ {\isacharparenleft}iprover\ elim{\isacharcolon}\ allE{\isacharparenright}\isanewline
\ \ \ \ \ \ \ \ \isacommand{then}\isamarkupfalse%
\ \isacommand{have}\isamarkupfalse%
\ {\isachardoublequoteopen}G\ \isactrlbold {\isasymor}\ H\ {\isasymin}\ S\ {\isasymlongrightarrow}\ G\ {\isasymin}\ S\ {\isasymor}\ H\ {\isasymin}\ S{\isachardoublequoteclose}\isanewline
\ \ \ \ \ \ \ \ \ \ \isacommand{using}\isamarkupfalse%
\ {\isacartoucheopen}Dis\ {\isacharparenleft}G\ \isactrlbold {\isasymor}\ H{\isacharparenright}\ G\ H{\isacartoucheclose}\ \isacommand{by}\isamarkupfalse%
\ {\isacharparenleft}rule\ mp{\isacharparenright}\isanewline
\ \ \ \ \ \ \ \ \isacommand{thus}\isamarkupfalse%
\ {\isachardoublequoteopen}G\ {\isasymin}\ S\ {\isasymor}\ H\ {\isasymin}\ S{\isachardoublequoteclose}\isanewline
\ \ \ \ \ \ \ \ \ \ \isacommand{using}\isamarkupfalse%
\ {\isacartoucheopen}G\ \isactrlbold {\isasymor}\ H\ {\isasymin}\ S{\isacartoucheclose}\ \isacommand{by}\isamarkupfalse%
\ {\isacharparenleft}rule\ mp{\isacharparenright}\isanewline
\ \ \ \ \ \ \isacommand{qed}\isamarkupfalse%
\isanewline
\ \ \ \ \isacommand{qed}\isamarkupfalse%
\isanewline
\ \ \ \ \isacommand{have}\isamarkupfalse%
\ C{\isadigit{5}}{\isacharcolon}{\isachardoublequoteopen}{\isasymforall}G\ H{\isachardot}\ G\ \isactrlbold {\isasymrightarrow}\ H\ {\isasymin}\ S\ {\isasymlongrightarrow}\ \isactrlbold {\isasymnot}\ G\ {\isasymin}\ S\ {\isasymor}\ H\ {\isasymin}\ S{\isachardoublequoteclose}\isanewline
\ \ \ \ \isacommand{proof}\isamarkupfalse%
\ {\isacharparenleft}rule\ allI{\isacharparenright}{\isacharplus}\isanewline
\ \ \ \ \ \ \isacommand{fix}\isamarkupfalse%
\ G\ H\isanewline
\ \ \ \ \ \ \isacommand{show}\isamarkupfalse%
\ {\isachardoublequoteopen}G\ \isactrlbold {\isasymrightarrow}\ H\ {\isasymin}\ S\ {\isasymlongrightarrow}\ \isactrlbold {\isasymnot}\ G\ {\isasymin}\ S\ {\isasymor}\ H\ {\isasymin}\ S{\isachardoublequoteclose}\isanewline
\ \ \ \ \ \ \isacommand{proof}\isamarkupfalse%
\ {\isacharparenleft}rule\ impI{\isacharparenright}\isanewline
\ \ \ \ \ \ \ \ \isacommand{assume}\isamarkupfalse%
\ {\isachardoublequoteopen}G\ \isactrlbold {\isasymrightarrow}\ H\ {\isasymin}\ S{\isachardoublequoteclose}\ \isanewline
\ \ \ \ \ \ \ \ \isacommand{have}\isamarkupfalse%
\ {\isachardoublequoteopen}Dis\ {\isacharparenleft}G\ \isactrlbold {\isasymrightarrow}\ H{\isacharparenright}\ {\isacharparenleft}\isactrlbold {\isasymnot}\ G{\isacharparenright}\ H{\isachardoublequoteclose}\isanewline
\ \ \ \ \ \ \ \ \ \ \isacommand{by}\isamarkupfalse%
\ {\isacharparenleft}simp\ only{\isacharcolon}\ Dis{\isachardot}intros{\isacharparenleft}{\isadigit{2}}{\isacharparenright}{\isacharparenright}\isanewline
\ \ \ \ \ \ \ \ \isacommand{have}\isamarkupfalse%
\ {\isachardoublequoteopen}Dis\ {\isacharparenleft}G\ \isactrlbold {\isasymrightarrow}\ H{\isacharparenright}\ {\isacharparenleft}\isactrlbold {\isasymnot}\ G{\isacharparenright}\ H\ {\isasymlongrightarrow}\ G\ \isactrlbold {\isasymrightarrow}\ H\ {\isasymin}\ S\ {\isasymlongrightarrow}\ \isactrlbold {\isasymnot}\ G\ {\isasymin}\ S\ {\isasymor}\ H\ {\isasymin}\ S{\isachardoublequoteclose}\isanewline
\ \ \ \ \ \ \ \ \ \ \isacommand{using}\isamarkupfalse%
\ Dis\ \isacommand{by}\isamarkupfalse%
\ {\isacharparenleft}iprover\ elim{\isacharcolon}\ allE{\isacharparenright}\isanewline
\ \ \ \ \ \ \ \ \isacommand{then}\isamarkupfalse%
\ \isacommand{have}\isamarkupfalse%
\ {\isachardoublequoteopen}G\ \isactrlbold {\isasymrightarrow}\ H\ {\isasymin}\ S\ {\isasymlongrightarrow}\ \isactrlbold {\isasymnot}\ G\ {\isasymin}\ S\ {\isasymor}\ H\ {\isasymin}\ S{\isachardoublequoteclose}\ \isanewline
\ \ \ \ \ \ \ \ \ \ \isacommand{using}\isamarkupfalse%
\ {\isacartoucheopen}Dis\ {\isacharparenleft}G\ \isactrlbold {\isasymrightarrow}\ H{\isacharparenright}\ {\isacharparenleft}\isactrlbold {\isasymnot}\ G{\isacharparenright}\ H{\isacartoucheclose}\ \isacommand{by}\isamarkupfalse%
\ {\isacharparenleft}rule\ mp{\isacharparenright}\isanewline
\ \ \ \ \ \ \ \ \isacommand{thus}\isamarkupfalse%
\ {\isachardoublequoteopen}\isactrlbold {\isasymnot}\ G\ {\isasymin}\ S\ {\isasymor}\ H\ {\isasymin}\ S{\isachardoublequoteclose}\isanewline
\ \ \ \ \ \ \ \ \ \ \isacommand{using}\isamarkupfalse%
\ {\isacartoucheopen}G\ \isactrlbold {\isasymrightarrow}\ H\ {\isasymin}\ S{\isacartoucheclose}\ \isacommand{by}\isamarkupfalse%
\ {\isacharparenleft}rule\ mp{\isacharparenright}\isanewline
\ \ \ \ \ \ \isacommand{qed}\isamarkupfalse%
\isanewline
\ \ \ \ \isacommand{qed}\isamarkupfalse%
\isanewline
\ \ \ \ \isacommand{have}\isamarkupfalse%
\ C{\isadigit{6}}{\isacharcolon}{\isachardoublequoteopen}{\isasymforall}G{\isachardot}\ \isactrlbold {\isasymnot}{\isacharparenleft}\isactrlbold {\isasymnot}\ G{\isacharparenright}\ {\isasymin}\ S\ {\isasymlongrightarrow}\ G\ {\isasymin}\ S{\isachardoublequoteclose}\isanewline
\ \ \ \ \isacommand{proof}\isamarkupfalse%
\ {\isacharparenleft}rule\ allI{\isacharparenright}\isanewline
\ \ \ \ \ \ \isacommand{fix}\isamarkupfalse%
\ G\isanewline
\ \ \ \ \ \ \isacommand{show}\isamarkupfalse%
\ {\isachardoublequoteopen}\isactrlbold {\isasymnot}{\isacharparenleft}\isactrlbold {\isasymnot}\ G{\isacharparenright}\ {\isasymin}\ S\ {\isasymlongrightarrow}\ G\ {\isasymin}\ S{\isachardoublequoteclose}\isanewline
\ \ \ \ \ \ \isacommand{proof}\isamarkupfalse%
\ {\isacharparenleft}rule\ impI{\isacharparenright}\isanewline
\ \ \ \ \ \ \ \ \isacommand{assume}\isamarkupfalse%
\ {\isachardoublequoteopen}\isactrlbold {\isasymnot}\ {\isacharparenleft}\isactrlbold {\isasymnot}\ G{\isacharparenright}\ {\isasymin}\ S{\isachardoublequoteclose}\ \isanewline
\ \ \ \ \ \ \ \ \isacommand{have}\isamarkupfalse%
\ {\isachardoublequoteopen}Con\ {\isacharparenleft}\isactrlbold {\isasymnot}\ {\isacharparenleft}\isactrlbold {\isasymnot}\ G{\isacharparenright}{\isacharparenright}\ G\ G{\isachardoublequoteclose}\isanewline
\ \ \ \ \ \ \ \ \ \ \isacommand{by}\isamarkupfalse%
\ {\isacharparenleft}simp\ only{\isacharcolon}\ Con{\isachardot}intros{\isacharparenleft}{\isadigit{4}}{\isacharparenright}{\isacharparenright}\isanewline
\ \ \ \ \ \ \ \ \isacommand{have}\isamarkupfalse%
\ {\isachardoublequoteopen}Con\ {\isacharparenleft}\isactrlbold {\isasymnot}{\isacharparenleft}\isactrlbold {\isasymnot}\ G{\isacharparenright}{\isacharparenright}\ G\ G\ {\isasymlongrightarrow}\ {\isacharparenleft}\isactrlbold {\isasymnot}{\isacharparenleft}\isactrlbold {\isasymnot}\ G{\isacharparenright}{\isacharparenright}\ {\isasymin}\ S\ {\isasymlongrightarrow}\ G\ {\isasymin}\ S\ {\isasymand}\ G\ {\isasymin}\ S{\isachardoublequoteclose}\isanewline
\ \ \ \ \ \ \ \ \ \ \isacommand{using}\isamarkupfalse%
\ Con\ \isacommand{by}\isamarkupfalse%
\ {\isacharparenleft}iprover\ elim{\isacharcolon}\ allE{\isacharparenright}\isanewline
\ \ \ \ \ \ \ \ \isacommand{then}\isamarkupfalse%
\ \isacommand{have}\isamarkupfalse%
\ {\isachardoublequoteopen}{\isacharparenleft}\isactrlbold {\isasymnot}{\isacharparenleft}\isactrlbold {\isasymnot}\ G{\isacharparenright}{\isacharparenright}\ {\isasymin}\ S\ {\isasymlongrightarrow}\ G\ {\isasymin}\ S\ {\isasymand}\ G\ {\isasymin}\ S{\isachardoublequoteclose}\isanewline
\ \ \ \ \ \ \ \ \ \ \isacommand{using}\isamarkupfalse%
\ {\isacartoucheopen}Con\ {\isacharparenleft}\isactrlbold {\isasymnot}\ {\isacharparenleft}\isactrlbold {\isasymnot}\ G{\isacharparenright}{\isacharparenright}\ G\ G{\isacartoucheclose}\ \isacommand{by}\isamarkupfalse%
\ {\isacharparenleft}rule\ mp{\isacharparenright}\isanewline
\ \ \ \ \ \ \ \ \isacommand{then}\isamarkupfalse%
\ \isacommand{have}\isamarkupfalse%
\ {\isachardoublequoteopen}G\ {\isasymin}\ S\ {\isasymand}\ G\ {\isasymin}\ S{\isachardoublequoteclose}\isanewline
\ \ \ \ \ \ \ \ \ \ \isacommand{using}\isamarkupfalse%
\ {\isacartoucheopen}\isactrlbold {\isasymnot}\ {\isacharparenleft}\isactrlbold {\isasymnot}\ G{\isacharparenright}\ {\isasymin}\ S{\isacartoucheclose}\ \isacommand{by}\isamarkupfalse%
\ {\isacharparenleft}rule\ mp{\isacharparenright}\isanewline
\ \ \ \ \ \ \ \ \isacommand{thus}\isamarkupfalse%
\ {\isachardoublequoteopen}G\ {\isasymin}\ S{\isachardoublequoteclose}\isanewline
\ \ \ \ \ \ \ \ \ \ \isacommand{by}\isamarkupfalse%
\ {\isacharparenleft}simp\ only{\isacharcolon}\ conj{\isacharunderscore}absorb{\isacharparenright}\isanewline
\ \ \ \ \ \ \isacommand{qed}\isamarkupfalse%
\isanewline
\ \ \ \ \isacommand{qed}\isamarkupfalse%
\isanewline
\ \ \ \ \isacommand{have}\isamarkupfalse%
\ C{\isadigit{7}}{\isacharcolon}{\isachardoublequoteopen}{\isasymforall}G\ H{\isachardot}\ \isactrlbold {\isasymnot}{\isacharparenleft}G\ \isactrlbold {\isasymand}\ H{\isacharparenright}\ {\isasymin}\ S\ {\isasymlongrightarrow}\ \isactrlbold {\isasymnot}\ G\ {\isasymin}\ S\ {\isasymor}\ \isactrlbold {\isasymnot}\ H\ {\isasymin}\ S{\isachardoublequoteclose}\isanewline
\ \ \ \ \isacommand{proof}\isamarkupfalse%
\ {\isacharparenleft}rule\ allI{\isacharparenright}{\isacharplus}\isanewline
\ \ \ \ \ \ \isacommand{fix}\isamarkupfalse%
\ G\ H\isanewline
\ \ \ \ \ \ \isacommand{show}\isamarkupfalse%
\ {\isachardoublequoteopen}\isactrlbold {\isasymnot}{\isacharparenleft}G\ \isactrlbold {\isasymand}\ H{\isacharparenright}\ {\isasymin}\ S\ {\isasymlongrightarrow}\ \isactrlbold {\isasymnot}\ G\ {\isasymin}\ S\ {\isasymor}\ \isactrlbold {\isasymnot}\ H\ {\isasymin}\ S{\isachardoublequoteclose}\isanewline
\ \ \ \ \ \ \isacommand{proof}\isamarkupfalse%
\ {\isacharparenleft}rule\ impI{\isacharparenright}\isanewline
\ \ \ \ \ \ \ \ \isacommand{assume}\isamarkupfalse%
\ {\isachardoublequoteopen}\isactrlbold {\isasymnot}{\isacharparenleft}G\ \isactrlbold {\isasymand}\ H{\isacharparenright}\ {\isasymin}\ S{\isachardoublequoteclose}\isanewline
\ \ \ \ \ \ \ \ \isacommand{have}\isamarkupfalse%
\ {\isachardoublequoteopen}Dis\ {\isacharparenleft}\isactrlbold {\isasymnot}{\isacharparenleft}G\ \isactrlbold {\isasymand}\ H{\isacharparenright}{\isacharparenright}\ {\isacharparenleft}\isactrlbold {\isasymnot}\ G{\isacharparenright}\ {\isacharparenleft}\isactrlbold {\isasymnot}\ H{\isacharparenright}{\isachardoublequoteclose}\isanewline
\ \ \ \ \ \ \ \ \ \ \isacommand{by}\isamarkupfalse%
\ {\isacharparenleft}simp\ only{\isacharcolon}\ Dis{\isachardot}intros{\isacharparenleft}{\isadigit{3}}{\isacharparenright}{\isacharparenright}\isanewline
\ \ \ \ \ \ \ \ \isacommand{have}\isamarkupfalse%
\ {\isachardoublequoteopen}Dis\ {\isacharparenleft}\isactrlbold {\isasymnot}{\isacharparenleft}G\ \isactrlbold {\isasymand}\ H{\isacharparenright}{\isacharparenright}\ {\isacharparenleft}\isactrlbold {\isasymnot}\ G{\isacharparenright}\ {\isacharparenleft}\isactrlbold {\isasymnot}\ H{\isacharparenright}\ {\isasymlongrightarrow}\ \isactrlbold {\isasymnot}{\isacharparenleft}G\ \isactrlbold {\isasymand}\ H{\isacharparenright}\ {\isasymin}\ S\ {\isasymlongrightarrow}\ \isactrlbold {\isasymnot}\ G\ {\isasymin}\ S\ {\isasymor}\ \isactrlbold {\isasymnot}\ H\ {\isasymin}\ S{\isachardoublequoteclose}\isanewline
\ \ \ \ \ \ \ \ \ \ \isacommand{using}\isamarkupfalse%
\ Dis\ \isacommand{by}\isamarkupfalse%
\ {\isacharparenleft}iprover\ elim{\isacharcolon}\ allE{\isacharparenright}\isanewline
\ \ \ \ \ \ \ \ \isacommand{then}\isamarkupfalse%
\ \isacommand{have}\isamarkupfalse%
\ {\isachardoublequoteopen}\isactrlbold {\isasymnot}{\isacharparenleft}G\ \isactrlbold {\isasymand}\ H{\isacharparenright}\ {\isasymin}\ S\ {\isasymlongrightarrow}\ \isactrlbold {\isasymnot}\ G\ {\isasymin}\ S\ {\isasymor}\ \isactrlbold {\isasymnot}\ H\ {\isasymin}\ S{\isachardoublequoteclose}\isanewline
\ \ \ \ \ \ \ \ \ \ \isacommand{using}\isamarkupfalse%
\ {\isacartoucheopen}Dis\ {\isacharparenleft}\isactrlbold {\isasymnot}{\isacharparenleft}G\ \isactrlbold {\isasymand}\ H{\isacharparenright}{\isacharparenright}\ {\isacharparenleft}\isactrlbold {\isasymnot}\ G{\isacharparenright}\ {\isacharparenleft}\isactrlbold {\isasymnot}\ H{\isacharparenright}{\isacartoucheclose}\ \isacommand{by}\isamarkupfalse%
\ {\isacharparenleft}rule\ mp{\isacharparenright}\isanewline
\ \ \ \ \ \ \ \ \isacommand{thus}\isamarkupfalse%
\ {\isachardoublequoteopen}\isactrlbold {\isasymnot}\ G\ {\isasymin}\ S\ {\isasymor}\ \isactrlbold {\isasymnot}\ H\ {\isasymin}\ S{\isachardoublequoteclose}\isanewline
\ \ \ \ \ \ \ \ \ \ \isacommand{using}\isamarkupfalse%
\ {\isacartoucheopen}\isactrlbold {\isasymnot}{\isacharparenleft}G\ \isactrlbold {\isasymand}\ H{\isacharparenright}\ {\isasymin}\ S{\isacartoucheclose}\ \isacommand{by}\isamarkupfalse%
\ {\isacharparenleft}rule\ mp{\isacharparenright}\isanewline
\ \ \ \ \ \ \isacommand{qed}\isamarkupfalse%
\isanewline
\ \ \ \ \isacommand{qed}\isamarkupfalse%
\isanewline
\ \ \ \ \isacommand{have}\isamarkupfalse%
\ C{\isadigit{8}}{\isacharcolon}{\isachardoublequoteopen}{\isasymforall}G\ H{\isachardot}\ \isactrlbold {\isasymnot}{\isacharparenleft}G\ \isactrlbold {\isasymor}\ H{\isacharparenright}\ {\isasymin}\ S\ {\isasymlongrightarrow}\ \isactrlbold {\isasymnot}\ G\ {\isasymin}\ S\ {\isasymand}\ \isactrlbold {\isasymnot}\ H\ {\isasymin}\ S{\isachardoublequoteclose}\isanewline
\ \ \ \ \isacommand{proof}\isamarkupfalse%
\ {\isacharparenleft}rule\ allI{\isacharparenright}{\isacharplus}\isanewline
\ \ \ \ \ \ \isacommand{fix}\isamarkupfalse%
\ G\ H\isanewline
\ \ \ \ \ \ \isacommand{show}\isamarkupfalse%
\ {\isachardoublequoteopen}\isactrlbold {\isasymnot}{\isacharparenleft}G\ \isactrlbold {\isasymor}\ H{\isacharparenright}\ {\isasymin}\ S\ {\isasymlongrightarrow}\ \isactrlbold {\isasymnot}\ G\ {\isasymin}\ S\ {\isasymand}\ \isactrlbold {\isasymnot}\ H\ {\isasymin}\ S{\isachardoublequoteclose}\isanewline
\ \ \ \ \ \ \isacommand{proof}\isamarkupfalse%
\ {\isacharparenleft}rule\ impI{\isacharparenright}\isanewline
\ \ \ \ \ \ \ \ \isacommand{assume}\isamarkupfalse%
\ {\isachardoublequoteopen}\isactrlbold {\isasymnot}{\isacharparenleft}G\ \isactrlbold {\isasymor}\ H{\isacharparenright}\ {\isasymin}\ S{\isachardoublequoteclose}\isanewline
\ \ \ \ \ \ \ \ \isacommand{have}\isamarkupfalse%
\ {\isachardoublequoteopen}Con\ {\isacharparenleft}\isactrlbold {\isasymnot}{\isacharparenleft}G\ \isactrlbold {\isasymor}\ H{\isacharparenright}{\isacharparenright}\ {\isacharparenleft}\isactrlbold {\isasymnot}\ G{\isacharparenright}\ {\isacharparenleft}\isactrlbold {\isasymnot}\ H{\isacharparenright}{\isachardoublequoteclose}\isanewline
\ \ \ \ \ \ \ \ \ \ \isacommand{by}\isamarkupfalse%
\ {\isacharparenleft}simp\ only{\isacharcolon}\ Con{\isachardot}intros{\isacharparenleft}{\isadigit{2}}{\isacharparenright}{\isacharparenright}\isanewline
\ \ \ \ \ \ \ \ \isacommand{have}\isamarkupfalse%
\ {\isachardoublequoteopen}Con\ {\isacharparenleft}\isactrlbold {\isasymnot}{\isacharparenleft}G\ \isactrlbold {\isasymor}\ H{\isacharparenright}{\isacharparenright}\ {\isacharparenleft}\isactrlbold {\isasymnot}\ G{\isacharparenright}\ {\isacharparenleft}\isactrlbold {\isasymnot}\ H{\isacharparenright}\ {\isasymlongrightarrow}\ \isactrlbold {\isasymnot}{\isacharparenleft}G\ \isactrlbold {\isasymor}\ H{\isacharparenright}\ {\isasymin}\ S\ {\isasymlongrightarrow}\ \isactrlbold {\isasymnot}\ G\ {\isasymin}\ S\ {\isasymand}\ \isactrlbold {\isasymnot}\ H\ {\isasymin}\ S{\isachardoublequoteclose}\isanewline
\ \ \ \ \ \ \ \ \ \ \isacommand{using}\isamarkupfalse%
\ Con\ \isacommand{by}\isamarkupfalse%
\ {\isacharparenleft}iprover\ elim{\isacharcolon}\ allE{\isacharparenright}\isanewline
\ \ \ \ \ \ \ \ \isacommand{then}\isamarkupfalse%
\ \isacommand{have}\isamarkupfalse%
\ {\isachardoublequoteopen}\isactrlbold {\isasymnot}{\isacharparenleft}G\ \isactrlbold {\isasymor}\ H{\isacharparenright}\ {\isasymin}\ S\ {\isasymlongrightarrow}\ \isactrlbold {\isasymnot}\ G\ {\isasymin}\ S\ {\isasymand}\ \isactrlbold {\isasymnot}\ H\ {\isasymin}\ S{\isachardoublequoteclose}\isanewline
\ \ \ \ \ \ \ \ \ \ \isacommand{using}\isamarkupfalse%
\ {\isacartoucheopen}Con\ {\isacharparenleft}\isactrlbold {\isasymnot}{\isacharparenleft}G\ \isactrlbold {\isasymor}\ H{\isacharparenright}{\isacharparenright}\ {\isacharparenleft}\isactrlbold {\isasymnot}\ G{\isacharparenright}\ {\isacharparenleft}\isactrlbold {\isasymnot}\ H{\isacharparenright}{\isacartoucheclose}\ \isacommand{by}\isamarkupfalse%
\ {\isacharparenleft}rule\ mp{\isacharparenright}\isanewline
\ \ \ \ \ \ \ \ \isacommand{thus}\isamarkupfalse%
\ {\isachardoublequoteopen}\isactrlbold {\isasymnot}\ G\ {\isasymin}\ S\ {\isasymand}\ \isactrlbold {\isasymnot}\ H\ {\isasymin}\ S{\isachardoublequoteclose}\isanewline
\ \ \ \ \ \ \ \ \ \ \isacommand{using}\isamarkupfalse%
\ {\isacartoucheopen}\isactrlbold {\isasymnot}{\isacharparenleft}G\ \isactrlbold {\isasymor}\ H{\isacharparenright}\ {\isasymin}\ S{\isacartoucheclose}\ \isacommand{by}\isamarkupfalse%
\ {\isacharparenleft}rule\ mp{\isacharparenright}\isanewline
\ \ \ \ \ \ \isacommand{qed}\isamarkupfalse%
\isanewline
\ \ \ \ \isacommand{qed}\isamarkupfalse%
\isanewline
\ \ \ \ \isacommand{have}\isamarkupfalse%
\ C{\isadigit{9}}{\isacharcolon}{\isachardoublequoteopen}{\isasymforall}G\ H{\isachardot}\ \isactrlbold {\isasymnot}{\isacharparenleft}G\ \isactrlbold {\isasymrightarrow}\ H{\isacharparenright}\ {\isasymin}\ S\ {\isasymlongrightarrow}\ G\ {\isasymin}\ S\ {\isasymand}\ \isactrlbold {\isasymnot}\ H\ {\isasymin}\ S{\isachardoublequoteclose}\isanewline
\ \ \ \ \isacommand{proof}\isamarkupfalse%
\ {\isacharparenleft}rule\ allI{\isacharparenright}{\isacharplus}\isanewline
\ \ \ \ \ \ \isacommand{fix}\isamarkupfalse%
\ G\ H\isanewline
\ \ \ \ \ \ \isacommand{show}\isamarkupfalse%
\ {\isachardoublequoteopen}\isactrlbold {\isasymnot}{\isacharparenleft}G\ \isactrlbold {\isasymrightarrow}\ H{\isacharparenright}\ {\isasymin}\ S\ {\isasymlongrightarrow}\ G\ {\isasymin}\ S\ {\isasymand}\ \isactrlbold {\isasymnot}\ H\ {\isasymin}\ S{\isachardoublequoteclose}\isanewline
\ \ \ \ \ \ \isacommand{proof}\isamarkupfalse%
\ {\isacharparenleft}rule\ impI{\isacharparenright}\isanewline
\ \ \ \ \ \ \ \ \isacommand{assume}\isamarkupfalse%
\ {\isachardoublequoteopen}\isactrlbold {\isasymnot}{\isacharparenleft}G\ \isactrlbold {\isasymrightarrow}\ H{\isacharparenright}\ {\isasymin}\ S{\isachardoublequoteclose}\isanewline
\ \ \ \ \ \ \ \ \isacommand{have}\isamarkupfalse%
\ {\isachardoublequoteopen}Con\ {\isacharparenleft}\isactrlbold {\isasymnot}{\isacharparenleft}G\ \isactrlbold {\isasymrightarrow}\ H{\isacharparenright}{\isacharparenright}\ G\ {\isacharparenleft}\isactrlbold {\isasymnot}\ H{\isacharparenright}{\isachardoublequoteclose}\isanewline
\ \ \ \ \ \ \ \ \ \ \isacommand{by}\isamarkupfalse%
\ {\isacharparenleft}simp\ only{\isacharcolon}\ Con{\isachardot}intros{\isacharparenleft}{\isadigit{3}}{\isacharparenright}{\isacharparenright}\isanewline
\ \ \ \ \ \ \ \ \isacommand{have}\isamarkupfalse%
\ {\isachardoublequoteopen}Con\ {\isacharparenleft}\isactrlbold {\isasymnot}{\isacharparenleft}G\ \isactrlbold {\isasymrightarrow}\ H{\isacharparenright}{\isacharparenright}\ G\ {\isacharparenleft}\isactrlbold {\isasymnot}\ H{\isacharparenright}\ {\isasymlongrightarrow}\ \isactrlbold {\isasymnot}{\isacharparenleft}G\ \isactrlbold {\isasymrightarrow}\ H{\isacharparenright}\ {\isasymin}\ S\ {\isasymlongrightarrow}\ G\ {\isasymin}\ S\ {\isasymand}\ \isactrlbold {\isasymnot}\ H\ {\isasymin}\ S{\isachardoublequoteclose}\isanewline
\ \ \ \ \ \ \ \ \ \ \isacommand{using}\isamarkupfalse%
\ Con\ \isacommand{by}\isamarkupfalse%
\ {\isacharparenleft}iprover\ elim{\isacharcolon}\ allE{\isacharparenright}\isanewline
\ \ \ \ \ \ \ \ \isacommand{then}\isamarkupfalse%
\ \isacommand{have}\isamarkupfalse%
\ {\isachardoublequoteopen}\isactrlbold {\isasymnot}{\isacharparenleft}G\ \isactrlbold {\isasymrightarrow}\ H{\isacharparenright}\ {\isasymin}\ S\ {\isasymlongrightarrow}\ G\ {\isasymin}\ S\ {\isasymand}\ \isactrlbold {\isasymnot}\ H\ {\isasymin}\ S{\isachardoublequoteclose}\isanewline
\ \ \ \ \ \ \ \ \ \ \isacommand{using}\isamarkupfalse%
\ {\isacartoucheopen}Con\ {\isacharparenleft}\isactrlbold {\isasymnot}{\isacharparenleft}G\ \isactrlbold {\isasymrightarrow}\ H{\isacharparenright}{\isacharparenright}\ G\ {\isacharparenleft}\isactrlbold {\isasymnot}\ H{\isacharparenright}{\isacartoucheclose}\ \isacommand{by}\isamarkupfalse%
\ {\isacharparenleft}rule\ mp{\isacharparenright}\isanewline
\ \ \ \ \ \ \ \ \isacommand{thus}\isamarkupfalse%
\ {\isachardoublequoteopen}G\ {\isasymin}\ S\ {\isasymand}\ \isactrlbold {\isasymnot}\ H\ {\isasymin}\ S{\isachardoublequoteclose}\isanewline
\ \ \ \ \ \ \ \ \ \ \isacommand{using}\isamarkupfalse%
\ {\isacartoucheopen}\isactrlbold {\isasymnot}{\isacharparenleft}G\ \isactrlbold {\isasymrightarrow}\ H{\isacharparenright}\ {\isasymin}\ S{\isacartoucheclose}\ \isacommand{by}\isamarkupfalse%
\ {\isacharparenleft}rule\ mp{\isacharparenright}\isanewline
\ \ \ \ \ \ \isacommand{qed}\isamarkupfalse%
\isanewline
\ \ \ \ \isacommand{qed}\isamarkupfalse%
\isanewline
\ \ \ \ \isacommand{have}\isamarkupfalse%
\ A{\isacharcolon}{\isachardoublequoteopen}{\isasymbottom}\ {\isasymnotin}\ S\isanewline
\ \ \ \ {\isasymand}\ {\isacharparenleft}{\isasymforall}k{\isachardot}\ Atom\ k\ {\isasymin}\ S\ {\isasymlongrightarrow}\ \isactrlbold {\isasymnot}\ {\isacharparenleft}Atom\ k{\isacharparenright}\ {\isasymin}\ S\ {\isasymlongrightarrow}\ False{\isacharparenright}\isanewline
\ \ \ \ {\isasymand}\ {\isacharparenleft}{\isasymforall}G\ H{\isachardot}\ G\ \isactrlbold {\isasymand}\ H\ {\isasymin}\ S\ {\isasymlongrightarrow}\ G\ {\isasymin}\ S\ {\isasymand}\ H\ {\isasymin}\ S{\isacharparenright}\isanewline
\ \ \ \ {\isasymand}\ {\isacharparenleft}{\isasymforall}G\ H{\isachardot}\ G\ \isactrlbold {\isasymor}\ H\ {\isasymin}\ S\ {\isasymlongrightarrow}\ G\ {\isasymin}\ S\ {\isasymor}\ H\ {\isasymin}\ S{\isacharparenright}\isanewline
\ \ \ \ {\isasymand}\ {\isacharparenleft}{\isasymforall}G\ H{\isachardot}\ G\ \isactrlbold {\isasymrightarrow}\ H\ {\isasymin}\ S\ {\isasymlongrightarrow}\ \isactrlbold {\isasymnot}G\ {\isasymin}\ S\ {\isasymor}\ H\ {\isasymin}\ S{\isacharparenright}{\isachardoublequoteclose}\isanewline
\ \ \ \ \ \ \isacommand{using}\isamarkupfalse%
\ C{\isadigit{1}}\ C{\isadigit{2}}\ C{\isadigit{3}}\ C{\isadigit{4}}\ C{\isadigit{5}}\ \isacommand{by}\isamarkupfalse%
\ {\isacharparenleft}iprover\ intro{\isacharcolon}\ conjI{\isacharparenright}\isanewline
\ \ \ \ \isacommand{have}\isamarkupfalse%
\ B{\isacharcolon}{\isachardoublequoteopen}{\isacharparenleft}{\isasymforall}G{\isachardot}\ \isactrlbold {\isasymnot}\ {\isacharparenleft}\isactrlbold {\isasymnot}G{\isacharparenright}\ {\isasymin}\ S\ {\isasymlongrightarrow}\ G\ {\isasymin}\ S{\isacharparenright}\isanewline
\ \ \ \ {\isasymand}\ {\isacharparenleft}{\isasymforall}G\ H{\isachardot}\ \isactrlbold {\isasymnot}{\isacharparenleft}G\ \isactrlbold {\isasymand}\ H{\isacharparenright}\ {\isasymin}\ S\ {\isasymlongrightarrow}\ \isactrlbold {\isasymnot}\ G\ {\isasymin}\ S\ {\isasymor}\ \isactrlbold {\isasymnot}\ H\ {\isasymin}\ S{\isacharparenright}\isanewline
\ \ \ \ {\isasymand}\ {\isacharparenleft}{\isasymforall}G\ H{\isachardot}\ \isactrlbold {\isasymnot}{\isacharparenleft}G\ \isactrlbold {\isasymor}\ H{\isacharparenright}\ {\isasymin}\ S\ {\isasymlongrightarrow}\ \isactrlbold {\isasymnot}\ G\ {\isasymin}\ S\ {\isasymand}\ \isactrlbold {\isasymnot}\ H\ {\isasymin}\ S{\isacharparenright}\isanewline
\ \ \ \ {\isasymand}\ {\isacharparenleft}{\isasymforall}G\ H{\isachardot}\ \isactrlbold {\isasymnot}{\isacharparenleft}G\ \isactrlbold {\isasymrightarrow}\ H{\isacharparenright}\ {\isasymin}\ S\ {\isasymlongrightarrow}\ G\ {\isasymin}\ S\ {\isasymand}\ \isactrlbold {\isasymnot}\ H\ {\isasymin}\ S{\isacharparenright}{\isachardoublequoteclose}\isanewline
\ \ \ \ \ \ \isacommand{using}\isamarkupfalse%
\ C{\isadigit{6}}\ C{\isadigit{7}}\ C{\isadigit{8}}\ C{\isadigit{9}}\ \isacommand{by}\isamarkupfalse%
\ {\isacharparenleft}iprover\ intro{\isacharcolon}\ conjI{\isacharparenright}\isanewline
\ \ \ \ \isacommand{have}\isamarkupfalse%
\ {\isachardoublequoteopen}{\isacharparenleft}{\isasymbottom}\ {\isasymnotin}\ S\isanewline
\ \ \ \ {\isasymand}\ {\isacharparenleft}{\isasymforall}k{\isachardot}\ Atom\ k\ {\isasymin}\ S\ {\isasymlongrightarrow}\ \isactrlbold {\isasymnot}\ {\isacharparenleft}Atom\ k{\isacharparenright}\ {\isasymin}\ S\ {\isasymlongrightarrow}\ False{\isacharparenright}\isanewline
\ \ \ \ {\isasymand}\ {\isacharparenleft}{\isasymforall}G\ H{\isachardot}\ G\ \isactrlbold {\isasymand}\ H\ {\isasymin}\ S\ {\isasymlongrightarrow}\ G\ {\isasymin}\ S\ {\isasymand}\ H\ {\isasymin}\ S{\isacharparenright}\isanewline
\ \ \ \ {\isasymand}\ {\isacharparenleft}{\isasymforall}G\ H{\isachardot}\ G\ \isactrlbold {\isasymor}\ H\ {\isasymin}\ S\ {\isasymlongrightarrow}\ G\ {\isasymin}\ S\ {\isasymor}\ H\ {\isasymin}\ S{\isacharparenright}\isanewline
\ \ \ \ {\isasymand}\ {\isacharparenleft}{\isasymforall}G\ H{\isachardot}\ G\ \isactrlbold {\isasymrightarrow}\ H\ {\isasymin}\ S\ {\isasymlongrightarrow}\ \isactrlbold {\isasymnot}G\ {\isasymin}\ S\ {\isasymor}\ H\ {\isasymin}\ S{\isacharparenright}{\isacharparenright}\isanewline
\ \ \ \ {\isasymand}\ {\isacharparenleft}{\isacharparenleft}{\isasymforall}G{\isachardot}\ \isactrlbold {\isasymnot}\ {\isacharparenleft}\isactrlbold {\isasymnot}G{\isacharparenright}\ {\isasymin}\ S\ {\isasymlongrightarrow}\ G\ {\isasymin}\ S{\isacharparenright}\isanewline
\ \ \ \ {\isasymand}\ {\isacharparenleft}{\isasymforall}G\ H{\isachardot}\ \isactrlbold {\isasymnot}{\isacharparenleft}G\ \isactrlbold {\isasymand}\ H{\isacharparenright}\ {\isasymin}\ S\ {\isasymlongrightarrow}\ \isactrlbold {\isasymnot}\ G\ {\isasymin}\ S\ {\isasymor}\ \isactrlbold {\isasymnot}\ H\ {\isasymin}\ S{\isacharparenright}\isanewline
\ \ \ \ {\isasymand}\ {\isacharparenleft}{\isasymforall}G\ H{\isachardot}\ \isactrlbold {\isasymnot}{\isacharparenleft}G\ \isactrlbold {\isasymor}\ H{\isacharparenright}\ {\isasymin}\ S\ {\isasymlongrightarrow}\ \isactrlbold {\isasymnot}\ G\ {\isasymin}\ S\ {\isasymand}\ \isactrlbold {\isasymnot}\ H\ {\isasymin}\ S{\isacharparenright}\isanewline
\ \ \ \ {\isasymand}\ {\isacharparenleft}{\isasymforall}G\ H{\isachardot}\ \isactrlbold {\isasymnot}{\isacharparenleft}G\ \isactrlbold {\isasymrightarrow}\ H{\isacharparenright}\ {\isasymin}\ S\ {\isasymlongrightarrow}\ G\ {\isasymin}\ S\ {\isasymand}\ \isactrlbold {\isasymnot}\ H\ {\isasymin}\ S{\isacharparenright}{\isacharparenright}{\isachardoublequoteclose}\isanewline
\ \ \ \ \ \ \isacommand{using}\isamarkupfalse%
\ A\ B\ \isacommand{by}\isamarkupfalse%
\ {\isacharparenleft}rule\ conjI{\isacharparenright}\isanewline
\ \ \ \ \isacommand{thus}\isamarkupfalse%
\ {\isachardoublequoteopen}{\isasymbottom}\ {\isasymnotin}\ S\isanewline
\ \ \ \ {\isasymand}\ {\isacharparenleft}{\isasymforall}k{\isachardot}\ Atom\ k\ {\isasymin}\ S\ {\isasymlongrightarrow}\ \isactrlbold {\isasymnot}\ {\isacharparenleft}Atom\ k{\isacharparenright}\ {\isasymin}\ S\ {\isasymlongrightarrow}\ False{\isacharparenright}\isanewline
\ \ \ \ {\isasymand}\ {\isacharparenleft}{\isasymforall}G\ H{\isachardot}\ G\ \isactrlbold {\isasymand}\ H\ {\isasymin}\ S\ {\isasymlongrightarrow}\ G\ {\isasymin}\ S\ {\isasymand}\ H\ {\isasymin}\ S{\isacharparenright}\isanewline
\ \ \ \ {\isasymand}\ {\isacharparenleft}{\isasymforall}G\ H{\isachardot}\ G\ \isactrlbold {\isasymor}\ H\ {\isasymin}\ S\ {\isasymlongrightarrow}\ G\ {\isasymin}\ S\ {\isasymor}\ H\ {\isasymin}\ S{\isacharparenright}\isanewline
\ \ \ \ {\isasymand}\ {\isacharparenleft}{\isasymforall}G\ H{\isachardot}\ G\ \isactrlbold {\isasymrightarrow}\ H\ {\isasymin}\ S\ {\isasymlongrightarrow}\ \isactrlbold {\isasymnot}G\ {\isasymin}\ S\ {\isasymor}\ H\ {\isasymin}\ S{\isacharparenright}\isanewline
\ \ \ \ {\isasymand}\ {\isacharparenleft}{\isasymforall}G{\isachardot}\ \isactrlbold {\isasymnot}\ {\isacharparenleft}\isactrlbold {\isasymnot}G{\isacharparenright}\ {\isasymin}\ S\ {\isasymlongrightarrow}\ G\ {\isasymin}\ S{\isacharparenright}\isanewline
\ \ \ \ {\isasymand}\ {\isacharparenleft}{\isasymforall}G\ H{\isachardot}\ \isactrlbold {\isasymnot}{\isacharparenleft}G\ \isactrlbold {\isasymand}\ H{\isacharparenright}\ {\isasymin}\ S\ {\isasymlongrightarrow}\ \isactrlbold {\isasymnot}\ G\ {\isasymin}\ S\ {\isasymor}\ \isactrlbold {\isasymnot}\ H\ {\isasymin}\ S{\isacharparenright}\isanewline
\ \ \ \ {\isasymand}\ {\isacharparenleft}{\isasymforall}G\ H{\isachardot}\ \isactrlbold {\isasymnot}{\isacharparenleft}G\ \isactrlbold {\isasymor}\ H{\isacharparenright}\ {\isasymin}\ S\ {\isasymlongrightarrow}\ \isactrlbold {\isasymnot}\ G\ {\isasymin}\ S\ {\isasymand}\ \isactrlbold {\isasymnot}\ H\ {\isasymin}\ S{\isacharparenright}\isanewline
\ \ \ \ {\isasymand}\ {\isacharparenleft}{\isasymforall}G\ H{\isachardot}\ \isactrlbold {\isasymnot}{\isacharparenleft}G\ \isactrlbold {\isasymrightarrow}\ H{\isacharparenright}\ {\isasymin}\ S\ {\isasymlongrightarrow}\ G\ {\isasymin}\ S\ {\isasymand}\ \isactrlbold {\isasymnot}\ H\ {\isasymin}\ S{\isacharparenright}{\isachardoublequoteclose}\ \isanewline
\ \ \ \ \ \ \isacommand{by}\isamarkupfalse%
\ {\isacharparenleft}iprover\ intro{\isacharcolon}\ conj{\isacharunderscore}assoc{\isacharparenright}\isanewline
\ \ \isacommand{qed}\isamarkupfalse%
\isanewline
\ \ \isacommand{thus}\isamarkupfalse%
\ {\isachardoublequoteopen}Hintikka\ S{\isachardoublequoteclose}\isanewline
\ \ \ \ \isacommand{unfolding}\isamarkupfalse%
\ Hintikka{\isacharunderscore}def\ \isacommand{by}\isamarkupfalse%
\ this\isanewline
\isacommand{qed}\isamarkupfalse%
%
\endisatagproof
{\isafoldproof}%
%
\isadelimproof
%
\endisadelimproof
%
\begin{isamarkuptext}%
En conclusión, el lema completo se demuestra detalladamente en Isabelle/HOL como sigue.%
\end{isamarkuptext}\isamarkuptrue%
\isacommand{lemma}\isamarkupfalse%
\ {\isachardoublequoteopen}Hintikka\ S\ {\isacharequal}\ {\isacharparenleft}{\isasymbottom}\ {\isasymnotin}\ S\isanewline
{\isasymand}\ {\isacharparenleft}{\isasymforall}k{\isachardot}\ Atom\ k\ {\isasymin}\ S\ {\isasymlongrightarrow}\ \isactrlbold {\isasymnot}\ {\isacharparenleft}Atom\ k{\isacharparenright}\ {\isasymin}\ S\ {\isasymlongrightarrow}\ False{\isacharparenright}\isanewline
{\isasymand}\ {\isacharparenleft}{\isasymforall}F\ G\ H{\isachardot}\ Con\ F\ G\ H\ {\isasymlongrightarrow}\ F\ {\isasymin}\ S\ {\isasymlongrightarrow}\ G\ {\isasymin}\ S\ {\isasymand}\ H\ {\isasymin}\ S{\isacharparenright}\isanewline
{\isasymand}\ {\isacharparenleft}{\isasymforall}F\ G\ H{\isachardot}\ Dis\ F\ G\ H\ {\isasymlongrightarrow}\ F\ {\isasymin}\ S\ {\isasymlongrightarrow}\ G\ {\isasymin}\ S\ {\isasymor}\ H\ {\isasymin}\ S{\isacharparenright}{\isacharparenright}{\isachardoublequoteclose}\ \ \isanewline
%
\isadelimproof
%
\endisadelimproof
%
\isatagproof
\isacommand{proof}\isamarkupfalse%
\ {\isacharparenleft}rule\ iffI{\isacharparenright}\isanewline
\ \ \isacommand{assume}\isamarkupfalse%
\ {\isachardoublequoteopen}Hintikka\ S{\isachardoublequoteclose}\isanewline
\ \ \isacommand{thus}\isamarkupfalse%
\ {\isachardoublequoteopen}{\isacharparenleft}{\isasymbottom}\ {\isasymnotin}\ S\isanewline
\ \ {\isasymand}\ {\isacharparenleft}{\isasymforall}k{\isachardot}\ Atom\ k\ {\isasymin}\ S\ {\isasymlongrightarrow}\ \isactrlbold {\isasymnot}\ {\isacharparenleft}Atom\ k{\isacharparenright}\ {\isasymin}\ S\ {\isasymlongrightarrow}\ False{\isacharparenright}\isanewline
\ \ {\isasymand}\ {\isacharparenleft}{\isasymforall}F\ G\ H{\isachardot}\ Con\ F\ G\ H\ {\isasymlongrightarrow}\ F\ {\isasymin}\ S\ {\isasymlongrightarrow}\ G\ {\isasymin}\ S\ {\isasymand}\ H\ {\isasymin}\ S{\isacharparenright}\isanewline
\ \ {\isasymand}\ {\isacharparenleft}{\isasymforall}F\ G\ H{\isachardot}\ Dis\ F\ G\ H\ {\isasymlongrightarrow}\ F\ {\isasymin}\ S\ {\isasymlongrightarrow}\ G\ {\isasymin}\ S\ {\isasymor}\ H\ {\isasymin}\ S{\isacharparenright}{\isacharparenright}{\isachardoublequoteclose}\isanewline
\ \ \ \ \isacommand{by}\isamarkupfalse%
\ {\isacharparenleft}rule\ Hintikka{\isacharunderscore}alt{\isadigit{1}}{\isacharparenright}\isanewline
\isacommand{next}\isamarkupfalse%
\isanewline
\ \ \isacommand{assume}\isamarkupfalse%
\ {\isachardoublequoteopen}{\isacharparenleft}{\isasymbottom}\ {\isasymnotin}\ S\isanewline
\ \ {\isasymand}\ {\isacharparenleft}{\isasymforall}k{\isachardot}\ Atom\ k\ {\isasymin}\ S\ {\isasymlongrightarrow}\ \isactrlbold {\isasymnot}\ {\isacharparenleft}Atom\ k{\isacharparenright}\ {\isasymin}\ S\ {\isasymlongrightarrow}\ False{\isacharparenright}\isanewline
\ \ {\isasymand}\ {\isacharparenleft}{\isasymforall}F\ G\ H{\isachardot}\ Con\ F\ G\ H\ {\isasymlongrightarrow}\ F\ {\isasymin}\ S\ {\isasymlongrightarrow}\ G\ {\isasymin}\ S\ {\isasymand}\ H\ {\isasymin}\ S{\isacharparenright}\isanewline
\ \ {\isasymand}\ {\isacharparenleft}{\isasymforall}F\ G\ H{\isachardot}\ Dis\ F\ G\ H\ {\isasymlongrightarrow}\ F\ {\isasymin}\ S\ {\isasymlongrightarrow}\ G\ {\isasymin}\ S\ {\isasymor}\ H\ {\isasymin}\ S{\isacharparenright}{\isacharparenright}{\isachardoublequoteclose}\isanewline
\ \ \isacommand{thus}\isamarkupfalse%
\ {\isachardoublequoteopen}Hintikka\ S{\isachardoublequoteclose}\isanewline
\ \ \ \ \isacommand{by}\isamarkupfalse%
\ {\isacharparenleft}rule\ Hintikka{\isacharunderscore}alt{\isadigit{2}}{\isacharparenright}\isanewline
\isacommand{qed}\isamarkupfalse%
%
\endisatagproof
{\isafoldproof}%
%
\isadelimproof
%
\endisadelimproof
%
\begin{isamarkuptext}%
Por último, veamos su demostración automática.%
\end{isamarkuptext}\isamarkuptrue%
\isacommand{lemma}\isamarkupfalse%
\ Hintikka{\isacharunderscore}alt{\isacharcolon}\ {\isachardoublequoteopen}Hintikka\ S\ {\isacharequal}\ {\isacharparenleft}{\isasymbottom}\ {\isasymnotin}\ S\isanewline
{\isasymand}\ {\isacharparenleft}{\isasymforall}k{\isachardot}\ Atom\ k\ {\isasymin}\ S\ {\isasymlongrightarrow}\ \isactrlbold {\isasymnot}\ {\isacharparenleft}Atom\ k{\isacharparenright}\ {\isasymin}\ S\ {\isasymlongrightarrow}\ False{\isacharparenright}\isanewline
{\isasymand}\ {\isacharparenleft}{\isasymforall}F\ G\ H{\isachardot}\ Con\ F\ G\ H\ {\isasymlongrightarrow}\ F\ {\isasymin}\ S\ {\isasymlongrightarrow}\ G\ {\isasymin}\ S\ {\isasymand}\ H\ {\isasymin}\ S{\isacharparenright}\isanewline
{\isasymand}\ {\isacharparenleft}{\isasymforall}F\ G\ H{\isachardot}\ Dis\ F\ G\ H\ {\isasymlongrightarrow}\ F\ {\isasymin}\ S\ {\isasymlongrightarrow}\ G\ {\isasymin}\ S\ {\isasymor}\ H\ {\isasymin}\ S{\isacharparenright}{\isacharparenright}{\isachardoublequoteclose}\ \ \isanewline
%
\isadelimproof
\ \ %
\endisadelimproof
%
\isatagproof
\isacommand{apply}\isamarkupfalse%
{\isacharparenleft}simp\ add{\isacharcolon}\ Hintikka{\isacharunderscore}def\ con{\isacharunderscore}dis{\isacharunderscore}simps{\isacharparenright}\isanewline
\ \ \isacommand{apply}\isamarkupfalse%
{\isacharparenleft}rule\ iffI{\isacharparenright}\isanewline
\ \ \ \isacommand{subgoal}\isamarkupfalse%
\ \isacommand{by}\isamarkupfalse%
\ blast\isanewline
\ \ \isacommand{subgoal}\isamarkupfalse%
\ \isacommand{by}\isamarkupfalse%
\ safe\ metis{\isacharplus}\isanewline
\ \ \isacommand{done}\isamarkupfalse%
\isanewline
%
\endisatagproof
{\isafoldproof}%
%
\isadelimproof
%
\endisadelimproof
%
\isadelimtheory
%
\endisadelimtheory
%
\isatagtheory
%
\endisatagtheory
{\isafoldtheory}%
%
\isadelimtheory
%
\endisadelimtheory
%
\end{isabellebody}%
\endinput
%:%file=~/TFM/TFM/Notunif.thy%:%
%:%19=10%:%
%:%20=11%:%
%:%21=12%:%
%:%22=13%:%
%:%23=14%:%
%:%24=15%:%
%:%25=16%:%
%:%26=17%:%
%:%27=18%:%
%:%28=19%:%
%:%29=20%:%
%:%30=21%:%
%:%31=22%:%
%:%32=23%:%
%:%33=24%:%
%:%34=25%:%
%:%35=26%:%
%:%36=27%:%
%:%38=29%:%
%:%39=29%:%
%:%41=31%:%
%:%42=32%:%
%:%44=34%:%
%:%45=34%:%
%:%48=35%:%
%:%52=35%:%
%:%53=35%:%
%:%58=35%:%
%:%61=36%:%
%:%62=37%:%
%:%63=37%:%
%:%66=38%:%
%:%70=38%:%
%:%71=38%:%
%:%76=38%:%
%:%79=39%:%
%:%80=40%:%
%:%81=40%:%
%:%84=41%:%
%:%88=41%:%
%:%89=41%:%
%:%94=41%:%
%:%97=42%:%
%:%98=43%:%
%:%99=43%:%
%:%102=44%:%
%:%106=44%:%
%:%107=44%:%
%:%112=44%:%
%:%115=45%:%
%:%116=46%:%
%:%117=46%:%
%:%120=47%:%
%:%124=47%:%
%:%125=47%:%
%:%130=47%:%
%:%133=48%:%
%:%134=49%:%
%:%135=49%:%
%:%138=50%:%
%:%142=50%:%
%:%143=50%:%
%:%148=50%:%
%:%151=51%:%
%:%152=52%:%
%:%153=52%:%
%:%156=53%:%
%:%160=53%:%
%:%161=53%:%
%:%166=53%:%
%:%169=54%:%
%:%170=55%:%
%:%171=55%:%
%:%174=56%:%
%:%178=56%:%
%:%179=56%:%
%:%184=56%:%
%:%187=57%:%
%:%188=58%:%
%:%189=58%:%
%:%192=59%:%
%:%196=59%:%
%:%197=59%:%
%:%202=59%:%
%:%205=60%:%
%:%206=61%:%
%:%207=61%:%
%:%210=62%:%
%:%214=62%:%
%:%215=62%:%
%:%224=64%:%
%:%225=65%:%
%:%227=67%:%
%:%228=67%:%
%:%231=68%:%
%:%235=68%:%
%:%236=68%:%
%:%241=68%:%
%:%244=69%:%
%:%245=70%:%
%:%246=70%:%
%:%249=71%:%
%:%253=71%:%
%:%254=71%:%
%:%263=73%:%
%:%264=74%:%
%:%265=75%:%
%:%266=76%:%
%:%267=77%:%
%:%268=78%:%
%:%269=79%:%
%:%270=80%:%
%:%271=81%:%
%:%272=82%:%
%:%273=83%:%
%:%274=84%:%
%:%275=85%:%
%:%276=86%:%
%:%277=87%:%
%:%278=88%:%
%:%279=89%:%
%:%280=90%:%
%:%281=91%:%
%:%282=92%:%
%:%283=93%:%
%:%284=94%:%
%:%285=95%:%
%:%286=96%:%
%:%287=97%:%
%:%288=98%:%
%:%290=100%:%
%:%291=100%:%
%:%292=101%:%
%:%293=102%:%
%:%294=103%:%
%:%295=104%:%
%:%297=106%:%
%:%298=107%:%
%:%299=108%:%
%:%300=109%:%
%:%301=110%:%
%:%304=110%:%
%:%305=111%:%
%:%306=112%:%
%:%307=113%:%
%:%308=114%:%
%:%309=115%:%
%:%310=116%:%
%:%311=117%:%
%:%312=118%:%
%:%313=119%:%
%:%314=120%:%
%:%315=121%:%
%:%316=122%:%
%:%317=123%:%
%:%318=124%:%
%:%319=125%:%
%:%320=126%:%
%:%321=127%:%
%:%322=128%:%
%:%323=129%:%
%:%324=130%:%
%:%325=131%:%
%:%326=132%:%
%:%327=133%:%
%:%329=135%:%
%:%330=135%:%
%:%331=136%:%
%:%332=137%:%
%:%333=138%:%
%:%334=139%:%
%:%336=141%:%
%:%337=142%:%
%:%338=143%:%
%:%339=144%:%
%:%342=144%:%
%:%343=145%:%
%:%344=146%:%
%:%345=147%:%
%:%346=148%:%
%:%347=149%:%
%:%348=150%:%
%:%349=151%:%
%:%350=152%:%
%:%351=153%:%
%:%352=154%:%
%:%353=155%:%
%:%355=157%:%
%:%356=157%:%
%:%359=158%:%
%:%363=158%:%
%:%364=158%:%
%:%365=158%:%
%:%374=160%:%
%:%375=161%:%
%:%376=162%:%
%:%378=164%:%
%:%379=164%:%
%:%382=165%:%
%:%386=165%:%
%:%387=165%:%
%:%392=165%:%
%:%395=166%:%
%:%396=167%:%
%:%397=167%:%
%:%400=168%:%
%:%404=168%:%
%:%405=168%:%
%:%414=170%:%
%:%415=171%:%
%:%417=173%:%
%:%418=173%:%
%:%421=174%:%
%:%425=174%:%
%:%426=174%:%
%:%435=176%:%
%:%436=177%:%
%:%438=179%:%
%:%439=179%:%
%:%440=180%:%
%:%443=183%:%
%:%444=184%:%
%:%447=187%:%
%:%450=188%:%
%:%454=188%:%
%:%455=188%:%
%:%464=190%:%
%:%465=191%:%
%:%466=192%:%
%:%467=193%:%
%:%468=194%:%
%:%469=195%:%
%:%470=196%:%
%:%471=197%:%
%:%472=198%:%
%:%473=199%:%
%:%474=200%:%
%:%475=201%:%
%:%476=202%:%
%:%477=203%:%
%:%478=204%:%
%:%479=205%:%
%:%480=206%:%
%:%481=207%:%
%:%482=208%:%
%:%483=209%:%
%:%484=210%:%
%:%485=211%:%
%:%487=213%:%
%:%488=213%:%
%:%491=216%:%
%:%494=217%:%
%:%498=217%:%
%:%508=219%:%
%:%509=220%:%
%:%510=221%:%
%:%511=222%:%
%:%512=223%:%
%:%513=224%:%
%:%514=225%:%
%:%515=226%:%
%:%516=227%:%
%:%517=228%:%
%:%518=229%:%
%:%519=230%:%
%:%520=231%:%
%:%521=232%:%
%:%522=233%:%
%:%523=234%:%
%:%524=235%:%
%:%525=236%:%
%:%526=237%:%
%:%527=238%:%
%:%528=239%:%
%:%529=240%:%
%:%530=241%:%
%:%531=242%:%
%:%532=243%:%
%:%533=244%:%
%:%534=245%:%
%:%535=246%:%
%:%536=247%:%
%:%537=248%:%
%:%538=249%:%
%:%539=250%:%
%:%540=251%:%
%:%541=252%:%
%:%542=253%:%
%:%543=254%:%
%:%544=255%:%
%:%545=256%:%
%:%546=257%:%
%:%547=258%:%
%:%548=259%:%
%:%549=260%:%
%:%550=261%:%
%:%551=262%:%
%:%552=263%:%
%:%553=264%:%
%:%554=265%:%
%:%555=266%:%
%:%556=267%:%
%:%557=268%:%
%:%558=269%:%
%:%559=270%:%
%:%560=271%:%
%:%561=272%:%
%:%562=273%:%
%:%563=274%:%
%:%564=275%:%
%:%565=276%:%
%:%566=277%:%
%:%567=278%:%
%:%568=279%:%
%:%569=280%:%
%:%570=281%:%
%:%571=282%:%
%:%572=283%:%
%:%573=284%:%
%:%574=285%:%
%:%575=286%:%
%:%576=287%:%
%:%577=288%:%
%:%578=289%:%
%:%579=290%:%
%:%580=291%:%
%:%581=292%:%
%:%582=293%:%
%:%583=294%:%
%:%584=295%:%
%:%585=296%:%
%:%586=297%:%
%:%587=298%:%
%:%588=299%:%
%:%589=300%:%
%:%590=301%:%
%:%591=302%:%
%:%592=303%:%
%:%593=304%:%
%:%594=305%:%
%:%595=306%:%
%:%596=307%:%
%:%597=308%:%
%:%598=309%:%
%:%599=310%:%
%:%600=311%:%
%:%601=312%:%
%:%602=313%:%
%:%603=314%:%
%:%604=315%:%
%:%605=316%:%
%:%606=317%:%
%:%607=318%:%
%:%608=319%:%
%:%609=320%:%
%:%610=321%:%
%:%611=322%:%
%:%612=323%:%
%:%613=324%:%
%:%614=325%:%
%:%615=326%:%
%:%616=327%:%
%:%617=328%:%
%:%618=329%:%
%:%619=330%:%
%:%620=331%:%
%:%621=332%:%
%:%622=333%:%
%:%623=334%:%
%:%624=335%:%
%:%625=336%:%
%:%626=337%:%
%:%627=338%:%
%:%628=339%:%
%:%629=340%:%
%:%630=341%:%
%:%631=342%:%
%:%632=343%:%
%:%633=344%:%
%:%634=345%:%
%:%635=346%:%
%:%636=347%:%
%:%637=348%:%
%:%638=349%:%
%:%639=350%:%
%:%640=351%:%
%:%641=352%:%
%:%642=353%:%
%:%643=354%:%
%:%644=355%:%
%:%645=356%:%
%:%646=357%:%
%:%647=358%:%
%:%648=359%:%
%:%649=360%:%
%:%650=361%:%
%:%651=362%:%
%:%652=363%:%
%:%653=364%:%
%:%654=365%:%
%:%655=366%:%
%:%656=367%:%
%:%657=368%:%
%:%658=369%:%
%:%659=370%:%
%:%660=371%:%
%:%661=372%:%
%:%662=373%:%
%:%663=374%:%
%:%664=375%:%
%:%665=376%:%
%:%666=377%:%
%:%667=378%:%
%:%668=379%:%
%:%669=380%:%
%:%670=381%:%
%:%672=383%:%
%:%673=383%:%
%:%674=384%:%
%:%677=387%:%
%:%678=388%:%
%:%685=389%:%
%:%686=389%:%
%:%687=390%:%
%:%688=390%:%
%:%689=391%:%
%:%690=391%:%
%:%691=391%:%
%:%694=394%:%
%:%695=395%:%
%:%696=395%:%
%:%697=396%:%
%:%698=396%:%
%:%699=397%:%
%:%700=397%:%
%:%701=398%:%
%:%702=398%:%
%:%703=399%:%
%:%704=399%:%
%:%705=400%:%
%:%706=400%:%
%:%707=400%:%
%:%708=401%:%
%:%709=401%:%
%:%710=402%:%
%:%711=402%:%
%:%712=402%:%
%:%713=403%:%
%:%714=403%:%
%:%715=404%:%
%:%716=404%:%
%:%718=406%:%
%:%719=407%:%
%:%720=407%:%
%:%721=408%:%
%:%722=408%:%
%:%723=409%:%
%:%724=409%:%
%:%725=410%:%
%:%726=410%:%
%:%727=411%:%
%:%728=411%:%
%:%729=411%:%
%:%730=412%:%
%:%731=412%:%
%:%732=412%:%
%:%733=413%:%
%:%734=413%:%
%:%735=414%:%
%:%736=414%:%
%:%737=415%:%
%:%738=415%:%
%:%739=415%:%
%:%740=416%:%
%:%741=416%:%
%:%742=417%:%
%:%743=417%:%
%:%744=417%:%
%:%745=418%:%
%:%746=418%:%
%:%747=419%:%
%:%748=419%:%
%:%749=419%:%
%:%750=420%:%
%:%751=420%:%
%:%752=421%:%
%:%753=421%:%
%:%754=421%:%
%:%755=422%:%
%:%756=422%:%
%:%757=423%:%
%:%758=423%:%
%:%759=424%:%
%:%760=425%:%
%:%761=425%:%
%:%762=426%:%
%:%763=426%:%
%:%764=427%:%
%:%765=427%:%
%:%766=428%:%
%:%767=428%:%
%:%768=429%:%
%:%769=429%:%
%:%770=429%:%
%:%771=430%:%
%:%772=430%:%
%:%773=431%:%
%:%774=431%:%
%:%775=431%:%
%:%776=432%:%
%:%777=432%:%
%:%778=433%:%
%:%779=433%:%
%:%780=433%:%
%:%781=434%:%
%:%782=434%:%
%:%783=435%:%
%:%784=435%:%
%:%785=435%:%
%:%786=436%:%
%:%787=436%:%
%:%788=437%:%
%:%789=437%:%
%:%790=437%:%
%:%791=438%:%
%:%792=438%:%
%:%793=439%:%
%:%794=439%:%
%:%795=440%:%
%:%796=440%:%
%:%797=440%:%
%:%798=441%:%
%:%799=441%:%
%:%800=442%:%
%:%801=442%:%
%:%802=443%:%
%:%803=443%:%
%:%804=443%:%
%:%805=444%:%
%:%806=444%:%
%:%807=445%:%
%:%808=445%:%
%:%809=445%:%
%:%810=446%:%
%:%811=446%:%
%:%812=446%:%
%:%813=447%:%
%:%814=447%:%
%:%815=448%:%
%:%816=448%:%
%:%817=448%:%
%:%818=449%:%
%:%819=449%:%
%:%820=450%:%
%:%821=450%:%
%:%822=450%:%
%:%823=451%:%
%:%824=451%:%
%:%825=452%:%
%:%826=452%:%
%:%827=453%:%
%:%828=453%:%
%:%829=454%:%
%:%830=454%:%
%:%831=455%:%
%:%832=455%:%
%:%833=456%:%
%:%834=456%:%
%:%835=457%:%
%:%845=459%:%
%:%846=460%:%
%:%847=461%:%
%:%848=462%:%
%:%849=463%:%
%:%850=464%:%
%:%852=466%:%
%:%853=466%:%
%:%854=467%:%
%:%857=470%:%
%:%858=471%:%
%:%865=472%:%
%:%866=472%:%
%:%867=473%:%
%:%868=473%:%
%:%869=474%:%
%:%870=474%:%
%:%871=474%:%
%:%874=477%:%
%:%875=478%:%
%:%876=478%:%
%:%877=479%:%
%:%878=479%:%
%:%879=480%:%
%:%880=480%:%
%:%881=481%:%
%:%882=481%:%
%:%883=482%:%
%:%884=482%:%
%:%885=483%:%
%:%886=483%:%
%:%887=483%:%
%:%888=484%:%
%:%889=484%:%
%:%890=485%:%
%:%891=485%:%
%:%892=485%:%
%:%893=486%:%
%:%894=486%:%
%:%895=487%:%
%:%896=487%:%
%:%898=489%:%
%:%899=490%:%
%:%900=490%:%
%:%901=491%:%
%:%902=491%:%
%:%903=492%:%
%:%904=492%:%
%:%905=493%:%
%:%906=493%:%
%:%907=494%:%
%:%908=494%:%
%:%909=494%:%
%:%910=495%:%
%:%911=495%:%
%:%912=496%:%
%:%913=496%:%
%:%914=496%:%
%:%915=497%:%
%:%916=497%:%
%:%917=498%:%
%:%918=498%:%
%:%919=498%:%
%:%920=499%:%
%:%921=499%:%
%:%922=500%:%
%:%923=500%:%
%:%924=500%:%
%:%925=501%:%
%:%926=501%:%
%:%927=502%:%
%:%928=502%:%
%:%929=502%:%
%:%930=503%:%
%:%931=503%:%
%:%932=504%:%
%:%933=504%:%
%:%934=504%:%
%:%935=505%:%
%:%936=505%:%
%:%937=506%:%
%:%938=506%:%
%:%939=507%:%
%:%940=508%:%
%:%941=508%:%
%:%942=509%:%
%:%943=509%:%
%:%944=510%:%
%:%945=510%:%
%:%946=511%:%
%:%947=511%:%
%:%948=512%:%
%:%949=512%:%
%:%950=512%:%
%:%951=513%:%
%:%952=513%:%
%:%953=514%:%
%:%954=514%:%
%:%955=514%:%
%:%956=515%:%
%:%957=515%:%
%:%958=516%:%
%:%959=516%:%
%:%960=516%:%
%:%961=517%:%
%:%962=517%:%
%:%963=518%:%
%:%964=518%:%
%:%965=518%:%
%:%966=519%:%
%:%967=519%:%
%:%968=520%:%
%:%969=520%:%
%:%970=520%:%
%:%971=521%:%
%:%972=521%:%
%:%973=522%:%
%:%974=522%:%
%:%975=522%:%
%:%976=523%:%
%:%977=523%:%
%:%978=524%:%
%:%979=524%:%
%:%980=525%:%
%:%981=525%:%
%:%982=525%:%
%:%983=526%:%
%:%984=526%:%
%:%985=527%:%
%:%986=527%:%
%:%987=528%:%
%:%988=528%:%
%:%989=528%:%
%:%990=529%:%
%:%991=529%:%
%:%992=530%:%
%:%993=530%:%
%:%994=530%:%
%:%995=531%:%
%:%996=531%:%
%:%997=531%:%
%:%998=532%:%
%:%999=532%:%
%:%1000=533%:%
%:%1001=533%:%
%:%1002=533%:%
%:%1003=534%:%
%:%1004=534%:%
%:%1005=535%:%
%:%1006=535%:%
%:%1007=535%:%
%:%1008=536%:%
%:%1009=536%:%
%:%1010=537%:%
%:%1011=537%:%
%:%1012=538%:%
%:%1013=538%:%
%:%1014=539%:%
%:%1015=539%:%
%:%1016=540%:%
%:%1017=540%:%
%:%1018=541%:%
%:%1019=541%:%
%:%1020=542%:%
%:%1030=544%:%
%:%1031=545%:%
%:%1033=547%:%
%:%1034=547%:%
%:%1035=548%:%
%:%1036=549%:%
%:%1039=552%:%
%:%1046=553%:%
%:%1047=553%:%
%:%1048=554%:%
%:%1049=554%:%
%:%1057=562%:%
%:%1058=563%:%
%:%1059=563%:%
%:%1060=563%:%
%:%1061=564%:%
%:%1062=564%:%
%:%1063=564%:%
%:%1064=565%:%
%:%1065=565%:%
%:%1066=566%:%
%:%1067=566%:%
%:%1068=567%:%
%:%1069=567%:%
%:%1070=567%:%
%:%1071=568%:%
%:%1072=568%:%
%:%1073=569%:%
%:%1074=569%:%
%:%1075=570%:%
%:%1076=570%:%
%:%1077=571%:%
%:%1078=571%:%
%:%1079=572%:%
%:%1080=572%:%
%:%1081=572%:%
%:%1082=573%:%
%:%1083=573%:%
%:%1084=574%:%
%:%1085=574%:%
%:%1086=574%:%
%:%1087=575%:%
%:%1088=575%:%
%:%1089=576%:%
%:%1090=576%:%
%:%1091=576%:%
%:%1092=577%:%
%:%1093=577%:%
%:%1094=578%:%
%:%1095=578%:%
%:%1096=578%:%
%:%1097=579%:%
%:%1098=579%:%
%:%1101=582%:%
%:%1102=583%:%
%:%1103=583%:%
%:%1104=583%:%
%:%1105=584%:%
%:%1106=584%:%
%:%1107=585%:%
%:%1108=585%:%
%:%1109=586%:%
%:%1110=586%:%
%:%1111=587%:%
%:%1112=587%:%
%:%1113=588%:%
%:%1114=588%:%
%:%1115=589%:%
%:%1116=589%:%
%:%1117=590%:%
%:%1118=590%:%
%:%1119=591%:%
%:%1120=591%:%
%:%1121=591%:%
%:%1122=592%:%
%:%1123=592%:%
%:%1124=593%:%
%:%1125=593%:%
%:%1126=593%:%
%:%1127=594%:%
%:%1128=594%:%
%:%1129=595%:%
%:%1130=595%:%
%:%1131=595%:%
%:%1132=596%:%
%:%1133=596%:%
%:%1134=597%:%
%:%1135=597%:%
%:%1136=597%:%
%:%1137=598%:%
%:%1138=598%:%
%:%1141=601%:%
%:%1142=602%:%
%:%1143=602%:%
%:%1144=602%:%
%:%1145=603%:%
%:%1146=603%:%
%:%1147=604%:%
%:%1148=604%:%
%:%1149=605%:%
%:%1150=605%:%
%:%1151=606%:%
%:%1152=606%:%
%:%1155=609%:%
%:%1156=610%:%
%:%1157=610%:%
%:%1158=610%:%
%:%1159=611%:%
%:%1169=613%:%
%:%1170=614%:%
%:%1172=616%:%
%:%1173=616%:%
%:%1174=617%:%
%:%1177=620%:%
%:%1178=621%:%
%:%1185=622%:%
%:%1186=622%:%
%:%1187=623%:%
%:%1188=623%:%
%:%1189=624%:%
%:%1190=624%:%
%:%1191=624%:%
%:%1192=625%:%
%:%1193=625%:%
%:%1194=626%:%
%:%1195=626%:%
%:%1196=626%:%
%:%1197=627%:%
%:%1198=627%:%
%:%1206=635%:%
%:%1207=636%:%
%:%1208=636%:%
%:%1209=637%:%
%:%1210=637%:%
%:%1211=638%:%
%:%1212=638%:%
%:%1213=638%:%
%:%1214=639%:%
%:%1215=639%:%
%:%1216=640%:%
%:%1217=640%:%
%:%1218=640%:%
%:%1219=641%:%
%:%1220=641%:%
%:%1221=642%:%
%:%1222=642%:%
%:%1223=643%:%
%:%1224=643%:%
%:%1225=644%:%
%:%1226=644%:%
%:%1227=645%:%
%:%1228=645%:%
%:%1229=646%:%
%:%1230=646%:%
%:%1231=647%:%
%:%1232=647%:%
%:%1233=648%:%
%:%1234=648%:%
%:%1235=649%:%
%:%1236=649%:%
%:%1237=650%:%
%:%1238=650%:%
%:%1239=650%:%
%:%1240=651%:%
%:%1241=651%:%
%:%1242=651%:%
%:%1243=652%:%
%:%1244=652%:%
%:%1245=652%:%
%:%1246=653%:%
%:%1247=653%:%
%:%1248=654%:%
%:%1249=654%:%
%:%1250=654%:%
%:%1251=655%:%
%:%1252=655%:%
%:%1253=656%:%
%:%1254=656%:%
%:%1255=657%:%
%:%1256=657%:%
%:%1257=658%:%
%:%1258=658%:%
%:%1259=659%:%
%:%1260=659%:%
%:%1261=660%:%
%:%1262=660%:%
%:%1263=661%:%
%:%1264=661%:%
%:%1265=662%:%
%:%1266=662%:%
%:%1267=663%:%
%:%1268=663%:%
%:%1269=664%:%
%:%1270=664%:%
%:%1271=665%:%
%:%1272=665%:%
%:%1273=666%:%
%:%1274=666%:%
%:%1275=666%:%
%:%1276=667%:%
%:%1277=667%:%
%:%1278=667%:%
%:%1279=668%:%
%:%1280=668%:%
%:%1281=668%:%
%:%1282=669%:%
%:%1283=669%:%
%:%1284=670%:%
%:%1285=670%:%
%:%1286=670%:%
%:%1287=671%:%
%:%1288=671%:%
%:%1289=672%:%
%:%1290=672%:%
%:%1291=673%:%
%:%1292=673%:%
%:%1293=674%:%
%:%1294=674%:%
%:%1295=675%:%
%:%1296=675%:%
%:%1297=676%:%
%:%1298=676%:%
%:%1299=677%:%
%:%1300=677%:%
%:%1301=678%:%
%:%1302=678%:%
%:%1303=679%:%
%:%1304=679%:%
%:%1305=680%:%
%:%1306=680%:%
%:%1307=681%:%
%:%1308=681%:%
%:%1309=682%:%
%:%1310=682%:%
%:%1311=682%:%
%:%1312=683%:%
%:%1313=683%:%
%:%1314=683%:%
%:%1315=684%:%
%:%1316=684%:%
%:%1317=684%:%
%:%1318=685%:%
%:%1319=685%:%
%:%1320=686%:%
%:%1321=686%:%
%:%1322=686%:%
%:%1323=687%:%
%:%1324=687%:%
%:%1325=688%:%
%:%1326=688%:%
%:%1327=689%:%
%:%1328=689%:%
%:%1329=690%:%
%:%1330=690%:%
%:%1331=691%:%
%:%1332=691%:%
%:%1333=692%:%
%:%1334=692%:%
%:%1335=693%:%
%:%1336=693%:%
%:%1337=694%:%
%:%1338=694%:%
%:%1339=695%:%
%:%1340=695%:%
%:%1341=696%:%
%:%1342=696%:%
%:%1343=697%:%
%:%1344=697%:%
%:%1345=698%:%
%:%1346=698%:%
%:%1347=698%:%
%:%1348=699%:%
%:%1349=699%:%
%:%1350=699%:%
%:%1351=700%:%
%:%1352=700%:%
%:%1353=700%:%
%:%1354=701%:%
%:%1355=701%:%
%:%1356=701%:%
%:%1357=702%:%
%:%1358=702%:%
%:%1359=702%:%
%:%1360=703%:%
%:%1361=703%:%
%:%1362=704%:%
%:%1363=704%:%
%:%1364=705%:%
%:%1365=705%:%
%:%1366=706%:%
%:%1367=706%:%
%:%1368=707%:%
%:%1369=707%:%
%:%1370=708%:%
%:%1371=708%:%
%:%1372=709%:%
%:%1373=709%:%
%:%1374=710%:%
%:%1375=710%:%
%:%1376=711%:%
%:%1377=711%:%
%:%1378=712%:%
%:%1379=712%:%
%:%1380=713%:%
%:%1381=713%:%
%:%1382=714%:%
%:%1383=714%:%
%:%1384=715%:%
%:%1385=715%:%
%:%1386=716%:%
%:%1387=716%:%
%:%1388=716%:%
%:%1389=717%:%
%:%1390=717%:%
%:%1391=717%:%
%:%1392=718%:%
%:%1393=718%:%
%:%1394=718%:%
%:%1395=719%:%
%:%1396=719%:%
%:%1397=720%:%
%:%1398=720%:%
%:%1399=720%:%
%:%1400=721%:%
%:%1401=721%:%
%:%1402=722%:%
%:%1403=722%:%
%:%1404=723%:%
%:%1405=723%:%
%:%1406=724%:%
%:%1407=724%:%
%:%1408=725%:%
%:%1409=725%:%
%:%1410=726%:%
%:%1411=726%:%
%:%1412=727%:%
%:%1413=727%:%
%:%1414=728%:%
%:%1415=728%:%
%:%1416=729%:%
%:%1417=729%:%
%:%1418=730%:%
%:%1419=730%:%
%:%1420=731%:%
%:%1421=731%:%
%:%1422=732%:%
%:%1423=732%:%
%:%1424=732%:%
%:%1425=733%:%
%:%1426=733%:%
%:%1427=733%:%
%:%1428=734%:%
%:%1429=734%:%
%:%1430=734%:%
%:%1431=735%:%
%:%1432=735%:%
%:%1433=736%:%
%:%1434=736%:%
%:%1435=736%:%
%:%1436=737%:%
%:%1437=737%:%
%:%1438=738%:%
%:%1439=738%:%
%:%1440=739%:%
%:%1441=739%:%
%:%1442=740%:%
%:%1443=740%:%
%:%1444=741%:%
%:%1445=741%:%
%:%1446=742%:%
%:%1447=742%:%
%:%1448=743%:%
%:%1449=743%:%
%:%1450=744%:%
%:%1451=744%:%
%:%1452=745%:%
%:%1453=745%:%
%:%1454=746%:%
%:%1455=746%:%
%:%1456=747%:%
%:%1457=747%:%
%:%1458=748%:%
%:%1459=748%:%
%:%1460=748%:%
%:%1461=749%:%
%:%1462=749%:%
%:%1463=749%:%
%:%1464=750%:%
%:%1465=750%:%
%:%1466=750%:%
%:%1467=751%:%
%:%1468=751%:%
%:%1469=752%:%
%:%1470=752%:%
%:%1471=752%:%
%:%1472=753%:%
%:%1473=753%:%
%:%1474=754%:%
%:%1475=754%:%
%:%1476=755%:%
%:%1477=755%:%
%:%1481=759%:%
%:%1482=760%:%
%:%1483=760%:%
%:%1484=760%:%
%:%1485=761%:%
%:%1486=761%:%
%:%1489=764%:%
%:%1490=765%:%
%:%1491=765%:%
%:%1492=765%:%
%:%1493=766%:%
%:%1494=766%:%
%:%1502=774%:%
%:%1503=775%:%
%:%1504=775%:%
%:%1505=775%:%
%:%1506=776%:%
%:%1507=776%:%
%:%1515=784%:%
%:%1516=785%:%
%:%1517=785%:%
%:%1518=786%:%
%:%1519=786%:%
%:%1520=787%:%
%:%1521=787%:%
%:%1522=788%:%
%:%1523=788%:%
%:%1524=788%:%
%:%1525=789%:%
%:%1535=791%:%
%:%1537=793%:%
%:%1538=793%:%
%:%1541=796%:%
%:%1548=797%:%
%:%1549=797%:%
%:%1550=798%:%
%:%1551=798%:%
%:%1552=799%:%
%:%1553=799%:%
%:%1556=802%:%
%:%1557=803%:%
%:%1558=803%:%
%:%1559=804%:%
%:%1560=804%:%
%:%1561=805%:%
%:%1562=805%:%
%:%1565=808%:%
%:%1566=809%:%
%:%1567=809%:%
%:%1568=810%:%
%:%1569=810%:%
%:%1570=811%:%
%:%1580=813%:%
%:%1582=815%:%
%:%1583=815%:%
%:%1586=818%:%
%:%1589=819%:%
%:%1593=819%:%
%:%1594=819%:%
%:%1595=820%:%
%:%1596=820%:%
%:%1597=821%:%
%:%1598=821%:%
%:%1599=821%:%
%:%1600=822%:%
%:%1601=822%:%
%:%1602=822%:%
%:%1603=823%:%
%:%1604=823%:%