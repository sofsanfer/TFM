%
\begin{isabellebody}%
\setisabellecontext{Sintaxis}%
%
\isadelimtheory
%
\endisadelimtheory
%
\isatagtheory
%
\endisatagtheory
{\isafoldtheory}%
%
\isadelimtheory
%
\endisadelimtheory
%
\isadelimdocument
%
\endisadelimdocument
%
\isatagdocument
%
\isamarkupsection{Fórmulas%
}
\isamarkuptrue%
%
\endisatagdocument
{\isafolddocument}%
%
\isadelimdocument
%
\endisadelimdocument
%
\begin{isamarkuptext}%
En esta sección presentaremos una formalización en Isabelle de la 
  sintaxis de la lógica proposicional, junto con resultados y pruebas 
  sobre la misma. En líneas generales, primero daremos las nociones de 
  forma clásica y, a continuación, su correspondiente formalización.

  En primer lugar, supondremos que disponemos de los siguientes 
  elementos:
  \begin{description}
    \item[Alfabeto:] Es una lista infinita de variables proposicionales. 
      También pueden ser llamadas átomos o símbolos proposicionales.
    \item[Conectivas:] Conjunto finito cuyos elementos interactúan con 
      las variables. Pueden ser monarias que afectan a un único elemento 
      o binarias que afectan a dos. En el primer grupo se encuentra la 
      negación (\isa{{\isasymnot}}) y en el segundo la conjunción (\isa{{\isasymand}}), la disyunción 
      (\isa{{\isasymor}}) y la implicación (\isa{{\isasymlongrightarrow}}).
  \end{description}

  A continuación definiremos la estructura de fórmula sobre los 
  elementos anteriores. Para ello daremos una definición recursiva 
  basada en dos elementos: un conjunto de fórmulas básicas y una serie 
  de procedimientos de definición de fórmulas a partir de otras. El 
  conjunto de las fórmulas será el menor conjunto de estructuras 
  sintácticas con dicho alfabeto y conectivas que contiene a las básicas 
  y es cerrado mediante los procedimientos de definición que mostraremos 
  a continuación.

  \begin{definicion}
    El conjunto de las fórmulas proposicionales está formado por las 
    siguientes:
    \begin{itemize}
      \item Las fórmulas atómicas, constituidas únicamente por una 
        variable del alfabeto. 
      \item La constante \isa{{\isasymbottom}}.
      \item Dada una fórmula \isa{F}, la negación \isa{{\isasymnot}\ F} es una fórmula.
      \item Dadas dos fórmulas \isa{F} y \isa{G}, la conjunción \isa{F\ {\isasymand}\ G} es una
        fórmula.
      \item Dadas dos fórmulas \isa{F} y \isa{G}, la disyunción \isa{F\ {\isasymor}\ G} es una
        fórmula.
      \item Dadas dos fórmulas \isa{F} y \isa{G}, la implicación \isa{F\ {\isasymlongrightarrow}\ G} es 
        una fórmula.
    \end{itemize}
  \end{definicion}

  Intuitivamente, las fórmulas proposicionales son entendidas como un 
  tipo de árbol sintáctico cuyos nodos son las conectivas y sus hojas
  las fórmulas atómicas. Veamos, por ejemplo, el árbol sintáctico de
  la fórmula \isa{p\ {\isasymrightarrow}\ {\isacharparenleft}{\isasymnot}\ q\ {\isasymor}\ p{\isacharparenright}}.

 \begin{forest} for tree = {parent anchor = south, child anchor = north}
        [\isa{p\ {\isasymrightarrow}\ {\isacharparenleft}{\isasymnot}\ q\ {\isasymor}\ p{\isacharparenright}}
            [\isa{p}]
            [\isa{{\isasymnot}\ q\ {\isasymor}\ p}
                [\isa{{\isasymnot}\ q}
                  [\isa{q}]]
                [\isa{p}]]
        ]
 \end{forest}

  A continuación, veamos la representación en Isabelle de la estructura
  de las fórmulas proposicionales.%
\end{isamarkuptext}\isamarkuptrue%
\isacommand{datatype}\isamarkupfalse%
\ {\isacharparenleft}atoms{\isacharcolon}\ {\isacharprime}a{\isacharparenright}\ formula\ {\isacharequal}\ \isanewline
\ \ Atom\ {\isacharprime}a\isanewline
{\isacharbar}\ Bot\ \ \ \ \ \ \ \ \ \ \ \ \ \ \ \ \ \ \ \ \ \ \ \ \ \ \ \ \ \ {\isacharparenleft}{\isachardoublequoteopen}{\isasymbottom}{\isachardoublequoteclose}{\isacharparenright}\ \ \isanewline
{\isacharbar}\ Not\ {\isachardoublequoteopen}{\isacharprime}a\ formula{\isachardoublequoteclose}\ \ \ \ \ \ \ \ \ \ \ \ \ \ \ \ \ {\isacharparenleft}{\isachardoublequoteopen}\isactrlbold {\isasymnot}{\isachardoublequoteclose}{\isacharparenright}\isanewline
{\isacharbar}\ And\ {\isachardoublequoteopen}{\isacharprime}a\ formula{\isachardoublequoteclose}\ {\isachardoublequoteopen}{\isacharprime}a\ formula{\isachardoublequoteclose}\ \ \ \ {\isacharparenleft}\isakeyword{infix}\ {\isachardoublequoteopen}\isactrlbold {\isasymand}{\isachardoublequoteclose}\ {\isadigit{6}}{\isadigit{8}}{\isacharparenright}\isanewline
{\isacharbar}\ Or\ {\isachardoublequoteopen}{\isacharprime}a\ formula{\isachardoublequoteclose}\ {\isachardoublequoteopen}{\isacharprime}a\ formula{\isachardoublequoteclose}\ \ \ \ \ {\isacharparenleft}\isakeyword{infix}\ {\isachardoublequoteopen}\isactrlbold {\isasymor}{\isachardoublequoteclose}\ {\isadigit{6}}{\isadigit{8}}{\isacharparenright}\isanewline
{\isacharbar}\ Imp\ {\isachardoublequoteopen}{\isacharprime}a\ formula{\isachardoublequoteclose}\ {\isachardoublequoteopen}{\isacharprime}a\ formula{\isachardoublequoteclose}\ \ \ \ {\isacharparenleft}\isakeyword{infixr}\ {\isachardoublequoteopen}\isactrlbold {\isasymrightarrow}{\isachardoublequoteclose}\ {\isadigit{6}}{\isadigit{8}}{\isacharparenright}%
\begin{isamarkuptext}%
Como podemos observar representamos las fórmulas proposicionales
  mediante un tipo de dato recursivo, \isa{formula}, con los 
  siguientes constructores sobre un tipo cualquiera:

  \begin{description}
    \item[Fórmulas básicas]:
      \begin{itemize}
        \item \isa{Atom\ {\isacharcolon}{\isacharcolon}\ {\isacharprime}a\ {\isasymRightarrow}\ {\isacharprime}a\ formula}
        \item \isa{{\isasymbottom}\ {\isacharcolon}{\isacharcolon}\ {\isacharprime}a\ formula}
      \end{itemize}
    \item [Fórmulas compuestas]:
      \begin{itemize}
        \item \isa{\isactrlbold {\isasymnot}\ {\isacharcolon}{\isacharcolon}\ {\isacharprime}a\ formula\ {\isasymRightarrow}\ {\isacharprime}a\ formula}
        \item \isa{{\isacharparenleft}\isactrlbold {\isasymand}{\isacharparenright}\ {\isacharcolon}{\isacharcolon}\ {\isacharprime}a\ formula\ {\isasymRightarrow}\ {\isacharprime}a\ formula\ {\isasymRightarrow}\ {\isacharprime}a\ formula}
        \item \isa{{\isacharparenleft}\isactrlbold {\isasymor}{\isacharparenright}\ {\isacharcolon}{\isacharcolon}\ {\isacharprime}a\ formula\ {\isasymRightarrow}\ {\isacharprime}a\ formula\ {\isasymRightarrow}\ {\isacharprime}a\ formula}
        \item \isa{{\isacharparenleft}\isactrlbold {\isasymrightarrow}{\isacharparenright}\ {\isacharcolon}{\isacharcolon}\ {\isacharprime}a\ formula\ {\isasymRightarrow}\ {\isacharprime}a\ formula\ {\isasymRightarrow}\ {\isacharprime}a\ formula}
      \end{itemize}
  \end{description}

  Cabe señalar que los términos \isa{infix} e \isa{infixr} nos señalan que 
  los constructores que representan a las conectivas se pueden usar de
  forma infija. En particular, \isa{infixr} se trata de un infijo asociado a 
  la derecha.

  Por otro lado, la definición de \isa{formula} 
  genera automáticamente los siguientes lemas sobre la función 
  \isa{atoms\ {\isacharcolon}{\isacharcolon}\ {\isacharprime}a\ formula\ {\isasymRightarrow}\ {\isacharprime}a\ set}, que obtiene el conjunto de átomos de una fórmula.

  \begin{itemize}
    \item[] \isa{atoms\ {\isacharparenleft}Atom\ x{\isadigit{1}}{\isachardot}{\isadigit{0}}{\isacharparenright}\ {\isacharequal}\ {\isacharbraceleft}x{\isadigit{1}}{\isachardot}{\isadigit{0}}{\isacharbraceright}\isasep\isanewline%
atoms\ {\isasymbottom}\ {\isacharequal}\ {\isasymemptyset}\isasep\isanewline%
atoms\ {\isacharparenleft}\isactrlbold {\isasymnot}\ x{\isadigit{3}}{\isachardot}{\isadigit{0}}{\isacharparenright}\ {\isacharequal}\ atoms\ x{\isadigit{3}}{\isachardot}{\isadigit{0}}\isasep\isanewline%
atoms\ {\isacharparenleft}x{\isadigit{4}}{\isadigit{1}}{\isachardot}{\isadigit{0}}\ \isactrlbold {\isasymand}\ x{\isadigit{4}}{\isadigit{2}}{\isachardot}{\isadigit{0}}{\isacharparenright}\ {\isacharequal}\ atoms\ x{\isadigit{4}}{\isadigit{1}}{\isachardot}{\isadigit{0}}\ {\isasymunion}\ atoms\ x{\isadigit{4}}{\isadigit{2}}{\isachardot}{\isadigit{0}}\isasep\isanewline%
atoms\ {\isacharparenleft}x{\isadigit{5}}{\isadigit{1}}{\isachardot}{\isadigit{0}}\ \isactrlbold {\isasymor}\ x{\isadigit{5}}{\isadigit{2}}{\isachardot}{\isadigit{0}}{\isacharparenright}\ {\isacharequal}\ atoms\ x{\isadigit{5}}{\isadigit{1}}{\isachardot}{\isadigit{0}}\ {\isasymunion}\ atoms\ x{\isadigit{5}}{\isadigit{2}}{\isachardot}{\isadigit{0}}\isasep\isanewline%
atoms\ {\isacharparenleft}x{\isadigit{6}}{\isadigit{1}}{\isachardot}{\isadigit{0}}\ \isactrlbold {\isasymrightarrow}\ x{\isadigit{6}}{\isadigit{2}}{\isachardot}{\isadigit{0}}{\isacharparenright}\ {\isacharequal}\ atoms\ x{\isadigit{6}}{\isadigit{1}}{\isachardot}{\isadigit{0}}\ {\isasymunion}\ atoms\ x{\isadigit{6}}{\isadigit{2}}{\isachardot}{\isadigit{0}}}
  \end{itemize} 

  A continuación veremos varios ejemplos de fórmulas y el conjunto de 
  sus variables proposicionales obtenido mediante \isa{atoms}. Se 
  observa que, por ser conjuntos, no contienen elementos repetidos.%
\end{isamarkuptext}\isamarkuptrue%
\isacommand{notepad}\isamarkupfalse%
\ \isanewline
\isakeyword{begin}\isanewline
%
\isadelimproof
\ \ %
\endisadelimproof
%
\isatagproof
\isacommand{fix}\isamarkupfalse%
\ p\ q\ r\ {\isacharcolon}{\isacharcolon}\ {\isacharprime}a\isanewline
\isanewline
\ \ \isacommand{have}\isamarkupfalse%
\ {\isachardoublequoteopen}atoms\ {\isacharparenleft}Atom\ p{\isacharparenright}\ {\isacharequal}\ {\isacharbraceleft}p{\isacharbraceright}{\isachardoublequoteclose}\isanewline
\ \ \ \ \isacommand{by}\isamarkupfalse%
\ {\isacharparenleft}simp\ only{\isacharcolon}\ formula{\isachardot}set{\isacharparenright}\isanewline
\isanewline
\ \ \isacommand{have}\isamarkupfalse%
\ {\isachardoublequoteopen}atoms\ {\isacharparenleft}\isactrlbold {\isasymnot}\ {\isacharparenleft}Atom\ p{\isacharparenright}{\isacharparenright}\ {\isacharequal}\ {\isacharbraceleft}p{\isacharbraceright}{\isachardoublequoteclose}\isanewline
\ \ \ \ \isacommand{by}\isamarkupfalse%
\ {\isacharparenleft}simp\ only{\isacharcolon}\ formula{\isachardot}set{\isacharparenright}\isanewline
\isanewline
\ \ \isacommand{have}\isamarkupfalse%
\ {\isachardoublequoteopen}atoms\ {\isacharparenleft}{\isacharparenleft}Atom\ p\ \isactrlbold {\isasymrightarrow}\ Atom\ q{\isacharparenright}\ \isactrlbold {\isasymor}\ Atom\ r{\isacharparenright}\ {\isacharequal}\ {\isacharbraceleft}p{\isacharcomma}q{\isacharcomma}r{\isacharbraceright}{\isachardoublequoteclose}\isanewline
\ \ \ \ \isacommand{by}\isamarkupfalse%
\ auto\isanewline
\isanewline
\ \ \isacommand{have}\isamarkupfalse%
\ {\isachardoublequoteopen}atoms\ {\isacharparenleft}{\isacharparenleft}Atom\ p\ \isactrlbold {\isasymrightarrow}\ Atom\ p{\isacharparenright}\ \isactrlbold {\isasymor}\ Atom\ r{\isacharparenright}\ {\isacharequal}\ {\isacharbraceleft}p{\isacharcomma}r{\isacharbraceright}{\isachardoublequoteclose}\isanewline
\ \ \ \ \isacommand{by}\isamarkupfalse%
\ auto%
\endisatagproof
{\isafoldproof}%
%
\isadelimproof
\ \ \isanewline
%
\endisadelimproof
\isacommand{end}\isamarkupfalse%
%
\begin{isamarkuptext}%
En particular, el conjunto de símbolos proposicionales de la 
  fórmula \isa{Bot} es vacío. Además, para calcular esta constante es 
  necesario especificar el tipo sobre el que se construye la fórmula.%
\end{isamarkuptext}\isamarkuptrue%
\isacommand{notepad}\isamarkupfalse%
\ \isanewline
\isakeyword{begin}\isanewline
%
\isadelimproof
\ \ %
\endisadelimproof
%
\isatagproof
\isacommand{fix}\isamarkupfalse%
\ p\ {\isacharcolon}{\isacharcolon}\ {\isacharprime}a\isanewline
\isanewline
\ \ \isacommand{have}\isamarkupfalse%
\ {\isachardoublequoteopen}atoms\ {\isasymbottom}\ {\isacharequal}\ {\isasymemptyset}{\isachardoublequoteclose}\isanewline
\ \ \ \ \isacommand{by}\isamarkupfalse%
\ {\isacharparenleft}simp\ only{\isacharcolon}\ formula{\isachardot}set{\isacharparenright}\isanewline
\isanewline
\ \ \isacommand{have}\isamarkupfalse%
\ {\isachardoublequoteopen}atoms\ {\isacharparenleft}Atom\ p\ \isactrlbold {\isasymor}\ {\isasymbottom}{\isacharparenright}\ {\isacharequal}\ {\isacharbraceleft}p{\isacharbraceright}{\isachardoublequoteclose}\isanewline
\ \ \isacommand{proof}\isamarkupfalse%
\ {\isacharminus}\isanewline
\ \ \ \ \isacommand{have}\isamarkupfalse%
\ {\isachardoublequoteopen}atoms\ {\isacharparenleft}Atom\ p\ \isactrlbold {\isasymor}\ {\isasymbottom}{\isacharparenright}\ {\isacharequal}\ atoms\ {\isacharparenleft}Atom\ p{\isacharparenright}\ {\isasymunion}\ atoms\ Bot{\isachardoublequoteclose}\isanewline
\ \ \ \ \ \ \isacommand{by}\isamarkupfalse%
\ {\isacharparenleft}simp\ only{\isacharcolon}\ formula{\isachardot}set{\isacharparenleft}{\isadigit{5}}{\isacharparenright}{\isacharparenright}\isanewline
\ \ \ \ \isacommand{also}\isamarkupfalse%
\ \isacommand{have}\isamarkupfalse%
\ {\isachardoublequoteopen}{\isasymdots}\ {\isacharequal}\ {\isacharbraceleft}p{\isacharbraceright}\ {\isasymunion}\ atoms\ Bot{\isachardoublequoteclose}\isanewline
\ \ \ \ \ \ \isacommand{by}\isamarkupfalse%
\ {\isacharparenleft}simp\ only{\isacharcolon}\ formula{\isachardot}set{\isacharparenleft}{\isadigit{1}}{\isacharparenright}{\isacharparenright}\isanewline
\ \ \ \ \isacommand{also}\isamarkupfalse%
\ \isacommand{have}\isamarkupfalse%
\ {\isachardoublequoteopen}{\isasymdots}\ {\isacharequal}\ {\isacharbraceleft}p{\isacharbraceright}\ {\isasymunion}\ {\isasymemptyset}{\isachardoublequoteclose}\isanewline
\ \ \ \ \ \ \isacommand{by}\isamarkupfalse%
\ {\isacharparenleft}simp\ only{\isacharcolon}\ formula{\isachardot}set{\isacharparenleft}{\isadigit{2}}{\isacharparenright}{\isacharparenright}\isanewline
\ \ \ \ \isacommand{also}\isamarkupfalse%
\ \isacommand{have}\isamarkupfalse%
\ {\isachardoublequoteopen}{\isasymdots}\ {\isacharequal}\ {\isacharbraceleft}p{\isacharbraceright}{\isachardoublequoteclose}\isanewline
\ \ \ \ \ \ \isacommand{by}\isamarkupfalse%
\ {\isacharparenleft}simp\ only{\isacharcolon}\ Un{\isacharunderscore}empty{\isacharunderscore}right{\isacharparenright}\isanewline
\ \ \ \ \isacommand{finally}\isamarkupfalse%
\ \isacommand{show}\isamarkupfalse%
\ {\isachardoublequoteopen}atoms\ {\isacharparenleft}Atom\ p\ \isactrlbold {\isasymor}\ {\isasymbottom}{\isacharparenright}\ {\isacharequal}\ {\isacharbraceleft}p{\isacharbraceright}{\isachardoublequoteclose}\isanewline
\ \ \ \ \ \ \isacommand{by}\isamarkupfalse%
\ this\isanewline
\ \ \isacommand{qed}\isamarkupfalse%
\isanewline
\isanewline
\ \ \isacommand{have}\isamarkupfalse%
\ {\isachardoublequoteopen}atoms\ {\isacharparenleft}Atom\ p\ \isactrlbold {\isasymor}\ {\isasymbottom}{\isacharparenright}\ {\isacharequal}\ {\isacharbraceleft}p{\isacharbraceright}{\isachardoublequoteclose}\isanewline
\ \ \ \ \isacommand{by}\isamarkupfalse%
\ {\isacharparenleft}simp\ only{\isacharcolon}\ formula{\isachardot}set\ Un{\isacharunderscore}empty{\isacharunderscore}right{\isacharparenright}%
\endisatagproof
{\isafoldproof}%
%
\isadelimproof
\isanewline
%
\endisadelimproof
\isacommand{end}\isamarkupfalse%
\isanewline
\isanewline
\isacommand{value}\isamarkupfalse%
\ {\isachardoublequoteopen}{\isacharparenleft}Bot{\isacharcolon}{\isacharcolon}nat\ formula{\isacharparenright}{\isachardoublequoteclose}%
\begin{isamarkuptext}%
Una vez definida la estructura de las fórmulas, vamos a introducir 
  el método de demostración que seguirán los resultados que aquí 
  presentaremos, tanto en la teoría clásica como en Isabelle. 

  Según la definición recursiva de las fórmulas, dispondremos de un 
  esquema de inducción sobre las mismas:

  \begin{teorema}[Principio de inducción sobre fórmulas
  proposicionales]
    Sea \isa{{\isasymP}} una propiedad sobre fórmulas que verifica las siguientes 
    condiciones:
    \begin{itemize}
      \item Las fórmulas atómicas la cumplen.
      \item La constante \isa{{\isasymbottom}} la cumple.
      \item Dada \isa{F} fórmula que la cumple, entonces \isa{{\isasymnot}\ F} la cumple.
      \item Dadas \isa{F} y \isa{G} fórmulas que la cumplen, entonces \isa{F\ {\isacharasterisk}\ G} la 
        cumple, donde \isa{{\isacharasterisk}} simboliza cualquier conectiva binaria.
    \end{itemize}
    Entonces, todas las fórmulas proposicionales tienen la propiedad 
    \isa{{\isasymP}}.
  \end{teorema}

  Análogamente, como las fórmulas proposicionales están definidas 
  mediante un tipo de datos recursivo, Isabelle genera de forma 
  automática el esquema de inducción correspondiente. De este modo, en 
  las pruebas formalizadas utilizaremos la táctica \isa{induction}, 
  que corresponde al siguiente esquema.

  \isa{{\isasymAnd}x{\isachardot}\ P{\isacharparenleft}Atom\ x{\isacharparenright}}

  \isa{P\ {\isasymbottom}}

  \isa{{\isasymAnd}x{\isachardot}\ P\ x\ {\isasymLongrightarrow}\ P{\isacharparenleft}\isactrlbold {\isasymnot}\ x{\isacharparenright}}

  \isa{{\isasymAnd}x{\isadigit{1}}\ x{\isadigit{2}}{\isachardot}\ P\ x{\isadigit{1}}\ {\isasymand}\ P\ x{\isadigit{2}}\ {\isasymLongrightarrow}\ P\ {\isacharparenleft}x{\isadigit{1}}\ \isactrlbold {\isasymand}\ x{\isadigit{2}}{\isacharparenright}}

  \isa{{\isasymAnd}x{\isadigit{1}}\ x{\isadigit{2}}{\isachardot}\ P\ x{\isadigit{1}}\ {\isasymand}\ P\ x{\isadigit{2}}\ {\isasymLongrightarrow}\ P\ {\isacharparenleft}x{\isadigit{1}}\ \isactrlbold {\isasymor}\ x{\isadigit{2}}{\isacharparenright}}

  \isa{{\isasymAnd}x{\isadigit{1}}\ x{\isadigit{2}}{\isachardot}\ P\ x{\isadigit{1}}\ {\isasymand}\ P\ x{\isadigit{2}}\ {\isasymLongrightarrow}\ P\ {\isacharparenleft}x{\isadigit{1}}\ \isactrlbold {\isasymrightarrow}\ x{\isadigit{2}}{\isacharparenright}}

  \rule{70mm}{0.1mm}

  \isa{P\ formula}

  Como hemos señalado, el esquema inductivo genera así seis casos 
  distintos como se muestra anteriormente. Además, todas las 
  demostraciones sobre casos de conectivas binarias son equivalentes en 
  esta sección, pues la construcción sintáctica de fórmulas es idéntica 
  entre ellas. Estas se diferencian esencialmente en la connotación 
  semántica que veremos más adelante.

  A continuación el primer resultado de este apartado:

  \begin{lema}
    El conjunto de los átomos de una fórmula proposicional es finito.
  \end{lema}

  Para proceder a la demostración, consideremos la siguiente
  definición inductiva de conjunto finito. Cabe añadir que la 
  demostración seguirá el esquema inductivo relativo a la estructura de 
  fórmula, y no el que induce esta última definición.

  \begin{definicion}
    Los conjuntos finitos son:
      \begin{itemize}
        \item El vacío.
        \item Dado un conjunto finito \isa{A} y un elemento cualquiera \isa{a}, 
          entonces \isa{{\isacharbraceleft}a{\isacharbraceright}\ {\isasymunion}\ A} es finito.
      \end{itemize}
  \end{definicion}

  La formalización en Isabelle de la definición anterior es precisamente 
  \isa{finite} perteneciente a la teoría 
  \href{https://n9.cl/x86r}{FiniteSet.thy}. Dicha definición inductiva
  genera dos reglas análogas a las anteriores que definen a los 
  conjuntos finitos y que emplearemos en la demostración del resultado.

  \begin{itemize}
    \item[] \isa{fold{\isacharunderscore}graph\ f\ z\ {\isasymemptyset}\ z} 
      \hfill (\isa{emptyI})
  \end{itemize}

  \begin{itemize}
    \item[] \isa{\mbox{}\inferrule{\mbox{x\ {\isasymnotin}\ A\ {\isasymand}\ fold{\isacharunderscore}graph\ f\ z\ A\ y}}{\mbox{fold{\isacharunderscore}graph\ f\ z\ {\isacharparenleft}{\isacharbraceleft}x{\isacharbraceright}\ {\isasymunion}\ A{\isacharparenright}\ {\isacharparenleft}f\ x\ y{\isacharparenright}}}} 
      \hfill (\isa{insertI})
  \end{itemize}

  De este modo, en Isabelle podemos especificar el lema como sigue.%
\end{isamarkuptext}\isamarkuptrue%
\isacommand{lemma}\isamarkupfalse%
\ {\isachardoublequoteopen}finite\ {\isacharparenleft}atoms\ F{\isacharparenright}{\isachardoublequoteclose}\isanewline
%
\isadelimproof
\ \ %
\endisadelimproof
%
\isatagproof
\isacommand{oops}\isamarkupfalse%
%
\endisatagproof
{\isafoldproof}%
%
\isadelimproof
%
\endisadelimproof
%
\begin{isamarkuptext}%
A continuación, veamos la demostración clásica del lema. 

  \begin{demostracion}
  La prueba es por inducción sobre el tipo recursivo de las fórmulas. 
  Veamos cada caso.
  
  Consideremos una fórmula atómica \isa{p} cualquiera. Entonces, 
  su conjunto de variables proposicionales es \isa{{\isacharbraceleft}p{\isacharbraceright}}, finito.

  Sea la fórmula \isa{{\isasymbottom}}. Entonces, su conjunto de átomos es vacío y, por 
  lo tanto, finito.
  
  Sea \isa{F} una fórmula cuyo conjunto de variables proposicionales sea 
  finito. Entonces, por definición, \isa{{\isasymnot}\ F} y \isa{F} tienen igual conjunto de
  átomos y, por hipótesis de inducción, es finito.

  Consideremos las fórmulas \isa{F} y \isa{G} cuyos conjuntos de átomos son 
  finitos. Por\\ construcción, el conjunto de variables de \isa{F{\isacharasterisk}G} es la 
  unión de sus respectivos conjuntos de átomos para cualquier \isa{{\isacharasterisk}} 
  conectiva binaria. Por lo tanto, usando la hipótesis de inducción, 
  dicho conjunto es finito. 
  \end{demostracion} 

  Veamos ahora la prueba detallada en Isabelle. Mostraremos con detalle 
  todos los casos de conectivas binarias, aunque se puede observar que 
  son completamente análogos. Para facilitar la lectura, primero 
  demostraremos por separado cada uno de los casos según el esquema 
  inductivo de fórmulas, y finalmente añadiremos la prueba para una 
  fórmula cualquiera a partir de los anteriores.%
\end{isamarkuptext}\isamarkuptrue%
\isacommand{lemma}\isamarkupfalse%
\ atoms{\isacharunderscore}finite{\isacharunderscore}atom{\isacharcolon}\isanewline
\ \ {\isachardoublequoteopen}finite\ {\isacharparenleft}atoms\ {\isacharparenleft}Atom\ x{\isacharparenright}{\isacharparenright}{\isachardoublequoteclose}\isanewline
%
\isadelimproof
%
\endisadelimproof
%
\isatagproof
\isacommand{proof}\isamarkupfalse%
\ {\isacharminus}\isanewline
\ \ \isacommand{have}\isamarkupfalse%
\ {\isachardoublequoteopen}finite\ {\isasymemptyset}{\isachardoublequoteclose}\isanewline
\ \ \ \ \isacommand{by}\isamarkupfalse%
\ {\isacharparenleft}simp\ only{\isacharcolon}\ finite{\isachardot}emptyI{\isacharparenright}\isanewline
\ \ \isacommand{then}\isamarkupfalse%
\ \isacommand{have}\isamarkupfalse%
\ {\isachardoublequoteopen}finite\ {\isacharbraceleft}x{\isacharbraceright}{\isachardoublequoteclose}\isanewline
\ \ \ \ \isacommand{by}\isamarkupfalse%
\ {\isacharparenleft}simp\ only{\isacharcolon}\ finite{\isacharunderscore}insert{\isacharparenright}\isanewline
\ \ \isacommand{then}\isamarkupfalse%
\ \isacommand{show}\isamarkupfalse%
\ {\isachardoublequoteopen}finite\ {\isacharparenleft}atoms\ {\isacharparenleft}Atom\ x{\isacharparenright}{\isacharparenright}{\isachardoublequoteclose}\isanewline
\ \ \ \ \isacommand{by}\isamarkupfalse%
\ {\isacharparenleft}simp\ only{\isacharcolon}\ formula{\isachardot}set{\isacharparenleft}{\isadigit{1}}{\isacharparenright}{\isacharparenright}\ \isanewline
\isacommand{qed}\isamarkupfalse%
%
\endisatagproof
{\isafoldproof}%
%
\isadelimproof
\isanewline
%
\endisadelimproof
\isanewline
\isacommand{lemma}\isamarkupfalse%
\ atoms{\isacharunderscore}finite{\isacharunderscore}bot{\isacharcolon}\isanewline
\ \ {\isachardoublequoteopen}finite\ {\isacharparenleft}atoms\ {\isasymbottom}{\isacharparenright}{\isachardoublequoteclose}\isanewline
%
\isadelimproof
%
\endisadelimproof
%
\isatagproof
\isacommand{proof}\isamarkupfalse%
\ {\isacharminus}\isanewline
\ \ \isacommand{have}\isamarkupfalse%
\ {\isachardoublequoteopen}finite\ {\isasymemptyset}{\isachardoublequoteclose}\isanewline
\ \ \ \ \isacommand{by}\isamarkupfalse%
\ {\isacharparenleft}simp\ only{\isacharcolon}\ finite{\isachardot}emptyI{\isacharparenright}\isanewline
\ \ \isacommand{then}\isamarkupfalse%
\ \isacommand{show}\isamarkupfalse%
\ {\isachardoublequoteopen}finite\ {\isacharparenleft}atoms\ {\isasymbottom}{\isacharparenright}{\isachardoublequoteclose}\isanewline
\ \ \ \ \isacommand{by}\isamarkupfalse%
\ {\isacharparenleft}simp\ only{\isacharcolon}\ formula{\isachardot}set{\isacharparenleft}{\isadigit{2}}{\isacharparenright}{\isacharparenright}\ \isanewline
\isacommand{qed}\isamarkupfalse%
%
\endisatagproof
{\isafoldproof}%
%
\isadelimproof
\isanewline
%
\endisadelimproof
\isanewline
\isacommand{lemma}\isamarkupfalse%
\ atoms{\isacharunderscore}finite{\isacharunderscore}not{\isacharcolon}\isanewline
\ \ \isakeyword{assumes}\ {\isachardoublequoteopen}finite\ {\isacharparenleft}atoms\ F{\isacharparenright}{\isachardoublequoteclose}\ \isanewline
\ \ \isakeyword{shows}\ \ \ {\isachardoublequoteopen}finite\ {\isacharparenleft}atoms\ {\isacharparenleft}\isactrlbold {\isasymnot}\ F{\isacharparenright}{\isacharparenright}{\isachardoublequoteclose}\isanewline
%
\isadelimproof
\ \ %
\endisadelimproof
%
\isatagproof
\isacommand{using}\isamarkupfalse%
\ assms\isanewline
\ \ \isacommand{by}\isamarkupfalse%
\ {\isacharparenleft}simp\ only{\isacharcolon}\ formula{\isachardot}set{\isacharparenleft}{\isadigit{3}}{\isacharparenright}{\isacharparenright}%
\endisatagproof
{\isafoldproof}%
%
\isadelimproof
\ \isanewline
%
\endisadelimproof
\isanewline
\isacommand{lemma}\isamarkupfalse%
\ atoms{\isacharunderscore}finite{\isacharunderscore}and{\isacharcolon}\isanewline
\ \ \isakeyword{assumes}\ {\isachardoublequoteopen}finite\ {\isacharparenleft}atoms\ F{\isadigit{1}}{\isacharparenright}{\isachardoublequoteclose}\isanewline
\ \ \ \ \ \ \ \ \ \ {\isachardoublequoteopen}finite\ {\isacharparenleft}atoms\ F{\isadigit{2}}{\isacharparenright}{\isachardoublequoteclose}\isanewline
\ \ \isakeyword{shows}\ \ \ {\isachardoublequoteopen}finite\ {\isacharparenleft}atoms\ {\isacharparenleft}F{\isadigit{1}}\ \isactrlbold {\isasymand}\ F{\isadigit{2}}{\isacharparenright}{\isacharparenright}{\isachardoublequoteclose}\isanewline
%
\isadelimproof
%
\endisadelimproof
%
\isatagproof
\isacommand{proof}\isamarkupfalse%
\ {\isacharminus}\isanewline
\ \ \isacommand{have}\isamarkupfalse%
\ {\isachardoublequoteopen}finite\ {\isacharparenleft}atoms\ F{\isadigit{1}}\ {\isasymunion}\ atoms\ F{\isadigit{2}}{\isacharparenright}{\isachardoublequoteclose}\isanewline
\ \ \ \ \isacommand{using}\isamarkupfalse%
\ assms\isanewline
\ \ \ \ \isacommand{by}\isamarkupfalse%
\ {\isacharparenleft}simp\ only{\isacharcolon}\ finite{\isacharunderscore}UnI{\isacharparenright}\isanewline
\ \ \isacommand{then}\isamarkupfalse%
\ \isacommand{show}\isamarkupfalse%
\ {\isachardoublequoteopen}finite\ {\isacharparenleft}atoms\ {\isacharparenleft}F{\isadigit{1}}\ \isactrlbold {\isasymand}\ F{\isadigit{2}}{\isacharparenright}{\isacharparenright}{\isachardoublequoteclose}\ \ \isanewline
\ \ \ \ \isacommand{by}\isamarkupfalse%
\ {\isacharparenleft}simp\ only{\isacharcolon}\ formula{\isachardot}set{\isacharparenleft}{\isadigit{4}}{\isacharparenright}{\isacharparenright}\isanewline
\isacommand{qed}\isamarkupfalse%
%
\endisatagproof
{\isafoldproof}%
%
\isadelimproof
\isanewline
%
\endisadelimproof
\isanewline
\isacommand{lemma}\isamarkupfalse%
\ atoms{\isacharunderscore}finite{\isacharunderscore}or{\isacharcolon}\isanewline
\ \ \isakeyword{assumes}\ {\isachardoublequoteopen}finite\ {\isacharparenleft}atoms\ F{\isadigit{1}}{\isacharparenright}{\isachardoublequoteclose}\isanewline
\ \ \ \ \ \ \ \ \ \ {\isachardoublequoteopen}finite\ {\isacharparenleft}atoms\ F{\isadigit{2}}{\isacharparenright}{\isachardoublequoteclose}\isanewline
\ \ \isakeyword{shows}\ \ \ {\isachardoublequoteopen}finite\ {\isacharparenleft}atoms\ {\isacharparenleft}F{\isadigit{1}}\ \isactrlbold {\isasymor}\ F{\isadigit{2}}{\isacharparenright}{\isacharparenright}{\isachardoublequoteclose}\isanewline
%
\isadelimproof
%
\endisadelimproof
%
\isatagproof
\isacommand{proof}\isamarkupfalse%
\ {\isacharminus}\isanewline
\ \ \isacommand{have}\isamarkupfalse%
\ {\isachardoublequoteopen}finite\ {\isacharparenleft}atoms\ F{\isadigit{1}}\ {\isasymunion}\ atoms\ F{\isadigit{2}}{\isacharparenright}{\isachardoublequoteclose}\isanewline
\ \ \ \ \isacommand{using}\isamarkupfalse%
\ assms\isanewline
\ \ \ \ \isacommand{by}\isamarkupfalse%
\ {\isacharparenleft}simp\ only{\isacharcolon}\ finite{\isacharunderscore}UnI{\isacharparenright}\isanewline
\ \ \isacommand{then}\isamarkupfalse%
\ \isacommand{show}\isamarkupfalse%
\ {\isachardoublequoteopen}finite\ {\isacharparenleft}atoms\ {\isacharparenleft}F{\isadigit{1}}\ \isactrlbold {\isasymor}\ F{\isadigit{2}}{\isacharparenright}{\isacharparenright}{\isachardoublequoteclose}\ \ \isanewline
\ \ \ \ \isacommand{by}\isamarkupfalse%
\ {\isacharparenleft}simp\ only{\isacharcolon}\ formula{\isachardot}set{\isacharparenleft}{\isadigit{5}}{\isacharparenright}{\isacharparenright}\isanewline
\isacommand{qed}\isamarkupfalse%
%
\endisatagproof
{\isafoldproof}%
%
\isadelimproof
\isanewline
%
\endisadelimproof
\isanewline
\isacommand{lemma}\isamarkupfalse%
\ atoms{\isacharunderscore}finite{\isacharunderscore}imp{\isacharcolon}\isanewline
\ \ \isakeyword{assumes}\ {\isachardoublequoteopen}finite\ {\isacharparenleft}atoms\ F{\isadigit{1}}{\isacharparenright}{\isachardoublequoteclose}\isanewline
\ \ \ \ \ \ \ \ \ \ {\isachardoublequoteopen}finite\ {\isacharparenleft}atoms\ F{\isadigit{2}}{\isacharparenright}{\isachardoublequoteclose}\isanewline
\ \ \isakeyword{shows}\ \ \ {\isachardoublequoteopen}finite\ {\isacharparenleft}atoms\ {\isacharparenleft}F{\isadigit{1}}\ \isactrlbold {\isasymrightarrow}\ F{\isadigit{2}}{\isacharparenright}{\isacharparenright}{\isachardoublequoteclose}\isanewline
%
\isadelimproof
%
\endisadelimproof
%
\isatagproof
\isacommand{proof}\isamarkupfalse%
\ {\isacharminus}\isanewline
\ \ \isacommand{have}\isamarkupfalse%
\ {\isachardoublequoteopen}finite\ {\isacharparenleft}atoms\ F{\isadigit{1}}\ {\isasymunion}\ atoms\ F{\isadigit{2}}{\isacharparenright}{\isachardoublequoteclose}\isanewline
\ \ \ \ \isacommand{using}\isamarkupfalse%
\ assms\isanewline
\ \ \ \ \isacommand{by}\isamarkupfalse%
\ {\isacharparenleft}simp\ only{\isacharcolon}\ finite{\isacharunderscore}UnI{\isacharparenright}\isanewline
\ \ \isacommand{then}\isamarkupfalse%
\ \isacommand{show}\isamarkupfalse%
\ {\isachardoublequoteopen}finite\ {\isacharparenleft}atoms\ {\isacharparenleft}F{\isadigit{1}}\ \isactrlbold {\isasymrightarrow}\ F{\isadigit{2}}{\isacharparenright}{\isacharparenright}{\isachardoublequoteclose}\ \ \isanewline
\ \ \ \ \isacommand{by}\isamarkupfalse%
\ {\isacharparenleft}simp\ only{\isacharcolon}\ formula{\isachardot}set{\isacharparenleft}{\isadigit{6}}{\isacharparenright}{\isacharparenright}\isanewline
\isacommand{qed}\isamarkupfalse%
%
\endisatagproof
{\isafoldproof}%
%
\isadelimproof
\isanewline
%
\endisadelimproof
\isanewline
\isacommand{lemma}\isamarkupfalse%
\ atoms{\isacharunderscore}finite{\isacharcolon}\ {\isachardoublequoteopen}finite\ {\isacharparenleft}atoms\ F{\isacharparenright}{\isachardoublequoteclose}\isanewline
%
\isadelimproof
%
\endisadelimproof
%
\isatagproof
\isacommand{proof}\isamarkupfalse%
\ {\isacharparenleft}induction\ F{\isacharparenright}\isanewline
\ \ \isacommand{case}\isamarkupfalse%
\ {\isacharparenleft}Atom\ x{\isacharparenright}\isanewline
\ \ \isacommand{then}\isamarkupfalse%
\ \isacommand{show}\isamarkupfalse%
\ {\isacharquery}case\ \isacommand{by}\isamarkupfalse%
\ {\isacharparenleft}simp\ only{\isacharcolon}\ atoms{\isacharunderscore}finite{\isacharunderscore}atom{\isacharparenright}\isanewline
\isacommand{next}\isamarkupfalse%
\isanewline
\ \ \isacommand{case}\isamarkupfalse%
\ Bot\isanewline
\ \ \isacommand{then}\isamarkupfalse%
\ \isacommand{show}\isamarkupfalse%
\ {\isacharquery}case\ \isacommand{by}\isamarkupfalse%
\ {\isacharparenleft}simp\ only{\isacharcolon}\ atoms{\isacharunderscore}finite{\isacharunderscore}bot{\isacharparenright}\isanewline
\isacommand{next}\isamarkupfalse%
\isanewline
\ \ \isacommand{case}\isamarkupfalse%
\ {\isacharparenleft}Not\ F{\isacharparenright}\isanewline
\ \ \isacommand{then}\isamarkupfalse%
\ \isacommand{show}\isamarkupfalse%
\ {\isacharquery}case\ \isacommand{by}\isamarkupfalse%
\ {\isacharparenleft}simp\ only{\isacharcolon}\ atoms{\isacharunderscore}finite{\isacharunderscore}not{\isacharparenright}\isanewline
\isacommand{next}\isamarkupfalse%
\isanewline
\ \ \isacommand{case}\isamarkupfalse%
\ {\isacharparenleft}And\ F{\isadigit{1}}\ F{\isadigit{2}}{\isacharparenright}\isanewline
\ \ \isacommand{then}\isamarkupfalse%
\ \isacommand{show}\isamarkupfalse%
\ {\isacharquery}case\ \isacommand{by}\isamarkupfalse%
\ {\isacharparenleft}simp\ only{\isacharcolon}\ atoms{\isacharunderscore}finite{\isacharunderscore}and{\isacharparenright}\isanewline
\isacommand{next}\isamarkupfalse%
\isanewline
\ \ \isacommand{case}\isamarkupfalse%
\ {\isacharparenleft}Or\ F{\isadigit{1}}\ F{\isadigit{2}}{\isacharparenright}\isanewline
\ \ \isacommand{then}\isamarkupfalse%
\ \isacommand{show}\isamarkupfalse%
\ {\isacharquery}case\ \isacommand{by}\isamarkupfalse%
\ {\isacharparenleft}simp\ only{\isacharcolon}\ atoms{\isacharunderscore}finite{\isacharunderscore}or{\isacharparenright}\isanewline
\isacommand{next}\isamarkupfalse%
\isanewline
\ \ \isacommand{case}\isamarkupfalse%
\ {\isacharparenleft}Imp\ F{\isadigit{1}}\ F{\isadigit{2}}{\isacharparenright}\isanewline
\ \ \isacommand{then}\isamarkupfalse%
\ \isacommand{show}\isamarkupfalse%
\ {\isacharquery}case\ \isacommand{by}\isamarkupfalse%
\ {\isacharparenleft}simp\ only{\isacharcolon}\ atoms{\isacharunderscore}finite{\isacharunderscore}imp{\isacharparenright}\isanewline
\isacommand{qed}\isamarkupfalse%
%
\endisatagproof
{\isafoldproof}%
%
\isadelimproof
%
\endisadelimproof
%
\begin{isamarkuptext}%
Su demostración automática es la siguiente.%
\end{isamarkuptext}\isamarkuptrue%
\isacommand{lemma}\isamarkupfalse%
\ {\isachardoublequoteopen}finite\ {\isacharparenleft}atoms\ F{\isacharparenright}{\isachardoublequoteclose}\ \isanewline
%
\isadelimproof
\ \ %
\endisadelimproof
%
\isatagproof
\isacommand{by}\isamarkupfalse%
\ {\isacharparenleft}induction\ F{\isacharparenright}\ simp{\isacharunderscore}all%
\endisatagproof
{\isafoldproof}%
%
\isadelimproof
%
\endisadelimproof
%
\isadelimdocument
%
\endisadelimdocument
%
\isatagdocument
%
\isamarkupsection{Subfórmulas%
}
\isamarkuptrue%
%
\endisatagdocument
{\isafolddocument}%
%
\isadelimdocument
%
\endisadelimdocument
%
\begin{isamarkuptext}%
Veamos la noción de subfórmulas.

  \begin{definicion}
  El conjunto de subfórmulas de una fórmula \isa{F}, notada \isa{Subf{\isacharparenleft}F{\isacharparenright}}, se 
  define recursivamente como:
    \begin{itemize}
      \item \isa{{\isacharbraceleft}F{\isacharbraceright}} si \isa{F} es una fórmula atómica.
      \item \isa{{\isacharbraceleft}{\isasymbottom}{\isacharbraceright}} si \isa{F} es \isa{{\isasymbottom}}.
      \item \isa{{\isacharbraceleft}F{\isacharbraceright}\ {\isasymunion}\ Subf{\isacharparenleft}G{\isacharparenright}} si \isa{F} es \isa{{\isasymnot}G}.
      \item \isa{{\isacharbraceleft}F{\isacharbraceright}\ {\isasymunion}\ Subf{\isacharparenleft}G{\isacharparenright}\ {\isasymunion}\ Subf{\isacharparenleft}H{\isacharparenright}} si \isa{F} es \isa{G{\isacharasterisk}H} donde \isa{{\isacharasterisk}} es 
        cualquier conectiva binaria.
    \end{itemize}
  \end{definicion}

  Para proceder a la formalización de Isabelle, seguiremos dos etapas. 
  En primer lugar, definimos la función primitiva recursiva 
  \isa{subformulae}. Esta nos devolverá la lista de todas las 
  subfórmulas de una fórmula original obtenidas recursivamente.%
\end{isamarkuptext}\isamarkuptrue%
\isacommand{primrec}\isamarkupfalse%
\ subformulae\ {\isacharcolon}{\isacharcolon}\ {\isachardoublequoteopen}{\isacharprime}a\ formula\ {\isasymRightarrow}\ {\isacharprime}a\ formula\ list{\isachardoublequoteclose}\ \isakeyword{where}\isanewline
\ \ {\isachardoublequoteopen}subformulae\ {\isacharparenleft}Atom\ k{\isacharparenright}\ {\isacharequal}\ {\isacharbrackleft}Atom\ k{\isacharbrackright}{\isachardoublequoteclose}\ \isanewline
{\isacharbar}\ {\isachardoublequoteopen}subformulae\ {\isasymbottom}\ \ \ \ \ \ \ \ {\isacharequal}\ {\isacharbrackleft}{\isasymbottom}{\isacharbrackright}{\isachardoublequoteclose}\ \isanewline
{\isacharbar}\ {\isachardoublequoteopen}subformulae\ {\isacharparenleft}\isactrlbold {\isasymnot}\ F{\isacharparenright}\ \ \ \ {\isacharequal}\ {\isacharparenleft}\isactrlbold {\isasymnot}\ F{\isacharparenright}\ {\isacharhash}\ subformulae\ F{\isachardoublequoteclose}\ \isanewline
{\isacharbar}\ {\isachardoublequoteopen}subformulae\ {\isacharparenleft}F\ \isactrlbold {\isasymand}\ G{\isacharparenright}\ \ {\isacharequal}\ {\isacharparenleft}F\ \isactrlbold {\isasymand}\ G{\isacharparenright}\ {\isacharhash}\ subformulae\ F\ {\isacharat}\ subformulae\ G{\isachardoublequoteclose}\ \isanewline
{\isacharbar}\ {\isachardoublequoteopen}subformulae\ {\isacharparenleft}F\ \isactrlbold {\isasymor}\ G{\isacharparenright}\ \ {\isacharequal}\ {\isacharparenleft}F\ \isactrlbold {\isasymor}\ G{\isacharparenright}\ {\isacharhash}\ subformulae\ F\ {\isacharat}\ subformulae\ G{\isachardoublequoteclose}\isanewline
{\isacharbar}\ {\isachardoublequoteopen}subformulae\ {\isacharparenleft}F\ \isactrlbold {\isasymrightarrow}\ G{\isacharparenright}\ {\isacharequal}\ {\isacharparenleft}F\ \isactrlbold {\isasymrightarrow}\ G{\isacharparenright}\ {\isacharhash}\ subformulae\ F\ {\isacharat}\ subformulae\ G{\isachardoublequoteclose}%
\begin{isamarkuptext}%
Observemos que, en la definición anterior, \isa{{\isacharhash}} es el operador que 
  añade un elemento al comienzo de una lista y \isa{{\isacharat}} concatena varias 
  listas. 

  Siguiendo con los ejemplos, apliquemos \isa{subformulae} en
  las distintas fórmulas. En particular, al tratarse de una lista pueden 
  aparecer elementos repetidos como se muestra a continuación.%
\end{isamarkuptext}\isamarkuptrue%
\isacommand{notepad}\isamarkupfalse%
\isanewline
\isakeyword{begin}\isanewline
%
\isadelimproof
\ \ %
\endisadelimproof
%
\isatagproof
\isacommand{fix}\isamarkupfalse%
\ p\ {\isacharcolon}{\isacharcolon}\ {\isacharprime}a\isanewline
\isanewline
\ \ \isacommand{have}\isamarkupfalse%
\ {\isachardoublequoteopen}subformulae\ {\isacharparenleft}Atom\ p{\isacharparenright}\ {\isacharequal}\ {\isacharbrackleft}Atom\ p{\isacharbrackright}{\isachardoublequoteclose}\isanewline
\ \ \ \ \isacommand{by}\isamarkupfalse%
\ simp\isanewline
\isanewline
\ \ \isacommand{have}\isamarkupfalse%
\ {\isachardoublequoteopen}subformulae\ {\isacharparenleft}\isactrlbold {\isasymnot}\ {\isacharparenleft}Atom\ p{\isacharparenright}{\isacharparenright}\ {\isacharequal}\ {\isacharbrackleft}\isactrlbold {\isasymnot}\ {\isacharparenleft}Atom\ p{\isacharparenright}{\isacharcomma}\ Atom\ p{\isacharbrackright}{\isachardoublequoteclose}\isanewline
\ \ \ \ \isacommand{by}\isamarkupfalse%
\ simp\isanewline
\isanewline
\ \ \isacommand{have}\isamarkupfalse%
\ {\isachardoublequoteopen}subformulae\ {\isacharparenleft}{\isacharparenleft}Atom\ p\ \isactrlbold {\isasymrightarrow}\ Atom\ q{\isacharparenright}\ \isactrlbold {\isasymor}\ Atom\ r{\isacharparenright}\ {\isacharequal}\ \isanewline
\ \ \ \ \ \ \ {\isacharbrackleft}{\isacharparenleft}Atom\ p\ \isactrlbold {\isasymrightarrow}\ Atom\ q{\isacharparenright}\ \isactrlbold {\isasymor}\ Atom\ r{\isacharcomma}\ Atom\ p\ \isactrlbold {\isasymrightarrow}\ Atom\ q{\isacharcomma}\ Atom\ p{\isacharcomma}\ \isanewline
\ \ \ \ \ \ \ \ Atom\ q{\isacharcomma}\ Atom\ r{\isacharbrackright}{\isachardoublequoteclose}\isanewline
\ \ \ \ \isacommand{by}\isamarkupfalse%
\ simp\isanewline
\isanewline
\ \ \isacommand{have}\isamarkupfalse%
\ {\isachardoublequoteopen}subformulae\ {\isacharparenleft}Atom\ p\ \isactrlbold {\isasymand}\ {\isasymbottom}{\isacharparenright}\ {\isacharequal}\ {\isacharbrackleft}Atom\ p\ \isactrlbold {\isasymand}\ {\isasymbottom}{\isacharcomma}\ Atom\ p{\isacharcomma}\ {\isasymbottom}{\isacharbrackright}{\isachardoublequoteclose}\isanewline
\ \ \ \ \isacommand{by}\isamarkupfalse%
\ simp\isanewline
\isanewline
\ \ \isacommand{have}\isamarkupfalse%
\ {\isachardoublequoteopen}subformulae\ {\isacharparenleft}Atom\ p\ \isactrlbold {\isasymor}\ Atom\ p{\isacharparenright}\ {\isacharequal}\ \isanewline
\ \ \ \ \ \ \ {\isacharbrackleft}Atom\ p\ \isactrlbold {\isasymor}\ Atom\ p{\isacharcomma}\ Atom\ p{\isacharcomma}\ Atom\ p{\isacharbrackright}{\isachardoublequoteclose}\isanewline
\ \ \ \ \isacommand{by}\isamarkupfalse%
\ simp%
\endisatagproof
{\isafoldproof}%
%
\isadelimproof
\isanewline
%
\endisadelimproof
\isacommand{end}\isamarkupfalse%
%
\begin{isamarkuptext}%
En la segunda etapa de formalización, definimos 
  \isa{setSubformulae}, que convierte al tipo conjunto la lista de 
  subfórmulas anterior.%
\end{isamarkuptext}\isamarkuptrue%
\isacommand{abbreviation}\isamarkupfalse%
\ setSubformulae\ {\isacharcolon}{\isacharcolon}\ {\isachardoublequoteopen}{\isacharprime}a\ formula\ {\isasymRightarrow}\ {\isacharprime}a\ formula\ set{\isachardoublequoteclose}\ \isakeyword{where}\isanewline
\ \ {\isachardoublequoteopen}setSubformulae\ F\ {\isasymequiv}\ set\ {\isacharparenleft}subformulae\ F{\isacharparenright}{\isachardoublequoteclose}%
\begin{isamarkuptext}%
De este modo, la función \isa{setSubformulae} es la formalización
  en Isabelle de \isa{Subf{\isacharparenleft}·{\isacharparenright}}. En Isabelle, primero hemos definido la lista 
  de subfórmulas pues, en algunos casos, es más sencilla la prueba de 
  resultados sobre este tipo. 
  Algunas de las ventajas del tipo conjuntos son la eliminación de 
  elementos repetidos o las operaciones propias de teoría de conjuntos. 
  Observemos los siguientes ejemplos con el tipo de conjuntos.%
\end{isamarkuptext}\isamarkuptrue%
\isacommand{notepad}\isamarkupfalse%
\isanewline
\isakeyword{begin}\isanewline
%
\isadelimproof
\ \ %
\endisadelimproof
%
\isatagproof
\isacommand{fix}\isamarkupfalse%
\ p\ q\ r\ {\isacharcolon}{\isacharcolon}\ {\isacharprime}a\isanewline
\isanewline
\ \ \isacommand{have}\isamarkupfalse%
\ {\isachardoublequoteopen}setSubformulae\ {\isacharparenleft}Atom\ p\ \isactrlbold {\isasymor}\ Atom\ p{\isacharparenright}\ {\isacharequal}\ {\isacharbraceleft}Atom\ p\ \isactrlbold {\isasymor}\ Atom\ p{\isacharcomma}\ Atom\ p{\isacharbraceright}{\isachardoublequoteclose}\isanewline
\ \ \ \ \isacommand{by}\isamarkupfalse%
\ simp\isanewline
\ \ \isanewline
\ \ \isacommand{have}\isamarkupfalse%
\ {\isachardoublequoteopen}setSubformulae\ {\isacharparenleft}{\isacharparenleft}Atom\ p\ \isactrlbold {\isasymrightarrow}\ Atom\ q{\isacharparenright}\ \isactrlbold {\isasymor}\ Atom\ r{\isacharparenright}\ {\isacharequal}\isanewline
\ \ \ \ \ \ \ \ {\isacharbraceleft}{\isacharparenleft}Atom\ p\ \isactrlbold {\isasymrightarrow}\ Atom\ q{\isacharparenright}\ \isactrlbold {\isasymor}\ Atom\ r{\isacharcomma}\ Atom\ p\ \isactrlbold {\isasymrightarrow}\ Atom\ q{\isacharcomma}\ Atom\ p{\isacharcomma}\ \isanewline
\ \ \ \ \ \ \ \ \ Atom\ q{\isacharcomma}\ Atom\ r{\isacharbraceright}{\isachardoublequoteclose}\isanewline
\ \ \isacommand{by}\isamarkupfalse%
\ auto%
\endisatagproof
{\isafoldproof}%
%
\isadelimproof
\ \ \ \isanewline
%
\endisadelimproof
\isacommand{end}\isamarkupfalse%
%
\begin{isamarkuptext}%
Por otro lado, debemos señalar que el uso de 
  \isa{abbreviation} para definir \isa{setSubformulae} no es 
  arbitrario. No es una definición propiamente dicha, sino 
  una forma de nombrar la composición de las funciones \isa{set} y 
  \isa{subformulae}.


  En primer lugar, veamos que \isa{setSubformulae} es una
  formalización de \isa{Subf} en Isabelle. Para ello 
  utilizaremos el siguiente resultado sobre listas, probado como sigue.%
\end{isamarkuptext}\isamarkuptrue%
\isacommand{lemma}\isamarkupfalse%
\ set{\isacharunderscore}insert{\isacharcolon}\ {\isachardoublequoteopen}set\ {\isacharparenleft}x\ {\isacharhash}\ ys{\isacharparenright}\ {\isacharequal}\ {\isacharbraceleft}x{\isacharbraceright}\ {\isasymunion}\ set\ ys{\isachardoublequoteclose}\isanewline
%
\isadelimproof
\ \ %
\endisadelimproof
%
\isatagproof
\isacommand{by}\isamarkupfalse%
\ {\isacharparenleft}simp\ only{\isacharcolon}\ list{\isachardot}set{\isacharparenleft}{\isadigit{2}}{\isacharparenright}\ Un{\isacharunderscore}insert{\isacharunderscore}left\ sup{\isacharunderscore}bot{\isachardot}left{\isacharunderscore}neutral{\isacharparenright}%
\endisatagproof
{\isafoldproof}%
%
\isadelimproof
%
\endisadelimproof
%
\begin{isamarkuptext}%
Por tanto, obtenemos la equivalencia como resultado de los 
  siguientes lemas, que aparecen demostrados de manera detallada.%
\end{isamarkuptext}\isamarkuptrue%
\isacommand{lemma}\isamarkupfalse%
\ setSubformulae{\isacharunderscore}atom{\isacharcolon}\isanewline
\ \ {\isachardoublequoteopen}setSubformulae\ {\isacharparenleft}Atom\ p{\isacharparenright}\ {\isacharequal}\ {\isacharbraceleft}Atom\ p{\isacharbraceright}{\isachardoublequoteclose}\isanewline
%
\isadelimproof
\ \ \ \ %
\endisadelimproof
%
\isatagproof
\isacommand{by}\isamarkupfalse%
\ {\isacharparenleft}simp\ only{\isacharcolon}\ subformulae{\isachardot}simps{\isacharparenleft}{\isadigit{1}}{\isacharparenright}\ list{\isachardot}set{\isacharparenright}%
\endisatagproof
{\isafoldproof}%
%
\isadelimproof
\isanewline
%
\endisadelimproof
\isanewline
\isacommand{lemma}\isamarkupfalse%
\ setSubformulae{\isacharunderscore}bot{\isacharcolon}\isanewline
\ \ {\isachardoublequoteopen}setSubformulae\ {\isacharparenleft}{\isasymbottom}{\isacharparenright}\ {\isacharequal}\ {\isacharbraceleft}{\isasymbottom}{\isacharbraceright}{\isachardoublequoteclose}\isanewline
%
\isadelimproof
\ \ \ \ %
\endisadelimproof
%
\isatagproof
\isacommand{by}\isamarkupfalse%
\ {\isacharparenleft}simp\ only{\isacharcolon}\ subformulae{\isachardot}simps{\isacharparenleft}{\isadigit{2}}{\isacharparenright}\ list{\isachardot}set{\isacharparenright}%
\endisatagproof
{\isafoldproof}%
%
\isadelimproof
\isanewline
%
\endisadelimproof
\isanewline
\isacommand{lemma}\isamarkupfalse%
\ setSubformulae{\isacharunderscore}not{\isacharcolon}\isanewline
\ \ \isakeyword{shows}\ {\isachardoublequoteopen}setSubformulae\ {\isacharparenleft}\isactrlbold {\isasymnot}\ F{\isacharparenright}\ {\isacharequal}\ {\isacharbraceleft}\isactrlbold {\isasymnot}\ F{\isacharbraceright}\ {\isasymunion}\ setSubformulae\ F{\isachardoublequoteclose}\isanewline
%
\isadelimproof
%
\endisadelimproof
%
\isatagproof
\isacommand{proof}\isamarkupfalse%
\ {\isacharminus}\isanewline
\ \ \isacommand{have}\isamarkupfalse%
\ {\isachardoublequoteopen}setSubformulae\ {\isacharparenleft}\isactrlbold {\isasymnot}\ F{\isacharparenright}\ {\isacharequal}\ set\ {\isacharparenleft}\isactrlbold {\isasymnot}\ F\ {\isacharhash}\ subformulae\ F{\isacharparenright}{\isachardoublequoteclose}\isanewline
\ \ \ \ \isacommand{by}\isamarkupfalse%
\ {\isacharparenleft}simp\ only{\isacharcolon}\ subformulae{\isachardot}simps{\isacharparenleft}{\isadigit{3}}{\isacharparenright}{\isacharparenright}\isanewline
\ \ \isacommand{also}\isamarkupfalse%
\ \isacommand{have}\isamarkupfalse%
\ {\isachardoublequoteopen}{\isasymdots}\ {\isacharequal}\ {\isacharbraceleft}\isactrlbold {\isasymnot}\ F{\isacharbraceright}\ {\isasymunion}\ set\ {\isacharparenleft}subformulae\ F{\isacharparenright}{\isachardoublequoteclose}\isanewline
\ \ \ \ \isacommand{by}\isamarkupfalse%
\ {\isacharparenleft}simp\ only{\isacharcolon}\ set{\isacharunderscore}insert{\isacharparenright}\isanewline
\ \ \isacommand{finally}\isamarkupfalse%
\ \isacommand{show}\isamarkupfalse%
\ {\isacharquery}thesis\isanewline
\ \ \ \ \isacommand{by}\isamarkupfalse%
\ this\isanewline
\isacommand{qed}\isamarkupfalse%
%
\endisatagproof
{\isafoldproof}%
%
\isadelimproof
\isanewline
%
\endisadelimproof
\isanewline
\isacommand{lemma}\isamarkupfalse%
\ setSubformulae{\isacharunderscore}and{\isacharcolon}\ \isanewline
\ \ {\isachardoublequoteopen}setSubformulae\ {\isacharparenleft}F{\isadigit{1}}\ \isactrlbold {\isasymand}\ F{\isadigit{2}}{\isacharparenright}\ \isanewline
\ \ \ {\isacharequal}\ {\isacharbraceleft}F{\isadigit{1}}\ \isactrlbold {\isasymand}\ F{\isadigit{2}}{\isacharbraceright}\ {\isasymunion}\ {\isacharparenleft}setSubformulae\ F{\isadigit{1}}\ {\isasymunion}\ setSubformulae\ F{\isadigit{2}}{\isacharparenright}{\isachardoublequoteclose}\isanewline
%
\isadelimproof
%
\endisadelimproof
%
\isatagproof
\isacommand{proof}\isamarkupfalse%
\ {\isacharminus}\isanewline
\ \ \isacommand{have}\isamarkupfalse%
\ {\isachardoublequoteopen}setSubformulae\ {\isacharparenleft}F{\isadigit{1}}\ \isactrlbold {\isasymand}\ F{\isadigit{2}}{\isacharparenright}\ \isanewline
\ \ \ \ \ \ \ \ {\isacharequal}\ set\ {\isacharparenleft}{\isacharparenleft}F{\isadigit{1}}\ \isactrlbold {\isasymand}\ F{\isadigit{2}}{\isacharparenright}\ {\isacharhash}\ {\isacharparenleft}subformulae\ F{\isadigit{1}}\ {\isacharat}\ subformulae\ F{\isadigit{2}}{\isacharparenright}{\isacharparenright}{\isachardoublequoteclose}\isanewline
\ \ \ \ \isacommand{by}\isamarkupfalse%
\ {\isacharparenleft}simp\ only{\isacharcolon}\ subformulae{\isachardot}simps{\isacharparenleft}{\isadigit{4}}{\isacharparenright}{\isacharparenright}\isanewline
\ \ \isacommand{also}\isamarkupfalse%
\ \isacommand{have}\isamarkupfalse%
\ {\isachardoublequoteopen}{\isasymdots}\ {\isacharequal}\ {\isacharbraceleft}F{\isadigit{1}}\ \isactrlbold {\isasymand}\ F{\isadigit{2}}{\isacharbraceright}\ {\isasymunion}\ {\isacharparenleft}set\ {\isacharparenleft}subformulae\ F{\isadigit{1}}\ {\isacharat}\ subformulae\ F{\isadigit{2}}{\isacharparenright}{\isacharparenright}{\isachardoublequoteclose}\isanewline
\ \ \ \ \isacommand{by}\isamarkupfalse%
\ {\isacharparenleft}simp\ only{\isacharcolon}\ set{\isacharunderscore}insert{\isacharparenright}\isanewline
\ \ \isacommand{also}\isamarkupfalse%
\ \isacommand{have}\isamarkupfalse%
\ {\isachardoublequoteopen}{\isasymdots}\ {\isacharequal}\ {\isacharbraceleft}F{\isadigit{1}}\ \isactrlbold {\isasymand}\ F{\isadigit{2}}{\isacharbraceright}\ {\isasymunion}\ {\isacharparenleft}setSubformulae\ F{\isadigit{1}}\ {\isasymunion}\ setSubformulae\ F{\isadigit{2}}{\isacharparenright}{\isachardoublequoteclose}\isanewline
\ \ \ \ \isacommand{by}\isamarkupfalse%
\ {\isacharparenleft}simp\ only{\isacharcolon}\ set{\isacharunderscore}append{\isacharparenright}\isanewline
\ \ \isacommand{finally}\isamarkupfalse%
\ \isacommand{show}\isamarkupfalse%
\ {\isacharquery}thesis\isanewline
\ \ \ \ \isacommand{by}\isamarkupfalse%
\ this\isanewline
\isacommand{qed}\isamarkupfalse%
%
\endisatagproof
{\isafoldproof}%
%
\isadelimproof
\isanewline
%
\endisadelimproof
\isanewline
\isacommand{lemma}\isamarkupfalse%
\ setSubformulae{\isacharunderscore}or{\isacharcolon}\ \isanewline
\ \ {\isachardoublequoteopen}setSubformulae\ {\isacharparenleft}F{\isadigit{1}}\ \isactrlbold {\isasymor}\ F{\isadigit{2}}{\isacharparenright}\ \isanewline
\ \ \ {\isacharequal}\ {\isacharbraceleft}F{\isadigit{1}}\ \isactrlbold {\isasymor}\ F{\isadigit{2}}{\isacharbraceright}\ {\isasymunion}\ {\isacharparenleft}setSubformulae\ F{\isadigit{1}}\ {\isasymunion}\ setSubformulae\ F{\isadigit{2}}{\isacharparenright}{\isachardoublequoteclose}\isanewline
%
\isadelimproof
%
\endisadelimproof
%
\isatagproof
\isacommand{proof}\isamarkupfalse%
\ {\isacharminus}\isanewline
\ \ \isacommand{have}\isamarkupfalse%
\ {\isachardoublequoteopen}setSubformulae\ {\isacharparenleft}F{\isadigit{1}}\ \isactrlbold {\isasymor}\ F{\isadigit{2}}{\isacharparenright}\ \isanewline
\ \ \ \ \ \ \ \ {\isacharequal}\ set\ {\isacharparenleft}{\isacharparenleft}F{\isadigit{1}}\ \isactrlbold {\isasymor}\ F{\isadigit{2}}{\isacharparenright}\ {\isacharhash}\ {\isacharparenleft}subformulae\ F{\isadigit{1}}\ {\isacharat}\ subformulae\ F{\isadigit{2}}{\isacharparenright}{\isacharparenright}{\isachardoublequoteclose}\isanewline
\ \ \ \ \isacommand{by}\isamarkupfalse%
\ {\isacharparenleft}simp\ only{\isacharcolon}\ subformulae{\isachardot}simps{\isacharparenleft}{\isadigit{5}}{\isacharparenright}{\isacharparenright}\isanewline
\ \ \isacommand{also}\isamarkupfalse%
\ \isacommand{have}\isamarkupfalse%
\ {\isachardoublequoteopen}{\isasymdots}\ {\isacharequal}\ {\isacharbraceleft}F{\isadigit{1}}\ \isactrlbold {\isasymor}\ F{\isadigit{2}}{\isacharbraceright}\ {\isasymunion}\ {\isacharparenleft}set\ {\isacharparenleft}subformulae\ F{\isadigit{1}}\ {\isacharat}\ subformulae\ F{\isadigit{2}}{\isacharparenright}{\isacharparenright}{\isachardoublequoteclose}\isanewline
\ \ \ \ \isacommand{by}\isamarkupfalse%
\ {\isacharparenleft}simp\ only{\isacharcolon}\ set{\isacharunderscore}insert{\isacharparenright}\isanewline
\ \ \isacommand{also}\isamarkupfalse%
\ \isacommand{have}\isamarkupfalse%
\ {\isachardoublequoteopen}{\isasymdots}\ {\isacharequal}\ {\isacharbraceleft}F{\isadigit{1}}\ \isactrlbold {\isasymor}\ F{\isadigit{2}}{\isacharbraceright}\ {\isasymunion}\ {\isacharparenleft}setSubformulae\ F{\isadigit{1}}\ {\isasymunion}\ setSubformulae\ F{\isadigit{2}}{\isacharparenright}{\isachardoublequoteclose}\isanewline
\ \ \ \ \isacommand{by}\isamarkupfalse%
\ {\isacharparenleft}simp\ only{\isacharcolon}\ set{\isacharunderscore}append{\isacharparenright}\isanewline
\ \ \isacommand{finally}\isamarkupfalse%
\ \isacommand{show}\isamarkupfalse%
\ {\isacharquery}thesis\isanewline
\ \ \ \ \isacommand{by}\isamarkupfalse%
\ this\isanewline
\isacommand{qed}\isamarkupfalse%
%
\endisatagproof
{\isafoldproof}%
%
\isadelimproof
\isanewline
%
\endisadelimproof
\isanewline
\isacommand{lemma}\isamarkupfalse%
\ setSubformulae{\isacharunderscore}imp{\isacharcolon}\ \isanewline
\ \ {\isachardoublequoteopen}setSubformulae\ {\isacharparenleft}F{\isadigit{1}}\ \isactrlbold {\isasymrightarrow}\ F{\isadigit{2}}{\isacharparenright}\ \isanewline
\ \ \ {\isacharequal}\ {\isacharbraceleft}F{\isadigit{1}}\ \isactrlbold {\isasymrightarrow}\ F{\isadigit{2}}{\isacharbraceright}\ {\isasymunion}\ {\isacharparenleft}setSubformulae\ F{\isadigit{1}}\ {\isasymunion}\ setSubformulae\ F{\isadigit{2}}{\isacharparenright}{\isachardoublequoteclose}\isanewline
%
\isadelimproof
%
\endisadelimproof
%
\isatagproof
\isacommand{proof}\isamarkupfalse%
\ {\isacharminus}\isanewline
\ \ \isacommand{have}\isamarkupfalse%
\ {\isachardoublequoteopen}setSubformulae\ {\isacharparenleft}F{\isadigit{1}}\ \isactrlbold {\isasymrightarrow}\ F{\isadigit{2}}{\isacharparenright}\ \isanewline
\ \ \ \ \ \ \ \ {\isacharequal}\ set\ {\isacharparenleft}{\isacharparenleft}F{\isadigit{1}}\ \isactrlbold {\isasymrightarrow}\ F{\isadigit{2}}{\isacharparenright}\ {\isacharhash}\ {\isacharparenleft}subformulae\ F{\isadigit{1}}\ {\isacharat}\ subformulae\ F{\isadigit{2}}{\isacharparenright}{\isacharparenright}{\isachardoublequoteclose}\isanewline
\ \ \ \ \isacommand{by}\isamarkupfalse%
\ {\isacharparenleft}simp\ only{\isacharcolon}\ subformulae{\isachardot}simps{\isacharparenleft}{\isadigit{6}}{\isacharparenright}{\isacharparenright}\isanewline
\ \ \isacommand{also}\isamarkupfalse%
\ \isacommand{have}\isamarkupfalse%
\ {\isachardoublequoteopen}{\isasymdots}\ {\isacharequal}\ {\isacharbraceleft}F{\isadigit{1}}\ \isactrlbold {\isasymrightarrow}\ F{\isadigit{2}}{\isacharbraceright}\ {\isasymunion}\ {\isacharparenleft}set\ {\isacharparenleft}subformulae\ F{\isadigit{1}}\ {\isacharat}\ subformulae\ F{\isadigit{2}}{\isacharparenright}{\isacharparenright}{\isachardoublequoteclose}\isanewline
\ \ \ \ \isacommand{by}\isamarkupfalse%
\ {\isacharparenleft}simp\ only{\isacharcolon}\ set{\isacharunderscore}insert{\isacharparenright}\isanewline
\ \ \isacommand{also}\isamarkupfalse%
\ \isacommand{have}\isamarkupfalse%
\ {\isachardoublequoteopen}{\isasymdots}\ {\isacharequal}\ {\isacharbraceleft}F{\isadigit{1}}\ \isactrlbold {\isasymrightarrow}\ F{\isadigit{2}}{\isacharbraceright}\ {\isasymunion}\ {\isacharparenleft}setSubformulae\ F{\isadigit{1}}\ {\isasymunion}\ setSubformulae\ F{\isadigit{2}}{\isacharparenright}{\isachardoublequoteclose}\isanewline
\ \ \ \ \isacommand{by}\isamarkupfalse%
\ {\isacharparenleft}simp\ only{\isacharcolon}\ set{\isacharunderscore}append{\isacharparenright}\isanewline
\ \ \isacommand{finally}\isamarkupfalse%
\ \isacommand{show}\isamarkupfalse%
\ {\isacharquery}thesis\isanewline
\ \ \ \ \isacommand{by}\isamarkupfalse%
\ this\isanewline
\isacommand{qed}\isamarkupfalse%
%
\endisatagproof
{\isafoldproof}%
%
\isadelimproof
%
\endisadelimproof
%
\begin{isamarkuptext}%
Una vez probada la equivalencia, comencemos con los resultados 
  correspondientes a las subfórmulas. En primer lugar, tenemos la 
  siguiente propiedad como consecuencia directa de la equivalencia de 
  funciones anterior.

  \begin{lema}
    Toda fórmula es subfórmula de ella misma.
  \end{lema}

  \begin{demostracion}
    La demostración se hace en cada caso de la estructura de las 
    fórmulas.
  
    Sea \isa{p} fórmula atómica cualquiera. Por definición, tenemos que su
    conjunto de subfórmulas es \isa{{\isacharbraceleft}p{\isacharbraceright}}, luego se tiene la propiedad.
  
    Sea la fórmula \isa{{\isasymbottom}}. Por definición, su conjunto de subfórmulas es
    \isa{{\isacharbraceleft}{\isasymbottom}{\isacharbraceright}}, luego se verifica el resultado.

    Sea la fórmula \isa{{\isasymnot}\ F}. Veamos que pertenece a su conjunto de
    subfórmulas.
    Por definición, tenemos que el conjunto de subfórmulas de \isa{{\isasymnot}\ F} es
    \isa{{\isacharbraceleft}{\isasymnot}\ F{\isacharbraceright}\ {\isasymunion}\ Subf{\isacharparenleft}F{\isacharparenright}}. Por tanto, \isa{{\isasymnot}\ F} pertence a su propio conjunto
    de subfórmulas como queríamos demostrar.

    Sea \isa{{\isacharasterisk}} una conectiva binaria cualquiera y las fórmulas \isa{F} y \isa{G}
    Veamos que \isa{F{\isacharasterisk}G} pertenece a su conjunto de subfórmulas.
    Por definición, tenemos que el conjunto de subfórmulas de \isa{F{\isacharasterisk}G} es
    \isa{{\isacharbraceleft}F{\isacharasterisk}G{\isacharbraceright}\ {\isasymunion}\ Subf{\isacharparenleft}F{\isacharparenright}\ {\isasymunion}\ Subf{\isacharparenleft}G{\isacharparenright}}. Por tanto, \isa{F{\isacharasterisk}G} pertence a su propio 
    conjunto de subfórmulas como queríamos demostrar.
  \end{demostracion}

  Formalicemos ahora el lema con su correspondiente demostración 
  detallada.%
\end{isamarkuptext}\isamarkuptrue%
\isacommand{lemma}\isamarkupfalse%
\ subformulae{\isacharunderscore}self{\isacharcolon}\ {\isachardoublequoteopen}F\ {\isasymin}\ setSubformulae\ F{\isachardoublequoteclose}\isanewline
%
\isadelimproof
%
\endisadelimproof
%
\isatagproof
\isacommand{proof}\isamarkupfalse%
\ {\isacharparenleft}cases\ F{\isacharparenright}\isanewline
\ \ \isacommand{case}\isamarkupfalse%
\ {\isacharparenleft}Atom\ x{\isadigit{1}}{\isacharparenright}\isanewline
\ \ \isacommand{then}\isamarkupfalse%
\ \isacommand{show}\isamarkupfalse%
\ {\isacharquery}thesis\ \isanewline
\ \ \ \ \isacommand{by}\isamarkupfalse%
\ {\isacharparenleft}simp\ only{\isacharcolon}\ singletonI\ setSubformulae{\isacharunderscore}atom{\isacharparenright}\isanewline
\isacommand{next}\isamarkupfalse%
\isanewline
\ \ \isacommand{case}\isamarkupfalse%
\ Bot\isanewline
\ \ \isacommand{then}\isamarkupfalse%
\ \isacommand{show}\isamarkupfalse%
\ {\isacharquery}thesis\isanewline
\ \ \ \ \isacommand{by}\isamarkupfalse%
\ {\isacharparenleft}simp\ only{\isacharcolon}\ singletonI\ setSubformulae{\isacharunderscore}bot{\isacharparenright}\isanewline
\isacommand{next}\isamarkupfalse%
\isanewline
\ \ \isacommand{case}\isamarkupfalse%
\ {\isacharparenleft}Not\ F{\isacharparenright}\isanewline
\ \ \isacommand{then}\isamarkupfalse%
\ \isacommand{show}\isamarkupfalse%
\ {\isacharquery}thesis\isanewline
\ \ \ \ \isacommand{by}\isamarkupfalse%
\ {\isacharparenleft}simp\ only{\isacharcolon}\ singletonI\ UnI{\isadigit{1}}\ setSubformulae{\isacharunderscore}not{\isacharparenright}\isanewline
\isacommand{next}\isamarkupfalse%
\isanewline
\ \ \isacommand{case}\isamarkupfalse%
\ {\isacharparenleft}And\ F{\isadigit{1}}\ F{\isadigit{2}}{\isacharparenright}\isanewline
\ \ \isacommand{then}\isamarkupfalse%
\ \isacommand{show}\isamarkupfalse%
\ {\isacharquery}thesis\isanewline
\ \ \ \isacommand{by}\isamarkupfalse%
\ {\isacharparenleft}simp\ only{\isacharcolon}\ singletonI\ UnI{\isadigit{1}}\ setSubformulae{\isacharunderscore}and{\isacharparenright}\isanewline
\isacommand{next}\isamarkupfalse%
\isanewline
\ \ \isacommand{case}\isamarkupfalse%
\ {\isacharparenleft}Or\ F{\isadigit{1}}\ F{\isadigit{2}}{\isacharparenright}\isanewline
\ \ \isacommand{then}\isamarkupfalse%
\ \isacommand{show}\isamarkupfalse%
\ {\isacharquery}thesis\isanewline
\ \ \ \isacommand{by}\isamarkupfalse%
\ {\isacharparenleft}simp\ only{\isacharcolon}\ singletonI\ UnI{\isadigit{1}}\ setSubformulae{\isacharunderscore}or{\isacharparenright}\isanewline
\isacommand{next}\isamarkupfalse%
\isanewline
\ \ \isacommand{case}\isamarkupfalse%
\ {\isacharparenleft}Imp\ F{\isadigit{1}}\ F{\isadigit{2}}{\isacharparenright}\isanewline
\ \ \isacommand{then}\isamarkupfalse%
\ \isacommand{show}\isamarkupfalse%
\ {\isacharquery}thesis\isanewline
\ \ \ \isacommand{by}\isamarkupfalse%
\ {\isacharparenleft}simp\ only{\isacharcolon}\ singletonI\ UnI{\isadigit{1}}\ setSubformulae{\isacharunderscore}imp{\isacharparenright}\isanewline
\isacommand{qed}\isamarkupfalse%
%
\endisatagproof
{\isafoldproof}%
%
\isadelimproof
%
\endisadelimproof
%
\begin{isamarkuptext}%
La demostración automática es la siguiente.%
\end{isamarkuptext}\isamarkuptrue%
\isacommand{lemma}\isamarkupfalse%
\ {\isachardoublequoteopen}F\ {\isasymin}\ setSubformulae\ F{\isachardoublequoteclose}\isanewline
%
\isadelimproof
\ \ %
\endisadelimproof
%
\isatagproof
\isacommand{by}\isamarkupfalse%
\ {\isacharparenleft}cases\ F{\isacharparenright}\ simp{\isacharunderscore}all%
\endisatagproof
{\isafoldproof}%
%
\isadelimproof
%
\endisadelimproof
%
\begin{isamarkuptext}%
Procedamos con los demás resultados de la sección. Como hemos 
  señalado con anterioridad, utilizaremos varias propiedades de 
  conjuntos pertenecientes a la teoría 
  \href{https://n9.cl/qatp}{Set.thy} de Isabelle, que apareceran en 
  el glosario final. 

  Además, definiremos dos reglas adicionales que utilizaremos con 
  frecuencia.%
\end{isamarkuptext}\isamarkuptrue%
\isacommand{lemma}\isamarkupfalse%
\ subContUnionRev{\isadigit{1}}{\isacharcolon}\ \isanewline
\ \ \isakeyword{assumes}\ {\isachardoublequoteopen}A\ {\isasymunion}\ B\ {\isasymsubseteq}\ C{\isachardoublequoteclose}\ \isanewline
\ \ \isakeyword{shows}\ \ \ {\isachardoublequoteopen}A\ {\isasymsubseteq}\ C{\isachardoublequoteclose}\isanewline
%
\isadelimproof
%
\endisadelimproof
%
\isatagproof
\isacommand{proof}\isamarkupfalse%
\ {\isacharminus}\isanewline
\ \ \isacommand{have}\isamarkupfalse%
\ {\isachardoublequoteopen}A\ {\isasymsubseteq}\ C\ {\isasymand}\ B\ {\isasymsubseteq}\ C{\isachardoublequoteclose}\isanewline
\ \ \ \ \isacommand{using}\isamarkupfalse%
\ assms\isanewline
\ \ \ \ \isacommand{by}\isamarkupfalse%
\ {\isacharparenleft}simp\ only{\isacharcolon}\ sup{\isachardot}bounded{\isacharunderscore}iff{\isacharparenright}\isanewline
\ \ \isacommand{then}\isamarkupfalse%
\ \isacommand{show}\isamarkupfalse%
\ {\isachardoublequoteopen}A\ {\isasymsubseteq}\ C{\isachardoublequoteclose}\isanewline
\ \ \ \ \isacommand{by}\isamarkupfalse%
\ {\isacharparenleft}rule\ conjunct{\isadigit{1}}{\isacharparenright}\isanewline
\isacommand{qed}\isamarkupfalse%
%
\endisatagproof
{\isafoldproof}%
%
\isadelimproof
\isanewline
%
\endisadelimproof
\isanewline
\isacommand{lemma}\isamarkupfalse%
\ subContUnionRev{\isadigit{2}}{\isacharcolon}\ \isanewline
\ \ \isakeyword{assumes}\ {\isachardoublequoteopen}A\ {\isasymunion}\ B\ {\isasymsubseteq}\ C{\isachardoublequoteclose}\ \isanewline
\ \ \isakeyword{shows}\ \ \ {\isachardoublequoteopen}B\ {\isasymsubseteq}\ C{\isachardoublequoteclose}\isanewline
%
\isadelimproof
%
\endisadelimproof
%
\isatagproof
\isacommand{proof}\isamarkupfalse%
\ {\isacharminus}\isanewline
\ \ \isacommand{have}\isamarkupfalse%
\ {\isachardoublequoteopen}A\ {\isasymsubseteq}\ C\ {\isasymand}\ B\ {\isasymsubseteq}\ C{\isachardoublequoteclose}\isanewline
\ \ \ \ \isacommand{using}\isamarkupfalse%
\ assms\isanewline
\ \ \ \ \isacommand{by}\isamarkupfalse%
\ {\isacharparenleft}simp\ only{\isacharcolon}\ sup{\isachardot}bounded{\isacharunderscore}iff{\isacharparenright}\isanewline
\ \ \isacommand{then}\isamarkupfalse%
\ \isacommand{show}\isamarkupfalse%
\ {\isachardoublequoteopen}B\ {\isasymsubseteq}\ C{\isachardoublequoteclose}\isanewline
\ \ \ \ \isacommand{by}\isamarkupfalse%
\ {\isacharparenleft}rule\ conjunct{\isadigit{2}}{\isacharparenright}\isanewline
\isacommand{qed}\isamarkupfalse%
%
\endisatagproof
{\isafoldproof}%
%
\isadelimproof
%
\endisadelimproof
%
\begin{isamarkuptext}%
Sus correspondientes demostraciones automáticas se muestran a 
  continuación.%
\end{isamarkuptext}\isamarkuptrue%
\isacommand{lemma}\isamarkupfalse%
\ {\isachardoublequoteopen}A\ {\isasymunion}\ B\ {\isasymsubseteq}\ C\ {\isasymLongrightarrow}\ A\ {\isasymsubseteq}\ C{\isachardoublequoteclose}\isanewline
%
\isadelimproof
\ \ %
\endisadelimproof
%
\isatagproof
\isacommand{by}\isamarkupfalse%
\ simp%
\endisatagproof
{\isafoldproof}%
%
\isadelimproof
\isanewline
%
\endisadelimproof
\isanewline
\isacommand{lemma}\isamarkupfalse%
\ {\isachardoublequoteopen}A\ {\isasymunion}\ B\ {\isasymsubseteq}\ C\ {\isasymLongrightarrow}\ B\ {\isasymsubseteq}\ C{\isachardoublequoteclose}\isanewline
%
\isadelimproof
\ \ %
\endisadelimproof
%
\isatagproof
\isacommand{by}\isamarkupfalse%
\ simp%
\endisatagproof
{\isafoldproof}%
%
\isadelimproof
%
\endisadelimproof
%
\begin{isamarkuptext}%
Veamos ahora los distintos resultados sobre subfórmulas.

  \begin{lema}
    Todas las fórmulas atómicas de una fórmula son subfórmulas.
  \end{lema}

  \begin{demostracion}
    Aclaremos que el conjunto de las fórmulas atómicas de una fórmula 
    cualquiera está formado a partir de cada elemento de su conjunto de 
    variables proposicionales. 
    Queremos demostrar que este conjunto está contenido en el conjunto 
    de\\ subfórmulas de dicha fórmula.
    De este modo, la prueba seguirá el esquema inductivo para la 
    estructura de fórmulas. Veamos cada caso:
  
    Consideremos la fórmula atómica \isa{p} cualquiera. Como su
    conjunto de átomos es \isa{{\isacharbraceleft}p{\isacharbraceright}}, el conjunto de sus fórmulas atómicas
    correspondiente será \isa{{\isacharbraceleft}p{\isacharbraceright}}. Por otro lado, su conjunto de
    subfórmulas es también \isa{{\isacharbraceleft}p{\isacharbraceright}}, luego el conjunto de sus fórmulas 
    atómicas está contenido en el conjunto de sus subfórmulas como 
    queríamos demostrar.

    Sea la fórmula \isa{{\isasymbottom}}. Como su conjunto de átomos es vacío, es claro 
    que el conjunto de sus fórmulas atómicas es también el vacío y, por
    tanto, está contenido en el conjunto de sus subfórmulas.

    Sea la fórmula \isa{F} tal que el conjunto de sus fórmulas atómicas está
    contenido en el conjunto de sus subfórmulas. Probemos el resultado 
    para \isa{{\isasymnot}\ F}. 
    En primer lugar, sabemos que los 
    conjuntos de variables proposicionales de \isa{F} y \isa{{\isasymnot}\ F} coinciden, 
    luego tendrán igual conjunto de fórmulas atómicas. Por lo tanto,
    por hipótesis de inducción tenemos que el conjunto de fórmulas
    atómicas de \isa{F} está contenido en el conjunto de subfórmulas de 
    \isa{F}. Por otro lado, como el conjunto de subfórmulas de \isa{{\isasymnot}\ F} está 
    definido como\\ \isa{Subf{\isacharparenleft}{\isasymnot}\ F{\isacharparenright}\ {\isacharequal}\ {\isacharbraceleft}{\isasymnot}\ F{\isacharbraceright}\ {\isasymunion}\ Subf{\isacharparenleft}F{\isacharparenright}}, tenemos que el 
    el conjunto de subfórmulas de \isa{F} está contenido en el de \isa{{\isasymnot}\ F}.
    Por tanto, por propiedades de contención, 
    tenemos que el conjunto de fórmulas atómicas de \isa{{\isasymnot}\ F} está 
    contenido en el conjunto de subfórmulas de \isa{{\isasymnot}\ F} como queríamos 
    demostrar.

    Sean las fórmulas \isa{F} y \isa{G} tales que sus conjuntos de fórmulas 
    atómicas están contenidos en sus conjuntos de subfórmulas 
    respectivamente. Probemos ahora el resultado para \isa{F{\isacharasterisk}G}, donde \isa{{\isacharasterisk}}
    simboliza una conectiva binaria cualquiera.
    En primer lugar, sabemos que el conjunto de átomos de \isa{F{\isacharasterisk}G}
    es la unión de sus correspondientes conjuntos de átomos. De este
    modo, el conjunto de fórmulas atómicas de \isa{F{\isacharasterisk}G} será la unión del 
    conjunto de fórmulas atómicas de \isa{F} y el correspondiente de \isa{G}. 
    Por tanto, por hipótesis de inducción tenemos que el conjunto de 
    fórmulas atómicas de \isa{F{\isacharasterisk}G} está contenido en la unión del conjunto
    de subfórmulas de \isa{F} y el conjunto de subfórmulas de \isa{G}. Como el
    conjunto de subfórmulas de \isa{F{\isacharasterisk}G} se define como\\
    \isa{Subf{\isacharparenleft}F{\isacharasterisk}G{\isacharparenright}\ {\isacharequal}\ {\isacharbraceleft}F{\isacharasterisk}G{\isacharbraceright}\ {\isasymunion}\ Subf{\isacharparenleft}F{\isacharparenright}\ {\isasymunion}\ Subf{\isacharparenleft}G{\isacharparenright}}, tenemos que la unión
    de los conjuntos de subfórmulas de \isa{F} y \isa{G} está contenida en el
    conjunto de subfórmulas de \isa{F{\isacharasterisk}G}. Por tanto, por propiedades
    de la contención, tenemos que le conjunto de fórmulas atómicas de
    \isa{F{\isacharasterisk}G} está contenido en el conjunto de subfórmulas de \isa{F{\isacharasterisk}G} como 
    queríamos demostrar.  
  \end{demostracion}

  En Isabelle, se especifica como sigue.%
\end{isamarkuptext}\isamarkuptrue%
\isacommand{lemma}\isamarkupfalse%
\ {\isachardoublequoteopen}Atom\ {\isacharbackquote}\ atoms\ F\ {\isasymsubseteq}\ setSubformulae\ F{\isachardoublequoteclose}\isanewline
%
\isadelimproof
\ \ %
\endisadelimproof
%
\isatagproof
\isacommand{oops}\isamarkupfalse%
%
\endisatagproof
{\isafoldproof}%
%
\isadelimproof
%
\endisadelimproof
%
\begin{isamarkuptext}%
Debemos observar que \isa{Atom\ {\isacharbackquote}\ atoms\ F} construye las fórmulas 
  atómicas a partir de cada uno de los elementos de \isa{atoms\ F}, creando 
  un conjunto de fórmulas atómicas. Para ello emplea el infijo \isa{{\isacharbackquote}} 
  definido como notación abreviada de \isa{{\isacharparenleft}{\isacharbackquote}{\isacharparenright}} que calcula la 
  imagen de un conjunto en la teoría \href{https://n9.cl/qatp}{Set.thy}.

  \begin{itemize}
    \item[] \isa{f\ {\isacharbackquote}\ A\ {\isacharequal}\ {\isacharbraceleft}y\ {\isacharbar}\ {\isasymexists}x{\isasymin}A{\isachardot}\ y\ {\isacharequal}\ f\ x{\isacharbraceright}} 
      \hfill (\isa{image{\isacharunderscore}def})
  \end{itemize}

  Para aclarar su funcionamiento, veamos ejemplos para distintos casos 
  de fórmulas.%
\end{isamarkuptext}\isamarkuptrue%
\isacommand{notepad}\isamarkupfalse%
\isanewline
\isakeyword{begin}\isanewline
%
\isadelimproof
\ \ %
\endisadelimproof
%
\isatagproof
\isacommand{fix}\isamarkupfalse%
\ p\ q\ r\ {\isacharcolon}{\isacharcolon}\ {\isacharprime}a\isanewline
\isanewline
\ \ \isacommand{have}\isamarkupfalse%
\ {\isachardoublequoteopen}Atom\ {\isacharbackquote}atoms\ {\isacharparenleft}Atom\ p\ \isactrlbold {\isasymor}\ {\isasymbottom}{\isacharparenright}\ {\isacharequal}\ {\isacharbraceleft}Atom\ p{\isacharbraceright}{\isachardoublequoteclose}\isanewline
\ \ \ \ \isacommand{by}\isamarkupfalse%
\ simp\isanewline
\isanewline
\ \ \isacommand{have}\isamarkupfalse%
\ {\isachardoublequoteopen}Atom\ {\isacharbackquote}atoms\ {\isacharparenleft}{\isacharparenleft}Atom\ p\ \isactrlbold {\isasymrightarrow}\ Atom\ q{\isacharparenright}\ \isactrlbold {\isasymor}\ Atom\ r{\isacharparenright}\ {\isacharequal}\ \isanewline
\ \ \ \ \ \ \ {\isacharbraceleft}Atom\ p{\isacharcomma}\ Atom\ q{\isacharcomma}\ Atom\ r{\isacharbraceright}{\isachardoublequoteclose}\isanewline
\ \ \ \ \isacommand{by}\isamarkupfalse%
\ auto\ \isanewline
\isanewline
\ \ \isacommand{have}\isamarkupfalse%
\ {\isachardoublequoteopen}Atom\ {\isacharbackquote}atoms\ {\isacharparenleft}{\isacharparenleft}Atom\ p\ \isactrlbold {\isasymrightarrow}\ Atom\ p{\isacharparenright}\ \isactrlbold {\isasymor}\ Atom\ r{\isacharparenright}\ {\isacharequal}\ \isanewline
\ \ \ \ \ \ \ {\isacharbraceleft}Atom\ p{\isacharcomma}\ Atom\ r{\isacharbraceright}{\isachardoublequoteclose}\isanewline
\ \ \ \ \isacommand{by}\isamarkupfalse%
\ auto%
\endisatagproof
{\isafoldproof}%
%
\isadelimproof
\isanewline
%
\endisadelimproof
\isacommand{end}\isamarkupfalse%
%
\begin{isamarkuptext}%
Además, esta función tiene las siguientes propiedades sobre 
  conjuntos que utilizaremos en la demostración.

  \begin{itemize}
    \item[] \isa{f\ {\isacharbackquote}\ {\isacharparenleft}A\ {\isasymunion}\ B{\isacharparenright}\ {\isacharequal}\ f\ {\isacharbackquote}\ A\ {\isasymunion}\ f\ {\isacharbackquote}\ B} 
      \hfill (\isa{image{\isacharunderscore}Un})
    \item[] \isa{f\ {\isacharbackquote}\ {\isacharparenleft}{\isacharbraceleft}a{\isacharbraceright}\ {\isasymunion}\ B{\isacharparenright}\ {\isacharequal}\ {\isacharbraceleft}f\ a{\isacharbraceright}\ {\isasymunion}\ f\ {\isacharbackquote}\ B} 
      \hfill (\isa{image{\isacharunderscore}insert})
    \item[] \isa{f\ {\isacharbackquote}\ {\isasymemptyset}\ {\isacharequal}\ {\isasymemptyset}} 
      \hfill (\isa{image{\isacharunderscore}empty})
  \end{itemize}

  Una vez hechas las aclaraciones necesarias, comencemos la demostración 
  estructurada. Esta seguirá el esquema inductivo señalado con 
  anterioridad.%
\end{isamarkuptext}\isamarkuptrue%
\isacommand{lemma}\isamarkupfalse%
\ atoms{\isacharunderscore}are{\isacharunderscore}subformulae{\isacharunderscore}atom{\isacharcolon}\ \isanewline
\ \ {\isachardoublequoteopen}Atom\ {\isacharbackquote}\ atoms\ {\isacharparenleft}Atom\ x{\isacharparenright}\ {\isasymsubseteq}\ setSubformulae\ {\isacharparenleft}Atom\ x{\isacharparenright}{\isachardoublequoteclose}\ \isanewline
%
\isadelimproof
%
\endisadelimproof
%
\isatagproof
\isacommand{proof}\isamarkupfalse%
\ {\isacharminus}\isanewline
\ \ \isacommand{have}\isamarkupfalse%
\ {\isachardoublequoteopen}Atom\ {\isacharbackquote}\ atoms\ {\isacharparenleft}Atom\ x{\isacharparenright}\ {\isacharequal}\ Atom\ {\isacharbackquote}\ {\isacharbraceleft}x{\isacharbraceright}{\isachardoublequoteclose}\isanewline
\ \ \ \ \isacommand{by}\isamarkupfalse%
\ {\isacharparenleft}simp\ only{\isacharcolon}\ formula{\isachardot}set{\isacharparenleft}{\isadigit{1}}{\isacharparenright}{\isacharparenright}\isanewline
\ \ \isacommand{also}\isamarkupfalse%
\ \isacommand{have}\isamarkupfalse%
\ {\isachardoublequoteopen}{\isasymdots}\ {\isacharequal}\ {\isacharbraceleft}Atom\ x{\isacharbraceright}{\isachardoublequoteclose}\isanewline
\ \ \ \ \isacommand{by}\isamarkupfalse%
\ {\isacharparenleft}simp\ only{\isacharcolon}\ image{\isacharunderscore}insert\ image{\isacharunderscore}empty{\isacharparenright}\isanewline
\ \ \isacommand{also}\isamarkupfalse%
\ \isacommand{have}\isamarkupfalse%
\ {\isachardoublequoteopen}{\isasymdots}\ {\isacharequal}\ set\ {\isacharbrackleft}Atom\ x{\isacharbrackright}{\isachardoublequoteclose}\isanewline
\ \ \ \ \isacommand{by}\isamarkupfalse%
\ {\isacharparenleft}simp\ only{\isacharcolon}\ list{\isachardot}set{\isacharparenleft}{\isadigit{1}}{\isacharparenright}\ list{\isachardot}set{\isacharparenleft}{\isadigit{2}}{\isacharparenright}{\isacharparenright}\isanewline
\ \ \isacommand{also}\isamarkupfalse%
\ \isacommand{have}\isamarkupfalse%
\ {\isachardoublequoteopen}{\isasymdots}\ {\isacharequal}\ set\ {\isacharparenleft}subformulae\ {\isacharparenleft}Atom\ x{\isacharparenright}{\isacharparenright}{\isachardoublequoteclose}\isanewline
\ \ \ \ \isacommand{by}\isamarkupfalse%
\ {\isacharparenleft}simp\ only{\isacharcolon}\ subformulae{\isachardot}simps{\isacharparenleft}{\isadigit{1}}{\isacharparenright}{\isacharparenright}\isanewline
\ \ \isacommand{finally}\isamarkupfalse%
\ \isacommand{have}\isamarkupfalse%
\ {\isachardoublequoteopen}Atom\ {\isacharbackquote}\ atoms\ {\isacharparenleft}Atom\ x{\isacharparenright}\ {\isacharequal}\ set\ {\isacharparenleft}subformulae\ {\isacharparenleft}Atom\ x{\isacharparenright}{\isacharparenright}{\isachardoublequoteclose}\isanewline
\ \ \ \ \isacommand{by}\isamarkupfalse%
\ this\isanewline
\ \ \isacommand{then}\isamarkupfalse%
\ \isacommand{show}\isamarkupfalse%
\ {\isacharquery}thesis\ \isanewline
\ \ \ \ \isacommand{by}\isamarkupfalse%
\ {\isacharparenleft}simp\ only{\isacharcolon}\ subset{\isacharunderscore}refl{\isacharparenright}\isanewline
\isacommand{qed}\isamarkupfalse%
%
\endisatagproof
{\isafoldproof}%
%
\isadelimproof
\isanewline
%
\endisadelimproof
\isanewline
\isacommand{lemma}\isamarkupfalse%
\ atoms{\isacharunderscore}are{\isacharunderscore}subformulae{\isacharunderscore}bot{\isacharcolon}\ \isanewline
\ \ {\isachardoublequoteopen}Atom\ {\isacharbackquote}\ atoms\ {\isasymbottom}\ {\isasymsubseteq}\ setSubformulae\ {\isasymbottom}{\isachardoublequoteclose}\ \ \isanewline
%
\isadelimproof
%
\endisadelimproof
%
\isatagproof
\isacommand{proof}\isamarkupfalse%
\ {\isacharminus}\isanewline
\ \ \isacommand{have}\isamarkupfalse%
\ {\isachardoublequoteopen}Atom\ {\isacharbackquote}\ atoms\ {\isasymbottom}\ {\isacharequal}\ Atom\ {\isacharbackquote}\ {\isasymemptyset}{\isachardoublequoteclose}\isanewline
\ \ \ \ \isacommand{by}\isamarkupfalse%
\ {\isacharparenleft}simp\ only{\isacharcolon}\ formula{\isachardot}set{\isacharparenleft}{\isadigit{2}}{\isacharparenright}{\isacharparenright}\isanewline
\ \ \isacommand{also}\isamarkupfalse%
\ \isacommand{have}\isamarkupfalse%
\ {\isachardoublequoteopen}{\isasymdots}\ {\isacharequal}\ {\isasymemptyset}{\isachardoublequoteclose}\isanewline
\ \ \ \ \isacommand{by}\isamarkupfalse%
\ {\isacharparenleft}simp\ only{\isacharcolon}\ image{\isacharunderscore}empty{\isacharparenright}\isanewline
\ \ \isacommand{also}\isamarkupfalse%
\ \isacommand{have}\isamarkupfalse%
\ {\isachardoublequoteopen}{\isasymdots}\ {\isasymsubseteq}\ setSubformulae\ {\isasymbottom}{\isachardoublequoteclose}\isanewline
\ \ \ \ \isacommand{by}\isamarkupfalse%
\ {\isacharparenleft}simp\ only{\isacharcolon}\ empty{\isacharunderscore}subsetI{\isacharparenright}\isanewline
\ \ \isacommand{finally}\isamarkupfalse%
\ \isacommand{show}\isamarkupfalse%
\ {\isacharquery}thesis\isanewline
\ \ \ \ \isacommand{by}\isamarkupfalse%
\ this\isanewline
\isacommand{qed}\isamarkupfalse%
%
\endisatagproof
{\isafoldproof}%
%
\isadelimproof
\isanewline
%
\endisadelimproof
\isanewline
\isacommand{lemma}\isamarkupfalse%
\ atoms{\isacharunderscore}are{\isacharunderscore}subformulae{\isacharunderscore}not{\isacharcolon}\ \isanewline
\ \ \isakeyword{assumes}\ {\isachardoublequoteopen}Atom\ {\isacharbackquote}\ atoms\ F\ {\isasymsubseteq}\ setSubformulae\ F{\isachardoublequoteclose}\ \isanewline
\ \ \isakeyword{shows}\ \ \ {\isachardoublequoteopen}Atom\ {\isacharbackquote}\ atoms\ {\isacharparenleft}\isactrlbold {\isasymnot}\ F{\isacharparenright}\ {\isasymsubseteq}\ setSubformulae\ {\isacharparenleft}\isactrlbold {\isasymnot}\ F{\isacharparenright}{\isachardoublequoteclose}\isanewline
%
\isadelimproof
%
\endisadelimproof
%
\isatagproof
\isacommand{proof}\isamarkupfalse%
\ {\isacharminus}\isanewline
\ \ \isacommand{have}\isamarkupfalse%
\ {\isachardoublequoteopen}Atom\ {\isacharbackquote}\ atoms\ {\isacharparenleft}\isactrlbold {\isasymnot}\ F{\isacharparenright}\ {\isacharequal}\ Atom\ {\isacharbackquote}\ atoms\ F{\isachardoublequoteclose}\isanewline
\ \ \ \ \isacommand{by}\isamarkupfalse%
\ {\isacharparenleft}simp\ only{\isacharcolon}\ formula{\isachardot}set{\isacharparenleft}{\isadigit{3}}{\isacharparenright}{\isacharparenright}\isanewline
\ \ \isacommand{also}\isamarkupfalse%
\ \isacommand{have}\isamarkupfalse%
\ {\isachardoublequoteopen}{\isasymdots}\ {\isasymsubseteq}\ setSubformulae\ F{\isachardoublequoteclose}\isanewline
\ \ \ \ \isacommand{by}\isamarkupfalse%
\ {\isacharparenleft}simp\ only{\isacharcolon}\ assms{\isacharparenright}\isanewline
\ \ \isacommand{also}\isamarkupfalse%
\ \isacommand{have}\isamarkupfalse%
\ {\isachardoublequoteopen}{\isasymdots}\ {\isasymsubseteq}\ {\isacharbraceleft}\isactrlbold {\isasymnot}\ F{\isacharbraceright}\ {\isasymunion}\ setSubformulae\ F{\isachardoublequoteclose}\isanewline
\ \ \ \ \isacommand{by}\isamarkupfalse%
\ {\isacharparenleft}simp\ only{\isacharcolon}\ Un{\isacharunderscore}upper{\isadigit{2}}{\isacharparenright}\isanewline
\ \ \isacommand{also}\isamarkupfalse%
\ \isacommand{have}\isamarkupfalse%
\ {\isachardoublequoteopen}{\isasymdots}\ {\isacharequal}\ setSubformulae\ {\isacharparenleft}\isactrlbold {\isasymnot}\ F{\isacharparenright}{\isachardoublequoteclose}\isanewline
\ \ \ \ \isacommand{by}\isamarkupfalse%
\ {\isacharparenleft}simp\ only{\isacharcolon}\ setSubformulae{\isacharunderscore}not{\isacharparenright}\isanewline
\ \ \isacommand{finally}\isamarkupfalse%
\ \isacommand{show}\isamarkupfalse%
\ {\isacharquery}thesis\isanewline
\ \ \ \ \isacommand{by}\isamarkupfalse%
\ this\isanewline
\isacommand{qed}\isamarkupfalse%
%
\endisatagproof
{\isafoldproof}%
%
\isadelimproof
\isanewline
%
\endisadelimproof
\isanewline
\isacommand{lemma}\isamarkupfalse%
\ atoms{\isacharunderscore}are{\isacharunderscore}subformulae{\isacharunderscore}and{\isacharcolon}\ \isanewline
\ \ \isakeyword{assumes}\ {\isachardoublequoteopen}Atom\ {\isacharbackquote}\ atoms\ F{\isadigit{1}}\ {\isasymsubseteq}\ setSubformulae\ F{\isadigit{1}}{\isachardoublequoteclose}\isanewline
\ \ \ \ \ \ \ \ \ \ {\isachardoublequoteopen}Atom\ {\isacharbackquote}\ atoms\ F{\isadigit{2}}\ {\isasymsubseteq}\ setSubformulae\ F{\isadigit{2}}{\isachardoublequoteclose}\isanewline
\ \ \isakeyword{shows}\ \ \ {\isachardoublequoteopen}Atom\ {\isacharbackquote}\ atoms\ {\isacharparenleft}F{\isadigit{1}}\ \isactrlbold {\isasymand}\ F{\isadigit{2}}{\isacharparenright}\ {\isasymsubseteq}\ setSubformulae\ {\isacharparenleft}F{\isadigit{1}}\ \isactrlbold {\isasymand}\ F{\isadigit{2}}{\isacharparenright}{\isachardoublequoteclose}\isanewline
%
\isadelimproof
%
\endisadelimproof
%
\isatagproof
\isacommand{proof}\isamarkupfalse%
\ {\isacharminus}\isanewline
\ \ \isacommand{have}\isamarkupfalse%
\ {\isachardoublequoteopen}Atom\ {\isacharbackquote}\ atoms\ {\isacharparenleft}F{\isadigit{1}}\ \isactrlbold {\isasymand}\ F{\isadigit{2}}{\isacharparenright}\ {\isacharequal}\ Atom\ {\isacharbackquote}\ {\isacharparenleft}atoms\ F{\isadigit{1}}\ {\isasymunion}\ atoms\ F{\isadigit{2}}{\isacharparenright}{\isachardoublequoteclose}\isanewline
\ \ \ \ \isacommand{by}\isamarkupfalse%
\ {\isacharparenleft}simp\ only{\isacharcolon}\ formula{\isachardot}set{\isacharparenleft}{\isadigit{4}}{\isacharparenright}{\isacharparenright}\isanewline
\ \ \isacommand{also}\isamarkupfalse%
\ \isacommand{have}\isamarkupfalse%
\ {\isachardoublequoteopen}{\isasymdots}\ {\isacharequal}\ Atom\ {\isacharbackquote}\ atoms\ F{\isadigit{1}}\ {\isasymunion}\ Atom\ {\isacharbackquote}\ atoms\ F{\isadigit{2}}{\isachardoublequoteclose}\ \isanewline
\ \ \ \ \isacommand{by}\isamarkupfalse%
\ {\isacharparenleft}rule\ image{\isacharunderscore}Un{\isacharparenright}\isanewline
\ \ \isacommand{also}\isamarkupfalse%
\ \isacommand{have}\isamarkupfalse%
\ {\isachardoublequoteopen}{\isasymdots}\ {\isasymsubseteq}\ setSubformulae\ F{\isadigit{1}}\ {\isasymunion}\ setSubformulae\ F{\isadigit{2}}{\isachardoublequoteclose}\isanewline
\ \ \ \ \isacommand{using}\isamarkupfalse%
\ assms\isanewline
\ \ \ \ \isacommand{by}\isamarkupfalse%
\ {\isacharparenleft}rule\ Un{\isacharunderscore}mono{\isacharparenright}\isanewline
\ \ \isacommand{also}\isamarkupfalse%
\ \isacommand{have}\isamarkupfalse%
\ {\isachardoublequoteopen}{\isasymdots}\ {\isasymsubseteq}\ {\isacharbraceleft}F{\isadigit{1}}\ \isactrlbold {\isasymand}\ F{\isadigit{2}}{\isacharbraceright}\ {\isasymunion}\ {\isacharparenleft}setSubformulae\ F{\isadigit{1}}\ {\isasymunion}\ setSubformulae\ F{\isadigit{2}}{\isacharparenright}{\isachardoublequoteclose}\isanewline
\ \ \ \ \isacommand{by}\isamarkupfalse%
\ {\isacharparenleft}simp\ only{\isacharcolon}\ Un{\isacharunderscore}upper{\isadigit{2}}{\isacharparenright}\isanewline
\ \ \isacommand{also}\isamarkupfalse%
\ \isacommand{have}\isamarkupfalse%
\ {\isachardoublequoteopen}{\isasymdots}\ {\isacharequal}\ setSubformulae\ {\isacharparenleft}F{\isadigit{1}}\ \isactrlbold {\isasymand}\ F{\isadigit{2}}{\isacharparenright}{\isachardoublequoteclose}\isanewline
\ \ \ \ \isacommand{by}\isamarkupfalse%
\ {\isacharparenleft}simp\ only{\isacharcolon}\ setSubformulae{\isacharunderscore}and{\isacharparenright}\isanewline
\ \ \isacommand{finally}\isamarkupfalse%
\ \isacommand{show}\isamarkupfalse%
\ {\isacharquery}thesis\isanewline
\ \ \ \ \isacommand{by}\isamarkupfalse%
\ this\isanewline
\isacommand{qed}\isamarkupfalse%
%
\endisatagproof
{\isafoldproof}%
%
\isadelimproof
\isanewline
%
\endisadelimproof
\isanewline
\isacommand{lemma}\isamarkupfalse%
\ atoms{\isacharunderscore}are{\isacharunderscore}subformulae{\isacharunderscore}or{\isacharcolon}\ \isanewline
\ \ \isakeyword{assumes}\ {\isachardoublequoteopen}Atom\ {\isacharbackquote}\ atoms\ F{\isadigit{1}}\ {\isasymsubseteq}\ setSubformulae\ F{\isadigit{1}}{\isachardoublequoteclose}\isanewline
\ \ \ \ \ \ \ \ \ \ {\isachardoublequoteopen}Atom\ {\isacharbackquote}\ atoms\ F{\isadigit{2}}\ {\isasymsubseteq}\ setSubformulae\ F{\isadigit{2}}{\isachardoublequoteclose}\isanewline
\ \ \isakeyword{shows}\ \ \ {\isachardoublequoteopen}Atom\ {\isacharbackquote}\ atoms\ {\isacharparenleft}F{\isadigit{1}}\ \isactrlbold {\isasymor}\ F{\isadigit{2}}{\isacharparenright}\ {\isasymsubseteq}\ setSubformulae\ {\isacharparenleft}F{\isadigit{1}}\ \isactrlbold {\isasymor}\ F{\isadigit{2}}{\isacharparenright}{\isachardoublequoteclose}\isanewline
%
\isadelimproof
%
\endisadelimproof
%
\isatagproof
\isacommand{proof}\isamarkupfalse%
\ {\isacharminus}\isanewline
\ \ \isacommand{have}\isamarkupfalse%
\ {\isachardoublequoteopen}Atom\ {\isacharbackquote}\ atoms\ {\isacharparenleft}F{\isadigit{1}}\ \isactrlbold {\isasymor}\ F{\isadigit{2}}{\isacharparenright}\ {\isacharequal}\ Atom\ {\isacharbackquote}\ {\isacharparenleft}atoms\ F{\isadigit{1}}\ {\isasymunion}\ atoms\ F{\isadigit{2}}{\isacharparenright}{\isachardoublequoteclose}\isanewline
\ \ \ \ \isacommand{by}\isamarkupfalse%
\ {\isacharparenleft}simp\ only{\isacharcolon}\ formula{\isachardot}set{\isacharparenleft}{\isadigit{5}}{\isacharparenright}{\isacharparenright}\isanewline
\ \ \isacommand{also}\isamarkupfalse%
\ \isacommand{have}\isamarkupfalse%
\ {\isachardoublequoteopen}{\isasymdots}\ {\isacharequal}\ Atom\ {\isacharbackquote}\ atoms\ F{\isadigit{1}}\ {\isasymunion}\ Atom\ {\isacharbackquote}\ atoms\ F{\isadigit{2}}{\isachardoublequoteclose}\ \isanewline
\ \ \ \ \isacommand{by}\isamarkupfalse%
\ {\isacharparenleft}rule\ image{\isacharunderscore}Un{\isacharparenright}\isanewline
\ \ \isacommand{also}\isamarkupfalse%
\ \isacommand{have}\isamarkupfalse%
\ {\isachardoublequoteopen}{\isasymdots}\ {\isasymsubseteq}\ setSubformulae\ F{\isadigit{1}}\ {\isasymunion}\ setSubformulae\ F{\isadigit{2}}{\isachardoublequoteclose}\isanewline
\ \ \ \ \isacommand{using}\isamarkupfalse%
\ assms\isanewline
\ \ \ \ \isacommand{by}\isamarkupfalse%
\ {\isacharparenleft}rule\ Un{\isacharunderscore}mono{\isacharparenright}\isanewline
\ \ \isacommand{also}\isamarkupfalse%
\ \isacommand{have}\isamarkupfalse%
\ {\isachardoublequoteopen}{\isasymdots}\ {\isasymsubseteq}\ {\isacharbraceleft}F{\isadigit{1}}\ \isactrlbold {\isasymor}\ F{\isadigit{2}}{\isacharbraceright}\ {\isasymunion}\ {\isacharparenleft}setSubformulae\ F{\isadigit{1}}\ {\isasymunion}\ setSubformulae\ F{\isadigit{2}}{\isacharparenright}{\isachardoublequoteclose}\isanewline
\ \ \ \ \isacommand{by}\isamarkupfalse%
\ {\isacharparenleft}simp\ only{\isacharcolon}\ Un{\isacharunderscore}upper{\isadigit{2}}{\isacharparenright}\isanewline
\ \ \isacommand{also}\isamarkupfalse%
\ \isacommand{have}\isamarkupfalse%
\ {\isachardoublequoteopen}{\isasymdots}\ {\isacharequal}\ setSubformulae\ {\isacharparenleft}F{\isadigit{1}}\ \isactrlbold {\isasymor}\ F{\isadigit{2}}{\isacharparenright}{\isachardoublequoteclose}\isanewline
\ \ \ \ \isacommand{by}\isamarkupfalse%
\ {\isacharparenleft}simp\ only{\isacharcolon}\ setSubformulae{\isacharunderscore}or{\isacharparenright}\isanewline
\ \ \isacommand{finally}\isamarkupfalse%
\ \isacommand{show}\isamarkupfalse%
\ {\isacharquery}thesis\isanewline
\ \ \ \ \isacommand{by}\isamarkupfalse%
\ this\isanewline
\isacommand{qed}\isamarkupfalse%
%
\endisatagproof
{\isafoldproof}%
%
\isadelimproof
\isanewline
%
\endisadelimproof
\isanewline
\isacommand{lemma}\isamarkupfalse%
\ atoms{\isacharunderscore}are{\isacharunderscore}subformulae{\isacharunderscore}imp{\isacharcolon}\ \isanewline
\ \ \isakeyword{assumes}\ {\isachardoublequoteopen}Atom\ {\isacharbackquote}\ atoms\ F{\isadigit{1}}\ {\isasymsubseteq}\ setSubformulae\ F{\isadigit{1}}{\isachardoublequoteclose}\isanewline
\ \ \ \ \ \ \ \ \ \ {\isachardoublequoteopen}Atom\ {\isacharbackquote}\ atoms\ F{\isadigit{2}}\ {\isasymsubseteq}\ setSubformulae\ F{\isadigit{2}}{\isachardoublequoteclose}\isanewline
\ \ \isakeyword{shows}\ \ \ {\isachardoublequoteopen}Atom\ {\isacharbackquote}\ atoms\ {\isacharparenleft}F{\isadigit{1}}\ \isactrlbold {\isasymrightarrow}\ F{\isadigit{2}}{\isacharparenright}\ {\isasymsubseteq}\ setSubformulae\ {\isacharparenleft}F{\isadigit{1}}\ \isactrlbold {\isasymrightarrow}\ F{\isadigit{2}}{\isacharparenright}{\isachardoublequoteclose}\isanewline
%
\isadelimproof
%
\endisadelimproof
%
\isatagproof
\isacommand{proof}\isamarkupfalse%
\ {\isacharminus}\isanewline
\ \ \isacommand{have}\isamarkupfalse%
\ {\isachardoublequoteopen}Atom\ {\isacharbackquote}\ atoms\ {\isacharparenleft}F{\isadigit{1}}\ \isactrlbold {\isasymrightarrow}\ F{\isadigit{2}}{\isacharparenright}\ {\isacharequal}\ Atom\ {\isacharbackquote}\ {\isacharparenleft}atoms\ F{\isadigit{1}}\ {\isasymunion}\ atoms\ F{\isadigit{2}}{\isacharparenright}{\isachardoublequoteclose}\isanewline
\ \ \ \ \isacommand{by}\isamarkupfalse%
\ {\isacharparenleft}simp\ only{\isacharcolon}\ formula{\isachardot}set{\isacharparenleft}{\isadigit{6}}{\isacharparenright}{\isacharparenright}\isanewline
\ \ \isacommand{also}\isamarkupfalse%
\ \isacommand{have}\isamarkupfalse%
\ {\isachardoublequoteopen}{\isasymdots}\ {\isacharequal}\ Atom\ {\isacharbackquote}\ atoms\ F{\isadigit{1}}\ {\isasymunion}\ Atom\ {\isacharbackquote}\ atoms\ F{\isadigit{2}}{\isachardoublequoteclose}\ \isanewline
\ \ \ \ \isacommand{by}\isamarkupfalse%
\ {\isacharparenleft}rule\ image{\isacharunderscore}Un{\isacharparenright}\isanewline
\ \ \isacommand{also}\isamarkupfalse%
\ \isacommand{have}\isamarkupfalse%
\ {\isachardoublequoteopen}{\isasymdots}\ {\isasymsubseteq}\ setSubformulae\ F{\isadigit{1}}\ {\isasymunion}\ setSubformulae\ F{\isadigit{2}}{\isachardoublequoteclose}\isanewline
\ \ \ \ \isacommand{using}\isamarkupfalse%
\ assms\isanewline
\ \ \ \ \isacommand{by}\isamarkupfalse%
\ {\isacharparenleft}rule\ Un{\isacharunderscore}mono{\isacharparenright}\isanewline
\ \ \isacommand{also}\isamarkupfalse%
\ \isacommand{have}\isamarkupfalse%
\ {\isachardoublequoteopen}{\isasymdots}\ {\isasymsubseteq}\ {\isacharbraceleft}F{\isadigit{1}}\ \isactrlbold {\isasymrightarrow}\ F{\isadigit{2}}{\isacharbraceright}\ {\isasymunion}\ {\isacharparenleft}setSubformulae\ F{\isadigit{1}}\ {\isasymunion}\ setSubformulae\ F{\isadigit{2}}{\isacharparenright}{\isachardoublequoteclose}\isanewline
\ \ \ \ \isacommand{by}\isamarkupfalse%
\ {\isacharparenleft}simp\ only{\isacharcolon}\ Un{\isacharunderscore}upper{\isadigit{2}}{\isacharparenright}\isanewline
\ \ \isacommand{also}\isamarkupfalse%
\ \isacommand{have}\isamarkupfalse%
\ {\isachardoublequoteopen}{\isasymdots}\ {\isacharequal}\ setSubformulae\ {\isacharparenleft}F{\isadigit{1}}\ \isactrlbold {\isasymrightarrow}\ F{\isadigit{2}}{\isacharparenright}{\isachardoublequoteclose}\isanewline
\ \ \ \ \isacommand{by}\isamarkupfalse%
\ {\isacharparenleft}simp\ only{\isacharcolon}\ setSubformulae{\isacharunderscore}imp{\isacharparenright}\isanewline
\ \ \isacommand{finally}\isamarkupfalse%
\ \isacommand{show}\isamarkupfalse%
\ {\isacharquery}thesis\isanewline
\ \ \ \ \isacommand{by}\isamarkupfalse%
\ this\isanewline
\isacommand{qed}\isamarkupfalse%
%
\endisatagproof
{\isafoldproof}%
%
\isadelimproof
\isanewline
%
\endisadelimproof
\isanewline
\isacommand{lemma}\isamarkupfalse%
\ atoms{\isacharunderscore}are{\isacharunderscore}subformulae{\isacharcolon}\ \isanewline
\ \ {\isachardoublequoteopen}Atom\ {\isacharbackquote}\ atoms\ F\ {\isasymsubseteq}\ setSubformulae\ F{\isachardoublequoteclose}\isanewline
%
\isadelimproof
%
\endisadelimproof
%
\isatagproof
\isacommand{proof}\isamarkupfalse%
\ {\isacharparenleft}induction\ F{\isacharparenright}\isanewline
\ \ \isacommand{case}\isamarkupfalse%
\ {\isacharparenleft}Atom\ x{\isacharparenright}\isanewline
\ \ \isacommand{then}\isamarkupfalse%
\ \isacommand{show}\isamarkupfalse%
\ {\isacharquery}case\ \isacommand{by}\isamarkupfalse%
\ {\isacharparenleft}simp\ only{\isacharcolon}\ atoms{\isacharunderscore}are{\isacharunderscore}subformulae{\isacharunderscore}atom{\isacharparenright}\ \isanewline
\isacommand{next}\isamarkupfalse%
\isanewline
\ \ \isacommand{case}\isamarkupfalse%
\ Bot\isanewline
\ \ \isacommand{then}\isamarkupfalse%
\ \isacommand{show}\isamarkupfalse%
\ {\isacharquery}case\ \isacommand{by}\isamarkupfalse%
\ {\isacharparenleft}simp\ only{\isacharcolon}\ atoms{\isacharunderscore}are{\isacharunderscore}subformulae{\isacharunderscore}bot{\isacharparenright}\ \isanewline
\isacommand{next}\isamarkupfalse%
\isanewline
\ \ \isacommand{case}\isamarkupfalse%
\ {\isacharparenleft}Not\ F{\isacharparenright}\isanewline
\ \ \isacommand{then}\isamarkupfalse%
\ \isacommand{show}\isamarkupfalse%
\ {\isacharquery}case\ \isacommand{by}\isamarkupfalse%
\ {\isacharparenleft}simp\ only{\isacharcolon}\ atoms{\isacharunderscore}are{\isacharunderscore}subformulae{\isacharunderscore}not{\isacharparenright}\ \isanewline
\isacommand{next}\isamarkupfalse%
\isanewline
\ \ \isacommand{case}\isamarkupfalse%
\ {\isacharparenleft}And\ F{\isadigit{1}}\ F{\isadigit{2}}{\isacharparenright}\isanewline
\ \ \isacommand{then}\isamarkupfalse%
\ \isacommand{show}\isamarkupfalse%
\ {\isacharquery}case\ \isacommand{by}\isamarkupfalse%
\ {\isacharparenleft}simp\ only{\isacharcolon}\ atoms{\isacharunderscore}are{\isacharunderscore}subformulae{\isacharunderscore}and{\isacharparenright}\ \isanewline
\isacommand{next}\isamarkupfalse%
\isanewline
\ \ \isacommand{case}\isamarkupfalse%
\ {\isacharparenleft}Or\ F{\isadigit{1}}\ F{\isadigit{2}}{\isacharparenright}\isanewline
\ \ \isacommand{then}\isamarkupfalse%
\ \isacommand{show}\isamarkupfalse%
\ {\isacharquery}case\ \isacommand{by}\isamarkupfalse%
\ {\isacharparenleft}simp\ only{\isacharcolon}\ atoms{\isacharunderscore}are{\isacharunderscore}subformulae{\isacharunderscore}or{\isacharparenright}\isanewline
\isacommand{next}\isamarkupfalse%
\isanewline
\ \ \isacommand{case}\isamarkupfalse%
\ {\isacharparenleft}Imp\ F{\isadigit{1}}\ F{\isadigit{2}}{\isacharparenright}\isanewline
\ \ \isacommand{then}\isamarkupfalse%
\ \isacommand{show}\isamarkupfalse%
\ {\isacharquery}case\ \isacommand{by}\isamarkupfalse%
\ {\isacharparenleft}simp\ only{\isacharcolon}\ atoms{\isacharunderscore}are{\isacharunderscore}subformulae{\isacharunderscore}imp{\isacharparenright}\isanewline
\isacommand{qed}\isamarkupfalse%
%
\endisatagproof
{\isafoldproof}%
%
\isadelimproof
%
\endisadelimproof
%
\begin{isamarkuptext}%
La demostración automática queda igualmente expuesta a 
  continuación.%
\end{isamarkuptext}\isamarkuptrue%
\isacommand{lemma}\isamarkupfalse%
\ {\isachardoublequoteopen}Atom\ {\isacharbackquote}\ atoms\ F\ {\isasymsubseteq}\ setSubformulae\ F{\isachardoublequoteclose}\isanewline
%
\isadelimproof
\ \ %
\endisadelimproof
%
\isatagproof
\isacommand{by}\isamarkupfalse%
\ {\isacharparenleft}induction\ F{\isacharparenright}\ \ auto%
\endisatagproof
{\isafoldproof}%
%
\isadelimproof
%
\endisadelimproof
%
\begin{isamarkuptext}%
La siguiente propiedad declara que el conjunto de átomos de una 
  subfórmula está contenido en el conjunto de átomos de la propia 
  fórmula.

  \begin{lema}
  Dada una fórmula, los átomos de sus subfórmulas son átomos de ella
  misma.
  \end{lema}

  \begin{demostracion}
  Procedemos mediante inducción en la estructura de las fórmulas según 
  los distintos casos:

  Sea \isa{p} una fórmula atómica cualquiera. Por definición de su conjunto
  de subfórmulas, su única subfórmula es ella misma, luego se verifica
  el resultado.

  Sea la fórmula \isa{{\isasymbottom}}. Por definición de su conjunto de subfórmulas, su 
  única subfórmula es ella misma, luego se verifica análogamente la
  propiedad en este caso.

  Sea la fórmula \isa{F} tal que para cualquier subfórmula suya se verifica 
  que el conjunto de sus átomos está contenido en el conjunto de átomos 
  de \isa{F}. Supongamos \isa{G} subfórmula cualquiera de \isa{{\isasymnot}\ F}. Vamos a
  probar que el conjunto de átomos de \isa{G} está contenido en el de 
  \isa{{\isasymnot}\ F}.
  Por definición, tenemos que el conjunto de subfórmulas de \isa{{\isasymnot}\ F} es de
  la forma \\ \isa{Subf{\isacharparenleft}{\isasymnot}\ F{\isacharparenright}\ {\isacharequal}\ {\isacharbraceleft}{\isasymnot}\ F{\isacharbraceright}\ {\isasymunion}\ Subf{\isacharparenleft}F{\isacharparenright}}. De este modo, tenemos dos 
  opciones posibles:\\ \isa{G\ {\isasymin}\ {\isacharbraceleft}{\isasymnot}\ F{\isacharbraceright}} o \isa{G\ {\isasymin}\ Subf{\isacharparenleft}F{\isacharparenright}}. 
  Del primer caso se deduce \isa{G\ {\isacharequal}\ {\isasymnot}\ F} 
  y, por tanto, tienen igual conjunto de átomos.
  Observando el segundo caso, por hipótesis de inducción, se tiene que 
  el conjunto de átomos de \isa{G} está contenido en el de \isa{F}. Además, como 
  el conjunto de átomos de \isa{F} y \isa{{\isasymnot}\ F} coinciden, se verifica el 
  resultado.

  Sea \isa{F{\isadigit{1}}} una fórmula proposicional tal que el conjunto de los átomos 
  de cualquier subfórmula suya está contenido en el conjunto de átomos 
  de \isa{F{\isadigit{1}}}. Sea también \isa{F{\isadigit{2}}}\\ cumpliendo dicha hipótesis de inducción 
  para sus correspondientes subfórmulas. Supongamos además que \isa{G} es
  subfórmula de \isa{F{\isadigit{1}}{\isacharasterisk}F{\isadigit{2}}}, donde \isa{{\isacharasterisk}} simboliza una conectiva binaria 
  cualquiera. Vamos a probar que el conjunto de átomos de \isa{G} está 
  contenido en el conjunto de átomos de \isa{F{\isadigit{1}}{\isacharasterisk}F{\isadigit{2}}}.
  En primer lugar, por definición tenemos que el conjunto de
  subfórmulas de \isa{F{\isadigit{1}}{\isacharasterisk}F{\isadigit{2}}} es de la forma\\
  \isa{Subf{\isacharparenleft}F{\isadigit{1}}{\isacharasterisk}F{\isadigit{2}}{\isacharparenright}\ {\isacharequal}\ {\isacharbraceleft}F{\isadigit{1}}{\isacharasterisk}F{\isadigit{2}}{\isacharbraceright}\ {\isasymunion}\ {\isacharparenleft}Subf{\isacharparenleft}F{\isadigit{1}}{\isacharparenright}\ {\isasymunion}\ Subf{\isacharparenleft}F{\isadigit{2}}{\isacharparenright}{\isacharparenright}}. De este modo, 
  tenemos dos posibles opciones:
  \isa{G\ {\isasymin}\ {\isacharbraceleft}F{\isadigit{1}}{\isacharasterisk}F{\isadigit{2}}{\isacharbraceright}} o \isa{G\ {\isasymin}\ Subf{\isacharparenleft}F{\isadigit{1}}{\isacharparenright}\ {\isasymunion}\ Subf{\isacharparenleft}F{\isadigit{2}}{\isacharparenright}}.
  Si \isa{G\ {\isasymin}\ {\isacharbraceleft}F{\isadigit{1}}{\isacharasterisk}F{\isadigit{2}}{\isacharbraceright}}, entonces \isa{G\ {\isacharequal}\ F{\isadigit{1}}{\isacharasterisk}F{\isadigit{2}}} y tienen igual conjunto de 
  átomos.
  Por otro lado, si \isa{G\ {\isasymin}\ Subf{\isacharparenleft}F{\isadigit{1}}{\isacharparenright}\ {\isasymunion}\ Subf{\isacharparenleft}F{\isadigit{2}}{\isacharparenright}} tenemos dos nuevas
  posibilidades: \isa{G} es subfórmula de \isa{F{\isadigit{1}}} o \isa{G} es subfórmula de \isa{F{\isadigit{2}}}.
  Suponiendo que fuese subfórmula de \isa{F{\isadigit{1}}}, aplicando hipótesis de
  inducción tendríamos que el conjunto de átomos de \isa{G} está contenido 
  en el de \isa{F{\isadigit{1}}}. De este modo, como el conjunto de átomos de \isa{F{\isadigit{1}}{\isacharasterisk}F{\isadigit{2}}} se
  define como la unión de los conjuntos de átomos de \isa{F{\isadigit{1}}} y \isa{F{\isadigit{2}}}, por
  propiedades de la contención se verifica que el conjunto de átomos de
  \isa{G} está contenido en el de \isa{F{\isadigit{1}}{\isacharasterisk}F{\isadigit{2}}}. Observemos que si \isa{G} es 
  subfórmula de \isa{F{\isadigit{2}}}, se demuestra análogamente cambiando los
  subíndices correspondientes. Por tanto, se tiene el resultado.      
  \end{demostracion}

  Formalizado en Isabelle:%
\end{isamarkuptext}\isamarkuptrue%
\isacommand{lemma}\isamarkupfalse%
\ {\isachardoublequoteopen}G\ {\isasymin}\ setSubformulae\ F\ {\isasymLongrightarrow}\ atoms\ G\ {\isasymsubseteq}\ atoms\ F{\isachardoublequoteclose}\isanewline
%
\isadelimproof
\ \ %
\endisadelimproof
%
\isatagproof
\isacommand{oops}\isamarkupfalse%
%
\endisatagproof
{\isafoldproof}%
%
\isadelimproof
%
\endisadelimproof
%
\begin{isamarkuptext}%
Veamos su demostración estructurada.%
\end{isamarkuptext}\isamarkuptrue%
\isacommand{lemma}\isamarkupfalse%
\ subformulas{\isacharunderscore}atoms{\isacharunderscore}atom{\isacharcolon}\isanewline
\ \ \isakeyword{assumes}\ {\isachardoublequoteopen}G\ {\isasymin}\ setSubformulae\ {\isacharparenleft}Atom\ x{\isacharparenright}{\isachardoublequoteclose}\ \isanewline
\ \ \isakeyword{shows}\ \ \ {\isachardoublequoteopen}atoms\ G\ {\isasymsubseteq}\ atoms\ {\isacharparenleft}Atom\ x{\isacharparenright}{\isachardoublequoteclose}\isanewline
%
\isadelimproof
%
\endisadelimproof
%
\isatagproof
\isacommand{proof}\isamarkupfalse%
\ {\isacharminus}\isanewline
\ \ \isacommand{have}\isamarkupfalse%
\ {\isachardoublequoteopen}G\ {\isasymin}\ {\isacharbraceleft}Atom\ x{\isacharbraceright}{\isachardoublequoteclose}\isanewline
\ \ \ \ \isacommand{using}\isamarkupfalse%
\ assms\isanewline
\ \ \ \ \isacommand{by}\isamarkupfalse%
\ {\isacharparenleft}simp\ only{\isacharcolon}\ setSubformulae{\isacharunderscore}atom{\isacharparenright}\isanewline
\ \ \isacommand{then}\isamarkupfalse%
\ \isacommand{have}\isamarkupfalse%
\ {\isachardoublequoteopen}G\ {\isacharequal}\ Atom\ x{\isachardoublequoteclose}\isanewline
\ \ \ \ \isacommand{by}\isamarkupfalse%
\ {\isacharparenleft}simp\ only{\isacharcolon}\ singletonD{\isacharparenright}\isanewline
\ \ \isacommand{then}\isamarkupfalse%
\ \isacommand{show}\isamarkupfalse%
\ {\isacharquery}thesis\isanewline
\ \ \ \ \isacommand{by}\isamarkupfalse%
\ {\isacharparenleft}simp\ only{\isacharcolon}\ subset{\isacharunderscore}refl{\isacharparenright}\isanewline
\isacommand{qed}\isamarkupfalse%
%
\endisatagproof
{\isafoldproof}%
%
\isadelimproof
\isanewline
%
\endisadelimproof
\isanewline
\isacommand{lemma}\isamarkupfalse%
\ subformulas{\isacharunderscore}atoms{\isacharunderscore}bot{\isacharcolon}\isanewline
\ \ \isakeyword{assumes}\ {\isachardoublequoteopen}G\ {\isasymin}\ setSubformulae\ {\isasymbottom}{\isachardoublequoteclose}\ \isanewline
\ \ \isakeyword{shows}\ \ \ {\isachardoublequoteopen}atoms\ G\ {\isasymsubseteq}\ atoms\ {\isasymbottom}{\isachardoublequoteclose}\isanewline
%
\isadelimproof
%
\endisadelimproof
%
\isatagproof
\isacommand{proof}\isamarkupfalse%
\ {\isacharminus}\isanewline
\ \ \isacommand{have}\isamarkupfalse%
\ {\isachardoublequoteopen}G\ {\isasymin}\ {\isacharbraceleft}{\isasymbottom}{\isacharbraceright}{\isachardoublequoteclose}\isanewline
\ \ \ \ \isacommand{using}\isamarkupfalse%
\ assms\isanewline
\ \ \ \ \isacommand{by}\isamarkupfalse%
\ {\isacharparenleft}simp\ only{\isacharcolon}\ setSubformulae{\isacharunderscore}bot{\isacharparenright}\isanewline
\ \ \isacommand{then}\isamarkupfalse%
\ \isacommand{have}\isamarkupfalse%
\ {\isachardoublequoteopen}G\ {\isacharequal}\ {\isasymbottom}{\isachardoublequoteclose}\isanewline
\ \ \ \ \isacommand{by}\isamarkupfalse%
\ {\isacharparenleft}simp\ only{\isacharcolon}\ singletonD{\isacharparenright}\isanewline
\ \ \isacommand{then}\isamarkupfalse%
\ \isacommand{show}\isamarkupfalse%
\ {\isacharquery}thesis\isanewline
\ \ \ \ \isacommand{by}\isamarkupfalse%
\ {\isacharparenleft}simp\ only{\isacharcolon}\ subset{\isacharunderscore}refl{\isacharparenright}\isanewline
\isacommand{qed}\isamarkupfalse%
%
\endisatagproof
{\isafoldproof}%
%
\isadelimproof
\isanewline
%
\endisadelimproof
\isanewline
\isacommand{lemma}\isamarkupfalse%
\ subformulas{\isacharunderscore}atoms{\isacharunderscore}not{\isacharcolon}\isanewline
\ \ \isakeyword{assumes}\ {\isachardoublequoteopen}G\ {\isasymin}\ setSubformulae\ F\ {\isasymLongrightarrow}\ atoms\ G\ {\isasymsubseteq}\ atoms\ F{\isachardoublequoteclose}\isanewline
\ \ \ \ \ \ \ \ \ \ {\isachardoublequoteopen}G\ {\isasymin}\ setSubformulae\ {\isacharparenleft}\isactrlbold {\isasymnot}\ F{\isacharparenright}{\isachardoublequoteclose}\isanewline
\ \ \isakeyword{shows}\ \ \ {\isachardoublequoteopen}atoms\ G\ {\isasymsubseteq}\ atoms\ {\isacharparenleft}\isactrlbold {\isasymnot}\ F{\isacharparenright}{\isachardoublequoteclose}\isanewline
%
\isadelimproof
%
\endisadelimproof
%
\isatagproof
\isacommand{proof}\isamarkupfalse%
\ {\isacharminus}\isanewline
\ \ \isacommand{have}\isamarkupfalse%
\ {\isachardoublequoteopen}G\ {\isasymin}\ {\isacharbraceleft}\isactrlbold {\isasymnot}\ F{\isacharbraceright}\ {\isasymunion}\ setSubformulae\ F{\isachardoublequoteclose}\isanewline
\ \ \ \ \isacommand{using}\isamarkupfalse%
\ assms{\isacharparenleft}{\isadigit{2}}{\isacharparenright}\isanewline
\ \ \ \ \isacommand{by}\isamarkupfalse%
\ {\isacharparenleft}simp\ only{\isacharcolon}\ setSubformulae{\isacharunderscore}not{\isacharparenright}\ \isanewline
\ \ \isacommand{then}\isamarkupfalse%
\ \isacommand{have}\isamarkupfalse%
\ {\isachardoublequoteopen}G\ {\isasymin}\ {\isacharbraceleft}\isactrlbold {\isasymnot}\ F{\isacharbraceright}\ {\isasymor}\ G\ {\isasymin}\ setSubformulae\ F{\isachardoublequoteclose}\isanewline
\ \ \ \ \isacommand{by}\isamarkupfalse%
\ {\isacharparenleft}simp\ only{\isacharcolon}\ Un{\isacharunderscore}iff{\isacharparenright}\isanewline
\ \ \isacommand{then}\isamarkupfalse%
\ \isacommand{show}\isamarkupfalse%
\ {\isachardoublequoteopen}atoms\ G\ {\isasymsubseteq}\ atoms\ {\isacharparenleft}\isactrlbold {\isasymnot}\ F{\isacharparenright}{\isachardoublequoteclose}\isanewline
\ \ \isacommand{proof}\isamarkupfalse%
\ {\isacharparenleft}rule\ disjE{\isacharparenright}\isanewline
\ \ \ \ \isacommand{assume}\isamarkupfalse%
\ {\isachardoublequoteopen}G\ {\isasymin}\ {\isacharbraceleft}\isactrlbold {\isasymnot}\ F{\isacharbraceright}{\isachardoublequoteclose}\isanewline
\ \ \ \ \isacommand{then}\isamarkupfalse%
\ \isacommand{have}\isamarkupfalse%
\ {\isachardoublequoteopen}G\ {\isacharequal}\ \isactrlbold {\isasymnot}\ F{\isachardoublequoteclose}\isanewline
\ \ \ \ \ \ \isacommand{by}\isamarkupfalse%
\ {\isacharparenleft}simp\ only{\isacharcolon}\ singletonD{\isacharparenright}\isanewline
\ \ \ \ \isacommand{then}\isamarkupfalse%
\ \isacommand{show}\isamarkupfalse%
\ {\isacharquery}thesis\isanewline
\ \ \ \ \ \ \isacommand{by}\isamarkupfalse%
\ {\isacharparenleft}simp\ only{\isacharcolon}\ subset{\isacharunderscore}refl{\isacharparenright}\isanewline
\ \ \isacommand{next}\isamarkupfalse%
\isanewline
\ \ \ \ \isacommand{assume}\isamarkupfalse%
\ {\isachardoublequoteopen}G\ {\isasymin}\ setSubformulae\ F{\isachardoublequoteclose}\isanewline
\ \ \ \ \isacommand{then}\isamarkupfalse%
\ \isacommand{have}\isamarkupfalse%
\ {\isachardoublequoteopen}atoms\ G\ {\isasymsubseteq}\ atoms\ F{\isachardoublequoteclose}\isanewline
\ \ \ \ \ \ \isacommand{by}\isamarkupfalse%
\ {\isacharparenleft}simp\ only{\isacharcolon}\ assms{\isacharparenleft}{\isadigit{1}}{\isacharparenright}{\isacharparenright}\isanewline
\ \ \ \ \isacommand{also}\isamarkupfalse%
\ \isacommand{have}\isamarkupfalse%
\ {\isachardoublequoteopen}{\isasymdots}\ {\isacharequal}\ atoms\ {\isacharparenleft}\isactrlbold {\isasymnot}\ F{\isacharparenright}{\isachardoublequoteclose}\isanewline
\ \ \ \ \ \ \isacommand{by}\isamarkupfalse%
\ {\isacharparenleft}simp\ only{\isacharcolon}\ formula{\isachardot}set{\isacharparenleft}{\isadigit{3}}{\isacharparenright}{\isacharparenright}\isanewline
\ \ \ \ \isacommand{finally}\isamarkupfalse%
\ \isacommand{show}\isamarkupfalse%
\ {\isacharquery}thesis\isanewline
\ \ \ \ \ \ \isacommand{by}\isamarkupfalse%
\ this\isanewline
\ \ \isacommand{qed}\isamarkupfalse%
\isanewline
\isacommand{qed}\isamarkupfalse%
%
\endisatagproof
{\isafoldproof}%
%
\isadelimproof
\isanewline
%
\endisadelimproof
\isanewline
\isacommand{lemma}\isamarkupfalse%
\ subformulas{\isacharunderscore}atoms{\isacharunderscore}and{\isacharcolon}\isanewline
\ \ \isakeyword{assumes}\ {\isachardoublequoteopen}G\ {\isasymin}\ setSubformulae\ F{\isadigit{1}}\ {\isasymLongrightarrow}\ atoms\ G\ {\isasymsubseteq}\ atoms\ F{\isadigit{1}}{\isachardoublequoteclose}\isanewline
\ \ \ \ \ \ \ \ \ \ {\isachardoublequoteopen}G\ {\isasymin}\ setSubformulae\ F{\isadigit{2}}\ {\isasymLongrightarrow}\ atoms\ G\ {\isasymsubseteq}\ atoms\ F{\isadigit{2}}{\isachardoublequoteclose}\isanewline
\ \ \ \ \ \ \ \ \ \ {\isachardoublequoteopen}G\ {\isasymin}\ setSubformulae\ {\isacharparenleft}F{\isadigit{1}}\ \isactrlbold {\isasymand}\ F{\isadigit{2}}{\isacharparenright}{\isachardoublequoteclose}\isanewline
\ \ \isakeyword{shows}\ \ \ {\isachardoublequoteopen}atoms\ G\ {\isasymsubseteq}\ atoms\ {\isacharparenleft}F{\isadigit{1}}\ \isactrlbold {\isasymand}\ F{\isadigit{2}}{\isacharparenright}{\isachardoublequoteclose}\isanewline
%
\isadelimproof
%
\endisadelimproof
%
\isatagproof
\isacommand{proof}\isamarkupfalse%
\ {\isacharminus}\isanewline
\ \ \isacommand{have}\isamarkupfalse%
\ {\isachardoublequoteopen}G\ {\isasymin}\ {\isacharbraceleft}F{\isadigit{1}}\ \isactrlbold {\isasymand}\ F{\isadigit{2}}{\isacharbraceright}\ {\isasymunion}\ {\isacharparenleft}setSubformulae\ F{\isadigit{1}}\ {\isasymunion}\ setSubformulae\ F{\isadigit{2}}{\isacharparenright}{\isachardoublequoteclose}\isanewline
\ \ \ \ \isacommand{using}\isamarkupfalse%
\ assms{\isacharparenleft}{\isadigit{3}}{\isacharparenright}\ \isanewline
\ \ \ \ \isacommand{by}\isamarkupfalse%
\ {\isacharparenleft}simp\ only{\isacharcolon}\ setSubformulae{\isacharunderscore}and{\isacharparenright}\isanewline
\ \ \isacommand{then}\isamarkupfalse%
\ \isacommand{have}\isamarkupfalse%
\ {\isachardoublequoteopen}G\ {\isasymin}\ {\isacharbraceleft}F{\isadigit{1}}\ \isactrlbold {\isasymand}\ F{\isadigit{2}}{\isacharbraceright}\ {\isasymor}\ G\ {\isasymin}\ setSubformulae\ F{\isadigit{1}}\ {\isasymunion}\ setSubformulae\ F{\isadigit{2}}{\isachardoublequoteclose}\isanewline
\ \ \ \ \isacommand{by}\isamarkupfalse%
\ {\isacharparenleft}simp\ only{\isacharcolon}\ Un{\isacharunderscore}iff{\isacharparenright}\isanewline
\ \ \isacommand{then}\isamarkupfalse%
\ \isacommand{show}\isamarkupfalse%
\ {\isacharquery}thesis\isanewline
\ \ \isacommand{proof}\isamarkupfalse%
\ {\isacharparenleft}rule\ disjE{\isacharparenright}\isanewline
\ \ \ \ \isacommand{assume}\isamarkupfalse%
\ {\isachardoublequoteopen}G\ {\isasymin}\ {\isacharbraceleft}F{\isadigit{1}}\ \isactrlbold {\isasymand}\ F{\isadigit{2}}{\isacharbraceright}{\isachardoublequoteclose}\isanewline
\ \ \ \ \isacommand{then}\isamarkupfalse%
\ \isacommand{have}\isamarkupfalse%
\ {\isachardoublequoteopen}G\ {\isacharequal}\ F{\isadigit{1}}\ \isactrlbold {\isasymand}\ F{\isadigit{2}}{\isachardoublequoteclose}\isanewline
\ \ \ \ \ \ \isacommand{by}\isamarkupfalse%
\ {\isacharparenleft}simp\ only{\isacharcolon}\ singletonD{\isacharparenright}\isanewline
\ \ \ \ \isacommand{then}\isamarkupfalse%
\ \isacommand{show}\isamarkupfalse%
\ {\isacharquery}thesis\isanewline
\ \ \ \ \ \ \isacommand{by}\isamarkupfalse%
\ {\isacharparenleft}simp\ only{\isacharcolon}\ subset{\isacharunderscore}refl{\isacharparenright}\isanewline
\ \ \isacommand{next}\isamarkupfalse%
\isanewline
\ \ \ \ \isacommand{assume}\isamarkupfalse%
\ {\isachardoublequoteopen}G\ {\isasymin}\ setSubformulae\ F{\isadigit{1}}\ {\isasymunion}\ setSubformulae\ F{\isadigit{2}}{\isachardoublequoteclose}\isanewline
\ \ \ \ \isacommand{then}\isamarkupfalse%
\ \isacommand{have}\isamarkupfalse%
\ {\isachardoublequoteopen}G\ {\isasymin}\ setSubformulae\ F{\isadigit{1}}\ {\isasymor}\ G\ {\isasymin}\ setSubformulae\ F{\isadigit{2}}{\isachardoublequoteclose}\ \ \isanewline
\ \ \ \ \ \ \isacommand{by}\isamarkupfalse%
\ {\isacharparenleft}simp\ only{\isacharcolon}\ Un{\isacharunderscore}iff{\isacharparenright}\isanewline
\ \ \ \ \isacommand{then}\isamarkupfalse%
\ \isacommand{show}\isamarkupfalse%
\ {\isacharquery}thesis\isanewline
\ \ \ \ \isacommand{proof}\isamarkupfalse%
\ {\isacharparenleft}rule\ disjE{\isacharparenright}\isanewline
\ \ \ \ \ \ \isacommand{assume}\isamarkupfalse%
\ {\isachardoublequoteopen}G\ {\isasymin}\ setSubformulae\ F{\isadigit{1}}{\isachardoublequoteclose}\isanewline
\ \ \ \ \ \ \isacommand{then}\isamarkupfalse%
\ \isacommand{have}\isamarkupfalse%
\ {\isachardoublequoteopen}atoms\ G\ {\isasymsubseteq}\ atoms\ F{\isadigit{1}}{\isachardoublequoteclose}\isanewline
\ \ \ \ \ \ \ \ \isacommand{by}\isamarkupfalse%
\ {\isacharparenleft}rule\ assms{\isacharparenleft}{\isadigit{1}}{\isacharparenright}{\isacharparenright}\isanewline
\ \ \ \ \ \ \isacommand{also}\isamarkupfalse%
\ \isacommand{have}\isamarkupfalse%
\ {\isachardoublequoteopen}{\isasymdots}\ {\isasymsubseteq}\ atoms\ F{\isadigit{1}}\ {\isasymunion}\ atoms\ F{\isadigit{2}}{\isachardoublequoteclose}\isanewline
\ \ \ \ \ \ \ \ \isacommand{by}\isamarkupfalse%
\ {\isacharparenleft}simp\ only{\isacharcolon}\ Un{\isacharunderscore}upper{\isadigit{1}}{\isacharparenright}\isanewline
\ \ \ \ \ \ \isacommand{also}\isamarkupfalse%
\ \isacommand{have}\isamarkupfalse%
\ {\isachardoublequoteopen}{\isasymdots}\ {\isacharequal}\ atoms\ {\isacharparenleft}F{\isadigit{1}}\ \isactrlbold {\isasymand}\ F{\isadigit{2}}{\isacharparenright}{\isachardoublequoteclose}\isanewline
\ \ \ \ \ \ \ \ \isacommand{by}\isamarkupfalse%
\ {\isacharparenleft}simp\ only{\isacharcolon}\ formula{\isachardot}set{\isacharparenleft}{\isadigit{4}}{\isacharparenright}{\isacharparenright}\isanewline
\ \ \ \ \ \ \isacommand{finally}\isamarkupfalse%
\ \isacommand{show}\isamarkupfalse%
\ {\isacharquery}thesis\isanewline
\ \ \ \ \ \ \ \ \isacommand{by}\isamarkupfalse%
\ this\isanewline
\ \ \ \ \isacommand{next}\isamarkupfalse%
\isanewline
\ \ \ \ \ \ \isacommand{assume}\isamarkupfalse%
\ {\isachardoublequoteopen}G\ {\isasymin}\ setSubformulae\ F{\isadigit{2}}{\isachardoublequoteclose}\isanewline
\ \ \ \ \ \ \isacommand{then}\isamarkupfalse%
\ \isacommand{have}\isamarkupfalse%
\ {\isachardoublequoteopen}atoms\ G\ {\isasymsubseteq}\ atoms\ F{\isadigit{2}}{\isachardoublequoteclose}\isanewline
\ \ \ \ \ \ \ \ \isacommand{by}\isamarkupfalse%
\ {\isacharparenleft}rule\ assms{\isacharparenleft}{\isadigit{2}}{\isacharparenright}{\isacharparenright}\isanewline
\ \ \ \ \ \ \isacommand{also}\isamarkupfalse%
\ \isacommand{have}\isamarkupfalse%
\ {\isachardoublequoteopen}{\isasymdots}\ {\isasymsubseteq}\ atoms\ F{\isadigit{1}}\ {\isasymunion}\ atoms\ F{\isadigit{2}}{\isachardoublequoteclose}\isanewline
\ \ \ \ \ \ \ \ \isacommand{by}\isamarkupfalse%
\ {\isacharparenleft}simp\ only{\isacharcolon}\ Un{\isacharunderscore}upper{\isadigit{2}}{\isacharparenright}\isanewline
\ \ \ \ \ \ \isacommand{also}\isamarkupfalse%
\ \isacommand{have}\isamarkupfalse%
\ {\isachardoublequoteopen}{\isasymdots}\ {\isacharequal}\ atoms\ {\isacharparenleft}F{\isadigit{1}}\ \isactrlbold {\isasymand}\ F{\isadigit{2}}{\isacharparenright}{\isachardoublequoteclose}\isanewline
\ \ \ \ \ \ \ \ \isacommand{by}\isamarkupfalse%
\ {\isacharparenleft}simp\ only{\isacharcolon}\ formula{\isachardot}set{\isacharparenleft}{\isadigit{4}}{\isacharparenright}{\isacharparenright}\isanewline
\ \ \ \ \ \ \isacommand{finally}\isamarkupfalse%
\ \isacommand{show}\isamarkupfalse%
\ {\isacharquery}thesis\isanewline
\ \ \ \ \ \ \ \ \isacommand{by}\isamarkupfalse%
\ this\isanewline
\ \ \ \ \isacommand{qed}\isamarkupfalse%
\isanewline
\ \ \isacommand{qed}\isamarkupfalse%
\isanewline
\isacommand{qed}\isamarkupfalse%
%
\endisatagproof
{\isafoldproof}%
%
\isadelimproof
\isanewline
%
\endisadelimproof
\isanewline
\isacommand{lemma}\isamarkupfalse%
\ subformulas{\isacharunderscore}atoms{\isacharunderscore}or{\isacharcolon}\isanewline
\ \ \isakeyword{assumes}\ {\isachardoublequoteopen}G\ {\isasymin}\ setSubformulae\ F{\isadigit{1}}\ {\isasymLongrightarrow}\ atoms\ G\ {\isasymsubseteq}\ atoms\ F{\isadigit{1}}{\isachardoublequoteclose}\isanewline
\ \ \ \ \ \ \ \ \ \ {\isachardoublequoteopen}G\ {\isasymin}\ setSubformulae\ F{\isadigit{2}}\ {\isasymLongrightarrow}\ atoms\ G\ {\isasymsubseteq}\ atoms\ F{\isadigit{2}}{\isachardoublequoteclose}\isanewline
\ \ \ \ \ \ \ \ \ \ {\isachardoublequoteopen}G\ {\isasymin}\ setSubformulae\ {\isacharparenleft}F{\isadigit{1}}\ \isactrlbold {\isasymor}\ F{\isadigit{2}}{\isacharparenright}{\isachardoublequoteclose}\isanewline
\ \ \isakeyword{shows}\ \ \ {\isachardoublequoteopen}atoms\ G\ {\isasymsubseteq}\ atoms\ {\isacharparenleft}F{\isadigit{1}}\ \isactrlbold {\isasymor}\ F{\isadigit{2}}{\isacharparenright}{\isachardoublequoteclose}\isanewline
%
\isadelimproof
%
\endisadelimproof
%
\isatagproof
\isacommand{proof}\isamarkupfalse%
\ {\isacharminus}\isanewline
\ \ \isacommand{have}\isamarkupfalse%
\ {\isachardoublequoteopen}G\ {\isasymin}\ {\isacharbraceleft}F{\isadigit{1}}\ \isactrlbold {\isasymor}\ F{\isadigit{2}}{\isacharbraceright}\ {\isasymunion}\ {\isacharparenleft}setSubformulae\ F{\isadigit{1}}\ {\isasymunion}\ setSubformulae\ F{\isadigit{2}}{\isacharparenright}{\isachardoublequoteclose}\isanewline
\ \ \ \ \isacommand{using}\isamarkupfalse%
\ assms{\isacharparenleft}{\isadigit{3}}{\isacharparenright}\ \isanewline
\ \ \ \ \isacommand{by}\isamarkupfalse%
\ {\isacharparenleft}simp\ only{\isacharcolon}\ setSubformulae{\isacharunderscore}or{\isacharparenright}\isanewline
\ \ \isacommand{then}\isamarkupfalse%
\ \isacommand{have}\isamarkupfalse%
\ {\isachardoublequoteopen}G\ {\isasymin}\ {\isacharbraceleft}F{\isadigit{1}}\ \isactrlbold {\isasymor}\ F{\isadigit{2}}{\isacharbraceright}\ {\isasymor}\ G\ {\isasymin}\ setSubformulae\ F{\isadigit{1}}\ {\isasymunion}\ setSubformulae\ F{\isadigit{2}}{\isachardoublequoteclose}\isanewline
\ \ \ \ \isacommand{by}\isamarkupfalse%
\ {\isacharparenleft}simp\ only{\isacharcolon}\ Un{\isacharunderscore}iff{\isacharparenright}\isanewline
\ \ \isacommand{then}\isamarkupfalse%
\ \isacommand{show}\isamarkupfalse%
\ {\isacharquery}thesis\isanewline
\ \ \isacommand{proof}\isamarkupfalse%
\ {\isacharparenleft}rule\ disjE{\isacharparenright}\isanewline
\ \ \ \ \isacommand{assume}\isamarkupfalse%
\ {\isachardoublequoteopen}G\ {\isasymin}\ {\isacharbraceleft}F{\isadigit{1}}\ \isactrlbold {\isasymor}\ F{\isadigit{2}}{\isacharbraceright}{\isachardoublequoteclose}\isanewline
\ \ \ \ \isacommand{then}\isamarkupfalse%
\ \isacommand{have}\isamarkupfalse%
\ {\isachardoublequoteopen}G\ {\isacharequal}\ F{\isadigit{1}}\ \isactrlbold {\isasymor}\ F{\isadigit{2}}{\isachardoublequoteclose}\isanewline
\ \ \ \ \ \ \isacommand{by}\isamarkupfalse%
\ {\isacharparenleft}simp\ only{\isacharcolon}\ singletonD{\isacharparenright}\isanewline
\ \ \ \ \isacommand{then}\isamarkupfalse%
\ \isacommand{show}\isamarkupfalse%
\ {\isacharquery}thesis\isanewline
\ \ \ \ \ \ \isacommand{by}\isamarkupfalse%
\ {\isacharparenleft}simp\ only{\isacharcolon}\ subset{\isacharunderscore}refl{\isacharparenright}\isanewline
\ \ \isacommand{next}\isamarkupfalse%
\isanewline
\ \ \ \ \isacommand{assume}\isamarkupfalse%
\ {\isachardoublequoteopen}G\ {\isasymin}\ setSubformulae\ F{\isadigit{1}}\ {\isasymunion}\ setSubformulae\ F{\isadigit{2}}{\isachardoublequoteclose}\isanewline
\ \ \ \ \isacommand{then}\isamarkupfalse%
\ \isacommand{have}\isamarkupfalse%
\ {\isachardoublequoteopen}G\ {\isasymin}\ setSubformulae\ F{\isadigit{1}}\ {\isasymor}\ G\ {\isasymin}\ setSubformulae\ F{\isadigit{2}}{\isachardoublequoteclose}\ \ \isanewline
\ \ \ \ \ \ \isacommand{by}\isamarkupfalse%
\ {\isacharparenleft}simp\ only{\isacharcolon}\ Un{\isacharunderscore}iff{\isacharparenright}\isanewline
\ \ \ \ \isacommand{then}\isamarkupfalse%
\ \isacommand{show}\isamarkupfalse%
\ {\isacharquery}thesis\isanewline
\ \ \ \ \isacommand{proof}\isamarkupfalse%
\ {\isacharparenleft}rule\ disjE{\isacharparenright}\isanewline
\ \ \ \ \ \ \isacommand{assume}\isamarkupfalse%
\ {\isachardoublequoteopen}G\ {\isasymin}\ setSubformulae\ F{\isadigit{1}}{\isachardoublequoteclose}\isanewline
\ \ \ \ \ \ \isacommand{then}\isamarkupfalse%
\ \isacommand{have}\isamarkupfalse%
\ {\isachardoublequoteopen}atoms\ G\ {\isasymsubseteq}\ atoms\ F{\isadigit{1}}{\isachardoublequoteclose}\isanewline
\ \ \ \ \ \ \ \ \isacommand{by}\isamarkupfalse%
\ {\isacharparenleft}rule\ assms{\isacharparenleft}{\isadigit{1}}{\isacharparenright}{\isacharparenright}\isanewline
\ \ \ \ \ \ \isacommand{also}\isamarkupfalse%
\ \isacommand{have}\isamarkupfalse%
\ {\isachardoublequoteopen}{\isasymdots}\ {\isasymsubseteq}\ atoms\ F{\isadigit{1}}\ {\isasymunion}\ atoms\ F{\isadigit{2}}{\isachardoublequoteclose}\isanewline
\ \ \ \ \ \ \ \ \isacommand{by}\isamarkupfalse%
\ {\isacharparenleft}simp\ only{\isacharcolon}\ Un{\isacharunderscore}upper{\isadigit{1}}{\isacharparenright}\isanewline
\ \ \ \ \ \ \isacommand{also}\isamarkupfalse%
\ \isacommand{have}\isamarkupfalse%
\ {\isachardoublequoteopen}{\isasymdots}\ {\isacharequal}\ atoms\ {\isacharparenleft}F{\isadigit{1}}\ \isactrlbold {\isasymor}\ F{\isadigit{2}}{\isacharparenright}{\isachardoublequoteclose}\isanewline
\ \ \ \ \ \ \ \ \isacommand{by}\isamarkupfalse%
\ {\isacharparenleft}simp\ only{\isacharcolon}\ formula{\isachardot}set{\isacharparenleft}{\isadigit{5}}{\isacharparenright}{\isacharparenright}\isanewline
\ \ \ \ \ \ \isacommand{finally}\isamarkupfalse%
\ \isacommand{show}\isamarkupfalse%
\ {\isacharquery}thesis\isanewline
\ \ \ \ \ \ \ \ \isacommand{by}\isamarkupfalse%
\ this\isanewline
\ \ \ \ \isacommand{next}\isamarkupfalse%
\isanewline
\ \ \ \ \ \ \isacommand{assume}\isamarkupfalse%
\ {\isachardoublequoteopen}G\ {\isasymin}\ setSubformulae\ F{\isadigit{2}}{\isachardoublequoteclose}\isanewline
\ \ \ \ \ \ \isacommand{then}\isamarkupfalse%
\ \isacommand{have}\isamarkupfalse%
\ {\isachardoublequoteopen}atoms\ G\ {\isasymsubseteq}\ atoms\ F{\isadigit{2}}{\isachardoublequoteclose}\isanewline
\ \ \ \ \ \ \ \ \isacommand{by}\isamarkupfalse%
\ {\isacharparenleft}rule\ assms{\isacharparenleft}{\isadigit{2}}{\isacharparenright}{\isacharparenright}\isanewline
\ \ \ \ \ \ \isacommand{also}\isamarkupfalse%
\ \isacommand{have}\isamarkupfalse%
\ {\isachardoublequoteopen}{\isasymdots}\ {\isasymsubseteq}\ atoms\ F{\isadigit{1}}\ {\isasymunion}\ atoms\ F{\isadigit{2}}{\isachardoublequoteclose}\isanewline
\ \ \ \ \ \ \ \ \isacommand{by}\isamarkupfalse%
\ {\isacharparenleft}simp\ only{\isacharcolon}\ Un{\isacharunderscore}upper{\isadigit{2}}{\isacharparenright}\isanewline
\ \ \ \ \ \ \isacommand{also}\isamarkupfalse%
\ \isacommand{have}\isamarkupfalse%
\ {\isachardoublequoteopen}{\isasymdots}\ {\isacharequal}\ atoms\ {\isacharparenleft}F{\isadigit{1}}\ \isactrlbold {\isasymor}\ F{\isadigit{2}}{\isacharparenright}{\isachardoublequoteclose}\isanewline
\ \ \ \ \ \ \ \ \isacommand{by}\isamarkupfalse%
\ {\isacharparenleft}simp\ only{\isacharcolon}\ formula{\isachardot}set{\isacharparenleft}{\isadigit{5}}{\isacharparenright}{\isacharparenright}\isanewline
\ \ \ \ \ \ \isacommand{finally}\isamarkupfalse%
\ \isacommand{show}\isamarkupfalse%
\ {\isacharquery}thesis\isanewline
\ \ \ \ \ \ \ \ \isacommand{by}\isamarkupfalse%
\ this\isanewline
\ \ \ \ \isacommand{qed}\isamarkupfalse%
\isanewline
\ \ \isacommand{qed}\isamarkupfalse%
\isanewline
\isacommand{qed}\isamarkupfalse%
%
\endisatagproof
{\isafoldproof}%
%
\isadelimproof
\isanewline
%
\endisadelimproof
\isanewline
\isacommand{lemma}\isamarkupfalse%
\ subformulas{\isacharunderscore}atoms{\isacharunderscore}imp{\isacharcolon}\isanewline
\ \ \isakeyword{assumes}\ {\isachardoublequoteopen}G\ {\isasymin}\ setSubformulae\ F{\isadigit{1}}\ {\isasymLongrightarrow}\ atoms\ G\ {\isasymsubseteq}\ atoms\ F{\isadigit{1}}{\isachardoublequoteclose}\isanewline
\ \ \ \ \ \ \ \ \ \ {\isachardoublequoteopen}G\ {\isasymin}\ setSubformulae\ F{\isadigit{2}}\ {\isasymLongrightarrow}\ atoms\ G\ {\isasymsubseteq}\ atoms\ F{\isadigit{2}}{\isachardoublequoteclose}\isanewline
\ \ \ \ \ \ \ \ \ \ {\isachardoublequoteopen}G\ {\isasymin}\ setSubformulae\ {\isacharparenleft}F{\isadigit{1}}\ \isactrlbold {\isasymrightarrow}\ F{\isadigit{2}}{\isacharparenright}{\isachardoublequoteclose}\isanewline
\ \ \isakeyword{shows}\ \ \ {\isachardoublequoteopen}atoms\ G\ {\isasymsubseteq}\ atoms\ {\isacharparenleft}F{\isadigit{1}}\ \isactrlbold {\isasymrightarrow}\ F{\isadigit{2}}{\isacharparenright}{\isachardoublequoteclose}\isanewline
%
\isadelimproof
%
\endisadelimproof
%
\isatagproof
\isacommand{proof}\isamarkupfalse%
\ {\isacharminus}\isanewline
\ \ \isacommand{have}\isamarkupfalse%
\ {\isachardoublequoteopen}G\ {\isasymin}\ {\isacharbraceleft}F{\isadigit{1}}\ \isactrlbold {\isasymrightarrow}\ F{\isadigit{2}}{\isacharbraceright}\ {\isasymunion}\ {\isacharparenleft}setSubformulae\ F{\isadigit{1}}\ {\isasymunion}\ setSubformulae\ F{\isadigit{2}}{\isacharparenright}{\isachardoublequoteclose}\isanewline
\ \ \ \ \isacommand{using}\isamarkupfalse%
\ assms{\isacharparenleft}{\isadigit{3}}{\isacharparenright}\ \isanewline
\ \ \ \ \isacommand{by}\isamarkupfalse%
\ {\isacharparenleft}simp\ only{\isacharcolon}\ setSubformulae{\isacharunderscore}imp{\isacharparenright}\isanewline
\ \ \isacommand{then}\isamarkupfalse%
\ \isacommand{have}\isamarkupfalse%
\ {\isachardoublequoteopen}G\ {\isasymin}\ {\isacharbraceleft}F{\isadigit{1}}\ \isactrlbold {\isasymrightarrow}\ F{\isadigit{2}}{\isacharbraceright}\ {\isasymor}\ G\ {\isasymin}\ setSubformulae\ F{\isadigit{1}}\ {\isasymunion}\ setSubformulae\ F{\isadigit{2}}{\isachardoublequoteclose}\isanewline
\ \ \ \ \isacommand{by}\isamarkupfalse%
\ {\isacharparenleft}simp\ only{\isacharcolon}\ Un{\isacharunderscore}iff{\isacharparenright}\isanewline
\ \ \isacommand{then}\isamarkupfalse%
\ \isacommand{show}\isamarkupfalse%
\ {\isacharquery}thesis\isanewline
\ \ \isacommand{proof}\isamarkupfalse%
\ {\isacharparenleft}rule\ disjE{\isacharparenright}\isanewline
\ \ \ \ \isacommand{assume}\isamarkupfalse%
\ {\isachardoublequoteopen}G\ {\isasymin}\ {\isacharbraceleft}F{\isadigit{1}}\ \isactrlbold {\isasymrightarrow}\ F{\isadigit{2}}{\isacharbraceright}{\isachardoublequoteclose}\isanewline
\ \ \ \ \isacommand{then}\isamarkupfalse%
\ \isacommand{have}\isamarkupfalse%
\ {\isachardoublequoteopen}G\ {\isacharequal}\ F{\isadigit{1}}\ \isactrlbold {\isasymrightarrow}\ F{\isadigit{2}}{\isachardoublequoteclose}\isanewline
\ \ \ \ \ \ \isacommand{by}\isamarkupfalse%
\ {\isacharparenleft}simp\ only{\isacharcolon}\ singletonD{\isacharparenright}\isanewline
\ \ \ \ \isacommand{then}\isamarkupfalse%
\ \isacommand{show}\isamarkupfalse%
\ {\isacharquery}thesis\isanewline
\ \ \ \ \ \ \isacommand{by}\isamarkupfalse%
\ {\isacharparenleft}simp\ only{\isacharcolon}\ subset{\isacharunderscore}refl{\isacharparenright}\isanewline
\ \ \isacommand{next}\isamarkupfalse%
\isanewline
\ \ \ \ \isacommand{assume}\isamarkupfalse%
\ {\isachardoublequoteopen}G\ {\isasymin}\ setSubformulae\ F{\isadigit{1}}\ {\isasymunion}\ setSubformulae\ F{\isadigit{2}}{\isachardoublequoteclose}\isanewline
\ \ \ \ \isacommand{then}\isamarkupfalse%
\ \isacommand{have}\isamarkupfalse%
\ {\isachardoublequoteopen}G\ {\isasymin}\ setSubformulae\ F{\isadigit{1}}\ {\isasymor}\ G\ {\isasymin}\ setSubformulae\ F{\isadigit{2}}{\isachardoublequoteclose}\ \ \isanewline
\ \ \ \ \ \ \isacommand{by}\isamarkupfalse%
\ {\isacharparenleft}simp\ only{\isacharcolon}\ Un{\isacharunderscore}iff{\isacharparenright}\isanewline
\ \ \ \ \isacommand{then}\isamarkupfalse%
\ \isacommand{show}\isamarkupfalse%
\ {\isacharquery}thesis\isanewline
\ \ \ \ \isacommand{proof}\isamarkupfalse%
\ {\isacharparenleft}rule\ disjE{\isacharparenright}\isanewline
\ \ \ \ \ \ \isacommand{assume}\isamarkupfalse%
\ {\isachardoublequoteopen}G\ {\isasymin}\ setSubformulae\ F{\isadigit{1}}{\isachardoublequoteclose}\isanewline
\ \ \ \ \ \ \isacommand{then}\isamarkupfalse%
\ \isacommand{have}\isamarkupfalse%
\ {\isachardoublequoteopen}atoms\ G\ {\isasymsubseteq}\ atoms\ F{\isadigit{1}}{\isachardoublequoteclose}\isanewline
\ \ \ \ \ \ \ \ \isacommand{by}\isamarkupfalse%
\ {\isacharparenleft}rule\ assms{\isacharparenleft}{\isadigit{1}}{\isacharparenright}{\isacharparenright}\isanewline
\ \ \ \ \ \ \isacommand{also}\isamarkupfalse%
\ \isacommand{have}\isamarkupfalse%
\ {\isachardoublequoteopen}{\isasymdots}\ {\isasymsubseteq}\ atoms\ F{\isadigit{1}}\ {\isasymunion}\ atoms\ F{\isadigit{2}}{\isachardoublequoteclose}\isanewline
\ \ \ \ \ \ \ \ \isacommand{by}\isamarkupfalse%
\ {\isacharparenleft}simp\ only{\isacharcolon}\ Un{\isacharunderscore}upper{\isadigit{1}}{\isacharparenright}\isanewline
\ \ \ \ \ \ \isacommand{also}\isamarkupfalse%
\ \isacommand{have}\isamarkupfalse%
\ {\isachardoublequoteopen}{\isasymdots}\ {\isacharequal}\ atoms\ {\isacharparenleft}F{\isadigit{1}}\ \isactrlbold {\isasymrightarrow}\ F{\isadigit{2}}{\isacharparenright}{\isachardoublequoteclose}\isanewline
\ \ \ \ \ \ \ \ \isacommand{by}\isamarkupfalse%
\ {\isacharparenleft}simp\ only{\isacharcolon}\ formula{\isachardot}set{\isacharparenleft}{\isadigit{6}}{\isacharparenright}{\isacharparenright}\isanewline
\ \ \ \ \ \ \isacommand{finally}\isamarkupfalse%
\ \isacommand{show}\isamarkupfalse%
\ {\isacharquery}thesis\isanewline
\ \ \ \ \ \ \ \ \isacommand{by}\isamarkupfalse%
\ this\isanewline
\ \ \ \ \isacommand{next}\isamarkupfalse%
\isanewline
\ \ \ \ \ \ \isacommand{assume}\isamarkupfalse%
\ {\isachardoublequoteopen}G\ {\isasymin}\ setSubformulae\ F{\isadigit{2}}{\isachardoublequoteclose}\isanewline
\ \ \ \ \ \ \isacommand{then}\isamarkupfalse%
\ \isacommand{have}\isamarkupfalse%
\ {\isachardoublequoteopen}atoms\ G\ {\isasymsubseteq}\ atoms\ F{\isadigit{2}}{\isachardoublequoteclose}\isanewline
\ \ \ \ \ \ \ \ \isacommand{by}\isamarkupfalse%
\ {\isacharparenleft}rule\ assms{\isacharparenleft}{\isadigit{2}}{\isacharparenright}{\isacharparenright}\isanewline
\ \ \ \ \ \ \isacommand{also}\isamarkupfalse%
\ \isacommand{have}\isamarkupfalse%
\ {\isachardoublequoteopen}{\isasymdots}\ {\isasymsubseteq}\ atoms\ F{\isadigit{1}}\ {\isasymunion}\ atoms\ F{\isadigit{2}}{\isachardoublequoteclose}\isanewline
\ \ \ \ \ \ \ \ \isacommand{by}\isamarkupfalse%
\ {\isacharparenleft}simp\ only{\isacharcolon}\ Un{\isacharunderscore}upper{\isadigit{2}}{\isacharparenright}\isanewline
\ \ \ \ \ \ \isacommand{also}\isamarkupfalse%
\ \isacommand{have}\isamarkupfalse%
\ {\isachardoublequoteopen}{\isasymdots}\ {\isacharequal}\ atoms\ {\isacharparenleft}F{\isadigit{1}}\ \isactrlbold {\isasymrightarrow}\ F{\isadigit{2}}{\isacharparenright}{\isachardoublequoteclose}\isanewline
\ \ \ \ \ \ \ \ \isacommand{by}\isamarkupfalse%
\ {\isacharparenleft}simp\ only{\isacharcolon}\ formula{\isachardot}set{\isacharparenleft}{\isadigit{6}}{\isacharparenright}{\isacharparenright}\isanewline
\ \ \ \ \ \ \isacommand{finally}\isamarkupfalse%
\ \isacommand{show}\isamarkupfalse%
\ {\isacharquery}thesis\isanewline
\ \ \ \ \ \ \ \ \isacommand{by}\isamarkupfalse%
\ this\isanewline
\ \ \ \ \isacommand{qed}\isamarkupfalse%
\isanewline
\ \ \isacommand{qed}\isamarkupfalse%
\isanewline
\isacommand{qed}\isamarkupfalse%
%
\endisatagproof
{\isafoldproof}%
%
\isadelimproof
\isanewline
%
\endisadelimproof
\isanewline
\isacommand{lemma}\isamarkupfalse%
\ subformulae{\isacharunderscore}atoms{\isacharcolon}\ \isanewline
\ \ {\isachardoublequoteopen}G\ {\isasymin}\ setSubformulae\ F\ {\isasymLongrightarrow}\ atoms\ G\ {\isasymsubseteq}\ atoms\ F{\isachardoublequoteclose}\isanewline
%
\isadelimproof
%
\endisadelimproof
%
\isatagproof
\isacommand{proof}\isamarkupfalse%
\ {\isacharparenleft}induction\ F{\isacharparenright}\isanewline
\ \ \isacommand{case}\isamarkupfalse%
\ {\isacharparenleft}Atom\ x{\isacharparenright}\isanewline
\ \ \isacommand{then}\isamarkupfalse%
\ \isacommand{show}\isamarkupfalse%
\ {\isacharquery}case\ \isacommand{by}\isamarkupfalse%
\ {\isacharparenleft}simp\ only{\isacharcolon}\ subformulas{\isacharunderscore}atoms{\isacharunderscore}atom{\isacharparenright}\ \isanewline
\isacommand{next}\isamarkupfalse%
\isanewline
\ \ \isacommand{case}\isamarkupfalse%
\ Bot\isanewline
\ \ \isacommand{then}\isamarkupfalse%
\ \isacommand{show}\isamarkupfalse%
\ {\isacharquery}case\ \isacommand{by}\isamarkupfalse%
\ {\isacharparenleft}simp\ only{\isacharcolon}\ subformulas{\isacharunderscore}atoms{\isacharunderscore}bot{\isacharparenright}\isanewline
\isacommand{next}\isamarkupfalse%
\isanewline
\ \ \isacommand{case}\isamarkupfalse%
\ {\isacharparenleft}Not\ F{\isacharparenright}\isanewline
\ \ \isacommand{then}\isamarkupfalse%
\ \isacommand{show}\isamarkupfalse%
\ {\isacharquery}case\ \isacommand{by}\isamarkupfalse%
\ {\isacharparenleft}simp\ only{\isacharcolon}\ subformulas{\isacharunderscore}atoms{\isacharunderscore}not{\isacharparenright}\isanewline
\isacommand{next}\isamarkupfalse%
\isanewline
\ \ \isacommand{case}\isamarkupfalse%
\ {\isacharparenleft}And\ F{\isadigit{1}}\ F{\isadigit{2}}{\isacharparenright}\isanewline
\ \ \isacommand{then}\isamarkupfalse%
\ \isacommand{show}\isamarkupfalse%
\ {\isacharquery}case\ \isacommand{by}\isamarkupfalse%
\ {\isacharparenleft}simp\ only{\isacharcolon}\ subformulas{\isacharunderscore}atoms{\isacharunderscore}and{\isacharparenright}\isanewline
\isacommand{next}\isamarkupfalse%
\isanewline
\ \ \isacommand{case}\isamarkupfalse%
\ {\isacharparenleft}Or\ F{\isadigit{1}}\ F{\isadigit{2}}{\isacharparenright}\isanewline
\ \ \isacommand{then}\isamarkupfalse%
\ \isacommand{show}\isamarkupfalse%
\ {\isacharquery}case\ \isacommand{by}\isamarkupfalse%
\ {\isacharparenleft}simp\ only{\isacharcolon}\ subformulas{\isacharunderscore}atoms{\isacharunderscore}or{\isacharparenright}\isanewline
\isacommand{next}\isamarkupfalse%
\isanewline
\ \ \isacommand{case}\isamarkupfalse%
\ {\isacharparenleft}Imp\ F{\isadigit{1}}\ F{\isadigit{2}}{\isacharparenright}\isanewline
\ \ \isacommand{then}\isamarkupfalse%
\ \isacommand{show}\isamarkupfalse%
\ {\isacharquery}case\ \isacommand{by}\isamarkupfalse%
\ {\isacharparenleft}simp\ only{\isacharcolon}\ subformulas{\isacharunderscore}atoms{\isacharunderscore}imp{\isacharparenright}\isanewline
\isacommand{qed}\isamarkupfalse%
%
\endisatagproof
{\isafoldproof}%
%
\isadelimproof
%
\endisadelimproof
%
\begin{isamarkuptext}%
Por último, su demostración automática.%
\end{isamarkuptext}\isamarkuptrue%
\isacommand{lemma}\isamarkupfalse%
\ {\isachardoublequoteopen}G\ {\isasymin}\ setSubformulae\ F\ {\isasymLongrightarrow}\ atoms\ G\ {\isasymsubseteq}\ atoms\ F{\isachardoublequoteclose}\isanewline
%
\isadelimproof
\ \ %
\endisadelimproof
%
\isatagproof
\isacommand{by}\isamarkupfalse%
\ {\isacharparenleft}induction\ F{\isacharparenright}\ auto%
\endisatagproof
{\isafoldproof}%
%
\isadelimproof
%
\endisadelimproof
%
\begin{isamarkuptext}%
A continuación vamos a introducir un lema para facilitar
   las siguientes demostraciones detalladas mediante contenciones en 
   cadena.

  \begin{lema}
    Sea \isa{G} una subfórmula de \isa{F}, entonces el conjunto de subfórmulas 
    de \isa{G} está contenido en el de \isa{F}.
  \end{lema} 

  \begin{demostracion}
  La prueba es por inducción en la estructura de fórmula.
  
  Sea \isa{p} una fórmula atómica cualquiera. Por definición, el conjunto de
  sus subfórmulas es \isa{{\isacharbraceleft}p{\isacharbraceright}}, luego su única subfórmula es ella misma y,
  por tanto, tienen igual conjunto de subfórmulas.

  Sea la fórmula \isa{{\isasymbottom}}. Por definición, el conjunto de
  sus subfórmulas es \isa{{\isacharbraceleft}{\isasymbottom}{\isacharbraceright}}, luego su única subfórmula es ella misma y,
  por tanto, tienen igual conjunto de subfórmulas.

  Sea una fórmula \isa{F} tal que para toda subfórmula suya se tiene que el
  conjunto de sus subfórmulas está contenido en el conjunto de 
  subfórmulas de \isa{F}.
  Supongamos \isa{G} subfórmula de \isa{{\isasymnot}\ F}. Vamos a probar que el conjunto de
  subfórmulas de \isa{G} está contenido en el de \isa{{\isasymnot}\ F}.
  En primer lugar, por definición se cumple que el conjunto de
  subfórmulas de \isa{{\isasymnot}\ F} es de la forma \isa{Subf{\isacharparenleft}{\isasymnot}\ F{\isacharparenright}\ {\isacharequal}\ {\isacharbraceleft}{\isasymnot}\ F{\isacharbraceright}\ {\isasymunion}\ Subf{\isacharparenleft}F{\isacharparenright}}.
  Como hemos supuesto \isa{G} subfórmula de \isa{{\isasymnot}\ F}, hay dos opciones 
  posibles: \isa{G\ {\isasymin}\ {\isacharbraceleft}{\isasymnot}\ F{\isacharbraceright}} o \isa{G\ {\isasymin}\ Subf{\isacharparenleft}F{\isacharparenright}}. 
  Del primer caso se obtiene que \isa{G\ {\isacharequal}\ {\isasymnot}\ F} y, por tanto, tienen igual 
  conjunto de subfórmulas. 
  Por otro lado si suponemos que \isa{G} es subfórmula de \isa{F}, por hipótesis
  de inducción tenemos que el conjunto de subfórmulas de \isa{G} está 
  contenido en el de \isa{F}. Como, a su vez, el conjunto de subfórmulas
  de \isa{F} está contenido en el de \isa{{\isasymnot}\ F} según la definición anterior, 
  por propiedades de la contención de verifica que el conjunto de 
  subfórmulas de \isa{G} está contenido en el de \isa{{\isasymnot}\ F}, como queríamos 
  demostrar.

  Sean las fórmulas \isa{F{\isadigit{1}}} y \isa{F{\isadigit{2}}} tales que para cualquier subfórmula
  de \isa{F{\isadigit{1}}} el conjunto de sus subfórmulas está contenido en el conjunto 
  de subfórmulas de \isa{F{\isadigit{1}}}, y para cualquier subfórmula de \isa{F{\isadigit{2}}} el 
  conjunto de sus subfórmulas está contenido en el conjunto de 
  subfórmulas de \isa{F{\isadigit{2}}}. Supongamos \isa{G} 
  subfórmula de \isa{F{\isadigit{1}}{\isacharasterisk}F{\isadigit{2}}} donde \isa{{\isacharasterisk}} simboliza una conectiva binaria 
  cualquiera. Vamos a probar que el conjunto de subfórmulas de \isa{G} está
  contenido en el de \isa{F{\isadigit{1}}{\isacharasterisk}F{\isadigit{2}}}. 
  En primer lugar, por definición se cumple que el conjunto de 
  subfórmulas de \isa{F{\isadigit{1}}{\isacharasterisk}F{\isadigit{2}}} es de la forma
  \isa{{\isacharbraceleft}F{\isadigit{1}}{\isacharasterisk}F{\isadigit{2}}{\isacharbraceright}\ {\isasymunion}\ {\isacharparenleft}Subf{\isacharparenleft}F{\isadigit{1}}{\isacharparenright}\ {\isasymunion}\ Subf{\isacharparenleft}F{\isadigit{2}}{\isacharparenright}{\isacharparenright}}. De este modo,
  tenemos dos opciones: \isa{G\ {\isasymin}\ {\isacharbraceleft}F{\isadigit{1}}{\isacharasterisk}F{\isadigit{2}}{\isacharbraceright}} o \isa{G\ {\isasymin}\ Subf{\isacharparenleft}F{\isadigit{1}}{\isacharparenright}\ {\isasymunion}\ Subf{\isacharparenleft}F{\isadigit{2}}{\isacharparenright}}.
  De la primera opción se deduce \isa{G\ {\isacharequal}\ F{\isadigit{1}}{\isacharasterisk}F{\isadigit{2}}} y, por
  tanto, tienen igual conjunto de subfórmulas. 
  Por otro lado, si \isa{G\ {\isasymin}\ Subf{\isacharparenleft}F{\isadigit{1}}{\isacharparenright}\ {\isasymunion}\ Subf{\isacharparenleft}F{\isadigit{2}}{\isacharparenright}}, tenemos a su vez dos 
  opciones: \isa{G} es subfórmula de \isa{F{\isadigit{1}}} o \isa{G} es subfórmula de \isa{F{\isadigit{2}}}.
  Supongamos que fuese subfórmula de \isa{F{\isadigit{1}}}. En este caso, por hipótesis 
  de inducción se tiene que el conjunto de subfórmulas de \isa{G} está 
  contenido en el de \isa{F{\isadigit{1}}}. Por la definición anterior del conjunto de 
  subfórmulas de \isa{F{\isadigit{1}}{\isacharasterisk}F{\isadigit{2}}}, se verifica que el conjunto de subfórmulas de 
  \isa{F{\isadigit{1}}} está contenido en el de \isa{F{\isadigit{1}}{\isacharasterisk}F{\isadigit{2}}}. Por tanto, por propiedades de
  contención se tiene que el conjunto de subfórmulas de \isa{G} está 
  contenido en el conjunto de subfórmulas de \isa{F{\isadigit{1}}{\isacharasterisk}F{\isadigit{2}}}. El caso de \isa{G} 
  subfórmula de \isa{F{\isadigit{2}}} se demuestra análogamente cambiando el 
  índice de la fórmula correspondiente. Por tanto, se verifica el 
  resultado en este caso.  
  \end{demostracion}

Veamos su formalización en Isabelle junto con su demostración 
  estructurada.%
\end{isamarkuptext}\isamarkuptrue%
\isacommand{lemma}\isamarkupfalse%
\ subContsubformulae{\isacharunderscore}atom{\isacharcolon}\ \isanewline
\ \ \isakeyword{assumes}\ {\isachardoublequoteopen}G\ {\isasymin}\ setSubformulae\ {\isacharparenleft}Atom\ x{\isacharparenright}{\isachardoublequoteclose}\ \isanewline
\ \ \isakeyword{shows}\ {\isachardoublequoteopen}setSubformulae\ G\ {\isasymsubseteq}\ setSubformulae\ {\isacharparenleft}Atom\ x{\isacharparenright}{\isachardoublequoteclose}\isanewline
%
\isadelimproof
%
\endisadelimproof
%
\isatagproof
\isacommand{proof}\isamarkupfalse%
\ {\isacharminus}\ \isanewline
\ \ \isacommand{have}\isamarkupfalse%
\ {\isachardoublequoteopen}G\ {\isasymin}\ {\isacharbraceleft}Atom\ x{\isacharbraceright}{\isachardoublequoteclose}\ \isacommand{using}\isamarkupfalse%
\ assms\ \isanewline
\ \ \ \ \isacommand{by}\isamarkupfalse%
\ {\isacharparenleft}simp\ only{\isacharcolon}\ setSubformulae{\isacharunderscore}atom{\isacharparenright}\isanewline
\ \ \isacommand{then}\isamarkupfalse%
\ \isacommand{have}\isamarkupfalse%
\ {\isachardoublequoteopen}G\ {\isacharequal}\ Atom\ x{\isachardoublequoteclose}\isanewline
\ \ \ \ \isacommand{by}\isamarkupfalse%
\ {\isacharparenleft}simp\ only{\isacharcolon}\ singletonD{\isacharparenright}\isanewline
\ \ \isacommand{then}\isamarkupfalse%
\ \isacommand{show}\isamarkupfalse%
\ {\isacharquery}thesis\isanewline
\ \ \ \ \isacommand{by}\isamarkupfalse%
\ {\isacharparenleft}simp\ only{\isacharcolon}\ subset{\isacharunderscore}refl{\isacharparenright}\isanewline
\isacommand{qed}\isamarkupfalse%
%
\endisatagproof
{\isafoldproof}%
%
\isadelimproof
\isanewline
%
\endisadelimproof
\isanewline
\isacommand{lemma}\isamarkupfalse%
\ subContsubformulae{\isacharunderscore}bot{\isacharcolon}\isanewline
\ \ \isakeyword{assumes}\ {\isachardoublequoteopen}G\ {\isasymin}\ setSubformulae\ {\isasymbottom}{\isachardoublequoteclose}\ \isanewline
\ \ \isakeyword{shows}\ \ \ {\isachardoublequoteopen}setSubformulae\ G\ {\isasymsubseteq}\ setSubformulae\ {\isasymbottom}{\isachardoublequoteclose}\isanewline
%
\isadelimproof
%
\endisadelimproof
%
\isatagproof
\isacommand{proof}\isamarkupfalse%
\ {\isacharminus}\isanewline
\ \ \isacommand{have}\isamarkupfalse%
\ {\isachardoublequoteopen}G\ {\isasymin}\ {\isacharbraceleft}{\isasymbottom}{\isacharbraceright}{\isachardoublequoteclose}\isanewline
\ \ \ \ \isacommand{using}\isamarkupfalse%
\ assms\isanewline
\ \ \ \ \isacommand{by}\isamarkupfalse%
\ {\isacharparenleft}simp\ only{\isacharcolon}\ setSubformulae{\isacharunderscore}bot{\isacharparenright}\isanewline
\ \ \isacommand{then}\isamarkupfalse%
\ \isacommand{have}\isamarkupfalse%
\ {\isachardoublequoteopen}G\ {\isacharequal}\ {\isasymbottom}{\isachardoublequoteclose}\isanewline
\ \ \ \ \isacommand{by}\isamarkupfalse%
\ {\isacharparenleft}simp\ only{\isacharcolon}\ singletonD{\isacharparenright}\isanewline
\ \ \isacommand{then}\isamarkupfalse%
\ \isacommand{show}\isamarkupfalse%
\ {\isacharquery}thesis\isanewline
\ \ \ \ \isacommand{by}\isamarkupfalse%
\ {\isacharparenleft}simp\ only{\isacharcolon}\ subset{\isacharunderscore}refl{\isacharparenright}\isanewline
\isacommand{qed}\isamarkupfalse%
%
\endisatagproof
{\isafoldproof}%
%
\isadelimproof
\isanewline
%
\endisadelimproof
\isanewline
\isacommand{lemma}\isamarkupfalse%
\ subContsubformulae{\isacharunderscore}not{\isacharcolon}\isanewline
\ \ \isakeyword{assumes}\ {\isachardoublequoteopen}G\ {\isasymin}\ setSubformulae\ F\ {\isasymLongrightarrow}\ setSubformulae\ G\ {\isasymsubseteq}\ setSubformulae\ F{\isachardoublequoteclose}\isanewline
\ \ \ \ \ \ \ \ \ \ {\isachardoublequoteopen}G\ {\isasymin}\ setSubformulae\ {\isacharparenleft}\isactrlbold {\isasymnot}\ F{\isacharparenright}{\isachardoublequoteclose}\isanewline
\ \ \isakeyword{shows}\ \ \ {\isachardoublequoteopen}setSubformulae\ G\ {\isasymsubseteq}\ setSubformulae\ {\isacharparenleft}\isactrlbold {\isasymnot}\ F{\isacharparenright}{\isachardoublequoteclose}\isanewline
%
\isadelimproof
%
\endisadelimproof
%
\isatagproof
\isacommand{proof}\isamarkupfalse%
\ {\isacharminus}\isanewline
\ \ \isacommand{have}\isamarkupfalse%
\ {\isachardoublequoteopen}G\ {\isasymin}\ {\isacharbraceleft}\isactrlbold {\isasymnot}\ F{\isacharbraceright}\ {\isasymunion}\ setSubformulae\ F{\isachardoublequoteclose}\isanewline
\ \ \ \ \isacommand{using}\isamarkupfalse%
\ assms{\isacharparenleft}{\isadigit{2}}{\isacharparenright}\isanewline
\ \ \ \ \isacommand{by}\isamarkupfalse%
\ {\isacharparenleft}simp\ only{\isacharcolon}\ setSubformulae{\isacharunderscore}not{\isacharparenright}\ \isanewline
\ \ \isacommand{then}\isamarkupfalse%
\ \isacommand{have}\isamarkupfalse%
\ {\isachardoublequoteopen}G\ {\isasymin}\ {\isacharbraceleft}\isactrlbold {\isasymnot}\ F{\isacharbraceright}\ {\isasymor}\ G\ {\isasymin}\ setSubformulae\ F{\isachardoublequoteclose}\isanewline
\ \ \ \ \isacommand{by}\isamarkupfalse%
\ {\isacharparenleft}simp\ only{\isacharcolon}\ Un{\isacharunderscore}iff{\isacharparenright}\isanewline
\ \ \isacommand{then}\isamarkupfalse%
\ \isacommand{show}\isamarkupfalse%
\ {\isachardoublequoteopen}setSubformulae\ G\ {\isasymsubseteq}\ setSubformulae\ {\isacharparenleft}\isactrlbold {\isasymnot}\ F{\isacharparenright}{\isachardoublequoteclose}\isanewline
\ \ \isacommand{proof}\isamarkupfalse%
\isanewline
\ \ \ \ \isacommand{assume}\isamarkupfalse%
\ {\isachardoublequoteopen}G\ {\isasymin}\ {\isacharbraceleft}\isactrlbold {\isasymnot}\ F{\isacharbraceright}{\isachardoublequoteclose}\isanewline
\ \ \ \ \isacommand{then}\isamarkupfalse%
\ \isacommand{have}\isamarkupfalse%
\ {\isachardoublequoteopen}G\ {\isacharequal}\ \isactrlbold {\isasymnot}\ F{\isachardoublequoteclose}\isanewline
\ \ \ \ \ \ \isacommand{by}\isamarkupfalse%
\ {\isacharparenleft}simp\ only{\isacharcolon}\ singletonD{\isacharparenright}\isanewline
\ \ \ \ \isacommand{then}\isamarkupfalse%
\ \isacommand{show}\isamarkupfalse%
\ {\isacharquery}thesis\isanewline
\ \ \ \ \ \ \isacommand{by}\isamarkupfalse%
\ {\isacharparenleft}simp\ only{\isacharcolon}\ subset{\isacharunderscore}refl{\isacharparenright}\isanewline
\ \ \isacommand{next}\isamarkupfalse%
\isanewline
\ \ \ \ \isacommand{assume}\isamarkupfalse%
\ {\isachardoublequoteopen}G\ {\isasymin}\ setSubformulae\ F{\isachardoublequoteclose}\isanewline
\ \ \ \ \isacommand{then}\isamarkupfalse%
\ \isacommand{have}\isamarkupfalse%
\ {\isachardoublequoteopen}setSubformulae\ G\ {\isasymsubseteq}\ setSubformulae\ F{\isachardoublequoteclose}\isanewline
\ \ \ \ \ \ \isacommand{by}\isamarkupfalse%
\ {\isacharparenleft}simp\ only{\isacharcolon}\ assms{\isacharparenleft}{\isadigit{1}}{\isacharparenright}{\isacharparenright}\isanewline
\ \ \ \ \isacommand{also}\isamarkupfalse%
\ \isacommand{have}\isamarkupfalse%
\ {\isachardoublequoteopen}setSubformulae\ F\ {\isasymsubseteq}\ setSubformulae\ {\isacharparenleft}\isactrlbold {\isasymnot}\ F{\isacharparenright}{\isachardoublequoteclose}\isanewline
\ \ \ \ \ \ \isacommand{by}\isamarkupfalse%
\ {\isacharparenleft}simp\ only{\isacharcolon}\ setSubformulae{\isacharunderscore}not\ Un{\isacharunderscore}upper{\isadigit{2}}{\isacharparenright}\isanewline
\ \ \ \ \isacommand{finally}\isamarkupfalse%
\ \isacommand{show}\isamarkupfalse%
\ {\isacharquery}thesis\ \isanewline
\ \ \ \ \ \ \isacommand{by}\isamarkupfalse%
\ this\isanewline
\ \ \isacommand{qed}\isamarkupfalse%
\isanewline
\isacommand{qed}\isamarkupfalse%
%
\endisatagproof
{\isafoldproof}%
%
\isadelimproof
\isanewline
%
\endisadelimproof
\isanewline
\isacommand{lemma}\isamarkupfalse%
\ subContsubformulae{\isacharunderscore}and{\isacharcolon}\isanewline
\ \ \isakeyword{assumes}\ {\isachardoublequoteopen}G\ {\isasymin}\ setSubformulae\ F{\isadigit{1}}\ \isanewline
\ \ \ \ \ \ \ \ \ \ \ \ {\isasymLongrightarrow}\ setSubformulae\ G\ {\isasymsubseteq}\ setSubformulae\ F{\isadigit{1}}{\isachardoublequoteclose}\isanewline
\ \ \ \ \ \ \ \ \ \ {\isachardoublequoteopen}G\ {\isasymin}\ setSubformulae\ F{\isadigit{2}}\ \isanewline
\ \ \ \ \ \ \ \ \ \ \ \ {\isasymLongrightarrow}\ setSubformulae\ G\ {\isasymsubseteq}\ setSubformulae\ F{\isadigit{2}}{\isachardoublequoteclose}\isanewline
\ \ \ \ \ \ \ \ \ \ {\isachardoublequoteopen}G\ {\isasymin}\ setSubformulae\ {\isacharparenleft}F{\isadigit{1}}\ \isactrlbold {\isasymand}\ F{\isadigit{2}}{\isacharparenright}{\isachardoublequoteclose}\isanewline
\ \ \isakeyword{shows}\ \ \ {\isachardoublequoteopen}setSubformulae\ G\ {\isasymsubseteq}\ setSubformulae\ {\isacharparenleft}F{\isadigit{1}}\ \isactrlbold {\isasymand}\ F{\isadigit{2}}{\isacharparenright}{\isachardoublequoteclose}\isanewline
%
\isadelimproof
%
\endisadelimproof
%
\isatagproof
\isacommand{proof}\isamarkupfalse%
\ {\isacharminus}\isanewline
\ \ \isacommand{have}\isamarkupfalse%
\ {\isachardoublequoteopen}G\ {\isasymin}\ {\isacharbraceleft}F{\isadigit{1}}\ \isactrlbold {\isasymand}\ F{\isadigit{2}}{\isacharbraceright}\ {\isasymunion}\ {\isacharparenleft}setSubformulae\ F{\isadigit{1}}\ {\isasymunion}\ setSubformulae\ F{\isadigit{2}}{\isacharparenright}{\isachardoublequoteclose}\isanewline
\ \ \ \ \isacommand{using}\isamarkupfalse%
\ assms{\isacharparenleft}{\isadigit{3}}{\isacharparenright}\ \isanewline
\ \ \ \ \isacommand{by}\isamarkupfalse%
\ {\isacharparenleft}simp\ only{\isacharcolon}\ setSubformulae{\isacharunderscore}and{\isacharparenright}\isanewline
\ \ \isacommand{then}\isamarkupfalse%
\ \isacommand{have}\isamarkupfalse%
\ {\isachardoublequoteopen}G\ {\isasymin}\ {\isacharbraceleft}F{\isadigit{1}}\ \isactrlbold {\isasymand}\ F{\isadigit{2}}{\isacharbraceright}\ {\isasymor}\ G\ {\isasymin}\ setSubformulae\ F{\isadigit{1}}\ {\isasymunion}\ setSubformulae\ F{\isadigit{2}}{\isachardoublequoteclose}\isanewline
\ \ \ \ \isacommand{by}\isamarkupfalse%
\ {\isacharparenleft}simp\ only{\isacharcolon}\ Un{\isacharunderscore}iff{\isacharparenright}\isanewline
\ \ \isacommand{then}\isamarkupfalse%
\ \isacommand{show}\isamarkupfalse%
\ {\isacharquery}thesis\isanewline
\ \ \isacommand{proof}\isamarkupfalse%
\ {\isacharparenleft}rule\ disjE{\isacharparenright}\isanewline
\ \ \ \ \isacommand{assume}\isamarkupfalse%
\ {\isachardoublequoteopen}G\ {\isasymin}\ {\isacharbraceleft}F{\isadigit{1}}\ \isactrlbold {\isasymand}\ F{\isadigit{2}}{\isacharbraceright}{\isachardoublequoteclose}\isanewline
\ \ \ \ \isacommand{then}\isamarkupfalse%
\ \isacommand{have}\isamarkupfalse%
\ {\isachardoublequoteopen}G\ {\isacharequal}\ F{\isadigit{1}}\ \isactrlbold {\isasymand}\ F{\isadigit{2}}{\isachardoublequoteclose}\isanewline
\ \ \ \ \ \ \isacommand{by}\isamarkupfalse%
\ {\isacharparenleft}simp\ only{\isacharcolon}\ singletonD{\isacharparenright}\isanewline
\ \ \ \ \isacommand{then}\isamarkupfalse%
\ \isacommand{show}\isamarkupfalse%
\ {\isacharquery}thesis\isanewline
\ \ \ \ \ \ \isacommand{by}\isamarkupfalse%
\ {\isacharparenleft}simp\ only{\isacharcolon}\ subset{\isacharunderscore}refl{\isacharparenright}\isanewline
\ \ \isacommand{next}\isamarkupfalse%
\isanewline
\ \ \ \ \isacommand{assume}\isamarkupfalse%
\ {\isachardoublequoteopen}G\ {\isasymin}\ setSubformulae\ F{\isadigit{1}}\ {\isasymunion}\ setSubformulae\ F{\isadigit{2}}{\isachardoublequoteclose}\isanewline
\ \ \ \ \isacommand{then}\isamarkupfalse%
\ \isacommand{have}\isamarkupfalse%
\ {\isachardoublequoteopen}G\ {\isasymin}\ setSubformulae\ F{\isadigit{1}}\ {\isasymor}\ G\ {\isasymin}\ setSubformulae\ F{\isadigit{2}}{\isachardoublequoteclose}\ \ \isanewline
\ \ \ \ \ \ \isacommand{by}\isamarkupfalse%
\ {\isacharparenleft}simp\ only{\isacharcolon}\ Un{\isacharunderscore}iff{\isacharparenright}\isanewline
\ \ \ \ \isacommand{then}\isamarkupfalse%
\ \isacommand{show}\isamarkupfalse%
\ {\isacharquery}thesis\isanewline
\ \ \ \ \isacommand{proof}\isamarkupfalse%
\ \isanewline
\ \ \ \ \ \ \isacommand{assume}\isamarkupfalse%
\ {\isachardoublequoteopen}G\ {\isasymin}\ setSubformulae\ F{\isadigit{1}}{\isachardoublequoteclose}\isanewline
\ \ \ \ \ \ \isacommand{then}\isamarkupfalse%
\ \isacommand{have}\isamarkupfalse%
\ {\isachardoublequoteopen}setSubformulae\ G\ {\isasymsubseteq}\ setSubformulae\ F{\isadigit{1}}{\isachardoublequoteclose}\isanewline
\ \ \ \ \ \ \ \ \isacommand{by}\isamarkupfalse%
\ {\isacharparenleft}simp\ only{\isacharcolon}\ assms{\isacharparenleft}{\isadigit{1}}{\isacharparenright}{\isacharparenright}\isanewline
\ \ \ \ \ \ \isacommand{also}\isamarkupfalse%
\ \isacommand{have}\isamarkupfalse%
\ {\isachardoublequoteopen}{\isasymdots}\ {\isasymsubseteq}\ setSubformulae\ F{\isadigit{1}}\ {\isasymunion}\ setSubformulae\ F{\isadigit{2}}{\isachardoublequoteclose}\isanewline
\ \ \ \ \ \ \ \ \isacommand{by}\isamarkupfalse%
\ {\isacharparenleft}simp\ only{\isacharcolon}\ Un{\isacharunderscore}upper{\isadigit{1}}{\isacharparenright}\isanewline
\ \ \ \ \ \ \isacommand{also}\isamarkupfalse%
\ \isacommand{have}\isamarkupfalse%
\ {\isachardoublequoteopen}{\isasymdots}\ {\isasymsubseteq}\ setSubformulae\ {\isacharparenleft}F{\isadigit{1}}\ \isactrlbold {\isasymand}\ F{\isadigit{2}}{\isacharparenright}{\isachardoublequoteclose}\isanewline
\ \ \ \ \ \ \ \ \isacommand{by}\isamarkupfalse%
\ {\isacharparenleft}simp\ only{\isacharcolon}\ setSubformulae{\isacharunderscore}and\ Un{\isacharunderscore}upper{\isadigit{2}}{\isacharparenright}\isanewline
\ \ \ \ \ \ \isacommand{finally}\isamarkupfalse%
\ \isacommand{show}\isamarkupfalse%
\ {\isacharquery}thesis\isanewline
\ \ \ \ \ \ \ \ \isacommand{by}\isamarkupfalse%
\ this\isanewline
\ \ \ \ \isacommand{next}\isamarkupfalse%
\isanewline
\ \ \ \ \ \ \isacommand{assume}\isamarkupfalse%
\ {\isachardoublequoteopen}G\ {\isasymin}\ setSubformulae\ F{\isadigit{2}}{\isachardoublequoteclose}\isanewline
\ \ \ \ \ \ \isacommand{then}\isamarkupfalse%
\ \isacommand{have}\isamarkupfalse%
\ {\isachardoublequoteopen}setSubformulae\ G\ {\isasymsubseteq}\ setSubformulae\ F{\isadigit{2}}{\isachardoublequoteclose}\isanewline
\ \ \ \ \ \ \ \ \isacommand{by}\isamarkupfalse%
\ {\isacharparenleft}rule\ assms{\isacharparenleft}{\isadigit{2}}{\isacharparenright}{\isacharparenright}\isanewline
\ \ \ \ \ \ \isacommand{also}\isamarkupfalse%
\ \isacommand{have}\isamarkupfalse%
\ {\isachardoublequoteopen}{\isasymdots}\ {\isasymsubseteq}\ setSubformulae\ F{\isadigit{1}}\ {\isasymunion}\ setSubformulae\ F{\isadigit{2}}{\isachardoublequoteclose}\isanewline
\ \ \ \ \ \ \ \ \isacommand{by}\isamarkupfalse%
\ {\isacharparenleft}simp\ only{\isacharcolon}\ Un{\isacharunderscore}upper{\isadigit{2}}{\isacharparenright}\isanewline
\ \ \ \ \ \ \isacommand{also}\isamarkupfalse%
\ \isacommand{have}\isamarkupfalse%
\ {\isachardoublequoteopen}{\isasymdots}\ {\isasymsubseteq}\ setSubformulae\ {\isacharparenleft}F{\isadigit{1}}\ \isactrlbold {\isasymand}\ F{\isadigit{2}}{\isacharparenright}{\isachardoublequoteclose}\isanewline
\ \ \ \ \ \ \ \ \isacommand{by}\isamarkupfalse%
\ {\isacharparenleft}simp\ only{\isacharcolon}\ setSubformulae{\isacharunderscore}and\ Un{\isacharunderscore}upper{\isadigit{2}}{\isacharparenright}\isanewline
\ \ \ \ \ \ \isacommand{finally}\isamarkupfalse%
\ \isacommand{show}\isamarkupfalse%
\ {\isacharquery}thesis\isanewline
\ \ \ \ \ \ \ \ \isacommand{by}\isamarkupfalse%
\ this\isanewline
\ \ \ \ \isacommand{qed}\isamarkupfalse%
\isanewline
\ \ \isacommand{qed}\isamarkupfalse%
\isanewline
\isacommand{qed}\isamarkupfalse%
%
\endisatagproof
{\isafoldproof}%
%
\isadelimproof
\isanewline
%
\endisadelimproof
\isanewline
\isacommand{lemma}\isamarkupfalse%
\ subContsubformulae{\isacharunderscore}or{\isacharcolon}\isanewline
\ \ \isakeyword{assumes}\ {\isachardoublequoteopen}G\ {\isasymin}\ setSubformulae\ F{\isadigit{1}}\ \isanewline
\ \ \ \ \ \ \ \ \ \ \ \ {\isasymLongrightarrow}\ setSubformulae\ G\ {\isasymsubseteq}\ setSubformulae\ F{\isadigit{1}}{\isachardoublequoteclose}\isanewline
\ \ \ \ \ \ \ \ \ \ {\isachardoublequoteopen}G\ {\isasymin}\ setSubformulae\ F{\isadigit{2}}\ \isanewline
\ \ \ \ \ \ \ \ \ \ \ \ {\isasymLongrightarrow}\ setSubformulae\ G\ {\isasymsubseteq}\ setSubformulae\ F{\isadigit{2}}{\isachardoublequoteclose}\isanewline
\ \ \ \ \ \ \ \ \ \ {\isachardoublequoteopen}G\ {\isasymin}\ setSubformulae\ {\isacharparenleft}F{\isadigit{1}}\ \isactrlbold {\isasymor}\ F{\isadigit{2}}{\isacharparenright}{\isachardoublequoteclose}\isanewline
\ \ \isakeyword{shows}\ \ \ {\isachardoublequoteopen}setSubformulae\ G\ {\isasymsubseteq}\ setSubformulae\ {\isacharparenleft}F{\isadigit{1}}\ \isactrlbold {\isasymor}\ F{\isadigit{2}}{\isacharparenright}{\isachardoublequoteclose}\isanewline
%
\isadelimproof
%
\endisadelimproof
%
\isatagproof
\isacommand{proof}\isamarkupfalse%
\ {\isacharminus}\isanewline
\ \ \isacommand{have}\isamarkupfalse%
\ {\isachardoublequoteopen}G\ {\isasymin}\ {\isacharbraceleft}F{\isadigit{1}}\ \isactrlbold {\isasymor}\ F{\isadigit{2}}{\isacharbraceright}\ {\isasymunion}\ {\isacharparenleft}setSubformulae\ F{\isadigit{1}}\ {\isasymunion}\ setSubformulae\ F{\isadigit{2}}{\isacharparenright}{\isachardoublequoteclose}\isanewline
\ \ \ \ \isacommand{using}\isamarkupfalse%
\ assms{\isacharparenleft}{\isadigit{3}}{\isacharparenright}\ \isanewline
\ \ \ \ \isacommand{by}\isamarkupfalse%
\ {\isacharparenleft}simp\ only{\isacharcolon}\ setSubformulae{\isacharunderscore}or{\isacharparenright}\isanewline
\ \ \isacommand{then}\isamarkupfalse%
\ \isacommand{have}\isamarkupfalse%
\ {\isachardoublequoteopen}G\ {\isasymin}\ {\isacharbraceleft}F{\isadigit{1}}\ \isactrlbold {\isasymor}\ F{\isadigit{2}}{\isacharbraceright}\ {\isasymor}\ G\ {\isasymin}\ setSubformulae\ F{\isadigit{1}}\ {\isasymunion}\ setSubformulae\ F{\isadigit{2}}{\isachardoublequoteclose}\isanewline
\ \ \ \ \isacommand{by}\isamarkupfalse%
\ {\isacharparenleft}simp\ only{\isacharcolon}\ Un{\isacharunderscore}iff{\isacharparenright}\isanewline
\ \ \isacommand{then}\isamarkupfalse%
\ \isacommand{show}\isamarkupfalse%
\ {\isacharquery}thesis\isanewline
\ \ \isacommand{proof}\isamarkupfalse%
\ {\isacharparenleft}rule\ disjE{\isacharparenright}\isanewline
\ \ \ \ \isacommand{assume}\isamarkupfalse%
\ {\isachardoublequoteopen}G\ {\isasymin}\ {\isacharbraceleft}F{\isadigit{1}}\ \isactrlbold {\isasymor}\ F{\isadigit{2}}{\isacharbraceright}{\isachardoublequoteclose}\isanewline
\ \ \ \ \isacommand{then}\isamarkupfalse%
\ \isacommand{have}\isamarkupfalse%
\ {\isachardoublequoteopen}G\ {\isacharequal}\ F{\isadigit{1}}\ \isactrlbold {\isasymor}\ F{\isadigit{2}}{\isachardoublequoteclose}\isanewline
\ \ \ \ \ \ \isacommand{by}\isamarkupfalse%
\ {\isacharparenleft}simp\ only{\isacharcolon}\ singletonD{\isacharparenright}\isanewline
\ \ \ \ \isacommand{then}\isamarkupfalse%
\ \isacommand{show}\isamarkupfalse%
\ {\isacharquery}thesis\isanewline
\ \ \ \ \ \ \isacommand{by}\isamarkupfalse%
\ {\isacharparenleft}simp\ only{\isacharcolon}\ subset{\isacharunderscore}refl{\isacharparenright}\isanewline
\ \ \isacommand{next}\isamarkupfalse%
\isanewline
\ \ \ \ \isacommand{assume}\isamarkupfalse%
\ {\isachardoublequoteopen}G\ {\isasymin}\ setSubformulae\ F{\isadigit{1}}\ {\isasymunion}\ setSubformulae\ F{\isadigit{2}}{\isachardoublequoteclose}\isanewline
\ \ \ \ \isacommand{then}\isamarkupfalse%
\ \isacommand{have}\isamarkupfalse%
\ {\isachardoublequoteopen}G\ {\isasymin}\ setSubformulae\ F{\isadigit{1}}\ {\isasymor}\ G\ {\isasymin}\ setSubformulae\ F{\isadigit{2}}{\isachardoublequoteclose}\ \ \isanewline
\ \ \ \ \ \ \isacommand{by}\isamarkupfalse%
\ {\isacharparenleft}simp\ only{\isacharcolon}\ Un{\isacharunderscore}iff{\isacharparenright}\isanewline
\ \ \ \ \isacommand{then}\isamarkupfalse%
\ \isacommand{show}\isamarkupfalse%
\ {\isacharquery}thesis\isanewline
\ \ \ \ \isacommand{proof}\isamarkupfalse%
\ {\isacharparenleft}rule\ disjE{\isacharparenright}\isanewline
\ \ \ \ \ \ \isacommand{assume}\isamarkupfalse%
\ {\isachardoublequoteopen}G\ {\isasymin}\ setSubformulae\ F{\isadigit{1}}{\isachardoublequoteclose}\isanewline
\ \ \ \ \ \ \isacommand{then}\isamarkupfalse%
\ \isacommand{have}\isamarkupfalse%
\ {\isachardoublequoteopen}setSubformulae\ G\ {\isasymsubseteq}\ setSubformulae\ F{\isadigit{1}}{\isachardoublequoteclose}\isanewline
\ \ \ \ \ \ \ \ \isacommand{by}\isamarkupfalse%
\ {\isacharparenleft}simp\ only{\isacharcolon}\ assms{\isacharparenleft}{\isadigit{1}}{\isacharparenright}{\isacharparenright}\isanewline
\ \ \ \ \ \ \isacommand{also}\isamarkupfalse%
\ \isacommand{have}\isamarkupfalse%
\ {\isachardoublequoteopen}{\isasymdots}\ {\isasymsubseteq}\ setSubformulae\ F{\isadigit{1}}\ {\isasymunion}\ setSubformulae\ F{\isadigit{2}}{\isachardoublequoteclose}\isanewline
\ \ \ \ \ \ \ \ \isacommand{by}\isamarkupfalse%
\ {\isacharparenleft}simp\ only{\isacharcolon}\ Un{\isacharunderscore}upper{\isadigit{1}}{\isacharparenright}\isanewline
\ \ \ \ \ \ \isacommand{also}\isamarkupfalse%
\ \isacommand{have}\isamarkupfalse%
\ {\isachardoublequoteopen}{\isasymdots}\ {\isasymsubseteq}\ setSubformulae\ {\isacharparenleft}F{\isadigit{1}}\ \isactrlbold {\isasymor}\ F{\isadigit{2}}{\isacharparenright}{\isachardoublequoteclose}\isanewline
\ \ \ \ \ \ \ \ \isacommand{by}\isamarkupfalse%
\ {\isacharparenleft}simp\ only{\isacharcolon}\ setSubformulae{\isacharunderscore}or\ Un{\isacharunderscore}upper{\isadigit{2}}{\isacharparenright}\isanewline
\ \ \ \ \ \ \isacommand{finally}\isamarkupfalse%
\ \isacommand{show}\isamarkupfalse%
\ {\isacharquery}thesis\isanewline
\ \ \ \ \ \ \ \ \isacommand{by}\isamarkupfalse%
\ this\isanewline
\ \ \ \ \isacommand{next}\isamarkupfalse%
\isanewline
\ \ \ \ \ \ \isacommand{assume}\isamarkupfalse%
\ {\isachardoublequoteopen}G\ {\isasymin}\ setSubformulae\ F{\isadigit{2}}{\isachardoublequoteclose}\isanewline
\ \ \ \ \ \ \isacommand{then}\isamarkupfalse%
\ \isacommand{have}\isamarkupfalse%
\ {\isachardoublequoteopen}setSubformulae\ G\ {\isasymsubseteq}\ setSubformulae\ F{\isadigit{2}}{\isachardoublequoteclose}\isanewline
\ \ \ \ \ \ \ \ \isacommand{by}\isamarkupfalse%
\ {\isacharparenleft}rule\ assms{\isacharparenleft}{\isadigit{2}}{\isacharparenright}{\isacharparenright}\isanewline
\ \ \ \ \ \ \isacommand{also}\isamarkupfalse%
\ \isacommand{have}\isamarkupfalse%
\ {\isachardoublequoteopen}{\isasymdots}\ {\isasymsubseteq}\ setSubformulae\ F{\isadigit{1}}\ {\isasymunion}\ setSubformulae\ F{\isadigit{2}}{\isachardoublequoteclose}\isanewline
\ \ \ \ \ \ \ \ \isacommand{by}\isamarkupfalse%
\ {\isacharparenleft}simp\ only{\isacharcolon}\ Un{\isacharunderscore}upper{\isadigit{2}}{\isacharparenright}\isanewline
\ \ \ \ \ \ \isacommand{also}\isamarkupfalse%
\ \isacommand{have}\isamarkupfalse%
\ {\isachardoublequoteopen}{\isasymdots}\ {\isasymsubseteq}\ setSubformulae\ {\isacharparenleft}F{\isadigit{1}}\ \isactrlbold {\isasymor}\ F{\isadigit{2}}{\isacharparenright}{\isachardoublequoteclose}\isanewline
\ \ \ \ \ \ \ \ \isacommand{by}\isamarkupfalse%
\ {\isacharparenleft}simp\ only{\isacharcolon}\ setSubformulae{\isacharunderscore}or\ Un{\isacharunderscore}upper{\isadigit{2}}{\isacharparenright}\isanewline
\ \ \ \ \ \ \isacommand{finally}\isamarkupfalse%
\ \isacommand{show}\isamarkupfalse%
\ {\isacharquery}thesis\isanewline
\ \ \ \ \ \ \ \ \isacommand{by}\isamarkupfalse%
\ this\isanewline
\ \ \ \ \isacommand{qed}\isamarkupfalse%
\isanewline
\ \ \isacommand{qed}\isamarkupfalse%
\isanewline
\isacommand{qed}\isamarkupfalse%
%
\endisatagproof
{\isafoldproof}%
%
\isadelimproof
\isanewline
%
\endisadelimproof
\isanewline
\isacommand{lemma}\isamarkupfalse%
\ subContsubformulae{\isacharunderscore}imp{\isacharcolon}\isanewline
\ \ \isakeyword{assumes}\ {\isachardoublequoteopen}G\ {\isasymin}\ setSubformulae\ F{\isadigit{1}}\ \isanewline
\ \ \ \ \ \ \ \ \ \ \ \ {\isasymLongrightarrow}\ setSubformulae\ G\ {\isasymsubseteq}\ setSubformulae\ F{\isadigit{1}}{\isachardoublequoteclose}\isanewline
\ \ \ \ \ \ \ \ \ \ {\isachardoublequoteopen}G\ {\isasymin}\ setSubformulae\ F{\isadigit{2}}\ \isanewline
\ \ \ \ \ \ \ \ \ \ \ \ {\isasymLongrightarrow}\ setSubformulae\ G\ {\isasymsubseteq}\ setSubformulae\ F{\isadigit{2}}{\isachardoublequoteclose}\isanewline
\ \ \ \ \ \ \ \ \ \ {\isachardoublequoteopen}G\ {\isasymin}\ setSubformulae\ {\isacharparenleft}F{\isadigit{1}}\ \isactrlbold {\isasymrightarrow}\ F{\isadigit{2}}{\isacharparenright}{\isachardoublequoteclose}\isanewline
\ \ \isakeyword{shows}\ \ \ {\isachardoublequoteopen}setSubformulae\ G\ {\isasymsubseteq}\ setSubformulae\ {\isacharparenleft}F{\isadigit{1}}\ \isactrlbold {\isasymrightarrow}\ F{\isadigit{2}}{\isacharparenright}{\isachardoublequoteclose}\isanewline
%
\isadelimproof
%
\endisadelimproof
%
\isatagproof
\isacommand{proof}\isamarkupfalse%
\ {\isacharminus}\isanewline
\ \ \isacommand{have}\isamarkupfalse%
\ {\isachardoublequoteopen}G\ {\isasymin}\ {\isacharbraceleft}F{\isadigit{1}}\ \isactrlbold {\isasymrightarrow}\ F{\isadigit{2}}{\isacharbraceright}\ {\isasymunion}\ {\isacharparenleft}setSubformulae\ F{\isadigit{1}}\ {\isasymunion}\ setSubformulae\ F{\isadigit{2}}{\isacharparenright}{\isachardoublequoteclose}\isanewline
\ \ \ \ \isacommand{using}\isamarkupfalse%
\ assms{\isacharparenleft}{\isadigit{3}}{\isacharparenright}\ \isanewline
\ \ \ \ \isacommand{by}\isamarkupfalse%
\ {\isacharparenleft}simp\ only{\isacharcolon}\ setSubformulae{\isacharunderscore}imp{\isacharparenright}\isanewline
\ \ \isacommand{then}\isamarkupfalse%
\ \isacommand{have}\isamarkupfalse%
\ {\isachardoublequoteopen}G\ {\isasymin}\ {\isacharbraceleft}F{\isadigit{1}}\ \isactrlbold {\isasymrightarrow}\ F{\isadigit{2}}{\isacharbraceright}\ {\isasymor}\ G\ {\isasymin}\ setSubformulae\ F{\isadigit{1}}\ {\isasymunion}\ setSubformulae\ F{\isadigit{2}}{\isachardoublequoteclose}\isanewline
\ \ \ \ \isacommand{by}\isamarkupfalse%
\ {\isacharparenleft}simp\ only{\isacharcolon}\ Un{\isacharunderscore}iff{\isacharparenright}\isanewline
\ \ \isacommand{then}\isamarkupfalse%
\ \isacommand{show}\isamarkupfalse%
\ {\isacharquery}thesis\isanewline
\ \ \isacommand{proof}\isamarkupfalse%
\ {\isacharparenleft}rule\ disjE{\isacharparenright}\isanewline
\ \ \ \ \isacommand{assume}\isamarkupfalse%
\ {\isachardoublequoteopen}G\ {\isasymin}\ {\isacharbraceleft}F{\isadigit{1}}\ \isactrlbold {\isasymrightarrow}\ F{\isadigit{2}}{\isacharbraceright}{\isachardoublequoteclose}\isanewline
\ \ \ \ \isacommand{then}\isamarkupfalse%
\ \isacommand{have}\isamarkupfalse%
\ {\isachardoublequoteopen}G\ {\isacharequal}\ F{\isadigit{1}}\ \isactrlbold {\isasymrightarrow}\ F{\isadigit{2}}{\isachardoublequoteclose}\isanewline
\ \ \ \ \ \ \isacommand{by}\isamarkupfalse%
\ {\isacharparenleft}simp\ only{\isacharcolon}\ singletonD{\isacharparenright}\isanewline
\ \ \ \ \isacommand{then}\isamarkupfalse%
\ \isacommand{show}\isamarkupfalse%
\ {\isacharquery}thesis\isanewline
\ \ \ \ \ \ \isacommand{by}\isamarkupfalse%
\ {\isacharparenleft}simp\ only{\isacharcolon}\ subset{\isacharunderscore}refl{\isacharparenright}\isanewline
\ \ \isacommand{next}\isamarkupfalse%
\isanewline
\ \ \ \ \isacommand{assume}\isamarkupfalse%
\ {\isachardoublequoteopen}G\ {\isasymin}\ setSubformulae\ F{\isadigit{1}}\ {\isasymunion}\ setSubformulae\ F{\isadigit{2}}{\isachardoublequoteclose}\isanewline
\ \ \ \ \isacommand{then}\isamarkupfalse%
\ \isacommand{have}\isamarkupfalse%
\ {\isachardoublequoteopen}G\ {\isasymin}\ setSubformulae\ F{\isadigit{1}}\ {\isasymor}\ G\ {\isasymin}\ setSubformulae\ F{\isadigit{2}}{\isachardoublequoteclose}\ \ \isanewline
\ \ \ \ \ \ \isacommand{by}\isamarkupfalse%
\ {\isacharparenleft}simp\ only{\isacharcolon}\ Un{\isacharunderscore}iff{\isacharparenright}\isanewline
\ \ \ \ \isacommand{then}\isamarkupfalse%
\ \isacommand{show}\isamarkupfalse%
\ {\isacharquery}thesis\isanewline
\ \ \ \ \isacommand{proof}\isamarkupfalse%
\ {\isacharparenleft}rule\ disjE{\isacharparenright}\isanewline
\ \ \ \ \ \ \isacommand{assume}\isamarkupfalse%
\ {\isachardoublequoteopen}G\ {\isasymin}\ setSubformulae\ F{\isadigit{1}}{\isachardoublequoteclose}\isanewline
\ \ \ \ \ \ \isacommand{then}\isamarkupfalse%
\ \isacommand{have}\isamarkupfalse%
\ {\isachardoublequoteopen}setSubformulae\ G\ {\isasymsubseteq}\ setSubformulae\ F{\isadigit{1}}{\isachardoublequoteclose}\isanewline
\ \ \ \ \ \ \ \ \isacommand{by}\isamarkupfalse%
\ {\isacharparenleft}simp\ only{\isacharcolon}\ assms{\isacharparenleft}{\isadigit{1}}{\isacharparenright}{\isacharparenright}\isanewline
\ \ \ \ \ \ \isacommand{also}\isamarkupfalse%
\ \isacommand{have}\isamarkupfalse%
\ {\isachardoublequoteopen}{\isasymdots}\ {\isasymsubseteq}\ setSubformulae\ F{\isadigit{1}}\ {\isasymunion}\ setSubformulae\ F{\isadigit{2}}{\isachardoublequoteclose}\isanewline
\ \ \ \ \ \ \ \ \isacommand{by}\isamarkupfalse%
\ {\isacharparenleft}simp\ only{\isacharcolon}\ Un{\isacharunderscore}upper{\isadigit{1}}{\isacharparenright}\isanewline
\ \ \ \ \ \ \isacommand{also}\isamarkupfalse%
\ \isacommand{have}\isamarkupfalse%
\ {\isachardoublequoteopen}{\isasymdots}\ {\isasymsubseteq}\ setSubformulae\ {\isacharparenleft}F{\isadigit{1}}\ \isactrlbold {\isasymrightarrow}\ F{\isadigit{2}}{\isacharparenright}{\isachardoublequoteclose}\isanewline
\ \ \ \ \ \ \ \ \isacommand{by}\isamarkupfalse%
\ {\isacharparenleft}simp\ only{\isacharcolon}\ setSubformulae{\isacharunderscore}imp\ Un{\isacharunderscore}upper{\isadigit{2}}{\isacharparenright}\isanewline
\ \ \ \ \ \ \isacommand{finally}\isamarkupfalse%
\ \isacommand{show}\isamarkupfalse%
\ {\isacharquery}thesis\isanewline
\ \ \ \ \ \ \ \ \isacommand{by}\isamarkupfalse%
\ this\isanewline
\ \ \ \ \isacommand{next}\isamarkupfalse%
\isanewline
\ \ \ \ \ \ \isacommand{assume}\isamarkupfalse%
\ {\isachardoublequoteopen}G\ {\isasymin}\ setSubformulae\ F{\isadigit{2}}{\isachardoublequoteclose}\isanewline
\ \ \ \ \ \ \isacommand{then}\isamarkupfalse%
\ \isacommand{have}\isamarkupfalse%
\ {\isachardoublequoteopen}setSubformulae\ G\ {\isasymsubseteq}\ setSubformulae\ F{\isadigit{2}}{\isachardoublequoteclose}\isanewline
\ \ \ \ \ \ \ \ \isacommand{by}\isamarkupfalse%
\ {\isacharparenleft}rule\ assms{\isacharparenleft}{\isadigit{2}}{\isacharparenright}{\isacharparenright}\isanewline
\ \ \ \ \ \ \isacommand{also}\isamarkupfalse%
\ \isacommand{have}\isamarkupfalse%
\ {\isachardoublequoteopen}{\isasymdots}\ {\isasymsubseteq}\ setSubformulae\ F{\isadigit{1}}\ {\isasymunion}\ setSubformulae\ F{\isadigit{2}}{\isachardoublequoteclose}\isanewline
\ \ \ \ \ \ \ \ \isacommand{by}\isamarkupfalse%
\ {\isacharparenleft}simp\ only{\isacharcolon}\ Un{\isacharunderscore}upper{\isadigit{2}}{\isacharparenright}\isanewline
\ \ \ \ \ \ \isacommand{also}\isamarkupfalse%
\ \isacommand{have}\isamarkupfalse%
\ {\isachardoublequoteopen}{\isasymdots}\ {\isasymsubseteq}\ setSubformulae\ {\isacharparenleft}F{\isadigit{1}}\ \isactrlbold {\isasymrightarrow}\ F{\isadigit{2}}{\isacharparenright}{\isachardoublequoteclose}\isanewline
\ \ \ \ \ \ \ \ \isacommand{by}\isamarkupfalse%
\ {\isacharparenleft}simp\ only{\isacharcolon}\ setSubformulae{\isacharunderscore}imp\ Un{\isacharunderscore}upper{\isadigit{2}}{\isacharparenright}\isanewline
\ \ \ \ \ \ \isacommand{finally}\isamarkupfalse%
\ \isacommand{show}\isamarkupfalse%
\ {\isacharquery}thesis\isanewline
\ \ \ \ \ \ \ \ \isacommand{by}\isamarkupfalse%
\ this\isanewline
\ \ \ \ \isacommand{qed}\isamarkupfalse%
\isanewline
\ \ \isacommand{qed}\isamarkupfalse%
\isanewline
\isacommand{qed}\isamarkupfalse%
%
\endisatagproof
{\isafoldproof}%
%
\isadelimproof
\isanewline
%
\endisadelimproof
\isanewline
\isacommand{lemma}\isamarkupfalse%
\isanewline
\ \ {\isachardoublequoteopen}G\ {\isasymin}\ setSubformulae\ F\ {\isasymLongrightarrow}\ setSubformulae\ G\ {\isasymsubseteq}\ setSubformulae\ F{\isachardoublequoteclose}\isanewline
%
\isadelimproof
%
\endisadelimproof
%
\isatagproof
\isacommand{proof}\isamarkupfalse%
\ {\isacharparenleft}induction\ F{\isacharparenright}\isanewline
\ \ \isacommand{case}\isamarkupfalse%
\ {\isacharparenleft}Atom\ x{\isacharparenright}\isanewline
\ \ \isacommand{then}\isamarkupfalse%
\ \isacommand{show}\isamarkupfalse%
\ {\isacharquery}case\ \isacommand{by}\isamarkupfalse%
\ {\isacharparenleft}rule\ subContsubformulae{\isacharunderscore}atom{\isacharparenright}\isanewline
\isacommand{next}\isamarkupfalse%
\isanewline
\ \ \isacommand{case}\isamarkupfalse%
\ Bot\isanewline
\ \ \isacommand{then}\isamarkupfalse%
\ \isacommand{show}\isamarkupfalse%
\ {\isacharquery}case\ \isacommand{by}\isamarkupfalse%
\ {\isacharparenleft}rule\ subContsubformulae{\isacharunderscore}bot{\isacharparenright}\isanewline
\isacommand{next}\isamarkupfalse%
\isanewline
\ \ \isacommand{case}\isamarkupfalse%
\ {\isacharparenleft}Not\ F{\isacharparenright}\isanewline
\ \ \isacommand{then}\isamarkupfalse%
\ \isacommand{show}\isamarkupfalse%
\ {\isacharquery}case\ \isacommand{by}\isamarkupfalse%
\ {\isacharparenleft}rule\ subContsubformulae{\isacharunderscore}not{\isacharparenright}\isanewline
\isacommand{next}\isamarkupfalse%
\isanewline
\ \ \isacommand{case}\isamarkupfalse%
\ {\isacharparenleft}And\ F{\isadigit{1}}\ F{\isadigit{2}}{\isacharparenright}\isanewline
\ \ \isacommand{then}\isamarkupfalse%
\ \isacommand{show}\isamarkupfalse%
\ {\isacharquery}case\ \isacommand{by}\isamarkupfalse%
\ {\isacharparenleft}rule\ subContsubformulae{\isacharunderscore}and{\isacharparenright}\isanewline
\isacommand{next}\isamarkupfalse%
\isanewline
\ \ \isacommand{case}\isamarkupfalse%
\ {\isacharparenleft}Or\ F{\isadigit{1}}\ F{\isadigit{2}}{\isacharparenright}\isanewline
\ \ \isacommand{then}\isamarkupfalse%
\ \isacommand{show}\isamarkupfalse%
\ {\isacharquery}case\ \isacommand{by}\isamarkupfalse%
\ {\isacharparenleft}rule\ subContsubformulae{\isacharunderscore}or{\isacharparenright}\isanewline
\isacommand{next}\isamarkupfalse%
\isanewline
\ \ \isacommand{case}\isamarkupfalse%
\ {\isacharparenleft}Imp\ F{\isadigit{1}}\ F{\isadigit{2}}{\isacharparenright}\isanewline
\ \ \isacommand{then}\isamarkupfalse%
\ \isacommand{show}\isamarkupfalse%
\ {\isacharquery}case\ \isacommand{by}\isamarkupfalse%
\ {\isacharparenleft}rule\ subContsubformulae{\isacharunderscore}imp{\isacharparenright}\isanewline
\isacommand{qed}\isamarkupfalse%
%
\endisatagproof
{\isafoldproof}%
%
\isadelimproof
%
\endisadelimproof
%
\begin{isamarkuptext}%
Finalmente, su demostración automática se muestra a continuación.%
\end{isamarkuptext}\isamarkuptrue%
\isacommand{lemma}\isamarkupfalse%
\ subContsubformulae{\isacharcolon}\isanewline
\ \ {\isachardoublequoteopen}G\ {\isasymin}\ setSubformulae\ F\ {\isasymLongrightarrow}\ setSubformulae\ G\ {\isasymsubseteq}\ setSubformulae\ F{\isachardoublequoteclose}\isanewline
%
\isadelimproof
\ \ %
\endisadelimproof
%
\isatagproof
\isacommand{by}\isamarkupfalse%
\ {\isacharparenleft}induction\ F{\isacharparenright}\ auto%
\endisatagproof
{\isafoldproof}%
%
\isadelimproof
%
\endisadelimproof
%
\begin{isamarkuptext}%
El siguiente lema nos da la noción de transitividad de contención 
  en cadena de las subfórmulas, de modo que la subfórmula de una 
  subfórmula es del mismo modo subfórmula de la mayor.

  \begin{lema}
    Sea \isa{H} una subfórmula de \isa{G} que es a su vez subfórmula de \isa{F}, 
    entonces \isa{H} es subfórmula de \isa{F}.
  \end{lema}

  \begin{demostracion}
  La prueba está basada en el lema anterior. Hemos demostrado que si 
  \isa{H} es subfórmula de \isa{G}, entonces el conjunto de subfórmulas de \isa{H} 
  está contenido en el conjunto de subfórmulas de \isa{G}. Del mismo modo, 
  como \isa{G} es subfórmula de \isa{F}, su conjunto de subfórmulas está 
  contenido en el conjunto de subfórmulas de \isa{F}. Por la
  transitividad de la contención, tenemos que el conjunto de subfórmulas
  de \isa{H} está contenido en el de \isa{F}. Por otro lema anterior, 
  como \isa{H} es subfórmula de ella misma, es decir, pertenece a su 
  conjunto de subfórmulas, por la contención anterior se verifica que
  pertenece al conjunto de subfórmulas de \isa{F} como queríamos demostrar. 
  \end{demostracion}

  Veamos su formalización y prueba estructurada en Isabelle.%
\end{isamarkuptext}\isamarkuptrue%
\isacommand{lemma}\isamarkupfalse%
\isanewline
\ \ \isakeyword{assumes}\ {\isachardoublequoteopen}G\ {\isasymin}\ setSubformulae\ F{\isachardoublequoteclose}\ \isanewline
\ \ \ \ \ \ \ \ \ \ {\isachardoublequoteopen}H\ {\isasymin}\ setSubformulae\ G{\isachardoublequoteclose}\isanewline
\ \ \isakeyword{shows}\ \ \ {\isachardoublequoteopen}H\ {\isasymin}\ setSubformulae\ F{\isachardoublequoteclose}\isanewline
%
\isadelimproof
%
\endisadelimproof
%
\isatagproof
\isacommand{proof}\isamarkupfalse%
\ {\isacharminus}\isanewline
\ \ \isacommand{have}\isamarkupfalse%
\ {\isadigit{1}}{\isacharcolon}\ {\isachardoublequoteopen}setSubformulae\ G\ {\isasymsubseteq}\ setSubformulae\ F{\isachardoublequoteclose}\ \isanewline
\ \ \ \ \isacommand{using}\isamarkupfalse%
\ assms{\isacharparenleft}{\isadigit{1}}{\isacharparenright}\ \isanewline
\ \ \ \ \isacommand{by}\isamarkupfalse%
\ {\isacharparenleft}rule\ subContsubformulae{\isacharparenright}\isanewline
\ \ \isacommand{have}\isamarkupfalse%
\ {\isachardoublequoteopen}setSubformulae\ H\ {\isasymsubseteq}\ setSubformulae\ G{\isachardoublequoteclose}\ \isanewline
\ \ \ \ \isacommand{using}\isamarkupfalse%
\ assms{\isacharparenleft}{\isadigit{2}}{\isacharparenright}\ \isanewline
\ \ \ \ \isacommand{by}\isamarkupfalse%
\ {\isacharparenleft}rule\ subContsubformulae{\isacharparenright}\isanewline
\ \ \isacommand{then}\isamarkupfalse%
\ \isacommand{have}\isamarkupfalse%
\ {\isadigit{2}}{\isacharcolon}\ {\isachardoublequoteopen}setSubformulae\ H\ {\isasymsubseteq}\ setSubformulae\ F{\isachardoublequoteclose}\ \isanewline
\ \ \ \ \isacommand{using}\isamarkupfalse%
\ {\isadigit{1}}\ \isanewline
\ \ \ \ \isacommand{by}\isamarkupfalse%
\ {\isacharparenleft}rule\ subset{\isacharunderscore}trans{\isacharparenright}\isanewline
\ \ \isacommand{have}\isamarkupfalse%
\ {\isachardoublequoteopen}H\ {\isasymin}\ setSubformulae\ H{\isachardoublequoteclose}\ \isanewline
\ \ \ \ \isacommand{by}\isamarkupfalse%
\ {\isacharparenleft}simp\ only{\isacharcolon}\ subformulae{\isacharunderscore}self{\isacharparenright}\isanewline
\ \ \isacommand{then}\isamarkupfalse%
\ \isacommand{show}\isamarkupfalse%
\ {\isachardoublequoteopen}H\ {\isasymin}\ setSubformulae\ F{\isachardoublequoteclose}\ \isanewline
\ \ \ \ \isacommand{using}\isamarkupfalse%
\ {\isadigit{2}}\ \isanewline
\ \ \ \ \isacommand{by}\isamarkupfalse%
\ {\isacharparenleft}rule\ rev{\isacharunderscore}subsetD{\isacharparenright}\isanewline
\isacommand{qed}\isamarkupfalse%
%
\endisatagproof
{\isafoldproof}%
%
\isadelimproof
%
\endisadelimproof
%
\begin{isamarkuptext}%
A continuación su demostración automática.%
\end{isamarkuptext}\isamarkuptrue%
\isacommand{lemma}\isamarkupfalse%
\ subsubformulae{\isacharcolon}\ \isanewline
\ \ {\isachardoublequoteopen}G\ {\isasymin}\ setSubformulae\ F\ \isanewline
\ \ \ {\isasymLongrightarrow}\ H\ {\isasymin}\ setSubformulae\ G\ \isanewline
\ \ \ {\isasymLongrightarrow}\ H\ {\isasymin}\ setSubformulae\ F{\isachardoublequoteclose}\isanewline
%
\isadelimproof
\ \ %
\endisadelimproof
%
\isatagproof
\isacommand{by}\isamarkupfalse%
\ {\isacharparenleft}drule\ subContsubformulae{\isacharcomma}\ erule\ subsetD{\isacharparenright}%
\endisatagproof
{\isafoldproof}%
%
\isadelimproof
%
\endisadelimproof
%
\begin{isamarkuptext}%
Presentemos ahora otro resultado que relaciona las conectivas
  con los conjuntos de subfórmulas.%
\end{isamarkuptext}\isamarkuptrue%
%
\begin{isamarkuptext}%
Para la demostración en Isabelle, probaremos cada caso de forma
 independiente.%
\end{isamarkuptext}\isamarkuptrue%
\isacommand{lemma}\isamarkupfalse%
\ subformulas{\isacharunderscore}in{\isacharunderscore}subformulas{\isacharunderscore}not{\isacharcolon}\isanewline
\ \ \isakeyword{assumes}\ {\isachardoublequoteopen}\isactrlbold {\isasymnot}\ G\ {\isasymin}\ setSubformulae\ F{\isachardoublequoteclose}\isanewline
\ \ \isakeyword{shows}\ {\isachardoublequoteopen}G\ {\isasymin}\ setSubformulae\ F{\isachardoublequoteclose}\isanewline
%
\isadelimproof
%
\endisadelimproof
%
\isatagproof
\isacommand{proof}\isamarkupfalse%
\ {\isacharminus}\isanewline
\ \ \isacommand{have}\isamarkupfalse%
\ {\isachardoublequoteopen}G\ {\isasymin}\ setSubformulae\ G{\isachardoublequoteclose}\isanewline
\ \ \ \ \isacommand{by}\isamarkupfalse%
\ {\isacharparenleft}simp\ only{\isacharcolon}\ subformulae{\isacharunderscore}self{\isacharparenright}\isanewline
\ \ \isacommand{then}\isamarkupfalse%
\ \isacommand{have}\isamarkupfalse%
\ {\isachardoublequoteopen}G\ {\isasymin}\ {\isacharbraceleft}\isactrlbold {\isasymnot}\ G{\isacharbraceright}\ {\isasymunion}\ setSubformulae\ G{\isachardoublequoteclose}\isanewline
\ \ \ \ \isacommand{by}\isamarkupfalse%
\ {\isacharparenleft}simp\ only{\isacharcolon}\ UnI{\isadigit{2}}{\isacharparenright}\isanewline
\ \ \isacommand{then}\isamarkupfalse%
\ \isacommand{have}\isamarkupfalse%
\ {\isadigit{1}}{\isacharcolon}{\isachardoublequoteopen}G\ {\isasymin}\ setSubformulae\ {\isacharparenleft}\isactrlbold {\isasymnot}\ G{\isacharparenright}{\isachardoublequoteclose}\ \isanewline
\ \ \ \ \isacommand{by}\isamarkupfalse%
\ {\isacharparenleft}simp\ only{\isacharcolon}\ setSubformulae{\isacharunderscore}not{\isacharparenright}\isanewline
\ \ \isacommand{show}\isamarkupfalse%
\ {\isachardoublequoteopen}G\ {\isasymin}\ setSubformulae\ F{\isachardoublequoteclose}\ \isacommand{using}\isamarkupfalse%
\ assms\ {\isadigit{1}}\ \isanewline
\ \ \ \ \isacommand{by}\isamarkupfalse%
\ {\isacharparenleft}rule\ subsubformulae{\isacharparenright}\isanewline
\isacommand{qed}\isamarkupfalse%
%
\endisatagproof
{\isafoldproof}%
%
\isadelimproof
\isanewline
%
\endisadelimproof
\isanewline
\isacommand{lemma}\isamarkupfalse%
\ subformulas{\isacharunderscore}in{\isacharunderscore}subformulas{\isacharunderscore}and{\isacharcolon}\isanewline
\ \ \isakeyword{assumes}\ {\isachardoublequoteopen}G\ \isactrlbold {\isasymand}\ H\ {\isasymin}\ setSubformulae\ F{\isachardoublequoteclose}\ \isanewline
\ \ \isakeyword{shows}\ {\isachardoublequoteopen}G\ {\isasymin}\ setSubformulae\ F\ {\isasymand}\ H\ {\isasymin}\ setSubformulae\ F{\isachardoublequoteclose}\isanewline
%
\isadelimproof
%
\endisadelimproof
%
\isatagproof
\isacommand{proof}\isamarkupfalse%
\ {\isacharparenleft}rule\ conjI{\isacharparenright}\isanewline
\ \ \isacommand{have}\isamarkupfalse%
\ {\isachardoublequoteopen}G\ {\isasymin}\ setSubformulae\ {\isacharparenleft}G\ \isactrlbold {\isasymand}\ H{\isacharparenright}{\isachardoublequoteclose}\ \isanewline
\ \ \ \ \isacommand{by}\isamarkupfalse%
\ {\isacharparenleft}simp\ only{\isacharcolon}\ subformulae{\isacharunderscore}self\ UnI{\isadigit{2}}\ UnI{\isadigit{1}}\ setSubformulae{\isacharunderscore}and{\isacharparenright}\isanewline
\ \ \isacommand{with}\isamarkupfalse%
\ assms\ \isacommand{show}\isamarkupfalse%
\ {\isachardoublequoteopen}G\ {\isasymin}\ setSubformulae\ F{\isachardoublequoteclose}\ \isanewline
\ \ \ \ \isacommand{by}\isamarkupfalse%
\ {\isacharparenleft}rule\ subsubformulae{\isacharparenright}\isanewline
\isacommand{next}\isamarkupfalse%
\isanewline
\ \ \isacommand{have}\isamarkupfalse%
\ {\isachardoublequoteopen}H\ {\isasymin}\ setSubformulae\ {\isacharparenleft}G\ \isactrlbold {\isasymand}\ H{\isacharparenright}{\isachardoublequoteclose}\ \ \isanewline
\ \ \ \ \isacommand{by}\isamarkupfalse%
\ {\isacharparenleft}simp\ only{\isacharcolon}\ subformulae{\isacharunderscore}self\ UnI{\isadigit{2}}\ UnI{\isadigit{1}}\ setSubformulae{\isacharunderscore}and{\isacharparenright}\isanewline
\ \ \isacommand{with}\isamarkupfalse%
\ assms\ \isacommand{show}\isamarkupfalse%
\ {\isachardoublequoteopen}H\ {\isasymin}\ setSubformulae\ F{\isachardoublequoteclose}\ \isanewline
\ \ \ \ \isacommand{by}\isamarkupfalse%
\ {\isacharparenleft}rule\ subsubformulae{\isacharparenright}\isanewline
\isacommand{qed}\isamarkupfalse%
%
\endisatagproof
{\isafoldproof}%
%
\isadelimproof
\isanewline
%
\endisadelimproof
\isanewline
\isacommand{lemma}\isamarkupfalse%
\ subformulas{\isacharunderscore}in{\isacharunderscore}subformulas{\isacharunderscore}or{\isacharcolon}\isanewline
\ \ \isakeyword{assumes}\ {\isachardoublequoteopen}G\ \isactrlbold {\isasymor}\ H\ {\isasymin}\ setSubformulae\ F{\isachardoublequoteclose}\ \isanewline
\ \ \isakeyword{shows}\ {\isachardoublequoteopen}G\ {\isasymin}\ setSubformulae\ F\ {\isasymand}\ H\ {\isasymin}\ setSubformulae\ F{\isachardoublequoteclose}\isanewline
%
\isadelimproof
%
\endisadelimproof
%
\isatagproof
\isacommand{proof}\isamarkupfalse%
\ {\isacharparenleft}rule\ conjI{\isacharparenright}\isanewline
\ \ \isacommand{have}\isamarkupfalse%
\ {\isachardoublequoteopen}G\ {\isasymin}\ setSubformulae\ {\isacharparenleft}G\ \isactrlbold {\isasymor}\ H{\isacharparenright}{\isachardoublequoteclose}\ \isanewline
\ \ \ \ \isacommand{by}\isamarkupfalse%
\ {\isacharparenleft}simp\ only{\isacharcolon}\ subformulae{\isacharunderscore}self\ UnI{\isadigit{2}}\ UnI{\isadigit{1}}\ setSubformulae{\isacharunderscore}or{\isacharparenright}\isanewline
\ \ \isacommand{with}\isamarkupfalse%
\ assms\ \isacommand{show}\isamarkupfalse%
\ {\isachardoublequoteopen}G\ {\isasymin}\ setSubformulae\ F{\isachardoublequoteclose}\ \isanewline
\ \ \ \ \isacommand{by}\isamarkupfalse%
\ {\isacharparenleft}rule\ subsubformulae{\isacharparenright}\isanewline
\isacommand{next}\isamarkupfalse%
\isanewline
\ \ \isacommand{have}\isamarkupfalse%
\ {\isachardoublequoteopen}H\ {\isasymin}\ setSubformulae\ {\isacharparenleft}G\ \isactrlbold {\isasymor}\ H{\isacharparenright}{\isachardoublequoteclose}\ \ \isanewline
\ \ \ \ \isacommand{by}\isamarkupfalse%
\ {\isacharparenleft}simp\ only{\isacharcolon}\ subformulae{\isacharunderscore}self\ UnI{\isadigit{2}}\ UnI{\isadigit{1}}\ setSubformulae{\isacharunderscore}or{\isacharparenright}\isanewline
\ \ \isacommand{with}\isamarkupfalse%
\ assms\ \isacommand{show}\isamarkupfalse%
\ {\isachardoublequoteopen}H\ {\isasymin}\ setSubformulae\ F{\isachardoublequoteclose}\ \isanewline
\ \ \ \ \isacommand{by}\isamarkupfalse%
\ {\isacharparenleft}rule\ subsubformulae{\isacharparenright}\isanewline
\isacommand{qed}\isamarkupfalse%
%
\endisatagproof
{\isafoldproof}%
%
\isadelimproof
\isanewline
%
\endisadelimproof
\isanewline
\isacommand{lemma}\isamarkupfalse%
\ subformulas{\isacharunderscore}in{\isacharunderscore}subformulas{\isacharunderscore}imp{\isacharcolon}\isanewline
\ \ \isakeyword{assumes}\ {\isachardoublequoteopen}G\ \isactrlbold {\isasymrightarrow}\ H\ {\isasymin}\ setSubformulae\ F{\isachardoublequoteclose}\ \isanewline
\ \ \isakeyword{shows}\ {\isachardoublequoteopen}G\ {\isasymin}\ setSubformulae\ F\ {\isasymand}\ H\ {\isasymin}\ setSubformulae\ F{\isachardoublequoteclose}\isanewline
%
\isadelimproof
%
\endisadelimproof
%
\isatagproof
\isacommand{proof}\isamarkupfalse%
\ {\isacharparenleft}rule\ conjI{\isacharparenright}\isanewline
\ \ \isacommand{have}\isamarkupfalse%
\ {\isachardoublequoteopen}G\ {\isasymin}\ setSubformulae\ {\isacharparenleft}G\ \isactrlbold {\isasymrightarrow}\ H{\isacharparenright}{\isachardoublequoteclose}\ \isanewline
\ \ \ \ \isacommand{by}\isamarkupfalse%
\ {\isacharparenleft}simp\ only{\isacharcolon}\ subformulae{\isacharunderscore}self\ UnI{\isadigit{2}}\ UnI{\isadigit{1}}\ setSubformulae{\isacharunderscore}imp{\isacharparenright}\isanewline
\ \ \isacommand{with}\isamarkupfalse%
\ assms\ \isacommand{show}\isamarkupfalse%
\ {\isachardoublequoteopen}G\ {\isasymin}\ setSubformulae\ F{\isachardoublequoteclose}\ \isanewline
\ \ \ \ \isacommand{by}\isamarkupfalse%
\ {\isacharparenleft}rule\ subsubformulae{\isacharparenright}\isanewline
\isacommand{next}\isamarkupfalse%
\isanewline
\ \ \isacommand{have}\isamarkupfalse%
\ {\isachardoublequoteopen}H\ {\isasymin}\ setSubformulae\ {\isacharparenleft}G\ \isactrlbold {\isasymrightarrow}\ H{\isacharparenright}{\isachardoublequoteclose}\ \ \isanewline
\ \ \ \ \isacommand{by}\isamarkupfalse%
\ {\isacharparenleft}simp\ only{\isacharcolon}\ subformulae{\isacharunderscore}self\ UnI{\isadigit{2}}\ UnI{\isadigit{1}}\ setSubformulae{\isacharunderscore}imp{\isacharparenright}\isanewline
\ \ \isacommand{with}\isamarkupfalse%
\ assms\ \isacommand{show}\isamarkupfalse%
\ {\isachardoublequoteopen}H\ {\isasymin}\ setSubformulae\ F{\isachardoublequoteclose}\ \isanewline
\ \ \ \ \isacommand{by}\isamarkupfalse%
\ {\isacharparenleft}rule\ subsubformulae{\isacharparenright}\isanewline
\isacommand{qed}\isamarkupfalse%
%
\endisatagproof
{\isafoldproof}%
%
\isadelimproof
\isanewline
%
\endisadelimproof
\isanewline
\isacommand{lemmas}\isamarkupfalse%
\ subformulas{\isacharunderscore}in{\isacharunderscore}subformulas\ {\isacharequal}\isanewline
\ \ subformulas{\isacharunderscore}in{\isacharunderscore}subformulas{\isacharunderscore}and\isanewline
\ \ subformulas{\isacharunderscore}in{\isacharunderscore}subformulas{\isacharunderscore}or\isanewline
\ \ subformulas{\isacharunderscore}in{\isacharunderscore}subformulas{\isacharunderscore}imp\isanewline
\ \ subformulas{\isacharunderscore}in{\isacharunderscore}subformulas{\isacharunderscore}not%
\isadelimdocument
%
\endisadelimdocument
%
\isatagdocument
%
\isamarkupsection{Conectivas generalizadas%
}
\isamarkuptrue%
%
\endisatagdocument
{\isafolddocument}%
%
\isadelimdocument
%
\endisadelimdocument
%
\begin{isamarkuptext}%
En esta sección definiremos nuevas conectivas y fórmulas a partir 
  de las ya definidas en el apartado anterior, junto con varios 
  resultados sobre las mismas. Veamos el primero.

  \begin{definicion}
    Se define la fórmula \isa{{\isasymtop}} como la implicación \isa{{\isasymbottom}\ {\isasymlongrightarrow}\ {\isasymbottom}}.
  \end{definicion}

  Se formaliza del siguiente modo.%
\end{isamarkuptext}\isamarkuptrue%
\isacommand{definition}\isamarkupfalse%
\ Top\ {\isacharparenleft}{\isachardoublequoteopen}{\isasymtop}{\isachardoublequoteclose}{\isacharparenright}\ \isakeyword{where}\isanewline
\ \ {\isachardoublequoteopen}{\isasymtop}\ {\isasymequiv}\ {\isasymbottom}\ \isactrlbold {\isasymrightarrow}\ {\isasymbottom}{\isachardoublequoteclose}%
\begin{isamarkuptext}%
Como podemos observar, se define mediante una relación de 
  equivalencia con otra fórmula ya conocida. El uso de dicha 
  equivalencia justifica el tipo \isa{definition} empleado en este 
  caso. 

  Por la propia definición, es claro que \isa{{\isasymtop}} no contiene ninguna
  variable proposicional, como se verifica a continuación en Isabelle.%
\end{isamarkuptext}\isamarkuptrue%
\isacommand{lemma}\isamarkupfalse%
\ {\isachardoublequoteopen}atoms\ {\isasymtop}\ {\isacharequal}\ {\isasymemptyset}{\isachardoublequoteclose}\isanewline
%
\isadelimproof
\ \ \ %
\endisadelimproof
%
\isatagproof
\isacommand{by}\isamarkupfalse%
\ {\isacharparenleft}simp\ only{\isacharcolon}\ Top{\isacharunderscore}def\ formula{\isachardot}set\ Un{\isacharunderscore}absorb{\isacharparenright}%
\endisatagproof
{\isafoldproof}%
%
\isadelimproof
%
\endisadelimproof
%
\begin{isamarkuptext}%
A continuación vamos a definir dos conectivas que generalizan la 
  conjunción y la disyunción para una lista finita de fórmulas. 

  En Isabelle está predefinido el tipo listas de la siguiente manera:

  \begin{definicion}
    Las listas de un tipo de elemento cualquiera se definen
    recursivamente como sigue.
    \begin{itemize}
      \item[] La lista vacía es una lista.
      \item[] Sea \isa{x} un elemento, y \isa{xs} una lista de elementos de su
      mismo tipo. Entonces, \isa{x{\isacharhash}xs} es una lista.
    \end{itemize}
  \end{definicion}

  La conjunción y disyunción generalizadas se definen sobre listas de
  fórmulas de manera recursiva:

  \begin{definicion}
  La conjunción generalizada de una lista de fórmulas se define 
  recursivamente como:
    \begin{itemize}
      \item La conjunción generalizada de la lista vacía es \isa{{\isasymnot}{\isasymbottom}}.
      \item Sea \isa{F} una fórmula y \isa{Fs} una lista de fórmulas. Entonces,
  la conjunción generalizada de \isa{F{\isacharhash}Fs} es la conjunción de \isa{F} con la 
  conjunción generalizada de \isa{Fs}.
    \end{itemize}
  \end{definicion}

  \begin{definicion}
  La disyunción generalizada de una lista de fórmulas se define 
  recursivamente como:
    \begin{itemize}
      \item La disyunción generalizada de la lista vacía es \isa{{\isasymbottom}}.
      \item Sea \isa{F} una fórmula y \isa{Fs} una lista de fórmulas. Entonces,
  la disyunción generalizada de \isa{F{\isacharhash}Fs} es la disyunción de \isa{F} con la 
  disyunción generalizada de \isa{Fs}.
    \end{itemize}
  \end{definicion}

  Notemos que al referirnos simplemente a disyunción o conjunción en las
  siguientes definiciones nos referiremos a la de dos elementos.

  Su formalización en Isabelle es la siguiente:%
\end{isamarkuptext}\isamarkuptrue%
\isacommand{primrec}\isamarkupfalse%
\ BigAnd\ {\isacharcolon}{\isacharcolon}\ {\isachardoublequoteopen}{\isacharprime}a\ formula\ list\ {\isasymRightarrow}\ {\isacharprime}a\ formula{\isachardoublequoteclose}\ {\isacharparenleft}{\isachardoublequoteopen}\isactrlbold {\isasymAnd}{\isacharunderscore}{\isachardoublequoteclose}{\isacharparenright}\ \isakeyword{where}\isanewline
\ \ {\isachardoublequoteopen}\isactrlbold {\isasymAnd}{\isacharbrackleft}{\isacharbrackright}\ {\isacharequal}\ {\isacharparenleft}\isactrlbold {\isasymnot}{\isasymbottom}{\isacharparenright}{\isachardoublequoteclose}\ \isanewline
{\isacharbar}\ {\isachardoublequoteopen}\isactrlbold {\isasymAnd}{\isacharparenleft}F{\isacharhash}Fs{\isacharparenright}\ {\isacharequal}\ F\ \isactrlbold {\isasymand}\ \isactrlbold {\isasymAnd}Fs{\isachardoublequoteclose}\isanewline
\isanewline
\isacommand{primrec}\isamarkupfalse%
\ BigOr\ {\isacharcolon}{\isacharcolon}\ {\isachardoublequoteopen}{\isacharprime}a\ formula\ list\ {\isasymRightarrow}\ {\isacharprime}a\ formula{\isachardoublequoteclose}\ {\isacharparenleft}{\isachardoublequoteopen}\isactrlbold {\isasymOr}{\isacharunderscore}{\isachardoublequoteclose}{\isacharparenright}\ \isakeyword{where}\isanewline
\ \ {\isachardoublequoteopen}\isactrlbold {\isasymOr}{\isacharbrackleft}{\isacharbrackright}\ {\isacharequal}\ {\isasymbottom}{\isachardoublequoteclose}\ \isanewline
{\isacharbar}\ {\isachardoublequoteopen}\isactrlbold {\isasymOr}{\isacharparenleft}F{\isacharhash}Fs{\isacharparenright}\ {\isacharequal}\ F\ \isactrlbold {\isasymor}\ \isactrlbold {\isasymOr}Fs{\isachardoublequoteclose}%
\begin{isamarkuptext}%
Ambas nuevas conectivas se definen con el tipo funciones 
  primitivas recursivas. Estas se basan en los dos casos descritos
  anteriormente según la definición recursiva de listas que se genera en
  Isabelle: la lista vacía representada como \isa{{\isacharbrackleft}{\isacharbrackright}} y la lista
  construida añadiendo una fórmula a una lista de fórmulas. 
  Además, se observa en cada definición el nuevo símbolo de 
  notación que aparece entre paréntesis.

  Por otro lado, como es habitual, de acuerdo a la definición recursiva
  de listas, Isabelle genera automáticamente un esquema inductivo que 
  emplearemos más adelante.

  Vamos a mostrar una propiedad sobre la conjunción plural.

  \begin{lema}
  El conjunto de átomos de la conjunción generalizada de una lista 
  de fórmulas es la unión de los conjuntos de átomos de cada fórmula de 
  la lista.
  \end{lema}

  \begin{demostracion}
  La prueba se hace por inducción sobre listas, en particular,
  listas de fórmulas. Para ello, demostremos el resultado en los casos
  siguientes. 

  En primer lugar lo probaremos para la lista vacía de fórmulas. Es
  claro por definición que la conjunción generalizada de la lista vacía
  es \isa{{\isasymnot}\ {\isasymbottom}}. De este modo, su conjunto de átomos coincide con los de 
  \isa{{\isasymbottom}}, luego es el vacío. Por tanto, queda demostrado el resultado, 
  pues el vacío es igual a la unión del conjunto de átomos de cada 
  elemento de la lista vacía de fórmulas. 

  Supongamos ahora una lista de fórmulas \isa{Fs} verificando el enunciado.
  Sea la fórmula \isa{F}, vamos a probar que \isa{F{\isacharhash}Fs} cumple la propiedad.
  Por definición de la nueva conectiva, el conjunto de átomos de 
  la conjunción generalizada de \isa{F{\isacharhash}Fs} es igual al conjunto de átomos de 
  la conjunción de \isa{F} con la conjunción generalizada de \isa{Fs}. De este
  modo, por propiedades del conjunto de átomos de la conjunción, tenemos
  que dicho conjunto es la unión del conjunto de átomos de \isa{F} y el
  conjunto de átomos de la conjunción generalizada de \isa{Fs}. Aplicando 
  ahora la hipótesis de inducción sobre \isa{Fs}, tenemos que lo anterior es 
  igual a la unión del conjunto de átomos de \isa{F} con la unión
 (generalizada) de los conjuntos de átomos de cada fórmula de \isa{Fs}. 
  Luego, por propiedades de la unión, es equivalente a la unión de los 
  conjuntos de átomos de cada elemento de \isa{F{\isacharhash}Fs} como queríamos 
  demostrar.
  \end{demostracion}

  En Isabelle se formaliza como sigue.%
\end{isamarkuptext}\isamarkuptrue%
\isacommand{lemma}\isamarkupfalse%
\ atoms{\isacharunderscore}BigAnd{\isacharcolon}\ \isanewline
\ \ {\isachardoublequoteopen}atoms\ {\isacharparenleft}\isactrlbold {\isasymAnd}Fs{\isacharparenright}\ {\isacharequal}\ {\isasymUnion}{\isacharparenleft}atoms\ {\isacharbackquote}\ set\ Fs{\isacharparenright}{\isachardoublequoteclose}\isanewline
%
\isadelimproof
\ \ %
\endisadelimproof
%
\isatagproof
\isacommand{oops}\isamarkupfalse%
%
\endisatagproof
{\isafoldproof}%
%
\isadelimproof
%
\endisadelimproof
%
\begin{isamarkuptext}%
Observemos el lado izquierdo de la igualdad. \isa{Fs} es una 
  lista de fórmulas, luego la conjunción generalizada de dicha lista se 
  trata de una fórmula. Al aplicarle \isa{atoms} a dicha fórmula, obtenemos 
  finalmente el conjunto de sus átomos. Por otro lado, en el lado 
  derecho de la igualdad tenemos el conjunto \isa{set\ Fs} cuyos elementos 
  son las fórmulas de la lista \isa{Fs}. De este modo, al aplicar \isa{atoms\ {\isacharbackquote}} 
  a dicho conjunto obtenemos la imagen por \isa{atoms} de cada uno de sus
  elementos, es decir, un conjunto cuyos elementos son los 
  conjuntos de átomos de cada fórmula de \isa{Fs}. Por último, mediante la 
  unión se obtiene el conjunto de los átomos de cada fórmula de la 
  lista inicial.

  Veamos ahora la demostración detallada. Esta seguirá el esquema de 
  inducción sobre listas. Previamente vamos a probar cada caso por
  separado.%
\end{isamarkuptext}\isamarkuptrue%
\isacommand{lemma}\isamarkupfalse%
\ atoms{\isacharunderscore}BigAnd{\isacharunderscore}nil{\isacharcolon}\ \isanewline
\ \ {\isachardoublequoteopen}atoms\ {\isacharparenleft}\isactrlbold {\isasymAnd}{\isacharbrackleft}{\isacharbrackright}{\isacharparenright}\ {\isacharequal}\ {\isasymUnion}\ {\isacharparenleft}atoms\ {\isacharbackquote}\ set\ Nil{\isacharparenright}{\isachardoublequoteclose}\isanewline
%
\isadelimproof
%
\endisadelimproof
%
\isatagproof
\isacommand{proof}\isamarkupfalse%
\ {\isacharminus}\isanewline
\ \ \isacommand{have}\isamarkupfalse%
\ {\isachardoublequoteopen}atoms\ {\isacharparenleft}\isactrlbold {\isasymAnd}{\isacharbrackleft}{\isacharbrackright}{\isacharparenright}\ {\isacharequal}\ atoms\ {\isacharparenleft}\isactrlbold {\isasymnot}\ {\isasymbottom}{\isacharparenright}{\isachardoublequoteclose}\ \isanewline
\ \ \ \ \isacommand{by}\isamarkupfalse%
\ {\isacharparenleft}simp\ only{\isacharcolon}\ BigAnd{\isachardot}simps{\isacharparenleft}{\isadigit{1}}{\isacharparenright}{\isacharparenright}\isanewline
\ \ \isacommand{also}\isamarkupfalse%
\ \isacommand{have}\isamarkupfalse%
\ {\isachardoublequoteopen}{\isasymdots}\ {\isacharequal}\ atoms\ {\isasymbottom}{\isachardoublequoteclose}\ \isanewline
\ \ \ \ \isacommand{by}\isamarkupfalse%
\ {\isacharparenleft}simp\ only{\isacharcolon}\ formula{\isachardot}set{\isacharparenleft}{\isadigit{3}}{\isacharparenright}{\isacharparenright}\isanewline
\ \ \isacommand{also}\isamarkupfalse%
\ \isacommand{have}\isamarkupfalse%
\ {\isachardoublequoteopen}{\isasymdots}\ {\isacharequal}\ {\isasymemptyset}{\isachardoublequoteclose}\ \isanewline
\ \ \ \ \isacommand{by}\isamarkupfalse%
\ {\isacharparenleft}simp\ only{\isacharcolon}\ formula{\isachardot}set{\isacharparenleft}{\isadigit{2}}{\isacharparenright}{\isacharparenright}\isanewline
\ \ \isacommand{also}\isamarkupfalse%
\ \isacommand{have}\isamarkupfalse%
\ {\isachardoublequoteopen}{\isasymdots}\ {\isacharequal}\ {\isasymUnion}\ {\isasymemptyset}{\isachardoublequoteclose}\isanewline
\ \ \ \ \isacommand{by}\isamarkupfalse%
\ {\isacharparenleft}simp\ only{\isacharcolon}\ Union{\isacharunderscore}empty{\isacharparenright}\isanewline
\ \ \isacommand{also}\isamarkupfalse%
\ \isacommand{have}\isamarkupfalse%
\ {\isachardoublequoteopen}{\isasymdots}\ {\isacharequal}\ \ {\isasymUnion}\ {\isacharparenleft}atoms\ {\isacharbackquote}\ {\isasymemptyset}{\isacharparenright}{\isachardoublequoteclose}\isanewline
\ \ \ \ \isacommand{by}\isamarkupfalse%
\ {\isacharparenleft}simp\ only{\isacharcolon}\ image{\isacharunderscore}empty{\isacharparenright}\isanewline
\ \ \isacommand{also}\isamarkupfalse%
\ \isacommand{have}\isamarkupfalse%
\ {\isachardoublequoteopen}{\isasymdots}\ {\isacharequal}\ {\isasymUnion}\ {\isacharparenleft}atoms\ {\isacharbackquote}\ set\ {\isacharbrackleft}{\isacharbrackright}{\isacharparenright}{\isachardoublequoteclose}\isanewline
\ \ \ \ \isacommand{by}\isamarkupfalse%
\ {\isacharparenleft}simp\ only{\isacharcolon}\ list{\isachardot}set{\isacharparenright}\isanewline
\ \ \isacommand{finally}\isamarkupfalse%
\ \isacommand{show}\isamarkupfalse%
\ {\isacharquery}thesis\isanewline
\ \ \ \ \isacommand{by}\isamarkupfalse%
\ this\isanewline
\isacommand{qed}\isamarkupfalse%
%
\endisatagproof
{\isafoldproof}%
%
\isadelimproof
%
\endisadelimproof
%
\begin{isamarkuptext}%
Mostramos el siguiente lema auxiliar que utilizaremos en la
  demostración del último caso de inducción.%
\end{isamarkuptext}\isamarkuptrue%
\isacommand{lemma}\isamarkupfalse%
\ union{\isacharunderscore}imagen{\isacharcolon}\ {\isachardoublequoteopen}f\ a\ {\isasymunion}\ {\isasymUnion}\ {\isacharparenleft}f\ {\isacharbackquote}\ B{\isacharparenright}\ {\isacharequal}\ {\isasymUnion}\ {\isacharparenleft}f\ {\isacharbackquote}\ {\isacharparenleft}{\isacharbraceleft}a{\isacharbraceright}\ {\isasymunion}\ B{\isacharparenright}{\isacharparenright}{\isachardoublequoteclose}\isanewline
%
\isadelimproof
\ \ %
\endisadelimproof
%
\isatagproof
\isacommand{by}\isamarkupfalse%
\ {\isacharparenleft}simp\ only{\isacharcolon}\ Union{\isacharunderscore}image{\isacharunderscore}insert\isanewline
\ \ \ \ \ \ \ \ \ \ \ \ \ \ \ \ \ insert{\isacharunderscore}is{\isacharunderscore}Un{\isacharbrackleft}THEN\ sym{\isacharbrackright}{\isacharparenright}%
\endisatagproof
{\isafoldproof}%
%
\isadelimproof
%
\endisadelimproof
%
\begin{isamarkuptext}%
Se trata de una modificación del lema \isa{Union{\isacharunderscore}image{\isacharunderscore}insert} en 
  Isabelle para adaptarlo al caso particular. 

  \begin{itemize}
    \item[] \isa{{\isasymUnion}\ {\isacharparenleft}f\ {\isacharbackquote}\ {\isacharparenleft}{\isacharbraceleft}a{\isacharbraceright}\ {\isasymunion}\ B{\isacharparenright}{\isacharparenright}\ {\isacharequal}\ f\ a\ {\isasymunion}\ {\isasymUnion}\ {\isacharparenleft}f\ {\isacharbackquote}\ B{\isacharparenright}} 
      \hfill (\isa{Union{\isacharunderscore}image{\isacharunderscore}insert})
  \end{itemize}

  Para ello empleamos el lema \isa{insert{\isacharunderscore}is{\isacharunderscore}Un}.

  \begin{itemize}
    \item[] \isa{insert\ a\ A\ {\isacharequal}\ {\isacharbraceleft}a{\isacharbraceright}\ {\isasymunion}\ A} \hspace{5cm} \isa{{\isacharparenleft}insert{\isacharunderscore}is{\isacharunderscore}Un{\isacharparenright}}
  \end{itemize}

  De esta manera, la unión de un conjunto de un solo elemento y otro 
  conjunto cualquiera es equivalente a insertar dicho elemento en el 
  conjunto. Además, aplicamos el lema seguido de \isa{{\isacharbrackleft}THEN\ sym{\isacharbrackright}} para 
  mostrar la equivalencia en el sentido en el que acaba de ser
  enunciada por simetría, pues en Isabelle aparece en sentido opuesto.
  Por tanto, el lema auxiliar \isa{union{\isacharunderscore}imagen} es fundamentalmente el
  lema de Isabelle\\ \isa{Union{\isacharunderscore}image{\isacharunderscore}insert} teniendo en cuenta las
  equivalencias anteriores.

  Procedamos a la demostración del último caso de inducción.%
\end{isamarkuptext}\isamarkuptrue%
\isacommand{lemma}\isamarkupfalse%
\ atoms{\isacharunderscore}BigAnd{\isacharunderscore}cons{\isacharcolon}\isanewline
\ \ \isakeyword{assumes}\ {\isachardoublequoteopen}atoms\ {\isacharparenleft}\isactrlbold {\isasymAnd}Fs{\isacharparenright}\ {\isacharequal}\ {\isasymUnion}\ {\isacharparenleft}atoms\ {\isacharbackquote}\ set\ Fs{\isacharparenright}{\isachardoublequoteclose}\isanewline
\ \ \isakeyword{shows}\ {\isachardoublequoteopen}atoms\ {\isacharparenleft}\isactrlbold {\isasymAnd}{\isacharparenleft}F{\isacharhash}Fs{\isacharparenright}{\isacharparenright}\ {\isacharequal}\ {\isasymUnion}\ {\isacharparenleft}atoms\ {\isacharbackquote}\ set\ {\isacharparenleft}F{\isacharhash}Fs{\isacharparenright}{\isacharparenright}{\isachardoublequoteclose}\isanewline
%
\isadelimproof
%
\endisadelimproof
%
\isatagproof
\isacommand{proof}\isamarkupfalse%
\ {\isacharminus}\isanewline
\ \ \isacommand{have}\isamarkupfalse%
\ {\isachardoublequoteopen}atoms\ {\isacharparenleft}\isactrlbold {\isasymAnd}{\isacharparenleft}F{\isacharhash}Fs{\isacharparenright}{\isacharparenright}\ {\isacharequal}\ atoms\ {\isacharparenleft}F\ \isactrlbold {\isasymand}\ \isactrlbold {\isasymAnd}Fs{\isacharparenright}{\isachardoublequoteclose}\isanewline
\ \ \ \ \isacommand{by}\isamarkupfalse%
\ {\isacharparenleft}simp\ only{\isacharcolon}\ BigAnd{\isachardot}simps{\isacharparenleft}{\isadigit{2}}{\isacharparenright}{\isacharparenright}\isanewline
\ \ \isacommand{also}\isamarkupfalse%
\ \isacommand{have}\isamarkupfalse%
\ {\isachardoublequoteopen}{\isasymdots}\ {\isacharequal}\ atoms\ F\ {\isasymunion}\ atoms\ {\isacharparenleft}\isactrlbold {\isasymAnd}Fs{\isacharparenright}{\isachardoublequoteclose}\isanewline
\ \ \ \ \isacommand{by}\isamarkupfalse%
\ {\isacharparenleft}simp\ only{\isacharcolon}\ formula{\isachardot}set{\isacharparenleft}{\isadigit{4}}{\isacharparenright}{\isacharparenright}\isanewline
\ \ \isacommand{also}\isamarkupfalse%
\ \isacommand{have}\isamarkupfalse%
\ {\isachardoublequoteopen}{\isasymdots}\ {\isacharequal}\ atoms\ F\ {\isasymunion}\ {\isasymUnion}{\isacharparenleft}atoms\ {\isacharbackquote}\ set\ Fs{\isacharparenright}{\isachardoublequoteclose}\isanewline
\ \ \ \ \isacommand{by}\isamarkupfalse%
\ {\isacharparenleft}simp\ only{\isacharcolon}\ assms{\isacharparenright}\isanewline
\ \ \isacommand{also}\isamarkupfalse%
\ \isacommand{have}\isamarkupfalse%
\ {\isachardoublequoteopen}{\isasymdots}\ {\isacharequal}\ {\isasymUnion}{\isacharparenleft}atoms\ {\isacharbackquote}\ {\isacharparenleft}{\isacharbraceleft}F{\isacharbraceright}\ {\isasymunion}\ set\ Fs{\isacharparenright}{\isacharparenright}{\isachardoublequoteclose}\ \isanewline
\ \ \ \ \isacommand{by}\isamarkupfalse%
\ {\isacharparenleft}simp\ only{\isacharcolon}\ union{\isacharunderscore}imagen{\isacharparenright}\isanewline
\ \ \isacommand{also}\isamarkupfalse%
\ \isacommand{have}\isamarkupfalse%
\ {\isachardoublequoteopen}{\isasymdots}\ {\isacharequal}\ \ {\isasymUnion}{\isacharparenleft}atoms\ {\isacharbackquote}\ set\ {\isacharparenleft}F{\isacharhash}Fs{\isacharparenright}{\isacharparenright}{\isachardoublequoteclose}\isanewline
\ \ \ \ \isacommand{by}\isamarkupfalse%
\ {\isacharparenleft}simp\ only{\isacharcolon}\ set{\isacharunderscore}insert{\isacharparenright}\isanewline
\ \ \isacommand{finally}\isamarkupfalse%
\ \isacommand{show}\isamarkupfalse%
\ \ {\isachardoublequoteopen}atoms\ {\isacharparenleft}\isactrlbold {\isasymAnd}{\isacharparenleft}F{\isacharhash}Fs{\isacharparenright}{\isacharparenright}\ {\isacharequal}\ {\isasymUnion}{\isacharparenleft}atoms\ {\isacharbackquote}\ set\ {\isacharparenleft}F{\isacharhash}Fs{\isacharparenright}{\isacharparenright}{\isachardoublequoteclose}\ \isanewline
\ \ \ \ \isacommand{by}\isamarkupfalse%
\ this\isanewline
\isacommand{qed}\isamarkupfalse%
%
\endisatagproof
{\isafoldproof}%
%
\isadelimproof
%
\endisadelimproof
%
\begin{isamarkuptext}%
Por tanto, la demostración detallada completa es la siguiente.%
\end{isamarkuptext}\isamarkuptrue%
\isacommand{lemma}\isamarkupfalse%
\ {\isachardoublequoteopen}atoms\ {\isacharparenleft}\isactrlbold {\isasymAnd}Fs{\isacharparenright}\ {\isacharequal}\ {\isasymUnion}\ {\isacharparenleft}atoms\ {\isacharbackquote}\ set\ Fs{\isacharparenright}{\isachardoublequoteclose}\isanewline
%
\isadelimproof
%
\endisadelimproof
%
\isatagproof
\isacommand{proof}\isamarkupfalse%
\ {\isacharparenleft}induction\ Fs{\isacharparenright}\isanewline
\ \ \isacommand{case}\isamarkupfalse%
\ Nil\isanewline
\ \ \isacommand{then}\isamarkupfalse%
\ \isacommand{show}\isamarkupfalse%
\ {\isacharquery}case\ \isacommand{by}\isamarkupfalse%
\ {\isacharparenleft}rule\ atoms{\isacharunderscore}BigAnd{\isacharunderscore}nil{\isacharparenright}\isanewline
\isacommand{next}\isamarkupfalse%
\isanewline
\ \ \isacommand{case}\isamarkupfalse%
\ {\isacharparenleft}Cons\ a\ Fs{\isacharparenright}\isanewline
\ \ \isacommand{assume}\isamarkupfalse%
\ {\isachardoublequoteopen}atoms\ {\isacharparenleft}\isactrlbold {\isasymAnd}Fs{\isacharparenright}\ {\isacharequal}\ {\isasymUnion}{\isacharparenleft}atoms\ {\isacharbackquote}\ set\ Fs{\isacharparenright}{\isachardoublequoteclose}\ \isanewline
\ \ \isacommand{then}\isamarkupfalse%
\ \isacommand{show}\isamarkupfalse%
\ {\isacharquery}case\ \isanewline
\ \ \ \ \isacommand{by}\isamarkupfalse%
\ {\isacharparenleft}rule\ atoms{\isacharunderscore}BigAnd{\isacharunderscore}cons{\isacharparenright}\isanewline
\isacommand{qed}\isamarkupfalse%
%
\endisatagproof
{\isafoldproof}%
%
\isadelimproof
%
\endisadelimproof
%
\begin{isamarkuptext}%
Por último, su demostración automática.%
\end{isamarkuptext}\isamarkuptrue%
\isacommand{lemma}\isamarkupfalse%
\ atoms{\isacharunderscore}BigAnd{\isacharcolon}\ \isanewline
\ \ {\isachardoublequoteopen}atoms\ {\isacharparenleft}\isactrlbold {\isasymAnd}Fs{\isacharparenright}\ {\isacharequal}\ {\isasymUnion}{\isacharparenleft}atoms\ {\isacharbackquote}\ set\ Fs{\isacharparenright}{\isachardoublequoteclose}\isanewline
%
\isadelimproof
\ \ %
\endisadelimproof
%
\isatagproof
\isacommand{by}\isamarkupfalse%
\ {\isacharparenleft}induction\ Fs{\isacharparenright}\ simp{\isacharunderscore}all%
\endisatagproof
{\isafoldproof}%
%
\isadelimproof
%
\endisadelimproof
%
\isadelimtheory
%
\endisadelimtheory
%
\isatagtheory
%
\endisatagtheory
{\isafoldtheory}%
%
\isadelimtheory
%
\endisadelimtheory
%
\end{isabellebody}%
\endinput
%:%file=~/Desktop/TFM/TFM1/Sintaxis.thy%:%
%:%24=13%:%
%:%36=15%:%
%:%37=16%:%
%:%38=17%:%
%:%39=18%:%
%:%40=19%:%
%:%41=20%:%
%:%42=21%:%
%:%43=22%:%
%:%44=23%:%
%:%45=24%:%
%:%46=25%:%
%:%47=26%:%
%:%48=27%:%
%:%49=28%:%
%:%50=29%:%
%:%51=30%:%
%:%52=31%:%
%:%53=32%:%
%:%54=33%:%
%:%55=34%:%
%:%56=35%:%
%:%57=36%:%
%:%58=37%:%
%:%59=38%:%
%:%60=39%:%
%:%61=40%:%
%:%62=41%:%
%:%63=42%:%
%:%64=43%:%
%:%65=44%:%
%:%66=45%:%
%:%67=46%:%
%:%68=47%:%
%:%69=48%:%
%:%70=49%:%
%:%71=50%:%
%:%72=51%:%
%:%73=52%:%
%:%74=53%:%
%:%75=54%:%
%:%76=55%:%
%:%77=56%:%
%:%78=57%:%
%:%79=58%:%
%:%80=59%:%
%:%81=60%:%
%:%82=61%:%
%:%83=62%:%
%:%84=63%:%
%:%85=64%:%
%:%86=65%:%
%:%87=66%:%
%:%88=67%:%
%:%89=68%:%
%:%90=69%:%
%:%91=70%:%
%:%92=71%:%
%:%93=72%:%
%:%94=73%:%
%:%95=74%:%
%:%97=76%:%
%:%98=76%:%
%:%99=77%:%
%:%100=78%:%
%:%101=79%:%
%:%102=80%:%
%:%103=81%:%
%:%104=82%:%
%:%106=84%:%
%:%107=85%:%
%:%108=86%:%
%:%109=87%:%
%:%110=88%:%
%:%111=89%:%
%:%112=90%:%
%:%113=91%:%
%:%114=92%:%
%:%115=93%:%
%:%116=94%:%
%:%117=95%:%
%:%118=96%:%
%:%119=97%:%
%:%120=98%:%
%:%121=99%:%
%:%122=100%:%
%:%123=101%:%
%:%124=102%:%
%:%125=103%:%
%:%126=104%:%
%:%127=105%:%
%:%128=106%:%
%:%129=107%:%
%:%130=108%:%
%:%131=109%:%
%:%132=110%:%
%:%133=111%:%
%:%134=112%:%
%:%135=113%:%
%:%140=113%:%
%:%141=114%:%
%:%142=115%:%
%:%143=116%:%
%:%144=117%:%
%:%145=118%:%
%:%147=120%:%
%:%148=120%:%
%:%149=121%:%
%:%152=122%:%
%:%156=122%:%
%:%157=122%:%
%:%158=123%:%
%:%159=124%:%
%:%160=124%:%
%:%161=125%:%
%:%162=125%:%
%:%163=126%:%
%:%164=127%:%
%:%165=127%:%
%:%166=128%:%
%:%167=128%:%
%:%168=129%:%
%:%169=130%:%
%:%170=130%:%
%:%171=131%:%
%:%172=131%:%
%:%173=132%:%
%:%174=133%:%
%:%175=133%:%
%:%176=134%:%
%:%177=134%:%
%:%182=134%:%
%:%185=135%:%
%:%188=137%:%
%:%189=138%:%
%:%190=139%:%
%:%192=141%:%
%:%193=141%:%
%:%194=142%:%
%:%197=143%:%
%:%201=143%:%
%:%202=143%:%
%:%203=144%:%
%:%204=145%:%
%:%205=145%:%
%:%206=146%:%
%:%207=146%:%
%:%208=147%:%
%:%209=148%:%
%:%210=148%:%
%:%211=149%:%
%:%212=149%:%
%:%213=150%:%
%:%214=150%:%
%:%215=151%:%
%:%216=151%:%
%:%217=152%:%
%:%218=152%:%
%:%219=152%:%
%:%220=153%:%
%:%221=153%:%
%:%222=154%:%
%:%223=154%:%
%:%224=154%:%
%:%225=155%:%
%:%226=155%:%
%:%227=156%:%
%:%228=156%:%
%:%229=156%:%
%:%230=157%:%
%:%231=157%:%
%:%232=158%:%
%:%233=158%:%
%:%234=158%:%
%:%235=159%:%
%:%236=159%:%
%:%237=160%:%
%:%238=160%:%
%:%239=161%:%
%:%240=162%:%
%:%241=162%:%
%:%242=163%:%
%:%243=163%:%
%:%248=163%:%
%:%251=164%:%
%:%252=164%:%
%:%253=165%:%
%:%254=166%:%
%:%255=166%:%
%:%257=168%:%
%:%258=169%:%
%:%259=170%:%
%:%260=171%:%
%:%261=172%:%
%:%262=173%:%
%:%263=174%:%
%:%264=175%:%
%:%265=176%:%
%:%266=177%:%
%:%267=178%:%
%:%268=179%:%
%:%269=180%:%
%:%270=181%:%
%:%271=182%:%
%:%272=183%:%
%:%273=184%:%
%:%274=185%:%
%:%275=186%:%
%:%276=187%:%
%:%277=188%:%
%:%278=189%:%
%:%279=190%:%
%:%280=191%:%
%:%281=192%:%
%:%282=193%:%
%:%283=194%:%
%:%284=195%:%
%:%285=196%:%
%:%286=197%:%
%:%287=198%:%
%:%288=199%:%
%:%289=200%:%
%:%290=201%:%
%:%291=202%:%
%:%292=203%:%
%:%293=204%:%
%:%294=205%:%
%:%295=206%:%
%:%296=207%:%
%:%297=208%:%
%:%298=209%:%
%:%299=210%:%
%:%300=211%:%
%:%301=212%:%
%:%302=213%:%
%:%303=214%:%
%:%304=215%:%
%:%305=216%:%
%:%306=217%:%
%:%307=218%:%
%:%308=219%:%
%:%309=220%:%
%:%310=221%:%
%:%311=222%:%
%:%312=223%:%
%:%313=224%:%
%:%314=225%:%
%:%315=226%:%
%:%316=227%:%
%:%317=228%:%
%:%318=229%:%
%:%319=230%:%
%:%320=231%:%
%:%321=232%:%
%:%322=233%:%
%:%323=234%:%
%:%324=235%:%
%:%325=236%:%
%:%326=237%:%
%:%327=238%:%
%:%328=239%:%
%:%329=240%:%
%:%330=241%:%
%:%331=242%:%
%:%332=243%:%
%:%333=244%:%
%:%334=245%:%
%:%335=246%:%
%:%336=247%:%
%:%337=248%:%
%:%338=249%:%
%:%339=250%:%
%:%340=251%:%
%:%341=252%:%
%:%342=253%:%
%:%343=254%:%
%:%344=255%:%
%:%346=257%:%
%:%347=257%:%
%:%350=258%:%
%:%354=258%:%
%:%364=260%:%
%:%365=261%:%
%:%366=262%:%
%:%367=263%:%
%:%368=264%:%
%:%369=265%:%
%:%370=266%:%
%:%371=267%:%
%:%372=268%:%
%:%373=269%:%
%:%374=270%:%
%:%375=271%:%
%:%376=272%:%
%:%377=273%:%
%:%378=274%:%
%:%379=275%:%
%:%380=276%:%
%:%381=277%:%
%:%382=278%:%
%:%383=279%:%
%:%384=280%:%
%:%385=281%:%
%:%386=282%:%
%:%387=283%:%
%:%388=284%:%
%:%389=285%:%
%:%390=286%:%
%:%391=287%:%
%:%392=288%:%
%:%394=290%:%
%:%395=290%:%
%:%396=291%:%
%:%403=292%:%
%:%404=292%:%
%:%405=293%:%
%:%406=293%:%
%:%407=294%:%
%:%408=294%:%
%:%409=295%:%
%:%410=295%:%
%:%411=295%:%
%:%412=296%:%
%:%413=296%:%
%:%414=297%:%
%:%415=297%:%
%:%416=297%:%
%:%417=298%:%
%:%418=298%:%
%:%419=299%:%
%:%425=299%:%
%:%428=300%:%
%:%429=301%:%
%:%430=301%:%
%:%431=302%:%
%:%438=303%:%
%:%439=303%:%
%:%440=304%:%
%:%441=304%:%
%:%442=305%:%
%:%443=305%:%
%:%444=306%:%
%:%445=306%:%
%:%446=306%:%
%:%447=307%:%
%:%448=307%:%
%:%449=308%:%
%:%455=308%:%
%:%458=309%:%
%:%459=310%:%
%:%460=310%:%
%:%461=311%:%
%:%462=312%:%
%:%465=313%:%
%:%469=313%:%
%:%470=313%:%
%:%471=314%:%
%:%472=314%:%
%:%477=314%:%
%:%480=315%:%
%:%481=316%:%
%:%482=316%:%
%:%483=317%:%
%:%484=318%:%
%:%485=319%:%
%:%492=320%:%
%:%493=320%:%
%:%494=321%:%
%:%495=321%:%
%:%496=322%:%
%:%497=322%:%
%:%498=323%:%
%:%499=323%:%
%:%500=324%:%
%:%501=324%:%
%:%502=324%:%
%:%503=325%:%
%:%504=325%:%
%:%505=326%:%
%:%511=326%:%
%:%514=327%:%
%:%515=328%:%
%:%516=328%:%
%:%517=329%:%
%:%518=330%:%
%:%519=331%:%
%:%526=332%:%
%:%527=332%:%
%:%528=333%:%
%:%529=333%:%
%:%530=334%:%
%:%531=334%:%
%:%532=335%:%
%:%533=335%:%
%:%534=336%:%
%:%535=336%:%
%:%536=336%:%
%:%537=337%:%
%:%538=337%:%
%:%539=338%:%
%:%545=338%:%
%:%548=339%:%
%:%549=340%:%
%:%550=340%:%
%:%551=341%:%
%:%552=342%:%
%:%553=343%:%
%:%560=344%:%
%:%561=344%:%
%:%562=345%:%
%:%563=345%:%
%:%564=346%:%
%:%565=346%:%
%:%566=347%:%
%:%567=347%:%
%:%568=348%:%
%:%569=348%:%
%:%570=348%:%
%:%571=349%:%
%:%572=349%:%
%:%573=350%:%
%:%579=350%:%
%:%582=351%:%
%:%583=352%:%
%:%584=352%:%
%:%591=353%:%
%:%592=353%:%
%:%593=354%:%
%:%594=354%:%
%:%595=355%:%
%:%596=355%:%
%:%597=355%:%
%:%598=355%:%
%:%599=356%:%
%:%600=356%:%
%:%601=357%:%
%:%602=357%:%
%:%603=358%:%
%:%604=358%:%
%:%605=358%:%
%:%606=358%:%
%:%607=359%:%
%:%608=359%:%
%:%609=360%:%
%:%610=360%:%
%:%611=361%:%
%:%612=361%:%
%:%613=361%:%
%:%614=361%:%
%:%615=362%:%
%:%616=362%:%
%:%617=363%:%
%:%618=363%:%
%:%619=364%:%
%:%620=364%:%
%:%621=364%:%
%:%622=364%:%
%:%623=365%:%
%:%624=365%:%
%:%625=366%:%
%:%626=366%:%
%:%627=367%:%
%:%628=367%:%
%:%629=367%:%
%:%630=367%:%
%:%631=368%:%
%:%632=368%:%
%:%633=369%:%
%:%634=369%:%
%:%635=370%:%
%:%636=370%:%
%:%637=370%:%
%:%638=370%:%
%:%639=371%:%
%:%649=373%:%
%:%651=375%:%
%:%652=375%:%
%:%655=376%:%
%:%659=376%:%
%:%660=376%:%
%:%674=378%:%
%:%686=380%:%
%:%687=381%:%
%:%688=382%:%
%:%689=383%:%
%:%690=384%:%
%:%691=385%:%
%:%692=386%:%
%:%693=387%:%
%:%694=388%:%
%:%695=389%:%
%:%696=390%:%
%:%697=391%:%
%:%698=392%:%
%:%699=393%:%
%:%700=394%:%
%:%701=395%:%
%:%702=396%:%
%:%703=397%:%
%:%705=399%:%
%:%706=399%:%
%:%707=400%:%
%:%708=401%:%
%:%709=402%:%
%:%710=403%:%
%:%711=404%:%
%:%712=405%:%
%:%714=407%:%
%:%715=408%:%
%:%716=409%:%
%:%717=410%:%
%:%718=411%:%
%:%719=412%:%
%:%720=413%:%
%:%722=416%:%
%:%723=416%:%
%:%724=417%:%
%:%727=418%:%
%:%731=418%:%
%:%732=418%:%
%:%733=419%:%
%:%734=420%:%
%:%735=420%:%
%:%736=421%:%
%:%737=421%:%
%:%738=422%:%
%:%739=423%:%
%:%740=423%:%
%:%741=424%:%
%:%742=424%:%
%:%743=425%:%
%:%744=426%:%
%:%745=426%:%
%:%747=428%:%
%:%748=429%:%
%:%749=429%:%
%:%750=430%:%
%:%751=431%:%
%:%752=431%:%
%:%753=432%:%
%:%754=432%:%
%:%755=433%:%
%:%756=434%:%
%:%757=434%:%
%:%758=435%:%
%:%759=436%:%
%:%760=436%:%
%:%765=436%:%
%:%768=437%:%
%:%771=439%:%
%:%772=440%:%
%:%773=441%:%
%:%775=443%:%
%:%776=443%:%
%:%777=444%:%
%:%779=446%:%
%:%780=447%:%
%:%781=448%:%
%:%782=449%:%
%:%783=450%:%
%:%784=451%:%
%:%785=452%:%
%:%787=455%:%
%:%788=455%:%
%:%789=456%:%
%:%792=457%:%
%:%796=457%:%
%:%797=457%:%
%:%798=458%:%
%:%799=459%:%
%:%800=459%:%
%:%801=460%:%
%:%802=460%:%
%:%803=461%:%
%:%804=462%:%
%:%805=462%:%
%:%807=464%:%
%:%808=465%:%
%:%809=465%:%
%:%814=465%:%
%:%817=466%:%
%:%820=468%:%
%:%821=469%:%
%:%822=470%:%
%:%823=471%:%
%:%824=472%:%
%:%825=473%:%
%:%826=474%:%
%:%827=475%:%
%:%828=476%:%
%:%829=477%:%
%:%831=479%:%
%:%832=479%:%
%:%835=480%:%
%:%839=480%:%
%:%840=480%:%
%:%849=482%:%
%:%850=483%:%
%:%852=485%:%
%:%853=485%:%
%:%854=486%:%
%:%857=487%:%
%:%861=487%:%
%:%862=487%:%
%:%867=487%:%
%:%870=488%:%
%:%871=489%:%
%:%872=489%:%
%:%873=490%:%
%:%876=491%:%
%:%880=491%:%
%:%881=491%:%
%:%886=491%:%
%:%889=492%:%
%:%890=493%:%
%:%891=493%:%
%:%892=494%:%
%:%899=495%:%
%:%900=495%:%
%:%901=496%:%
%:%902=496%:%
%:%903=497%:%
%:%904=497%:%
%:%905=498%:%
%:%906=498%:%
%:%907=498%:%
%:%908=499%:%
%:%909=499%:%
%:%910=500%:%
%:%911=500%:%
%:%912=500%:%
%:%913=501%:%
%:%914=501%:%
%:%915=502%:%
%:%921=502%:%
%:%924=503%:%
%:%925=504%:%
%:%926=504%:%
%:%927=505%:%
%:%928=506%:%
%:%935=507%:%
%:%936=507%:%
%:%937=508%:%
%:%938=508%:%
%:%939=509%:%
%:%940=510%:%
%:%941=510%:%
%:%942=511%:%
%:%943=511%:%
%:%944=511%:%
%:%945=512%:%
%:%946=512%:%
%:%947=513%:%
%:%948=513%:%
%:%949=513%:%
%:%950=514%:%
%:%951=514%:%
%:%952=515%:%
%:%953=515%:%
%:%954=515%:%
%:%955=516%:%
%:%956=516%:%
%:%957=517%:%
%:%963=517%:%
%:%966=518%:%
%:%967=519%:%
%:%968=519%:%
%:%969=520%:%
%:%970=521%:%
%:%977=522%:%
%:%978=522%:%
%:%979=523%:%
%:%980=523%:%
%:%981=524%:%
%:%982=525%:%
%:%983=525%:%
%:%984=526%:%
%:%985=526%:%
%:%986=526%:%
%:%987=527%:%
%:%988=527%:%
%:%989=528%:%
%:%990=528%:%
%:%991=528%:%
%:%992=529%:%
%:%993=529%:%
%:%994=530%:%
%:%995=530%:%
%:%996=530%:%
%:%997=531%:%
%:%998=531%:%
%:%999=532%:%
%:%1005=532%:%
%:%1008=533%:%
%:%1009=534%:%
%:%1010=534%:%
%:%1011=535%:%
%:%1012=536%:%
%:%1019=537%:%
%:%1020=537%:%
%:%1021=538%:%
%:%1022=538%:%
%:%1023=539%:%
%:%1024=540%:%
%:%1025=540%:%
%:%1026=541%:%
%:%1027=541%:%
%:%1028=541%:%
%:%1029=542%:%
%:%1030=542%:%
%:%1031=543%:%
%:%1032=543%:%
%:%1033=543%:%
%:%1034=544%:%
%:%1035=544%:%
%:%1036=545%:%
%:%1037=545%:%
%:%1038=545%:%
%:%1039=546%:%
%:%1040=546%:%
%:%1041=547%:%
%:%1051=549%:%
%:%1052=550%:%
%:%1053=551%:%
%:%1054=552%:%
%:%1055=553%:%
%:%1056=554%:%
%:%1057=555%:%
%:%1058=556%:%
%:%1059=557%:%
%:%1060=558%:%
%:%1061=559%:%
%:%1062=560%:%
%:%1063=561%:%
%:%1064=562%:%
%:%1065=563%:%
%:%1066=564%:%
%:%1067=565%:%
%:%1068=566%:%
%:%1069=567%:%
%:%1070=568%:%
%:%1071=569%:%
%:%1072=570%:%
%:%1073=571%:%
%:%1074=572%:%
%:%1075=573%:%
%:%1076=574%:%
%:%1077=575%:%
%:%1078=576%:%
%:%1079=577%:%
%:%1080=578%:%
%:%1081=579%:%
%:%1082=580%:%
%:%1083=581%:%
%:%1084=582%:%
%:%1086=584%:%
%:%1087=584%:%
%:%1094=585%:%
%:%1095=585%:%
%:%1096=586%:%
%:%1097=586%:%
%:%1098=587%:%
%:%1099=587%:%
%:%1100=587%:%
%:%1101=588%:%
%:%1102=588%:%
%:%1103=589%:%
%:%1104=589%:%
%:%1105=590%:%
%:%1106=590%:%
%:%1107=591%:%
%:%1108=591%:%
%:%1109=591%:%
%:%1110=592%:%
%:%1111=592%:%
%:%1112=593%:%
%:%1113=593%:%
%:%1114=594%:%
%:%1115=594%:%
%:%1116=595%:%
%:%1117=595%:%
%:%1118=595%:%
%:%1119=596%:%
%:%1120=596%:%
%:%1121=597%:%
%:%1122=597%:%
%:%1123=598%:%
%:%1124=598%:%
%:%1125=599%:%
%:%1126=599%:%
%:%1127=599%:%
%:%1128=600%:%
%:%1129=600%:%
%:%1130=601%:%
%:%1131=601%:%
%:%1132=602%:%
%:%1133=602%:%
%:%1134=603%:%
%:%1135=603%:%
%:%1136=603%:%
%:%1137=604%:%
%:%1138=604%:%
%:%1139=605%:%
%:%1140=605%:%
%:%1141=606%:%
%:%1142=606%:%
%:%1143=607%:%
%:%1144=607%:%
%:%1145=607%:%
%:%1146=608%:%
%:%1147=608%:%
%:%1148=609%:%
%:%1158=611%:%
%:%1160=613%:%
%:%1161=613%:%
%:%1164=614%:%
%:%1168=614%:%
%:%1169=614%:%
%:%1178=616%:%
%:%1179=617%:%
%:%1180=618%:%
%:%1181=619%:%
%:%1182=620%:%
%:%1183=621%:%
%:%1184=622%:%
%:%1185=623%:%
%:%1187=625%:%
%:%1188=625%:%
%:%1189=626%:%
%:%1190=627%:%
%:%1197=628%:%
%:%1198=628%:%
%:%1199=629%:%
%:%1200=629%:%
%:%1201=630%:%
%:%1202=630%:%
%:%1203=631%:%
%:%1204=631%:%
%:%1205=632%:%
%:%1206=632%:%
%:%1207=632%:%
%:%1208=633%:%
%:%1209=633%:%
%:%1210=634%:%
%:%1216=634%:%
%:%1219=635%:%
%:%1220=636%:%
%:%1221=636%:%
%:%1222=637%:%
%:%1223=638%:%
%:%1230=639%:%
%:%1231=639%:%
%:%1232=640%:%
%:%1233=640%:%
%:%1234=641%:%
%:%1235=641%:%
%:%1236=642%:%
%:%1237=642%:%
%:%1238=643%:%
%:%1239=643%:%
%:%1240=643%:%
%:%1241=644%:%
%:%1242=644%:%
%:%1243=645%:%
%:%1253=647%:%
%:%1254=648%:%
%:%1256=650%:%
%:%1257=650%:%
%:%1260=651%:%
%:%1264=651%:%
%:%1265=651%:%
%:%1270=651%:%
%:%1273=652%:%
%:%1274=653%:%
%:%1275=653%:%
%:%1278=654%:%
%:%1282=654%:%
%:%1283=654%:%
%:%1292=656%:%
%:%1293=657%:%
%:%1294=658%:%
%:%1295=659%:%
%:%1296=660%:%
%:%1297=661%:%
%:%1298=662%:%
%:%1299=663%:%
%:%1300=664%:%
%:%1301=665%:%
%:%1302=666%:%
%:%1303=667%:%
%:%1304=668%:%
%:%1305=669%:%
%:%1306=670%:%
%:%1307=671%:%
%:%1308=672%:%
%:%1309=673%:%
%:%1310=674%:%
%:%1311=675%:%
%:%1312=676%:%
%:%1313=677%:%
%:%1314=678%:%
%:%1315=679%:%
%:%1316=680%:%
%:%1317=681%:%
%:%1318=682%:%
%:%1319=683%:%
%:%1320=684%:%
%:%1321=685%:%
%:%1322=686%:%
%:%1323=687%:%
%:%1324=688%:%
%:%1325=689%:%
%:%1326=690%:%
%:%1327=691%:%
%:%1328=692%:%
%:%1329=693%:%
%:%1330=694%:%
%:%1331=695%:%
%:%1332=696%:%
%:%1333=697%:%
%:%1334=698%:%
%:%1335=699%:%
%:%1336=700%:%
%:%1337=701%:%
%:%1338=702%:%
%:%1339=703%:%
%:%1340=704%:%
%:%1341=705%:%
%:%1342=706%:%
%:%1343=707%:%
%:%1344=708%:%
%:%1345=709%:%
%:%1346=710%:%
%:%1347=711%:%
%:%1348=712%:%
%:%1349=713%:%
%:%1350=714%:%
%:%1351=715%:%
%:%1352=716%:%
%:%1353=717%:%
%:%1354=718%:%
%:%1356=720%:%
%:%1357=720%:%
%:%1360=721%:%
%:%1364=721%:%
%:%1374=723%:%
%:%1375=724%:%
%:%1376=725%:%
%:%1377=726%:%
%:%1378=727%:%
%:%1379=728%:%
%:%1380=729%:%
%:%1381=730%:%
%:%1382=731%:%
%:%1383=732%:%
%:%1384=733%:%
%:%1385=734%:%
%:%1386=735%:%
%:%1388=737%:%
%:%1389=737%:%
%:%1390=738%:%
%:%1393=739%:%
%:%1397=739%:%
%:%1398=739%:%
%:%1399=740%:%
%:%1400=741%:%
%:%1401=741%:%
%:%1402=742%:%
%:%1403=742%:%
%:%1404=743%:%
%:%1405=744%:%
%:%1406=744%:%
%:%1407=745%:%
%:%1408=746%:%
%:%1409=746%:%
%:%1410=747%:%
%:%1411=748%:%
%:%1412=748%:%
%:%1413=749%:%
%:%1414=750%:%
%:%1415=750%:%
%:%1420=750%:%
%:%1423=751%:%
%:%1426=753%:%
%:%1427=754%:%
%:%1428=755%:%
%:%1429=756%:%
%:%1430=757%:%
%:%1431=758%:%
%:%1432=759%:%
%:%1433=760%:%
%:%1434=761%:%
%:%1435=762%:%
%:%1436=763%:%
%:%1437=764%:%
%:%1438=765%:%
%:%1439=766%:%
%:%1440=767%:%
%:%1442=769%:%
%:%1443=769%:%
%:%1444=770%:%
%:%1451=771%:%
%:%1452=771%:%
%:%1453=772%:%
%:%1454=772%:%
%:%1455=773%:%
%:%1456=773%:%
%:%1457=774%:%
%:%1458=774%:%
%:%1459=774%:%
%:%1460=775%:%
%:%1461=775%:%
%:%1462=776%:%
%:%1463=776%:%
%:%1464=776%:%
%:%1465=777%:%
%:%1466=777%:%
%:%1467=778%:%
%:%1468=778%:%
%:%1469=778%:%
%:%1470=779%:%
%:%1471=779%:%
%:%1472=780%:%
%:%1473=780%:%
%:%1474=780%:%
%:%1475=781%:%
%:%1476=781%:%
%:%1477=782%:%
%:%1478=782%:%
%:%1479=782%:%
%:%1480=783%:%
%:%1481=783%:%
%:%1482=784%:%
%:%1488=784%:%
%:%1491=785%:%
%:%1492=786%:%
%:%1493=786%:%
%:%1494=787%:%
%:%1501=788%:%
%:%1502=788%:%
%:%1503=789%:%
%:%1504=789%:%
%:%1505=790%:%
%:%1506=790%:%
%:%1507=791%:%
%:%1508=791%:%
%:%1509=791%:%
%:%1510=792%:%
%:%1511=792%:%
%:%1512=793%:%
%:%1513=793%:%
%:%1514=793%:%
%:%1515=794%:%
%:%1516=794%:%
%:%1517=795%:%
%:%1518=795%:%
%:%1519=795%:%
%:%1520=796%:%
%:%1521=796%:%
%:%1522=797%:%
%:%1528=797%:%
%:%1531=798%:%
%:%1532=799%:%
%:%1533=799%:%
%:%1534=800%:%
%:%1535=801%:%
%:%1542=802%:%
%:%1543=802%:%
%:%1544=803%:%
%:%1545=803%:%
%:%1546=804%:%
%:%1547=804%:%
%:%1548=805%:%
%:%1549=805%:%
%:%1550=805%:%
%:%1551=806%:%
%:%1552=806%:%
%:%1553=807%:%
%:%1554=807%:%
%:%1555=807%:%
%:%1556=808%:%
%:%1557=808%:%
%:%1558=809%:%
%:%1559=809%:%
%:%1560=809%:%
%:%1561=810%:%
%:%1562=810%:%
%:%1563=811%:%
%:%1564=811%:%
%:%1565=811%:%
%:%1566=812%:%
%:%1567=812%:%
%:%1568=813%:%
%:%1574=813%:%
%:%1577=814%:%
%:%1578=815%:%
%:%1579=815%:%
%:%1580=816%:%
%:%1581=817%:%
%:%1582=818%:%
%:%1589=819%:%
%:%1590=819%:%
%:%1591=820%:%
%:%1592=820%:%
%:%1593=821%:%
%:%1594=821%:%
%:%1595=822%:%
%:%1596=822%:%
%:%1597=822%:%
%:%1598=823%:%
%:%1599=823%:%
%:%1600=824%:%
%:%1601=824%:%
%:%1602=824%:%
%:%1603=825%:%
%:%1604=825%:%
%:%1605=826%:%
%:%1606=826%:%
%:%1607=827%:%
%:%1608=827%:%
%:%1609=827%:%
%:%1610=828%:%
%:%1611=828%:%
%:%1612=829%:%
%:%1613=829%:%
%:%1614=829%:%
%:%1615=830%:%
%:%1616=830%:%
%:%1617=831%:%
%:%1618=831%:%
%:%1619=831%:%
%:%1620=832%:%
%:%1621=832%:%
%:%1622=833%:%
%:%1628=833%:%
%:%1631=834%:%
%:%1632=835%:%
%:%1633=835%:%
%:%1634=836%:%
%:%1635=837%:%
%:%1636=838%:%
%:%1643=839%:%
%:%1644=839%:%
%:%1645=840%:%
%:%1646=840%:%
%:%1647=841%:%
%:%1648=841%:%
%:%1649=842%:%
%:%1650=842%:%
%:%1651=842%:%
%:%1652=843%:%
%:%1653=843%:%
%:%1654=844%:%
%:%1655=844%:%
%:%1656=844%:%
%:%1657=845%:%
%:%1658=845%:%
%:%1659=846%:%
%:%1660=846%:%
%:%1661=847%:%
%:%1662=847%:%
%:%1663=847%:%
%:%1664=848%:%
%:%1665=848%:%
%:%1666=849%:%
%:%1667=849%:%
%:%1668=849%:%
%:%1669=850%:%
%:%1670=850%:%
%:%1671=851%:%
%:%1672=851%:%
%:%1673=851%:%
%:%1674=852%:%
%:%1675=852%:%
%:%1676=853%:%
%:%1682=853%:%
%:%1685=854%:%
%:%1686=855%:%
%:%1687=855%:%
%:%1688=856%:%
%:%1689=857%:%
%:%1690=858%:%
%:%1697=859%:%
%:%1698=859%:%
%:%1699=860%:%
%:%1700=860%:%
%:%1701=861%:%
%:%1702=861%:%
%:%1703=862%:%
%:%1704=862%:%
%:%1705=862%:%
%:%1706=863%:%
%:%1707=863%:%
%:%1708=864%:%
%:%1709=864%:%
%:%1710=864%:%
%:%1711=865%:%
%:%1712=865%:%
%:%1713=866%:%
%:%1714=866%:%
%:%1715=867%:%
%:%1716=867%:%
%:%1717=867%:%
%:%1718=868%:%
%:%1719=868%:%
%:%1720=869%:%
%:%1721=869%:%
%:%1722=869%:%
%:%1723=870%:%
%:%1724=870%:%
%:%1725=871%:%
%:%1726=871%:%
%:%1727=871%:%
%:%1728=872%:%
%:%1729=872%:%
%:%1730=873%:%
%:%1736=873%:%
%:%1739=874%:%
%:%1740=875%:%
%:%1741=875%:%
%:%1742=876%:%
%:%1749=877%:%
%:%1750=877%:%
%:%1751=878%:%
%:%1752=878%:%
%:%1753=879%:%
%:%1754=879%:%
%:%1755=879%:%
%:%1756=879%:%
%:%1757=880%:%
%:%1758=880%:%
%:%1759=881%:%
%:%1760=881%:%
%:%1761=882%:%
%:%1762=882%:%
%:%1763=882%:%
%:%1764=882%:%
%:%1765=883%:%
%:%1766=883%:%
%:%1767=884%:%
%:%1768=884%:%
%:%1769=885%:%
%:%1770=885%:%
%:%1771=885%:%
%:%1772=885%:%
%:%1773=886%:%
%:%1774=886%:%
%:%1775=887%:%
%:%1776=887%:%
%:%1777=888%:%
%:%1778=888%:%
%:%1779=888%:%
%:%1780=888%:%
%:%1781=889%:%
%:%1782=889%:%
%:%1783=890%:%
%:%1784=890%:%
%:%1785=891%:%
%:%1786=891%:%
%:%1787=891%:%
%:%1788=891%:%
%:%1789=892%:%
%:%1790=892%:%
%:%1791=893%:%
%:%1792=893%:%
%:%1793=894%:%
%:%1794=894%:%
%:%1795=894%:%
%:%1796=894%:%
%:%1797=895%:%
%:%1807=897%:%
%:%1808=898%:%
%:%1810=900%:%
%:%1811=900%:%
%:%1814=901%:%
%:%1818=901%:%
%:%1819=901%:%
%:%1828=903%:%
%:%1829=904%:%
%:%1830=905%:%
%:%1831=906%:%
%:%1832=907%:%
%:%1833=908%:%
%:%1834=909%:%
%:%1835=910%:%
%:%1836=911%:%
%:%1837=912%:%
%:%1838=913%:%
%:%1839=914%:%
%:%1840=915%:%
%:%1841=916%:%
%:%1842=917%:%
%:%1843=918%:%
%:%1844=919%:%
%:%1845=920%:%
%:%1846=921%:%
%:%1847=922%:%
%:%1848=923%:%
%:%1849=924%:%
%:%1850=925%:%
%:%1851=926%:%
%:%1852=927%:%
%:%1853=928%:%
%:%1854=929%:%
%:%1855=930%:%
%:%1856=931%:%
%:%1857=932%:%
%:%1858=933%:%
%:%1859=934%:%
%:%1860=935%:%
%:%1861=936%:%
%:%1862=937%:%
%:%1863=938%:%
%:%1864=939%:%
%:%1865=940%:%
%:%1866=941%:%
%:%1867=942%:%
%:%1868=943%:%
%:%1869=944%:%
%:%1870=945%:%
%:%1871=946%:%
%:%1872=947%:%
%:%1873=948%:%
%:%1874=949%:%
%:%1875=950%:%
%:%1876=951%:%
%:%1877=952%:%
%:%1878=953%:%
%:%1879=954%:%
%:%1880=955%:%
%:%1881=956%:%
%:%1882=957%:%
%:%1883=958%:%
%:%1884=959%:%
%:%1885=960%:%
%:%1886=961%:%
%:%1887=962%:%
%:%1888=963%:%
%:%1889=964%:%
%:%1890=965%:%
%:%1892=967%:%
%:%1893=967%:%
%:%1896=968%:%
%:%1900=968%:%
%:%1910=970%:%
%:%1912=972%:%
%:%1913=972%:%
%:%1914=973%:%
%:%1915=974%:%
%:%1922=975%:%
%:%1923=975%:%
%:%1924=976%:%
%:%1925=976%:%
%:%1926=977%:%
%:%1927=977%:%
%:%1928=978%:%
%:%1929=978%:%
%:%1930=979%:%
%:%1931=979%:%
%:%1932=979%:%
%:%1933=980%:%
%:%1934=980%:%
%:%1935=981%:%
%:%1936=981%:%
%:%1937=981%:%
%:%1938=982%:%
%:%1939=982%:%
%:%1940=983%:%
%:%1946=983%:%
%:%1949=984%:%
%:%1950=985%:%
%:%1951=985%:%
%:%1952=986%:%
%:%1953=987%:%
%:%1960=988%:%
%:%1961=988%:%
%:%1962=989%:%
%:%1963=989%:%
%:%1964=990%:%
%:%1965=990%:%
%:%1966=991%:%
%:%1967=991%:%
%:%1968=992%:%
%:%1969=992%:%
%:%1970=992%:%
%:%1971=993%:%
%:%1972=993%:%
%:%1973=994%:%
%:%1974=994%:%
%:%1975=994%:%
%:%1976=995%:%
%:%1977=995%:%
%:%1978=996%:%
%:%1984=996%:%
%:%1987=997%:%
%:%1988=998%:%
%:%1989=998%:%
%:%1990=999%:%
%:%1991=1000%:%
%:%1992=1001%:%
%:%1999=1002%:%
%:%2000=1002%:%
%:%2001=1003%:%
%:%2002=1003%:%
%:%2003=1004%:%
%:%2004=1004%:%
%:%2005=1005%:%
%:%2006=1005%:%
%:%2007=1006%:%
%:%2008=1006%:%
%:%2009=1006%:%
%:%2010=1007%:%
%:%2011=1007%:%
%:%2012=1008%:%
%:%2013=1008%:%
%:%2014=1008%:%
%:%2015=1009%:%
%:%2016=1009%:%
%:%2017=1010%:%
%:%2018=1010%:%
%:%2019=1011%:%
%:%2020=1011%:%
%:%2021=1011%:%
%:%2022=1012%:%
%:%2023=1012%:%
%:%2024=1013%:%
%:%2025=1013%:%
%:%2026=1013%:%
%:%2027=1014%:%
%:%2028=1014%:%
%:%2029=1015%:%
%:%2030=1015%:%
%:%2031=1016%:%
%:%2032=1016%:%
%:%2033=1017%:%
%:%2034=1017%:%
%:%2035=1017%:%
%:%2036=1018%:%
%:%2037=1018%:%
%:%2038=1019%:%
%:%2039=1019%:%
%:%2040=1019%:%
%:%2041=1020%:%
%:%2042=1020%:%
%:%2043=1021%:%
%:%2044=1021%:%
%:%2045=1021%:%
%:%2046=1022%:%
%:%2047=1022%:%
%:%2048=1023%:%
%:%2049=1023%:%
%:%2050=1024%:%
%:%2056=1024%:%
%:%2059=1025%:%
%:%2060=1026%:%
%:%2061=1026%:%
%:%2062=1027%:%
%:%2063=1028%:%
%:%2064=1029%:%
%:%2065=1030%:%
%:%2072=1031%:%
%:%2073=1031%:%
%:%2074=1032%:%
%:%2075=1032%:%
%:%2076=1033%:%
%:%2077=1033%:%
%:%2078=1034%:%
%:%2079=1034%:%
%:%2080=1035%:%
%:%2081=1035%:%
%:%2082=1035%:%
%:%2083=1036%:%
%:%2084=1036%:%
%:%2085=1037%:%
%:%2086=1037%:%
%:%2087=1037%:%
%:%2088=1038%:%
%:%2089=1038%:%
%:%2090=1039%:%
%:%2091=1039%:%
%:%2092=1040%:%
%:%2093=1040%:%
%:%2094=1040%:%
%:%2095=1041%:%
%:%2096=1041%:%
%:%2097=1042%:%
%:%2098=1042%:%
%:%2099=1042%:%
%:%2100=1043%:%
%:%2101=1043%:%
%:%2102=1044%:%
%:%2103=1044%:%
%:%2104=1045%:%
%:%2105=1045%:%
%:%2106=1046%:%
%:%2107=1046%:%
%:%2108=1046%:%
%:%2109=1047%:%
%:%2110=1047%:%
%:%2111=1048%:%
%:%2112=1048%:%
%:%2113=1048%:%
%:%2114=1049%:%
%:%2115=1049%:%
%:%2116=1050%:%
%:%2117=1050%:%
%:%2118=1051%:%
%:%2119=1051%:%
%:%2120=1051%:%
%:%2121=1052%:%
%:%2122=1052%:%
%:%2123=1053%:%
%:%2124=1053%:%
%:%2125=1053%:%
%:%2126=1054%:%
%:%2127=1054%:%
%:%2128=1055%:%
%:%2129=1055%:%
%:%2130=1055%:%
%:%2131=1056%:%
%:%2132=1056%:%
%:%2133=1057%:%
%:%2134=1057%:%
%:%2135=1057%:%
%:%2136=1058%:%
%:%2137=1058%:%
%:%2138=1059%:%
%:%2139=1059%:%
%:%2140=1060%:%
%:%2141=1060%:%
%:%2142=1061%:%
%:%2143=1061%:%
%:%2144=1061%:%
%:%2145=1062%:%
%:%2146=1062%:%
%:%2147=1063%:%
%:%2148=1063%:%
%:%2149=1063%:%
%:%2150=1064%:%
%:%2151=1064%:%
%:%2152=1065%:%
%:%2153=1065%:%
%:%2154=1065%:%
%:%2155=1066%:%
%:%2156=1066%:%
%:%2157=1067%:%
%:%2158=1067%:%
%:%2159=1067%:%
%:%2160=1068%:%
%:%2161=1068%:%
%:%2162=1069%:%
%:%2163=1069%:%
%:%2164=1070%:%
%:%2165=1070%:%
%:%2166=1071%:%
%:%2172=1071%:%
%:%2175=1072%:%
%:%2176=1073%:%
%:%2177=1073%:%
%:%2178=1074%:%
%:%2179=1075%:%
%:%2180=1076%:%
%:%2181=1077%:%
%:%2188=1078%:%
%:%2189=1078%:%
%:%2190=1079%:%
%:%2191=1079%:%
%:%2192=1080%:%
%:%2193=1080%:%
%:%2194=1081%:%
%:%2195=1081%:%
%:%2196=1082%:%
%:%2197=1082%:%
%:%2198=1082%:%
%:%2199=1083%:%
%:%2200=1083%:%
%:%2201=1084%:%
%:%2202=1084%:%
%:%2203=1084%:%
%:%2204=1085%:%
%:%2205=1085%:%
%:%2206=1086%:%
%:%2207=1086%:%
%:%2208=1087%:%
%:%2209=1087%:%
%:%2210=1087%:%
%:%2211=1088%:%
%:%2212=1088%:%
%:%2213=1089%:%
%:%2214=1089%:%
%:%2215=1089%:%
%:%2216=1090%:%
%:%2217=1090%:%
%:%2218=1091%:%
%:%2219=1091%:%
%:%2220=1092%:%
%:%2221=1092%:%
%:%2222=1093%:%
%:%2223=1093%:%
%:%2224=1093%:%
%:%2225=1094%:%
%:%2226=1094%:%
%:%2227=1095%:%
%:%2228=1095%:%
%:%2229=1095%:%
%:%2230=1096%:%
%:%2231=1096%:%
%:%2232=1097%:%
%:%2233=1097%:%
%:%2234=1098%:%
%:%2235=1098%:%
%:%2236=1098%:%
%:%2237=1099%:%
%:%2238=1099%:%
%:%2239=1100%:%
%:%2240=1100%:%
%:%2241=1100%:%
%:%2242=1101%:%
%:%2243=1101%:%
%:%2244=1102%:%
%:%2245=1102%:%
%:%2246=1102%:%
%:%2247=1103%:%
%:%2248=1103%:%
%:%2249=1104%:%
%:%2250=1104%:%
%:%2251=1104%:%
%:%2252=1105%:%
%:%2253=1105%:%
%:%2254=1106%:%
%:%2255=1106%:%
%:%2256=1107%:%
%:%2257=1107%:%
%:%2258=1108%:%
%:%2259=1108%:%
%:%2260=1108%:%
%:%2261=1109%:%
%:%2262=1109%:%
%:%2263=1110%:%
%:%2264=1110%:%
%:%2265=1110%:%
%:%2266=1111%:%
%:%2267=1111%:%
%:%2268=1112%:%
%:%2269=1112%:%
%:%2270=1112%:%
%:%2271=1113%:%
%:%2272=1113%:%
%:%2273=1114%:%
%:%2274=1114%:%
%:%2275=1114%:%
%:%2276=1115%:%
%:%2277=1115%:%
%:%2278=1116%:%
%:%2279=1116%:%
%:%2280=1117%:%
%:%2281=1117%:%
%:%2282=1118%:%
%:%2288=1118%:%
%:%2291=1119%:%
%:%2292=1120%:%
%:%2293=1120%:%
%:%2294=1121%:%
%:%2295=1122%:%
%:%2296=1123%:%
%:%2297=1124%:%
%:%2304=1125%:%
%:%2305=1125%:%
%:%2306=1126%:%
%:%2307=1126%:%
%:%2308=1127%:%
%:%2309=1127%:%
%:%2310=1128%:%
%:%2311=1128%:%
%:%2312=1129%:%
%:%2313=1129%:%
%:%2314=1129%:%
%:%2315=1130%:%
%:%2316=1130%:%
%:%2317=1131%:%
%:%2318=1131%:%
%:%2319=1131%:%
%:%2320=1132%:%
%:%2321=1132%:%
%:%2322=1133%:%
%:%2323=1133%:%
%:%2324=1134%:%
%:%2325=1134%:%
%:%2326=1134%:%
%:%2327=1135%:%
%:%2328=1135%:%
%:%2329=1136%:%
%:%2330=1136%:%
%:%2331=1136%:%
%:%2332=1137%:%
%:%2333=1137%:%
%:%2334=1138%:%
%:%2335=1138%:%
%:%2336=1139%:%
%:%2337=1139%:%
%:%2338=1140%:%
%:%2339=1140%:%
%:%2340=1140%:%
%:%2341=1141%:%
%:%2342=1141%:%
%:%2343=1142%:%
%:%2344=1142%:%
%:%2345=1142%:%
%:%2346=1143%:%
%:%2347=1143%:%
%:%2348=1144%:%
%:%2349=1144%:%
%:%2350=1145%:%
%:%2351=1145%:%
%:%2352=1145%:%
%:%2353=1146%:%
%:%2354=1146%:%
%:%2355=1147%:%
%:%2356=1147%:%
%:%2357=1147%:%
%:%2358=1148%:%
%:%2359=1148%:%
%:%2360=1149%:%
%:%2361=1149%:%
%:%2362=1149%:%
%:%2363=1150%:%
%:%2364=1150%:%
%:%2365=1151%:%
%:%2366=1151%:%
%:%2367=1151%:%
%:%2368=1152%:%
%:%2369=1152%:%
%:%2370=1153%:%
%:%2371=1153%:%
%:%2372=1154%:%
%:%2373=1154%:%
%:%2374=1155%:%
%:%2375=1155%:%
%:%2376=1155%:%
%:%2377=1156%:%
%:%2378=1156%:%
%:%2379=1157%:%
%:%2380=1157%:%
%:%2381=1157%:%
%:%2382=1158%:%
%:%2383=1158%:%
%:%2384=1159%:%
%:%2385=1159%:%
%:%2386=1159%:%
%:%2387=1160%:%
%:%2388=1160%:%
%:%2389=1161%:%
%:%2390=1161%:%
%:%2391=1161%:%
%:%2392=1162%:%
%:%2393=1162%:%
%:%2394=1163%:%
%:%2395=1163%:%
%:%2396=1164%:%
%:%2397=1164%:%
%:%2398=1165%:%
%:%2404=1165%:%
%:%2407=1166%:%
%:%2408=1167%:%
%:%2409=1167%:%
%:%2410=1168%:%
%:%2417=1169%:%
%:%2418=1169%:%
%:%2419=1170%:%
%:%2420=1170%:%
%:%2421=1171%:%
%:%2422=1171%:%
%:%2423=1171%:%
%:%2424=1171%:%
%:%2425=1172%:%
%:%2426=1172%:%
%:%2427=1173%:%
%:%2428=1173%:%
%:%2429=1174%:%
%:%2430=1174%:%
%:%2431=1174%:%
%:%2432=1174%:%
%:%2433=1175%:%
%:%2434=1175%:%
%:%2435=1176%:%
%:%2436=1176%:%
%:%2437=1177%:%
%:%2438=1177%:%
%:%2439=1177%:%
%:%2440=1177%:%
%:%2441=1178%:%
%:%2442=1178%:%
%:%2443=1179%:%
%:%2444=1179%:%
%:%2445=1180%:%
%:%2446=1180%:%
%:%2447=1180%:%
%:%2448=1180%:%
%:%2449=1181%:%
%:%2450=1181%:%
%:%2451=1182%:%
%:%2452=1182%:%
%:%2453=1183%:%
%:%2454=1183%:%
%:%2455=1183%:%
%:%2456=1183%:%
%:%2457=1184%:%
%:%2458=1184%:%
%:%2459=1185%:%
%:%2460=1185%:%
%:%2461=1186%:%
%:%2462=1186%:%
%:%2463=1186%:%
%:%2464=1186%:%
%:%2465=1187%:%
%:%2475=1189%:%
%:%2477=1191%:%
%:%2478=1191%:%
%:%2481=1192%:%
%:%2485=1192%:%
%:%2486=1192%:%
%:%2495=1194%:%
%:%2496=1195%:%
%:%2497=1196%:%
%:%2498=1197%:%
%:%2499=1198%:%
%:%2500=1199%:%
%:%2501=1200%:%
%:%2502=1201%:%
%:%2503=1202%:%
%:%2504=1203%:%
%:%2505=1204%:%
%:%2506=1205%:%
%:%2507=1206%:%
%:%2508=1207%:%
%:%2509=1208%:%
%:%2510=1209%:%
%:%2511=1210%:%
%:%2512=1211%:%
%:%2513=1212%:%
%:%2514=1213%:%
%:%2515=1214%:%
%:%2516=1215%:%
%:%2517=1216%:%
%:%2518=1217%:%
%:%2519=1218%:%
%:%2520=1219%:%
%:%2521=1220%:%
%:%2522=1221%:%
%:%2523=1222%:%
%:%2524=1223%:%
%:%2525=1224%:%
%:%2526=1225%:%
%:%2527=1226%:%
%:%2528=1227%:%
%:%2529=1228%:%
%:%2530=1229%:%
%:%2531=1230%:%
%:%2532=1231%:%
%:%2533=1232%:%
%:%2534=1233%:%
%:%2535=1234%:%
%:%2536=1235%:%
%:%2537=1236%:%
%:%2538=1237%:%
%:%2539=1238%:%
%:%2540=1239%:%
%:%2541=1240%:%
%:%2542=1241%:%
%:%2543=1242%:%
%:%2544=1243%:%
%:%2545=1244%:%
%:%2546=1245%:%
%:%2547=1246%:%
%:%2548=1247%:%
%:%2549=1248%:%
%:%2550=1249%:%
%:%2551=1250%:%
%:%2552=1251%:%
%:%2553=1252%:%
%:%2554=1253%:%
%:%2555=1254%:%
%:%2556=1255%:%
%:%2557=1256%:%
%:%2558=1257%:%
%:%2559=1258%:%
%:%2560=1259%:%
%:%2561=1260%:%
%:%2562=1261%:%
%:%2563=1262%:%
%:%2565=1264%:%
%:%2566=1264%:%
%:%2567=1265%:%
%:%2568=1266%:%
%:%2575=1267%:%
%:%2576=1267%:%
%:%2577=1268%:%
%:%2578=1268%:%
%:%2579=1268%:%
%:%2580=1269%:%
%:%2581=1269%:%
%:%2582=1270%:%
%:%2583=1270%:%
%:%2584=1270%:%
%:%2585=1271%:%
%:%2586=1271%:%
%:%2587=1272%:%
%:%2588=1272%:%
%:%2589=1272%:%
%:%2590=1273%:%
%:%2591=1273%:%
%:%2592=1274%:%
%:%2598=1274%:%
%:%2601=1275%:%
%:%2602=1276%:%
%:%2603=1276%:%
%:%2604=1277%:%
%:%2605=1278%:%
%:%2612=1279%:%
%:%2613=1279%:%
%:%2614=1280%:%
%:%2615=1280%:%
%:%2616=1281%:%
%:%2617=1281%:%
%:%2618=1282%:%
%:%2619=1282%:%
%:%2620=1283%:%
%:%2621=1283%:%
%:%2622=1283%:%
%:%2623=1284%:%
%:%2624=1284%:%
%:%2625=1285%:%
%:%2626=1285%:%
%:%2627=1285%:%
%:%2628=1286%:%
%:%2629=1286%:%
%:%2630=1287%:%
%:%2636=1287%:%
%:%2639=1288%:%
%:%2640=1289%:%
%:%2641=1289%:%
%:%2642=1290%:%
%:%2643=1291%:%
%:%2644=1292%:%
%:%2651=1293%:%
%:%2652=1293%:%
%:%2653=1294%:%
%:%2654=1294%:%
%:%2655=1295%:%
%:%2656=1295%:%
%:%2657=1296%:%
%:%2658=1296%:%
%:%2659=1297%:%
%:%2660=1297%:%
%:%2661=1297%:%
%:%2662=1298%:%
%:%2663=1298%:%
%:%2664=1299%:%
%:%2665=1299%:%
%:%2666=1299%:%
%:%2667=1300%:%
%:%2668=1300%:%
%:%2669=1301%:%
%:%2670=1301%:%
%:%2671=1302%:%
%:%2672=1302%:%
%:%2673=1302%:%
%:%2674=1303%:%
%:%2675=1303%:%
%:%2676=1304%:%
%:%2677=1304%:%
%:%2678=1304%:%
%:%2679=1305%:%
%:%2680=1305%:%
%:%2681=1306%:%
%:%2682=1306%:%
%:%2683=1307%:%
%:%2684=1307%:%
%:%2685=1308%:%
%:%2686=1308%:%
%:%2687=1308%:%
%:%2688=1309%:%
%:%2689=1309%:%
%:%2690=1310%:%
%:%2691=1310%:%
%:%2692=1310%:%
%:%2693=1311%:%
%:%2694=1311%:%
%:%2695=1312%:%
%:%2696=1312%:%
%:%2697=1312%:%
%:%2698=1313%:%
%:%2699=1313%:%
%:%2700=1314%:%
%:%2701=1314%:%
%:%2702=1315%:%
%:%2708=1315%:%
%:%2711=1316%:%
%:%2712=1317%:%
%:%2713=1317%:%
%:%2714=1318%:%
%:%2715=1319%:%
%:%2716=1320%:%
%:%2717=1321%:%
%:%2718=1322%:%
%:%2719=1323%:%
%:%2726=1324%:%
%:%2727=1324%:%
%:%2728=1325%:%
%:%2729=1325%:%
%:%2730=1326%:%
%:%2731=1326%:%
%:%2732=1327%:%
%:%2733=1327%:%
%:%2734=1328%:%
%:%2735=1328%:%
%:%2736=1328%:%
%:%2737=1329%:%
%:%2738=1329%:%
%:%2739=1330%:%
%:%2740=1330%:%
%:%2741=1330%:%
%:%2742=1331%:%
%:%2743=1331%:%
%:%2744=1332%:%
%:%2745=1332%:%
%:%2746=1333%:%
%:%2747=1333%:%
%:%2748=1333%:%
%:%2749=1334%:%
%:%2750=1334%:%
%:%2751=1335%:%
%:%2752=1335%:%
%:%2753=1335%:%
%:%2754=1336%:%
%:%2755=1336%:%
%:%2756=1337%:%
%:%2757=1337%:%
%:%2758=1338%:%
%:%2759=1338%:%
%:%2760=1339%:%
%:%2761=1339%:%
%:%2762=1339%:%
%:%2763=1340%:%
%:%2764=1340%:%
%:%2765=1341%:%
%:%2766=1341%:%
%:%2767=1341%:%
%:%2768=1342%:%
%:%2769=1342%:%
%:%2770=1343%:%
%:%2771=1343%:%
%:%2772=1344%:%
%:%2773=1344%:%
%:%2774=1344%:%
%:%2775=1345%:%
%:%2776=1345%:%
%:%2777=1346%:%
%:%2778=1346%:%
%:%2779=1346%:%
%:%2780=1347%:%
%:%2781=1347%:%
%:%2782=1348%:%
%:%2783=1348%:%
%:%2784=1348%:%
%:%2785=1349%:%
%:%2786=1349%:%
%:%2787=1350%:%
%:%2788=1350%:%
%:%2789=1350%:%
%:%2790=1351%:%
%:%2791=1351%:%
%:%2792=1352%:%
%:%2793=1352%:%
%:%2794=1353%:%
%:%2795=1353%:%
%:%2796=1354%:%
%:%2797=1354%:%
%:%2798=1354%:%
%:%2799=1355%:%
%:%2800=1355%:%
%:%2801=1356%:%
%:%2802=1356%:%
%:%2803=1356%:%
%:%2804=1357%:%
%:%2805=1357%:%
%:%2806=1358%:%
%:%2807=1358%:%
%:%2808=1358%:%
%:%2809=1359%:%
%:%2810=1359%:%
%:%2811=1360%:%
%:%2812=1360%:%
%:%2813=1360%:%
%:%2814=1361%:%
%:%2815=1361%:%
%:%2816=1362%:%
%:%2817=1362%:%
%:%2818=1363%:%
%:%2819=1363%:%
%:%2820=1364%:%
%:%2826=1364%:%
%:%2829=1365%:%
%:%2830=1366%:%
%:%2831=1366%:%
%:%2832=1367%:%
%:%2833=1368%:%
%:%2834=1369%:%
%:%2835=1370%:%
%:%2836=1371%:%
%:%2837=1372%:%
%:%2844=1373%:%
%:%2845=1373%:%
%:%2846=1374%:%
%:%2847=1374%:%
%:%2848=1375%:%
%:%2849=1375%:%
%:%2850=1376%:%
%:%2851=1376%:%
%:%2852=1377%:%
%:%2853=1377%:%
%:%2854=1377%:%
%:%2855=1378%:%
%:%2856=1378%:%
%:%2857=1379%:%
%:%2858=1379%:%
%:%2859=1379%:%
%:%2860=1380%:%
%:%2861=1380%:%
%:%2862=1381%:%
%:%2863=1381%:%
%:%2864=1382%:%
%:%2865=1382%:%
%:%2866=1382%:%
%:%2867=1383%:%
%:%2868=1383%:%
%:%2869=1384%:%
%:%2870=1384%:%
%:%2871=1384%:%
%:%2872=1385%:%
%:%2873=1385%:%
%:%2874=1386%:%
%:%2875=1386%:%
%:%2876=1387%:%
%:%2877=1387%:%
%:%2878=1388%:%
%:%2879=1388%:%
%:%2880=1388%:%
%:%2881=1389%:%
%:%2882=1389%:%
%:%2883=1390%:%
%:%2884=1390%:%
%:%2885=1390%:%
%:%2886=1391%:%
%:%2887=1391%:%
%:%2888=1392%:%
%:%2889=1392%:%
%:%2890=1393%:%
%:%2891=1393%:%
%:%2892=1393%:%
%:%2893=1394%:%
%:%2894=1394%:%
%:%2895=1395%:%
%:%2896=1395%:%
%:%2897=1395%:%
%:%2898=1396%:%
%:%2899=1396%:%
%:%2900=1397%:%
%:%2901=1397%:%
%:%2902=1397%:%
%:%2903=1398%:%
%:%2904=1398%:%
%:%2905=1399%:%
%:%2906=1399%:%
%:%2907=1399%:%
%:%2908=1400%:%
%:%2909=1400%:%
%:%2910=1401%:%
%:%2911=1401%:%
%:%2912=1402%:%
%:%2913=1402%:%
%:%2914=1403%:%
%:%2915=1403%:%
%:%2916=1403%:%
%:%2917=1404%:%
%:%2918=1404%:%
%:%2919=1405%:%
%:%2920=1405%:%
%:%2921=1405%:%
%:%2922=1406%:%
%:%2923=1406%:%
%:%2924=1407%:%
%:%2925=1407%:%
%:%2926=1407%:%
%:%2927=1408%:%
%:%2928=1408%:%
%:%2929=1409%:%
%:%2930=1409%:%
%:%2931=1409%:%
%:%2932=1410%:%
%:%2933=1410%:%
%:%2934=1411%:%
%:%2935=1411%:%
%:%2936=1412%:%
%:%2937=1412%:%
%:%2938=1413%:%
%:%2944=1413%:%
%:%2947=1414%:%
%:%2948=1415%:%
%:%2949=1415%:%
%:%2950=1416%:%
%:%2951=1417%:%
%:%2952=1418%:%
%:%2953=1419%:%
%:%2954=1420%:%
%:%2955=1421%:%
%:%2962=1422%:%
%:%2963=1422%:%
%:%2964=1423%:%
%:%2965=1423%:%
%:%2966=1424%:%
%:%2967=1424%:%
%:%2968=1425%:%
%:%2969=1425%:%
%:%2970=1426%:%
%:%2971=1426%:%
%:%2972=1426%:%
%:%2973=1427%:%
%:%2974=1427%:%
%:%2975=1428%:%
%:%2976=1428%:%
%:%2977=1428%:%
%:%2978=1429%:%
%:%2979=1429%:%
%:%2980=1430%:%
%:%2981=1430%:%
%:%2982=1431%:%
%:%2983=1431%:%
%:%2984=1431%:%
%:%2985=1432%:%
%:%2986=1432%:%
%:%2987=1433%:%
%:%2988=1433%:%
%:%2989=1433%:%
%:%2990=1434%:%
%:%2991=1434%:%
%:%2992=1435%:%
%:%2993=1435%:%
%:%2994=1436%:%
%:%2995=1436%:%
%:%2996=1437%:%
%:%2997=1437%:%
%:%2998=1437%:%
%:%2999=1438%:%
%:%3000=1438%:%
%:%3001=1439%:%
%:%3002=1439%:%
%:%3003=1439%:%
%:%3004=1440%:%
%:%3005=1440%:%
%:%3006=1441%:%
%:%3007=1441%:%
%:%3008=1442%:%
%:%3009=1442%:%
%:%3010=1442%:%
%:%3011=1443%:%
%:%3012=1443%:%
%:%3013=1444%:%
%:%3014=1444%:%
%:%3015=1444%:%
%:%3016=1445%:%
%:%3017=1445%:%
%:%3018=1446%:%
%:%3019=1446%:%
%:%3020=1446%:%
%:%3021=1447%:%
%:%3022=1447%:%
%:%3023=1448%:%
%:%3024=1448%:%
%:%3025=1448%:%
%:%3026=1449%:%
%:%3027=1449%:%
%:%3028=1450%:%
%:%3029=1450%:%
%:%3030=1451%:%
%:%3031=1451%:%
%:%3032=1452%:%
%:%3033=1452%:%
%:%3034=1452%:%
%:%3035=1453%:%
%:%3036=1453%:%
%:%3037=1454%:%
%:%3038=1454%:%
%:%3039=1454%:%
%:%3040=1455%:%
%:%3041=1455%:%
%:%3042=1456%:%
%:%3043=1456%:%
%:%3044=1456%:%
%:%3045=1457%:%
%:%3046=1457%:%
%:%3047=1458%:%
%:%3048=1458%:%
%:%3049=1458%:%
%:%3050=1459%:%
%:%3051=1459%:%
%:%3052=1460%:%
%:%3053=1460%:%
%:%3054=1461%:%
%:%3055=1461%:%
%:%3056=1462%:%
%:%3062=1462%:%
%:%3065=1463%:%
%:%3066=1464%:%
%:%3067=1464%:%
%:%3068=1465%:%
%:%3075=1466%:%
%:%3076=1466%:%
%:%3077=1467%:%
%:%3078=1467%:%
%:%3079=1468%:%
%:%3080=1468%:%
%:%3081=1468%:%
%:%3082=1468%:%
%:%3083=1469%:%
%:%3084=1469%:%
%:%3085=1470%:%
%:%3086=1470%:%
%:%3087=1471%:%
%:%3088=1471%:%
%:%3089=1471%:%
%:%3090=1471%:%
%:%3091=1472%:%
%:%3092=1472%:%
%:%3093=1473%:%
%:%3094=1473%:%
%:%3095=1474%:%
%:%3096=1474%:%
%:%3097=1474%:%
%:%3098=1474%:%
%:%3099=1475%:%
%:%3100=1475%:%
%:%3101=1476%:%
%:%3102=1476%:%
%:%3103=1477%:%
%:%3104=1477%:%
%:%3105=1477%:%
%:%3106=1477%:%
%:%3107=1478%:%
%:%3108=1478%:%
%:%3109=1479%:%
%:%3110=1479%:%
%:%3111=1480%:%
%:%3112=1480%:%
%:%3113=1480%:%
%:%3114=1480%:%
%:%3115=1481%:%
%:%3116=1481%:%
%:%3117=1482%:%
%:%3118=1482%:%
%:%3119=1483%:%
%:%3120=1483%:%
%:%3121=1483%:%
%:%3122=1483%:%
%:%3123=1484%:%
%:%3133=1486%:%
%:%3135=1488%:%
%:%3136=1488%:%
%:%3137=1489%:%
%:%3140=1490%:%
%:%3144=1490%:%
%:%3145=1490%:%
%:%3154=1492%:%
%:%3155=1493%:%
%:%3156=1494%:%
%:%3157=1495%:%
%:%3158=1496%:%
%:%3159=1497%:%
%:%3160=1498%:%
%:%3161=1499%:%
%:%3162=1500%:%
%:%3163=1501%:%
%:%3164=1502%:%
%:%3165=1503%:%
%:%3166=1504%:%
%:%3167=1505%:%
%:%3168=1506%:%
%:%3169=1507%:%
%:%3170=1508%:%
%:%3171=1509%:%
%:%3172=1510%:%
%:%3173=1511%:%
%:%3174=1512%:%
%:%3175=1513%:%
%:%3176=1514%:%
%:%3178=1516%:%
%:%3179=1516%:%
%:%3180=1517%:%
%:%3181=1518%:%
%:%3182=1519%:%
%:%3189=1520%:%
%:%3190=1520%:%
%:%3191=1521%:%
%:%3192=1521%:%
%:%3193=1522%:%
%:%3194=1522%:%
%:%3195=1523%:%
%:%3196=1523%:%
%:%3197=1524%:%
%:%3198=1524%:%
%:%3199=1525%:%
%:%3200=1525%:%
%:%3201=1526%:%
%:%3202=1526%:%
%:%3203=1527%:%
%:%3204=1527%:%
%:%3205=1527%:%
%:%3206=1528%:%
%:%3207=1528%:%
%:%3208=1529%:%
%:%3209=1529%:%
%:%3210=1530%:%
%:%3211=1530%:%
%:%3212=1531%:%
%:%3213=1531%:%
%:%3214=1532%:%
%:%3215=1532%:%
%:%3216=1532%:%
%:%3217=1533%:%
%:%3218=1533%:%
%:%3219=1534%:%
%:%3220=1534%:%
%:%3221=1535%:%
%:%3231=1537%:%
%:%3233=1539%:%
%:%3234=1539%:%
%:%3235=1540%:%
%:%3237=1542%:%
%:%3240=1543%:%
%:%3244=1543%:%
%:%3245=1543%:%
%:%3254=1545%:%
%:%3255=1546%:%
%:%3259=1548%:%
%:%3260=1549%:%
%:%3262=1551%:%
%:%3263=1551%:%
%:%3264=1552%:%
%:%3265=1553%:%
%:%3272=1554%:%
%:%3273=1554%:%
%:%3274=1555%:%
%:%3275=1555%:%
%:%3276=1556%:%
%:%3277=1556%:%
%:%3278=1557%:%
%:%3279=1557%:%
%:%3280=1557%:%
%:%3281=1558%:%
%:%3282=1558%:%
%:%3283=1559%:%
%:%3284=1559%:%
%:%3285=1559%:%
%:%3286=1560%:%
%:%3287=1560%:%
%:%3288=1561%:%
%:%3289=1561%:%
%:%3290=1561%:%
%:%3291=1562%:%
%:%3292=1562%:%
%:%3293=1563%:%
%:%3299=1563%:%
%:%3302=1564%:%
%:%3303=1565%:%
%:%3304=1565%:%
%:%3305=1566%:%
%:%3306=1567%:%
%:%3313=1568%:%
%:%3314=1568%:%
%:%3315=1569%:%
%:%3316=1569%:%
%:%3317=1570%:%
%:%3318=1570%:%
%:%3319=1571%:%
%:%3320=1571%:%
%:%3321=1571%:%
%:%3322=1572%:%
%:%3323=1572%:%
%:%3324=1573%:%
%:%3325=1573%:%
%:%3326=1574%:%
%:%3327=1574%:%
%:%3328=1575%:%
%:%3329=1575%:%
%:%3330=1576%:%
%:%3331=1576%:%
%:%3332=1576%:%
%:%3333=1577%:%
%:%3334=1577%:%
%:%3335=1578%:%
%:%3341=1578%:%
%:%3344=1579%:%
%:%3345=1580%:%
%:%3346=1580%:%
%:%3347=1581%:%
%:%3348=1582%:%
%:%3355=1583%:%
%:%3356=1583%:%
%:%3357=1584%:%
%:%3358=1584%:%
%:%3359=1585%:%
%:%3360=1585%:%
%:%3361=1586%:%
%:%3362=1586%:%
%:%3363=1586%:%
%:%3364=1587%:%
%:%3365=1587%:%
%:%3366=1588%:%
%:%3367=1588%:%
%:%3368=1589%:%
%:%3369=1589%:%
%:%3370=1590%:%
%:%3371=1590%:%
%:%3372=1591%:%
%:%3373=1591%:%
%:%3374=1591%:%
%:%3375=1592%:%
%:%3376=1592%:%
%:%3377=1593%:%
%:%3383=1593%:%
%:%3386=1594%:%
%:%3387=1595%:%
%:%3388=1595%:%
%:%3389=1596%:%
%:%3390=1597%:%
%:%3397=1598%:%
%:%3398=1598%:%
%:%3399=1599%:%
%:%3400=1599%:%
%:%3401=1600%:%
%:%3402=1600%:%
%:%3403=1601%:%
%:%3404=1601%:%
%:%3405=1601%:%
%:%3406=1602%:%
%:%3407=1602%:%
%:%3408=1603%:%
%:%3409=1603%:%
%:%3410=1604%:%
%:%3411=1604%:%
%:%3412=1605%:%
%:%3413=1605%:%
%:%3414=1606%:%
%:%3415=1606%:%
%:%3416=1606%:%
%:%3417=1607%:%
%:%3418=1607%:%
%:%3419=1608%:%
%:%3425=1608%:%
%:%3428=1609%:%
%:%3429=1610%:%
%:%3430=1610%:%
%:%3431=1611%:%
%:%3432=1612%:%
%:%3433=1613%:%
%:%3434=1614%:%
%:%3441=1616%:%
%:%3453=1618%:%
%:%3454=1619%:%
%:%3455=1620%:%
%:%3456=1621%:%
%:%3457=1622%:%
%:%3458=1623%:%
%:%3459=1624%:%
%:%3460=1625%:%
%:%3461=1626%:%
%:%3463=1628%:%
%:%3464=1628%:%
%:%3465=1629%:%
%:%3467=1631%:%
%:%3468=1632%:%
%:%3469=1633%:%
%:%3470=1634%:%
%:%3471=1635%:%
%:%3472=1636%:%
%:%3473=1637%:%
%:%3475=1639%:%
%:%3476=1639%:%
%:%3479=1640%:%
%:%3483=1640%:%
%:%3484=1640%:%
%:%3493=1642%:%
%:%3494=1643%:%
%:%3495=1644%:%
%:%3496=1645%:%
%:%3497=1646%:%
%:%3498=1647%:%
%:%3499=1648%:%
%:%3500=1649%:%
%:%3501=1650%:%
%:%3502=1651%:%
%:%3503=1652%:%
%:%3504=1653%:%
%:%3505=1654%:%
%:%3506=1655%:%
%:%3507=1656%:%
%:%3508=1657%:%
%:%3509=1658%:%
%:%3510=1659%:%
%:%3511=1660%:%
%:%3512=1661%:%
%:%3513=1662%:%
%:%3514=1663%:%
%:%3515=1664%:%
%:%3516=1665%:%
%:%3517=1666%:%
%:%3518=1667%:%
%:%3519=1668%:%
%:%3520=1669%:%
%:%3521=1670%:%
%:%3522=1671%:%
%:%3523=1672%:%
%:%3524=1673%:%
%:%3525=1674%:%
%:%3526=1675%:%
%:%3527=1676%:%
%:%3528=1677%:%
%:%3529=1678%:%
%:%3530=1679%:%
%:%3531=1680%:%
%:%3532=1681%:%
%:%3533=1682%:%
%:%3534=1683%:%
%:%3535=1684%:%
%:%3536=1685%:%
%:%3538=1687%:%
%:%3539=1687%:%
%:%3540=1688%:%
%:%3541=1689%:%
%:%3542=1690%:%
%:%3543=1691%:%
%:%3544=1691%:%
%:%3545=1692%:%
%:%3546=1693%:%
%:%3548=1695%:%
%:%3549=1696%:%
%:%3550=1697%:%
%:%3551=1698%:%
%:%3552=1699%:%
%:%3553=1700%:%
%:%3554=1701%:%
%:%3555=1702%:%
%:%3556=1703%:%
%:%3557=1704%:%
%:%3558=1705%:%
%:%3559=1706%:%
%:%3560=1707%:%
%:%3561=1708%:%
%:%3562=1709%:%
%:%3563=1710%:%
%:%3564=1711%:%
%:%3565=1712%:%
%:%3566=1713%:%
%:%3567=1714%:%
%:%3568=1715%:%
%:%3569=1716%:%
%:%3570=1717%:%
%:%3571=1718%:%
%:%3572=1719%:%
%:%3573=1720%:%
%:%3574=1721%:%
%:%3575=1722%:%
%:%3576=1723%:%
%:%3577=1724%:%
%:%3578=1725%:%
%:%3579=1726%:%
%:%3580=1727%:%
%:%3581=1728%:%
%:%3582=1729%:%
%:%3583=1730%:%
%:%3584=1731%:%
%:%3585=1732%:%
%:%3586=1733%:%
%:%3587=1734%:%
%:%3588=1735%:%
%:%3589=1736%:%
%:%3590=1737%:%
%:%3591=1738%:%
%:%3592=1739%:%
%:%3593=1740%:%
%:%3594=1741%:%
%:%3595=1742%:%
%:%3596=1743%:%
%:%3598=1745%:%
%:%3599=1745%:%
%:%3600=1746%:%
%:%3603=1747%:%
%:%3607=1747%:%
%:%3617=1749%:%
%:%3618=1750%:%
%:%3619=1751%:%
%:%3620=1752%:%
%:%3621=1753%:%
%:%3622=1754%:%
%:%3623=1755%:%
%:%3624=1756%:%
%:%3625=1757%:%
%:%3626=1758%:%
%:%3627=1759%:%
%:%3628=1760%:%
%:%3629=1761%:%
%:%3630=1762%:%
%:%3631=1763%:%
%:%3633=1765%:%
%:%3634=1765%:%
%:%3635=1766%:%
%:%3642=1767%:%
%:%3643=1767%:%
%:%3644=1768%:%
%:%3645=1768%:%
%:%3646=1769%:%
%:%3647=1769%:%
%:%3648=1770%:%
%:%3649=1770%:%
%:%3650=1770%:%
%:%3651=1771%:%
%:%3652=1771%:%
%:%3653=1772%:%
%:%3654=1772%:%
%:%3655=1772%:%
%:%3656=1773%:%
%:%3657=1773%:%
%:%3658=1774%:%
%:%3659=1774%:%
%:%3660=1774%:%
%:%3661=1775%:%
%:%3662=1775%:%
%:%3663=1776%:%
%:%3664=1776%:%
%:%3665=1776%:%
%:%3666=1777%:%
%:%3667=1777%:%
%:%3668=1778%:%
%:%3669=1778%:%
%:%3670=1778%:%
%:%3671=1779%:%
%:%3672=1779%:%
%:%3673=1780%:%
%:%3674=1780%:%
%:%3675=1780%:%
%:%3676=1781%:%
%:%3677=1781%:%
%:%3678=1782%:%
%:%3688=1784%:%
%:%3689=1785%:%
%:%3691=1787%:%
%:%3692=1787%:%
%:%3695=1788%:%
%:%3699=1788%:%
%:%3700=1788%:%
%:%3701=1789%:%
%:%3710=1791%:%
%:%3711=1792%:%
%:%3712=1793%:%
%:%3713=1794%:%
%:%3714=1795%:%
%:%3715=1796%:%
%:%3716=1797%:%
%:%3717=1798%:%
%:%3718=1799%:%
%:%3719=1800%:%
%:%3720=1801%:%
%:%3721=1802%:%
%:%3722=1803%:%
%:%3723=1804%:%
%:%3724=1805%:%
%:%3725=1806%:%
%:%3726=1807%:%
%:%3727=1808%:%
%:%3728=1809%:%
%:%3729=1810%:%
%:%3730=1811%:%
%:%3731=1812%:%
%:%3732=1813%:%
%:%3733=1814%:%
%:%3735=1816%:%
%:%3736=1816%:%
%:%3737=1817%:%
%:%3738=1818%:%
%:%3745=1819%:%
%:%3746=1819%:%
%:%3747=1820%:%
%:%3748=1820%:%
%:%3749=1821%:%
%:%3750=1821%:%
%:%3751=1822%:%
%:%3752=1822%:%
%:%3753=1822%:%
%:%3754=1823%:%
%:%3755=1823%:%
%:%3756=1824%:%
%:%3757=1824%:%
%:%3758=1824%:%
%:%3759=1825%:%
%:%3760=1825%:%
%:%3761=1826%:%
%:%3762=1826%:%
%:%3763=1826%:%
%:%3764=1827%:%
%:%3765=1827%:%
%:%3766=1828%:%
%:%3767=1828%:%
%:%3768=1828%:%
%:%3769=1829%:%
%:%3770=1829%:%
%:%3771=1830%:%
%:%3772=1830%:%
%:%3773=1830%:%
%:%3774=1831%:%
%:%3775=1831%:%
%:%3776=1832%:%
%:%3786=1834%:%
%:%3788=1836%:%
%:%3789=1836%:%
%:%3796=1837%:%
%:%3797=1837%:%
%:%3798=1838%:%
%:%3799=1838%:%
%:%3800=1839%:%
%:%3801=1839%:%
%:%3802=1839%:%
%:%3803=1839%:%
%:%3804=1840%:%
%:%3805=1840%:%
%:%3806=1841%:%
%:%3807=1841%:%
%:%3808=1842%:%
%:%3809=1842%:%
%:%3810=1843%:%
%:%3811=1843%:%
%:%3812=1843%:%
%:%3813=1844%:%
%:%3814=1844%:%
%:%3815=1845%:%
%:%3825=1847%:%
%:%3827=1849%:%
%:%3828=1849%:%
%:%3829=1850%:%
%:%3832=1851%:%
%:%3836=1851%:%
%:%3837=1851%:%