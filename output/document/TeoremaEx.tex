%
\begin{isabellebody}%
\setisabellecontext{TeoremaEx}%
%
\isadelimtheory
%
\endisadelimtheory
%
\isatagtheory
%
\endisatagtheory
{\isafoldtheory}%
%
\isadelimtheory
%
\endisadelimtheory
%
\begin{isamarkuptext}%
\comentario{Añadir introducción.}%
\end{isamarkuptext}\isamarkuptrue%
%
\begin{isamarkuptext}%
\comentario{Cambiar referencias de los lemas tras el cambio de índice.}%
\end{isamarkuptext}\isamarkuptrue%
%
\isadelimdocument
%
\endisadelimdocument
%
\isatagdocument
%
\isamarkupsection{Sucesiones de conjuntos de una colección%
}
\isamarkuptrue%
%
\endisatagdocument
{\isafolddocument}%
%
\isadelimdocument
%
\endisadelimdocument
%
\begin{isamarkuptext}%
En este apartado daremos una introducción sobre sucesiones de conjuntos de fórmulas a 
  partir de una colección y un conjunto de la misma. De este modo, se mostrarán distintas 
  características sobre las sucesiones y se definirá su límite. En la siguiente sección 
  probaremos que dicho límite constituye un conjunto satisfacible por el lema de Hintikka.

\comentario{Revisar el párrafo anterior al final}

  Recordemos que el conjunto de las fórmulas proposicionales se define recursivamente a partir 
  de un alfabeto numerable de variables proposicionales. Por lo tanto, el conjunto de fórmulas 
  proposicionales es igualmente numerable, de modo que es posible enumerar sus elementos. Una vez 
  realizada esta observación, veamos la definición de sucesión de conjuntos de fórmulas 
  proposicionales a partir de una colección y un conjunto de la misma.

\begin{definicion}
  Sea \isa{C} una colección, \isa{S\ {\isasymin}\ C} y \isa{F\isactrlsub {\isadigit{1}}{\isacharcomma}\ F\isactrlsub {\isadigit{2}}{\isacharcomma}\ F\isactrlsub {\isadigit{3}}\ {\isasymdots}} una enumeración de 
  las fórmulas proposicionales. Se define la \isa{sucesión\ de\ conjuntos\ de\ C\ a\ partir\ de\ S} como sigue:

  $S_{0} = S$

  $S_{n+1} = \left\{ \begin{array}{lcc} S_{n} \cup \{F_{n}\} &  si  & S_{n} \cup \{F_{n}\} \in C \\ \\ S_{n} & c.c \end{array} \right.$ 
\end{definicion}

  Para su formalización en Isabelle se ha introducido una instancia en la teoría de \isa{Sintaxis} que 
  indica explícitamente que el conjunto de las fórmulas proposicionales es numerable, probado
  mediante el método \isa{countable{\isacharunderscore}datatype} de Isabelle.

  \isa{instance\ formula\ {\isacharcolon}{\isacharcolon}\ {\isacharparenleft}countable{\isacharparenright}\ countable\ by\ countable{\isacharunderscore}datatype}

  De esta manera se genera en Isabelle una enumeración predeterminada de los elementos del conjunto,
  junto con herramientas para probar propiedades referentes a la numerabilidad. En particular, en la 
  formalización de la definición \isa{{\isadigit{4}}{\isachardot}{\isadigit{1}}{\isachardot}{\isadigit{1}}} se utilizará la función \isa{from{\isacharunderscore}nat} que, al aplicarla a un 
  número natural \isa{n}, nos devuelve la \isa{n}-ésima fórmula proposicional según una enumeración 
  predeterminada en Isabelle. 

  Puesto que la definición de las sucesiones en \isa{{\isadigit{4}}{\isachardot}{\isadigit{1}}{\isachardot}{\isadigit{1}}} se trata de una definición 
  recursiva sobre la estructura recursiva de los números naturales, se formalizará en Isabelle
  mediante el tipo de funciones primitivas recursivas de la siguiente manera.%
\end{isamarkuptext}\isamarkuptrue%
\isacommand{primrec}\isamarkupfalse%
\ pcp{\isacharunderscore}seq\ \isakeyword{where}\isanewline
{\isachardoublequoteopen}pcp{\isacharunderscore}seq\ C\ S\ {\isadigit{0}}\ {\isacharequal}\ S{\isachardoublequoteclose}\ {\isacharbar}\isanewline
{\isachardoublequoteopen}pcp{\isacharunderscore}seq\ C\ S\ {\isacharparenleft}Suc\ n{\isacharparenright}\ {\isacharequal}\ {\isacharparenleft}let\ Sn\ {\isacharequal}\ pcp{\isacharunderscore}seq\ C\ S\ n{\isacharsemicolon}\ Sn{\isadigit{1}}\ {\isacharequal}\ insert\ {\isacharparenleft}from{\isacharunderscore}nat\ n{\isacharparenright}\ Sn\ in\isanewline
\ \ \ \ \ \ \ \ \ \ \ \ \ \ \ \ \ \ \ \ \ \ \ \ if\ Sn{\isadigit{1}}\ {\isasymin}\ C\ then\ Sn{\isadigit{1}}\ else\ Sn{\isacharparenright}{\isachardoublequoteclose}%
\begin{isamarkuptext}%
Veamos el primer resultado sobre dichas sucesiones.

  \begin{lema}
    Sea \isa{C} una colección de conjuntos con la propiedad de consistencia proposicional,\\ \isa{S\ {\isasymin}\ C} y 
    \isa{{\isacharbraceleft}S\isactrlsub n{\isacharbraceright}} la sucesión de conjuntos de \isa{C} a partir de \isa{S} construida según la definición \isa{{\isadigit{4}}{\isachardot}{\isadigit{1}}{\isachardot}{\isadigit{1}}}. 
    Entonces, para todo \isa{n\ {\isasymin}\ {\isasymnat}} se verifica que \isa{S\isactrlsub n\ {\isasymin}\ C}.
  \end{lema}

  Procedamos con su demostración.

  \begin{demostracion}
    El resultado se prueba por inducción en los números naturales que conforman los subíndices de la 
    sucesión.

    En primer lugar, tenemos que \isa{S\isactrlsub {\isadigit{0}}\ {\isacharequal}\ S} pertenece a \isa{C} por hipótesis.

    Por otro lado, supongamos que \isa{S\isactrlsub n\ {\isasymin}\ C}. Probemos que \isa{S\isactrlsub n\isactrlsub {\isacharplus}\isactrlsub {\isadigit{1}}\ {\isasymin}\ C}. Si suponemos que \isa{S\isactrlsub n\ {\isasymunion}\ {\isacharbraceleft}F\isactrlsub n{\isacharbraceright}\ {\isasymin}\ C},
    por definición tenemos que \isa{S\isactrlsub n\isactrlsub {\isacharplus}\isactrlsub {\isadigit{1}}\ {\isacharequal}\ S\isactrlsub n\ {\isasymunion}\ {\isacharbraceleft}F\isactrlsub n{\isacharbraceright}}, luego pertenece a \isa{C}. En caso contrario, si
    suponemos que \isa{S\isactrlsub n\ {\isasymunion}\ {\isacharbraceleft}F\isactrlsub n{\isacharbraceright}\ {\isasymnotin}\ C}, por definición tenemos que \isa{S\isactrlsub n\isactrlsub {\isacharplus}\isactrlsub {\isadigit{1}}\ {\isacharequal}\ S\isactrlsub n}, que pertenece igualmente
    a \isa{C} por hipótesis de inducción. Por tanto, queda probado el resultado.
  \end{demostracion}

  La formalización y demostración detallada del lema en Isabelle son las siguientes.%
\end{isamarkuptext}\isamarkuptrue%
\isacommand{lemma}\isamarkupfalse%
\ \isanewline
\ \ \isakeyword{assumes}\ {\isachardoublequoteopen}pcp\ C{\isachardoublequoteclose}\ \isanewline
\ \ \ \ \ \ \ \ \ \ {\isachardoublequoteopen}S\ {\isasymin}\ C{\isachardoublequoteclose}\isanewline
\ \ \ \ \ \ \ \ \isakeyword{shows}\ {\isachardoublequoteopen}pcp{\isacharunderscore}seq\ C\ S\ n\ {\isasymin}\ C{\isachardoublequoteclose}\isanewline
%
\isadelimproof
%
\endisadelimproof
%
\isatagproof
\isacommand{proof}\isamarkupfalse%
\ {\isacharparenleft}induction\ n{\isacharparenright}\isanewline
\ \ \isacommand{show}\isamarkupfalse%
\ {\isachardoublequoteopen}pcp{\isacharunderscore}seq\ C\ S\ {\isadigit{0}}\ {\isasymin}\ C{\isachardoublequoteclose}\isanewline
\ \ \ \ \isacommand{by}\isamarkupfalse%
\ {\isacharparenleft}simp\ only{\isacharcolon}\ pcp{\isacharunderscore}seq{\isachardot}simps{\isacharparenleft}{\isadigit{1}}{\isacharparenright}\ {\isacartoucheopen}S\ {\isasymin}\ C{\isacartoucheclose}{\isacharparenright}\isanewline
\isacommand{next}\isamarkupfalse%
\isanewline
\ \ \isacommand{fix}\isamarkupfalse%
\ n\isanewline
\ \ \isacommand{assume}\isamarkupfalse%
\ HI{\isacharcolon}{\isachardoublequoteopen}pcp{\isacharunderscore}seq\ C\ S\ n\ {\isasymin}\ C{\isachardoublequoteclose}\isanewline
\ \ \isacommand{have}\isamarkupfalse%
\ {\isachardoublequoteopen}pcp{\isacharunderscore}seq\ C\ S\ {\isacharparenleft}Suc\ n{\isacharparenright}\ {\isacharequal}\ {\isacharparenleft}let\ Sn\ {\isacharequal}\ pcp{\isacharunderscore}seq\ C\ S\ n{\isacharsemicolon}\ Sn{\isadigit{1}}\ {\isacharequal}\ insert\ {\isacharparenleft}from{\isacharunderscore}nat\ n{\isacharparenright}\ Sn\ in\isanewline
\ \ \ \ \ \ \ \ \ \ \ \ \ \ \ \ \ \ \ \ \ \ \ \ if\ Sn{\isadigit{1}}\ {\isasymin}\ C\ then\ Sn{\isadigit{1}}\ else\ Sn{\isacharparenright}{\isachardoublequoteclose}\ \isanewline
\ \ \ \ \isacommand{by}\isamarkupfalse%
\ {\isacharparenleft}simp\ only{\isacharcolon}\ pcp{\isacharunderscore}seq{\isachardot}simps{\isacharparenleft}{\isadigit{2}}{\isacharparenright}{\isacharparenright}\isanewline
\ \ \isacommand{then}\isamarkupfalse%
\ \isacommand{have}\isamarkupfalse%
\ SucDef{\isacharcolon}{\isachardoublequoteopen}pcp{\isacharunderscore}seq\ C\ S\ {\isacharparenleft}Suc\ n{\isacharparenright}\ {\isacharequal}\ {\isacharparenleft}if\ insert\ {\isacharparenleft}from{\isacharunderscore}nat\ n{\isacharparenright}\ {\isacharparenleft}pcp{\isacharunderscore}seq\ C\ S\ n{\isacharparenright}\ {\isasymin}\ C\ then\ \isanewline
\ \ \ \ \ \ \ \ \ \ \ \ \ \ \ \ \ \ \ \ insert\ {\isacharparenleft}from{\isacharunderscore}nat\ n{\isacharparenright}\ {\isacharparenleft}pcp{\isacharunderscore}seq\ C\ S\ n{\isacharparenright}\ else\ pcp{\isacharunderscore}seq\ C\ S\ n{\isacharparenright}{\isachardoublequoteclose}\ \isanewline
\ \ \ \ \isacommand{by}\isamarkupfalse%
\ {\isacharparenleft}simp\ only{\isacharcolon}\ Let{\isacharunderscore}def{\isacharparenright}\isanewline
\ \ \isacommand{show}\isamarkupfalse%
\ {\isachardoublequoteopen}pcp{\isacharunderscore}seq\ C\ S\ {\isacharparenleft}Suc\ n{\isacharparenright}\ {\isasymin}\ C{\isachardoublequoteclose}\isanewline
\ \ \isacommand{proof}\isamarkupfalse%
\ {\isacharparenleft}cases{\isacharparenright}\isanewline
\ \ \ \ \isacommand{assume}\isamarkupfalse%
\ {\isadigit{1}}{\isacharcolon}{\isachardoublequoteopen}insert\ {\isacharparenleft}from{\isacharunderscore}nat\ n{\isacharparenright}\ {\isacharparenleft}pcp{\isacharunderscore}seq\ C\ S\ n{\isacharparenright}\ {\isasymin}\ C{\isachardoublequoteclose}\isanewline
\ \ \ \ \isacommand{have}\isamarkupfalse%
\ {\isachardoublequoteopen}pcp{\isacharunderscore}seq\ C\ S\ {\isacharparenleft}Suc\ n{\isacharparenright}\ {\isacharequal}\ insert\ {\isacharparenleft}from{\isacharunderscore}nat\ n{\isacharparenright}\ {\isacharparenleft}pcp{\isacharunderscore}seq\ C\ S\ n{\isacharparenright}{\isachardoublequoteclose}\isanewline
\ \ \ \ \ \ \isacommand{using}\isamarkupfalse%
\ SucDef\ {\isadigit{1}}\ \isacommand{by}\isamarkupfalse%
\ {\isacharparenleft}simp\ only{\isacharcolon}\ if{\isacharunderscore}True{\isacharparenright}\isanewline
\ \ \ \ \isacommand{thus}\isamarkupfalse%
\ {\isachardoublequoteopen}pcp{\isacharunderscore}seq\ C\ S\ {\isacharparenleft}Suc\ n{\isacharparenright}\ {\isasymin}\ C{\isachardoublequoteclose}\isanewline
\ \ \ \ \ \ \isacommand{by}\isamarkupfalse%
\ {\isacharparenleft}simp\ only{\isacharcolon}\ {\isadigit{1}}{\isacharparenright}\isanewline
\ \ \isacommand{next}\isamarkupfalse%
\isanewline
\ \ \ \ \isacommand{assume}\isamarkupfalse%
\ {\isadigit{2}}{\isacharcolon}{\isachardoublequoteopen}insert\ {\isacharparenleft}from{\isacharunderscore}nat\ n{\isacharparenright}\ {\isacharparenleft}pcp{\isacharunderscore}seq\ C\ S\ n{\isacharparenright}\ {\isasymnotin}\ C{\isachardoublequoteclose}\isanewline
\ \ \ \ \isacommand{have}\isamarkupfalse%
\ {\isachardoublequoteopen}pcp{\isacharunderscore}seq\ C\ S\ {\isacharparenleft}Suc\ n{\isacharparenright}\ {\isacharequal}\ pcp{\isacharunderscore}seq\ C\ S\ n{\isachardoublequoteclose}\isanewline
\ \ \ \ \ \ \isacommand{using}\isamarkupfalse%
\ SucDef\ {\isadigit{2}}\ \isacommand{by}\isamarkupfalse%
\ {\isacharparenleft}simp\ only{\isacharcolon}\ if{\isacharunderscore}False{\isacharparenright}\isanewline
\ \ \ \ \isacommand{thus}\isamarkupfalse%
\ {\isachardoublequoteopen}pcp{\isacharunderscore}seq\ C\ S\ {\isacharparenleft}Suc\ n{\isacharparenright}\ {\isasymin}\ C{\isachardoublequoteclose}\isanewline
\ \ \ \ \ \ \isacommand{by}\isamarkupfalse%
\ {\isacharparenleft}simp\ only{\isacharcolon}\ HI{\isacharparenright}\isanewline
\ \ \isacommand{qed}\isamarkupfalse%
\isanewline
\isacommand{qed}\isamarkupfalse%
%
\endisatagproof
{\isafoldproof}%
%
\isadelimproof
%
\endisadelimproof
%
\begin{isamarkuptext}%
Del mismo modo, podemos probar el lema de manera automática en Isabelle.%
\end{isamarkuptext}\isamarkuptrue%
\isacommand{lemma}\isamarkupfalse%
\ pcp{\isacharunderscore}seq{\isacharunderscore}in{\isacharcolon}\ {\isachardoublequoteopen}pcp\ C\ {\isasymLongrightarrow}\ S\ {\isasymin}\ C\ {\isasymLongrightarrow}\ pcp{\isacharunderscore}seq\ C\ S\ n\ {\isasymin}\ C{\isachardoublequoteclose}\isanewline
%
\isadelimproof
%
\endisadelimproof
%
\isatagproof
\isacommand{proof}\isamarkupfalse%
{\isacharparenleft}induction\ n{\isacharparenright}\isanewline
\ \ \isacommand{case}\isamarkupfalse%
\ {\isacharparenleft}Suc\ n{\isacharparenright}\ \ \isanewline
\ \ \isacommand{hence}\isamarkupfalse%
\ {\isachardoublequoteopen}pcp{\isacharunderscore}seq\ C\ S\ n\ {\isasymin}\ C{\isachardoublequoteclose}\ \isacommand{by}\isamarkupfalse%
\ simp\isanewline
\ \ \isacommand{thus}\isamarkupfalse%
\ {\isacharquery}case\ \isacommand{by}\isamarkupfalse%
\ {\isacharparenleft}simp\ add{\isacharcolon}\ Let{\isacharunderscore}def{\isacharparenright}\isanewline
\isacommand{qed}\isamarkupfalse%
\ simp%
\endisatagproof
{\isafoldproof}%
%
\isadelimproof
%
\endisadelimproof
%
\begin{isamarkuptext}%
Por otro lado, veamos la monotonía de dichas sucesiones.

  \begin{lema}
    Toda sucesión de conjuntos construida a partir de una colección y un conjunto según la
    definición \isa{{\isadigit{4}}{\isachardot}{\isadigit{1}}{\isachardot}{\isadigit{1}}} es monótona.
  \end{lema}

  En Isabelle, se formaliza de la siguiente forma.%
\end{isamarkuptext}\isamarkuptrue%
\isacommand{lemma}\isamarkupfalse%
\ {\isachardoublequoteopen}pcp{\isacharunderscore}seq\ C\ S\ n\ {\isasymsubseteq}\ pcp{\isacharunderscore}seq\ C\ S\ {\isacharparenleft}Suc\ n{\isacharparenright}{\isachardoublequoteclose}\isanewline
%
\isadelimproof
\ \ %
\endisadelimproof
%
\isatagproof
\isacommand{oops}\isamarkupfalse%
%
\endisatagproof
{\isafoldproof}%
%
\isadelimproof
%
\endisadelimproof
%
\begin{isamarkuptext}%
Procedamos con la demostración del lema.

  \begin{demostracion}
    Sea una colección de conjuntos \isa{C}, \isa{S\ {\isasymin}\ C} y \isa{{\isacharbraceleft}S\isactrlsub n{\isacharbraceright}} la sucesión de conjuntos de \isa{C} a partir de 
    \isa{S} según la definición \isa{{\isadigit{4}}{\isachardot}{\isadigit{1}}{\isachardot}{\isadigit{1}}}. Para probar que \isa{{\isacharbraceleft}S\isactrlsub n{\isacharbraceright}} es monótona, basta probar que \isa{S\isactrlsub n\ {\isasymsubseteq}\ S\isactrlsub n\isactrlsub {\isacharplus}\isactrlsub {\isadigit{1}}} 
    para todo \isa{n\ {\isasymin}\ {\isasymnat}}. En efecto, el resultado es inmediato al considerar dos casos para todo 
    \isa{n\ {\isasymin}\ {\isasymnat}}: \isa{S\isactrlsub n\ {\isasymunion}\ {\isacharbraceleft}F\isactrlsub n{\isacharbraceright}\ {\isasymin}\ C} o \isa{S\isactrlsub n\ {\isasymunion}\ {\isacharbraceleft}F\isactrlsub n{\isacharbraceright}\ {\isasymnotin}\ C}. Si suponemos que\\ \isa{S\isactrlsub n\ {\isasymunion}\ {\isacharbraceleft}F\isactrlsub n{\isacharbraceright}\ {\isasymin}\ C}, por definición 
    tenemos que \isa{S\isactrlsub n\isactrlsub {\isacharplus}\isactrlsub {\isadigit{1}}\ {\isacharequal}\ S\isactrlsub n\ {\isasymunion}\ {\isacharbraceleft}F\isactrlsub n{\isacharbraceright}}, luego es claro que\\ \isa{S\isactrlsub n\ {\isasymsubseteq}\ S\isactrlsub n\isactrlsub {\isacharplus}\isactrlsub {\isadigit{1}}}. En caso contrario, si 
    \isa{S\isactrlsub n\ {\isasymunion}\ {\isacharbraceleft}F\isactrlsub n{\isacharbraceright}\ {\isasymnotin}\ C}, por definición se tiene que \isa{S\isactrlsub n\isactrlsub {\isacharplus}\isactrlsub {\isadigit{1}}\ {\isacharequal}\ S\isactrlsub n}, obteniéndose igualmente el resultado
    por la propiedad reflexiva de la contención de conjuntos. 
  \end{demostracion}

  La prueba detallada en Isabelle se muestra a continuación.%
\end{isamarkuptext}\isamarkuptrue%
\isacommand{lemma}\isamarkupfalse%
\ {\isachardoublequoteopen}pcp{\isacharunderscore}seq\ C\ S\ n\ {\isasymsubseteq}\ pcp{\isacharunderscore}seq\ C\ S\ {\isacharparenleft}Suc\ n{\isacharparenright}{\isachardoublequoteclose}\isanewline
%
\isadelimproof
%
\endisadelimproof
%
\isatagproof
\isacommand{proof}\isamarkupfalse%
\ {\isacharminus}\isanewline
\ \ \isacommand{have}\isamarkupfalse%
\ {\isachardoublequoteopen}pcp{\isacharunderscore}seq\ C\ S\ {\isacharparenleft}Suc\ n{\isacharparenright}\ {\isacharequal}\ {\isacharparenleft}let\ Sn\ {\isacharequal}\ pcp{\isacharunderscore}seq\ C\ S\ n{\isacharsemicolon}\ Sn{\isadigit{1}}\ {\isacharequal}\ insert\ {\isacharparenleft}from{\isacharunderscore}nat\ n{\isacharparenright}\ Sn\ in\isanewline
\ \ \ \ \ \ \ \ \ \ \ \ \ \ \ \ \ \ \ \ \ \ \ \ if\ Sn{\isadigit{1}}\ {\isasymin}\ C\ then\ Sn{\isadigit{1}}\ else\ Sn{\isacharparenright}{\isachardoublequoteclose}\ \isanewline
\ \ \ \ \isacommand{by}\isamarkupfalse%
\ {\isacharparenleft}simp\ only{\isacharcolon}\ pcp{\isacharunderscore}seq{\isachardot}simps{\isacharparenleft}{\isadigit{2}}{\isacharparenright}{\isacharparenright}\isanewline
\ \ \isacommand{then}\isamarkupfalse%
\ \isacommand{have}\isamarkupfalse%
\ SucDef{\isacharcolon}{\isachardoublequoteopen}pcp{\isacharunderscore}seq\ C\ S\ {\isacharparenleft}Suc\ n{\isacharparenright}\ {\isacharequal}\ {\isacharparenleft}if\ insert\ {\isacharparenleft}from{\isacharunderscore}nat\ n{\isacharparenright}\ {\isacharparenleft}pcp{\isacharunderscore}seq\ C\ S\ n{\isacharparenright}\ {\isasymin}\ C\ then\ \isanewline
\ \ \ \ \ \ \ \ \ \ \ \ \ \ \ \ \ \ \ \ insert\ {\isacharparenleft}from{\isacharunderscore}nat\ n{\isacharparenright}\ {\isacharparenleft}pcp{\isacharunderscore}seq\ C\ S\ n{\isacharparenright}\ else\ pcp{\isacharunderscore}seq\ C\ S\ n{\isacharparenright}{\isachardoublequoteclose}\ \isanewline
\ \ \ \ \isacommand{by}\isamarkupfalse%
\ {\isacharparenleft}simp\ only{\isacharcolon}\ Let{\isacharunderscore}def{\isacharparenright}\isanewline
\ \ \isacommand{thus}\isamarkupfalse%
\ {\isachardoublequoteopen}pcp{\isacharunderscore}seq\ C\ S\ n\ {\isasymsubseteq}\ pcp{\isacharunderscore}seq\ C\ S\ {\isacharparenleft}Suc\ n{\isacharparenright}{\isachardoublequoteclose}\isanewline
\ \ \isacommand{proof}\isamarkupfalse%
\ {\isacharparenleft}cases{\isacharparenright}\isanewline
\ \ \ \ \isacommand{assume}\isamarkupfalse%
\ {\isadigit{1}}{\isacharcolon}{\isachardoublequoteopen}insert\ {\isacharparenleft}from{\isacharunderscore}nat\ n{\isacharparenright}\ {\isacharparenleft}pcp{\isacharunderscore}seq\ C\ S\ n{\isacharparenright}\ {\isasymin}\ C{\isachardoublequoteclose}\isanewline
\ \ \ \ \isacommand{have}\isamarkupfalse%
\ {\isachardoublequoteopen}pcp{\isacharunderscore}seq\ C\ S\ {\isacharparenleft}Suc\ n{\isacharparenright}\ {\isacharequal}\ insert\ {\isacharparenleft}from{\isacharunderscore}nat\ n{\isacharparenright}\ {\isacharparenleft}pcp{\isacharunderscore}seq\ C\ S\ n{\isacharparenright}{\isachardoublequoteclose}\isanewline
\ \ \ \ \ \ \isacommand{using}\isamarkupfalse%
\ SucDef\ {\isadigit{1}}\ \isacommand{by}\isamarkupfalse%
\ {\isacharparenleft}simp\ only{\isacharcolon}\ if{\isacharunderscore}True{\isacharparenright}\isanewline
\ \ \ \ \isacommand{thus}\isamarkupfalse%
\ {\isachardoublequoteopen}pcp{\isacharunderscore}seq\ C\ S\ n\ {\isasymsubseteq}\ pcp{\isacharunderscore}seq\ C\ S\ {\isacharparenleft}Suc\ n{\isacharparenright}{\isachardoublequoteclose}\isanewline
\ \ \ \ \ \ \isacommand{by}\isamarkupfalse%
\ {\isacharparenleft}simp\ only{\isacharcolon}\ subset{\isacharunderscore}insertI{\isacharparenright}\isanewline
\ \ \isacommand{next}\isamarkupfalse%
\isanewline
\ \ \ \ \isacommand{assume}\isamarkupfalse%
\ {\isadigit{2}}{\isacharcolon}{\isachardoublequoteopen}insert\ {\isacharparenleft}from{\isacharunderscore}nat\ n{\isacharparenright}\ {\isacharparenleft}pcp{\isacharunderscore}seq\ C\ S\ n{\isacharparenright}\ {\isasymnotin}\ C{\isachardoublequoteclose}\isanewline
\ \ \ \ \isacommand{have}\isamarkupfalse%
\ {\isachardoublequoteopen}pcp{\isacharunderscore}seq\ C\ S\ {\isacharparenleft}Suc\ n{\isacharparenright}\ {\isacharequal}\ pcp{\isacharunderscore}seq\ C\ S\ n{\isachardoublequoteclose}\isanewline
\ \ \ \ \ \ \isacommand{using}\isamarkupfalse%
\ SucDef\ {\isadigit{2}}\ \isacommand{by}\isamarkupfalse%
\ {\isacharparenleft}simp\ only{\isacharcolon}\ if{\isacharunderscore}False{\isacharparenright}\isanewline
\ \ \ \ \isacommand{thus}\isamarkupfalse%
\ {\isachardoublequoteopen}pcp{\isacharunderscore}seq\ C\ S\ n\ {\isasymsubseteq}\ pcp{\isacharunderscore}seq\ C\ S\ {\isacharparenleft}Suc\ n{\isacharparenright}{\isachardoublequoteclose}\isanewline
\ \ \ \ \ \ \isacommand{by}\isamarkupfalse%
\ {\isacharparenleft}simp\ only{\isacharcolon}\ subset{\isacharunderscore}refl{\isacharparenright}\isanewline
\ \ \isacommand{qed}\isamarkupfalse%
\isanewline
\isacommand{qed}\isamarkupfalse%
%
\endisatagproof
{\isafoldproof}%
%
\isadelimproof
%
\endisadelimproof
%
\begin{isamarkuptext}%
Del mismo modo, se puede probar automáticamente en Isabelle/HOL.%
\end{isamarkuptext}\isamarkuptrue%
\isacommand{lemma}\isamarkupfalse%
\ pcp{\isacharunderscore}seq{\isacharunderscore}monotonicity{\isacharcolon}{\isachardoublequoteopen}pcp{\isacharunderscore}seq\ C\ S\ n\ {\isasymsubseteq}\ pcp{\isacharunderscore}seq\ C\ S\ {\isacharparenleft}Suc\ n{\isacharparenright}{\isachardoublequoteclose}\isanewline
%
\isadelimproof
\ \ %
\endisadelimproof
%
\isatagproof
\isacommand{by}\isamarkupfalse%
\ {\isacharparenleft}smt\ eq{\isacharunderscore}iff\ pcp{\isacharunderscore}seq{\isachardot}simps{\isacharparenleft}{\isadigit{2}}{\isacharparenright}\ subset{\isacharunderscore}insertI{\isacharparenright}%
\endisatagproof
{\isafoldproof}%
%
\isadelimproof
%
\endisadelimproof
%
\begin{isamarkuptext}%
Por otra lado, para facilitar posteriores demostraciones en Isabelle/HOL, vamos a formalizar 
  el lema anterior empleando la siguiente definición generalizada de monotonía.%
\end{isamarkuptext}\isamarkuptrue%
\isacommand{lemma}\isamarkupfalse%
\ pcp{\isacharunderscore}seq{\isacharunderscore}mono{\isacharcolon}\isanewline
\ \ \isakeyword{assumes}\ {\isachardoublequoteopen}n\ {\isasymle}\ m{\isachardoublequoteclose}\ \isanewline
\ \ \isakeyword{shows}\ {\isachardoublequoteopen}pcp{\isacharunderscore}seq\ C\ S\ n\ {\isasymsubseteq}\ pcp{\isacharunderscore}seq\ C\ S\ m{\isachardoublequoteclose}\isanewline
%
\isadelimproof
\ \ %
\endisadelimproof
%
\isatagproof
\isacommand{using}\isamarkupfalse%
\ pcp{\isacharunderscore}seq{\isacharunderscore}monotonicity\ assms\ \isacommand{by}\isamarkupfalse%
\ {\isacharparenleft}rule\ lift{\isacharunderscore}Suc{\isacharunderscore}mono{\isacharunderscore}le{\isacharparenright}%
\endisatagproof
{\isafoldproof}%
%
\isadelimproof
%
\endisadelimproof
%
\begin{isamarkuptext}%
A continuación daremos un lema que permite caracterizar un elemento de la sucesión en función 
  de los anteriores.

\begin{lema}
  Sea \isa{C} una colección de conjuntos, \isa{S\ {\isasymin}\ C} y \isa{{\isacharbraceleft}S\isactrlsub n{\isacharbraceright}} la sucesión de conjuntos de \isa{C} a partir de 
  \isa{S} construida según la definición \isa{{\isadigit{4}}{\isachardot}{\isadigit{1}}{\isachardot}{\isadigit{1}}}. Entonces, para todos \isa{n}, \isa{m\ {\isasymin}\ {\isasymnat}} 
  se verifica $\bigcup_{n \leq m} S_{n} = S_{m}$
\end{lema}

\begin{demostracion}
  En las condiciones del enunciado, la prueba se realiza por inducción en \isa{m\ {\isasymin}\ {\isasymnat}}.

  En primer lugar, consideremos el caso base \isa{m\ {\isacharequal}\ {\isadigit{0}}}. El resultado se obtiene directamente:

  $\bigcup_{n \leq 0} S_{n} = \bigcup_{n = 0} S_{n} = S_{0} = S_{m}$

  Por otro lado, supongamos por hipótesis de inducción que $\bigcup_{n \leq m} S_{n} = S_{m}$.
  Veamos que se verifica $\bigcup_{n \leq m + 1} S_{n} = S_{m + 1}$. Observemos que si \isa{n\ {\isasymle}\ m\ {\isacharplus}\ {\isadigit{1}}},
  entonces se tiene que, o bien \isa{n\ {\isasymle}\ m}, o bien \isa{n\ {\isacharequal}\ m\ {\isacharplus}\ {\isadigit{1}}}. De este modo, aplicando la 
  hipótesis de inducción, deducimos lo siguiente.

  $\bigcup_{n \leq m + 1} S_{n} = \bigcup_{n \leq m} S_{n} \cup \bigcup_{n = m + 1} S_{n} = \bigcup_{n \leq m} S_{n} \cup S_{m + 1} = S_{m} \cup S_{m + 1}$

  Por la monotonía de la sucesión, se tiene que \isa{S\isactrlsub m\ {\isasymsubseteq}\ S\isactrlsub m\isactrlsub {\isacharplus}\isactrlsub {\isadigit{1}}}. Luego, se verifica:

  $\bigcup_{n \leq m + 1} S_{n} = S_{m} \cup S_{m + 1} = S_{m + 1}$

  Lo que prueba el resultado.
\end{demostracion}

  Procedamos a su formalización y demostración detallada. Para ello, emplearemos la unión 
  generalizada en Isabelle/HOL perteneciente a la teoría 
  \href{https://n9.cl/gtf5x}{Complete-Lattices.thy}, junto con distintas propiedades sobre la misma
  definidas en dicha teoría. En Isabelle, los conjuntos se formalizan como equivalentes a los 
  predicados, de manera que un elemento pertenece a un conjunto si verifica el predicado que lo 
  caracteriza. De este modo, cada conjunto conforma un retículo cuyo orden parcial establecido es la 
  relación de contención. En consecuencia, los conjuntos de conjuntos forman un retículo completo 
  con dicho orden parcial, de manera que la unión generalizada de conjuntos se formaliza en Isabelle 
  como el supremo del retículo completo que conforman.

\comentario{No sé si está bien explicado.}

  Veamos la prueba detallada del resultado en Isabelle/HOL.%
\end{isamarkuptext}\isamarkuptrue%
\isacommand{lemma}\isamarkupfalse%
\ {\isachardoublequoteopen}{\isasymUnion}{\isacharbraceleft}pcp{\isacharunderscore}seq\ C\ S\ n{\isacharbar}n{\isachardot}\ n\ {\isasymle}\ m{\isacharbraceright}\ {\isacharequal}\ pcp{\isacharunderscore}seq\ C\ S\ m{\isachardoublequoteclose}\isanewline
%
\isadelimproof
%
\endisadelimproof
%
\isatagproof
\isacommand{proof}\isamarkupfalse%
\ {\isacharparenleft}induct\ m{\isacharparenright}\isanewline
\ \ \isacommand{have}\isamarkupfalse%
\ \ {\isachardoublequoteopen}{\isasymUnion}{\isacharbraceleft}pcp{\isacharunderscore}seq\ C\ S\ n{\isacharbar}n{\isachardot}\ n\ {\isasymle}\ {\isadigit{0}}{\isacharbraceright}\ {\isacharequal}\ {\isasymUnion}{\isacharbraceleft}pcp{\isacharunderscore}seq\ C\ S\ n{\isacharbar}n{\isachardot}\ n\ {\isacharequal}\ {\isadigit{0}}{\isacharbraceright}{\isachardoublequoteclose}\isanewline
\ \ \ \ \isacommand{by}\isamarkupfalse%
\ {\isacharparenleft}simp\ only{\isacharcolon}\ le{\isacharunderscore}zero{\isacharunderscore}eq{\isacharparenright}\isanewline
\ \ \isacommand{also}\isamarkupfalse%
\ \isacommand{have}\isamarkupfalse%
\ {\isachardoublequoteopen}{\isasymdots}\ {\isacharequal}\ {\isasymUnion}{\isacharparenleft}{\isacharparenleft}pcp{\isacharunderscore}seq\ C\ S{\isacharparenright}{\isacharbackquote}{\isacharbraceleft}n{\isachardot}\ n\ {\isacharequal}\ {\isadigit{0}}{\isacharbraceright}{\isacharparenright}{\isachardoublequoteclose}\isanewline
\ \ \ \ \isacommand{by}\isamarkupfalse%
\ {\isacharparenleft}simp\ only{\isacharcolon}\ image{\isacharunderscore}Collect{\isacharparenright}\isanewline
\ \ \isacommand{also}\isamarkupfalse%
\ \isacommand{have}\isamarkupfalse%
\ {\isachardoublequoteopen}{\isasymdots}\ {\isacharequal}\ {\isasymUnion}{\isacharbraceleft}pcp{\isacharunderscore}seq\ C\ S\ {\isadigit{0}}{\isacharbraceright}{\isachardoublequoteclose}\isanewline
\ \ \ \ \isacommand{by}\isamarkupfalse%
\ {\isacharparenleft}simp\ only{\isacharcolon}\ singleton{\isacharunderscore}conv\ image{\isacharunderscore}insert\ image{\isacharunderscore}empty{\isacharparenright}\isanewline
\ \ \isacommand{also}\isamarkupfalse%
\ \isacommand{have}\isamarkupfalse%
\ {\isachardoublequoteopen}{\isasymdots}\ {\isacharequal}\ pcp{\isacharunderscore}seq\ C\ S\ {\isadigit{0}}{\isachardoublequoteclose}\ \isanewline
\ \ \ \ \isacommand{by}\isamarkupfalse%
\ \ {\isacharparenleft}simp\ only{\isacharcolon}cSup{\isacharunderscore}singleton{\isacharparenright}\isanewline
\ \ \isacommand{finally}\isamarkupfalse%
\ \isacommand{show}\isamarkupfalse%
\ {\isachardoublequoteopen}{\isasymUnion}{\isacharbraceleft}pcp{\isacharunderscore}seq\ C\ S\ n{\isacharbar}n{\isachardot}\ n\ {\isasymle}\ {\isadigit{0}}{\isacharbraceright}\ {\isacharequal}\ pcp{\isacharunderscore}seq\ C\ S\ {\isadigit{0}}{\isachardoublequoteclose}\ \isanewline
\ \ \ \ \isacommand{by}\isamarkupfalse%
\ this\isanewline
\isacommand{next}\isamarkupfalse%
\isanewline
\ \ \isacommand{fix}\isamarkupfalse%
\ m\isanewline
\ \ \isacommand{assume}\isamarkupfalse%
\ HI{\isacharcolon}{\isachardoublequoteopen}{\isasymUnion}{\isacharbraceleft}pcp{\isacharunderscore}seq\ C\ S\ n{\isacharbar}n{\isachardot}\ n\ {\isasymle}\ m{\isacharbraceright}\ {\isacharequal}\ pcp{\isacharunderscore}seq\ C\ S\ m{\isachardoublequoteclose}\isanewline
\ \ \isacommand{have}\isamarkupfalse%
\ {\isachardoublequoteopen}m\ {\isasymle}\ Suc\ m{\isachardoublequoteclose}\ \isanewline
\ \ \ \ \isacommand{by}\isamarkupfalse%
\ {\isacharparenleft}simp\ only{\isacharcolon}\ add{\isacharunderscore}{\isadigit{0}}{\isacharunderscore}right{\isacharparenright}\isanewline
\ \ \isacommand{then}\isamarkupfalse%
\ \isacommand{have}\isamarkupfalse%
\ Mon{\isacharcolon}{\isachardoublequoteopen}pcp{\isacharunderscore}seq\ C\ S\ m\ {\isasymsubseteq}\ pcp{\isacharunderscore}seq\ C\ S\ {\isacharparenleft}Suc\ m{\isacharparenright}{\isachardoublequoteclose}\isanewline
\ \ \ \ \isacommand{by}\isamarkupfalse%
\ {\isacharparenleft}rule\ pcp{\isacharunderscore}seq{\isacharunderscore}mono{\isacharparenright}\isanewline
\ \ \isacommand{have}\isamarkupfalse%
\ {\isachardoublequoteopen}{\isasymUnion}{\isacharbraceleft}pcp{\isacharunderscore}seq\ C\ S\ n\ {\isacharbar}\ n{\isachardot}\ n\ {\isasymle}\ Suc\ m{\isacharbraceright}\ {\isacharequal}\ {\isasymUnion}{\isacharparenleft}{\isacharparenleft}pcp{\isacharunderscore}seq\ C\ S{\isacharparenright}{\isacharbackquote}{\isacharparenleft}{\isacharbraceleft}n{\isachardot}\ n\ {\isasymle}\ Suc\ m{\isacharbraceright}{\isacharparenright}{\isacharparenright}{\isachardoublequoteclose}\isanewline
\ \ \ \ \isacommand{by}\isamarkupfalse%
\ {\isacharparenleft}simp\ only{\isacharcolon}\ image{\isacharunderscore}Collect{\isacharparenright}\isanewline
\ \ \isacommand{also}\isamarkupfalse%
\ \isacommand{have}\isamarkupfalse%
\ {\isachardoublequoteopen}{\isasymdots}\ {\isacharequal}\ {\isasymUnion}{\isacharparenleft}{\isacharparenleft}pcp{\isacharunderscore}seq\ C\ S{\isacharparenright}{\isacharbackquote}{\isacharparenleft}{\isacharbraceleft}Suc\ m{\isacharbraceright}\ {\isasymunion}\ {\isacharbraceleft}n{\isachardot}\ n\ {\isasymle}\ m{\isacharbraceright}{\isacharparenright}{\isacharparenright}{\isachardoublequoteclose}\isanewline
\ \ \ \ \isacommand{by}\isamarkupfalse%
\ {\isacharparenleft}simp\ only{\isacharcolon}\ le{\isacharunderscore}Suc{\isacharunderscore}eq\ Collect{\isacharunderscore}disj{\isacharunderscore}eq\ Un{\isacharunderscore}commute\ singleton{\isacharunderscore}conv{\isacharparenright}\isanewline
\ \ \isacommand{also}\isamarkupfalse%
\ \isacommand{have}\isamarkupfalse%
\ {\isachardoublequoteopen}{\isasymdots}\ {\isacharequal}\ {\isasymUnion}{\isacharparenleft}{\isacharbraceleft}pcp{\isacharunderscore}seq\ C\ S\ {\isacharparenleft}Suc\ m{\isacharparenright}{\isacharbraceright}\ {\isasymunion}\ {\isacharbraceleft}pcp{\isacharunderscore}seq\ C\ S\ n\ {\isacharbar}\ n{\isachardot}\ n\ {\isasymle}\ m{\isacharbraceright}{\isacharparenright}{\isachardoublequoteclose}\isanewline
\ \ \ \ \isacommand{by}\isamarkupfalse%
\ {\isacharparenleft}simp\ only{\isacharcolon}\ image{\isacharunderscore}Un\ image{\isacharunderscore}insert\ image{\isacharunderscore}empty\ image{\isacharunderscore}Collect{\isacharparenright}\isanewline
\ \ \isacommand{also}\isamarkupfalse%
\ \isacommand{have}\isamarkupfalse%
\ {\isachardoublequoteopen}{\isasymdots}\ {\isacharequal}\ {\isasymUnion}{\isacharbraceleft}pcp{\isacharunderscore}seq\ C\ S\ {\isacharparenleft}Suc\ m{\isacharparenright}{\isacharbraceright}\ {\isasymunion}\ {\isasymUnion}{\isacharbraceleft}pcp{\isacharunderscore}seq\ C\ S\ n\ {\isacharbar}\ n{\isachardot}\ n\ {\isasymle}\ m{\isacharbraceright}{\isachardoublequoteclose}\isanewline
\ \ \ \ \isacommand{by}\isamarkupfalse%
\ {\isacharparenleft}simp\ only{\isacharcolon}\ Union{\isacharunderscore}Un{\isacharunderscore}distrib{\isacharparenright}\isanewline
\ \ \isacommand{also}\isamarkupfalse%
\ \isacommand{have}\isamarkupfalse%
\ {\isachardoublequoteopen}{\isasymdots}\ {\isacharequal}\ {\isacharparenleft}pcp{\isacharunderscore}seq\ C\ S\ {\isacharparenleft}Suc\ m{\isacharparenright}{\isacharparenright}\ {\isasymunion}\ {\isasymUnion}{\isacharbraceleft}pcp{\isacharunderscore}seq\ C\ S\ n\ {\isacharbar}\ n{\isachardot}\ n\ {\isasymle}\ m{\isacharbraceright}{\isachardoublequoteclose}\isanewline
\ \ \ \ \isacommand{by}\isamarkupfalse%
\ {\isacharparenleft}simp\ only{\isacharcolon}\ cSup{\isacharunderscore}singleton{\isacharparenright}\isanewline
\ \ \isacommand{also}\isamarkupfalse%
\ \isacommand{have}\isamarkupfalse%
\ {\isachardoublequoteopen}{\isasymdots}\ {\isacharequal}\ {\isacharparenleft}pcp{\isacharunderscore}seq\ C\ S\ {\isacharparenleft}Suc\ m{\isacharparenright}{\isacharparenright}\ {\isasymunion}\ {\isacharparenleft}pcp{\isacharunderscore}seq\ C\ S\ m{\isacharparenright}{\isachardoublequoteclose}\isanewline
\ \ \ \ \isacommand{by}\isamarkupfalse%
\ {\isacharparenleft}simp\ only{\isacharcolon}\ HI{\isacharparenright}\isanewline
\ \ \isacommand{also}\isamarkupfalse%
\ \isacommand{have}\isamarkupfalse%
\ {\isachardoublequoteopen}{\isasymdots}\ {\isacharequal}\ pcp{\isacharunderscore}seq\ C\ S\ {\isacharparenleft}Suc\ m{\isacharparenright}{\isachardoublequoteclose}\isanewline
\ \ \ \ \isacommand{using}\isamarkupfalse%
\ Mon\ \isacommand{by}\isamarkupfalse%
\ {\isacharparenleft}simp\ only{\isacharcolon}\ Un{\isacharunderscore}absorb{\isadigit{2}}{\isacharparenright}\isanewline
\ \ \isacommand{finally}\isamarkupfalse%
\ \isacommand{show}\isamarkupfalse%
\ {\isachardoublequoteopen}{\isasymUnion}{\isacharbraceleft}pcp{\isacharunderscore}seq\ C\ S\ n{\isacharbar}n{\isachardot}\ n\ {\isasymle}\ {\isacharparenleft}Suc\ m{\isacharparenright}{\isacharbraceright}\ {\isacharequal}\ pcp{\isacharunderscore}seq\ C\ S\ {\isacharparenleft}Suc\ m{\isacharparenright}{\isachardoublequoteclose}\isanewline
\ \ \ \ \isacommand{by}\isamarkupfalse%
\ this\isanewline
\isacommand{qed}\isamarkupfalse%
%
\endisatagproof
{\isafoldproof}%
%
\isadelimproof
%
\endisadelimproof
%
\begin{isamarkuptext}%
Análogamente, podemos dar una prueba automática.%
\end{isamarkuptext}\isamarkuptrue%
\isacommand{lemma}\isamarkupfalse%
\ pcp{\isacharunderscore}seq{\isacharunderscore}UN{\isacharcolon}\ {\isachardoublequoteopen}{\isasymUnion}{\isacharbraceleft}pcp{\isacharunderscore}seq\ C\ S\ n{\isacharbar}n{\isachardot}\ n\ {\isasymle}\ m{\isacharbraceright}\ {\isacharequal}\ pcp{\isacharunderscore}seq\ C\ S\ m{\isachardoublequoteclose}\isanewline
%
\isadelimproof
%
\endisadelimproof
%
\isatagproof
\isacommand{proof}\isamarkupfalse%
{\isacharparenleft}induction\ m{\isacharparenright}\isanewline
\ \ \isacommand{case}\isamarkupfalse%
\ {\isacharparenleft}Suc\ m{\isacharparenright}\isanewline
\ \ \isacommand{have}\isamarkupfalse%
\ {\isachardoublequoteopen}{\isacharbraceleft}f\ n\ {\isacharbar}n{\isachardot}\ n\ {\isasymle}\ Suc\ m{\isacharbraceright}\ {\isacharequal}\ insert\ {\isacharparenleft}f\ {\isacharparenleft}Suc\ m{\isacharparenright}{\isacharparenright}\ {\isacharbraceleft}f\ n\ {\isacharbar}n{\isachardot}\ n\ {\isasymle}\ m{\isacharbraceright}{\isachardoublequoteclose}\ \isanewline
\ \ \ \ \isakeyword{for}\ f\ \isacommand{using}\isamarkupfalse%
\ le{\isacharunderscore}Suc{\isacharunderscore}eq\ \isacommand{by}\isamarkupfalse%
\ auto\isanewline
\ \ \isacommand{hence}\isamarkupfalse%
\ {\isachardoublequoteopen}{\isacharbraceleft}pcp{\isacharunderscore}seq\ C\ S\ n\ {\isacharbar}n{\isachardot}\ n\ {\isasymle}\ Suc\ m{\isacharbraceright}\ {\isacharequal}\ \isanewline
\ \ \ \ \ \ \ \ \ \ insert\ {\isacharparenleft}pcp{\isacharunderscore}seq\ C\ S\ {\isacharparenleft}Suc\ m{\isacharparenright}{\isacharparenright}\ {\isacharbraceleft}pcp{\isacharunderscore}seq\ C\ S\ n\ {\isacharbar}n{\isachardot}\ n\ {\isasymle}\ m{\isacharbraceright}{\isachardoublequoteclose}\ \isacommand{{\isachardot}}\isamarkupfalse%
\isanewline
\ \ \isacommand{hence}\isamarkupfalse%
\ {\isachardoublequoteopen}{\isasymUnion}{\isacharbraceleft}pcp{\isacharunderscore}seq\ C\ S\ n\ {\isacharbar}n{\isachardot}\ n\ {\isasymle}\ Suc\ m{\isacharbraceright}\ {\isacharequal}\ \isanewline
\ \ \ \ \ \ \ \ \ {\isasymUnion}{\isacharbraceleft}pcp{\isacharunderscore}seq\ C\ S\ n\ {\isacharbar}n{\isachardot}\ n\ {\isasymle}\ m{\isacharbraceright}\ {\isasymunion}\ pcp{\isacharunderscore}seq\ C\ S\ {\isacharparenleft}Suc\ m{\isacharparenright}{\isachardoublequoteclose}\ \isacommand{by}\isamarkupfalse%
\ auto\isanewline
\ \ \isacommand{thus}\isamarkupfalse%
\ {\isacharquery}case\ \isacommand{using}\isamarkupfalse%
\ Suc\ pcp{\isacharunderscore}seq{\isacharunderscore}mono\ \isacommand{by}\isamarkupfalse%
\ blast\isanewline
\isacommand{qed}\isamarkupfalse%
\ simp%
\endisatagproof
{\isafoldproof}%
%
\isadelimproof
%
\endisadelimproof
%
\begin{isamarkuptext}%
Finalmente, definamos el límite de las sucesiones presentadas en la definición \isa{{\isadigit{4}}{\isachardot}{\isadigit{1}}{\isachardot}{\isadigit{1}}}.

 \begin{definicion}
  Sea \isa{C} una colección, \isa{S\ {\isasymin}\ C} y \isa{{\isacharbraceleft}S\isactrlsub n{\isacharbraceright}} la sucesión de conjuntos de \isa{C} a partir de \isa{S} según la
  definición \isa{{\isadigit{4}}{\isachardot}{\isadigit{1}}{\isachardot}{\isadigit{1}}}. Se define el \isa{límite\ de\ la\ sucesión\ de\ conjuntos\ de\ C\ a\ partir\ de\ S} como 
  $\bigcup_{n = 0}^{\infty} S_{n}$
 \end{definicion}

  La definición del límite se formaliza utilizando la unión generalizada de Isabelle como sigue.%
\end{isamarkuptext}\isamarkuptrue%
\isacommand{definition}\isamarkupfalse%
\ {\isachardoublequoteopen}pcp{\isacharunderscore}lim\ C\ S\ {\isasymequiv}\ {\isasymUnion}{\isacharbraceleft}pcp{\isacharunderscore}seq\ C\ S\ n{\isacharbar}n{\isachardot}\ True{\isacharbraceright}{\isachardoublequoteclose}%
\begin{isamarkuptext}%
Veamos el primer resultado sobre el límite.

\begin{lema}
  Sea \isa{C} una colección de conjuntos, \isa{S\ {\isasymin}\ C} y \isa{{\isacharbraceleft}S\isactrlsub n{\isacharbraceright}} la sucesión de conjuntos de \isa{C} a partir de
  \isa{S} según la definición \isa{{\isadigit{4}}{\isachardot}{\isadigit{1}}{\isachardot}{\isadigit{1}}}. Entonces, para todo \isa{n\ {\isasymin}\ {\isasymnat}}, se verifica:

  $S_{n} \subseteq \bigcup_{n = 0}^{\infty} S_{n}$
\end{lema}

\begin{demostracion}
  El resultado se obtiene de manera inmediata ya que, para todo \isa{n\ {\isasymin}\ {\isasymnat}}, se verifica que 
  $S_{n} \in \{S_{n}\}_{n = 0}^{\infty}$. Por tanto, es claro que 
  $S_{n} \subseteq \bigcup_{n = 0}^{\infty} S_{n}$.
\end{demostracion}

  Su formalización y demostración detallada en Isabelle se muestran a continuación.%
\end{isamarkuptext}\isamarkuptrue%
\isacommand{lemma}\isamarkupfalse%
\ {\isachardoublequoteopen}pcp{\isacharunderscore}seq\ C\ S\ n\ {\isasymsubseteq}\ pcp{\isacharunderscore}lim\ C\ S{\isachardoublequoteclose}\isanewline
%
\isadelimproof
\ \ %
\endisadelimproof
%
\isatagproof
\isacommand{unfolding}\isamarkupfalse%
\ pcp{\isacharunderscore}lim{\isacharunderscore}def\isanewline
\isacommand{proof}\isamarkupfalse%
\ {\isacharminus}\isanewline
\ \ \isacommand{have}\isamarkupfalse%
\ {\isachardoublequoteopen}n\ {\isasymin}\ {\isacharbraceleft}n\ {\isacharbar}\ n{\isachardot}\ True{\isacharbraceright}{\isachardoublequoteclose}\ \isanewline
\ \ \ \ \isacommand{by}\isamarkupfalse%
\ {\isacharparenleft}simp\ only{\isacharcolon}\ simp{\isacharunderscore}thms{\isacharparenleft}{\isadigit{2}}{\isadigit{1}}{\isacharcomma}{\isadigit{3}}{\isadigit{8}}{\isacharparenright}\ Collect{\isacharunderscore}const\ if{\isacharunderscore}True\ UNIV{\isacharunderscore}I{\isacharparenright}\ \isanewline
\ \ \isacommand{then}\isamarkupfalse%
\ \isacommand{have}\isamarkupfalse%
\ {\isachardoublequoteopen}pcp{\isacharunderscore}seq\ C\ S\ n\ {\isasymin}\ {\isacharparenleft}pcp{\isacharunderscore}seq\ C\ S{\isacharparenright}{\isacharbackquote}{\isacharbraceleft}n\ {\isacharbar}\ n{\isachardot}\ True{\isacharbraceright}{\isachardoublequoteclose}\isanewline
\ \ \ \ \isacommand{by}\isamarkupfalse%
\ {\isacharparenleft}simp\ only{\isacharcolon}\ imageI{\isacharparenright}\isanewline
\ \ \isacommand{then}\isamarkupfalse%
\ \isacommand{have}\isamarkupfalse%
\ {\isachardoublequoteopen}pcp{\isacharunderscore}seq\ C\ S\ n\ {\isasymin}\ {\isacharbraceleft}pcp{\isacharunderscore}seq\ C\ S\ n\ {\isacharbar}\ n{\isachardot}\ True{\isacharbraceright}{\isachardoublequoteclose}\isanewline
\ \ \ \ \isacommand{by}\isamarkupfalse%
\ {\isacharparenleft}simp\ only{\isacharcolon}\ image{\isacharunderscore}Collect\ simp{\isacharunderscore}thms{\isacharparenleft}{\isadigit{4}}{\isadigit{0}}{\isacharparenright}{\isacharparenright}\isanewline
\ \ \isacommand{thus}\isamarkupfalse%
\ {\isachardoublequoteopen}pcp{\isacharunderscore}seq\ C\ S\ n\ {\isasymsubseteq}\ {\isasymUnion}{\isacharbraceleft}pcp{\isacharunderscore}seq\ C\ S\ n\ {\isacharbar}\ n{\isachardot}\ True{\isacharbraceright}{\isachardoublequoteclose}\isanewline
\ \ \ \ \isacommand{by}\isamarkupfalse%
\ {\isacharparenleft}simp\ only{\isacharcolon}\ Union{\isacharunderscore}upper{\isacharparenright}\isanewline
\isacommand{qed}\isamarkupfalse%
%
\endisatagproof
{\isafoldproof}%
%
\isadelimproof
%
\endisadelimproof
%
\begin{isamarkuptext}%
Podemos probarlo de manera automática como sigue.%
\end{isamarkuptext}\isamarkuptrue%
\isacommand{lemma}\isamarkupfalse%
\ pcp{\isacharunderscore}seq{\isacharunderscore}sub{\isacharcolon}\ {\isachardoublequoteopen}pcp{\isacharunderscore}seq\ C\ S\ n\ {\isasymsubseteq}\ pcp{\isacharunderscore}lim\ C\ S{\isachardoublequoteclose}\ \isanewline
%
\isadelimproof
\ \ %
\endisadelimproof
%
\isatagproof
\isacommand{unfolding}\isamarkupfalse%
\ pcp{\isacharunderscore}lim{\isacharunderscore}def\ \isacommand{by}\isamarkupfalse%
\ blast%
\endisatagproof
{\isafoldproof}%
%
\isadelimproof
%
\endisadelimproof
%
\begin{isamarkuptext}%
Mostremos otro resultado. 

  \begin{lema}
    Sea \isa{C} una colección de conjuntos de fórmulas proposicionales, \isa{S\ {\isasymin}\ C} y \isa{{\isacharbraceleft}S\isactrlsub n{\isacharbraceright}} la sucesión de 
    conjuntos de \isa{C} a partir de \isa{S} según la definición \isa{{\isadigit{4}}{\isachardot}{\isadigit{1}}{\isachardot}{\isadigit{1}}}. Si \isa{F} es una fórmula tal que
    $F \in \bigcup_{n = 0}^{\infty} S_{n}$, entonces existe un \isa{k\ {\isasymin}\ {\isasymnat}} tal que \isa{F\ {\isasymin}\ S\isactrlsub k}. 
  \end{lema}

  \begin{demostracion}
    La prueba es inmediata de la definición del límite de la sucesión de conjuntos \isa{{\isacharbraceleft}S\isactrlsub n{\isacharbraceright}}: si
    \isa{F} pertenece a la unión generalizada $\bigcup_{n = 0}^{\infty} S_{n}$, entonces existe algún
    conjunto \isa{S\isactrlsub k} tal que \isa{F\ {\isasymin}\ S\isactrlsub k}. Es decir, existe \isa{k\ {\isasymin}\ {\isasymnat}} tal que \isa{F\ {\isasymin}\ S\isactrlsub k}, como queríamos
    demostrar.
  \end{demostracion} 

  Su prueba detallada en Isabelle/HOL es la siguiente.%
\end{isamarkuptext}\isamarkuptrue%
\isacommand{lemma}\isamarkupfalse%
\ \isanewline
\ \ \isakeyword{assumes}\ {\isachardoublequoteopen}F\ {\isasymin}\ pcp{\isacharunderscore}lim\ C\ S{\isachardoublequoteclose}\isanewline
\ \ \isakeyword{shows}\ {\isachardoublequoteopen}{\isasymexists}k{\isachardot}\ F\ {\isasymin}\ pcp{\isacharunderscore}seq\ C\ S\ k{\isachardoublequoteclose}\ \isanewline
%
\isadelimproof
%
\endisadelimproof
%
\isatagproof
\isacommand{proof}\isamarkupfalse%
\ {\isacharminus}\isanewline
\ \ \isacommand{have}\isamarkupfalse%
\ {\isachardoublequoteopen}F\ {\isasymin}\ {\isasymUnion}{\isacharparenleft}{\isacharparenleft}pcp{\isacharunderscore}seq\ C\ S{\isacharparenright}\ {\isacharbackquote}\ {\isacharbraceleft}n\ {\isacharbar}\ n{\isachardot}\ True{\isacharbraceright}{\isacharparenright}{\isachardoublequoteclose}\isanewline
\ \ \ \ \isacommand{using}\isamarkupfalse%
\ assms\ \isacommand{by}\isamarkupfalse%
\ {\isacharparenleft}simp\ only{\isacharcolon}\ pcp{\isacharunderscore}lim{\isacharunderscore}def\ image{\isacharunderscore}Collect\ simp{\isacharunderscore}thms{\isacharparenleft}{\isadigit{4}}{\isadigit{0}}{\isacharparenright}{\isacharparenright}\isanewline
\ \ \isacommand{then}\isamarkupfalse%
\ \isacommand{have}\isamarkupfalse%
\ {\isachardoublequoteopen}{\isasymexists}k\ {\isasymin}\ {\isacharbraceleft}n{\isachardot}\ True{\isacharbraceright}{\isachardot}\ F\ {\isasymin}\ pcp{\isacharunderscore}seq\ C\ S\ k{\isachardoublequoteclose}\isanewline
\ \ \ \ \isacommand{by}\isamarkupfalse%
\ {\isacharparenleft}simp\ only{\isacharcolon}\ UN{\isacharunderscore}iff\ simp{\isacharunderscore}thms{\isacharparenleft}{\isadigit{4}}{\isadigit{0}}{\isacharparenright}{\isacharparenright}\isanewline
\ \ \isacommand{then}\isamarkupfalse%
\ \isacommand{have}\isamarkupfalse%
\ {\isachardoublequoteopen}{\isasymexists}k\ {\isasymin}\ UNIV{\isachardot}\ F\ {\isasymin}\ pcp{\isacharunderscore}seq\ C\ S\ k{\isachardoublequoteclose}\ \isanewline
\ \ \ \ \isacommand{by}\isamarkupfalse%
\ {\isacharparenleft}simp\ only{\isacharcolon}\ UNIV{\isacharunderscore}def{\isacharparenright}\isanewline
\ \ \isacommand{thus}\isamarkupfalse%
\ {\isachardoublequoteopen}{\isasymexists}k{\isachardot}\ F\ {\isasymin}\ pcp{\isacharunderscore}seq\ C\ S\ k{\isachardoublequoteclose}\ \isanewline
\ \ \ \ \isacommand{by}\isamarkupfalse%
\ {\isacharparenleft}simp\ only{\isacharcolon}\ bex{\isacharunderscore}UNIV{\isacharparenright}\isanewline
\isacommand{qed}\isamarkupfalse%
%
\endisatagproof
{\isafoldproof}%
%
\isadelimproof
%
\endisadelimproof
%
\begin{isamarkuptext}%
Mostremos, a continuación, la demostración automática del resultado.%
\end{isamarkuptext}\isamarkuptrue%
\isacommand{lemma}\isamarkupfalse%
\ pcp{\isacharunderscore}lim{\isacharunderscore}inserted{\isacharunderscore}at{\isacharunderscore}ex{\isacharcolon}\ \isanewline
\ \ \ \ {\isachardoublequoteopen}S{\isacharprime}\ {\isasymin}\ pcp{\isacharunderscore}lim\ C\ S\ {\isasymLongrightarrow}\ {\isasymexists}k{\isachardot}\ S{\isacharprime}\ {\isasymin}\ pcp{\isacharunderscore}seq\ C\ S\ k{\isachardoublequoteclose}\isanewline
%
\isadelimproof
\ \ %
\endisadelimproof
%
\isatagproof
\isacommand{unfolding}\isamarkupfalse%
\ pcp{\isacharunderscore}lim{\isacharunderscore}def\ \isacommand{by}\isamarkupfalse%
\ blast%
\endisatagproof
{\isafoldproof}%
%
\isadelimproof
%
\endisadelimproof
%
\begin{isamarkuptext}%
Por último, veamos la siguiente propiedad sobre conjuntos finitos contenidos en el límite de 
  las sucesiones definido en \isa{{\isadigit{4}}{\isachardot}{\isadigit{1}}{\isachardot}{\isadigit{5}}}.

\begin{lema}
  Sea \isa{C} una colección, \isa{S\ {\isasymin}\ C} y \isa{{\isacharbraceleft}S\isactrlsub n{\isacharbraceright}} la sucesión de conjuntos de \isa{C} a partir de \isa{S} según la
  definición \isa{{\isadigit{4}}{\isachardot}{\isadigit{1}}{\isachardot}{\isadigit{1}}}. Si \isa{S{\isacharprime}} es un conjunto finito tal que \isa{S{\isacharprime}\ {\isasymsubseteq}} $\bigcup_{n = 0}^{\infty} S_{n}$, 
  entonces existe un\\ \isa{k\ {\isasymin}\ {\isasymnat}} tal que \isa{S{\isacharprime}\ {\isasymsubseteq}\ S\isactrlsub k}.
\end{lema}

\begin{demostracion}
  La prueba del resultado se realiza por inducción sobre la estructura recursiva de los conjuntos 
  finitos.

  En primer lugar, probemos el caso base correspondiente al conjunto vacío. Supongamos que \isa{{\isacharbraceleft}{\isacharbraceright}} está 
  contenido en el límite de la sucesión de conjuntos de \isa{C} a partir de \isa{S}. Como \isa{{\isacharbraceleft}{\isacharbraceright}} es 
  subconjunto de todo conjunto, en particular lo es de \isa{S\ {\isacharequal}\ S\isactrlsub {\isadigit{0}}}, probando así el primer caso.

  Por otra parte, probemos el paso de inducción. Sea \isa{S{\isacharprime}} un conjunto finito verificando la 
  hipótesis de inducción: si \isa{S{\isacharprime}} está contenido en el límite de la sucesión de conjuntos de 
  \isa{C} a partir de \isa{S}, entonces también está contenido en algún \isa{S\isactrlsub k\isactrlsub {\isacharprime}} para cierto \isa{k{\isacharprime}\ {\isasymin}\ {\isasymnat}}. Sea 
  \isa{F} una fórmula tal que \isa{F\ {\isasymnotin}\ S{\isacharprime}}. Vamos a probar que si \isa{{\isacharbraceleft}F{\isacharbraceright}\ {\isasymunion}\ S{\isacharprime}} está contenido en el límite, 
  entonces está contenido en \isa{S\isactrlsub k} para algún \isa{k\ {\isasymin}\ {\isasymnat}}. 

  Como hemos supuesto que \isa{{\isacharbraceleft}F{\isacharbraceright}\ {\isasymunion}\ S{\isacharprime}} está contenido en el límite, entonces se verifica que \isa{F}
  pertenece al límite y \isa{S{\isacharprime}} está contenido en él. Por el lema \isa{{\isadigit{4}}{\isachardot}{\isadigit{1}}{\isachardot}{\isadigit{7}}}, como \isa{F} pertenece al 
  límite, deducimos que existe un \isa{k\ {\isasymin}\ {\isasymnat}} tal que \isa{F\ {\isasymin}\ S\isactrlsub k}. Por otro lado, como \isa{S{\isacharprime}} está contenido
  en el límite, por hipótesis de inducción existe algún \isa{k{\isacharprime}\ {\isasymin}\ {\isasymnat}} tal que \isa{S{\isacharprime}\ {\isasymsubseteq}\ S\isactrlsub k\isactrlsub {\isacharprime}}. El resultado 
  se obtiene considerando el máximo entre \isa{k} y \isa{k{\isacharprime}}, que notaremos por \isa{k{\isacharprime}{\isacharprime}}. En efecto, por la 
  monotonía de la sucesión, se verifica que tanto \isa{S\isactrlsub k} como \isa{S\isactrlsub k\isactrlsub {\isacharprime}} están contenidos en \isa{S\isactrlsub k\isactrlsub {\isacharprime}\isactrlsub {\isacharprime}}. De este 
  modo, como \isa{S{\isacharprime}\ {\isasymsubseteq}\ S\isactrlsub k\isactrlsub {\isacharprime}}, por la transitividad de la contención de conjuntos se tiene que 
  \isa{S{\isacharprime}\ {\isasymsubseteq}\ S\isactrlsub k\isactrlsub {\isacharprime}\isactrlsub {\isacharprime}}. Además, como \isa{F\ {\isasymin}\ S\isactrlsub k}, se tiene que \isa{F\ {\isasymin}\ S\isactrlsub k\isactrlsub {\isacharprime}\isactrlsub {\isacharprime}}. Por lo tanto, \isa{{\isacharbraceleft}F{\isacharbraceright}\ {\isasymunion}\ S{\isacharprime}\ {\isasymsubseteq}\ S\isactrlsub k\isactrlsub {\isacharprime}\isactrlsub {\isacharprime}}, como 
  queríamos demostrar. 
\end{demostracion}

  Procedamos con la demostración detallada en Isabelle.%
\end{isamarkuptext}\isamarkuptrue%
\isacommand{lemma}\isamarkupfalse%
\ \isanewline
\ \ \isakeyword{assumes}\ {\isachardoublequoteopen}finite\ S{\isacharprime}{\isachardoublequoteclose}\isanewline
\ \ \ \ \ \ \ \ \ \ {\isachardoublequoteopen}S{\isacharprime}\ {\isasymsubseteq}\ pcp{\isacharunderscore}lim\ C\ S{\isachardoublequoteclose}\isanewline
\ \ \ \ \ \ \ \ \isakeyword{shows}\ {\isachardoublequoteopen}{\isasymexists}k{\isachardot}\ S{\isacharprime}\ {\isasymsubseteq}\ pcp{\isacharunderscore}seq\ C\ S\ k{\isachardoublequoteclose}\isanewline
%
\isadelimproof
\ \ %
\endisadelimproof
%
\isatagproof
\isacommand{using}\isamarkupfalse%
\ assms\isanewline
\isacommand{proof}\isamarkupfalse%
\ {\isacharparenleft}induction\ S{\isacharprime}\ rule{\isacharcolon}\ finite{\isacharunderscore}induct{\isacharparenright}\isanewline
\ \ \isacommand{case}\isamarkupfalse%
\ empty\isanewline
\ \ \isacommand{have}\isamarkupfalse%
\ {\isachardoublequoteopen}pcp{\isacharunderscore}seq\ C\ S\ {\isadigit{0}}\ {\isacharequal}\ S{\isachardoublequoteclose}\isanewline
\ \ \ \ \isacommand{by}\isamarkupfalse%
\ {\isacharparenleft}simp\ only{\isacharcolon}\ pcp{\isacharunderscore}seq{\isachardot}simps{\isacharparenleft}{\isadigit{1}}{\isacharparenright}{\isacharparenright}\isanewline
\ \ \isacommand{have}\isamarkupfalse%
\ {\isachardoublequoteopen}{\isacharbraceleft}{\isacharbraceright}\ {\isasymsubseteq}\ S{\isachardoublequoteclose}\isanewline
\ \ \ \ \isacommand{by}\isamarkupfalse%
\ {\isacharparenleft}rule\ order{\isacharunderscore}bot{\isacharunderscore}class{\isachardot}bot{\isachardot}extremum{\isacharparenright}\isanewline
\ \ \isacommand{then}\isamarkupfalse%
\ \isacommand{have}\isamarkupfalse%
\ {\isachardoublequoteopen}{\isacharbraceleft}{\isacharbraceright}\ {\isasymsubseteq}\ pcp{\isacharunderscore}seq\ C\ S\ {\isadigit{0}}{\isachardoublequoteclose}\isanewline
\ \ \ \ \isacommand{by}\isamarkupfalse%
\ {\isacharparenleft}simp\ only{\isacharcolon}\ {\isacartoucheopen}pcp{\isacharunderscore}seq\ C\ S\ {\isadigit{0}}\ {\isacharequal}\ S{\isacartoucheclose}{\isacharparenright}\isanewline
\ \ \isacommand{then}\isamarkupfalse%
\ \isacommand{show}\isamarkupfalse%
\ {\isacharquery}case\ \isanewline
\ \ \ \ \isacommand{by}\isamarkupfalse%
\ {\isacharparenleft}rule\ exI{\isacharparenright}\isanewline
\isacommand{next}\isamarkupfalse%
\isanewline
\ \ \isacommand{case}\isamarkupfalse%
\ {\isacharparenleft}insert\ F\ S{\isacharprime}{\isacharparenright}\isanewline
\ \ \isacommand{then}\isamarkupfalse%
\ \isacommand{have}\isamarkupfalse%
\ {\isachardoublequoteopen}insert\ F\ S{\isacharprime}\ {\isasymsubseteq}\ pcp{\isacharunderscore}lim\ C\ S{\isachardoublequoteclose}\isanewline
\ \ \ \ \isacommand{by}\isamarkupfalse%
\ {\isacharparenleft}simp\ only{\isacharcolon}\ insert{\isachardot}prems{\isacharparenright}\isanewline
\ \ \isacommand{then}\isamarkupfalse%
\ \isacommand{have}\isamarkupfalse%
\ C{\isacharcolon}{\isachardoublequoteopen}F\ {\isasymin}\ {\isacharparenleft}pcp{\isacharunderscore}lim\ C\ S{\isacharparenright}\ {\isasymand}\ S{\isacharprime}\ {\isasymsubseteq}\ pcp{\isacharunderscore}lim\ C\ S{\isachardoublequoteclose}\isanewline
\ \ \ \ \isacommand{by}\isamarkupfalse%
\ {\isacharparenleft}simp\ only{\isacharcolon}\ insert{\isacharunderscore}subset{\isacharparenright}\ \isanewline
\ \ \isacommand{then}\isamarkupfalse%
\ \isacommand{have}\isamarkupfalse%
\ {\isachardoublequoteopen}S{\isacharprime}\ {\isasymsubseteq}\ pcp{\isacharunderscore}lim\ C\ S{\isachardoublequoteclose}\isanewline
\ \ \ \ \isacommand{by}\isamarkupfalse%
\ {\isacharparenleft}rule\ conjunct{\isadigit{2}}{\isacharparenright}\isanewline
\ \ \isacommand{then}\isamarkupfalse%
\ \isacommand{have}\isamarkupfalse%
\ EX{\isadigit{1}}{\isacharcolon}{\isachardoublequoteopen}{\isasymexists}k{\isachardot}\ S{\isacharprime}\ {\isasymsubseteq}\ pcp{\isacharunderscore}seq\ C\ S\ k{\isachardoublequoteclose}\isanewline
\ \ \ \ \isacommand{by}\isamarkupfalse%
\ {\isacharparenleft}simp\ only{\isacharcolon}\ insert{\isachardot}IH{\isacharparenright}\isanewline
\ \ \isacommand{obtain}\isamarkupfalse%
\ k{\isadigit{1}}\ \isakeyword{where}\ {\isachardoublequoteopen}S{\isacharprime}\ {\isasymsubseteq}\ pcp{\isacharunderscore}seq\ C\ S\ k{\isadigit{1}}{\isachardoublequoteclose}\isanewline
\ \ \ \ \isacommand{using}\isamarkupfalse%
\ EX{\isadigit{1}}\ \isacommand{by}\isamarkupfalse%
\ {\isacharparenleft}rule\ exE{\isacharparenright}\isanewline
\ \ \isacommand{have}\isamarkupfalse%
\ {\isachardoublequoteopen}F\ {\isasymin}\ pcp{\isacharunderscore}lim\ C\ S{\isachardoublequoteclose}\isanewline
\ \ \ \ \isacommand{using}\isamarkupfalse%
\ C\ \isacommand{by}\isamarkupfalse%
\ {\isacharparenleft}rule\ conjunct{\isadigit{1}}{\isacharparenright}\isanewline
\ \ \isacommand{then}\isamarkupfalse%
\ \isacommand{have}\isamarkupfalse%
\ EX{\isadigit{2}}{\isacharcolon}{\isachardoublequoteopen}{\isasymexists}k{\isachardot}\ F\ {\isasymin}\ pcp{\isacharunderscore}seq\ C\ S\ k{\isachardoublequoteclose}\isanewline
\ \ \ \ \isacommand{by}\isamarkupfalse%
\ {\isacharparenleft}rule\ pcp{\isacharunderscore}lim{\isacharunderscore}inserted{\isacharunderscore}at{\isacharunderscore}ex{\isacharparenright}\isanewline
\ \ \isacommand{obtain}\isamarkupfalse%
\ k{\isadigit{2}}\ \isakeyword{where}\ {\isachardoublequoteopen}F\ {\isasymin}\ pcp{\isacharunderscore}seq\ C\ S\ k{\isadigit{2}}{\isachardoublequoteclose}\ \isanewline
\ \ \ \ \isacommand{using}\isamarkupfalse%
\ EX{\isadigit{2}}\ \isacommand{by}\isamarkupfalse%
\ {\isacharparenleft}rule\ exE{\isacharparenright}\isanewline
\ \ \isacommand{have}\isamarkupfalse%
\ {\isachardoublequoteopen}k{\isadigit{1}}\ {\isasymle}\ max\ k{\isadigit{1}}\ k{\isadigit{2}}{\isachardoublequoteclose}\isanewline
\ \ \ \ \isacommand{by}\isamarkupfalse%
\ {\isacharparenleft}simp\ only{\isacharcolon}\ linorder{\isacharunderscore}class{\isachardot}max{\isachardot}cobounded{\isadigit{1}}{\isacharparenright}\isanewline
\ \ \isacommand{then}\isamarkupfalse%
\ \isacommand{have}\isamarkupfalse%
\ {\isachardoublequoteopen}pcp{\isacharunderscore}seq\ C\ S\ k{\isadigit{1}}\ {\isasymsubseteq}\ pcp{\isacharunderscore}seq\ C\ S\ {\isacharparenleft}max\ k{\isadigit{1}}\ k{\isadigit{2}}{\isacharparenright}{\isachardoublequoteclose}\isanewline
\ \ \ \ \isacommand{by}\isamarkupfalse%
\ {\isacharparenleft}rule\ pcp{\isacharunderscore}seq{\isacharunderscore}mono{\isacharparenright}\isanewline
\ \ \isacommand{have}\isamarkupfalse%
\ {\isachardoublequoteopen}k{\isadigit{2}}\ {\isasymle}\ max\ k{\isadigit{1}}\ k{\isadigit{2}}{\isachardoublequoteclose}\isanewline
\ \ \ \ \isacommand{by}\isamarkupfalse%
\ {\isacharparenleft}simp\ only{\isacharcolon}\ linorder{\isacharunderscore}class{\isachardot}max{\isachardot}cobounded{\isadigit{2}}{\isacharparenright}\isanewline
\ \ \isacommand{then}\isamarkupfalse%
\ \isacommand{have}\isamarkupfalse%
\ {\isachardoublequoteopen}pcp{\isacharunderscore}seq\ C\ S\ k{\isadigit{2}}\ {\isasymsubseteq}\ pcp{\isacharunderscore}seq\ C\ S\ {\isacharparenleft}max\ k{\isadigit{1}}\ k{\isadigit{2}}{\isacharparenright}{\isachardoublequoteclose}\isanewline
\ \ \ \ \isacommand{by}\isamarkupfalse%
\ {\isacharparenleft}rule\ pcp{\isacharunderscore}seq{\isacharunderscore}mono{\isacharparenright}\isanewline
\ \ \isacommand{have}\isamarkupfalse%
\ {\isachardoublequoteopen}S{\isacharprime}\ {\isasymsubseteq}\ pcp{\isacharunderscore}seq\ C\ S\ {\isacharparenleft}max\ k{\isadigit{1}}\ k{\isadigit{2}}{\isacharparenright}{\isachardoublequoteclose}\isanewline
\ \ \ \ \isacommand{using}\isamarkupfalse%
\ {\isacartoucheopen}S{\isacharprime}\ {\isasymsubseteq}\ pcp{\isacharunderscore}seq\ C\ S\ k{\isadigit{1}}{\isacartoucheclose}\ {\isacartoucheopen}pcp{\isacharunderscore}seq\ C\ S\ k{\isadigit{1}}\ {\isasymsubseteq}\ pcp{\isacharunderscore}seq\ C\ S\ {\isacharparenleft}max\ k{\isadigit{1}}\ k{\isadigit{2}}{\isacharparenright}{\isacartoucheclose}\ \isacommand{by}\isamarkupfalse%
\ {\isacharparenleft}rule\ subset{\isacharunderscore}trans{\isacharparenright}\isanewline
\ \ \isacommand{have}\isamarkupfalse%
\ {\isachardoublequoteopen}F\ {\isasymin}\ pcp{\isacharunderscore}seq\ C\ S\ {\isacharparenleft}max\ k{\isadigit{1}}\ k{\isadigit{2}}{\isacharparenright}{\isachardoublequoteclose}\isanewline
\ \ \ \ \isacommand{using}\isamarkupfalse%
\ {\isacartoucheopen}F\ {\isasymin}\ pcp{\isacharunderscore}seq\ C\ S\ k{\isadigit{2}}{\isacartoucheclose}\ {\isacartoucheopen}pcp{\isacharunderscore}seq\ C\ S\ k{\isadigit{2}}\ {\isasymsubseteq}\ pcp{\isacharunderscore}seq\ C\ S\ {\isacharparenleft}max\ k{\isadigit{1}}\ k{\isadigit{2}}{\isacharparenright}{\isacartoucheclose}\ \isacommand{by}\isamarkupfalse%
\ {\isacharparenleft}rule\ rev{\isacharunderscore}subsetD{\isacharparenright}\isanewline
\ \ \isacommand{then}\isamarkupfalse%
\ \isacommand{have}\isamarkupfalse%
\ {\isadigit{1}}{\isacharcolon}{\isachardoublequoteopen}insert\ F\ S{\isacharprime}\ {\isasymsubseteq}\ pcp{\isacharunderscore}seq\ C\ S\ {\isacharparenleft}max\ k{\isadigit{1}}\ k{\isadigit{2}}{\isacharparenright}{\isachardoublequoteclose}\isanewline
\ \ \ \ \isacommand{using}\isamarkupfalse%
\ {\isacartoucheopen}S{\isacharprime}\ {\isasymsubseteq}\ pcp{\isacharunderscore}seq\ C\ S\ {\isacharparenleft}max\ k{\isadigit{1}}\ k{\isadigit{2}}{\isacharparenright}{\isacartoucheclose}\ \isacommand{by}\isamarkupfalse%
\ {\isacharparenleft}simp\ only{\isacharcolon}\ insert{\isacharunderscore}subset{\isacharparenright}\isanewline
\ \ \isacommand{thus}\isamarkupfalse%
\ {\isacharquery}case\isanewline
\ \ \ \ \isacommand{by}\isamarkupfalse%
\ {\isacharparenleft}rule\ exI{\isacharparenright}\isanewline
\isacommand{qed}\isamarkupfalse%
%
\endisatagproof
{\isafoldproof}%
%
\isadelimproof
%
\endisadelimproof
%
\begin{isamarkuptext}%
Finalmente, su demostración automática en Isabelle/HOL es la siguiente.%
\end{isamarkuptext}\isamarkuptrue%
\isacommand{lemma}\isamarkupfalse%
\ finite{\isacharunderscore}pcp{\isacharunderscore}lim{\isacharunderscore}EX{\isacharcolon}\isanewline
\ \ \isakeyword{assumes}\ {\isachardoublequoteopen}finite\ S{\isacharprime}{\isachardoublequoteclose}\isanewline
\ \ \ \ \ \ \ \ \ \ {\isachardoublequoteopen}S{\isacharprime}\ {\isasymsubseteq}\ pcp{\isacharunderscore}lim\ C\ S{\isachardoublequoteclose}\isanewline
\ \ \ \ \ \ \ \ \isakeyword{shows}\ {\isachardoublequoteopen}{\isasymexists}k{\isachardot}\ S{\isacharprime}\ {\isasymsubseteq}\ pcp{\isacharunderscore}seq\ C\ S\ k{\isachardoublequoteclose}\isanewline
%
\isadelimproof
\ \ %
\endisadelimproof
%
\isatagproof
\isacommand{using}\isamarkupfalse%
\ assms\isanewline
\isacommand{proof}\isamarkupfalse%
{\isacharparenleft}induction\ S{\isacharprime}\ rule{\isacharcolon}\ finite{\isacharunderscore}induct{\isacharparenright}\ \isanewline
\ \ \isacommand{case}\isamarkupfalse%
\ {\isacharparenleft}insert\ F\ S{\isacharprime}{\isacharparenright}\isanewline
\ \ \isacommand{hence}\isamarkupfalse%
\ {\isachardoublequoteopen}{\isasymexists}k{\isachardot}\ S{\isacharprime}\ {\isasymsubseteq}\ pcp{\isacharunderscore}seq\ C\ S\ k{\isachardoublequoteclose}\ \isacommand{by}\isamarkupfalse%
\ fast\isanewline
\ \ \isacommand{then}\isamarkupfalse%
\ \isacommand{guess}\isamarkupfalse%
\ k{\isadigit{1}}\ \isacommand{{\isachardot}{\isachardot}}\isamarkupfalse%
\isanewline
\ \ \isacommand{moreover}\isamarkupfalse%
\ \isacommand{obtain}\isamarkupfalse%
\ k{\isadigit{2}}\ \isakeyword{where}\ {\isachardoublequoteopen}F\ {\isasymin}\ pcp{\isacharunderscore}seq\ C\ S\ k{\isadigit{2}}{\isachardoublequoteclose}\isanewline
\ \ \ \ \isacommand{by}\isamarkupfalse%
\ {\isacharparenleft}meson\ pcp{\isacharunderscore}lim{\isacharunderscore}inserted{\isacharunderscore}at{\isacharunderscore}ex\ insert{\isachardot}prems\ insert{\isacharunderscore}subset{\isacharparenright}\isanewline
\ \ \isacommand{ultimately}\isamarkupfalse%
\ \isacommand{have}\isamarkupfalse%
\ {\isachardoublequoteopen}insert\ F\ S{\isacharprime}\ {\isasymsubseteq}\ pcp{\isacharunderscore}seq\ C\ S\ {\isacharparenleft}max\ k{\isadigit{1}}\ k{\isadigit{2}}{\isacharparenright}{\isachardoublequoteclose}\isanewline
\ \ \ \ \isacommand{by}\isamarkupfalse%
\ {\isacharparenleft}meson\ pcp{\isacharunderscore}seq{\isacharunderscore}mono\ dual{\isacharunderscore}order{\isachardot}trans\ insert{\isacharunderscore}subset\ max{\isachardot}bounded{\isacharunderscore}iff\ order{\isacharunderscore}refl\ subsetCE{\isacharparenright}\isanewline
\ \ \isacommand{thus}\isamarkupfalse%
\ {\isacharquery}case\ \isacommand{by}\isamarkupfalse%
\ blast\isanewline
\isacommand{qed}\isamarkupfalse%
\ simp%
\endisatagproof
{\isafoldproof}%
%
\isadelimproof
%
\endisadelimproof
%
\isadelimdocument
%
\endisadelimdocument
%
\isatagdocument
%
\isamarkupsection{El teorema de existencia de modelo%
}
\isamarkuptrue%
%
\endisatagdocument
{\isafolddocument}%
%
\isadelimdocument
%
\endisadelimdocument
%
\begin{isamarkuptext}%
En esta sección demostraremos finalmente el 
  \isa{teorema\ de\ existencia\ de\ modelo}, el cual prueba que todo conjunto de fórmulas perteneciente a 
  una colección que verifique la propiedad de consistencia proposicional es satisfacible. Para ello, 
  considerando una colección \isa{C} cualquiera y \isa{S\ {\isasymin}\ C}, empleando resultados anteriores extenderemos 
  la colección a una colección \isa{C{\isacharprime}{\isacharprime}} que tenga la propiedad de consistencia proposicional, sea
  cerrada bajo subconjuntos y sea de carácter finito. De este modo, en esta sección probaremos que el 
  límite de la sucesión formada a partir de una colección que tenga dichas condiciones y un conjunto
  cualquiera \isa{S} como se indica en la definición \isa{{\isadigit{1}}{\isachardot}{\isadigit{4}}{\isachardot}{\isadigit{1}}} pertenece a la colección. Es más, 
  demostraremos que dicho límite se trata de un conjunto de \isa{Hintikka} luego, por el \isa{teorema\ de\ Hintikka}, es satisfacible. Finalmente, como \isa{S} está contenido en el límite, quedará demostrada 
  la satisfacibilidad del conjunto \isa{S} al heredarla por contención.

  \comentario{Habrá que modificar el párrafo anterior al final.}%
\end{isamarkuptext}\isamarkuptrue%
%
\begin{isamarkuptext}%
En primer lugar, probemos que si \isa{C} es una colección que verifica la propiedad de 
  consistencia proposicional, es cerrada bajo subconjuntos y es de carácter finito, entonces el 
  límite de toda sucesión de conjuntos de \isa{C} según la definición \isa{{\isadigit{4}}{\isachardot}{\isadigit{1}}{\isachardot}{\isadigit{1}}} pertenece a \isa{C}.

  \begin{lema}
    Sea \isa{C} una colección de conjuntos que verifica la propiedad de consistencia proposicional, es 
    cerrada bajo subconjuntos y es de carácter finito. Sea \isa{S\ {\isasymin}\ C} y \isa{{\isacharbraceleft}S\isactrlsub n{\isacharbraceright}} la sucesión de conjuntos
    de \isa{C} a partir de \isa{S} según la definición \isa{{\isadigit{4}}{\isachardot}{\isadigit{1}}{\isachardot}{\isadigit{1}}}. Entonces, el límite de la sucesión está en
    \isa{C}.
  \end{lema}

  \begin{demostracion}
    Por definición, como \isa{C} es de carácter finito, para todo conjunto son equivalentes:
    \begin{enumerate}
      \item El conjunto pertenece a \isa{C}.
      \item Todo subconjunto finito suyo pertenece a \isa{C}.
    \end{enumerate}

    De este modo, para demostrar que el límite de la sucesión \isa{{\isacharbraceleft}S\isactrlsub n{\isacharbraceright}} pertenece a \isa{C}, basta probar
    que todo subconjunto finito suyo está en \isa{C}.

    Sea \isa{S{\isacharprime}} un subconjunto finito del límite de la sucesión. Por el lema \isa{{\isadigit{1}}{\isachardot}{\isadigit{4}}{\isachardot}{\isadigit{8}}}, existe un
    \isa{k\ {\isasymin}\ {\isasymnat}} tal que \isa{S{\isacharprime}\ {\isasymsubseteq}\ S\isactrlsub k}. Por tanto, como \isa{S\isactrlsub k\ {\isasymin}\ C} para todo \isa{k\ {\isasymin}\ {\isasymnat}} y \isa{C} es cerrada bajo
    subconjuntos, por definición se tiene que \isa{S{\isacharprime}\ {\isasymin}\ C}, como queríamos demostrar.
  \end{demostracion}

  En Isabelle se formaliza y demuestra detalladamente como sigue.%
\end{isamarkuptext}\isamarkuptrue%
\isacommand{lemma}\isamarkupfalse%
\isanewline
\ \ \isakeyword{assumes}\ {\isachardoublequoteopen}pcp\ C{\isachardoublequoteclose}\isanewline
\ \ \ \ \ \ \ \ \ \ {\isachardoublequoteopen}S\ {\isasymin}\ C{\isachardoublequoteclose}\isanewline
\ \ \ \ \ \ \ \ \ \ {\isachardoublequoteopen}subset{\isacharunderscore}closed\ C{\isachardoublequoteclose}\isanewline
\ \ \ \ \ \ \ \ \ \ {\isachardoublequoteopen}finite{\isacharunderscore}character\ C{\isachardoublequoteclose}\isanewline
\ \ \isakeyword{shows}\ {\isachardoublequoteopen}pcp{\isacharunderscore}lim\ C\ S\ {\isasymin}\ C{\isachardoublequoteclose}\ \isanewline
%
\isadelimproof
%
\endisadelimproof
%
\isatagproof
\isacommand{proof}\isamarkupfalse%
\ {\isacharminus}\isanewline
\ \ \isacommand{have}\isamarkupfalse%
\ {\isachardoublequoteopen}{\isasymforall}S{\isachardot}\ S\ {\isasymin}\ C\ {\isasymlongleftrightarrow}\ {\isacharparenleft}{\isasymforall}S{\isacharprime}\ {\isasymsubseteq}\ S{\isachardot}\ finite\ S{\isacharprime}\ {\isasymlongrightarrow}\ S{\isacharprime}\ {\isasymin}\ C{\isacharparenright}{\isachardoublequoteclose}\isanewline
\ \ \ \ \isacommand{using}\isamarkupfalse%
\ assms{\isacharparenleft}{\isadigit{4}}{\isacharparenright}\ \isacommand{unfolding}\isamarkupfalse%
\ finite{\isacharunderscore}character{\isacharunderscore}def\ \isacommand{by}\isamarkupfalse%
\ this\isanewline
\ \ \isacommand{then}\isamarkupfalse%
\ \isacommand{have}\isamarkupfalse%
\ FC{\isadigit{1}}{\isacharcolon}{\isachardoublequoteopen}pcp{\isacharunderscore}lim\ C\ S\ {\isasymin}\ C\ {\isasymlongleftrightarrow}\ {\isacharparenleft}{\isasymforall}S{\isacharprime}\ {\isasymsubseteq}\ {\isacharparenleft}pcp{\isacharunderscore}lim\ C\ S{\isacharparenright}{\isachardot}\ finite\ S{\isacharprime}\ {\isasymlongrightarrow}\ S{\isacharprime}\ {\isasymin}\ C{\isacharparenright}{\isachardoublequoteclose}\isanewline
\ \ \ \ \isacommand{by}\isamarkupfalse%
\ {\isacharparenleft}rule\ allE{\isacharparenright}\isanewline
\ \ \isacommand{have}\isamarkupfalse%
\ SC{\isacharcolon}{\isachardoublequoteopen}{\isasymforall}S\ {\isasymin}\ C{\isachardot}\ {\isasymforall}S{\isacharprime}{\isasymsubseteq}S{\isachardot}\ S{\isacharprime}\ {\isasymin}\ C{\isachardoublequoteclose}\isanewline
\ \ \ \ \isacommand{using}\isamarkupfalse%
\ assms{\isacharparenleft}{\isadigit{3}}{\isacharparenright}\ \isacommand{unfolding}\isamarkupfalse%
\ subset{\isacharunderscore}closed{\isacharunderscore}def\ \isacommand{by}\isamarkupfalse%
\ this\isanewline
\ \ \isacommand{have}\isamarkupfalse%
\ FC{\isadigit{2}}{\isacharcolon}{\isachardoublequoteopen}{\isasymforall}S{\isacharprime}\ {\isasymsubseteq}\ pcp{\isacharunderscore}lim\ C\ S{\isachardot}\ finite\ S{\isacharprime}\ {\isasymlongrightarrow}\ S{\isacharprime}\ {\isasymin}\ C{\isachardoublequoteclose}\isanewline
\ \ \isacommand{proof}\isamarkupfalse%
\ {\isacharparenleft}rule\ sallI{\isacharparenright}\isanewline
\ \ \ \ \isacommand{fix}\isamarkupfalse%
\ S{\isacharprime}\ {\isacharcolon}{\isacharcolon}\ {\isachardoublequoteopen}{\isacharprime}a\ formula\ set{\isachardoublequoteclose}\isanewline
\ \ \ \ \isacommand{assume}\isamarkupfalse%
\ {\isachardoublequoteopen}S{\isacharprime}\ {\isasymsubseteq}\ pcp{\isacharunderscore}lim\ C\ S{\isachardoublequoteclose}\isanewline
\ \ \ \ \isacommand{show}\isamarkupfalse%
\ {\isachardoublequoteopen}finite\ S{\isacharprime}\ {\isasymlongrightarrow}\ S{\isacharprime}\ {\isasymin}\ C{\isachardoublequoteclose}\isanewline
\ \ \ \ \isacommand{proof}\isamarkupfalse%
\ {\isacharparenleft}rule\ impI{\isacharparenright}\isanewline
\ \ \ \ \ \ \isacommand{assume}\isamarkupfalse%
\ {\isachardoublequoteopen}finite\ S{\isacharprime}{\isachardoublequoteclose}\isanewline
\ \ \ \ \ \ \isacommand{then}\isamarkupfalse%
\ \isacommand{have}\isamarkupfalse%
\ EX{\isacharcolon}{\isachardoublequoteopen}{\isasymexists}k{\isachardot}\ S{\isacharprime}\ {\isasymsubseteq}\ pcp{\isacharunderscore}seq\ C\ S\ k{\isachardoublequoteclose}\ \isanewline
\ \ \ \ \ \ \ \ \isacommand{using}\isamarkupfalse%
\ {\isacartoucheopen}S{\isacharprime}\ {\isasymsubseteq}\ pcp{\isacharunderscore}lim\ C\ S{\isacartoucheclose}\ \isacommand{by}\isamarkupfalse%
\ {\isacharparenleft}rule\ finite{\isacharunderscore}pcp{\isacharunderscore}lim{\isacharunderscore}EX{\isacharparenright}\isanewline
\ \ \ \ \ \ \isacommand{obtain}\isamarkupfalse%
\ k\ \isakeyword{where}\ {\isachardoublequoteopen}S{\isacharprime}\ {\isasymsubseteq}\ pcp{\isacharunderscore}seq\ C\ S\ k{\isachardoublequoteclose}\isanewline
\ \ \ \ \ \ \ \ \isacommand{using}\isamarkupfalse%
\ EX\ \isacommand{by}\isamarkupfalse%
\ {\isacharparenleft}rule\ exE{\isacharparenright}\isanewline
\ \ \ \ \ \ \isacommand{have}\isamarkupfalse%
\ {\isachardoublequoteopen}pcp{\isacharunderscore}seq\ C\ S\ k\ {\isasymin}\ C{\isachardoublequoteclose}\isanewline
\ \ \ \ \ \ \ \ \isacommand{using}\isamarkupfalse%
\ assms{\isacharparenleft}{\isadigit{1}}{\isacharparenright}\ assms{\isacharparenleft}{\isadigit{2}}{\isacharparenright}\ \isacommand{by}\isamarkupfalse%
\ {\isacharparenleft}rule\ pcp{\isacharunderscore}seq{\isacharunderscore}in{\isacharparenright}\isanewline
\ \ \ \ \ \ \isacommand{have}\isamarkupfalse%
\ {\isachardoublequoteopen}{\isasymforall}S{\isacharprime}\ {\isasymsubseteq}\ {\isacharparenleft}pcp{\isacharunderscore}seq\ C\ S\ k{\isacharparenright}{\isachardot}\ S{\isacharprime}\ {\isasymin}\ C{\isachardoublequoteclose}\isanewline
\ \ \ \ \ \ \ \ \isacommand{using}\isamarkupfalse%
\ SC\ {\isacartoucheopen}pcp{\isacharunderscore}seq\ C\ S\ k\ {\isasymin}\ C{\isacartoucheclose}\ \isacommand{by}\isamarkupfalse%
\ {\isacharparenleft}rule\ bspec{\isacharparenright}\isanewline
\ \ \ \ \ \ \isacommand{thus}\isamarkupfalse%
\ {\isachardoublequoteopen}S{\isacharprime}\ {\isasymin}\ C{\isachardoublequoteclose}\isanewline
\ \ \ \ \ \ \ \ \isacommand{using}\isamarkupfalse%
\ {\isacartoucheopen}S{\isacharprime}\ {\isasymsubseteq}\ pcp{\isacharunderscore}seq\ C\ S\ k{\isacartoucheclose}\ \isacommand{by}\isamarkupfalse%
\ {\isacharparenleft}rule\ sspec{\isacharparenright}\isanewline
\ \ \ \ \isacommand{qed}\isamarkupfalse%
\isanewline
\ \ \isacommand{qed}\isamarkupfalse%
\isanewline
\ \ \isacommand{show}\isamarkupfalse%
\ {\isachardoublequoteopen}pcp{\isacharunderscore}lim\ C\ S\ {\isasymin}\ C{\isachardoublequoteclose}\ \isanewline
\ \ \ \ \isacommand{using}\isamarkupfalse%
\ FC{\isadigit{1}}\ FC{\isadigit{2}}\ \isacommand{by}\isamarkupfalse%
\ {\isacharparenleft}rule\ forw{\isacharunderscore}subst{\isacharparenright}\isanewline
\isacommand{qed}\isamarkupfalse%
%
\endisatagproof
{\isafoldproof}%
%
\isadelimproof
%
\endisadelimproof
%
\begin{isamarkuptext}%
Por otra parte, podemos dar una prueba automática del resultado.%
\end{isamarkuptext}\isamarkuptrue%
\isacommand{lemma}\isamarkupfalse%
\ pcp{\isacharunderscore}lim{\isacharunderscore}in{\isacharcolon}\isanewline
\ \ \isakeyword{assumes}\ c{\isacharcolon}\ {\isachardoublequoteopen}pcp\ C{\isachardoublequoteclose}\isanewline
\ \ \isakeyword{and}\ el{\isacharcolon}\ {\isachardoublequoteopen}S\ {\isasymin}\ C{\isachardoublequoteclose}\isanewline
\ \ \isakeyword{and}\ sc{\isacharcolon}\ {\isachardoublequoteopen}subset{\isacharunderscore}closed\ C{\isachardoublequoteclose}\isanewline
\ \ \isakeyword{and}\ fc{\isacharcolon}\ {\isachardoublequoteopen}finite{\isacharunderscore}character\ C{\isachardoublequoteclose}\isanewline
\ \ \isakeyword{shows}\ {\isachardoublequoteopen}pcp{\isacharunderscore}lim\ C\ S\ {\isasymin}\ C{\isachardoublequoteclose}\ {\isacharparenleft}\isakeyword{is}\ {\isachardoublequoteopen}{\isacharquery}cl\ {\isasymin}\ C{\isachardoublequoteclose}{\isacharparenright}\isanewline
%
\isadelimproof
%
\endisadelimproof
%
\isatagproof
\isacommand{proof}\isamarkupfalse%
\ {\isacharminus}\isanewline
\ \ \isacommand{from}\isamarkupfalse%
\ pcp{\isacharunderscore}seq{\isacharunderscore}in{\isacharbrackleft}OF\ c\ el{\isacharcomma}\ THEN\ allI{\isacharbrackright}\ \isacommand{have}\isamarkupfalse%
\ {\isachardoublequoteopen}{\isasymforall}n{\isachardot}\ pcp{\isacharunderscore}seq\ C\ S\ n\ {\isasymin}\ C{\isachardoublequoteclose}\ \isacommand{{\isachardot}}\isamarkupfalse%
\isanewline
\ \ \isacommand{hence}\isamarkupfalse%
\ {\isachardoublequoteopen}{\isasymforall}m{\isachardot}\ {\isasymUnion}{\isacharbraceleft}pcp{\isacharunderscore}seq\ C\ S\ n{\isacharbar}n{\isachardot}\ n\ {\isasymle}\ m{\isacharbraceright}\ {\isasymin}\ C{\isachardoublequoteclose}\ \isacommand{unfolding}\isamarkupfalse%
\ pcp{\isacharunderscore}seq{\isacharunderscore}UN\ \isacommand{{\isachardot}}\isamarkupfalse%
\isanewline
\ \ \isacommand{have}\isamarkupfalse%
\ {\isachardoublequoteopen}{\isasymforall}S{\isacharprime}\ {\isasymsubseteq}\ {\isacharquery}cl{\isachardot}\ finite\ S{\isacharprime}\ {\isasymlongrightarrow}\ S{\isacharprime}\ {\isasymin}\ C{\isachardoublequoteclose}\isanewline
\ \ \isacommand{proof}\isamarkupfalse%
\ safe\isanewline
\ \ \ \ \isacommand{fix}\isamarkupfalse%
\ S{\isacharprime}\ {\isacharcolon}{\isacharcolon}\ {\isachardoublequoteopen}{\isacharprime}a\ formula\ set{\isachardoublequoteclose}\isanewline
\ \ \ \ \isacommand{have}\isamarkupfalse%
\ {\isachardoublequoteopen}pcp{\isacharunderscore}seq\ C\ S\ {\isacharparenleft}Suc\ {\isacharparenleft}Max\ {\isacharparenleft}to{\isacharunderscore}nat\ {\isacharbackquote}\ S{\isacharprime}{\isacharparenright}{\isacharparenright}{\isacharparenright}\ {\isasymsubseteq}\ pcp{\isacharunderscore}lim\ C\ S{\isachardoublequoteclose}\ \isanewline
\ \ \ \ \ \ \isacommand{using}\isamarkupfalse%
\ pcp{\isacharunderscore}seq{\isacharunderscore}sub\ \isacommand{by}\isamarkupfalse%
\ blast\isanewline
\ \ \ \ \isacommand{assume}\isamarkupfalse%
\ {\isacartoucheopen}finite\ S{\isacharprime}{\isacartoucheclose}\ {\isacartoucheopen}S{\isacharprime}\ {\isasymsubseteq}\ pcp{\isacharunderscore}lim\ C\ S{\isacartoucheclose}\isanewline
\ \ \ \ \isacommand{hence}\isamarkupfalse%
\ {\isachardoublequoteopen}{\isasymexists}k{\isachardot}\ S{\isacharprime}\ {\isasymsubseteq}\ pcp{\isacharunderscore}seq\ C\ S\ k{\isachardoublequoteclose}\ \isanewline
\ \ \ \ \isacommand{proof}\isamarkupfalse%
{\isacharparenleft}induction\ S{\isacharprime}\ rule{\isacharcolon}\ finite{\isacharunderscore}induct{\isacharparenright}\ \isanewline
\ \ \ \ \ \ \isacommand{case}\isamarkupfalse%
\ {\isacharparenleft}insert\ x\ S{\isacharprime}{\isacharparenright}\isanewline
\ \ \ \ \ \ \isacommand{hence}\isamarkupfalse%
\ {\isachardoublequoteopen}{\isasymexists}k{\isachardot}\ S{\isacharprime}\ {\isasymsubseteq}\ pcp{\isacharunderscore}seq\ C\ S\ k{\isachardoublequoteclose}\ \isacommand{by}\isamarkupfalse%
\ fast\isanewline
\ \ \ \ \ \ \isacommand{then}\isamarkupfalse%
\ \isacommand{guess}\isamarkupfalse%
\ k{\isadigit{1}}\ \isacommand{{\isachardot}{\isachardot}}\isamarkupfalse%
\isanewline
\ \ \ \ \ \ \isacommand{moreover}\isamarkupfalse%
\ \isacommand{obtain}\isamarkupfalse%
\ k{\isadigit{2}}\ \isakeyword{where}\ {\isachardoublequoteopen}x\ {\isasymin}\ pcp{\isacharunderscore}seq\ C\ S\ k{\isadigit{2}}{\isachardoublequoteclose}\isanewline
\ \ \ \ \ \ \ \ \isacommand{by}\isamarkupfalse%
\ {\isacharparenleft}meson\ pcp{\isacharunderscore}lim{\isacharunderscore}inserted{\isacharunderscore}at{\isacharunderscore}ex\ insert{\isachardot}prems\ insert{\isacharunderscore}subset{\isacharparenright}\isanewline
\ \ \ \ \ \ \isacommand{ultimately}\isamarkupfalse%
\ \isacommand{have}\isamarkupfalse%
\ {\isachardoublequoteopen}insert\ x\ S{\isacharprime}\ {\isasymsubseteq}\ pcp{\isacharunderscore}seq\ C\ S\ {\isacharparenleft}max\ k{\isadigit{1}}\ k{\isadigit{2}}{\isacharparenright}{\isachardoublequoteclose}\isanewline
\ \ \ \ \ \ \ \ \isacommand{by}\isamarkupfalse%
\ {\isacharparenleft}meson\ pcp{\isacharunderscore}seq{\isacharunderscore}mono\ dual{\isacharunderscore}order{\isachardot}trans\ insert{\isacharunderscore}subset\ max{\isachardot}bounded{\isacharunderscore}iff\ order{\isacharunderscore}refl\ subsetCE{\isacharparenright}\isanewline
\ \ \ \ \ \ \isacommand{thus}\isamarkupfalse%
\ {\isacharquery}case\ \isacommand{by}\isamarkupfalse%
\ blast\isanewline
\ \ \ \ \isacommand{qed}\isamarkupfalse%
\ simp\isanewline
\ \ \ \ \isacommand{with}\isamarkupfalse%
\ pcp{\isacharunderscore}seq{\isacharunderscore}in{\isacharbrackleft}OF\ c\ el{\isacharbrackright}\ sc\isanewline
\ \ \ \ \isacommand{show}\isamarkupfalse%
\ {\isachardoublequoteopen}S{\isacharprime}\ {\isasymin}\ C{\isachardoublequoteclose}\ \isacommand{unfolding}\isamarkupfalse%
\ subset{\isacharunderscore}closed{\isacharunderscore}def\ \isacommand{by}\isamarkupfalse%
\ blast\isanewline
\ \ \isacommand{qed}\isamarkupfalse%
\isanewline
\ \ \isacommand{thus}\isamarkupfalse%
\ {\isachardoublequoteopen}{\isacharquery}cl\ {\isasymin}\ C{\isachardoublequoteclose}\ \isacommand{using}\isamarkupfalse%
\ fc\ \isacommand{unfolding}\isamarkupfalse%
\ finite{\isacharunderscore}character{\isacharunderscore}def\ \isacommand{by}\isamarkupfalse%
\ blast\isanewline
\isacommand{qed}\isamarkupfalse%
%
\endisatagproof
{\isafoldproof}%
%
\isadelimproof
%
\endisadelimproof
%
\begin{isamarkuptext}%
Probemos que, además, el límite de las sucesión definida en \isa{{\isadigit{4}}{\isachardot}{\isadigit{1}}{\isachardot}{\isadigit{1}}} se trata de un elemento 
  maximal de la colección que lo define si esta verifica la propiedad de consistencia proposicional
  y es cerrada bajo subconjuntos.

  \begin{lema}
    Sea \isa{C} una colección de conjuntos que verifica la propiedad de consistencia proposicional y
    es cerrada bajo subconjuntos, \isa{S} un conjunto y \isa{{\isacharbraceleft}S\isactrlsub n{\isacharbraceright}} la sucesión de conjuntos de \isa{C} a partir 
    de \isa{S} según la definición \isa{{\isadigit{4}}{\isachardot}{\isadigit{1}}{\isachardot}{\isadigit{1}}}. Entonces, el límite de la sucesión \isa{{\isacharbraceleft}S\isactrlsub n{\isacharbraceright}} es un elemento 
    maximal de \isa{C}.
  \end{lema}

  \begin{demostracion}
    Por definición de elemento maximal, basta probar que para cualquier conjunto \isa{K\ {\isasymin}\ C} que
    contenga al límite de la sucesión se tiene que \isa{K} y el límite coinciden.

    La demostración se realizará por reducción al absurdo. Consideremos un conjunto \isa{K\ {\isasymin}\ C} que 
    contenga estrictamente al límite de la sucesión \isa{{\isacharbraceleft}S\isactrlsub n{\isacharbraceright}}. De este modo, existe una fórmula \isa{F} tal 
    que \isa{F\ {\isasymin}\ K} y \isa{F} no está en el límite. Supongamos que \isa{F} es la \isa{n}-ésima fórmula según la 
    enumeración de la definición \isa{{\isadigit{4}}{\isachardot}{\isadigit{1}}{\isachardot}{\isadigit{1}}} utilizada para construir la sucesión. 

    Por un lado, hemos probado que todo elemento de la sucesión está contenido en el límite, luego 
    en particular obtenemos que \isa{S\isactrlsub n\isactrlsub {\isacharplus}\isactrlsub {\isadigit{1}}} está contenido en el límite. De este modo, como \isa{F} no 
    pertenece al límite, es claro que \isa{F\ {\isasymnotin}\ S\isactrlsub n\isactrlsub {\isacharplus}\isactrlsub {\isadigit{1}}}. Además, \isa{{\isacharbraceleft}F{\isacharbraceright}\ {\isasymunion}\ S\isactrlsub n\ {\isasymnotin}\ C} ya que, en caso contrario, 
    por la definición \isa{{\isadigit{4}}{\isachardot}{\isadigit{1}}{\isachardot}{\isadigit{1}}} de la sucesión obtendríamos que\\ \isa{S\isactrlsub n\isactrlsub {\isacharplus}\isactrlsub {\isadigit{1}}\ {\isacharequal}\ {\isacharbraceleft}F{\isacharbraceright}\ {\isasymunion}\ S\isactrlsub n}, lo que contradice 
    que \isa{F\ {\isasymnotin}\ S\isactrlsub n\isactrlsub {\isacharplus}\isactrlsub {\isadigit{1}}}. 

    Por otro lado, como \isa{S\isactrlsub n} también está contenida en el límite que, a su vez, está contenido en 
    \isa{K}, se obtiene por transitividad que \isa{S\isactrlsub n\ {\isasymsubseteq}\ K}. Además, como \isa{F\ {\isasymin}\ K}, se verifica que 
    \isa{{\isacharbraceleft}F{\isacharbraceright}\ {\isasymunion}\ S\isactrlsub n\ {\isasymsubseteq}\ K}. Como \isa{C} es una colección cerrada bajo subconjuntos por hipótesis y \isa{K\ {\isasymin}\ C}, 
    por definición se tiene que \isa{{\isacharbraceleft}F{\isacharbraceright}\ {\isasymunion}\ S\isactrlsub n\ {\isasymin}\ C}, llegando así a una contradicción con lo demostrado 
    anteriormente.
  \end{demostracion}

  Su formalización y prueba detallada en Isabelle/HOL se muestran a continuación.%
\end{isamarkuptext}\isamarkuptrue%
\isacommand{lemma}\isamarkupfalse%
\isanewline
\ \ \isakeyword{assumes}\ {\isachardoublequoteopen}pcp\ C{\isachardoublequoteclose}\isanewline
\ \ \ \ \ \ \ \ \ \ {\isachardoublequoteopen}subset{\isacharunderscore}closed\ C{\isachardoublequoteclose}\isanewline
\ \ \ \ \ \ \ \ \ \ {\isachardoublequoteopen}K\ {\isasymin}\ C{\isachardoublequoteclose}\isanewline
\ \ \ \ \ \ \ \ \ \ {\isachardoublequoteopen}pcp{\isacharunderscore}lim\ C\ S\ {\isasymsubseteq}\ K{\isachardoublequoteclose}\isanewline
\ \ \isakeyword{shows}\ {\isachardoublequoteopen}pcp{\isacharunderscore}lim\ C\ S\ {\isacharequal}\ K{\isachardoublequoteclose}\isanewline
%
\isadelimproof
%
\endisadelimproof
%
\isatagproof
\isacommand{proof}\isamarkupfalse%
\ {\isacharparenleft}rule\ ccontr{\isacharparenright}\isanewline
\ \ \isacommand{assume}\isamarkupfalse%
\ H{\isacharcolon}{\isachardoublequoteopen}{\isasymnot}{\isacharparenleft}pcp{\isacharunderscore}lim\ C\ S\ {\isacharequal}\ K{\isacharparenright}{\isachardoublequoteclose}\isanewline
\ \ \isacommand{have}\isamarkupfalse%
\ CE{\isacharcolon}{\isachardoublequoteopen}pcp{\isacharunderscore}lim\ C\ S\ {\isasymsubseteq}\ K\ {\isasymand}\ pcp{\isacharunderscore}lim\ C\ S\ {\isasymnoteq}\ K{\isachardoublequoteclose}\isanewline
\ \ \ \ \isacommand{using}\isamarkupfalse%
\ assms{\isacharparenleft}{\isadigit{4}}{\isacharparenright}\ H\ \isacommand{by}\isamarkupfalse%
\ {\isacharparenleft}rule\ conjI{\isacharparenright}\isanewline
\ \ \isacommand{have}\isamarkupfalse%
\ {\isachardoublequoteopen}pcp{\isacharunderscore}lim\ C\ S\ {\isasymsubseteq}\ K\ {\isasymand}\ pcp{\isacharunderscore}lim\ C\ S\ {\isasymnoteq}\ K\ {\isasymlongleftrightarrow}\ pcp{\isacharunderscore}lim\ C\ S\ {\isasymsubset}\ K{\isachardoublequoteclose}\isanewline
\ \ \ \ \isacommand{by}\isamarkupfalse%
\ {\isacharparenleft}simp\ only{\isacharcolon}\ psubset{\isacharunderscore}eq{\isacharparenright}\isanewline
\ \ \isacommand{then}\isamarkupfalse%
\ \isacommand{have}\isamarkupfalse%
\ {\isachardoublequoteopen}pcp{\isacharunderscore}lim\ C\ S\ {\isasymsubset}\ K{\isachardoublequoteclose}\ \isanewline
\ \ \ \ \isacommand{using}\isamarkupfalse%
\ CE\ \isacommand{by}\isamarkupfalse%
\ {\isacharparenleft}rule\ iffD{\isadigit{1}}{\isacharparenright}\isanewline
\ \ \isacommand{then}\isamarkupfalse%
\ \isacommand{have}\isamarkupfalse%
\ {\isachardoublequoteopen}{\isasymexists}F{\isachardot}\ F\ {\isasymin}\ {\isacharparenleft}K\ {\isacharminus}\ {\isacharparenleft}pcp{\isacharunderscore}lim\ C\ S{\isacharparenright}{\isacharparenright}{\isachardoublequoteclose}\isanewline
\ \ \ \ \isacommand{by}\isamarkupfalse%
\ {\isacharparenleft}simp\ only{\isacharcolon}\ psubset{\isacharunderscore}imp{\isacharunderscore}ex{\isacharunderscore}mem{\isacharparenright}\ \isanewline
\ \ \isacommand{then}\isamarkupfalse%
\ \isacommand{have}\isamarkupfalse%
\ E{\isacharcolon}{\isachardoublequoteopen}{\isasymexists}F{\isachardot}\ F\ {\isasymin}\ K\ {\isasymand}\ F\ {\isasymnotin}\ {\isacharparenleft}pcp{\isacharunderscore}lim\ C\ S{\isacharparenright}{\isachardoublequoteclose}\isanewline
\ \ \ \ \isacommand{by}\isamarkupfalse%
\ {\isacharparenleft}simp\ only{\isacharcolon}\ Diff{\isacharunderscore}iff{\isacharparenright}\isanewline
\ \ \isacommand{obtain}\isamarkupfalse%
\ F\ \isakeyword{where}\ F{\isacharcolon}{\isachardoublequoteopen}F\ {\isasymin}\ K\ {\isasymand}\ F\ {\isasymnotin}\ pcp{\isacharunderscore}lim\ C\ S{\isachardoublequoteclose}\ \isanewline
\ \ \ \ \isacommand{using}\isamarkupfalse%
\ E\ \isacommand{by}\isamarkupfalse%
\ {\isacharparenleft}rule\ exE{\isacharparenright}\isanewline
\ \ \isacommand{have}\isamarkupfalse%
\ {\isachardoublequoteopen}F\ {\isasymin}\ K{\isachardoublequoteclose}\ \isanewline
\ \ \ \ \isacommand{using}\isamarkupfalse%
\ F\ \isacommand{by}\isamarkupfalse%
\ {\isacharparenleft}rule\ conjunct{\isadigit{1}}{\isacharparenright}\isanewline
\ \ \isacommand{have}\isamarkupfalse%
\ {\isachardoublequoteopen}F\ {\isasymnotin}\ pcp{\isacharunderscore}lim\ C\ S{\isachardoublequoteclose}\isanewline
\ \ \ \ \isacommand{using}\isamarkupfalse%
\ F\ \isacommand{by}\isamarkupfalse%
\ {\isacharparenleft}rule\ conjunct{\isadigit{2}}{\isacharparenright}\isanewline
\ \ \isacommand{have}\isamarkupfalse%
\ {\isachardoublequoteopen}pcp{\isacharunderscore}seq\ C\ S\ {\isacharparenleft}Suc\ {\isacharparenleft}to{\isacharunderscore}nat\ F{\isacharparenright}{\isacharparenright}\ {\isasymsubseteq}\ pcp{\isacharunderscore}lim\ C\ S{\isachardoublequoteclose}\isanewline
\ \ \ \ \isacommand{by}\isamarkupfalse%
\ {\isacharparenleft}rule\ pcp{\isacharunderscore}seq{\isacharunderscore}sub{\isacharparenright}\isanewline
\ \ \isacommand{then}\isamarkupfalse%
\ \isacommand{have}\isamarkupfalse%
\ {\isachardoublequoteopen}F\ {\isasymin}\ pcp{\isacharunderscore}seq\ C\ S\ {\isacharparenleft}Suc\ {\isacharparenleft}to{\isacharunderscore}nat\ F{\isacharparenright}{\isacharparenright}\ {\isasymlongrightarrow}\ F\ {\isasymin}\ pcp{\isacharunderscore}lim\ C\ S{\isachardoublequoteclose}\isanewline
\ \ \ \ \isacommand{by}\isamarkupfalse%
\ {\isacharparenleft}rule\ in{\isacharunderscore}mono{\isacharparenright}\isanewline
\ \ \isacommand{then}\isamarkupfalse%
\ \isacommand{have}\isamarkupfalse%
\ {\isadigit{1}}{\isacharcolon}{\isachardoublequoteopen}F\ {\isasymnotin}\ pcp{\isacharunderscore}seq\ C\ S\ {\isacharparenleft}Suc\ {\isacharparenleft}to{\isacharunderscore}nat\ F{\isacharparenright}{\isacharparenright}{\isachardoublequoteclose}\isanewline
\ \ \ \ \isacommand{using}\isamarkupfalse%
\ {\isacartoucheopen}F\ {\isasymnotin}\ pcp{\isacharunderscore}lim\ C\ S{\isacartoucheclose}\ \isacommand{by}\isamarkupfalse%
\ {\isacharparenleft}rule\ mt{\isacharparenright}\isanewline
\ \ \isacommand{have}\isamarkupfalse%
\ {\isadigit{2}}{\isacharcolon}\ {\isachardoublequoteopen}insert\ F\ {\isacharparenleft}pcp{\isacharunderscore}seq\ C\ S\ {\isacharparenleft}to{\isacharunderscore}nat\ F{\isacharparenright}{\isacharparenright}\ {\isasymnotin}\ C{\isachardoublequoteclose}\ \isanewline
\ \ \isacommand{proof}\isamarkupfalse%
\ {\isacharparenleft}rule\ ccontr{\isacharparenright}\isanewline
\ \ \ \ \isacommand{assume}\isamarkupfalse%
\ {\isachardoublequoteopen}{\isasymnot}{\isacharparenleft}insert\ F\ {\isacharparenleft}pcp{\isacharunderscore}seq\ C\ S\ {\isacharparenleft}to{\isacharunderscore}nat\ F{\isacharparenright}{\isacharparenright}\ {\isasymnotin}\ C{\isacharparenright}{\isachardoublequoteclose}\isanewline
\ \ \ \ \isacommand{then}\isamarkupfalse%
\ \isacommand{have}\isamarkupfalse%
\ {\isachardoublequoteopen}insert\ F\ {\isacharparenleft}pcp{\isacharunderscore}seq\ C\ S\ {\isacharparenleft}to{\isacharunderscore}nat\ F{\isacharparenright}{\isacharparenright}\ {\isasymin}\ C{\isachardoublequoteclose}\isanewline
\ \ \ \ \ \ \isacommand{by}\isamarkupfalse%
\ {\isacharparenleft}rule\ notnotD{\isacharparenright}\isanewline
\ \ \ \ \isacommand{then}\isamarkupfalse%
\ \isacommand{have}\isamarkupfalse%
\ C{\isacharcolon}{\isachardoublequoteopen}insert\ {\isacharparenleft}from{\isacharunderscore}nat\ {\isacharparenleft}to{\isacharunderscore}nat\ F{\isacharparenright}{\isacharparenright}\ {\isacharparenleft}pcp{\isacharunderscore}seq\ C\ S\ {\isacharparenleft}to{\isacharunderscore}nat\ F{\isacharparenright}{\isacharparenright}\ {\isasymin}\ C{\isachardoublequoteclose}\isanewline
\ \ \ \ \ \ \isacommand{by}\isamarkupfalse%
\ {\isacharparenleft}simp\ only{\isacharcolon}\ from{\isacharunderscore}nat{\isacharunderscore}to{\isacharunderscore}nat{\isacharparenright}\isanewline
\ \ \ \ \isacommand{have}\isamarkupfalse%
\ {\isachardoublequoteopen}pcp{\isacharunderscore}seq\ C\ S\ {\isacharparenleft}Suc\ {\isacharparenleft}to{\isacharunderscore}nat\ F{\isacharparenright}{\isacharparenright}\ {\isacharequal}\ {\isacharparenleft}let\ Sn\ {\isacharequal}\ pcp{\isacharunderscore}seq\ C\ S\ {\isacharparenleft}to{\isacharunderscore}nat\ F{\isacharparenright}{\isacharsemicolon}\ \isanewline
\ \ \ \ \ \ \ \ \ \ Sn{\isadigit{1}}\ {\isacharequal}\ insert\ {\isacharparenleft}from{\isacharunderscore}nat\ {\isacharparenleft}to{\isacharunderscore}nat\ F{\isacharparenright}{\isacharparenright}\ Sn\ in\ if\ Sn{\isadigit{1}}\ {\isasymin}\ C\ then\ Sn{\isadigit{1}}\ else\ Sn{\isacharparenright}{\isachardoublequoteclose}\ \isanewline
\ \ \ \ \ \ \isacommand{by}\isamarkupfalse%
\ {\isacharparenleft}simp\ only{\isacharcolon}\ pcp{\isacharunderscore}seq{\isachardot}simps{\isacharparenleft}{\isadigit{2}}{\isacharparenright}{\isacharparenright}\isanewline
\ \ \ \ \isacommand{then}\isamarkupfalse%
\ \isacommand{have}\isamarkupfalse%
\ SucDef{\isacharcolon}{\isachardoublequoteopen}pcp{\isacharunderscore}seq\ C\ S\ {\isacharparenleft}Suc\ {\isacharparenleft}to{\isacharunderscore}nat\ F{\isacharparenright}{\isacharparenright}\ {\isacharequal}\ {\isacharparenleft}if\ insert\ {\isacharparenleft}from{\isacharunderscore}nat\ {\isacharparenleft}to{\isacharunderscore}nat\ F{\isacharparenright}{\isacharparenright}\ {\isacharparenleft}pcp{\isacharunderscore}seq\ C\ S\ {\isacharparenleft}to{\isacharunderscore}nat\ F{\isacharparenright}{\isacharparenright}\ {\isasymin}\ C\ \isanewline
\ \ \ \ \ \ \ \ \ \ then\ insert\ {\isacharparenleft}from{\isacharunderscore}nat\ {\isacharparenleft}to{\isacharunderscore}nat\ F{\isacharparenright}{\isacharparenright}\ {\isacharparenleft}pcp{\isacharunderscore}seq\ C\ S\ {\isacharparenleft}to{\isacharunderscore}nat\ F{\isacharparenright}{\isacharparenright}\ else\ pcp{\isacharunderscore}seq\ C\ S\ {\isacharparenleft}to{\isacharunderscore}nat\ F{\isacharparenright}{\isacharparenright}{\isachardoublequoteclose}\ \isanewline
\ \ \ \ \ \ \isacommand{by}\isamarkupfalse%
\ {\isacharparenleft}simp\ only{\isacharcolon}\ Let{\isacharunderscore}def{\isacharparenright}\isanewline
\ \ \ \ \isacommand{then}\isamarkupfalse%
\ \isacommand{have}\isamarkupfalse%
\ {\isachardoublequoteopen}pcp{\isacharunderscore}seq\ C\ S\ {\isacharparenleft}Suc\ {\isacharparenleft}to{\isacharunderscore}nat\ F{\isacharparenright}{\isacharparenright}\ {\isacharequal}\ insert\ {\isacharparenleft}from{\isacharunderscore}nat\ {\isacharparenleft}to{\isacharunderscore}nat\ F{\isacharparenright}{\isacharparenright}\ {\isacharparenleft}pcp{\isacharunderscore}seq\ C\ S\ {\isacharparenleft}to{\isacharunderscore}nat\ F{\isacharparenright}{\isacharparenright}{\isachardoublequoteclose}\ \isanewline
\ \ \ \ \ \ \isacommand{using}\isamarkupfalse%
\ C\ \isacommand{by}\isamarkupfalse%
\ {\isacharparenleft}simp\ only{\isacharcolon}\ if{\isacharunderscore}True{\isacharparenright}\isanewline
\ \ \ \ \isacommand{then}\isamarkupfalse%
\ \isacommand{have}\isamarkupfalse%
\ {\isachardoublequoteopen}pcp{\isacharunderscore}seq\ C\ S\ {\isacharparenleft}Suc\ {\isacharparenleft}to{\isacharunderscore}nat\ F{\isacharparenright}{\isacharparenright}\ {\isacharequal}\ insert\ F\ {\isacharparenleft}pcp{\isacharunderscore}seq\ C\ S\ {\isacharparenleft}to{\isacharunderscore}nat\ F{\isacharparenright}{\isacharparenright}{\isachardoublequoteclose}\isanewline
\ \ \ \ \ \ \isacommand{by}\isamarkupfalse%
\ {\isacharparenleft}simp\ only{\isacharcolon}\ from{\isacharunderscore}nat{\isacharunderscore}to{\isacharunderscore}nat{\isacharparenright}\isanewline
\ \ \ \ \isacommand{then}\isamarkupfalse%
\ \isacommand{have}\isamarkupfalse%
\ {\isachardoublequoteopen}F\ {\isasymin}\ pcp{\isacharunderscore}seq\ C\ S\ {\isacharparenleft}Suc\ {\isacharparenleft}to{\isacharunderscore}nat\ F{\isacharparenright}{\isacharparenright}{\isachardoublequoteclose}\isanewline
\ \ \ \ \ \ \isacommand{by}\isamarkupfalse%
\ {\isacharparenleft}simp\ only{\isacharcolon}\ insertI{\isadigit{1}}{\isacharparenright}\isanewline
\ \ \ \ \isacommand{show}\isamarkupfalse%
\ {\isachardoublequoteopen}False{\isachardoublequoteclose}\isanewline
\ \ \ \ \ \ \isacommand{using}\isamarkupfalse%
\ {\isacartoucheopen}F\ {\isasymnotin}\ pcp{\isacharunderscore}seq\ C\ S\ {\isacharparenleft}Suc\ {\isacharparenleft}to{\isacharunderscore}nat\ F{\isacharparenright}{\isacharparenright}{\isacartoucheclose}\ {\isacartoucheopen}F\ {\isasymin}\ pcp{\isacharunderscore}seq\ C\ S\ {\isacharparenleft}Suc\ {\isacharparenleft}to{\isacharunderscore}nat\ F{\isacharparenright}{\isacharparenright}{\isacartoucheclose}\ \isacommand{by}\isamarkupfalse%
\ {\isacharparenleft}rule\ notE{\isacharparenright}\isanewline
\ \ \isacommand{qed}\isamarkupfalse%
\isanewline
\ \ \isacommand{have}\isamarkupfalse%
\ {\isachardoublequoteopen}pcp{\isacharunderscore}seq\ C\ S\ {\isacharparenleft}to{\isacharunderscore}nat\ F{\isacharparenright}\ {\isasymsubseteq}\ pcp{\isacharunderscore}lim\ C\ S{\isachardoublequoteclose}\isanewline
\ \ \ \ \isacommand{by}\isamarkupfalse%
\ {\isacharparenleft}rule\ pcp{\isacharunderscore}seq{\isacharunderscore}sub{\isacharparenright}\isanewline
\ \ \isacommand{then}\isamarkupfalse%
\ \isacommand{have}\isamarkupfalse%
\ {\isachardoublequoteopen}pcp{\isacharunderscore}seq\ C\ S\ {\isacharparenleft}to{\isacharunderscore}nat\ F{\isacharparenright}\ {\isasymsubseteq}\ K{\isachardoublequoteclose}\isanewline
\ \ \ \ \isacommand{using}\isamarkupfalse%
\ assms{\isacharparenleft}{\isadigit{4}}{\isacharparenright}\ \isacommand{by}\isamarkupfalse%
\ {\isacharparenleft}rule\ subset{\isacharunderscore}trans{\isacharparenright}\isanewline
\ \ \isacommand{then}\isamarkupfalse%
\ \isacommand{have}\isamarkupfalse%
\ {\isachardoublequoteopen}insert\ F\ {\isacharparenleft}pcp{\isacharunderscore}seq\ C\ S\ {\isacharparenleft}to{\isacharunderscore}nat\ F{\isacharparenright}{\isacharparenright}\ {\isasymsubseteq}\ K{\isachardoublequoteclose}\ \isanewline
\ \ \ \ \isacommand{using}\isamarkupfalse%
\ {\isacartoucheopen}F\ {\isasymin}\ K{\isacartoucheclose}\ \isacommand{by}\isamarkupfalse%
\ {\isacharparenleft}simp\ only{\isacharcolon}\ insert{\isacharunderscore}subset{\isacharparenright}\isanewline
\ \ \isacommand{have}\isamarkupfalse%
\ {\isachardoublequoteopen}{\isasymforall}S\ {\isasymin}\ C{\isachardot}\ {\isasymforall}s{\isasymsubseteq}S{\isachardot}\ s\ {\isasymin}\ C{\isachardoublequoteclose}\isanewline
\ \ \ \ \isacommand{using}\isamarkupfalse%
\ assms{\isacharparenleft}{\isadigit{2}}{\isacharparenright}\ \isacommand{by}\isamarkupfalse%
\ {\isacharparenleft}simp\ only{\isacharcolon}\ subset{\isacharunderscore}closed{\isacharunderscore}def{\isacharparenright}\isanewline
\ \ \isacommand{then}\isamarkupfalse%
\ \isacommand{have}\isamarkupfalse%
\ {\isachardoublequoteopen}{\isasymforall}s\ {\isasymsubseteq}\ K{\isachardot}\ s\ {\isasymin}\ C{\isachardoublequoteclose}\isanewline
\ \ \ \ \isacommand{using}\isamarkupfalse%
\ assms{\isacharparenleft}{\isadigit{3}}{\isacharparenright}\ \isacommand{by}\isamarkupfalse%
\ {\isacharparenleft}rule\ bspec{\isacharparenright}\isanewline
\ \ \isacommand{then}\isamarkupfalse%
\ \isacommand{have}\isamarkupfalse%
\ {\isadigit{3}}{\isacharcolon}{\isachardoublequoteopen}insert\ F\ {\isacharparenleft}pcp{\isacharunderscore}seq\ C\ S\ {\isacharparenleft}to{\isacharunderscore}nat\ F{\isacharparenright}{\isacharparenright}\ {\isasymin}\ C{\isachardoublequoteclose}\ \isanewline
\ \ \ \ \isacommand{using}\isamarkupfalse%
\ {\isacartoucheopen}insert\ F\ {\isacharparenleft}pcp{\isacharunderscore}seq\ C\ S\ {\isacharparenleft}to{\isacharunderscore}nat\ F{\isacharparenright}{\isacharparenright}\ {\isasymsubseteq}\ K{\isacartoucheclose}\ \isacommand{by}\isamarkupfalse%
\ {\isacharparenleft}rule\ sspec{\isacharparenright}\isanewline
\ \ \isacommand{show}\isamarkupfalse%
\ {\isachardoublequoteopen}False{\isachardoublequoteclose}\isanewline
\ \ \ \ \isacommand{using}\isamarkupfalse%
\ {\isadigit{2}}\ {\isadigit{3}}\ \isacommand{by}\isamarkupfalse%
\ {\isacharparenleft}rule\ notE{\isacharparenright}\isanewline
\isacommand{qed}\isamarkupfalse%
%
\endisatagproof
{\isafoldproof}%
%
\isadelimproof
%
\endisadelimproof
%
\begin{isamarkuptext}%
Análogamente a resultados anteriores, veamos su prueba automática.%
\end{isamarkuptext}\isamarkuptrue%
\isacommand{lemma}\isamarkupfalse%
\ cl{\isacharunderscore}max{\isacharcolon}\isanewline
\ \ \isakeyword{assumes}\ c{\isacharcolon}\ {\isachardoublequoteopen}pcp\ C{\isachardoublequoteclose}\isanewline
\ \ \isakeyword{assumes}\ sc{\isacharcolon}\ {\isachardoublequoteopen}subset{\isacharunderscore}closed\ C{\isachardoublequoteclose}\isanewline
\ \ \isakeyword{assumes}\ el{\isacharcolon}\ {\isachardoublequoteopen}K\ {\isasymin}\ C{\isachardoublequoteclose}\isanewline
\ \ \isakeyword{assumes}\ su{\isacharcolon}\ {\isachardoublequoteopen}pcp{\isacharunderscore}lim\ C\ S\ {\isasymsubseteq}\ K{\isachardoublequoteclose}\isanewline
\ \ \isakeyword{shows}\ {\isachardoublequoteopen}pcp{\isacharunderscore}lim\ C\ S\ {\isacharequal}\ K{\isachardoublequoteclose}\ {\isacharparenleft}\isakeyword{is}\ {\isacharquery}e{\isacharparenright}\isanewline
%
\isadelimproof
%
\endisadelimproof
%
\isatagproof
\isacommand{proof}\isamarkupfalse%
\ {\isacharparenleft}rule\ ccontr{\isacharparenright}\isanewline
\ \ \isacommand{assume}\isamarkupfalse%
\ {\isacartoucheopen}{\isasymnot}{\isacharquery}e{\isacartoucheclose}\isanewline
\ \ \isacommand{with}\isamarkupfalse%
\ su\ \isacommand{have}\isamarkupfalse%
\ {\isachardoublequoteopen}pcp{\isacharunderscore}lim\ C\ S\ {\isasymsubset}\ K{\isachardoublequoteclose}\ \isacommand{by}\isamarkupfalse%
\ simp\isanewline
\ \ \isacommand{then}\isamarkupfalse%
\ \isacommand{obtain}\isamarkupfalse%
\ F\ \isakeyword{where}\ e{\isacharcolon}\ {\isachardoublequoteopen}F\ {\isasymin}\ K{\isachardoublequoteclose}\ \isakeyword{and}\ ne{\isacharcolon}\ {\isachardoublequoteopen}F\ {\isasymnotin}\ pcp{\isacharunderscore}lim\ C\ S{\isachardoublequoteclose}\ \isacommand{by}\isamarkupfalse%
\ blast\isanewline
\ \ \isacommand{from}\isamarkupfalse%
\ ne\ \isacommand{have}\isamarkupfalse%
\ {\isachardoublequoteopen}F\ {\isasymnotin}\ pcp{\isacharunderscore}seq\ C\ S\ {\isacharparenleft}Suc\ {\isacharparenleft}to{\isacharunderscore}nat\ F{\isacharparenright}{\isacharparenright}{\isachardoublequoteclose}\ \isacommand{using}\isamarkupfalse%
\ pcp{\isacharunderscore}seq{\isacharunderscore}sub\ \isacommand{by}\isamarkupfalse%
\ fast\isanewline
\ \ \isacommand{hence}\isamarkupfalse%
\ {\isadigit{1}}{\isacharcolon}\ {\isachardoublequoteopen}insert\ F\ {\isacharparenleft}pcp{\isacharunderscore}seq\ C\ S\ {\isacharparenleft}to{\isacharunderscore}nat\ F{\isacharparenright}{\isacharparenright}\ {\isasymnotin}\ C{\isachardoublequoteclose}\ \isacommand{by}\isamarkupfalse%
\ {\isacharparenleft}simp\ add{\isacharcolon}\ Let{\isacharunderscore}def\ split{\isacharcolon}\ if{\isacharunderscore}splits{\isacharparenright}\isanewline
\ \ \isacommand{have}\isamarkupfalse%
\ {\isachardoublequoteopen}insert\ F\ {\isacharparenleft}pcp{\isacharunderscore}seq\ C\ S\ {\isacharparenleft}to{\isacharunderscore}nat\ F{\isacharparenright}{\isacharparenright}\ {\isasymsubseteq}\ K{\isachardoublequoteclose}\ \isacommand{using}\isamarkupfalse%
\ pcp{\isacharunderscore}seq{\isacharunderscore}sub\ e\ su\ \isacommand{by}\isamarkupfalse%
\ blast\isanewline
\ \ \isacommand{hence}\isamarkupfalse%
\ {\isachardoublequoteopen}insert\ F\ {\isacharparenleft}pcp{\isacharunderscore}seq\ C\ S\ {\isacharparenleft}to{\isacharunderscore}nat\ F{\isacharparenright}{\isacharparenright}\ {\isasymin}\ C{\isachardoublequoteclose}\ \isacommand{using}\isamarkupfalse%
\ sc\ \isanewline
\ \ \ \ \isacommand{unfolding}\isamarkupfalse%
\ subset{\isacharunderscore}closed{\isacharunderscore}def\ \isacommand{using}\isamarkupfalse%
\ el\ \isacommand{by}\isamarkupfalse%
\ blast\isanewline
\ \ \isacommand{with}\isamarkupfalse%
\ {\isadigit{1}}\ \isacommand{show}\isamarkupfalse%
\ False\ \isacommand{{\isachardot}{\isachardot}}\isamarkupfalse%
\isanewline
\isacommand{qed}\isamarkupfalse%
%
\endisatagproof
{\isafoldproof}%
%
\isadelimproof
%
\endisadelimproof
%
\begin{isamarkuptext}%
A continuación mostremos un resultado sobre el límite de la sucesión de \isa{{\isadigit{4}}{\isachardot}{\isadigit{1}}{\isachardot}{\isadigit{1}}} que es 
  consecuencia de que dicho límite sea un elemento maximal de la colección que lo define si esta
  verifica la propiedad de consistencia proposicional y es cerrada bajo subconjuntos.
  
  \begin{corolario}
    Sea \isa{C} una colección de conjuntos que verifica la propiedad de consistencia proposicional y
    es cerrada bajo subconjuntos, \isa{S} un conjunto, \isa{{\isacharbraceleft}S\isactrlsub n{\isacharbraceright}} la sucesión de conjuntos de \isa{C} a partir 
    de \isa{S} según la definición \isa{{\isadigit{4}}{\isachardot}{\isadigit{1}}{\isachardot}{\isadigit{1}}} y \isa{F} una fórmula proposicional. Entonces, si\\
    $\{F\} \cup \bigcup_{n = 0}^{\infty} S_{n} \in C$, se verifica que 
    $F \in \bigcup_{n = 0}^{\infty} S_{n}$. 
  \end{corolario}

  \begin{demostracion}
    Como \isa{C} es una colección que verifica la propiedad de consistencia proposicional y es cerrada 
    bajo subconjuntos, se tiene que el límite $\bigcup_{n = 0}^{\infty} S_{n}$ es maximal en \isa{C}. Por 
    lo tanto, si suponemos que $\{F\} \cup \bigcup_{n = 0}^{\infty} S_{n} \in C$, como el límite 
    está contenido en dicho conjunto, se cumple que 
    $\{F\} \cup \bigcup_{n = 0}^{\infty} S_{n} = \bigcup_{n = 0}^{\infty} S_{n}$, luego \isa{F} 
    pertenece al límite, como queríamos demostrar.
  \end{demostracion}

  Veamos su formalización y prueba detallada en Isabelle/HOL.%
\end{isamarkuptext}\isamarkuptrue%
\isacommand{lemma}\isamarkupfalse%
\isanewline
\ \ \isakeyword{assumes}\ {\isachardoublequoteopen}pcp\ C{\isachardoublequoteclose}\isanewline
\ \ \isakeyword{assumes}\ {\isachardoublequoteopen}subset{\isacharunderscore}closed\ C{\isachardoublequoteclose}\isanewline
\ \ \isakeyword{shows}\ {\isachardoublequoteopen}insert\ F\ {\isacharparenleft}pcp{\isacharunderscore}lim\ C\ S{\isacharparenright}\ {\isasymin}\ C\ {\isasymLongrightarrow}\ F\ {\isasymin}\ pcp{\isacharunderscore}lim\ C\ S{\isachardoublequoteclose}\isanewline
%
\isadelimproof
%
\endisadelimproof
%
\isatagproof
\isacommand{proof}\isamarkupfalse%
\ {\isacharminus}\isanewline
\ \ \isacommand{assume}\isamarkupfalse%
\ {\isachardoublequoteopen}insert\ F\ {\isacharparenleft}pcp{\isacharunderscore}lim\ C\ S{\isacharparenright}\ {\isasymin}\ C{\isachardoublequoteclose}\isanewline
\ \ \isacommand{have}\isamarkupfalse%
\ {\isachardoublequoteopen}pcp{\isacharunderscore}lim\ C\ S\ {\isasymsubseteq}\ insert\ F\ {\isacharparenleft}pcp{\isacharunderscore}lim\ C\ S{\isacharparenright}{\isachardoublequoteclose}\isanewline
\ \ \ \ \isacommand{by}\isamarkupfalse%
\ {\isacharparenleft}rule\ subset{\isacharunderscore}insertI{\isacharparenright}\ \isanewline
\ \ \isacommand{have}\isamarkupfalse%
\ {\isachardoublequoteopen}pcp{\isacharunderscore}lim\ C\ S\ {\isacharequal}\ insert\ F\ {\isacharparenleft}pcp{\isacharunderscore}lim\ C\ S{\isacharparenright}{\isachardoublequoteclose}\isanewline
\ \ \ \ \isacommand{using}\isamarkupfalse%
\ assms{\isacharparenleft}{\isadigit{1}}{\isacharparenright}\ assms{\isacharparenleft}{\isadigit{2}}{\isacharparenright}\ {\isacartoucheopen}insert\ F\ {\isacharparenleft}pcp{\isacharunderscore}lim\ C\ S{\isacharparenright}\ {\isasymin}\ C{\isacartoucheclose}\ {\isacartoucheopen}pcp{\isacharunderscore}lim\ C\ S\ {\isasymsubseteq}\ insert\ F\ {\isacharparenleft}pcp{\isacharunderscore}lim\ C\ S{\isacharparenright}{\isacartoucheclose}\ \isacommand{by}\isamarkupfalse%
\ {\isacharparenleft}rule\ cl{\isacharunderscore}max{\isacharparenright}\isanewline
\ \ \isacommand{then}\isamarkupfalse%
\ \isacommand{have}\isamarkupfalse%
\ {\isachardoublequoteopen}insert\ F\ {\isacharparenleft}pcp{\isacharunderscore}lim\ C\ S{\isacharparenright}\ {\isasymsubseteq}\ pcp{\isacharunderscore}lim\ C\ S{\isachardoublequoteclose}\isanewline
\ \ \ \ \isacommand{by}\isamarkupfalse%
\ {\isacharparenleft}rule\ equalityD{\isadigit{2}}{\isacharparenright}\isanewline
\ \ \isacommand{then}\isamarkupfalse%
\ \isacommand{have}\isamarkupfalse%
\ {\isachardoublequoteopen}F\ {\isasymin}\ pcp{\isacharunderscore}lim\ C\ S\ {\isasymand}\ pcp{\isacharunderscore}lim\ C\ S\ {\isasymsubseteq}\ pcp{\isacharunderscore}lim\ C\ S{\isachardoublequoteclose}\isanewline
\ \ \ \ \isacommand{by}\isamarkupfalse%
\ {\isacharparenleft}simp\ only{\isacharcolon}\ insert{\isacharunderscore}subset{\isacharparenright}\isanewline
\ \ \isacommand{thus}\isamarkupfalse%
\ {\isachardoublequoteopen}F\ {\isasymin}\ pcp{\isacharunderscore}lim\ C\ S{\isachardoublequoteclose}\isanewline
\ \ \ \ \isacommand{by}\isamarkupfalse%
\ {\isacharparenleft}rule\ conjunct{\isadigit{1}}{\isacharparenright}\isanewline
\isacommand{qed}\isamarkupfalse%
%
\endisatagproof
{\isafoldproof}%
%
\isadelimproof
%
\endisadelimproof
%
\begin{isamarkuptext}%
Mostremos su demostración automática.%
\end{isamarkuptext}\isamarkuptrue%
\isacommand{lemma}\isamarkupfalse%
\ cl{\isacharunderscore}max{\isacharprime}{\isacharcolon}\isanewline
\ \ \isakeyword{assumes}\ c{\isacharcolon}\ {\isachardoublequoteopen}pcp\ C{\isachardoublequoteclose}\isanewline
\ \ \isakeyword{assumes}\ sc{\isacharcolon}\ {\isachardoublequoteopen}subset{\isacharunderscore}closed\ C{\isachardoublequoteclose}\isanewline
\ \ \isakeyword{shows}\ {\isachardoublequoteopen}insert\ F\ {\isacharparenleft}pcp{\isacharunderscore}lim\ C\ S{\isacharparenright}\ {\isasymin}\ C\ {\isasymLongrightarrow}\ F\ {\isasymin}\ pcp{\isacharunderscore}lim\ C\ S{\isachardoublequoteclose}\isanewline
%
\isadelimproof
\ \ %
\endisadelimproof
%
\isatagproof
\isacommand{using}\isamarkupfalse%
\ cl{\isacharunderscore}max{\isacharbrackleft}OF\ assms{\isacharbrackright}\ \isacommand{by}\isamarkupfalse%
\ blast{\isacharplus}%
\endisatagproof
{\isafoldproof}%
%
\isadelimproof
%
\endisadelimproof
%
\begin{isamarkuptext}%
El siguiente resultado prueba que el límite de la sucesión definida en \isa{{\isadigit{4}}{\isachardot}{\isadigit{1}}{\isachardot}{\isadigit{1}}} es un conjunto
  de Hintikka si la colección que lo define verifica la propiedad de consistencia proposicional, es
  es cerrada bajo subconjuntos y es de carácter finito. Como consecuencia del \isa{teorema\ de\ Hintikka},
  se trata en particular de un conjunto satisfacible. 

  \begin{lema}
    Sea \isa{C} una colección de conjuntos que verifica la propiedad de consistencia proposicional, es
    es cerrada bajo subconjuntos y es de carácter finito. Sea \isa{S\ {\isasymin}\ C} y \isa{{\isacharbraceleft}S\isactrlsub n{\isacharbraceright}} la sucesión de
    conjuntos de \isa{C} a partir de \isa{S} según la definición \isa{{\isadigit{4}}{\isachardot}{\isadigit{1}}{\isachardot}{\isadigit{1}}}. Entonces, el límite de la sucesión
    \isa{{\isacharbraceleft}S\isactrlsub n{\isacharbraceright}} es un conjunto de Hintikka.
  \end{lema}

  \begin{demostracion}
    Para facilitar la lectura, vamos a notar por \isa{L\isactrlsub S\isactrlsub C} al límite de la sucesión \isa{{\isacharbraceleft}S\isactrlsub n{\isacharbraceright}} descrita 
    en el enunciado.

    Por resultados anteriores, como \isa{C} verifica la propiedad de consistencia proposicional, es
    es cerrada bajo subconjuntos y es de carácter finito, se tiene que \isa{L\isactrlsub S\isactrlsub C\ {\isasymin}\ C}. En particular, por 
    verificar la propiedad de consistencia proposicional, por el lema de\\ caracterización de dicha
    propiedad mediante notación uniforme, se cumplen las siguientes condiciones para \isa{L\isactrlsub S\isactrlsub C}:

    \begin{itemize}
      \item \isa{{\isasymbottom}\ {\isasymnotin}\ L\isactrlsub S\isactrlsub C}.
      \item Dada \isa{p} una fórmula atómica cualquiera, no se tiene 
      simultáneamente que\\ \isa{p\ {\isasymin}\ L\isactrlsub S\isactrlsub C} y \isa{{\isasymnot}\ p\ {\isasymin}\ L\isactrlsub S\isactrlsub C}.
      \item Para toda fórmula de tipo \isa{{\isasymalpha}} con componentes \isa{{\isasymalpha}\isactrlsub {\isadigit{1}}} y \isa{{\isasymalpha}\isactrlsub {\isadigit{2}}} tal que \isa{{\isasymalpha}}
      pertenece a \isa{L\isactrlsub S\isactrlsub C}, se tiene que \isa{{\isacharbraceleft}{\isasymalpha}\isactrlsub {\isadigit{1}}{\isacharcomma}{\isasymalpha}\isactrlsub {\isadigit{2}}{\isacharbraceright}\ {\isasymunion}\ L\isactrlsub S\isactrlsub C} pertenece a \isa{C}.
      \item Para toda fórmula de tipo \isa{{\isasymbeta}} con componentes \isa{{\isasymbeta}\isactrlsub {\isadigit{1}}} y \isa{{\isasymbeta}\isactrlsub {\isadigit{2}}} tal que \isa{{\isasymbeta}}
      pertenece a \isa{L\isactrlsub S\isactrlsub C}, se tiene que o bien \isa{{\isacharbraceleft}{\isasymbeta}\isactrlsub {\isadigit{1}}{\isacharbraceright}\ {\isasymunion}\ L\isactrlsub S\isactrlsub C} pertenece a \isa{C} o 
      bien \isa{{\isacharbraceleft}{\isasymbeta}\isactrlsub {\isadigit{2}}{\isacharbraceright}\ {\isasymunion}\ L\isactrlsub S\isactrlsub C} pertenece a \isa{C}.
    \end{itemize}

    Veamos que \isa{L\isactrlsub S\isactrlsub C} es un conjunto de Hintikka probando que cumple las condiciones del
    lema de caracterización de los conjuntos de Hintikka mediante notación uniforme, es decir,
    probaremos que \isa{L\isactrlsub S\isactrlsub C} verifica:

    \begin{itemize}
      \item \isa{{\isasymbottom}\ {\isasymnotin}\ L\isactrlsub S\isactrlsub C}.
      \item Dada \isa{p} una fórmula atómica cualquiera, no se tiene 
      simultáneamente que\\ \isa{p\ {\isasymin}\ L\isactrlsub S\isactrlsub C} y \isa{{\isasymnot}\ p\ {\isasymin}\ L\isactrlsub S\isactrlsub C}.
      \item Para toda fórmula de tipo \isa{{\isasymalpha}} con componentes \isa{{\isasymalpha}\isactrlsub {\isadigit{1}}} y \isa{{\isasymalpha}\isactrlsub {\isadigit{2}}} se verifica 
      que si la fórmula pertenece a \isa{L\isactrlsub S\isactrlsub C}, entonces \isa{{\isasymalpha}\isactrlsub {\isadigit{1}}} y \isa{{\isasymalpha}\isactrlsub {\isadigit{2}}} también.
      \item Para toda fórmula de tipo \isa{{\isasymbeta}} con componentes \isa{{\isasymbeta}\isactrlsub {\isadigit{1}}} y \isa{{\isasymbeta}\isactrlsub {\isadigit{2}}} se verifica 
      que si la fórmula pertenece a \isa{L\isactrlsub S\isactrlsub C}, entonces o bien \isa{{\isasymbeta}\isactrlsub {\isadigit{1}}} pertenece
      a \isa{L\isactrlsub S\isactrlsub C} o bien \isa{{\isasymbeta}\isactrlsub {\isadigit{2}}} pertenece a \isa{L\isactrlsub S\isactrlsub C}.
    \end{itemize} 

    Observemos que las dos primeras condiciones coinciden con las obtenidas anteriormente para \isa{L\isactrlsub S\isactrlsub C} 
    por el lema de caracterización de la propiedad de consistencia proposicional mediante notación
    uniforme. Veamos que, en efecto, se cumplen el resto de condiciones.

    En primer lugar, probemos que para una fórmula \isa{F} de tipo \isa{{\isasymalpha}} y componentes \isa{{\isasymalpha}\isactrlsub {\isadigit{1}}} y \isa{{\isasymalpha}\isactrlsub {\isadigit{2}}} tal que 
    \isa{F\ {\isasymin}\ L\isactrlsub S\isactrlsub C} se verifica que tanto \isa{{\isasymalpha}\isactrlsub {\isadigit{1}}} como \isa{{\isasymalpha}\isactrlsub {\isadigit{2}}} pertenecen a \isa{L\isactrlsub S\isactrlsub C}. Por la tercera condición 
    obtenida anteriormente para \isa{L\isactrlsub S\isactrlsub C} por el lema de caracterización de la propiedad de consistencia 
    proposicional mediante notación uniforme, se cumple que\\ \isa{{\isacharbraceleft}{\isasymalpha}\isactrlsub {\isadigit{1}}{\isacharcomma}{\isasymalpha}\isactrlsub {\isadigit{2}}{\isacharbraceright}\ {\isasymunion}\ L\isactrlsub S\isactrlsub C\ {\isasymin}\ C}. Observemos que
    se verifica \isa{L\isactrlsub S\isactrlsub C\ {\isasymsubseteq}\ {\isacharbraceleft}{\isasymalpha}\isactrlsub {\isadigit{1}}{\isacharcomma}{\isasymalpha}\isactrlsub {\isadigit{2}}{\isacharbraceright}\ {\isasymunion}\ L\isactrlsub S\isactrlsub C}. De este modo, como \isa{C} es una colección con la propiedad de 
    consistencia proposicional y cerrada bajo subconjuntos, por el lema \isa{{\isadigit{4}}{\isachardot}{\isadigit{2}}{\isachardot}{\isadigit{2}}} se tiene que 
    los conjuntos \isa{L\isactrlsub S\isactrlsub C} y \isa{{\isacharbraceleft}{\isasymalpha}\isactrlsub {\isadigit{1}}{\isacharcomma}{\isasymalpha}\isactrlsub {\isadigit{2}}{\isacharbraceright}\ {\isasymunion}\ L\isactrlsub S\isactrlsub C} coinciden. Por tanto, es claro que \isa{{\isasymalpha}\isactrlsub {\isadigit{1}}\ {\isasymin}\ L\isactrlsub S\isactrlsub C} y \isa{{\isasymalpha}\isactrlsub {\isadigit{2}}\ {\isasymin}\ L\isactrlsub S\isactrlsub C}, 
    como queríamos demostrar.

    Por último, demostremos que para una fórmula \isa{F} de tipo \isa{{\isasymbeta}} y componentes \isa{{\isasymbeta}\isactrlsub {\isadigit{1}}} y \isa{{\isasymbeta}\isactrlsub {\isadigit{2}}} tal que
    \isa{F\ {\isasymin}\ L\isactrlsub S\isactrlsub C} se verifica que o bien \isa{{\isasymbeta}\isactrlsub {\isadigit{1}}\ {\isasymin}\ L\isactrlsub S\isactrlsub C} o bien \isa{{\isasymbeta}\isactrlsub {\isadigit{2}}\ {\isasymin}\ L\isactrlsub S\isactrlsub C}. Por la cuarta condición obtenida 
    anteriormente para \isa{L\isactrlsub S\isactrlsub C} por el lema de caracterización de la propiedad de consistencia 
    proposicional mediante notación uniforme, se cumple que o bien\\ \isa{{\isacharbraceleft}{\isasymbeta}\isactrlsub {\isadigit{1}}{\isacharbraceright}\ {\isasymunion}\ L\isactrlsub S\isactrlsub C\ {\isasymin}\ C} o bien 
    \isa{{\isacharbraceleft}{\isasymbeta}\isactrlsub {\isadigit{2}}{\isacharbraceright}\ {\isasymunion}\ L\isactrlsub S\isactrlsub C\ {\isasymin}\ C}. De este modo, si suponemos que \isa{{\isacharbraceleft}{\isasymbeta}\isactrlsub {\isadigit{1}}{\isacharbraceright}\ {\isasymunion}\ L\isactrlsub S\isactrlsub C\ {\isasymin}\ C}, como \isa{C} tiene la propiedad de 
    consistencia proposicional y es cerrada bajo subconjuntos, por el corolario \isa{{\isadigit{4}}{\isachardot}{\isadigit{2}}{\isachardot}{\isadigit{3}}} se tiene 
    que \isa{{\isasymbeta}\isactrlsub {\isadigit{1}}\ {\isasymin}\ L\isactrlsub S\isactrlsub C}. Por tanto, se cumple que o bien \isa{{\isasymbeta}\isactrlsub {\isadigit{1}}\ {\isasymin}\ L\isactrlsub S\isactrlsub C} o bien \isa{{\isasymbeta}\isactrlsub {\isadigit{2}}\ {\isasymin}\ L\isactrlsub S\isactrlsub C}. Si suponemos que 
    \isa{{\isacharbraceleft}{\isasymbeta}\isactrlsub {\isadigit{2}}{\isacharbraceright}\ {\isasymunion}\ L\isactrlsub S\isactrlsub C\ {\isasymin}\ C}, se observa fácilmente que llegamos a la misma conclusión de manera análoga. 
    Por lo tanto, queda probado el resultado.
  \end{demostracion}

  Veamos su formalización y prueba detallada en Isabelle.%
\end{isamarkuptext}\isamarkuptrue%
\isacommand{lemma}\isamarkupfalse%
\isanewline
\ \ \isakeyword{assumes}\ {\isachardoublequoteopen}pcp\ C{\isachardoublequoteclose}\isanewline
\ \ \isakeyword{assumes}\ {\isachardoublequoteopen}subset{\isacharunderscore}closed\ C{\isachardoublequoteclose}\isanewline
\ \ \isakeyword{assumes}\ {\isachardoublequoteopen}finite{\isacharunderscore}character\ C{\isachardoublequoteclose}\isanewline
\ \ \isakeyword{assumes}\ {\isachardoublequoteopen}S\ {\isasymin}\ C{\isachardoublequoteclose}\isanewline
\ \ \isakeyword{shows}\ {\isachardoublequoteopen}Hintikka\ {\isacharparenleft}pcp{\isacharunderscore}lim\ C\ S{\isacharparenright}{\isachardoublequoteclose}\isanewline
%
\isadelimproof
%
\endisadelimproof
%
\isatagproof
\isacommand{proof}\isamarkupfalse%
\ {\isacharparenleft}rule\ Hintikka{\isacharunderscore}alt{\isadigit{2}}{\isacharparenright}\isanewline
\ \ \isacommand{let}\isamarkupfalse%
\ {\isacharquery}cl\ {\isacharequal}\ {\isachardoublequoteopen}pcp{\isacharunderscore}lim\ C\ S{\isachardoublequoteclose}\isanewline
\ \ \isacommand{have}\isamarkupfalse%
\ {\isachardoublequoteopen}{\isacharquery}cl\ {\isasymin}\ C{\isachardoublequoteclose}\isanewline
\ \ \ \ \isacommand{using}\isamarkupfalse%
\ assms{\isacharparenleft}{\isadigit{1}}{\isacharparenright}\ assms{\isacharparenleft}{\isadigit{4}}{\isacharparenright}\ assms{\isacharparenleft}{\isadigit{2}}{\isacharparenright}\ assms{\isacharparenleft}{\isadigit{3}}{\isacharparenright}\ \isacommand{by}\isamarkupfalse%
\ {\isacharparenleft}rule\ pcp{\isacharunderscore}lim{\isacharunderscore}in{\isacharparenright}\isanewline
\ \ \isacommand{have}\isamarkupfalse%
\ {\isachardoublequoteopen}{\isacharparenleft}{\isasymforall}S\ {\isasymin}\ C{\isachardot}\isanewline
\ \ {\isasymbottom}\ {\isasymnotin}\ S\isanewline
{\isasymand}\ {\isacharparenleft}{\isasymforall}k{\isachardot}\ Atom\ k\ {\isasymin}\ S\ {\isasymlongrightarrow}\ \isactrlbold {\isasymnot}\ {\isacharparenleft}Atom\ k{\isacharparenright}\ {\isasymin}\ S\ {\isasymlongrightarrow}\ False{\isacharparenright}\isanewline
{\isasymand}\ {\isacharparenleft}{\isasymforall}F\ G\ H{\isachardot}\ Con\ F\ G\ H\ {\isasymlongrightarrow}\ F\ {\isasymin}\ S\ {\isasymlongrightarrow}\ {\isacharbraceleft}G{\isacharcomma}H{\isacharbraceright}\ {\isasymunion}\ S\ {\isasymin}\ C{\isacharparenright}\isanewline
{\isasymand}\ {\isacharparenleft}{\isasymforall}F\ G\ H{\isachardot}\ Dis\ F\ G\ H\ {\isasymlongrightarrow}\ F\ {\isasymin}\ S\ {\isasymlongrightarrow}\ {\isacharbraceleft}G{\isacharbraceright}\ {\isasymunion}\ S\ {\isasymin}\ C\ {\isasymor}\ {\isacharbraceleft}H{\isacharbraceright}\ {\isasymunion}\ S\ {\isasymin}\ C{\isacharparenright}{\isacharparenright}{\isachardoublequoteclose}\isanewline
\ \ \ \ \isacommand{using}\isamarkupfalse%
\ assms{\isacharparenleft}{\isadigit{1}}{\isacharparenright}\ \isacommand{by}\isamarkupfalse%
\ {\isacharparenleft}rule\ pcp{\isacharunderscore}alt{\isadigit{1}}{\isacharparenright}\isanewline
\ \ \isacommand{then}\isamarkupfalse%
\ \isacommand{have}\isamarkupfalse%
\ d{\isacharcolon}{\isachardoublequoteopen}{\isasymbottom}\ {\isasymnotin}\ {\isacharquery}cl\isanewline
{\isasymand}\ {\isacharparenleft}{\isasymforall}k{\isachardot}\ Atom\ k\ {\isasymin}\ {\isacharquery}cl\ {\isasymlongrightarrow}\ \isactrlbold {\isasymnot}\ {\isacharparenleft}Atom\ k{\isacharparenright}\ {\isasymin}\ {\isacharquery}cl\ {\isasymlongrightarrow}\ False{\isacharparenright}\isanewline
{\isasymand}\ {\isacharparenleft}{\isasymforall}F\ G\ H{\isachardot}\ Con\ F\ G\ H\ {\isasymlongrightarrow}\ F\ {\isasymin}\ {\isacharquery}cl\ {\isasymlongrightarrow}\ {\isacharbraceleft}G{\isacharcomma}H{\isacharbraceright}\ {\isasymunion}\ {\isacharquery}cl\ {\isasymin}\ C{\isacharparenright}\isanewline
{\isasymand}\ {\isacharparenleft}{\isasymforall}F\ G\ H{\isachardot}\ Dis\ F\ G\ H\ {\isasymlongrightarrow}\ F\ {\isasymin}\ {\isacharquery}cl\ {\isasymlongrightarrow}\ {\isacharbraceleft}G{\isacharbraceright}\ {\isasymunion}\ {\isacharquery}cl\ {\isasymin}\ C\ {\isasymor}\ {\isacharbraceleft}H{\isacharbraceright}\ {\isasymunion}\ {\isacharquery}cl\ {\isasymin}\ C{\isacharparenright}{\isachardoublequoteclose}\isanewline
\ \ \ \ \isacommand{using}\isamarkupfalse%
\ {\isacartoucheopen}{\isacharquery}cl\ {\isasymin}\ C{\isacartoucheclose}\ \isacommand{by}\isamarkupfalse%
\ {\isacharparenleft}rule\ bspec{\isacharparenright}\isanewline
\ \ \isacommand{then}\isamarkupfalse%
\ \isacommand{have}\isamarkupfalse%
\ H{\isadigit{1}}{\isacharcolon}{\isachardoublequoteopen}{\isasymbottom}\ {\isasymnotin}\ {\isacharquery}cl{\isachardoublequoteclose}\isanewline
\ \ \ \ \isacommand{by}\isamarkupfalse%
\ {\isacharparenleft}rule\ conjunct{\isadigit{1}}{\isacharparenright}\isanewline
\ \ \isacommand{have}\isamarkupfalse%
\ H{\isadigit{2}}{\isacharcolon}{\isachardoublequoteopen}{\isasymforall}k{\isachardot}\ Atom\ k\ {\isasymin}\ {\isacharquery}cl\ {\isasymlongrightarrow}\ \isactrlbold {\isasymnot}\ {\isacharparenleft}Atom\ k{\isacharparenright}\ {\isasymin}\ {\isacharquery}cl\ {\isasymlongrightarrow}\ False{\isachardoublequoteclose}\isanewline
\ \ \ \ \isacommand{using}\isamarkupfalse%
\ d\ \isacommand{by}\isamarkupfalse%
\ {\isacharparenleft}iprover\ elim{\isacharcolon}\ conjunct{\isadigit{2}}\ conjunct{\isadigit{1}}{\isacharparenright}\isanewline
\ \ \isacommand{have}\isamarkupfalse%
\ Con{\isacharcolon}{\isachardoublequoteopen}{\isasymforall}F\ G\ H{\isachardot}\ Con\ F\ G\ H\ {\isasymlongrightarrow}\ F\ {\isasymin}\ {\isacharquery}cl\ {\isasymlongrightarrow}\ {\isacharbraceleft}G{\isacharcomma}H{\isacharbraceright}\ {\isasymunion}\ {\isacharquery}cl\ {\isasymin}\ C{\isachardoublequoteclose}\isanewline
\ \ \ \ \isacommand{using}\isamarkupfalse%
\ d\ \isacommand{by}\isamarkupfalse%
\ {\isacharparenleft}iprover\ elim{\isacharcolon}\ conjunct{\isadigit{2}}\ conjunct{\isadigit{1}}{\isacharparenright}\isanewline
\ \ \isacommand{have}\isamarkupfalse%
\ H{\isadigit{3}}{\isacharcolon}{\isachardoublequoteopen}{\isasymforall}F\ G\ H{\isachardot}\ Con\ F\ G\ H\ {\isasymlongrightarrow}\ F\ {\isasymin}\ {\isacharquery}cl\ {\isasymlongrightarrow}\ G\ {\isasymin}\ {\isacharquery}cl\ {\isasymand}\ H\ {\isasymin}\ {\isacharquery}cl{\isachardoublequoteclose}\isanewline
\ \ \isacommand{proof}\isamarkupfalse%
\ {\isacharparenleft}rule\ allI{\isacharparenright}{\isacharplus}\isanewline
\ \ \ \ \isacommand{fix}\isamarkupfalse%
\ F\ G\ H\isanewline
\ \ \ \ \isacommand{show}\isamarkupfalse%
\ {\isachardoublequoteopen}Con\ F\ G\ H\ {\isasymlongrightarrow}\ F\ {\isasymin}\ {\isacharquery}cl\ {\isasymlongrightarrow}\ G\ {\isasymin}\ {\isacharquery}cl\ {\isasymand}\ H\ {\isasymin}\ {\isacharquery}cl{\isachardoublequoteclose}\isanewline
\ \ \ \ \isacommand{proof}\isamarkupfalse%
\ {\isacharparenleft}rule\ impI{\isacharparenright}{\isacharplus}\isanewline
\ \ \ \ \ \ \isacommand{assume}\isamarkupfalse%
\ {\isachardoublequoteopen}Con\ F\ G\ H{\isachardoublequoteclose}\isanewline
\ \ \ \ \ \ \isacommand{assume}\isamarkupfalse%
\ {\isachardoublequoteopen}F\ {\isasymin}\ {\isacharquery}cl{\isachardoublequoteclose}\isanewline
\ \ \ \ \ \ \isacommand{have}\isamarkupfalse%
\ {\isachardoublequoteopen}Con\ F\ G\ H\ {\isasymlongrightarrow}\ F\ {\isasymin}\ {\isacharquery}cl\ {\isasymlongrightarrow}\ {\isacharbraceleft}G{\isacharcomma}H{\isacharbraceright}\ {\isasymunion}\ {\isacharquery}cl\ {\isasymin}\ C{\isachardoublequoteclose}\isanewline
\ \ \ \ \ \ \ \ \isacommand{using}\isamarkupfalse%
\ Con\ \isacommand{by}\isamarkupfalse%
\ {\isacharparenleft}iprover\ elim{\isacharcolon}\ allE{\isacharparenright}\isanewline
\ \ \ \ \ \ \isacommand{then}\isamarkupfalse%
\ \isacommand{have}\isamarkupfalse%
\ {\isachardoublequoteopen}F\ {\isasymin}\ {\isacharquery}cl\ {\isasymlongrightarrow}\ {\isacharbraceleft}G{\isacharcomma}H{\isacharbraceright}\ {\isasymunion}\ {\isacharquery}cl\ {\isasymin}\ C{\isachardoublequoteclose}\isanewline
\ \ \ \ \ \ \ \ \isacommand{using}\isamarkupfalse%
\ {\isacartoucheopen}Con\ F\ G\ H{\isacartoucheclose}\ \isacommand{by}\isamarkupfalse%
\ {\isacharparenleft}rule\ mp{\isacharparenright}\isanewline
\ \ \ \ \ \ \isacommand{then}\isamarkupfalse%
\ \isacommand{have}\isamarkupfalse%
\ {\isachardoublequoteopen}{\isacharbraceleft}G{\isacharcomma}H{\isacharbraceright}\ {\isasymunion}\ {\isacharquery}cl\ {\isasymin}\ C{\isachardoublequoteclose}\isanewline
\ \ \ \ \ \ \ \ \isacommand{using}\isamarkupfalse%
\ {\isacartoucheopen}F\ {\isasymin}\ {\isacharquery}cl{\isacartoucheclose}\ \isacommand{by}\isamarkupfalse%
\ {\isacharparenleft}rule\ mp{\isacharparenright}\isanewline
\ \ \ \ \ \ \isacommand{have}\isamarkupfalse%
\ {\isachardoublequoteopen}{\isacharparenleft}insert\ G\ {\isacharparenleft}insert\ H\ {\isacharquery}cl{\isacharparenright}{\isacharparenright}\ {\isacharequal}\ {\isacharbraceleft}G{\isacharcomma}H{\isacharbraceright}\ {\isasymunion}\ {\isacharquery}cl{\isachardoublequoteclose}\isanewline
\ \ \ \ \ \ \ \ \isacommand{by}\isamarkupfalse%
\ {\isacharparenleft}rule\ insertSetElem{\isacharparenright}\isanewline
\ \ \ \ \ \ \isacommand{then}\isamarkupfalse%
\ \isacommand{have}\isamarkupfalse%
\ {\isachardoublequoteopen}{\isacharparenleft}insert\ G\ {\isacharparenleft}insert\ H\ {\isacharquery}cl{\isacharparenright}{\isacharparenright}\ {\isasymin}\ C{\isachardoublequoteclose}\isanewline
\ \ \ \ \ \ \ \ \isacommand{using}\isamarkupfalse%
\ {\isacartoucheopen}{\isacharbraceleft}G{\isacharcomma}H{\isacharbraceright}\ {\isasymunion}\ {\isacharquery}cl\ {\isasymin}\ C{\isacartoucheclose}\ \isacommand{by}\isamarkupfalse%
\ {\isacharparenleft}simp\ only{\isacharcolon}\ {\isacartoucheopen}{\isacharparenleft}insert\ G\ {\isacharparenleft}insert\ H\ {\isacharquery}cl{\isacharparenright}{\isacharparenright}\ {\isacharequal}\ {\isacharbraceleft}G{\isacharcomma}H{\isacharbraceright}\ {\isasymunion}\ {\isacharquery}cl{\isacartoucheclose}{\isacharparenright}\isanewline
\ \ \ \ \ \ \isacommand{have}\isamarkupfalse%
\ {\isachardoublequoteopen}{\isacharquery}cl\ {\isasymsubseteq}\ insert\ H\ {\isacharquery}cl{\isachardoublequoteclose}\isanewline
\ \ \ \ \ \ \ \ \isacommand{by}\isamarkupfalse%
\ {\isacharparenleft}rule\ subset{\isacharunderscore}insertI{\isacharparenright}\isanewline
\ \ \ \ \ \ \isacommand{then}\isamarkupfalse%
\ \isacommand{have}\isamarkupfalse%
\ {\isachardoublequoteopen}{\isacharquery}cl\ {\isasymsubseteq}\ insert\ G\ {\isacharparenleft}insert\ H\ {\isacharquery}cl{\isacharparenright}{\isachardoublequoteclose}\isanewline
\ \ \ \ \ \ \ \ \isacommand{by}\isamarkupfalse%
\ {\isacharparenleft}rule\ subset{\isacharunderscore}insertI{\isadigit{2}}{\isacharparenright}\isanewline
\ \ \ \ \ \ \isacommand{have}\isamarkupfalse%
\ {\isachardoublequoteopen}{\isacharquery}cl\ {\isacharequal}\ insert\ G\ {\isacharparenleft}insert\ H\ {\isacharquery}cl{\isacharparenright}{\isachardoublequoteclose}\ \isanewline
\ \ \ \ \ \ \ \ \isacommand{using}\isamarkupfalse%
\ assms{\isacharparenleft}{\isadigit{1}}{\isacharparenright}\ assms{\isacharparenleft}{\isadigit{2}}{\isacharparenright}\ {\isacartoucheopen}insert\ G\ {\isacharparenleft}insert\ H\ {\isacharquery}cl{\isacharparenright}\ {\isasymin}\ C{\isacartoucheclose}\ {\isacartoucheopen}{\isacharquery}cl\ {\isasymsubseteq}\ insert\ G\ {\isacharparenleft}insert\ H\ {\isacharquery}cl{\isacharparenright}{\isacartoucheclose}\ \isacommand{by}\isamarkupfalse%
\ {\isacharparenleft}rule\ cl{\isacharunderscore}max{\isacharparenright}\isanewline
\ \ \ \ \ \ \isacommand{then}\isamarkupfalse%
\ \isacommand{have}\isamarkupfalse%
\ {\isachardoublequoteopen}insert\ G\ {\isacharparenleft}insert\ H\ {\isacharquery}cl{\isacharparenright}\ {\isasymsubseteq}\ {\isacharquery}cl{\isachardoublequoteclose}\isanewline
\ \ \ \ \ \ \ \ \isacommand{by}\isamarkupfalse%
\ {\isacharparenleft}simp\ only{\isacharcolon}\ equalityD{\isadigit{2}}{\isacharparenright}\isanewline
\ \ \ \ \ \ \isacommand{then}\isamarkupfalse%
\ \isacommand{have}\isamarkupfalse%
\ {\isachardoublequoteopen}G\ {\isasymin}\ {\isacharquery}cl\ {\isasymand}\ insert\ H\ {\isacharquery}cl\ {\isasymsubseteq}\ {\isacharquery}cl{\isachardoublequoteclose}\isanewline
\ \ \ \ \ \ \ \ \isacommand{by}\isamarkupfalse%
\ {\isacharparenleft}simp\ only{\isacharcolon}\ insert{\isacharunderscore}subset{\isacharparenright}\isanewline
\ \ \ \ \ \ \isacommand{then}\isamarkupfalse%
\ \isacommand{have}\isamarkupfalse%
\ {\isachardoublequoteopen}G\ {\isasymin}\ {\isacharquery}cl{\isachardoublequoteclose}\isanewline
\ \ \ \ \ \ \ \ \isacommand{by}\isamarkupfalse%
\ {\isacharparenleft}rule\ conjunct{\isadigit{1}}{\isacharparenright}\isanewline
\ \ \ \ \ \ \isacommand{have}\isamarkupfalse%
\ {\isachardoublequoteopen}insert\ H\ {\isacharquery}cl\ {\isasymsubseteq}\ {\isacharquery}cl{\isachardoublequoteclose}\isanewline
\ \ \ \ \ \ \ \ \isacommand{using}\isamarkupfalse%
\ {\isacartoucheopen}G\ {\isasymin}\ {\isacharquery}cl\ {\isasymand}\ insert\ H\ {\isacharquery}cl\ {\isasymsubseteq}\ {\isacharquery}cl{\isacartoucheclose}\ \isacommand{by}\isamarkupfalse%
\ {\isacharparenleft}rule\ conjunct{\isadigit{2}}{\isacharparenright}\isanewline
\ \ \ \ \ \ \isacommand{then}\isamarkupfalse%
\ \isacommand{have}\isamarkupfalse%
\ {\isachardoublequoteopen}H\ {\isasymin}\ {\isacharquery}cl\ {\isasymand}\ {\isacharquery}cl\ {\isasymsubseteq}\ {\isacharquery}cl{\isachardoublequoteclose}\isanewline
\ \ \ \ \ \ \ \ \isacommand{by}\isamarkupfalse%
\ {\isacharparenleft}simp\ only{\isacharcolon}\ insert{\isacharunderscore}subset{\isacharparenright}\isanewline
\ \ \ \ \ \ \isacommand{then}\isamarkupfalse%
\ \isacommand{have}\isamarkupfalse%
\ {\isachardoublequoteopen}H\ {\isasymin}\ {\isacharquery}cl{\isachardoublequoteclose}\isanewline
\ \ \ \ \ \ \ \ \isacommand{by}\isamarkupfalse%
\ {\isacharparenleft}rule\ conjunct{\isadigit{1}}{\isacharparenright}\isanewline
\ \ \ \ \ \ \isacommand{show}\isamarkupfalse%
\ {\isachardoublequoteopen}G\ {\isasymin}\ {\isacharquery}cl\ {\isasymand}\ H\ {\isasymin}\ {\isacharquery}cl{\isachardoublequoteclose}\isanewline
\ \ \ \ \ \ \ \ \isacommand{using}\isamarkupfalse%
\ {\isacartoucheopen}G\ {\isasymin}\ {\isacharquery}cl{\isacartoucheclose}\ {\isacartoucheopen}H\ {\isasymin}\ {\isacharquery}cl{\isacartoucheclose}\ \isacommand{by}\isamarkupfalse%
\ {\isacharparenleft}rule\ conjI{\isacharparenright}\isanewline
\ \ \ \ \isacommand{qed}\isamarkupfalse%
\isanewline
\ \ \isacommand{qed}\isamarkupfalse%
\isanewline
\ \ \isacommand{have}\isamarkupfalse%
\ Dis{\isacharcolon}{\isachardoublequoteopen}{\isasymforall}F\ G\ H{\isachardot}\ Dis\ F\ G\ H\ {\isasymlongrightarrow}\ F\ {\isasymin}\ {\isacharquery}cl\ {\isasymlongrightarrow}\ {\isacharbraceleft}G{\isacharbraceright}\ {\isasymunion}\ {\isacharquery}cl\ {\isasymin}\ C\ {\isasymor}\ {\isacharbraceleft}H{\isacharbraceright}\ {\isasymunion}\ {\isacharquery}cl\ {\isasymin}\ C{\isachardoublequoteclose}\isanewline
\ \ \ \ \isacommand{using}\isamarkupfalse%
\ d\ \isacommand{by}\isamarkupfalse%
\ {\isacharparenleft}iprover\ elim{\isacharcolon}\ conjunct{\isadigit{2}}\ conjunct{\isadigit{1}}{\isacharparenright}\isanewline
\ \ \isacommand{have}\isamarkupfalse%
\ H{\isadigit{4}}{\isacharcolon}{\isachardoublequoteopen}{\isasymforall}F\ G\ H{\isachardot}\ Dis\ F\ G\ H\ {\isasymlongrightarrow}\ F\ {\isasymin}\ {\isacharquery}cl\ {\isasymlongrightarrow}\ G\ {\isasymin}\ {\isacharquery}cl\ {\isasymor}\ H\ {\isasymin}\ {\isacharquery}cl{\isachardoublequoteclose}\isanewline
\ \ \isacommand{proof}\isamarkupfalse%
\ {\isacharparenleft}rule\ allI{\isacharparenright}{\isacharplus}\isanewline
\ \ \ \ \isacommand{fix}\isamarkupfalse%
\ F\ G\ H\isanewline
\ \ \ \ \isacommand{show}\isamarkupfalse%
\ {\isachardoublequoteopen}Dis\ F\ G\ H\ {\isasymlongrightarrow}\ F\ {\isasymin}\ {\isacharquery}cl\ {\isasymlongrightarrow}\ G\ {\isasymin}\ {\isacharquery}cl\ {\isasymor}\ H\ {\isasymin}\ {\isacharquery}cl{\isachardoublequoteclose}\isanewline
\ \ \ \ \isacommand{proof}\isamarkupfalse%
\ {\isacharparenleft}rule\ impI{\isacharparenright}{\isacharplus}\isanewline
\ \ \ \ \ \ \isacommand{assume}\isamarkupfalse%
\ {\isachardoublequoteopen}Dis\ F\ G\ H{\isachardoublequoteclose}\isanewline
\ \ \ \ \ \ \isacommand{assume}\isamarkupfalse%
\ {\isachardoublequoteopen}F\ {\isasymin}\ {\isacharquery}cl{\isachardoublequoteclose}\isanewline
\ \ \ \ \ \ \isacommand{have}\isamarkupfalse%
\ {\isachardoublequoteopen}Dis\ F\ G\ H\ {\isasymlongrightarrow}\ F\ {\isasymin}\ {\isacharquery}cl\ {\isasymlongrightarrow}\ {\isacharbraceleft}G{\isacharbraceright}\ {\isasymunion}\ {\isacharquery}cl\ {\isasymin}\ C\ {\isasymor}\ {\isacharbraceleft}H{\isacharbraceright}\ {\isasymunion}\ {\isacharquery}cl\ {\isasymin}\ C{\isachardoublequoteclose}\isanewline
\ \ \ \ \ \ \ \ \isacommand{using}\isamarkupfalse%
\ Dis\ \isacommand{by}\isamarkupfalse%
\ {\isacharparenleft}iprover\ elim{\isacharcolon}\ allE{\isacharparenright}\isanewline
\ \ \ \ \ \ \isacommand{then}\isamarkupfalse%
\ \isacommand{have}\isamarkupfalse%
\ {\isachardoublequoteopen}F\ {\isasymin}\ {\isacharquery}cl\ {\isasymlongrightarrow}\ {\isacharbraceleft}G{\isacharbraceright}\ {\isasymunion}\ {\isacharquery}cl\ {\isasymin}\ C\ {\isasymor}\ {\isacharbraceleft}H{\isacharbraceright}\ {\isasymunion}\ {\isacharquery}cl\ {\isasymin}\ C{\isachardoublequoteclose}\isanewline
\ \ \ \ \ \ \ \ \isacommand{using}\isamarkupfalse%
\ {\isacartoucheopen}Dis\ F\ G\ H{\isacartoucheclose}\ \isacommand{by}\isamarkupfalse%
\ {\isacharparenleft}rule\ mp{\isacharparenright}\isanewline
\ \ \ \ \ \ \isacommand{then}\isamarkupfalse%
\ \isacommand{have}\isamarkupfalse%
\ {\isachardoublequoteopen}{\isacharbraceleft}G{\isacharbraceright}\ {\isasymunion}\ {\isacharquery}cl\ {\isasymin}\ C\ {\isasymor}\ {\isacharbraceleft}H{\isacharbraceright}\ {\isasymunion}\ {\isacharquery}cl\ {\isasymin}\ C{\isachardoublequoteclose}\isanewline
\ \ \ \ \ \ \ \ \isacommand{using}\isamarkupfalse%
\ {\isacartoucheopen}F\ {\isasymin}\ {\isacharquery}cl{\isacartoucheclose}\ \isacommand{by}\isamarkupfalse%
\ {\isacharparenleft}rule\ mp{\isacharparenright}\isanewline
\ \ \ \ \ \ \isacommand{thus}\isamarkupfalse%
\ {\isachardoublequoteopen}G\ {\isasymin}\ {\isacharquery}cl\ {\isasymor}\ H\ {\isasymin}\ {\isacharquery}cl{\isachardoublequoteclose}\isanewline
\ \ \ \ \ \ \isacommand{proof}\isamarkupfalse%
\ {\isacharparenleft}rule\ disjE{\isacharparenright}\isanewline
\ \ \ \ \ \ \ \ \isacommand{assume}\isamarkupfalse%
\ {\isachardoublequoteopen}{\isacharbraceleft}G{\isacharbraceright}\ {\isasymunion}\ {\isacharquery}cl\ {\isasymin}\ C{\isachardoublequoteclose}\isanewline
\ \ \ \ \ \ \ \ \isacommand{have}\isamarkupfalse%
\ {\isachardoublequoteopen}insert\ G\ {\isacharquery}cl\ {\isacharequal}\ {\isacharbraceleft}G{\isacharbraceright}\ {\isasymunion}\ {\isacharquery}cl{\isachardoublequoteclose}\isanewline
\ \ \ \ \ \ \ \ \ \ \isacommand{by}\isamarkupfalse%
\ {\isacharparenleft}rule\ insert{\isacharunderscore}is{\isacharunderscore}Un{\isacharparenright}\isanewline
\ \ \ \ \ \ \ \ \isacommand{have}\isamarkupfalse%
\ {\isachardoublequoteopen}insert\ G\ {\isacharquery}cl\ {\isasymin}\ C{\isachardoublequoteclose}\isanewline
\ \ \ \ \ \ \ \ \ \ \isacommand{using}\isamarkupfalse%
\ {\isacartoucheopen}{\isacharbraceleft}G{\isacharbraceright}\ {\isasymunion}\ {\isacharquery}cl\ {\isasymin}\ C{\isacartoucheclose}\ \isacommand{by}\isamarkupfalse%
\ {\isacharparenleft}simp\ only{\isacharcolon}\ {\isacartoucheopen}insert\ G\ {\isacharquery}cl\ {\isacharequal}\ {\isacharbraceleft}G{\isacharbraceright}\ {\isasymunion}\ {\isacharquery}cl{\isacartoucheclose}{\isacharparenright}\isanewline
\ \ \ \ \ \ \ \ \isacommand{have}\isamarkupfalse%
\ {\isachardoublequoteopen}insert\ G\ {\isacharquery}cl\ {\isasymin}\ C\ {\isasymLongrightarrow}\ G\ {\isasymin}\ {\isacharquery}cl{\isachardoublequoteclose}\isanewline
\ \ \ \ \ \ \ \ \ \ \isacommand{using}\isamarkupfalse%
\ assms{\isacharparenleft}{\isadigit{1}}{\isacharparenright}\ assms{\isacharparenleft}{\isadigit{2}}{\isacharparenright}\ \isacommand{by}\isamarkupfalse%
\ {\isacharparenleft}rule\ cl{\isacharunderscore}max{\isacharprime}{\isacharparenright}\isanewline
\ \ \ \ \ \ \ \ \isacommand{then}\isamarkupfalse%
\ \isacommand{have}\isamarkupfalse%
\ {\isachardoublequoteopen}G\ {\isasymin}\ {\isacharquery}cl{\isachardoublequoteclose}\isanewline
\ \ \ \ \ \ \ \ \ \ \isacommand{by}\isamarkupfalse%
\ {\isacharparenleft}simp\ only{\isacharcolon}\ {\isacartoucheopen}insert\ G\ {\isacharquery}cl\ {\isasymin}\ C{\isacartoucheclose}{\isacharparenright}\isanewline
\ \ \ \ \ \ \ \ \isacommand{thus}\isamarkupfalse%
\ {\isachardoublequoteopen}G\ {\isasymin}\ {\isacharquery}cl\ {\isasymor}\ H\ {\isasymin}\ {\isacharquery}cl{\isachardoublequoteclose}\isanewline
\ \ \ \ \ \ \ \ \ \ \isacommand{by}\isamarkupfalse%
\ {\isacharparenleft}rule\ disjI{\isadigit{1}}{\isacharparenright}\isanewline
\ \ \ \ \ \ \isacommand{next}\isamarkupfalse%
\isanewline
\ \ \ \ \ \ \ \ \isacommand{assume}\isamarkupfalse%
\ {\isachardoublequoteopen}{\isacharbraceleft}H{\isacharbraceright}\ {\isasymunion}\ {\isacharquery}cl\ {\isasymin}\ C{\isachardoublequoteclose}\isanewline
\ \ \ \ \ \ \ \ \isacommand{have}\isamarkupfalse%
\ {\isachardoublequoteopen}insert\ H\ {\isacharquery}cl\ {\isacharequal}\ {\isacharbraceleft}H{\isacharbraceright}\ {\isasymunion}\ {\isacharquery}cl{\isachardoublequoteclose}\isanewline
\ \ \ \ \ \ \ \ \ \ \isacommand{by}\isamarkupfalse%
\ {\isacharparenleft}rule\ insert{\isacharunderscore}is{\isacharunderscore}Un{\isacharparenright}\isanewline
\ \ \ \ \ \ \ \ \isacommand{have}\isamarkupfalse%
\ {\isachardoublequoteopen}insert\ H\ {\isacharquery}cl\ {\isasymin}\ C{\isachardoublequoteclose}\isanewline
\ \ \ \ \ \ \ \ \ \ \isacommand{using}\isamarkupfalse%
\ {\isacartoucheopen}{\isacharbraceleft}H{\isacharbraceright}\ {\isasymunion}\ {\isacharquery}cl\ {\isasymin}\ C{\isacartoucheclose}\ \isacommand{by}\isamarkupfalse%
\ {\isacharparenleft}simp\ only{\isacharcolon}\ {\isacartoucheopen}insert\ H\ {\isacharquery}cl\ {\isacharequal}\ {\isacharbraceleft}H{\isacharbraceright}\ {\isasymunion}\ {\isacharquery}cl{\isacartoucheclose}{\isacharparenright}\isanewline
\ \ \ \ \ \ \ \ \isacommand{have}\isamarkupfalse%
\ {\isachardoublequoteopen}insert\ H\ {\isacharquery}cl\ {\isasymin}\ C\ {\isasymLongrightarrow}\ H\ {\isasymin}\ {\isacharquery}cl{\isachardoublequoteclose}\isanewline
\ \ \ \ \ \ \ \ \ \ \isacommand{using}\isamarkupfalse%
\ assms{\isacharparenleft}{\isadigit{1}}{\isacharparenright}\ assms{\isacharparenleft}{\isadigit{2}}{\isacharparenright}\ \isacommand{by}\isamarkupfalse%
\ {\isacharparenleft}rule\ cl{\isacharunderscore}max{\isacharprime}{\isacharparenright}\isanewline
\ \ \ \ \ \ \ \ \isacommand{then}\isamarkupfalse%
\ \isacommand{have}\isamarkupfalse%
\ {\isachardoublequoteopen}H\ {\isasymin}\ {\isacharquery}cl{\isachardoublequoteclose}\isanewline
\ \ \ \ \ \ \ \ \ \ \isacommand{by}\isamarkupfalse%
\ {\isacharparenleft}simp\ only{\isacharcolon}\ {\isacartoucheopen}insert\ H\ {\isacharquery}cl\ {\isasymin}\ C{\isacartoucheclose}{\isacharparenright}\isanewline
\ \ \ \ \ \ \ \ \isacommand{thus}\isamarkupfalse%
\ {\isachardoublequoteopen}G\ {\isasymin}\ {\isacharquery}cl\ {\isasymor}\ H\ {\isasymin}\ {\isacharquery}cl{\isachardoublequoteclose}\isanewline
\ \ \ \ \ \ \ \ \ \ \isacommand{by}\isamarkupfalse%
\ {\isacharparenleft}rule\ disjI{\isadigit{2}}{\isacharparenright}\isanewline
\ \ \ \ \ \ \isacommand{qed}\isamarkupfalse%
\isanewline
\ \ \ \ \isacommand{qed}\isamarkupfalse%
\isanewline
\ \ \isacommand{qed}\isamarkupfalse%
\isanewline
\ \ \isacommand{show}\isamarkupfalse%
\ {\isachardoublequoteopen}{\isasymbottom}\ {\isasymnotin}\ {\isacharquery}cl\ {\isasymand}\isanewline
\ \ \ \ {\isacharparenleft}{\isasymforall}k{\isachardot}\ Atom\ k\ {\isasymin}\ {\isacharquery}cl\ {\isasymlongrightarrow}\ \isactrlbold {\isasymnot}\ {\isacharparenleft}Atom\ k{\isacharparenright}\ {\isasymin}\ {\isacharquery}cl\ {\isasymlongrightarrow}\ False{\isacharparenright}\ {\isasymand}\isanewline
\ \ \ \ {\isacharparenleft}{\isasymforall}F\ G\ H{\isachardot}\ Con\ F\ G\ H\ {\isasymlongrightarrow}\ F\ {\isasymin}\ {\isacharquery}cl\ {\isasymlongrightarrow}\ G\ {\isasymin}\ {\isacharquery}cl\ {\isasymand}\ H\ {\isasymin}\ {\isacharquery}cl{\isacharparenright}\ {\isasymand}\isanewline
\ \ \ \ {\isacharparenleft}{\isasymforall}F\ G\ H{\isachardot}\ Dis\ F\ G\ H\ {\isasymlongrightarrow}\ F\ {\isasymin}\ {\isacharquery}cl\ {\isasymlongrightarrow}\ G\ {\isasymin}\ {\isacharquery}cl\ {\isasymor}\ H\ {\isasymin}\ {\isacharquery}cl{\isacharparenright}{\isachardoublequoteclose}\isanewline
\ \ \ \ \isacommand{using}\isamarkupfalse%
\ H{\isadigit{1}}\ H{\isadigit{2}}\ H{\isadigit{3}}\ H{\isadigit{4}}\ \isacommand{by}\isamarkupfalse%
\ {\isacharparenleft}iprover\ intro{\isacharcolon}\ conjI{\isacharparenright}\isanewline
\isacommand{qed}\isamarkupfalse%
%
\endisatagproof
{\isafoldproof}%
%
\isadelimproof
%
\endisadelimproof
%
\begin{isamarkuptext}%
Del mismo modo, podemos probar el resultado de manera automática como sigue.%
\end{isamarkuptext}\isamarkuptrue%
\isacommand{lemma}\isamarkupfalse%
\ pcp{\isacharunderscore}lim{\isacharunderscore}Hintikka{\isacharcolon}\isanewline
\ \ \isakeyword{assumes}\ c{\isacharcolon}\ {\isachardoublequoteopen}pcp\ C{\isachardoublequoteclose}\isanewline
\ \ \isakeyword{assumes}\ sc{\isacharcolon}\ {\isachardoublequoteopen}subset{\isacharunderscore}closed\ C{\isachardoublequoteclose}\isanewline
\ \ \isakeyword{assumes}\ fc{\isacharcolon}\ {\isachardoublequoteopen}finite{\isacharunderscore}character\ C{\isachardoublequoteclose}\isanewline
\ \ \isakeyword{assumes}\ el{\isacharcolon}\ {\isachardoublequoteopen}S\ {\isasymin}\ C{\isachardoublequoteclose}\isanewline
\ \ \isakeyword{shows}\ {\isachardoublequoteopen}Hintikka\ {\isacharparenleft}pcp{\isacharunderscore}lim\ C\ S{\isacharparenright}{\isachardoublequoteclose}\isanewline
%
\isadelimproof
%
\endisadelimproof
%
\isatagproof
\isacommand{proof}\isamarkupfalse%
\ {\isacharminus}\isanewline
\ \ \isacommand{let}\isamarkupfalse%
\ {\isacharquery}cl\ {\isacharequal}\ {\isachardoublequoteopen}pcp{\isacharunderscore}lim\ C\ S{\isachardoublequoteclose}\isanewline
\ \ \isacommand{have}\isamarkupfalse%
\ {\isachardoublequoteopen}{\isacharquery}cl\ {\isasymin}\ C{\isachardoublequoteclose}\ \isacommand{using}\isamarkupfalse%
\ pcp{\isacharunderscore}lim{\isacharunderscore}in{\isacharbrackleft}OF\ c\ el\ sc\ fc{\isacharbrackright}\ \isacommand{{\isachardot}}\isamarkupfalse%
\isanewline
\ \ \isacommand{from}\isamarkupfalse%
\ c{\isacharbrackleft}unfolded\ pcp{\isacharunderscore}alt{\isacharcomma}\ THEN\ bspec{\isacharcomma}\ OF\ this{\isacharbrackright}\isanewline
\ \ \isacommand{have}\isamarkupfalse%
\ d{\isacharcolon}\ {\isachardoublequoteopen}{\isasymbottom}\ {\isasymnotin}\ {\isacharquery}cl{\isachardoublequoteclose}\isanewline
\ \ \ \ {\isachardoublequoteopen}Atom\ k\ {\isasymin}\ {\isacharquery}cl\ {\isasymLongrightarrow}\ \isactrlbold {\isasymnot}\ {\isacharparenleft}Atom\ k{\isacharparenright}\ {\isasymin}\ {\isacharquery}cl\ {\isasymLongrightarrow}\ False{\isachardoublequoteclose}\isanewline
\ \ \ \ {\isachardoublequoteopen}Con\ F\ G\ H\ {\isasymLongrightarrow}\ F\ {\isasymin}\ {\isacharquery}cl\ {\isasymLongrightarrow}\ insert\ G\ {\isacharparenleft}insert\ H\ {\isacharquery}cl{\isacharparenright}\ {\isasymin}\ C{\isachardoublequoteclose}\isanewline
\ \ \ \ {\isachardoublequoteopen}Dis\ F\ G\ H\ {\isasymLongrightarrow}\ F\ {\isasymin}\ {\isacharquery}cl\ {\isasymLongrightarrow}\ insert\ G\ {\isacharquery}cl\ {\isasymin}\ C\ {\isasymor}\ insert\ H\ {\isacharquery}cl\ {\isasymin}\ C{\isachardoublequoteclose}\isanewline
\ \ \ \ \isakeyword{for}\ k\ F\ G\ H\ \isacommand{by}\isamarkupfalse%
\ force{\isacharplus}\isanewline
\ \ \isacommand{with}\isamarkupfalse%
\ d{\isacharparenleft}{\isadigit{1}}{\isacharcomma}{\isadigit{2}}{\isacharparenright}\ \isacommand{show}\isamarkupfalse%
\ {\isachardoublequoteopen}Hintikka\ {\isacharquery}cl{\isachardoublequoteclose}\ \isacommand{unfolding}\isamarkupfalse%
\ Hintikka{\isacharunderscore}alt\ \isanewline
\ \ \ \ \isacommand{using}\isamarkupfalse%
\ c\ cl{\isacharunderscore}max\ cl{\isacharunderscore}max{\isacharprime}\ d{\isacharparenleft}{\isadigit{4}}{\isacharparenright}\ sc\ \isacommand{by}\isamarkupfalse%
\ blast\isanewline
\isacommand{qed}\isamarkupfalse%
%
\endisatagproof
{\isafoldproof}%
%
\isadelimproof
%
\endisadelimproof
%
\begin{isamarkuptext}%
Finalmente, vamos a demostrar el \isa{teorema\ de\ existencia\ de\ modelo}. Para ello precisaremos de
  un resultado que indica que la satisfacibilidad de conjuntos de fórmulas se hereda por la 
  contención.

  \begin{lema}
    Todo subconjunto de un conjunto de fórmulas satisfacible es satisfacible.
  \end{lema}

  \begin{demostracion}
    Sea \isa{B} un conjunto de fórmulas satisfacible y \isa{A\ {\isasymsubseteq}\ B}. Veamos que \isa{A} es satisfacible.
    Por definición, como \isa{B} es satisfacible, existe una interpretación \isa{{\isasymA}} que es modelo de cada 
    fórmula de \isa{B}. Como \isa{A\ {\isasymsubseteq}\ B}, en particular \isa{{\isasymA}} es modelo de toda fórmula de \isa{A}. Por tanto, 
    \isa{A} es satisfacible, ya que existe una interpretación que es modelo de todas sus fórmulas.
  \end{demostracion}

  Su prueba detallada en Isabelle/HOL es la siguiente.%
\end{isamarkuptext}\isamarkuptrue%
\isacommand{lemma}\isamarkupfalse%
\ sat{\isacharunderscore}mono{\isacharcolon}\isanewline
\ \ \isakeyword{assumes}\ {\isachardoublequoteopen}A\ {\isasymsubseteq}\ B{\isachardoublequoteclose}\isanewline
\ \ \ \ \ \ \ \ \ \ {\isachardoublequoteopen}sat\ B{\isachardoublequoteclose}\isanewline
\ \ \ \ \ \ \ \ \isakeyword{shows}\ {\isachardoublequoteopen}sat\ A{\isachardoublequoteclose}\isanewline
%
\isadelimproof
\ \ %
\endisadelimproof
%
\isatagproof
\isacommand{unfolding}\isamarkupfalse%
\ sat{\isacharunderscore}def\isanewline
\isacommand{proof}\isamarkupfalse%
\ {\isacharminus}\isanewline
\ \isacommand{have}\isamarkupfalse%
\ satB{\isacharcolon}{\isachardoublequoteopen}{\isasymexists}{\isasymA}{\isachardot}\ {\isasymforall}F\ {\isasymin}\ B{\isachardot}\ {\isasymA}\ {\isasymTurnstile}\ F{\isachardoublequoteclose}\isanewline
\ \ \ \isacommand{using}\isamarkupfalse%
\ assms{\isacharparenleft}{\isadigit{2}}{\isacharparenright}\ \isacommand{by}\isamarkupfalse%
\ {\isacharparenleft}simp\ only{\isacharcolon}\ sat{\isacharunderscore}def{\isacharparenright}\isanewline
\ \isacommand{obtain}\isamarkupfalse%
\ {\isasymA}\ \isakeyword{where}\ {\isachardoublequoteopen}{\isasymforall}F\ {\isasymin}\ B{\isachardot}\ {\isasymA}\ {\isasymTurnstile}\ F{\isachardoublequoteclose}\isanewline
\ \ \ \ \isacommand{using}\isamarkupfalse%
\ satB\ \isacommand{by}\isamarkupfalse%
\ {\isacharparenleft}rule\ exE{\isacharparenright}\isanewline
\ \isacommand{have}\isamarkupfalse%
\ {\isachardoublequoteopen}{\isasymforall}F\ {\isasymin}\ A{\isachardot}\ {\isasymA}\ {\isasymTurnstile}\ F{\isachardoublequoteclose}\isanewline
\ \ \isacommand{proof}\isamarkupfalse%
\ {\isacharparenleft}rule\ ballI{\isacharparenright}\isanewline
\ \ \ \ \isacommand{fix}\isamarkupfalse%
\ F\isanewline
\ \ \ \ \isacommand{assume}\isamarkupfalse%
\ {\isachardoublequoteopen}F\ {\isasymin}\ A{\isachardoublequoteclose}\isanewline
\ \ \ \ \isacommand{have}\isamarkupfalse%
\ {\isachardoublequoteopen}F\ {\isasymin}\ A\ {\isasymlongrightarrow}\ F\ {\isasymin}\ B{\isachardoublequoteclose}\isanewline
\ \ \ \ \ \ \isacommand{using}\isamarkupfalse%
\ assms{\isacharparenleft}{\isadigit{1}}{\isacharparenright}\ \isacommand{by}\isamarkupfalse%
\ {\isacharparenleft}rule\ in{\isacharunderscore}mono{\isacharparenright}\isanewline
\ \ \ \ \isacommand{then}\isamarkupfalse%
\ \isacommand{have}\isamarkupfalse%
\ {\isachardoublequoteopen}F\ {\isasymin}\ B{\isachardoublequoteclose}\isanewline
\ \ \ \ \ \ \isacommand{using}\isamarkupfalse%
\ {\isacartoucheopen}F\ {\isasymin}\ A{\isacartoucheclose}\ \isacommand{by}\isamarkupfalse%
\ {\isacharparenleft}rule\ mp{\isacharparenright}\isanewline
\ \ \ \ \isacommand{show}\isamarkupfalse%
\ {\isachardoublequoteopen}{\isasymA}\ {\isasymTurnstile}\ F{\isachardoublequoteclose}\isanewline
\ \ \ \ \ \ \isacommand{using}\isamarkupfalse%
\ {\isacartoucheopen}{\isasymforall}F\ {\isasymin}\ B{\isachardot}\ {\isasymA}\ {\isasymTurnstile}\ F{\isacartoucheclose}\ {\isacartoucheopen}F\ {\isasymin}\ B{\isacartoucheclose}\ \isacommand{by}\isamarkupfalse%
\ {\isacharparenleft}rule\ bspec{\isacharparenright}\isanewline
\ \ \isacommand{qed}\isamarkupfalse%
\isanewline
\ \ \isacommand{thus}\isamarkupfalse%
\ {\isachardoublequoteopen}{\isasymexists}{\isasymA}{\isachardot}\ {\isasymforall}F\ {\isasymin}\ A{\isachardot}\ {\isasymA}\ {\isasymTurnstile}\ F{\isachardoublequoteclose}\isanewline
\ \ \ \ \isacommand{by}\isamarkupfalse%
\ {\isacharparenleft}simp\ only{\isacharcolon}\ exI{\isacharparenright}\isanewline
\isacommand{qed}\isamarkupfalse%
%
\endisatagproof
{\isafoldproof}%
%
\isadelimproof
%
\endisadelimproof
%
\begin{isamarkuptext}%
De este modo, procedamos finalmente con la demostración del teorema.

  \begin{teorema}[Teorema de Existencia de Modelo]
    Todo conjunto de fórmulas perteneciente a una colección que verifique la propiedad de
    consistencia proposicional es satisfacible. 
  \end{teorema}

  \begin{demostracion}
    Sea \isa{C} una colección de conjuntos de fórmulas proposicionales que verifique la propiedad de 
    consistencia proposicional y \isa{S\ {\isasymin}\ C}. Vamos a probar que \isa{S} es satisfacible.

    En primer lugar, como \isa{C} verifica la propiedad de consistencia proposicional, por el lema 
    \isa{{\isadigit{3}}{\isachardot}{\isadigit{0}}{\isachardot}{\isadigit{3}}} podemos extenderla a una colección \isa{C{\isacharprime}} que también verifique la propiedad y
    sea cerrada bajo subconjuntos. A su vez, por el lema \isa{{\isadigit{3}}{\isachardot}{\isadigit{0}}{\isachardot}{\isadigit{5}}}, como la extensión 
    \isa{C{\isacharprime}} es una colección con la propiedad de consistencia proposicional y cerrada bajo 
    subconjuntos, podemos extenderla a otra colección \isa{C{\isacharprime}{\isacharprime}} que también verifica la propiedad de
    consistencia proposicional y sea de carácter finito. De este modo, por la transitividad de la 
    contención, es claro que \isa{C{\isacharprime}{\isacharprime}} es una extensión de \isa{C}, luego \isa{S\ {\isasymin}\ C{\isacharprime}{\isacharprime}} por hipótesis. 
    Por otro lado, por el lema \isa{{\isadigit{3}}{\isachardot}{\isadigit{0}}{\isachardot}{\isadigit{4}}}, como \isa{C{\isacharprime}{\isacharprime}} es de carácter finito, se tiene que es 
    cerrada bajo subconjuntos. 

    En suma, \isa{C{\isacharprime}{\isacharprime}} es una extensión de \isa{C} que verifica la propiedad de consistencia proposicional, 
    es cerrada bajo subconjuntos y es de carácter finito. Luego, por el lema \isa{{\isadigit{4}}{\isachardot}{\isadigit{2}}{\isachardot}{\isadigit{4}}}, el límite de 
    la sucesión \isa{{\isacharbraceleft}S\isactrlsub n{\isacharbraceright}} de conjuntos de \isa{C{\isacharprime}{\isacharprime}} a partir de \isa{S} según la definición \isa{{\isadigit{4}}{\isachardot}{\isadigit{1}}{\isachardot}{\isadigit{1}}} es un 
    conjunto de Hintikka. Por tanto, por el \isa{teorema\ de\ Hintikka}, se trata de un conjunto 
    satisfacible. 

    Finalmente, puesto que para todo \isa{n\ {\isasymin}\ {\isasymnat}} se tiene que \isa{S\isactrlsub n} está contenido en el límite, en 
    particular el conjunto \isa{S\isactrlsub {\isadigit{0}}} está contenido en él. Por definición de la sucesión, dicho conjunto 
    coincide con \isa{S}. Por tanto, como \isa{S} está contenido en el límite que es un conjunto 
    satisfacible, queda demostrada la satisfacibilidad de \isa{S}.
  \end{demostracion}

  \comentario{Tal vez sería buena idea hacer un grafo similar al de ex3.}

  Mostremos su formalización y demostración detallada en Isabelle.%
\end{isamarkuptext}\isamarkuptrue%
\isacommand{theorem}\isamarkupfalse%
\isanewline
\ \ \isakeyword{fixes}\ S\ {\isacharcolon}{\isacharcolon}\ {\isachardoublequoteopen}{\isacharprime}a\ {\isacharcolon}{\isacharcolon}\ countable\ formula\ set{\isachardoublequoteclose}\isanewline
\ \ \isakeyword{assumes}\ {\isachardoublequoteopen}pcp\ C{\isachardoublequoteclose}\isanewline
\ \ \isakeyword{assumes}\ {\isachardoublequoteopen}S\ {\isasymin}\ C{\isachardoublequoteclose}\isanewline
\ \ \isakeyword{shows}\ {\isachardoublequoteopen}sat\ S{\isachardoublequoteclose}\isanewline
%
\isadelimproof
%
\endisadelimproof
%
\isatagproof
\isacommand{proof}\isamarkupfalse%
\ {\isacharminus}\isanewline
\ \ \isacommand{have}\isamarkupfalse%
\ {\isachardoublequoteopen}pcp\ C\ {\isasymLongrightarrow}\ {\isasymexists}C{\isacharprime}{\isachardot}\ C\ {\isasymsubseteq}\ C{\isacharprime}\ {\isasymand}\ pcp\ C{\isacharprime}\ {\isasymand}\ subset{\isacharunderscore}closed\ C{\isacharprime}{\isachardoublequoteclose}\isanewline
\ \ \ \ \isacommand{by}\isamarkupfalse%
\ {\isacharparenleft}rule\ ex{\isadigit{1}}{\isacharparenright}\isanewline
\ \ \isacommand{then}\isamarkupfalse%
\ \isacommand{have}\isamarkupfalse%
\ E{\isadigit{1}}{\isacharcolon}{\isachardoublequoteopen}{\isasymexists}C{\isacharprime}{\isachardot}\ C\ {\isasymsubseteq}\ C{\isacharprime}\ {\isasymand}\ pcp\ C{\isacharprime}\ {\isasymand}\ subset{\isacharunderscore}closed\ C{\isacharprime}{\isachardoublequoteclose}\isanewline
\ \ \ \ \isacommand{by}\isamarkupfalse%
\ {\isacharparenleft}simp\ only{\isacharcolon}\ assms{\isacharparenleft}{\isadigit{1}}{\isacharparenright}{\isacharparenright}\isanewline
\ \ \isacommand{obtain}\isamarkupfalse%
\ Ce{\isacharprime}\ \isakeyword{where}\ H{\isadigit{1}}{\isacharcolon}{\isachardoublequoteopen}C\ {\isasymsubseteq}\ Ce{\isacharprime}\ {\isasymand}\ pcp\ Ce{\isacharprime}\ {\isasymand}\ subset{\isacharunderscore}closed\ Ce{\isacharprime}{\isachardoublequoteclose}\isanewline
\ \ \ \ \isacommand{using}\isamarkupfalse%
\ E{\isadigit{1}}\ \isacommand{by}\isamarkupfalse%
\ {\isacharparenleft}rule\ exE{\isacharparenright}\isanewline
\ \ \isacommand{have}\isamarkupfalse%
\ {\isachardoublequoteopen}C\ {\isasymsubseteq}\ Ce{\isacharprime}{\isachardoublequoteclose}\isanewline
\ \ \ \ \isacommand{using}\isamarkupfalse%
\ H{\isadigit{1}}\ \isacommand{by}\isamarkupfalse%
\ {\isacharparenleft}rule\ conjunct{\isadigit{1}}{\isacharparenright}\isanewline
\ \ \isacommand{have}\isamarkupfalse%
\ {\isachardoublequoteopen}pcp\ Ce{\isacharprime}{\isachardoublequoteclose}\isanewline
\ \ \ \ \isacommand{using}\isamarkupfalse%
\ H{\isadigit{1}}\ \isacommand{by}\isamarkupfalse%
\ {\isacharparenleft}iprover\ elim{\isacharcolon}\ conjunct{\isadigit{2}}\ conjunct{\isadigit{1}}{\isacharparenright}\isanewline
\ \ \isacommand{have}\isamarkupfalse%
\ {\isachardoublequoteopen}subset{\isacharunderscore}closed\ Ce{\isacharprime}{\isachardoublequoteclose}\isanewline
\ \ \ \ \isacommand{using}\isamarkupfalse%
\ H{\isadigit{1}}\ \isacommand{by}\isamarkupfalse%
\ {\isacharparenleft}iprover\ elim{\isacharcolon}\ conjunct{\isadigit{2}}\ conjunct{\isadigit{1}}{\isacharparenright}\isanewline
\ \ \isacommand{have}\isamarkupfalse%
\ E{\isadigit{2}}{\isacharcolon}{\isachardoublequoteopen}{\isasymexists}Ce{\isachardot}\ Ce{\isacharprime}\ {\isasymsubseteq}\ Ce\ {\isasymand}\ pcp\ Ce\ {\isasymand}\ finite{\isacharunderscore}character\ Ce{\isachardoublequoteclose}\isanewline
\ \ \ \ \isacommand{using}\isamarkupfalse%
\ {\isacartoucheopen}pcp\ Ce{\isacharprime}{\isacartoucheclose}\ {\isacartoucheopen}subset{\isacharunderscore}closed\ Ce{\isacharprime}{\isacartoucheclose}\ \isacommand{by}\isamarkupfalse%
\ {\isacharparenleft}rule\ ex{\isadigit{3}}{\isacharparenright}\isanewline
\ \ \isacommand{obtain}\isamarkupfalse%
\ Ce\ \isakeyword{where}\ H{\isadigit{2}}{\isacharcolon}{\isachardoublequoteopen}Ce{\isacharprime}\ {\isasymsubseteq}\ Ce\ {\isasymand}\ pcp\ Ce\ {\isasymand}\ finite{\isacharunderscore}character\ Ce{\isachardoublequoteclose}\isanewline
\ \ \ \ \isacommand{using}\isamarkupfalse%
\ E{\isadigit{2}}\ \isacommand{by}\isamarkupfalse%
\ {\isacharparenleft}rule\ exE{\isacharparenright}\isanewline
\ \ \isacommand{have}\isamarkupfalse%
\ {\isachardoublequoteopen}Ce{\isacharprime}\ {\isasymsubseteq}\ Ce{\isachardoublequoteclose}\isanewline
\ \ \ \ \isacommand{using}\isamarkupfalse%
\ H{\isadigit{2}}\ \isacommand{by}\isamarkupfalse%
\ {\isacharparenleft}rule\ conjunct{\isadigit{1}}{\isacharparenright}\isanewline
\ \ \isacommand{then}\isamarkupfalse%
\ \isacommand{have}\isamarkupfalse%
\ Subset{\isacharcolon}{\isachardoublequoteopen}C\ {\isasymsubseteq}\ Ce{\isachardoublequoteclose}\isanewline
\ \ \ \ \isacommand{using}\isamarkupfalse%
\ {\isacartoucheopen}C\ {\isasymsubseteq}\ Ce{\isacharprime}{\isacartoucheclose}\ \isacommand{by}\isamarkupfalse%
\ {\isacharparenleft}simp\ only{\isacharcolon}\ subset{\isacharunderscore}trans{\isacharparenright}\isanewline
\ \ \isacommand{have}\isamarkupfalse%
\ Pcp{\isacharcolon}{\isachardoublequoteopen}pcp\ Ce{\isachardoublequoteclose}\isanewline
\ \ \ \ \isacommand{using}\isamarkupfalse%
\ H{\isadigit{2}}\ \isacommand{by}\isamarkupfalse%
\ {\isacharparenleft}iprover\ elim{\isacharcolon}\ conjunct{\isadigit{2}}\ conjunct{\isadigit{1}}{\isacharparenright}\isanewline
\ \ \isacommand{have}\isamarkupfalse%
\ FC{\isacharcolon}{\isachardoublequoteopen}finite{\isacharunderscore}character\ Ce{\isachardoublequoteclose}\isanewline
\ \ \ \ \isacommand{using}\isamarkupfalse%
\ H{\isadigit{2}}\ \isacommand{by}\isamarkupfalse%
\ {\isacharparenleft}iprover\ elim{\isacharcolon}\ conjunct{\isadigit{2}}\ conjunct{\isadigit{1}}{\isacharparenright}\isanewline
\ \ \isacommand{then}\isamarkupfalse%
\ \isacommand{have}\isamarkupfalse%
\ SC{\isacharcolon}{\isachardoublequoteopen}subset{\isacharunderscore}closed\ Ce{\isachardoublequoteclose}\isanewline
\ \ \ \ \isacommand{by}\isamarkupfalse%
\ {\isacharparenleft}rule\ ex{\isadigit{2}}{\isacharparenright}\isanewline
\ \ \isacommand{have}\isamarkupfalse%
\ {\isachardoublequoteopen}S\ {\isasymin}\ C\ {\isasymlongrightarrow}\ S\ {\isasymin}\ Ce{\isachardoublequoteclose}\isanewline
\ \ \ \ \isacommand{using}\isamarkupfalse%
\ {\isacartoucheopen}C\ {\isasymsubseteq}\ Ce{\isacartoucheclose}\ \isacommand{by}\isamarkupfalse%
\ {\isacharparenleft}rule\ in{\isacharunderscore}mono{\isacharparenright}\isanewline
\ \ \isacommand{then}\isamarkupfalse%
\ \isacommand{have}\isamarkupfalse%
\ {\isachardoublequoteopen}S\ {\isasymin}\ Ce{\isachardoublequoteclose}\ \isanewline
\ \ \ \ \isacommand{using}\isamarkupfalse%
\ assms{\isacharparenleft}{\isadigit{2}}{\isacharparenright}\ \isacommand{by}\isamarkupfalse%
\ {\isacharparenleft}rule\ mp{\isacharparenright}\isanewline
\ \ \isacommand{have}\isamarkupfalse%
\ {\isachardoublequoteopen}Hintikka\ {\isacharparenleft}pcp{\isacharunderscore}lim\ Ce\ S{\isacharparenright}{\isachardoublequoteclose}\isanewline
\ \ \ \ \isacommand{using}\isamarkupfalse%
\ Pcp\ SC\ FC\ {\isacartoucheopen}S\ {\isasymin}\ Ce{\isacartoucheclose}\ \isacommand{by}\isamarkupfalse%
\ {\isacharparenleft}rule\ pcp{\isacharunderscore}lim{\isacharunderscore}Hintikka{\isacharparenright}\isanewline
\ \ \isacommand{then}\isamarkupfalse%
\ \isacommand{have}\isamarkupfalse%
\ {\isachardoublequoteopen}sat\ {\isacharparenleft}pcp{\isacharunderscore}lim\ Ce\ S{\isacharparenright}{\isachardoublequoteclose}\isanewline
\ \ \ \ \isacommand{by}\isamarkupfalse%
\ {\isacharparenleft}rule\ Hintikkaslemma{\isacharparenright}\isanewline
\ \ \isacommand{have}\isamarkupfalse%
\ {\isachardoublequoteopen}pcp{\isacharunderscore}seq\ Ce\ S\ {\isadigit{0}}\ {\isacharequal}\ S{\isachardoublequoteclose}\isanewline
\ \ \ \ \isacommand{by}\isamarkupfalse%
\ {\isacharparenleft}simp\ only{\isacharcolon}\ pcp{\isacharunderscore}seq{\isachardot}simps{\isacharparenleft}{\isadigit{1}}{\isacharparenright}{\isacharparenright}\isanewline
\ \ \isacommand{have}\isamarkupfalse%
\ {\isachardoublequoteopen}pcp{\isacharunderscore}seq\ Ce\ S\ {\isadigit{0}}\ {\isasymsubseteq}\ pcp{\isacharunderscore}lim\ Ce\ S{\isachardoublequoteclose}\isanewline
\ \ \ \ \isacommand{by}\isamarkupfalse%
\ {\isacharparenleft}rule\ pcp{\isacharunderscore}seq{\isacharunderscore}sub{\isacharparenright}\isanewline
\ \ \isacommand{then}\isamarkupfalse%
\ \isacommand{have}\isamarkupfalse%
\ {\isachardoublequoteopen}S\ {\isasymsubseteq}\ pcp{\isacharunderscore}lim\ Ce\ S{\isachardoublequoteclose}\isanewline
\ \ \ \ \isacommand{by}\isamarkupfalse%
\ {\isacharparenleft}simp\ only{\isacharcolon}\ {\isacartoucheopen}pcp{\isacharunderscore}seq\ Ce\ S\ {\isadigit{0}}\ {\isacharequal}\ S{\isacartoucheclose}{\isacharparenright}\isanewline
\ \ \isacommand{thus}\isamarkupfalse%
\ {\isachardoublequoteopen}sat\ S{\isachardoublequoteclose}\isanewline
\ \ \ \ \isacommand{using}\isamarkupfalse%
\ {\isacartoucheopen}sat\ {\isacharparenleft}pcp{\isacharunderscore}lim\ Ce\ S{\isacharparenright}{\isacartoucheclose}\ \isacommand{by}\isamarkupfalse%
\ {\isacharparenleft}rule\ sat{\isacharunderscore}mono{\isacharparenright}\isanewline
\isacommand{qed}\isamarkupfalse%
%
\endisatagproof
{\isafoldproof}%
%
\isadelimproof
%
\endisadelimproof
%
\begin{isamarkuptext}%
Finalmente, demostremos el teorema de manera automática.%
\end{isamarkuptext}\isamarkuptrue%
\isacommand{theorem}\isamarkupfalse%
\ pcp{\isacharunderscore}sat{\isacharcolon}\isanewline
\ \ \isakeyword{fixes}\ S\ {\isacharcolon}{\isacharcolon}\ {\isachardoublequoteopen}{\isacharprime}a\ {\isacharcolon}{\isacharcolon}\ countable\ formula\ set{\isachardoublequoteclose}\isanewline
\ \ \isakeyword{assumes}\ c{\isacharcolon}\ {\isachardoublequoteopen}pcp\ C{\isachardoublequoteclose}\isanewline
\ \ \isakeyword{assumes}\ el{\isacharcolon}\ {\isachardoublequoteopen}S\ {\isasymin}\ C{\isachardoublequoteclose}\isanewline
\ \ \isakeyword{shows}\ {\isachardoublequoteopen}sat\ S{\isachardoublequoteclose}\isanewline
%
\isadelimproof
%
\endisadelimproof
%
\isatagproof
\isacommand{proof}\isamarkupfalse%
\ {\isacharminus}\isanewline
\ \ \isacommand{from}\isamarkupfalse%
\ c\ \isacommand{obtain}\isamarkupfalse%
\ Ce\ \isakeyword{where}\ \isanewline
\ \ \ \ \ \ {\isachardoublequoteopen}C\ {\isasymsubseteq}\ Ce{\isachardoublequoteclose}\ {\isachardoublequoteopen}pcp\ Ce{\isachardoublequoteclose}\ {\isachardoublequoteopen}subset{\isacharunderscore}closed\ Ce{\isachardoublequoteclose}\ {\isachardoublequoteopen}finite{\isacharunderscore}character\ Ce{\isachardoublequoteclose}\ \isanewline
\ \ \ \ \ \ \isacommand{using}\isamarkupfalse%
\ ex{\isadigit{1}}{\isacharbrackleft}\isakeyword{where}\ {\isacharprime}a{\isacharequal}{\isacharprime}a{\isacharbrackright}\ ex{\isadigit{2}}{\isacharbrackleft}\isakeyword{where}\ {\isacharprime}a{\isacharequal}{\isacharprime}a{\isacharbrackright}\ ex{\isadigit{3}}{\isacharbrackleft}\isakeyword{where}\ {\isacharprime}a{\isacharequal}{\isacharprime}a{\isacharbrackright}\isanewline
\ \ \ \ \isacommand{by}\isamarkupfalse%
\ {\isacharparenleft}meson\ dual{\isacharunderscore}order{\isachardot}trans\ ex{\isadigit{2}}{\isacharparenright}\isanewline
\ \ \isacommand{have}\isamarkupfalse%
\ {\isachardoublequoteopen}S\ {\isasymin}\ Ce{\isachardoublequoteclose}\ \isacommand{using}\isamarkupfalse%
\ {\isacartoucheopen}C\ {\isasymsubseteq}\ Ce{\isacartoucheclose}\ el\ \isacommand{{\isachardot}{\isachardot}}\isamarkupfalse%
\isanewline
\ \ \isacommand{with}\isamarkupfalse%
\ pcp{\isacharunderscore}lim{\isacharunderscore}Hintikka\ {\isacartoucheopen}pcp\ Ce{\isacartoucheclose}\ {\isacartoucheopen}subset{\isacharunderscore}closed\ Ce{\isacartoucheclose}\ {\isacartoucheopen}finite{\isacharunderscore}character\ Ce{\isacartoucheclose}\isanewline
\ \ \isacommand{have}\isamarkupfalse%
\ \ {\isachardoublequoteopen}Hintikka\ {\isacharparenleft}pcp{\isacharunderscore}lim\ Ce\ S{\isacharparenright}{\isachardoublequoteclose}\ \isacommand{{\isachardot}}\isamarkupfalse%
\isanewline
\ \ \isacommand{with}\isamarkupfalse%
\ Hintikkaslemma\ \isacommand{have}\isamarkupfalse%
\ {\isachardoublequoteopen}sat\ {\isacharparenleft}pcp{\isacharunderscore}lim\ Ce\ S{\isacharparenright}{\isachardoublequoteclose}\ \isacommand{{\isachardot}}\isamarkupfalse%
\isanewline
\ \ \isacommand{moreover}\isamarkupfalse%
\ \isacommand{have}\isamarkupfalse%
\ {\isachardoublequoteopen}S\ {\isasymsubseteq}\ pcp{\isacharunderscore}lim\ Ce\ S{\isachardoublequoteclose}\ \isanewline
\ \ \ \ \isacommand{using}\isamarkupfalse%
\ pcp{\isacharunderscore}seq{\isachardot}simps{\isacharparenleft}{\isadigit{1}}{\isacharparenright}\ pcp{\isacharunderscore}seq{\isacharunderscore}sub\ \isacommand{by}\isamarkupfalse%
\ fast\isanewline
\ \ \isacommand{ultimately}\isamarkupfalse%
\ \isacommand{show}\isamarkupfalse%
\ {\isacharquery}thesis\ \isacommand{unfolding}\isamarkupfalse%
\ sat{\isacharunderscore}def\ \isacommand{by}\isamarkupfalse%
\ fast\isanewline
\isacommand{qed}\isamarkupfalse%
%
\endisatagproof
{\isafoldproof}%
%
\isadelimproof
%
\endisadelimproof
%
\isadelimdocument
%
\endisadelimdocument
%
\isatagdocument
%
\isamarkupsection{Teorema de Compacidad%
}
\isamarkuptrue%
%
\endisatagdocument
{\isafolddocument}%
%
\isadelimdocument
%
\endisadelimdocument
%
\begin{isamarkuptext}%
\comentario{Añadir explicación nexo.}

  \begin{teorema}[Teorema de Compacidad]
    Todo conjunto de fórmulas finitamente satisfacible es satisfacible.
  \end{teorema}

  \comentario{Añadir demostracion y nexos.}%
\end{isamarkuptext}\isamarkuptrue%
\isacommand{definition}\isamarkupfalse%
\ colecComp\ {\isacharcolon}{\isacharcolon}\ {\isachardoublequoteopen}{\isacharprime}a\ formula\ set\ {\isasymRightarrow}\ {\isacharparenleft}{\isacharprime}a\ formula\ set{\isacharparenright}\ set{\isachardoublequoteclose}\isanewline
\ \ \isakeyword{where}\ colecComp{\isacharcolon}\ {\isachardoublequoteopen}colecComp\ S\ {\isacharequal}\ {\isacharbraceleft}W{\isachardot}\ fin{\isacharunderscore}sat\ W{\isacharbraceright}{\isachardoublequoteclose}\isanewline
\isanewline
\isacommand{lemma}\isamarkupfalse%
\ set{\isacharunderscore}in{\isacharunderscore}colecComp{\isacharcolon}\ \isanewline
\ \ \isakeyword{assumes}\ {\isachardoublequoteopen}fin{\isacharunderscore}sat\ S{\isachardoublequoteclose}\isanewline
\ \ \isakeyword{shows}\ {\isachardoublequoteopen}S\ {\isasymin}\ colecComp\ S{\isachardoublequoteclose}\isanewline
%
\isadelimproof
\ \ %
\endisadelimproof
%
\isatagproof
\isacommand{unfolding}\isamarkupfalse%
\ colecComp\ \isacommand{using}\isamarkupfalse%
\ assms\ \isacommand{unfolding}\isamarkupfalse%
\ fin{\isacharunderscore}sat{\isacharunderscore}def\ \isacommand{by}\isamarkupfalse%
\ {\isacharparenleft}rule\ CollectI{\isacharparenright}%
\endisatagproof
{\isafoldproof}%
%
\isadelimproof
%
\endisadelimproof
%
\begin{isamarkuptext}%
\comentario{No sé por qué no me permite probar el sorry. El caso de bot está comentado.}%
\end{isamarkuptext}\isamarkuptrue%
\isacommand{lemma}\isamarkupfalse%
\ pcp{\isacharunderscore}colecComp{\isacharunderscore}atoms{\isacharcolon}\isanewline
\ \ \isakeyword{assumes}\ {\isachardoublequoteopen}W\ {\isasymin}\ {\isacharparenleft}colecComp\ S{\isacharparenright}{\isachardoublequoteclose}\isanewline
\ \ \isakeyword{shows}\ {\isachardoublequoteopen}{\isasymforall}k{\isachardot}\ Atom\ k\ {\isasymin}\ W\ {\isasymlongrightarrow}\ \isactrlbold {\isasymnot}\ {\isacharparenleft}Atom\ k{\isacharparenright}\ {\isasymin}\ W\ {\isasymlongrightarrow}\ False{\isachardoublequoteclose}\isanewline
%
\isadelimproof
%
\endisadelimproof
%
\isatagproof
\isacommand{proof}\isamarkupfalse%
\ {\isacharparenleft}rule\ allI{\isacharparenright}\isanewline
\ \ \isacommand{fix}\isamarkupfalse%
\ k\isanewline
\ \ \isacommand{show}\isamarkupfalse%
\ {\isachardoublequoteopen}Atom\ k\ {\isasymin}\ W\ {\isasymlongrightarrow}\ \isactrlbold {\isasymnot}\ {\isacharparenleft}Atom\ k{\isacharparenright}\ {\isasymin}\ W\ {\isasymlongrightarrow}\ False{\isachardoublequoteclose}\isanewline
\ \ \isacommand{proof}\isamarkupfalse%
\ {\isacharparenleft}rule\ impI{\isacharparenright}{\isacharplus}\isanewline
\ \ \ \ \isacommand{assume}\isamarkupfalse%
\ {\isadigit{1}}{\isacharcolon}{\isachardoublequoteopen}Atom\ k\ {\isasymin}\ W{\isachardoublequoteclose}\isanewline
\ \ \ \ \isacommand{assume}\isamarkupfalse%
\ {\isadigit{2}}{\isacharcolon}{\isachardoublequoteopen}\isactrlbold {\isasymnot}\ {\isacharparenleft}Atom\ k{\isacharparenright}\ {\isasymin}\ W{\isachardoublequoteclose}\isanewline
\ \ \ \ \isacommand{have}\isamarkupfalse%
\ {\isachardoublequoteopen}{\isacharbraceleft}Atom\ k{\isacharcomma}\ \isactrlbold {\isasymnot}{\isacharparenleft}Atom\ k{\isacharparenright}{\isacharbraceright}\ {\isasymsubseteq}\ W{\isachardoublequoteclose}\isanewline
\ \ \ \ \ \ \isacommand{using}\isamarkupfalse%
\ {\isadigit{1}}\ {\isadigit{2}}\ \isacommand{by}\isamarkupfalse%
\ blast\ \isanewline
\ \ \ \ \isacommand{have}\isamarkupfalse%
\ {\isachardoublequoteopen}finite\ {\isacharbraceleft}Atom\ k{\isacharcomma}\ \isactrlbold {\isasymnot}{\isacharparenleft}Atom\ k{\isacharparenright}{\isacharbraceright}{\isachardoublequoteclose}\isanewline
\ \ \ \ \ \ \isacommand{by}\isamarkupfalse%
\ blast\ \isanewline
\ \ \ \ \isacommand{have}\isamarkupfalse%
\ {\isachardoublequoteopen}{\isasymforall}S{\isacharprime}\ {\isasymsubseteq}\ W{\isachardot}\ finite\ S{\isacharprime}\ {\isasymlongrightarrow}\ sat\ S{\isacharprime}{\isachardoublequoteclose}\isanewline
\ \ \ \ \ \ \isacommand{using}\isamarkupfalse%
\ assms\ \isacommand{unfolding}\isamarkupfalse%
\ colecComp\ fin{\isacharunderscore}sat{\isacharunderscore}def\ \isacommand{by}\isamarkupfalse%
\ {\isacharparenleft}rule\ CollectD{\isacharparenright}\isanewline
\ \ \ \ \isacommand{then}\isamarkupfalse%
\ \isacommand{have}\isamarkupfalse%
\ {\isachardoublequoteopen}finite\ {\isacharbraceleft}Atom\ k{\isacharcomma}\ \isactrlbold {\isasymnot}{\isacharparenleft}Atom\ k{\isacharparenright}{\isacharbraceright}\ {\isasymlongrightarrow}\ sat\ {\isacharbraceleft}Atom\ k{\isacharcomma}\ \isactrlbold {\isasymnot}{\isacharparenleft}Atom\ k{\isacharparenright}{\isacharbraceright}{\isachardoublequoteclose}\isanewline
\ \ \ \ \ \ \isacommand{using}\isamarkupfalse%
\ {\isacartoucheopen}{\isacharbraceleft}Atom\ k{\isacharcomma}\ \isactrlbold {\isasymnot}{\isacharparenleft}Atom\ k{\isacharparenright}{\isacharbraceright}\ {\isasymsubseteq}\ W{\isacartoucheclose}\ \isacommand{by}\isamarkupfalse%
\ {\isacharparenleft}rule\ sspec{\isacharparenright}\isanewline
\ \ \ \ \isacommand{then}\isamarkupfalse%
\ \isacommand{have}\isamarkupfalse%
\ {\isachardoublequoteopen}sat\ {\isacharbraceleft}Atom\ k{\isacharcomma}\ \isactrlbold {\isasymnot}{\isacharparenleft}Atom\ k{\isacharparenright}{\isacharbraceright}{\isachardoublequoteclose}\isanewline
\ \ \ \ \ \ \isacommand{using}\isamarkupfalse%
\ {\isacartoucheopen}finite\ {\isacharbraceleft}Atom\ k{\isacharcomma}\ \isactrlbold {\isasymnot}{\isacharparenleft}Atom\ k{\isacharparenright}{\isacharbraceright}{\isacartoucheclose}\ \isacommand{by}\isamarkupfalse%
\ {\isacharparenleft}rule\ mp{\isacharparenright}\isanewline
\ \ \ \ \isacommand{then}\isamarkupfalse%
\ \isacommand{have}\isamarkupfalse%
\ Sat{\isacharcolon}{\isachardoublequoteopen}{\isasymexists}{\isasymA}{\isachardot}\ {\isasymforall}F\ {\isasymin}\ {\isacharbraceleft}Atom\ k{\isacharcomma}\ \isactrlbold {\isasymnot}{\isacharparenleft}Atom\ k{\isacharparenright}{\isacharbraceright}{\isachardot}\ {\isasymA}\ {\isasymTurnstile}\ F{\isachardoublequoteclose}\isanewline
\ \ \ \ \ \ \isacommand{by}\isamarkupfalse%
\ {\isacharparenleft}simp\ add{\isacharcolon}\ sat{\isacharunderscore}def{\isacharparenright}\ \isanewline
\ \ \ \ \isacommand{obtain}\isamarkupfalse%
\ {\isasymA}\ \isakeyword{where}\ H{\isacharcolon}{\isachardoublequoteopen}{\isasymforall}F\ {\isasymin}\ {\isacharbraceleft}Atom\ k{\isacharcomma}\ \isactrlbold {\isasymnot}{\isacharparenleft}Atom\ k{\isacharparenright}{\isacharbraceright}{\isachardot}\ {\isasymA}\ {\isasymTurnstile}\ F{\isachardoublequoteclose}\isanewline
\ \ \ \ \ \ \isacommand{using}\isamarkupfalse%
\ Sat\ \isacommand{by}\isamarkupfalse%
\ {\isacharparenleft}rule\ exE{\isacharparenright}\isanewline
\ \ \ \ \isacommand{have}\isamarkupfalse%
\ {\isachardoublequoteopen}Atom\ k\ {\isasymin}\ {\isacharbraceleft}Atom\ k{\isacharcomma}\ \isactrlbold {\isasymnot}{\isacharparenleft}Atom\ k{\isacharparenright}{\isacharbraceright}{\isachardoublequoteclose}\isanewline
\ \ \ \ \ \ \isacommand{by}\isamarkupfalse%
\ blast\ \isanewline
\ \ \ \ \isacommand{have}\isamarkupfalse%
\ {\isachardoublequoteopen}{\isasymA}\ {\isasymTurnstile}\ Atom\ k{\isachardoublequoteclose}\isanewline
\ \ \ \ \ \ \isacommand{using}\isamarkupfalse%
\ H\ {\isacartoucheopen}Atom\ k\ {\isasymin}\ {\isacharbraceleft}Atom\ k{\isacharcomma}\ \isactrlbold {\isasymnot}{\isacharparenleft}Atom\ k{\isacharparenright}{\isacharbraceright}{\isacartoucheclose}\ \isacommand{by}\isamarkupfalse%
\ {\isacharparenleft}rule\ bspec{\isacharparenright}\isanewline
\ \ \ \ \isacommand{have}\isamarkupfalse%
\ {\isachardoublequoteopen}\isactrlbold {\isasymnot}{\isacharparenleft}Atom\ k{\isacharparenright}\ {\isasymin}\ {\isacharbraceleft}Atom\ k{\isacharcomma}\ \isactrlbold {\isasymnot}{\isacharparenleft}Atom\ k{\isacharparenright}{\isacharbraceright}{\isachardoublequoteclose}\isanewline
\ \ \ \ \ \ \isacommand{by}\isamarkupfalse%
\ blast\ \isanewline
\ \ \ \ \isacommand{have}\isamarkupfalse%
\ {\isachardoublequoteopen}{\isasymA}\ {\isasymTurnstile}\ \isactrlbold {\isasymnot}{\isacharparenleft}Atom\ k{\isacharparenright}{\isachardoublequoteclose}\isanewline
\ \ \ \ \ \ \isacommand{using}\isamarkupfalse%
\ H\ {\isacartoucheopen}\isactrlbold {\isasymnot}{\isacharparenleft}Atom\ k{\isacharparenright}\ {\isasymin}\ {\isacharbraceleft}Atom\ k{\isacharcomma}\ \isactrlbold {\isasymnot}{\isacharparenleft}Atom\ k{\isacharparenright}{\isacharbraceright}{\isacartoucheclose}\ \isacommand{by}\isamarkupfalse%
\ {\isacharparenleft}rule\ bspec{\isacharparenright}\isanewline
\ \ \ \ \isacommand{then}\isamarkupfalse%
\ \isacommand{have}\isamarkupfalse%
\ {\isachardoublequoteopen}{\isasymnot}\ {\isasymA}\ {\isasymTurnstile}\ Atom\ k{\isachardoublequoteclose}\isanewline
\ \ \ \ \ \ \isacommand{by}\isamarkupfalse%
\ {\isacharparenleft}simp\ add{\isacharcolon}\ formula{\isacharunderscore}semantics{\isachardot}simps{\isacharparenleft}{\isadigit{3}}{\isacharparenright}{\isacharparenright}\ \isanewline
\ \ \ \ \isacommand{thus}\isamarkupfalse%
\ {\isachardoublequoteopen}False{\isachardoublequoteclose}\isanewline
\ \ \ \ \ \ \isacommand{using}\isamarkupfalse%
\ {\isacartoucheopen}{\isasymA}\ {\isasymTurnstile}\ Atom\ k{\isacartoucheclose}\ \isacommand{by}\isamarkupfalse%
\ {\isacharparenleft}rule\ notE{\isacharparenright}\isanewline
\ \ \isacommand{qed}\isamarkupfalse%
\isanewline
\isacommand{qed}\isamarkupfalse%
%
\endisatagproof
{\isafoldproof}%
%
\isadelimproof
\isanewline
%
\endisadelimproof
\isanewline
\isacommand{lemma}\isamarkupfalse%
\ finite{\isacharunderscore}subset{\isacharunderscore}insert{\isadigit{1}}{\isacharcolon}\isanewline
\ \ \isakeyword{assumes}\ {\isachardoublequoteopen}finite\ S{\isacharprime}{\isachardoublequoteclose}\isanewline
\ \ \ \ \ \ \ \ \ \ {\isachardoublequoteopen}S{\isacharprime}\ {\isasymsubseteq}\ {\isacharbraceleft}a{\isacharbraceright}\ {\isasymunion}\ B{\isachardoublequoteclose}\isanewline
\ \ \ \ \ \ \ \ \isakeyword{shows}\ {\isachardoublequoteopen}{\isasymexists}Wo\ {\isasymsubseteq}\ B{\isachardot}\ finite\ Wo\ {\isasymand}\ {\isacharparenleft}S{\isacharprime}\ {\isacharequal}\ {\isacharbraceleft}a{\isacharbraceright}\ {\isasymunion}\ Wo\ {\isasymor}\ S{\isacharprime}\ {\isacharequal}\ Wo{\isacharparenright}{\isachardoublequoteclose}\isanewline
%
\isadelimproof
\ \ %
\endisadelimproof
%
\isatagproof
\isacommand{by}\isamarkupfalse%
\ {\isacharparenleft}metis\ Diff{\isacharunderscore}empty\ Diff{\isacharunderscore}insert{\isadigit{0}}\ Diff{\isacharunderscore}subset{\isacharunderscore}conv\ Un{\isacharunderscore}Diff{\isacharunderscore}cancel\ assms{\isacharparenleft}{\isadigit{1}}{\isacharparenright}\ assms{\isacharparenleft}{\isadigit{2}}{\isacharparenright}\ finite{\isacharunderscore}Diff\ insert{\isacharunderscore}Diff\ insert{\isacharunderscore}is{\isacharunderscore}Un{\isacharparenright}%
\endisatagproof
{\isafoldproof}%
%
\isadelimproof
\isanewline
%
\endisadelimproof
\isanewline
\isacommand{lemma}\isamarkupfalse%
\ finite{\isacharunderscore}subset{\isacharunderscore}insert{\isadigit{2}}{\isacharcolon}\isanewline
\ \ \isakeyword{assumes}\ {\isachardoublequoteopen}finite\ S{\isacharprime}{\isachardoublequoteclose}\isanewline
\ \ \ \ \ \ \ \ \ \ {\isachardoublequoteopen}S{\isacharprime}\ {\isasymsubseteq}\ {\isacharbraceleft}a{\isacharcomma}b{\isacharbraceright}\ {\isasymunion}\ B{\isachardoublequoteclose}\isanewline
\ \ \ \ \ \ \ \ \isakeyword{shows}\ {\isachardoublequoteopen}{\isasymexists}Wo\ {\isasymsubseteq}\ B{\isachardot}\ finite\ Wo\ {\isasymand}\ {\isacharparenleft}S{\isacharprime}\ {\isacharequal}\ {\isacharbraceleft}a{\isacharcomma}b{\isacharbraceright}\ {\isasymunion}\ Wo\ {\isasymor}\ S{\isacharprime}\ {\isacharequal}\ {\isacharbraceleft}a{\isacharbraceright}\ {\isasymunion}\ Wo\ {\isasymor}\ S{\isacharprime}\ {\isacharequal}\ {\isacharbraceleft}b{\isacharbraceright}\ {\isasymunion}\ Wo\ {\isasymor}\ S{\isacharprime}\ {\isacharequal}\ Wo{\isacharparenright}{\isachardoublequoteclose}\isanewline
%
\isadelimproof
%
\endisadelimproof
%
\isatagproof
\isacommand{proof}\isamarkupfalse%
\ {\isacharminus}\isanewline
\ \ \isacommand{have}\isamarkupfalse%
\ {\isachardoublequoteopen}S{\isacharprime}\ {\isasymsubseteq}\ {\isacharbraceleft}a{\isacharbraceright}\ {\isasymunion}\ {\isacharparenleft}{\isacharbraceleft}b{\isacharbraceright}\ {\isasymunion}\ B{\isacharparenright}{\isachardoublequoteclose}\isanewline
\ \ \ \ \isacommand{using}\isamarkupfalse%
\ assms{\isacharparenleft}{\isadigit{2}}{\isacharparenright}\ \isacommand{by}\isamarkupfalse%
\ blast\ \isanewline
\ \ \isacommand{then}\isamarkupfalse%
\ \isacommand{have}\isamarkupfalse%
\ Ex{\isadigit{1}}{\isacharcolon}{\isachardoublequoteopen}{\isasymexists}Wo\ {\isasymsubseteq}\ {\isacharparenleft}{\isacharbraceleft}b{\isacharbraceright}\ {\isasymunion}\ B{\isacharparenright}{\isachardot}\ finite\ Wo\ {\isasymand}\ {\isacharparenleft}S{\isacharprime}\ {\isacharequal}\ {\isacharbraceleft}a{\isacharbraceright}\ {\isasymunion}\ Wo\ {\isasymor}\ S{\isacharprime}\ {\isacharequal}\ Wo{\isacharparenright}{\isachardoublequoteclose}\isanewline
\ \ \ \ \isacommand{by}\isamarkupfalse%
\ {\isacharparenleft}meson\ assms{\isacharparenleft}{\isadigit{1}}{\isacharparenright}\ finite{\isacharunderscore}subset{\isacharunderscore}insert{\isadigit{1}}{\isacharparenright}\ \isanewline
\ \ \isacommand{obtain}\isamarkupfalse%
\ Wo\ \isakeyword{where}\ {\isachardoublequoteopen}Wo\ {\isasymsubseteq}\ {\isacharbraceleft}b{\isacharbraceright}\ {\isasymunion}\ B{\isachardoublequoteclose}\ \isakeyword{and}\ {\isadigit{1}}{\isacharcolon}{\isachardoublequoteopen}finite\ Wo\ {\isasymand}\ {\isacharparenleft}S{\isacharprime}\ {\isacharequal}\ {\isacharbraceleft}a{\isacharbraceright}\ {\isasymunion}\ Wo\ {\isasymor}\ S{\isacharprime}\ {\isacharequal}\ Wo{\isacharparenright}{\isachardoublequoteclose}\isanewline
\ \ \ \ \isacommand{using}\isamarkupfalse%
\ Ex{\isadigit{1}}\ \isacommand{by}\isamarkupfalse%
\ blast\ \isanewline
\ \ \isacommand{have}\isamarkupfalse%
\ {\isachardoublequoteopen}finite\ Wo{\isachardoublequoteclose}\isanewline
\ \ \ \ \isacommand{using}\isamarkupfalse%
\ {\isadigit{1}}\ \isacommand{by}\isamarkupfalse%
\ {\isacharparenleft}rule\ conjunct{\isadigit{1}}{\isacharparenright}\isanewline
\ \ \isacommand{have}\isamarkupfalse%
\ Ex{\isadigit{2}}{\isacharcolon}{\isachardoublequoteopen}{\isasymexists}Wo{\isacharprime}\ {\isasymsubseteq}\ B{\isachardot}\ finite\ Wo{\isacharprime}\ {\isasymand}\ {\isacharparenleft}Wo\ {\isacharequal}\ {\isacharbraceleft}b{\isacharbraceright}\ {\isasymunion}\ Wo{\isacharprime}\ {\isasymor}\ Wo\ {\isacharequal}\ Wo{\isacharprime}{\isacharparenright}{\isachardoublequoteclose}\isanewline
\ \ \ \ \isacommand{using}\isamarkupfalse%
\ {\isacartoucheopen}finite\ Wo{\isacartoucheclose}\ {\isacartoucheopen}Wo\ {\isasymsubseteq}\ {\isacharbraceleft}b{\isacharbraceright}\ {\isasymunion}\ B{\isacartoucheclose}\ \isacommand{by}\isamarkupfalse%
\ {\isacharparenleft}meson\ finite{\isacharunderscore}subset{\isacharunderscore}insert{\isadigit{1}}{\isacharparenright}\ \isanewline
\ \ \isacommand{obtain}\isamarkupfalse%
\ Wo{\isacharprime}\ \isakeyword{where}\ {\isachardoublequoteopen}Wo{\isacharprime}\ {\isasymsubseteq}\ B{\isachardoublequoteclose}\ \isakeyword{and}\ {\isadigit{2}}{\isacharcolon}{\isachardoublequoteopen}finite\ Wo{\isacharprime}\ {\isasymand}\ {\isacharparenleft}Wo\ {\isacharequal}\ {\isacharbraceleft}b{\isacharbraceright}\ {\isasymunion}\ Wo{\isacharprime}\ {\isasymor}\ Wo\ {\isacharequal}\ Wo{\isacharprime}{\isacharparenright}{\isachardoublequoteclose}\isanewline
\ \ \ \ \isacommand{using}\isamarkupfalse%
\ Ex{\isadigit{2}}\ \isacommand{by}\isamarkupfalse%
\ blast\ \isanewline
\ \ \isacommand{have}\isamarkupfalse%
\ {\isachardoublequoteopen}finite\ Wo{\isacharprime}{\isachardoublequoteclose}\isanewline
\ \ \ \ \isacommand{using}\isamarkupfalse%
\ {\isadigit{2}}\ \isacommand{by}\isamarkupfalse%
\ {\isacharparenleft}rule\ conjunct{\isadigit{1}}{\isacharparenright}\isanewline
\ \ \isacommand{have}\isamarkupfalse%
\ {\isachardoublequoteopen}Wo\ {\isacharequal}\ {\isacharbraceleft}b{\isacharbraceright}\ {\isasymunion}\ Wo{\isacharprime}\ {\isasymor}\ Wo\ {\isacharequal}\ Wo{\isacharprime}{\isachardoublequoteclose}\isanewline
\ \ \ \ \isacommand{using}\isamarkupfalse%
\ {\isadigit{2}}\ \isacommand{by}\isamarkupfalse%
\ {\isacharparenleft}rule\ conjunct{\isadigit{2}}{\isacharparenright}\isanewline
\ \ \isacommand{thus}\isamarkupfalse%
\ {\isachardoublequoteopen}{\isasymexists}Wo\ {\isasymsubseteq}\ B{\isachardot}\ finite\ Wo\ {\isasymand}\ {\isacharparenleft}S{\isacharprime}\ {\isacharequal}\ {\isacharbraceleft}a{\isacharcomma}b{\isacharbraceright}\ {\isasymunion}\ Wo\ {\isasymor}\ S{\isacharprime}\ {\isacharequal}\ {\isacharbraceleft}a{\isacharbraceright}\ {\isasymunion}\ Wo\ {\isasymor}\ S{\isacharprime}\ {\isacharequal}\ {\isacharbraceleft}b{\isacharbraceright}\ {\isasymunion}\ Wo\ {\isasymor}\ S{\isacharprime}\ {\isacharequal}\ Wo{\isacharparenright}{\isachardoublequoteclose}\isanewline
\ \ \isacommand{proof}\isamarkupfalse%
\ {\isacharparenleft}rule\ disjE{\isacharparenright}\isanewline
\ \ \ \ \isacommand{assume}\isamarkupfalse%
\ {\isachardoublequoteopen}Wo\ {\isacharequal}\ {\isacharbraceleft}b{\isacharbraceright}\ {\isasymunion}\ Wo{\isacharprime}{\isachardoublequoteclose}\isanewline
\ \ \ \ \isacommand{have}\isamarkupfalse%
\ {\isachardoublequoteopen}S{\isacharprime}\ {\isacharequal}\ {\isacharbraceleft}a{\isacharbraceright}\ {\isasymunion}\ Wo\ {\isasymor}\ S{\isacharprime}\ {\isacharequal}\ Wo{\isachardoublequoteclose}\isanewline
\ \ \ \ \ \ \isacommand{using}\isamarkupfalse%
\ {\isadigit{1}}\ \isacommand{by}\isamarkupfalse%
\ {\isacharparenleft}rule\ conjunct{\isadigit{2}}{\isacharparenright}\isanewline
\ \ \ \ \isacommand{thus}\isamarkupfalse%
\ {\isachardoublequoteopen}{\isasymexists}Wo\ {\isasymsubseteq}\ B{\isachardot}\ finite\ Wo\ {\isasymand}\ {\isacharparenleft}S{\isacharprime}\ {\isacharequal}\ {\isacharbraceleft}a{\isacharcomma}b{\isacharbraceright}\ {\isasymunion}\ Wo\ {\isasymor}\ S{\isacharprime}\ {\isacharequal}\ {\isacharbraceleft}a{\isacharbraceright}\ {\isasymunion}\ Wo\ {\isasymor}\ S{\isacharprime}\ {\isacharequal}\ {\isacharbraceleft}b{\isacharbraceright}\ {\isasymunion}\ Wo\ {\isasymor}\ S{\isacharprime}\ {\isacharequal}\ Wo{\isacharparenright}{\isachardoublequoteclose}\isanewline
\ \ \ \ \isacommand{proof}\isamarkupfalse%
\ {\isacharparenleft}rule\ disjE{\isacharparenright}\isanewline
\ \ \ \ \ \ \isacommand{assume}\isamarkupfalse%
\ {\isachardoublequoteopen}S{\isacharprime}\ {\isacharequal}\ {\isacharbraceleft}a{\isacharbraceright}\ {\isasymunion}\ Wo{\isachardoublequoteclose}\isanewline
\ \ \ \ \ \ \isacommand{then}\isamarkupfalse%
\ \isacommand{have}\isamarkupfalse%
\ {\isachardoublequoteopen}S{\isacharprime}\ {\isacharequal}\ {\isacharbraceleft}a{\isacharbraceright}\ {\isasymunion}\ {\isacharbraceleft}b{\isacharbraceright}\ {\isasymunion}\ Wo{\isacharprime}{\isachardoublequoteclose}\isanewline
\ \ \ \ \ \ \ \ \isacommand{by}\isamarkupfalse%
\ {\isacharparenleft}simp\ add{\isacharcolon}\ {\isacartoucheopen}Wo\ {\isacharequal}\ {\isacharbraceleft}b{\isacharbraceright}\ {\isasymunion}\ Wo{\isacharprime}{\isacartoucheclose}{\isacharparenright}\ \isanewline
\ \ \ \ \ \ \isacommand{then}\isamarkupfalse%
\ \isacommand{have}\isamarkupfalse%
\ {\isachardoublequoteopen}S{\isacharprime}\ {\isacharequal}\ {\isacharbraceleft}a{\isacharcomma}b{\isacharbraceright}\ {\isasymunion}\ Wo{\isacharprime}{\isachardoublequoteclose}\isanewline
\ \ \ \ \ \ \ \ \isacommand{by}\isamarkupfalse%
\ blast\ \isanewline
\ \ \ \ \ \ \isacommand{then}\isamarkupfalse%
\ \isacommand{have}\isamarkupfalse%
\ {\isachardoublequoteopen}S{\isacharprime}\ {\isacharequal}\ {\isacharbraceleft}a{\isacharcomma}b{\isacharbraceright}\ {\isasymunion}\ Wo{\isacharprime}\ {\isasymor}\ S{\isacharprime}\ {\isacharequal}\ {\isacharbraceleft}a{\isacharbraceright}\ {\isasymunion}\ Wo{\isacharprime}\ {\isasymor}\ S{\isacharprime}\ {\isacharequal}\ {\isacharbraceleft}b{\isacharbraceright}\ {\isasymunion}\ Wo{\isacharprime}\ {\isasymor}\ S{\isacharprime}\ {\isacharequal}\ Wo{\isacharprime}{\isachardoublequoteclose}\isanewline
\ \ \ \ \ \ \ \ \isacommand{by}\isamarkupfalse%
\ {\isacharparenleft}iprover\ intro{\isacharcolon}\ disjI{\isadigit{1}}{\isacharparenright}\isanewline
\ \ \ \ \ \ \isacommand{then}\isamarkupfalse%
\ \isacommand{have}\isamarkupfalse%
\ {\isachardoublequoteopen}finite\ Wo{\isacharprime}\ {\isasymand}\ {\isacharparenleft}S{\isacharprime}\ {\isacharequal}\ {\isacharbraceleft}a{\isacharcomma}b{\isacharbraceright}\ {\isasymunion}\ Wo{\isacharprime}\ {\isasymor}\ S{\isacharprime}\ {\isacharequal}\ {\isacharbraceleft}a{\isacharbraceright}\ {\isasymunion}\ Wo{\isacharprime}\ {\isasymor}\ S{\isacharprime}\ {\isacharequal}\ {\isacharbraceleft}b{\isacharbraceright}\ {\isasymunion}\ Wo{\isacharprime}\ {\isasymor}\ S{\isacharprime}\ {\isacharequal}\ Wo{\isacharprime}{\isacharparenright}{\isachardoublequoteclose}\isanewline
\ \ \ \ \ \ \ \ \isacommand{using}\isamarkupfalse%
\ {\isacartoucheopen}finite\ Wo{\isacharprime}{\isacartoucheclose}\ \isacommand{by}\isamarkupfalse%
\ {\isacharparenleft}iprover\ intro{\isacharcolon}\ conjI{\isacharparenright}\isanewline
\ \ \ \ \ \ \isacommand{thus}\isamarkupfalse%
\ {\isachardoublequoteopen}{\isasymexists}Wo\ {\isasymsubseteq}\ B{\isachardot}\ finite\ Wo\ {\isasymand}\ {\isacharparenleft}S{\isacharprime}\ {\isacharequal}\ {\isacharbraceleft}a{\isacharcomma}b{\isacharbraceright}\ {\isasymunion}\ Wo\ {\isasymor}\ S{\isacharprime}\ {\isacharequal}\ {\isacharbraceleft}a{\isacharbraceright}\ {\isasymunion}\ Wo\ {\isasymor}\ S{\isacharprime}\ {\isacharequal}\ {\isacharbraceleft}b{\isacharbraceright}\ {\isasymunion}\ Wo\ {\isasymor}\ S{\isacharprime}\ {\isacharequal}\ Wo{\isacharparenright}{\isachardoublequoteclose}\isanewline
\ \ \ \ \ \ \ \ \isacommand{using}\isamarkupfalse%
\ {\isacartoucheopen}Wo{\isacharprime}\ {\isasymsubseteq}\ B{\isacartoucheclose}\ \isacommand{by}\isamarkupfalse%
\ blast\ \isanewline
\ \ \ \ \isacommand{next}\isamarkupfalse%
\isanewline
\ \ \ \ \ \ \isacommand{assume}\isamarkupfalse%
\ {\isachardoublequoteopen}S{\isacharprime}\ {\isacharequal}\ Wo{\isachardoublequoteclose}\isanewline
\ \ \ \ \ \ \isacommand{then}\isamarkupfalse%
\ \isacommand{have}\isamarkupfalse%
\ {\isachardoublequoteopen}S{\isacharprime}\ {\isacharequal}\ {\isacharbraceleft}b{\isacharbraceright}\ {\isasymunion}\ Wo{\isacharprime}{\isachardoublequoteclose}\isanewline
\ \ \ \ \ \ \ \ \isacommand{by}\isamarkupfalse%
\ {\isacharparenleft}simp\ add{\isacharcolon}\ {\isacartoucheopen}Wo\ {\isacharequal}\ {\isacharbraceleft}b{\isacharbraceright}\ {\isasymunion}\ Wo{\isacharprime}{\isacartoucheclose}{\isacharparenright}\isanewline
\ \ \ \ \ \ \isacommand{then}\isamarkupfalse%
\ \isacommand{have}\isamarkupfalse%
\ {\isachardoublequoteopen}S{\isacharprime}\ {\isacharequal}\ {\isacharbraceleft}a{\isacharcomma}b{\isacharbraceright}\ {\isasymunion}\ Wo{\isacharprime}\ {\isasymor}\ S{\isacharprime}\ {\isacharequal}\ {\isacharbraceleft}a{\isacharbraceright}\ {\isasymunion}\ Wo{\isacharprime}\ {\isasymor}\ S{\isacharprime}\ {\isacharequal}\ {\isacharbraceleft}b{\isacharbraceright}\ {\isasymunion}\ Wo{\isacharprime}\ {\isasymor}\ S{\isacharprime}\ {\isacharequal}\ Wo{\isacharprime}{\isachardoublequoteclose}\isanewline
\ \ \ \ \ \ \ \ \isacommand{by}\isamarkupfalse%
\ {\isacharparenleft}iprover\ intro{\isacharcolon}\ disjI{\isadigit{1}}{\isacharparenright}\isanewline
\ \ \ \ \ \ \isacommand{then}\isamarkupfalse%
\ \isacommand{have}\isamarkupfalse%
\ {\isachardoublequoteopen}finite\ Wo{\isacharprime}\ {\isasymand}\ {\isacharparenleft}S{\isacharprime}\ {\isacharequal}\ {\isacharbraceleft}a{\isacharcomma}b{\isacharbraceright}\ {\isasymunion}\ Wo{\isacharprime}\ {\isasymor}\ S{\isacharprime}\ {\isacharequal}\ {\isacharbraceleft}a{\isacharbraceright}\ {\isasymunion}\ Wo{\isacharprime}\ {\isasymor}\ S{\isacharprime}\ {\isacharequal}\ {\isacharbraceleft}b{\isacharbraceright}\ {\isasymunion}\ Wo{\isacharprime}\ {\isasymor}\ S{\isacharprime}\ {\isacharequal}\ Wo{\isacharprime}{\isacharparenright}{\isachardoublequoteclose}\isanewline
\ \ \ \ \ \ \ \ \isacommand{using}\isamarkupfalse%
\ {\isacartoucheopen}finite\ Wo{\isacharprime}{\isacartoucheclose}\ \isacommand{by}\isamarkupfalse%
\ {\isacharparenleft}iprover\ intro{\isacharcolon}\ conjI{\isacharparenright}\isanewline
\ \ \ \ \ \ \isacommand{thus}\isamarkupfalse%
\ {\isachardoublequoteopen}{\isasymexists}Wo\ {\isasymsubseteq}\ B{\isachardot}\ finite\ Wo\ {\isasymand}\ {\isacharparenleft}S{\isacharprime}\ {\isacharequal}\ {\isacharbraceleft}a{\isacharcomma}b{\isacharbraceright}\ {\isasymunion}\ Wo\ {\isasymor}\ S{\isacharprime}\ {\isacharequal}\ {\isacharbraceleft}a{\isacharbraceright}\ {\isasymunion}\ Wo\ {\isasymor}\ S{\isacharprime}\ {\isacharequal}\ {\isacharbraceleft}b{\isacharbraceright}\ {\isasymunion}\ Wo\ {\isasymor}\ S{\isacharprime}\ {\isacharequal}\ Wo{\isacharparenright}{\isachardoublequoteclose}\isanewline
\ \ \ \ \ \ \ \ \isacommand{using}\isamarkupfalse%
\ {\isacartoucheopen}Wo{\isacharprime}\ {\isasymsubseteq}\ B{\isacartoucheclose}\ \isacommand{by}\isamarkupfalse%
\ blast\ \isanewline
\ \ \ \ \isacommand{qed}\isamarkupfalse%
\isanewline
\ \ \isacommand{next}\isamarkupfalse%
\isanewline
\ \ \ \ \isacommand{assume}\isamarkupfalse%
\ {\isachardoublequoteopen}Wo\ {\isacharequal}\ Wo{\isacharprime}{\isachardoublequoteclose}\isanewline
\ \ \ \ \isacommand{have}\isamarkupfalse%
\ {\isachardoublequoteopen}S{\isacharprime}\ {\isacharequal}\ {\isacharbraceleft}a{\isacharbraceright}\ {\isasymunion}\ Wo\ {\isasymor}\ S{\isacharprime}\ {\isacharequal}\ Wo{\isachardoublequoteclose}\isanewline
\ \ \ \ \ \ \isacommand{using}\isamarkupfalse%
\ {\isadigit{1}}\ \isacommand{by}\isamarkupfalse%
\ {\isacharparenleft}rule\ conjunct{\isadigit{2}}{\isacharparenright}\isanewline
\ \ \ \ \isacommand{thus}\isamarkupfalse%
\ {\isachardoublequoteopen}{\isasymexists}Wo\ {\isasymsubseteq}\ B{\isachardot}\ finite\ Wo\ {\isasymand}\ {\isacharparenleft}S{\isacharprime}\ {\isacharequal}\ {\isacharbraceleft}a{\isacharcomma}b{\isacharbraceright}\ {\isasymunion}\ Wo\ {\isasymor}\ S{\isacharprime}\ {\isacharequal}\ {\isacharbraceleft}a{\isacharbraceright}\ {\isasymunion}\ Wo\ {\isasymor}\ S{\isacharprime}\ {\isacharequal}\ {\isacharbraceleft}b{\isacharbraceright}\ {\isasymunion}\ Wo\ {\isasymor}\ S{\isacharprime}\ {\isacharequal}\ Wo{\isacharparenright}{\isachardoublequoteclose}\isanewline
\ \ \ \ \isacommand{proof}\isamarkupfalse%
\ {\isacharparenleft}rule\ disjE{\isacharparenright}\isanewline
\ \ \ \ \ \ \isacommand{assume}\isamarkupfalse%
\ {\isachardoublequoteopen}S{\isacharprime}\ {\isacharequal}\ {\isacharbraceleft}a{\isacharbraceright}\ {\isasymunion}\ Wo{\isachardoublequoteclose}\isanewline
\ \ \ \ \ \ \isacommand{then}\isamarkupfalse%
\ \isacommand{have}\isamarkupfalse%
\ {\isachardoublequoteopen}S{\isacharprime}\ {\isacharequal}\ {\isacharbraceleft}a{\isacharbraceright}\ {\isasymunion}\ Wo{\isacharprime}{\isachardoublequoteclose}\isanewline
\ \ \ \ \ \ \ \ \isacommand{by}\isamarkupfalse%
\ {\isacharparenleft}simp\ add{\isacharcolon}\ {\isacartoucheopen}Wo\ {\isacharequal}\ Wo{\isacharprime}{\isacartoucheclose}{\isacharparenright}\ \isanewline
\ \ \ \ \ \ \isacommand{then}\isamarkupfalse%
\ \isacommand{have}\isamarkupfalse%
\ {\isachardoublequoteopen}S{\isacharprime}\ {\isacharequal}\ {\isacharbraceleft}a{\isacharcomma}b{\isacharbraceright}\ {\isasymunion}\ Wo{\isacharprime}\ {\isasymor}\ S{\isacharprime}\ {\isacharequal}\ {\isacharbraceleft}a{\isacharbraceright}\ {\isasymunion}\ Wo{\isacharprime}\ {\isasymor}\ S{\isacharprime}\ {\isacharequal}\ {\isacharbraceleft}b{\isacharbraceright}\ {\isasymunion}\ Wo{\isacharprime}\ {\isasymor}\ S{\isacharprime}\ {\isacharequal}\ Wo{\isacharprime}{\isachardoublequoteclose}\isanewline
\ \ \ \ \ \ \ \ \isacommand{by}\isamarkupfalse%
\ {\isacharparenleft}iprover\ intro{\isacharcolon}\ disjI{\isadigit{1}}{\isacharparenright}\isanewline
\ \ \ \ \ \ \isacommand{then}\isamarkupfalse%
\ \isacommand{have}\isamarkupfalse%
\ {\isachardoublequoteopen}finite\ Wo{\isacharprime}\ {\isasymand}\ {\isacharparenleft}S{\isacharprime}\ {\isacharequal}\ {\isacharbraceleft}a{\isacharcomma}b{\isacharbraceright}\ {\isasymunion}\ Wo{\isacharprime}\ {\isasymor}\ S{\isacharprime}\ {\isacharequal}\ {\isacharbraceleft}a{\isacharbraceright}\ {\isasymunion}\ Wo{\isacharprime}\ {\isasymor}\ S{\isacharprime}\ {\isacharequal}\ {\isacharbraceleft}b{\isacharbraceright}\ {\isasymunion}\ Wo{\isacharprime}\ {\isasymor}\ S{\isacharprime}\ {\isacharequal}\ Wo{\isacharprime}{\isacharparenright}{\isachardoublequoteclose}\isanewline
\ \ \ \ \ \ \ \ \isacommand{using}\isamarkupfalse%
\ {\isacartoucheopen}finite\ Wo{\isacharprime}{\isacartoucheclose}\ \isacommand{by}\isamarkupfalse%
\ {\isacharparenleft}iprover\ intro{\isacharcolon}\ conjI{\isacharparenright}\isanewline
\ \ \ \ \ \ \isacommand{thus}\isamarkupfalse%
\ {\isachardoublequoteopen}{\isasymexists}Wo\ {\isasymsubseteq}\ B{\isachardot}\ finite\ Wo\ {\isasymand}\ {\isacharparenleft}S{\isacharprime}\ {\isacharequal}\ {\isacharbraceleft}a{\isacharcomma}b{\isacharbraceright}\ {\isasymunion}\ Wo\ {\isasymor}\ S{\isacharprime}\ {\isacharequal}\ {\isacharbraceleft}a{\isacharbraceright}\ {\isasymunion}\ Wo\ {\isasymor}\ S{\isacharprime}\ {\isacharequal}\ {\isacharbraceleft}b{\isacharbraceright}\ {\isasymunion}\ Wo\ {\isasymor}\ S{\isacharprime}\ {\isacharequal}\ Wo{\isacharparenright}{\isachardoublequoteclose}\isanewline
\ \ \ \ \ \ \ \ \isacommand{using}\isamarkupfalse%
\ {\isacartoucheopen}Wo{\isacharprime}\ {\isasymsubseteq}\ B{\isacartoucheclose}\ \isacommand{by}\isamarkupfalse%
\ blast\ \isanewline
\ \ \ \ \isacommand{next}\isamarkupfalse%
\isanewline
\ \ \ \ \ \ \isacommand{assume}\isamarkupfalse%
\ {\isachardoublequoteopen}S{\isacharprime}\ {\isacharequal}\ Wo{\isachardoublequoteclose}\isanewline
\ \ \ \ \ \ \isacommand{then}\isamarkupfalse%
\ \isacommand{have}\isamarkupfalse%
\ {\isachardoublequoteopen}S{\isacharprime}\ {\isacharequal}\ Wo{\isacharprime}{\isachardoublequoteclose}\isanewline
\ \ \ \ \ \ \ \ \isacommand{by}\isamarkupfalse%
\ {\isacharparenleft}simp\ add{\isacharcolon}\ {\isacartoucheopen}Wo\ {\isacharequal}\ Wo{\isacharprime}{\isacartoucheclose}{\isacharparenright}\isanewline
\ \ \ \ \ \ \isacommand{then}\isamarkupfalse%
\ \isacommand{have}\isamarkupfalse%
\ {\isachardoublequoteopen}S{\isacharprime}\ {\isacharequal}\ {\isacharbraceleft}a{\isacharcomma}b{\isacharbraceright}\ {\isasymunion}\ Wo{\isacharprime}\ {\isasymor}\ S{\isacharprime}\ {\isacharequal}\ {\isacharbraceleft}a{\isacharbraceright}\ {\isasymunion}\ Wo{\isacharprime}\ {\isasymor}\ S{\isacharprime}\ {\isacharequal}\ {\isacharbraceleft}b{\isacharbraceright}\ {\isasymunion}\ Wo{\isacharprime}\ {\isasymor}\ S{\isacharprime}\ {\isacharequal}\ Wo{\isacharprime}{\isachardoublequoteclose}\isanewline
\ \ \ \ \ \ \ \ \isacommand{by}\isamarkupfalse%
\ {\isacharparenleft}iprover\ intro{\isacharcolon}\ disjI{\isadigit{1}}{\isacharparenright}\isanewline
\ \ \ \ \ \ \isacommand{then}\isamarkupfalse%
\ \isacommand{have}\isamarkupfalse%
\ {\isachardoublequoteopen}finite\ Wo{\isacharprime}\ {\isasymand}\ {\isacharparenleft}S{\isacharprime}\ {\isacharequal}\ {\isacharbraceleft}a{\isacharcomma}b{\isacharbraceright}\ {\isasymunion}\ Wo{\isacharprime}\ {\isasymor}\ S{\isacharprime}\ {\isacharequal}\ {\isacharbraceleft}a{\isacharbraceright}\ {\isasymunion}\ Wo{\isacharprime}\ {\isasymor}\ S{\isacharprime}\ {\isacharequal}\ {\isacharbraceleft}b{\isacharbraceright}\ {\isasymunion}\ Wo{\isacharprime}\ {\isasymor}\ S{\isacharprime}\ {\isacharequal}\ Wo{\isacharprime}{\isacharparenright}{\isachardoublequoteclose}\isanewline
\ \ \ \ \ \ \ \ \isacommand{using}\isamarkupfalse%
\ {\isacartoucheopen}finite\ Wo{\isacharprime}{\isacartoucheclose}\ \isacommand{by}\isamarkupfalse%
\ {\isacharparenleft}iprover\ intro{\isacharcolon}\ conjI{\isacharparenright}\isanewline
\ \ \ \ \ \ \isacommand{thus}\isamarkupfalse%
\ {\isachardoublequoteopen}{\isasymexists}Wo\ {\isasymsubseteq}\ B{\isachardot}\ finite\ Wo\ {\isasymand}\ {\isacharparenleft}S{\isacharprime}\ {\isacharequal}\ {\isacharbraceleft}a{\isacharcomma}b{\isacharbraceright}\ {\isasymunion}\ Wo\ {\isasymor}\ S{\isacharprime}\ {\isacharequal}\ {\isacharbraceleft}a{\isacharbraceright}\ {\isasymunion}\ Wo\ {\isasymor}\ S{\isacharprime}\ {\isacharequal}\ {\isacharbraceleft}b{\isacharbraceright}\ {\isasymunion}\ Wo\ {\isasymor}\ S{\isacharprime}\ {\isacharequal}\ Wo{\isacharparenright}{\isachardoublequoteclose}\isanewline
\ \ \ \ \ \ \ \ \isacommand{using}\isamarkupfalse%
\ {\isacartoucheopen}Wo{\isacharprime}\ {\isasymsubseteq}\ B{\isacartoucheclose}\ \isacommand{by}\isamarkupfalse%
\ blast\ \isanewline
\ \ \ \ \isacommand{qed}\isamarkupfalse%
\isanewline
\ \ \isacommand{qed}\isamarkupfalse%
\isanewline
\isacommand{qed}\isamarkupfalse%
%
\endisatagproof
{\isafoldproof}%
%
\isadelimproof
\isanewline
%
\endisadelimproof
\isanewline
\isacommand{lemma}\isamarkupfalse%
\ pcp{\isacharunderscore}colecComp{\isacharunderscore}CON{\isacharunderscore}both{\isacharcolon}\isanewline
\ \ \isakeyword{assumes}\ {\isachardoublequoteopen}W\ {\isasymin}\ {\isacharparenleft}colecComp\ S{\isacharparenright}{\isachardoublequoteclose}\isanewline
\ \ \ \ \ \ \ \ \ \ {\isachardoublequoteopen}Con\ F\ G\ H{\isachardoublequoteclose}\isanewline
\ \ \ \ \ \ \ \ \ \ {\isachardoublequoteopen}F\ {\isasymin}\ W{\isachardoublequoteclose}\isanewline
\ \ \ \ \ \ \ \ \ \ {\isachardoublequoteopen}finite\ Wo{\isachardoublequoteclose}\isanewline
\ \ \ \ \ \ \ \ \ \ {\isachardoublequoteopen}Wo\ {\isasymsubseteq}\ W{\isachardoublequoteclose}\isanewline
\ \ \ \ \ \ \ \ \isakeyword{shows}\ {\isachardoublequoteopen}sat\ {\isacharparenleft}{\isacharbraceleft}G{\isacharcomma}H{\isacharbraceright}\ {\isasymunion}\ Wo{\isacharparenright}{\isachardoublequoteclose}\isanewline
%
\isadelimproof
%
\endisadelimproof
%
\isatagproof
\isacommand{proof}\isamarkupfalse%
\ {\isacharminus}\isanewline
\ \ \isacommand{have}\isamarkupfalse%
\ WcolecComp{\isacharcolon}{\isachardoublequoteopen}{\isasymforall}S{\isacharprime}\ {\isasymsubseteq}\ W{\isachardot}\ finite\ S{\isacharprime}\ {\isasymlongrightarrow}\ sat\ S{\isacharprime}{\isachardoublequoteclose}\isanewline
\ \ \ \ \isacommand{using}\isamarkupfalse%
\ assms{\isacharparenleft}{\isadigit{1}}{\isacharparenright}\ \isacommand{unfolding}\isamarkupfalse%
\ colecComp\ fin{\isacharunderscore}sat{\isacharunderscore}def\ \isacommand{by}\isamarkupfalse%
\ {\isacharparenleft}rule\ CollectD{\isacharparenright}\isanewline
\ \ \isacommand{then}\isamarkupfalse%
\ \isacommand{have}\isamarkupfalse%
\ {\isachardoublequoteopen}finite\ Wo\ {\isasymlongrightarrow}\ sat\ Wo{\isachardoublequoteclose}\isanewline
\ \ \ \ \isacommand{using}\isamarkupfalse%
\ assms{\isacharparenleft}{\isadigit{5}}{\isacharparenright}\ \isacommand{by}\isamarkupfalse%
\ blast\ \isanewline
\ \ \isacommand{then}\isamarkupfalse%
\ \isacommand{have}\isamarkupfalse%
\ {\isachardoublequoteopen}sat\ Wo{\isachardoublequoteclose}\isanewline
\ \ \ \ \isacommand{using}\isamarkupfalse%
\ assms{\isacharparenleft}{\isadigit{4}}{\isacharparenright}\ \isacommand{by}\isamarkupfalse%
\ {\isacharparenleft}rule\ mp{\isacharparenright}\isanewline
\ \ \isacommand{have}\isamarkupfalse%
\ {\isachardoublequoteopen}sat\ {\isacharparenleft}{\isacharbraceleft}F{\isacharbraceright}\ {\isasymunion}\ Wo{\isacharparenright}{\isachardoublequoteclose}\isanewline
\ \ \isacommand{proof}\isamarkupfalse%
\ {\isacharminus}\isanewline
\ \ \ \ \isacommand{have}\isamarkupfalse%
\ {\isachardoublequoteopen}finite\ {\isacharparenleft}{\isacharbraceleft}F{\isacharbraceright}\ {\isasymunion}\ Wo{\isacharparenright}{\isachardoublequoteclose}\isanewline
\ \ \ \ \ \ \isacommand{using}\isamarkupfalse%
\ assms{\isacharparenleft}{\isadigit{4}}{\isacharparenright}\ \isacommand{by}\isamarkupfalse%
\ blast\ \isanewline
\ \ \ \ \isacommand{have}\isamarkupfalse%
\ {\isachardoublequoteopen}{\isacharbraceleft}F{\isacharbraceright}\ {\isasymunion}\ Wo\ {\isasymsubseteq}\ W{\isachardoublequoteclose}\isanewline
\ \ \ \ \ \ \isacommand{using}\isamarkupfalse%
\ assms{\isacharparenleft}{\isadigit{3}}{\isacharparenright}\ assms{\isacharparenleft}{\isadigit{5}}{\isacharparenright}\ \isacommand{by}\isamarkupfalse%
\ blast\ \isanewline
\ \ \ \ \isacommand{have}\isamarkupfalse%
\ {\isachardoublequoteopen}finite\ {\isacharparenleft}{\isacharbraceleft}F{\isacharbraceright}\ {\isasymunion}\ Wo{\isacharparenright}\ {\isasymlongrightarrow}\ sat\ {\isacharparenleft}{\isacharbraceleft}F{\isacharbraceright}\ {\isasymunion}\ Wo{\isacharparenright}{\isachardoublequoteclose}\isanewline
\ \ \ \ \ \ \isacommand{using}\isamarkupfalse%
\ WcolecComp\ {\isacartoucheopen}{\isacharbraceleft}F{\isacharbraceright}\ {\isasymunion}\ Wo\ {\isasymsubseteq}\ W{\isacartoucheclose}\ \isacommand{by}\isamarkupfalse%
\ {\isacharparenleft}rule\ sspec{\isacharparenright}\isanewline
\ \ \ \ \isacommand{thus}\isamarkupfalse%
\ {\isachardoublequoteopen}sat\ {\isacharparenleft}{\isacharbraceleft}F{\isacharbraceright}\ {\isasymunion}\ Wo{\isacharparenright}{\isachardoublequoteclose}\isanewline
\ \ \ \ \ \ \isacommand{using}\isamarkupfalse%
\ {\isacartoucheopen}finite\ {\isacharparenleft}{\isacharbraceleft}F{\isacharbraceright}\ {\isasymunion}\ Wo{\isacharparenright}{\isacartoucheclose}\ \isacommand{by}\isamarkupfalse%
\ {\isacharparenleft}rule\ mp{\isacharparenright}\isanewline
\ \ \isacommand{qed}\isamarkupfalse%
\isanewline
\ \ \isacommand{have}\isamarkupfalse%
\ {\isachardoublequoteopen}F\ {\isasymin}\ {\isacharbraceleft}F{\isacharbraceright}\ {\isasymunion}\ Wo{\isachardoublequoteclose}\isanewline
\ \ \ \ \isacommand{by}\isamarkupfalse%
\ simp\ \isanewline
\ \ \isacommand{have}\isamarkupfalse%
\ Ex{\isadigit{1}}{\isacharcolon}{\isachardoublequoteopen}{\isasymexists}{\isasymA}{\isachardot}\ {\isasymforall}F\ {\isasymin}\ {\isacharparenleft}{\isacharbraceleft}F{\isacharbraceright}\ {\isasymunion}\ Wo{\isacharparenright}{\isachardot}\ {\isasymA}\ {\isasymTurnstile}\ F{\isachardoublequoteclose}\isanewline
\ \ \ \ \isacommand{using}\isamarkupfalse%
\ {\isacartoucheopen}sat\ {\isacharparenleft}{\isacharbraceleft}F{\isacharbraceright}\ {\isasymunion}\ Wo{\isacharparenright}{\isacartoucheclose}\ \isacommand{by}\isamarkupfalse%
\ {\isacharparenleft}simp\ only{\isacharcolon}\ sat{\isacharunderscore}def{\isacharparenright}\isanewline
\ \ \isacommand{obtain}\isamarkupfalse%
\ {\isasymA}\ \isakeyword{where}\ Forall{\isadigit{1}}{\isacharcolon}{\isachardoublequoteopen}{\isasymforall}F\ {\isasymin}\ {\isacharparenleft}{\isacharbraceleft}F{\isacharbraceright}\ {\isasymunion}\ Wo{\isacharparenright}{\isachardot}\ {\isasymA}\ {\isasymTurnstile}\ F{\isachardoublequoteclose}\isanewline
\ \ \ \ \isacommand{using}\isamarkupfalse%
\ Ex{\isadigit{1}}\ \isacommand{by}\isamarkupfalse%
\ {\isacharparenleft}rule\ exE{\isacharparenright}\isanewline
\ \ \isacommand{have}\isamarkupfalse%
\ {\isachardoublequoteopen}{\isasymA}\ {\isasymTurnstile}\ F{\isachardoublequoteclose}\isanewline
\ \ \ \ \isacommand{using}\isamarkupfalse%
\ Forall{\isadigit{1}}\ {\isacartoucheopen}F\ {\isasymin}\ {\isacharbraceleft}F{\isacharbraceright}\ {\isasymunion}\ Wo{\isacartoucheclose}\ \isacommand{by}\isamarkupfalse%
\ {\isacharparenleft}rule\ bspec{\isacharparenright}\isanewline
\ \ \isacommand{have}\isamarkupfalse%
\ {\isachardoublequoteopen}{\isacharbraceleft}G{\isacharcomma}H{\isacharbraceright}\ {\isasymunion}\ Wo\ {\isasymsubseteq}\ {\isacharbraceleft}G{\isacharcomma}H{\isacharcomma}F{\isacharbraceright}\ {\isasymunion}\ Wo{\isachardoublequoteclose}\isanewline
\ \ \ \ \isacommand{by}\isamarkupfalse%
\ blast\ \isanewline
\ \ \isacommand{have}\isamarkupfalse%
\ {\isachardoublequoteopen}F\ {\isacharequal}\ G\ \isactrlbold {\isasymand}\ H\ {\isasymor}\ \isanewline
\ \ \ \ {\isacharparenleft}{\isasymexists}F{\isadigit{1}}\ G{\isadigit{1}}{\isachardot}\ F\ {\isacharequal}\ \isactrlbold {\isasymnot}\ {\isacharparenleft}F{\isadigit{1}}\ \isactrlbold {\isasymor}\ G{\isadigit{1}}{\isacharparenright}\ {\isasymand}\ G\ {\isacharequal}\ \isactrlbold {\isasymnot}\ F{\isadigit{1}}\ {\isasymand}\ H\ {\isacharequal}\ \isactrlbold {\isasymnot}\ G{\isadigit{1}}{\isacharparenright}\ {\isasymor}\ \isanewline
\ \ \ \ {\isacharparenleft}{\isasymexists}H{\isadigit{1}}{\isachardot}\ F\ {\isacharequal}\ \isactrlbold {\isasymnot}\ {\isacharparenleft}G\ \isactrlbold {\isasymrightarrow}\ H{\isadigit{1}}{\isacharparenright}\ {\isasymand}\ H\ {\isacharequal}\ \isactrlbold {\isasymnot}\ H{\isadigit{1}}{\isacharparenright}\ {\isasymor}\ \isanewline
\ \ \ \ F\ {\isacharequal}\ \isactrlbold {\isasymnot}\ {\isacharparenleft}\isactrlbold {\isasymnot}\ G{\isacharparenright}\ {\isasymand}\ H\ {\isacharequal}\ G{\isachardoublequoteclose}\isanewline
\ \ \ \ \isacommand{using}\isamarkupfalse%
\ assms{\isacharparenleft}{\isadigit{2}}{\isacharparenright}\ \isacommand{by}\isamarkupfalse%
\ {\isacharparenleft}simp\ only{\isacharcolon}\ con{\isacharunderscore}dis{\isacharunderscore}simps{\isacharparenleft}{\isadigit{1}}{\isacharparenright}{\isacharparenright}\isanewline
\ \ \isacommand{then}\isamarkupfalse%
\ \isacommand{have}\isamarkupfalse%
\ {\isachardoublequoteopen}sat\ {\isacharparenleft}{\isacharbraceleft}G{\isacharcomma}H{\isacharcomma}F{\isacharbraceright}\ {\isasymunion}\ Wo{\isacharparenright}{\isachardoublequoteclose}\isanewline
\ \ \isacommand{proof}\isamarkupfalse%
\ {\isacharparenleft}rule\ disjE{\isacharparenright}\isanewline
\ \ \ \ \isacommand{assume}\isamarkupfalse%
\ {\isachardoublequoteopen}F\ {\isacharequal}\ G\ \isactrlbold {\isasymand}\ H{\isachardoublequoteclose}\isanewline
\ \ \ \ \isacommand{then}\isamarkupfalse%
\ \isacommand{have}\isamarkupfalse%
\ {\isachardoublequoteopen}{\isasymA}\ {\isasymTurnstile}\ {\isacharparenleft}G\ \isactrlbold {\isasymand}\ H{\isacharparenright}{\isachardoublequoteclose}\isanewline
\ \ \ \ \ \ \isacommand{using}\isamarkupfalse%
\ {\isacartoucheopen}{\isasymA}\ {\isasymTurnstile}\ F{\isacartoucheclose}\ \isacommand{by}\isamarkupfalse%
\ {\isacharparenleft}simp\ only{\isacharcolon}\ {\isacartoucheopen}{\isasymA}\ {\isasymTurnstile}\ F{\isacartoucheclose}{\isacharparenright}\isanewline
\ \ \ \ \isacommand{then}\isamarkupfalse%
\ \isacommand{have}\isamarkupfalse%
\ {\isachardoublequoteopen}{\isasymA}\ {\isasymTurnstile}\ G\ {\isasymand}\ {\isasymA}\ {\isasymTurnstile}\ H{\isachardoublequoteclose}\isanewline
\ \ \ \ \ \ \isacommand{by}\isamarkupfalse%
\ {\isacharparenleft}simp\ only{\isacharcolon}\ formula{\isacharunderscore}semantics{\isachardot}simps{\isacharparenleft}{\isadigit{4}}{\isacharparenright}{\isacharparenright}\isanewline
\ \ \ \ \isacommand{then}\isamarkupfalse%
\ \isacommand{have}\isamarkupfalse%
\ {\isachardoublequoteopen}{\isasymA}\ {\isasymTurnstile}\ G{\isachardoublequoteclose}\isanewline
\ \ \ \ \ \ \isacommand{by}\isamarkupfalse%
\ {\isacharparenleft}rule\ conjunct{\isadigit{1}}{\isacharparenright}\isanewline
\ \ \ \ \isacommand{have}\isamarkupfalse%
\ {\isachardoublequoteopen}{\isasymA}\ {\isasymTurnstile}\ H{\isachardoublequoteclose}\isanewline
\ \ \ \ \ \ \isacommand{using}\isamarkupfalse%
\ {\isacartoucheopen}{\isasymA}\ {\isasymTurnstile}\ G\ {\isasymand}\ {\isasymA}\ {\isasymTurnstile}\ H{\isacartoucheclose}\ \isacommand{by}\isamarkupfalse%
\ {\isacharparenleft}rule\ conjunct{\isadigit{2}}{\isacharparenright}\isanewline
\ \ \ \ \isacommand{have}\isamarkupfalse%
\ {\isachardoublequoteopen}{\isasymforall}F\ {\isasymin}\ {\isacharbraceleft}G{\isacharcomma}H{\isacharcomma}F{\isacharbraceright}\ {\isasymunion}\ Wo{\isachardot}\ {\isasymA}\ {\isasymTurnstile}\ F{\isachardoublequoteclose}\isanewline
\ \ \ \ \ \ \isacommand{using}\isamarkupfalse%
\ Forall{\isadigit{1}}\ {\isacartoucheopen}{\isasymA}\ {\isasymTurnstile}\ G{\isacartoucheclose}\ {\isacartoucheopen}{\isasymA}\ {\isasymTurnstile}\ H{\isacartoucheclose}\ \isacommand{by}\isamarkupfalse%
\ blast\ \isanewline
\ \ \ \ \isacommand{then}\isamarkupfalse%
\ \isacommand{have}\isamarkupfalse%
\ {\isachardoublequoteopen}{\isasymexists}{\isasymA}{\isachardot}\ {\isasymforall}F\ {\isasymin}\ {\isacharparenleft}{\isacharbraceleft}G{\isacharcomma}H{\isacharcomma}F{\isacharbraceright}\ {\isasymunion}\ Wo{\isacharparenright}{\isachardot}\ {\isasymA}\ {\isasymTurnstile}\ F{\isachardoublequoteclose}\isanewline
\ \ \ \ \ \ \isacommand{by}\isamarkupfalse%
\ blast\ \isanewline
\ \ \ \ \isacommand{thus}\isamarkupfalse%
\ {\isachardoublequoteopen}sat\ {\isacharparenleft}{\isacharbraceleft}G{\isacharcomma}H{\isacharcomma}F{\isacharbraceright}\ {\isasymunion}\ Wo{\isacharparenright}{\isachardoublequoteclose}\isanewline
\ \ \ \ \ \ \isacommand{by}\isamarkupfalse%
\ {\isacharparenleft}simp\ only{\isacharcolon}\ sat{\isacharunderscore}def{\isacharparenright}\isanewline
\ \ \isacommand{next}\isamarkupfalse%
\isanewline
\ \ \ \ \isacommand{assume}\isamarkupfalse%
\ {\isachardoublequoteopen}{\isacharparenleft}{\isasymexists}F{\isadigit{1}}\ G{\isadigit{1}}{\isachardot}\ F\ {\isacharequal}\ \isactrlbold {\isasymnot}\ {\isacharparenleft}F{\isadigit{1}}\ \isactrlbold {\isasymor}\ G{\isadigit{1}}{\isacharparenright}\ {\isasymand}\ G\ {\isacharequal}\ \isactrlbold {\isasymnot}\ F{\isadigit{1}}\ {\isasymand}\ H\ {\isacharequal}\ \isactrlbold {\isasymnot}\ G{\isadigit{1}}{\isacharparenright}\ {\isasymor}\ \isanewline
\ \ \ \ {\isacharparenleft}{\isasymexists}H{\isadigit{1}}{\isachardot}\ F\ {\isacharequal}\ \isactrlbold {\isasymnot}\ {\isacharparenleft}G\ \isactrlbold {\isasymrightarrow}\ H{\isadigit{1}}{\isacharparenright}\ {\isasymand}\ H\ {\isacharequal}\ \isactrlbold {\isasymnot}\ H{\isadigit{1}}{\isacharparenright}\ {\isasymor}\ \isanewline
\ \ \ \ F\ {\isacharequal}\ \isactrlbold {\isasymnot}\ {\isacharparenleft}\isactrlbold {\isasymnot}\ G{\isacharparenright}\ {\isasymand}\ H\ {\isacharequal}\ G{\isachardoublequoteclose}\isanewline
\ \ \ \ \isacommand{thus}\isamarkupfalse%
\ {\isachardoublequoteopen}sat\ {\isacharparenleft}{\isacharbraceleft}G{\isacharcomma}H{\isacharcomma}F{\isacharbraceright}\ {\isasymunion}\ Wo{\isacharparenright}{\isachardoublequoteclose}\isanewline
\ \ \ \ \isacommand{proof}\isamarkupfalse%
\ {\isacharparenleft}rule\ disjE{\isacharparenright}\isanewline
\ \ \ \ \ \ \isacommand{assume}\isamarkupfalse%
\ Ex{\isadigit{2}}{\isacharcolon}{\isachardoublequoteopen}{\isasymexists}F{\isadigit{1}}\ G{\isadigit{1}}{\isachardot}\ F\ {\isacharequal}\ \isactrlbold {\isasymnot}\ {\isacharparenleft}F{\isadigit{1}}\ \isactrlbold {\isasymor}\ G{\isadigit{1}}{\isacharparenright}\ {\isasymand}\ G\ {\isacharequal}\ \isactrlbold {\isasymnot}\ F{\isadigit{1}}\ {\isasymand}\ H\ {\isacharequal}\ \isactrlbold {\isasymnot}\ G{\isadigit{1}}{\isachardoublequoteclose}\ \isanewline
\ \ \ \ \ \ \isacommand{obtain}\isamarkupfalse%
\ F{\isadigit{1}}\ G{\isadigit{1}}\ \isakeyword{where}\ {\isadigit{2}}{\isacharcolon}{\isachardoublequoteopen}F\ {\isacharequal}\ \isactrlbold {\isasymnot}{\isacharparenleft}F{\isadigit{1}}\ \isactrlbold {\isasymor}\ G{\isadigit{1}}{\isacharparenright}\ {\isasymand}\ G\ {\isacharequal}\ \isactrlbold {\isasymnot}\ F{\isadigit{1}}\ {\isasymand}\ H\ {\isacharequal}\ \isactrlbold {\isasymnot}\ G{\isadigit{1}}{\isachardoublequoteclose}\isanewline
\ \ \ \ \ \ \ \ \isacommand{using}\isamarkupfalse%
\ Ex{\isadigit{2}}\ \isacommand{by}\isamarkupfalse%
\ {\isacharparenleft}iprover\ elim{\isacharcolon}\ exE{\isacharparenright}\isanewline
\ \ \ \ \ \ \isacommand{have}\isamarkupfalse%
\ {\isachardoublequoteopen}G\ {\isacharequal}\ \isactrlbold {\isasymnot}\ F{\isadigit{1}}{\isachardoublequoteclose}\isanewline
\ \ \ \ \ \ \ \ \isacommand{using}\isamarkupfalse%
\ {\isadigit{2}}\ \isacommand{by}\isamarkupfalse%
\ {\isacharparenleft}iprover\ elim{\isacharcolon}\ conjunct{\isadigit{1}}{\isacharparenright}\isanewline
\ \ \ \ \ \ \isacommand{have}\isamarkupfalse%
\ {\isachardoublequoteopen}H\ {\isacharequal}\ \isactrlbold {\isasymnot}\ G{\isadigit{1}}{\isachardoublequoteclose}\isanewline
\ \ \ \ \ \ \ \ \isacommand{using}\isamarkupfalse%
\ {\isadigit{2}}\ \isacommand{by}\isamarkupfalse%
\ {\isacharparenleft}iprover\ elim{\isacharcolon}\ conjunct{\isadigit{2}}{\isacharparenright}\isanewline
\ \ \ \ \ \ \isacommand{have}\isamarkupfalse%
\ {\isachardoublequoteopen}F\ {\isacharequal}\ \isactrlbold {\isasymnot}{\isacharparenleft}F{\isadigit{1}}\ \isactrlbold {\isasymor}\ G{\isadigit{1}}{\isacharparenright}{\isachardoublequoteclose}\isanewline
\ \ \ \ \ \ \ \ \isacommand{using}\isamarkupfalse%
\ {\isadigit{2}}\ \isacommand{by}\isamarkupfalse%
\ {\isacharparenleft}rule\ conjunct{\isadigit{1}}{\isacharparenright}\isanewline
\ \ \ \ \ \ \isacommand{then}\isamarkupfalse%
\ \isacommand{have}\isamarkupfalse%
\ {\isachardoublequoteopen}{\isasymA}\ {\isasymTurnstile}\ \isactrlbold {\isasymnot}{\isacharparenleft}F{\isadigit{1}}\ \isactrlbold {\isasymor}\ G{\isadigit{1}}{\isacharparenright}{\isachardoublequoteclose}\isanewline
\ \ \ \ \ \ \ \ \isacommand{using}\isamarkupfalse%
\ {\isacartoucheopen}{\isasymA}\ {\isasymTurnstile}\ F{\isacartoucheclose}\ \isacommand{by}\isamarkupfalse%
\ {\isacharparenleft}simp\ only{\isacharcolon}\ {\isacartoucheopen}{\isasymA}\ {\isasymTurnstile}\ F{\isacartoucheclose}{\isacharparenright}\isanewline
\ \ \ \ \ \ \isacommand{then}\isamarkupfalse%
\ \isacommand{have}\isamarkupfalse%
\ {\isachardoublequoteopen}{\isasymnot}{\isacharparenleft}{\isasymA}\ {\isasymTurnstile}\ {\isacharparenleft}F{\isadigit{1}}\ \isactrlbold {\isasymor}\ G{\isadigit{1}}{\isacharparenright}{\isacharparenright}{\isachardoublequoteclose}\isanewline
\ \ \ \ \ \ \ \ \isacommand{by}\isamarkupfalse%
\ {\isacharparenleft}simp\ add{\isacharcolon}\ formula{\isacharunderscore}semantics{\isachardot}simps{\isacharparenleft}{\isadigit{3}}{\isacharparenright}{\isacharparenright}\ \isanewline
\ \ \ \ \ \ \isacommand{then}\isamarkupfalse%
\ \isacommand{have}\isamarkupfalse%
\ {\isachardoublequoteopen}{\isasymnot}{\isacharparenleft}{\isasymA}\ {\isasymTurnstile}\ F{\isadigit{1}}\ {\isasymor}\ {\isasymA}\ {\isasymTurnstile}\ G{\isadigit{1}}{\isacharparenright}{\isachardoublequoteclose}\isanewline
\ \ \ \ \ \ \ \ \isacommand{by}\isamarkupfalse%
\ {\isacharparenleft}simp\ add{\isacharcolon}\ formula{\isacharunderscore}semantics{\isachardot}simps{\isacharparenleft}{\isadigit{5}}{\isacharparenright}{\isacharparenright}\ \isanewline
\ \ \ \ \ \ \isacommand{then}\isamarkupfalse%
\ \isacommand{have}\isamarkupfalse%
\ {\isachardoublequoteopen}{\isasymnot}\ {\isasymA}\ {\isasymTurnstile}\ F{\isadigit{1}}\ {\isasymand}\ {\isasymnot}\ {\isasymA}\ {\isasymTurnstile}\ G{\isadigit{1}}{\isachardoublequoteclose}\isanewline
\ \ \ \ \ \ \ \ \isacommand{by}\isamarkupfalse%
\ simp\ \isanewline
\ \ \ \ \ \ \isacommand{then}\isamarkupfalse%
\ \isacommand{have}\isamarkupfalse%
\ {\isachardoublequoteopen}{\isasymA}\ {\isasymTurnstile}\ \isactrlbold {\isasymnot}\ F{\isadigit{1}}\ {\isasymand}\ {\isasymA}\ {\isasymTurnstile}\ \isactrlbold {\isasymnot}\ G{\isadigit{1}}{\isachardoublequoteclose}\isanewline
\ \ \ \ \ \ \ \ \isacommand{by}\isamarkupfalse%
\ {\isacharparenleft}simp\ add{\isacharcolon}\ formula{\isacharunderscore}semantics{\isachardot}simps{\isacharparenleft}{\isadigit{3}}{\isacharparenright}{\isacharparenright}\ \isanewline
\ \ \ \ \ \ \isacommand{then}\isamarkupfalse%
\ \isacommand{have}\isamarkupfalse%
\ {\isachardoublequoteopen}{\isasymA}\ {\isasymTurnstile}\ G\ {\isasymand}\ {\isasymA}\ {\isasymTurnstile}\ H{\isachardoublequoteclose}\isanewline
\ \ \ \ \ \ \ \ \isacommand{by}\isamarkupfalse%
\ {\isacharparenleft}simp\ only{\isacharcolon}\ {\isacartoucheopen}G\ {\isacharequal}\ \isactrlbold {\isasymnot}\ F{\isadigit{1}}{\isacartoucheclose}\ {\isacartoucheopen}H\ {\isacharequal}\ \isactrlbold {\isasymnot}\ G{\isadigit{1}}{\isacartoucheclose}{\isacharparenright}\isanewline
\ \ \ \ \ \ \isacommand{then}\isamarkupfalse%
\ \isacommand{have}\isamarkupfalse%
\ {\isachardoublequoteopen}{\isasymA}\ {\isasymTurnstile}\ G{\isachardoublequoteclose}\isanewline
\ \ \ \ \ \ \ \ \isacommand{by}\isamarkupfalse%
\ {\isacharparenleft}rule\ conjunct{\isadigit{1}}{\isacharparenright}\isanewline
\ \ \ \ \ \ \isacommand{have}\isamarkupfalse%
\ {\isachardoublequoteopen}{\isasymA}\ {\isasymTurnstile}\ H{\isachardoublequoteclose}\isanewline
\ \ \ \ \ \ \ \ \isacommand{using}\isamarkupfalse%
\ {\isacartoucheopen}{\isasymA}\ {\isasymTurnstile}\ G\ {\isasymand}\ {\isasymA}\ {\isasymTurnstile}\ H{\isacartoucheclose}\ \isacommand{by}\isamarkupfalse%
\ {\isacharparenleft}rule\ conjunct{\isadigit{2}}{\isacharparenright}\ \isanewline
\ \ \ \ \ \ \isacommand{have}\isamarkupfalse%
\ {\isachardoublequoteopen}{\isasymforall}F\ {\isasymin}\ {\isacharbraceleft}G{\isacharcomma}H{\isacharcomma}F{\isacharbraceright}\ {\isasymunion}\ Wo{\isachardot}\ {\isasymA}\ {\isasymTurnstile}\ F{\isachardoublequoteclose}\isanewline
\ \ \ \ \ \ \ \ \isacommand{using}\isamarkupfalse%
\ Forall{\isadigit{1}}\ {\isacartoucheopen}{\isasymA}\ {\isasymTurnstile}\ G{\isacartoucheclose}\ {\isacartoucheopen}{\isasymA}\ {\isasymTurnstile}\ H{\isacartoucheclose}\ \isacommand{by}\isamarkupfalse%
\ blast\ \isanewline
\ \ \ \ \ \ \isacommand{then}\isamarkupfalse%
\ \isacommand{have}\isamarkupfalse%
\ {\isachardoublequoteopen}{\isasymexists}{\isasymA}{\isachardot}\ {\isasymforall}F\ {\isasymin}\ {\isacharparenleft}{\isacharbraceleft}G{\isacharcomma}H{\isacharcomma}F{\isacharbraceright}\ {\isasymunion}\ Wo{\isacharparenright}{\isachardot}\ {\isasymA}\ {\isasymTurnstile}\ F{\isachardoublequoteclose}\isanewline
\ \ \ \ \ \ \ \ \isacommand{by}\isamarkupfalse%
\ blast\ \isanewline
\ \ \ \ \ \ \isacommand{thus}\isamarkupfalse%
\ {\isachardoublequoteopen}sat\ {\isacharparenleft}{\isacharbraceleft}G{\isacharcomma}H{\isacharcomma}F{\isacharbraceright}\ {\isasymunion}\ Wo{\isacharparenright}{\isachardoublequoteclose}\isanewline
\ \ \ \ \ \ \ \ \isacommand{by}\isamarkupfalse%
\ {\isacharparenleft}simp\ only{\isacharcolon}\ sat{\isacharunderscore}def{\isacharparenright}\isanewline
\ \ \ \ \isacommand{next}\isamarkupfalse%
\isanewline
\ \ \ \ \ \ \isacommand{assume}\isamarkupfalse%
\ {\isachardoublequoteopen}{\isacharparenleft}{\isasymexists}H{\isadigit{1}}{\isachardot}\ F\ {\isacharequal}\ \isactrlbold {\isasymnot}\ {\isacharparenleft}G\ \isactrlbold {\isasymrightarrow}\ H{\isadigit{1}}{\isacharparenright}\ {\isasymand}\ H\ {\isacharequal}\ \isactrlbold {\isasymnot}\ H{\isadigit{1}}{\isacharparenright}\ {\isasymor}\ \isanewline
\ \ \ \ \ \ \ \ \ \ \ \ \ \ F\ {\isacharequal}\ \isactrlbold {\isasymnot}\ {\isacharparenleft}\isactrlbold {\isasymnot}\ G{\isacharparenright}\ {\isasymand}\ H\ {\isacharequal}\ G{\isachardoublequoteclose}\isanewline
\ \ \ \ \ \ \isacommand{thus}\isamarkupfalse%
\ {\isachardoublequoteopen}sat\ {\isacharparenleft}{\isacharbraceleft}G{\isacharcomma}H{\isacharcomma}F{\isacharbraceright}\ {\isasymunion}\ Wo{\isacharparenright}{\isachardoublequoteclose}\isanewline
\ \ \ \ \ \ \isacommand{proof}\isamarkupfalse%
\ {\isacharparenleft}rule\ disjE{\isacharparenright}\isanewline
\ \ \ \ \ \ \ \ \isacommand{assume}\isamarkupfalse%
\ Ex{\isadigit{3}}{\isacharcolon}{\isachardoublequoteopen}{\isasymexists}H{\isadigit{1}}{\isachardot}\ F\ {\isacharequal}\ \isactrlbold {\isasymnot}\ {\isacharparenleft}G\ \isactrlbold {\isasymrightarrow}\ H{\isadigit{1}}{\isacharparenright}\ {\isasymand}\ H\ {\isacharequal}\ \isactrlbold {\isasymnot}\ H{\isadigit{1}}{\isachardoublequoteclose}\isanewline
\ \ \ \ \ \ \ \ \isacommand{obtain}\isamarkupfalse%
\ H{\isadigit{1}}\ \isakeyword{where}\ {\isadigit{3}}{\isacharcolon}{\isachardoublequoteopen}F\ {\isacharequal}\ \isactrlbold {\isasymnot}{\isacharparenleft}G\ \isactrlbold {\isasymrightarrow}\ H{\isadigit{1}}{\isacharparenright}\ {\isasymand}\ H\ {\isacharequal}\ \isactrlbold {\isasymnot}\ H{\isadigit{1}}{\isachardoublequoteclose}\isanewline
\ \ \ \ \ \ \ \ \ \ \isacommand{using}\isamarkupfalse%
\ Ex{\isadigit{3}}\ \isacommand{by}\isamarkupfalse%
\ {\isacharparenleft}rule\ exE{\isacharparenright}\isanewline
\ \ \ \ \ \ \ \ \isacommand{have}\isamarkupfalse%
\ {\isachardoublequoteopen}H\ {\isacharequal}\ \isactrlbold {\isasymnot}\ H{\isadigit{1}}{\isachardoublequoteclose}\isanewline
\ \ \ \ \ \ \ \ \ \ \isacommand{using}\isamarkupfalse%
\ {\isadigit{3}}\ \isacommand{by}\isamarkupfalse%
\ {\isacharparenleft}rule\ conjunct{\isadigit{2}}{\isacharparenright}\isanewline
\ \ \ \ \ \ \ \ \isacommand{have}\isamarkupfalse%
\ {\isachardoublequoteopen}F\ {\isacharequal}\ \isactrlbold {\isasymnot}{\isacharparenleft}G\ \isactrlbold {\isasymrightarrow}\ H{\isadigit{1}}{\isacharparenright}{\isachardoublequoteclose}\isanewline
\ \ \ \ \ \ \ \ \ \ \isacommand{using}\isamarkupfalse%
\ {\isadigit{3}}\ \isacommand{by}\isamarkupfalse%
\ {\isacharparenleft}rule\ conjunct{\isadigit{1}}{\isacharparenright}\isanewline
\ \ \ \ \ \ \ \ \isacommand{then}\isamarkupfalse%
\ \isacommand{have}\isamarkupfalse%
\ {\isachardoublequoteopen}{\isasymA}\ {\isasymTurnstile}\ \isactrlbold {\isasymnot}{\isacharparenleft}G\ \isactrlbold {\isasymrightarrow}\ H{\isadigit{1}}{\isacharparenright}{\isachardoublequoteclose}\isanewline
\ \ \ \ \ \ \ \ \ \ \isacommand{using}\isamarkupfalse%
\ {\isacartoucheopen}{\isasymA}\ {\isasymTurnstile}\ F{\isacartoucheclose}\ \isacommand{by}\isamarkupfalse%
\ {\isacharparenleft}simp\ only{\isacharcolon}\ {\isacartoucheopen}{\isasymA}\ {\isasymTurnstile}\ F{\isacartoucheclose}{\isacharparenright}\isanewline
\ \ \ \ \ \ \ \ \isacommand{then}\isamarkupfalse%
\ \isacommand{have}\isamarkupfalse%
\ {\isachardoublequoteopen}{\isasymnot}{\isacharparenleft}{\isasymA}\ {\isasymTurnstile}\ {\isacharparenleft}G\ \isactrlbold {\isasymrightarrow}\ H{\isadigit{1}}{\isacharparenright}{\isacharparenright}{\isachardoublequoteclose}\isanewline
\ \ \ \ \ \ \ \ \ \ \isacommand{by}\isamarkupfalse%
\ {\isacharparenleft}simp\ add{\isacharcolon}\ formula{\isacharunderscore}semantics{\isachardot}simps{\isacharparenleft}{\isadigit{3}}{\isacharparenright}{\isacharparenright}\ \isanewline
\ \ \ \ \ \ \ \ \isacommand{then}\isamarkupfalse%
\ \isacommand{have}\isamarkupfalse%
\ {\isachardoublequoteopen}{\isasymnot}{\isacharparenleft}{\isasymA}\ {\isasymTurnstile}\ G\ {\isasymlongrightarrow}\ {\isasymA}\ {\isasymTurnstile}\ H{\isadigit{1}}{\isacharparenright}{\isachardoublequoteclose}\isanewline
\ \ \ \ \ \ \ \ \ \ \isacommand{by}\isamarkupfalse%
\ {\isacharparenleft}simp\ add{\isacharcolon}\ formula{\isacharunderscore}semantics{\isachardot}simps{\isacharparenleft}{\isadigit{6}}{\isacharparenright}{\isacharparenright}\ \isanewline
\ \ \ \ \ \ \ \ \isacommand{then}\isamarkupfalse%
\ \isacommand{have}\isamarkupfalse%
\ {\isachardoublequoteopen}{\isasymA}\ {\isasymTurnstile}\ G\ {\isasymand}\ {\isasymnot}\ {\isacharparenleft}{\isasymA}\ {\isasymTurnstile}\ H{\isadigit{1}}{\isacharparenright}{\isachardoublequoteclose}\isanewline
\ \ \ \ \ \ \ \ \ \ \isacommand{by}\isamarkupfalse%
\ simp\ \isanewline
\ \ \ \ \ \ \ \ \isacommand{then}\isamarkupfalse%
\ \isacommand{have}\isamarkupfalse%
\ {\isachardoublequoteopen}{\isasymA}\ {\isasymTurnstile}\ G\ {\isasymand}\ {\isasymA}\ {\isasymTurnstile}\ \isactrlbold {\isasymnot}\ H{\isadigit{1}}{\isachardoublequoteclose}\isanewline
\ \ \ \ \ \ \ \ \ \ \isacommand{by}\isamarkupfalse%
\ {\isacharparenleft}simp\ add{\isacharcolon}\ formula{\isacharunderscore}semantics{\isachardot}simps{\isacharparenleft}{\isadigit{3}}{\isacharparenright}{\isacharparenright}\isanewline
\ \ \ \ \ \ \ \ \isacommand{then}\isamarkupfalse%
\ \isacommand{have}\isamarkupfalse%
\ {\isachardoublequoteopen}{\isasymA}\ {\isasymTurnstile}\ G\ {\isasymand}\ {\isasymA}\ {\isasymTurnstile}\ H{\isachardoublequoteclose}\isanewline
\ \ \ \ \ \ \ \ \ \ \isacommand{by}\isamarkupfalse%
\ {\isacharparenleft}simp\ only{\isacharcolon}\ {\isacartoucheopen}H\ {\isacharequal}\ \isactrlbold {\isasymnot}\ H{\isadigit{1}}{\isacartoucheclose}{\isacharparenright}\isanewline
\ \ \ \ \ \ \ \ \isacommand{then}\isamarkupfalse%
\ \isacommand{have}\isamarkupfalse%
\ {\isachardoublequoteopen}{\isasymA}\ {\isasymTurnstile}\ G{\isachardoublequoteclose}\isanewline
\ \ \ \ \ \ \ \ \ \ \isacommand{by}\isamarkupfalse%
\ {\isacharparenleft}rule\ conjunct{\isadigit{1}}{\isacharparenright}\isanewline
\ \ \ \ \ \ \ \ \isacommand{have}\isamarkupfalse%
\ {\isachardoublequoteopen}{\isasymA}\ {\isasymTurnstile}\ H{\isachardoublequoteclose}\isanewline
\ \ \ \ \ \ \ \ \ \ \isacommand{using}\isamarkupfalse%
\ {\isacartoucheopen}{\isasymA}\ {\isasymTurnstile}\ G\ {\isasymand}\ {\isasymA}\ {\isasymTurnstile}\ H{\isacartoucheclose}\ \isacommand{by}\isamarkupfalse%
\ {\isacharparenleft}rule\ conjunct{\isadigit{2}}{\isacharparenright}\ \isanewline
\ \ \ \ \ \ \ \ \isacommand{have}\isamarkupfalse%
\ {\isachardoublequoteopen}{\isasymforall}F\ {\isasymin}\ {\isacharbraceleft}G{\isacharcomma}H{\isacharcomma}F{\isacharbraceright}\ {\isasymunion}\ Wo{\isachardot}\ {\isasymA}\ {\isasymTurnstile}\ F{\isachardoublequoteclose}\isanewline
\ \ \ \ \ \ \ \ \ \ \isacommand{using}\isamarkupfalse%
\ Forall{\isadigit{1}}\ {\isacartoucheopen}{\isasymA}\ {\isasymTurnstile}\ G{\isacartoucheclose}\ {\isacartoucheopen}{\isasymA}\ {\isasymTurnstile}\ H{\isacartoucheclose}\ \isacommand{by}\isamarkupfalse%
\ blast\ \isanewline
\ \ \ \ \ \ \ \ \isacommand{then}\isamarkupfalse%
\ \isacommand{have}\isamarkupfalse%
\ {\isachardoublequoteopen}{\isasymexists}{\isasymA}{\isachardot}\ {\isasymforall}F\ {\isasymin}\ {\isacharparenleft}{\isacharbraceleft}G{\isacharcomma}H{\isacharcomma}F{\isacharbraceright}\ {\isasymunion}\ Wo{\isacharparenright}{\isachardot}\ {\isasymA}\ {\isasymTurnstile}\ F{\isachardoublequoteclose}\isanewline
\ \ \ \ \ \ \ \ \ \ \isacommand{by}\isamarkupfalse%
\ blast\ \isanewline
\ \ \ \ \ \ \ \ \isacommand{thus}\isamarkupfalse%
\ {\isachardoublequoteopen}sat\ {\isacharparenleft}{\isacharbraceleft}G{\isacharcomma}H{\isacharcomma}F{\isacharbraceright}\ {\isasymunion}\ Wo{\isacharparenright}{\isachardoublequoteclose}\isanewline
\ \ \ \ \ \ \ \ \ \ \isacommand{by}\isamarkupfalse%
\ {\isacharparenleft}simp\ only{\isacharcolon}\ sat{\isacharunderscore}def{\isacharparenright}\isanewline
\ \ \ \ \ \ \isacommand{next}\isamarkupfalse%
\isanewline
\ \ \ \ \ \ \ \ \isacommand{assume}\isamarkupfalse%
\ {\isachardoublequoteopen}F\ {\isacharequal}\ \isactrlbold {\isasymnot}\ {\isacharparenleft}\isactrlbold {\isasymnot}\ G{\isacharparenright}\ {\isasymand}\ H\ {\isacharequal}\ G{\isachardoublequoteclose}\isanewline
\ \ \ \ \ \ \ \ \isacommand{then}\isamarkupfalse%
\ \isacommand{have}\isamarkupfalse%
\ {\isachardoublequoteopen}H\ {\isacharequal}\ G{\isachardoublequoteclose}\isanewline
\ \ \ \ \ \ \ \ \ \ \isacommand{by}\isamarkupfalse%
\ {\isacharparenleft}rule\ conjunct{\isadigit{2}}{\isacharparenright}\isanewline
\ \ \ \ \ \ \ \ \isacommand{have}\isamarkupfalse%
\ {\isachardoublequoteopen}F\ {\isacharequal}\ \isactrlbold {\isasymnot}\ {\isacharparenleft}\isactrlbold {\isasymnot}\ G{\isacharparenright}{\isachardoublequoteclose}\isanewline
\ \ \ \ \ \ \ \ \ \ \isacommand{using}\isamarkupfalse%
\ {\isacartoucheopen}F\ {\isacharequal}\ \isactrlbold {\isasymnot}\ {\isacharparenleft}\isactrlbold {\isasymnot}\ G{\isacharparenright}\ {\isasymand}\ H\ {\isacharequal}\ G{\isacartoucheclose}\ \isacommand{by}\isamarkupfalse%
\ {\isacharparenleft}rule\ conjunct{\isadigit{1}}{\isacharparenright}\isanewline
\ \ \ \ \ \ \ \ \isacommand{then}\isamarkupfalse%
\ \isacommand{have}\isamarkupfalse%
\ {\isachardoublequoteopen}{\isasymA}\ {\isasymTurnstile}\ \isactrlbold {\isasymnot}\ {\isacharparenleft}\isactrlbold {\isasymnot}\ G{\isacharparenright}{\isachardoublequoteclose}\isanewline
\ \ \ \ \ \ \ \ \ \ \isacommand{using}\isamarkupfalse%
\ {\isacartoucheopen}{\isasymA}\ {\isasymTurnstile}\ F{\isacartoucheclose}\ \isacommand{by}\isamarkupfalse%
\ {\isacharparenleft}simp\ only{\isacharcolon}\ {\isacartoucheopen}{\isasymA}\ {\isasymTurnstile}\ F{\isacartoucheclose}{\isacharparenright}\isanewline
\ \ \ \ \ \ \ \ \isacommand{then}\isamarkupfalse%
\ \isacommand{have}\isamarkupfalse%
\ {\isachardoublequoteopen}{\isasymnot}\ {\isasymA}\ {\isasymTurnstile}\ \isactrlbold {\isasymnot}\ G{\isachardoublequoteclose}\isanewline
\ \ \ \ \ \ \ \ \ \ \isacommand{by}\isamarkupfalse%
\ {\isacharparenleft}simp\ add{\isacharcolon}\ formula{\isacharunderscore}semantics{\isachardot}simps{\isacharparenleft}{\isadigit{3}}{\isacharparenright}{\isacharparenright}\ \isanewline
\ \ \ \ \ \ \ \ \isacommand{then}\isamarkupfalse%
\ \isacommand{have}\isamarkupfalse%
\ {\isachardoublequoteopen}{\isasymnot}\ {\isasymnot}{\isasymA}\ {\isasymTurnstile}\ G{\isachardoublequoteclose}\isanewline
\ \ \ \ \ \ \ \ \ \ \isacommand{by}\isamarkupfalse%
\ {\isacharparenleft}simp\ add{\isacharcolon}\ formula{\isacharunderscore}semantics{\isachardot}simps{\isacharparenleft}{\isadigit{3}}{\isacharparenright}{\isacharparenright}\isanewline
\ \ \ \ \ \ \ \ \isacommand{then}\isamarkupfalse%
\ \isacommand{have}\isamarkupfalse%
\ {\isachardoublequoteopen}{\isasymA}\ {\isasymTurnstile}\ G{\isachardoublequoteclose}\isanewline
\ \ \ \ \ \ \ \ \ \ \isacommand{by}\isamarkupfalse%
\ {\isacharparenleft}rule\ notnotD{\isacharparenright}\isanewline
\ \ \ \ \ \ \ \ \isacommand{then}\isamarkupfalse%
\ \isacommand{have}\isamarkupfalse%
\ {\isachardoublequoteopen}{\isasymA}\ {\isasymTurnstile}\ H{\isachardoublequoteclose}\isanewline
\ \ \ \ \ \ \ \ \ \ \isacommand{by}\isamarkupfalse%
\ {\isacharparenleft}simp\ only{\isacharcolon}\ {\isacartoucheopen}H\ {\isacharequal}\ G{\isacartoucheclose}{\isacharparenright}\isanewline
\ \ \ \ \ \ \ \ \isacommand{have}\isamarkupfalse%
\ {\isachardoublequoteopen}{\isasymforall}F\ {\isasymin}\ {\isacharbraceleft}G{\isacharcomma}H{\isacharcomma}F{\isacharbraceright}\ {\isasymunion}\ Wo{\isachardot}\ {\isasymA}\ {\isasymTurnstile}\ F{\isachardoublequoteclose}\isanewline
\ \ \ \ \ \ \ \ \ \ \isacommand{using}\isamarkupfalse%
\ Forall{\isadigit{1}}\ {\isacartoucheopen}{\isasymA}\ {\isasymTurnstile}\ G{\isacartoucheclose}\ {\isacartoucheopen}{\isasymA}\ {\isasymTurnstile}\ H{\isacartoucheclose}\ \isacommand{by}\isamarkupfalse%
\ blast\ \isanewline
\ \ \ \ \ \ \ \ \isacommand{then}\isamarkupfalse%
\ \isacommand{have}\isamarkupfalse%
\ {\isachardoublequoteopen}{\isasymexists}{\isasymA}{\isachardot}\ {\isasymforall}F\ {\isasymin}\ {\isacharparenleft}{\isacharbraceleft}G{\isacharcomma}H{\isacharcomma}F{\isacharbraceright}\ {\isasymunion}\ Wo{\isacharparenright}{\isachardot}\ {\isasymA}\ {\isasymTurnstile}\ F{\isachardoublequoteclose}\isanewline
\ \ \ \ \ \ \ \ \ \ \isacommand{by}\isamarkupfalse%
\ blast\ \isanewline
\ \ \ \ \ \ \ \ \isacommand{thus}\isamarkupfalse%
\ {\isachardoublequoteopen}sat\ {\isacharparenleft}{\isacharbraceleft}G{\isacharcomma}H{\isacharcomma}F{\isacharbraceright}\ {\isasymunion}\ Wo{\isacharparenright}{\isachardoublequoteclose}\isanewline
\ \ \ \ \ \ \ \ \ \ \isacommand{by}\isamarkupfalse%
\ {\isacharparenleft}simp\ only{\isacharcolon}\ sat{\isacharunderscore}def{\isacharparenright}\isanewline
\ \ \ \ \ \ \isacommand{qed}\isamarkupfalse%
\isanewline
\ \ \ \ \isacommand{qed}\isamarkupfalse%
\isanewline
\ \ \isacommand{qed}\isamarkupfalse%
\isanewline
\ \ \isacommand{thus}\isamarkupfalse%
\ {\isachardoublequoteopen}sat\ {\isacharparenleft}{\isacharbraceleft}G{\isacharcomma}H{\isacharbraceright}\ {\isasymunion}\ Wo{\isacharparenright}{\isachardoublequoteclose}\isanewline
\ \ \ \ \isacommand{using}\isamarkupfalse%
\ {\isacartoucheopen}{\isacharbraceleft}G{\isacharcomma}H{\isacharbraceright}\ {\isasymunion}\ Wo\ {\isasymsubseteq}\ {\isacharbraceleft}G{\isacharcomma}H{\isacharcomma}F{\isacharbraceright}\ {\isasymunion}\ Wo{\isacartoucheclose}\ \isacommand{by}\isamarkupfalse%
\ {\isacharparenleft}simp\ only{\isacharcolon}\ sat{\isacharunderscore}mono{\isacharparenright}\isanewline
\isacommand{qed}\isamarkupfalse%
%
\endisatagproof
{\isafoldproof}%
%
\isadelimproof
\isanewline
%
\endisadelimproof
\ \ \ \ \ \ \isanewline
\isacommand{lemma}\isamarkupfalse%
\ pcp{\isacharunderscore}colecComp{\isacharunderscore}CON{\isacharcolon}\isanewline
\ \ \isakeyword{assumes}\ {\isachardoublequoteopen}W\ {\isasymin}\ {\isacharparenleft}colecComp\ S{\isacharparenright}{\isachardoublequoteclose}\isanewline
\ \ \isakeyword{shows}\ {\isachardoublequoteopen}{\isasymforall}F\ G\ H{\isachardot}\ Con\ F\ G\ H\ {\isasymlongrightarrow}\ F\ {\isasymin}\ W\ {\isasymlongrightarrow}\ {\isacharbraceleft}G{\isacharcomma}H{\isacharbraceright}\ {\isasymunion}\ W\ {\isasymin}\ {\isacharparenleft}colecComp\ S{\isacharparenright}{\isachardoublequoteclose}\isanewline
%
\isadelimproof
%
\endisadelimproof
%
\isatagproof
\isacommand{proof}\isamarkupfalse%
\ {\isacharparenleft}rule\ allI{\isacharparenright}{\isacharplus}\isanewline
\ \ \isacommand{fix}\isamarkupfalse%
\ F\ G\ H\isanewline
\ \ \isacommand{have}\isamarkupfalse%
\ Hip{\isacharcolon}{\isachardoublequoteopen}{\isasymforall}S{\isacharprime}\ {\isasymsubseteq}\ W{\isachardot}\ finite\ S{\isacharprime}\ {\isasymlongrightarrow}\ sat\ S{\isacharprime}{\isachardoublequoteclose}\isanewline
\ \ \ \ \isacommand{using}\isamarkupfalse%
\ assms\ \isacommand{unfolding}\isamarkupfalse%
\ colecComp\ fin{\isacharunderscore}sat{\isacharunderscore}def\ \isacommand{by}\isamarkupfalse%
\ {\isacharparenleft}rule\ CollectD{\isacharparenright}\isanewline
\ \ \isacommand{show}\isamarkupfalse%
\ {\isachardoublequoteopen}Con\ F\ G\ H\ {\isasymlongrightarrow}\ F\ {\isasymin}\ W\ {\isasymlongrightarrow}\ {\isacharbraceleft}G{\isacharcomma}H{\isacharbraceright}\ {\isasymunion}\ W\ {\isasymin}\ {\isacharparenleft}colecComp\ S{\isacharparenright}{\isachardoublequoteclose}\isanewline
\ \ \isacommand{proof}\isamarkupfalse%
\ {\isacharparenleft}rule\ impI{\isacharparenright}{\isacharplus}\isanewline
\ \ \ \ \isacommand{assume}\isamarkupfalse%
\ {\isachardoublequoteopen}Con\ F\ G\ H{\isachardoublequoteclose}\isanewline
\ \ \ \ \isacommand{assume}\isamarkupfalse%
\ {\isachardoublequoteopen}F\ {\isasymin}\ W{\isachardoublequoteclose}\isanewline
\ \ \ \ \isacommand{show}\isamarkupfalse%
\ {\isachardoublequoteopen}{\isacharbraceleft}G{\isacharcomma}H{\isacharbraceright}\ {\isasymunion}\ W\ {\isasymin}\ {\isacharparenleft}colecComp\ S{\isacharparenright}{\isachardoublequoteclose}\isanewline
\ \ \ \ \ \ \isacommand{unfolding}\isamarkupfalse%
\ colecComp\ fin{\isacharunderscore}sat{\isacharunderscore}def\isanewline
\ \ \ \ \isacommand{proof}\isamarkupfalse%
\ {\isacharparenleft}rule\ CollectI{\isacharparenright}\isanewline
\ \ \ \ \ \ \isacommand{show}\isamarkupfalse%
\ {\isachardoublequoteopen}{\isasymforall}S{\isacharprime}\ {\isasymsubseteq}\ {\isacharbraceleft}G{\isacharcomma}H{\isacharbraceright}\ {\isasymunion}\ W{\isachardot}\ finite\ S{\isacharprime}\ {\isasymlongrightarrow}\ sat\ S{\isacharprime}{\isachardoublequoteclose}\isanewline
\ \ \ \ \ \ \isacommand{proof}\isamarkupfalse%
\ {\isacharparenleft}rule\ sallI{\isacharparenright}\isanewline
\ \ \ \ \ \ \ \ \isacommand{fix}\isamarkupfalse%
\ S{\isacharprime}\isanewline
\ \ \ \ \ \ \ \ \isacommand{assume}\isamarkupfalse%
\ {\isachardoublequoteopen}S{\isacharprime}\ {\isasymsubseteq}\ {\isacharbraceleft}G{\isacharcomma}H{\isacharbraceright}\ {\isasymunion}\ W{\isachardoublequoteclose}\isanewline
\ \ \ \ \ \ \ \ \isacommand{then}\isamarkupfalse%
\ \isacommand{have}\isamarkupfalse%
\ {\isachardoublequoteopen}S{\isacharprime}\ {\isasymsubseteq}\ {\isacharbraceleft}G{\isacharbraceright}\ {\isasymunion}\ {\isacharparenleft}{\isacharbraceleft}H{\isacharbraceright}\ {\isasymunion}\ W{\isacharparenright}{\isachardoublequoteclose}\isanewline
\ \ \ \ \ \ \ \ \ \ \isacommand{by}\isamarkupfalse%
\ blast\ \isanewline
\ \ \ \ \ \ \ \ \isacommand{show}\isamarkupfalse%
\ {\isachardoublequoteopen}finite\ S{\isacharprime}\ {\isasymlongrightarrow}\ sat\ S{\isacharprime}{\isachardoublequoteclose}\isanewline
\ \ \ \ \ \ \ \ \isacommand{proof}\isamarkupfalse%
\ {\isacharparenleft}rule\ impI{\isacharparenright}\isanewline
\ \ \ \ \ \ \ \ \ \ \isacommand{assume}\isamarkupfalse%
\ {\isachardoublequoteopen}finite\ S{\isacharprime}{\isachardoublequoteclose}\ \isanewline
\ \ \ \ \ \ \ \ \ \ \isacommand{have}\isamarkupfalse%
\ Ex{\isacharcolon}{\isachardoublequoteopen}{\isasymexists}Wo\ {\isasymsubseteq}\ W{\isachardot}\ finite\ Wo\ {\isasymand}\ {\isacharparenleft}S{\isacharprime}\ {\isacharequal}\ {\isacharbraceleft}G{\isacharcomma}H{\isacharbraceright}\ {\isasymunion}\ Wo\ {\isasymor}\ S{\isacharprime}\ {\isacharequal}\ {\isacharbraceleft}G{\isacharbraceright}\ {\isasymunion}\ Wo\ {\isasymor}\ S{\isacharprime}\ {\isacharequal}\ {\isacharbraceleft}H{\isacharbraceright}\ {\isasymunion}\ Wo\ {\isasymor}\ S{\isacharprime}\ {\isacharequal}\ Wo{\isacharparenright}{\isachardoublequoteclose}\isanewline
\ \ \ \ \ \ \ \ \ \ \ \ \isacommand{using}\isamarkupfalse%
\ {\isacartoucheopen}finite\ S{\isacharprime}{\isacartoucheclose}\ {\isacartoucheopen}S{\isacharprime}\ {\isasymsubseteq}\ {\isacharbraceleft}G{\isacharcomma}H{\isacharbraceright}\ {\isasymunion}\ W{\isacartoucheclose}\ \isacommand{by}\isamarkupfalse%
\ {\isacharparenleft}rule\ finite{\isacharunderscore}subset{\isacharunderscore}insert{\isadigit{2}}{\isacharparenright}\isanewline
\ \ \ \ \ \ \ \ \ \ \isacommand{obtain}\isamarkupfalse%
\ Wo{\isacharprime}\ \isakeyword{where}\ {\isachardoublequoteopen}Wo{\isacharprime}\ {\isasymsubseteq}\ W{\isachardoublequoteclose}\ \isakeyword{and}\ {\isadigit{1}}{\isacharcolon}{\isachardoublequoteopen}finite\ Wo{\isacharprime}\ {\isasymand}\ {\isacharparenleft}S{\isacharprime}\ {\isacharequal}\ {\isacharbraceleft}G{\isacharcomma}H{\isacharbraceright}\ {\isasymunion}\ Wo{\isacharprime}\ {\isasymor}\ S{\isacharprime}\ {\isacharequal}\ {\isacharbraceleft}G{\isacharbraceright}\ {\isasymunion}\ Wo{\isacharprime}\ {\isasymor}\ S{\isacharprime}\ {\isacharequal}\ {\isacharbraceleft}H{\isacharbraceright}\ {\isasymunion}\ Wo{\isacharprime}\ {\isasymor}\ S{\isacharprime}\ {\isacharequal}\ Wo{\isacharprime}{\isacharparenright}{\isachardoublequoteclose}\isanewline
\ \ \ \ \ \ \ \ \ \ \ \ \isacommand{using}\isamarkupfalse%
\ Ex\ \isacommand{by}\isamarkupfalse%
\ blast\ \isanewline
\ \ \ \ \ \ \ \ \ \ \isacommand{have}\isamarkupfalse%
\ {\isachardoublequoteopen}finite\ Wo{\isacharprime}{\isachardoublequoteclose}\isanewline
\ \ \ \ \ \ \ \ \ \ \ \ \isacommand{using}\isamarkupfalse%
\ {\isadigit{1}}\ \isacommand{by}\isamarkupfalse%
\ {\isacharparenleft}rule\ conjunct{\isadigit{1}}{\isacharparenright}\isanewline
\ \ \ \ \ \ \ \ \ \ \ \ \isacommand{have}\isamarkupfalse%
\ {\isachardoublequoteopen}sat\ {\isacharparenleft}{\isacharbraceleft}G{\isacharcomma}H{\isacharbraceright}\ {\isasymunion}\ Wo{\isacharprime}{\isacharparenright}{\isachardoublequoteclose}\ \isanewline
\ \ \ \ \ \ \ \ \ \ \ \ \ \ \isacommand{using}\isamarkupfalse%
\ {\isacartoucheopen}W\ {\isasymin}\ {\isacharparenleft}colecComp\ S{\isacharparenright}{\isacartoucheclose}\ {\isacartoucheopen}Con\ F\ G\ H{\isacartoucheclose}\ {\isacartoucheopen}F\ {\isasymin}\ W{\isacartoucheclose}\ {\isacartoucheopen}finite\ Wo{\isacharprime}{\isacartoucheclose}\ {\isacartoucheopen}Wo{\isacharprime}\ {\isasymsubseteq}\ W{\isacartoucheclose}\ \isacommand{by}\isamarkupfalse%
\ {\isacharparenleft}rule\ pcp{\isacharunderscore}colecComp{\isacharunderscore}CON{\isacharunderscore}both{\isacharparenright}\isanewline
\ \ \ \ \ \ \ \ \ \ \isacommand{have}\isamarkupfalse%
\ {\isachardoublequoteopen}S{\isacharprime}\ {\isacharequal}\ {\isacharbraceleft}G{\isacharcomma}H{\isacharbraceright}\ {\isasymunion}\ Wo{\isacharprime}\ {\isasymor}\ S{\isacharprime}\ {\isacharequal}\ {\isacharbraceleft}G{\isacharbraceright}\ {\isasymunion}\ Wo{\isacharprime}\ {\isasymor}\ S{\isacharprime}\ {\isacharequal}\ {\isacharbraceleft}H{\isacharbraceright}\ {\isasymunion}\ Wo{\isacharprime}\ {\isasymor}\ S{\isacharprime}\ {\isacharequal}\ Wo{\isacharprime}{\isachardoublequoteclose}\isanewline
\ \ \ \ \ \ \ \ \ \ \ \ \isacommand{using}\isamarkupfalse%
\ {\isadigit{1}}\ \isacommand{by}\isamarkupfalse%
\ {\isacharparenleft}rule\ conjunct{\isadigit{2}}{\isacharparenright}\isanewline
\ \ \ \ \ \ \ \ \ \ \isacommand{thus}\isamarkupfalse%
\ {\isachardoublequoteopen}sat\ S{\isacharprime}{\isachardoublequoteclose}\isanewline
\ \ \ \ \ \ \ \ \ \ \isacommand{proof}\isamarkupfalse%
\ {\isacharparenleft}rule\ disjE{\isacharparenright}\isanewline
\ \ \ \ \ \ \ \ \ \ \ \ \isacommand{assume}\isamarkupfalse%
\ {\isachardoublequoteopen}S{\isacharprime}\ {\isacharequal}\ {\isacharbraceleft}G{\isacharcomma}H{\isacharbraceright}\ {\isasymunion}\ Wo{\isacharprime}{\isachardoublequoteclose}\isanewline
\ \ \ \ \ \ \ \ \ \ \ \ \isacommand{show}\isamarkupfalse%
\ {\isachardoublequoteopen}sat\ S{\isacharprime}{\isachardoublequoteclose}\isanewline
\ \ \ \ \ \ \ \ \ \ \ \ \ \ \isacommand{using}\isamarkupfalse%
\ {\isacartoucheopen}sat{\isacharparenleft}{\isacharbraceleft}G{\isacharcomma}H{\isacharbraceright}\ {\isasymunion}\ Wo{\isacharprime}{\isacharparenright}{\isacartoucheclose}\ \isacommand{by}\isamarkupfalse%
\ {\isacharparenleft}simp\ only{\isacharcolon}\ {\isacartoucheopen}S{\isacharprime}\ {\isacharequal}\ {\isacharbraceleft}G{\isacharcomma}H{\isacharbraceright}\ {\isasymunion}\ Wo{\isacharprime}{\isacartoucheclose}{\isacharparenright}\isanewline
\ \ \ \ \ \ \ \ \ \ \isacommand{next}\isamarkupfalse%
\isanewline
\ \ \ \ \ \ \ \ \ \ \ \ \isacommand{assume}\isamarkupfalse%
\ {\isachardoublequoteopen}S{\isacharprime}\ {\isacharequal}\ {\isacharbraceleft}G{\isacharbraceright}\ {\isasymunion}\ Wo{\isacharprime}\ {\isasymor}\ S{\isacharprime}\ {\isacharequal}\ {\isacharbraceleft}H{\isacharbraceright}\ {\isasymunion}\ Wo{\isacharprime}\ {\isasymor}\ S{\isacharprime}\ {\isacharequal}\ Wo{\isacharprime}{\isachardoublequoteclose}\isanewline
\ \ \ \ \ \ \ \ \ \ \ \ \isacommand{thus}\isamarkupfalse%
\ {\isachardoublequoteopen}sat\ S{\isacharprime}{\isachardoublequoteclose}\isanewline
\ \ \ \ \ \ \ \ \ \ \ \ \isacommand{proof}\isamarkupfalse%
\ {\isacharparenleft}rule\ disjE{\isacharparenright}\isanewline
\ \ \ \ \ \ \ \ \ \ \ \ \ \ \isacommand{assume}\isamarkupfalse%
\ {\isachardoublequoteopen}S{\isacharprime}\ {\isacharequal}\ {\isacharbraceleft}G{\isacharbraceright}\ {\isasymunion}\ Wo{\isacharprime}{\isachardoublequoteclose}\isanewline
\ \ \ \ \ \ \ \ \ \ \ \ \ \ \isacommand{then}\isamarkupfalse%
\ \isacommand{have}\isamarkupfalse%
\ {\isachardoublequoteopen}S{\isacharprime}\ {\isasymsubseteq}\ {\isacharbraceleft}G{\isacharcomma}H{\isacharbraceright}\ {\isasymunion}\ Wo{\isacharprime}{\isachardoublequoteclose}\isanewline
\ \ \ \ \ \ \ \ \ \ \ \ \ \ \ \ \isacommand{by}\isamarkupfalse%
\ blast\ \isanewline
\ \ \ \ \ \ \ \ \ \ \ \ \ \ \isacommand{thus}\isamarkupfalse%
\ {\isachardoublequoteopen}sat\ S{\isacharprime}{\isachardoublequoteclose}\isanewline
\ \ \ \ \ \ \ \ \ \ \ \ \ \ \ \ \isacommand{using}\isamarkupfalse%
\ {\isacartoucheopen}sat{\isacharparenleft}{\isacharbraceleft}G{\isacharcomma}H{\isacharbraceright}\ {\isasymunion}\ Wo{\isacharprime}{\isacharparenright}{\isacartoucheclose}\ \isacommand{by}\isamarkupfalse%
\ {\isacharparenleft}rule\ sat{\isacharunderscore}mono{\isacharparenright}\isanewline
\ \ \ \ \ \ \ \ \ \ \ \ \isacommand{next}\isamarkupfalse%
\isanewline
\ \ \ \ \ \ \ \ \ \ \ \ \ \ \isacommand{assume}\isamarkupfalse%
\ {\isachardoublequoteopen}S{\isacharprime}\ {\isacharequal}\ {\isacharbraceleft}H{\isacharbraceright}\ {\isasymunion}\ Wo{\isacharprime}\ {\isasymor}\ S{\isacharprime}\ {\isacharequal}\ Wo{\isacharprime}{\isachardoublequoteclose}\isanewline
\ \ \ \ \ \ \ \ \ \ \ \ \ \ \isacommand{thus}\isamarkupfalse%
\ {\isachardoublequoteopen}sat\ S{\isacharprime}{\isachardoublequoteclose}\isanewline
\ \ \ \ \ \ \ \ \ \ \ \ \ \ \isacommand{proof}\isamarkupfalse%
\ {\isacharparenleft}rule\ disjE{\isacharparenright}\isanewline
\ \ \ \ \ \ \ \ \ \ \ \ \ \ \ \ \isacommand{assume}\isamarkupfalse%
\ {\isachardoublequoteopen}S{\isacharprime}\ {\isacharequal}\ {\isacharbraceleft}H{\isacharbraceright}\ {\isasymunion}\ Wo{\isacharprime}{\isachardoublequoteclose}\isanewline
\ \ \ \ \ \ \ \ \ \ \ \ \ \ \ \ \isacommand{then}\isamarkupfalse%
\ \isacommand{have}\isamarkupfalse%
\ {\isachardoublequoteopen}S{\isacharprime}\ {\isasymsubseteq}\ {\isacharbraceleft}G{\isacharcomma}H{\isacharbraceright}\ {\isasymunion}\ Wo{\isacharprime}{\isachardoublequoteclose}\isanewline
\ \ \ \ \ \ \ \ \ \ \ \ \ \ \ \ \ \ \isacommand{by}\isamarkupfalse%
\ blast\ \isanewline
\ \ \ \ \ \ \ \ \ \ \ \ \ \ \ \ \isacommand{thus}\isamarkupfalse%
\ {\isachardoublequoteopen}sat\ S{\isacharprime}{\isachardoublequoteclose}\isanewline
\ \ \ \ \ \ \ \ \ \ \ \ \ \ \ \ \ \ \isacommand{using}\isamarkupfalse%
\ {\isacartoucheopen}sat{\isacharparenleft}{\isacharbraceleft}G{\isacharcomma}H{\isacharbraceright}\ {\isasymunion}\ Wo{\isacharprime}{\isacharparenright}{\isacartoucheclose}\ \isacommand{by}\isamarkupfalse%
\ {\isacharparenleft}rule\ sat{\isacharunderscore}mono{\isacharparenright}\isanewline
\ \ \ \ \ \ \ \ \ \ \ \ \ \ \isacommand{next}\isamarkupfalse%
\isanewline
\ \ \ \ \ \ \ \ \ \ \ \ \ \ \ \ \isacommand{assume}\isamarkupfalse%
\ {\isachardoublequoteopen}S{\isacharprime}\ {\isacharequal}\ Wo{\isacharprime}{\isachardoublequoteclose}\isanewline
\ \ \ \ \ \ \ \ \ \ \ \ \ \ \ \ \isacommand{then}\isamarkupfalse%
\ \isacommand{have}\isamarkupfalse%
\ {\isachardoublequoteopen}S{\isacharprime}\ {\isasymsubseteq}\ {\isacharbraceleft}G{\isacharcomma}H{\isacharbraceright}\ {\isasymunion}\ Wo{\isacharprime}{\isachardoublequoteclose}\isanewline
\ \ \ \ \ \ \ \ \ \ \ \ \ \ \ \ \ \ \isacommand{by}\isamarkupfalse%
\ blast\ \isanewline
\ \ \ \ \ \ \ \ \ \ \ \ \ \ \ \ \isacommand{thus}\isamarkupfalse%
\ {\isachardoublequoteopen}sat\ S{\isacharprime}{\isachardoublequoteclose}\isanewline
\ \ \ \ \ \ \ \ \ \ \ \ \ \ \ \ \ \ \isacommand{using}\isamarkupfalse%
\ {\isacartoucheopen}sat{\isacharparenleft}{\isacharbraceleft}G{\isacharcomma}H{\isacharbraceright}\ {\isasymunion}\ Wo{\isacharprime}{\isacharparenright}{\isacartoucheclose}\ \isacommand{by}\isamarkupfalse%
\ {\isacharparenleft}rule\ sat{\isacharunderscore}mono{\isacharparenright}\isanewline
\ \ \ \ \ \ \ \ \ \ \ \ \ \ \isacommand{qed}\isamarkupfalse%
\isanewline
\ \ \ \ \ \ \ \ \ \ \ \ \isacommand{qed}\isamarkupfalse%
\isanewline
\ \ \ \ \ \ \ \ \ \ \isacommand{qed}\isamarkupfalse%
\isanewline
\ \ \ \ \ \ \ \ \isacommand{qed}\isamarkupfalse%
\isanewline
\ \ \ \ \ \ \isacommand{qed}\isamarkupfalse%
\isanewline
\ \ \ \ \isacommand{qed}\isamarkupfalse%
\isanewline
\ \ \isacommand{qed}\isamarkupfalse%
\isanewline
\isacommand{qed}\isamarkupfalse%
%
\endisatagproof
{\isafoldproof}%
%
\isadelimproof
\isanewline
%
\endisadelimproof
\isanewline
\isanewline
\isanewline
\isacommand{lemma}\isamarkupfalse%
\ prop{\isacharunderscore}Compactness{\isacharcolon}\isanewline
\ \ \isakeyword{assumes}\ {\isachardoublequoteopen}fin{\isacharunderscore}sat\ S{\isachardoublequoteclose}\isanewline
\ \ \isakeyword{shows}\ {\isachardoublequoteopen}sat\ S{\isachardoublequoteclose}\isanewline
%
\isadelimproof
\ \ %
\endisadelimproof
%
\isatagproof
\isacommand{oops}\isamarkupfalse%
%
\endisatagproof
{\isafoldproof}%
%
\isadelimproof
\isanewline
%
\endisadelimproof
\isanewline
\isacommand{lemma}\isamarkupfalse%
\ prop{\isacharunderscore}Compactness{\isacharunderscore}aut{\isacharcolon}{\isachardoublequoteopen}fin{\isacharunderscore}sat\ S\ {\isasymLongrightarrow}\ sat\ S{\isachardoublequoteclose}\isanewline
%
\isadelimproof
\ \ %
\endisadelimproof
%
\isatagproof
\isacommand{oops}\isamarkupfalse%
\isanewline
%
\endisatagproof
{\isafoldproof}%
%
\isadelimproof
%
\endisadelimproof
%
\isadelimtheory
%
\endisadelimtheory
%
\isatagtheory
%
\endisatagtheory
{\isafoldtheory}%
%
\isadelimtheory
%
\endisadelimtheory
%
\end{isabellebody}%
\endinput
%:%file=~/TFM/TFM/TeoremaEx.thy%:%
%:%19=13%:%
%:%23=15%:%
%:%32=17%:%
%:%44=19%:%
%:%45=20%:%
%:%46=21%:%
%:%47=22%:%
%:%48=23%:%
%:%49=24%:%
%:%50=25%:%
%:%51=26%:%
%:%52=27%:%
%:%53=28%:%
%:%54=29%:%
%:%55=30%:%
%:%56=31%:%
%:%57=32%:%
%:%58=33%:%
%:%59=34%:%
%:%60=35%:%
%:%61=36%:%
%:%62=37%:%
%:%63=38%:%
%:%64=39%:%
%:%65=40%:%
%:%66=41%:%
%:%67=42%:%
%:%68=43%:%
%:%69=44%:%
%:%70=45%:%
%:%71=46%:%
%:%72=47%:%
%:%73=48%:%
%:%74=49%:%
%:%75=50%:%
%:%76=51%:%
%:%77=52%:%
%:%78=53%:%
%:%79=54%:%
%:%80=55%:%
%:%82=57%:%
%:%83=57%:%
%:%84=58%:%
%:%85=59%:%
%:%88=62%:%
%:%89=63%:%
%:%90=64%:%
%:%91=65%:%
%:%92=66%:%
%:%93=67%:%
%:%94=68%:%
%:%95=69%:%
%:%96=70%:%
%:%97=71%:%
%:%98=72%:%
%:%99=73%:%
%:%100=74%:%
%:%101=75%:%
%:%102=76%:%
%:%103=77%:%
%:%104=78%:%
%:%105=79%:%
%:%106=80%:%
%:%107=81%:%
%:%108=82%:%
%:%109=83%:%
%:%110=84%:%
%:%112=86%:%
%:%113=86%:%
%:%114=87%:%
%:%115=88%:%
%:%116=89%:%
%:%123=90%:%
%:%124=90%:%
%:%125=91%:%
%:%126=91%:%
%:%127=92%:%
%:%128=92%:%
%:%129=93%:%
%:%130=93%:%
%:%131=94%:%
%:%132=94%:%
%:%133=95%:%
%:%134=95%:%
%:%135=96%:%
%:%136=96%:%
%:%137=97%:%
%:%138=98%:%
%:%139=98%:%
%:%140=99%:%
%:%141=99%:%
%:%142=99%:%
%:%143=100%:%
%:%144=101%:%
%:%145=101%:%
%:%146=102%:%
%:%147=102%:%
%:%148=103%:%
%:%149=103%:%
%:%150=104%:%
%:%151=104%:%
%:%152=105%:%
%:%153=105%:%
%:%154=106%:%
%:%155=106%:%
%:%156=106%:%
%:%157=107%:%
%:%158=107%:%
%:%159=108%:%
%:%160=108%:%
%:%161=109%:%
%:%162=109%:%
%:%163=110%:%
%:%164=110%:%
%:%165=111%:%
%:%166=111%:%
%:%167=112%:%
%:%168=112%:%
%:%169=112%:%
%:%170=113%:%
%:%171=113%:%
%:%172=114%:%
%:%173=114%:%
%:%174=115%:%
%:%175=115%:%
%:%176=116%:%
%:%186=118%:%
%:%188=120%:%
%:%189=120%:%
%:%196=121%:%
%:%197=121%:%
%:%198=122%:%
%:%199=122%:%
%:%200=123%:%
%:%201=123%:%
%:%202=123%:%
%:%203=124%:%
%:%204=124%:%
%:%205=124%:%
%:%206=125%:%
%:%207=125%:%
%:%216=127%:%
%:%217=128%:%
%:%218=129%:%
%:%219=130%:%
%:%220=131%:%
%:%221=132%:%
%:%222=133%:%
%:%223=134%:%
%:%225=136%:%
%:%226=136%:%
%:%229=137%:%
%:%233=137%:%
%:%243=139%:%
%:%244=140%:%
%:%245=141%:%
%:%246=142%:%
%:%247=143%:%
%:%248=144%:%
%:%249=145%:%
%:%250=146%:%
%:%251=147%:%
%:%252=148%:%
%:%253=149%:%
%:%254=150%:%
%:%255=151%:%
%:%257=153%:%
%:%258=153%:%
%:%265=154%:%
%:%266=154%:%
%:%267=155%:%
%:%268=155%:%
%:%269=156%:%
%:%270=157%:%
%:%271=157%:%
%:%272=158%:%
%:%273=158%:%
%:%274=158%:%
%:%275=159%:%
%:%276=160%:%
%:%277=160%:%
%:%278=161%:%
%:%279=161%:%
%:%280=162%:%
%:%281=162%:%
%:%282=163%:%
%:%283=163%:%
%:%284=164%:%
%:%285=164%:%
%:%286=165%:%
%:%287=165%:%
%:%288=165%:%
%:%289=166%:%
%:%290=166%:%
%:%291=167%:%
%:%292=167%:%
%:%293=168%:%
%:%294=168%:%
%:%295=169%:%
%:%296=169%:%
%:%297=170%:%
%:%298=170%:%
%:%299=171%:%
%:%300=171%:%
%:%301=171%:%
%:%302=172%:%
%:%303=172%:%
%:%304=173%:%
%:%305=173%:%
%:%306=174%:%
%:%307=174%:%
%:%308=175%:%
%:%318=177%:%
%:%320=179%:%
%:%321=179%:%
%:%324=180%:%
%:%328=180%:%
%:%329=180%:%
%:%338=182%:%
%:%339=183%:%
%:%341=185%:%
%:%342=185%:%
%:%343=186%:%
%:%344=187%:%
%:%347=188%:%
%:%351=188%:%
%:%352=188%:%
%:%353=188%:%
%:%362=190%:%
%:%363=191%:%
%:%364=192%:%
%:%365=193%:%
%:%366=194%:%
%:%367=195%:%
%:%368=196%:%
%:%369=197%:%
%:%370=198%:%
%:%371=199%:%
%:%372=200%:%
%:%373=201%:%
%:%374=202%:%
%:%375=203%:%
%:%376=204%:%
%:%377=205%:%
%:%378=206%:%
%:%379=207%:%
%:%380=208%:%
%:%381=209%:%
%:%382=210%:%
%:%383=211%:%
%:%384=212%:%
%:%385=213%:%
%:%386=214%:%
%:%387=215%:%
%:%388=216%:%
%:%389=217%:%
%:%390=218%:%
%:%391=219%:%
%:%392=220%:%
%:%393=221%:%
%:%394=222%:%
%:%395=223%:%
%:%396=224%:%
%:%397=225%:%
%:%398=226%:%
%:%399=227%:%
%:%400=228%:%
%:%401=229%:%
%:%402=230%:%
%:%403=231%:%
%:%404=232%:%
%:%406=234%:%
%:%407=234%:%
%:%414=235%:%
%:%415=235%:%
%:%416=236%:%
%:%417=236%:%
%:%418=237%:%
%:%419=237%:%
%:%420=238%:%
%:%421=238%:%
%:%422=238%:%
%:%423=239%:%
%:%424=239%:%
%:%425=240%:%
%:%426=240%:%
%:%427=240%:%
%:%428=241%:%
%:%429=241%:%
%:%430=242%:%
%:%431=242%:%
%:%432=242%:%
%:%433=243%:%
%:%434=243%:%
%:%435=244%:%
%:%436=244%:%
%:%437=244%:%
%:%438=245%:%
%:%439=245%:%
%:%440=246%:%
%:%441=246%:%
%:%442=247%:%
%:%443=247%:%
%:%444=248%:%
%:%445=248%:%
%:%446=249%:%
%:%447=249%:%
%:%448=250%:%
%:%449=250%:%
%:%450=251%:%
%:%451=251%:%
%:%452=251%:%
%:%453=252%:%
%:%454=252%:%
%:%455=253%:%
%:%456=253%:%
%:%457=254%:%
%:%458=254%:%
%:%459=255%:%
%:%460=255%:%
%:%461=255%:%
%:%462=256%:%
%:%463=256%:%
%:%464=257%:%
%:%465=257%:%
%:%466=257%:%
%:%467=258%:%
%:%468=258%:%
%:%469=259%:%
%:%470=259%:%
%:%471=259%:%
%:%472=260%:%
%:%473=260%:%
%:%474=261%:%
%:%475=261%:%
%:%476=261%:%
%:%477=262%:%
%:%478=262%:%
%:%479=263%:%
%:%480=263%:%
%:%481=263%:%
%:%482=264%:%
%:%483=264%:%
%:%484=265%:%
%:%485=265%:%
%:%486=265%:%
%:%487=266%:%
%:%488=266%:%
%:%489=266%:%
%:%490=267%:%
%:%491=267%:%
%:%492=267%:%
%:%493=268%:%
%:%494=268%:%
%:%495=269%:%
%:%505=271%:%
%:%507=273%:%
%:%508=273%:%
%:%515=274%:%
%:%516=274%:%
%:%517=275%:%
%:%518=275%:%
%:%519=276%:%
%:%520=276%:%
%:%521=277%:%
%:%522=277%:%
%:%523=277%:%
%:%524=278%:%
%:%525=278%:%
%:%526=279%:%
%:%527=279%:%
%:%528=280%:%
%:%529=280%:%
%:%530=281%:%
%:%531=281%:%
%:%532=282%:%
%:%533=282%:%
%:%534=282%:%
%:%535=282%:%
%:%536=283%:%
%:%537=283%:%
%:%546=285%:%
%:%547=286%:%
%:%548=287%:%
%:%549=288%:%
%:%550=289%:%
%:%551=290%:%
%:%552=291%:%
%:%553=292%:%
%:%554=293%:%
%:%556=295%:%
%:%557=295%:%
%:%559=297%:%
%:%560=298%:%
%:%561=299%:%
%:%562=300%:%
%:%563=301%:%
%:%564=302%:%
%:%565=303%:%
%:%566=304%:%
%:%567=305%:%
%:%568=306%:%
%:%569=307%:%
%:%570=308%:%
%:%571=309%:%
%:%572=310%:%
%:%573=311%:%
%:%574=312%:%
%:%576=314%:%
%:%577=314%:%
%:%580=315%:%
%:%584=315%:%
%:%585=315%:%
%:%586=316%:%
%:%587=316%:%
%:%588=317%:%
%:%589=317%:%
%:%590=318%:%
%:%591=318%:%
%:%592=319%:%
%:%593=319%:%
%:%594=319%:%
%:%595=320%:%
%:%596=320%:%
%:%597=321%:%
%:%598=321%:%
%:%599=321%:%
%:%600=322%:%
%:%601=322%:%
%:%602=323%:%
%:%603=323%:%
%:%604=324%:%
%:%605=324%:%
%:%606=325%:%
%:%616=327%:%
%:%618=329%:%
%:%619=329%:%
%:%622=330%:%
%:%626=330%:%
%:%627=330%:%
%:%628=330%:%
%:%637=332%:%
%:%638=333%:%
%:%639=334%:%
%:%640=335%:%
%:%641=336%:%
%:%642=337%:%
%:%643=338%:%
%:%644=339%:%
%:%645=340%:%
%:%646=341%:%
%:%647=342%:%
%:%648=343%:%
%:%649=344%:%
%:%650=345%:%
%:%651=346%:%
%:%652=347%:%
%:%654=349%:%
%:%655=349%:%
%:%656=350%:%
%:%657=351%:%
%:%664=352%:%
%:%665=352%:%
%:%666=353%:%
%:%667=353%:%
%:%668=354%:%
%:%669=354%:%
%:%670=354%:%
%:%671=355%:%
%:%672=355%:%
%:%673=355%:%
%:%674=356%:%
%:%675=356%:%
%:%676=357%:%
%:%677=357%:%
%:%678=357%:%
%:%679=358%:%
%:%680=358%:%
%:%681=359%:%
%:%682=359%:%
%:%683=360%:%
%:%684=360%:%
%:%685=361%:%
%:%695=363%:%
%:%697=365%:%
%:%698=365%:%
%:%699=366%:%
%:%702=367%:%
%:%706=367%:%
%:%707=367%:%
%:%708=367%:%
%:%717=369%:%
%:%718=370%:%
%:%719=371%:%
%:%720=372%:%
%:%721=373%:%
%:%722=374%:%
%:%723=375%:%
%:%724=376%:%
%:%725=377%:%
%:%726=378%:%
%:%727=379%:%
%:%728=380%:%
%:%729=381%:%
%:%730=382%:%
%:%731=383%:%
%:%732=384%:%
%:%733=385%:%
%:%734=386%:%
%:%735=387%:%
%:%736=388%:%
%:%737=389%:%
%:%738=390%:%
%:%739=391%:%
%:%740=392%:%
%:%741=393%:%
%:%742=394%:%
%:%743=395%:%
%:%744=396%:%
%:%745=397%:%
%:%746=398%:%
%:%747=399%:%
%:%748=400%:%
%:%749=401%:%
%:%750=402%:%
%:%751=403%:%
%:%753=405%:%
%:%754=405%:%
%:%755=406%:%
%:%756=407%:%
%:%757=408%:%
%:%760=409%:%
%:%764=409%:%
%:%765=409%:%
%:%766=410%:%
%:%767=410%:%
%:%768=411%:%
%:%769=411%:%
%:%770=412%:%
%:%771=412%:%
%:%772=413%:%
%:%773=413%:%
%:%774=414%:%
%:%775=414%:%
%:%776=415%:%
%:%777=415%:%
%:%778=416%:%
%:%779=416%:%
%:%780=416%:%
%:%781=417%:%
%:%782=417%:%
%:%783=418%:%
%:%784=418%:%
%:%785=418%:%
%:%786=419%:%
%:%787=419%:%
%:%788=420%:%
%:%789=420%:%
%:%790=421%:%
%:%791=421%:%
%:%792=422%:%
%:%793=422%:%
%:%794=422%:%
%:%795=423%:%
%:%796=423%:%
%:%797=424%:%
%:%798=424%:%
%:%799=424%:%
%:%800=425%:%
%:%801=425%:%
%:%802=426%:%
%:%803=426%:%
%:%804=426%:%
%:%805=427%:%
%:%806=427%:%
%:%807=428%:%
%:%808=428%:%
%:%809=428%:%
%:%810=429%:%
%:%811=429%:%
%:%812=430%:%
%:%813=430%:%
%:%814=431%:%
%:%815=431%:%
%:%816=431%:%
%:%817=432%:%
%:%818=432%:%
%:%819=433%:%
%:%820=433%:%
%:%821=433%:%
%:%822=434%:%
%:%823=434%:%
%:%824=434%:%
%:%825=435%:%
%:%826=435%:%
%:%827=436%:%
%:%828=436%:%
%:%829=437%:%
%:%830=437%:%
%:%831=437%:%
%:%832=438%:%
%:%833=438%:%
%:%834=439%:%
%:%835=439%:%
%:%836=440%:%
%:%837=440%:%
%:%838=440%:%
%:%839=441%:%
%:%840=441%:%
%:%841=442%:%
%:%842=442%:%
%:%843=443%:%
%:%844=443%:%
%:%845=444%:%
%:%846=444%:%
%:%847=444%:%
%:%848=445%:%
%:%849=445%:%
%:%850=446%:%
%:%851=446%:%
%:%852=447%:%
%:%853=447%:%
%:%854=447%:%
%:%855=448%:%
%:%856=448%:%
%:%857=449%:%
%:%858=449%:%
%:%859=449%:%
%:%860=450%:%
%:%861=450%:%
%:%862=450%:%
%:%863=451%:%
%:%864=451%:%
%:%865=451%:%
%:%866=452%:%
%:%867=452%:%
%:%868=453%:%
%:%869=453%:%
%:%870=454%:%
%:%880=456%:%
%:%882=458%:%
%:%883=458%:%
%:%884=459%:%
%:%885=460%:%
%:%886=461%:%
%:%889=462%:%
%:%893=462%:%
%:%894=462%:%
%:%895=463%:%
%:%896=463%:%
%:%897=464%:%
%:%898=464%:%
%:%899=465%:%
%:%900=465%:%
%:%901=465%:%
%:%902=466%:%
%:%903=466%:%
%:%904=466%:%
%:%905=466%:%
%:%906=467%:%
%:%907=467%:%
%:%908=467%:%
%:%909=468%:%
%:%910=468%:%
%:%911=469%:%
%:%912=469%:%
%:%913=469%:%
%:%914=470%:%
%:%915=470%:%
%:%916=471%:%
%:%917=471%:%
%:%918=471%:%
%:%919=472%:%
%:%920=472%:%
%:%934=474%:%
%:%946=476%:%
%:%947=477%:%
%:%948=478%:%
%:%949=479%:%
%:%950=480%:%
%:%951=481%:%
%:%952=482%:%
%:%953=483%:%
%:%954=484%:%
%:%954=485%:%
%:%955=486%:%
%:%956=487%:%
%:%957=488%:%
%:%961=490%:%
%:%962=491%:%
%:%963=492%:%
%:%964=493%:%
%:%965=494%:%
%:%966=495%:%
%:%967=496%:%
%:%968=497%:%
%:%969=498%:%
%:%970=499%:%
%:%971=500%:%
%:%972=501%:%
%:%973=502%:%
%:%974=503%:%
%:%975=504%:%
%:%976=505%:%
%:%977=506%:%
%:%978=507%:%
%:%979=508%:%
%:%980=509%:%
%:%981=510%:%
%:%982=511%:%
%:%983=512%:%
%:%984=513%:%
%:%985=514%:%
%:%986=515%:%
%:%987=516%:%
%:%989=518%:%
%:%990=518%:%
%:%991=519%:%
%:%992=520%:%
%:%993=521%:%
%:%994=522%:%
%:%995=523%:%
%:%1002=524%:%
%:%1003=524%:%
%:%1004=525%:%
%:%1005=525%:%
%:%1006=526%:%
%:%1007=526%:%
%:%1008=526%:%
%:%1009=526%:%
%:%1010=527%:%
%:%1011=527%:%
%:%1012=527%:%
%:%1013=528%:%
%:%1014=528%:%
%:%1015=529%:%
%:%1016=529%:%
%:%1017=530%:%
%:%1018=530%:%
%:%1019=530%:%
%:%1020=530%:%
%:%1021=531%:%
%:%1022=531%:%
%:%1023=532%:%
%:%1024=532%:%
%:%1025=533%:%
%:%1026=533%:%
%:%1027=534%:%
%:%1028=534%:%
%:%1029=535%:%
%:%1030=535%:%
%:%1031=536%:%
%:%1032=536%:%
%:%1033=537%:%
%:%1034=537%:%
%:%1035=538%:%
%:%1036=538%:%
%:%1037=538%:%
%:%1038=539%:%
%:%1039=539%:%
%:%1040=539%:%
%:%1041=540%:%
%:%1042=540%:%
%:%1043=541%:%
%:%1044=541%:%
%:%1045=541%:%
%:%1046=542%:%
%:%1047=542%:%
%:%1048=543%:%
%:%1049=543%:%
%:%1050=543%:%
%:%1051=544%:%
%:%1052=544%:%
%:%1053=545%:%
%:%1054=545%:%
%:%1055=545%:%
%:%1056=546%:%
%:%1057=546%:%
%:%1058=547%:%
%:%1059=547%:%
%:%1060=547%:%
%:%1061=548%:%
%:%1062=548%:%
%:%1063=549%:%
%:%1064=549%:%
%:%1065=550%:%
%:%1066=550%:%
%:%1067=551%:%
%:%1068=551%:%
%:%1069=551%:%
%:%1070=552%:%
%:%1080=554%:%
%:%1082=556%:%
%:%1083=556%:%
%:%1084=557%:%
%:%1085=558%:%
%:%1086=559%:%
%:%1087=560%:%
%:%1088=561%:%
%:%1095=562%:%
%:%1096=562%:%
%:%1097=563%:%
%:%1098=563%:%
%:%1099=563%:%
%:%1100=563%:%
%:%1101=564%:%
%:%1102=564%:%
%:%1103=564%:%
%:%1104=564%:%
%:%1105=565%:%
%:%1106=565%:%
%:%1107=566%:%
%:%1108=566%:%
%:%1109=567%:%
%:%1110=567%:%
%:%1111=568%:%
%:%1112=568%:%
%:%1113=569%:%
%:%1114=569%:%
%:%1115=569%:%
%:%1116=570%:%
%:%1117=570%:%
%:%1118=571%:%
%:%1119=571%:%
%:%1120=572%:%
%:%1121=572%:%
%:%1122=573%:%
%:%1123=573%:%
%:%1124=574%:%
%:%1125=574%:%
%:%1126=574%:%
%:%1127=575%:%
%:%1128=575%:%
%:%1129=575%:%
%:%1130=575%:%
%:%1131=576%:%
%:%1132=576%:%
%:%1133=576%:%
%:%1134=577%:%
%:%1135=577%:%
%:%1136=578%:%
%:%1137=578%:%
%:%1138=578%:%
%:%1139=579%:%
%:%1140=579%:%
%:%1141=580%:%
%:%1142=580%:%
%:%1143=580%:%
%:%1144=581%:%
%:%1145=581%:%
%:%1146=582%:%
%:%1147=582%:%
%:%1148=583%:%
%:%1149=583%:%
%:%1150=583%:%
%:%1151=583%:%
%:%1152=584%:%
%:%1153=584%:%
%:%1154=585%:%
%:%1155=585%:%
%:%1156=585%:%
%:%1157=585%:%
%:%1158=585%:%
%:%1159=586%:%
%:%1169=588%:%
%:%1170=589%:%
%:%1171=590%:%
%:%1172=591%:%
%:%1173=592%:%
%:%1174=593%:%
%:%1175=594%:%
%:%1176=595%:%
%:%1177=596%:%
%:%1178=597%:%
%:%1179=598%:%
%:%1180=599%:%
%:%1181=600%:%
%:%1182=601%:%
%:%1183=602%:%
%:%1184=603%:%
%:%1185=604%:%
%:%1186=605%:%
%:%1187=606%:%
%:%1188=607%:%
%:%1189=608%:%
%:%1190=609%:%
%:%1191=610%:%
%:%1192=611%:%
%:%1193=612%:%
%:%1194=613%:%
%:%1195=614%:%
%:%1196=615%:%
%:%1197=616%:%
%:%1198=617%:%
%:%1199=618%:%
%:%1200=619%:%
%:%1201=620%:%
%:%1202=621%:%
%:%1204=623%:%
%:%1205=623%:%
%:%1206=624%:%
%:%1207=625%:%
%:%1208=626%:%
%:%1209=627%:%
%:%1210=628%:%
%:%1217=629%:%
%:%1218=629%:%
%:%1219=630%:%
%:%1220=630%:%
%:%1221=631%:%
%:%1222=631%:%
%:%1223=632%:%
%:%1224=632%:%
%:%1225=632%:%
%:%1226=633%:%
%:%1227=633%:%
%:%1228=634%:%
%:%1229=634%:%
%:%1230=635%:%
%:%1231=635%:%
%:%1232=635%:%
%:%1233=636%:%
%:%1234=636%:%
%:%1235=636%:%
%:%1236=637%:%
%:%1237=637%:%
%:%1238=637%:%
%:%1239=638%:%
%:%1240=638%:%
%:%1241=639%:%
%:%1242=639%:%
%:%1243=639%:%
%:%1244=640%:%
%:%1245=640%:%
%:%1246=641%:%
%:%1247=641%:%
%:%1248=642%:%
%:%1249=642%:%
%:%1250=642%:%
%:%1251=643%:%
%:%1252=643%:%
%:%1253=644%:%
%:%1254=644%:%
%:%1255=644%:%
%:%1256=645%:%
%:%1257=645%:%
%:%1258=646%:%
%:%1259=646%:%
%:%1260=646%:%
%:%1261=647%:%
%:%1262=647%:%
%:%1263=648%:%
%:%1264=648%:%
%:%1265=649%:%
%:%1266=649%:%
%:%1267=649%:%
%:%1268=650%:%
%:%1269=650%:%
%:%1270=651%:%
%:%1271=651%:%
%:%1272=651%:%
%:%1273=652%:%
%:%1274=652%:%
%:%1275=652%:%
%:%1276=653%:%
%:%1277=653%:%
%:%1278=654%:%
%:%1279=654%:%
%:%1280=655%:%
%:%1281=655%:%
%:%1282=656%:%
%:%1283=656%:%
%:%1284=656%:%
%:%1285=657%:%
%:%1286=657%:%
%:%1287=658%:%
%:%1288=658%:%
%:%1289=658%:%
%:%1290=659%:%
%:%1291=659%:%
%:%1292=660%:%
%:%1293=660%:%
%:%1294=661%:%
%:%1295=662%:%
%:%1296=662%:%
%:%1297=663%:%
%:%1298=663%:%
%:%1299=663%:%
%:%1300=664%:%
%:%1301=665%:%
%:%1302=665%:%
%:%1303=666%:%
%:%1304=666%:%
%:%1305=666%:%
%:%1306=667%:%
%:%1307=667%:%
%:%1308=667%:%
%:%1309=668%:%
%:%1310=668%:%
%:%1311=668%:%
%:%1312=669%:%
%:%1313=669%:%
%:%1314=670%:%
%:%1315=670%:%
%:%1316=670%:%
%:%1317=671%:%
%:%1318=671%:%
%:%1319=672%:%
%:%1320=672%:%
%:%1321=673%:%
%:%1322=673%:%
%:%1323=673%:%
%:%1324=674%:%
%:%1325=674%:%
%:%1326=675%:%
%:%1327=675%:%
%:%1328=676%:%
%:%1329=676%:%
%:%1330=677%:%
%:%1331=677%:%
%:%1332=677%:%
%:%1333=678%:%
%:%1334=678%:%
%:%1335=678%:%
%:%1336=679%:%
%:%1337=679%:%
%:%1338=679%:%
%:%1339=680%:%
%:%1340=680%:%
%:%1341=680%:%
%:%1342=681%:%
%:%1343=681%:%
%:%1344=682%:%
%:%1345=682%:%
%:%1346=682%:%
%:%1347=683%:%
%:%1348=683%:%
%:%1349=683%:%
%:%1350=684%:%
%:%1351=684%:%
%:%1352=684%:%
%:%1353=685%:%
%:%1354=685%:%
%:%1355=685%:%
%:%1356=686%:%
%:%1357=686%:%
%:%1358=686%:%
%:%1359=687%:%
%:%1360=687%:%
%:%1361=688%:%
%:%1362=688%:%
%:%1363=688%:%
%:%1364=689%:%
%:%1374=691%:%
%:%1376=693%:%
%:%1377=693%:%
%:%1378=694%:%
%:%1379=695%:%
%:%1380=696%:%
%:%1381=697%:%
%:%1382=698%:%
%:%1389=699%:%
%:%1390=699%:%
%:%1391=700%:%
%:%1392=700%:%
%:%1393=701%:%
%:%1394=701%:%
%:%1395=701%:%
%:%1396=701%:%
%:%1397=702%:%
%:%1398=702%:%
%:%1399=702%:%
%:%1400=702%:%
%:%1401=703%:%
%:%1402=703%:%
%:%1403=703%:%
%:%1404=703%:%
%:%1405=703%:%
%:%1406=704%:%
%:%1407=704%:%
%:%1408=704%:%
%:%1409=705%:%
%:%1410=705%:%
%:%1411=705%:%
%:%1412=705%:%
%:%1413=706%:%
%:%1414=706%:%
%:%1415=706%:%
%:%1416=707%:%
%:%1417=707%:%
%:%1418=707%:%
%:%1419=707%:%
%:%1420=708%:%
%:%1421=708%:%
%:%1422=708%:%
%:%1423=708%:%
%:%1424=709%:%
%:%1434=711%:%
%:%1435=712%:%
%:%1436=713%:%
%:%1437=714%:%
%:%1438=715%:%
%:%1439=716%:%
%:%1440=717%:%
%:%1441=718%:%
%:%1442=719%:%
%:%1443=720%:%
%:%1444=721%:%
%:%1445=722%:%
%:%1446=723%:%
%:%1447=724%:%
%:%1448=725%:%
%:%1449=726%:%
%:%1450=727%:%
%:%1451=728%:%
%:%1452=729%:%
%:%1453=730%:%
%:%1454=731%:%
%:%1455=732%:%
%:%1457=734%:%
%:%1458=734%:%
%:%1459=735%:%
%:%1460=736%:%
%:%1461=737%:%
%:%1468=738%:%
%:%1469=738%:%
%:%1470=739%:%
%:%1471=739%:%
%:%1472=740%:%
%:%1473=740%:%
%:%1474=741%:%
%:%1475=741%:%
%:%1476=742%:%
%:%1477=742%:%
%:%1478=743%:%
%:%1479=743%:%
%:%1480=743%:%
%:%1481=744%:%
%:%1482=744%:%
%:%1483=744%:%
%:%1484=745%:%
%:%1485=745%:%
%:%1486=746%:%
%:%1487=746%:%
%:%1488=746%:%
%:%1489=747%:%
%:%1490=747%:%
%:%1491=748%:%
%:%1492=748%:%
%:%1493=749%:%
%:%1494=749%:%
%:%1495=750%:%
%:%1505=752%:%
%:%1507=754%:%
%:%1508=754%:%
%:%1509=755%:%
%:%1510=756%:%
%:%1511=757%:%
%:%1514=758%:%
%:%1518=758%:%
%:%1519=758%:%
%:%1520=758%:%
%:%1529=760%:%
%:%1530=761%:%
%:%1531=762%:%
%:%1532=763%:%
%:%1533=764%:%
%:%1534=765%:%
%:%1535=766%:%
%:%1536=767%:%
%:%1537=768%:%
%:%1538=769%:%
%:%1539=770%:%
%:%1540=771%:%
%:%1541=772%:%
%:%1542=773%:%
%:%1543=774%:%
%:%1544=775%:%
%:%1545=776%:%
%:%1546=777%:%
%:%1547=778%:%
%:%1548=779%:%
%:%1549=780%:%
%:%1550=781%:%
%:%1551=782%:%
%:%1552=783%:%
%:%1553=784%:%
%:%1554=785%:%
%:%1555=786%:%
%:%1556=787%:%
%:%1557=788%:%
%:%1558=789%:%
%:%1559=790%:%
%:%1560=791%:%
%:%1561=792%:%
%:%1562=793%:%
%:%1563=794%:%
%:%1564=795%:%
%:%1565=796%:%
%:%1566=797%:%
%:%1567=798%:%
%:%1568=799%:%
%:%1569=800%:%
%:%1570=801%:%
%:%1571=802%:%
%:%1572=803%:%
%:%1573=804%:%
%:%1574=805%:%
%:%1575=806%:%
%:%1576=807%:%
%:%1577=808%:%
%:%1578=809%:%
%:%1579=810%:%
%:%1580=811%:%
%:%1581=812%:%
%:%1582=813%:%
%:%1583=814%:%
%:%1584=815%:%
%:%1585=816%:%
%:%1586=817%:%
%:%1587=818%:%
%:%1588=819%:%
%:%1589=820%:%
%:%1590=821%:%
%:%1591=822%:%
%:%1592=823%:%
%:%1593=824%:%
%:%1594=825%:%
%:%1595=826%:%
%:%1596=827%:%
%:%1597=828%:%
%:%1598=829%:%
%:%1599=830%:%
%:%1600=831%:%
%:%1602=833%:%
%:%1603=833%:%
%:%1604=834%:%
%:%1605=835%:%
%:%1606=836%:%
%:%1607=837%:%
%:%1608=838%:%
%:%1615=839%:%
%:%1616=839%:%
%:%1617=840%:%
%:%1618=840%:%
%:%1619=841%:%
%:%1620=841%:%
%:%1621=842%:%
%:%1622=842%:%
%:%1623=842%:%
%:%1624=843%:%
%:%1625=843%:%
%:%1629=847%:%
%:%1630=848%:%
%:%1631=848%:%
%:%1632=848%:%
%:%1633=849%:%
%:%1634=849%:%
%:%1635=849%:%
%:%1638=852%:%
%:%1639=853%:%
%:%1640=853%:%
%:%1641=853%:%
%:%1642=854%:%
%:%1643=854%:%
%:%1644=854%:%
%:%1645=855%:%
%:%1646=855%:%
%:%1647=856%:%
%:%1648=856%:%
%:%1649=857%:%
%:%1650=857%:%
%:%1651=857%:%
%:%1652=858%:%
%:%1653=858%:%
%:%1654=859%:%
%:%1655=859%:%
%:%1656=859%:%
%:%1657=860%:%
%:%1658=860%:%
%:%1659=861%:%
%:%1660=861%:%
%:%1661=862%:%
%:%1662=862%:%
%:%1663=863%:%
%:%1664=863%:%
%:%1665=864%:%
%:%1666=864%:%
%:%1667=865%:%
%:%1668=865%:%
%:%1669=866%:%
%:%1670=866%:%
%:%1671=867%:%
%:%1672=867%:%
%:%1673=868%:%
%:%1674=868%:%
%:%1675=868%:%
%:%1676=869%:%
%:%1677=869%:%
%:%1678=869%:%
%:%1679=870%:%
%:%1680=870%:%
%:%1681=870%:%
%:%1682=871%:%
%:%1683=871%:%
%:%1684=871%:%
%:%1685=872%:%
%:%1686=872%:%
%:%1687=872%:%
%:%1688=873%:%
%:%1689=873%:%
%:%1690=874%:%
%:%1691=874%:%
%:%1692=875%:%
%:%1693=875%:%
%:%1694=875%:%
%:%1695=876%:%
%:%1696=876%:%
%:%1697=876%:%
%:%1698=877%:%
%:%1699=877%:%
%:%1700=878%:%
%:%1701=878%:%
%:%1702=879%:%
%:%1703=879%:%
%:%1704=879%:%
%:%1705=880%:%
%:%1706=880%:%
%:%1707=881%:%
%:%1708=881%:%
%:%1709=882%:%
%:%1710=882%:%
%:%1711=882%:%
%:%1712=883%:%
%:%1713=883%:%
%:%1714=883%:%
%:%1715=884%:%
%:%1716=884%:%
%:%1717=885%:%
%:%1718=885%:%
%:%1719=885%:%
%:%1720=886%:%
%:%1721=886%:%
%:%1722=887%:%
%:%1723=887%:%
%:%1724=887%:%
%:%1725=888%:%
%:%1726=888%:%
%:%1727=889%:%
%:%1728=889%:%
%:%1729=890%:%
%:%1730=890%:%
%:%1731=890%:%
%:%1732=891%:%
%:%1733=891%:%
%:%1734=891%:%
%:%1735=892%:%
%:%1736=892%:%
%:%1737=893%:%
%:%1738=893%:%
%:%1739=893%:%
%:%1740=894%:%
%:%1741=894%:%
%:%1742=895%:%
%:%1743=895%:%
%:%1744=896%:%
%:%1745=896%:%
%:%1746=896%:%
%:%1747=897%:%
%:%1748=897%:%
%:%1749=898%:%
%:%1750=898%:%
%:%1751=899%:%
%:%1752=899%:%
%:%1753=900%:%
%:%1754=900%:%
%:%1755=900%:%
%:%1756=901%:%
%:%1757=901%:%
%:%1758=902%:%
%:%1759=902%:%
%:%1760=903%:%
%:%1761=903%:%
%:%1762=904%:%
%:%1763=904%:%
%:%1764=905%:%
%:%1765=905%:%
%:%1766=906%:%
%:%1767=906%:%
%:%1768=907%:%
%:%1769=907%:%
%:%1770=908%:%
%:%1771=908%:%
%:%1772=909%:%
%:%1773=909%:%
%:%1774=909%:%
%:%1775=910%:%
%:%1776=910%:%
%:%1777=910%:%
%:%1778=911%:%
%:%1779=911%:%
%:%1780=911%:%
%:%1781=912%:%
%:%1782=912%:%
%:%1783=912%:%
%:%1784=913%:%
%:%1785=913%:%
%:%1786=913%:%
%:%1787=914%:%
%:%1788=914%:%
%:%1789=915%:%
%:%1790=915%:%
%:%1791=916%:%
%:%1792=916%:%
%:%1793=917%:%
%:%1794=917%:%
%:%1795=918%:%
%:%1796=918%:%
%:%1797=919%:%
%:%1798=919%:%
%:%1799=920%:%
%:%1800=920%:%
%:%1801=920%:%
%:%1802=921%:%
%:%1803=921%:%
%:%1804=922%:%
%:%1805=922%:%
%:%1806=922%:%
%:%1807=923%:%
%:%1808=923%:%
%:%1809=923%:%
%:%1810=924%:%
%:%1811=924%:%
%:%1812=925%:%
%:%1813=925%:%
%:%1814=926%:%
%:%1815=926%:%
%:%1816=927%:%
%:%1817=927%:%
%:%1818=928%:%
%:%1819=928%:%
%:%1820=929%:%
%:%1821=929%:%
%:%1822=930%:%
%:%1823=930%:%
%:%1824=931%:%
%:%1825=931%:%
%:%1826=932%:%
%:%1827=932%:%
%:%1828=932%:%
%:%1829=933%:%
%:%1830=933%:%
%:%1831=934%:%
%:%1832=934%:%
%:%1833=934%:%
%:%1834=935%:%
%:%1835=935%:%
%:%1836=935%:%
%:%1837=936%:%
%:%1838=936%:%
%:%1839=937%:%
%:%1840=937%:%
%:%1841=938%:%
%:%1842=938%:%
%:%1843=939%:%
%:%1844=939%:%
%:%1845=940%:%
%:%1846=940%:%
%:%1847=941%:%
%:%1848=941%:%
%:%1849=942%:%
%:%1850=942%:%
%:%1853=945%:%
%:%1854=946%:%
%:%1855=946%:%
%:%1856=946%:%
%:%1857=947%:%
%:%1867=949%:%
%:%1869=951%:%
%:%1870=951%:%
%:%1871=952%:%
%:%1872=953%:%
%:%1873=954%:%
%:%1874=955%:%
%:%1875=956%:%
%:%1882=957%:%
%:%1883=957%:%
%:%1884=958%:%
%:%1885=958%:%
%:%1886=959%:%
%:%1887=959%:%
%:%1888=959%:%
%:%1889=959%:%
%:%1890=960%:%
%:%1891=960%:%
%:%1892=961%:%
%:%1893=961%:%
%:%1894=962%:%
%:%1895=963%:%
%:%1896=964%:%
%:%1897=965%:%
%:%1898=965%:%
%:%1899=966%:%
%:%1900=966%:%
%:%1901=966%:%
%:%1902=966%:%
%:%1903=967%:%
%:%1904=967%:%
%:%1905=967%:%
%:%1906=968%:%
%:%1916=970%:%
%:%1917=971%:%
%:%1918=972%:%
%:%1919=973%:%
%:%1920=974%:%
%:%1921=975%:%
%:%1922=976%:%
%:%1923=977%:%
%:%1924=978%:%
%:%1925=979%:%
%:%1926=980%:%
%:%1927=981%:%
%:%1928=982%:%
%:%1929=983%:%
%:%1930=984%:%
%:%1931=985%:%
%:%1933=987%:%
%:%1934=987%:%
%:%1935=988%:%
%:%1936=989%:%
%:%1937=990%:%
%:%1940=991%:%
%:%1944=991%:%
%:%1945=991%:%
%:%1946=992%:%
%:%1947=992%:%
%:%1948=993%:%
%:%1949=993%:%
%:%1950=994%:%
%:%1951=994%:%
%:%1952=994%:%
%:%1953=995%:%
%:%1954=995%:%
%:%1955=996%:%
%:%1956=996%:%
%:%1957=996%:%
%:%1958=997%:%
%:%1959=997%:%
%:%1960=998%:%
%:%1961=998%:%
%:%1962=999%:%
%:%1963=999%:%
%:%1964=1000%:%
%:%1965=1000%:%
%:%1966=1001%:%
%:%1967=1001%:%
%:%1968=1002%:%
%:%1969=1002%:%
%:%1970=1002%:%
%:%1971=1003%:%
%:%1972=1003%:%
%:%1973=1003%:%
%:%1974=1004%:%
%:%1975=1004%:%
%:%1976=1004%:%
%:%1977=1005%:%
%:%1978=1005%:%
%:%1979=1006%:%
%:%1980=1006%:%
%:%1981=1006%:%
%:%1982=1007%:%
%:%1983=1007%:%
%:%1984=1008%:%
%:%1985=1008%:%
%:%1986=1009%:%
%:%1987=1009%:%
%:%1988=1010%:%
%:%1998=1012%:%
%:%1999=1013%:%
%:%2000=1014%:%
%:%2001=1015%:%
%:%2002=1016%:%
%:%2003=1017%:%
%:%2004=1018%:%
%:%2005=1019%:%
%:%2006=1020%:%
%:%2007=1021%:%
%:%2008=1022%:%
%:%2009=1023%:%
%:%2010=1024%:%
%:%2011=1025%:%
%:%2012=1026%:%
%:%2013=1027%:%
%:%2014=1028%:%
%:%2015=1029%:%
%:%2016=1030%:%
%:%2017=1031%:%
%:%2018=1032%:%
%:%2019=1033%:%
%:%2020=1034%:%
%:%2021=1035%:%
%:%2022=1036%:%
%:%2023=1037%:%
%:%2024=1038%:%
%:%2025=1039%:%
%:%2026=1040%:%
%:%2027=1041%:%
%:%2028=1042%:%
%:%2029=1043%:%
%:%2030=1044%:%
%:%2031=1045%:%
%:%2032=1046%:%
%:%2033=1047%:%
%:%2035=1049%:%
%:%2036=1049%:%
%:%2037=1050%:%
%:%2038=1051%:%
%:%2039=1052%:%
%:%2040=1053%:%
%:%2047=1054%:%
%:%2048=1054%:%
%:%2049=1055%:%
%:%2050=1055%:%
%:%2051=1056%:%
%:%2052=1056%:%
%:%2053=1057%:%
%:%2054=1057%:%
%:%2055=1057%:%
%:%2056=1058%:%
%:%2057=1058%:%
%:%2058=1059%:%
%:%2059=1059%:%
%:%2060=1060%:%
%:%2061=1060%:%
%:%2062=1060%:%
%:%2063=1061%:%
%:%2064=1061%:%
%:%2065=1062%:%
%:%2066=1062%:%
%:%2067=1062%:%
%:%2068=1063%:%
%:%2069=1063%:%
%:%2070=1064%:%
%:%2071=1064%:%
%:%2072=1064%:%
%:%2073=1065%:%
%:%2074=1065%:%
%:%2075=1066%:%
%:%2076=1066%:%
%:%2077=1066%:%
%:%2078=1067%:%
%:%2079=1067%:%
%:%2080=1068%:%
%:%2081=1068%:%
%:%2082=1068%:%
%:%2083=1069%:%
%:%2084=1069%:%
%:%2085=1070%:%
%:%2086=1070%:%
%:%2087=1070%:%
%:%2088=1071%:%
%:%2089=1071%:%
%:%2090=1072%:%
%:%2091=1072%:%
%:%2092=1072%:%
%:%2093=1073%:%
%:%2094=1073%:%
%:%2095=1073%:%
%:%2096=1074%:%
%:%2097=1074%:%
%:%2098=1074%:%
%:%2099=1075%:%
%:%2100=1075%:%
%:%2101=1076%:%
%:%2102=1076%:%
%:%2103=1076%:%
%:%2104=1077%:%
%:%2105=1077%:%
%:%2106=1078%:%
%:%2107=1078%:%
%:%2108=1078%:%
%:%2109=1079%:%
%:%2110=1079%:%
%:%2111=1079%:%
%:%2112=1080%:%
%:%2113=1080%:%
%:%2114=1081%:%
%:%2115=1081%:%
%:%2116=1082%:%
%:%2117=1082%:%
%:%2118=1082%:%
%:%2119=1083%:%
%:%2120=1083%:%
%:%2121=1083%:%
%:%2122=1084%:%
%:%2123=1084%:%
%:%2124=1084%:%
%:%2125=1085%:%
%:%2126=1085%:%
%:%2127=1086%:%
%:%2128=1086%:%
%:%2129=1086%:%
%:%2130=1087%:%
%:%2131=1087%:%
%:%2132=1087%:%
%:%2133=1088%:%
%:%2134=1088%:%
%:%2135=1089%:%
%:%2136=1089%:%
%:%2137=1090%:%
%:%2138=1090%:%
%:%2139=1091%:%
%:%2140=1091%:%
%:%2141=1092%:%
%:%2142=1092%:%
%:%2143=1093%:%
%:%2144=1093%:%
%:%2145=1093%:%
%:%2146=1094%:%
%:%2147=1094%:%
%:%2148=1095%:%
%:%2149=1095%:%
%:%2150=1096%:%
%:%2151=1096%:%
%:%2152=1096%:%
%:%2153=1097%:%
%:%2163=1099%:%
%:%2165=1101%:%
%:%2166=1101%:%
%:%2167=1102%:%
%:%2168=1103%:%
%:%2169=1104%:%
%:%2170=1105%:%
%:%2177=1106%:%
%:%2178=1106%:%
%:%2179=1107%:%
%:%2180=1107%:%
%:%2181=1107%:%
%:%2182=1108%:%
%:%2183=1109%:%
%:%2184=1109%:%
%:%2185=1110%:%
%:%2186=1110%:%
%:%2187=1111%:%
%:%2188=1111%:%
%:%2189=1111%:%
%:%2190=1111%:%
%:%2191=1112%:%
%:%2192=1112%:%
%:%2193=1113%:%
%:%2194=1113%:%
%:%2195=1113%:%
%:%2196=1114%:%
%:%2197=1114%:%
%:%2198=1114%:%
%:%2199=1114%:%
%:%2200=1115%:%
%:%2201=1115%:%
%:%2202=1115%:%
%:%2203=1116%:%
%:%2204=1116%:%
%:%2205=1116%:%
%:%2206=1117%:%
%:%2207=1117%:%
%:%2208=1117%:%
%:%2209=1117%:%
%:%2210=1117%:%
%:%2211=1118%:%
%:%2226=1120%:%
%:%2238=1122%:%
%:%2239=1123%:%
%:%2240=1124%:%
%:%2241=1125%:%
%:%2242=1126%:%
%:%2243=1127%:%
%:%2244=1128%:%
%:%2246=1130%:%
%:%2247=1130%:%
%:%2248=1131%:%
%:%2249=1132%:%
%:%2250=1133%:%
%:%2251=1133%:%
%:%2252=1134%:%
%:%2253=1135%:%
%:%2256=1136%:%
%:%2260=1136%:%
%:%2261=1136%:%
%:%2262=1136%:%
%:%2263=1136%:%
%:%2264=1136%:%
%:%2273=1169%:%
%:%2275=1171%:%
%:%2276=1171%:%
%:%2277=1172%:%
%:%2278=1173%:%
%:%2285=1174%:%
%:%2286=1174%:%
%:%2287=1175%:%
%:%2288=1175%:%
%:%2289=1176%:%
%:%2290=1176%:%
%:%2291=1177%:%
%:%2292=1177%:%
%:%2293=1178%:%
%:%2294=1178%:%
%:%2295=1179%:%
%:%2296=1179%:%
%:%2297=1180%:%
%:%2298=1180%:%
%:%2299=1181%:%
%:%2300=1181%:%
%:%2301=1181%:%
%:%2302=1182%:%
%:%2303=1182%:%
%:%2304=1183%:%
%:%2305=1183%:%
%:%2306=1184%:%
%:%2307=1184%:%
%:%2308=1185%:%
%:%2309=1185%:%
%:%2310=1185%:%
%:%2311=1185%:%
%:%2312=1186%:%
%:%2313=1186%:%
%:%2314=1186%:%
%:%2315=1187%:%
%:%2316=1187%:%
%:%2317=1187%:%
%:%2318=1188%:%
%:%2319=1188%:%
%:%2320=1188%:%
%:%2321=1189%:%
%:%2322=1189%:%
%:%2323=1189%:%
%:%2324=1190%:%
%:%2325=1190%:%
%:%2326=1190%:%
%:%2327=1191%:%
%:%2328=1191%:%
%:%2329=1192%:%
%:%2330=1192%:%
%:%2331=1193%:%
%:%2332=1193%:%
%:%2333=1193%:%
%:%2334=1194%:%
%:%2335=1194%:%
%:%2336=1195%:%
%:%2337=1195%:%
%:%2338=1196%:%
%:%2339=1196%:%
%:%2340=1197%:%
%:%2341=1197%:%
%:%2342=1197%:%
%:%2343=1198%:%
%:%2344=1198%:%
%:%2345=1199%:%
%:%2346=1199%:%
%:%2347=1200%:%
%:%2348=1200%:%
%:%2349=1201%:%
%:%2350=1201%:%
%:%2351=1201%:%
%:%2352=1202%:%
%:%2353=1202%:%
%:%2354=1202%:%
%:%2355=1203%:%
%:%2356=1203%:%
%:%2357=1204%:%
%:%2358=1204%:%
%:%2359=1205%:%
%:%2360=1205%:%
%:%2361=1205%:%
%:%2362=1206%:%
%:%2363=1206%:%
%:%2364=1207%:%
%:%2370=1207%:%
%:%2373=1208%:%
%:%2374=1209%:%
%:%2375=1209%:%
%:%2376=1210%:%
%:%2377=1211%:%
%:%2378=1212%:%
%:%2381=1213%:%
%:%2385=1213%:%
%:%2386=1213%:%
%:%2391=1213%:%
%:%2394=1214%:%
%:%2395=1215%:%
%:%2396=1215%:%
%:%2397=1216%:%
%:%2398=1217%:%
%:%2399=1218%:%
%:%2406=1219%:%
%:%2407=1219%:%
%:%2408=1220%:%
%:%2409=1220%:%
%:%2410=1221%:%
%:%2411=1221%:%
%:%2412=1221%:%
%:%2413=1222%:%
%:%2414=1222%:%
%:%2415=1222%:%
%:%2416=1223%:%
%:%2417=1223%:%
%:%2418=1224%:%
%:%2419=1224%:%
%:%2420=1225%:%
%:%2421=1225%:%
%:%2422=1225%:%
%:%2423=1226%:%
%:%2424=1226%:%
%:%2425=1227%:%
%:%2426=1227%:%
%:%2427=1227%:%
%:%2428=1228%:%
%:%2429=1228%:%
%:%2430=1229%:%
%:%2431=1229%:%
%:%2432=1229%:%
%:%2433=1230%:%
%:%2434=1230%:%
%:%2435=1231%:%
%:%2436=1231%:%
%:%2437=1231%:%
%:%2438=1232%:%
%:%2439=1232%:%
%:%2440=1233%:%
%:%2441=1233%:%
%:%2442=1233%:%
%:%2443=1234%:%
%:%2444=1234%:%
%:%2445=1235%:%
%:%2446=1235%:%
%:%2447=1235%:%
%:%2448=1236%:%
%:%2449=1236%:%
%:%2450=1237%:%
%:%2451=1237%:%
%:%2452=1238%:%
%:%2453=1238%:%
%:%2454=1239%:%
%:%2455=1239%:%
%:%2456=1240%:%
%:%2457=1240%:%
%:%2458=1240%:%
%:%2459=1241%:%
%:%2460=1241%:%
%:%2461=1242%:%
%:%2462=1242%:%
%:%2463=1243%:%
%:%2464=1243%:%
%:%2465=1244%:%
%:%2466=1244%:%
%:%2467=1244%:%
%:%2468=1245%:%
%:%2469=1245%:%
%:%2470=1246%:%
%:%2471=1246%:%
%:%2472=1246%:%
%:%2473=1247%:%
%:%2474=1247%:%
%:%2475=1248%:%
%:%2476=1248%:%
%:%2477=1248%:%
%:%2478=1249%:%
%:%2479=1249%:%
%:%2480=1250%:%
%:%2481=1250%:%
%:%2482=1250%:%
%:%2483=1251%:%
%:%2484=1251%:%
%:%2485=1251%:%
%:%2486=1252%:%
%:%2487=1252%:%
%:%2488=1253%:%
%:%2489=1253%:%
%:%2490=1253%:%
%:%2491=1254%:%
%:%2492=1254%:%
%:%2493=1255%:%
%:%2494=1255%:%
%:%2495=1256%:%
%:%2496=1256%:%
%:%2497=1256%:%
%:%2498=1257%:%
%:%2499=1257%:%
%:%2500=1258%:%
%:%2501=1258%:%
%:%2502=1258%:%
%:%2503=1259%:%
%:%2504=1259%:%
%:%2505=1260%:%
%:%2506=1260%:%
%:%2507=1260%:%
%:%2508=1261%:%
%:%2509=1261%:%
%:%2510=1261%:%
%:%2511=1262%:%
%:%2512=1262%:%
%:%2513=1263%:%
%:%2514=1263%:%
%:%2515=1263%:%
%:%2516=1264%:%
%:%2517=1264%:%
%:%2518=1265%:%
%:%2519=1265%:%
%:%2520=1266%:%
%:%2521=1266%:%
%:%2522=1267%:%
%:%2523=1267%:%
%:%2524=1268%:%
%:%2525=1268%:%
%:%2526=1268%:%
%:%2527=1269%:%
%:%2528=1269%:%
%:%2529=1270%:%
%:%2530=1270%:%
%:%2531=1271%:%
%:%2532=1271%:%
%:%2533=1272%:%
%:%2534=1272%:%
%:%2535=1272%:%
%:%2536=1273%:%
%:%2537=1273%:%
%:%2538=1274%:%
%:%2539=1274%:%
%:%2540=1274%:%
%:%2541=1275%:%
%:%2542=1275%:%
%:%2543=1276%:%
%:%2544=1276%:%
%:%2545=1276%:%
%:%2546=1277%:%
%:%2547=1277%:%
%:%2548=1277%:%
%:%2549=1278%:%
%:%2550=1278%:%
%:%2551=1279%:%
%:%2552=1279%:%
%:%2553=1279%:%
%:%2554=1280%:%
%:%2555=1280%:%
%:%2556=1281%:%
%:%2557=1281%:%
%:%2558=1282%:%
%:%2559=1282%:%
%:%2560=1282%:%
%:%2561=1283%:%
%:%2562=1283%:%
%:%2563=1284%:%
%:%2564=1284%:%
%:%2565=1284%:%
%:%2566=1285%:%
%:%2567=1285%:%
%:%2568=1286%:%
%:%2569=1286%:%
%:%2570=1286%:%
%:%2571=1287%:%
%:%2572=1287%:%
%:%2573=1287%:%
%:%2574=1288%:%
%:%2575=1288%:%
%:%2576=1289%:%
%:%2577=1289%:%
%:%2578=1289%:%
%:%2579=1290%:%
%:%2580=1290%:%
%:%2581=1291%:%
%:%2582=1291%:%
%:%2583=1292%:%
%:%2589=1292%:%
%:%2592=1293%:%
%:%2593=1294%:%
%:%2594=1294%:%
%:%2595=1295%:%
%:%2596=1296%:%
%:%2597=1297%:%
%:%2598=1298%:%
%:%2599=1299%:%
%:%2600=1300%:%
%:%2607=1301%:%
%:%2608=1301%:%
%:%2609=1302%:%
%:%2610=1302%:%
%:%2611=1303%:%
%:%2612=1303%:%
%:%2613=1303%:%
%:%2614=1303%:%
%:%2615=1304%:%
%:%2616=1304%:%
%:%2617=1304%:%
%:%2618=1305%:%
%:%2619=1305%:%
%:%2620=1305%:%
%:%2621=1306%:%
%:%2622=1306%:%
%:%2623=1306%:%
%:%2624=1307%:%
%:%2625=1307%:%
%:%2626=1307%:%
%:%2627=1308%:%
%:%2628=1308%:%
%:%2629=1309%:%
%:%2630=1309%:%
%:%2631=1310%:%
%:%2632=1310%:%
%:%2633=1311%:%
%:%2634=1311%:%
%:%2635=1311%:%
%:%2636=1312%:%
%:%2637=1312%:%
%:%2638=1313%:%
%:%2639=1313%:%
%:%2640=1313%:%
%:%2641=1314%:%
%:%2642=1314%:%
%:%2643=1315%:%
%:%2644=1315%:%
%:%2645=1315%:%
%:%2646=1316%:%
%:%2647=1316%:%
%:%2648=1317%:%
%:%2649=1317%:%
%:%2650=1317%:%
%:%2651=1318%:%
%:%2652=1318%:%
%:%2653=1319%:%
%:%2654=1319%:%
%:%2655=1320%:%
%:%2656=1320%:%
%:%2657=1321%:%
%:%2658=1321%:%
%:%2659=1322%:%
%:%2660=1322%:%
%:%2661=1322%:%
%:%2662=1323%:%
%:%2663=1323%:%
%:%2664=1324%:%
%:%2665=1324%:%
%:%2666=1324%:%
%:%2667=1325%:%
%:%2668=1325%:%
%:%2669=1326%:%
%:%2670=1326%:%
%:%2671=1326%:%
%:%2672=1327%:%
%:%2673=1327%:%
%:%2674=1328%:%
%:%2675=1328%:%
%:%2676=1329%:%
%:%2677=1329%:%
%:%2680=1332%:%
%:%2681=1333%:%
%:%2682=1333%:%
%:%2683=1333%:%
%:%2684=1334%:%
%:%2685=1334%:%
%:%2686=1334%:%
%:%2687=1335%:%
%:%2688=1335%:%
%:%2689=1336%:%
%:%2690=1336%:%
%:%2691=1337%:%
%:%2692=1337%:%
%:%2693=1337%:%
%:%2694=1338%:%
%:%2695=1338%:%
%:%2696=1338%:%
%:%2697=1339%:%
%:%2698=1339%:%
%:%2699=1339%:%
%:%2700=1340%:%
%:%2701=1340%:%
%:%2702=1341%:%
%:%2703=1341%:%
%:%2704=1341%:%
%:%2705=1342%:%
%:%2706=1342%:%
%:%2707=1343%:%
%:%2708=1343%:%
%:%2709=1344%:%
%:%2710=1344%:%
%:%2711=1344%:%
%:%2712=1345%:%
%:%2713=1345%:%
%:%2714=1346%:%
%:%2715=1346%:%
%:%2716=1346%:%
%:%2717=1347%:%
%:%2718=1347%:%
%:%2719=1347%:%
%:%2720=1348%:%
%:%2721=1348%:%
%:%2722=1349%:%
%:%2723=1349%:%
%:%2724=1350%:%
%:%2725=1350%:%
%:%2726=1351%:%
%:%2727=1351%:%
%:%2728=1352%:%
%:%2729=1352%:%
%:%2731=1354%:%
%:%2732=1355%:%
%:%2733=1355%:%
%:%2734=1356%:%
%:%2735=1356%:%
%:%2736=1357%:%
%:%2737=1357%:%
%:%2738=1358%:%
%:%2739=1358%:%
%:%2740=1359%:%
%:%2741=1359%:%
%:%2742=1359%:%
%:%2743=1360%:%
%:%2744=1360%:%
%:%2745=1361%:%
%:%2746=1361%:%
%:%2747=1361%:%
%:%2748=1362%:%
%:%2749=1362%:%
%:%2750=1363%:%
%:%2751=1363%:%
%:%2752=1363%:%
%:%2753=1364%:%
%:%2754=1364%:%
%:%2755=1365%:%
%:%2756=1365%:%
%:%2757=1365%:%
%:%2758=1366%:%
%:%2759=1366%:%
%:%2760=1366%:%
%:%2761=1367%:%
%:%2762=1367%:%
%:%2763=1367%:%
%:%2764=1368%:%
%:%2765=1368%:%
%:%2766=1368%:%
%:%2767=1369%:%
%:%2768=1369%:%
%:%2769=1370%:%
%:%2770=1370%:%
%:%2771=1370%:%
%:%2772=1371%:%
%:%2773=1371%:%
%:%2774=1372%:%
%:%2775=1372%:%
%:%2776=1372%:%
%:%2777=1373%:%
%:%2778=1373%:%
%:%2779=1374%:%
%:%2780=1374%:%
%:%2781=1374%:%
%:%2782=1375%:%
%:%2783=1375%:%
%:%2784=1376%:%
%:%2785=1376%:%
%:%2786=1376%:%
%:%2787=1377%:%
%:%2788=1377%:%
%:%2789=1378%:%
%:%2790=1378%:%
%:%2791=1378%:%
%:%2792=1379%:%
%:%2793=1379%:%
%:%2794=1380%:%
%:%2795=1380%:%
%:%2796=1381%:%
%:%2797=1381%:%
%:%2798=1381%:%
%:%2799=1382%:%
%:%2800=1382%:%
%:%2801=1383%:%
%:%2802=1383%:%
%:%2803=1383%:%
%:%2804=1384%:%
%:%2805=1384%:%
%:%2806=1384%:%
%:%2807=1385%:%
%:%2808=1385%:%
%:%2809=1386%:%
%:%2810=1386%:%
%:%2811=1387%:%
%:%2812=1387%:%
%:%2813=1388%:%
%:%2814=1388%:%
%:%2815=1389%:%
%:%2816=1389%:%
%:%2817=1390%:%
%:%2818=1391%:%
%:%2819=1391%:%
%:%2820=1392%:%
%:%2821=1392%:%
%:%2822=1393%:%
%:%2823=1393%:%
%:%2824=1394%:%
%:%2825=1394%:%
%:%2826=1395%:%
%:%2827=1395%:%
%:%2828=1395%:%
%:%2829=1396%:%
%:%2830=1396%:%
%:%2831=1397%:%
%:%2832=1397%:%
%:%2833=1397%:%
%:%2834=1398%:%
%:%2835=1398%:%
%:%2836=1399%:%
%:%2837=1399%:%
%:%2838=1399%:%
%:%2839=1400%:%
%:%2840=1400%:%
%:%2841=1400%:%
%:%2842=1401%:%
%:%2843=1401%:%
%:%2844=1401%:%
%:%2845=1402%:%
%:%2846=1402%:%
%:%2847=1402%:%
%:%2848=1403%:%
%:%2849=1403%:%
%:%2850=1404%:%
%:%2851=1404%:%
%:%2852=1404%:%
%:%2853=1405%:%
%:%2854=1405%:%
%:%2855=1406%:%
%:%2856=1406%:%
%:%2857=1406%:%
%:%2858=1407%:%
%:%2859=1407%:%
%:%2860=1408%:%
%:%2861=1408%:%
%:%2862=1408%:%
%:%2863=1409%:%
%:%2864=1409%:%
%:%2865=1410%:%
%:%2866=1410%:%
%:%2867=1410%:%
%:%2868=1411%:%
%:%2869=1411%:%
%:%2870=1412%:%
%:%2871=1412%:%
%:%2872=1412%:%
%:%2873=1413%:%
%:%2874=1413%:%
%:%2875=1414%:%
%:%2876=1414%:%
%:%2877=1415%:%
%:%2878=1415%:%
%:%2879=1415%:%
%:%2880=1416%:%
%:%2881=1416%:%
%:%2882=1417%:%
%:%2883=1417%:%
%:%2884=1417%:%
%:%2885=1418%:%
%:%2886=1418%:%
%:%2887=1418%:%
%:%2888=1419%:%
%:%2889=1419%:%
%:%2890=1420%:%
%:%2891=1420%:%
%:%2892=1421%:%
%:%2893=1421%:%
%:%2894=1422%:%
%:%2895=1422%:%
%:%2896=1423%:%
%:%2897=1423%:%
%:%2898=1424%:%
%:%2899=1424%:%
%:%2900=1424%:%
%:%2901=1425%:%
%:%2902=1425%:%
%:%2903=1426%:%
%:%2904=1426%:%
%:%2905=1427%:%
%:%2906=1427%:%
%:%2907=1427%:%
%:%2908=1428%:%
%:%2909=1428%:%
%:%2910=1428%:%
%:%2911=1429%:%
%:%2912=1429%:%
%:%2913=1429%:%
%:%2914=1430%:%
%:%2915=1430%:%
%:%2916=1430%:%
%:%2917=1431%:%
%:%2918=1431%:%
%:%2919=1432%:%
%:%2920=1432%:%
%:%2921=1432%:%
%:%2922=1433%:%
%:%2923=1433%:%
%:%2924=1434%:%
%:%2925=1434%:%
%:%2926=1434%:%
%:%2927=1435%:%
%:%2928=1435%:%
%:%2929=1436%:%
%:%2930=1436%:%
%:%2931=1436%:%
%:%2932=1437%:%
%:%2933=1437%:%
%:%2934=1438%:%
%:%2935=1438%:%
%:%2936=1439%:%
%:%2937=1439%:%
%:%2938=1439%:%
%:%2939=1440%:%
%:%2940=1440%:%
%:%2941=1440%:%
%:%2942=1441%:%
%:%2943=1441%:%
%:%2944=1442%:%
%:%2945=1442%:%
%:%2946=1443%:%
%:%2947=1443%:%
%:%2948=1444%:%
%:%2949=1444%:%
%:%2950=1445%:%
%:%2951=1445%:%
%:%2952=1446%:%
%:%2953=1446%:%
%:%2954=1447%:%
%:%2955=1447%:%
%:%2956=1448%:%
%:%2957=1448%:%
%:%2958=1448%:%
%:%2959=1449%:%
%:%2965=1449%:%
%:%2968=1450%:%
%:%2969=1451%:%
%:%2970=1451%:%
%:%2971=1452%:%
%:%2972=1453%:%
%:%2979=1454%:%
%:%2980=1454%:%
%:%2981=1455%:%
%:%2982=1455%:%
%:%2983=1456%:%
%:%2984=1456%:%
%:%2985=1457%:%
%:%2986=1457%:%
%:%2987=1457%:%
%:%2988=1457%:%
%:%2989=1458%:%
%:%2990=1458%:%
%:%2991=1459%:%
%:%2992=1459%:%
%:%2993=1460%:%
%:%2994=1460%:%
%:%2995=1461%:%
%:%2996=1461%:%
%:%2997=1462%:%
%:%2998=1462%:%
%:%2999=1463%:%
%:%3000=1463%:%
%:%3001=1464%:%
%:%3002=1464%:%
%:%3003=1465%:%
%:%3004=1465%:%
%:%3005=1466%:%
%:%3006=1466%:%
%:%3007=1467%:%
%:%3008=1467%:%
%:%3009=1468%:%
%:%3010=1468%:%
%:%3011=1469%:%
%:%3012=1469%:%
%:%3013=1469%:%
%:%3014=1470%:%
%:%3015=1470%:%
%:%3016=1471%:%
%:%3017=1471%:%
%:%3018=1472%:%
%:%3019=1472%:%
%:%3020=1473%:%
%:%3021=1473%:%
%:%3022=1474%:%
%:%3023=1474%:%
%:%3024=1475%:%
%:%3025=1475%:%
%:%3026=1475%:%
%:%3027=1476%:%
%:%3028=1476%:%
%:%3029=1477%:%
%:%3030=1477%:%
%:%3031=1477%:%
%:%3032=1478%:%
%:%3033=1478%:%
%:%3034=1479%:%
%:%3035=1479%:%
%:%3036=1479%:%
%:%3037=1480%:%
%:%3038=1480%:%
%:%3039=1481%:%
%:%3040=1481%:%
%:%3041=1481%:%
%:%3042=1482%:%
%:%3043=1482%:%
%:%3044=1483%:%
%:%3045=1483%:%
%:%3046=1483%:%
%:%3047=1484%:%
%:%3048=1484%:%
%:%3049=1485%:%
%:%3050=1485%:%
%:%3051=1486%:%
%:%3052=1486%:%
%:%3053=1487%:%
%:%3054=1487%:%
%:%3055=1488%:%
%:%3056=1488%:%
%:%3057=1488%:%
%:%3058=1489%:%
%:%3059=1489%:%
%:%3060=1490%:%
%:%3061=1490%:%
%:%3062=1491%:%
%:%3063=1491%:%
%:%3064=1492%:%
%:%3065=1492%:%
%:%3066=1493%:%
%:%3067=1493%:%
%:%3068=1494%:%
%:%3069=1494%:%
%:%3070=1494%:%
%:%3071=1495%:%
%:%3072=1495%:%
%:%3073=1496%:%
%:%3074=1496%:%
%:%3075=1497%:%
%:%3076=1497%:%
%:%3077=1497%:%
%:%3078=1498%:%
%:%3079=1498%:%
%:%3080=1499%:%
%:%3081=1499%:%
%:%3082=1500%:%
%:%3083=1500%:%
%:%3084=1501%:%
%:%3085=1501%:%
%:%3086=1502%:%
%:%3087=1502%:%
%:%3088=1503%:%
%:%3089=1503%:%
%:%3090=1503%:%
%:%3091=1504%:%
%:%3092=1504%:%
%:%3093=1505%:%
%:%3094=1505%:%
%:%3095=1506%:%
%:%3096=1506%:%
%:%3097=1506%:%
%:%3098=1507%:%
%:%3099=1507%:%
%:%3100=1508%:%
%:%3101=1508%:%
%:%3102=1509%:%
%:%3103=1509%:%
%:%3104=1509%:%
%:%3105=1510%:%
%:%3106=1510%:%
%:%3107=1511%:%
%:%3108=1511%:%
%:%3109=1512%:%
%:%3110=1512%:%
%:%3111=1512%:%
%:%3112=1513%:%
%:%3113=1513%:%
%:%3114=1514%:%
%:%3115=1514%:%
%:%3116=1515%:%
%:%3117=1515%:%
%:%3118=1516%:%
%:%3119=1516%:%
%:%3120=1517%:%
%:%3121=1517%:%
%:%3122=1518%:%
%:%3123=1518%:%
%:%3124=1519%:%
%:%3125=1519%:%
%:%3126=1520%:%
%:%3132=1520%:%
%:%3135=1521%:%
%:%3136=1545%:%
%:%3137=1546%:%
%:%3138=1547%:%
%:%3139=1547%:%
%:%3140=1548%:%
%:%3141=1549%:%
%:%3144=1550%:%
%:%3148=1550%:%
%:%3154=1550%:%
%:%3157=1551%:%
%:%3158=1552%:%
%:%3159=1552%:%
%:%3162=1553%:%
%:%3166=1553%:%
%:%3167=1553%:%