%
\begin{isabellebody}%
\setisabellecontext{TeoremaEx}%
%
\isadelimtheory
%
\endisadelimtheory
%
\isatagtheory
%
\endisatagtheory
{\isafoldtheory}%
%
\isadelimtheory
%
\endisadelimtheory
%
\begin{isamarkuptext}%
\comentario{Tutores: cambiar José Antonio por Joaquín}%
\end{isamarkuptext}\isamarkuptrue%
%
\isadelimdocument
%
\endisadelimdocument
%
\isatagdocument
%
\isamarkupsection{Sucesiones de conjuntos%
}
\isamarkuptrue%
%
\endisatagdocument
{\isafolddocument}%
%
\isadelimdocument
%
\endisadelimdocument
%
\begin{isamarkuptext}%
En este apartado vamos a definir ciertas sucesiones monótonas \isa{{\isacharbraceleft}S\isactrlsub n{\isacharbraceright}} de conjuntos de fórmulas a 
  partir de una colección \isa{C} y un conjunto \isa{S\ {\isasymin}\ C}. De este modo, se demostrará que si
  \isa{C} verifica la propiedad de consistencia proposicional, entonces todo elemento de la suceción
  \isa{{\isacharbraceleft}S\isactrlsub n{\isacharbraceright}} pertenece a la colección. Igualmente, daremos un resultado que permite caracterizar 
  conjuntos de \isa{{\isacharbraceleft}S\isactrlsub n{\isacharbraceright}} en función de los anteriores. Por otro lado, definiremos el límite de dichas 
  sucesiones, probando que todo conjunto de \isa{{\isacharbraceleft}S\isactrlsub n{\isacharbraceright}} está contenido en el límite. Además, se
  demostrará que si una fórmula pertenece al límite, entonces pertenece a algún conjunto de
  la sucesión \isa{{\isacharbraceleft}S\isactrlsub n{\isacharbraceright}}. Finalmente, mostraremos un resultado sobre conjuntos finitos contenidos en
  el límite.

  Recordemos que el conjunto de las fórmulas proposicionales se define recursivamente a partir 
  de un alfabeto numerable de variables proposicionales. Por lo tanto, el conjunto de fórmulas 
  proposicionales es igualmente numerable, de modo que es posible enumerar sus elementos. Una vez 
  realizada esta observación, veamos la definición de sucesión de conjuntos de fórmulas 
  proposicionales a partir de una colección y un conjunto de la misma.

\begin{definicion}
  Sea \isa{C} una colección de conjuntos de fórmulas proposicionales, \isa{S\ {\isasymin}\ C} y \isa{F\isactrlsub {\isadigit{1}}{\isacharcomma}\ F\isactrlsub {\isadigit{2}}{\isacharcomma}\ F\isactrlsub {\isadigit{3}}\ {\isasymdots}} una 
  enumeración de las fórmulas proposicionales. Se define la \isa{sucesión\ de\ conjuntos\ de\ C\ a\ partir\ de\ S} como sigue:

  $S_{0} = S$

  $S_{n+1} = \left\{ \begin{array}{lcc} S_{n} \cup \{F_{n}\} &  si  & S_{n} \cup \{F_{n}\} \in C \\ \\ S_{n} & c.c \end{array} \right.$ 
\end{definicion}

  Para su formalización en Isabelle se ha introducido una instancia en la teoría de \isa{Sintaxis} que 
  indica explícitamente que el conjunto de las fórmulas proposicionales es numerable, probado
  mediante el método \isa{countable{\isacharunderscore}datatype} de Isabelle.

  \isa{instance\ formula\ {\isacharcolon}{\isacharcolon}\ {\isacharparenleft}countable{\isacharparenright}\ countable\ by\ countable{\isacharunderscore}datatype}

  De esta manera se genera en Isabelle una enumeración predeterminada de los elementos del conjunto,
  junto con herramientas para probar propiedades referentes a la numerabilidad. En particular, en la 
  formalización de la definición \isa{{\isadigit{4}}{\isachardot}{\isadigit{1}}{\isachardot}{\isadigit{1}}} se utilizará la función \isa{from{\isacharunderscore}nat} que, al aplicarla a un 
  número natural \isa{n}, nos devuelve la \isa{n}-ésima fórmula proposicional según una enumeración 
  predeterminada en Isabelle. 

  Puesto que la definición de las sucesiones en \isa{{\isadigit{4}}{\isachardot}{\isadigit{1}}{\isachardot}{\isadigit{1}}} se trata de una definición 
  recursiva sobre la estructura recursiva de los números naturales, se formalizará en Isabelle
  mediante el tipo de funciones primitivas recursivas de la siguiente manera.%
\end{isamarkuptext}\isamarkuptrue%
\isacommand{primrec}\isamarkupfalse%
\ pcp{\isacharunderscore}seq\ \isakeyword{where}\isanewline
{\isachardoublequoteopen}pcp{\isacharunderscore}seq\ C\ S\ {\isadigit{0}}\ {\isacharequal}\ S{\isachardoublequoteclose}\ {\isacharbar}\isanewline
{\isachardoublequoteopen}pcp{\isacharunderscore}seq\ C\ S\ {\isacharparenleft}Suc\ n{\isacharparenright}\ {\isacharequal}\ {\isacharparenleft}let\ Sn\ {\isacharequal}\ pcp{\isacharunderscore}seq\ C\ S\ n{\isacharsemicolon}\ Sn{\isadigit{1}}\ {\isacharequal}\ insert\ {\isacharparenleft}from{\isacharunderscore}nat\ n{\isacharparenright}\ Sn\ in\isanewline
\ \ \ \ \ \ \ \ \ \ \ \ \ \ \ \ \ \ \ \ \ \ \ \ if\ Sn{\isadigit{1}}\ {\isasymin}\ C\ then\ Sn{\isadigit{1}}\ else\ Sn{\isacharparenright}{\isachardoublequoteclose}%
\begin{isamarkuptext}%
Veamos el primer resultado sobre dichas sucesiones.

  \begin{lema}
    Sea \isa{C} una colección de conjuntos con la propiedad de consistencia proposicional,\\ \isa{S\ {\isasymin}\ C} y 
    \isa{{\isacharbraceleft}S\isactrlsub n{\isacharbraceright}} la sucesión de conjuntos de \isa{C} a partir de \isa{S} construida según la definición \isa{{\isadigit{4}}{\isachardot}{\isadigit{1}}{\isachardot}{\isadigit{1}}}. 
    Entonces, para todo \isa{n\ {\isasymin}\ {\isasymnat}} se verifica que \isa{S\isactrlsub n\ {\isasymin}\ C}.
  \end{lema}

  Procedamos con su demostración.

  \begin{demostracion}
    El resultado se prueba por inducción en los números naturales que conforman los subíndices de la 
    sucesión.

    En primer lugar, tenemos que \isa{S\isactrlsub {\isadigit{0}}\ {\isacharequal}\ S} pertenece a \isa{C} por hipótesis.

    Por otro lado, supongamos que \isa{S\isactrlsub n\ {\isasymin}\ C}. Probemos que \isa{S\isactrlsub n\isactrlsub {\isacharplus}\isactrlsub {\isadigit{1}}\ {\isasymin}\ C}. Si suponemos que \isa{S\isactrlsub n\ {\isasymunion}\ {\isacharbraceleft}F\isactrlsub n{\isacharbraceright}\ {\isasymin}\ C},
    por definición tenemos que \isa{S\isactrlsub n\isactrlsub {\isacharplus}\isactrlsub {\isadigit{1}}\ {\isacharequal}\ S\isactrlsub n\ {\isasymunion}\ {\isacharbraceleft}F\isactrlsub n{\isacharbraceright}}, luego pertenece a \isa{C}. En caso contrario, si
    suponemos que \isa{S\isactrlsub n\ {\isasymunion}\ {\isacharbraceleft}F\isactrlsub n{\isacharbraceright}\ {\isasymnotin}\ C}, por definición tenemos que \isa{S\isactrlsub n\isactrlsub {\isacharplus}\isactrlsub {\isadigit{1}}\ {\isacharequal}\ S\isactrlsub n}, que pertenece igualmente
    a \isa{C} por hipótesis de inducción. Por tanto, queda probado el resultado.
  \end{demostracion}

  La formalización y demostración detallada del lema en Isabelle son las siguientes.%
\end{isamarkuptext}\isamarkuptrue%
\isacommand{lemma}\isamarkupfalse%
\ \isanewline
\ \ \isakeyword{assumes}\ {\isachardoublequoteopen}pcp\ C{\isachardoublequoteclose}\ \isanewline
\ \ \ \ \ \ \ \ \ \ {\isachardoublequoteopen}S\ {\isasymin}\ C{\isachardoublequoteclose}\isanewline
\ \ \ \ \ \ \ \ \isakeyword{shows}\ {\isachardoublequoteopen}pcp{\isacharunderscore}seq\ C\ S\ n\ {\isasymin}\ C{\isachardoublequoteclose}\isanewline
%
\isadelimproof
%
\endisadelimproof
%
\isatagproof
\isacommand{proof}\isamarkupfalse%
\ {\isacharparenleft}induction\ n{\isacharparenright}\isanewline
\ \ \isacommand{show}\isamarkupfalse%
\ {\isachardoublequoteopen}pcp{\isacharunderscore}seq\ C\ S\ {\isadigit{0}}\ {\isasymin}\ C{\isachardoublequoteclose}\isanewline
\ \ \ \ \isacommand{by}\isamarkupfalse%
\ {\isacharparenleft}simp\ only{\isacharcolon}\ pcp{\isacharunderscore}seq{\isachardot}simps{\isacharparenleft}{\isadigit{1}}{\isacharparenright}\ {\isacartoucheopen}S\ {\isasymin}\ C{\isacartoucheclose}{\isacharparenright}\isanewline
\isacommand{next}\isamarkupfalse%
\isanewline
\ \ \isacommand{fix}\isamarkupfalse%
\ n\isanewline
\ \ \isacommand{assume}\isamarkupfalse%
\ HI{\isacharcolon}{\isachardoublequoteopen}pcp{\isacharunderscore}seq\ C\ S\ n\ {\isasymin}\ C{\isachardoublequoteclose}\isanewline
\ \ \isacommand{have}\isamarkupfalse%
\ {\isachardoublequoteopen}pcp{\isacharunderscore}seq\ C\ S\ {\isacharparenleft}Suc\ n{\isacharparenright}\ {\isacharequal}\ {\isacharparenleft}let\ Sn\ {\isacharequal}\ pcp{\isacharunderscore}seq\ C\ S\ n{\isacharsemicolon}\ Sn{\isadigit{1}}\ {\isacharequal}\ insert\ {\isacharparenleft}from{\isacharunderscore}nat\ n{\isacharparenright}\ Sn\ in\isanewline
\ \ \ \ \ \ \ \ \ \ \ \ \ \ \ \ \ \ \ \ \ \ \ \ if\ Sn{\isadigit{1}}\ {\isasymin}\ C\ then\ Sn{\isadigit{1}}\ else\ Sn{\isacharparenright}{\isachardoublequoteclose}\ \isanewline
\ \ \ \ \isacommand{by}\isamarkupfalse%
\ {\isacharparenleft}simp\ only{\isacharcolon}\ pcp{\isacharunderscore}seq{\isachardot}simps{\isacharparenleft}{\isadigit{2}}{\isacharparenright}{\isacharparenright}\isanewline
\ \ \isacommand{then}\isamarkupfalse%
\ \isacommand{have}\isamarkupfalse%
\ SucDef{\isacharcolon}{\isachardoublequoteopen}pcp{\isacharunderscore}seq\ C\ S\ {\isacharparenleft}Suc\ n{\isacharparenright}\ {\isacharequal}\ {\isacharparenleft}if\ insert\ {\isacharparenleft}from{\isacharunderscore}nat\ n{\isacharparenright}\ {\isacharparenleft}pcp{\isacharunderscore}seq\ C\ S\ n{\isacharparenright}\ {\isasymin}\ C\ then\ \isanewline
\ \ \ \ \ \ \ \ \ \ \ \ \ \ \ \ \ \ \ \ insert\ {\isacharparenleft}from{\isacharunderscore}nat\ n{\isacharparenright}\ {\isacharparenleft}pcp{\isacharunderscore}seq\ C\ S\ n{\isacharparenright}\ else\ pcp{\isacharunderscore}seq\ C\ S\ n{\isacharparenright}{\isachardoublequoteclose}\ \isanewline
\ \ \ \ \isacommand{by}\isamarkupfalse%
\ {\isacharparenleft}simp\ only{\isacharcolon}\ Let{\isacharunderscore}def{\isacharparenright}\isanewline
\ \ \isacommand{show}\isamarkupfalse%
\ {\isachardoublequoteopen}pcp{\isacharunderscore}seq\ C\ S\ {\isacharparenleft}Suc\ n{\isacharparenright}\ {\isasymin}\ C{\isachardoublequoteclose}\isanewline
\ \ \isacommand{proof}\isamarkupfalse%
\ {\isacharparenleft}cases{\isacharparenright}\isanewline
\ \ \ \ \isacommand{assume}\isamarkupfalse%
\ {\isadigit{1}}{\isacharcolon}{\isachardoublequoteopen}insert\ {\isacharparenleft}from{\isacharunderscore}nat\ n{\isacharparenright}\ {\isacharparenleft}pcp{\isacharunderscore}seq\ C\ S\ n{\isacharparenright}\ {\isasymin}\ C{\isachardoublequoteclose}\isanewline
\ \ \ \ \isacommand{have}\isamarkupfalse%
\ {\isachardoublequoteopen}pcp{\isacharunderscore}seq\ C\ S\ {\isacharparenleft}Suc\ n{\isacharparenright}\ {\isacharequal}\ insert\ {\isacharparenleft}from{\isacharunderscore}nat\ n{\isacharparenright}\ {\isacharparenleft}pcp{\isacharunderscore}seq\ C\ S\ n{\isacharparenright}{\isachardoublequoteclose}\isanewline
\ \ \ \ \ \ \isacommand{using}\isamarkupfalse%
\ SucDef\ {\isadigit{1}}\ \isacommand{by}\isamarkupfalse%
\ {\isacharparenleft}simp\ only{\isacharcolon}\ if{\isacharunderscore}True{\isacharparenright}\isanewline
\ \ \ \ \isacommand{thus}\isamarkupfalse%
\ {\isachardoublequoteopen}pcp{\isacharunderscore}seq\ C\ S\ {\isacharparenleft}Suc\ n{\isacharparenright}\ {\isasymin}\ C{\isachardoublequoteclose}\isanewline
\ \ \ \ \ \ \isacommand{by}\isamarkupfalse%
\ {\isacharparenleft}simp\ only{\isacharcolon}\ {\isadigit{1}}{\isacharparenright}\isanewline
\ \ \isacommand{next}\isamarkupfalse%
\isanewline
\ \ \ \ \isacommand{assume}\isamarkupfalse%
\ {\isadigit{2}}{\isacharcolon}{\isachardoublequoteopen}insert\ {\isacharparenleft}from{\isacharunderscore}nat\ n{\isacharparenright}\ {\isacharparenleft}pcp{\isacharunderscore}seq\ C\ S\ n{\isacharparenright}\ {\isasymnotin}\ C{\isachardoublequoteclose}\isanewline
\ \ \ \ \isacommand{have}\isamarkupfalse%
\ {\isachardoublequoteopen}pcp{\isacharunderscore}seq\ C\ S\ {\isacharparenleft}Suc\ n{\isacharparenright}\ {\isacharequal}\ pcp{\isacharunderscore}seq\ C\ S\ n{\isachardoublequoteclose}\isanewline
\ \ \ \ \ \ \isacommand{using}\isamarkupfalse%
\ SucDef\ {\isadigit{2}}\ \isacommand{by}\isamarkupfalse%
\ {\isacharparenleft}simp\ only{\isacharcolon}\ if{\isacharunderscore}False{\isacharparenright}\isanewline
\ \ \ \ \isacommand{thus}\isamarkupfalse%
\ {\isachardoublequoteopen}pcp{\isacharunderscore}seq\ C\ S\ {\isacharparenleft}Suc\ n{\isacharparenright}\ {\isasymin}\ C{\isachardoublequoteclose}\isanewline
\ \ \ \ \ \ \isacommand{by}\isamarkupfalse%
\ {\isacharparenleft}simp\ only{\isacharcolon}\ HI{\isacharparenright}\isanewline
\ \ \isacommand{qed}\isamarkupfalse%
\isanewline
\isacommand{qed}\isamarkupfalse%
%
\endisatagproof
{\isafoldproof}%
%
\isadelimproof
%
\endisadelimproof
%
\begin{isamarkuptext}%
Del mismo modo, podemos probar el lema de manera automática en Isabelle.%
\end{isamarkuptext}\isamarkuptrue%
\isacommand{lemma}\isamarkupfalse%
\ pcp{\isacharunderscore}seq{\isacharunderscore}in{\isacharcolon}\ {\isachardoublequoteopen}pcp\ C\ {\isasymLongrightarrow}\ S\ {\isasymin}\ C\ {\isasymLongrightarrow}\ pcp{\isacharunderscore}seq\ C\ S\ n\ {\isasymin}\ C{\isachardoublequoteclose}\isanewline
%
\isadelimproof
%
\endisadelimproof
%
\isatagproof
\isacommand{proof}\isamarkupfalse%
{\isacharparenleft}induction\ n{\isacharparenright}\isanewline
\ \ \isacommand{case}\isamarkupfalse%
\ {\isacharparenleft}Suc\ n{\isacharparenright}\ \ \isanewline
\ \ \isacommand{hence}\isamarkupfalse%
\ {\isachardoublequoteopen}pcp{\isacharunderscore}seq\ C\ S\ n\ {\isasymin}\ C{\isachardoublequoteclose}\ \isacommand{by}\isamarkupfalse%
\ simp\isanewline
\ \ \isacommand{thus}\isamarkupfalse%
\ {\isacharquery}case\ \isacommand{by}\isamarkupfalse%
\ {\isacharparenleft}simp\ add{\isacharcolon}\ Let{\isacharunderscore}def{\isacharparenright}\isanewline
\isacommand{qed}\isamarkupfalse%
\ simp%
\endisatagproof
{\isafoldproof}%
%
\isadelimproof
%
\endisadelimproof
%
\begin{isamarkuptext}%
Por otro lado, veamos la monotonía de dichas sucesiones.

  \begin{lema}
    Toda sucesión de conjuntos construida a partir de una colección y un conjunto según la
    definición \isa{{\isadigit{4}}{\isachardot}{\isadigit{1}}{\isachardot}{\isadigit{1}}} es monótona.
  \end{lema}

  En Isabelle, se formaliza de la siguiente forma.%
\end{isamarkuptext}\isamarkuptrue%
\isacommand{lemma}\isamarkupfalse%
\ {\isachardoublequoteopen}pcp{\isacharunderscore}seq\ C\ S\ n\ {\isasymsubseteq}\ pcp{\isacharunderscore}seq\ C\ S\ {\isacharparenleft}Suc\ n{\isacharparenright}{\isachardoublequoteclose}\isanewline
%
\isadelimproof
\ \ %
\endisadelimproof
%
\isatagproof
\isacommand{oops}\isamarkupfalse%
%
\endisatagproof
{\isafoldproof}%
%
\isadelimproof
%
\endisadelimproof
%
\begin{isamarkuptext}%
Procedamos con la demostración del lema.

  \begin{demostracion}
    Sea una colección de conjuntos \isa{C}, \isa{S\ {\isasymin}\ C} y \isa{{\isacharbraceleft}S\isactrlsub n{\isacharbraceright}} la sucesión de conjuntos de \isa{C} a partir de 
    \isa{S} según la definición \isa{{\isadigit{4}}{\isachardot}{\isadigit{1}}{\isachardot}{\isadigit{1}}}. Para probar que \isa{{\isacharbraceleft}S\isactrlsub n{\isacharbraceright}} es monótona, basta probar que \isa{S\isactrlsub n\ {\isasymsubseteq}\ S\isactrlsub n\isactrlsub {\isacharplus}\isactrlsub {\isadigit{1}}} 
    para todo \isa{n\ {\isasymin}\ {\isasymnat}}. En efecto, el resultado es inmediato al considerar dos casos para todo 
    \isa{n\ {\isasymin}\ {\isasymnat}}: \isa{S\isactrlsub n\ {\isasymunion}\ {\isacharbraceleft}F\isactrlsub n{\isacharbraceright}\ {\isasymin}\ C} o \isa{S\isactrlsub n\ {\isasymunion}\ {\isacharbraceleft}F\isactrlsub n{\isacharbraceright}\ {\isasymnotin}\ C}. Si suponemos que\\ \isa{S\isactrlsub n\ {\isasymunion}\ {\isacharbraceleft}F\isactrlsub n{\isacharbraceright}\ {\isasymin}\ C}, por definición 
    tenemos que \isa{S\isactrlsub n\isactrlsub {\isacharplus}\isactrlsub {\isadigit{1}}\ {\isacharequal}\ S\isactrlsub n\ {\isasymunion}\ {\isacharbraceleft}F\isactrlsub n{\isacharbraceright}}, luego es claro que\\ \isa{S\isactrlsub n\ {\isasymsubseteq}\ S\isactrlsub n\isactrlsub {\isacharplus}\isactrlsub {\isadigit{1}}}. En caso contrario, si 
    \isa{S\isactrlsub n\ {\isasymunion}\ {\isacharbraceleft}F\isactrlsub n{\isacharbraceright}\ {\isasymnotin}\ C}, por definición se tiene que \isa{S\isactrlsub n\isactrlsub {\isacharplus}\isactrlsub {\isadigit{1}}\ {\isacharequal}\ S\isactrlsub n}, obteniéndose igualmente el resultado
    por la propiedad reflexiva de la contención de conjuntos. 
  \end{demostracion}

  La prueba detallada en Isabelle se muestra a continuación.%
\end{isamarkuptext}\isamarkuptrue%
\isacommand{lemma}\isamarkupfalse%
\ {\isachardoublequoteopen}pcp{\isacharunderscore}seq\ C\ S\ n\ {\isasymsubseteq}\ pcp{\isacharunderscore}seq\ C\ S\ {\isacharparenleft}Suc\ n{\isacharparenright}{\isachardoublequoteclose}\isanewline
%
\isadelimproof
%
\endisadelimproof
%
\isatagproof
\isacommand{proof}\isamarkupfalse%
\ {\isacharminus}\isanewline
\ \ \isacommand{have}\isamarkupfalse%
\ {\isachardoublequoteopen}pcp{\isacharunderscore}seq\ C\ S\ {\isacharparenleft}Suc\ n{\isacharparenright}\ {\isacharequal}\ {\isacharparenleft}let\ Sn\ {\isacharequal}\ pcp{\isacharunderscore}seq\ C\ S\ n{\isacharsemicolon}\ Sn{\isadigit{1}}\ {\isacharequal}\ insert\ {\isacharparenleft}from{\isacharunderscore}nat\ n{\isacharparenright}\ Sn\ in\isanewline
\ \ \ \ \ \ \ \ \ \ \ \ \ \ \ \ \ \ \ \ \ \ \ \ if\ Sn{\isadigit{1}}\ {\isasymin}\ C\ then\ Sn{\isadigit{1}}\ else\ Sn{\isacharparenright}{\isachardoublequoteclose}\ \isanewline
\ \ \ \ \isacommand{by}\isamarkupfalse%
\ {\isacharparenleft}simp\ only{\isacharcolon}\ pcp{\isacharunderscore}seq{\isachardot}simps{\isacharparenleft}{\isadigit{2}}{\isacharparenright}{\isacharparenright}\isanewline
\ \ \isacommand{then}\isamarkupfalse%
\ \isacommand{have}\isamarkupfalse%
\ SucDef{\isacharcolon}{\isachardoublequoteopen}pcp{\isacharunderscore}seq\ C\ S\ {\isacharparenleft}Suc\ n{\isacharparenright}\ {\isacharequal}\ {\isacharparenleft}if\ insert\ {\isacharparenleft}from{\isacharunderscore}nat\ n{\isacharparenright}\ {\isacharparenleft}pcp{\isacharunderscore}seq\ C\ S\ n{\isacharparenright}\ {\isasymin}\ C\ then\ \isanewline
\ \ \ \ \ \ \ \ \ \ \ \ \ \ \ \ \ \ \ \ insert\ {\isacharparenleft}from{\isacharunderscore}nat\ n{\isacharparenright}\ {\isacharparenleft}pcp{\isacharunderscore}seq\ C\ S\ n{\isacharparenright}\ else\ pcp{\isacharunderscore}seq\ C\ S\ n{\isacharparenright}{\isachardoublequoteclose}\ \isanewline
\ \ \ \ \isacommand{by}\isamarkupfalse%
\ {\isacharparenleft}simp\ only{\isacharcolon}\ Let{\isacharunderscore}def{\isacharparenright}\isanewline
\ \ \isacommand{thus}\isamarkupfalse%
\ {\isachardoublequoteopen}pcp{\isacharunderscore}seq\ C\ S\ n\ {\isasymsubseteq}\ pcp{\isacharunderscore}seq\ C\ S\ {\isacharparenleft}Suc\ n{\isacharparenright}{\isachardoublequoteclose}\isanewline
\ \ \isacommand{proof}\isamarkupfalse%
\ {\isacharparenleft}cases{\isacharparenright}\isanewline
\ \ \ \ \isacommand{assume}\isamarkupfalse%
\ {\isadigit{1}}{\isacharcolon}{\isachardoublequoteopen}insert\ {\isacharparenleft}from{\isacharunderscore}nat\ n{\isacharparenright}\ {\isacharparenleft}pcp{\isacharunderscore}seq\ C\ S\ n{\isacharparenright}\ {\isasymin}\ C{\isachardoublequoteclose}\isanewline
\ \ \ \ \isacommand{have}\isamarkupfalse%
\ {\isachardoublequoteopen}pcp{\isacharunderscore}seq\ C\ S\ {\isacharparenleft}Suc\ n{\isacharparenright}\ {\isacharequal}\ insert\ {\isacharparenleft}from{\isacharunderscore}nat\ n{\isacharparenright}\ {\isacharparenleft}pcp{\isacharunderscore}seq\ C\ S\ n{\isacharparenright}{\isachardoublequoteclose}\isanewline
\ \ \ \ \ \ \isacommand{using}\isamarkupfalse%
\ SucDef\ {\isadigit{1}}\ \isacommand{by}\isamarkupfalse%
\ {\isacharparenleft}simp\ only{\isacharcolon}\ if{\isacharunderscore}True{\isacharparenright}\isanewline
\ \ \ \ \isacommand{thus}\isamarkupfalse%
\ {\isachardoublequoteopen}pcp{\isacharunderscore}seq\ C\ S\ n\ {\isasymsubseteq}\ pcp{\isacharunderscore}seq\ C\ S\ {\isacharparenleft}Suc\ n{\isacharparenright}{\isachardoublequoteclose}\isanewline
\ \ \ \ \ \ \isacommand{by}\isamarkupfalse%
\ {\isacharparenleft}simp\ only{\isacharcolon}\ subset{\isacharunderscore}insertI{\isacharparenright}\isanewline
\ \ \isacommand{next}\isamarkupfalse%
\isanewline
\ \ \ \ \isacommand{assume}\isamarkupfalse%
\ {\isadigit{2}}{\isacharcolon}{\isachardoublequoteopen}insert\ {\isacharparenleft}from{\isacharunderscore}nat\ n{\isacharparenright}\ {\isacharparenleft}pcp{\isacharunderscore}seq\ C\ S\ n{\isacharparenright}\ {\isasymnotin}\ C{\isachardoublequoteclose}\isanewline
\ \ \ \ \isacommand{have}\isamarkupfalse%
\ {\isachardoublequoteopen}pcp{\isacharunderscore}seq\ C\ S\ {\isacharparenleft}Suc\ n{\isacharparenright}\ {\isacharequal}\ pcp{\isacharunderscore}seq\ C\ S\ n{\isachardoublequoteclose}\isanewline
\ \ \ \ \ \ \isacommand{using}\isamarkupfalse%
\ SucDef\ {\isadigit{2}}\ \isacommand{by}\isamarkupfalse%
\ {\isacharparenleft}simp\ only{\isacharcolon}\ if{\isacharunderscore}False{\isacharparenright}\isanewline
\ \ \ \ \isacommand{thus}\isamarkupfalse%
\ {\isachardoublequoteopen}pcp{\isacharunderscore}seq\ C\ S\ n\ {\isasymsubseteq}\ pcp{\isacharunderscore}seq\ C\ S\ {\isacharparenleft}Suc\ n{\isacharparenright}{\isachardoublequoteclose}\isanewline
\ \ \ \ \ \ \isacommand{by}\isamarkupfalse%
\ {\isacharparenleft}simp\ only{\isacharcolon}\ subset{\isacharunderscore}refl{\isacharparenright}\isanewline
\ \ \isacommand{qed}\isamarkupfalse%
\isanewline
\isacommand{qed}\isamarkupfalse%
%
\endisatagproof
{\isafoldproof}%
%
\isadelimproof
%
\endisadelimproof
%
\begin{isamarkuptext}%
Del mismo modo, se puede probar automáticamente en Isabelle/HOL.%
\end{isamarkuptext}\isamarkuptrue%
\isacommand{lemma}\isamarkupfalse%
\ pcp{\isacharunderscore}seq{\isacharunderscore}monotonicity{\isacharcolon}{\isachardoublequoteopen}pcp{\isacharunderscore}seq\ C\ S\ n\ {\isasymsubseteq}\ pcp{\isacharunderscore}seq\ C\ S\ {\isacharparenleft}Suc\ n{\isacharparenright}{\isachardoublequoteclose}\isanewline
%
\isadelimproof
\ \ %
\endisadelimproof
%
\isatagproof
\isacommand{by}\isamarkupfalse%
\ {\isacharparenleft}smt\ eq{\isacharunderscore}iff\ pcp{\isacharunderscore}seq{\isachardot}simps{\isacharparenleft}{\isadigit{2}}{\isacharparenright}\ subset{\isacharunderscore}insertI{\isacharparenright}%
\endisatagproof
{\isafoldproof}%
%
\isadelimproof
%
\endisadelimproof
%
\begin{isamarkuptext}%
Por otra lado, para facilitar posteriores demostraciones en Isabelle/HOL, vamos a formalizar 
  el lema anterior empleando la siguiente definición generalizada de monotonía.%
\end{isamarkuptext}\isamarkuptrue%
\isacommand{lemma}\isamarkupfalse%
\ pcp{\isacharunderscore}seq{\isacharunderscore}mono{\isacharcolon}\isanewline
\ \ \isakeyword{assumes}\ {\isachardoublequoteopen}n\ {\isasymle}\ m{\isachardoublequoteclose}\ \isanewline
\ \ \isakeyword{shows}\ {\isachardoublequoteopen}pcp{\isacharunderscore}seq\ C\ S\ n\ {\isasymsubseteq}\ pcp{\isacharunderscore}seq\ C\ S\ m{\isachardoublequoteclose}\isanewline
%
\isadelimproof
\ \ %
\endisadelimproof
%
\isatagproof
\isacommand{using}\isamarkupfalse%
\ pcp{\isacharunderscore}seq{\isacharunderscore}monotonicity\ assms\ \isacommand{by}\isamarkupfalse%
\ {\isacharparenleft}rule\ lift{\isacharunderscore}Suc{\isacharunderscore}mono{\isacharunderscore}le{\isacharparenright}%
\endisatagproof
{\isafoldproof}%
%
\isadelimproof
%
\endisadelimproof
%
\begin{isamarkuptext}%
A continuación daremos un lema que permite caracterizar un elemento de la sucesión en función 
  de los anteriores.

\begin{lema}
  Sea \isa{C} una colección de conjuntos, \isa{S\ {\isasymin}\ C} y \isa{{\isacharbraceleft}S\isactrlsub n{\isacharbraceright}} la sucesión de conjuntos de \isa{C} a partir de 
  \isa{S} construida según la definición \isa{{\isadigit{4}}{\isachardot}{\isadigit{1}}{\isachardot}{\isadigit{1}}}. Entonces, para todos \isa{n}, \isa{m\ {\isasymin}\ {\isasymnat}} 
  se verifica $\bigcup_{n \leq m} S_{n} = S_{m}$
\end{lema}

\begin{demostracion}
  En las condiciones del enunciado, la prueba se realiza por inducción en \isa{m\ {\isasymin}\ {\isasymnat}}.

  En primer lugar, consideremos el caso base \isa{m\ {\isacharequal}\ {\isadigit{0}}}. El resultado se obtiene directamente:

  $\bigcup_{n \leq 0} S_{n} = \bigcup_{n = 0} S_{n} = S_{0} = S_{m}$

  Por otro lado, supongamos por hipótesis de inducción que $\bigcup_{n \leq m} S_{n} = S_{m}$.
  Veamos que se verifica $\bigcup_{n \leq m + 1} S_{n} = S_{m + 1}$. Observemos que si \isa{n\ {\isasymle}\ m\ {\isacharplus}\ {\isadigit{1}}},
  entonces se tiene que, o bien \isa{n\ {\isasymle}\ m}, o bien \isa{n\ {\isacharequal}\ m\ {\isacharplus}\ {\isadigit{1}}}. De este modo, aplicando la 
  hipótesis de inducción, deducimos lo siguiente.

  $\bigcup_{n \leq m + 1} S_{n} = \bigcup_{n \leq m} S_{n} \cup \bigcup_{n = m + 1} S_{n} = \bigcup_{n \leq m} S_{n} \cup S_{m + 1} = S_{m} \cup S_{m + 1}$

  Por la monotonía de la sucesión, se tiene que \isa{S\isactrlsub m\ {\isasymsubseteq}\ S\isactrlsub m\isactrlsub {\isacharplus}\isactrlsub {\isadigit{1}}}. Luego, se verifica:

  $\bigcup_{n \leq m + 1} S_{n} = S_{m} \cup S_{m + 1} = S_{m + 1}$

  Lo que prueba el resultado.
\end{demostracion}

  Para formalizar dicho resultado y su demostración en Isabelle, hay que tener en cuenta cómo está
  formalizada la \isa{Teoría\ de\ Conjuntos} \href{https://bit.ly/3ibCuje}{Set.thy}. Los conjuntos están 
  formalizados como predicados, junto con la función \isa{Collect} y el predicado \isa{member}, verificando 
  los siguientes axiomas:

  \isa{mem{\isacharunderscore}Collect{\isacharunderscore}eq{\isacharcolon}}  "\isa{member\ a\ {\isacharparenleft}Collect\ P{\isacharparenright}\ {\isacharequal}\ P\ a}"

  \isa{Collect{\isacharunderscore}mem{\isacharunderscore}eq{\isacharcolon}} "\isa{Collect\ {\isacharparenleft}{\isasymlambda}x{\isachardot}\ member\ x\ A{\isacharparenright}\ {\isacharequal}\ A}"

  Se demuestra también que el tipo de los conjuntos constituye un álgebra de \isa{Boole}, en el que el 
  supremo de dos conjuntos es la unión y el ínfimo es la intersección. De esta forma, podemos usar 
  la unión generalizada, definida en la la teoría de retículos completos de Isabelle 
  \href{https://n9.cl/gtf5x}{Complete-Lattices.thy}.

  Veamos la prueba detallada del resultado en Isabelle/HOL.%
\end{isamarkuptext}\isamarkuptrue%
\isacommand{lemma}\isamarkupfalse%
\ {\isachardoublequoteopen}{\isasymUnion}{\isacharbraceleft}pcp{\isacharunderscore}seq\ C\ S\ n{\isacharbar}n{\isachardot}\ n\ {\isasymle}\ m{\isacharbraceright}\ {\isacharequal}\ pcp{\isacharunderscore}seq\ C\ S\ m{\isachardoublequoteclose}\isanewline
%
\isadelimproof
%
\endisadelimproof
%
\isatagproof
\isacommand{proof}\isamarkupfalse%
\ {\isacharparenleft}induct\ m{\isacharparenright}\isanewline
\ \ \isacommand{have}\isamarkupfalse%
\ \ {\isachardoublequoteopen}{\isasymUnion}{\isacharbraceleft}pcp{\isacharunderscore}seq\ C\ S\ n{\isacharbar}n{\isachardot}\ n\ {\isasymle}\ {\isadigit{0}}{\isacharbraceright}\ {\isacharequal}\ {\isasymUnion}{\isacharbraceleft}pcp{\isacharunderscore}seq\ C\ S\ n{\isacharbar}n{\isachardot}\ n\ {\isacharequal}\ {\isadigit{0}}{\isacharbraceright}{\isachardoublequoteclose}\isanewline
\ \ \ \ \isacommand{by}\isamarkupfalse%
\ {\isacharparenleft}simp\ only{\isacharcolon}\ le{\isacharunderscore}zero{\isacharunderscore}eq{\isacharparenright}\isanewline
\ \ \isacommand{also}\isamarkupfalse%
\ \isacommand{have}\isamarkupfalse%
\ {\isachardoublequoteopen}{\isasymdots}\ {\isacharequal}\ {\isasymUnion}{\isacharparenleft}{\isacharparenleft}pcp{\isacharunderscore}seq\ C\ S{\isacharparenright}{\isacharbackquote}{\isacharbraceleft}n{\isachardot}\ n\ {\isacharequal}\ {\isadigit{0}}{\isacharbraceright}{\isacharparenright}{\isachardoublequoteclose}\isanewline
\ \ \ \ \isacommand{by}\isamarkupfalse%
\ {\isacharparenleft}simp\ only{\isacharcolon}\ image{\isacharunderscore}Collect{\isacharparenright}\isanewline
\ \ \isacommand{also}\isamarkupfalse%
\ \isacommand{have}\isamarkupfalse%
\ {\isachardoublequoteopen}{\isasymdots}\ {\isacharequal}\ {\isasymUnion}{\isacharbraceleft}pcp{\isacharunderscore}seq\ C\ S\ {\isadigit{0}}{\isacharbraceright}{\isachardoublequoteclose}\isanewline
\ \ \ \ \isacommand{by}\isamarkupfalse%
\ {\isacharparenleft}simp\ only{\isacharcolon}\ singleton{\isacharunderscore}conv\ image{\isacharunderscore}insert\ image{\isacharunderscore}empty{\isacharparenright}\isanewline
\ \ \isacommand{also}\isamarkupfalse%
\ \isacommand{have}\isamarkupfalse%
\ {\isachardoublequoteopen}{\isasymdots}\ {\isacharequal}\ pcp{\isacharunderscore}seq\ C\ S\ {\isadigit{0}}{\isachardoublequoteclose}\ \isanewline
\ \ \ \ \isacommand{by}\isamarkupfalse%
\ \ {\isacharparenleft}simp\ only{\isacharcolon}cSup{\isacharunderscore}singleton{\isacharparenright}\isanewline
\ \ \isacommand{finally}\isamarkupfalse%
\ \isacommand{show}\isamarkupfalse%
\ {\isachardoublequoteopen}{\isasymUnion}{\isacharbraceleft}pcp{\isacharunderscore}seq\ C\ S\ n{\isacharbar}n{\isachardot}\ n\ {\isasymle}\ {\isadigit{0}}{\isacharbraceright}\ {\isacharequal}\ pcp{\isacharunderscore}seq\ C\ S\ {\isadigit{0}}{\isachardoublequoteclose}\ \isanewline
\ \ \ \ \isacommand{by}\isamarkupfalse%
\ this\isanewline
\isacommand{next}\isamarkupfalse%
\isanewline
\ \ \isacommand{fix}\isamarkupfalse%
\ m\isanewline
\ \ \isacommand{assume}\isamarkupfalse%
\ HI{\isacharcolon}{\isachardoublequoteopen}{\isasymUnion}{\isacharbraceleft}pcp{\isacharunderscore}seq\ C\ S\ n{\isacharbar}n{\isachardot}\ n\ {\isasymle}\ m{\isacharbraceright}\ {\isacharequal}\ pcp{\isacharunderscore}seq\ C\ S\ m{\isachardoublequoteclose}\isanewline
\ \ \isacommand{have}\isamarkupfalse%
\ {\isachardoublequoteopen}m\ {\isasymle}\ Suc\ m{\isachardoublequoteclose}\ \isanewline
\ \ \ \ \isacommand{by}\isamarkupfalse%
\ {\isacharparenleft}simp\ only{\isacharcolon}\ add{\isacharunderscore}{\isadigit{0}}{\isacharunderscore}right{\isacharparenright}\isanewline
\ \ \isacommand{then}\isamarkupfalse%
\ \isacommand{have}\isamarkupfalse%
\ Mon{\isacharcolon}{\isachardoublequoteopen}pcp{\isacharunderscore}seq\ C\ S\ m\ {\isasymsubseteq}\ pcp{\isacharunderscore}seq\ C\ S\ {\isacharparenleft}Suc\ m{\isacharparenright}{\isachardoublequoteclose}\isanewline
\ \ \ \ \isacommand{by}\isamarkupfalse%
\ {\isacharparenleft}rule\ pcp{\isacharunderscore}seq{\isacharunderscore}mono{\isacharparenright}\isanewline
\ \ \isacommand{have}\isamarkupfalse%
\ {\isachardoublequoteopen}{\isasymUnion}{\isacharbraceleft}pcp{\isacharunderscore}seq\ C\ S\ n\ {\isacharbar}\ n{\isachardot}\ n\ {\isasymle}\ Suc\ m{\isacharbraceright}\ {\isacharequal}\ {\isasymUnion}{\isacharparenleft}{\isacharparenleft}pcp{\isacharunderscore}seq\ C\ S{\isacharparenright}{\isacharbackquote}{\isacharparenleft}{\isacharbraceleft}n{\isachardot}\ n\ {\isasymle}\ Suc\ m{\isacharbraceright}{\isacharparenright}{\isacharparenright}{\isachardoublequoteclose}\isanewline
\ \ \ \ \isacommand{by}\isamarkupfalse%
\ {\isacharparenleft}simp\ only{\isacharcolon}\ image{\isacharunderscore}Collect{\isacharparenright}\isanewline
\ \ \isacommand{also}\isamarkupfalse%
\ \isacommand{have}\isamarkupfalse%
\ {\isachardoublequoteopen}{\isasymdots}\ {\isacharequal}\ {\isasymUnion}{\isacharparenleft}{\isacharparenleft}pcp{\isacharunderscore}seq\ C\ S{\isacharparenright}{\isacharbackquote}{\isacharparenleft}{\isacharbraceleft}Suc\ m{\isacharbraceright}\ {\isasymunion}\ {\isacharbraceleft}n{\isachardot}\ n\ {\isasymle}\ m{\isacharbraceright}{\isacharparenright}{\isacharparenright}{\isachardoublequoteclose}\isanewline
\ \ \ \ \isacommand{by}\isamarkupfalse%
\ {\isacharparenleft}simp\ only{\isacharcolon}\ le{\isacharunderscore}Suc{\isacharunderscore}eq\ Collect{\isacharunderscore}disj{\isacharunderscore}eq\ Un{\isacharunderscore}commute\ singleton{\isacharunderscore}conv{\isacharparenright}\isanewline
\ \ \isacommand{also}\isamarkupfalse%
\ \isacommand{have}\isamarkupfalse%
\ {\isachardoublequoteopen}{\isasymdots}\ {\isacharequal}\ {\isasymUnion}{\isacharparenleft}{\isacharbraceleft}pcp{\isacharunderscore}seq\ C\ S\ {\isacharparenleft}Suc\ m{\isacharparenright}{\isacharbraceright}\ {\isasymunion}\ {\isacharbraceleft}pcp{\isacharunderscore}seq\ C\ S\ n\ {\isacharbar}\ n{\isachardot}\ n\ {\isasymle}\ m{\isacharbraceright}{\isacharparenright}{\isachardoublequoteclose}\isanewline
\ \ \ \ \isacommand{by}\isamarkupfalse%
\ {\isacharparenleft}simp\ only{\isacharcolon}\ image{\isacharunderscore}Un\ image{\isacharunderscore}insert\ image{\isacharunderscore}empty\ image{\isacharunderscore}Collect{\isacharparenright}\isanewline
\ \ \isacommand{also}\isamarkupfalse%
\ \isacommand{have}\isamarkupfalse%
\ {\isachardoublequoteopen}{\isasymdots}\ {\isacharequal}\ {\isasymUnion}{\isacharbraceleft}pcp{\isacharunderscore}seq\ C\ S\ {\isacharparenleft}Suc\ m{\isacharparenright}{\isacharbraceright}\ {\isasymunion}\ {\isasymUnion}{\isacharbraceleft}pcp{\isacharunderscore}seq\ C\ S\ n\ {\isacharbar}\ n{\isachardot}\ n\ {\isasymle}\ m{\isacharbraceright}{\isachardoublequoteclose}\isanewline
\ \ \ \ \isacommand{by}\isamarkupfalse%
\ {\isacharparenleft}simp\ only{\isacharcolon}\ Union{\isacharunderscore}Un{\isacharunderscore}distrib{\isacharparenright}\isanewline
\ \ \isacommand{also}\isamarkupfalse%
\ \isacommand{have}\isamarkupfalse%
\ {\isachardoublequoteopen}{\isasymdots}\ {\isacharequal}\ {\isacharparenleft}pcp{\isacharunderscore}seq\ C\ S\ {\isacharparenleft}Suc\ m{\isacharparenright}{\isacharparenright}\ {\isasymunion}\ {\isasymUnion}{\isacharbraceleft}pcp{\isacharunderscore}seq\ C\ S\ n\ {\isacharbar}\ n{\isachardot}\ n\ {\isasymle}\ m{\isacharbraceright}{\isachardoublequoteclose}\isanewline
\ \ \ \ \isacommand{by}\isamarkupfalse%
\ {\isacharparenleft}simp\ only{\isacharcolon}\ cSup{\isacharunderscore}singleton{\isacharparenright}\isanewline
\ \ \isacommand{also}\isamarkupfalse%
\ \isacommand{have}\isamarkupfalse%
\ {\isachardoublequoteopen}{\isasymdots}\ {\isacharequal}\ {\isacharparenleft}pcp{\isacharunderscore}seq\ C\ S\ {\isacharparenleft}Suc\ m{\isacharparenright}{\isacharparenright}\ {\isasymunion}\ {\isacharparenleft}pcp{\isacharunderscore}seq\ C\ S\ m{\isacharparenright}{\isachardoublequoteclose}\isanewline
\ \ \ \ \isacommand{by}\isamarkupfalse%
\ {\isacharparenleft}simp\ only{\isacharcolon}\ HI{\isacharparenright}\isanewline
\ \ \isacommand{also}\isamarkupfalse%
\ \isacommand{have}\isamarkupfalse%
\ {\isachardoublequoteopen}{\isasymdots}\ {\isacharequal}\ pcp{\isacharunderscore}seq\ C\ S\ {\isacharparenleft}Suc\ m{\isacharparenright}{\isachardoublequoteclose}\isanewline
\ \ \ \ \isacommand{using}\isamarkupfalse%
\ Mon\ \isacommand{by}\isamarkupfalse%
\ {\isacharparenleft}simp\ only{\isacharcolon}\ Un{\isacharunderscore}absorb{\isadigit{2}}{\isacharparenright}\isanewline
\ \ \isacommand{finally}\isamarkupfalse%
\ \isacommand{show}\isamarkupfalse%
\ {\isachardoublequoteopen}{\isasymUnion}{\isacharbraceleft}pcp{\isacharunderscore}seq\ C\ S\ n{\isacharbar}n{\isachardot}\ n\ {\isasymle}\ {\isacharparenleft}Suc\ m{\isacharparenright}{\isacharbraceright}\ {\isacharequal}\ pcp{\isacharunderscore}seq\ C\ S\ {\isacharparenleft}Suc\ m{\isacharparenright}{\isachardoublequoteclose}\isanewline
\ \ \ \ \isacommand{by}\isamarkupfalse%
\ this\isanewline
\isacommand{qed}\isamarkupfalse%
%
\endisatagproof
{\isafoldproof}%
%
\isadelimproof
%
\endisadelimproof
%
\begin{isamarkuptext}%
Análogamente, podemos dar una prueba automática.%
\end{isamarkuptext}\isamarkuptrue%
\isacommand{lemma}\isamarkupfalse%
\ pcp{\isacharunderscore}seq{\isacharunderscore}UN{\isacharcolon}\ {\isachardoublequoteopen}{\isasymUnion}{\isacharbraceleft}pcp{\isacharunderscore}seq\ C\ S\ n{\isacharbar}n{\isachardot}\ n\ {\isasymle}\ m{\isacharbraceright}\ {\isacharequal}\ pcp{\isacharunderscore}seq\ C\ S\ m{\isachardoublequoteclose}\isanewline
%
\isadelimproof
%
\endisadelimproof
%
\isatagproof
\isacommand{proof}\isamarkupfalse%
{\isacharparenleft}induction\ m{\isacharparenright}\isanewline
\ \ \isacommand{case}\isamarkupfalse%
\ {\isacharparenleft}Suc\ m{\isacharparenright}\isanewline
\ \ \isacommand{have}\isamarkupfalse%
\ {\isachardoublequoteopen}{\isacharbraceleft}f\ n\ {\isacharbar}n{\isachardot}\ n\ {\isasymle}\ Suc\ m{\isacharbraceright}\ {\isacharequal}\ insert\ {\isacharparenleft}f\ {\isacharparenleft}Suc\ m{\isacharparenright}{\isacharparenright}\ {\isacharbraceleft}f\ n\ {\isacharbar}n{\isachardot}\ n\ {\isasymle}\ m{\isacharbraceright}{\isachardoublequoteclose}\ \isanewline
\ \ \ \ \isakeyword{for}\ f\ \isacommand{using}\isamarkupfalse%
\ le{\isacharunderscore}Suc{\isacharunderscore}eq\ \isacommand{by}\isamarkupfalse%
\ auto\isanewline
\ \ \isacommand{hence}\isamarkupfalse%
\ {\isachardoublequoteopen}{\isacharbraceleft}pcp{\isacharunderscore}seq\ C\ S\ n\ {\isacharbar}n{\isachardot}\ n\ {\isasymle}\ Suc\ m{\isacharbraceright}\ {\isacharequal}\ \isanewline
\ \ \ \ \ \ \ \ \ \ insert\ {\isacharparenleft}pcp{\isacharunderscore}seq\ C\ S\ {\isacharparenleft}Suc\ m{\isacharparenright}{\isacharparenright}\ {\isacharbraceleft}pcp{\isacharunderscore}seq\ C\ S\ n\ {\isacharbar}n{\isachardot}\ n\ {\isasymle}\ m{\isacharbraceright}{\isachardoublequoteclose}\ \isacommand{{\isachardot}}\isamarkupfalse%
\isanewline
\ \ \isacommand{hence}\isamarkupfalse%
\ {\isachardoublequoteopen}{\isasymUnion}{\isacharbraceleft}pcp{\isacharunderscore}seq\ C\ S\ n\ {\isacharbar}n{\isachardot}\ n\ {\isasymle}\ Suc\ m{\isacharbraceright}\ {\isacharequal}\ \isanewline
\ \ \ \ \ \ \ \ \ {\isasymUnion}{\isacharbraceleft}pcp{\isacharunderscore}seq\ C\ S\ n\ {\isacharbar}n{\isachardot}\ n\ {\isasymle}\ m{\isacharbraceright}\ {\isasymunion}\ pcp{\isacharunderscore}seq\ C\ S\ {\isacharparenleft}Suc\ m{\isacharparenright}{\isachardoublequoteclose}\ \isacommand{by}\isamarkupfalse%
\ auto\isanewline
\ \ \isacommand{thus}\isamarkupfalse%
\ {\isacharquery}case\ \isacommand{using}\isamarkupfalse%
\ Suc\ pcp{\isacharunderscore}seq{\isacharunderscore}mono\ \isacommand{by}\isamarkupfalse%
\ blast\isanewline
\isacommand{qed}\isamarkupfalse%
\ simp%
\endisatagproof
{\isafoldproof}%
%
\isadelimproof
%
\endisadelimproof
%
\begin{isamarkuptext}%
Finalmente, definamos el límite de las sucesiones presentadas en la definición \isa{{\isadigit{4}}{\isachardot}{\isadigit{1}}{\isachardot}{\isadigit{1}}}.

 \begin{definicion}
  Sea \isa{C} una colección, \isa{S\ {\isasymin}\ C} y \isa{{\isacharbraceleft}S\isactrlsub n{\isacharbraceright}} la sucesión de conjuntos de \isa{C} a partir de \isa{S} según la
  definición \isa{{\isadigit{4}}{\isachardot}{\isadigit{1}}{\isachardot}{\isadigit{1}}}. Se define el \isa{límite\ de\ la\ sucesión\ de\ conjuntos\ de\ C\ a\ partir\ de\ S} como 
  $\bigcup_{n = 0}^{\infty} S_{n}$
 \end{definicion}

  La definición del límite se formaliza utilizando la unión generalizada de Isabelle como sigue.%
\end{isamarkuptext}\isamarkuptrue%
\isacommand{definition}\isamarkupfalse%
\ {\isachardoublequoteopen}pcp{\isacharunderscore}lim\ C\ S\ {\isasymequiv}\ {\isasymUnion}{\isacharbraceleft}pcp{\isacharunderscore}seq\ C\ S\ n{\isacharbar}n{\isachardot}\ True{\isacharbraceright}{\isachardoublequoteclose}%
\begin{isamarkuptext}%
Veamos el primer resultado sobre el límite.

\begin{lema}
  Sea \isa{C} una colección de conjuntos, \isa{S\ {\isasymin}\ C} y \isa{{\isacharbraceleft}S\isactrlsub n{\isacharbraceright}} la sucesión de conjuntos de \isa{C} a partir de
  \isa{S} según la definición \isa{{\isadigit{4}}{\isachardot}{\isadigit{1}}{\isachardot}{\isadigit{1}}}. Entonces, para todo \isa{n\ {\isasymin}\ {\isasymnat}}, se verifica:

  $S_{n} \subseteq \bigcup_{n = 0}^{\infty} S_{n}$
\end{lema}

\begin{demostracion}
  El resultado se obtiene de manera inmediata ya que, para todo \isa{n\ {\isasymin}\ {\isasymnat}}, se verifica que 
  $S_{n} \in \{S_{n}\}_{n = 0}^{\infty}$. Por tanto, es claro que 
  $S_{n} \subseteq \bigcup_{n = 0}^{\infty} S_{n}$.
\end{demostracion}

  Su formalización y demostración detallada en Isabelle se muestran a continuación.%
\end{isamarkuptext}\isamarkuptrue%
\isacommand{lemma}\isamarkupfalse%
\ {\isachardoublequoteopen}pcp{\isacharunderscore}seq\ C\ S\ n\ {\isasymsubseteq}\ pcp{\isacharunderscore}lim\ C\ S{\isachardoublequoteclose}\isanewline
%
\isadelimproof
\ \ %
\endisadelimproof
%
\isatagproof
\isacommand{unfolding}\isamarkupfalse%
\ pcp{\isacharunderscore}lim{\isacharunderscore}def\isanewline
\isacommand{proof}\isamarkupfalse%
\ {\isacharminus}\isanewline
\ \ \isacommand{have}\isamarkupfalse%
\ {\isachardoublequoteopen}n\ {\isasymin}\ {\isacharbraceleft}n\ {\isacharbar}\ n{\isachardot}\ True{\isacharbraceright}{\isachardoublequoteclose}\ \isanewline
\ \ \ \ \isacommand{by}\isamarkupfalse%
\ {\isacharparenleft}simp\ only{\isacharcolon}\ simp{\isacharunderscore}thms{\isacharparenleft}{\isadigit{2}}{\isadigit{1}}{\isacharcomma}{\isadigit{3}}{\isadigit{8}}{\isacharparenright}\ Collect{\isacharunderscore}const\ if{\isacharunderscore}True\ UNIV{\isacharunderscore}I{\isacharparenright}\ \isanewline
\ \ \isacommand{then}\isamarkupfalse%
\ \isacommand{have}\isamarkupfalse%
\ {\isachardoublequoteopen}pcp{\isacharunderscore}seq\ C\ S\ n\ {\isasymin}\ {\isacharparenleft}pcp{\isacharunderscore}seq\ C\ S{\isacharparenright}{\isacharbackquote}{\isacharbraceleft}n\ {\isacharbar}\ n{\isachardot}\ True{\isacharbraceright}{\isachardoublequoteclose}\isanewline
\ \ \ \ \isacommand{by}\isamarkupfalse%
\ {\isacharparenleft}simp\ only{\isacharcolon}\ imageI{\isacharparenright}\isanewline
\ \ \isacommand{then}\isamarkupfalse%
\ \isacommand{have}\isamarkupfalse%
\ {\isachardoublequoteopen}pcp{\isacharunderscore}seq\ C\ S\ n\ {\isasymin}\ {\isacharbraceleft}pcp{\isacharunderscore}seq\ C\ S\ n\ {\isacharbar}\ n{\isachardot}\ True{\isacharbraceright}{\isachardoublequoteclose}\isanewline
\ \ \ \ \isacommand{by}\isamarkupfalse%
\ {\isacharparenleft}simp\ only{\isacharcolon}\ image{\isacharunderscore}Collect\ simp{\isacharunderscore}thms{\isacharparenleft}{\isadigit{4}}{\isadigit{0}}{\isacharparenright}{\isacharparenright}\isanewline
\ \ \isacommand{thus}\isamarkupfalse%
\ {\isachardoublequoteopen}pcp{\isacharunderscore}seq\ C\ S\ n\ {\isasymsubseteq}\ {\isasymUnion}{\isacharbraceleft}pcp{\isacharunderscore}seq\ C\ S\ n\ {\isacharbar}\ n{\isachardot}\ True{\isacharbraceright}{\isachardoublequoteclose}\isanewline
\ \ \ \ \isacommand{by}\isamarkupfalse%
\ {\isacharparenleft}simp\ only{\isacharcolon}\ Union{\isacharunderscore}upper{\isacharparenright}\isanewline
\isacommand{qed}\isamarkupfalse%
%
\endisatagproof
{\isafoldproof}%
%
\isadelimproof
%
\endisadelimproof
%
\begin{isamarkuptext}%
Podemos probarlo de manera automática como sigue.%
\end{isamarkuptext}\isamarkuptrue%
\isacommand{lemma}\isamarkupfalse%
\ pcp{\isacharunderscore}seq{\isacharunderscore}sub{\isacharcolon}\ {\isachardoublequoteopen}pcp{\isacharunderscore}seq\ C\ S\ n\ {\isasymsubseteq}\ pcp{\isacharunderscore}lim\ C\ S{\isachardoublequoteclose}\ \isanewline
%
\isadelimproof
\ \ %
\endisadelimproof
%
\isatagproof
\isacommand{unfolding}\isamarkupfalse%
\ pcp{\isacharunderscore}lim{\isacharunderscore}def\ \isacommand{by}\isamarkupfalse%
\ blast%
\endisatagproof
{\isafoldproof}%
%
\isadelimproof
%
\endisadelimproof
%
\begin{isamarkuptext}%
Mostremos otro resultado. 

  \begin{lema}
    Sea \isa{C} una colección de conjuntos de fórmulas proposicionales, \isa{S\ {\isasymin}\ C} y \isa{{\isacharbraceleft}S\isactrlsub n{\isacharbraceright}} la sucesión de 
    conjuntos de \isa{C} a partir de \isa{S} según la definición \isa{{\isadigit{4}}{\isachardot}{\isadigit{1}}{\isachardot}{\isadigit{1}}}. Si \isa{F} es una fórmula tal que
    $F \in \bigcup_{n = 0}^{\infty} S_{n}$, entonces existe un \isa{k\ {\isasymin}\ {\isasymnat}} tal que \isa{F\ {\isasymin}\ S\isactrlsub k}. 
  \end{lema}

  \begin{demostracion}
    La prueba es inmediata de la definición del límite de la sucesión de conjuntos \isa{{\isacharbraceleft}S\isactrlsub n{\isacharbraceright}}: si
    \isa{F} pertenece a la unión generalizada $\bigcup_{n = 0}^{\infty} S_{n}$, entonces existe algún
    conjunto \isa{S\isactrlsub k} tal que \isa{F\ {\isasymin}\ S\isactrlsub k}. Es decir, existe \isa{k\ {\isasymin}\ {\isasymnat}} tal que \isa{F\ {\isasymin}\ S\isactrlsub k}, como queríamos
    demostrar.
  \end{demostracion} 

  Su prueba detallada en Isabelle/HOL es la siguiente.%
\end{isamarkuptext}\isamarkuptrue%
\isacommand{lemma}\isamarkupfalse%
\ \isanewline
\ \ \isakeyword{assumes}\ {\isachardoublequoteopen}F\ {\isasymin}\ pcp{\isacharunderscore}lim\ C\ S{\isachardoublequoteclose}\isanewline
\ \ \isakeyword{shows}\ {\isachardoublequoteopen}{\isasymexists}k{\isachardot}\ F\ {\isasymin}\ pcp{\isacharunderscore}seq\ C\ S\ k{\isachardoublequoteclose}\ \isanewline
%
\isadelimproof
%
\endisadelimproof
%
\isatagproof
\isacommand{proof}\isamarkupfalse%
\ {\isacharminus}\isanewline
\ \ \isacommand{have}\isamarkupfalse%
\ {\isachardoublequoteopen}F\ {\isasymin}\ {\isasymUnion}{\isacharparenleft}{\isacharparenleft}pcp{\isacharunderscore}seq\ C\ S{\isacharparenright}\ {\isacharbackquote}\ {\isacharbraceleft}n\ {\isacharbar}\ n{\isachardot}\ True{\isacharbraceright}{\isacharparenright}{\isachardoublequoteclose}\isanewline
\ \ \ \ \isacommand{using}\isamarkupfalse%
\ assms\ \isacommand{by}\isamarkupfalse%
\ {\isacharparenleft}simp\ only{\isacharcolon}\ pcp{\isacharunderscore}lim{\isacharunderscore}def\ image{\isacharunderscore}Collect\ simp{\isacharunderscore}thms{\isacharparenleft}{\isadigit{4}}{\isadigit{0}}{\isacharparenright}{\isacharparenright}\isanewline
\ \ \isacommand{then}\isamarkupfalse%
\ \isacommand{have}\isamarkupfalse%
\ {\isachardoublequoteopen}{\isasymexists}k\ {\isasymin}\ {\isacharbraceleft}n{\isachardot}\ True{\isacharbraceright}{\isachardot}\ F\ {\isasymin}\ pcp{\isacharunderscore}seq\ C\ S\ k{\isachardoublequoteclose}\isanewline
\ \ \ \ \isacommand{by}\isamarkupfalse%
\ {\isacharparenleft}simp\ only{\isacharcolon}\ UN{\isacharunderscore}iff\ simp{\isacharunderscore}thms{\isacharparenleft}{\isadigit{4}}{\isadigit{0}}{\isacharparenright}{\isacharparenright}\isanewline
\ \ \isacommand{then}\isamarkupfalse%
\ \isacommand{have}\isamarkupfalse%
\ {\isachardoublequoteopen}{\isasymexists}k\ {\isasymin}\ UNIV{\isachardot}\ F\ {\isasymin}\ pcp{\isacharunderscore}seq\ C\ S\ k{\isachardoublequoteclose}\ \isanewline
\ \ \ \ \isacommand{by}\isamarkupfalse%
\ {\isacharparenleft}simp\ only{\isacharcolon}\ UNIV{\isacharunderscore}def{\isacharparenright}\isanewline
\ \ \isacommand{thus}\isamarkupfalse%
\ {\isachardoublequoteopen}{\isasymexists}k{\isachardot}\ F\ {\isasymin}\ pcp{\isacharunderscore}seq\ C\ S\ k{\isachardoublequoteclose}\ \isanewline
\ \ \ \ \isacommand{by}\isamarkupfalse%
\ {\isacharparenleft}simp\ only{\isacharcolon}\ bex{\isacharunderscore}UNIV{\isacharparenright}\isanewline
\isacommand{qed}\isamarkupfalse%
%
\endisatagproof
{\isafoldproof}%
%
\isadelimproof
%
\endisadelimproof
%
\begin{isamarkuptext}%
Mostremos, a continuación, la demostración automática del resultado.%
\end{isamarkuptext}\isamarkuptrue%
\isacommand{lemma}\isamarkupfalse%
\ pcp{\isacharunderscore}lim{\isacharunderscore}inserted{\isacharunderscore}at{\isacharunderscore}ex{\isacharcolon}\ \isanewline
\ \ \ \ {\isachardoublequoteopen}S{\isacharprime}\ {\isasymin}\ pcp{\isacharunderscore}lim\ C\ S\ {\isasymLongrightarrow}\ {\isasymexists}k{\isachardot}\ S{\isacharprime}\ {\isasymin}\ pcp{\isacharunderscore}seq\ C\ S\ k{\isachardoublequoteclose}\isanewline
%
\isadelimproof
\ \ %
\endisadelimproof
%
\isatagproof
\isacommand{unfolding}\isamarkupfalse%
\ pcp{\isacharunderscore}lim{\isacharunderscore}def\ \isacommand{by}\isamarkupfalse%
\ blast%
\endisatagproof
{\isafoldproof}%
%
\isadelimproof
%
\endisadelimproof
%
\begin{isamarkuptext}%
Por último, veamos la siguiente propiedad sobre conjuntos finitos contenidos en el límite de 
  las sucesiones definido en \isa{{\isadigit{4}}{\isachardot}{\isadigit{1}}{\isachardot}{\isadigit{5}}}.

\begin{lema}
  Sea \isa{C} una colección, \isa{S\ {\isasymin}\ C} y \isa{{\isacharbraceleft}S\isactrlsub n{\isacharbraceright}} la sucesión de conjuntos de \isa{C} a partir de \isa{S} según la
  definición \isa{{\isadigit{4}}{\isachardot}{\isadigit{1}}{\isachardot}{\isadigit{1}}}. Si \isa{S{\isacharprime}} es un conjunto finito tal que \isa{S{\isacharprime}\ {\isasymsubseteq}} $\bigcup_{n = 0}^{\infty} S_{n}$, 
  entonces existe un\\ \isa{k\ {\isasymin}\ {\isasymnat}} tal que \isa{S{\isacharprime}\ {\isasymsubseteq}\ S\isactrlsub k}.
\end{lema}

\begin{demostracion}
  La prueba del resultado se realiza por inducción sobre la estructura recursiva de los conjuntos 
  finitos.

  En primer lugar, probemos el caso base correspondiente al conjunto vacío. Supongamos que \isa{{\isacharbraceleft}{\isacharbraceright}} está 
  contenido en el límite de la sucesión de conjuntos de \isa{C} a partir de \isa{S}. Como \isa{{\isacharbraceleft}{\isacharbraceright}} es 
  subconjunto de todo conjunto, en particular lo es de \isa{S\ {\isacharequal}\ S\isactrlsub {\isadigit{0}}}, probando así el primer caso.

  Por otra parte, probemos el paso de inducción. Sea \isa{S{\isacharprime}} un conjunto finito verificando la 
  hipótesis de inducción: si \isa{S{\isacharprime}} está contenido en el límite de la sucesión de conjuntos de 
  \isa{C} a partir de \isa{S}, entonces también está contenido en algún \isa{S\isactrlsub k\isactrlsub {\isacharprime}} para cierto \isa{k{\isacharprime}\ {\isasymin}\ {\isasymnat}}. Sea 
  \isa{F} una fórmula tal que \isa{F\ {\isasymnotin}\ S{\isacharprime}}. Vamos a probar que si \isa{{\isacharbraceleft}F{\isacharbraceright}\ {\isasymunion}\ S{\isacharprime}} está contenido en el límite, 
  entonces está contenido en \isa{S\isactrlsub k} para algún \isa{k\ {\isasymin}\ {\isasymnat}}. 

  Como hemos supuesto que \isa{{\isacharbraceleft}F{\isacharbraceright}\ {\isasymunion}\ S{\isacharprime}} está contenido en el límite, entonces se verifica que \isa{F}
  pertenece al límite y \isa{S{\isacharprime}} está contenido en él. Por el lema \isa{{\isadigit{4}}{\isachardot}{\isadigit{1}}{\isachardot}{\isadigit{7}}}, como \isa{F} pertenece al 
  límite, deducimos que existe un \isa{k\ {\isasymin}\ {\isasymnat}} tal que \isa{F\ {\isasymin}\ S\isactrlsub k}. Por otro lado, como \isa{S{\isacharprime}} está contenido
  en el límite, por hipótesis de inducción existe algún \isa{k{\isacharprime}\ {\isasymin}\ {\isasymnat}} tal que \isa{S{\isacharprime}\ {\isasymsubseteq}\ S\isactrlsub k\isactrlsub {\isacharprime}}. El resultado 
  se obtiene considerando el máximo entre \isa{k} y \isa{k{\isacharprime}}, que notaremos por \isa{k{\isacharprime}{\isacharprime}}. En efecto, por la 
  monotonía de la sucesión, se verifica que tanto \isa{S\isactrlsub k} como \isa{S\isactrlsub k\isactrlsub {\isacharprime}} están contenidos en \isa{S\isactrlsub k\isactrlsub {\isacharprime}\isactrlsub {\isacharprime}}. De este 
  modo, como \isa{S{\isacharprime}\ {\isasymsubseteq}\ S\isactrlsub k\isactrlsub {\isacharprime}}, por la transitividad de la contención de conjuntos se tiene que 
  \isa{S{\isacharprime}\ {\isasymsubseteq}\ S\isactrlsub k\isactrlsub {\isacharprime}\isactrlsub {\isacharprime}}. Además, como \isa{F\ {\isasymin}\ S\isactrlsub k}, se tiene que \isa{F\ {\isasymin}\ S\isactrlsub k\isactrlsub {\isacharprime}\isactrlsub {\isacharprime}}. Por lo tanto, \isa{{\isacharbraceleft}F{\isacharbraceright}\ {\isasymunion}\ S{\isacharprime}\ {\isasymsubseteq}\ S\isactrlsub k\isactrlsub {\isacharprime}\isactrlsub {\isacharprime}}, como 
  queríamos demostrar. 
\end{demostracion}

  Procedamos con la demostración detallada en Isabelle.%
\end{isamarkuptext}\isamarkuptrue%
\isacommand{lemma}\isamarkupfalse%
\ \isanewline
\ \ \isakeyword{assumes}\ {\isachardoublequoteopen}finite\ S{\isacharprime}{\isachardoublequoteclose}\isanewline
\ \ \ \ \ \ \ \ \ \ {\isachardoublequoteopen}S{\isacharprime}\ {\isasymsubseteq}\ pcp{\isacharunderscore}lim\ C\ S{\isachardoublequoteclose}\isanewline
\ \ \ \ \ \ \ \ \isakeyword{shows}\ {\isachardoublequoteopen}{\isasymexists}k{\isachardot}\ S{\isacharprime}\ {\isasymsubseteq}\ pcp{\isacharunderscore}seq\ C\ S\ k{\isachardoublequoteclose}\isanewline
%
\isadelimproof
\ \ %
\endisadelimproof
%
\isatagproof
\isacommand{using}\isamarkupfalse%
\ assms\isanewline
\isacommand{proof}\isamarkupfalse%
\ {\isacharparenleft}induction\ S{\isacharprime}\ rule{\isacharcolon}\ finite{\isacharunderscore}induct{\isacharparenright}\isanewline
\ \ \isacommand{case}\isamarkupfalse%
\ empty\isanewline
\ \ \isacommand{have}\isamarkupfalse%
\ {\isachardoublequoteopen}pcp{\isacharunderscore}seq\ C\ S\ {\isadigit{0}}\ {\isacharequal}\ S{\isachardoublequoteclose}\isanewline
\ \ \ \ \isacommand{by}\isamarkupfalse%
\ {\isacharparenleft}simp\ only{\isacharcolon}\ pcp{\isacharunderscore}seq{\isachardot}simps{\isacharparenleft}{\isadigit{1}}{\isacharparenright}{\isacharparenright}\isanewline
\ \ \isacommand{have}\isamarkupfalse%
\ {\isachardoublequoteopen}{\isacharbraceleft}{\isacharbraceright}\ {\isasymsubseteq}\ S{\isachardoublequoteclose}\isanewline
\ \ \ \ \isacommand{by}\isamarkupfalse%
\ {\isacharparenleft}rule\ order{\isacharunderscore}bot{\isacharunderscore}class{\isachardot}bot{\isachardot}extremum{\isacharparenright}\isanewline
\ \ \isacommand{then}\isamarkupfalse%
\ \isacommand{have}\isamarkupfalse%
\ {\isachardoublequoteopen}{\isacharbraceleft}{\isacharbraceright}\ {\isasymsubseteq}\ pcp{\isacharunderscore}seq\ C\ S\ {\isadigit{0}}{\isachardoublequoteclose}\isanewline
\ \ \ \ \isacommand{by}\isamarkupfalse%
\ {\isacharparenleft}simp\ only{\isacharcolon}\ {\isacartoucheopen}pcp{\isacharunderscore}seq\ C\ S\ {\isadigit{0}}\ {\isacharequal}\ S{\isacartoucheclose}{\isacharparenright}\isanewline
\ \ \isacommand{then}\isamarkupfalse%
\ \isacommand{show}\isamarkupfalse%
\ {\isacharquery}case\ \isanewline
\ \ \ \ \isacommand{by}\isamarkupfalse%
\ {\isacharparenleft}rule\ exI{\isacharparenright}\isanewline
\isacommand{next}\isamarkupfalse%
\isanewline
\ \ \isacommand{case}\isamarkupfalse%
\ {\isacharparenleft}insert\ F\ S{\isacharprime}{\isacharparenright}\isanewline
\ \ \isacommand{then}\isamarkupfalse%
\ \isacommand{have}\isamarkupfalse%
\ {\isachardoublequoteopen}insert\ F\ S{\isacharprime}\ {\isasymsubseteq}\ pcp{\isacharunderscore}lim\ C\ S{\isachardoublequoteclose}\isanewline
\ \ \ \ \isacommand{by}\isamarkupfalse%
\ {\isacharparenleft}simp\ only{\isacharcolon}\ insert{\isachardot}prems{\isacharparenright}\isanewline
\ \ \isacommand{then}\isamarkupfalse%
\ \isacommand{have}\isamarkupfalse%
\ C{\isacharcolon}{\isachardoublequoteopen}F\ {\isasymin}\ {\isacharparenleft}pcp{\isacharunderscore}lim\ C\ S{\isacharparenright}\ {\isasymand}\ S{\isacharprime}\ {\isasymsubseteq}\ pcp{\isacharunderscore}lim\ C\ S{\isachardoublequoteclose}\isanewline
\ \ \ \ \isacommand{by}\isamarkupfalse%
\ {\isacharparenleft}simp\ only{\isacharcolon}\ insert{\isacharunderscore}subset{\isacharparenright}\ \isanewline
\ \ \isacommand{then}\isamarkupfalse%
\ \isacommand{have}\isamarkupfalse%
\ {\isachardoublequoteopen}S{\isacharprime}\ {\isasymsubseteq}\ pcp{\isacharunderscore}lim\ C\ S{\isachardoublequoteclose}\isanewline
\ \ \ \ \isacommand{by}\isamarkupfalse%
\ {\isacharparenleft}rule\ conjunct{\isadigit{2}}{\isacharparenright}\isanewline
\ \ \isacommand{then}\isamarkupfalse%
\ \isacommand{have}\isamarkupfalse%
\ EX{\isadigit{1}}{\isacharcolon}{\isachardoublequoteopen}{\isasymexists}k{\isachardot}\ S{\isacharprime}\ {\isasymsubseteq}\ pcp{\isacharunderscore}seq\ C\ S\ k{\isachardoublequoteclose}\isanewline
\ \ \ \ \isacommand{by}\isamarkupfalse%
\ {\isacharparenleft}simp\ only{\isacharcolon}\ insert{\isachardot}IH{\isacharparenright}\isanewline
\ \ \isacommand{obtain}\isamarkupfalse%
\ k{\isadigit{1}}\ \isakeyword{where}\ {\isachardoublequoteopen}S{\isacharprime}\ {\isasymsubseteq}\ pcp{\isacharunderscore}seq\ C\ S\ k{\isadigit{1}}{\isachardoublequoteclose}\isanewline
\ \ \ \ \isacommand{using}\isamarkupfalse%
\ EX{\isadigit{1}}\ \isacommand{by}\isamarkupfalse%
\ {\isacharparenleft}rule\ exE{\isacharparenright}\isanewline
\ \ \isacommand{have}\isamarkupfalse%
\ {\isachardoublequoteopen}F\ {\isasymin}\ pcp{\isacharunderscore}lim\ C\ S{\isachardoublequoteclose}\isanewline
\ \ \ \ \isacommand{using}\isamarkupfalse%
\ C\ \isacommand{by}\isamarkupfalse%
\ {\isacharparenleft}rule\ conjunct{\isadigit{1}}{\isacharparenright}\isanewline
\ \ \isacommand{then}\isamarkupfalse%
\ \isacommand{have}\isamarkupfalse%
\ EX{\isadigit{2}}{\isacharcolon}{\isachardoublequoteopen}{\isasymexists}k{\isachardot}\ F\ {\isasymin}\ pcp{\isacharunderscore}seq\ C\ S\ k{\isachardoublequoteclose}\isanewline
\ \ \ \ \isacommand{by}\isamarkupfalse%
\ {\isacharparenleft}rule\ pcp{\isacharunderscore}lim{\isacharunderscore}inserted{\isacharunderscore}at{\isacharunderscore}ex{\isacharparenright}\isanewline
\ \ \isacommand{obtain}\isamarkupfalse%
\ k{\isadigit{2}}\ \isakeyword{where}\ {\isachardoublequoteopen}F\ {\isasymin}\ pcp{\isacharunderscore}seq\ C\ S\ k{\isadigit{2}}{\isachardoublequoteclose}\ \isanewline
\ \ \ \ \isacommand{using}\isamarkupfalse%
\ EX{\isadigit{2}}\ \isacommand{by}\isamarkupfalse%
\ {\isacharparenleft}rule\ exE{\isacharparenright}\isanewline
\ \ \isacommand{have}\isamarkupfalse%
\ {\isachardoublequoteopen}k{\isadigit{1}}\ {\isasymle}\ max\ k{\isadigit{1}}\ k{\isadigit{2}}{\isachardoublequoteclose}\isanewline
\ \ \ \ \isacommand{by}\isamarkupfalse%
\ {\isacharparenleft}simp\ only{\isacharcolon}\ linorder{\isacharunderscore}class{\isachardot}max{\isachardot}cobounded{\isadigit{1}}{\isacharparenright}\isanewline
\ \ \isacommand{then}\isamarkupfalse%
\ \isacommand{have}\isamarkupfalse%
\ {\isachardoublequoteopen}pcp{\isacharunderscore}seq\ C\ S\ k{\isadigit{1}}\ {\isasymsubseteq}\ pcp{\isacharunderscore}seq\ C\ S\ {\isacharparenleft}max\ k{\isadigit{1}}\ k{\isadigit{2}}{\isacharparenright}{\isachardoublequoteclose}\isanewline
\ \ \ \ \isacommand{by}\isamarkupfalse%
\ {\isacharparenleft}rule\ pcp{\isacharunderscore}seq{\isacharunderscore}mono{\isacharparenright}\isanewline
\ \ \isacommand{have}\isamarkupfalse%
\ {\isachardoublequoteopen}k{\isadigit{2}}\ {\isasymle}\ max\ k{\isadigit{1}}\ k{\isadigit{2}}{\isachardoublequoteclose}\isanewline
\ \ \ \ \isacommand{by}\isamarkupfalse%
\ {\isacharparenleft}simp\ only{\isacharcolon}\ linorder{\isacharunderscore}class{\isachardot}max{\isachardot}cobounded{\isadigit{2}}{\isacharparenright}\isanewline
\ \ \isacommand{then}\isamarkupfalse%
\ \isacommand{have}\isamarkupfalse%
\ {\isachardoublequoteopen}pcp{\isacharunderscore}seq\ C\ S\ k{\isadigit{2}}\ {\isasymsubseteq}\ pcp{\isacharunderscore}seq\ C\ S\ {\isacharparenleft}max\ k{\isadigit{1}}\ k{\isadigit{2}}{\isacharparenright}{\isachardoublequoteclose}\isanewline
\ \ \ \ \isacommand{by}\isamarkupfalse%
\ {\isacharparenleft}rule\ pcp{\isacharunderscore}seq{\isacharunderscore}mono{\isacharparenright}\isanewline
\ \ \isacommand{have}\isamarkupfalse%
\ {\isachardoublequoteopen}S{\isacharprime}\ {\isasymsubseteq}\ pcp{\isacharunderscore}seq\ C\ S\ {\isacharparenleft}max\ k{\isadigit{1}}\ k{\isadigit{2}}{\isacharparenright}{\isachardoublequoteclose}\isanewline
\ \ \ \ \isacommand{using}\isamarkupfalse%
\ {\isacartoucheopen}S{\isacharprime}\ {\isasymsubseteq}\ pcp{\isacharunderscore}seq\ C\ S\ k{\isadigit{1}}{\isacartoucheclose}\ {\isacartoucheopen}pcp{\isacharunderscore}seq\ C\ S\ k{\isadigit{1}}\ {\isasymsubseteq}\ pcp{\isacharunderscore}seq\ C\ S\ {\isacharparenleft}max\ k{\isadigit{1}}\ k{\isadigit{2}}{\isacharparenright}{\isacartoucheclose}\ \isacommand{by}\isamarkupfalse%
\ {\isacharparenleft}rule\ subset{\isacharunderscore}trans{\isacharparenright}\isanewline
\ \ \isacommand{have}\isamarkupfalse%
\ {\isachardoublequoteopen}F\ {\isasymin}\ pcp{\isacharunderscore}seq\ C\ S\ {\isacharparenleft}max\ k{\isadigit{1}}\ k{\isadigit{2}}{\isacharparenright}{\isachardoublequoteclose}\isanewline
\ \ \ \ \isacommand{using}\isamarkupfalse%
\ {\isacartoucheopen}F\ {\isasymin}\ pcp{\isacharunderscore}seq\ C\ S\ k{\isadigit{2}}{\isacartoucheclose}\ {\isacartoucheopen}pcp{\isacharunderscore}seq\ C\ S\ k{\isadigit{2}}\ {\isasymsubseteq}\ pcp{\isacharunderscore}seq\ C\ S\ {\isacharparenleft}max\ k{\isadigit{1}}\ k{\isadigit{2}}{\isacharparenright}{\isacartoucheclose}\ \isacommand{by}\isamarkupfalse%
\ {\isacharparenleft}rule\ rev{\isacharunderscore}subsetD{\isacharparenright}\isanewline
\ \ \isacommand{then}\isamarkupfalse%
\ \isacommand{have}\isamarkupfalse%
\ {\isadigit{1}}{\isacharcolon}{\isachardoublequoteopen}insert\ F\ S{\isacharprime}\ {\isasymsubseteq}\ pcp{\isacharunderscore}seq\ C\ S\ {\isacharparenleft}max\ k{\isadigit{1}}\ k{\isadigit{2}}{\isacharparenright}{\isachardoublequoteclose}\isanewline
\ \ \ \ \isacommand{using}\isamarkupfalse%
\ {\isacartoucheopen}S{\isacharprime}\ {\isasymsubseteq}\ pcp{\isacharunderscore}seq\ C\ S\ {\isacharparenleft}max\ k{\isadigit{1}}\ k{\isadigit{2}}{\isacharparenright}{\isacartoucheclose}\ \isacommand{by}\isamarkupfalse%
\ {\isacharparenleft}simp\ only{\isacharcolon}\ insert{\isacharunderscore}subset{\isacharparenright}\isanewline
\ \ \isacommand{thus}\isamarkupfalse%
\ {\isacharquery}case\isanewline
\ \ \ \ \isacommand{by}\isamarkupfalse%
\ {\isacharparenleft}rule\ exI{\isacharparenright}\isanewline
\isacommand{qed}\isamarkupfalse%
%
\endisatagproof
{\isafoldproof}%
%
\isadelimproof
%
\endisadelimproof
%
\begin{isamarkuptext}%
Finalmente, su demostración automática en Isabelle/HOL es la siguiente.%
\end{isamarkuptext}\isamarkuptrue%
\isacommand{lemma}\isamarkupfalse%
\ finite{\isacharunderscore}pcp{\isacharunderscore}lim{\isacharunderscore}EX{\isacharcolon}\isanewline
\ \ \isakeyword{assumes}\ {\isachardoublequoteopen}finite\ S{\isacharprime}{\isachardoublequoteclose}\isanewline
\ \ \ \ \ \ \ \ \ \ {\isachardoublequoteopen}S{\isacharprime}\ {\isasymsubseteq}\ pcp{\isacharunderscore}lim\ C\ S{\isachardoublequoteclose}\isanewline
\ \ \ \ \ \ \ \ \isakeyword{shows}\ {\isachardoublequoteopen}{\isasymexists}k{\isachardot}\ S{\isacharprime}\ {\isasymsubseteq}\ pcp{\isacharunderscore}seq\ C\ S\ k{\isachardoublequoteclose}\isanewline
%
\isadelimproof
\ \ %
\endisadelimproof
%
\isatagproof
\isacommand{using}\isamarkupfalse%
\ assms\isanewline
\isacommand{proof}\isamarkupfalse%
{\isacharparenleft}induction\ S{\isacharprime}\ rule{\isacharcolon}\ finite{\isacharunderscore}induct{\isacharparenright}\ \isanewline
\ \ \isacommand{case}\isamarkupfalse%
\ {\isacharparenleft}insert\ F\ S{\isacharprime}{\isacharparenright}\isanewline
\ \ \isacommand{hence}\isamarkupfalse%
\ {\isachardoublequoteopen}{\isasymexists}k{\isachardot}\ S{\isacharprime}\ {\isasymsubseteq}\ pcp{\isacharunderscore}seq\ C\ S\ k{\isachardoublequoteclose}\ \isacommand{by}\isamarkupfalse%
\ fast\isanewline
\ \ \isacommand{then}\isamarkupfalse%
\ \isacommand{guess}\isamarkupfalse%
\ k{\isadigit{1}}\ \isacommand{{\isachardot}{\isachardot}}\isamarkupfalse%
\isanewline
\ \ \isacommand{moreover}\isamarkupfalse%
\ \isacommand{obtain}\isamarkupfalse%
\ k{\isadigit{2}}\ \isakeyword{where}\ {\isachardoublequoteopen}F\ {\isasymin}\ pcp{\isacharunderscore}seq\ C\ S\ k{\isadigit{2}}{\isachardoublequoteclose}\isanewline
\ \ \ \ \isacommand{by}\isamarkupfalse%
\ {\isacharparenleft}meson\ pcp{\isacharunderscore}lim{\isacharunderscore}inserted{\isacharunderscore}at{\isacharunderscore}ex\ insert{\isachardot}prems\ insert{\isacharunderscore}subset{\isacharparenright}\isanewline
\ \ \isacommand{ultimately}\isamarkupfalse%
\ \isacommand{have}\isamarkupfalse%
\ {\isachardoublequoteopen}insert\ F\ S{\isacharprime}\ {\isasymsubseteq}\ pcp{\isacharunderscore}seq\ C\ S\ {\isacharparenleft}max\ k{\isadigit{1}}\ k{\isadigit{2}}{\isacharparenright}{\isachardoublequoteclose}\isanewline
\ \ \ \ \isacommand{by}\isamarkupfalse%
\ {\isacharparenleft}meson\ pcp{\isacharunderscore}seq{\isacharunderscore}mono\ dual{\isacharunderscore}order{\isachardot}trans\ insert{\isacharunderscore}subset\ max{\isachardot}bounded{\isacharunderscore}iff\ order{\isacharunderscore}refl\ subsetCE{\isacharparenright}\isanewline
\ \ \isacommand{thus}\isamarkupfalse%
\ {\isacharquery}case\ \isacommand{by}\isamarkupfalse%
\ blast\isanewline
\isacommand{qed}\isamarkupfalse%
\ simp%
\endisatagproof
{\isafoldproof}%
%
\isadelimproof
%
\endisadelimproof
%
\isadelimdocument
%
\endisadelimdocument
%
\isatagdocument
%
\isamarkupsection{El Teorema de Existencia de Modelo%
}
\isamarkuptrue%
%
\endisatagdocument
{\isafolddocument}%
%
\isadelimdocument
%
\endisadelimdocument
%
\begin{isamarkuptext}%
En esta sección demostraremos finalmente el \isa{Teorema\ de\ Existencia\ de\ Modelo}, el cual prueba 
  que todo conjunto de fórmulas \isa{S} perteneciente a una colección \isa{C} que verifique la propiedad de 
  consistencia proposicional es satisfacible. Para ello, extenderemos la colección \isa{C} a otra \isa{C{\isacharprime}} 
  que tenga la propiedad de consistencia proposicional, sea cerrada bajo subconjuntos y sea de 
  carácter finito. De este modo, introduciremos distintos resultados sobre colecciones \isa{C{\isacharprime}} con 
  las características anteriores. En primer lugar, probaremos que el límite de la sucesión definida
  según la definición \isa{{\isadigit{4}}{\isachardot}{\isadigit{1}}{\isachardot}{\isadigit{1}}} pertenece a la colección \isa{C{\isacharprime}} que la define si esta verifica la 
  propiedad de consistencia proposicional, es cerrada bajo subconjuntos y es de carácter finito. De 
  hecho, probaremos que dicho límite es un elemento maximal de la colección que lo define si esta es 
  cerrada bajo subconjuntos y verifica la propiedad de consistencia proposicional. Por otra parte 
  demostraremos que el límite se trata de un conjunto de \isa{Hintikka} si está definido a partir de una 
  colección \isa{C{\isacharprime}} que tenga la propiedad de consistencia proposicional, sea cerrada bajo subconjuntos 
  y sea de carácter finito. Por tanto, por el \isa{Teorema\ de\ Hintikka}, en las condiciones anteriores 
  el límite es un conjunto satisfacible. Finalmente, como \isa{S\ {\isasymin}\ C} pertenece también a la extensión 
  \isa{C{\isacharprime}}, se verifica que está contenido en el límite de la sucesión definida según la definición 
  \isa{{\isadigit{4}}{\isachardot}{\isadigit{1}}{\isachardot}{\isadigit{1}}} a partir de \isa{C{\isacharprime}} y \isa{S\ {\isasymin}\ C{\isacharprime}}. Por tanto, como \isa{C{\isacharprime}} verifica las características anteriores, 
  quedará demostrada la satisfacibilidad del conjunto \isa{S} al heredarla por contención del límite, lo 
  que prueba el \isa{Teorema\ de\ Existencia\ de\ Modelo}.

  En primer lugar, probemos que si \isa{C} es una colección que verifica la propiedad de 
  consistencia proposicional, es cerrada bajo subconjuntos y es de carácter finito, entonces el 
  límite de toda sucesión de conjuntos de \isa{C} según la definición \isa{{\isadigit{4}}{\isachardot}{\isadigit{1}}{\isachardot}{\isadigit{1}}} pertenece a \isa{C}.

  \begin{lema}
    Sea \isa{C} una colección de conjuntos que verifica la propiedad de consistencia proposicional, es 
    cerrada bajo subconjuntos y es de carácter finito. Sea \isa{S\ {\isasymin}\ C} y \isa{{\isacharbraceleft}S\isactrlsub n{\isacharbraceright}} la sucesión de conjuntos
    de \isa{C} a partir de \isa{S} según la definición \isa{{\isadigit{4}}{\isachardot}{\isadigit{1}}{\isachardot}{\isadigit{1}}}. Entonces, el límite de la sucesión está en
    \isa{C}.
  \end{lema}

  \begin{demostracion}
    Por definición, como \isa{C} es de carácter finito, para todo conjunto son equivalentes:
    \begin{enumerate}
      \item El conjunto pertenece a \isa{C}.
      \item Todo subconjunto finito suyo pertenece a \isa{C}.
    \end{enumerate}

    De este modo, para demostrar que el límite de la sucesión \isa{{\isacharbraceleft}S\isactrlsub n{\isacharbraceright}} pertenece a \isa{C}, basta probar
    que todo subconjunto finito suyo está en \isa{C}.

    Sea \isa{S{\isacharprime}} un subconjunto finito del límite de la sucesión. Por el lema \isa{{\isadigit{1}}{\isachardot}{\isadigit{4}}{\isachardot}{\isadigit{8}}}, existe un
    \isa{k\ {\isasymin}\ {\isasymnat}} tal que \isa{S{\isacharprime}\ {\isasymsubseteq}\ S\isactrlsub k}. Por tanto, como \isa{S\isactrlsub k\ {\isasymin}\ C} para todo \isa{k\ {\isasymin}\ {\isasymnat}} y \isa{C} es cerrada bajo
    subconjuntos, por definición se tiene que \isa{S{\isacharprime}\ {\isasymin}\ C}, como queríamos demostrar.
  \end{demostracion}

  En Isabelle se formaliza y demuestra detalladamente como sigue.%
\end{isamarkuptext}\isamarkuptrue%
\isacommand{lemma}\isamarkupfalse%
\isanewline
\ \ \isakeyword{assumes}\ {\isachardoublequoteopen}pcp\ C{\isachardoublequoteclose}\isanewline
\ \ \ \ \ \ \ \ \ \ {\isachardoublequoteopen}S\ {\isasymin}\ C{\isachardoublequoteclose}\isanewline
\ \ \ \ \ \ \ \ \ \ {\isachardoublequoteopen}subset{\isacharunderscore}closed\ C{\isachardoublequoteclose}\isanewline
\ \ \ \ \ \ \ \ \ \ {\isachardoublequoteopen}finite{\isacharunderscore}character\ C{\isachardoublequoteclose}\isanewline
\ \ \isakeyword{shows}\ {\isachardoublequoteopen}pcp{\isacharunderscore}lim\ C\ S\ {\isasymin}\ C{\isachardoublequoteclose}\ \isanewline
%
\isadelimproof
%
\endisadelimproof
%
\isatagproof
\isacommand{proof}\isamarkupfalse%
\ {\isacharminus}\isanewline
\ \ \isacommand{have}\isamarkupfalse%
\ {\isachardoublequoteopen}{\isasymforall}S{\isachardot}\ S\ {\isasymin}\ C\ {\isasymlongleftrightarrow}\ {\isacharparenleft}{\isasymforall}S{\isacharprime}\ {\isasymsubseteq}\ S{\isachardot}\ finite\ S{\isacharprime}\ {\isasymlongrightarrow}\ S{\isacharprime}\ {\isasymin}\ C{\isacharparenright}{\isachardoublequoteclose}\isanewline
\ \ \ \ \isacommand{using}\isamarkupfalse%
\ assms{\isacharparenleft}{\isadigit{4}}{\isacharparenright}\ \isacommand{unfolding}\isamarkupfalse%
\ finite{\isacharunderscore}character{\isacharunderscore}def\ \isacommand{by}\isamarkupfalse%
\ this\isanewline
\ \ \isacommand{then}\isamarkupfalse%
\ \isacommand{have}\isamarkupfalse%
\ FC{\isadigit{1}}{\isacharcolon}{\isachardoublequoteopen}pcp{\isacharunderscore}lim\ C\ S\ {\isasymin}\ C\ {\isasymlongleftrightarrow}\ {\isacharparenleft}{\isasymforall}S{\isacharprime}\ {\isasymsubseteq}\ {\isacharparenleft}pcp{\isacharunderscore}lim\ C\ S{\isacharparenright}{\isachardot}\ finite\ S{\isacharprime}\ {\isasymlongrightarrow}\ S{\isacharprime}\ {\isasymin}\ C{\isacharparenright}{\isachardoublequoteclose}\isanewline
\ \ \ \ \isacommand{by}\isamarkupfalse%
\ {\isacharparenleft}rule\ allE{\isacharparenright}\isanewline
\ \ \isacommand{have}\isamarkupfalse%
\ SC{\isacharcolon}{\isachardoublequoteopen}{\isasymforall}S\ {\isasymin}\ C{\isachardot}\ {\isasymforall}S{\isacharprime}{\isasymsubseteq}S{\isachardot}\ S{\isacharprime}\ {\isasymin}\ C{\isachardoublequoteclose}\isanewline
\ \ \ \ \isacommand{using}\isamarkupfalse%
\ assms{\isacharparenleft}{\isadigit{3}}{\isacharparenright}\ \isacommand{unfolding}\isamarkupfalse%
\ subset{\isacharunderscore}closed{\isacharunderscore}def\ \isacommand{by}\isamarkupfalse%
\ this\isanewline
\ \ \isacommand{have}\isamarkupfalse%
\ FC{\isadigit{2}}{\isacharcolon}{\isachardoublequoteopen}{\isasymforall}S{\isacharprime}\ {\isasymsubseteq}\ pcp{\isacharunderscore}lim\ C\ S{\isachardot}\ finite\ S{\isacharprime}\ {\isasymlongrightarrow}\ S{\isacharprime}\ {\isasymin}\ C{\isachardoublequoteclose}\isanewline
\ \ \isacommand{proof}\isamarkupfalse%
\ {\isacharparenleft}rule\ sallI{\isacharparenright}\isanewline
\ \ \ \ \isacommand{fix}\isamarkupfalse%
\ S{\isacharprime}\ {\isacharcolon}{\isacharcolon}\ {\isachardoublequoteopen}{\isacharprime}a\ formula\ set{\isachardoublequoteclose}\isanewline
\ \ \ \ \isacommand{assume}\isamarkupfalse%
\ {\isachardoublequoteopen}S{\isacharprime}\ {\isasymsubseteq}\ pcp{\isacharunderscore}lim\ C\ S{\isachardoublequoteclose}\isanewline
\ \ \ \ \isacommand{show}\isamarkupfalse%
\ {\isachardoublequoteopen}finite\ S{\isacharprime}\ {\isasymlongrightarrow}\ S{\isacharprime}\ {\isasymin}\ C{\isachardoublequoteclose}\isanewline
\ \ \ \ \isacommand{proof}\isamarkupfalse%
\ {\isacharparenleft}rule\ impI{\isacharparenright}\isanewline
\ \ \ \ \ \ \isacommand{assume}\isamarkupfalse%
\ {\isachardoublequoteopen}finite\ S{\isacharprime}{\isachardoublequoteclose}\isanewline
\ \ \ \ \ \ \isacommand{then}\isamarkupfalse%
\ \isacommand{have}\isamarkupfalse%
\ EX{\isacharcolon}{\isachardoublequoteopen}{\isasymexists}k{\isachardot}\ S{\isacharprime}\ {\isasymsubseteq}\ pcp{\isacharunderscore}seq\ C\ S\ k{\isachardoublequoteclose}\ \isanewline
\ \ \ \ \ \ \ \ \isacommand{using}\isamarkupfalse%
\ {\isacartoucheopen}S{\isacharprime}\ {\isasymsubseteq}\ pcp{\isacharunderscore}lim\ C\ S{\isacartoucheclose}\ \isacommand{by}\isamarkupfalse%
\ {\isacharparenleft}rule\ finite{\isacharunderscore}pcp{\isacharunderscore}lim{\isacharunderscore}EX{\isacharparenright}\isanewline
\ \ \ \ \ \ \isacommand{obtain}\isamarkupfalse%
\ k\ \isakeyword{where}\ {\isachardoublequoteopen}S{\isacharprime}\ {\isasymsubseteq}\ pcp{\isacharunderscore}seq\ C\ S\ k{\isachardoublequoteclose}\isanewline
\ \ \ \ \ \ \ \ \isacommand{using}\isamarkupfalse%
\ EX\ \isacommand{by}\isamarkupfalse%
\ {\isacharparenleft}rule\ exE{\isacharparenright}\isanewline
\ \ \ \ \ \ \isacommand{have}\isamarkupfalse%
\ {\isachardoublequoteopen}pcp{\isacharunderscore}seq\ C\ S\ k\ {\isasymin}\ C{\isachardoublequoteclose}\isanewline
\ \ \ \ \ \ \ \ \isacommand{using}\isamarkupfalse%
\ assms{\isacharparenleft}{\isadigit{1}}{\isacharparenright}\ assms{\isacharparenleft}{\isadigit{2}}{\isacharparenright}\ \isacommand{by}\isamarkupfalse%
\ {\isacharparenleft}rule\ pcp{\isacharunderscore}seq{\isacharunderscore}in{\isacharparenright}\isanewline
\ \ \ \ \ \ \isacommand{have}\isamarkupfalse%
\ {\isachardoublequoteopen}{\isasymforall}S{\isacharprime}\ {\isasymsubseteq}\ {\isacharparenleft}pcp{\isacharunderscore}seq\ C\ S\ k{\isacharparenright}{\isachardot}\ S{\isacharprime}\ {\isasymin}\ C{\isachardoublequoteclose}\isanewline
\ \ \ \ \ \ \ \ \isacommand{using}\isamarkupfalse%
\ SC\ {\isacartoucheopen}pcp{\isacharunderscore}seq\ C\ S\ k\ {\isasymin}\ C{\isacartoucheclose}\ \isacommand{by}\isamarkupfalse%
\ {\isacharparenleft}rule\ bspec{\isacharparenright}\isanewline
\ \ \ \ \ \ \isacommand{thus}\isamarkupfalse%
\ {\isachardoublequoteopen}S{\isacharprime}\ {\isasymin}\ C{\isachardoublequoteclose}\isanewline
\ \ \ \ \ \ \ \ \isacommand{using}\isamarkupfalse%
\ {\isacartoucheopen}S{\isacharprime}\ {\isasymsubseteq}\ pcp{\isacharunderscore}seq\ C\ S\ k{\isacartoucheclose}\ \isacommand{by}\isamarkupfalse%
\ {\isacharparenleft}rule\ sspec{\isacharparenright}\isanewline
\ \ \ \ \isacommand{qed}\isamarkupfalse%
\isanewline
\ \ \isacommand{qed}\isamarkupfalse%
\isanewline
\ \ \isacommand{show}\isamarkupfalse%
\ {\isachardoublequoteopen}pcp{\isacharunderscore}lim\ C\ S\ {\isasymin}\ C{\isachardoublequoteclose}\ \isanewline
\ \ \ \ \isacommand{using}\isamarkupfalse%
\ FC{\isadigit{1}}\ FC{\isadigit{2}}\ \isacommand{by}\isamarkupfalse%
\ {\isacharparenleft}rule\ forw{\isacharunderscore}subst{\isacharparenright}\isanewline
\isacommand{qed}\isamarkupfalse%
%
\endisatagproof
{\isafoldproof}%
%
\isadelimproof
%
\endisadelimproof
%
\begin{isamarkuptext}%
Por otra parte, podemos dar una prueba automática del resultado.%
\end{isamarkuptext}\isamarkuptrue%
\isacommand{lemma}\isamarkupfalse%
\ pcp{\isacharunderscore}lim{\isacharunderscore}in{\isacharcolon}\isanewline
\ \ \isakeyword{assumes}\ c{\isacharcolon}\ {\isachardoublequoteopen}pcp\ C{\isachardoublequoteclose}\isanewline
\ \ \isakeyword{and}\ el{\isacharcolon}\ {\isachardoublequoteopen}S\ {\isasymin}\ C{\isachardoublequoteclose}\isanewline
\ \ \isakeyword{and}\ sc{\isacharcolon}\ {\isachardoublequoteopen}subset{\isacharunderscore}closed\ C{\isachardoublequoteclose}\isanewline
\ \ \isakeyword{and}\ fc{\isacharcolon}\ {\isachardoublequoteopen}finite{\isacharunderscore}character\ C{\isachardoublequoteclose}\isanewline
\ \ \isakeyword{shows}\ {\isachardoublequoteopen}pcp{\isacharunderscore}lim\ C\ S\ {\isasymin}\ C{\isachardoublequoteclose}\ {\isacharparenleft}\isakeyword{is}\ {\isachardoublequoteopen}{\isacharquery}cl\ {\isasymin}\ C{\isachardoublequoteclose}{\isacharparenright}\isanewline
%
\isadelimproof
%
\endisadelimproof
%
\isatagproof
\isacommand{proof}\isamarkupfalse%
\ {\isacharminus}\isanewline
\ \ \isacommand{from}\isamarkupfalse%
\ pcp{\isacharunderscore}seq{\isacharunderscore}in{\isacharbrackleft}OF\ c\ el{\isacharcomma}\ THEN\ allI{\isacharbrackright}\ \isacommand{have}\isamarkupfalse%
\ {\isachardoublequoteopen}{\isasymforall}n{\isachardot}\ pcp{\isacharunderscore}seq\ C\ S\ n\ {\isasymin}\ C{\isachardoublequoteclose}\ \isacommand{{\isachardot}}\isamarkupfalse%
\isanewline
\ \ \isacommand{hence}\isamarkupfalse%
\ {\isachardoublequoteopen}{\isasymforall}m{\isachardot}\ {\isasymUnion}{\isacharbraceleft}pcp{\isacharunderscore}seq\ C\ S\ n{\isacharbar}n{\isachardot}\ n\ {\isasymle}\ m{\isacharbraceright}\ {\isasymin}\ C{\isachardoublequoteclose}\ \isacommand{unfolding}\isamarkupfalse%
\ pcp{\isacharunderscore}seq{\isacharunderscore}UN\ \isacommand{{\isachardot}}\isamarkupfalse%
\isanewline
\ \ \isacommand{have}\isamarkupfalse%
\ {\isachardoublequoteopen}{\isasymforall}S{\isacharprime}\ {\isasymsubseteq}\ {\isacharquery}cl{\isachardot}\ finite\ S{\isacharprime}\ {\isasymlongrightarrow}\ S{\isacharprime}\ {\isasymin}\ C{\isachardoublequoteclose}\isanewline
\ \ \isacommand{proof}\isamarkupfalse%
\ safe\isanewline
\ \ \ \ \isacommand{fix}\isamarkupfalse%
\ S{\isacharprime}\ {\isacharcolon}{\isacharcolon}\ {\isachardoublequoteopen}{\isacharprime}a\ formula\ set{\isachardoublequoteclose}\isanewline
\ \ \ \ \isacommand{have}\isamarkupfalse%
\ {\isachardoublequoteopen}pcp{\isacharunderscore}seq\ C\ S\ {\isacharparenleft}Suc\ {\isacharparenleft}Max\ {\isacharparenleft}to{\isacharunderscore}nat\ {\isacharbackquote}\ S{\isacharprime}{\isacharparenright}{\isacharparenright}{\isacharparenright}\ {\isasymsubseteq}\ pcp{\isacharunderscore}lim\ C\ S{\isachardoublequoteclose}\ \isanewline
\ \ \ \ \ \ \isacommand{using}\isamarkupfalse%
\ pcp{\isacharunderscore}seq{\isacharunderscore}sub\ \isacommand{by}\isamarkupfalse%
\ blast\isanewline
\ \ \ \ \isacommand{assume}\isamarkupfalse%
\ {\isacartoucheopen}finite\ S{\isacharprime}{\isacartoucheclose}\ {\isacartoucheopen}S{\isacharprime}\ {\isasymsubseteq}\ pcp{\isacharunderscore}lim\ C\ S{\isacartoucheclose}\isanewline
\ \ \ \ \isacommand{hence}\isamarkupfalse%
\ {\isachardoublequoteopen}{\isasymexists}k{\isachardot}\ S{\isacharprime}\ {\isasymsubseteq}\ pcp{\isacharunderscore}seq\ C\ S\ k{\isachardoublequoteclose}\ \isanewline
\ \ \ \ \isacommand{proof}\isamarkupfalse%
{\isacharparenleft}induction\ S{\isacharprime}\ rule{\isacharcolon}\ finite{\isacharunderscore}induct{\isacharparenright}\ \isanewline
\ \ \ \ \ \ \isacommand{case}\isamarkupfalse%
\ {\isacharparenleft}insert\ x\ S{\isacharprime}{\isacharparenright}\isanewline
\ \ \ \ \ \ \isacommand{hence}\isamarkupfalse%
\ {\isachardoublequoteopen}{\isasymexists}k{\isachardot}\ S{\isacharprime}\ {\isasymsubseteq}\ pcp{\isacharunderscore}seq\ C\ S\ k{\isachardoublequoteclose}\ \isacommand{by}\isamarkupfalse%
\ fast\isanewline
\ \ \ \ \ \ \isacommand{then}\isamarkupfalse%
\ \isacommand{guess}\isamarkupfalse%
\ k{\isadigit{1}}\ \isacommand{{\isachardot}{\isachardot}}\isamarkupfalse%
\isanewline
\ \ \ \ \ \ \isacommand{moreover}\isamarkupfalse%
\ \isacommand{obtain}\isamarkupfalse%
\ k{\isadigit{2}}\ \isakeyword{where}\ {\isachardoublequoteopen}x\ {\isasymin}\ pcp{\isacharunderscore}seq\ C\ S\ k{\isadigit{2}}{\isachardoublequoteclose}\isanewline
\ \ \ \ \ \ \ \ \isacommand{by}\isamarkupfalse%
\ {\isacharparenleft}meson\ pcp{\isacharunderscore}lim{\isacharunderscore}inserted{\isacharunderscore}at{\isacharunderscore}ex\ insert{\isachardot}prems\ insert{\isacharunderscore}subset{\isacharparenright}\isanewline
\ \ \ \ \ \ \isacommand{ultimately}\isamarkupfalse%
\ \isacommand{have}\isamarkupfalse%
\ {\isachardoublequoteopen}insert\ x\ S{\isacharprime}\ {\isasymsubseteq}\ pcp{\isacharunderscore}seq\ C\ S\ {\isacharparenleft}max\ k{\isadigit{1}}\ k{\isadigit{2}}{\isacharparenright}{\isachardoublequoteclose}\isanewline
\ \ \ \ \ \ \ \ \isacommand{by}\isamarkupfalse%
\ {\isacharparenleft}meson\ pcp{\isacharunderscore}seq{\isacharunderscore}mono\ dual{\isacharunderscore}order{\isachardot}trans\ insert{\isacharunderscore}subset\ max{\isachardot}bounded{\isacharunderscore}iff\ order{\isacharunderscore}refl\ subsetCE{\isacharparenright}\isanewline
\ \ \ \ \ \ \isacommand{thus}\isamarkupfalse%
\ {\isacharquery}case\ \isacommand{by}\isamarkupfalse%
\ blast\isanewline
\ \ \ \ \isacommand{qed}\isamarkupfalse%
\ simp\isanewline
\ \ \ \ \isacommand{with}\isamarkupfalse%
\ pcp{\isacharunderscore}seq{\isacharunderscore}in{\isacharbrackleft}OF\ c\ el{\isacharbrackright}\ sc\isanewline
\ \ \ \ \isacommand{show}\isamarkupfalse%
\ {\isachardoublequoteopen}S{\isacharprime}\ {\isasymin}\ C{\isachardoublequoteclose}\ \isacommand{unfolding}\isamarkupfalse%
\ subset{\isacharunderscore}closed{\isacharunderscore}def\ \isacommand{by}\isamarkupfalse%
\ blast\isanewline
\ \ \isacommand{qed}\isamarkupfalse%
\isanewline
\ \ \isacommand{thus}\isamarkupfalse%
\ {\isachardoublequoteopen}{\isacharquery}cl\ {\isasymin}\ C{\isachardoublequoteclose}\ \isacommand{using}\isamarkupfalse%
\ fc\ \isacommand{unfolding}\isamarkupfalse%
\ finite{\isacharunderscore}character{\isacharunderscore}def\ \isacommand{by}\isamarkupfalse%
\ blast\isanewline
\isacommand{qed}\isamarkupfalse%
%
\endisatagproof
{\isafoldproof}%
%
\isadelimproof
%
\endisadelimproof
%
\begin{isamarkuptext}%
Probemos que, además, el límite de las sucesión definida en \isa{{\isadigit{4}}{\isachardot}{\isadigit{1}}{\isachardot}{\isadigit{1}}} se trata de un elemento 
  maximal de la colección que lo define si esta verifica la propiedad de consistencia proposicional
  y es cerrada bajo subconjuntos.

  \begin{lema}
    Sea \isa{C} una colección de conjuntos que verifica la propiedad de consistencia proposicional y
    es cerrada bajo subconjuntos, \isa{S} un conjunto y \isa{{\isacharbraceleft}S\isactrlsub n{\isacharbraceright}} la sucesión de conjuntos de \isa{C} a partir 
    de \isa{S} según la definición \isa{{\isadigit{4}}{\isachardot}{\isadigit{1}}{\isachardot}{\isadigit{1}}}. Entonces, el límite de la sucesión \isa{{\isacharbraceleft}S\isactrlsub n{\isacharbraceright}} es un elemento 
    maximal de \isa{C}.
  \end{lema}

  \begin{demostracion}
    Por definición de elemento maximal, basta probar que para cualquier conjunto \isa{K\ {\isasymin}\ C} que
    contenga al límite de la sucesión se tiene que \isa{K} y el límite coinciden.

    La demostración se realizará por reducción al absurdo. Consideremos un conjunto \isa{K\ {\isasymin}\ C} que 
    contenga estrictamente al límite de la sucesión \isa{{\isacharbraceleft}S\isactrlsub n{\isacharbraceright}}. De este modo, existe una fórmula \isa{F} tal 
    que \isa{F\ {\isasymin}\ K} y \isa{F} no está en el límite. Supongamos que \isa{F} es la \isa{n}-ésima fórmula según la 
    enumeración de la definición \isa{{\isadigit{4}}{\isachardot}{\isadigit{1}}{\isachardot}{\isadigit{1}}} utilizada para construir la sucesión. 

    Por un lado, hemos probado que todo elemento de la sucesión está contenido en el límite, luego 
    en particular obtenemos que \isa{S\isactrlsub n\isactrlsub {\isacharplus}\isactrlsub {\isadigit{1}}} está contenido en el límite. De este modo, como \isa{F} no 
    pertenece al límite, es claro que \isa{F\ {\isasymnotin}\ S\isactrlsub n\isactrlsub {\isacharplus}\isactrlsub {\isadigit{1}}}. Además, \isa{{\isacharbraceleft}F{\isacharbraceright}\ {\isasymunion}\ S\isactrlsub n\ {\isasymnotin}\ C} ya que, en caso contrario, 
    por la definición \isa{{\isadigit{4}}{\isachardot}{\isadigit{1}}{\isachardot}{\isadigit{1}}} de la sucesión obtendríamos que\\ \isa{S\isactrlsub n\isactrlsub {\isacharplus}\isactrlsub {\isadigit{1}}\ {\isacharequal}\ {\isacharbraceleft}F{\isacharbraceright}\ {\isasymunion}\ S\isactrlsub n}, lo que contradice 
    que \isa{F\ {\isasymnotin}\ S\isactrlsub n\isactrlsub {\isacharplus}\isactrlsub {\isadigit{1}}}. 

    Por otro lado, como \isa{S\isactrlsub n} también está contenida en el límite que, a su vez, está contenido en 
    \isa{K}, se obtiene por transitividad que \isa{S\isactrlsub n\ {\isasymsubseteq}\ K}. Además, como \isa{F\ {\isasymin}\ K}, se verifica que 
    \isa{{\isacharbraceleft}F{\isacharbraceright}\ {\isasymunion}\ S\isactrlsub n\ {\isasymsubseteq}\ K}. Como \isa{C} es una colección cerrada bajo subconjuntos por hipótesis y \isa{K\ {\isasymin}\ C}, 
    por definición se tiene que \isa{{\isacharbraceleft}F{\isacharbraceright}\ {\isasymunion}\ S\isactrlsub n\ {\isasymin}\ C}, llegando así a una contradicción con lo demostrado 
    anteriormente.
  \end{demostracion}

  Su formalización y prueba detallada en Isabelle/HOL se muestran a continuación.%
\end{isamarkuptext}\isamarkuptrue%
\isacommand{lemma}\isamarkupfalse%
\isanewline
\ \ \isakeyword{assumes}\ {\isachardoublequoteopen}pcp\ C{\isachardoublequoteclose}\isanewline
\ \ \ \ \ \ \ \ \ \ {\isachardoublequoteopen}subset{\isacharunderscore}closed\ C{\isachardoublequoteclose}\isanewline
\ \ \ \ \ \ \ \ \ \ {\isachardoublequoteopen}K\ {\isasymin}\ C{\isachardoublequoteclose}\isanewline
\ \ \ \ \ \ \ \ \ \ {\isachardoublequoteopen}pcp{\isacharunderscore}lim\ C\ S\ {\isasymsubseteq}\ K{\isachardoublequoteclose}\isanewline
\ \ \isakeyword{shows}\ {\isachardoublequoteopen}pcp{\isacharunderscore}lim\ C\ S\ {\isacharequal}\ K{\isachardoublequoteclose}\isanewline
%
\isadelimproof
%
\endisadelimproof
%
\isatagproof
\isacommand{proof}\isamarkupfalse%
\ {\isacharparenleft}rule\ ccontr{\isacharparenright}\isanewline
\ \ \isacommand{assume}\isamarkupfalse%
\ H{\isacharcolon}{\isachardoublequoteopen}{\isasymnot}{\isacharparenleft}pcp{\isacharunderscore}lim\ C\ S\ {\isacharequal}\ K{\isacharparenright}{\isachardoublequoteclose}\isanewline
\ \ \isacommand{have}\isamarkupfalse%
\ CE{\isacharcolon}{\isachardoublequoteopen}pcp{\isacharunderscore}lim\ C\ S\ {\isasymsubseteq}\ K\ {\isasymand}\ pcp{\isacharunderscore}lim\ C\ S\ {\isasymnoteq}\ K{\isachardoublequoteclose}\isanewline
\ \ \ \ \isacommand{using}\isamarkupfalse%
\ assms{\isacharparenleft}{\isadigit{4}}{\isacharparenright}\ H\ \isacommand{by}\isamarkupfalse%
\ {\isacharparenleft}rule\ conjI{\isacharparenright}\isanewline
\ \ \isacommand{have}\isamarkupfalse%
\ {\isachardoublequoteopen}pcp{\isacharunderscore}lim\ C\ S\ {\isasymsubseteq}\ K\ {\isasymand}\ pcp{\isacharunderscore}lim\ C\ S\ {\isasymnoteq}\ K\ {\isasymlongleftrightarrow}\ pcp{\isacharunderscore}lim\ C\ S\ {\isasymsubset}\ K{\isachardoublequoteclose}\isanewline
\ \ \ \ \isacommand{by}\isamarkupfalse%
\ {\isacharparenleft}simp\ only{\isacharcolon}\ psubset{\isacharunderscore}eq{\isacharparenright}\isanewline
\ \ \isacommand{then}\isamarkupfalse%
\ \isacommand{have}\isamarkupfalse%
\ {\isachardoublequoteopen}pcp{\isacharunderscore}lim\ C\ S\ {\isasymsubset}\ K{\isachardoublequoteclose}\ \isanewline
\ \ \ \ \isacommand{using}\isamarkupfalse%
\ CE\ \isacommand{by}\isamarkupfalse%
\ {\isacharparenleft}rule\ iffD{\isadigit{1}}{\isacharparenright}\isanewline
\ \ \isacommand{then}\isamarkupfalse%
\ \isacommand{have}\isamarkupfalse%
\ {\isachardoublequoteopen}{\isasymexists}F{\isachardot}\ F\ {\isasymin}\ {\isacharparenleft}K\ {\isacharminus}\ {\isacharparenleft}pcp{\isacharunderscore}lim\ C\ S{\isacharparenright}{\isacharparenright}{\isachardoublequoteclose}\isanewline
\ \ \ \ \isacommand{by}\isamarkupfalse%
\ {\isacharparenleft}simp\ only{\isacharcolon}\ psubset{\isacharunderscore}imp{\isacharunderscore}ex{\isacharunderscore}mem{\isacharparenright}\ \isanewline
\ \ \isacommand{then}\isamarkupfalse%
\ \isacommand{have}\isamarkupfalse%
\ E{\isacharcolon}{\isachardoublequoteopen}{\isasymexists}F{\isachardot}\ F\ {\isasymin}\ K\ {\isasymand}\ F\ {\isasymnotin}\ {\isacharparenleft}pcp{\isacharunderscore}lim\ C\ S{\isacharparenright}{\isachardoublequoteclose}\isanewline
\ \ \ \ \isacommand{by}\isamarkupfalse%
\ {\isacharparenleft}simp\ only{\isacharcolon}\ Diff{\isacharunderscore}iff{\isacharparenright}\isanewline
\ \ \isacommand{obtain}\isamarkupfalse%
\ F\ \isakeyword{where}\ F{\isacharcolon}{\isachardoublequoteopen}F\ {\isasymin}\ K\ {\isasymand}\ F\ {\isasymnotin}\ pcp{\isacharunderscore}lim\ C\ S{\isachardoublequoteclose}\ \isanewline
\ \ \ \ \isacommand{using}\isamarkupfalse%
\ E\ \isacommand{by}\isamarkupfalse%
\ {\isacharparenleft}rule\ exE{\isacharparenright}\isanewline
\ \ \isacommand{have}\isamarkupfalse%
\ {\isachardoublequoteopen}F\ {\isasymin}\ K{\isachardoublequoteclose}\ \isanewline
\ \ \ \ \isacommand{using}\isamarkupfalse%
\ F\ \isacommand{by}\isamarkupfalse%
\ {\isacharparenleft}rule\ conjunct{\isadigit{1}}{\isacharparenright}\isanewline
\ \ \isacommand{have}\isamarkupfalse%
\ {\isachardoublequoteopen}F\ {\isasymnotin}\ pcp{\isacharunderscore}lim\ C\ S{\isachardoublequoteclose}\isanewline
\ \ \ \ \isacommand{using}\isamarkupfalse%
\ F\ \isacommand{by}\isamarkupfalse%
\ {\isacharparenleft}rule\ conjunct{\isadigit{2}}{\isacharparenright}\isanewline
\ \ \isacommand{have}\isamarkupfalse%
\ {\isachardoublequoteopen}pcp{\isacharunderscore}seq\ C\ S\ {\isacharparenleft}Suc\ {\isacharparenleft}to{\isacharunderscore}nat\ F{\isacharparenright}{\isacharparenright}\ {\isasymsubseteq}\ pcp{\isacharunderscore}lim\ C\ S{\isachardoublequoteclose}\isanewline
\ \ \ \ \isacommand{by}\isamarkupfalse%
\ {\isacharparenleft}rule\ pcp{\isacharunderscore}seq{\isacharunderscore}sub{\isacharparenright}\isanewline
\ \ \isacommand{then}\isamarkupfalse%
\ \isacommand{have}\isamarkupfalse%
\ {\isachardoublequoteopen}F\ {\isasymin}\ pcp{\isacharunderscore}seq\ C\ S\ {\isacharparenleft}Suc\ {\isacharparenleft}to{\isacharunderscore}nat\ F{\isacharparenright}{\isacharparenright}\ {\isasymlongrightarrow}\ F\ {\isasymin}\ pcp{\isacharunderscore}lim\ C\ S{\isachardoublequoteclose}\isanewline
\ \ \ \ \isacommand{by}\isamarkupfalse%
\ {\isacharparenleft}rule\ in{\isacharunderscore}mono{\isacharparenright}\isanewline
\ \ \isacommand{then}\isamarkupfalse%
\ \isacommand{have}\isamarkupfalse%
\ {\isadigit{1}}{\isacharcolon}{\isachardoublequoteopen}F\ {\isasymnotin}\ pcp{\isacharunderscore}seq\ C\ S\ {\isacharparenleft}Suc\ {\isacharparenleft}to{\isacharunderscore}nat\ F{\isacharparenright}{\isacharparenright}{\isachardoublequoteclose}\isanewline
\ \ \ \ \isacommand{using}\isamarkupfalse%
\ {\isacartoucheopen}F\ {\isasymnotin}\ pcp{\isacharunderscore}lim\ C\ S{\isacartoucheclose}\ \isacommand{by}\isamarkupfalse%
\ {\isacharparenleft}rule\ mt{\isacharparenright}\isanewline
\ \ \isacommand{have}\isamarkupfalse%
\ {\isadigit{2}}{\isacharcolon}\ {\isachardoublequoteopen}insert\ F\ {\isacharparenleft}pcp{\isacharunderscore}seq\ C\ S\ {\isacharparenleft}to{\isacharunderscore}nat\ F{\isacharparenright}{\isacharparenright}\ {\isasymnotin}\ C{\isachardoublequoteclose}\ \isanewline
\ \ \isacommand{proof}\isamarkupfalse%
\ {\isacharparenleft}rule\ ccontr{\isacharparenright}\isanewline
\ \ \ \ \isacommand{assume}\isamarkupfalse%
\ {\isachardoublequoteopen}{\isasymnot}{\isacharparenleft}insert\ F\ {\isacharparenleft}pcp{\isacharunderscore}seq\ C\ S\ {\isacharparenleft}to{\isacharunderscore}nat\ F{\isacharparenright}{\isacharparenright}\ {\isasymnotin}\ C{\isacharparenright}{\isachardoublequoteclose}\isanewline
\ \ \ \ \isacommand{then}\isamarkupfalse%
\ \isacommand{have}\isamarkupfalse%
\ {\isachardoublequoteopen}insert\ F\ {\isacharparenleft}pcp{\isacharunderscore}seq\ C\ S\ {\isacharparenleft}to{\isacharunderscore}nat\ F{\isacharparenright}{\isacharparenright}\ {\isasymin}\ C{\isachardoublequoteclose}\isanewline
\ \ \ \ \ \ \isacommand{by}\isamarkupfalse%
\ {\isacharparenleft}rule\ notnotD{\isacharparenright}\isanewline
\ \ \ \ \isacommand{then}\isamarkupfalse%
\ \isacommand{have}\isamarkupfalse%
\ C{\isacharcolon}{\isachardoublequoteopen}insert\ {\isacharparenleft}from{\isacharunderscore}nat\ {\isacharparenleft}to{\isacharunderscore}nat\ F{\isacharparenright}{\isacharparenright}\ {\isacharparenleft}pcp{\isacharunderscore}seq\ C\ S\ {\isacharparenleft}to{\isacharunderscore}nat\ F{\isacharparenright}{\isacharparenright}\ {\isasymin}\ C{\isachardoublequoteclose}\isanewline
\ \ \ \ \ \ \isacommand{by}\isamarkupfalse%
\ {\isacharparenleft}simp\ only{\isacharcolon}\ from{\isacharunderscore}nat{\isacharunderscore}to{\isacharunderscore}nat{\isacharparenright}\isanewline
\ \ \ \ \isacommand{have}\isamarkupfalse%
\ {\isachardoublequoteopen}pcp{\isacharunderscore}seq\ C\ S\ {\isacharparenleft}Suc\ {\isacharparenleft}to{\isacharunderscore}nat\ F{\isacharparenright}{\isacharparenright}\ {\isacharequal}\ {\isacharparenleft}let\ Sn\ {\isacharequal}\ pcp{\isacharunderscore}seq\ C\ S\ {\isacharparenleft}to{\isacharunderscore}nat\ F{\isacharparenright}{\isacharsemicolon}\ \isanewline
\ \ \ \ \ \ \ \ \ \ Sn{\isadigit{1}}\ {\isacharequal}\ insert\ {\isacharparenleft}from{\isacharunderscore}nat\ {\isacharparenleft}to{\isacharunderscore}nat\ F{\isacharparenright}{\isacharparenright}\ Sn\ in\ if\ Sn{\isadigit{1}}\ {\isasymin}\ C\ then\ Sn{\isadigit{1}}\ else\ Sn{\isacharparenright}{\isachardoublequoteclose}\ \isanewline
\ \ \ \ \ \ \isacommand{by}\isamarkupfalse%
\ {\isacharparenleft}simp\ only{\isacharcolon}\ pcp{\isacharunderscore}seq{\isachardot}simps{\isacharparenleft}{\isadigit{2}}{\isacharparenright}{\isacharparenright}\isanewline
\ \ \ \ \isacommand{then}\isamarkupfalse%
\ \isacommand{have}\isamarkupfalse%
\ SucDef{\isacharcolon}{\isachardoublequoteopen}pcp{\isacharunderscore}seq\ C\ S\ {\isacharparenleft}Suc\ {\isacharparenleft}to{\isacharunderscore}nat\ F{\isacharparenright}{\isacharparenright}\ {\isacharequal}\ {\isacharparenleft}if\ insert\ {\isacharparenleft}from{\isacharunderscore}nat\ {\isacharparenleft}to{\isacharunderscore}nat\ F{\isacharparenright}{\isacharparenright}\ {\isacharparenleft}pcp{\isacharunderscore}seq\ C\ S\ {\isacharparenleft}to{\isacharunderscore}nat\ F{\isacharparenright}{\isacharparenright}\ {\isasymin}\ C\ \isanewline
\ \ \ \ \ \ \ \ \ \ then\ insert\ {\isacharparenleft}from{\isacharunderscore}nat\ {\isacharparenleft}to{\isacharunderscore}nat\ F{\isacharparenright}{\isacharparenright}\ {\isacharparenleft}pcp{\isacharunderscore}seq\ C\ S\ {\isacharparenleft}to{\isacharunderscore}nat\ F{\isacharparenright}{\isacharparenright}\ else\ pcp{\isacharunderscore}seq\ C\ S\ {\isacharparenleft}to{\isacharunderscore}nat\ F{\isacharparenright}{\isacharparenright}{\isachardoublequoteclose}\ \isanewline
\ \ \ \ \ \ \isacommand{by}\isamarkupfalse%
\ {\isacharparenleft}simp\ only{\isacharcolon}\ Let{\isacharunderscore}def{\isacharparenright}\isanewline
\ \ \ \ \isacommand{then}\isamarkupfalse%
\ \isacommand{have}\isamarkupfalse%
\ {\isachardoublequoteopen}pcp{\isacharunderscore}seq\ C\ S\ {\isacharparenleft}Suc\ {\isacharparenleft}to{\isacharunderscore}nat\ F{\isacharparenright}{\isacharparenright}\ {\isacharequal}\ insert\ {\isacharparenleft}from{\isacharunderscore}nat\ {\isacharparenleft}to{\isacharunderscore}nat\ F{\isacharparenright}{\isacharparenright}\ {\isacharparenleft}pcp{\isacharunderscore}seq\ C\ S\ {\isacharparenleft}to{\isacharunderscore}nat\ F{\isacharparenright}{\isacharparenright}{\isachardoublequoteclose}\ \isanewline
\ \ \ \ \ \ \isacommand{using}\isamarkupfalse%
\ C\ \isacommand{by}\isamarkupfalse%
\ {\isacharparenleft}simp\ only{\isacharcolon}\ if{\isacharunderscore}True{\isacharparenright}\isanewline
\ \ \ \ \isacommand{then}\isamarkupfalse%
\ \isacommand{have}\isamarkupfalse%
\ {\isachardoublequoteopen}pcp{\isacharunderscore}seq\ C\ S\ {\isacharparenleft}Suc\ {\isacharparenleft}to{\isacharunderscore}nat\ F{\isacharparenright}{\isacharparenright}\ {\isacharequal}\ insert\ F\ {\isacharparenleft}pcp{\isacharunderscore}seq\ C\ S\ {\isacharparenleft}to{\isacharunderscore}nat\ F{\isacharparenright}{\isacharparenright}{\isachardoublequoteclose}\isanewline
\ \ \ \ \ \ \isacommand{by}\isamarkupfalse%
\ {\isacharparenleft}simp\ only{\isacharcolon}\ from{\isacharunderscore}nat{\isacharunderscore}to{\isacharunderscore}nat{\isacharparenright}\isanewline
\ \ \ \ \isacommand{then}\isamarkupfalse%
\ \isacommand{have}\isamarkupfalse%
\ {\isachardoublequoteopen}F\ {\isasymin}\ pcp{\isacharunderscore}seq\ C\ S\ {\isacharparenleft}Suc\ {\isacharparenleft}to{\isacharunderscore}nat\ F{\isacharparenright}{\isacharparenright}{\isachardoublequoteclose}\isanewline
\ \ \ \ \ \ \isacommand{by}\isamarkupfalse%
\ {\isacharparenleft}simp\ only{\isacharcolon}\ insertI{\isadigit{1}}{\isacharparenright}\isanewline
\ \ \ \ \isacommand{show}\isamarkupfalse%
\ {\isachardoublequoteopen}False{\isachardoublequoteclose}\isanewline
\ \ \ \ \ \ \isacommand{using}\isamarkupfalse%
\ {\isacartoucheopen}F\ {\isasymnotin}\ pcp{\isacharunderscore}seq\ C\ S\ {\isacharparenleft}Suc\ {\isacharparenleft}to{\isacharunderscore}nat\ F{\isacharparenright}{\isacharparenright}{\isacartoucheclose}\ {\isacartoucheopen}F\ {\isasymin}\ pcp{\isacharunderscore}seq\ C\ S\ {\isacharparenleft}Suc\ {\isacharparenleft}to{\isacharunderscore}nat\ F{\isacharparenright}{\isacharparenright}{\isacartoucheclose}\ \isacommand{by}\isamarkupfalse%
\ {\isacharparenleft}rule\ notE{\isacharparenright}\isanewline
\ \ \isacommand{qed}\isamarkupfalse%
\isanewline
\ \ \isacommand{have}\isamarkupfalse%
\ {\isachardoublequoteopen}pcp{\isacharunderscore}seq\ C\ S\ {\isacharparenleft}to{\isacharunderscore}nat\ F{\isacharparenright}\ {\isasymsubseteq}\ pcp{\isacharunderscore}lim\ C\ S{\isachardoublequoteclose}\isanewline
\ \ \ \ \isacommand{by}\isamarkupfalse%
\ {\isacharparenleft}rule\ pcp{\isacharunderscore}seq{\isacharunderscore}sub{\isacharparenright}\isanewline
\ \ \isacommand{then}\isamarkupfalse%
\ \isacommand{have}\isamarkupfalse%
\ {\isachardoublequoteopen}pcp{\isacharunderscore}seq\ C\ S\ {\isacharparenleft}to{\isacharunderscore}nat\ F{\isacharparenright}\ {\isasymsubseteq}\ K{\isachardoublequoteclose}\isanewline
\ \ \ \ \isacommand{using}\isamarkupfalse%
\ assms{\isacharparenleft}{\isadigit{4}}{\isacharparenright}\ \isacommand{by}\isamarkupfalse%
\ {\isacharparenleft}rule\ subset{\isacharunderscore}trans{\isacharparenright}\isanewline
\ \ \isacommand{then}\isamarkupfalse%
\ \isacommand{have}\isamarkupfalse%
\ {\isachardoublequoteopen}insert\ F\ {\isacharparenleft}pcp{\isacharunderscore}seq\ C\ S\ {\isacharparenleft}to{\isacharunderscore}nat\ F{\isacharparenright}{\isacharparenright}\ {\isasymsubseteq}\ K{\isachardoublequoteclose}\ \isanewline
\ \ \ \ \isacommand{using}\isamarkupfalse%
\ {\isacartoucheopen}F\ {\isasymin}\ K{\isacartoucheclose}\ \isacommand{by}\isamarkupfalse%
\ {\isacharparenleft}simp\ only{\isacharcolon}\ insert{\isacharunderscore}subset{\isacharparenright}\isanewline
\ \ \isacommand{have}\isamarkupfalse%
\ {\isachardoublequoteopen}{\isasymforall}S\ {\isasymin}\ C{\isachardot}\ {\isasymforall}s{\isasymsubseteq}S{\isachardot}\ s\ {\isasymin}\ C{\isachardoublequoteclose}\isanewline
\ \ \ \ \isacommand{using}\isamarkupfalse%
\ assms{\isacharparenleft}{\isadigit{2}}{\isacharparenright}\ \isacommand{by}\isamarkupfalse%
\ {\isacharparenleft}simp\ only{\isacharcolon}\ subset{\isacharunderscore}closed{\isacharunderscore}def{\isacharparenright}\isanewline
\ \ \isacommand{then}\isamarkupfalse%
\ \isacommand{have}\isamarkupfalse%
\ {\isachardoublequoteopen}{\isasymforall}s\ {\isasymsubseteq}\ K{\isachardot}\ s\ {\isasymin}\ C{\isachardoublequoteclose}\isanewline
\ \ \ \ \isacommand{using}\isamarkupfalse%
\ assms{\isacharparenleft}{\isadigit{3}}{\isacharparenright}\ \isacommand{by}\isamarkupfalse%
\ {\isacharparenleft}rule\ bspec{\isacharparenright}\isanewline
\ \ \isacommand{then}\isamarkupfalse%
\ \isacommand{have}\isamarkupfalse%
\ {\isadigit{3}}{\isacharcolon}{\isachardoublequoteopen}insert\ F\ {\isacharparenleft}pcp{\isacharunderscore}seq\ C\ S\ {\isacharparenleft}to{\isacharunderscore}nat\ F{\isacharparenright}{\isacharparenright}\ {\isasymin}\ C{\isachardoublequoteclose}\ \isanewline
\ \ \ \ \isacommand{using}\isamarkupfalse%
\ {\isacartoucheopen}insert\ F\ {\isacharparenleft}pcp{\isacharunderscore}seq\ C\ S\ {\isacharparenleft}to{\isacharunderscore}nat\ F{\isacharparenright}{\isacharparenright}\ {\isasymsubseteq}\ K{\isacartoucheclose}\ \isacommand{by}\isamarkupfalse%
\ {\isacharparenleft}rule\ sspec{\isacharparenright}\isanewline
\ \ \isacommand{show}\isamarkupfalse%
\ {\isachardoublequoteopen}False{\isachardoublequoteclose}\isanewline
\ \ \ \ \isacommand{using}\isamarkupfalse%
\ {\isadigit{2}}\ {\isadigit{3}}\ \isacommand{by}\isamarkupfalse%
\ {\isacharparenleft}rule\ notE{\isacharparenright}\isanewline
\isacommand{qed}\isamarkupfalse%
%
\endisatagproof
{\isafoldproof}%
%
\isadelimproof
%
\endisadelimproof
%
\begin{isamarkuptext}%
Análogamente a resultados anteriores, veamos su prueba automática.%
\end{isamarkuptext}\isamarkuptrue%
\isacommand{lemma}\isamarkupfalse%
\ cl{\isacharunderscore}max{\isacharcolon}\isanewline
\ \ \isakeyword{assumes}\ c{\isacharcolon}\ {\isachardoublequoteopen}pcp\ C{\isachardoublequoteclose}\isanewline
\ \ \isakeyword{assumes}\ sc{\isacharcolon}\ {\isachardoublequoteopen}subset{\isacharunderscore}closed\ C{\isachardoublequoteclose}\isanewline
\ \ \isakeyword{assumes}\ el{\isacharcolon}\ {\isachardoublequoteopen}K\ {\isasymin}\ C{\isachardoublequoteclose}\isanewline
\ \ \isakeyword{assumes}\ su{\isacharcolon}\ {\isachardoublequoteopen}pcp{\isacharunderscore}lim\ C\ S\ {\isasymsubseteq}\ K{\isachardoublequoteclose}\isanewline
\ \ \isakeyword{shows}\ {\isachardoublequoteopen}pcp{\isacharunderscore}lim\ C\ S\ {\isacharequal}\ K{\isachardoublequoteclose}\ {\isacharparenleft}\isakeyword{is}\ {\isacharquery}e{\isacharparenright}\isanewline
%
\isadelimproof
%
\endisadelimproof
%
\isatagproof
\isacommand{proof}\isamarkupfalse%
\ {\isacharparenleft}rule\ ccontr{\isacharparenright}\isanewline
\ \ \isacommand{assume}\isamarkupfalse%
\ {\isacartoucheopen}{\isasymnot}{\isacharquery}e{\isacartoucheclose}\isanewline
\ \ \isacommand{with}\isamarkupfalse%
\ su\ \isacommand{have}\isamarkupfalse%
\ {\isachardoublequoteopen}pcp{\isacharunderscore}lim\ C\ S\ {\isasymsubset}\ K{\isachardoublequoteclose}\ \isacommand{by}\isamarkupfalse%
\ simp\isanewline
\ \ \isacommand{then}\isamarkupfalse%
\ \isacommand{obtain}\isamarkupfalse%
\ F\ \isakeyword{where}\ e{\isacharcolon}\ {\isachardoublequoteopen}F\ {\isasymin}\ K{\isachardoublequoteclose}\ \isakeyword{and}\ ne{\isacharcolon}\ {\isachardoublequoteopen}F\ {\isasymnotin}\ pcp{\isacharunderscore}lim\ C\ S{\isachardoublequoteclose}\ \isacommand{by}\isamarkupfalse%
\ blast\isanewline
\ \ \isacommand{from}\isamarkupfalse%
\ ne\ \isacommand{have}\isamarkupfalse%
\ {\isachardoublequoteopen}F\ {\isasymnotin}\ pcp{\isacharunderscore}seq\ C\ S\ {\isacharparenleft}Suc\ {\isacharparenleft}to{\isacharunderscore}nat\ F{\isacharparenright}{\isacharparenright}{\isachardoublequoteclose}\ \isacommand{using}\isamarkupfalse%
\ pcp{\isacharunderscore}seq{\isacharunderscore}sub\ \isacommand{by}\isamarkupfalse%
\ fast\isanewline
\ \ \isacommand{hence}\isamarkupfalse%
\ {\isadigit{1}}{\isacharcolon}\ {\isachardoublequoteopen}insert\ F\ {\isacharparenleft}pcp{\isacharunderscore}seq\ C\ S\ {\isacharparenleft}to{\isacharunderscore}nat\ F{\isacharparenright}{\isacharparenright}\ {\isasymnotin}\ C{\isachardoublequoteclose}\ \isacommand{by}\isamarkupfalse%
\ {\isacharparenleft}simp\ add{\isacharcolon}\ Let{\isacharunderscore}def\ split{\isacharcolon}\ if{\isacharunderscore}splits{\isacharparenright}\isanewline
\ \ \isacommand{have}\isamarkupfalse%
\ {\isachardoublequoteopen}insert\ F\ {\isacharparenleft}pcp{\isacharunderscore}seq\ C\ S\ {\isacharparenleft}to{\isacharunderscore}nat\ F{\isacharparenright}{\isacharparenright}\ {\isasymsubseteq}\ K{\isachardoublequoteclose}\ \isacommand{using}\isamarkupfalse%
\ pcp{\isacharunderscore}seq{\isacharunderscore}sub\ e\ su\ \isacommand{by}\isamarkupfalse%
\ blast\isanewline
\ \ \isacommand{hence}\isamarkupfalse%
\ {\isachardoublequoteopen}insert\ F\ {\isacharparenleft}pcp{\isacharunderscore}seq\ C\ S\ {\isacharparenleft}to{\isacharunderscore}nat\ F{\isacharparenright}{\isacharparenright}\ {\isasymin}\ C{\isachardoublequoteclose}\ \isacommand{using}\isamarkupfalse%
\ sc\ \isanewline
\ \ \ \ \isacommand{unfolding}\isamarkupfalse%
\ subset{\isacharunderscore}closed{\isacharunderscore}def\ \isacommand{using}\isamarkupfalse%
\ el\ \isacommand{by}\isamarkupfalse%
\ blast\isanewline
\ \ \isacommand{with}\isamarkupfalse%
\ {\isadigit{1}}\ \isacommand{show}\isamarkupfalse%
\ False\ \isacommand{{\isachardot}{\isachardot}}\isamarkupfalse%
\isanewline
\isacommand{qed}\isamarkupfalse%
%
\endisatagproof
{\isafoldproof}%
%
\isadelimproof
%
\endisadelimproof
%
\begin{isamarkuptext}%
A continuación mostremos un resultado sobre el límite de la sucesión de \isa{{\isadigit{4}}{\isachardot}{\isadigit{1}}{\isachardot}{\isadigit{1}}} que es 
  consecuencia de que dicho límite sea un elemento maximal de la colección que lo define si esta
  verifica la propiedad de consistencia proposicional y es cerrada bajo subconjuntos.
  
  \begin{corolario}
    Sea \isa{C} una colección de conjuntos que verifica la propiedad de consistencia proposicional y
    es cerrada bajo subconjuntos, \isa{S} un conjunto, \isa{{\isacharbraceleft}S\isactrlsub n{\isacharbraceright}} la sucesión de conjuntos de \isa{C} a partir 
    de \isa{S} según la definición \isa{{\isadigit{4}}{\isachardot}{\isadigit{1}}{\isachardot}{\isadigit{1}}} y \isa{F} una fórmula proposicional. Entonces, si\\
    $\{F\} \cup \bigcup_{n = 0}^{\infty} S_{n} \in C$, se verifica que 
    $F \in \bigcup_{n = 0}^{\infty} S_{n}$. 
  \end{corolario}

  \begin{demostracion}
    Como \isa{C} es una colección que verifica la propiedad de consistencia proposicional y es cerrada 
    bajo subconjuntos, se tiene que el límite $\bigcup_{n = 0}^{\infty} S_{n}$ es maximal en \isa{C}. Por 
    lo tanto, si suponemos que $\{F\} \cup \bigcup_{n = 0}^{\infty} S_{n} \in C$, como el límite 
    está contenido en dicho conjunto, se cumple que 
    $\{F\} \cup \bigcup_{n = 0}^{\infty} S_{n} = \bigcup_{n = 0}^{\infty} S_{n}$, luego \isa{F} 
    pertenece al límite, como queríamos demostrar.
  \end{demostracion}

  Veamos su formalización y prueba detallada en Isabelle/HOL.%
\end{isamarkuptext}\isamarkuptrue%
\isacommand{lemma}\isamarkupfalse%
\isanewline
\ \ \isakeyword{assumes}\ {\isachardoublequoteopen}pcp\ C{\isachardoublequoteclose}\isanewline
\ \ \isakeyword{assumes}\ {\isachardoublequoteopen}subset{\isacharunderscore}closed\ C{\isachardoublequoteclose}\isanewline
\ \ \isakeyword{shows}\ {\isachardoublequoteopen}insert\ F\ {\isacharparenleft}pcp{\isacharunderscore}lim\ C\ S{\isacharparenright}\ {\isasymin}\ C\ {\isasymLongrightarrow}\ F\ {\isasymin}\ pcp{\isacharunderscore}lim\ C\ S{\isachardoublequoteclose}\isanewline
%
\isadelimproof
%
\endisadelimproof
%
\isatagproof
\isacommand{proof}\isamarkupfalse%
\ {\isacharminus}\isanewline
\ \ \isacommand{assume}\isamarkupfalse%
\ {\isachardoublequoteopen}insert\ F\ {\isacharparenleft}pcp{\isacharunderscore}lim\ C\ S{\isacharparenright}\ {\isasymin}\ C{\isachardoublequoteclose}\isanewline
\ \ \isacommand{have}\isamarkupfalse%
\ {\isachardoublequoteopen}pcp{\isacharunderscore}lim\ C\ S\ {\isasymsubseteq}\ insert\ F\ {\isacharparenleft}pcp{\isacharunderscore}lim\ C\ S{\isacharparenright}{\isachardoublequoteclose}\isanewline
\ \ \ \ \isacommand{by}\isamarkupfalse%
\ {\isacharparenleft}rule\ subset{\isacharunderscore}insertI{\isacharparenright}\ \isanewline
\ \ \isacommand{have}\isamarkupfalse%
\ {\isachardoublequoteopen}pcp{\isacharunderscore}lim\ C\ S\ {\isacharequal}\ insert\ F\ {\isacharparenleft}pcp{\isacharunderscore}lim\ C\ S{\isacharparenright}{\isachardoublequoteclose}\isanewline
\ \ \ \ \isacommand{using}\isamarkupfalse%
\ assms{\isacharparenleft}{\isadigit{1}}{\isacharparenright}\ assms{\isacharparenleft}{\isadigit{2}}{\isacharparenright}\ {\isacartoucheopen}insert\ F\ {\isacharparenleft}pcp{\isacharunderscore}lim\ C\ S{\isacharparenright}\ {\isasymin}\ C{\isacartoucheclose}\ {\isacartoucheopen}pcp{\isacharunderscore}lim\ C\ S\ {\isasymsubseteq}\ insert\ F\ {\isacharparenleft}pcp{\isacharunderscore}lim\ C\ S{\isacharparenright}{\isacartoucheclose}\ \isacommand{by}\isamarkupfalse%
\ {\isacharparenleft}rule\ cl{\isacharunderscore}max{\isacharparenright}\isanewline
\ \ \isacommand{then}\isamarkupfalse%
\ \isacommand{have}\isamarkupfalse%
\ {\isachardoublequoteopen}insert\ F\ {\isacharparenleft}pcp{\isacharunderscore}lim\ C\ S{\isacharparenright}\ {\isasymsubseteq}\ pcp{\isacharunderscore}lim\ C\ S{\isachardoublequoteclose}\isanewline
\ \ \ \ \isacommand{by}\isamarkupfalse%
\ {\isacharparenleft}rule\ equalityD{\isadigit{2}}{\isacharparenright}\isanewline
\ \ \isacommand{then}\isamarkupfalse%
\ \isacommand{have}\isamarkupfalse%
\ {\isachardoublequoteopen}F\ {\isasymin}\ pcp{\isacharunderscore}lim\ C\ S\ {\isasymand}\ pcp{\isacharunderscore}lim\ C\ S\ {\isasymsubseteq}\ pcp{\isacharunderscore}lim\ C\ S{\isachardoublequoteclose}\isanewline
\ \ \ \ \isacommand{by}\isamarkupfalse%
\ {\isacharparenleft}simp\ only{\isacharcolon}\ insert{\isacharunderscore}subset{\isacharparenright}\isanewline
\ \ \isacommand{thus}\isamarkupfalse%
\ {\isachardoublequoteopen}F\ {\isasymin}\ pcp{\isacharunderscore}lim\ C\ S{\isachardoublequoteclose}\isanewline
\ \ \ \ \isacommand{by}\isamarkupfalse%
\ {\isacharparenleft}rule\ conjunct{\isadigit{1}}{\isacharparenright}\isanewline
\isacommand{qed}\isamarkupfalse%
%
\endisatagproof
{\isafoldproof}%
%
\isadelimproof
%
\endisadelimproof
%
\begin{isamarkuptext}%
Mostremos su demostración automática.%
\end{isamarkuptext}\isamarkuptrue%
\isacommand{lemma}\isamarkupfalse%
\ cl{\isacharunderscore}max{\isacharprime}{\isacharcolon}\isanewline
\ \ \isakeyword{assumes}\ c{\isacharcolon}\ {\isachardoublequoteopen}pcp\ C{\isachardoublequoteclose}\isanewline
\ \ \isakeyword{assumes}\ sc{\isacharcolon}\ {\isachardoublequoteopen}subset{\isacharunderscore}closed\ C{\isachardoublequoteclose}\isanewline
\ \ \isakeyword{shows}\ {\isachardoublequoteopen}insert\ F\ {\isacharparenleft}pcp{\isacharunderscore}lim\ C\ S{\isacharparenright}\ {\isasymin}\ C\ {\isasymLongrightarrow}\ F\ {\isasymin}\ pcp{\isacharunderscore}lim\ C\ S{\isachardoublequoteclose}\isanewline
%
\isadelimproof
\ \ %
\endisadelimproof
%
\isatagproof
\isacommand{using}\isamarkupfalse%
\ cl{\isacharunderscore}max{\isacharbrackleft}OF\ assms{\isacharbrackright}\ \isacommand{by}\isamarkupfalse%
\ blast{\isacharplus}%
\endisatagproof
{\isafoldproof}%
%
\isadelimproof
%
\endisadelimproof
%
\begin{isamarkuptext}%
El siguiente resultado prueba que el límite de la sucesión definida en \isa{{\isadigit{4}}{\isachardot}{\isadigit{1}}{\isachardot}{\isadigit{1}}} es un conjunto
  de Hintikka si la colección que lo define verifica la propiedad de consistencia proposicional, es
  es cerrada bajo subconjuntos y es de carácter finito. Como consecuencia del \isa{teorema\ de\ Hintikka},
  se trata en particular de un conjunto satisfacible. 

  \begin{lema}
    Sea \isa{C} una colección de conjuntos que verifica la propiedad de consistencia proposicional, es
    es cerrada bajo subconjuntos y es de carácter finito. Sea \isa{S\ {\isasymin}\ C} y \isa{{\isacharbraceleft}S\isactrlsub n{\isacharbraceright}} la sucesión de
    conjuntos de \isa{C} a partir de \isa{S} según la definición \isa{{\isadigit{4}}{\isachardot}{\isadigit{1}}{\isachardot}{\isadigit{1}}}. Entonces, el límite de la sucesión
    \isa{{\isacharbraceleft}S\isactrlsub n{\isacharbraceright}} es un conjunto de Hintikka.
  \end{lema}

  \begin{demostracion}
    Para facilitar la lectura, vamos a notar por \isa{L\isactrlsub S\isactrlsub C} al límite de la sucesión \isa{{\isacharbraceleft}S\isactrlsub n{\isacharbraceright}} descrita 
    en el enunciado.

    Por resultados anteriores, como \isa{C} verifica la propiedad de consistencia proposicional, es
    es cerrada bajo subconjuntos y es de carácter finito, se tiene que \isa{L\isactrlsub S\isactrlsub C\ {\isasymin}\ C}. En particular, por 
    verificar la propiedad de consistencia proposicional, por el lema de\\ caracterización de dicha
    propiedad mediante notación uniforme, se cumplen las siguientes condiciones para \isa{L\isactrlsub S\isactrlsub C}:

    \begin{itemize}
      \item \isa{{\isasymbottom}\ {\isasymnotin}\ L\isactrlsub S\isactrlsub C}.
      \item Dada \isa{p} una fórmula atómica cualquiera, no se tiene 
      simultáneamente que\\ \isa{p\ {\isasymin}\ L\isactrlsub S\isactrlsub C} y \isa{{\isasymnot}\ p\ {\isasymin}\ L\isactrlsub S\isactrlsub C}.
      \item Para toda fórmula de tipo \isa{{\isasymalpha}} con componentes \isa{{\isasymalpha}\isactrlsub {\isadigit{1}}} y \isa{{\isasymalpha}\isactrlsub {\isadigit{2}}} tal que \isa{{\isasymalpha}}
      pertenece a \isa{L\isactrlsub S\isactrlsub C}, se tiene que \isa{{\isacharbraceleft}{\isasymalpha}\isactrlsub {\isadigit{1}}{\isacharcomma}{\isasymalpha}\isactrlsub {\isadigit{2}}{\isacharbraceright}\ {\isasymunion}\ L\isactrlsub S\isactrlsub C} pertenece a \isa{C}.
      \item Para toda fórmula de tipo \isa{{\isasymbeta}} con componentes \isa{{\isasymbeta}\isactrlsub {\isadigit{1}}} y \isa{{\isasymbeta}\isactrlsub {\isadigit{2}}} tal que \isa{{\isasymbeta}}
      pertenece a \isa{L\isactrlsub S\isactrlsub C}, se tiene que o bien \isa{{\isacharbraceleft}{\isasymbeta}\isactrlsub {\isadigit{1}}{\isacharbraceright}\ {\isasymunion}\ L\isactrlsub S\isactrlsub C} pertenece a \isa{C} o 
      bien \isa{{\isacharbraceleft}{\isasymbeta}\isactrlsub {\isadigit{2}}{\isacharbraceright}\ {\isasymunion}\ L\isactrlsub S\isactrlsub C} pertenece a \isa{C}.
    \end{itemize}

    Veamos que \isa{L\isactrlsub S\isactrlsub C} es un conjunto de Hintikka probando que cumple las condiciones del
    lema de caracterización de los conjuntos de Hintikka mediante notación uniforme, es decir,
    probaremos que \isa{L\isactrlsub S\isactrlsub C} verifica:

    \begin{itemize}
      \item \isa{{\isasymbottom}\ {\isasymnotin}\ L\isactrlsub S\isactrlsub C}.
      \item Dada \isa{p} una fórmula atómica cualquiera, no se tiene 
      simultáneamente que\\ \isa{p\ {\isasymin}\ L\isactrlsub S\isactrlsub C} y \isa{{\isasymnot}\ p\ {\isasymin}\ L\isactrlsub S\isactrlsub C}.
      \item Para toda fórmula de tipo \isa{{\isasymalpha}} con componentes \isa{{\isasymalpha}\isactrlsub {\isadigit{1}}} y \isa{{\isasymalpha}\isactrlsub {\isadigit{2}}} se verifica 
      que si la fórmula pertenece a \isa{L\isactrlsub S\isactrlsub C}, entonces \isa{{\isasymalpha}\isactrlsub {\isadigit{1}}} y \isa{{\isasymalpha}\isactrlsub {\isadigit{2}}} también.
      \item Para toda fórmula de tipo \isa{{\isasymbeta}} con componentes \isa{{\isasymbeta}\isactrlsub {\isadigit{1}}} y \isa{{\isasymbeta}\isactrlsub {\isadigit{2}}} se verifica 
      que si la fórmula pertenece a \isa{L\isactrlsub S\isactrlsub C}, entonces o bien \isa{{\isasymbeta}\isactrlsub {\isadigit{1}}} pertenece
      a \isa{L\isactrlsub S\isactrlsub C} o bien \isa{{\isasymbeta}\isactrlsub {\isadigit{2}}} pertenece a \isa{L\isactrlsub S\isactrlsub C}.
    \end{itemize} 

    Observemos que las dos primeras condiciones coinciden con las obtenidas anteriormente para \isa{L\isactrlsub S\isactrlsub C} 
    por el lema de caracterización de la propiedad de consistencia proposicional mediante notación
    uniforme. Veamos que, en efecto, se cumplen el resto de condiciones.

    En primer lugar, probemos que para una fórmula \isa{F} de tipo \isa{{\isasymalpha}} y componentes \isa{{\isasymalpha}\isactrlsub {\isadigit{1}}} y \isa{{\isasymalpha}\isactrlsub {\isadigit{2}}} tal que 
    \isa{F\ {\isasymin}\ L\isactrlsub S\isactrlsub C} se verifica que tanto \isa{{\isasymalpha}\isactrlsub {\isadigit{1}}} como \isa{{\isasymalpha}\isactrlsub {\isadigit{2}}} pertenecen a \isa{L\isactrlsub S\isactrlsub C}. Por la tercera condición 
    obtenida anteriormente para \isa{L\isactrlsub S\isactrlsub C} por el lema de caracterización de la propiedad de consistencia 
    proposicional mediante notación uniforme, se cumple que\\ \isa{{\isacharbraceleft}{\isasymalpha}\isactrlsub {\isadigit{1}}{\isacharcomma}{\isasymalpha}\isactrlsub {\isadigit{2}}{\isacharbraceright}\ {\isasymunion}\ L\isactrlsub S\isactrlsub C\ {\isasymin}\ C}. Observemos que
    se verifica \isa{L\isactrlsub S\isactrlsub C\ {\isasymsubseteq}\ {\isacharbraceleft}{\isasymalpha}\isactrlsub {\isadigit{1}}{\isacharcomma}{\isasymalpha}\isactrlsub {\isadigit{2}}{\isacharbraceright}\ {\isasymunion}\ L\isactrlsub S\isactrlsub C}. De este modo, como \isa{C} es una colección con la propiedad de 
    consistencia proposicional y cerrada bajo subconjuntos, por el lema \isa{{\isadigit{4}}{\isachardot}{\isadigit{2}}{\isachardot}{\isadigit{2}}} se tiene que 
    los conjuntos \isa{L\isactrlsub S\isactrlsub C} y \isa{{\isacharbraceleft}{\isasymalpha}\isactrlsub {\isadigit{1}}{\isacharcomma}{\isasymalpha}\isactrlsub {\isadigit{2}}{\isacharbraceright}\ {\isasymunion}\ L\isactrlsub S\isactrlsub C} coinciden. Por tanto, es claro que \isa{{\isasymalpha}\isactrlsub {\isadigit{1}}\ {\isasymin}\ L\isactrlsub S\isactrlsub C} y \isa{{\isasymalpha}\isactrlsub {\isadigit{2}}\ {\isasymin}\ L\isactrlsub S\isactrlsub C}, 
    como queríamos demostrar.

    Por último, demostremos que para una fórmula \isa{F} de tipo \isa{{\isasymbeta}} y componentes \isa{{\isasymbeta}\isactrlsub {\isadigit{1}}} y \isa{{\isasymbeta}\isactrlsub {\isadigit{2}}} tal que
    \isa{F\ {\isasymin}\ L\isactrlsub S\isactrlsub C} se verifica que o bien \isa{{\isasymbeta}\isactrlsub {\isadigit{1}}\ {\isasymin}\ L\isactrlsub S\isactrlsub C} o bien \isa{{\isasymbeta}\isactrlsub {\isadigit{2}}\ {\isasymin}\ L\isactrlsub S\isactrlsub C}. Por la cuarta condición obtenida 
    anteriormente para \isa{L\isactrlsub S\isactrlsub C} por el lema de caracterización de la propiedad de consistencia 
    proposicional mediante notación uniforme, se cumple que o bien\\ \isa{{\isacharbraceleft}{\isasymbeta}\isactrlsub {\isadigit{1}}{\isacharbraceright}\ {\isasymunion}\ L\isactrlsub S\isactrlsub C\ {\isasymin}\ C} o bien 
    \isa{{\isacharbraceleft}{\isasymbeta}\isactrlsub {\isadigit{2}}{\isacharbraceright}\ {\isasymunion}\ L\isactrlsub S\isactrlsub C\ {\isasymin}\ C}. De este modo, si suponemos que \isa{{\isacharbraceleft}{\isasymbeta}\isactrlsub {\isadigit{1}}{\isacharbraceright}\ {\isasymunion}\ L\isactrlsub S\isactrlsub C\ {\isasymin}\ C}, como \isa{C} tiene la propiedad de 
    consistencia proposicional y es cerrada bajo subconjuntos, por el corolario \isa{{\isadigit{4}}{\isachardot}{\isadigit{2}}{\isachardot}{\isadigit{3}}} se tiene 
    que \isa{{\isasymbeta}\isactrlsub {\isadigit{1}}\ {\isasymin}\ L\isactrlsub S\isactrlsub C}. Por tanto, se cumple que o bien \isa{{\isasymbeta}\isactrlsub {\isadigit{1}}\ {\isasymin}\ L\isactrlsub S\isactrlsub C} o bien \isa{{\isasymbeta}\isactrlsub {\isadigit{2}}\ {\isasymin}\ L\isactrlsub S\isactrlsub C}. Si suponemos que 
    \isa{{\isacharbraceleft}{\isasymbeta}\isactrlsub {\isadigit{2}}{\isacharbraceright}\ {\isasymunion}\ L\isactrlsub S\isactrlsub C\ {\isasymin}\ C}, se observa fácilmente que llegamos a la misma conclusión de manera análoga. 
    Por lo tanto, queda probado el resultado.
  \end{demostracion}

  Veamos su formalización y prueba detallada en Isabelle.%
\end{isamarkuptext}\isamarkuptrue%
\isacommand{lemma}\isamarkupfalse%
\isanewline
\ \ \isakeyword{assumes}\ {\isachardoublequoteopen}pcp\ C{\isachardoublequoteclose}\isanewline
\ \ \isakeyword{assumes}\ {\isachardoublequoteopen}subset{\isacharunderscore}closed\ C{\isachardoublequoteclose}\isanewline
\ \ \isakeyword{assumes}\ {\isachardoublequoteopen}finite{\isacharunderscore}character\ C{\isachardoublequoteclose}\isanewline
\ \ \isakeyword{assumes}\ {\isachardoublequoteopen}S\ {\isasymin}\ C{\isachardoublequoteclose}\isanewline
\ \ \isakeyword{shows}\ {\isachardoublequoteopen}Hintikka\ {\isacharparenleft}pcp{\isacharunderscore}lim\ C\ S{\isacharparenright}{\isachardoublequoteclose}\isanewline
%
\isadelimproof
%
\endisadelimproof
%
\isatagproof
\isacommand{proof}\isamarkupfalse%
\ {\isacharparenleft}rule\ Hintikka{\isacharunderscore}alt{\isadigit{2}}{\isacharparenright}\isanewline
\ \ \isacommand{let}\isamarkupfalse%
\ {\isacharquery}cl\ {\isacharequal}\ {\isachardoublequoteopen}pcp{\isacharunderscore}lim\ C\ S{\isachardoublequoteclose}\isanewline
\ \ \isacommand{have}\isamarkupfalse%
\ {\isachardoublequoteopen}{\isacharquery}cl\ {\isasymin}\ C{\isachardoublequoteclose}\isanewline
\ \ \ \ \isacommand{using}\isamarkupfalse%
\ assms{\isacharparenleft}{\isadigit{1}}{\isacharparenright}\ assms{\isacharparenleft}{\isadigit{4}}{\isacharparenright}\ assms{\isacharparenleft}{\isadigit{2}}{\isacharparenright}\ assms{\isacharparenleft}{\isadigit{3}}{\isacharparenright}\ \isacommand{by}\isamarkupfalse%
\ {\isacharparenleft}rule\ pcp{\isacharunderscore}lim{\isacharunderscore}in{\isacharparenright}\isanewline
\ \ \isacommand{have}\isamarkupfalse%
\ {\isachardoublequoteopen}{\isacharparenleft}{\isasymforall}S\ {\isasymin}\ C{\isachardot}\isanewline
\ \ {\isasymbottom}\ {\isasymnotin}\ S\isanewline
{\isasymand}\ {\isacharparenleft}{\isasymforall}k{\isachardot}\ Atom\ k\ {\isasymin}\ S\ {\isasymlongrightarrow}\ \isactrlbold {\isasymnot}\ {\isacharparenleft}Atom\ k{\isacharparenright}\ {\isasymin}\ S\ {\isasymlongrightarrow}\ False{\isacharparenright}\isanewline
{\isasymand}\ {\isacharparenleft}{\isasymforall}F\ G\ H{\isachardot}\ Con\ F\ G\ H\ {\isasymlongrightarrow}\ F\ {\isasymin}\ S\ {\isasymlongrightarrow}\ {\isacharbraceleft}G{\isacharcomma}H{\isacharbraceright}\ {\isasymunion}\ S\ {\isasymin}\ C{\isacharparenright}\isanewline
{\isasymand}\ {\isacharparenleft}{\isasymforall}F\ G\ H{\isachardot}\ Dis\ F\ G\ H\ {\isasymlongrightarrow}\ F\ {\isasymin}\ S\ {\isasymlongrightarrow}\ {\isacharbraceleft}G{\isacharbraceright}\ {\isasymunion}\ S\ {\isasymin}\ C\ {\isasymor}\ {\isacharbraceleft}H{\isacharbraceright}\ {\isasymunion}\ S\ {\isasymin}\ C{\isacharparenright}{\isacharparenright}{\isachardoublequoteclose}\isanewline
\ \ \ \ \isacommand{using}\isamarkupfalse%
\ assms{\isacharparenleft}{\isadigit{1}}{\isacharparenright}\ \isacommand{by}\isamarkupfalse%
\ {\isacharparenleft}rule\ pcp{\isacharunderscore}alt{\isadigit{1}}{\isacharparenright}\isanewline
\ \ \isacommand{then}\isamarkupfalse%
\ \isacommand{have}\isamarkupfalse%
\ d{\isacharcolon}{\isachardoublequoteopen}{\isasymbottom}\ {\isasymnotin}\ {\isacharquery}cl\isanewline
{\isasymand}\ {\isacharparenleft}{\isasymforall}k{\isachardot}\ Atom\ k\ {\isasymin}\ {\isacharquery}cl\ {\isasymlongrightarrow}\ \isactrlbold {\isasymnot}\ {\isacharparenleft}Atom\ k{\isacharparenright}\ {\isasymin}\ {\isacharquery}cl\ {\isasymlongrightarrow}\ False{\isacharparenright}\isanewline
{\isasymand}\ {\isacharparenleft}{\isasymforall}F\ G\ H{\isachardot}\ Con\ F\ G\ H\ {\isasymlongrightarrow}\ F\ {\isasymin}\ {\isacharquery}cl\ {\isasymlongrightarrow}\ {\isacharbraceleft}G{\isacharcomma}H{\isacharbraceright}\ {\isasymunion}\ {\isacharquery}cl\ {\isasymin}\ C{\isacharparenright}\isanewline
{\isasymand}\ {\isacharparenleft}{\isasymforall}F\ G\ H{\isachardot}\ Dis\ F\ G\ H\ {\isasymlongrightarrow}\ F\ {\isasymin}\ {\isacharquery}cl\ {\isasymlongrightarrow}\ {\isacharbraceleft}G{\isacharbraceright}\ {\isasymunion}\ {\isacharquery}cl\ {\isasymin}\ C\ {\isasymor}\ {\isacharbraceleft}H{\isacharbraceright}\ {\isasymunion}\ {\isacharquery}cl\ {\isasymin}\ C{\isacharparenright}{\isachardoublequoteclose}\isanewline
\ \ \ \ \isacommand{using}\isamarkupfalse%
\ {\isacartoucheopen}{\isacharquery}cl\ {\isasymin}\ C{\isacartoucheclose}\ \isacommand{by}\isamarkupfalse%
\ {\isacharparenleft}rule\ bspec{\isacharparenright}\isanewline
\ \ \isacommand{then}\isamarkupfalse%
\ \isacommand{have}\isamarkupfalse%
\ H{\isadigit{1}}{\isacharcolon}{\isachardoublequoteopen}{\isasymbottom}\ {\isasymnotin}\ {\isacharquery}cl{\isachardoublequoteclose}\isanewline
\ \ \ \ \isacommand{by}\isamarkupfalse%
\ {\isacharparenleft}rule\ conjunct{\isadigit{1}}{\isacharparenright}\isanewline
\ \ \isacommand{have}\isamarkupfalse%
\ H{\isadigit{2}}{\isacharcolon}{\isachardoublequoteopen}{\isasymforall}k{\isachardot}\ Atom\ k\ {\isasymin}\ {\isacharquery}cl\ {\isasymlongrightarrow}\ \isactrlbold {\isasymnot}\ {\isacharparenleft}Atom\ k{\isacharparenright}\ {\isasymin}\ {\isacharquery}cl\ {\isasymlongrightarrow}\ False{\isachardoublequoteclose}\isanewline
\ \ \ \ \isacommand{using}\isamarkupfalse%
\ d\ \isacommand{by}\isamarkupfalse%
\ {\isacharparenleft}iprover\ elim{\isacharcolon}\ conjunct{\isadigit{2}}\ conjunct{\isadigit{1}}{\isacharparenright}\isanewline
\ \ \isacommand{have}\isamarkupfalse%
\ Con{\isacharcolon}{\isachardoublequoteopen}{\isasymforall}F\ G\ H{\isachardot}\ Con\ F\ G\ H\ {\isasymlongrightarrow}\ F\ {\isasymin}\ {\isacharquery}cl\ {\isasymlongrightarrow}\ {\isacharbraceleft}G{\isacharcomma}H{\isacharbraceright}\ {\isasymunion}\ {\isacharquery}cl\ {\isasymin}\ C{\isachardoublequoteclose}\isanewline
\ \ \ \ \isacommand{using}\isamarkupfalse%
\ d\ \isacommand{by}\isamarkupfalse%
\ {\isacharparenleft}iprover\ elim{\isacharcolon}\ conjunct{\isadigit{2}}\ conjunct{\isadigit{1}}{\isacharparenright}\isanewline
\ \ \isacommand{have}\isamarkupfalse%
\ H{\isadigit{3}}{\isacharcolon}{\isachardoublequoteopen}{\isasymforall}F\ G\ H{\isachardot}\ Con\ F\ G\ H\ {\isasymlongrightarrow}\ F\ {\isasymin}\ {\isacharquery}cl\ {\isasymlongrightarrow}\ G\ {\isasymin}\ {\isacharquery}cl\ {\isasymand}\ H\ {\isasymin}\ {\isacharquery}cl{\isachardoublequoteclose}\isanewline
\ \ \isacommand{proof}\isamarkupfalse%
\ {\isacharparenleft}rule\ allI{\isacharparenright}{\isacharplus}\isanewline
\ \ \ \ \isacommand{fix}\isamarkupfalse%
\ F\ G\ H\isanewline
\ \ \ \ \isacommand{show}\isamarkupfalse%
\ {\isachardoublequoteopen}Con\ F\ G\ H\ {\isasymlongrightarrow}\ F\ {\isasymin}\ {\isacharquery}cl\ {\isasymlongrightarrow}\ G\ {\isasymin}\ {\isacharquery}cl\ {\isasymand}\ H\ {\isasymin}\ {\isacharquery}cl{\isachardoublequoteclose}\isanewline
\ \ \ \ \isacommand{proof}\isamarkupfalse%
\ {\isacharparenleft}rule\ impI{\isacharparenright}{\isacharplus}\isanewline
\ \ \ \ \ \ \isacommand{assume}\isamarkupfalse%
\ {\isachardoublequoteopen}Con\ F\ G\ H{\isachardoublequoteclose}\isanewline
\ \ \ \ \ \ \isacommand{assume}\isamarkupfalse%
\ {\isachardoublequoteopen}F\ {\isasymin}\ {\isacharquery}cl{\isachardoublequoteclose}\isanewline
\ \ \ \ \ \ \isacommand{have}\isamarkupfalse%
\ {\isachardoublequoteopen}Con\ F\ G\ H\ {\isasymlongrightarrow}\ F\ {\isasymin}\ {\isacharquery}cl\ {\isasymlongrightarrow}\ {\isacharbraceleft}G{\isacharcomma}H{\isacharbraceright}\ {\isasymunion}\ {\isacharquery}cl\ {\isasymin}\ C{\isachardoublequoteclose}\isanewline
\ \ \ \ \ \ \ \ \isacommand{using}\isamarkupfalse%
\ Con\ \isacommand{by}\isamarkupfalse%
\ {\isacharparenleft}iprover\ elim{\isacharcolon}\ allE{\isacharparenright}\isanewline
\ \ \ \ \ \ \isacommand{then}\isamarkupfalse%
\ \isacommand{have}\isamarkupfalse%
\ {\isachardoublequoteopen}F\ {\isasymin}\ {\isacharquery}cl\ {\isasymlongrightarrow}\ {\isacharbraceleft}G{\isacharcomma}H{\isacharbraceright}\ {\isasymunion}\ {\isacharquery}cl\ {\isasymin}\ C{\isachardoublequoteclose}\isanewline
\ \ \ \ \ \ \ \ \isacommand{using}\isamarkupfalse%
\ {\isacartoucheopen}Con\ F\ G\ H{\isacartoucheclose}\ \isacommand{by}\isamarkupfalse%
\ {\isacharparenleft}rule\ mp{\isacharparenright}\isanewline
\ \ \ \ \ \ \isacommand{then}\isamarkupfalse%
\ \isacommand{have}\isamarkupfalse%
\ {\isachardoublequoteopen}{\isacharbraceleft}G{\isacharcomma}H{\isacharbraceright}\ {\isasymunion}\ {\isacharquery}cl\ {\isasymin}\ C{\isachardoublequoteclose}\isanewline
\ \ \ \ \ \ \ \ \isacommand{using}\isamarkupfalse%
\ {\isacartoucheopen}F\ {\isasymin}\ {\isacharquery}cl{\isacartoucheclose}\ \isacommand{by}\isamarkupfalse%
\ {\isacharparenleft}rule\ mp{\isacharparenright}\isanewline
\ \ \ \ \ \ \isacommand{have}\isamarkupfalse%
\ {\isachardoublequoteopen}{\isacharparenleft}insert\ G\ {\isacharparenleft}insert\ H\ {\isacharquery}cl{\isacharparenright}{\isacharparenright}\ {\isacharequal}\ {\isacharbraceleft}G{\isacharcomma}H{\isacharbraceright}\ {\isasymunion}\ {\isacharquery}cl{\isachardoublequoteclose}\isanewline
\ \ \ \ \ \ \ \ \isacommand{by}\isamarkupfalse%
\ {\isacharparenleft}rule\ insertSetElem{\isacharparenright}\isanewline
\ \ \ \ \ \ \isacommand{then}\isamarkupfalse%
\ \isacommand{have}\isamarkupfalse%
\ {\isachardoublequoteopen}{\isacharparenleft}insert\ G\ {\isacharparenleft}insert\ H\ {\isacharquery}cl{\isacharparenright}{\isacharparenright}\ {\isasymin}\ C{\isachardoublequoteclose}\isanewline
\ \ \ \ \ \ \ \ \isacommand{using}\isamarkupfalse%
\ {\isacartoucheopen}{\isacharbraceleft}G{\isacharcomma}H{\isacharbraceright}\ {\isasymunion}\ {\isacharquery}cl\ {\isasymin}\ C{\isacartoucheclose}\ \isacommand{by}\isamarkupfalse%
\ {\isacharparenleft}simp\ only{\isacharcolon}\ {\isacartoucheopen}{\isacharparenleft}insert\ G\ {\isacharparenleft}insert\ H\ {\isacharquery}cl{\isacharparenright}{\isacharparenright}\ {\isacharequal}\ {\isacharbraceleft}G{\isacharcomma}H{\isacharbraceright}\ {\isasymunion}\ {\isacharquery}cl{\isacartoucheclose}{\isacharparenright}\isanewline
\ \ \ \ \ \ \isacommand{have}\isamarkupfalse%
\ {\isachardoublequoteopen}{\isacharquery}cl\ {\isasymsubseteq}\ insert\ H\ {\isacharquery}cl{\isachardoublequoteclose}\isanewline
\ \ \ \ \ \ \ \ \isacommand{by}\isamarkupfalse%
\ {\isacharparenleft}rule\ subset{\isacharunderscore}insertI{\isacharparenright}\isanewline
\ \ \ \ \ \ \isacommand{then}\isamarkupfalse%
\ \isacommand{have}\isamarkupfalse%
\ {\isachardoublequoteopen}{\isacharquery}cl\ {\isasymsubseteq}\ insert\ G\ {\isacharparenleft}insert\ H\ {\isacharquery}cl{\isacharparenright}{\isachardoublequoteclose}\isanewline
\ \ \ \ \ \ \ \ \isacommand{by}\isamarkupfalse%
\ {\isacharparenleft}rule\ subset{\isacharunderscore}insertI{\isadigit{2}}{\isacharparenright}\isanewline
\ \ \ \ \ \ \isacommand{have}\isamarkupfalse%
\ {\isachardoublequoteopen}{\isacharquery}cl\ {\isacharequal}\ insert\ G\ {\isacharparenleft}insert\ H\ {\isacharquery}cl{\isacharparenright}{\isachardoublequoteclose}\ \isanewline
\ \ \ \ \ \ \ \ \isacommand{using}\isamarkupfalse%
\ assms{\isacharparenleft}{\isadigit{1}}{\isacharparenright}\ assms{\isacharparenleft}{\isadigit{2}}{\isacharparenright}\ {\isacartoucheopen}insert\ G\ {\isacharparenleft}insert\ H\ {\isacharquery}cl{\isacharparenright}\ {\isasymin}\ C{\isacartoucheclose}\ {\isacartoucheopen}{\isacharquery}cl\ {\isasymsubseteq}\ insert\ G\ {\isacharparenleft}insert\ H\ {\isacharquery}cl{\isacharparenright}{\isacartoucheclose}\ \isacommand{by}\isamarkupfalse%
\ {\isacharparenleft}rule\ cl{\isacharunderscore}max{\isacharparenright}\isanewline
\ \ \ \ \ \ \isacommand{then}\isamarkupfalse%
\ \isacommand{have}\isamarkupfalse%
\ {\isachardoublequoteopen}insert\ G\ {\isacharparenleft}insert\ H\ {\isacharquery}cl{\isacharparenright}\ {\isasymsubseteq}\ {\isacharquery}cl{\isachardoublequoteclose}\isanewline
\ \ \ \ \ \ \ \ \isacommand{by}\isamarkupfalse%
\ {\isacharparenleft}simp\ only{\isacharcolon}\ equalityD{\isadigit{2}}{\isacharparenright}\isanewline
\ \ \ \ \ \ \isacommand{then}\isamarkupfalse%
\ \isacommand{have}\isamarkupfalse%
\ {\isachardoublequoteopen}G\ {\isasymin}\ {\isacharquery}cl\ {\isasymand}\ insert\ H\ {\isacharquery}cl\ {\isasymsubseteq}\ {\isacharquery}cl{\isachardoublequoteclose}\isanewline
\ \ \ \ \ \ \ \ \isacommand{by}\isamarkupfalse%
\ {\isacharparenleft}simp\ only{\isacharcolon}\ insert{\isacharunderscore}subset{\isacharparenright}\isanewline
\ \ \ \ \ \ \isacommand{then}\isamarkupfalse%
\ \isacommand{have}\isamarkupfalse%
\ {\isachardoublequoteopen}G\ {\isasymin}\ {\isacharquery}cl{\isachardoublequoteclose}\isanewline
\ \ \ \ \ \ \ \ \isacommand{by}\isamarkupfalse%
\ {\isacharparenleft}rule\ conjunct{\isadigit{1}}{\isacharparenright}\isanewline
\ \ \ \ \ \ \isacommand{have}\isamarkupfalse%
\ {\isachardoublequoteopen}insert\ H\ {\isacharquery}cl\ {\isasymsubseteq}\ {\isacharquery}cl{\isachardoublequoteclose}\isanewline
\ \ \ \ \ \ \ \ \isacommand{using}\isamarkupfalse%
\ {\isacartoucheopen}G\ {\isasymin}\ {\isacharquery}cl\ {\isasymand}\ insert\ H\ {\isacharquery}cl\ {\isasymsubseteq}\ {\isacharquery}cl{\isacartoucheclose}\ \isacommand{by}\isamarkupfalse%
\ {\isacharparenleft}rule\ conjunct{\isadigit{2}}{\isacharparenright}\isanewline
\ \ \ \ \ \ \isacommand{then}\isamarkupfalse%
\ \isacommand{have}\isamarkupfalse%
\ {\isachardoublequoteopen}H\ {\isasymin}\ {\isacharquery}cl\ {\isasymand}\ {\isacharquery}cl\ {\isasymsubseteq}\ {\isacharquery}cl{\isachardoublequoteclose}\isanewline
\ \ \ \ \ \ \ \ \isacommand{by}\isamarkupfalse%
\ {\isacharparenleft}simp\ only{\isacharcolon}\ insert{\isacharunderscore}subset{\isacharparenright}\isanewline
\ \ \ \ \ \ \isacommand{then}\isamarkupfalse%
\ \isacommand{have}\isamarkupfalse%
\ {\isachardoublequoteopen}H\ {\isasymin}\ {\isacharquery}cl{\isachardoublequoteclose}\isanewline
\ \ \ \ \ \ \ \ \isacommand{by}\isamarkupfalse%
\ {\isacharparenleft}rule\ conjunct{\isadigit{1}}{\isacharparenright}\isanewline
\ \ \ \ \ \ \isacommand{show}\isamarkupfalse%
\ {\isachardoublequoteopen}G\ {\isasymin}\ {\isacharquery}cl\ {\isasymand}\ H\ {\isasymin}\ {\isacharquery}cl{\isachardoublequoteclose}\isanewline
\ \ \ \ \ \ \ \ \isacommand{using}\isamarkupfalse%
\ {\isacartoucheopen}G\ {\isasymin}\ {\isacharquery}cl{\isacartoucheclose}\ {\isacartoucheopen}H\ {\isasymin}\ {\isacharquery}cl{\isacartoucheclose}\ \isacommand{by}\isamarkupfalse%
\ {\isacharparenleft}rule\ conjI{\isacharparenright}\isanewline
\ \ \ \ \isacommand{qed}\isamarkupfalse%
\isanewline
\ \ \isacommand{qed}\isamarkupfalse%
\isanewline
\ \ \isacommand{have}\isamarkupfalse%
\ Dis{\isacharcolon}{\isachardoublequoteopen}{\isasymforall}F\ G\ H{\isachardot}\ Dis\ F\ G\ H\ {\isasymlongrightarrow}\ F\ {\isasymin}\ {\isacharquery}cl\ {\isasymlongrightarrow}\ {\isacharbraceleft}G{\isacharbraceright}\ {\isasymunion}\ {\isacharquery}cl\ {\isasymin}\ C\ {\isasymor}\ {\isacharbraceleft}H{\isacharbraceright}\ {\isasymunion}\ {\isacharquery}cl\ {\isasymin}\ C{\isachardoublequoteclose}\isanewline
\ \ \ \ \isacommand{using}\isamarkupfalse%
\ d\ \isacommand{by}\isamarkupfalse%
\ {\isacharparenleft}iprover\ elim{\isacharcolon}\ conjunct{\isadigit{2}}\ conjunct{\isadigit{1}}{\isacharparenright}\isanewline
\ \ \isacommand{have}\isamarkupfalse%
\ H{\isadigit{4}}{\isacharcolon}{\isachardoublequoteopen}{\isasymforall}F\ G\ H{\isachardot}\ Dis\ F\ G\ H\ {\isasymlongrightarrow}\ F\ {\isasymin}\ {\isacharquery}cl\ {\isasymlongrightarrow}\ G\ {\isasymin}\ {\isacharquery}cl\ {\isasymor}\ H\ {\isasymin}\ {\isacharquery}cl{\isachardoublequoteclose}\isanewline
\ \ \isacommand{proof}\isamarkupfalse%
\ {\isacharparenleft}rule\ allI{\isacharparenright}{\isacharplus}\isanewline
\ \ \ \ \isacommand{fix}\isamarkupfalse%
\ F\ G\ H\isanewline
\ \ \ \ \isacommand{show}\isamarkupfalse%
\ {\isachardoublequoteopen}Dis\ F\ G\ H\ {\isasymlongrightarrow}\ F\ {\isasymin}\ {\isacharquery}cl\ {\isasymlongrightarrow}\ G\ {\isasymin}\ {\isacharquery}cl\ {\isasymor}\ H\ {\isasymin}\ {\isacharquery}cl{\isachardoublequoteclose}\isanewline
\ \ \ \ \isacommand{proof}\isamarkupfalse%
\ {\isacharparenleft}rule\ impI{\isacharparenright}{\isacharplus}\isanewline
\ \ \ \ \ \ \isacommand{assume}\isamarkupfalse%
\ {\isachardoublequoteopen}Dis\ F\ G\ H{\isachardoublequoteclose}\isanewline
\ \ \ \ \ \ \isacommand{assume}\isamarkupfalse%
\ {\isachardoublequoteopen}F\ {\isasymin}\ {\isacharquery}cl{\isachardoublequoteclose}\isanewline
\ \ \ \ \ \ \isacommand{have}\isamarkupfalse%
\ {\isachardoublequoteopen}Dis\ F\ G\ H\ {\isasymlongrightarrow}\ F\ {\isasymin}\ {\isacharquery}cl\ {\isasymlongrightarrow}\ {\isacharbraceleft}G{\isacharbraceright}\ {\isasymunion}\ {\isacharquery}cl\ {\isasymin}\ C\ {\isasymor}\ {\isacharbraceleft}H{\isacharbraceright}\ {\isasymunion}\ {\isacharquery}cl\ {\isasymin}\ C{\isachardoublequoteclose}\isanewline
\ \ \ \ \ \ \ \ \isacommand{using}\isamarkupfalse%
\ Dis\ \isacommand{by}\isamarkupfalse%
\ {\isacharparenleft}iprover\ elim{\isacharcolon}\ allE{\isacharparenright}\isanewline
\ \ \ \ \ \ \isacommand{then}\isamarkupfalse%
\ \isacommand{have}\isamarkupfalse%
\ {\isachardoublequoteopen}F\ {\isasymin}\ {\isacharquery}cl\ {\isasymlongrightarrow}\ {\isacharbraceleft}G{\isacharbraceright}\ {\isasymunion}\ {\isacharquery}cl\ {\isasymin}\ C\ {\isasymor}\ {\isacharbraceleft}H{\isacharbraceright}\ {\isasymunion}\ {\isacharquery}cl\ {\isasymin}\ C{\isachardoublequoteclose}\isanewline
\ \ \ \ \ \ \ \ \isacommand{using}\isamarkupfalse%
\ {\isacartoucheopen}Dis\ F\ G\ H{\isacartoucheclose}\ \isacommand{by}\isamarkupfalse%
\ {\isacharparenleft}rule\ mp{\isacharparenright}\isanewline
\ \ \ \ \ \ \isacommand{then}\isamarkupfalse%
\ \isacommand{have}\isamarkupfalse%
\ {\isachardoublequoteopen}{\isacharbraceleft}G{\isacharbraceright}\ {\isasymunion}\ {\isacharquery}cl\ {\isasymin}\ C\ {\isasymor}\ {\isacharbraceleft}H{\isacharbraceright}\ {\isasymunion}\ {\isacharquery}cl\ {\isasymin}\ C{\isachardoublequoteclose}\isanewline
\ \ \ \ \ \ \ \ \isacommand{using}\isamarkupfalse%
\ {\isacartoucheopen}F\ {\isasymin}\ {\isacharquery}cl{\isacartoucheclose}\ \isacommand{by}\isamarkupfalse%
\ {\isacharparenleft}rule\ mp{\isacharparenright}\isanewline
\ \ \ \ \ \ \isacommand{thus}\isamarkupfalse%
\ {\isachardoublequoteopen}G\ {\isasymin}\ {\isacharquery}cl\ {\isasymor}\ H\ {\isasymin}\ {\isacharquery}cl{\isachardoublequoteclose}\isanewline
\ \ \ \ \ \ \isacommand{proof}\isamarkupfalse%
\ {\isacharparenleft}rule\ disjE{\isacharparenright}\isanewline
\ \ \ \ \ \ \ \ \isacommand{assume}\isamarkupfalse%
\ {\isachardoublequoteopen}{\isacharbraceleft}G{\isacharbraceright}\ {\isasymunion}\ {\isacharquery}cl\ {\isasymin}\ C{\isachardoublequoteclose}\isanewline
\ \ \ \ \ \ \ \ \isacommand{have}\isamarkupfalse%
\ {\isachardoublequoteopen}insert\ G\ {\isacharquery}cl\ {\isacharequal}\ {\isacharbraceleft}G{\isacharbraceright}\ {\isasymunion}\ {\isacharquery}cl{\isachardoublequoteclose}\isanewline
\ \ \ \ \ \ \ \ \ \ \isacommand{by}\isamarkupfalse%
\ {\isacharparenleft}rule\ insert{\isacharunderscore}is{\isacharunderscore}Un{\isacharparenright}\isanewline
\ \ \ \ \ \ \ \ \isacommand{have}\isamarkupfalse%
\ {\isachardoublequoteopen}insert\ G\ {\isacharquery}cl\ {\isasymin}\ C{\isachardoublequoteclose}\isanewline
\ \ \ \ \ \ \ \ \ \ \isacommand{using}\isamarkupfalse%
\ {\isacartoucheopen}{\isacharbraceleft}G{\isacharbraceright}\ {\isasymunion}\ {\isacharquery}cl\ {\isasymin}\ C{\isacartoucheclose}\ \isacommand{by}\isamarkupfalse%
\ {\isacharparenleft}simp\ only{\isacharcolon}\ {\isacartoucheopen}insert\ G\ {\isacharquery}cl\ {\isacharequal}\ {\isacharbraceleft}G{\isacharbraceright}\ {\isasymunion}\ {\isacharquery}cl{\isacartoucheclose}{\isacharparenright}\isanewline
\ \ \ \ \ \ \ \ \isacommand{have}\isamarkupfalse%
\ {\isachardoublequoteopen}insert\ G\ {\isacharquery}cl\ {\isasymin}\ C\ {\isasymLongrightarrow}\ G\ {\isasymin}\ {\isacharquery}cl{\isachardoublequoteclose}\isanewline
\ \ \ \ \ \ \ \ \ \ \isacommand{using}\isamarkupfalse%
\ assms{\isacharparenleft}{\isadigit{1}}{\isacharparenright}\ assms{\isacharparenleft}{\isadigit{2}}{\isacharparenright}\ \isacommand{by}\isamarkupfalse%
\ {\isacharparenleft}rule\ cl{\isacharunderscore}max{\isacharprime}{\isacharparenright}\isanewline
\ \ \ \ \ \ \ \ \isacommand{then}\isamarkupfalse%
\ \isacommand{have}\isamarkupfalse%
\ {\isachardoublequoteopen}G\ {\isasymin}\ {\isacharquery}cl{\isachardoublequoteclose}\isanewline
\ \ \ \ \ \ \ \ \ \ \isacommand{by}\isamarkupfalse%
\ {\isacharparenleft}simp\ only{\isacharcolon}\ {\isacartoucheopen}insert\ G\ {\isacharquery}cl\ {\isasymin}\ C{\isacartoucheclose}{\isacharparenright}\isanewline
\ \ \ \ \ \ \ \ \isacommand{thus}\isamarkupfalse%
\ {\isachardoublequoteopen}G\ {\isasymin}\ {\isacharquery}cl\ {\isasymor}\ H\ {\isasymin}\ {\isacharquery}cl{\isachardoublequoteclose}\isanewline
\ \ \ \ \ \ \ \ \ \ \isacommand{by}\isamarkupfalse%
\ {\isacharparenleft}rule\ disjI{\isadigit{1}}{\isacharparenright}\isanewline
\ \ \ \ \ \ \isacommand{next}\isamarkupfalse%
\isanewline
\ \ \ \ \ \ \ \ \isacommand{assume}\isamarkupfalse%
\ {\isachardoublequoteopen}{\isacharbraceleft}H{\isacharbraceright}\ {\isasymunion}\ {\isacharquery}cl\ {\isasymin}\ C{\isachardoublequoteclose}\isanewline
\ \ \ \ \ \ \ \ \isacommand{have}\isamarkupfalse%
\ {\isachardoublequoteopen}insert\ H\ {\isacharquery}cl\ {\isacharequal}\ {\isacharbraceleft}H{\isacharbraceright}\ {\isasymunion}\ {\isacharquery}cl{\isachardoublequoteclose}\isanewline
\ \ \ \ \ \ \ \ \ \ \isacommand{by}\isamarkupfalse%
\ {\isacharparenleft}rule\ insert{\isacharunderscore}is{\isacharunderscore}Un{\isacharparenright}\isanewline
\ \ \ \ \ \ \ \ \isacommand{have}\isamarkupfalse%
\ {\isachardoublequoteopen}insert\ H\ {\isacharquery}cl\ {\isasymin}\ C{\isachardoublequoteclose}\isanewline
\ \ \ \ \ \ \ \ \ \ \isacommand{using}\isamarkupfalse%
\ {\isacartoucheopen}{\isacharbraceleft}H{\isacharbraceright}\ {\isasymunion}\ {\isacharquery}cl\ {\isasymin}\ C{\isacartoucheclose}\ \isacommand{by}\isamarkupfalse%
\ {\isacharparenleft}simp\ only{\isacharcolon}\ {\isacartoucheopen}insert\ H\ {\isacharquery}cl\ {\isacharequal}\ {\isacharbraceleft}H{\isacharbraceright}\ {\isasymunion}\ {\isacharquery}cl{\isacartoucheclose}{\isacharparenright}\isanewline
\ \ \ \ \ \ \ \ \isacommand{have}\isamarkupfalse%
\ {\isachardoublequoteopen}insert\ H\ {\isacharquery}cl\ {\isasymin}\ C\ {\isasymLongrightarrow}\ H\ {\isasymin}\ {\isacharquery}cl{\isachardoublequoteclose}\isanewline
\ \ \ \ \ \ \ \ \ \ \isacommand{using}\isamarkupfalse%
\ assms{\isacharparenleft}{\isadigit{1}}{\isacharparenright}\ assms{\isacharparenleft}{\isadigit{2}}{\isacharparenright}\ \isacommand{by}\isamarkupfalse%
\ {\isacharparenleft}rule\ cl{\isacharunderscore}max{\isacharprime}{\isacharparenright}\isanewline
\ \ \ \ \ \ \ \ \isacommand{then}\isamarkupfalse%
\ \isacommand{have}\isamarkupfalse%
\ {\isachardoublequoteopen}H\ {\isasymin}\ {\isacharquery}cl{\isachardoublequoteclose}\isanewline
\ \ \ \ \ \ \ \ \ \ \isacommand{by}\isamarkupfalse%
\ {\isacharparenleft}simp\ only{\isacharcolon}\ {\isacartoucheopen}insert\ H\ {\isacharquery}cl\ {\isasymin}\ C{\isacartoucheclose}{\isacharparenright}\isanewline
\ \ \ \ \ \ \ \ \isacommand{thus}\isamarkupfalse%
\ {\isachardoublequoteopen}G\ {\isasymin}\ {\isacharquery}cl\ {\isasymor}\ H\ {\isasymin}\ {\isacharquery}cl{\isachardoublequoteclose}\isanewline
\ \ \ \ \ \ \ \ \ \ \isacommand{by}\isamarkupfalse%
\ {\isacharparenleft}rule\ disjI{\isadigit{2}}{\isacharparenright}\isanewline
\ \ \ \ \ \ \isacommand{qed}\isamarkupfalse%
\isanewline
\ \ \ \ \isacommand{qed}\isamarkupfalse%
\isanewline
\ \ \isacommand{qed}\isamarkupfalse%
\isanewline
\ \ \isacommand{show}\isamarkupfalse%
\ {\isachardoublequoteopen}{\isasymbottom}\ {\isasymnotin}\ {\isacharquery}cl\ {\isasymand}\isanewline
\ \ \ \ {\isacharparenleft}{\isasymforall}k{\isachardot}\ Atom\ k\ {\isasymin}\ {\isacharquery}cl\ {\isasymlongrightarrow}\ \isactrlbold {\isasymnot}\ {\isacharparenleft}Atom\ k{\isacharparenright}\ {\isasymin}\ {\isacharquery}cl\ {\isasymlongrightarrow}\ False{\isacharparenright}\ {\isasymand}\isanewline
\ \ \ \ {\isacharparenleft}{\isasymforall}F\ G\ H{\isachardot}\ Con\ F\ G\ H\ {\isasymlongrightarrow}\ F\ {\isasymin}\ {\isacharquery}cl\ {\isasymlongrightarrow}\ G\ {\isasymin}\ {\isacharquery}cl\ {\isasymand}\ H\ {\isasymin}\ {\isacharquery}cl{\isacharparenright}\ {\isasymand}\isanewline
\ \ \ \ {\isacharparenleft}{\isasymforall}F\ G\ H{\isachardot}\ Dis\ F\ G\ H\ {\isasymlongrightarrow}\ F\ {\isasymin}\ {\isacharquery}cl\ {\isasymlongrightarrow}\ G\ {\isasymin}\ {\isacharquery}cl\ {\isasymor}\ H\ {\isasymin}\ {\isacharquery}cl{\isacharparenright}{\isachardoublequoteclose}\isanewline
\ \ \ \ \isacommand{using}\isamarkupfalse%
\ H{\isadigit{1}}\ H{\isadigit{2}}\ H{\isadigit{3}}\ H{\isadigit{4}}\ \isacommand{by}\isamarkupfalse%
\ {\isacharparenleft}iprover\ intro{\isacharcolon}\ conjI{\isacharparenright}\isanewline
\isacommand{qed}\isamarkupfalse%
%
\endisatagproof
{\isafoldproof}%
%
\isadelimproof
%
\endisadelimproof
%
\begin{isamarkuptext}%
Del mismo modo, podemos probar el resultado de manera automática como sigue.%
\end{isamarkuptext}\isamarkuptrue%
\isacommand{lemma}\isamarkupfalse%
\ pcp{\isacharunderscore}lim{\isacharunderscore}Hintikka{\isacharcolon}\isanewline
\ \ \isakeyword{assumes}\ c{\isacharcolon}\ {\isachardoublequoteopen}pcp\ C{\isachardoublequoteclose}\isanewline
\ \ \isakeyword{assumes}\ sc{\isacharcolon}\ {\isachardoublequoteopen}subset{\isacharunderscore}closed\ C{\isachardoublequoteclose}\isanewline
\ \ \isakeyword{assumes}\ fc{\isacharcolon}\ {\isachardoublequoteopen}finite{\isacharunderscore}character\ C{\isachardoublequoteclose}\isanewline
\ \ \isakeyword{assumes}\ el{\isacharcolon}\ {\isachardoublequoteopen}S\ {\isasymin}\ C{\isachardoublequoteclose}\isanewline
\ \ \isakeyword{shows}\ {\isachardoublequoteopen}Hintikka\ {\isacharparenleft}pcp{\isacharunderscore}lim\ C\ S{\isacharparenright}{\isachardoublequoteclose}\isanewline
%
\isadelimproof
%
\endisadelimproof
%
\isatagproof
\isacommand{proof}\isamarkupfalse%
\ {\isacharminus}\isanewline
\ \ \isacommand{let}\isamarkupfalse%
\ {\isacharquery}cl\ {\isacharequal}\ {\isachardoublequoteopen}pcp{\isacharunderscore}lim\ C\ S{\isachardoublequoteclose}\isanewline
\ \ \isacommand{have}\isamarkupfalse%
\ {\isachardoublequoteopen}{\isacharquery}cl\ {\isasymin}\ C{\isachardoublequoteclose}\ \isacommand{using}\isamarkupfalse%
\ pcp{\isacharunderscore}lim{\isacharunderscore}in{\isacharbrackleft}OF\ c\ el\ sc\ fc{\isacharbrackright}\ \isacommand{{\isachardot}}\isamarkupfalse%
\isanewline
\ \ \isacommand{from}\isamarkupfalse%
\ c{\isacharbrackleft}unfolded\ pcp{\isacharunderscore}alt{\isacharcomma}\ THEN\ bspec{\isacharcomma}\ OF\ this{\isacharbrackright}\isanewline
\ \ \isacommand{have}\isamarkupfalse%
\ d{\isacharcolon}\ {\isachardoublequoteopen}{\isasymbottom}\ {\isasymnotin}\ {\isacharquery}cl{\isachardoublequoteclose}\isanewline
\ \ \ \ {\isachardoublequoteopen}Atom\ k\ {\isasymin}\ {\isacharquery}cl\ {\isasymLongrightarrow}\ \isactrlbold {\isasymnot}\ {\isacharparenleft}Atom\ k{\isacharparenright}\ {\isasymin}\ {\isacharquery}cl\ {\isasymLongrightarrow}\ False{\isachardoublequoteclose}\isanewline
\ \ \ \ {\isachardoublequoteopen}Con\ F\ G\ H\ {\isasymLongrightarrow}\ F\ {\isasymin}\ {\isacharquery}cl\ {\isasymLongrightarrow}\ insert\ G\ {\isacharparenleft}insert\ H\ {\isacharquery}cl{\isacharparenright}\ {\isasymin}\ C{\isachardoublequoteclose}\isanewline
\ \ \ \ {\isachardoublequoteopen}Dis\ F\ G\ H\ {\isasymLongrightarrow}\ F\ {\isasymin}\ {\isacharquery}cl\ {\isasymLongrightarrow}\ insert\ G\ {\isacharquery}cl\ {\isasymin}\ C\ {\isasymor}\ insert\ H\ {\isacharquery}cl\ {\isasymin}\ C{\isachardoublequoteclose}\isanewline
\ \ \ \ \isakeyword{for}\ k\ F\ G\ H\ \isacommand{by}\isamarkupfalse%
\ force{\isacharplus}\isanewline
\ \ \isacommand{with}\isamarkupfalse%
\ d{\isacharparenleft}{\isadigit{1}}{\isacharcomma}{\isadigit{2}}{\isacharparenright}\ \isacommand{show}\isamarkupfalse%
\ {\isachardoublequoteopen}Hintikka\ {\isacharquery}cl{\isachardoublequoteclose}\ \isacommand{unfolding}\isamarkupfalse%
\ Hintikka{\isacharunderscore}alt\ \isanewline
\ \ \ \ \isacommand{using}\isamarkupfalse%
\ c\ cl{\isacharunderscore}max\ cl{\isacharunderscore}max{\isacharprime}\ d{\isacharparenleft}{\isadigit{4}}{\isacharparenright}\ sc\ \isacommand{by}\isamarkupfalse%
\ blast\isanewline
\isacommand{qed}\isamarkupfalse%
%
\endisatagproof
{\isafoldproof}%
%
\isadelimproof
%
\endisadelimproof
%
\begin{isamarkuptext}%
Finalmente, vamos a demostrar el \isa{teorema\ de\ existencia\ de\ modelo}. Para ello precisaremos de
  un resultado que indica que la satisfacibilidad de conjuntos de fórmulas se hereda por la 
  contención.

  \begin{lema}
    Todo subconjunto de un conjunto de fórmulas satisfacible es satisfacible.
  \end{lema}

  \begin{demostracion}
    Sea \isa{B} un conjunto de fórmulas satisfacible y \isa{A\ {\isasymsubseteq}\ B}. Veamos que \isa{A} es satisfacible.
    Por definición, como \isa{B} es satisfacible, existe una interpretación \isa{{\isasymA}} que es modelo de cada 
    fórmula de \isa{B}. Como \isa{A\ {\isasymsubseteq}\ B}, en particular \isa{{\isasymA}} es modelo de toda fórmula de \isa{A}. Por tanto, 
    \isa{A} es satisfacible, ya que existe una interpretación que es modelo de todas sus fórmulas.
  \end{demostracion}

  Su prueba detallada en Isabelle/HOL es la siguiente.%
\end{isamarkuptext}\isamarkuptrue%
\isacommand{lemma}\isamarkupfalse%
\ sat{\isacharunderscore}mono{\isacharcolon}\isanewline
\ \ \isakeyword{assumes}\ {\isachardoublequoteopen}A\ {\isasymsubseteq}\ B{\isachardoublequoteclose}\isanewline
\ \ \ \ \ \ \ \ \ \ {\isachardoublequoteopen}sat\ B{\isachardoublequoteclose}\isanewline
\ \ \ \ \ \ \ \ \isakeyword{shows}\ {\isachardoublequoteopen}sat\ A{\isachardoublequoteclose}\isanewline
%
\isadelimproof
\ \ %
\endisadelimproof
%
\isatagproof
\isacommand{unfolding}\isamarkupfalse%
\ sat{\isacharunderscore}def\isanewline
\isacommand{proof}\isamarkupfalse%
\ {\isacharminus}\isanewline
\ \isacommand{have}\isamarkupfalse%
\ satB{\isacharcolon}{\isachardoublequoteopen}{\isasymexists}{\isasymA}{\isachardot}\ {\isasymforall}F\ {\isasymin}\ B{\isachardot}\ {\isasymA}\ {\isasymTurnstile}\ F{\isachardoublequoteclose}\isanewline
\ \ \ \isacommand{using}\isamarkupfalse%
\ assms{\isacharparenleft}{\isadigit{2}}{\isacharparenright}\ \isacommand{by}\isamarkupfalse%
\ {\isacharparenleft}simp\ only{\isacharcolon}\ sat{\isacharunderscore}def{\isacharparenright}\isanewline
\ \isacommand{obtain}\isamarkupfalse%
\ {\isasymA}\ \isakeyword{where}\ {\isachardoublequoteopen}{\isasymforall}F\ {\isasymin}\ B{\isachardot}\ {\isasymA}\ {\isasymTurnstile}\ F{\isachardoublequoteclose}\isanewline
\ \ \ \ \isacommand{using}\isamarkupfalse%
\ satB\ \isacommand{by}\isamarkupfalse%
\ {\isacharparenleft}rule\ exE{\isacharparenright}\isanewline
\ \isacommand{have}\isamarkupfalse%
\ {\isachardoublequoteopen}{\isasymforall}F\ {\isasymin}\ A{\isachardot}\ {\isasymA}\ {\isasymTurnstile}\ F{\isachardoublequoteclose}\isanewline
\ \ \isacommand{proof}\isamarkupfalse%
\ {\isacharparenleft}rule\ ballI{\isacharparenright}\isanewline
\ \ \ \ \isacommand{fix}\isamarkupfalse%
\ F\isanewline
\ \ \ \ \isacommand{assume}\isamarkupfalse%
\ {\isachardoublequoteopen}F\ {\isasymin}\ A{\isachardoublequoteclose}\isanewline
\ \ \ \ \isacommand{have}\isamarkupfalse%
\ {\isachardoublequoteopen}F\ {\isasymin}\ A\ {\isasymlongrightarrow}\ F\ {\isasymin}\ B{\isachardoublequoteclose}\isanewline
\ \ \ \ \ \ \isacommand{using}\isamarkupfalse%
\ assms{\isacharparenleft}{\isadigit{1}}{\isacharparenright}\ \isacommand{by}\isamarkupfalse%
\ {\isacharparenleft}rule\ in{\isacharunderscore}mono{\isacharparenright}\isanewline
\ \ \ \ \isacommand{then}\isamarkupfalse%
\ \isacommand{have}\isamarkupfalse%
\ {\isachardoublequoteopen}F\ {\isasymin}\ B{\isachardoublequoteclose}\isanewline
\ \ \ \ \ \ \isacommand{using}\isamarkupfalse%
\ {\isacartoucheopen}F\ {\isasymin}\ A{\isacartoucheclose}\ \isacommand{by}\isamarkupfalse%
\ {\isacharparenleft}rule\ mp{\isacharparenright}\isanewline
\ \ \ \ \isacommand{show}\isamarkupfalse%
\ {\isachardoublequoteopen}{\isasymA}\ {\isasymTurnstile}\ F{\isachardoublequoteclose}\isanewline
\ \ \ \ \ \ \isacommand{using}\isamarkupfalse%
\ {\isacartoucheopen}{\isasymforall}F\ {\isasymin}\ B{\isachardot}\ {\isasymA}\ {\isasymTurnstile}\ F{\isacartoucheclose}\ {\isacartoucheopen}F\ {\isasymin}\ B{\isacartoucheclose}\ \isacommand{by}\isamarkupfalse%
\ {\isacharparenleft}rule\ bspec{\isacharparenright}\isanewline
\ \ \isacommand{qed}\isamarkupfalse%
\isanewline
\ \ \isacommand{thus}\isamarkupfalse%
\ {\isachardoublequoteopen}{\isasymexists}{\isasymA}{\isachardot}\ {\isasymforall}F\ {\isasymin}\ A{\isachardot}\ {\isasymA}\ {\isasymTurnstile}\ F{\isachardoublequoteclose}\isanewline
\ \ \ \ \isacommand{by}\isamarkupfalse%
\ {\isacharparenleft}simp\ only{\isacharcolon}\ exI{\isacharparenright}\isanewline
\isacommand{qed}\isamarkupfalse%
%
\endisatagproof
{\isafoldproof}%
%
\isadelimproof
%
\endisadelimproof
%
\begin{isamarkuptext}%
De este modo, procedamos finalmente con la demostración del teorema.

  \begin{teorema}[Teorema de Existencia de Modelo]
    Todo conjunto de fórmulas perteneciente a una colección que verifique la propiedad de
    consistencia proposicional es satisfacible. 
  \end{teorema}

  \begin{demostracion}
    Sea \isa{C} una colección de conjuntos de fórmulas proposicionales que verifique la propiedad de 
    consistencia proposicional y \isa{S\ {\isasymin}\ C}. Vamos a probar que \isa{S} es satisfacible.

    En primer lugar, como \isa{C} verifica la propiedad de consistencia proposicional, por el lema 
    \isa{{\isadigit{3}}{\isachardot}{\isadigit{0}}{\isachardot}{\isadigit{3}}} podemos extenderla a una colección \isa{C{\isacharprime}} que también verifique la propiedad y
    sea cerrada bajo subconjuntos. A su vez, por el lema \isa{{\isadigit{3}}{\isachardot}{\isadigit{0}}{\isachardot}{\isadigit{5}}}, como la extensión 
    \isa{C{\isacharprime}} es una colección con la propiedad de consistencia proposicional y cerrada bajo 
    subconjuntos, podemos extenderla a otra colección \isa{C{\isacharprime}{\isacharprime}} que también verifica la propiedad de
    consistencia proposicional y sea de carácter finito. De este modo, por la transitividad de la 
    contención, es claro que \isa{C{\isacharprime}{\isacharprime}} es una extensión de \isa{C}, luego \isa{S\ {\isasymin}\ C{\isacharprime}{\isacharprime}} por hipótesis. 
    Por otro lado, por el lema \isa{{\isadigit{3}}{\isachardot}{\isadigit{0}}{\isachardot}{\isadigit{4}}}, como \isa{C{\isacharprime}{\isacharprime}} es de carácter finito, se tiene que es 
    cerrada bajo subconjuntos. 

    En suma, \isa{C{\isacharprime}{\isacharprime}} es una extensión de \isa{C} que verifica la propiedad de consistencia proposicional, 
    es cerrada bajo subconjuntos y es de carácter finito. Luego, por el lema \isa{{\isadigit{4}}{\isachardot}{\isadigit{2}}{\isachardot}{\isadigit{4}}}, el límite de 
    la sucesión \isa{{\isacharbraceleft}S\isactrlsub n{\isacharbraceright}} de conjuntos de \isa{C{\isacharprime}{\isacharprime}} a partir de \isa{S} según la definición \isa{{\isadigit{4}}{\isachardot}{\isadigit{1}}{\isachardot}{\isadigit{1}}} es un 
    conjunto de Hintikka. Por tanto, por el \isa{teorema\ de\ Hintikka}, se trata de un conjunto 
    satisfacible. 

    Finalmente, puesto que para todo \isa{n\ {\isasymin}\ {\isasymnat}} se tiene que \isa{S\isactrlsub n} está contenido en el límite, en 
    particular el conjunto \isa{S\isactrlsub {\isadigit{0}}} está contenido en él. Por definición de la sucesión, dicho conjunto 
    coincide con \isa{S}. Por tanto, como \isa{S} está contenido en el límite que es un conjunto 
    satisfacible, queda demostrada la satisfacibilidad de \isa{S}.
  \end{demostracion}

  Mostremos su formalización y demostración detallada en Isabelle.%
\end{isamarkuptext}\isamarkuptrue%
\isacommand{theorem}\isamarkupfalse%
\isanewline
\ \ \isakeyword{fixes}\ S\ {\isacharcolon}{\isacharcolon}\ {\isachardoublequoteopen}{\isacharprime}a\ {\isacharcolon}{\isacharcolon}\ countable\ formula\ set{\isachardoublequoteclose}\isanewline
\ \ \isakeyword{assumes}\ {\isachardoublequoteopen}pcp\ C{\isachardoublequoteclose}\isanewline
\ \ \isakeyword{assumes}\ {\isachardoublequoteopen}S\ {\isasymin}\ C{\isachardoublequoteclose}\isanewline
\ \ \isakeyword{shows}\ {\isachardoublequoteopen}sat\ S{\isachardoublequoteclose}\isanewline
%
\isadelimproof
%
\endisadelimproof
%
\isatagproof
\isacommand{proof}\isamarkupfalse%
\ {\isacharminus}\isanewline
\ \ \isacommand{have}\isamarkupfalse%
\ {\isachardoublequoteopen}pcp\ C\ {\isasymLongrightarrow}\ {\isasymexists}C{\isacharprime}{\isachardot}\ C\ {\isasymsubseteq}\ C{\isacharprime}\ {\isasymand}\ pcp\ C{\isacharprime}\ {\isasymand}\ subset{\isacharunderscore}closed\ C{\isacharprime}{\isachardoublequoteclose}\isanewline
\ \ \ \ \isacommand{by}\isamarkupfalse%
\ {\isacharparenleft}rule\ ex{\isadigit{1}}{\isacharparenright}\isanewline
\ \ \isacommand{then}\isamarkupfalse%
\ \isacommand{have}\isamarkupfalse%
\ E{\isadigit{1}}{\isacharcolon}{\isachardoublequoteopen}{\isasymexists}C{\isacharprime}{\isachardot}\ C\ {\isasymsubseteq}\ C{\isacharprime}\ {\isasymand}\ pcp\ C{\isacharprime}\ {\isasymand}\ subset{\isacharunderscore}closed\ C{\isacharprime}{\isachardoublequoteclose}\isanewline
\ \ \ \ \isacommand{by}\isamarkupfalse%
\ {\isacharparenleft}simp\ only{\isacharcolon}\ assms{\isacharparenleft}{\isadigit{1}}{\isacharparenright}{\isacharparenright}\isanewline
\ \ \isacommand{obtain}\isamarkupfalse%
\ Ce{\isacharprime}\ \isakeyword{where}\ H{\isadigit{1}}{\isacharcolon}{\isachardoublequoteopen}C\ {\isasymsubseteq}\ Ce{\isacharprime}\ {\isasymand}\ pcp\ Ce{\isacharprime}\ {\isasymand}\ subset{\isacharunderscore}closed\ Ce{\isacharprime}{\isachardoublequoteclose}\isanewline
\ \ \ \ \isacommand{using}\isamarkupfalse%
\ E{\isadigit{1}}\ \isacommand{by}\isamarkupfalse%
\ {\isacharparenleft}rule\ exE{\isacharparenright}\isanewline
\ \ \isacommand{have}\isamarkupfalse%
\ {\isachardoublequoteopen}C\ {\isasymsubseteq}\ Ce{\isacharprime}{\isachardoublequoteclose}\isanewline
\ \ \ \ \isacommand{using}\isamarkupfalse%
\ H{\isadigit{1}}\ \isacommand{by}\isamarkupfalse%
\ {\isacharparenleft}rule\ conjunct{\isadigit{1}}{\isacharparenright}\isanewline
\ \ \isacommand{have}\isamarkupfalse%
\ {\isachardoublequoteopen}pcp\ Ce{\isacharprime}{\isachardoublequoteclose}\isanewline
\ \ \ \ \isacommand{using}\isamarkupfalse%
\ H{\isadigit{1}}\ \isacommand{by}\isamarkupfalse%
\ {\isacharparenleft}iprover\ elim{\isacharcolon}\ conjunct{\isadigit{2}}\ conjunct{\isadigit{1}}{\isacharparenright}\isanewline
\ \ \isacommand{have}\isamarkupfalse%
\ {\isachardoublequoteopen}subset{\isacharunderscore}closed\ Ce{\isacharprime}{\isachardoublequoteclose}\isanewline
\ \ \ \ \isacommand{using}\isamarkupfalse%
\ H{\isadigit{1}}\ \isacommand{by}\isamarkupfalse%
\ {\isacharparenleft}iprover\ elim{\isacharcolon}\ conjunct{\isadigit{2}}\ conjunct{\isadigit{1}}{\isacharparenright}\isanewline
\ \ \isacommand{have}\isamarkupfalse%
\ E{\isadigit{2}}{\isacharcolon}{\isachardoublequoteopen}{\isasymexists}Ce{\isachardot}\ Ce{\isacharprime}\ {\isasymsubseteq}\ Ce\ {\isasymand}\ pcp\ Ce\ {\isasymand}\ finite{\isacharunderscore}character\ Ce{\isachardoublequoteclose}\isanewline
\ \ \ \ \isacommand{using}\isamarkupfalse%
\ {\isacartoucheopen}pcp\ Ce{\isacharprime}{\isacartoucheclose}\ {\isacartoucheopen}subset{\isacharunderscore}closed\ Ce{\isacharprime}{\isacartoucheclose}\ \isacommand{by}\isamarkupfalse%
\ {\isacharparenleft}rule\ ex{\isadigit{3}}{\isacharparenright}\isanewline
\ \ \isacommand{obtain}\isamarkupfalse%
\ Ce\ \isakeyword{where}\ H{\isadigit{2}}{\isacharcolon}{\isachardoublequoteopen}Ce{\isacharprime}\ {\isasymsubseteq}\ Ce\ {\isasymand}\ pcp\ Ce\ {\isasymand}\ finite{\isacharunderscore}character\ Ce{\isachardoublequoteclose}\isanewline
\ \ \ \ \isacommand{using}\isamarkupfalse%
\ E{\isadigit{2}}\ \isacommand{by}\isamarkupfalse%
\ {\isacharparenleft}rule\ exE{\isacharparenright}\isanewline
\ \ \isacommand{have}\isamarkupfalse%
\ {\isachardoublequoteopen}Ce{\isacharprime}\ {\isasymsubseteq}\ Ce{\isachardoublequoteclose}\isanewline
\ \ \ \ \isacommand{using}\isamarkupfalse%
\ H{\isadigit{2}}\ \isacommand{by}\isamarkupfalse%
\ {\isacharparenleft}rule\ conjunct{\isadigit{1}}{\isacharparenright}\isanewline
\ \ \isacommand{then}\isamarkupfalse%
\ \isacommand{have}\isamarkupfalse%
\ Subset{\isacharcolon}{\isachardoublequoteopen}C\ {\isasymsubseteq}\ Ce{\isachardoublequoteclose}\isanewline
\ \ \ \ \isacommand{using}\isamarkupfalse%
\ {\isacartoucheopen}C\ {\isasymsubseteq}\ Ce{\isacharprime}{\isacartoucheclose}\ \isacommand{by}\isamarkupfalse%
\ {\isacharparenleft}simp\ only{\isacharcolon}\ subset{\isacharunderscore}trans{\isacharparenright}\isanewline
\ \ \isacommand{have}\isamarkupfalse%
\ Pcp{\isacharcolon}{\isachardoublequoteopen}pcp\ Ce{\isachardoublequoteclose}\isanewline
\ \ \ \ \isacommand{using}\isamarkupfalse%
\ H{\isadigit{2}}\ \isacommand{by}\isamarkupfalse%
\ {\isacharparenleft}iprover\ elim{\isacharcolon}\ conjunct{\isadigit{2}}\ conjunct{\isadigit{1}}{\isacharparenright}\isanewline
\ \ \isacommand{have}\isamarkupfalse%
\ FC{\isacharcolon}{\isachardoublequoteopen}finite{\isacharunderscore}character\ Ce{\isachardoublequoteclose}\isanewline
\ \ \ \ \isacommand{using}\isamarkupfalse%
\ H{\isadigit{2}}\ \isacommand{by}\isamarkupfalse%
\ {\isacharparenleft}iprover\ elim{\isacharcolon}\ conjunct{\isadigit{2}}\ conjunct{\isadigit{1}}{\isacharparenright}\isanewline
\ \ \isacommand{then}\isamarkupfalse%
\ \isacommand{have}\isamarkupfalse%
\ SC{\isacharcolon}{\isachardoublequoteopen}subset{\isacharunderscore}closed\ Ce{\isachardoublequoteclose}\isanewline
\ \ \ \ \isacommand{by}\isamarkupfalse%
\ {\isacharparenleft}rule\ ex{\isadigit{2}}{\isacharparenright}\isanewline
\ \ \isacommand{have}\isamarkupfalse%
\ {\isachardoublequoteopen}S\ {\isasymin}\ C\ {\isasymlongrightarrow}\ S\ {\isasymin}\ Ce{\isachardoublequoteclose}\isanewline
\ \ \ \ \isacommand{using}\isamarkupfalse%
\ {\isacartoucheopen}C\ {\isasymsubseteq}\ Ce{\isacartoucheclose}\ \isacommand{by}\isamarkupfalse%
\ {\isacharparenleft}rule\ in{\isacharunderscore}mono{\isacharparenright}\isanewline
\ \ \isacommand{then}\isamarkupfalse%
\ \isacommand{have}\isamarkupfalse%
\ {\isachardoublequoteopen}S\ {\isasymin}\ Ce{\isachardoublequoteclose}\ \isanewline
\ \ \ \ \isacommand{using}\isamarkupfalse%
\ assms{\isacharparenleft}{\isadigit{2}}{\isacharparenright}\ \isacommand{by}\isamarkupfalse%
\ {\isacharparenleft}rule\ mp{\isacharparenright}\isanewline
\ \ \isacommand{have}\isamarkupfalse%
\ {\isachardoublequoteopen}Hintikka\ {\isacharparenleft}pcp{\isacharunderscore}lim\ Ce\ S{\isacharparenright}{\isachardoublequoteclose}\isanewline
\ \ \ \ \isacommand{using}\isamarkupfalse%
\ Pcp\ SC\ FC\ {\isacartoucheopen}S\ {\isasymin}\ Ce{\isacartoucheclose}\ \isacommand{by}\isamarkupfalse%
\ {\isacharparenleft}rule\ pcp{\isacharunderscore}lim{\isacharunderscore}Hintikka{\isacharparenright}\isanewline
\ \ \isacommand{then}\isamarkupfalse%
\ \isacommand{have}\isamarkupfalse%
\ {\isachardoublequoteopen}sat\ {\isacharparenleft}pcp{\isacharunderscore}lim\ Ce\ S{\isacharparenright}{\isachardoublequoteclose}\isanewline
\ \ \ \ \isacommand{by}\isamarkupfalse%
\ {\isacharparenleft}rule\ Hintikkaslemma{\isacharparenright}\isanewline
\ \ \isacommand{have}\isamarkupfalse%
\ {\isachardoublequoteopen}pcp{\isacharunderscore}seq\ Ce\ S\ {\isadigit{0}}\ {\isacharequal}\ S{\isachardoublequoteclose}\isanewline
\ \ \ \ \isacommand{by}\isamarkupfalse%
\ {\isacharparenleft}simp\ only{\isacharcolon}\ pcp{\isacharunderscore}seq{\isachardot}simps{\isacharparenleft}{\isadigit{1}}{\isacharparenright}{\isacharparenright}\isanewline
\ \ \isacommand{have}\isamarkupfalse%
\ {\isachardoublequoteopen}pcp{\isacharunderscore}seq\ Ce\ S\ {\isadigit{0}}\ {\isasymsubseteq}\ pcp{\isacharunderscore}lim\ Ce\ S{\isachardoublequoteclose}\isanewline
\ \ \ \ \isacommand{by}\isamarkupfalse%
\ {\isacharparenleft}rule\ pcp{\isacharunderscore}seq{\isacharunderscore}sub{\isacharparenright}\isanewline
\ \ \isacommand{then}\isamarkupfalse%
\ \isacommand{have}\isamarkupfalse%
\ {\isachardoublequoteopen}S\ {\isasymsubseteq}\ pcp{\isacharunderscore}lim\ Ce\ S{\isachardoublequoteclose}\isanewline
\ \ \ \ \isacommand{by}\isamarkupfalse%
\ {\isacharparenleft}simp\ only{\isacharcolon}\ {\isacartoucheopen}pcp{\isacharunderscore}seq\ Ce\ S\ {\isadigit{0}}\ {\isacharequal}\ S{\isacartoucheclose}{\isacharparenright}\isanewline
\ \ \isacommand{thus}\isamarkupfalse%
\ {\isachardoublequoteopen}sat\ S{\isachardoublequoteclose}\isanewline
\ \ \ \ \isacommand{using}\isamarkupfalse%
\ {\isacartoucheopen}sat\ {\isacharparenleft}pcp{\isacharunderscore}lim\ Ce\ S{\isacharparenright}{\isacartoucheclose}\ \isacommand{by}\isamarkupfalse%
\ {\isacharparenleft}rule\ sat{\isacharunderscore}mono{\isacharparenright}\isanewline
\isacommand{qed}\isamarkupfalse%
%
\endisatagproof
{\isafoldproof}%
%
\isadelimproof
%
\endisadelimproof
%
\begin{isamarkuptext}%
Finalmente, demostremos el teorema de manera automática.%
\end{isamarkuptext}\isamarkuptrue%
\isacommand{theorem}\isamarkupfalse%
\ pcp{\isacharunderscore}sat{\isacharcolon}\isanewline
\ \ \isakeyword{fixes}\ S\ {\isacharcolon}{\isacharcolon}\ {\isachardoublequoteopen}{\isacharprime}a\ {\isacharcolon}{\isacharcolon}\ countable\ formula\ set{\isachardoublequoteclose}\isanewline
\ \ \isakeyword{assumes}\ c{\isacharcolon}\ {\isachardoublequoteopen}pcp\ C{\isachardoublequoteclose}\isanewline
\ \ \isakeyword{assumes}\ el{\isacharcolon}\ {\isachardoublequoteopen}S\ {\isasymin}\ C{\isachardoublequoteclose}\isanewline
\ \ \isakeyword{shows}\ {\isachardoublequoteopen}sat\ S{\isachardoublequoteclose}\isanewline
%
\isadelimproof
%
\endisadelimproof
%
\isatagproof
\isacommand{proof}\isamarkupfalse%
\ {\isacharminus}\isanewline
\ \ \isacommand{from}\isamarkupfalse%
\ c\ \isacommand{obtain}\isamarkupfalse%
\ Ce\ \isakeyword{where}\ \isanewline
\ \ \ \ \ \ {\isachardoublequoteopen}C\ {\isasymsubseteq}\ Ce{\isachardoublequoteclose}\ {\isachardoublequoteopen}pcp\ Ce{\isachardoublequoteclose}\ {\isachardoublequoteopen}subset{\isacharunderscore}closed\ Ce{\isachardoublequoteclose}\ {\isachardoublequoteopen}finite{\isacharunderscore}character\ Ce{\isachardoublequoteclose}\ \isanewline
\ \ \ \ \ \ \isacommand{using}\isamarkupfalse%
\ ex{\isadigit{1}}{\isacharbrackleft}\isakeyword{where}\ {\isacharprime}a{\isacharequal}{\isacharprime}a{\isacharbrackright}\ ex{\isadigit{2}}{\isacharbrackleft}\isakeyword{where}\ {\isacharprime}a{\isacharequal}{\isacharprime}a{\isacharbrackright}\ ex{\isadigit{3}}{\isacharbrackleft}\isakeyword{where}\ {\isacharprime}a{\isacharequal}{\isacharprime}a{\isacharbrackright}\isanewline
\ \ \ \ \isacommand{by}\isamarkupfalse%
\ {\isacharparenleft}meson\ dual{\isacharunderscore}order{\isachardot}trans\ ex{\isadigit{2}}{\isacharparenright}\isanewline
\ \ \isacommand{have}\isamarkupfalse%
\ {\isachardoublequoteopen}S\ {\isasymin}\ Ce{\isachardoublequoteclose}\ \isacommand{using}\isamarkupfalse%
\ {\isacartoucheopen}C\ {\isasymsubseteq}\ Ce{\isacartoucheclose}\ el\ \isacommand{{\isachardot}{\isachardot}}\isamarkupfalse%
\isanewline
\ \ \isacommand{with}\isamarkupfalse%
\ pcp{\isacharunderscore}lim{\isacharunderscore}Hintikka\ {\isacartoucheopen}pcp\ Ce{\isacartoucheclose}\ {\isacartoucheopen}subset{\isacharunderscore}closed\ Ce{\isacartoucheclose}\ {\isacartoucheopen}finite{\isacharunderscore}character\ Ce{\isacartoucheclose}\isanewline
\ \ \isacommand{have}\isamarkupfalse%
\ \ {\isachardoublequoteopen}Hintikka\ {\isacharparenleft}pcp{\isacharunderscore}lim\ Ce\ S{\isacharparenright}{\isachardoublequoteclose}\ \isacommand{{\isachardot}}\isamarkupfalse%
\isanewline
\ \ \isacommand{with}\isamarkupfalse%
\ Hintikkaslemma\ \isacommand{have}\isamarkupfalse%
\ {\isachardoublequoteopen}sat\ {\isacharparenleft}pcp{\isacharunderscore}lim\ Ce\ S{\isacharparenright}{\isachardoublequoteclose}\ \isacommand{{\isachardot}}\isamarkupfalse%
\isanewline
\ \ \isacommand{moreover}\isamarkupfalse%
\ \isacommand{have}\isamarkupfalse%
\ {\isachardoublequoteopen}S\ {\isasymsubseteq}\ pcp{\isacharunderscore}lim\ Ce\ S{\isachardoublequoteclose}\ \isanewline
\ \ \ \ \isacommand{using}\isamarkupfalse%
\ pcp{\isacharunderscore}seq{\isachardot}simps{\isacharparenleft}{\isadigit{1}}{\isacharparenright}\ pcp{\isacharunderscore}seq{\isacharunderscore}sub\ \isacommand{by}\isamarkupfalse%
\ fast\isanewline
\ \ \isacommand{ultimately}\isamarkupfalse%
\ \isacommand{show}\isamarkupfalse%
\ {\isacharquery}thesis\ \isacommand{unfolding}\isamarkupfalse%
\ sat{\isacharunderscore}def\ \isacommand{by}\isamarkupfalse%
\ fast\isanewline
\isacommand{qed}\isamarkupfalse%
%
\endisatagproof
{\isafoldproof}%
%
\isadelimproof
%
\endisadelimproof
%
\isadelimdocument
%
\endisadelimdocument
%
\isatagdocument
%
\isamarkupsection{Teorema de Compacidad%
}
\isamarkuptrue%
%
\endisatagdocument
{\isafolddocument}%
%
\isadelimdocument
%
\endisadelimdocument
%
\begin{isamarkuptext}%
En esta sección vamos demostrar el \isa{Teorema\ de\ Compacidad} para la lógica proposicional
  como consecuencia del \isa{Teorema\ de\ Existencia\ de\ Modelo}.

  \begin{teorema}[Teorema de Compacidad]
    Todo conjunto de fórmulas finitamente satisfacible es satisfacible.
  \end{teorema}

  Para su demostración consideraremos la colección formada por los conjuntos de fórmulas finitamente 
  satisfacibles. Probaremos que dicha colección verifica la propiedad de consistencia proposicional
  y, por el \isa{Teorema\ de\ Existencia\ de\ Modelo}, todo conjunto perteneciente a ella será
  satisfacible, demostrando así el teorema.

  Mostremos previamente dos resultados sobre subconjuntos finitos que emplearemos en la 
  demostración del teorema.

  \begin{lema}
    Sea \isa{S} un conjunto finito tal que \isa{S\ {\isasymsubseteq}\ {\isacharbraceleft}a{\isacharbraceright}\ {\isasymunion}\ B}. Entonces, existe un conjunto finito \isa{S{\isacharprime}\ {\isasymsubseteq}\ B} 
    tal que o bien \isa{S\ {\isacharequal}\ {\isacharbraceleft}a{\isacharbraceright}\ {\isasymunion}\ S{\isacharprime}} o bien \isa{S\ {\isacharequal}\ S{\isacharprime}}.
  \end{lema}

  \begin{demostracion}
   La prueba se realiza considerando dos casos: \isa{a\ {\isasymin}\ S} o \isa{a\ {\isasymnotin}\ S}.

   Por un lado, si consideramos que \isa{a\ {\isasymin}\ S}, basta tomar el conjunto \isa{S{\isacharprime}\ {\isacharequal}\ S\ {\isacharminus}\ {\isacharbraceleft}a{\isacharbraceright}}. 
   De este modo, como \isa{S\ {\isasymsubseteq}\ {\isacharbraceleft}a{\isacharbraceright}\ {\isasymunion}\ B}, es claro que \isa{S{\isacharprime}\ {\isasymsubseteq}\ B}. Además, puesto que \isa{S} es finito,
   se tiene que \isa{S{\isacharprime}} también lo es. Finalmente, se observa fácilmente que cumple que \isa{S\ {\isacharequal}\ S{\isacharprime}\ {\isasymunion}\ {\isacharbraceleft}a{\isacharbraceright}}.

   Por otro lado, supongamos que \isa{a\ {\isasymnotin}\ S}. En este caso, si tomamos \isa{S{\isacharprime}\ {\isacharequal}\ S} se verifica que
   \isa{S{\isacharprime}\ {\isasymsubseteq}\ B} ya que \isa{S\ {\isasymsubseteq}\ {\isacharbraceleft}a{\isacharbraceright}\ {\isasymunion}\ B} y \isa{a\ {\isasymnotin}\ S}. Además, \isa{S{\isacharprime}} es finito por serlo \isa{S}. Luego cumple
   las condiciones del resultado, como queríamos demostrar.
  \end{demostracion}

  Procedamos con la prueba detallada y formalización en Isabelle. Para ello, hemos utilizado el
  siguiente lema auxiliar.%
\end{isamarkuptext}\isamarkuptrue%
\isacommand{lemma}\isamarkupfalse%
\ subexI\ {\isacharbrackleft}intro{\isacharbrackright}{\isacharcolon}\ {\isachardoublequoteopen}P\ A\ {\isasymLongrightarrow}\ A\ {\isasymsubseteq}\ B\ {\isasymLongrightarrow}\ {\isasymexists}A{\isasymsubseteq}B{\isachardot}\ P\ A{\isachardoublequoteclose}\isanewline
%
\isadelimproof
\ \ %
\endisadelimproof
%
\isatagproof
\isacommand{by}\isamarkupfalse%
\ blast%
\endisatagproof
{\isafoldproof}%
%
\isadelimproof
%
\endisadelimproof
%
\begin{isamarkuptext}%
De este modo, probemos detalladamente el resultado.%
\end{isamarkuptext}\isamarkuptrue%
\isacommand{lemma}\isamarkupfalse%
\ finite{\isacharunderscore}subset{\isacharunderscore}insert{\isadigit{1}}{\isacharcolon}\isanewline
\ \ \isakeyword{assumes}\ {\isachardoublequoteopen}finite\ S{\isachardoublequoteclose}\isanewline
\ \ \ \ \ \ \ \ \ \ {\isachardoublequoteopen}S\ {\isasymsubseteq}\ insert\ a\ B{\isachardoublequoteclose}\isanewline
\ \ \isakeyword{shows}\ {\isachardoublequoteopen}{\isasymexists}S{\isacharprime}\ {\isasymsubseteq}\ B{\isachardot}\ finite\ S{\isacharprime}\ {\isasymand}\ {\isacharparenleft}S\ {\isacharequal}\ insert\ a\ S{\isacharprime}\ {\isasymor}\ S\ {\isacharequal}\ S{\isacharprime}{\isacharparenright}{\isachardoublequoteclose}\isanewline
%
\isadelimproof
%
\endisadelimproof
%
\isatagproof
\isacommand{proof}\isamarkupfalse%
\ {\isacharparenleft}cases\ {\isachardoublequoteopen}a\ {\isasymin}\ S{\isachardoublequoteclose}{\isacharparenright}\isanewline
\ \ \isacommand{assume}\isamarkupfalse%
\ {\isachardoublequoteopen}a\ {\isasymin}\ S{\isachardoublequoteclose}\isanewline
\ \ \isacommand{then}\isamarkupfalse%
\ \isacommand{have}\isamarkupfalse%
\ {\isachardoublequoteopen}S\ {\isacharequal}\ insert\ a\ {\isacharparenleft}S\ {\isacharminus}\ {\isacharbraceleft}a{\isacharbraceright}{\isacharparenright}{\isachardoublequoteclose}\ \isanewline
\ \ \ \ \isacommand{by}\isamarkupfalse%
\ {\isacharparenleft}simp\ only{\isacharcolon}\ insert{\isacharunderscore}Diff{\isacharbrackleft}THEN\ sym{\isacharbrackright}{\isacharparenright}\isanewline
\ \ \isacommand{then}\isamarkupfalse%
\ \isacommand{have}\isamarkupfalse%
\ {\isadigit{1}}{\isacharcolon}{\isachardoublequoteopen}S\ {\isacharequal}\ insert\ a\ {\isacharparenleft}S\ {\isacharminus}\ {\isacharbraceleft}a{\isacharbraceright}{\isacharparenright}\ {\isasymor}\ S\ {\isacharequal}\ S\ {\isacharminus}\ {\isacharbraceleft}a{\isacharbraceright}{\isachardoublequoteclose}\isanewline
\ \ \ \ \isacommand{by}\isamarkupfalse%
\ {\isacharparenleft}rule\ disjI{\isadigit{1}}{\isacharparenright}\isanewline
\ \ \isacommand{have}\isamarkupfalse%
\ {\isadigit{2}}{\isacharcolon}{\isachardoublequoteopen}finite\ {\isacharparenleft}S\ {\isacharminus}\ {\isacharbraceleft}a{\isacharbraceright}{\isacharparenright}{\isachardoublequoteclose}\ \isanewline
\ \ \ \ \isacommand{using}\isamarkupfalse%
\ assms{\isacharparenleft}{\isadigit{1}}{\isacharparenright}\ \isacommand{by}\isamarkupfalse%
\ {\isacharparenleft}rule\ finite{\isacharunderscore}Diff{\isacharparenright}\isanewline
\ \ \isacommand{have}\isamarkupfalse%
\ {\isadigit{3}}{\isacharcolon}{\isachardoublequoteopen}{\isacharparenleft}S\ {\isacharminus}\ {\isacharbraceleft}a{\isacharbraceright}{\isacharparenright}\ {\isasymsubseteq}\ B{\isachardoublequoteclose}\ \isanewline
\ \ \ \ \isacommand{using}\isamarkupfalse%
\ assms{\isacharparenleft}{\isadigit{2}}{\isacharparenright}\ \isacommand{by}\isamarkupfalse%
\ {\isacharparenleft}simp\ add{\isacharcolon}\ Diff{\isacharunderscore}subset{\isacharunderscore}conv{\isacharparenright}\isanewline
\ \ \isacommand{have}\isamarkupfalse%
\ {\isachardoublequoteopen}finite\ {\isacharparenleft}S\ {\isacharminus}\ {\isacharbraceleft}a{\isacharbraceright}{\isacharparenright}\ {\isasymand}\ {\isacharparenleft}S\ {\isacharequal}\ insert\ a\ {\isacharparenleft}S\ {\isacharminus}\ {\isacharbraceleft}a{\isacharbraceright}{\isacharparenright}\ {\isasymor}\ S\ {\isacharequal}\ S\ {\isacharminus}\ {\isacharbraceleft}a{\isacharbraceright}{\isacharparenright}{\isachardoublequoteclose}\isanewline
\ \ \ \ \isacommand{using}\isamarkupfalse%
\ {\isadigit{2}}\ {\isadigit{1}}\ \isacommand{by}\isamarkupfalse%
\ {\isacharparenleft}rule\ conjI{\isacharparenright}\isanewline
\ \ \isacommand{thus}\isamarkupfalse%
\ {\isacharquery}thesis\ \isacommand{using}\isamarkupfalse%
\ {\isadigit{3}}\ \isacommand{by}\isamarkupfalse%
\ {\isacharparenleft}rule\ subexI{\isacharparenright}\isanewline
\isacommand{next}\isamarkupfalse%
\isanewline
\ \ \isacommand{assume}\isamarkupfalse%
\ {\isachardoublequoteopen}a\ {\isasymnotin}\ S{\isachardoublequoteclose}\isanewline
\ \ \isacommand{then}\isamarkupfalse%
\ \isacommand{have}\isamarkupfalse%
\ {\isadigit{1}}{\isacharcolon}{\isachardoublequoteopen}S\ {\isasymsubseteq}\ B{\isachardoublequoteclose}\ \isanewline
\ \ \ \ \isacommand{using}\isamarkupfalse%
\ assms{\isacharparenleft}{\isadigit{2}}{\isacharparenright}\ \isacommand{by}\isamarkupfalse%
\ blast\ \isanewline
\ \ \isacommand{have}\isamarkupfalse%
\ {\isadigit{2}}{\isacharcolon}{\isachardoublequoteopen}S\ {\isacharequal}\ insert\ a\ S\ {\isasymor}\ S\ {\isacharequal}\ S{\isachardoublequoteclose}\isanewline
\ \ \ \ \isacommand{by}\isamarkupfalse%
\ {\isacharparenleft}simp\ only{\isacharcolon}\ disjI{\isadigit{1}}\ simp{\isacharunderscore}thms{\isacharparenright}\isanewline
\ \ \isacommand{have}\isamarkupfalse%
\ {\isachardoublequoteopen}finite\ S\ {\isasymand}\ {\isacharparenleft}S\ {\isacharequal}\ insert\ a\ S\ {\isasymor}\ S\ {\isacharequal}\ S{\isacharparenright}{\isachardoublequoteclose}\isanewline
\ \ \ \ \isacommand{using}\isamarkupfalse%
\ assms{\isacharparenleft}{\isadigit{1}}{\isacharparenright}\ {\isadigit{2}}\ \isacommand{by}\isamarkupfalse%
\ {\isacharparenleft}rule\ conjI{\isacharparenright}\isanewline
\ \ \isacommand{thus}\isamarkupfalse%
\ {\isacharquery}thesis\ \isanewline
\ \ \ \ \isacommand{using}\isamarkupfalse%
\ {\isadigit{1}}\ \isacommand{by}\isamarkupfalse%
\ {\isacharparenleft}rule\ subexI{\isacharparenright}\isanewline
\isacommand{qed}\isamarkupfalse%
%
\endisatagproof
{\isafoldproof}%
%
\isadelimproof
%
\endisadelimproof
%
\begin{isamarkuptext}%
El segundo resultado sobre subconjuntos finitos es consecuencia del anterior.

\begin{lema}
  Sea \isa{S} un conjunto finito tal que \isa{S\ {\isasymsubseteq}\ {\isacharbraceleft}a{\isacharcomma}b{\isacharbraceright}\ {\isasymunion}\ B}. Entonces, existe un conjunto finito \isa{S{\isacharprime}\ {\isasymsubseteq}\ B} 
  tal que se verifica una de las condiciones siguientes: \isa{S\ {\isacharequal}\ {\isacharbraceleft}a{\isacharcomma}b{\isacharbraceright}\ {\isasymunion}\ S{\isacharprime}} o \isa{S\ {\isacharequal}\ {\isacharbraceleft}a{\isacharbraceright}\ {\isasymunion}\ S{\isacharprime}} o 
  \isa{S\ {\isacharequal}\ {\isacharbraceleft}b{\isacharbraceright}\ {\isasymunion}\ S{\isacharprime}} o \isa{S\ {\isacharequal}\ S{\isacharprime}}.
\end{lema}

\begin{demostracion}
  La prueba se realiza considerando cuatro casos:
  \begin{enumerate}
    \item \isa{a\ {\isasymin}\ S} y \isa{b\ {\isasymin}\ S}. 
    \item \isa{a\ {\isasymin}\ S} y \isa{b\ {\isasymnotin}\ S}. 
    \item \isa{a\ {\isasymnotin}\ S} y \isa{b\ {\isasymin}\ S}.
    \item \isa{a\ {\isasymnotin}\ S} y \isa{b\ {\isasymnotin}\ S}. 
  \end{enumerate}

  En primer lugar, supongamos que \isa{a} y \isa{b} están en \isa{S}. En este caso, basta tomar el conjunto
  \isa{S{\isacharprime}\ {\isacharequal}\ S\ {\isacharminus}\ {\isacharbraceleft}a{\isacharcomma}b{\isacharbraceright}}. Como \isa{S\ {\isasymsubseteq}\ {\isacharbraceleft}a{\isacharcomma}b{\isacharbraceright}\ {\isasymunion}\ B}, es claro que \isa{S{\isacharprime}\ {\isasymsubseteq}\ B}. Además, como \isa{S} es finito, se
  tiene que \isa{S{\isacharprime}} también es finito. Por último, verifica que \isa{S\ {\isacharequal}\ {\isacharbraceleft}a{\isacharcomma}b{\isacharbraceright}\ {\isasymunion}\ S{\isacharprime}}, lo que prueba el
  resultado para este caso.

  Supongamos ahora que \isa{a\ {\isasymin}\ S} y \isa{b\ {\isasymnotin}\ S}: considerando el conjunto \isa{S{\isacharprime}\ {\isacharequal}\ S\ {\isacharminus}\ {\isacharbraceleft}a{\isacharbraceright}} se tiene el
  resultado. Como \isa{S\ \ {\isasymsubseteq}\ {\isacharbraceleft}a{\isacharcomma}b{\isacharbraceright}\ {\isasymunion}\ B} y \isa{a\ {\isasymin}\ S} pero \isa{b\ {\isasymnotin}\ S}, es claro que \isa{S{\isacharprime}\ {\isasymsubseteq}\ B}. Como \isa{S} es
  finito, \isa{S{\isacharprime}} también lo es. Por último, se tiene que \isa{S\ {\isacharequal}\ {\isacharbraceleft}a{\isacharbraceright}\ {\isasymunion}\ S{\isacharprime}}, lo que verifica el resultado.

  Consideremos el caso \isa{a\ {\isasymnotin}\ S} y \isa{b\ {\isasymin}\ S}. Basta tomar el conjunto \isa{S{\isacharprime}\ {\isacharequal}\ S\ {\isacharminus}\ {\isacharbraceleft}b{\isacharbraceright}} y, de manera 
  completamente análoga al caso anterior, se demuestra el resultado para dicho caso.

  Finalmente, supongamos que \isa{a\ {\isasymnotin}\ S} y \isa{b\ {\isasymnotin}\ S}. Veamos que tomando el conjunto \isa{S{\isacharprime}\ {\isacharequal}\ S} se cumple
  el resultado. Como \isa{S\ {\isasymsubseteq}\ {\isacharbraceleft}a{\isacharcomma}b{\isacharbraceright}\ {\isasymunion}\ B} y ni \isa{a} ni \isa{b} están en \isa{S}, es claro que \isa{S{\isacharprime}\ {\isasymsubseteq}\ B}.
  Finalmente, como \isa{S} es finito, es claro que \isa{S{\isacharprime}} también lo es. Por lo tanto, queda probado el
  resultado.
\end{demostracion}

  Su formalización y prueba detallada en Isabelle/HOL son las siguientes.%
\end{isamarkuptext}\isamarkuptrue%
\isacommand{lemma}\isamarkupfalse%
\ finite{\isacharunderscore}subset{\isacharunderscore}insert{\isadigit{2}}{\isacharcolon}\isanewline
\ \ \isakeyword{assumes}\ {\isachardoublequoteopen}finite\ S{\isachardoublequoteclose}\isanewline
\ \ \ \ \ \ \ \ \ \ {\isachardoublequoteopen}S\ {\isasymsubseteq}\ {\isacharbraceleft}a{\isacharcomma}b{\isacharbraceright}\ {\isasymunion}\ B{\isachardoublequoteclose}\isanewline
\ \ \ \ \ \ \ \ \isakeyword{shows}\ {\isachardoublequoteopen}{\isasymexists}S{\isacharprime}\ {\isasymsubseteq}\ B{\isachardot}\ finite\ S{\isacharprime}\ {\isasymand}\ {\isacharparenleft}S\ {\isacharequal}\ {\isacharbraceleft}a{\isacharcomma}b{\isacharbraceright}\ {\isasymunion}\ S{\isacharprime}\ {\isasymor}\ S\ {\isacharequal}\ {\isacharbraceleft}a{\isacharbraceright}\ {\isasymunion}\ S{\isacharprime}\ {\isasymor}\ S\ {\isacharequal}\ {\isacharbraceleft}b{\isacharbraceright}\ {\isasymunion}\ S{\isacharprime}\ {\isasymor}\ S\ {\isacharequal}\ S{\isacharprime}{\isacharparenright}{\isachardoublequoteclose}\isanewline
%
\isadelimproof
%
\endisadelimproof
%
\isatagproof
\isacommand{proof}\isamarkupfalse%
\ {\isacharparenleft}cases\ {\isachardoublequoteopen}a\ {\isasymin}\ S\ {\isasymor}\ b\ {\isasymin}\ S{\isachardoublequoteclose}{\isacharparenright}\isanewline
\ \ \isacommand{assume}\isamarkupfalse%
\ {\isachardoublequoteopen}a\ {\isasymin}\ S\ {\isasymor}\ b\ {\isasymin}\ S{\isachardoublequoteclose}\isanewline
\ \ \isacommand{thus}\isamarkupfalse%
\ {\isacharquery}thesis\isanewline
\ \ \isacommand{proof}\isamarkupfalse%
\ {\isacharparenleft}rule\ disjE{\isacharparenright}\isanewline
\ \ \ \ \isacommand{assume}\isamarkupfalse%
\ {\isachardoublequoteopen}a\ {\isasymin}\ S{\isachardoublequoteclose}\isanewline
\ \ \ \ \isacommand{show}\isamarkupfalse%
\ {\isacharquery}thesis\isanewline
\ \ \ \ \isacommand{proof}\isamarkupfalse%
\ {\isacharparenleft}cases\ {\isachardoublequoteopen}b\ {\isasymin}\ S{\isachardoublequoteclose}{\isacharparenright}\isanewline
\ \ \ \ \ \ \isacommand{assume}\isamarkupfalse%
\ {\isachardoublequoteopen}b\ {\isasymin}\ S{\isachardoublequoteclose}\isanewline
\ \ \ \ \ \ \isacommand{have}\isamarkupfalse%
\ {\isadigit{1}}{\isacharcolon}{\isachardoublequoteopen}S\ {\isacharminus}\ {\isacharbraceleft}a{\isacharcomma}b{\isacharbraceright}\ {\isasymsubseteq}\ B{\isachardoublequoteclose}\isanewline
\ \ \ \ \ \ \ \ \isacommand{using}\isamarkupfalse%
\ assms{\isacharparenleft}{\isadigit{2}}{\isacharparenright}\ \isacommand{by}\isamarkupfalse%
\ blast\isanewline
\ \ \ \ \ \ \isacommand{have}\isamarkupfalse%
\ {\isachardoublequoteopen}{\isacharbraceleft}a{\isacharcomma}b{\isacharbraceright}\ {\isasymunion}\ S\ {\isacharequal}\ S{\isachardoublequoteclose}\isanewline
\ \ \ \ \ \ \ \ \isacommand{using}\isamarkupfalse%
\ {\isacartoucheopen}a\ {\isasymin}\ S{\isacartoucheclose}\ {\isacartoucheopen}b\ {\isasymin}\ S{\isacartoucheclose}\ \isacommand{by}\isamarkupfalse%
\ blast\isanewline
\ \ \ \ \ \ \isacommand{then}\isamarkupfalse%
\ \isacommand{have}\isamarkupfalse%
\ {\isachardoublequoteopen}S\ {\isacharequal}\ {\isacharbraceleft}a{\isacharcomma}b{\isacharbraceright}\ {\isasymunion}\ {\isacharparenleft}S\ {\isacharminus}\ {\isacharbraceleft}a{\isacharcomma}b{\isacharbraceright}{\isacharparenright}{\isachardoublequoteclose}\ \isanewline
\ \ \ \ \ \ \ \ \isacommand{by}\isamarkupfalse%
\ blast\isanewline
\ \ \ \ \ \ \isacommand{then}\isamarkupfalse%
\ \isacommand{have}\isamarkupfalse%
\ {\isadigit{2}}{\isacharcolon}{\isachardoublequoteopen}S\ {\isacharequal}\ {\isacharbraceleft}a{\isacharcomma}b{\isacharbraceright}\ {\isasymunion}\ {\isacharparenleft}S\ {\isacharminus}\ {\isacharbraceleft}a{\isacharcomma}b{\isacharbraceright}{\isacharparenright}\ {\isasymor}\ S\ {\isacharequal}\ {\isacharbraceleft}a{\isacharbraceright}\ {\isasymunion}\ {\isacharparenleft}S\ {\isacharminus}\ {\isacharbraceleft}a{\isacharcomma}b{\isacharbraceright}{\isacharparenright}\ {\isasymor}\ S\ {\isacharequal}\ {\isacharbraceleft}b{\isacharbraceright}\ {\isasymunion}\ {\isacharparenleft}S\ {\isacharminus}\ {\isacharbraceleft}a{\isacharcomma}b{\isacharbraceright}{\isacharparenright}\ {\isasymor}\ S\ {\isacharequal}\ {\isacharparenleft}S\ {\isacharminus}\ {\isacharbraceleft}a{\isacharcomma}b{\isacharbraceright}{\isacharparenright}{\isachardoublequoteclose}\isanewline
\ \ \ \ \ \ \ \ \isacommand{by}\isamarkupfalse%
\ {\isacharparenleft}iprover\ intro{\isacharcolon}\ disjI{\isadigit{1}}{\isacharparenright}\isanewline
\ \ \ \ \ \ \isacommand{have}\isamarkupfalse%
\ {\isachardoublequoteopen}finite\ {\isacharparenleft}S\ {\isacharminus}\ {\isacharbraceleft}a{\isacharcomma}b{\isacharbraceright}{\isacharparenright}{\isachardoublequoteclose}\isanewline
\ \ \ \ \ \ \ \ \isacommand{using}\isamarkupfalse%
\ assms{\isacharparenleft}{\isadigit{1}}{\isacharparenright}\ \isacommand{by}\isamarkupfalse%
\ {\isacharparenleft}rule\ finite{\isacharunderscore}Diff{\isacharparenright}\isanewline
\ \ \ \ \ \ \isacommand{then}\isamarkupfalse%
\ \isacommand{have}\isamarkupfalse%
\ {\isachardoublequoteopen}finite\ {\isacharparenleft}S\ {\isacharminus}\ {\isacharbraceleft}a{\isacharcomma}b{\isacharbraceright}{\isacharparenright}\ {\isasymand}\ {\isacharparenleft}S\ {\isacharequal}\ {\isacharbraceleft}a{\isacharcomma}b{\isacharbraceright}\ {\isasymunion}\ {\isacharparenleft}S\ {\isacharminus}\ {\isacharbraceleft}a{\isacharcomma}b{\isacharbraceright}{\isacharparenright}\ {\isasymor}\ S\ {\isacharequal}\ {\isacharbraceleft}a{\isacharbraceright}\ {\isasymunion}\ {\isacharparenleft}S\ {\isacharminus}\ {\isacharbraceleft}a{\isacharcomma}b{\isacharbraceright}{\isacharparenright}\ {\isasymor}\ S\ {\isacharequal}\ {\isacharbraceleft}b{\isacharbraceright}\ {\isasymunion}\ {\isacharparenleft}S\ {\isacharminus}\ {\isacharbraceleft}a{\isacharcomma}b{\isacharbraceright}{\isacharparenright}\ {\isasymor}\ S\ {\isacharequal}\ {\isacharparenleft}S\ {\isacharminus}\ {\isacharbraceleft}a{\isacharcomma}b{\isacharbraceright}{\isacharparenright}{\isacharparenright}{\isachardoublequoteclose}\isanewline
\ \ \ \ \ \ \ \ \isacommand{using}\isamarkupfalse%
\ {\isadigit{2}}\ \isacommand{by}\isamarkupfalse%
\ {\isacharparenleft}rule\ conjI{\isacharparenright}\isanewline
\ \ \ \ \ \ \isacommand{thus}\isamarkupfalse%
\ {\isacharquery}thesis\isanewline
\ \ \ \ \ \ \ \ \isacommand{using}\isamarkupfalse%
\ {\isadigit{1}}\ \isacommand{by}\isamarkupfalse%
\ {\isacharparenleft}rule\ subexI{\isacharparenright}\isanewline
\ \ \ \ \isacommand{next}\isamarkupfalse%
\isanewline
\ \ \ \ \ \ \isacommand{assume}\isamarkupfalse%
\ {\isachardoublequoteopen}b\ {\isasymnotin}\ S{\isachardoublequoteclose}\isanewline
\ \ \ \ \ \ \isacommand{then}\isamarkupfalse%
\ \isacommand{have}\isamarkupfalse%
\ {\isadigit{1}}{\isacharcolon}{\isachardoublequoteopen}S\ {\isacharminus}\ {\isacharbraceleft}a{\isacharbraceright}\ {\isasymsubseteq}\ B{\isachardoublequoteclose}\isanewline
\ \ \ \ \ \ \ \ \isacommand{using}\isamarkupfalse%
\ assms{\isacharparenleft}{\isadigit{2}}{\isacharparenright}\ \isacommand{by}\isamarkupfalse%
\ blast\isanewline
\ \ \ \ \ \ \isacommand{have}\isamarkupfalse%
\ {\isachardoublequoteopen}{\isacharbraceleft}a{\isacharbraceright}\ {\isasymunion}\ S\ {\isacharequal}\ S{\isachardoublequoteclose}\isanewline
\ \ \ \ \ \ \ \ \isacommand{using}\isamarkupfalse%
\ {\isacartoucheopen}a\ {\isasymin}\ S{\isacartoucheclose}\ \isacommand{by}\isamarkupfalse%
\ blast\isanewline
\ \ \ \ \ \ \isacommand{then}\isamarkupfalse%
\ \isacommand{have}\isamarkupfalse%
\ {\isachardoublequoteopen}S\ {\isacharequal}\ {\isacharbraceleft}a{\isacharbraceright}\ {\isasymunion}\ {\isacharparenleft}S\ {\isacharminus}\ {\isacharbraceleft}a{\isacharbraceright}{\isacharparenright}{\isachardoublequoteclose}\ \isanewline
\ \ \ \ \ \ \ \ \isacommand{by}\isamarkupfalse%
\ blast\isanewline
\ \ \ \ \ \ \isacommand{then}\isamarkupfalse%
\ \isacommand{have}\isamarkupfalse%
\ {\isadigit{2}}{\isacharcolon}{\isachardoublequoteopen}S\ {\isacharequal}\ {\isacharbraceleft}a{\isacharcomma}b{\isacharbraceright}\ {\isasymunion}\ {\isacharparenleft}S\ {\isacharminus}\ {\isacharbraceleft}a{\isacharbraceright}{\isacharparenright}\ {\isasymor}\ S\ {\isacharequal}\ {\isacharbraceleft}a{\isacharbraceright}\ {\isasymunion}\ {\isacharparenleft}S\ {\isacharminus}\ {\isacharbraceleft}a{\isacharbraceright}{\isacharparenright}\ {\isasymor}\ S\ {\isacharequal}\ {\isacharbraceleft}b{\isacharbraceright}\ {\isasymunion}\ {\isacharparenleft}S\ {\isacharminus}\ {\isacharbraceleft}a{\isacharbraceright}{\isacharparenright}\ {\isasymor}\ S\ {\isacharequal}\ {\isacharparenleft}S\ {\isacharminus}\ {\isacharbraceleft}a{\isacharbraceright}{\isacharparenright}{\isachardoublequoteclose}\isanewline
\ \ \ \ \ \ \ \ \isacommand{by}\isamarkupfalse%
\ {\isacharparenleft}iprover\ intro{\isacharcolon}\ disjI{\isadigit{1}}{\isacharparenright}\isanewline
\ \ \ \ \ \ \isacommand{have}\isamarkupfalse%
\ {\isachardoublequoteopen}finite\ {\isacharparenleft}S\ {\isacharminus}\ {\isacharbraceleft}a{\isacharbraceright}{\isacharparenright}{\isachardoublequoteclose}\isanewline
\ \ \ \ \ \ \ \ \isacommand{using}\isamarkupfalse%
\ assms{\isacharparenleft}{\isadigit{1}}{\isacharparenright}\ \isacommand{by}\isamarkupfalse%
\ {\isacharparenleft}rule\ finite{\isacharunderscore}Diff{\isacharparenright}\isanewline
\ \ \ \ \ \ \isacommand{then}\isamarkupfalse%
\ \isacommand{have}\isamarkupfalse%
\ {\isachardoublequoteopen}finite\ {\isacharparenleft}S\ {\isacharminus}\ {\isacharbraceleft}a{\isacharbraceright}{\isacharparenright}\ {\isasymand}\ {\isacharparenleft}S\ {\isacharequal}\ {\isacharbraceleft}a{\isacharcomma}b{\isacharbraceright}\ {\isasymunion}\ {\isacharparenleft}S\ {\isacharminus}\ {\isacharbraceleft}a{\isacharbraceright}{\isacharparenright}\ {\isasymor}\ S\ {\isacharequal}\ {\isacharbraceleft}a{\isacharbraceright}\ {\isasymunion}\ {\isacharparenleft}S\ {\isacharminus}\ {\isacharbraceleft}a{\isacharbraceright}{\isacharparenright}\ {\isasymor}\ S\ {\isacharequal}\ {\isacharbraceleft}b{\isacharbraceright}\ {\isasymunion}\ {\isacharparenleft}S\ {\isacharminus}\ {\isacharbraceleft}a{\isacharbraceright}{\isacharparenright}\ {\isasymor}\ S\ {\isacharequal}\ {\isacharparenleft}S\ {\isacharminus}\ {\isacharbraceleft}a{\isacharbraceright}{\isacharparenright}{\isacharparenright}{\isachardoublequoteclose}\isanewline
\ \ \ \ \ \ \ \ \isacommand{using}\isamarkupfalse%
\ {\isadigit{2}}\ \isacommand{by}\isamarkupfalse%
\ {\isacharparenleft}rule\ conjI{\isacharparenright}\isanewline
\ \ \ \ \ \ \isacommand{thus}\isamarkupfalse%
\ {\isacharquery}thesis\isanewline
\ \ \ \ \ \ \ \ \isacommand{using}\isamarkupfalse%
\ {\isadigit{1}}\ \isacommand{by}\isamarkupfalse%
\ {\isacharparenleft}rule\ subexI{\isacharparenright}\isanewline
\ \ \ \ \isacommand{qed}\isamarkupfalse%
\isanewline
\ \ \isacommand{next}\isamarkupfalse%
\isanewline
\ \ \ \ \isacommand{assume}\isamarkupfalse%
\ {\isachardoublequoteopen}b\ {\isasymin}\ S{\isachardoublequoteclose}\isanewline
\ \ \ \ \isacommand{show}\isamarkupfalse%
\ {\isacharquery}thesis\isanewline
\ \ \ \ \isacommand{proof}\isamarkupfalse%
\ {\isacharparenleft}cases\ {\isachardoublequoteopen}a\ {\isasymin}\ S{\isachardoublequoteclose}{\isacharparenright}\isanewline
\ \ \ \ \ \ \isacommand{assume}\isamarkupfalse%
\ {\isachardoublequoteopen}a\ {\isasymin}\ S{\isachardoublequoteclose}\isanewline
\ \ \ \ \ \ \isacommand{have}\isamarkupfalse%
\ {\isadigit{1}}{\isacharcolon}{\isachardoublequoteopen}S\ {\isacharminus}\ {\isacharbraceleft}a{\isacharcomma}b{\isacharbraceright}\ {\isasymsubseteq}\ B{\isachardoublequoteclose}\isanewline
\ \ \ \ \ \ \ \ \isacommand{using}\isamarkupfalse%
\ assms{\isacharparenleft}{\isadigit{2}}{\isacharparenright}\ \isacommand{by}\isamarkupfalse%
\ blast\isanewline
\ \ \ \ \ \ \isacommand{have}\isamarkupfalse%
\ {\isachardoublequoteopen}{\isacharbraceleft}a{\isacharcomma}b{\isacharbraceright}\ {\isasymunion}\ S\ {\isacharequal}\ S{\isachardoublequoteclose}\isanewline
\ \ \ \ \ \ \ \ \isacommand{using}\isamarkupfalse%
\ {\isacartoucheopen}a\ {\isasymin}\ S{\isacartoucheclose}\ {\isacartoucheopen}b\ {\isasymin}\ S{\isacartoucheclose}\ \isacommand{by}\isamarkupfalse%
\ blast\isanewline
\ \ \ \ \ \ \isacommand{then}\isamarkupfalse%
\ \isacommand{have}\isamarkupfalse%
\ {\isachardoublequoteopen}S\ {\isacharequal}\ {\isacharbraceleft}a{\isacharcomma}b{\isacharbraceright}\ {\isasymunion}\ {\isacharparenleft}S\ {\isacharminus}\ {\isacharbraceleft}a{\isacharcomma}b{\isacharbraceright}{\isacharparenright}{\isachardoublequoteclose}\ \isanewline
\ \ \ \ \ \ \ \ \isacommand{by}\isamarkupfalse%
\ blast\isanewline
\ \ \ \ \ \ \isacommand{then}\isamarkupfalse%
\ \isacommand{have}\isamarkupfalse%
\ {\isadigit{2}}{\isacharcolon}{\isachardoublequoteopen}S\ {\isacharequal}\ {\isacharbraceleft}a{\isacharcomma}b{\isacharbraceright}\ {\isasymunion}\ {\isacharparenleft}S\ {\isacharminus}\ {\isacharbraceleft}a{\isacharcomma}b{\isacharbraceright}{\isacharparenright}\ {\isasymor}\ S\ {\isacharequal}\ {\isacharbraceleft}a{\isacharbraceright}\ {\isasymunion}\ {\isacharparenleft}S\ {\isacharminus}\ {\isacharbraceleft}a{\isacharcomma}b{\isacharbraceright}{\isacharparenright}\ {\isasymor}\ S\ {\isacharequal}\ {\isacharbraceleft}b{\isacharbraceright}\ {\isasymunion}\ {\isacharparenleft}S\ {\isacharminus}\ {\isacharbraceleft}a{\isacharcomma}b{\isacharbraceright}{\isacharparenright}\ {\isasymor}\ S\ {\isacharequal}\ {\isacharparenleft}S\ {\isacharminus}\ {\isacharbraceleft}a{\isacharcomma}b{\isacharbraceright}{\isacharparenright}{\isachardoublequoteclose}\isanewline
\ \ \ \ \ \ \ \ \isacommand{by}\isamarkupfalse%
\ {\isacharparenleft}iprover\ intro{\isacharcolon}\ disjI{\isadigit{1}}{\isacharparenright}\isanewline
\ \ \ \ \ \ \isacommand{have}\isamarkupfalse%
\ {\isachardoublequoteopen}finite\ {\isacharparenleft}S\ {\isacharminus}\ {\isacharbraceleft}a{\isacharcomma}b{\isacharbraceright}{\isacharparenright}{\isachardoublequoteclose}\isanewline
\ \ \ \ \ \ \ \ \isacommand{using}\isamarkupfalse%
\ assms{\isacharparenleft}{\isadigit{1}}{\isacharparenright}\ \isacommand{by}\isamarkupfalse%
\ {\isacharparenleft}rule\ finite{\isacharunderscore}Diff{\isacharparenright}\isanewline
\ \ \ \ \ \ \isacommand{then}\isamarkupfalse%
\ \isacommand{have}\isamarkupfalse%
\ {\isachardoublequoteopen}finite\ {\isacharparenleft}S\ {\isacharminus}\ {\isacharbraceleft}a{\isacharcomma}b{\isacharbraceright}{\isacharparenright}\ {\isasymand}\ {\isacharparenleft}S\ {\isacharequal}\ {\isacharbraceleft}a{\isacharcomma}b{\isacharbraceright}\ {\isasymunion}\ {\isacharparenleft}S\ {\isacharminus}\ {\isacharbraceleft}a{\isacharcomma}b{\isacharbraceright}{\isacharparenright}\ {\isasymor}\ S\ {\isacharequal}\ {\isacharbraceleft}a{\isacharbraceright}\ {\isasymunion}\ {\isacharparenleft}S\ {\isacharminus}\ {\isacharbraceleft}a{\isacharcomma}b{\isacharbraceright}{\isacharparenright}\ {\isasymor}\ S\ {\isacharequal}\ {\isacharbraceleft}b{\isacharbraceright}\ {\isasymunion}\ {\isacharparenleft}S\ {\isacharminus}\ {\isacharbraceleft}a{\isacharcomma}b{\isacharbraceright}{\isacharparenright}\ {\isasymor}\ S\ {\isacharequal}\ {\isacharparenleft}S\ {\isacharminus}\ {\isacharbraceleft}a{\isacharcomma}b{\isacharbraceright}{\isacharparenright}{\isacharparenright}{\isachardoublequoteclose}\isanewline
\ \ \ \ \ \ \ \ \isacommand{using}\isamarkupfalse%
\ {\isadigit{2}}\ \isacommand{by}\isamarkupfalse%
\ {\isacharparenleft}rule\ conjI{\isacharparenright}\isanewline
\ \ \ \ \ \ \isacommand{thus}\isamarkupfalse%
\ {\isacharquery}thesis\isanewline
\ \ \ \ \ \ \ \ \isacommand{using}\isamarkupfalse%
\ {\isadigit{1}}\ \isacommand{by}\isamarkupfalse%
\ {\isacharparenleft}rule\ subexI{\isacharparenright}\isanewline
\ \ \ \ \isacommand{next}\isamarkupfalse%
\isanewline
\ \ \ \ \ \ \isacommand{assume}\isamarkupfalse%
\ {\isachardoublequoteopen}a\ {\isasymnotin}\ S{\isachardoublequoteclose}\isanewline
\ \ \ \ \ \ \isacommand{then}\isamarkupfalse%
\ \isacommand{have}\isamarkupfalse%
\ {\isadigit{1}}{\isacharcolon}{\isachardoublequoteopen}S\ {\isacharminus}\ {\isacharbraceleft}b{\isacharbraceright}\ {\isasymsubseteq}\ B{\isachardoublequoteclose}\isanewline
\ \ \ \ \ \ \ \ \isacommand{using}\isamarkupfalse%
\ assms{\isacharparenleft}{\isadigit{2}}{\isacharparenright}\ \isacommand{by}\isamarkupfalse%
\ blast\isanewline
\ \ \ \ \ \ \isacommand{have}\isamarkupfalse%
\ {\isachardoublequoteopen}{\isacharbraceleft}b{\isacharbraceright}\ {\isasymunion}\ S\ {\isacharequal}\ S{\isachardoublequoteclose}\isanewline
\ \ \ \ \ \ \ \ \isacommand{using}\isamarkupfalse%
\ {\isacartoucheopen}b\ {\isasymin}\ S{\isacartoucheclose}\ \isacommand{by}\isamarkupfalse%
\ blast\isanewline
\ \ \ \ \ \ \isacommand{then}\isamarkupfalse%
\ \isacommand{have}\isamarkupfalse%
\ {\isachardoublequoteopen}S\ {\isacharequal}\ {\isacharbraceleft}b{\isacharbraceright}\ {\isasymunion}\ {\isacharparenleft}S\ {\isacharminus}\ {\isacharbraceleft}b{\isacharbraceright}{\isacharparenright}{\isachardoublequoteclose}\ \isanewline
\ \ \ \ \ \ \ \ \isacommand{by}\isamarkupfalse%
\ blast\isanewline
\ \ \ \ \ \ \isacommand{then}\isamarkupfalse%
\ \isacommand{have}\isamarkupfalse%
\ {\isadigit{2}}{\isacharcolon}{\isachardoublequoteopen}S\ {\isacharequal}\ {\isacharbraceleft}a{\isacharcomma}b{\isacharbraceright}\ {\isasymunion}\ {\isacharparenleft}S\ {\isacharminus}\ {\isacharbraceleft}b{\isacharbraceright}{\isacharparenright}\ {\isasymor}\ S\ {\isacharequal}\ {\isacharbraceleft}a{\isacharbraceright}\ {\isasymunion}\ {\isacharparenleft}S\ {\isacharminus}\ {\isacharbraceleft}b{\isacharbraceright}{\isacharparenright}\ {\isasymor}\ S\ {\isacharequal}\ {\isacharbraceleft}b{\isacharbraceright}\ {\isasymunion}\ {\isacharparenleft}S\ {\isacharminus}\ {\isacharbraceleft}b{\isacharbraceright}{\isacharparenright}\ {\isasymor}\ S\ {\isacharequal}\ {\isacharparenleft}S\ {\isacharminus}\ {\isacharbraceleft}b{\isacharbraceright}{\isacharparenright}{\isachardoublequoteclose}\isanewline
\ \ \ \ \ \ \ \ \isacommand{by}\isamarkupfalse%
\ {\isacharparenleft}iprover\ intro{\isacharcolon}\ disjI{\isadigit{1}}{\isacharparenright}\isanewline
\ \ \ \ \ \ \isacommand{have}\isamarkupfalse%
\ {\isachardoublequoteopen}finite\ {\isacharparenleft}S\ {\isacharminus}\ {\isacharbraceleft}b{\isacharbraceright}{\isacharparenright}{\isachardoublequoteclose}\isanewline
\ \ \ \ \ \ \ \ \isacommand{using}\isamarkupfalse%
\ assms{\isacharparenleft}{\isadigit{1}}{\isacharparenright}\ \isacommand{by}\isamarkupfalse%
\ {\isacharparenleft}rule\ finite{\isacharunderscore}Diff{\isacharparenright}\isanewline
\ \ \ \ \ \ \isacommand{then}\isamarkupfalse%
\ \isacommand{have}\isamarkupfalse%
\ {\isachardoublequoteopen}finite\ {\isacharparenleft}S\ {\isacharminus}\ {\isacharbraceleft}b{\isacharbraceright}{\isacharparenright}\ {\isasymand}\ {\isacharparenleft}S\ {\isacharequal}\ {\isacharbraceleft}a{\isacharcomma}b{\isacharbraceright}\ {\isasymunion}\ {\isacharparenleft}S\ {\isacharminus}\ {\isacharbraceleft}b{\isacharbraceright}{\isacharparenright}\ {\isasymor}\ S\ {\isacharequal}\ {\isacharbraceleft}a{\isacharbraceright}\ {\isasymunion}\ {\isacharparenleft}S\ {\isacharminus}\ {\isacharbraceleft}b{\isacharbraceright}{\isacharparenright}\ {\isasymor}\ S\ {\isacharequal}\ {\isacharbraceleft}b{\isacharbraceright}\ {\isasymunion}\ {\isacharparenleft}S\ {\isacharminus}\ {\isacharbraceleft}b{\isacharbraceright}{\isacharparenright}\ {\isasymor}\ S\ {\isacharequal}\ {\isacharparenleft}S\ {\isacharminus}\ {\isacharbraceleft}b{\isacharbraceright}{\isacharparenright}{\isacharparenright}{\isachardoublequoteclose}\isanewline
\ \ \ \ \ \ \ \ \isacommand{using}\isamarkupfalse%
\ {\isadigit{2}}\ \isacommand{by}\isamarkupfalse%
\ {\isacharparenleft}rule\ conjI{\isacharparenright}\isanewline
\ \ \ \ \ \ \isacommand{thus}\isamarkupfalse%
\ {\isacharquery}thesis\isanewline
\ \ \ \ \ \ \ \ \isacommand{using}\isamarkupfalse%
\ {\isadigit{1}}\ \isacommand{by}\isamarkupfalse%
\ {\isacharparenleft}rule\ subexI{\isacharparenright}\isanewline
\ \ \ \ \isacommand{qed}\isamarkupfalse%
\isanewline
\ \ \isacommand{qed}\isamarkupfalse%
\isanewline
\isacommand{next}\isamarkupfalse%
\isanewline
\ \ \isacommand{assume}\isamarkupfalse%
\ {\isachardoublequoteopen}{\isasymnot}{\isacharparenleft}a\ {\isasymin}\ S\ {\isasymor}\ b\ {\isasymin}\ S{\isacharparenright}{\isachardoublequoteclose}\isanewline
\ \ \isacommand{then}\isamarkupfalse%
\ \isacommand{have}\isamarkupfalse%
\ {\isachardoublequoteopen}a\ {\isasymnotin}\ S\ {\isasymand}\ b\ {\isasymnotin}\ S{\isachardoublequoteclose}\isanewline
\ \ \ \ \isacommand{by}\isamarkupfalse%
\ {\isacharparenleft}simp\ only{\isacharcolon}\ de{\isacharunderscore}Morgan{\isacharunderscore}disj\ simp{\isacharunderscore}thms{\isacharparenleft}{\isadigit{8}}{\isacharparenright}{\isacharparenright}\isanewline
\ \ \isacommand{then}\isamarkupfalse%
\ \isacommand{have}\isamarkupfalse%
\ {\isadigit{1}}{\isacharcolon}{\isachardoublequoteopen}S\ {\isasymsubseteq}\ B{\isachardoublequoteclose}\isanewline
\ \ \ \ \isacommand{using}\isamarkupfalse%
\ assms{\isacharparenleft}{\isadigit{2}}{\isacharparenright}\ \isacommand{by}\isamarkupfalse%
\ blast\isanewline
\ \ \isacommand{have}\isamarkupfalse%
\ {\isadigit{2}}{\isacharcolon}{\isachardoublequoteopen}S\ {\isacharequal}\ {\isacharbraceleft}a{\isacharcomma}b{\isacharbraceright}\ {\isasymunion}\ S\ {\isasymor}\ S\ {\isacharequal}\ {\isacharbraceleft}a{\isacharbraceright}\ {\isasymunion}\ S\ {\isasymor}\ S\ {\isacharequal}\ {\isacharbraceleft}b{\isacharbraceright}\ {\isasymunion}\ S\ {\isasymor}\ S\ {\isacharequal}\ S{\isachardoublequoteclose}\isanewline
\ \ \ \ \isacommand{by}\isamarkupfalse%
\ {\isacharparenleft}iprover\ intro{\isacharcolon}\ disjI{\isadigit{1}}\ simp{\isacharunderscore}thms{\isacharparenright}\isanewline
\ \ \isacommand{have}\isamarkupfalse%
\ {\isachardoublequoteopen}finite\ S\ {\isasymand}\ {\isacharparenleft}S\ {\isacharequal}\ {\isacharbraceleft}a{\isacharcomma}b{\isacharbraceright}\ {\isasymunion}\ S\ {\isasymor}\ S\ {\isacharequal}\ {\isacharbraceleft}a{\isacharbraceright}\ {\isasymunion}\ S\ {\isasymor}\ S\ {\isacharequal}\ {\isacharbraceleft}b{\isacharbraceright}\ {\isasymunion}\ S\ {\isasymor}\ S\ {\isacharequal}\ S{\isacharparenright}{\isachardoublequoteclose}\isanewline
\ \ \ \ \isacommand{using}\isamarkupfalse%
\ assms{\isacharparenleft}{\isadigit{1}}{\isacharparenright}\ {\isadigit{2}}\ \isacommand{by}\isamarkupfalse%
\ {\isacharparenleft}rule\ conjI{\isacharparenright}\isanewline
\ \ \isacommand{thus}\isamarkupfalse%
\ {\isacharquery}thesis\isanewline
\ \ \ \ \isacommand{using}\isamarkupfalse%
\ {\isadigit{1}}\ \isacommand{by}\isamarkupfalse%
\ {\isacharparenleft}rule\ subexI{\isacharparenright}\isanewline
\isacommand{qed}\isamarkupfalse%
%
\endisatagproof
{\isafoldproof}%
%
\isadelimproof
%
\endisadelimproof
%
\begin{isamarkuptext}%
Una vez introducidos los resultados anteriores, procedamos con la prueba del \isa{Teorema\ de\ Compacidad}.

  \begin{demostracion}
    Consideremos la colección \isa{C} formada por los conjuntos de fórmulas finitamente satisfacibles.
    Recordemos que un conjunto de fórmulas es finitamente satisfacible si todo subconjunto finito 
    suyo es satisfacible. Vamos a probar que dicha colección verifica la propiedad de consistencia 
    proposicional y, por el \isa{Teorema\ de\ Existencia\ de\ Modelo}, quedará probado que todo conjunto de 
    \isa{C} es satisfacible, lo que demuestra el teorema.

    Para probar que \isa{C} verifica la propiedad de consistencia proposicional, por el lema \isa{{\isadigit{2}}{\isachardot}{\isadigit{0}}{\isachardot}{\isadigit{2}}} de 
    caracterización mediante notación uniforme, basta demostrar que se verifican las siguientes 
    condiciones para todo conjunto \isa{W\ {\isasymin}\ C}:
    \begin{itemize}
     \item \isa{{\isasymbottom}\ {\isasymnotin}\ W}.
     \item Dada \isa{p} una fórmula atómica cualquiera, no se tiene 
      simultáneamente que\\ \isa{p\ {\isasymin}\ W} y \isa{{\isasymnot}\ p\ {\isasymin}\ W}.
     \item Para toda fórmula de tipo \isa{{\isasymalpha}} con componentes \isa{{\isasymalpha}\isactrlsub {\isadigit{1}}} y \isa{{\isasymalpha}\isactrlsub {\isadigit{2}}} tal que \isa{{\isasymalpha}}
      pertenece a \isa{W}, se tiene que \isa{{\isacharbraceleft}{\isasymalpha}\isactrlsub {\isadigit{1}}{\isacharcomma}{\isasymalpha}\isactrlsub {\isadigit{2}}{\isacharbraceright}\ {\isasymunion}\ W} pertenece a \isa{C}.
     \item Para toda fórmula de tipo \isa{{\isasymbeta}} con componentes \isa{{\isasymbeta}\isactrlsub {\isadigit{1}}} y \isa{{\isasymbeta}\isactrlsub {\isadigit{2}}} tal que \isa{{\isasymbeta}}
      pertenece a \isa{W}, se tiene que o bien \isa{{\isacharbraceleft}{\isasymbeta}\isactrlsub {\isadigit{1}}{\isacharbraceright}\ {\isasymunion}\ W} pertenece a \isa{C} o 
      bien \isa{{\isacharbraceleft}{\isasymbeta}\isactrlsub {\isadigit{2}}{\isacharbraceright}\ {\isasymunion}\ W} pertenece a \isa{C}.
    \end{itemize}

    De este modo, consideremos un conjunto cualquiera \isa{W\ {\isasymin}\ C} y procedamos a probar cada una de las
    condiciones anteriores.

    La primera condición se demuestra por reducción al absurdo. En efecto, si suponemos que 
    \isa{{\isasymbottom}\ {\isasymin}\ W}, es claro que \isa{{\isacharbraceleft}{\isasymbottom}{\isacharbraceright}} es un subconjunto finito de \isa{W}. Como \isa{W} es un conjunto
    finitamente satisfacible por pertenecer a \isa{C}, se tiene por lo anterior que \isa{{\isacharbraceleft}{\isasymbottom}{\isacharbraceright}} es 
    satisfacible. De este modo, llegamos a una contradicción pues, por definición, no existe ninguna 
    interpretación que sea modelo de \isa{{\isasymbottom}}.

    A continuación probaremos que, si \isa{W\ {\isasymin}\ C}, entonces dada \isa{p} una fórmula atómica cualquiera, no 
    se tiene simultáneamente que \isa{p\ {\isasymin}\ W} y \isa{{\isasymnot}\ p\ {\isasymin}\ W}. Veamos dicho resultado por reducción al 
    absurdo, suponiendo que tanto \isa{p} como \isa{{\isasymnot}\ p} están en \isa{W}. En este caso, \isa{{\isacharbraceleft}p{\isacharcomma}{\isasymnot}\ p{\isacharbraceright}} sería un
    subconjunto finito de \isa{W} y, por ser \isa{W} finitamente satisfacible ya que\\ \isa{W\ {\isasymin}\ C}, obtendríamos 
    que \isa{{\isacharbraceleft}p{\isacharcomma}{\isasymnot}\ p{\isacharbraceright}} es satisfacible. Sin embargo esto no es cierto ya que, en ese caso, existiría
    una interpretación que sería modelo tanto de \isa{p} como de \isa{{\isasymnot}\ p}, llegando así a una 
    contradicción.

    Probemos ahora que dada una fórmula \isa{F} de tipo \isa{{\isasymalpha}} con componentes \isa{{\isasymalpha}\isactrlsub {\isadigit{1}}} y \isa{{\isasymalpha}\isactrlsub {\isadigit{2}}} tal que \isa{F\ {\isasymin}\ W},
    se tiene que \isa{{\isacharbraceleft}{\isasymalpha}\isactrlsub {\isadigit{1}}{\isacharcomma}{\isasymalpha}\isactrlsub {\isadigit{2}}{\isacharbraceright}\ {\isasymunion}\ W} pertenece a \isa{C}. Por definición de la colección, basta probar que 
    \isa{{\isacharbraceleft}{\isasymalpha}\isactrlsub {\isadigit{1}}{\isacharcomma}{\isasymalpha}\isactrlsub {\isadigit{2}}{\isacharbraceright}\ {\isasymunion}\ W} es finitamente satisfacible, es decir, que todo subconjunto finito suyo es
    satisfacible. Consideremos un subconjunto finito \isa{S} de \isa{{\isacharbraceleft}{\isasymalpha}\isactrlsub {\isadigit{1}}{\isacharcomma}{\isasymalpha}\isactrlsub {\isadigit{2}}{\isacharbraceright}\ {\isasymunion}\ W}. En estas condiciones,
    por el lema \isa{{\isadigit{4}}{\isachardot}{\isadigit{3}}{\isachardot}{\isadigit{3}}}, existe un subconjunto finito \isa{W\isactrlsub {\isadigit{0}}} de \isa{W} tal que\\ \isa{S\ {\isacharequal}\ {\isacharbraceleft}{\isasymalpha}\isactrlsub {\isadigit{1}}{\isacharcomma}{\isasymalpha}\isactrlsub {\isadigit{2}}{\isacharbraceright}\ {\isasymunion}\ W\isactrlsub {\isadigit{0}}},
    \isa{S\ {\isacharequal}\ {\isacharbraceleft}{\isasymalpha}\isactrlsub {\isadigit{1}}{\isacharbraceright}\ {\isasymunion}\ W\isactrlsub {\isadigit{0}}}, \isa{S\ {\isacharequal}\ {\isacharbraceleft}{\isasymalpha}\isactrlsub {\isadigit{2}}{\isacharbraceright}\ {\isasymunion}\ W\isactrlsub {\isadigit{0}}} o \isa{S\ {\isacharequal}\ W\isactrlsub {\isadigit{0}}}. Para probar que \isa{S} es satisfacible en cada uno de 
    estos posibles casos, basta demostrar que el conjunto\\ \isa{{\isacharbraceleft}{\isasymalpha}\isactrlsub {\isadigit{1}}{\isacharcomma}{\isasymalpha}\isactrlsub {\isadigit{2}}{\isacharcomma}F{\isacharbraceright}\ {\isasymunion}\ W\isactrlsub {\isadigit{0}}} es satisfacible. De este
    modo, puesto que todas las opciones posibles de \isa{S} están contenidas en dicho conjunto, se
    tiene la satisfacibilidad de cada una de ellas.

    Para probar que el conjunto \isa{{\isacharbraceleft}{\isasymalpha}\isactrlsub {\isadigit{1}}{\isacharcomma}{\isasymalpha}\isactrlsub {\isadigit{2}}{\isacharcomma}F{\isacharbraceright}\ {\isasymunion}\ W\isactrlsub {\isadigit{0}}} es satisfacible en estas condiciones, demostremos 
    que se verifica para cada caso de la fórmula \isa{F} de tipo \isa{{\isasymalpha}}:

      $\textbf{\isa{{\isasymone}{\isacharparenright}\ F\ {\isacharequal}\ G\ {\isasymand}\ H{\isacharcomma}\ para\ ciertas\ fórmulas\ G\ y\ H}:}$ Observemos que, como \isa{W\isactrlsub {\isadigit{0}}} es un subconjunto 
      finito de \isa{W} y \isa{F\ {\isasymin}\ W} por hipótesis, tenemos que \isa{{\isacharbraceleft}F{\isacharbraceright}\ {\isasymunion}\ W\isactrlsub {\isadigit{0}}} es un subconjunto finito de \isa{W}. 
      Como \isa{W} es finitamente satisfacible ya que pertenece a \isa{C}, se tiene que \isa{{\isacharbraceleft}F{\isacharbraceright}\ {\isasymunion}\ W\isactrlsub {\isadigit{0}}} es 
      satisfacible. Luego, por definición, existe una interpretación \isa{{\isasymA}} que es modelo de todas sus 
      fórmulas y, en particular, \isa{{\isasymA}} es modelo de \isa{F}. Como \isa{F\ {\isacharequal}\ G\ {\isasymand}\ H}, obtenemos por definición 
      del valor de una fórmula en una interpretación que \isa{{\isasymA}} es modelo de \isa{G} y de \isa{H}. En este caso,
      las componentes conjuntivas son \isa{{\isasymalpha}\isactrlsub {\isadigit{1}}\ {\isacharequal}\ G} y \isa{{\isasymalpha}\isactrlsub {\isadigit{2}}\ {\isacharequal}\ H}, luego \isa{{\isasymA}} es modelo de ambas componentes.
      Por lo tanto, \isa{{\isasymA}} es modelo de todas las fórmulas del conjunto \isa{{\isacharbraceleft}{\isasymalpha}\isactrlsub {\isadigit{1}}{\isacharcomma}{\isasymalpha}\isactrlsub {\isadigit{2}}{\isacharcomma}F{\isacharbraceright}\ {\isasymunion}\ W\isactrlsub {\isadigit{0}}}, lo que prueba 
      que se trata de un conjunto satisfacible.

      $\textbf{\isa{{\isasymtwo}{\isacharparenright}\ F\ {\isacharequal}\ {\isasymnot}{\isacharparenleft}G\ {\isasymor}\ H{\isacharparenright}{\isacharcomma}\ para\ ciertas\ fórmulas\ G\ y\ H}:}$ Análogamente al caso anterior, obtenemos 
      que el conjunto \isa{{\isacharbraceleft}F{\isacharbraceright}\ {\isasymunion}\ W\isactrlsub {\isadigit{0}}} es satisfacible. Luego, por definición, existe una interpretación 
      \isa{{\isasymA}} que es modelo de todas sus fórmulas y, en particular, de \isa{F}. Por definición del valor de 
      una fórmula en una interpretación, como \isa{F\ {\isacharequal}\ {\isasymnot}{\isacharparenleft}G\ {\isasymor}\ H{\isacharparenright}}, obtenemos que no es cierto que \isa{{\isasymA}} 
      sea modelo de \isa{G\ {\isasymor}\ H}. Aplicando de nuevo la definición del valor de una fórmula en una 
      interpretación, se obtiene que no es cierto que \isa{{\isasymA}} se modelo de \isa{G} o de \isa{H}. Por las leyes 
      de \isa{Morgan}, obtenemos equivalentemente que \isa{{\isasymA}} no es modelo de \isa{G} y \isa{{\isasymA}} no es modelo de \isa{H}. 
      Por lo tanto, por el valor de una fórmula en una interpretación, obtenemos que \isa{{\isasymA}} es 
      modelo de \isa{{\isasymnot}\ G} y \isa{{\isasymA}} es modelo de \isa{{\isasymnot}\ H}. Como las componentes conjuntivas en este caso son 
      \isa{{\isasymalpha}\isactrlsub {\isadigit{1}}\ {\isacharequal}\ {\isasymnot}\ G} y \isa{{\isasymalpha}\isactrlsub {\isadigit{2}}\ {\isacharequal}\ {\isasymnot}\ H}, es claro que \isa{{\isasymA}} es modelo de \isa{{\isasymalpha}\isactrlsub {\isadigit{1}}} y de \isa{{\isasymalpha}\isactrlsub {\isadigit{2}}}. Por lo tanto, la 
      interpretación \isa{{\isasymA}} es modelo de todas las fórmulas del conjunto \isa{{\isacharbraceleft}{\isasymalpha}\isactrlsub {\isadigit{1}}{\isacharcomma}{\isasymalpha}\isactrlsub {\isadigit{2}}{\isacharcomma}F{\isacharbraceright}\ {\isasymunion}\ W\isactrlsub {\isadigit{0}}}, lo que 
      prueba por definición que se trata de un conjunto satisfacible. 

      $\textbf{\isa{{\isasymthree}{\isacharparenright}\ F\ {\isacharequal}\ {\isasymnot}{\isacharparenleft}G\ {\isasymlongrightarrow}\ H{\isacharparenright}{\isacharcomma}\ para\ ciertas\ fórmulas\ G\ y\ H}:}$ Como hemos visto que \isa{{\isacharbraceleft}F{\isacharbraceright}\ {\isasymunion}\ W\isactrlsub {\isadigit{0}}} 
      es un conjunto satisfacible, existe una interpretación \isa{{\isasymA}} que es modelo de todas sus 
      fórmulas. En particular, \isa{{\isasymA}} es modelo de \isa{F} luego, por definición del valor de una fórmula 
      en una interpretación, es claro que \isa{{\isasymA}} no es modelo de \isa{G\ {\isasymlongrightarrow}\ H}. De nuevo por el valor de 
      una fórmula en una interpretación, obtenemos que no es cierto que si \isa{{\isasymA}} es modelo de \isa{G}, 
      entonces sea modelo de \isa{H}. Equivalentemente, \isa{{\isasymA}} es modelo de \isa{G} y no es modelo de \isa{H}. Por 
      lo tanto, por la definición del valor de una fórmula en una interpretación, se obtiene que 
      \isa{{\isasymA}} es modelo de \isa{G} y de \isa{{\isasymnot}\ H}. Como en este caso las componentes conjuntivas son \isa{{\isasymalpha}\isactrlsub {\isadigit{1}}\ {\isacharequal}\ G} y
      \isa{{\isasymalpha}\isactrlsub {\isadigit{2}}\ {\isacharequal}\ {\isasymnot}\ H}, es claro que \isa{{\isasymA}} es modelo de \isa{{\isasymalpha}\isactrlsub {\isadigit{1}}} y de \isa{{\isasymalpha}\isactrlsub {\isadigit{2}}}. Por lo tanto, \isa{{\isasymA}} es modelo de 
      todas las fórmulas del conjunto  \isa{{\isacharbraceleft}{\isasymalpha}\isactrlsub {\isadigit{1}}{\isacharcomma}{\isasymalpha}\isactrlsub {\isadigit{2}}{\isacharcomma}F{\isacharbraceright}\ {\isasymunion}\ W\isactrlsub {\isadigit{0}}}, probando así su satisfacibilidad.

      $\textbf{\isa{{\isasymfour}{\isacharparenright}\ F\ {\isacharequal}\ {\isasymnot}{\isacharparenleft}{\isasymnot}\ G{\isacharparenright}{\isacharcomma}\ para\ cierta\ fórmula\ G}:}$ Análogamente a los casos anteriores, se 
      prueba que existe una interpretación \isa{{\isasymA}} que es modelo de todas las fórmulas del conjunto\\ 
      \isa{{\isacharbraceleft}F{\isacharbraceright}\ {\isasymunion}\ W\isactrlsub {\isadigit{0}}} por ser este satisfacible. En particular, \isa{{\isasymA}} es modelo de \isa{F} luego, por 
      definición del valor de una fórmula en una interpretación, obtenemos que no es cierto que \isa{{\isasymA}} 
      no es modelo de \isa{G}. Es decir, \isa{{\isasymA}} es modelo de \isa{G} y, como en este caso ambas componentes 
      disyuntivas son \isa{G}, es claro que \isa{{\isasymA}} es modelo de \isa{{\isasymalpha}\isactrlsub {\isadigit{1}}} y de \isa{{\isasymalpha}\isactrlsub {\isadigit{2}}}. Por tanto, \isa{{\isasymA}} es modelo 
      de todas las fórmulas del conjunto \isa{{\isacharbraceleft}{\isasymalpha}\isactrlsub {\isadigit{1}}{\isacharcomma}{\isasymalpha}\isactrlsub {\isadigit{2}}{\isacharcomma}F{\isacharbraceright}\ {\isasymunion}\ W\isactrlsub {\isadigit{0}}}, lo que prueba su satisfacibilidad.

    Por lo tanto, \isa{{\isacharbraceleft}{\isasymalpha}\isactrlsub {\isadigit{1}}{\isacharcomma}{\isasymalpha}\isactrlsub {\isadigit{2}}{\isacharcomma}F{\isacharbraceright}\ {\isasymunion}\ W\isactrlsub {\isadigit{0}}} es un conjunto satisfacible para todos los casos de la fórmula
    \isa{F} de tipo \isa{{\isasymalpha}}. De este modo, como el subconjunto finito \isa{S} de \isa{{\isacharbraceleft}{\isasymalpha}\isactrlsub {\isadigit{1}}{\isacharcomma}{\isasymalpha}\isactrlsub {\isadigit{2}}{\isacharbraceright}\ {\isasymunion}\ W} es de la forma
    \isa{S\ {\isacharequal}\ {\isacharbraceleft}{\isasymalpha}\isactrlsub {\isadigit{1}}{\isacharcomma}{\isasymalpha}\isactrlsub {\isadigit{2}}{\isacharbraceright}\ {\isasymunion}\ W\isactrlsub {\isadigit{0}}}, \isa{S\ {\isacharequal}\ {\isacharbraceleft}{\isasymalpha}\isactrlsub {\isadigit{1}}{\isacharbraceright}\ {\isasymunion}\ W\isactrlsub {\isadigit{0}}}, \isa{S\ {\isacharequal}\ {\isacharbraceleft}{\isasymalpha}\isactrlsub {\isadigit{2}}{\isacharbraceright}\ {\isasymunion}\ W\isactrlsub {\isadigit{0}}} o \isa{S\ {\isacharequal}\ W\isactrlsub {\isadigit{0}}}, se prueba la satisfacibilidad
    de \isa{S} para cada uno de los casos por estar contenidos en el conjunto satisfacible
    \isa{{\isacharbraceleft}{\isasymalpha}\isactrlsub {\isadigit{1}}{\isacharcomma}{\isasymalpha}\isactrlsub {\isadigit{2}}{\isacharcomma}F{\isacharbraceright}\ {\isasymunion}\ W\isactrlsub {\isadigit{0}}}. Por lo tanto, \isa{{\isacharbraceleft}{\isasymalpha}\isactrlsub {\isadigit{1}}{\isacharcomma}{\isasymalpha}\isactrlsub {\isadigit{2}}{\isacharbraceright}\ {\isasymunion}\ W} es finitamente satisfacible y, por definición de 
    la colección \isa{C}, pertenece a ella como queríamos demostrar.

    Finalmente probemos que para toda fórmula \isa{F} de tipo \isa{{\isasymbeta}} con componentes \isa{{\isasymbeta}\isactrlsub {\isadigit{1}}} y \isa{{\isasymbeta}\isactrlsub {\isadigit{2}}} tal que 
    \isa{F\ {\isasymin}\ W}, se tiene que o bien \isa{{\isacharbraceleft}{\isasymbeta}\isactrlsub {\isadigit{1}}{\isacharbraceright}\ {\isasymunion}\ W\ {\isasymin}\ C} o bien \isa{{\isacharbraceleft}{\isasymbeta}\isactrlsub {\isadigit{2}}{\isacharbraceright}\ {\isasymunion}\ W\ {\isasymin}\ C}. La
    demostración se realizará por reducción al absurdo, luego supongamos en estas condiciones que\\
    \isa{{\isacharbraceleft}{\isasymbeta}\isactrlsub {\isadigit{1}}{\isacharbraceright}\ {\isasymunion}\ W\ {\isasymnotin}\ C} y \isa{{\isacharbraceleft}{\isasymbeta}\isactrlsub {\isadigit{2}}{\isacharbraceright}\ {\isasymunion}\ W\ {\isasymnotin}\ C}. 

    En primer lugar, veamos que si \isa{{\isacharbraceleft}{\isasymbeta}\isactrlsub i{\isacharbraceright}\ {\isasymunion}\ W\ {\isasymnotin}\ C}, entonces existe un subconjunto finito \isa{W\isactrlsub i} de 
    \isa{W} tal que el conjunto \isa{{\isacharbraceleft}{\isasymbeta}\isactrlsub i{\isacharbraceright}\ {\isasymunion}\ W\isactrlsub i} no es satisfacible. En efecto, si \isa{{\isacharbraceleft}{\isasymbeta}\isactrlsub i{\isacharbraceright}\ {\isasymunion}\ W\ {\isasymnotin}\ C}, por 
    definición de la colección \isa{C} tenemos que \isa{{\isacharbraceleft}{\isasymbeta}\isactrlsub i{\isacharbraceright}\ {\isasymunion}\ W} no es finitamente satisfacible. Por lo 
    tanto, existe un subconjunto finito \isa{W\isactrlsub i{\isacharprime}} de \isa{{\isacharbraceleft}{\isasymbeta}\isactrlsub i{\isacharbraceright}\ {\isasymunion}\ W} que no es satisfacible. Por el lema 
    \isa{{\isadigit{4}}{\isachardot}{\isadigit{3}}{\isachardot}{\isadigit{2}}}, obtenemos que existe un subconjunto finito \isa{W\isactrlsub i} de \isa{W} tal que o bien\\ \isa{W\isactrlsub i{\isacharprime}\ {\isacharequal}\ {\isacharbraceleft}{\isasymbeta}\isactrlsub i{\isacharbraceright}\ {\isasymunion}\ W\isactrlsub i} 
    o bien \isa{W\isactrlsub i{\isacharprime}\ {\isacharequal}\ W\isactrlsub i}. En efecto, si \isa{W\isactrlsub i{\isacharprime}\ {\isacharequal}\ {\isacharbraceleft}{\isasymbeta}\isactrlsub i{\isacharbraceright}\ {\isasymunion}\ W\isactrlsub i}, como \isa{W\isactrlsub i{\isacharprime}} no es satisfacible, se obtiene el
    resultado para \isa{W\isactrlsub i}. Por otro lado, supongamos que\\ \isa{W\isactrlsub i{\isacharprime}\ {\isacharequal}\ W\isactrlsub i}. Como \isa{W\isactrlsub i{\isacharprime}} no es satisfacible, 
    entonces \isa{{\isacharbraceleft}{\isasymbeta}\isactrlsub i{\isacharbraceright}\ {\isasymunion}\ W\isactrlsub i} tampoco es satisfacible ya que, en caso contrario, obtendríamos que
    \isa{W\isactrlsub i\ {\isacharequal}\ W\isactrlsub i{\isacharprime}} es satisfacible. Luego se verifica también el resultado para \isa{W\isactrlsub i}.

    De este modo, como \isa{{\isacharbraceleft}{\isasymbeta}\isactrlsub {\isadigit{1}}{\isacharbraceright}\ {\isasymunion}\ W\ {\isasymnotin}\ C} y \isa{{\isacharbraceleft}{\isasymbeta}\isactrlsub {\isadigit{2}}{\isacharbraceright}\ {\isasymunion}\ W\ {\isasymnotin}\ C}, obtenemos que existen subconjuntos finitos 
    \isa{W\isactrlsub {\isadigit{1}}} y \isa{W\isactrlsub {\isadigit{2}}} de \isa{W} tales que los conjunto \isa{{\isacharbraceleft}{\isasymbeta}\isactrlsub {\isadigit{1}}{\isacharbraceright}\ {\isasymunion}\ W\isactrlsub {\isadigit{1}}} y \isa{{\isacharbraceleft}{\isasymbeta}\isactrlsub {\isadigit{2}}{\isacharbraceright}\ {\isasymunion}\ W\isactrlsub {\isadigit{2}}} no son satisfacibles. 
    Consideremos el conjunto \isa{W\isactrlsub o\ {\isacharequal}\ W\isactrlsub {\isadigit{1}}\ {\isasymunion}\ W\isactrlsub {\isadigit{2}}}. Es claro que se tiene que\\ \isa{{\isacharbraceleft}{\isasymbeta}\isactrlsub {\isadigit{1}}{\isacharbraceright}\ {\isasymunion}\ W\isactrlsub {\isadigit{1}}\ {\isasymsubseteq}\ {\isacharbraceleft}{\isasymbeta}\isactrlsub {\isadigit{1}}{\isacharcomma}F{\isacharbraceright}\ {\isasymunion}\ W\isactrlsub o} y 
    que \isa{{\isacharbraceleft}{\isasymbeta}\isactrlsub {\isadigit{2}}{\isacharbraceright}\ {\isasymunion}\ W\isactrlsub {\isadigit{2}}\ {\isasymsubseteq}\ {\isacharbraceleft}{\isasymbeta}\isactrlsub {\isadigit{2}}{\isacharcomma}F{\isacharbraceright}\ {\isasymunion}\ W\isactrlsub {\isadigit{0}}}. Por lo tanto, los conjuntos \isa{{\isacharbraceleft}{\isasymbeta}\isactrlsub {\isadigit{1}}{\isacharcomma}F{\isacharbraceright}\ {\isasymunion}\ W\isactrlsub o} y \isa{{\isacharbraceleft}{\isasymbeta}\isactrlsub {\isadigit{2}}{\isacharcomma}F{\isacharbraceright}\ {\isasymunion}\ W\isactrlsub o} no son 
    satisfacibles ya que, en caso contrario, \isa{{\isacharbraceleft}{\isasymbeta}\isactrlsub {\isadigit{1}}{\isacharbraceright}\ {\isasymunion}\ W\isactrlsub {\isadigit{1}}} y \isa{{\isacharbraceleft}{\isasymbeta}\isactrlsub {\isadigit{2}}{\isacharbraceright}\ {\isasymunion}\ W\isactrlsub {\isadigit{2}}} serían satisfacibles. Para 
    llegar a la contradicción, basta probar que o bien \isa{{\isacharbraceleft}{\isasymbeta}\isactrlsub {\isadigit{1}}{\isacharcomma}F{\isacharbraceright}\ {\isasymunion}\ W\isactrlsub o} es satisfacible o bien 
    \isa{{\isacharbraceleft}{\isasymbeta}\isactrlsub {\isadigit{2}}{\isacharcomma}F{\isacharbraceright}\ {\isasymunion}\ W\isactrlsub o} es satisfacible. Para ello, veamos que se verifica el resultado para cada uno de 
    los casos posibles fórmula de tipo \isa{{\isasymbeta}} para \isa{F}.

      $\textbf{\isa{{\isasymone}{\isacharparenright}\ F\ {\isacharequal}\ G\ {\isasymor}\ H{\isacharcomma}\ para\ ciertas\ fórmulas\ G\ y\ H}:}$ Observemos que \isa{W\isactrlsub {\isadigit{0}}\ {\isacharequal}\ W\isactrlsub {\isadigit{1}}\ {\isasymunion}\ W\isactrlsub {\isadigit{2}}} es un 
      subconjunto finito de \isa{W} por ser \isa{W\isactrlsub {\isadigit{1}}} y \isa{W\isactrlsub {\isadigit{2}}} subconjuntos finitos de \isa{W}. Además, como 
      \isa{F\ {\isasymin}\ W} por hipótesis, tenemos que \isa{{\isacharbraceleft}F{\isacharbraceright}\ {\isasymunion}\ W\isactrlsub {\isadigit{0}}} es un subconjunto finito de \isa{W}. Como \isa{W} es 
      finitamente satisfacible ya que pertenece a \isa{C}, se tiene que \isa{{\isacharbraceleft}F{\isacharbraceright}\ {\isasymunion}\ W\isactrlsub {\isadigit{0}}} es satisfacible. 
      Luego, por definición, existe una interpretación \isa{{\isasymA}} que es modelo de todas sus fórmulas y, 
      en particular, \isa{{\isasymA}} es modelo de \isa{F}. Por definición del valor de una fórmula en una
      interpretación, obtenemos que o bien \isa{{\isasymA}} es modelo de \isa{G} o bien \isa{{\isasymA}} es modelo de \isa{H}.
      Como en este caso las componentes disyuntivas son \isa{{\isasymbeta}\isactrlsub {\isadigit{1}}\ {\isacharequal}\ G} y \isa{{\isasymbeta}\isactrlsub {\isadigit{2}}\ {\isacharequal}\ H}, se tiene que o bien \isa{{\isasymA}}
      es modelo de \isa{{\isasymbeta}\isactrlsub {\isadigit{1}}} o bien \isa{{\isasymA}} es modelo de \isa{{\isasymbeta}\isactrlsub {\isadigit{2}}}. Por lo tanto, es claro que o bien \isa{{\isasymA}} es
      modelo de todas las fórmulas del conjunto \isa{{\isacharbraceleft}{\isasymbeta}\isactrlsub {\isadigit{1}}{\isacharcomma}F{\isacharbraceright}\ {\isasymunion}\ W\isactrlsub {\isadigit{0}}} o bien es modelo de todas las fórmulas
      de \isa{{\isacharbraceleft}{\isasymbeta}\isactrlsub {\isadigit{2}}{\isacharcomma}F{\isacharbraceright}\ {\isasymunion}\ W\isactrlsub {\isadigit{0}}}. Luego, por definición de conjunto satisfacible tenemos que o bien\\ 
      \isa{{\isacharbraceleft}{\isasymbeta}\isactrlsub {\isadigit{1}}{\isacharcomma}F{\isacharbraceright}\ {\isasymunion}\ W\isactrlsub {\isadigit{0}}} es satisfacible o bien \isa{{\isacharbraceleft}{\isasymbeta}\isactrlsub {\isadigit{2}}{\isacharcomma}F{\isacharbraceright}\ {\isasymunion}\ W\isactrlsub {\isadigit{0}}} es satisfacible, como queríamos demostrar.

      $\textbf{\isa{{\isasymtwo}{\isacharparenright}\ F\ {\isacharequal}\ G\ {\isasymlongrightarrow}\ H{\isacharcomma}\ para\ ciertas\ fórmulas\ G\ y\ H}:}$ Análogamente se tiene que el 
      conjunto \isa{{\isacharbraceleft}F{\isacharbraceright}\ {\isasymunion}\ W\isactrlsub {\isadigit{0}}} es satisfacible, luego existe una interpretación \isa{{\isasymA}} que es modelo de 
      todas sus fórmulas. En particular, \isa{{\isasymA}} es modelo de \isa{F} y, por definición del valor de una 
      fórmula en una interpretación, se obtiene que si \isa{{\isasymA}} es modelo de \isa{G}, entonces es modelo de 
      \isa{H}. Equivalentemente, tenemos que \isa{{\isasymA}} no es modelo de \isa{G} o \isa{{\isasymA}} es modelo de \isa{H}. Por un 
      lado, si suponemos que \isa{{\isasymA}} no es modelo de \isa{G}, por definición obtenemos que \isa{{\isasymA}} es modelo de 
      \isa{{\isasymnot}\ G}. Como en este caso tenemos que \isa{{\isasymbeta}\isactrlsub {\isadigit{1}}\ {\isacharequal}\ {\isasymnot}\ G}, es claro que \isa{{\isasymA}} es modelo de \isa{{\isasymbeta}\isactrlsub {\isadigit{1}}}. Por 
      tanto, es modelo de todas las fórmulas de \isa{{\isacharbraceleft}{\isasymbeta}\isactrlsub {\isadigit{1}}{\isacharcomma}F{\isacharbraceright}\ {\isasymunion}\ W\isactrlsub {\isadigit{0}}}, luego es un conjunto satisfacible y 
      se verifica el resultado para este caso. Por otro lado, si suponemos que \isa{{\isasymA}} es modelo de \isa{H}, 
      como \isa{{\isasymbeta}\isactrlsub {\isadigit{2}}\ {\isacharequal}\ H}, obtenemos que \isa{{\isasymA}} es modelo de \isa{{\isasymbeta}\isactrlsub {\isadigit{2}}}. Luego, análogamente, \isa{{\isasymA}} es modelo de toda
      fórmula de \isa{{\isacharbraceleft}{\isasymbeta}\isactrlsub {\isadigit{2}}{\isacharcomma}F{\isacharbraceright}\ {\isasymunion}\ W\isactrlsub {\isadigit{0}}}, lo que prueba que se trata de un conjunto satisfacible por
      definición, probando el resultado. 

      $\textbf{\isa{{\isasymthree}{\isacharparenright}\ F\ {\isacharequal}\ {\isasymnot}{\isacharparenleft}G\ {\isasymand}\ H{\isacharparenright}{\isacharcomma}\ para\ ciertas\ fórmulas\ G\ y\ H}:}$ Como \isa{{\isacharbraceleft}F{\isacharbraceright}\ {\isasymunion}\ W\isactrlsub {\isadigit{0}}} es satisfacible,
      existe una interpretación \isa{{\isasymA}} que es modelo de todas sus fórmulas y, en particular, de \isa{F}.
      Luego, por definición del valor de una fórmula en una interpretación, obtenemos que \isa{{\isasymA}} no
      es modelo de \isa{G\ {\isasymand}\ H}. De nuevo por definición, esto implica que no es cierto que \isa{{\isasymA}} sea 
      modelo de \isa{G} y de \isa{H}. Es decir, o bien \isa{{\isasymA}} no es modelo de \isa{G} o bien \isa{{\isasymA}} no es modelo de
      \isa{H}. Si suponemos que no es modelo de \isa{G}, por definición se obtiene que \isa{{\isasymA}} es modelo de\\
      \isa{{\isasymnot}\ G}. Como en este caso la componente disyuntiva \isa{{\isasymbeta}\isactrlsub {\isadigit{1}}} es \isa{{\isasymnot}\ G}, se deduce que \isa{{\isasymA}} es modelo
      de \isa{{\isasymbeta}\isactrlsub {\isadigit{1}}}. Por tanto, \isa{{\isasymA}} es modelo de todas las fórmulas del conjunto \isa{{\isacharbraceleft}{\isasymbeta}\isactrlsub {\isadigit{1}}{\isacharcomma}F{\isacharbraceright}\ {\isasymunion}\ W\isactrlsub {\isadigit{0}}}, por lo que
      se demuestra que dicho conjunto es satisfacible, probando el resultado. Por otro lado, si
      suponemos que \isa{{\isasymA}} no es modelo de \isa{H}, se tiene que sí lo es de \isa{{\isasymnot}\ H}. Como \isa{{\isasymbeta}\isactrlsub {\isadigit{2}}} es \isa{{\isasymnot}\ H}
      en este caso, obtenemos que \isa{{\isasymA}} es modelo de \isa{{\isasymbeta}\isactrlsub {\isadigit{2}}}. Luego \isa{{\isasymA}} es modelo de todas las fórmulas
      de \isa{{\isacharbraceleft}{\isasymbeta}\isactrlsub {\isadigit{2}}{\isacharcomma}F{\isacharbraceright}\ {\isasymunion}\ W\isactrlsub {\isadigit{0}}}, demostrando así que es un conjunto satisfacible. Por tanto, se demuestra
      el resultado en ambos casos.

      $\textbf{\isa{{\isasymfour}{\isacharparenright}\ F\ {\isacharequal}\ {\isasymnot}{\isacharparenleft}{\isasymnot}\ G{\isacharparenright}{\isacharcomma}\ para\ cierta\ fórmula\ G}:}$ Puesto que \isa{{\isacharbraceleft}F{\isacharbraceright}\ {\isasymunion}\ W\isactrlsub {\isadigit{0}}} es satisfacible, 
      existe una interpretación \isa{{\isasymA}} modelo de todas sus fórmulas y, en particular, modelo de \isa{F}. 
      Luego, por definición del valor de una fórmula en una interpretación obtenemos que no es 
      cierto que \isa{{\isasymA}} no sea modelo de \isa{G}, es decir, \isa{{\isasymA}} es modelo de \isa{G}. Como las componentes 
      \isa{{\isasymbeta}\isactrlsub {\isadigit{1}}} y \isa{{\isasymbeta}\isactrlsub {\isadigit{2}}} son ambas \isa{G} en este caso, se obtiene que \isa{{\isasymA}} es modelo suyo. En particular, lo 
      es de \isa{{\isasymbeta}\isactrlsub {\isadigit{1}}}, de modo que \isa{{\isasymA}} es modelo de todas las fórmulas de \isa{{\isacharbraceleft}{\isasymbeta}\isactrlsub {\isadigit{1}}{\isacharcomma}F{\isacharbraceright}\ {\isasymunion}\ W\isactrlsub {\isadigit{0}}}, probando así que 
      es satisfacible. Por lo tanto, se verifica el resultado.
    
    En conclusión, hemos probado que o bien \isa{{\isacharbraceleft}{\isasymbeta}\isactrlsub {\isadigit{1}}{\isacharcomma}F{\isacharbraceright}\ {\isasymunion}\ W\isactrlsub o} es satisfacible o bien\\ \isa{{\isacharbraceleft}{\isasymbeta}\isactrlsub {\isadigit{2}}{\isacharcomma}F{\isacharbraceright}\ {\isasymunion}\ W\isactrlsub o} es 
    satisfacible. Por lo tanto, se tiene que no es cierto que ninguno de los dos conjuntos sea
    insatisfacible. Esto contradice lo demostrado anteriormente, llegando así a una contradicción
    que prueba por reducción al absurdo la última condición del lema \isa{{\isadigit{2}}{\isachardot}{\isadigit{0}}{\isachardot}{\isadigit{2}}}. De este modo, queda
    probado que la colección formada por los conjuntos de fórmulas finitamente satisfacibles 
    verifica la propiedad de consistencia proposicional y, por el \isa{Teorema\ de\ Existencia\ de\ Modelo}, 
    todo conjunto perteneciente a ella es satisfacible, lo que demuestra el teorema.
  \end{demostracion}

  Procedamos con la demostración detallada del \isa{Teorema\ de\ Compacidad} en Isabelle/HOL. Para ello, 
  definamos la colección de conjuntos finitamente satisfacibles en Isabelle/HOL. En adelante
  notaremos por \isa{C} a dicha colección.%
\end{isamarkuptext}\isamarkuptrue%
\isacommand{definition}\isamarkupfalse%
\ colecComp\ {\isacharcolon}{\isacharcolon}\ {\isachardoublequoteopen}{\isacharparenleft}{\isacharprime}a\ formula\ set{\isacharparenright}\ set{\isachardoublequoteclose}\isanewline
\ \ \isakeyword{where}\ colecComp{\isacharcolon}\ {\isachardoublequoteopen}colecComp\ {\isacharequal}\ {\isacharbraceleft}W{\isachardot}\ fin{\isacharunderscore}sat\ W{\isacharbraceright}{\isachardoublequoteclose}%
\begin{isamarkuptext}%
Para facilitar la demostración introduciremos el siguiente lema auxiliar que prueba que
  todo subconjunto finito de un conjunto perteneciente a la colección anterior es satisfacible.%
\end{isamarkuptext}\isamarkuptrue%
\isacommand{lemma}\isamarkupfalse%
\ colecComp{\isacharunderscore}subset{\isacharunderscore}finite{\isacharcolon}\ \isanewline
\ \ \isakeyword{assumes}\ {\isachardoublequoteopen}W\ {\isasymin}\ colecComp{\isachardoublequoteclose}\isanewline
\ \ \ \ \ \ \ \ \ \ {\isachardoublequoteopen}Wo\ {\isasymsubseteq}\ W{\isachardoublequoteclose}\isanewline
\ \ \ \ \ \ \ \ \ \ {\isachardoublequoteopen}finite\ Wo{\isachardoublequoteclose}\isanewline
\ \ \isakeyword{shows}\ {\isachardoublequoteopen}sat\ Wo{\isachardoublequoteclose}\ \isanewline
%
\isadelimproof
%
\endisadelimproof
%
\isatagproof
\isacommand{proof}\isamarkupfalse%
\ {\isacharminus}\isanewline
\ \ \isacommand{have}\isamarkupfalse%
\ {\isachardoublequoteopen}{\isasymforall}Wo\ {\isasymsubseteq}\ W{\isachardot}\ finite\ Wo\ {\isasymlongrightarrow}\ sat\ Wo{\isachardoublequoteclose}\isanewline
\ \ \ \ \isacommand{using}\isamarkupfalse%
\ assms{\isacharparenleft}{\isadigit{1}}{\isacharparenright}\ \isacommand{unfolding}\isamarkupfalse%
\ colecComp\ fin{\isacharunderscore}sat{\isacharunderscore}def\ \isacommand{by}\isamarkupfalse%
\ {\isacharparenleft}rule\ CollectD{\isacharparenright}\isanewline
\ \ \isacommand{then}\isamarkupfalse%
\ \isacommand{have}\isamarkupfalse%
\ {\isachardoublequoteopen}finite\ Wo\ {\isasymlongrightarrow}\ sat\ Wo{\isachardoublequoteclose}\isanewline
\ \ \ \ \isacommand{using}\isamarkupfalse%
\ {\isacartoucheopen}Wo\ {\isasymsubseteq}\ W{\isacartoucheclose}\ \isacommand{by}\isamarkupfalse%
\ {\isacharparenleft}rule\ sspec{\isacharparenright}\isanewline
\ \ \isacommand{thus}\isamarkupfalse%
\ {\isachardoublequoteopen}sat\ Wo{\isachardoublequoteclose}\isanewline
\ \ \ \ \isacommand{using}\isamarkupfalse%
\ {\isacartoucheopen}finite\ Wo{\isacartoucheclose}\ \isacommand{by}\isamarkupfalse%
\ {\isacharparenleft}rule\ mp{\isacharparenright}\isanewline
\isacommand{qed}\isamarkupfalse%
%
\endisatagproof
{\isafoldproof}%
%
\isadelimproof
%
\endisadelimproof
%
\begin{isamarkuptext}%
Para facilitar la comprensión de la demostración, mostraremos a continuación un grafo que 
  estructura las relaciones de necesidad de los lemas auxiliares empleados.

\comentario{Poner grafo bien.}

\begin{tikzpicture}
  [
    grow                    = down,
    level 1/.style          = {sibling distance=6cm},
    level 2/.style          = {sibling distance=4.5cm},
    level 3/.style          = {sibling distance=8cm}
    level 4/.style          = {sibling distance=4cm}
    level 5/.style          = {sibling distance=5cm}
    level 6/.style          = {sibling distance=5cm}
    level 7/.style          = {sibling distance=5cm};
    level distance          = 4.5cm,
    edge from parent/.style = {draw},
    every node/.style       = {font=\tiny},
    sloped
  ]
\raggedright
  \node [root] {\isa{prop{\isacharunderscore}Compactness}\\ \isa{{\isacharparenleft}Teorema\ de\ Compacidad\ {\isacharparenleft}{\isadigit{4}}{\isachardot}{\isadigit{3}}{\isachardot}{\isadigit{1}}{\isacharparenright}{\isacharparenright}}}
    child { node [env] {\isa{pcp{\isacharunderscore}colecComp}\\ \isa{{\isacharparenleft}C\ tiene\ la\ propiedad\ de\ consistencia\ proposicional{\isacharparenright}}}
          child { node [env] {\isa{pcp{\isacharunderscore}colecComp{\isacharunderscore}bot}\\ \isa{{\isacharparenleft}{\isasymbottom}\ {\isasymnotin}\ W{\isacharparenright}}}
              child { node [env] {\isa{not{\isacharunderscore}sat{\isacharunderscore}bot}\\ \isa{{\isacharparenleft}{\isacharbraceleft}{\isasymbottom}{\isacharbraceright}\ es\ insatisfacible{\isacharparenright}}}}}
          child { node [env] {\isa{pcp{\isacharunderscore}colecComp{\isacharunderscore}atoms}\\ \isa{{\isacharparenleft}Cond{\isachardot}\ fórmulas\ atómicas{\isacharparenright}}}
              child { node [env] {\isa{not{\isacharunderscore}sat{\isacharunderscore}atoms}\\ \isa{{\isacharparenleft}{\isacharbraceleft}p{\isacharcomma}{\isasymnot}\ p{\isacharbraceright}\ es\ insatisfacible{\isacharparenright}}}}}
      		child { node [env] {\isa{pcp{\isacharunderscore}colecComp{\isacharunderscore}CON}\\ \isa{{\isacharparenleft}Cond{\isachardot}\ fórmulas\ de\ tipo\ {\isasymalpha}{\isacharparenright}}}
        			child { node [env] {\isa{pcp{\isacharunderscore}colecComp{\isacharunderscore}CON{\isacharunderscore}sat}\\ \isa{{\isacharparenleft}Resultado\ {\isasymone}{\isacharparenright}}}
                      child { node [env] {\isa{pcp{\isacharunderscore}colecComp{\isacharunderscore}CON{\isacharunderscore}sat{\isadigit{1}}}\\\isa{pcp{\isacharunderscore}colecComp{\isacharunderscore}CON{\isacharunderscore}sat{\isadigit{2}}}\\\isa{pcp{\isacharunderscore}colecComp{\isacharunderscore}CON{\isacharunderscore}sat{\isadigit{3}}}\\\isa{pcp{\isacharunderscore}colecComp{\isacharunderscore}CON{\isacharunderscore}sat{\isadigit{4}}}}}}}
        			child { node [env] {\isa{pcp{\isacharunderscore}colecComp{\isacharunderscore}DIS}\\ \isa{{\isacharparenleft}Cond{\isachardot}\ fórmulas\ de\ tipo\ {\isasymbeta}{\isacharparenright}}}
                      child { node [env] {\isa{not{\isacharunderscore}colecComp}\\ \isa{{\isacharparenleft}Resultado\ {\isasymtwo}{\isacharparenright}}}
                            child { node [env] {\isa{sat{\isacharunderscore}subset{\isacharunderscore}ccontr}\\ \isa{{\isacharparenleft}Los\ conjuntos\ que}\\ \isa{contienen\ algún}\\ \isa{subconjunto\ insatisfacible}\\ \isa{son\ insatisfacibles{\isacharparenright}}}}}
                                  child { node [env] {\isa{pcp{\isacharunderscore}colecComp{\isacharunderscore}DIS{\isacharunderscore}sat}\\ \isa{{\isacharparenleft}Resultado\ {\isasymthree}{\isacharparenright}}}
                                  child { node [env] {\isa{pcp{\isacharunderscore}colecComp{\isacharunderscore}DIS{\isacharunderscore}sat{\isadigit{1}}}\\\isa{pcp{\isacharunderscore}colecComp{\isacharunderscore}DIS{\isacharunderscore}sat{\isadigit{2}}}\\\isa{pcp{\isacharunderscore}colecComp{\isacharunderscore}DIS{\isacharunderscore}sat{\isadigit{3}}}\\\isa{pcp{\isacharunderscore}colecComp{\isacharunderscore}DIS{\isacharunderscore}sat{\isadigit{4}}}}}}}};
\end{tikzpicture}

  De este modo, el \isa{Teorema\ de\ Compacidad} se demuestra aplicando el \isa{Teorema\ de}\\ \isa{Existencia\ de\ Modelo} a la colección \isa{C}. Por tanto, basta probar que la colección \isa{C} verifica la propiedad de 
  consistencia proposicional (formalizado como \isa{pcp{\isacharunderscore}colecComp}), de modo que todo \isa{W\ {\isasymin}\ C} será
  satisfacible. Para ello, por el lema \isa{{\isadigit{2}}{\isachardot}{\isadigit{0}}{\isachardot}{\isadigit{2}}}, es suficiente probar las siguientes condiciones dado 
  un conjunto \isa{W\ {\isasymin}\ C} cualquiera:

    \begin{enumerate}
     \item \isa{{\isasymbottom}\ {\isasymnotin}\ W}. (\isa{{\isasymLongrightarrow}} formalizado como \isa{pcp{\isacharunderscore}colecComp{\isacharunderscore}sat})
     \item Dada \isa{p} una fórmula atómica cualquiera, no se tiene 
      simultáneamente que\\ \isa{p\ {\isasymin}\ W} y \isa{{\isasymnot}\ p\ {\isasymin}\ W}. (\isa{{\isasymLongrightarrow}} formalizado como \isa{pcp{\isacharunderscore}colecComp{\isacharunderscore}atoms})
     \item Para toda fórmula de tipo \isa{{\isasymalpha}} con componentes \isa{{\isasymalpha}\isactrlsub {\isadigit{1}}} y \isa{{\isasymalpha}\isactrlsub {\isadigit{2}}} tal que \isa{{\isasymalpha}}
      pertenece a \isa{W}, se tiene que \isa{{\isacharbraceleft}{\isasymalpha}\isactrlsub {\isadigit{1}}{\isacharcomma}{\isasymalpha}\isactrlsub {\isadigit{2}}{\isacharbraceright}\ {\isasymunion}\ W\ {\isasymin}\ C}. (\isa{{\isasymLongrightarrow}} formalizado como 
      \isa{pcp{\isacharunderscore}colecComp{\isacharunderscore}CON})
     \item Para toda fórmula de tipo \isa{{\isasymbeta}} con componentes \isa{{\isasymbeta}\isactrlsub {\isadigit{1}}} y \isa{{\isasymbeta}\isactrlsub {\isadigit{2}}} tal que \isa{{\isasymbeta}}
      pertenece a \isa{W}, se tiene que o bien \isa{{\isacharbraceleft}{\isasymbeta}\isactrlsub {\isadigit{1}}{\isacharbraceright}\ {\isasymunion}\ W\ {\isasymin}\ C} o 
      bien \isa{{\isacharbraceleft}{\isasymbeta}\isactrlsub {\isadigit{2}}{\isacharbraceright}\ {\isasymunion}\ W\ {\isasymin}\ C}.\\ (\isa{{\isasymLongrightarrow}} formalizado como \isa{pcp{\isacharunderscore}colecComp{\isacharunderscore}DIS})
    \end{enumerate}
  A su vez, cada uno de los lemas auxiliares que prueban las condiciones anteriores precisa de los
  siguientes lemas:

  \begin{itemize}
    \item \isa{pcp{\isacharunderscore}colecComp{\isacharunderscore}sat}: Se prueba por reducción al absurdo mediante el lema \isa{not{\isacharunderscore}sat{\isacharunderscore}bot} que
    demuestra la insatisfacibilidad del conjunto \isa{{\isacharbraceleft}{\isasymbottom}{\isacharbraceright}}.
    \item \isa{pcp{\isacharunderscore}colecComp{\isacharunderscore}atoms}: Su demostración es por reducción al absurdo empleando el lema
    \isa{not{\isacharunderscore}sat{\isacharunderscore}atoms} que prueba la insatisfacibilidad del conjunto \isa{{\isacharbraceleft}p{\isacharcomma}{\isasymnot}\ p{\isacharbraceright}} para cualquier fórmula
    atómica \isa{p}.
    \item \isa{pcp{\isacharunderscore}colecComp{\isacharunderscore}CON}: Para su prueba, se precisa del \isa{resultado\ {\isasymone}}, formalizado como 
    \isa{pcp{\isacharunderscore}colecComp{\isacharunderscore}CON{\isacharunderscore}sat}. Este demuestra que dados \isa{W\ {\isasymin}\ C}, \isa{F\ {\isasymin}\ W} una fórmula de tipo 
    \isa{{\isasymalpha}} con componentes \isa{{\isasymalpha}\isactrlsub {\isadigit{1}}} y \isa{{\isasymalpha}\isactrlsub {\isadigit{2}}} y \isa{W\isactrlsub {\isadigit{0}}} un subconjunto finito de \isa{W}, se verifica que 
    \isa{{\isacharbraceleft}{\isasymalpha}\isactrlsub {\isadigit{1}}{\isacharcomma}{\isasymalpha}\isactrlsub {\isadigit{2}}{\isacharcomma}F{\isacharbraceright}\ {\isasymunion}\ W\isactrlsub {\isadigit{0}}} es satisfacible. Para probar dicho resultado se emplean a su vez los lemas
    auxiliares \isa{pcp{\isacharunderscore}colecComp{\isacharunderscore}CON{\isacharunderscore}sat{\isadigit{1}}}, \isa{pcp{\isacharunderscore}colecComp{\isacharunderscore}CON{\isacharunderscore}sat{\isadigit{2}}}, \isa{pcp{\isacharunderscore}colecComp{\isacharunderscore}CON{\isacharunderscore}sat{\isadigit{3}}} y 
    \isa{pcp{\isacharunderscore}colecComp{\isacharunderscore}CON{\isacharunderscore}sat{\isadigit{4}}} que demuestran el enunciado para cada tipo de fórmula \isa{{\isasymalpha}}.
    \item \isa{pcp{\isacharunderscore}colecComp{\isacharunderscore}DIS}: La prueba se realizará por reducción al absurdo. Para ello
    precisa de dos resultados.
    \begin{itemize}
      \item \isa{Resultado\ {\isasymtwo}}: Dados \isa{W\ {\isasymin}\ C} y \isa{{\isasymbeta}\isactrlsub i} una fórmula cualquiera tal que\\ \isa{{\isacharbraceleft}{\isasymbeta}\isactrlsub i{\isacharbraceright}\ {\isasymunion}\ W\ {\isasymnotin}\ C}, 
      entonces existe un subconjunto finito \isa{W\isactrlsub i} de \isa{W} tal que el conjunto \isa{{\isacharbraceleft}{\isasymbeta}\isactrlsub i{\isacharbraceright}\ {\isasymunion}\ W\isactrlsub i} no es 
      satisfacible. En Isabelle ha sido formalizado como \isa{not{\isacharunderscore}colecComp}. A su vez, ha precisado
      para su prueba del lema auxiliar \isa{sat{\isacharunderscore}subset{\isacharunderscore}ccontr} que demuestra que todo conjunto de 
      fórmulas que tenga un subconjunto insatisfacible es también insatisfacible.
      \item \isa{Resultado\ {\isasymthree}}: Dados \isa{W\ {\isasymin}\ C}, \isa{F} una fórmula de tipo \isa{{\isasymbeta}} con componentes \isa{{\isasymbeta}\isactrlsub {\isadigit{1}}} y \isa{{\isasymbeta}\isactrlsub {\isadigit{2}}} 
      tal que \isa{F\ {\isasymin}\ W} y \isa{W\isactrlsub {\isadigit{0}}} un subconjunto finito de \isa{W}, entonces se tiene que o bien 
      \isa{{\isacharbraceleft}{\isasymbeta}\isactrlsub {\isadigit{1}}{\isacharcomma}F{\isacharbraceright}\ {\isasymunion}\ W\isactrlsub {\isadigit{0}}} es satisfacible o bien \isa{{\isacharbraceleft}{\isasymbeta}\isactrlsub {\isadigit{2}}{\isacharcomma}F{\isacharbraceright}\ {\isasymunion}\ W\isactrlsub {\isadigit{0}}} es satisfacible. En Isabelle se ha
      formalizado como \isa{pcp{\isacharunderscore}colecComp{\isacharunderscore}DIS{\isacharunderscore}sat}. Para su prueba, ha precisado de cuatro lemas
      auxiliares que prueban el resultado para cada caso de fórmula de tipo \isa{{\isasymbeta}}: 
      \isa{pcp{\isacharunderscore}colecComp{\isacharunderscore}DIS{\isacharunderscore}sat{\isadigit{1}}}, \isa{pcp{\isacharunderscore}colecComp{\isacharunderscore}DIS{\isacharunderscore}sat{\isadigit{2}}}, \isa{pcp{\isacharunderscore}colecComp{\isacharunderscore}DIS{\isacharunderscore}sat{\isadigit{3}}},
      \isa{pcp{\isacharunderscore}colecComp{\isacharunderscore}DIS{\isacharunderscore}sat{\isadigit{4}}}.
    \end{itemize}
  \end{itemize}

  Comencemos con las demostraciones de los lemas auxiliares empleados en la demostración del 
  teorema. Para probar que \isa{C} verifica la propiedad de consistencia proposicional, dado un conjunto 
  \isa{W\ {\isasymin}\ C} probaremos por separado que se verifican cada una de las condiciones del lema \isa{{\isadigit{2}}{\isachardot}{\isadigit{0}}{\isachardot}{\isadigit{2}}}.
  
  En primer lugar, veamos que \isa{{\isasymbottom}\ {\isasymnotin}\ W} si \isa{W\ {\isasymin}\ C}. Para ello, precisaremos del siguiente lema 
  auxiliar que prueba que el conjunto \isa{{\isacharbraceleft}{\isasymbottom}{\isacharbraceright}} no es satisfacible.%
\end{isamarkuptext}\isamarkuptrue%
\isacommand{lemma}\isamarkupfalse%
\ not{\isacharunderscore}sat{\isacharunderscore}bot{\isacharcolon}\ {\isachardoublequoteopen}{\isasymnot}\ sat\ {\isacharbraceleft}{\isasymbottom}{\isacharbraceright}{\isachardoublequoteclose}\isanewline
%
\isadelimproof
%
\endisadelimproof
%
\isatagproof
\isacommand{proof}\isamarkupfalse%
\ {\isacharparenleft}rule\ ccontr{\isacharparenright}\isanewline
\ \ \isacommand{assume}\isamarkupfalse%
\ {\isachardoublequoteopen}{\isasymnot}{\isacharparenleft}{\isasymnot}sat{\isacharbraceleft}{\isasymbottom}\ {\isacharcolon}{\isacharcolon}\ {\isacharprime}a\ formula{\isacharbraceright}{\isacharparenright}{\isachardoublequoteclose}\isanewline
\ \ \isacommand{then}\isamarkupfalse%
\ \isacommand{have}\isamarkupfalse%
\ {\isachardoublequoteopen}sat\ {\isacharbraceleft}{\isasymbottom}\ {\isacharcolon}{\isacharcolon}\ {\isacharprime}a\ formula{\isacharbraceright}{\isachardoublequoteclose}\isanewline
\ \ \ \ \isacommand{by}\isamarkupfalse%
\ {\isacharparenleft}rule\ notnotD{\isacharparenright}\isanewline
\ \ \isacommand{then}\isamarkupfalse%
\ \isacommand{have}\isamarkupfalse%
\ Ex{\isacharcolon}{\isachardoublequoteopen}{\isasymexists}{\isasymA}{\isachardot}\ {\isasymforall}F\ {\isasymin}\ {\isacharbraceleft}{\isasymbottom}\ {\isacharcolon}{\isacharcolon}\ {\isacharprime}a\ formula{\isacharbraceright}{\isachardot}\ {\isasymA}\ {\isasymTurnstile}\ F{\isachardoublequoteclose}\isanewline
\ \ \ \ \isacommand{by}\isamarkupfalse%
\ {\isacharparenleft}simp\ only{\isacharcolon}\ sat{\isacharunderscore}def{\isacharparenright}\isanewline
\ \ \isacommand{obtain}\isamarkupfalse%
\ {\isasymA}\ \isakeyword{where}\ {\isadigit{1}}{\isacharcolon}{\isachardoublequoteopen}{\isasymforall}F\ {\isasymin}\ {\isacharbraceleft}{\isasymbottom}\ {\isacharcolon}{\isacharcolon}\ {\isacharprime}a\ formula{\isacharbraceright}{\isachardot}\ {\isasymA}\ {\isasymTurnstile}\ F{\isachardoublequoteclose}\isanewline
\ \ \ \ \isacommand{using}\isamarkupfalse%
\ Ex\ \isacommand{by}\isamarkupfalse%
\ {\isacharparenleft}rule\ exE{\isacharparenright}\isanewline
\ \ \isacommand{have}\isamarkupfalse%
\ {\isadigit{2}}{\isacharcolon}{\isachardoublequoteopen}{\isasymbottom}\ {\isasymin}\ {\isacharbraceleft}{\isasymbottom}{\isacharcolon}{\isacharcolon}\ {\isacharprime}a\ formula{\isacharbraceright}{\isachardoublequoteclose}\isanewline
\ \ \ \ \isacommand{by}\isamarkupfalse%
\ {\isacharparenleft}simp\ only{\isacharcolon}\ singletonI{\isacharparenright}\isanewline
\ \ \isacommand{have}\isamarkupfalse%
\ {\isachardoublequoteopen}{\isasymA}\ {\isasymTurnstile}\ {\isasymbottom}{\isachardoublequoteclose}\isanewline
\ \ \ \ \isacommand{using}\isamarkupfalse%
\ {\isadigit{1}}\ {\isadigit{2}}\ \isacommand{by}\isamarkupfalse%
\ {\isacharparenleft}rule\ bspec{\isacharparenright}\isanewline
\ \ \isacommand{thus}\isamarkupfalse%
\ {\isachardoublequoteopen}False{\isachardoublequoteclose}\isanewline
\ \ \ \ \isacommand{by}\isamarkupfalse%
\ {\isacharparenleft}simp\ only{\isacharcolon}\ formula{\isacharunderscore}semantics{\isachardot}simps{\isacharparenleft}{\isadigit{2}}{\isacharparenright}{\isacharparenright}\isanewline
\isacommand{qed}\isamarkupfalse%
%
\endisatagproof
{\isafoldproof}%
%
\isadelimproof
%
\endisadelimproof
%
\begin{isamarkuptext}%
Por tanto, probemos que si \isa{W\ {\isasymin}\ C}, entonces \isa{{\isasymbottom}\ {\isasymnotin}\ W}.%
\end{isamarkuptext}\isamarkuptrue%
\isacommand{lemma}\isamarkupfalse%
\ pcp{\isacharunderscore}colecComp{\isacharunderscore}bot{\isacharcolon}\isanewline
\ \ \isakeyword{assumes}\ {\isachardoublequoteopen}W\ {\isasymin}\ colecComp{\isachardoublequoteclose}\isanewline
\ \ \isakeyword{shows}\ {\isachardoublequoteopen}{\isasymbottom}\ {\isasymnotin}\ W{\isachardoublequoteclose}\isanewline
%
\isadelimproof
%
\endisadelimproof
%
\isatagproof
\isacommand{proof}\isamarkupfalse%
\ {\isacharparenleft}rule\ ccontr{\isacharparenright}\isanewline
\ \ \isacommand{assume}\isamarkupfalse%
\ {\isachardoublequoteopen}{\isasymnot}{\isacharparenleft}{\isasymbottom}\ {\isasymnotin}\ W{\isacharparenright}{\isachardoublequoteclose}\isanewline
\ \ \isacommand{then}\isamarkupfalse%
\ \isacommand{have}\isamarkupfalse%
\ {\isachardoublequoteopen}{\isasymbottom}\ {\isasymin}\ W{\isachardoublequoteclose}\isanewline
\ \ \ \ \isacommand{by}\isamarkupfalse%
\ {\isacharparenleft}rule\ notnotD{\isacharparenright}\isanewline
\ \ \isacommand{have}\isamarkupfalse%
\ {\isachardoublequoteopen}{\isacharbraceleft}{\isacharbraceright}\ {\isasymsubseteq}\ W{\isachardoublequoteclose}\ \isanewline
\ \ \ \ \isacommand{by}\isamarkupfalse%
\ {\isacharparenleft}simp\ only{\isacharcolon}\ empty{\isacharunderscore}subsetI{\isacharparenright}\ \isanewline
\ \ \isacommand{have}\isamarkupfalse%
\ {\isachardoublequoteopen}{\isasymbottom}\ {\isasymin}\ W\ {\isasymand}\ {\isacharbraceleft}{\isacharbraceright}\ {\isasymsubseteq}\ W{\isachardoublequoteclose}\isanewline
\ \ \ \ \isacommand{using}\isamarkupfalse%
\ {\isacartoucheopen}{\isasymbottom}\ {\isasymin}\ W{\isacartoucheclose}\ {\isacartoucheopen}{\isacharbraceleft}{\isacharbraceright}\ {\isasymsubseteq}\ W{\isacartoucheclose}\ \isacommand{by}\isamarkupfalse%
\ {\isacharparenleft}rule\ conjI{\isacharparenright}\isanewline
\ \ \isacommand{then}\isamarkupfalse%
\ \isacommand{have}\isamarkupfalse%
\ {\isachardoublequoteopen}{\isacharbraceleft}{\isasymbottom}{\isacharbraceright}\ {\isasymsubseteq}\ W{\isachardoublequoteclose}\isanewline
\ \ \ \ \isacommand{by}\isamarkupfalse%
\ {\isacharparenleft}simp\ only{\isacharcolon}\ insert{\isacharunderscore}subset{\isacharparenright}\isanewline
\ \ \isacommand{have}\isamarkupfalse%
\ {\isachardoublequoteopen}finite\ {\isacharbraceleft}{\isasymbottom}{\isacharbraceright}{\isachardoublequoteclose}\ \isanewline
\ \ \ \ \isacommand{by}\isamarkupfalse%
\ {\isacharparenleft}simp\ only{\isacharcolon}\ finite{\isachardot}emptyI\ finite{\isacharunderscore}insert{\isacharparenright}\isanewline
\ \ \isacommand{have}\isamarkupfalse%
\ {\isachardoublequoteopen}sat\ {\isacharbraceleft}{\isasymbottom}\ {\isacharcolon}{\isacharcolon}\ {\isacharprime}a\ formula{\isacharbraceright}{\isachardoublequoteclose}\ \isanewline
\ \ \ \ \isacommand{using}\isamarkupfalse%
\ assms\ {\isacartoucheopen}{\isacharbraceleft}{\isasymbottom}{\isacharbraceright}\ {\isasymsubseteq}\ W{\isacartoucheclose}\ {\isacartoucheopen}finite\ {\isacharbraceleft}{\isasymbottom}{\isacharbraceright}{\isacartoucheclose}\ \isacommand{by}\isamarkupfalse%
\ {\isacharparenleft}rule\ colecComp{\isacharunderscore}subset{\isacharunderscore}finite{\isacharparenright}\isanewline
\ \ \isacommand{have}\isamarkupfalse%
\ {\isachardoublequoteopen}{\isasymnot}\ sat\ {\isacharbraceleft}{\isasymbottom}\ {\isacharcolon}{\isacharcolon}\ {\isacharprime}a\ formula{\isacharbraceright}{\isachardoublequoteclose}\ \isanewline
\ \ \ \ \isacommand{by}\isamarkupfalse%
\ {\isacharparenleft}rule\ not{\isacharunderscore}sat{\isacharunderscore}bot{\isacharparenright}\isanewline
\ \ \isacommand{then}\isamarkupfalse%
\ \isacommand{show}\isamarkupfalse%
\ False\ \isanewline
\ \ \ \ \isacommand{using}\isamarkupfalse%
\ {\isacartoucheopen}sat\ {\isacharbraceleft}{\isasymbottom}\ {\isacharcolon}{\isacharcolon}\ {\isacharprime}a\ formula{\isacharbraceright}{\isacartoucheclose}\ \isacommand{by}\isamarkupfalse%
\ {\isacharparenleft}rule\ notE{\isacharparenright}\isanewline
\isacommand{qed}\isamarkupfalse%
%
\endisatagproof
{\isafoldproof}%
%
\isadelimproof
%
\endisadelimproof
%
\begin{isamarkuptext}%
Por otro lado, vamos a probar que dado un conjunto \isa{W\ {\isasymin}\ C} y \isa{p} una fórmula atómica 
  cualquiera, no se tiene simultáneamente que \isa{p\ {\isasymin}\ W} y \isa{{\isasymnot}\ p\ {\isasymin}\ W}. Para ello, emplearemos el 
  siguiente lema auxiliar que prueba que el conjunto \isa{{\isacharbraceleft}p{\isacharcomma}{\isasymnot}\ p{\isacharbraceright}} es insatisfacible para cualquier 
  fórmula atómica \isa{p}.%
\end{isamarkuptext}\isamarkuptrue%
\isacommand{lemma}\isamarkupfalse%
\ not{\isacharunderscore}sat{\isacharunderscore}atoms{\isacharcolon}\ {\isachardoublequoteopen}{\isasymnot}\ sat{\isacharparenleft}{\isacharbraceleft}Atom\ k{\isacharcomma}\ \isactrlbold {\isasymnot}\ {\isacharparenleft}Atom\ k{\isacharparenright}{\isacharbraceright}{\isacharparenright}{\isachardoublequoteclose}\isanewline
%
\isadelimproof
%
\endisadelimproof
%
\isatagproof
\isacommand{proof}\isamarkupfalse%
\ {\isacharparenleft}rule\ ccontr{\isacharparenright}\isanewline
\ \ \isacommand{assume}\isamarkupfalse%
\ {\isachardoublequoteopen}{\isasymnot}\ {\isasymnot}\ sat{\isacharparenleft}{\isacharbraceleft}Atom\ k{\isacharcomma}\ \isactrlbold {\isasymnot}\ {\isacharparenleft}Atom\ k{\isacharparenright}{\isacharbraceright}{\isacharparenright}{\isachardoublequoteclose}\isanewline
\ \ \isacommand{then}\isamarkupfalse%
\ \isacommand{have}\isamarkupfalse%
\ {\isachardoublequoteopen}sat{\isacharparenleft}{\isacharbraceleft}Atom\ k{\isacharcomma}\ \isactrlbold {\isasymnot}\ {\isacharparenleft}Atom\ k{\isacharparenright}{\isacharbraceright}{\isacharparenright}{\isachardoublequoteclose}\isanewline
\ \ \ \ \isacommand{by}\isamarkupfalse%
\ {\isacharparenleft}rule\ notnotD{\isacharparenright}\isanewline
\ \ \isacommand{then}\isamarkupfalse%
\ \isacommand{have}\isamarkupfalse%
\ Sat{\isacharcolon}{\isachardoublequoteopen}{\isasymexists}{\isasymA}{\isachardot}\ {\isasymforall}F\ {\isasymin}\ {\isacharbraceleft}Atom\ k{\isacharcomma}\ \isactrlbold {\isasymnot}{\isacharparenleft}Atom\ k{\isacharparenright}{\isacharbraceright}{\isachardot}\ {\isasymA}\ {\isasymTurnstile}\ F{\isachardoublequoteclose}\isanewline
\ \ \ \ \isacommand{by}\isamarkupfalse%
\ {\isacharparenleft}simp\ only{\isacharcolon}\ sat{\isacharunderscore}def{\isacharparenright}\isanewline
\ \ \isacommand{obtain}\isamarkupfalse%
\ {\isasymA}\ \isakeyword{where}\ H{\isacharcolon}{\isachardoublequoteopen}{\isasymforall}F\ {\isasymin}\ {\isacharbraceleft}Atom\ k{\isacharcomma}\ \isactrlbold {\isasymnot}{\isacharparenleft}Atom\ k{\isacharparenright}{\isacharbraceright}{\isachardot}\ {\isasymA}\ {\isasymTurnstile}\ F{\isachardoublequoteclose}\isanewline
\ \ \ \ \isacommand{using}\isamarkupfalse%
\ Sat\ \isacommand{by}\isamarkupfalse%
\ {\isacharparenleft}rule\ exE{\isacharparenright}\isanewline
\ \ \isacommand{have}\isamarkupfalse%
\ {\isachardoublequoteopen}Atom\ k\ {\isasymin}\ {\isacharbraceleft}Atom\ k{\isacharcomma}\ \isactrlbold {\isasymnot}{\isacharparenleft}Atom\ k{\isacharparenright}{\isacharbraceright}{\isachardoublequoteclose}\isanewline
\ \ \ \ \isacommand{by}\isamarkupfalse%
\ simp\isanewline
\ \ \isacommand{have}\isamarkupfalse%
\ {\isachardoublequoteopen}{\isasymA}\ {\isasymTurnstile}\ Atom\ k{\isachardoublequoteclose}\isanewline
\ \ \ \ \isacommand{using}\isamarkupfalse%
\ H\ {\isacartoucheopen}Atom\ k\ {\isasymin}\ {\isacharbraceleft}Atom\ k{\isacharcomma}\ \isactrlbold {\isasymnot}{\isacharparenleft}Atom\ k{\isacharparenright}{\isacharbraceright}{\isacartoucheclose}\ \isacommand{by}\isamarkupfalse%
\ {\isacharparenleft}rule\ bspec{\isacharparenright}\isanewline
\ \ \isacommand{have}\isamarkupfalse%
\ {\isachardoublequoteopen}\isactrlbold {\isasymnot}{\isacharparenleft}Atom\ k{\isacharparenright}\ {\isasymin}\ {\isacharbraceleft}Atom\ k{\isacharcomma}\ \isactrlbold {\isasymnot}{\isacharparenleft}Atom\ k{\isacharparenright}{\isacharbraceright}{\isachardoublequoteclose}\isanewline
\ \ \ \ \isacommand{by}\isamarkupfalse%
\ simp\isanewline
\ \ \isacommand{have}\isamarkupfalse%
\ {\isachardoublequoteopen}{\isasymA}\ {\isasymTurnstile}\ \isactrlbold {\isasymnot}{\isacharparenleft}Atom\ k{\isacharparenright}{\isachardoublequoteclose}\isanewline
\ \ \ \ \isacommand{using}\isamarkupfalse%
\ H\ {\isacartoucheopen}\isactrlbold {\isasymnot}{\isacharparenleft}Atom\ k{\isacharparenright}\ {\isasymin}\ {\isacharbraceleft}Atom\ k{\isacharcomma}\ \isactrlbold {\isasymnot}{\isacharparenleft}Atom\ k{\isacharparenright}{\isacharbraceright}{\isacartoucheclose}\ \isacommand{by}\isamarkupfalse%
\ {\isacharparenleft}rule\ bspec{\isacharparenright}\isanewline
\ \ \isacommand{then}\isamarkupfalse%
\ \isacommand{have}\isamarkupfalse%
\ {\isachardoublequoteopen}{\isasymnot}\ {\isasymA}\ {\isasymTurnstile}\ Atom\ k{\isachardoublequoteclose}\ \isanewline
\ \ \ \ \isacommand{by}\isamarkupfalse%
\ {\isacharparenleft}simp\ only{\isacharcolon}\ simp{\isacharunderscore}thms{\isacharparenleft}{\isadigit{8}}{\isacharparenright}\ formula{\isacharunderscore}semantics{\isachardot}simps{\isacharparenleft}{\isadigit{3}}{\isacharparenright}{\isacharparenright}\isanewline
\ \ \isacommand{thus}\isamarkupfalse%
\ {\isachardoublequoteopen}False{\isachardoublequoteclose}\isanewline
\ \ \ \ \isacommand{using}\isamarkupfalse%
\ {\isacartoucheopen}{\isasymA}\ {\isasymTurnstile}\ Atom\ k{\isacartoucheclose}\ \isacommand{by}\isamarkupfalse%
\ {\isacharparenleft}rule\ notE{\isacharparenright}\isanewline
\isacommand{qed}\isamarkupfalse%
%
\endisatagproof
{\isafoldproof}%
%
\isadelimproof
%
\endisadelimproof
%
\begin{isamarkuptext}%
De este modo, podemos demostrar detalladamente la condición: dados \isa{W\ {\isasymin}\ C} y \isa{p} una fórmula
  atómica cualquiera, no se tiene simultáneamente que \isa{p\ {\isasymin}\ W} y \isa{{\isasymnot}\ p\ {\isasymin}\ W}.%
\end{isamarkuptext}\isamarkuptrue%
\isacommand{lemma}\isamarkupfalse%
\ pcp{\isacharunderscore}colecComp{\isacharunderscore}atoms{\isacharcolon}\isanewline
\ \ \isakeyword{assumes}\ {\isachardoublequoteopen}W\ {\isasymin}\ colecComp{\isachardoublequoteclose}\isanewline
\ \ \isakeyword{shows}\ {\isachardoublequoteopen}{\isasymforall}k{\isachardot}\ Atom\ k\ {\isasymin}\ W\ {\isasymlongrightarrow}\ \isactrlbold {\isasymnot}\ {\isacharparenleft}Atom\ k{\isacharparenright}\ {\isasymin}\ W\ {\isasymlongrightarrow}\ False{\isachardoublequoteclose}\isanewline
%
\isadelimproof
%
\endisadelimproof
%
\isatagproof
\isacommand{proof}\isamarkupfalse%
\ {\isacharparenleft}rule\ allI{\isacharparenright}\isanewline
\ \ \isacommand{fix}\isamarkupfalse%
\ k\isanewline
\ \ \isacommand{show}\isamarkupfalse%
\ {\isachardoublequoteopen}Atom\ k\ {\isasymin}\ W\ {\isasymlongrightarrow}\ \isactrlbold {\isasymnot}\ {\isacharparenleft}Atom\ k{\isacharparenright}\ {\isasymin}\ W\ {\isasymlongrightarrow}\ False{\isachardoublequoteclose}\isanewline
\ \ \isacommand{proof}\isamarkupfalse%
\ {\isacharparenleft}rule\ impI{\isacharparenright}{\isacharplus}\isanewline
\ \ \ \ \isacommand{assume}\isamarkupfalse%
\ {\isadigit{1}}{\isacharcolon}{\isachardoublequoteopen}Atom\ k\ {\isasymin}\ W{\isachardoublequoteclose}\isanewline
\ \ \ \ \isacommand{assume}\isamarkupfalse%
\ {\isadigit{2}}{\isacharcolon}{\isachardoublequoteopen}\isactrlbold {\isasymnot}\ {\isacharparenleft}Atom\ k{\isacharparenright}\ {\isasymin}\ W{\isachardoublequoteclose}\isanewline
\ \ \ \ \isacommand{have}\isamarkupfalse%
\ {\isachardoublequoteopen}{\isacharbraceleft}{\isacharbraceright}\ {\isasymsubseteq}\ W{\isachardoublequoteclose}\isanewline
\ \ \ \ \ \ \isacommand{by}\isamarkupfalse%
\ {\isacharparenleft}simp\ only{\isacharcolon}\ empty{\isacharunderscore}subsetI{\isacharparenright}\ \isanewline
\ \ \ \ \isacommand{have}\isamarkupfalse%
\ {\isachardoublequoteopen}Atom\ k\ {\isasymin}\ W\ {\isasymand}\ {\isacharbraceleft}{\isacharbraceright}\ {\isasymsubseteq}\ W{\isachardoublequoteclose}\isanewline
\ \ \ \ \ \ \isacommand{using}\isamarkupfalse%
\ {\isadigit{1}}\ {\isacartoucheopen}{\isacharbraceleft}{\isacharbraceright}\ {\isasymsubseteq}\ W{\isacartoucheclose}\ \isacommand{by}\isamarkupfalse%
\ {\isacharparenleft}rule\ conjI{\isacharparenright}\isanewline
\ \ \ \ \isacommand{then}\isamarkupfalse%
\ \isacommand{have}\isamarkupfalse%
\ {\isachardoublequoteopen}{\isacharbraceleft}Atom\ k{\isacharbraceright}\ {\isasymsubseteq}\ W{\isachardoublequoteclose}\isanewline
\ \ \ \ \ \ \isacommand{by}\isamarkupfalse%
\ {\isacharparenleft}simp\ only{\isacharcolon}\ insert{\isacharunderscore}subset{\isacharparenright}\isanewline
\ \ \ \ \isacommand{have}\isamarkupfalse%
\ {\isachardoublequoteopen}\isactrlbold {\isasymnot}\ {\isacharparenleft}Atom\ k{\isacharparenright}\ {\isasymin}\ W\ {\isasymand}\ {\isacharbraceleft}{\isacharbraceright}\ {\isasymsubseteq}\ W{\isachardoublequoteclose}\isanewline
\ \ \ \ \ \ \isacommand{using}\isamarkupfalse%
\ {\isadigit{2}}\ {\isacartoucheopen}{\isacharbraceleft}{\isacharbraceright}\ {\isasymsubseteq}\ W{\isacartoucheclose}\ \isacommand{by}\isamarkupfalse%
\ {\isacharparenleft}rule\ conjI{\isacharparenright}\isanewline
\ \ \ \ \isacommand{then}\isamarkupfalse%
\ \isacommand{have}\isamarkupfalse%
\ {\isachardoublequoteopen}{\isacharbraceleft}\isactrlbold {\isasymnot}{\isacharparenleft}Atom\ k{\isacharparenright}{\isacharbraceright}\ {\isasymsubseteq}\ W{\isachardoublequoteclose}\isanewline
\ \ \ \ \ \ \isacommand{by}\isamarkupfalse%
\ {\isacharparenleft}simp\ only{\isacharcolon}\ insert{\isacharunderscore}subset{\isacharparenright}\isanewline
\ \ \ \ \isacommand{have}\isamarkupfalse%
\ {\isachardoublequoteopen}{\isacharbraceleft}Atom\ k{\isacharbraceright}\ {\isasymunion}\ {\isacharbraceleft}\isactrlbold {\isasymnot}{\isacharparenleft}Atom\ k{\isacharparenright}{\isacharbraceright}\ {\isasymsubseteq}\ W{\isachardoublequoteclose}\isanewline
\ \ \ \ \ \ \isacommand{using}\isamarkupfalse%
\ {\isacartoucheopen}{\isacharbraceleft}Atom\ k{\isacharbraceright}\ {\isasymsubseteq}\ W{\isacartoucheclose}\ {\isacartoucheopen}{\isacharbraceleft}\isactrlbold {\isasymnot}{\isacharparenleft}Atom\ k{\isacharparenright}{\isacharbraceright}\ {\isasymsubseteq}\ W{\isacartoucheclose}\ \isacommand{by}\isamarkupfalse%
\ {\isacharparenleft}simp\ only{\isacharcolon}\ Un{\isacharunderscore}least{\isacharparenright}\isanewline
\ \ \ \ \isacommand{then}\isamarkupfalse%
\ \isacommand{have}\isamarkupfalse%
\ {\isachardoublequoteopen}{\isacharbraceleft}Atom\ k{\isacharcomma}\ \isactrlbold {\isasymnot}{\isacharparenleft}Atom\ k{\isacharparenright}{\isacharbraceright}\ {\isasymsubseteq}\ W{\isachardoublequoteclose}\isanewline
\ \ \ \ \ \ \isacommand{by}\isamarkupfalse%
\ simp\ \isanewline
\ \ \ \ \isacommand{have}\isamarkupfalse%
\ {\isachardoublequoteopen}finite\ {\isacharbraceleft}Atom\ k{\isacharcomma}\ \isactrlbold {\isasymnot}{\isacharparenleft}Atom\ k{\isacharparenright}{\isacharbraceright}{\isachardoublequoteclose}\isanewline
\ \ \ \ \ \ \isacommand{by}\isamarkupfalse%
\ blast\isanewline
\ \ \ \ \isacommand{have}\isamarkupfalse%
\ {\isachardoublequoteopen}sat\ {\isacharparenleft}{\isacharbraceleft}Atom\ k{\isacharcomma}\ \isactrlbold {\isasymnot}{\isacharparenleft}Atom\ k{\isacharparenright}{\isacharbraceright}{\isacharparenright}{\isachardoublequoteclose}\isanewline
\ \ \ \ \ \ \isacommand{using}\isamarkupfalse%
\ assms\ {\isacartoucheopen}{\isacharbraceleft}Atom\ k{\isacharcomma}\ \isactrlbold {\isasymnot}{\isacharparenleft}Atom\ k{\isacharparenright}{\isacharbraceright}\ {\isasymsubseteq}\ W{\isacartoucheclose}\ {\isacartoucheopen}finite\ {\isacharbraceleft}Atom\ k{\isacharcomma}\ \isactrlbold {\isasymnot}{\isacharparenleft}Atom\ k{\isacharparenright}{\isacharbraceright}{\isacartoucheclose}\ \isacommand{by}\isamarkupfalse%
\ {\isacharparenleft}rule\ colecComp{\isacharunderscore}subset{\isacharunderscore}finite{\isacharparenright}\isanewline
\ \ \ \ \isacommand{have}\isamarkupfalse%
\ {\isachardoublequoteopen}{\isasymnot}\ sat\ {\isacharparenleft}{\isacharbraceleft}Atom\ k{\isacharcomma}\ \isactrlbold {\isasymnot}{\isacharparenleft}Atom\ k{\isacharparenright}{\isacharbraceright}{\isacharparenright}{\isachardoublequoteclose}\isanewline
\ \ \ \ \ \ \isacommand{by}\isamarkupfalse%
\ {\isacharparenleft}rule\ not{\isacharunderscore}sat{\isacharunderscore}atoms{\isacharparenright}\isanewline
\ \ \ \ \isacommand{thus}\isamarkupfalse%
\ {\isachardoublequoteopen}False{\isachardoublequoteclose}\isanewline
\ \ \ \ \ \ \isacommand{using}\isamarkupfalse%
\ {\isacartoucheopen}sat\ {\isacharparenleft}{\isacharbraceleft}Atom\ k{\isacharcomma}\ \isactrlbold {\isasymnot}{\isacharparenleft}Atom\ k{\isacharparenright}{\isacharbraceright}{\isacharparenright}{\isacartoucheclose}\ \isacommand{by}\isamarkupfalse%
\ {\isacharparenleft}rule\ notE{\isacharparenright}\isanewline
\ \ \isacommand{qed}\isamarkupfalse%
\isanewline
\isacommand{qed}\isamarkupfalse%
%
\endisatagproof
{\isafoldproof}%
%
\isadelimproof
%
\endisadelimproof
%
\begin{isamarkuptext}%
Demostremos la tercera condición del lema \isa{{\isadigit{2}}{\isachardot}{\isadigit{0}}{\isachardot}{\isadigit{2}}}: dados \isa{W\ {\isasymin}\ C} y \isa{F} una fórmula de 
  tipo \isa{{\isasymalpha}} con componentes \isa{{\isasymalpha}\isactrlsub {\isadigit{1}}} y \isa{{\isasymalpha}\isactrlsub {\isadigit{2}}} tal que \isa{F\ {\isasymin}\ W}, se tiene que \isa{{\isacharbraceleft}{\isasymalpha}\isactrlsub {\isadigit{1}}{\isacharcomma}{\isasymalpha}\isactrlsub {\isadigit{2}}{\isacharbraceright}\ {\isasymunion}\ W\ {\isasymin}\ C}. Para probar 
  dicho resultado, emplearemos un lema auxiliar que demuestra que dado un subconjunto finito \isa{W\isactrlsub {\isadigit{0}}} de 
  \isa{W} se tiene que \isa{{\isacharbraceleft}{\isasymalpha}\isactrlsub {\isadigit{1}}{\isacharcomma}{\isasymalpha}\isactrlsub {\isadigit{2}}{\isacharcomma}F{\isacharbraceright}\ {\isasymunion}\ W\isactrlsub {\isadigit{0}}} es un conjunto satisfacible. Mostraremos la prueba para cada
  caso de fórmula de tipo \isa{{\isasymalpha}}. Para ello, precisaremos del siguiente lema auxiliar que demuestra que 
  dado un conjunto \isa{W\ {\isasymin}\ C}, \isa{F} una fórmula perteneciente a \isa{W} y \isa{W\isactrlsub {\isadigit{0}}} un subconjunto finito de \isa{W}, 
  entonces \isa{{\isacharbraceleft}F{\isacharbraceright}\ {\isasymunion}\ W\isactrlsub {\isadigit{0}}} es satisfacible.%
\end{isamarkuptext}\isamarkuptrue%
\isacommand{lemma}\isamarkupfalse%
\ pcp{\isacharunderscore}colecComp{\isacharunderscore}elem{\isacharunderscore}sat{\isacharcolon}\isanewline
\ \ \isakeyword{assumes}\ {\isachardoublequoteopen}W\ {\isasymin}\ colecComp{\isachardoublequoteclose}\isanewline
\ \ \ \ \ \ \ \ \ \ {\isachardoublequoteopen}F\ {\isasymin}\ W{\isachardoublequoteclose}\isanewline
\ \ \ \ \ \ \ \ \ \ {\isachardoublequoteopen}finite\ Wo{\isachardoublequoteclose}\isanewline
\ \ \ \ \ \ \ \ \ \ {\isachardoublequoteopen}Wo\ {\isasymsubseteq}\ W{\isachardoublequoteclose}\isanewline
\ \ \ \ \ \ \ \ \isakeyword{shows}\ {\isachardoublequoteopen}sat\ {\isacharparenleft}{\isacharbraceleft}F{\isacharbraceright}\ {\isasymunion}\ Wo{\isacharparenright}{\isachardoublequoteclose}\isanewline
%
\isadelimproof
%
\endisadelimproof
%
\isatagproof
\isacommand{proof}\isamarkupfalse%
\ {\isacharminus}\isanewline
\ \ \isacommand{have}\isamarkupfalse%
\ {\isadigit{1}}{\isacharcolon}{\isachardoublequoteopen}insert\ F\ Wo\ {\isacharequal}\ {\isacharbraceleft}F{\isacharbraceright}\ {\isasymunion}\ Wo{\isachardoublequoteclose}\isanewline
\ \ \ \ \isacommand{by}\isamarkupfalse%
\ {\isacharparenleft}rule\ insert{\isacharunderscore}is{\isacharunderscore}Un{\isacharparenright}\isanewline
\ \ \isacommand{have}\isamarkupfalse%
\ {\isachardoublequoteopen}finite\ {\isacharparenleft}insert\ F\ Wo{\isacharparenright}{\isachardoublequoteclose}\isanewline
\ \ \ \ \isacommand{using}\isamarkupfalse%
\ assms{\isacharparenleft}{\isadigit{3}}{\isacharparenright}\ \isacommand{by}\isamarkupfalse%
\ {\isacharparenleft}simp\ only{\isacharcolon}\ finite{\isacharunderscore}insert{\isacharparenright}\isanewline
\ \ \isacommand{then}\isamarkupfalse%
\ \isacommand{have}\isamarkupfalse%
\ {\isachardoublequoteopen}finite\ {\isacharparenleft}{\isacharbraceleft}F{\isacharbraceright}\ {\isasymunion}\ Wo{\isacharparenright}{\isachardoublequoteclose}\isanewline
\ \ \ \ \isacommand{by}\isamarkupfalse%
\ {\isacharparenleft}simp\ only{\isacharcolon}\ {\isadigit{1}}{\isacharparenright}\ \isanewline
\ \ \isacommand{have}\isamarkupfalse%
\ {\isachardoublequoteopen}F\ {\isasymin}\ W\ {\isasymand}\ Wo\ {\isasymsubseteq}\ W{\isachardoublequoteclose}\isanewline
\ \ \ \ \isacommand{using}\isamarkupfalse%
\ assms{\isacharparenleft}{\isadigit{2}}{\isacharparenright}\ assms{\isacharparenleft}{\isadigit{4}}{\isacharparenright}\ \isacommand{by}\isamarkupfalse%
\ {\isacharparenleft}rule\ conjI{\isacharparenright}\isanewline
\ \ \isacommand{then}\isamarkupfalse%
\ \isacommand{have}\isamarkupfalse%
\ {\isachardoublequoteopen}insert\ F\ Wo\ {\isasymsubseteq}\ W{\isachardoublequoteclose}\isanewline
\ \ \ \ \isacommand{by}\isamarkupfalse%
\ {\isacharparenleft}simp\ only{\isacharcolon}\ insert{\isacharunderscore}subset{\isacharparenright}\isanewline
\ \ \isacommand{then}\isamarkupfalse%
\ \isacommand{have}\isamarkupfalse%
\ {\isachardoublequoteopen}{\isacharbraceleft}F{\isacharbraceright}\ {\isasymunion}\ Wo\ {\isasymsubseteq}\ W{\isachardoublequoteclose}\isanewline
\ \ \ \ \isacommand{by}\isamarkupfalse%
\ {\isacharparenleft}simp\ only{\isacharcolon}\ {\isadigit{1}}{\isacharparenright}\isanewline
\ \ \isacommand{show}\isamarkupfalse%
\ {\isachardoublequoteopen}sat\ {\isacharparenleft}{\isacharbraceleft}F{\isacharbraceright}\ {\isasymunion}\ Wo{\isacharparenright}{\isachardoublequoteclose}\isanewline
\ \ \ \ \isacommand{using}\isamarkupfalse%
\ assms{\isacharparenleft}{\isadigit{1}}{\isacharparenright}\ {\isacartoucheopen}{\isacharbraceleft}F{\isacharbraceright}\ {\isasymunion}\ Wo\ {\isasymsubseteq}\ W{\isacartoucheclose}\ {\isacartoucheopen}finite\ {\isacharparenleft}{\isacharbraceleft}F{\isacharbraceright}\ {\isasymunion}\ Wo{\isacharparenright}{\isacartoucheclose}\ \isacommand{by}\isamarkupfalse%
\ {\isacharparenleft}rule\ colecComp{\isacharunderscore}subset{\isacharunderscore}finite{\isacharparenright}\isanewline
\isacommand{qed}\isamarkupfalse%
%
\endisatagproof
{\isafoldproof}%
%
\isadelimproof
%
\endisadelimproof
%
\begin{isamarkuptext}%
De este modo, vamos a probar para cada caso de fórmula \isa{{\isasymalpha}} que dados \isa{W\ {\isasymin}\ C}, \isa{F} una fórmula 
  de tipo \isa{{\isasymalpha}} con componentes \isa{{\isasymalpha}\isactrlsub {\isadigit{1}}} y \isa{{\isasymalpha}\isactrlsub {\isadigit{2}}} tal que \isa{F\ {\isasymin}\ W} y \isa{W\isactrlsub {\isadigit{0}}} un subconjunto finito de \isa{W}, se 
  verifica que \isa{{\isacharbraceleft}{\isasymalpha}\isactrlsub {\isadigit{1}}{\isacharcomma}{\isasymalpha}\isactrlsub {\isadigit{2}}{\isacharcomma}F{\isacharbraceright}\ {\isasymunion}\ W\isactrlsub {\isadigit{0}}} es satisfacible. Para ello, emplearemos el siguiente lema auxiliar
  en Isabelle.%
\end{isamarkuptext}\isamarkuptrue%
\isacommand{lemma}\isamarkupfalse%
\ ball{\isacharunderscore}Un{\isacharcolon}\ \isanewline
\ \ \isakeyword{assumes}\ {\isachardoublequoteopen}{\isasymforall}x\ {\isasymin}\ A{\isachardot}\ P\ x{\isachardoublequoteclose}\isanewline
\ \ \ \ \ \ \ \ \ \ {\isachardoublequoteopen}{\isasymforall}x\ {\isasymin}\ B{\isachardot}\ P\ x{\isachardoublequoteclose}\isanewline
\ \ \ \ \ \ \ \ \isakeyword{shows}\ {\isachardoublequoteopen}{\isasymforall}x\ {\isasymin}\ {\isacharparenleft}A\ {\isasymunion}\ B{\isacharparenright}{\isachardot}\ P\ x{\isachardoublequoteclose}\ \isanewline
%
\isadelimproof
\ \ %
\endisadelimproof
%
\isatagproof
\isacommand{using}\isamarkupfalse%
\ assms\ \isacommand{by}\isamarkupfalse%
\ blast%
\endisatagproof
{\isafoldproof}%
%
\isadelimproof
%
\endisadelimproof
%
\begin{isamarkuptext}%
En primer lugar, probemos que dados \isa{W\ {\isasymin}\ C}, una fórmula \isa{F\ {\isacharequal}\ G\ {\isasymand}\ H} para ciertas fórmulas \isa{G} 
  y \isa{H} tal que \isa{F\ {\isasymin}\ W} y \isa{W\isactrlsub {\isadigit{0}}} un subconjunto finito de \isa{W}, se verifica que\\ \isa{{\isacharbraceleft}G{\isacharcomma}H{\isacharcomma}F{\isacharbraceright}\ {\isasymunion}\ W\isactrlsub {\isadigit{0}}} es 
  satisfacible.%
\end{isamarkuptext}\isamarkuptrue%
\isacommand{lemma}\isamarkupfalse%
\ pcp{\isacharunderscore}colecComp{\isacharunderscore}CON{\isacharunderscore}sat{\isadigit{1}}{\isacharcolon}\isanewline
\ \ \isakeyword{assumes}\ {\isachardoublequoteopen}W\ {\isasymin}\ colecComp{\isachardoublequoteclose}\isanewline
\ \ \ \ \ \ \ \ \ \ {\isachardoublequoteopen}F\ {\isacharequal}\ G\ \isactrlbold {\isasymand}\ H{\isachardoublequoteclose}\isanewline
\ \ \ \ \ \ \ \ \ \ {\isachardoublequoteopen}F\ {\isasymin}\ W{\isachardoublequoteclose}\isanewline
\ \ \ \ \ \ \ \ \ \ {\isachardoublequoteopen}finite\ Wo{\isachardoublequoteclose}\isanewline
\ \ \ \ \ \ \ \ \ \ {\isachardoublequoteopen}Wo\ {\isasymsubseteq}\ W{\isachardoublequoteclose}\isanewline
\ \ \ \ \ \ \ \ \isakeyword{shows}\ {\isachardoublequoteopen}sat\ {\isacharparenleft}{\isacharbraceleft}G{\isacharcomma}H{\isacharcomma}F{\isacharbraceright}\ {\isasymunion}\ Wo{\isacharparenright}{\isachardoublequoteclose}\isanewline
%
\isadelimproof
%
\endisadelimproof
%
\isatagproof
\isacommand{proof}\isamarkupfalse%
\ {\isacharminus}\isanewline
\ \ \isacommand{have}\isamarkupfalse%
\ {\isachardoublequoteopen}sat\ {\isacharparenleft}{\isacharbraceleft}F{\isacharbraceright}\ {\isasymunion}\ Wo{\isacharparenright}{\isachardoublequoteclose}\isanewline
\ \ \ \ \isacommand{using}\isamarkupfalse%
\ assms{\isacharparenleft}{\isadigit{1}}{\isacharcomma}{\isadigit{3}}{\isacharcomma}{\isadigit{4}}{\isacharcomma}{\isadigit{5}}{\isacharparenright}\ \isacommand{by}\isamarkupfalse%
\ {\isacharparenleft}rule\ pcp{\isacharunderscore}colecComp{\isacharunderscore}elem{\isacharunderscore}sat{\isacharparenright}\isanewline
\ \ \isacommand{have}\isamarkupfalse%
\ {\isachardoublequoteopen}F\ {\isasymin}\ {\isacharbraceleft}F{\isacharbraceright}\ {\isasymunion}\ Wo{\isachardoublequoteclose}\isanewline
\ \ \ \ \isacommand{by}\isamarkupfalse%
\ {\isacharparenleft}simp\ add{\isacharcolon}\ insertI{\isadigit{1}}{\isacharparenright}\isanewline
\ \ \isacommand{have}\isamarkupfalse%
\ Ex{\isadigit{1}}{\isacharcolon}{\isachardoublequoteopen}{\isasymexists}{\isasymA}{\isachardot}\ {\isasymforall}F\ {\isasymin}\ {\isacharparenleft}{\isacharbraceleft}F{\isacharbraceright}\ {\isasymunion}\ Wo{\isacharparenright}{\isachardot}\ {\isasymA}\ {\isasymTurnstile}\ F{\isachardoublequoteclose}\isanewline
\ \ \ \ \isacommand{using}\isamarkupfalse%
\ {\isacartoucheopen}sat\ {\isacharparenleft}{\isacharbraceleft}F{\isacharbraceright}\ {\isasymunion}\ Wo{\isacharparenright}{\isacartoucheclose}\ \isacommand{by}\isamarkupfalse%
\ {\isacharparenleft}simp\ only{\isacharcolon}\ sat{\isacharunderscore}def{\isacharparenright}\isanewline
\ \ \isacommand{obtain}\isamarkupfalse%
\ {\isasymA}\ \isakeyword{where}\ Forall{\isadigit{1}}{\isacharcolon}{\isachardoublequoteopen}{\isasymforall}F\ {\isasymin}\ {\isacharparenleft}{\isacharbraceleft}F{\isacharbraceright}\ {\isasymunion}\ Wo{\isacharparenright}{\isachardot}\ {\isasymA}\ {\isasymTurnstile}\ F{\isachardoublequoteclose}\isanewline
\ \ \ \ \isacommand{using}\isamarkupfalse%
\ Ex{\isadigit{1}}\ \isacommand{by}\isamarkupfalse%
\ {\isacharparenleft}rule\ exE{\isacharparenright}\isanewline
\ \ \isacommand{have}\isamarkupfalse%
\ {\isachardoublequoteopen}{\isasymA}\ {\isasymTurnstile}\ F{\isachardoublequoteclose}\isanewline
\ \ \ \ \isacommand{using}\isamarkupfalse%
\ Forall{\isadigit{1}}\ {\isacartoucheopen}F\ {\isasymin}\ {\isacharbraceleft}F{\isacharbraceright}\ {\isasymunion}\ Wo{\isacartoucheclose}\ \isacommand{by}\isamarkupfalse%
\ {\isacharparenleft}rule\ bspec{\isacharparenright}\isanewline
\ \ \isacommand{then}\isamarkupfalse%
\ \isacommand{have}\isamarkupfalse%
\ {\isachardoublequoteopen}{\isasymA}\ {\isasymTurnstile}\ {\isacharparenleft}G\ \isactrlbold {\isasymand}\ H{\isacharparenright}{\isachardoublequoteclose}\isanewline
\ \ \ \ \isacommand{using}\isamarkupfalse%
\ assms{\isacharparenleft}{\isadigit{2}}{\isacharparenright}\ \isacommand{by}\isamarkupfalse%
\ {\isacharparenleft}simp\ only{\isacharcolon}\ {\isacartoucheopen}{\isasymA}\ {\isasymTurnstile}\ F{\isacartoucheclose}{\isacharparenright}\isanewline
\ \ \isacommand{then}\isamarkupfalse%
\ \isacommand{have}\isamarkupfalse%
\ {\isachardoublequoteopen}{\isasymA}\ {\isasymTurnstile}\ G\ {\isasymand}\ {\isasymA}\ {\isasymTurnstile}\ H{\isachardoublequoteclose}\isanewline
\ \ \ \ \isacommand{by}\isamarkupfalse%
\ {\isacharparenleft}simp\ only{\isacharcolon}\ formula{\isacharunderscore}semantics{\isachardot}simps{\isacharparenleft}{\isadigit{4}}{\isacharparenright}{\isacharparenright}\isanewline
\ \ \isacommand{then}\isamarkupfalse%
\ \isacommand{have}\isamarkupfalse%
\ {\isachardoublequoteopen}{\isasymA}\ {\isasymTurnstile}\ G{\isachardoublequoteclose}\isanewline
\ \ \ \ \isacommand{by}\isamarkupfalse%
\ {\isacharparenleft}rule\ conjunct{\isadigit{1}}{\isacharparenright}\isanewline
\ \ \isacommand{then}\isamarkupfalse%
\ \isacommand{have}\isamarkupfalse%
\ {\isadigit{1}}{\isacharcolon}{\isachardoublequoteopen}{\isasymforall}F\ {\isasymin}\ {\isacharbraceleft}G{\isacharbraceright}{\isachardot}\ {\isasymA}\ {\isasymTurnstile}\ F{\isachardoublequoteclose}\isanewline
\ \ \ \ \isacommand{by}\isamarkupfalse%
\ simp\isanewline
\ \ \isacommand{have}\isamarkupfalse%
\ {\isachardoublequoteopen}{\isasymA}\ {\isasymTurnstile}\ H{\isachardoublequoteclose}\isanewline
\ \ \ \ \isacommand{using}\isamarkupfalse%
\ {\isacartoucheopen}{\isasymA}\ {\isasymTurnstile}\ G\ {\isasymand}\ {\isasymA}\ {\isasymTurnstile}\ H{\isacartoucheclose}\ \isacommand{by}\isamarkupfalse%
\ {\isacharparenleft}rule\ conjunct{\isadigit{2}}{\isacharparenright}\isanewline
\ \ \isacommand{then}\isamarkupfalse%
\ \isacommand{have}\isamarkupfalse%
\ {\isadigit{2}}{\isacharcolon}{\isachardoublequoteopen}{\isasymforall}F\ {\isasymin}\ {\isacharbraceleft}H{\isacharbraceright}{\isachardot}\ {\isasymA}\ {\isasymTurnstile}\ F{\isachardoublequoteclose}\isanewline
\ \ \ \ \isacommand{by}\isamarkupfalse%
\ simp\isanewline
\ \ \isacommand{have}\isamarkupfalse%
\ {\isachardoublequoteopen}{\isasymforall}F\ {\isasymin}\ {\isacharparenleft}{\isacharbraceleft}G{\isacharbraceright}\ {\isasymunion}\ {\isacharbraceleft}H{\isacharbraceright}{\isacharparenright}\ {\isasymunion}\ {\isacharparenleft}{\isacharbraceleft}F{\isacharbraceright}\ {\isasymunion}\ Wo{\isacharparenright}{\isachardot}\ {\isasymA}\ {\isasymTurnstile}\ F{\isachardoublequoteclose}\isanewline
\ \ \ \ \isacommand{using}\isamarkupfalse%
\ Forall{\isadigit{1}}\ {\isadigit{1}}\ {\isadigit{2}}\ \isacommand{by}\isamarkupfalse%
\ {\isacharparenleft}iprover\ intro{\isacharcolon}\ ball{\isacharunderscore}Un{\isacharparenright}\isanewline
\ \ \isacommand{then}\isamarkupfalse%
\ \isacommand{have}\isamarkupfalse%
\ {\isachardoublequoteopen}{\isasymforall}F\ {\isasymin}\ {\isacharparenleft}{\isacharbraceleft}G{\isacharcomma}H{\isacharcomma}F{\isacharbraceright}\ {\isasymunion}\ Wo{\isacharparenright}{\isachardot}\ {\isasymA}\ {\isasymTurnstile}\ F{\isachardoublequoteclose}\isanewline
\ \ \ \ \isacommand{by}\isamarkupfalse%
\ simp\isanewline
\ \ \isacommand{then}\isamarkupfalse%
\ \isacommand{have}\isamarkupfalse%
\ {\isachardoublequoteopen}{\isasymexists}{\isasymA}{\isachardot}\ {\isasymforall}F\ {\isasymin}\ {\isacharparenleft}{\isacharbraceleft}G{\isacharcomma}H{\isacharcomma}F{\isacharbraceright}\ {\isasymunion}\ Wo{\isacharparenright}{\isachardot}\ {\isasymA}\ {\isasymTurnstile}\ F{\isachardoublequoteclose}\isanewline
\ \ \ \ \isacommand{by}\isamarkupfalse%
\ {\isacharparenleft}iprover\ intro{\isacharcolon}\ exI{\isacharparenright}\isanewline
\ \ \isacommand{thus}\isamarkupfalse%
\ {\isachardoublequoteopen}sat\ {\isacharparenleft}{\isacharbraceleft}G{\isacharcomma}H{\isacharcomma}F{\isacharbraceright}\ {\isasymunion}\ Wo{\isacharparenright}{\isachardoublequoteclose}\isanewline
\ \ \ \ \isacommand{by}\isamarkupfalse%
\ {\isacharparenleft}simp\ only{\isacharcolon}\ sat{\isacharunderscore}def{\isacharparenright}\isanewline
\isacommand{qed}\isamarkupfalse%
%
\endisatagproof
{\isafoldproof}%
%
\isadelimproof
%
\endisadelimproof
%
\begin{isamarkuptext}%
A continuación veamos la prueba detallada del resultado para el segundo caso de fórmula de 
  tipo \isa{{\isasymalpha}}: dados \isa{W\ {\isasymin}\ C}, una fórmula \isa{F\ {\isacharequal}\ {\isasymnot}{\isacharparenleft}G\ {\isasymor}\ H{\isacharparenright}} para ciertas fórmulas \isa{G} y \isa{H} tal que 
  \isa{F\ {\isasymin}\ W} y \isa{W\isactrlsub {\isadigit{0}}} un subconjunto finito de \isa{W}, se verifica que \isa{{\isacharbraceleft}{\isasymnot}\ G{\isacharcomma}{\isasymnot}\ H{\isacharcomma}F{\isacharbraceright}\ {\isasymunion}\ W\isactrlsub {\isadigit{0}}} es satisfacible.%
\end{isamarkuptext}\isamarkuptrue%
\isacommand{lemma}\isamarkupfalse%
\ pcp{\isacharunderscore}colecComp{\isacharunderscore}CON{\isacharunderscore}sat{\isadigit{2}}{\isacharcolon}\isanewline
\ \ \isakeyword{assumes}\ {\isachardoublequoteopen}W\ {\isasymin}\ colecComp{\isachardoublequoteclose}\isanewline
\ \ \ \ \ \ \ \ \ \ {\isachardoublequoteopen}F\ {\isacharequal}\ \isactrlbold {\isasymnot}{\isacharparenleft}G\ \isactrlbold {\isasymor}\ H{\isacharparenright}{\isachardoublequoteclose}\isanewline
\ \ \ \ \ \ \ \ \ \ {\isachardoublequoteopen}F\ {\isasymin}\ W{\isachardoublequoteclose}\isanewline
\ \ \ \ \ \ \ \ \ \ {\isachardoublequoteopen}finite\ Wo{\isachardoublequoteclose}\isanewline
\ \ \ \ \ \ \ \ \ \ {\isachardoublequoteopen}Wo\ {\isasymsubseteq}\ W{\isachardoublequoteclose}\isanewline
\ \ \ \ \ \ \ \ \isakeyword{shows}\ {\isachardoublequoteopen}sat\ {\isacharparenleft}{\isacharbraceleft}\isactrlbold {\isasymnot}\ G{\isacharcomma}\isactrlbold {\isasymnot}\ H{\isacharcomma}F{\isacharbraceright}\ {\isasymunion}\ Wo{\isacharparenright}{\isachardoublequoteclose}\isanewline
%
\isadelimproof
%
\endisadelimproof
%
\isatagproof
\isacommand{proof}\isamarkupfalse%
\ {\isacharminus}\isanewline
\ \ \isacommand{have}\isamarkupfalse%
\ {\isachardoublequoteopen}sat\ {\isacharparenleft}{\isacharbraceleft}F{\isacharbraceright}\ {\isasymunion}\ Wo{\isacharparenright}{\isachardoublequoteclose}\isanewline
\ \ \ \ \isacommand{using}\isamarkupfalse%
\ assms{\isacharparenleft}{\isadigit{1}}{\isacharcomma}{\isadigit{3}}{\isacharcomma}{\isadigit{4}}{\isacharcomma}{\isadigit{5}}{\isacharparenright}\ \isacommand{by}\isamarkupfalse%
\ {\isacharparenleft}rule\ pcp{\isacharunderscore}colecComp{\isacharunderscore}elem{\isacharunderscore}sat{\isacharparenright}\isanewline
\ \ \isacommand{have}\isamarkupfalse%
\ {\isachardoublequoteopen}F\ {\isasymin}\ {\isacharbraceleft}F{\isacharbraceright}\ {\isasymunion}\ Wo{\isachardoublequoteclose}\isanewline
\ \ \ \ \isacommand{by}\isamarkupfalse%
\ {\isacharparenleft}simp\ add{\isacharcolon}\ insertI{\isadigit{1}}{\isacharparenright}\isanewline
\ \ \isacommand{have}\isamarkupfalse%
\ Ex{\isadigit{1}}{\isacharcolon}{\isachardoublequoteopen}{\isasymexists}{\isasymA}{\isachardot}\ {\isasymforall}F\ {\isasymin}\ {\isacharparenleft}{\isacharbraceleft}F{\isacharbraceright}\ {\isasymunion}\ Wo{\isacharparenright}{\isachardot}\ {\isasymA}\ {\isasymTurnstile}\ F{\isachardoublequoteclose}\isanewline
\ \ \ \ \isacommand{using}\isamarkupfalse%
\ {\isacartoucheopen}sat\ {\isacharparenleft}{\isacharbraceleft}F{\isacharbraceright}\ {\isasymunion}\ Wo{\isacharparenright}{\isacartoucheclose}\ \isacommand{by}\isamarkupfalse%
\ {\isacharparenleft}simp\ only{\isacharcolon}\ sat{\isacharunderscore}def{\isacharparenright}\isanewline
\ \ \isacommand{obtain}\isamarkupfalse%
\ {\isasymA}\ \isakeyword{where}\ Forall{\isadigit{1}}{\isacharcolon}{\isachardoublequoteopen}{\isasymforall}F\ {\isasymin}\ {\isacharparenleft}{\isacharbraceleft}F{\isacharbraceright}\ {\isasymunion}\ Wo{\isacharparenright}{\isachardot}\ {\isasymA}\ {\isasymTurnstile}\ F{\isachardoublequoteclose}\isanewline
\ \ \ \ \isacommand{using}\isamarkupfalse%
\ Ex{\isadigit{1}}\ \isacommand{by}\isamarkupfalse%
\ {\isacharparenleft}rule\ exE{\isacharparenright}\isanewline
\ \ \isacommand{have}\isamarkupfalse%
\ {\isachardoublequoteopen}{\isasymA}\ {\isasymTurnstile}\ F{\isachardoublequoteclose}\isanewline
\ \ \ \ \isacommand{using}\isamarkupfalse%
\ Forall{\isadigit{1}}\ {\isacartoucheopen}F\ {\isasymin}\ {\isacharbraceleft}F{\isacharbraceright}\ {\isasymunion}\ Wo{\isacartoucheclose}\ \isacommand{by}\isamarkupfalse%
\ {\isacharparenleft}rule\ bspec{\isacharparenright}\isanewline
\ \ \isacommand{then}\isamarkupfalse%
\ \isacommand{have}\isamarkupfalse%
\ {\isachardoublequoteopen}{\isasymA}\ {\isasymTurnstile}\ \isactrlbold {\isasymnot}{\isacharparenleft}G\ \isactrlbold {\isasymor}\ H{\isacharparenright}{\isachardoublequoteclose}\isanewline
\ \ \ \ \isacommand{using}\isamarkupfalse%
\ assms{\isacharparenleft}{\isadigit{2}}{\isacharparenright}\ \isacommand{by}\isamarkupfalse%
\ {\isacharparenleft}simp\ only{\isacharcolon}\ {\isacartoucheopen}{\isasymA}\ {\isasymTurnstile}\ F{\isacartoucheclose}{\isacharparenright}\isanewline
\ \ \isacommand{then}\isamarkupfalse%
\ \isacommand{have}\isamarkupfalse%
\ {\isachardoublequoteopen}{\isasymnot}{\isacharparenleft}{\isasymA}\ {\isasymTurnstile}\ {\isacharparenleft}G\ \isactrlbold {\isasymor}\ H{\isacharparenright}{\isacharparenright}{\isachardoublequoteclose}\isanewline
\ \ \ \ \isacommand{by}\isamarkupfalse%
\ {\isacharparenleft}simp\ only{\isacharcolon}\ formula{\isacharunderscore}semantics{\isachardot}simps{\isacharparenleft}{\isadigit{3}}{\isacharparenright}\ simp{\isacharunderscore}thms{\isacharparenleft}{\isadigit{8}}{\isacharparenright}{\isacharparenright}\isanewline
\ \ \isacommand{then}\isamarkupfalse%
\ \isacommand{have}\isamarkupfalse%
\ {\isachardoublequoteopen}{\isasymnot}{\isacharparenleft}{\isasymA}\ {\isasymTurnstile}\ G\ {\isasymor}\ {\isasymA}\ {\isasymTurnstile}\ H{\isacharparenright}{\isachardoublequoteclose}\isanewline
\ \ \ \ \isacommand{by}\isamarkupfalse%
\ {\isacharparenleft}simp\ only{\isacharcolon}\ formula{\isacharunderscore}semantics{\isachardot}simps{\isacharparenleft}{\isadigit{5}}{\isacharparenright}\ simp{\isacharunderscore}thms{\isacharparenleft}{\isadigit{8}}{\isacharparenright}{\isacharparenright}\isanewline
\ \ \isacommand{then}\isamarkupfalse%
\ \isacommand{have}\isamarkupfalse%
\ {\isachardoublequoteopen}{\isasymnot}\ {\isasymA}\ {\isasymTurnstile}\ G\ {\isasymand}\ {\isasymnot}\ {\isasymA}\ {\isasymTurnstile}\ H{\isachardoublequoteclose}\ \isanewline
\ \ \ \ \isacommand{by}\isamarkupfalse%
\ {\isacharparenleft}simp\ only{\isacharcolon}\ de{\isacharunderscore}Morgan{\isacharunderscore}disj\ simp{\isacharunderscore}thms{\isacharparenleft}{\isadigit{8}}{\isacharparenright}{\isacharparenright}\isanewline
\ \ \isacommand{then}\isamarkupfalse%
\ \isacommand{have}\isamarkupfalse%
\ {\isachardoublequoteopen}{\isasymA}\ {\isasymTurnstile}\ \isactrlbold {\isasymnot}\ G\ {\isasymand}\ {\isasymA}\ {\isasymTurnstile}\ \isactrlbold {\isasymnot}\ H{\isachardoublequoteclose}\isanewline
\ \ \ \ \isacommand{by}\isamarkupfalse%
\ {\isacharparenleft}simp\ only{\isacharcolon}\ formula{\isacharunderscore}semantics{\isachardot}simps{\isacharparenleft}{\isadigit{3}}{\isacharparenright}\ simp{\isacharunderscore}thms{\isacharparenleft}{\isadigit{8}}{\isacharparenright}{\isacharparenright}\ \isanewline
\ \ \isacommand{then}\isamarkupfalse%
\ \isacommand{have}\isamarkupfalse%
\ {\isachardoublequoteopen}{\isasymA}\ {\isasymTurnstile}\ \isactrlbold {\isasymnot}\ G{\isachardoublequoteclose}\isanewline
\ \ \ \ \isacommand{by}\isamarkupfalse%
\ {\isacharparenleft}rule\ conjunct{\isadigit{1}}{\isacharparenright}\isanewline
\ \ \isacommand{then}\isamarkupfalse%
\ \isacommand{have}\isamarkupfalse%
\ {\isadigit{1}}{\isacharcolon}{\isachardoublequoteopen}{\isasymforall}F\ {\isasymin}\ {\isacharbraceleft}\isactrlbold {\isasymnot}\ G{\isacharbraceright}{\isachardot}\ {\isasymA}\ {\isasymTurnstile}\ F{\isachardoublequoteclose}\isanewline
\ \ \ \ \isacommand{by}\isamarkupfalse%
\ simp\isanewline
\ \ \isacommand{have}\isamarkupfalse%
\ {\isachardoublequoteopen}{\isasymA}\ {\isasymTurnstile}\ \isactrlbold {\isasymnot}\ H{\isachardoublequoteclose}\isanewline
\ \ \ \ \isacommand{using}\isamarkupfalse%
\ {\isacartoucheopen}{\isasymA}\ {\isasymTurnstile}\ \isactrlbold {\isasymnot}\ G\ {\isasymand}\ {\isasymA}\ {\isasymTurnstile}\ \isactrlbold {\isasymnot}\ H{\isacartoucheclose}\ \isacommand{by}\isamarkupfalse%
\ {\isacharparenleft}rule\ conjunct{\isadigit{2}}{\isacharparenright}\isanewline
\ \ \isacommand{then}\isamarkupfalse%
\ \isacommand{have}\isamarkupfalse%
\ {\isadigit{2}}{\isacharcolon}{\isachardoublequoteopen}{\isasymforall}F\ {\isasymin}\ {\isacharbraceleft}\isactrlbold {\isasymnot}\ H{\isacharbraceright}{\isachardot}\ {\isasymA}\ {\isasymTurnstile}\ F{\isachardoublequoteclose}\isanewline
\ \ \ \ \isacommand{by}\isamarkupfalse%
\ simp\isanewline
\ \ \isacommand{have}\isamarkupfalse%
\ {\isachardoublequoteopen}{\isasymforall}F\ {\isasymin}\ {\isacharparenleft}{\isacharbraceleft}\isactrlbold {\isasymnot}\ G{\isacharbraceright}\ {\isasymunion}\ {\isacharbraceleft}\isactrlbold {\isasymnot}\ H{\isacharbraceright}{\isacharparenright}\ {\isasymunion}\ {\isacharparenleft}{\isacharbraceleft}F{\isacharbraceright}\ {\isasymunion}\ Wo{\isacharparenright}{\isachardot}\ {\isasymA}\ {\isasymTurnstile}\ F{\isachardoublequoteclose}\isanewline
\ \ \ \ \isacommand{using}\isamarkupfalse%
\ Forall{\isadigit{1}}\ {\isadigit{1}}\ {\isadigit{2}}\ \isacommand{by}\isamarkupfalse%
\ {\isacharparenleft}iprover\ intro{\isacharcolon}\ ball{\isacharunderscore}Un{\isacharparenright}\isanewline
\ \ \isacommand{then}\isamarkupfalse%
\ \isacommand{have}\isamarkupfalse%
\ {\isachardoublequoteopen}{\isasymforall}F\ {\isasymin}\ {\isacharparenleft}{\isacharbraceleft}\isactrlbold {\isasymnot}\ G{\isacharcomma}\isactrlbold {\isasymnot}\ H{\isacharcomma}\ F{\isacharbraceright}\ {\isasymunion}\ Wo{\isacharparenright}{\isachardot}\ {\isasymA}\ {\isasymTurnstile}\ F{\isachardoublequoteclose}\isanewline
\ \ \ \ \isacommand{by}\isamarkupfalse%
\ simp\isanewline
\ \ \isacommand{then}\isamarkupfalse%
\ \isacommand{have}\isamarkupfalse%
\ {\isachardoublequoteopen}{\isasymexists}{\isasymA}{\isachardot}\ {\isasymforall}F\ {\isasymin}\ {\isacharparenleft}{\isacharbraceleft}\isactrlbold {\isasymnot}\ G{\isacharcomma}\isactrlbold {\isasymnot}\ H{\isacharcomma}F{\isacharbraceright}\ {\isasymunion}\ Wo{\isacharparenright}{\isachardot}\ {\isasymA}\ {\isasymTurnstile}\ F{\isachardoublequoteclose}\isanewline
\ \ \ \ \isacommand{by}\isamarkupfalse%
\ {\isacharparenleft}iprover\ intro{\isacharcolon}\ exI{\isacharparenright}\isanewline
\ \ \isacommand{thus}\isamarkupfalse%
\ {\isachardoublequoteopen}sat\ {\isacharparenleft}{\isacharbraceleft}\isactrlbold {\isasymnot}\ G{\isacharcomma}\isactrlbold {\isasymnot}\ H{\isacharcomma}F{\isacharbraceright}\ {\isasymunion}\ Wo{\isacharparenright}{\isachardoublequoteclose}\isanewline
\ \ \ \ \isacommand{by}\isamarkupfalse%
\ {\isacharparenleft}simp\ only{\isacharcolon}\ sat{\isacharunderscore}def{\isacharparenright}\isanewline
\isacommand{qed}\isamarkupfalse%
%
\endisatagproof
{\isafoldproof}%
%
\isadelimproof
%
\endisadelimproof
%
\begin{isamarkuptext}%
Probemos detalladamente el resultado para el tercer caso de fórmula de tipo \isa{{\isasymalpha}}: dados 
  \isa{W\ {\isasymin}\ C}, una fórmula \isa{F\ {\isacharequal}\ {\isasymnot}{\isacharparenleft}G\ {\isasymlongrightarrow}\ H{\isacharparenright}} para ciertas fórmulas \isa{G} y \isa{H} tal que \isa{F\ {\isasymin}\ W} y \isa{W\isactrlsub {\isadigit{0}}} un 
  subconjunto finito de \isa{W}, se verifica que \isa{{\isacharbraceleft}G{\isacharcomma}{\isasymnot}\ H{\isacharcomma}F{\isacharbraceright}\ {\isasymunion}\ W\isactrlsub {\isadigit{0}}} es satisfacible.%
\end{isamarkuptext}\isamarkuptrue%
\isacommand{lemma}\isamarkupfalse%
\ pcp{\isacharunderscore}colecComp{\isacharunderscore}CON{\isacharunderscore}sat{\isadigit{3}}{\isacharcolon}\isanewline
\ \ \isakeyword{assumes}\ {\isachardoublequoteopen}W\ {\isasymin}\ colecComp{\isachardoublequoteclose}\isanewline
\ \ \ \ \ \ \ \ \ \ {\isachardoublequoteopen}F\ {\isacharequal}\ \isactrlbold {\isasymnot}\ {\isacharparenleft}G\ \isactrlbold {\isasymrightarrow}\ H{\isacharparenright}{\isachardoublequoteclose}\isanewline
\ \ \ \ \ \ \ \ \ \ {\isachardoublequoteopen}F\ {\isasymin}\ W{\isachardoublequoteclose}\isanewline
\ \ \ \ \ \ \ \ \ \ {\isachardoublequoteopen}finite\ Wo{\isachardoublequoteclose}\isanewline
\ \ \ \ \ \ \ \ \ \ {\isachardoublequoteopen}Wo\ {\isasymsubseteq}\ W{\isachardoublequoteclose}\isanewline
\ \ \ \ \ \ \ \ \isakeyword{shows}\ {\isachardoublequoteopen}sat\ {\isacharparenleft}{\isacharbraceleft}G{\isacharcomma}\isactrlbold {\isasymnot}\ H{\isacharcomma}F{\isacharbraceright}\ {\isasymunion}\ Wo{\isacharparenright}{\isachardoublequoteclose}\isanewline
%
\isadelimproof
%
\endisadelimproof
%
\isatagproof
\isacommand{proof}\isamarkupfalse%
\ {\isacharminus}\isanewline
\ \ \isacommand{have}\isamarkupfalse%
\ {\isachardoublequoteopen}sat\ {\isacharparenleft}{\isacharbraceleft}F{\isacharbraceright}\ {\isasymunion}\ Wo{\isacharparenright}{\isachardoublequoteclose}\isanewline
\ \ \ \ \isacommand{using}\isamarkupfalse%
\ assms{\isacharparenleft}{\isadigit{1}}{\isacharcomma}{\isadigit{3}}{\isacharcomma}{\isadigit{4}}{\isacharcomma}{\isadigit{5}}{\isacharparenright}\ \isacommand{by}\isamarkupfalse%
\ {\isacharparenleft}rule\ pcp{\isacharunderscore}colecComp{\isacharunderscore}elem{\isacharunderscore}sat{\isacharparenright}\isanewline
\ \ \isacommand{have}\isamarkupfalse%
\ {\isachardoublequoteopen}F\ {\isasymin}\ {\isacharbraceleft}F{\isacharbraceright}\ {\isasymunion}\ Wo{\isachardoublequoteclose}\isanewline
\ \ \ \ \isacommand{by}\isamarkupfalse%
\ {\isacharparenleft}simp\ add{\isacharcolon}\ insertI{\isadigit{1}}{\isacharparenright}\isanewline
\ \ \isacommand{have}\isamarkupfalse%
\ Ex{\isadigit{1}}{\isacharcolon}{\isachardoublequoteopen}{\isasymexists}{\isasymA}{\isachardot}\ {\isasymforall}F\ {\isasymin}\ {\isacharparenleft}{\isacharbraceleft}F{\isacharbraceright}\ {\isasymunion}\ Wo{\isacharparenright}{\isachardot}\ {\isasymA}\ {\isasymTurnstile}\ F{\isachardoublequoteclose}\isanewline
\ \ \ \ \isacommand{using}\isamarkupfalse%
\ {\isacartoucheopen}sat\ {\isacharparenleft}{\isacharbraceleft}F{\isacharbraceright}\ {\isasymunion}\ Wo{\isacharparenright}{\isacartoucheclose}\ \isacommand{by}\isamarkupfalse%
\ {\isacharparenleft}simp\ only{\isacharcolon}\ sat{\isacharunderscore}def{\isacharparenright}\isanewline
\ \ \isacommand{obtain}\isamarkupfalse%
\ {\isasymA}\ \isakeyword{where}\ Forall{\isadigit{1}}{\isacharcolon}{\isachardoublequoteopen}{\isasymforall}F\ {\isasymin}\ {\isacharparenleft}{\isacharbraceleft}F{\isacharbraceright}\ {\isasymunion}\ Wo{\isacharparenright}{\isachardot}\ {\isasymA}\ {\isasymTurnstile}\ F{\isachardoublequoteclose}\isanewline
\ \ \ \ \isacommand{using}\isamarkupfalse%
\ Ex{\isadigit{1}}\ \isacommand{by}\isamarkupfalse%
\ {\isacharparenleft}rule\ exE{\isacharparenright}\isanewline
\ \ \isacommand{have}\isamarkupfalse%
\ {\isachardoublequoteopen}{\isasymA}\ {\isasymTurnstile}\ F{\isachardoublequoteclose}\isanewline
\ \ \ \ \isacommand{using}\isamarkupfalse%
\ Forall{\isadigit{1}}\ {\isacartoucheopen}F\ {\isasymin}\ {\isacharbraceleft}F{\isacharbraceright}\ {\isasymunion}\ Wo{\isacartoucheclose}\ \isacommand{by}\isamarkupfalse%
\ {\isacharparenleft}rule\ bspec{\isacharparenright}\isanewline
\ \ \isacommand{then}\isamarkupfalse%
\ \isacommand{have}\isamarkupfalse%
\ {\isachardoublequoteopen}{\isasymA}\ {\isasymTurnstile}\ \isactrlbold {\isasymnot}{\isacharparenleft}G\ \isactrlbold {\isasymrightarrow}\ H{\isacharparenright}{\isachardoublequoteclose}\isanewline
\ \ \ \ \isacommand{using}\isamarkupfalse%
\ assms{\isacharparenleft}{\isadigit{2}}{\isacharparenright}\ \isacommand{by}\isamarkupfalse%
\ {\isacharparenleft}simp\ only{\isacharcolon}\ {\isacartoucheopen}{\isasymA}\ {\isasymTurnstile}\ F{\isacartoucheclose}{\isacharparenright}\isanewline
\ \ \isacommand{then}\isamarkupfalse%
\ \isacommand{have}\isamarkupfalse%
\ {\isachardoublequoteopen}{\isasymnot}{\isacharparenleft}{\isasymA}\ {\isasymTurnstile}\ {\isacharparenleft}G\ \isactrlbold {\isasymrightarrow}\ H{\isacharparenright}{\isacharparenright}{\isachardoublequoteclose}\isanewline
\ \ \ \ \isacommand{by}\isamarkupfalse%
\ {\isacharparenleft}simp\ only{\isacharcolon}\ formula{\isacharunderscore}semantics{\isachardot}simps{\isacharparenleft}{\isadigit{3}}{\isacharparenright}\ simp{\isacharunderscore}thms{\isacharparenleft}{\isadigit{8}}{\isacharparenright}{\isacharparenright}\isanewline
\ \ \isacommand{then}\isamarkupfalse%
\ \isacommand{have}\isamarkupfalse%
\ {\isachardoublequoteopen}{\isasymnot}{\isacharparenleft}{\isasymA}\ {\isasymTurnstile}\ G\ {\isasymlongrightarrow}\ {\isasymA}\ {\isasymTurnstile}\ H{\isacharparenright}{\isachardoublequoteclose}\isanewline
\ \ \ \ \isacommand{by}\isamarkupfalse%
\ {\isacharparenleft}simp\ only{\isacharcolon}\ formula{\isacharunderscore}semantics{\isachardot}simps{\isacharparenleft}{\isadigit{6}}{\isacharparenright}\ simp{\isacharunderscore}thms{\isacharparenleft}{\isadigit{8}}{\isacharparenright}{\isacharparenright}\isanewline
\ \ \isacommand{then}\isamarkupfalse%
\ \isacommand{have}\isamarkupfalse%
\ {\isachardoublequoteopen}{\isasymA}\ {\isasymTurnstile}\ G\ {\isasymand}\ {\isasymnot}\ {\isasymA}\ {\isasymTurnstile}\ H{\isachardoublequoteclose}\isanewline
\ \ \ \ \isacommand{by}\isamarkupfalse%
\ {\isacharparenleft}simp\ only{\isacharcolon}\ not{\isacharunderscore}imp\ simp{\isacharunderscore}thms{\isacharparenleft}{\isadigit{8}}{\isacharparenright}{\isacharparenright}\isanewline
\ \ \isacommand{then}\isamarkupfalse%
\ \isacommand{have}\isamarkupfalse%
\ {\isachardoublequoteopen}{\isasymA}\ {\isasymTurnstile}\ G\ {\isasymand}\ {\isasymA}\ {\isasymTurnstile}\ \isactrlbold {\isasymnot}\ H{\isachardoublequoteclose}\isanewline
\ \ \ \ \isacommand{by}\isamarkupfalse%
\ {\isacharparenleft}simp\ only{\isacharcolon}\ formula{\isacharunderscore}semantics{\isachardot}simps{\isacharparenleft}{\isadigit{3}}{\isacharparenright}\ simp{\isacharunderscore}thms{\isacharparenleft}{\isadigit{8}}{\isacharparenright}{\isacharparenright}\ \isanewline
\ \ \isacommand{then}\isamarkupfalse%
\ \isacommand{have}\isamarkupfalse%
\ {\isachardoublequoteopen}{\isasymA}\ {\isasymTurnstile}\ G{\isachardoublequoteclose}\isanewline
\ \ \ \ \isacommand{by}\isamarkupfalse%
\ {\isacharparenleft}rule\ conjunct{\isadigit{1}}{\isacharparenright}\isanewline
\ \ \isacommand{then}\isamarkupfalse%
\ \isacommand{have}\isamarkupfalse%
\ {\isadigit{1}}{\isacharcolon}{\isachardoublequoteopen}{\isasymforall}F\ {\isasymin}\ {\isacharbraceleft}G{\isacharbraceright}{\isachardot}\ {\isasymA}\ {\isasymTurnstile}\ F{\isachardoublequoteclose}\isanewline
\ \ \ \ \isacommand{by}\isamarkupfalse%
\ simp\isanewline
\ \ \isacommand{have}\isamarkupfalse%
\ {\isachardoublequoteopen}{\isasymA}\ {\isasymTurnstile}\ \isactrlbold {\isasymnot}\ H{\isachardoublequoteclose}\isanewline
\ \ \ \ \isacommand{using}\isamarkupfalse%
\ {\isacartoucheopen}{\isasymA}\ {\isasymTurnstile}\ G\ {\isasymand}\ {\isasymA}\ {\isasymTurnstile}\ \isactrlbold {\isasymnot}\ H{\isacartoucheclose}\ \isacommand{by}\isamarkupfalse%
\ {\isacharparenleft}rule\ conjunct{\isadigit{2}}{\isacharparenright}\isanewline
\ \ \isacommand{then}\isamarkupfalse%
\ \isacommand{have}\isamarkupfalse%
\ {\isadigit{2}}{\isacharcolon}{\isachardoublequoteopen}{\isasymforall}F\ {\isasymin}\ {\isacharbraceleft}\isactrlbold {\isasymnot}\ H{\isacharbraceright}{\isachardot}\ {\isasymA}\ {\isasymTurnstile}\ F{\isachardoublequoteclose}\isanewline
\ \ \ \ \isacommand{by}\isamarkupfalse%
\ simp\isanewline
\ \ \isacommand{have}\isamarkupfalse%
\ {\isachardoublequoteopen}{\isasymforall}F\ {\isasymin}\ {\isacharparenleft}{\isacharbraceleft}G{\isacharbraceright}\ {\isasymunion}\ {\isacharbraceleft}\isactrlbold {\isasymnot}\ H{\isacharbraceright}{\isacharparenright}\ {\isasymunion}\ {\isacharparenleft}{\isacharbraceleft}F{\isacharbraceright}\ {\isasymunion}\ Wo{\isacharparenright}{\isachardot}\ {\isasymA}\ {\isasymTurnstile}\ F{\isachardoublequoteclose}\isanewline
\ \ \ \ \isacommand{using}\isamarkupfalse%
\ Forall{\isadigit{1}}\ {\isadigit{1}}\ {\isadigit{2}}\ \isacommand{by}\isamarkupfalse%
\ {\isacharparenleft}iprover\ intro{\isacharcolon}\ ball{\isacharunderscore}Un{\isacharparenright}\isanewline
\ \ \isacommand{then}\isamarkupfalse%
\ \isacommand{have}\isamarkupfalse%
\ {\isachardoublequoteopen}{\isasymforall}F\ {\isasymin}\ {\isacharbraceleft}G{\isacharcomma}\isactrlbold {\isasymnot}\ H{\isacharcomma}F{\isacharbraceright}\ {\isasymunion}\ Wo{\isachardot}\ {\isasymA}\ {\isasymTurnstile}\ F{\isachardoublequoteclose}\isanewline
\ \ \ \ \isacommand{by}\isamarkupfalse%
\ simp\isanewline
\ \ \isacommand{then}\isamarkupfalse%
\ \isacommand{have}\isamarkupfalse%
\ {\isachardoublequoteopen}{\isasymexists}{\isasymA}{\isachardot}\ {\isasymforall}F\ {\isasymin}\ {\isacharparenleft}{\isacharbraceleft}G{\isacharcomma}\isactrlbold {\isasymnot}\ H{\isacharcomma}F{\isacharbraceright}\ {\isasymunion}\ Wo{\isacharparenright}{\isachardot}\ {\isasymA}\ {\isasymTurnstile}\ F{\isachardoublequoteclose}\isanewline
\ \ \ \ \isacommand{by}\isamarkupfalse%
\ {\isacharparenleft}iprover\ intro{\isacharcolon}\ exI{\isacharparenright}\isanewline
\ \ \isacommand{thus}\isamarkupfalse%
\ {\isachardoublequoteopen}sat\ {\isacharparenleft}{\isacharbraceleft}G{\isacharcomma}\isactrlbold {\isasymnot}\ H{\isacharcomma}F{\isacharbraceright}\ {\isasymunion}\ Wo{\isacharparenright}{\isachardoublequoteclose}\isanewline
\ \ \ \ \isacommand{by}\isamarkupfalse%
\ {\isacharparenleft}simp\ only{\isacharcolon}\ sat{\isacharunderscore}def{\isacharparenright}\isanewline
\isacommand{qed}\isamarkupfalse%
%
\endisatagproof
{\isafoldproof}%
%
\isadelimproof
%
\endisadelimproof
%
\begin{isamarkuptext}%
Por último probemos que dados \isa{W\ {\isasymin}\ C}, una fórmula \isa{F\ {\isacharequal}\ {\isasymnot}{\isacharparenleft}{\isasymnot}\ G{\isacharparenright}} para cierta fórmula \isa{G} tal 
  que \isa{F\ {\isasymin}\ W} y \isa{W\isactrlsub {\isadigit{0}}} un subconjunto finito de \isa{W}, se verifica que \isa{{\isacharbraceleft}G{\isacharcomma}F{\isacharbraceright}\ {\isasymunion}\ W\isactrlsub {\isadigit{0}}} es satisfacible.%
\end{isamarkuptext}\isamarkuptrue%
\isacommand{lemma}\isamarkupfalse%
\ pcp{\isacharunderscore}colecComp{\isacharunderscore}CON{\isacharunderscore}sat{\isadigit{4}}{\isacharcolon}\isanewline
\ \ \isakeyword{assumes}\ {\isachardoublequoteopen}W\ {\isasymin}\ colecComp{\isachardoublequoteclose}\isanewline
\ \ \ \ \ \ \ \ \ \ {\isachardoublequoteopen}F\ {\isacharequal}\ \isactrlbold {\isasymnot}\ {\isacharparenleft}\isactrlbold {\isasymnot}\ G{\isacharparenright}{\isachardoublequoteclose}\isanewline
\ \ \ \ \ \ \ \ \ \ {\isachardoublequoteopen}F\ {\isasymin}\ W{\isachardoublequoteclose}\isanewline
\ \ \ \ \ \ \ \ \ \ {\isachardoublequoteopen}finite\ Wo{\isachardoublequoteclose}\isanewline
\ \ \ \ \ \ \ \ \ \ {\isachardoublequoteopen}Wo\ {\isasymsubseteq}\ W{\isachardoublequoteclose}\isanewline
\ \ \ \ \ \ \ \ \isakeyword{shows}\ {\isachardoublequoteopen}sat\ {\isacharparenleft}{\isacharbraceleft}G{\isacharcomma}F{\isacharbraceright}\ {\isasymunion}\ Wo{\isacharparenright}{\isachardoublequoteclose}\isanewline
%
\isadelimproof
%
\endisadelimproof
%
\isatagproof
\isacommand{proof}\isamarkupfalse%
\ {\isacharminus}\isanewline
\ \ \isacommand{have}\isamarkupfalse%
\ {\isachardoublequoteopen}sat\ {\isacharparenleft}{\isacharbraceleft}F{\isacharbraceright}\ {\isasymunion}\ Wo{\isacharparenright}{\isachardoublequoteclose}\isanewline
\ \ \ \ \isacommand{using}\isamarkupfalse%
\ assms{\isacharparenleft}{\isadigit{1}}{\isacharcomma}{\isadigit{3}}{\isacharcomma}{\isadigit{4}}{\isacharcomma}{\isadigit{5}}{\isacharparenright}\ \isacommand{by}\isamarkupfalse%
\ {\isacharparenleft}rule\ pcp{\isacharunderscore}colecComp{\isacharunderscore}elem{\isacharunderscore}sat{\isacharparenright}\isanewline
\ \ \isacommand{have}\isamarkupfalse%
\ {\isachardoublequoteopen}F\ {\isasymin}\ {\isacharbraceleft}F{\isacharbraceright}\ {\isasymunion}\ Wo{\isachardoublequoteclose}\isanewline
\ \ \ \ \isacommand{by}\isamarkupfalse%
\ {\isacharparenleft}simp\ add{\isacharcolon}\ insertI{\isadigit{1}}{\isacharparenright}\isanewline
\ \ \isacommand{have}\isamarkupfalse%
\ Ex{\isadigit{1}}{\isacharcolon}{\isachardoublequoteopen}{\isasymexists}{\isasymA}{\isachardot}\ {\isasymforall}F\ {\isasymin}\ {\isacharparenleft}{\isacharbraceleft}F{\isacharbraceright}\ {\isasymunion}\ Wo{\isacharparenright}{\isachardot}\ {\isasymA}\ {\isasymTurnstile}\ F{\isachardoublequoteclose}\isanewline
\ \ \ \ \isacommand{using}\isamarkupfalse%
\ {\isacartoucheopen}sat\ {\isacharparenleft}{\isacharbraceleft}F{\isacharbraceright}\ {\isasymunion}\ Wo{\isacharparenright}{\isacartoucheclose}\ \isacommand{by}\isamarkupfalse%
\ {\isacharparenleft}simp\ only{\isacharcolon}\ sat{\isacharunderscore}def{\isacharparenright}\isanewline
\ \ \isacommand{obtain}\isamarkupfalse%
\ {\isasymA}\ \isakeyword{where}\ Forall{\isadigit{1}}{\isacharcolon}{\isachardoublequoteopen}{\isasymforall}F\ {\isasymin}\ {\isacharparenleft}{\isacharbraceleft}F{\isacharbraceright}\ {\isasymunion}\ Wo{\isacharparenright}{\isachardot}\ {\isasymA}\ {\isasymTurnstile}\ F{\isachardoublequoteclose}\isanewline
\ \ \ \ \isacommand{using}\isamarkupfalse%
\ Ex{\isadigit{1}}\ \isacommand{by}\isamarkupfalse%
\ {\isacharparenleft}rule\ exE{\isacharparenright}\isanewline
\ \ \isacommand{have}\isamarkupfalse%
\ {\isachardoublequoteopen}{\isasymA}\ {\isasymTurnstile}\ F{\isachardoublequoteclose}\isanewline
\ \ \ \ \isacommand{using}\isamarkupfalse%
\ Forall{\isadigit{1}}\ {\isacartoucheopen}F\ {\isasymin}\ {\isacharbraceleft}F{\isacharbraceright}\ {\isasymunion}\ Wo{\isacartoucheclose}\ \isacommand{by}\isamarkupfalse%
\ {\isacharparenleft}rule\ bspec{\isacharparenright}\isanewline
\ \ \isacommand{then}\isamarkupfalse%
\ \isacommand{have}\isamarkupfalse%
\ {\isachardoublequoteopen}{\isasymA}\ {\isasymTurnstile}\ \isactrlbold {\isasymnot}{\isacharparenleft}\isactrlbold {\isasymnot}\ G{\isacharparenright}{\isachardoublequoteclose}\isanewline
\ \ \ \ \isacommand{using}\isamarkupfalse%
\ assms{\isacharparenleft}{\isadigit{2}}{\isacharparenright}\ \isacommand{by}\isamarkupfalse%
\ {\isacharparenleft}simp\ only{\isacharcolon}\ {\isacartoucheopen}{\isasymA}\ {\isasymTurnstile}\ F{\isacartoucheclose}{\isacharparenright}\isanewline
\ \ \isacommand{then}\isamarkupfalse%
\ \isacommand{have}\isamarkupfalse%
\ {\isachardoublequoteopen}{\isasymnot}\ {\isasymA}\ {\isasymTurnstile}\ \isactrlbold {\isasymnot}\ G{\isachardoublequoteclose}\isanewline
\ \ \ \ \isacommand{by}\isamarkupfalse%
\ {\isacharparenleft}simp\ only{\isacharcolon}\ formula{\isacharunderscore}semantics{\isachardot}simps{\isacharparenleft}{\isadigit{3}}{\isacharparenright}\ simp{\isacharunderscore}thms{\isacharparenleft}{\isadigit{8}}{\isacharparenright}{\isacharparenright}\isanewline
\ \ \isacommand{then}\isamarkupfalse%
\ \isacommand{have}\isamarkupfalse%
\ {\isachardoublequoteopen}{\isasymnot}\ {\isasymnot}{\isasymA}\ {\isasymTurnstile}\ G{\isachardoublequoteclose}\isanewline
\ \ \ \ \isacommand{by}\isamarkupfalse%
\ {\isacharparenleft}simp\ only{\isacharcolon}\ formula{\isacharunderscore}semantics{\isachardot}simps{\isacharparenleft}{\isadigit{3}}{\isacharparenright}\ simp{\isacharunderscore}thms{\isacharparenleft}{\isadigit{8}}{\isacharparenright}{\isacharparenright}\isanewline
\ \ \isacommand{then}\isamarkupfalse%
\ \isacommand{have}\isamarkupfalse%
\ {\isachardoublequoteopen}{\isasymA}\ {\isasymTurnstile}\ G{\isachardoublequoteclose}\isanewline
\ \ \ \ \isacommand{by}\isamarkupfalse%
\ {\isacharparenleft}rule\ notnotD{\isacharparenright}\isanewline
\ \ \isacommand{then}\isamarkupfalse%
\ \isacommand{have}\isamarkupfalse%
\ {\isadigit{1}}{\isacharcolon}{\isachardoublequoteopen}{\isasymforall}F\ {\isasymin}\ {\isacharbraceleft}G{\isacharbraceright}{\isachardot}\ {\isasymA}\ {\isasymTurnstile}\ F{\isachardoublequoteclose}\isanewline
\ \ \ \ \isacommand{by}\isamarkupfalse%
\ simp\isanewline
\ \ \isacommand{have}\isamarkupfalse%
\ {\isachardoublequoteopen}{\isasymforall}F\ {\isasymin}\ {\isacharparenleft}{\isacharbraceleft}G{\isacharbraceright}{\isacharparenright}\ {\isasymunion}\ {\isacharparenleft}{\isacharbraceleft}F{\isacharbraceright}\ {\isasymunion}\ Wo{\isacharparenright}{\isachardot}\ {\isasymA}\ {\isasymTurnstile}\ F{\isachardoublequoteclose}\isanewline
\ \ \ \ \isacommand{using}\isamarkupfalse%
\ Forall{\isadigit{1}}\ {\isadigit{1}}\ \isacommand{by}\isamarkupfalse%
\ {\isacharparenleft}iprover\ intro{\isacharcolon}\ ball{\isacharunderscore}Un{\isacharparenright}\isanewline
\ \ \isacommand{then}\isamarkupfalse%
\ \isacommand{have}\isamarkupfalse%
\ {\isachardoublequoteopen}{\isasymforall}F\ {\isasymin}\ {\isacharbraceleft}G{\isacharcomma}F{\isacharbraceright}\ {\isasymunion}\ Wo{\isachardot}\ {\isasymA}\ {\isasymTurnstile}\ F{\isachardoublequoteclose}\isanewline
\ \ \ \ \isacommand{by}\isamarkupfalse%
\ simp\isanewline
\ \ \isacommand{then}\isamarkupfalse%
\ \isacommand{have}\isamarkupfalse%
\ {\isachardoublequoteopen}{\isasymexists}{\isasymA}{\isachardot}\ {\isasymforall}F\ {\isasymin}\ {\isacharparenleft}{\isacharbraceleft}G{\isacharcomma}F{\isacharbraceright}\ {\isasymunion}\ Wo{\isacharparenright}{\isachardot}\ {\isasymA}\ {\isasymTurnstile}\ F{\isachardoublequoteclose}\isanewline
\ \ \ \ \isacommand{by}\isamarkupfalse%
\ {\isacharparenleft}iprover\ intro{\isacharcolon}\ exI{\isacharparenright}\isanewline
\ \ \isacommand{thus}\isamarkupfalse%
\ {\isachardoublequoteopen}sat\ {\isacharparenleft}{\isacharbraceleft}G{\isacharcomma}F{\isacharbraceright}\ {\isasymunion}\ Wo{\isacharparenright}{\isachardoublequoteclose}\isanewline
\ \ \ \ \isacommand{by}\isamarkupfalse%
\ {\isacharparenleft}simp\ only{\isacharcolon}\ sat{\isacharunderscore}def{\isacharparenright}\isanewline
\isacommand{qed}\isamarkupfalse%
%
\endisatagproof
{\isafoldproof}%
%
\isadelimproof
%
\endisadelimproof
%
\begin{isamarkuptext}%
Por tanto, por las pruebas detalladas de los casos anteriores, podemos demostrar que dados 
  \isa{W\ {\isasymin}\ C}, \isa{F\ {\isasymin}\ W} una fórmula de tipo \isa{{\isasymalpha}} con componentes \isa{{\isasymalpha}\isactrlsub {\isadigit{1}}} y \isa{{\isasymalpha}\isactrlsub {\isadigit{2}}} y \isa{W\isactrlsub {\isadigit{0}}} un subconjunto finito 
  de \isa{W}, se verifica que \isa{{\isacharbraceleft}{\isasymalpha}\isactrlsub {\isadigit{1}}{\isacharcomma}{\isasymalpha}\isactrlsub {\isadigit{2}}{\isacharcomma}F{\isacharbraceright}\ {\isasymunion}\ W\isactrlsub {\isadigit{0}}} es satisfacible.%
\end{isamarkuptext}\isamarkuptrue%
\isacommand{lemma}\isamarkupfalse%
\ pcp{\isacharunderscore}colecComp{\isacharunderscore}CON{\isacharunderscore}sat{\isacharcolon}\isanewline
\ \ \isakeyword{assumes}\ {\isachardoublequoteopen}W\ {\isasymin}\ colecComp{\isachardoublequoteclose}\isanewline
\ \ \ \ \ \ \ \ \ \ {\isachardoublequoteopen}Con\ F\ G\ H{\isachardoublequoteclose}\isanewline
\ \ \ \ \ \ \ \ \ \ {\isachardoublequoteopen}F\ {\isasymin}\ W{\isachardoublequoteclose}\isanewline
\ \ \ \ \ \ \ \ \ \ {\isachardoublequoteopen}finite\ Wo{\isachardoublequoteclose}\isanewline
\ \ \ \ \ \ \ \ \ \ {\isachardoublequoteopen}Wo\ {\isasymsubseteq}\ W{\isachardoublequoteclose}\isanewline
\ \ \ \ \ \ \ \ \isakeyword{shows}\ {\isachardoublequoteopen}sat\ {\isacharparenleft}{\isacharbraceleft}G{\isacharcomma}H{\isacharcomma}F{\isacharbraceright}\ {\isasymunion}\ Wo{\isacharparenright}{\isachardoublequoteclose}\isanewline
%
\isadelimproof
%
\endisadelimproof
%
\isatagproof
\isacommand{proof}\isamarkupfalse%
\ {\isacharminus}\isanewline
\ \ \isacommand{have}\isamarkupfalse%
\ {\isachardoublequoteopen}{\isacharbraceleft}G{\isacharcomma}H{\isacharbraceright}\ {\isasymunion}\ Wo\ {\isasymsubseteq}\ {\isacharbraceleft}G{\isacharcomma}H{\isacharcomma}F{\isacharbraceright}\ {\isasymunion}\ Wo{\isachardoublequoteclose}\isanewline
\ \ \ \ \isacommand{by}\isamarkupfalse%
\ blast\isanewline
\ \ \isacommand{have}\isamarkupfalse%
\ {\isachardoublequoteopen}F\ {\isacharequal}\ G\ \isactrlbold {\isasymand}\ H\ {\isasymor}\ \isanewline
\ \ \ \ {\isacharparenleft}{\isasymexists}F{\isadigit{1}}\ G{\isadigit{1}}{\isachardot}\ F\ {\isacharequal}\ \isactrlbold {\isasymnot}\ {\isacharparenleft}F{\isadigit{1}}\ \isactrlbold {\isasymor}\ G{\isadigit{1}}{\isacharparenright}\ {\isasymand}\ G\ {\isacharequal}\ \isactrlbold {\isasymnot}\ F{\isadigit{1}}\ {\isasymand}\ H\ {\isacharequal}\ \isactrlbold {\isasymnot}\ G{\isadigit{1}}{\isacharparenright}\ {\isasymor}\ \isanewline
\ \ \ \ {\isacharparenleft}{\isasymexists}H{\isadigit{1}}{\isachardot}\ F\ {\isacharequal}\ \isactrlbold {\isasymnot}\ {\isacharparenleft}G\ \isactrlbold {\isasymrightarrow}\ H{\isadigit{1}}{\isacharparenright}\ {\isasymand}\ H\ {\isacharequal}\ \isactrlbold {\isasymnot}\ H{\isadigit{1}}{\isacharparenright}\ {\isasymor}\ \isanewline
\ \ \ \ F\ {\isacharequal}\ \isactrlbold {\isasymnot}\ {\isacharparenleft}\isactrlbold {\isasymnot}\ G{\isacharparenright}\ {\isasymand}\ H\ {\isacharequal}\ G{\isachardoublequoteclose}\isanewline
\ \ \ \ \isacommand{using}\isamarkupfalse%
\ assms{\isacharparenleft}{\isadigit{2}}{\isacharparenright}\ \isacommand{by}\isamarkupfalse%
\ {\isacharparenleft}simp\ only{\isacharcolon}\ con{\isacharunderscore}dis{\isacharunderscore}simps{\isacharparenleft}{\isadigit{1}}{\isacharparenright}{\isacharparenright}\isanewline
\ \ \isacommand{thus}\isamarkupfalse%
\ {\isachardoublequoteopen}sat\ {\isacharparenleft}{\isacharbraceleft}G{\isacharcomma}H{\isacharcomma}F{\isacharbraceright}\ {\isasymunion}\ Wo{\isacharparenright}{\isachardoublequoteclose}\isanewline
\ \ \isacommand{proof}\isamarkupfalse%
\ {\isacharparenleft}rule\ disjE{\isacharparenright}\isanewline
\ \ \ \ \isacommand{assume}\isamarkupfalse%
\ {\isachardoublequoteopen}F\ {\isacharequal}\ G\ \isactrlbold {\isasymand}\ H{\isachardoublequoteclose}\isanewline
\ \ \ \ \isacommand{show}\isamarkupfalse%
\ {\isachardoublequoteopen}sat\ {\isacharparenleft}{\isacharbraceleft}G{\isacharcomma}H{\isacharcomma}F{\isacharbraceright}\ {\isasymunion}\ Wo{\isacharparenright}{\isachardoublequoteclose}\isanewline
\ \ \ \ \ \ \isacommand{using}\isamarkupfalse%
\ assms{\isacharparenleft}{\isadigit{1}}{\isacharparenright}\ {\isacartoucheopen}F\ {\isacharequal}\ G\ \isactrlbold {\isasymand}\ H{\isacartoucheclose}\ assms{\isacharparenleft}{\isadigit{3}}{\isacharcomma}{\isadigit{4}}{\isacharcomma}{\isadigit{5}}{\isacharparenright}\ \isacommand{by}\isamarkupfalse%
\ {\isacharparenleft}rule\ pcp{\isacharunderscore}colecComp{\isacharunderscore}CON{\isacharunderscore}sat{\isadigit{1}}{\isacharparenright}\isanewline
\ \ \isacommand{next}\isamarkupfalse%
\isanewline
\ \ \ \ \isacommand{assume}\isamarkupfalse%
\ {\isachardoublequoteopen}{\isacharparenleft}{\isasymexists}F{\isadigit{1}}\ G{\isadigit{1}}{\isachardot}\ F\ {\isacharequal}\ \isactrlbold {\isasymnot}\ {\isacharparenleft}F{\isadigit{1}}\ \isactrlbold {\isasymor}\ G{\isadigit{1}}{\isacharparenright}\ {\isasymand}\ G\ {\isacharequal}\ \isactrlbold {\isasymnot}\ F{\isadigit{1}}\ {\isasymand}\ H\ {\isacharequal}\ \isactrlbold {\isasymnot}\ G{\isadigit{1}}{\isacharparenright}\ {\isasymor}\ \isanewline
\ \ \ \ {\isacharparenleft}{\isasymexists}H{\isadigit{1}}{\isachardot}\ F\ {\isacharequal}\ \isactrlbold {\isasymnot}\ {\isacharparenleft}G\ \isactrlbold {\isasymrightarrow}\ H{\isadigit{1}}{\isacharparenright}\ {\isasymand}\ H\ {\isacharequal}\ \isactrlbold {\isasymnot}\ H{\isadigit{1}}{\isacharparenright}\ {\isasymor}\ \isanewline
\ \ \ \ F\ {\isacharequal}\ \isactrlbold {\isasymnot}\ {\isacharparenleft}\isactrlbold {\isasymnot}\ G{\isacharparenright}\ {\isasymand}\ H\ {\isacharequal}\ G{\isachardoublequoteclose}\isanewline
\ \ \ \ \isacommand{thus}\isamarkupfalse%
\ {\isachardoublequoteopen}sat\ {\isacharparenleft}{\isacharbraceleft}G{\isacharcomma}H{\isacharcomma}F{\isacharbraceright}\ {\isasymunion}\ Wo{\isacharparenright}{\isachardoublequoteclose}\isanewline
\ \ \ \ \isacommand{proof}\isamarkupfalse%
\ {\isacharparenleft}rule\ disjE{\isacharparenright}\isanewline
\ \ \ \ \ \ \isacommand{assume}\isamarkupfalse%
\ Ex{\isadigit{2}}{\isacharcolon}{\isachardoublequoteopen}{\isasymexists}F{\isadigit{1}}\ G{\isadigit{1}}{\isachardot}\ F\ {\isacharequal}\ \isactrlbold {\isasymnot}\ {\isacharparenleft}F{\isadigit{1}}\ \isactrlbold {\isasymor}\ G{\isadigit{1}}{\isacharparenright}\ {\isasymand}\ G\ {\isacharequal}\ \isactrlbold {\isasymnot}\ F{\isadigit{1}}\ {\isasymand}\ H\ {\isacharequal}\ \isactrlbold {\isasymnot}\ G{\isadigit{1}}{\isachardoublequoteclose}\ \isanewline
\ \ \ \ \ \ \isacommand{obtain}\isamarkupfalse%
\ F{\isadigit{1}}\ G{\isadigit{1}}\ \isakeyword{where}\ {\isadigit{2}}{\isacharcolon}{\isachardoublequoteopen}F\ {\isacharequal}\ \isactrlbold {\isasymnot}{\isacharparenleft}F{\isadigit{1}}\ \isactrlbold {\isasymor}\ G{\isadigit{1}}{\isacharparenright}\ {\isasymand}\ G\ {\isacharequal}\ \isactrlbold {\isasymnot}\ F{\isadigit{1}}\ {\isasymand}\ H\ {\isacharequal}\ \isactrlbold {\isasymnot}\ G{\isadigit{1}}{\isachardoublequoteclose}\isanewline
\ \ \ \ \ \ \ \ \isacommand{using}\isamarkupfalse%
\ Ex{\isadigit{2}}\ \isacommand{by}\isamarkupfalse%
\ {\isacharparenleft}iprover\ elim{\isacharcolon}\ exE{\isacharparenright}\isanewline
\ \ \ \ \ \ \isacommand{have}\isamarkupfalse%
\ {\isachardoublequoteopen}G\ {\isacharequal}\ \isactrlbold {\isasymnot}\ F{\isadigit{1}}{\isachardoublequoteclose}\isanewline
\ \ \ \ \ \ \ \ \isacommand{using}\isamarkupfalse%
\ {\isadigit{2}}\ \isacommand{by}\isamarkupfalse%
\ {\isacharparenleft}iprover\ elim{\isacharcolon}\ conjunct{\isadigit{1}}{\isacharparenright}\isanewline
\ \ \ \ \ \ \isacommand{have}\isamarkupfalse%
\ {\isachardoublequoteopen}H\ {\isacharequal}\ \isactrlbold {\isasymnot}\ G{\isadigit{1}}{\isachardoublequoteclose}\isanewline
\ \ \ \ \ \ \ \ \isacommand{using}\isamarkupfalse%
\ {\isadigit{2}}\ \isacommand{by}\isamarkupfalse%
\ {\isacharparenleft}iprover\ elim{\isacharcolon}\ conjunct{\isadigit{2}}{\isacharparenright}\isanewline
\ \ \ \ \ \ \isacommand{have}\isamarkupfalse%
\ {\isachardoublequoteopen}F\ {\isacharequal}\ \isactrlbold {\isasymnot}{\isacharparenleft}F{\isadigit{1}}\ \isactrlbold {\isasymor}\ G{\isadigit{1}}{\isacharparenright}{\isachardoublequoteclose}\isanewline
\ \ \ \ \ \ \ \ \isacommand{using}\isamarkupfalse%
\ {\isadigit{2}}\ \isacommand{by}\isamarkupfalse%
\ {\isacharparenleft}rule\ conjunct{\isadigit{1}}{\isacharparenright}\isanewline
\ \ \ \ \ \ \isacommand{have}\isamarkupfalse%
\ {\isachardoublequoteopen}sat\ {\isacharparenleft}{\isacharbraceleft}\isactrlbold {\isasymnot}\ F{\isadigit{1}}{\isacharcomma}\ \isactrlbold {\isasymnot}\ G{\isadigit{1}}{\isacharcomma}\ F{\isacharbraceright}\ {\isasymunion}\ Wo{\isacharparenright}{\isachardoublequoteclose}\isanewline
\ \ \ \ \ \ \ \ \isacommand{using}\isamarkupfalse%
\ assms{\isacharparenleft}{\isadigit{1}}{\isacharparenright}\ {\isacartoucheopen}F\ {\isacharequal}\ \isactrlbold {\isasymnot}{\isacharparenleft}F{\isadigit{1}}\ \isactrlbold {\isasymor}\ G{\isadigit{1}}{\isacharparenright}{\isacartoucheclose}\ assms{\isacharparenleft}{\isadigit{3}}{\isacharcomma}{\isadigit{4}}{\isacharcomma}{\isadigit{5}}{\isacharparenright}\ \isacommand{by}\isamarkupfalse%
\ {\isacharparenleft}rule\ pcp{\isacharunderscore}colecComp{\isacharunderscore}CON{\isacharunderscore}sat{\isadigit{2}}{\isacharparenright}\isanewline
\ \ \ \ \ \ \isacommand{thus}\isamarkupfalse%
\ {\isachardoublequoteopen}sat\ {\isacharparenleft}{\isacharbraceleft}G{\isacharcomma}H{\isacharcomma}F{\isacharbraceright}\ {\isasymunion}\ Wo{\isacharparenright}{\isachardoublequoteclose}\isanewline
\ \ \ \ \ \ \ \ \isacommand{by}\isamarkupfalse%
\ {\isacharparenleft}simp\ only{\isacharcolon}\ {\isacartoucheopen}G\ {\isacharequal}\ \isactrlbold {\isasymnot}\ F{\isadigit{1}}{\isacartoucheclose}\ {\isacartoucheopen}H\ {\isacharequal}\ \isactrlbold {\isasymnot}\ G{\isadigit{1}}{\isacartoucheclose}{\isacharparenright}\isanewline
\ \ \ \ \isacommand{next}\isamarkupfalse%
\isanewline
\ \ \ \ \ \ \isacommand{assume}\isamarkupfalse%
\ {\isachardoublequoteopen}{\isacharparenleft}{\isasymexists}H{\isadigit{1}}{\isachardot}\ F\ {\isacharequal}\ \isactrlbold {\isasymnot}\ {\isacharparenleft}G\ \isactrlbold {\isasymrightarrow}\ H{\isadigit{1}}{\isacharparenright}\ {\isasymand}\ H\ {\isacharequal}\ \isactrlbold {\isasymnot}\ H{\isadigit{1}}{\isacharparenright}\ {\isasymor}\ \isanewline
\ \ \ \ \ \ \ \ \ \ \ \ \ \ F\ {\isacharequal}\ \isactrlbold {\isasymnot}\ {\isacharparenleft}\isactrlbold {\isasymnot}\ G{\isacharparenright}\ {\isasymand}\ H\ {\isacharequal}\ G{\isachardoublequoteclose}\isanewline
\ \ \ \ \ \ \isacommand{thus}\isamarkupfalse%
\ {\isachardoublequoteopen}sat\ {\isacharparenleft}{\isacharbraceleft}G{\isacharcomma}H{\isacharcomma}F{\isacharbraceright}\ {\isasymunion}\ Wo{\isacharparenright}{\isachardoublequoteclose}\isanewline
\ \ \ \ \ \ \isacommand{proof}\isamarkupfalse%
\ {\isacharparenleft}rule\ disjE{\isacharparenright}\isanewline
\ \ \ \ \ \ \ \ \isacommand{assume}\isamarkupfalse%
\ Ex{\isadigit{3}}{\isacharcolon}{\isachardoublequoteopen}{\isasymexists}H{\isadigit{1}}{\isachardot}\ F\ {\isacharequal}\ \isactrlbold {\isasymnot}\ {\isacharparenleft}G\ \isactrlbold {\isasymrightarrow}\ H{\isadigit{1}}{\isacharparenright}\ {\isasymand}\ H\ {\isacharequal}\ \isactrlbold {\isasymnot}\ H{\isadigit{1}}{\isachardoublequoteclose}\isanewline
\ \ \ \ \ \ \ \ \isacommand{obtain}\isamarkupfalse%
\ H{\isadigit{1}}\ \isakeyword{where}\ {\isadigit{3}}{\isacharcolon}{\isachardoublequoteopen}F\ {\isacharequal}\ \isactrlbold {\isasymnot}{\isacharparenleft}G\ \isactrlbold {\isasymrightarrow}\ H{\isadigit{1}}{\isacharparenright}\ {\isasymand}\ H\ {\isacharequal}\ \isactrlbold {\isasymnot}\ H{\isadigit{1}}{\isachardoublequoteclose}\isanewline
\ \ \ \ \ \ \ \ \ \ \isacommand{using}\isamarkupfalse%
\ Ex{\isadigit{3}}\ \isacommand{by}\isamarkupfalse%
\ {\isacharparenleft}rule\ exE{\isacharparenright}\isanewline
\ \ \ \ \ \ \ \ \isacommand{have}\isamarkupfalse%
\ {\isachardoublequoteopen}H\ {\isacharequal}\ \isactrlbold {\isasymnot}\ H{\isadigit{1}}{\isachardoublequoteclose}\isanewline
\ \ \ \ \ \ \ \ \ \ \isacommand{using}\isamarkupfalse%
\ {\isadigit{3}}\ \isacommand{by}\isamarkupfalse%
\ {\isacharparenleft}rule\ conjunct{\isadigit{2}}{\isacharparenright}\isanewline
\ \ \ \ \ \ \ \ \isacommand{have}\isamarkupfalse%
\ {\isachardoublequoteopen}F\ {\isacharequal}\ \isactrlbold {\isasymnot}{\isacharparenleft}G\ \isactrlbold {\isasymrightarrow}\ H{\isadigit{1}}{\isacharparenright}{\isachardoublequoteclose}\isanewline
\ \ \ \ \ \ \ \ \ \ \isacommand{using}\isamarkupfalse%
\ {\isadigit{3}}\ \isacommand{by}\isamarkupfalse%
\ {\isacharparenleft}rule\ conjunct{\isadigit{1}}{\isacharparenright}\isanewline
\ \ \ \ \ \ \ \ \isacommand{have}\isamarkupfalse%
\ {\isachardoublequoteopen}sat\ {\isacharparenleft}{\isacharbraceleft}G{\isacharcomma}\ \isactrlbold {\isasymnot}\ H{\isadigit{1}}{\isacharcomma}\ F{\isacharbraceright}\ {\isasymunion}\ Wo{\isacharparenright}{\isachardoublequoteclose}\isanewline
\ \ \ \ \ \ \ \ \ \ \isacommand{using}\isamarkupfalse%
\ assms{\isacharparenleft}{\isadigit{1}}{\isacharparenright}\ {\isacartoucheopen}F\ {\isacharequal}\ \isactrlbold {\isasymnot}{\isacharparenleft}G\ \isactrlbold {\isasymrightarrow}\ H{\isadigit{1}}{\isacharparenright}{\isacartoucheclose}\ assms{\isacharparenleft}{\isadigit{3}}{\isacharcomma}{\isadigit{4}}{\isacharcomma}{\isadigit{5}}{\isacharparenright}\ \isacommand{by}\isamarkupfalse%
\ {\isacharparenleft}rule\ pcp{\isacharunderscore}colecComp{\isacharunderscore}CON{\isacharunderscore}sat{\isadigit{3}}{\isacharparenright}\isanewline
\ \ \ \ \ \ \ \ \isacommand{thus}\isamarkupfalse%
\ {\isachardoublequoteopen}sat\ {\isacharparenleft}{\isacharbraceleft}G{\isacharcomma}H{\isacharcomma}F{\isacharbraceright}\ {\isasymunion}\ Wo{\isacharparenright}{\isachardoublequoteclose}\isanewline
\ \ \ \ \ \ \ \ \ \ \isacommand{by}\isamarkupfalse%
\ {\isacharparenleft}simp\ only{\isacharcolon}\ {\isacartoucheopen}H\ {\isacharequal}\ \isactrlbold {\isasymnot}\ H{\isadigit{1}}{\isacartoucheclose}{\isacharparenright}\isanewline
\ \ \ \ \ \ \isacommand{next}\isamarkupfalse%
\isanewline
\ \ \ \ \ \ \ \ \isacommand{assume}\isamarkupfalse%
\ {\isachardoublequoteopen}F\ {\isacharequal}\ \isactrlbold {\isasymnot}\ {\isacharparenleft}\isactrlbold {\isasymnot}\ G{\isacharparenright}\ {\isasymand}\ H\ {\isacharequal}\ G{\isachardoublequoteclose}\isanewline
\ \ \ \ \ \ \ \ \isacommand{then}\isamarkupfalse%
\ \isacommand{have}\isamarkupfalse%
\ {\isachardoublequoteopen}H\ {\isacharequal}\ G{\isachardoublequoteclose}\isanewline
\ \ \ \ \ \ \ \ \ \ \isacommand{by}\isamarkupfalse%
\ {\isacharparenleft}rule\ conjunct{\isadigit{2}}{\isacharparenright}\isanewline
\ \ \ \ \ \ \ \ \isacommand{then}\isamarkupfalse%
\ \isacommand{have}\isamarkupfalse%
\ C{\isadigit{4}}{\isacharcolon}{\isachardoublequoteopen}{\isacharbraceleft}G{\isacharcomma}F{\isacharbraceright}\ {\isasymunion}\ Wo\ {\isacharequal}\ {\isacharbraceleft}G{\isacharcomma}H{\isacharcomma}F{\isacharbraceright}\ {\isasymunion}\ Wo{\isachardoublequoteclose}\isanewline
\ \ \ \ \ \ \ \ \ \ \isacommand{by}\isamarkupfalse%
\ blast\isanewline
\ \ \ \ \ \ \ \ \isacommand{have}\isamarkupfalse%
\ {\isachardoublequoteopen}F\ {\isacharequal}\ \isactrlbold {\isasymnot}\ {\isacharparenleft}\isactrlbold {\isasymnot}\ G{\isacharparenright}{\isachardoublequoteclose}\isanewline
\ \ \ \ \ \ \ \ \ \ \isacommand{using}\isamarkupfalse%
\ {\isacartoucheopen}F\ {\isacharequal}\ \isactrlbold {\isasymnot}\ {\isacharparenleft}\isactrlbold {\isasymnot}\ G{\isacharparenright}\ {\isasymand}\ H\ {\isacharequal}\ G{\isacartoucheclose}\ \isacommand{by}\isamarkupfalse%
\ {\isacharparenleft}rule\ conjunct{\isadigit{1}}{\isacharparenright}\isanewline
\ \ \ \ \ \ \ \ \isacommand{have}\isamarkupfalse%
\ {\isachardoublequoteopen}sat\ {\isacharparenleft}{\isacharbraceleft}G{\isacharcomma}F{\isacharbraceright}\ {\isasymunion}\ Wo{\isacharparenright}{\isachardoublequoteclose}\isanewline
\ \ \ \ \ \ \ \ \ \ \isacommand{using}\isamarkupfalse%
\ assms{\isacharparenleft}{\isadigit{1}}{\isacharparenright}\ {\isacartoucheopen}F\ {\isacharequal}\ \isactrlbold {\isasymnot}{\isacharparenleft}\isactrlbold {\isasymnot}\ G{\isacharparenright}{\isacartoucheclose}\ assms{\isacharparenleft}{\isadigit{3}}{\isacharcomma}{\isadigit{4}}{\isacharcomma}{\isadigit{5}}{\isacharparenright}\ \isacommand{by}\isamarkupfalse%
\ {\isacharparenleft}rule\ pcp{\isacharunderscore}colecComp{\isacharunderscore}CON{\isacharunderscore}sat{\isadigit{4}}{\isacharparenright}\isanewline
\ \ \ \ \ \ \ \ \isacommand{thus}\isamarkupfalse%
\ {\isachardoublequoteopen}sat\ {\isacharparenleft}{\isacharbraceleft}G{\isacharcomma}H{\isacharcomma}F{\isacharbraceright}\ {\isasymunion}\ Wo{\isacharparenright}{\isachardoublequoteclose}\isanewline
\ \ \ \ \ \ \ \ \ \ \isacommand{by}\isamarkupfalse%
\ {\isacharparenleft}simp\ only{\isacharcolon}\ C{\isadigit{4}}{\isacharparenright}\isanewline
\ \ \ \ \ \ \isacommand{qed}\isamarkupfalse%
\isanewline
\ \ \ \ \isacommand{qed}\isamarkupfalse%
\isanewline
\ \ \isacommand{qed}\isamarkupfalse%
\isanewline
\isacommand{qed}\isamarkupfalse%
%
\endisatagproof
{\isafoldproof}%
%
\isadelimproof
%
\endisadelimproof
%
\begin{isamarkuptext}%
Finalmente, con el resultado anterior, podemos probar la tercera condición del lema \isa{{\isadigit{2}}{\isachardot}{\isadigit{0}}{\isachardot}{\isadigit{2}}}: 
  dados \isa{W\ {\isasymin}\ C} y \isa{F} una fórmula de tipo \isa{{\isasymalpha}} con componentes \isa{{\isasymalpha}\isactrlsub {\isadigit{1}}} y \isa{{\isasymalpha}\isactrlsub {\isadigit{2}}} tal que \isa{F\ {\isasymin}\ W}, se tiene 
  que \isa{{\isacharbraceleft}{\isasymalpha}\isactrlsub {\isadigit{1}}{\isacharcomma}{\isasymalpha}\isactrlsub {\isadigit{2}}{\isacharbraceright}\ {\isasymunion}\ W\ {\isasymin}\ C}.%
\end{isamarkuptext}\isamarkuptrue%
\isacommand{lemma}\isamarkupfalse%
\ pcp{\isacharunderscore}colecComp{\isacharunderscore}CON{\isacharcolon}\isanewline
\ \ \isakeyword{assumes}\ {\isachardoublequoteopen}W\ {\isasymin}\ colecComp{\isachardoublequoteclose}\isanewline
\ \ \isakeyword{shows}\ {\isachardoublequoteopen}{\isasymforall}F\ G\ H{\isachardot}\ Con\ F\ G\ H\ {\isasymlongrightarrow}\ F\ {\isasymin}\ W\ {\isasymlongrightarrow}\ {\isacharbraceleft}G{\isacharcomma}H{\isacharbraceright}\ {\isasymunion}\ W\ {\isasymin}\ colecComp{\isachardoublequoteclose}\isanewline
%
\isadelimproof
%
\endisadelimproof
%
\isatagproof
\isacommand{proof}\isamarkupfalse%
\ {\isacharparenleft}rule\ allI{\isacharparenright}{\isacharplus}\isanewline
\ \ \isacommand{fix}\isamarkupfalse%
\ F\ G\ H\isanewline
\ \ \isacommand{show}\isamarkupfalse%
\ {\isachardoublequoteopen}Con\ F\ G\ H\ {\isasymlongrightarrow}\ F\ {\isasymin}\ W\ {\isasymlongrightarrow}\ {\isacharbraceleft}G{\isacharcomma}H{\isacharbraceright}\ {\isasymunion}\ W\ {\isasymin}\ colecComp{\isachardoublequoteclose}\isanewline
\ \ \isacommand{proof}\isamarkupfalse%
\ {\isacharparenleft}rule\ impI{\isacharparenright}{\isacharplus}\isanewline
\ \ \ \ \isacommand{assume}\isamarkupfalse%
\ {\isachardoublequoteopen}Con\ F\ G\ H{\isachardoublequoteclose}\isanewline
\ \ \ \ \isacommand{assume}\isamarkupfalse%
\ {\isachardoublequoteopen}F\ {\isasymin}\ W{\isachardoublequoteclose}\isanewline
\ \ \ \ \isacommand{show}\isamarkupfalse%
\ {\isachardoublequoteopen}{\isacharbraceleft}G{\isacharcomma}H{\isacharbraceright}\ {\isasymunion}\ W\ {\isasymin}\ colecComp{\isachardoublequoteclose}\isanewline
\ \ \ \ \ \ \isacommand{unfolding}\isamarkupfalse%
\ colecComp\ fin{\isacharunderscore}sat{\isacharunderscore}def\isanewline
\ \ \ \ \isacommand{proof}\isamarkupfalse%
\ {\isacharparenleft}rule\ CollectI{\isacharparenright}\isanewline
\ \ \ \ \ \ \isacommand{show}\isamarkupfalse%
\ {\isachardoublequoteopen}{\isasymforall}S\ {\isasymsubseteq}\ {\isacharbraceleft}G{\isacharcomma}H{\isacharbraceright}\ {\isasymunion}\ W{\isachardot}\ finite\ S\ {\isasymlongrightarrow}\ sat\ S{\isachardoublequoteclose}\isanewline
\ \ \ \ \ \ \isacommand{proof}\isamarkupfalse%
\ {\isacharparenleft}rule\ sallI{\isacharparenright}\isanewline
\ \ \ \ \ \ \ \ \isacommand{fix}\isamarkupfalse%
\ S\isanewline
\ \ \ \ \ \ \ \ \isacommand{assume}\isamarkupfalse%
\ {\isachardoublequoteopen}S\ {\isasymsubseteq}\ {\isacharbraceleft}G{\isacharcomma}H{\isacharbraceright}\ {\isasymunion}\ W{\isachardoublequoteclose}\isanewline
\ \ \ \ \ \ \ \ \isacommand{then}\isamarkupfalse%
\ \isacommand{have}\isamarkupfalse%
\ {\isachardoublequoteopen}S\ {\isasymsubseteq}\ {\isacharbraceleft}G{\isacharbraceright}\ {\isasymunion}\ {\isacharparenleft}{\isacharbraceleft}H{\isacharbraceright}\ {\isasymunion}\ W{\isacharparenright}{\isachardoublequoteclose}\isanewline
\ \ \ \ \ \ \ \ \ \ \isacommand{by}\isamarkupfalse%
\ blast\ \isanewline
\ \ \ \ \ \ \ \ \isacommand{show}\isamarkupfalse%
\ {\isachardoublequoteopen}finite\ S\ {\isasymlongrightarrow}\ sat\ S{\isachardoublequoteclose}\isanewline
\ \ \ \ \ \ \ \ \isacommand{proof}\isamarkupfalse%
\ {\isacharparenleft}rule\ impI{\isacharparenright}\isanewline
\ \ \ \ \ \ \ \ \ \ \isacommand{assume}\isamarkupfalse%
\ {\isachardoublequoteopen}finite\ S{\isachardoublequoteclose}\ \isanewline
\ \ \ \ \ \ \ \ \ \ \isacommand{have}\isamarkupfalse%
\ Ex{\isacharcolon}{\isachardoublequoteopen}{\isasymexists}Wo\ {\isasymsubseteq}\ W{\isachardot}\ finite\ Wo\ {\isasymand}\ {\isacharparenleft}S\ {\isacharequal}\ {\isacharbraceleft}G{\isacharcomma}H{\isacharbraceright}\ {\isasymunion}\ Wo\ {\isasymor}\ S\ {\isacharequal}\ {\isacharbraceleft}G{\isacharbraceright}\ {\isasymunion}\ Wo\ {\isasymor}\ S\ {\isacharequal}\ {\isacharbraceleft}H{\isacharbraceright}\ {\isasymunion}\ Wo\ {\isasymor}\ S\ {\isacharequal}\ Wo{\isacharparenright}{\isachardoublequoteclose}\isanewline
\ \ \ \ \ \ \ \ \ \ \ \ \isacommand{using}\isamarkupfalse%
\ {\isacartoucheopen}finite\ S{\isacartoucheclose}\ {\isacartoucheopen}S\ {\isasymsubseteq}\ {\isacharbraceleft}G{\isacharcomma}H{\isacharbraceright}\ {\isasymunion}\ W{\isacartoucheclose}\ \isacommand{by}\isamarkupfalse%
\ {\isacharparenleft}rule\ finite{\isacharunderscore}subset{\isacharunderscore}insert{\isadigit{2}}{\isacharparenright}\isanewline
\ \ \ \ \ \ \ \ \ \ \isacommand{obtain}\isamarkupfalse%
\ Wo\ \isakeyword{where}\ {\isachardoublequoteopen}Wo\ {\isasymsubseteq}\ W{\isachardoublequoteclose}\ \isakeyword{and}\ {\isadigit{1}}{\isacharcolon}{\isachardoublequoteopen}finite\ Wo\ {\isasymand}\ {\isacharparenleft}S\ {\isacharequal}\ {\isacharbraceleft}G{\isacharcomma}H{\isacharbraceright}\ {\isasymunion}\ Wo\ {\isasymor}\ S\ {\isacharequal}\ {\isacharbraceleft}G{\isacharbraceright}\ {\isasymunion}\ Wo\ {\isasymor}\ S\ {\isacharequal}\ {\isacharbraceleft}H{\isacharbraceright}\ {\isasymunion}\ Wo\ {\isasymor}\ S\ {\isacharequal}\ Wo{\isacharparenright}{\isachardoublequoteclose}\isanewline
\ \ \ \ \ \ \ \ \ \ \ \ \isacommand{using}\isamarkupfalse%
\ Ex\ \isacommand{by}\isamarkupfalse%
\ {\isacharparenleft}rule\ subexE{\isacharparenright}\isanewline
\ \ \ \ \ \ \ \ \ \ \isacommand{have}\isamarkupfalse%
\ {\isachardoublequoteopen}finite\ Wo{\isachardoublequoteclose}\isanewline
\ \ \ \ \ \ \ \ \ \ \ \ \isacommand{using}\isamarkupfalse%
\ {\isadigit{1}}\ \isacommand{by}\isamarkupfalse%
\ {\isacharparenleft}rule\ conjunct{\isadigit{1}}{\isacharparenright}\isanewline
\ \ \ \ \ \ \ \ \ \ \ \ \isacommand{have}\isamarkupfalse%
\ {\isachardoublequoteopen}sat\ {\isacharparenleft}{\isacharbraceleft}G{\isacharcomma}H{\isacharcomma}F{\isacharbraceright}\ {\isasymunion}\ Wo{\isacharparenright}{\isachardoublequoteclose}\ \isanewline
\ \ \ \ \ \ \ \ \ \ \ \ \ \ \isacommand{using}\isamarkupfalse%
\ {\isacartoucheopen}W\ {\isasymin}\ colecComp{\isacartoucheclose}\ {\isacartoucheopen}Con\ F\ G\ H{\isacartoucheclose}\ {\isacartoucheopen}F\ {\isasymin}\ W{\isacartoucheclose}\ {\isacartoucheopen}finite\ Wo{\isacartoucheclose}\ {\isacartoucheopen}Wo\ {\isasymsubseteq}\ W{\isacartoucheclose}\ \isacommand{by}\isamarkupfalse%
\ {\isacharparenleft}rule\ pcp{\isacharunderscore}colecComp{\isacharunderscore}CON{\isacharunderscore}sat{\isacharparenright}\isanewline
\ \ \ \ \ \ \ \ \ \ \isacommand{have}\isamarkupfalse%
\ {\isachardoublequoteopen}S\ {\isacharequal}\ {\isacharbraceleft}G{\isacharcomma}H{\isacharbraceright}\ {\isasymunion}\ Wo\ {\isasymor}\ S\ {\isacharequal}\ {\isacharbraceleft}G{\isacharbraceright}\ {\isasymunion}\ Wo\ {\isasymor}\ S\ {\isacharequal}\ {\isacharbraceleft}H{\isacharbraceright}\ {\isasymunion}\ Wo\ {\isasymor}\ S\ {\isacharequal}\ Wo{\isachardoublequoteclose}\isanewline
\ \ \ \ \ \ \ \ \ \ \ \ \isacommand{using}\isamarkupfalse%
\ {\isadigit{1}}\ \isacommand{by}\isamarkupfalse%
\ {\isacharparenleft}rule\ conjunct{\isadigit{2}}{\isacharparenright}\isanewline
\ \ \ \ \ \ \ \ \ \ \isacommand{thus}\isamarkupfalse%
\ {\isachardoublequoteopen}sat\ S{\isachardoublequoteclose}\isanewline
\ \ \ \ \ \ \ \ \ \ \isacommand{proof}\isamarkupfalse%
\ {\isacharparenleft}rule\ disjE{\isacharparenright}\isanewline
\ \ \ \ \ \ \ \ \ \ \ \ \isacommand{assume}\isamarkupfalse%
\ {\isachardoublequoteopen}S\ {\isacharequal}\ {\isacharbraceleft}G{\isacharcomma}H{\isacharbraceright}\ {\isasymunion}\ Wo{\isachardoublequoteclose}\isanewline
\ \ \ \ \ \ \ \ \ \ \ \ \isacommand{then}\isamarkupfalse%
\ \isacommand{have}\isamarkupfalse%
\ {\isachardoublequoteopen}S\ {\isasymsubseteq}\ {\isacharbraceleft}G{\isacharcomma}H{\isacharcomma}F{\isacharbraceright}\ {\isasymunion}\ Wo{\isachardoublequoteclose}\isanewline
\ \ \ \ \ \ \ \ \ \ \ \ \ \ \isacommand{by}\isamarkupfalse%
\ blast\isanewline
\ \ \ \ \ \ \ \ \ \ \ \ \isacommand{show}\isamarkupfalse%
\ {\isachardoublequoteopen}sat\ S{\isachardoublequoteclose}\isanewline
\ \ \ \ \ \ \ \ \ \ \ \ \ \ \isacommand{using}\isamarkupfalse%
\ {\isacartoucheopen}sat{\isacharparenleft}{\isacharbraceleft}G{\isacharcomma}H{\isacharcomma}F{\isacharbraceright}\ {\isasymunion}\ Wo{\isacharparenright}{\isacartoucheclose}\ {\isacartoucheopen}S\ {\isasymsubseteq}\ {\isacharbraceleft}G{\isacharcomma}H{\isacharcomma}F{\isacharbraceright}\ {\isasymunion}\ Wo{\isacartoucheclose}\ \isacommand{by}\isamarkupfalse%
\ {\isacharparenleft}simp\ only{\isacharcolon}\ sat{\isacharunderscore}mono{\isacharparenright}\isanewline
\ \ \ \ \ \ \ \ \ \ \isacommand{next}\isamarkupfalse%
\isanewline
\ \ \ \ \ \ \ \ \ \ \ \ \isacommand{assume}\isamarkupfalse%
\ {\isachardoublequoteopen}S\ {\isacharequal}\ {\isacharbraceleft}G{\isacharbraceright}\ {\isasymunion}\ Wo\ {\isasymor}\ S\ {\isacharequal}\ {\isacharbraceleft}H{\isacharbraceright}\ {\isasymunion}\ Wo\ {\isasymor}\ S\ {\isacharequal}\ Wo{\isachardoublequoteclose}\isanewline
\ \ \ \ \ \ \ \ \ \ \ \ \isacommand{thus}\isamarkupfalse%
\ {\isachardoublequoteopen}sat\ S{\isachardoublequoteclose}\isanewline
\ \ \ \ \ \ \ \ \ \ \ \ \isacommand{proof}\isamarkupfalse%
\ {\isacharparenleft}rule\ disjE{\isacharparenright}\isanewline
\ \ \ \ \ \ \ \ \ \ \ \ \ \ \isacommand{assume}\isamarkupfalse%
\ {\isachardoublequoteopen}S\ {\isacharequal}\ {\isacharbraceleft}G{\isacharbraceright}\ {\isasymunion}\ Wo{\isachardoublequoteclose}\isanewline
\ \ \ \ \ \ \ \ \ \ \ \ \ \ \isacommand{then}\isamarkupfalse%
\ \isacommand{have}\isamarkupfalse%
\ {\isachardoublequoteopen}S\ {\isasymsubseteq}\ {\isacharbraceleft}G{\isacharcomma}H{\isacharcomma}F{\isacharbraceright}\ {\isasymunion}\ Wo{\isachardoublequoteclose}\isanewline
\ \ \ \ \ \ \ \ \ \ \ \ \ \ \ \ \isacommand{by}\isamarkupfalse%
\ blast\ \isanewline
\ \ \ \ \ \ \ \ \ \ \ \ \ \ \isacommand{thus}\isamarkupfalse%
\ {\isachardoublequoteopen}sat\ S{\isachardoublequoteclose}\isanewline
\ \ \ \ \ \ \ \ \ \ \ \ \ \ \ \ \isacommand{using}\isamarkupfalse%
\ {\isacartoucheopen}sat{\isacharparenleft}{\isacharbraceleft}G{\isacharcomma}H{\isacharcomma}F{\isacharbraceright}\ {\isasymunion}\ Wo{\isacharparenright}{\isacartoucheclose}\ \isacommand{by}\isamarkupfalse%
\ {\isacharparenleft}rule\ sat{\isacharunderscore}mono{\isacharparenright}\isanewline
\ \ \ \ \ \ \ \ \ \ \ \ \isacommand{next}\isamarkupfalse%
\isanewline
\ \ \ \ \ \ \ \ \ \ \ \ \ \ \isacommand{assume}\isamarkupfalse%
\ {\isachardoublequoteopen}S\ {\isacharequal}\ {\isacharbraceleft}H{\isacharbraceright}\ {\isasymunion}\ Wo\ {\isasymor}\ S\ {\isacharequal}\ Wo{\isachardoublequoteclose}\isanewline
\ \ \ \ \ \ \ \ \ \ \ \ \ \ \isacommand{thus}\isamarkupfalse%
\ {\isachardoublequoteopen}sat\ S{\isachardoublequoteclose}\isanewline
\ \ \ \ \ \ \ \ \ \ \ \ \ \ \isacommand{proof}\isamarkupfalse%
\ {\isacharparenleft}rule\ disjE{\isacharparenright}\isanewline
\ \ \ \ \ \ \ \ \ \ \ \ \ \ \ \ \isacommand{assume}\isamarkupfalse%
\ {\isachardoublequoteopen}S\ {\isacharequal}\ {\isacharbraceleft}H{\isacharbraceright}\ {\isasymunion}\ Wo{\isachardoublequoteclose}\isanewline
\ \ \ \ \ \ \ \ \ \ \ \ \ \ \ \ \isacommand{then}\isamarkupfalse%
\ \isacommand{have}\isamarkupfalse%
\ {\isachardoublequoteopen}S\ {\isasymsubseteq}\ {\isacharbraceleft}G{\isacharcomma}H{\isacharcomma}F{\isacharbraceright}\ {\isasymunion}\ Wo{\isachardoublequoteclose}\isanewline
\ \ \ \ \ \ \ \ \ \ \ \ \ \ \ \ \ \ \isacommand{by}\isamarkupfalse%
\ blast\ \isanewline
\ \ \ \ \ \ \ \ \ \ \ \ \ \ \ \ \isacommand{thus}\isamarkupfalse%
\ {\isachardoublequoteopen}sat\ S{\isachardoublequoteclose}\isanewline
\ \ \ \ \ \ \ \ \ \ \ \ \ \ \ \ \ \ \isacommand{using}\isamarkupfalse%
\ {\isacartoucheopen}sat{\isacharparenleft}{\isacharbraceleft}G{\isacharcomma}H{\isacharcomma}F{\isacharbraceright}\ {\isasymunion}\ Wo{\isacharparenright}{\isacartoucheclose}\ \isacommand{by}\isamarkupfalse%
\ {\isacharparenleft}rule\ sat{\isacharunderscore}mono{\isacharparenright}\isanewline
\ \ \ \ \ \ \ \ \ \ \ \ \ \ \isacommand{next}\isamarkupfalse%
\isanewline
\ \ \ \ \ \ \ \ \ \ \ \ \ \ \ \ \isacommand{assume}\isamarkupfalse%
\ {\isachardoublequoteopen}S\ {\isacharequal}\ Wo{\isachardoublequoteclose}\isanewline
\ \ \ \ \ \ \ \ \ \ \ \ \ \ \ \ \isacommand{then}\isamarkupfalse%
\ \isacommand{have}\isamarkupfalse%
\ {\isachardoublequoteopen}S\ {\isasymsubseteq}\ {\isacharbraceleft}G{\isacharcomma}H{\isacharcomma}F{\isacharbraceright}\ {\isasymunion}\ Wo{\isachardoublequoteclose}\isanewline
\ \ \ \ \ \ \ \ \ \ \ \ \ \ \ \ \ \ \isacommand{by}\isamarkupfalse%
\ {\isacharparenleft}simp\ only{\isacharcolon}\ Un{\isacharunderscore}upper{\isadigit{2}}{\isacharparenright}\isanewline
\ \ \ \ \ \ \ \ \ \ \ \ \ \ \ \ \isacommand{thus}\isamarkupfalse%
\ {\isachardoublequoteopen}sat\ S{\isachardoublequoteclose}\isanewline
\ \ \ \ \ \ \ \ \ \ \ \ \ \ \ \ \ \ \isacommand{using}\isamarkupfalse%
\ {\isacartoucheopen}sat{\isacharparenleft}{\isacharbraceleft}G{\isacharcomma}H{\isacharcomma}F{\isacharbraceright}\ {\isasymunion}\ Wo{\isacharparenright}{\isacartoucheclose}\ \isacommand{by}\isamarkupfalse%
\ {\isacharparenleft}rule\ sat{\isacharunderscore}mono{\isacharparenright}\isanewline
\ \ \ \ \ \ \ \ \ \ \ \ \ \ \isacommand{qed}\isamarkupfalse%
\isanewline
\ \ \ \ \ \ \ \ \ \ \ \ \isacommand{qed}\isamarkupfalse%
\isanewline
\ \ \ \ \ \ \ \ \ \ \isacommand{qed}\isamarkupfalse%
\isanewline
\ \ \ \ \ \ \ \ \isacommand{qed}\isamarkupfalse%
\isanewline
\ \ \ \ \ \ \isacommand{qed}\isamarkupfalse%
\isanewline
\ \ \ \ \isacommand{qed}\isamarkupfalse%
\isanewline
\ \ \isacommand{qed}\isamarkupfalse%
\isanewline
\isacommand{qed}\isamarkupfalse%
%
\endisatagproof
{\isafoldproof}%
%
\isadelimproof
%
\endisadelimproof
%
\begin{isamarkuptext}%
Por último, probemos la cuarta condición del lema \isa{{\isadigit{2}}{\isachardot}{\isadigit{0}}{\isachardot}{\isadigit{2}}}: dados \isa{W\ {\isasymin}\ C} y \isa{F} una 
  fórmula de tipo \isa{{\isasymbeta}} con componentes \isa{{\isasymbeta}\isactrlsub {\isadigit{1}}} y \isa{{\isasymbeta}\isactrlsub {\isadigit{2}}} tal que \isa{F\ {\isasymin}\ W}, se tiene que o bien\\ \isa{{\isacharbraceleft}{\isasymbeta}\isactrlsub {\isadigit{1}}{\isacharbraceright}\ {\isasymunion}\ W\ {\isasymin}\ C} 
  o bien \isa{{\isacharbraceleft}{\isasymbeta}\isactrlsub {\isadigit{2}}{\isacharbraceright}\ {\isasymunion}\ W\ {\isasymin}\ C}. 
  
  Por un lado, precisaremos para ello de un lema auxiliar que demuestre que dado \isa{W\ {\isasymin}\ C} y \isa{{\isasymbeta}\isactrlsub i} una 
  fórmula cualquiera tal que \isa{{\isacharbraceleft}{\isasymbeta}\isactrlsub i{\isacharbraceright}\ {\isasymunion}\ W\ {\isasymnotin}\ C}, entonces existe un subconjunto finito \isa{W\isactrlsub i} de \isa{W} tal 
  que el conjunto \isa{{\isacharbraceleft}{\isasymbeta}\isactrlsub i{\isacharbraceright}\ {\isasymunion}\ W\isactrlsub i} no es satisfacible. A su vez, para su demostración emplearemos un lema 
  que prueba que todo conjunto que contiene un subconjunto insatisfacible es también 
  insatisfacible.%
\end{isamarkuptext}\isamarkuptrue%
\isacommand{lemma}\isamarkupfalse%
\ sat{\isacharunderscore}subset{\isacharunderscore}ccontr{\isacharcolon}\isanewline
\ \ \isakeyword{assumes}\ {\isachardoublequoteopen}A\ {\isasymsubseteq}\ B{\isachardoublequoteclose}\isanewline
\ \ \ \ \ \ \ \ \ \ {\isachardoublequoteopen}{\isasymnot}\ sat\ A{\isachardoublequoteclose}\isanewline
\ \ \ \ \ \ \ \ \isakeyword{shows}\ {\isachardoublequoteopen}{\isasymnot}\ sat\ B{\isachardoublequoteclose}\isanewline
%
\isadelimproof
%
\endisadelimproof
%
\isatagproof
\isacommand{proof}\isamarkupfalse%
\ {\isacharminus}\isanewline
\ \ \isacommand{have}\isamarkupfalse%
\ {\isachardoublequoteopen}A\ {\isasymsubseteq}\ B\ {\isasymand}\ sat\ B\ {\isasymlongrightarrow}\ sat\ A{\isachardoublequoteclose}\isanewline
\ \ \ \ \isacommand{using}\isamarkupfalse%
\ sat{\isacharunderscore}mono\ \isacommand{by}\isamarkupfalse%
\ blast\isanewline
\ \ \isacommand{then}\isamarkupfalse%
\ \isacommand{have}\isamarkupfalse%
\ {\isachardoublequoteopen}{\isasymnot}{\isacharparenleft}A\ {\isasymsubseteq}\ B\ {\isasymand}\ sat\ B{\isacharparenright}{\isachardoublequoteclose}\isanewline
\ \ \ \ \isacommand{using}\isamarkupfalse%
\ assms{\isacharparenleft}{\isadigit{2}}{\isacharparenright}\ \isacommand{by}\isamarkupfalse%
\ {\isacharparenleft}rule\ mt{\isacharparenright}\isanewline
\ \ \isacommand{then}\isamarkupfalse%
\ \isacommand{have}\isamarkupfalse%
\ {\isachardoublequoteopen}{\isasymnot}{\isacharparenleft}A\ {\isasymsubseteq}\ B{\isacharparenright}\ {\isasymor}\ {\isasymnot}{\isacharparenleft}sat\ B{\isacharparenright}{\isachardoublequoteclose}\isanewline
\ \ \ \ \isacommand{by}\isamarkupfalse%
\ {\isacharparenleft}simp\ only{\isacharcolon}\ de{\isacharunderscore}Morgan{\isacharunderscore}conj{\isacharparenright}\isanewline
\ \ \isacommand{thus}\isamarkupfalse%
\ {\isacharquery}thesis\isanewline
\ \ \isacommand{proof}\isamarkupfalse%
\ {\isacharparenleft}rule\ disjE{\isacharparenright}\isanewline
\ \ \ \ \isacommand{assume}\isamarkupfalse%
\ {\isachardoublequoteopen}{\isasymnot}{\isacharparenleft}A\ {\isasymsubseteq}\ B{\isacharparenright}{\isachardoublequoteclose}\isanewline
\ \ \ \ \isacommand{thus}\isamarkupfalse%
\ {\isacharquery}thesis\isanewline
\ \ \ \ \ \ \isacommand{using}\isamarkupfalse%
\ assms{\isacharparenleft}{\isadigit{1}}{\isacharparenright}\ \isacommand{by}\isamarkupfalse%
\ {\isacharparenleft}rule\ notE{\isacharparenright}\isanewline
\ \ \isacommand{next}\isamarkupfalse%
\isanewline
\ \ \ \ \isacommand{assume}\isamarkupfalse%
\ {\isachardoublequoteopen}{\isasymnot}{\isacharparenleft}sat\ B{\isacharparenright}{\isachardoublequoteclose}\isanewline
\ \ \ \ \isacommand{thus}\isamarkupfalse%
\ {\isacharquery}thesis\isanewline
\ \ \ \ \ \ \isacommand{by}\isamarkupfalse%
\ this\isanewline
\ \ \isacommand{qed}\isamarkupfalse%
\isanewline
\isacommand{qed}\isamarkupfalse%
%
\endisatagproof
{\isafoldproof}%
%
\isadelimproof
%
\endisadelimproof
%
\begin{isamarkuptext}%
De este modo, podemos demostrar que dados \isa{W\ {\isasymin}\ C} y \isa{{\isasymbeta}\isactrlsub i} una fórmula cualquiera tal que 
  \isa{{\isacharbraceleft}{\isasymbeta}\isactrlsub i{\isacharbraceright}\ {\isasymunion}\ W\ {\isasymnotin}\ C}, entonces existe un subconjunto finito \isa{W\isactrlsub i} de \isa{W} tal que el conjunto \isa{{\isacharbraceleft}{\isasymbeta}\isactrlsub i{\isacharbraceright}\ {\isasymunion}\ W\isactrlsub F} 
  no es satisfacible.%
\end{isamarkuptext}\isamarkuptrue%
\isacommand{lemma}\isamarkupfalse%
\ not{\isacharunderscore}colecComp{\isacharcolon}\isanewline
\ \ \isakeyword{assumes}\ {\isachardoublequoteopen}W\ {\isasymin}\ colecComp{\isachardoublequoteclose}\isanewline
\ \ \ \ \ \ \ \ \ \ {\isachardoublequoteopen}{\isacharbraceleft}Gi{\isacharbraceright}\ {\isasymunion}\ W\ {\isasymnotin}\ colecComp{\isachardoublequoteclose}\isanewline
\ \ \ \ \ \ \ \ \isakeyword{shows}\ {\isachardoublequoteopen}{\isasymexists}Wi\ {\isasymsubseteq}\ W{\isachardot}\ finite\ Wi\ {\isasymand}\ {\isasymnot}{\isacharparenleft}sat\ {\isacharparenleft}{\isacharbraceleft}Gi{\isacharbraceright}\ {\isasymunion}\ Wi{\isacharparenright}{\isacharparenright}{\isachardoublequoteclose}\isanewline
%
\isadelimproof
%
\endisadelimproof
%
\isatagproof
\isacommand{proof}\isamarkupfalse%
\ {\isacharminus}\isanewline
\ \ \isacommand{have}\isamarkupfalse%
\ WCol{\isacharcolon}{\isachardoublequoteopen}{\isasymforall}S{\isacharprime}\ {\isasymsubseteq}\ W{\isachardot}\ finite\ S{\isacharprime}\ {\isasymlongrightarrow}\ sat\ S{\isacharprime}{\isachardoublequoteclose}\isanewline
\ \ \ \ \isacommand{using}\isamarkupfalse%
\ assms{\isacharparenleft}{\isadigit{1}}{\isacharparenright}\ \isacommand{unfolding}\isamarkupfalse%
\ colecComp\ fin{\isacharunderscore}sat{\isacharunderscore}def\ \isacommand{by}\isamarkupfalse%
\ {\isacharparenleft}rule\ CollectD{\isacharparenright}\ \isanewline
\ \ \isacommand{have}\isamarkupfalse%
\ {\isachardoublequoteopen}{\isasymnot}{\isacharparenleft}{\isasymforall}Wo\ {\isasymsubseteq}\ {\isacharbraceleft}Gi{\isacharbraceright}\ {\isasymunion}\ W{\isachardot}\ finite\ Wo\ {\isasymlongrightarrow}\ sat\ Wo{\isacharparenright}{\isachardoublequoteclose}\isanewline
\ \ \ \ \isacommand{using}\isamarkupfalse%
\ assms{\isacharparenleft}{\isadigit{2}}{\isacharparenright}\ \isacommand{unfolding}\isamarkupfalse%
\ colecComp\ fin{\isacharunderscore}sat{\isacharunderscore}def\ \isacommand{by}\isamarkupfalse%
\ {\isacharparenleft}simp\ only{\isacharcolon}\ mem{\isacharunderscore}Collect{\isacharunderscore}eq\ simp{\isacharunderscore}thms{\isacharparenleft}{\isadigit{8}}{\isacharparenright}{\isacharparenright}\isanewline
\ \ \isacommand{then}\isamarkupfalse%
\ \isacommand{have}\isamarkupfalse%
\ {\isachardoublequoteopen}{\isasymexists}Wo\ {\isasymsubseteq}\ {\isacharbraceleft}Gi{\isacharbraceright}\ {\isasymunion}\ W{\isachardot}\ {\isasymnot}{\isacharparenleft}finite\ Wo\ {\isasymlongrightarrow}\ sat\ Wo{\isacharparenright}{\isachardoublequoteclose}\isanewline
\ \ \ \ \isacommand{by}\isamarkupfalse%
\ {\isacharparenleft}rule\ sall{\isacharunderscore}simps{\isacharunderscore}not{\isacharunderscore}all{\isacharparenright}\isanewline
\ \ \isacommand{then}\isamarkupfalse%
\ \isacommand{have}\isamarkupfalse%
\ Ex{\isadigit{1}}{\isacharcolon}{\isachardoublequoteopen}{\isasymexists}Wo\ {\isasymsubseteq}\ {\isacharbraceleft}Gi{\isacharbraceright}\ {\isasymunion}\ W{\isachardot}\ finite\ Wo\ {\isasymand}\ {\isasymnot}{\isacharparenleft}sat\ Wo{\isacharparenright}{\isachardoublequoteclose}\isanewline
\ \ \ \ \isacommand{by}\isamarkupfalse%
\ {\isacharparenleft}simp\ only{\isacharcolon}\ not{\isacharunderscore}imp{\isacharparenright}\isanewline
\ \ \isacommand{obtain}\isamarkupfalse%
\ Wo\ \isakeyword{where}\ {\isachardoublequoteopen}Wo\ {\isasymsubseteq}\ {\isacharbraceleft}Gi{\isacharbraceright}\ {\isasymunion}\ W{\isachardoublequoteclose}\ \isakeyword{and}\ C{\isadigit{1}}{\isacharcolon}{\isachardoublequoteopen}finite\ Wo\ {\isasymand}\ {\isasymnot}{\isacharparenleft}sat\ Wo{\isacharparenright}{\isachardoublequoteclose}\isanewline
\ \ \ \ \isacommand{using}\isamarkupfalse%
\ Ex{\isadigit{1}}\ \isacommand{by}\isamarkupfalse%
\ {\isacharparenleft}rule\ subexE{\isacharparenright}\isanewline
\ \ \isacommand{have}\isamarkupfalse%
\ {\isachardoublequoteopen}finite\ Wo{\isachardoublequoteclose}\isanewline
\ \ \ \ \isacommand{using}\isamarkupfalse%
\ C{\isadigit{1}}\ \isacommand{by}\isamarkupfalse%
\ {\isacharparenleft}rule\ conjunct{\isadigit{1}}{\isacharparenright}\isanewline
\ \ \isacommand{have}\isamarkupfalse%
\ {\isachardoublequoteopen}{\isasymnot}{\isacharparenleft}sat\ Wo{\isacharparenright}{\isachardoublequoteclose}\isanewline
\ \ \ \ \isacommand{using}\isamarkupfalse%
\ C{\isadigit{1}}\ \isacommand{by}\isamarkupfalse%
\ {\isacharparenleft}rule\ conjunct{\isadigit{2}}{\isacharparenright}\isanewline
\ \ \isacommand{have}\isamarkupfalse%
\ {\isachardoublequoteopen}Wo\ {\isasymsubseteq}\ insert\ Gi\ W{\isachardoublequoteclose}\isanewline
\ \ \ \ \isacommand{using}\isamarkupfalse%
\ {\isacartoucheopen}Wo\ {\isasymsubseteq}\ {\isacharbraceleft}Gi{\isacharbraceright}\ {\isasymunion}\ W{\isacartoucheclose}\ \isacommand{by}\isamarkupfalse%
\ blast\isanewline
\ \ \isacommand{have}\isamarkupfalse%
\ Ex{\isadigit{2}}{\isacharcolon}{\isachardoublequoteopen}{\isasymexists}Wo{\isacharprime}\ {\isasymsubseteq}\ W{\isachardot}\ finite\ Wo{\isacharprime}\ {\isasymand}\ {\isacharparenleft}Wo\ {\isacharequal}\ insert\ Gi\ Wo{\isacharprime}\ {\isasymor}\ Wo\ {\isacharequal}\ Wo{\isacharprime}{\isacharparenright}{\isachardoublequoteclose}\isanewline
\ \ \ \ \isacommand{using}\isamarkupfalse%
\ {\isacartoucheopen}finite\ Wo{\isacartoucheclose}\ {\isacartoucheopen}Wo\ {\isasymsubseteq}\ insert\ Gi\ W{\isacartoucheclose}\ \isacommand{by}\isamarkupfalse%
\ {\isacharparenleft}rule\ finite{\isacharunderscore}subset{\isacharunderscore}insert{\isadigit{1}}{\isacharparenright}\isanewline
\ \ \isacommand{obtain}\isamarkupfalse%
\ Wo{\isacharprime}\ \isakeyword{where}\ {\isachardoublequoteopen}Wo{\isacharprime}\ {\isasymsubseteq}\ W{\isachardoublequoteclose}\ \isakeyword{and}\ C{\isadigit{2}}{\isacharcolon}{\isachardoublequoteopen}finite\ Wo{\isacharprime}\ {\isasymand}\ {\isacharparenleft}Wo\ {\isacharequal}\ {\isacharbraceleft}Gi{\isacharbraceright}\ {\isasymunion}\ Wo{\isacharprime}\ {\isasymor}\ Wo\ {\isacharequal}\ Wo{\isacharprime}{\isacharparenright}{\isachardoublequoteclose}\isanewline
\ \ \ \ \isacommand{using}\isamarkupfalse%
\ Ex{\isadigit{2}}\ \isacommand{by}\isamarkupfalse%
\ blast\isanewline
\ \ \isacommand{have}\isamarkupfalse%
\ {\isachardoublequoteopen}finite\ Wo{\isacharprime}{\isachardoublequoteclose}\isanewline
\ \ \ \ \isacommand{using}\isamarkupfalse%
\ C{\isadigit{2}}\ \isacommand{by}\isamarkupfalse%
\ {\isacharparenleft}rule\ conjunct{\isadigit{1}}{\isacharparenright}\isanewline
\ \ \isacommand{have}\isamarkupfalse%
\ {\isachardoublequoteopen}Wo\ {\isacharequal}\ {\isacharbraceleft}Gi{\isacharbraceright}\ {\isasymunion}\ Wo{\isacharprime}\ {\isasymor}\ Wo\ {\isacharequal}\ Wo{\isacharprime}{\isachardoublequoteclose}\isanewline
\ \ \ \ \isacommand{using}\isamarkupfalse%
\ C{\isadigit{2}}\ \isacommand{by}\isamarkupfalse%
\ {\isacharparenleft}rule\ conjunct{\isadigit{2}}{\isacharparenright}\isanewline
\ \ \isacommand{thus}\isamarkupfalse%
\ {\isacharquery}thesis\isanewline
\ \ \isacommand{proof}\isamarkupfalse%
\ {\isacharparenleft}rule\ disjE{\isacharparenright}\isanewline
\ \ \ \ \isacommand{assume}\isamarkupfalse%
\ {\isachardoublequoteopen}Wo\ {\isacharequal}\ {\isacharbraceleft}Gi{\isacharbraceright}\ {\isasymunion}\ Wo{\isacharprime}{\isachardoublequoteclose}\isanewline
\ \ \ \ \isacommand{then}\isamarkupfalse%
\ \isacommand{have}\isamarkupfalse%
\ {\isachardoublequoteopen}{\isasymnot}{\isacharparenleft}sat\ {\isacharparenleft}{\isacharbraceleft}Gi{\isacharbraceright}\ {\isasymunion}\ Wo{\isacharprime}{\isacharparenright}{\isacharparenright}{\isachardoublequoteclose}\ \isanewline
\ \ \ \ \ \ \isacommand{using}\isamarkupfalse%
\ {\isacartoucheopen}{\isasymnot}\ sat\ Wo{\isacartoucheclose}\ \isacommand{by}\isamarkupfalse%
\ {\isacharparenleft}simp\ only{\isacharcolon}\ {\isacartoucheopen}Wo\ {\isacharequal}\ {\isacharbraceleft}Gi{\isacharbraceright}\ {\isasymunion}\ Wo{\isacharprime}{\isacartoucheclose}\ simp{\isacharunderscore}thms{\isacharparenleft}{\isadigit{8}}{\isacharparenright}{\isacharparenright}\isanewline
\ \ \ \ \isacommand{have}\isamarkupfalse%
\ {\isachardoublequoteopen}finite\ Wo{\isacharprime}\ {\isasymand}\ {\isasymnot}{\isacharparenleft}sat\ {\isacharparenleft}{\isacharbraceleft}Gi{\isacharbraceright}\ {\isasymunion}\ Wo{\isacharprime}{\isacharparenright}{\isacharparenright}{\isachardoublequoteclose}\isanewline
\ \ \ \ \ \ \isacommand{using}\isamarkupfalse%
\ {\isacartoucheopen}finite\ Wo{\isacharprime}{\isacartoucheclose}\ {\isacartoucheopen}{\isasymnot}{\isacharparenleft}sat\ {\isacharparenleft}{\isacharbraceleft}Gi{\isacharbraceright}\ {\isasymunion}\ Wo{\isacharprime}{\isacharparenright}{\isacharparenright}{\isacartoucheclose}\ \isacommand{by}\isamarkupfalse%
\ {\isacharparenleft}rule\ conjI{\isacharparenright}\isanewline
\ \ \ \ \isacommand{thus}\isamarkupfalse%
\ {\isacharquery}thesis\isanewline
\ \ \ \ \ \ \isacommand{using}\isamarkupfalse%
\ {\isacartoucheopen}Wo{\isacharprime}\ {\isasymsubseteq}\ W{\isacartoucheclose}\ \isacommand{by}\isamarkupfalse%
\ {\isacharparenleft}rule\ subexI{\isacharparenright}\isanewline
\ \ \isacommand{next}\isamarkupfalse%
\isanewline
\ \ \ \ \isacommand{assume}\isamarkupfalse%
\ {\isachardoublequoteopen}Wo\ {\isacharequal}\ Wo{\isacharprime}{\isachardoublequoteclose}\isanewline
\ \ \ \ \isacommand{then}\isamarkupfalse%
\ \isacommand{have}\isamarkupfalse%
\ {\isachardoublequoteopen}{\isasymnot}\ {\isacharparenleft}sat\ Wo{\isacharprime}{\isacharparenright}{\isachardoublequoteclose}\isanewline
\ \ \ \ \ \ \isacommand{using}\isamarkupfalse%
\ {\isacartoucheopen}{\isasymnot}\ sat\ Wo{\isacartoucheclose}\ \isacommand{by}\isamarkupfalse%
\ {\isacharparenleft}simp\ only{\isacharcolon}\ {\isacartoucheopen}Wo\ {\isacharequal}\ Wo{\isacharprime}{\isacartoucheclose}\ simp{\isacharunderscore}thms{\isacharparenleft}{\isadigit{8}}{\isacharparenright}{\isacharparenright}\isanewline
\ \ \ \ \isacommand{have}\isamarkupfalse%
\ {\isachardoublequoteopen}Wo{\isacharprime}\ {\isasymsubseteq}\ {\isacharbraceleft}Gi{\isacharbraceright}\ {\isasymunion}\ Wo{\isacharprime}{\isachardoublequoteclose}\isanewline
\ \ \ \ \ \ \isacommand{by}\isamarkupfalse%
\ blast\isanewline
\ \ \ \ \isacommand{then}\isamarkupfalse%
\ \isacommand{have}\isamarkupfalse%
\ {\isachardoublequoteopen}{\isasymnot}\ {\isacharparenleft}sat\ {\isacharparenleft}{\isacharbraceleft}Gi{\isacharbraceright}\ {\isasymunion}\ Wo{\isacharprime}{\isacharparenright}{\isacharparenright}{\isachardoublequoteclose}\isanewline
\ \ \ \ \ \ \isacommand{using}\isamarkupfalse%
\ {\isacartoucheopen}{\isasymnot}\ {\isacharparenleft}sat\ Wo{\isacharprime}{\isacharparenright}{\isacartoucheclose}\ \isacommand{by}\isamarkupfalse%
\ {\isacharparenleft}rule\ sat{\isacharunderscore}subset{\isacharunderscore}ccontr{\isacharparenright}\isanewline
\ \ \ \ \isacommand{have}\isamarkupfalse%
\ {\isachardoublequoteopen}finite\ Wo{\isacharprime}\ {\isasymand}\ {\isasymnot}{\isacharparenleft}sat\ {\isacharparenleft}{\isacharbraceleft}Gi{\isacharbraceright}\ {\isasymunion}\ Wo{\isacharprime}{\isacharparenright}{\isacharparenright}{\isachardoublequoteclose}\isanewline
\ \ \ \ \ \ \isacommand{using}\isamarkupfalse%
\ {\isacartoucheopen}finite\ Wo{\isacharprime}{\isacartoucheclose}\ {\isacartoucheopen}{\isasymnot}{\isacharparenleft}sat\ {\isacharparenleft}{\isacharbraceleft}Gi{\isacharbraceright}\ {\isasymunion}\ Wo{\isacharprime}{\isacharparenright}{\isacharparenright}{\isacartoucheclose}\ \isacommand{by}\isamarkupfalse%
\ {\isacharparenleft}rule\ conjI{\isacharparenright}\isanewline
\ \ \ \ \isacommand{thus}\isamarkupfalse%
\ {\isacharquery}thesis\isanewline
\ \ \ \ \ \ \isacommand{using}\isamarkupfalse%
\ {\isacartoucheopen}Wo{\isacharprime}\ {\isasymsubseteq}\ W{\isacartoucheclose}\ \isacommand{by}\isamarkupfalse%
\ {\isacharparenleft}rule\ subexI{\isacharparenright}\isanewline
\ \ \isacommand{qed}\isamarkupfalse%
\isanewline
\isacommand{qed}\isamarkupfalse%
%
\endisatagproof
{\isafoldproof}%
%
\isadelimproof
%
\endisadelimproof
%
\begin{isamarkuptext}%
Por otro lado, para demostrar la cuarta condición del lema \isa{{\isadigit{2}}{\isachardot}{\isadigit{0}}{\isachardot}{\isadigit{2}}} que demuestra que \isa{C} 
  verifica la propiedad de consistencia proposicional, precisaremos de un lema auxiliar que prueba 
  que dados \isa{W\ {\isasymin}\ C}, \isa{F} una fórmula de tipo \isa{{\isasymbeta}} y componentes \isa{{\isasymbeta}\isactrlsub {\isadigit{1}}} y \isa{{\isasymbeta}\isactrlsub {\isadigit{2}}} tal que \isa{F\ {\isasymin}\ W} y \isa{W\isactrlsub {\isadigit{0}}} un 
  subconjunto finito de \isa{W}, entonces se tiene que o bien \isa{{\isacharbraceleft}{\isasymbeta}\isactrlsub {\isadigit{1}}{\isacharcomma}F{\isacharbraceright}\ {\isasymunion}\ W\isactrlsub {\isadigit{0}}} es satisfacible o bien 
  \isa{{\isacharbraceleft}{\isasymbeta}\isactrlsub {\isadigit{2}}{\isacharcomma}F{\isacharbraceright}\ {\isasymunion}\ W\isactrlsub {\isadigit{0}}} es satisfacible. Vamos a probar que, en efecto, se tiene el resultado para cada tipo de fórmula \isa{{\isasymbeta}}.

  En primer lugar, probemos que dados \isa{W\ {\isasymin}\ C}, una fórmula \isa{F\ {\isacharequal}\ G\ {\isasymand}\ H} para ciertas fórmulas \isa{G} y 
  \isa{H} tal que \isa{F\ {\isasymin}\ W} y \isa{W\isactrlsub {\isadigit{0}}} un subconjunto finito de \isa{W}, entonces se tiene que o bien 
  \isa{{\isacharbraceleft}G{\isacharcomma}F{\isacharbraceright}\ {\isasymunion}\ W\isactrlsub {\isadigit{0}}} es satisfacible o bien \isa{{\isacharbraceleft}H{\isacharcomma}F{\isacharbraceright}\ {\isasymunion}\ W\isactrlsub {\isadigit{0}}} es satisfacible.%
\end{isamarkuptext}\isamarkuptrue%
\isacommand{lemma}\isamarkupfalse%
\ pcp{\isacharunderscore}colecComp{\isacharunderscore}DIS{\isacharunderscore}sat{\isadigit{1}}{\isacharcolon}\isanewline
\ \ \isakeyword{assumes}\ {\isachardoublequoteopen}W\ {\isasymin}\ colecComp{\isachardoublequoteclose}\isanewline
\ \ \ \ \ \ \ \ \ \ {\isachardoublequoteopen}F\ {\isacharequal}\ G\ \isactrlbold {\isasymor}\ H{\isachardoublequoteclose}\isanewline
\ \ \ \ \ \ \ \ \ \ {\isachardoublequoteopen}F\ {\isasymin}\ W{\isachardoublequoteclose}\isanewline
\ \ \ \ \ \ \ \ \ \ {\isachardoublequoteopen}finite\ Wo{\isachardoublequoteclose}\isanewline
\ \ \ \ \ \ \ \ \ \ {\isachardoublequoteopen}Wo\ {\isasymsubseteq}\ W{\isachardoublequoteclose}\isanewline
\ \ \ \ \ \ \ \ \isakeyword{shows}\ {\isachardoublequoteopen}sat\ {\isacharparenleft}{\isacharbraceleft}G{\isacharcomma}F{\isacharbraceright}\ {\isasymunion}\ Wo{\isacharparenright}\ {\isasymor}\ sat\ {\isacharparenleft}{\isacharbraceleft}H{\isacharcomma}F{\isacharbraceright}\ {\isasymunion}\ Wo{\isacharparenright}{\isachardoublequoteclose}\isanewline
%
\isadelimproof
%
\endisadelimproof
%
\isatagproof
\isacommand{proof}\isamarkupfalse%
\ {\isacharminus}\isanewline
\ \ \isacommand{have}\isamarkupfalse%
\ {\isachardoublequoteopen}sat\ {\isacharparenleft}{\isacharbraceleft}F{\isacharbraceright}\ {\isasymunion}\ Wo{\isacharparenright}{\isachardoublequoteclose}\isanewline
\ \ \ \ \isacommand{using}\isamarkupfalse%
\ assms{\isacharparenleft}{\isadigit{1}}{\isacharcomma}{\isadigit{3}}{\isacharcomma}{\isadigit{4}}{\isacharcomma}{\isadigit{5}}{\isacharparenright}\ \isacommand{by}\isamarkupfalse%
\ {\isacharparenleft}rule\ pcp{\isacharunderscore}colecComp{\isacharunderscore}elem{\isacharunderscore}sat{\isacharparenright}\isanewline
\ \ \isacommand{have}\isamarkupfalse%
\ {\isachardoublequoteopen}F\ {\isasymin}\ {\isacharbraceleft}F{\isacharbraceright}\ {\isasymunion}\ Wo{\isachardoublequoteclose}\isanewline
\ \ \ \ \isacommand{by}\isamarkupfalse%
\ simp\ \isanewline
\ \ \isacommand{have}\isamarkupfalse%
\ Ex{\isadigit{1}}{\isacharcolon}{\isachardoublequoteopen}{\isasymexists}{\isasymA}{\isachardot}\ {\isasymforall}F\ {\isasymin}\ {\isacharparenleft}{\isacharbraceleft}F{\isacharbraceright}\ {\isasymunion}\ Wo{\isacharparenright}{\isachardot}\ {\isasymA}\ {\isasymTurnstile}\ F{\isachardoublequoteclose}\isanewline
\ \ \ \ \isacommand{using}\isamarkupfalse%
\ {\isacartoucheopen}sat\ {\isacharparenleft}{\isacharbraceleft}F{\isacharbraceright}\ {\isasymunion}\ Wo{\isacharparenright}{\isacartoucheclose}\ \isacommand{by}\isamarkupfalse%
\ {\isacharparenleft}simp\ only{\isacharcolon}\ sat{\isacharunderscore}def{\isacharparenright}\isanewline
\ \ \isacommand{obtain}\isamarkupfalse%
\ {\isasymA}\ \isakeyword{where}\ Forall{\isadigit{1}}{\isacharcolon}{\isachardoublequoteopen}{\isasymforall}F\ {\isasymin}\ {\isacharparenleft}{\isacharbraceleft}F{\isacharbraceright}\ {\isasymunion}\ Wo{\isacharparenright}{\isachardot}\ {\isasymA}\ {\isasymTurnstile}\ F{\isachardoublequoteclose}\isanewline
\ \ \ \ \isacommand{using}\isamarkupfalse%
\ Ex{\isadigit{1}}\ \isacommand{by}\isamarkupfalse%
\ {\isacharparenleft}rule\ exE{\isacharparenright}\isanewline
\ \ \isacommand{have}\isamarkupfalse%
\ {\isachardoublequoteopen}{\isasymA}\ {\isasymTurnstile}\ F{\isachardoublequoteclose}\isanewline
\ \ \ \ \isacommand{using}\isamarkupfalse%
\ Forall{\isadigit{1}}\ {\isacartoucheopen}F\ {\isasymin}\ {\isacharbraceleft}F{\isacharbraceright}\ {\isasymunion}\ Wo{\isacartoucheclose}\ \isacommand{by}\isamarkupfalse%
\ {\isacharparenleft}rule\ bspec{\isacharparenright}\isanewline
\ \ \isacommand{then}\isamarkupfalse%
\ \isacommand{have}\isamarkupfalse%
\ {\isachardoublequoteopen}{\isasymA}\ {\isasymTurnstile}\ {\isacharparenleft}G\ \isactrlbold {\isasymor}\ H{\isacharparenright}{\isachardoublequoteclose}\isanewline
\ \ \ \ \isacommand{using}\isamarkupfalse%
\ assms{\isacharparenleft}{\isadigit{2}}{\isacharparenright}\ \isacommand{by}\isamarkupfalse%
\ {\isacharparenleft}simp\ only{\isacharcolon}\ {\isacartoucheopen}{\isasymA}\ {\isasymTurnstile}\ F{\isacartoucheclose}{\isacharparenright}\isanewline
\ \ \isacommand{then}\isamarkupfalse%
\ \isacommand{have}\isamarkupfalse%
\ {\isachardoublequoteopen}{\isasymA}\ {\isasymTurnstile}\ G\ {\isasymor}\ {\isasymA}\ {\isasymTurnstile}\ H{\isachardoublequoteclose}\isanewline
\ \ \ \ \isacommand{by}\isamarkupfalse%
\ {\isacharparenleft}simp\ only{\isacharcolon}\ formula{\isacharunderscore}semantics{\isachardot}simps{\isacharparenleft}{\isadigit{5}}{\isacharparenright}{\isacharparenright}\isanewline
\ \ \isacommand{thus}\isamarkupfalse%
\ {\isacharquery}thesis\isanewline
\ \ \isacommand{proof}\isamarkupfalse%
\ {\isacharparenleft}rule\ disjE{\isacharparenright}\isanewline
\ \ \ \ \isacommand{assume}\isamarkupfalse%
\ {\isachardoublequoteopen}{\isasymA}\ {\isasymTurnstile}\ G{\isachardoublequoteclose}\isanewline
\ \ \ \ \isacommand{then}\isamarkupfalse%
\ \isacommand{have}\isamarkupfalse%
\ {\isachardoublequoteopen}{\isasymforall}F\ {\isasymin}\ {\isacharbraceleft}G{\isacharbraceright}{\isachardot}\ {\isasymA}\ {\isasymTurnstile}\ F{\isachardoublequoteclose}\isanewline
\ \ \ \ \ \ \isacommand{by}\isamarkupfalse%
\ simp\isanewline
\ \ \ \ \isacommand{then}\isamarkupfalse%
\ \isacommand{have}\isamarkupfalse%
\ {\isachardoublequoteopen}{\isasymforall}F\ {\isasymin}\ {\isacharparenleft}{\isacharbraceleft}G{\isacharbraceright}\ {\isasymunion}\ {\isacharparenleft}{\isacharbraceleft}F{\isacharbraceright}\ {\isasymunion}\ Wo{\isacharparenright}{\isacharparenright}{\isachardot}\ {\isasymA}\ {\isasymTurnstile}\ F{\isachardoublequoteclose}\isanewline
\ \ \ \ \ \ \isacommand{using}\isamarkupfalse%
\ Forall{\isadigit{1}}\ \isacommand{by}\isamarkupfalse%
\ {\isacharparenleft}rule\ ball{\isacharunderscore}Un{\isacharparenright}\isanewline
\ \ \ \ \isacommand{then}\isamarkupfalse%
\ \isacommand{have}\isamarkupfalse%
\ {\isachardoublequoteopen}{\isasymforall}F\ {\isasymin}\ {\isacharbraceleft}G{\isacharcomma}F{\isacharbraceright}\ {\isasymunion}\ Wo{\isachardot}\ {\isasymA}\ {\isasymTurnstile}\ F{\isachardoublequoteclose}\isanewline
\ \ \ \ \ \ \isacommand{by}\isamarkupfalse%
\ simp\ \isanewline
\ \ \ \ \isacommand{then}\isamarkupfalse%
\ \isacommand{have}\isamarkupfalse%
\ {\isachardoublequoteopen}{\isasymexists}{\isasymA}{\isachardot}\ {\isasymforall}F\ {\isasymin}\ {\isacharparenleft}{\isacharbraceleft}G{\isacharcomma}F{\isacharbraceright}\ {\isasymunion}\ Wo{\isacharparenright}{\isachardot}\ {\isasymA}\ {\isasymTurnstile}\ F{\isachardoublequoteclose}\isanewline
\ \ \ \ \ \ \isacommand{by}\isamarkupfalse%
\ {\isacharparenleft}iprover\ intro{\isacharcolon}\ exI{\isacharparenright}\isanewline
\ \ \ \ \isacommand{then}\isamarkupfalse%
\ \isacommand{have}\isamarkupfalse%
\ {\isachardoublequoteopen}sat\ {\isacharparenleft}{\isacharbraceleft}G{\isacharcomma}F{\isacharbraceright}\ {\isasymunion}\ Wo{\isacharparenright}{\isachardoublequoteclose}\isanewline
\ \ \ \ \ \ \isacommand{by}\isamarkupfalse%
\ {\isacharparenleft}simp\ only{\isacharcolon}\ sat{\isacharunderscore}def{\isacharparenright}\isanewline
\ \ \ \ \isacommand{thus}\isamarkupfalse%
\ {\isacharquery}thesis\isanewline
\ \ \ \ \ \ \isacommand{by}\isamarkupfalse%
\ {\isacharparenleft}rule\ disjI{\isadigit{1}}{\isacharparenright}\isanewline
\ \ \isacommand{next}\isamarkupfalse%
\isanewline
\ \ \ \ \isacommand{assume}\isamarkupfalse%
\ {\isachardoublequoteopen}{\isasymA}\ {\isasymTurnstile}\ H{\isachardoublequoteclose}\isanewline
\ \ \ \ \isacommand{then}\isamarkupfalse%
\ \isacommand{have}\isamarkupfalse%
\ {\isachardoublequoteopen}{\isasymforall}F\ {\isasymin}\ {\isacharbraceleft}H{\isacharbraceright}{\isachardot}\ {\isasymA}\ {\isasymTurnstile}\ F{\isachardoublequoteclose}\isanewline
\ \ \ \ \ \ \isacommand{by}\isamarkupfalse%
\ simp\isanewline
\ \ \ \ \isacommand{then}\isamarkupfalse%
\ \isacommand{have}\isamarkupfalse%
\ {\isachardoublequoteopen}{\isasymforall}F\ {\isasymin}\ {\isacharparenleft}{\isacharbraceleft}H{\isacharbraceright}\ {\isasymunion}\ {\isacharparenleft}{\isacharbraceleft}F{\isacharbraceright}\ {\isasymunion}\ Wo{\isacharparenright}{\isacharparenright}{\isachardot}\ {\isasymA}\ {\isasymTurnstile}\ F{\isachardoublequoteclose}\isanewline
\ \ \ \ \ \ \isacommand{using}\isamarkupfalse%
\ Forall{\isadigit{1}}\ \isacommand{by}\isamarkupfalse%
\ {\isacharparenleft}rule\ ball{\isacharunderscore}Un{\isacharparenright}\isanewline
\ \ \ \ \isacommand{then}\isamarkupfalse%
\ \isacommand{have}\isamarkupfalse%
\ {\isachardoublequoteopen}{\isasymforall}F\ {\isasymin}\ {\isacharbraceleft}H{\isacharcomma}F{\isacharbraceright}\ {\isasymunion}\ Wo{\isachardot}\ {\isasymA}\ {\isasymTurnstile}\ F{\isachardoublequoteclose}\isanewline
\ \ \ \ \ \ \isacommand{by}\isamarkupfalse%
\ simp\isanewline
\ \ \ \ \isacommand{then}\isamarkupfalse%
\ \isacommand{have}\isamarkupfalse%
\ {\isachardoublequoteopen}{\isasymexists}{\isasymA}{\isachardot}\ {\isasymforall}F\ {\isasymin}\ {\isacharparenleft}{\isacharbraceleft}H{\isacharcomma}F{\isacharbraceright}\ {\isasymunion}\ Wo{\isacharparenright}{\isachardot}\ {\isasymA}\ {\isasymTurnstile}\ F{\isachardoublequoteclose}\isanewline
\ \ \ \ \ \ \isacommand{by}\isamarkupfalse%
\ {\isacharparenleft}iprover\ intro{\isacharcolon}\ exI{\isacharparenright}\isanewline
\ \ \ \ \isacommand{then}\isamarkupfalse%
\ \isacommand{have}\isamarkupfalse%
\ {\isachardoublequoteopen}sat\ {\isacharparenleft}{\isacharbraceleft}H{\isacharcomma}F{\isacharbraceright}\ {\isasymunion}\ Wo{\isacharparenright}{\isachardoublequoteclose}\isanewline
\ \ \ \ \ \ \isacommand{by}\isamarkupfalse%
\ {\isacharparenleft}simp\ only{\isacharcolon}\ sat{\isacharunderscore}def{\isacharparenright}\isanewline
\ \ \ \ \isacommand{thus}\isamarkupfalse%
\ {\isacharquery}thesis\isanewline
\ \ \ \ \ \ \isacommand{by}\isamarkupfalse%
\ {\isacharparenleft}rule\ disjI{\isadigit{2}}{\isacharparenright}\isanewline
\ \ \isacommand{qed}\isamarkupfalse%
\isanewline
\isacommand{qed}\isamarkupfalse%
%
\endisatagproof
{\isafoldproof}%
%
\isadelimproof
%
\endisadelimproof
%
\begin{isamarkuptext}%
El siguiente lema auxiliar demuestra que dados \isa{W\ {\isasymin}\ C}, una fórmula \isa{F\ {\isacharequal}\ G\ {\isasymlongrightarrow}\ H} para ciertas 
  fórmulas \isa{G} y \isa{H} tal que \isa{F\ {\isasymin}\ W} y \isa{W\isactrlsub {\isadigit{0}}} un subconjunto finito de \isa{W}, entonces se tiene que o 
  bien \isa{{\isacharbraceleft}{\isasymnot}\ G{\isacharcomma}F{\isacharbraceright}\ {\isasymunion}\ W\isactrlsub {\isadigit{0}}} es satisfacible o bien \isa{{\isacharbraceleft}H{\isacharcomma}F{\isacharbraceright}\ {\isasymunion}\ W\isactrlsub {\isadigit{0}}} es satisfacible.%
\end{isamarkuptext}\isamarkuptrue%
\isacommand{lemma}\isamarkupfalse%
\ pcp{\isacharunderscore}colecComp{\isacharunderscore}DIS{\isacharunderscore}sat{\isadigit{2}}{\isacharcolon}\isanewline
\ \ \isakeyword{assumes}\ {\isachardoublequoteopen}W\ {\isasymin}\ colecComp{\isachardoublequoteclose}\isanewline
\ \ \ \ \ \ \ \ \ \ {\isachardoublequoteopen}F\ {\isacharequal}\ G\ \isactrlbold {\isasymrightarrow}\ H{\isachardoublequoteclose}\isanewline
\ \ \ \ \ \ \ \ \ \ {\isachardoublequoteopen}F\ {\isasymin}\ W{\isachardoublequoteclose}\isanewline
\ \ \ \ \ \ \ \ \ \ {\isachardoublequoteopen}finite\ Wo{\isachardoublequoteclose}\isanewline
\ \ \ \ \ \ \ \ \ \ {\isachardoublequoteopen}Wo\ {\isasymsubseteq}\ W{\isachardoublequoteclose}\isanewline
\ \ \ \ \ \ \ \ \isakeyword{shows}\ {\isachardoublequoteopen}sat\ {\isacharparenleft}{\isacharbraceleft}\isactrlbold {\isasymnot}\ G{\isacharcomma}F{\isacharbraceright}\ {\isasymunion}\ Wo{\isacharparenright}\ {\isasymor}\ sat\ {\isacharparenleft}{\isacharbraceleft}H{\isacharcomma}F{\isacharbraceright}\ {\isasymunion}\ Wo{\isacharparenright}{\isachardoublequoteclose}\isanewline
%
\isadelimproof
%
\endisadelimproof
%
\isatagproof
\isacommand{proof}\isamarkupfalse%
\ {\isacharminus}\isanewline
\ \ \isacommand{have}\isamarkupfalse%
\ {\isachardoublequoteopen}sat\ {\isacharparenleft}{\isacharbraceleft}F{\isacharbraceright}\ {\isasymunion}\ Wo{\isacharparenright}{\isachardoublequoteclose}\isanewline
\ \ \ \ \isacommand{using}\isamarkupfalse%
\ assms{\isacharparenleft}{\isadigit{1}}{\isacharcomma}{\isadigit{3}}{\isacharcomma}{\isadigit{4}}{\isacharcomma}{\isadigit{5}}{\isacharparenright}\ \isacommand{by}\isamarkupfalse%
\ {\isacharparenleft}rule\ pcp{\isacharunderscore}colecComp{\isacharunderscore}elem{\isacharunderscore}sat{\isacharparenright}\isanewline
\ \ \isacommand{have}\isamarkupfalse%
\ {\isachardoublequoteopen}F\ {\isasymin}\ {\isacharbraceleft}F{\isacharbraceright}\ {\isasymunion}\ Wo{\isachardoublequoteclose}\isanewline
\ \ \ \ \isacommand{by}\isamarkupfalse%
\ simp\isanewline
\ \ \isacommand{have}\isamarkupfalse%
\ Ex{\isadigit{1}}{\isacharcolon}{\isachardoublequoteopen}{\isasymexists}{\isasymA}{\isachardot}\ {\isasymforall}F\ {\isasymin}\ {\isacharparenleft}{\isacharbraceleft}F{\isacharbraceright}\ {\isasymunion}\ Wo{\isacharparenright}{\isachardot}\ {\isasymA}\ {\isasymTurnstile}\ F{\isachardoublequoteclose}\isanewline
\ \ \ \ \isacommand{using}\isamarkupfalse%
\ {\isacartoucheopen}sat\ {\isacharparenleft}{\isacharbraceleft}F{\isacharbraceright}\ {\isasymunion}\ Wo{\isacharparenright}{\isacartoucheclose}\ \isacommand{by}\isamarkupfalse%
\ {\isacharparenleft}simp\ only{\isacharcolon}\ sat{\isacharunderscore}def{\isacharparenright}\isanewline
\ \ \isacommand{obtain}\isamarkupfalse%
\ {\isasymA}\ \isakeyword{where}\ Forall{\isadigit{1}}{\isacharcolon}{\isachardoublequoteopen}{\isasymforall}F\ {\isasymin}\ {\isacharparenleft}{\isacharbraceleft}F{\isacharbraceright}\ {\isasymunion}\ Wo{\isacharparenright}{\isachardot}\ {\isasymA}\ {\isasymTurnstile}\ F{\isachardoublequoteclose}\isanewline
\ \ \ \ \isacommand{using}\isamarkupfalse%
\ Ex{\isadigit{1}}\ \isacommand{by}\isamarkupfalse%
\ {\isacharparenleft}rule\ exE{\isacharparenright}\isanewline
\ \ \isacommand{have}\isamarkupfalse%
\ {\isachardoublequoteopen}{\isasymA}\ {\isasymTurnstile}\ F{\isachardoublequoteclose}\isanewline
\ \ \ \ \isacommand{using}\isamarkupfalse%
\ Forall{\isadigit{1}}\ {\isacartoucheopen}F\ {\isasymin}\ {\isacharbraceleft}F{\isacharbraceright}\ {\isasymunion}\ Wo{\isacartoucheclose}\ \isacommand{by}\isamarkupfalse%
\ {\isacharparenleft}rule\ bspec{\isacharparenright}\isanewline
\ \ \isacommand{then}\isamarkupfalse%
\ \isacommand{have}\isamarkupfalse%
\ {\isachardoublequoteopen}{\isasymA}\ {\isasymTurnstile}\ {\isacharparenleft}G\ \isactrlbold {\isasymrightarrow}\ H{\isacharparenright}{\isachardoublequoteclose}\isanewline
\ \ \ \ \isacommand{using}\isamarkupfalse%
\ assms{\isacharparenleft}{\isadigit{2}}{\isacharparenright}\ \isacommand{by}\isamarkupfalse%
\ {\isacharparenleft}simp\ only{\isacharcolon}\ {\isacartoucheopen}{\isasymA}\ {\isasymTurnstile}\ F{\isacartoucheclose}{\isacharparenright}\isanewline
\ \ \isacommand{then}\isamarkupfalse%
\ \isacommand{have}\isamarkupfalse%
\ {\isachardoublequoteopen}{\isasymA}\ {\isasymTurnstile}\ G\ {\isasymlongrightarrow}\ {\isasymA}\ {\isasymTurnstile}\ H{\isachardoublequoteclose}\isanewline
\ \ \ \ \isacommand{by}\isamarkupfalse%
\ {\isacharparenleft}simp\ only{\isacharcolon}\ formula{\isacharunderscore}semantics{\isachardot}simps{\isacharparenleft}{\isadigit{6}}{\isacharparenright}{\isacharparenright}\isanewline
\ \ \isacommand{then}\isamarkupfalse%
\ \isacommand{have}\isamarkupfalse%
\ {\isachardoublequoteopen}{\isacharparenleft}{\isasymnot}{\isacharparenleft}{\isasymnot}\ {\isasymA}\ {\isasymTurnstile}\ G{\isacharparenright}{\isacharparenright}\ {\isasymlongrightarrow}\ {\isasymA}\ {\isasymTurnstile}\ H{\isachardoublequoteclose}\isanewline
\ \ \ \ \isacommand{by}\isamarkupfalse%
\ {\isacharparenleft}simp\ only{\isacharcolon}\ not{\isacharunderscore}not{\isacharparenright}\isanewline
\ \ \isacommand{then}\isamarkupfalse%
\ \isacommand{have}\isamarkupfalse%
\ {\isachardoublequoteopen}{\isacharparenleft}{\isasymnot}\ {\isasymA}\ {\isasymTurnstile}\ G{\isacharparenright}\ {\isasymor}\ {\isasymA}\ {\isasymTurnstile}\ H{\isachardoublequoteclose}\isanewline
\ \ \ \ \isacommand{by}\isamarkupfalse%
\ {\isacharparenleft}simp\ only{\isacharcolon}\ disj{\isacharunderscore}imp{\isacharparenright}\isanewline
\ \ \isacommand{thus}\isamarkupfalse%
\ {\isacharquery}thesis\isanewline
\ \ \isacommand{proof}\isamarkupfalse%
\ {\isacharparenleft}rule\ disjE{\isacharparenright}\isanewline
\ \ \ \ \isacommand{assume}\isamarkupfalse%
\ {\isachardoublequoteopen}{\isasymnot}\ {\isasymA}\ {\isasymTurnstile}\ G{\isachardoublequoteclose}\isanewline
\ \ \ \ \isacommand{then}\isamarkupfalse%
\ \isacommand{have}\isamarkupfalse%
\ {\isachardoublequoteopen}{\isasymA}\ {\isasymTurnstile}\ {\isacharparenleft}\isactrlbold {\isasymnot}\ G{\isacharparenright}{\isachardoublequoteclose}\isanewline
\ \ \ \ \ \ \isacommand{by}\isamarkupfalse%
\ {\isacharparenleft}simp\ only{\isacharcolon}\ formula{\isacharunderscore}semantics{\isachardot}simps{\isacharparenleft}{\isadigit{3}}{\isacharparenright}\ simp{\isacharunderscore}thms{\isacharparenleft}{\isadigit{8}}{\isacharparenright}{\isacharparenright}\isanewline
\ \ \ \ \isacommand{then}\isamarkupfalse%
\ \isacommand{have}\isamarkupfalse%
\ {\isachardoublequoteopen}{\isasymforall}F\ {\isasymin}\ {\isacharbraceleft}\isactrlbold {\isasymnot}\ G{\isacharbraceright}{\isachardot}\ {\isasymA}\ {\isasymTurnstile}\ F{\isachardoublequoteclose}\isanewline
\ \ \ \ \ \ \isacommand{by}\isamarkupfalse%
\ simp\isanewline
\ \ \ \ \isacommand{then}\isamarkupfalse%
\ \isacommand{have}\isamarkupfalse%
\ {\isachardoublequoteopen}{\isasymforall}F\ {\isasymin}\ {\isacharparenleft}{\isacharbraceleft}\isactrlbold {\isasymnot}\ G{\isacharbraceright}\ {\isasymunion}\ {\isacharparenleft}{\isacharbraceleft}F{\isacharbraceright}\ {\isasymunion}\ Wo{\isacharparenright}{\isacharparenright}{\isachardot}\ {\isasymA}\ {\isasymTurnstile}\ F{\isachardoublequoteclose}\isanewline
\ \ \ \ \ \ \isacommand{using}\isamarkupfalse%
\ Forall{\isadigit{1}}\ \isacommand{by}\isamarkupfalse%
\ {\isacharparenleft}rule\ ball{\isacharunderscore}Un{\isacharparenright}\isanewline
\ \ \ \ \isacommand{then}\isamarkupfalse%
\ \isacommand{have}\isamarkupfalse%
\ {\isachardoublequoteopen}{\isasymforall}F\ {\isasymin}\ {\isacharbraceleft}\isactrlbold {\isasymnot}\ G{\isacharcomma}F{\isacharbraceright}\ {\isasymunion}\ Wo{\isachardot}\ {\isasymA}\ {\isasymTurnstile}\ F{\isachardoublequoteclose}\isanewline
\ \ \ \ \ \ \isacommand{by}\isamarkupfalse%
\ simp\isanewline
\ \ \ \ \isacommand{then}\isamarkupfalse%
\ \isacommand{have}\isamarkupfalse%
\ {\isachardoublequoteopen}{\isasymexists}{\isasymA}{\isachardot}\ {\isasymforall}F\ {\isasymin}\ {\isacharparenleft}{\isacharbraceleft}\isactrlbold {\isasymnot}\ G{\isacharcomma}F{\isacharbraceright}\ {\isasymunion}\ Wo{\isacharparenright}{\isachardot}\ {\isasymA}\ {\isasymTurnstile}\ F{\isachardoublequoteclose}\isanewline
\ \ \ \ \ \ \isacommand{by}\isamarkupfalse%
\ {\isacharparenleft}iprover\ intro{\isacharcolon}\ exI{\isacharparenright}\isanewline
\ \ \ \ \isacommand{then}\isamarkupfalse%
\ \isacommand{have}\isamarkupfalse%
\ {\isachardoublequoteopen}sat\ {\isacharparenleft}{\isacharbraceleft}\isactrlbold {\isasymnot}\ G{\isacharcomma}F{\isacharbraceright}\ {\isasymunion}\ Wo{\isacharparenright}{\isachardoublequoteclose}\isanewline
\ \ \ \ \ \ \isacommand{by}\isamarkupfalse%
\ {\isacharparenleft}simp\ only{\isacharcolon}\ sat{\isacharunderscore}def{\isacharparenright}\isanewline
\ \ \ \ \isacommand{thus}\isamarkupfalse%
\ {\isacharquery}thesis\isanewline
\ \ \ \ \ \ \isacommand{by}\isamarkupfalse%
\ {\isacharparenleft}rule\ disjI{\isadigit{1}}{\isacharparenright}\isanewline
\ \ \isacommand{next}\isamarkupfalse%
\isanewline
\ \ \ \ \isacommand{assume}\isamarkupfalse%
\ {\isachardoublequoteopen}{\isasymA}\ {\isasymTurnstile}\ H{\isachardoublequoteclose}\isanewline
\ \ \ \ \isacommand{then}\isamarkupfalse%
\ \isacommand{have}\isamarkupfalse%
\ {\isachardoublequoteopen}{\isasymforall}F\ {\isasymin}\ {\isacharbraceleft}H{\isacharbraceright}{\isachardot}\ {\isasymA}\ {\isasymTurnstile}\ F{\isachardoublequoteclose}\isanewline
\ \ \ \ \ \ \isacommand{by}\isamarkupfalse%
\ simp\isanewline
\ \ \ \ \isacommand{then}\isamarkupfalse%
\ \isacommand{have}\isamarkupfalse%
\ {\isachardoublequoteopen}{\isasymforall}F\ {\isasymin}\ {\isacharparenleft}{\isacharbraceleft}H{\isacharbraceright}\ {\isasymunion}\ {\isacharparenleft}{\isacharbraceleft}F{\isacharbraceright}\ {\isasymunion}\ Wo{\isacharparenright}{\isacharparenright}{\isachardot}\ {\isasymA}\ {\isasymTurnstile}\ F{\isachardoublequoteclose}\isanewline
\ \ \ \ \ \ \isacommand{using}\isamarkupfalse%
\ Forall{\isadigit{1}}\ \isacommand{by}\isamarkupfalse%
\ {\isacharparenleft}rule\ ball{\isacharunderscore}Un{\isacharparenright}\isanewline
\ \ \ \ \isacommand{then}\isamarkupfalse%
\ \isacommand{have}\isamarkupfalse%
\ {\isachardoublequoteopen}{\isasymforall}F\ {\isasymin}\ {\isacharbraceleft}H{\isacharcomma}F{\isacharbraceright}\ {\isasymunion}\ Wo{\isachardot}\ {\isasymA}\ {\isasymTurnstile}\ F{\isachardoublequoteclose}\isanewline
\ \ \ \ \ \ \isacommand{by}\isamarkupfalse%
\ simp\isanewline
\ \ \ \ \isacommand{then}\isamarkupfalse%
\ \isacommand{have}\isamarkupfalse%
\ {\isachardoublequoteopen}{\isasymexists}{\isasymA}{\isachardot}\ {\isasymforall}F\ {\isasymin}\ {\isacharparenleft}{\isacharbraceleft}H{\isacharcomma}F{\isacharbraceright}\ {\isasymunion}\ Wo{\isacharparenright}{\isachardot}\ {\isasymA}\ {\isasymTurnstile}\ F{\isachardoublequoteclose}\isanewline
\ \ \ \ \ \ \isacommand{by}\isamarkupfalse%
\ {\isacharparenleft}iprover\ intro{\isacharcolon}\ exI{\isacharparenright}\isanewline
\ \ \ \ \isacommand{then}\isamarkupfalse%
\ \isacommand{have}\isamarkupfalse%
\ {\isachardoublequoteopen}sat\ {\isacharparenleft}{\isacharbraceleft}H{\isacharcomma}F{\isacharbraceright}\ {\isasymunion}\ Wo{\isacharparenright}{\isachardoublequoteclose}\isanewline
\ \ \ \ \ \ \isacommand{by}\isamarkupfalse%
\ {\isacharparenleft}simp\ only{\isacharcolon}\ sat{\isacharunderscore}def{\isacharparenright}\isanewline
\ \ \ \ \isacommand{thus}\isamarkupfalse%
\ {\isacharquery}thesis\isanewline
\ \ \ \ \ \ \isacommand{by}\isamarkupfalse%
\ {\isacharparenleft}rule\ disjI{\isadigit{2}}{\isacharparenright}\isanewline
\ \ \isacommand{qed}\isamarkupfalse%
\isanewline
\isacommand{qed}\isamarkupfalse%
%
\endisatagproof
{\isafoldproof}%
%
\isadelimproof
%
\endisadelimproof
%
\begin{isamarkuptext}%
Por otro lado probemos que dados \isa{W\ {\isasymin}\ C}, una fórmula \isa{F\ {\isacharequal}\ {\isasymnot}{\isacharparenleft}G\ {\isasymand}\ H{\isacharparenright}} para ciertas fórmulas 
  \isa{G} y \isa{H} tal que \isa{F\ {\isasymin}\ W} y \isa{W\isactrlsub {\isadigit{0}}} un subconjunto finito de \isa{W}, entonces se tiene que o bien 
  \isa{{\isacharbraceleft}{\isasymnot}\ G{\isacharcomma}F{\isacharbraceright}\ {\isasymunion}\ W\isactrlsub {\isadigit{0}}} es satisfacible o bien \isa{{\isacharbraceleft}{\isasymnot}\ H{\isacharcomma}F{\isacharbraceright}\ {\isasymunion}\ W\isactrlsub {\isadigit{0}}} es satisfacible.%
\end{isamarkuptext}\isamarkuptrue%
\isacommand{lemma}\isamarkupfalse%
\ pcp{\isacharunderscore}colecComp{\isacharunderscore}DIS{\isacharunderscore}sat{\isadigit{3}}{\isacharcolon}\isanewline
\ \ \isakeyword{assumes}\ {\isachardoublequoteopen}W\ {\isasymin}\ colecComp{\isachardoublequoteclose}\isanewline
\ \ \ \ \ \ \ \ \ \ {\isachardoublequoteopen}F\ {\isacharequal}\ \isactrlbold {\isasymnot}\ {\isacharparenleft}G\ \isactrlbold {\isasymand}\ H{\isacharparenright}{\isachardoublequoteclose}\isanewline
\ \ \ \ \ \ \ \ \ \ {\isachardoublequoteopen}F\ {\isasymin}\ W{\isachardoublequoteclose}\isanewline
\ \ \ \ \ \ \ \ \ \ {\isachardoublequoteopen}finite\ Wo{\isachardoublequoteclose}\isanewline
\ \ \ \ \ \ \ \ \ \ {\isachardoublequoteopen}Wo\ {\isasymsubseteq}\ W{\isachardoublequoteclose}\isanewline
\ \ \ \ \ \ \ \ \isakeyword{shows}\ {\isachardoublequoteopen}sat\ {\isacharparenleft}{\isacharbraceleft}\isactrlbold {\isasymnot}\ G{\isacharcomma}F{\isacharbraceright}\ {\isasymunion}\ Wo{\isacharparenright}\ {\isasymor}\ sat\ {\isacharparenleft}{\isacharbraceleft}\isactrlbold {\isasymnot}\ H{\isacharcomma}F{\isacharbraceright}\ {\isasymunion}\ Wo{\isacharparenright}{\isachardoublequoteclose}\isanewline
%
\isadelimproof
%
\endisadelimproof
%
\isatagproof
\isacommand{proof}\isamarkupfalse%
\ {\isacharminus}\isanewline
\ \ \isacommand{have}\isamarkupfalse%
\ {\isachardoublequoteopen}sat\ {\isacharparenleft}{\isacharbraceleft}F{\isacharbraceright}\ {\isasymunion}\ Wo{\isacharparenright}{\isachardoublequoteclose}\isanewline
\ \ \ \ \isacommand{using}\isamarkupfalse%
\ assms{\isacharparenleft}{\isadigit{1}}{\isacharcomma}{\isadigit{3}}{\isacharcomma}{\isadigit{4}}{\isacharcomma}{\isadigit{5}}{\isacharparenright}\ \isacommand{by}\isamarkupfalse%
\ {\isacharparenleft}rule\ pcp{\isacharunderscore}colecComp{\isacharunderscore}elem{\isacharunderscore}sat{\isacharparenright}\isanewline
\ \ \isacommand{have}\isamarkupfalse%
\ {\isachardoublequoteopen}F\ {\isasymin}\ {\isacharbraceleft}F{\isacharbraceright}\ {\isasymunion}\ Wo{\isachardoublequoteclose}\isanewline
\ \ \ \ \isacommand{by}\isamarkupfalse%
\ simp\isanewline
\ \ \isacommand{have}\isamarkupfalse%
\ Ex{\isadigit{1}}{\isacharcolon}{\isachardoublequoteopen}{\isasymexists}{\isasymA}{\isachardot}\ {\isasymforall}F\ {\isasymin}\ {\isacharparenleft}{\isacharbraceleft}F{\isacharbraceright}\ {\isasymunion}\ Wo{\isacharparenright}{\isachardot}\ {\isasymA}\ {\isasymTurnstile}\ F{\isachardoublequoteclose}\isanewline
\ \ \ \ \isacommand{using}\isamarkupfalse%
\ {\isacartoucheopen}sat\ {\isacharparenleft}{\isacharbraceleft}F{\isacharbraceright}\ {\isasymunion}\ Wo{\isacharparenright}{\isacartoucheclose}\ \isacommand{by}\isamarkupfalse%
\ {\isacharparenleft}simp\ only{\isacharcolon}\ sat{\isacharunderscore}def{\isacharparenright}\isanewline
\ \ \isacommand{obtain}\isamarkupfalse%
\ {\isasymA}\ \isakeyword{where}\ Forall{\isadigit{1}}{\isacharcolon}{\isachardoublequoteopen}{\isasymforall}F\ {\isasymin}\ {\isacharparenleft}{\isacharbraceleft}F{\isacharbraceright}\ {\isasymunion}\ Wo{\isacharparenright}{\isachardot}\ {\isasymA}\ {\isasymTurnstile}\ F{\isachardoublequoteclose}\isanewline
\ \ \ \ \isacommand{using}\isamarkupfalse%
\ Ex{\isadigit{1}}\ \isacommand{by}\isamarkupfalse%
\ {\isacharparenleft}rule\ exE{\isacharparenright}\isanewline
\ \ \isacommand{have}\isamarkupfalse%
\ {\isachardoublequoteopen}{\isasymA}\ {\isasymTurnstile}\ F{\isachardoublequoteclose}\isanewline
\ \ \ \ \isacommand{using}\isamarkupfalse%
\ Forall{\isadigit{1}}\ {\isacartoucheopen}F\ {\isasymin}\ {\isacharbraceleft}F{\isacharbraceright}\ {\isasymunion}\ Wo{\isacartoucheclose}\ \isacommand{by}\isamarkupfalse%
\ {\isacharparenleft}rule\ bspec{\isacharparenright}\isanewline
\ \ \isacommand{then}\isamarkupfalse%
\ \isacommand{have}\isamarkupfalse%
\ {\isachardoublequoteopen}{\isasymA}\ {\isasymTurnstile}\ \isactrlbold {\isasymnot}\ {\isacharparenleft}G\ \isactrlbold {\isasymand}\ H{\isacharparenright}{\isachardoublequoteclose}\isanewline
\ \ \ \ \isacommand{using}\isamarkupfalse%
\ assms{\isacharparenleft}{\isadigit{2}}{\isacharparenright}\ \isacommand{by}\isamarkupfalse%
\ {\isacharparenleft}simp\ only{\isacharcolon}\ {\isacartoucheopen}{\isasymA}\ {\isasymTurnstile}\ F{\isacartoucheclose}{\isacharparenright}\isanewline
\ \ \isacommand{then}\isamarkupfalse%
\ \isacommand{have}\isamarkupfalse%
\ {\isachardoublequoteopen}{\isasymnot}\ {\isacharparenleft}{\isasymA}\ {\isasymTurnstile}\ {\isacharparenleft}G\ \isactrlbold {\isasymand}\ H{\isacharparenright}{\isacharparenright}{\isachardoublequoteclose}\isanewline
\ \ \ \ \isacommand{by}\isamarkupfalse%
\ {\isacharparenleft}simp\ only{\isacharcolon}\ formula{\isacharunderscore}semantics{\isachardot}simps{\isacharparenleft}{\isadigit{3}}{\isacharparenright}\ simp{\isacharunderscore}thms{\isacharparenleft}{\isadigit{8}}{\isacharparenright}{\isacharparenright}\isanewline
\ \ \isacommand{then}\isamarkupfalse%
\ \isacommand{have}\isamarkupfalse%
\ {\isachardoublequoteopen}{\isasymnot}{\isacharparenleft}{\isasymA}\ {\isasymTurnstile}\ G\ {\isasymand}\ {\isasymA}\ {\isasymTurnstile}\ H{\isacharparenright}{\isachardoublequoteclose}\isanewline
\ \ \ \ \isacommand{by}\isamarkupfalse%
\ {\isacharparenleft}simp\ only{\isacharcolon}\ formula{\isacharunderscore}semantics{\isachardot}simps{\isacharparenleft}{\isadigit{4}}{\isacharparenright}\ simp{\isacharunderscore}thms{\isacharparenleft}{\isadigit{8}}{\isacharparenright}{\isacharparenright}\isanewline
\ \ \isacommand{then}\isamarkupfalse%
\ \isacommand{have}\isamarkupfalse%
\ {\isachardoublequoteopen}{\isasymnot}\ {\isacharparenleft}{\isasymA}\ {\isasymTurnstile}\ G{\isacharparenright}\ {\isasymor}\ {\isasymnot}\ {\isacharparenleft}{\isasymA}\ {\isasymTurnstile}\ H{\isacharparenright}{\isachardoublequoteclose}\isanewline
\ \ \ \ \isacommand{by}\isamarkupfalse%
\ {\isacharparenleft}simp\ only{\isacharcolon}\ de{\isacharunderscore}Morgan{\isacharunderscore}conj{\isacharparenright}\isanewline
\ \ \isacommand{thus}\isamarkupfalse%
\ {\isacharquery}thesis\isanewline
\ \ \isacommand{proof}\isamarkupfalse%
\ {\isacharparenleft}rule\ disjE{\isacharparenright}\isanewline
\ \ \ \ \isacommand{assume}\isamarkupfalse%
\ {\isachardoublequoteopen}{\isasymnot}\ {\isacharparenleft}{\isasymA}\ {\isasymTurnstile}\ G{\isacharparenright}{\isachardoublequoteclose}\isanewline
\ \ \ \ \isacommand{then}\isamarkupfalse%
\ \isacommand{have}\isamarkupfalse%
\ {\isachardoublequoteopen}{\isasymA}\ {\isasymTurnstile}\ \isactrlbold {\isasymnot}\ G{\isachardoublequoteclose}\isanewline
\ \ \ \ \ \ \isacommand{by}\isamarkupfalse%
\ {\isacharparenleft}simp\ only{\isacharcolon}\ formula{\isacharunderscore}semantics{\isachardot}simps{\isacharparenleft}{\isadigit{3}}{\isacharparenright}\ simp{\isacharunderscore}thms{\isacharparenleft}{\isadigit{8}}{\isacharparenright}{\isacharparenright}\isanewline
\ \ \ \ \isacommand{then}\isamarkupfalse%
\ \isacommand{have}\isamarkupfalse%
\ {\isachardoublequoteopen}{\isasymforall}F\ {\isasymin}\ {\isacharbraceleft}\isactrlbold {\isasymnot}\ G{\isacharbraceright}{\isachardot}\ {\isasymA}\ {\isasymTurnstile}\ F{\isachardoublequoteclose}\isanewline
\ \ \ \ \ \ \isacommand{by}\isamarkupfalse%
\ simp\isanewline
\ \ \ \ \isacommand{then}\isamarkupfalse%
\ \isacommand{have}\isamarkupfalse%
\ {\isachardoublequoteopen}{\isasymforall}F\ {\isasymin}\ {\isacharparenleft}{\isacharbraceleft}\isactrlbold {\isasymnot}\ G{\isacharbraceright}\ {\isasymunion}\ {\isacharparenleft}{\isacharbraceleft}F{\isacharbraceright}\ {\isasymunion}\ Wo{\isacharparenright}{\isacharparenright}{\isachardot}\ {\isasymA}\ {\isasymTurnstile}\ F{\isachardoublequoteclose}\isanewline
\ \ \ \ \ \ \isacommand{using}\isamarkupfalse%
\ Forall{\isadigit{1}}\ \isacommand{by}\isamarkupfalse%
\ {\isacharparenleft}rule\ ball{\isacharunderscore}Un{\isacharparenright}\isanewline
\ \ \ \ \isacommand{then}\isamarkupfalse%
\ \isacommand{have}\isamarkupfalse%
\ {\isachardoublequoteopen}{\isasymforall}F\ {\isasymin}\ {\isacharbraceleft}\isactrlbold {\isasymnot}\ G{\isacharcomma}F{\isacharbraceright}\ {\isasymunion}\ Wo{\isachardot}\ {\isasymA}\ {\isasymTurnstile}\ F{\isachardoublequoteclose}\isanewline
\ \ \ \ \ \ \isacommand{by}\isamarkupfalse%
\ simp\isanewline
\ \ \ \ \isacommand{then}\isamarkupfalse%
\ \isacommand{have}\isamarkupfalse%
\ {\isachardoublequoteopen}{\isasymexists}{\isasymA}{\isachardot}\ {\isasymforall}F\ {\isasymin}\ {\isacharparenleft}{\isacharbraceleft}\isactrlbold {\isasymnot}\ G{\isacharcomma}F{\isacharbraceright}\ {\isasymunion}\ Wo{\isacharparenright}{\isachardot}\ {\isasymA}\ {\isasymTurnstile}\ F{\isachardoublequoteclose}\isanewline
\ \ \ \ \ \ \isacommand{by}\isamarkupfalse%
\ {\isacharparenleft}iprover\ intro{\isacharcolon}\ exI{\isacharparenright}\isanewline
\ \ \ \ \isacommand{then}\isamarkupfalse%
\ \isacommand{have}\isamarkupfalse%
\ {\isachardoublequoteopen}sat\ {\isacharparenleft}{\isacharbraceleft}\isactrlbold {\isasymnot}\ G{\isacharcomma}F{\isacharbraceright}\ {\isasymunion}\ Wo{\isacharparenright}{\isachardoublequoteclose}\isanewline
\ \ \ \ \ \ \isacommand{by}\isamarkupfalse%
\ {\isacharparenleft}simp\ only{\isacharcolon}\ sat{\isacharunderscore}def{\isacharparenright}\isanewline
\ \ \ \ \isacommand{thus}\isamarkupfalse%
\ {\isacharquery}thesis\isanewline
\ \ \ \ \ \ \isacommand{by}\isamarkupfalse%
\ {\isacharparenleft}rule\ disjI{\isadigit{1}}{\isacharparenright}\isanewline
\ \ \isacommand{next}\isamarkupfalse%
\isanewline
\ \ \ \ \isacommand{assume}\isamarkupfalse%
\ {\isachardoublequoteopen}{\isasymnot}\ {\isacharparenleft}{\isasymA}\ {\isasymTurnstile}\ H{\isacharparenright}{\isachardoublequoteclose}\isanewline
\ \ \ \ \isacommand{then}\isamarkupfalse%
\ \isacommand{have}\isamarkupfalse%
\ {\isachardoublequoteopen}{\isasymA}\ {\isasymTurnstile}\ \isactrlbold {\isasymnot}\ H{\isachardoublequoteclose}\isanewline
\ \ \ \ \ \ \isacommand{by}\isamarkupfalse%
\ {\isacharparenleft}simp\ only{\isacharcolon}\ formula{\isacharunderscore}semantics{\isachardot}simps{\isacharparenleft}{\isadigit{3}}{\isacharparenright}\ simp{\isacharunderscore}thms{\isacharparenleft}{\isadigit{8}}{\isacharparenright}{\isacharparenright}\isanewline
\ \ \ \ \isacommand{then}\isamarkupfalse%
\ \isacommand{have}\isamarkupfalse%
\ {\isachardoublequoteopen}{\isasymforall}F\ {\isasymin}\ {\isacharbraceleft}\isactrlbold {\isasymnot}\ H{\isacharbraceright}{\isachardot}\ {\isasymA}\ {\isasymTurnstile}\ F{\isachardoublequoteclose}\isanewline
\ \ \ \ \ \ \isacommand{by}\isamarkupfalse%
\ simp\isanewline
\ \ \ \ \isacommand{then}\isamarkupfalse%
\ \isacommand{have}\isamarkupfalse%
\ {\isachardoublequoteopen}{\isasymforall}F\ {\isasymin}\ {\isacharparenleft}{\isacharbraceleft}\isactrlbold {\isasymnot}\ H{\isacharbraceright}\ {\isasymunion}\ {\isacharparenleft}{\isacharbraceleft}F{\isacharbraceright}\ {\isasymunion}\ Wo{\isacharparenright}{\isacharparenright}{\isachardot}\ {\isasymA}\ {\isasymTurnstile}\ F{\isachardoublequoteclose}\isanewline
\ \ \ \ \ \ \isacommand{using}\isamarkupfalse%
\ Forall{\isadigit{1}}\ \isacommand{by}\isamarkupfalse%
\ {\isacharparenleft}rule\ ball{\isacharunderscore}Un{\isacharparenright}\isanewline
\ \ \ \ \isacommand{then}\isamarkupfalse%
\ \isacommand{have}\isamarkupfalse%
\ {\isachardoublequoteopen}{\isasymforall}F\ {\isasymin}\ {\isacharbraceleft}\isactrlbold {\isasymnot}\ H{\isacharcomma}F{\isacharbraceright}\ {\isasymunion}\ Wo{\isachardot}\ {\isasymA}\ {\isasymTurnstile}\ F{\isachardoublequoteclose}\isanewline
\ \ \ \ \ \ \isacommand{by}\isamarkupfalse%
\ simp\isanewline
\ \ \ \ \isacommand{then}\isamarkupfalse%
\ \isacommand{have}\isamarkupfalse%
\ {\isachardoublequoteopen}{\isasymexists}{\isasymA}{\isachardot}\ {\isasymforall}F\ {\isasymin}\ {\isacharparenleft}{\isacharbraceleft}\isactrlbold {\isasymnot}\ H{\isacharcomma}F{\isacharbraceright}\ {\isasymunion}\ Wo{\isacharparenright}{\isachardot}\ {\isasymA}\ {\isasymTurnstile}\ F{\isachardoublequoteclose}\isanewline
\ \ \ \ \ \ \isacommand{by}\isamarkupfalse%
\ {\isacharparenleft}iprover\ intro{\isacharcolon}\ exI{\isacharparenright}\isanewline
\ \ \ \ \isacommand{then}\isamarkupfalse%
\ \isacommand{have}\isamarkupfalse%
\ {\isachardoublequoteopen}sat\ {\isacharparenleft}{\isacharbraceleft}\isactrlbold {\isasymnot}\ H{\isacharcomma}F{\isacharbraceright}\ {\isasymunion}\ Wo{\isacharparenright}{\isachardoublequoteclose}\isanewline
\ \ \ \ \ \ \isacommand{by}\isamarkupfalse%
\ {\isacharparenleft}simp\ only{\isacharcolon}\ sat{\isacharunderscore}def{\isacharparenright}\isanewline
\ \ \ \ \isacommand{thus}\isamarkupfalse%
\ {\isacharquery}thesis\isanewline
\ \ \ \ \ \ \isacommand{by}\isamarkupfalse%
\ {\isacharparenleft}rule\ disjI{\isadigit{2}}{\isacharparenright}\isanewline
\ \ \isacommand{qed}\isamarkupfalse%
\isanewline
\isacommand{qed}\isamarkupfalse%
%
\endisatagproof
{\isafoldproof}%
%
\isadelimproof
%
\endisadelimproof
%
\begin{isamarkuptext}%
Por último, probemos que dados \isa{W\ {\isasymin}\ C}, una fórmula \isa{F\ {\isacharequal}\ {\isasymnot}\ {\isacharparenleft}{\isasymnot}\ G{\isacharparenright}} para cierta fórmula \isa{G} tal 
  que \isa{F\ {\isasymin}\ W} y \isa{W\isactrlsub {\isadigit{0}}} un subconjunto finito de \isa{W}, entonces se tiene que \isa{{\isacharbraceleft}G{\isacharcomma}F{\isacharbraceright}\ {\isasymunion}\ W\isactrlsub {\isadigit{0}}} es 
  satisfacible.%
\end{isamarkuptext}\isamarkuptrue%
\isacommand{lemma}\isamarkupfalse%
\ pcp{\isacharunderscore}colecComp{\isacharunderscore}DIS{\isacharunderscore}sat{\isadigit{4}}{\isacharcolon}\isanewline
\ \ \isakeyword{assumes}\ {\isachardoublequoteopen}W\ {\isasymin}\ colecComp{\isachardoublequoteclose}\isanewline
\ \ \ \ \ \ \ \ \ \ {\isachardoublequoteopen}F\ {\isacharequal}\ \isactrlbold {\isasymnot}\ {\isacharparenleft}\isactrlbold {\isasymnot}\ G{\isacharparenright}{\isachardoublequoteclose}\isanewline
\ \ \ \ \ \ \ \ \ \ {\isachardoublequoteopen}F\ {\isasymin}\ W{\isachardoublequoteclose}\isanewline
\ \ \ \ \ \ \ \ \ \ {\isachardoublequoteopen}finite\ Wo{\isachardoublequoteclose}\isanewline
\ \ \ \ \ \ \ \ \ \ {\isachardoublequoteopen}Wo\ {\isasymsubseteq}\ W{\isachardoublequoteclose}\isanewline
\ \ \ \ \ \ \ \ \isakeyword{shows}\ {\isachardoublequoteopen}sat\ {\isacharparenleft}{\isacharbraceleft}G{\isacharcomma}F{\isacharbraceright}\ {\isasymunion}\ Wo{\isacharparenright}{\isachardoublequoteclose}\isanewline
%
\isadelimproof
%
\endisadelimproof
%
\isatagproof
\isacommand{proof}\isamarkupfalse%
\ {\isacharminus}\isanewline
\ \ \isacommand{have}\isamarkupfalse%
\ {\isachardoublequoteopen}sat\ {\isacharparenleft}{\isacharbraceleft}F{\isacharbraceright}\ {\isasymunion}\ Wo{\isacharparenright}{\isachardoublequoteclose}\isanewline
\ \ \ \ \isacommand{using}\isamarkupfalse%
\ assms{\isacharparenleft}{\isadigit{1}}{\isacharcomma}{\isadigit{3}}{\isacharcomma}{\isadigit{4}}{\isacharcomma}{\isadigit{5}}{\isacharparenright}\ \isacommand{by}\isamarkupfalse%
\ {\isacharparenleft}rule\ pcp{\isacharunderscore}colecComp{\isacharunderscore}elem{\isacharunderscore}sat{\isacharparenright}\isanewline
\ \ \isacommand{have}\isamarkupfalse%
\ {\isachardoublequoteopen}F\ {\isasymin}\ {\isacharbraceleft}F{\isacharbraceright}\ {\isasymunion}\ Wo{\isachardoublequoteclose}\isanewline
\ \ \ \ \isacommand{by}\isamarkupfalse%
\ simp\ \isanewline
\ \ \isacommand{have}\isamarkupfalse%
\ Ex{\isadigit{1}}{\isacharcolon}{\isachardoublequoteopen}{\isasymexists}{\isasymA}{\isachardot}\ {\isasymforall}F\ {\isasymin}\ {\isacharparenleft}{\isacharbraceleft}F{\isacharbraceright}\ {\isasymunion}\ Wo{\isacharparenright}{\isachardot}\ {\isasymA}\ {\isasymTurnstile}\ F{\isachardoublequoteclose}\isanewline
\ \ \ \ \isacommand{using}\isamarkupfalse%
\ {\isacartoucheopen}sat\ {\isacharparenleft}{\isacharbraceleft}F{\isacharbraceright}\ {\isasymunion}\ Wo{\isacharparenright}{\isacartoucheclose}\ \isacommand{by}\isamarkupfalse%
\ {\isacharparenleft}simp\ only{\isacharcolon}\ sat{\isacharunderscore}def{\isacharparenright}\isanewline
\ \ \isacommand{obtain}\isamarkupfalse%
\ {\isasymA}\ \isakeyword{where}\ Forall{\isadigit{1}}{\isacharcolon}{\isachardoublequoteopen}{\isasymforall}F\ {\isasymin}\ {\isacharparenleft}{\isacharbraceleft}F{\isacharbraceright}\ {\isasymunion}\ Wo{\isacharparenright}{\isachardot}\ {\isasymA}\ {\isasymTurnstile}\ F{\isachardoublequoteclose}\isanewline
\ \ \ \ \isacommand{using}\isamarkupfalse%
\ Ex{\isadigit{1}}\ \isacommand{by}\isamarkupfalse%
\ {\isacharparenleft}rule\ exE{\isacharparenright}\isanewline
\ \ \isacommand{have}\isamarkupfalse%
\ {\isachardoublequoteopen}{\isasymA}\ {\isasymTurnstile}\ F{\isachardoublequoteclose}\isanewline
\ \ \ \ \isacommand{using}\isamarkupfalse%
\ Forall{\isadigit{1}}\ {\isacartoucheopen}F\ {\isasymin}\ {\isacharbraceleft}F{\isacharbraceright}\ {\isasymunion}\ Wo{\isacartoucheclose}\ \isacommand{by}\isamarkupfalse%
\ {\isacharparenleft}rule\ bspec{\isacharparenright}\isanewline
\ \ \isacommand{then}\isamarkupfalse%
\ \isacommand{have}\isamarkupfalse%
\ {\isachardoublequoteopen}{\isasymA}\ {\isasymTurnstile}\ \isactrlbold {\isasymnot}{\isacharparenleft}\isactrlbold {\isasymnot}\ G{\isacharparenright}{\isachardoublequoteclose}\isanewline
\ \ \ \ \isacommand{using}\isamarkupfalse%
\ assms{\isacharparenleft}{\isadigit{2}}{\isacharparenright}\ \isacommand{by}\isamarkupfalse%
\ {\isacharparenleft}simp\ only{\isacharcolon}\ {\isacartoucheopen}{\isasymA}\ {\isasymTurnstile}\ F{\isacartoucheclose}{\isacharparenright}\isanewline
\ \ \isacommand{then}\isamarkupfalse%
\ \isacommand{have}\isamarkupfalse%
\ {\isachardoublequoteopen}{\isasymnot}\ {\isasymA}\ {\isasymTurnstile}\ \isactrlbold {\isasymnot}\ G{\isachardoublequoteclose}\isanewline
\ \ \ \ \isacommand{by}\isamarkupfalse%
\ {\isacharparenleft}simp\ only{\isacharcolon}\ formula{\isacharunderscore}semantics{\isachardot}simps{\isacharparenleft}{\isadigit{3}}{\isacharparenright}\ simp{\isacharunderscore}thms{\isacharparenleft}{\isadigit{8}}{\isacharparenright}{\isacharparenright}\isanewline
\ \ \isacommand{then}\isamarkupfalse%
\ \isacommand{have}\isamarkupfalse%
\ {\isachardoublequoteopen}{\isasymnot}\ {\isasymnot}{\isasymA}\ {\isasymTurnstile}\ G{\isachardoublequoteclose}\isanewline
\ \ \ \ \isacommand{by}\isamarkupfalse%
\ {\isacharparenleft}simp\ only{\isacharcolon}\ formula{\isacharunderscore}semantics{\isachardot}simps{\isacharparenleft}{\isadigit{3}}{\isacharparenright}\ simp{\isacharunderscore}thms{\isacharparenleft}{\isadigit{8}}{\isacharparenright}{\isacharparenright}\isanewline
\ \ \isacommand{then}\isamarkupfalse%
\ \isacommand{have}\isamarkupfalse%
\ {\isachardoublequoteopen}{\isasymA}\ {\isasymTurnstile}\ G{\isachardoublequoteclose}\isanewline
\ \ \ \ \isacommand{by}\isamarkupfalse%
\ {\isacharparenleft}rule\ notnotD{\isacharparenright}\isanewline
\ \ \isacommand{then}\isamarkupfalse%
\ \isacommand{have}\isamarkupfalse%
\ {\isachardoublequoteopen}{\isasymforall}F\ {\isasymin}\ {\isacharbraceleft}G{\isacharbraceright}{\isachardot}\ {\isasymA}\ {\isasymTurnstile}\ F{\isachardoublequoteclose}\isanewline
\ \ \ \ \isacommand{by}\isamarkupfalse%
\ simp\isanewline
\ \ \isacommand{then}\isamarkupfalse%
\ \isacommand{have}\isamarkupfalse%
\ {\isachardoublequoteopen}{\isasymforall}F\ {\isasymin}\ {\isacharparenleft}{\isacharbraceleft}G{\isacharbraceright}\ {\isasymunion}\ {\isacharparenleft}{\isacharbraceleft}F{\isacharbraceright}\ {\isasymunion}\ Wo{\isacharparenright}{\isacharparenright}{\isachardot}\ {\isasymA}\ {\isasymTurnstile}\ F{\isachardoublequoteclose}\isanewline
\ \ \ \ \isacommand{using}\isamarkupfalse%
\ Forall{\isadigit{1}}\ \isacommand{by}\isamarkupfalse%
\ {\isacharparenleft}rule\ ball{\isacharunderscore}Un{\isacharparenright}\isanewline
\ \ \isacommand{then}\isamarkupfalse%
\ \isacommand{have}\isamarkupfalse%
\ {\isachardoublequoteopen}{\isasymforall}F\ {\isasymin}\ {\isacharbraceleft}G{\isacharcomma}F{\isacharbraceright}\ {\isasymunion}\ Wo{\isachardot}\ {\isasymA}\ {\isasymTurnstile}\ F{\isachardoublequoteclose}\isanewline
\ \ \ \ \isacommand{by}\isamarkupfalse%
\ simp\isanewline
\ \ \isacommand{then}\isamarkupfalse%
\ \isacommand{have}\isamarkupfalse%
\ {\isachardoublequoteopen}{\isasymexists}{\isasymA}{\isachardot}\ {\isasymforall}F\ {\isasymin}\ {\isacharparenleft}{\isacharbraceleft}G{\isacharcomma}F{\isacharbraceright}\ {\isasymunion}\ Wo{\isacharparenright}{\isachardot}\ {\isasymA}\ {\isasymTurnstile}\ F{\isachardoublequoteclose}\isanewline
\ \ \ \ \isacommand{by}\isamarkupfalse%
\ {\isacharparenleft}iprover\ intro{\isacharcolon}\ exI{\isacharparenright}\isanewline
\ \ \isacommand{thus}\isamarkupfalse%
\ {\isacharquery}thesis\isanewline
\ \ \ \ \isacommand{by}\isamarkupfalse%
\ {\isacharparenleft}simp\ only{\isacharcolon}\ sat{\isacharunderscore}def{\isacharparenright}\isanewline
\isacommand{qed}\isamarkupfalse%
%
\endisatagproof
{\isafoldproof}%
%
\isadelimproof
%
\endisadelimproof
%
\begin{isamarkuptext}%
De este modo, por los lemas anteriores para los distintos tipos de fórmula \isa{{\isasymbeta}}, se
  demuestra que dados \isa{W\ {\isasymin}\ C}, \isa{F} una fórmula de tipo \isa{{\isasymbeta}} con componentes \isa{{\isasymbeta}\isactrlsub {\isadigit{1}}} y \isa{{\isasymbeta}\isactrlsub {\isadigit{2}}} tal que 
  \isa{F\ {\isasymin}\ W} y \isa{W\isactrlsub {\isadigit{0}}} un subconjunto finito de \isa{W}, entonces se tiene que o bien \isa{{\isacharbraceleft}{\isasymbeta}\isactrlsub {\isadigit{1}}{\isacharcomma}F{\isacharbraceright}\ {\isasymunion}\ W\isactrlsub {\isadigit{0}}} es 
  satisfacible o bien \isa{{\isacharbraceleft}{\isasymbeta}\isactrlsub {\isadigit{2}}{\isacharcomma}F{\isacharbraceright}\ {\isasymunion}\ W\isactrlsub {\isadigit{0}}} es satisfacible.%
\end{isamarkuptext}\isamarkuptrue%
\isacommand{lemma}\isamarkupfalse%
\ pcp{\isacharunderscore}colecComp{\isacharunderscore}DIS{\isacharunderscore}sat{\isacharcolon}\isanewline
\ \ \isakeyword{assumes}\ {\isachardoublequoteopen}W\ {\isasymin}\ colecComp{\isachardoublequoteclose}\isanewline
\ \ \ \ \ \ \ \ \ \ {\isachardoublequoteopen}Dis\ F\ G\ H{\isachardoublequoteclose}\isanewline
\ \ \ \ \ \ \ \ \ \ {\isachardoublequoteopen}F\ {\isasymin}\ W{\isachardoublequoteclose}\isanewline
\ \ \ \ \ \ \ \ \ \ {\isachardoublequoteopen}finite\ Wo{\isachardoublequoteclose}\isanewline
\ \ \ \ \ \ \ \ \ \ {\isachardoublequoteopen}Wo\ {\isasymsubseteq}\ W{\isachardoublequoteclose}\isanewline
\ \ \ \ \ \ \ \ \isakeyword{shows}\ {\isachardoublequoteopen}sat\ {\isacharparenleft}{\isacharbraceleft}G{\isacharcomma}F{\isacharbraceright}\ {\isasymunion}\ Wo{\isacharparenright}\ {\isasymor}\ sat\ {\isacharparenleft}{\isacharbraceleft}H{\isacharcomma}F{\isacharbraceright}\ {\isasymunion}\ Wo{\isacharparenright}{\isachardoublequoteclose}\isanewline
%
\isadelimproof
%
\endisadelimproof
%
\isatagproof
\isacommand{proof}\isamarkupfalse%
\ {\isacharminus}\isanewline
\ \ \isacommand{have}\isamarkupfalse%
\ {\isachardoublequoteopen}{\isacharparenleft}F\ {\isacharequal}\ G\ \isactrlbold {\isasymor}\ H\ {\isasymor}\ \isanewline
\ \ \ \ \ \ \ \ {\isacharparenleft}{\isasymexists}G{\isadigit{1}}\ H{\isadigit{1}}{\isachardot}\ F\ {\isacharequal}\ G{\isadigit{1}}\ \isactrlbold {\isasymrightarrow}\ H{\isadigit{1}}\ {\isasymand}\ G\ {\isacharequal}\ \isactrlbold {\isasymnot}\ G{\isadigit{1}}\ {\isasymand}\ H\ {\isacharequal}\ H{\isadigit{1}}{\isacharparenright}\ {\isasymor}\ \isanewline
\ \ \ \ \ \ \ \ {\isacharparenleft}{\isasymexists}G{\isadigit{1}}\ H{\isadigit{1}}{\isachardot}\ F\ {\isacharequal}\ \isactrlbold {\isasymnot}\ {\isacharparenleft}G{\isadigit{1}}\ \isactrlbold {\isasymand}\ H{\isadigit{1}}{\isacharparenright}\ {\isasymand}\ G\ {\isacharequal}\ \isactrlbold {\isasymnot}\ G{\isadigit{1}}\ {\isasymand}\ H\ {\isacharequal}\ \isactrlbold {\isasymnot}\ H{\isadigit{1}}{\isacharparenright}\ {\isasymor}\ \isanewline
\ \ \ \ \ \ \ \ F\ {\isacharequal}\ \isactrlbold {\isasymnot}\ {\isacharparenleft}\isactrlbold {\isasymnot}\ G{\isacharparenright}\ {\isasymand}\ H\ {\isacharequal}\ G{\isacharparenright}{\isachardoublequoteclose}\isanewline
\ \ \ \ \isacommand{using}\isamarkupfalse%
\ assms{\isacharparenleft}{\isadigit{2}}{\isacharparenright}\ \isacommand{by}\isamarkupfalse%
\ {\isacharparenleft}simp\ only{\isacharcolon}\ con{\isacharunderscore}dis{\isacharunderscore}simps{\isacharparenleft}{\isadigit{2}}{\isacharparenright}{\isacharparenright}\isanewline
\ \ \isacommand{thus}\isamarkupfalse%
\ {\isacharquery}thesis\isanewline
\ \ \isacommand{proof}\isamarkupfalse%
\ {\isacharparenleft}rule\ disjE{\isacharparenright}\isanewline
\ \ \ \ \isacommand{assume}\isamarkupfalse%
\ {\isachardoublequoteopen}F\ {\isacharequal}\ G\ \isactrlbold {\isasymor}\ H{\isachardoublequoteclose}\isanewline
\ \ \ \ \isacommand{show}\isamarkupfalse%
\ {\isachardoublequoteopen}sat\ {\isacharparenleft}{\isacharbraceleft}G{\isacharcomma}F{\isacharbraceright}\ {\isasymunion}\ Wo{\isacharparenright}\ {\isasymor}\ sat\ {\isacharparenleft}{\isacharbraceleft}H{\isacharcomma}F{\isacharbraceright}\ {\isasymunion}\ Wo{\isacharparenright}{\isachardoublequoteclose}\isanewline
\ \ \ \ \ \ \isacommand{using}\isamarkupfalse%
\ assms{\isacharparenleft}{\isadigit{1}}{\isacharparenright}\ {\isacartoucheopen}F\ {\isacharequal}\ G\ \isactrlbold {\isasymor}\ H{\isacartoucheclose}\ assms{\isacharparenleft}{\isadigit{3}}{\isacharcomma}{\isadigit{4}}{\isacharcomma}{\isadigit{5}}{\isacharparenright}\ \isacommand{by}\isamarkupfalse%
\ {\isacharparenleft}rule\ pcp{\isacharunderscore}colecComp{\isacharunderscore}DIS{\isacharunderscore}sat{\isadigit{1}}{\isacharparenright}\isanewline
\ \ \isacommand{next}\isamarkupfalse%
\isanewline
\ \ \ \ \isacommand{assume}\isamarkupfalse%
\ {\isachardoublequoteopen}{\isacharparenleft}{\isasymexists}G{\isadigit{1}}\ H{\isadigit{1}}{\isachardot}\ F\ {\isacharequal}\ G{\isadigit{1}}\ \isactrlbold {\isasymrightarrow}\ H{\isadigit{1}}\ {\isasymand}\ G\ {\isacharequal}\ \isactrlbold {\isasymnot}\ G{\isadigit{1}}\ {\isasymand}\ H\ {\isacharequal}\ H{\isadigit{1}}{\isacharparenright}\ {\isasymor}\ \isanewline
\ \ \ \ \ \ \ \ {\isacharparenleft}{\isasymexists}G{\isadigit{1}}\ H{\isadigit{1}}{\isachardot}\ F\ {\isacharequal}\ \isactrlbold {\isasymnot}\ {\isacharparenleft}G{\isadigit{1}}\ \isactrlbold {\isasymand}\ H{\isadigit{1}}{\isacharparenright}\ {\isasymand}\ G\ {\isacharequal}\ \isactrlbold {\isasymnot}\ G{\isadigit{1}}\ {\isasymand}\ H\ {\isacharequal}\ \isactrlbold {\isasymnot}\ H{\isadigit{1}}{\isacharparenright}\ {\isasymor}\ \isanewline
\ \ \ \ \ \ \ \ F\ {\isacharequal}\ \isactrlbold {\isasymnot}\ {\isacharparenleft}\isactrlbold {\isasymnot}\ G{\isacharparenright}\ {\isasymand}\ H\ {\isacharequal}\ G{\isachardoublequoteclose}\isanewline
\ \ \ \ \isacommand{thus}\isamarkupfalse%
\ {\isacharquery}thesis\isanewline
\ \ \ \ \isacommand{proof}\isamarkupfalse%
\ {\isacharparenleft}rule\ disjE{\isacharparenright}\isanewline
\ \ \ \ \ \ \isacommand{assume}\isamarkupfalse%
\ Ex{\isadigit{1}}{\isacharcolon}{\isachardoublequoteopen}{\isasymexists}G{\isadigit{1}}\ H{\isadigit{1}}{\isachardot}\ F\ {\isacharequal}\ G{\isadigit{1}}\ \isactrlbold {\isasymrightarrow}\ H{\isadigit{1}}\ {\isasymand}\ G\ {\isacharequal}\ \isactrlbold {\isasymnot}\ G{\isadigit{1}}\ {\isasymand}\ H\ {\isacharequal}\ H{\isadigit{1}}{\isachardoublequoteclose}\isanewline
\ \ \ \ \ \ \isacommand{obtain}\isamarkupfalse%
\ G{\isadigit{1}}\ H{\isadigit{1}}\ \isakeyword{where}\ C{\isadigit{1}}{\isacharcolon}{\isachardoublequoteopen}F\ {\isacharequal}\ G{\isadigit{1}}\ \isactrlbold {\isasymrightarrow}\ H{\isadigit{1}}\ {\isasymand}\ G\ {\isacharequal}\ \isactrlbold {\isasymnot}\ G{\isadigit{1}}\ {\isasymand}\ H\ {\isacharequal}\ H{\isadigit{1}}{\isachardoublequoteclose}\isanewline
\ \ \ \ \ \ \ \ \isacommand{using}\isamarkupfalse%
\ Ex{\isadigit{1}}\ \isacommand{by}\isamarkupfalse%
\ {\isacharparenleft}iprover\ elim{\isacharcolon}\ exE{\isacharparenright}\isanewline
\ \ \ \ \ \ \isacommand{have}\isamarkupfalse%
\ {\isachardoublequoteopen}F\ {\isacharequal}\ G{\isadigit{1}}\ \isactrlbold {\isasymrightarrow}\ H{\isadigit{1}}{\isachardoublequoteclose}\isanewline
\ \ \ \ \ \ \ \ \isacommand{using}\isamarkupfalse%
\ C{\isadigit{1}}\ \isacommand{by}\isamarkupfalse%
\ {\isacharparenleft}rule\ conjunct{\isadigit{1}}{\isacharparenright}\isanewline
\ \ \ \ \ \ \isacommand{have}\isamarkupfalse%
\ {\isachardoublequoteopen}G\ {\isacharequal}\ \isactrlbold {\isasymnot}\ G{\isadigit{1}}{\isachardoublequoteclose}\isanewline
\ \ \ \ \ \ \ \ \isacommand{using}\isamarkupfalse%
\ C{\isadigit{1}}\ \isacommand{by}\isamarkupfalse%
\ {\isacharparenleft}iprover\ elim{\isacharcolon}\ conjunct{\isadigit{1}}{\isacharparenright}\isanewline
\ \ \ \ \ \ \isacommand{have}\isamarkupfalse%
\ {\isachardoublequoteopen}H\ {\isacharequal}\ H{\isadigit{1}}{\isachardoublequoteclose}\isanewline
\ \ \ \ \ \ \ \ \isacommand{using}\isamarkupfalse%
\ C{\isadigit{1}}\ \isacommand{by}\isamarkupfalse%
\ {\isacharparenleft}iprover\ elim{\isacharcolon}\ conjunct{\isadigit{2}}{\isacharparenright}\isanewline
\ \ \ \ \ \ \isacommand{have}\isamarkupfalse%
\ {\isachardoublequoteopen}sat\ {\isacharparenleft}{\isacharbraceleft}\isactrlbold {\isasymnot}\ G{\isadigit{1}}{\isacharcomma}F{\isacharbraceright}\ {\isasymunion}\ Wo{\isacharparenright}\ {\isasymor}\ sat\ {\isacharparenleft}{\isacharbraceleft}H{\isadigit{1}}{\isacharcomma}F{\isacharbraceright}\ {\isasymunion}\ Wo{\isacharparenright}{\isachardoublequoteclose}\isanewline
\ \ \ \ \ \ \ \ \isacommand{using}\isamarkupfalse%
\ assms{\isacharparenleft}{\isadigit{1}}{\isacharparenright}\ {\isacartoucheopen}F\ {\isacharequal}\ G{\isadigit{1}}\ \isactrlbold {\isasymrightarrow}\ H{\isadigit{1}}{\isacartoucheclose}\ assms{\isacharparenleft}{\isadigit{3}}{\isacharcomma}{\isadigit{4}}{\isacharcomma}{\isadigit{5}}{\isacharparenright}\ \isacommand{by}\isamarkupfalse%
\ {\isacharparenleft}rule\ pcp{\isacharunderscore}colecComp{\isacharunderscore}DIS{\isacharunderscore}sat{\isadigit{2}}{\isacharparenright}\isanewline
\ \ \ \ \ \ \isacommand{thus}\isamarkupfalse%
\ {\isachardoublequoteopen}sat\ {\isacharparenleft}{\isacharbraceleft}G{\isacharcomma}\ F{\isacharbraceright}\ {\isasymunion}\ Wo{\isacharparenright}\ {\isasymor}\ sat\ {\isacharparenleft}{\isacharbraceleft}H{\isacharcomma}\ F{\isacharbraceright}\ {\isasymunion}\ Wo{\isacharparenright}{\isachardoublequoteclose}\isanewline
\ \ \ \ \ \ \ \ \isacommand{by}\isamarkupfalse%
\ {\isacharparenleft}simp\ only{\isacharcolon}\ {\isacartoucheopen}G\ {\isacharequal}\ \isactrlbold {\isasymnot}\ G{\isadigit{1}}{\isacartoucheclose}\ {\isacartoucheopen}H\ {\isacharequal}\ H{\isadigit{1}}{\isacartoucheclose}{\isacharparenright}\ \isanewline
\ \ \ \ \isacommand{next}\isamarkupfalse%
\isanewline
\ \ \ \ \ \ \isacommand{assume}\isamarkupfalse%
\ {\isachardoublequoteopen}{\isacharparenleft}{\isasymexists}G{\isadigit{1}}\ H{\isadigit{1}}{\isachardot}\ F\ {\isacharequal}\ \isactrlbold {\isasymnot}\ {\isacharparenleft}G{\isadigit{1}}\ \isactrlbold {\isasymand}\ H{\isadigit{1}}{\isacharparenright}\ {\isasymand}\ G\ {\isacharequal}\ \isactrlbold {\isasymnot}\ G{\isadigit{1}}\ {\isasymand}\ H\ {\isacharequal}\ \isactrlbold {\isasymnot}\ H{\isadigit{1}}{\isacharparenright}\ {\isasymor}\ \isanewline
\ \ \ \ \ \ \ \ F\ {\isacharequal}\ \isactrlbold {\isasymnot}\ {\isacharparenleft}\isactrlbold {\isasymnot}\ G{\isacharparenright}\ {\isasymand}\ H\ {\isacharequal}\ G{\isachardoublequoteclose}\isanewline
\ \ \ \ \ \ \isacommand{thus}\isamarkupfalse%
\ {\isacharquery}thesis\isanewline
\ \ \ \ \ \ \isacommand{proof}\isamarkupfalse%
\ {\isacharparenleft}rule\ disjE{\isacharparenright}\isanewline
\ \ \ \ \ \ \ \ \isacommand{assume}\isamarkupfalse%
\ Ex{\isadigit{2}}{\isacharcolon}{\isachardoublequoteopen}{\isasymexists}G{\isadigit{1}}\ H{\isadigit{1}}{\isachardot}\ F\ {\isacharequal}\ \isactrlbold {\isasymnot}\ {\isacharparenleft}G{\isadigit{1}}\ \isactrlbold {\isasymand}\ H{\isadigit{1}}{\isacharparenright}\ {\isasymand}\ G\ {\isacharequal}\ \isactrlbold {\isasymnot}\ G{\isadigit{1}}\ {\isasymand}\ H\ {\isacharequal}\ \isactrlbold {\isasymnot}\ H{\isadigit{1}}{\isachardoublequoteclose}\isanewline
\ \ \ \ \ \ \ \ \isacommand{obtain}\isamarkupfalse%
\ G{\isadigit{1}}\ H{\isadigit{1}}\ \isakeyword{where}\ C{\isadigit{2}}{\isacharcolon}{\isachardoublequoteopen}F\ {\isacharequal}\ \isactrlbold {\isasymnot}\ {\isacharparenleft}G{\isadigit{1}}\ \isactrlbold {\isasymand}\ H{\isadigit{1}}{\isacharparenright}\ {\isasymand}\ G\ {\isacharequal}\ \isactrlbold {\isasymnot}\ G{\isadigit{1}}\ {\isasymand}\ H\ {\isacharequal}\ \isactrlbold {\isasymnot}\ H{\isadigit{1}}{\isachardoublequoteclose}\isanewline
\ \ \ \ \ \ \ \ \ \ \isacommand{using}\isamarkupfalse%
\ Ex{\isadigit{2}}\ \isacommand{by}\isamarkupfalse%
\ {\isacharparenleft}iprover\ elim{\isacharcolon}\ exE{\isacharparenright}\isanewline
\ \ \ \ \ \ \ \ \isacommand{have}\isamarkupfalse%
\ {\isachardoublequoteopen}F\ {\isacharequal}\ \isactrlbold {\isasymnot}\ {\isacharparenleft}G{\isadigit{1}}\ \isactrlbold {\isasymand}\ H{\isadigit{1}}{\isacharparenright}{\isachardoublequoteclose}\isanewline
\ \ \ \ \ \ \ \ \ \ \isacommand{using}\isamarkupfalse%
\ C{\isadigit{2}}\ \isacommand{by}\isamarkupfalse%
\ {\isacharparenleft}rule\ conjunct{\isadigit{1}}{\isacharparenright}\isanewline
\ \ \ \ \ \ \ \ \isacommand{have}\isamarkupfalse%
\ {\isachardoublequoteopen}G\ {\isacharequal}\ \isactrlbold {\isasymnot}\ G{\isadigit{1}}{\isachardoublequoteclose}\isanewline
\ \ \ \ \ \ \ \ \ \ \isacommand{using}\isamarkupfalse%
\ C{\isadigit{2}}\ \isacommand{by}\isamarkupfalse%
\ {\isacharparenleft}iprover\ elim{\isacharcolon}\ conjunct{\isadigit{1}}{\isacharparenright}\isanewline
\ \ \ \ \ \ \ \ \isacommand{have}\isamarkupfalse%
\ {\isachardoublequoteopen}H\ {\isacharequal}\ \isactrlbold {\isasymnot}\ H{\isadigit{1}}{\isachardoublequoteclose}\isanewline
\ \ \ \ \ \ \ \ \ \ \isacommand{using}\isamarkupfalse%
\ C{\isadigit{2}}\ \isacommand{by}\isamarkupfalse%
\ {\isacharparenleft}iprover\ elim{\isacharcolon}\ conjunct{\isadigit{2}}{\isacharparenright}\isanewline
\ \ \ \ \ \ \ \ \isacommand{have}\isamarkupfalse%
\ {\isachardoublequoteopen}sat\ {\isacharparenleft}{\isacharbraceleft}\isactrlbold {\isasymnot}\ G{\isadigit{1}}{\isacharcomma}F{\isacharbraceright}\ {\isasymunion}\ Wo{\isacharparenright}\ {\isasymor}\ sat\ {\isacharparenleft}{\isacharbraceleft}\isactrlbold {\isasymnot}\ H{\isadigit{1}}{\isacharcomma}F{\isacharbraceright}\ {\isasymunion}\ Wo{\isacharparenright}{\isachardoublequoteclose}\isanewline
\ \ \ \ \ \ \ \ \ \ \isacommand{using}\isamarkupfalse%
\ assms{\isacharparenleft}{\isadigit{1}}{\isacharparenright}\ {\isacartoucheopen}F\ {\isacharequal}\ \isactrlbold {\isasymnot}\ {\isacharparenleft}G{\isadigit{1}}\ \isactrlbold {\isasymand}\ H{\isadigit{1}}{\isacharparenright}{\isacartoucheclose}\ assms{\isacharparenleft}{\isadigit{3}}{\isacharcomma}{\isadigit{4}}{\isacharcomma}{\isadigit{5}}{\isacharparenright}\ \isacommand{by}\isamarkupfalse%
\ {\isacharparenleft}rule\ pcp{\isacharunderscore}colecComp{\isacharunderscore}DIS{\isacharunderscore}sat{\isadigit{3}}{\isacharparenright}\isanewline
\ \ \ \ \ \ \ \ \isacommand{thus}\isamarkupfalse%
\ {\isachardoublequoteopen}sat\ {\isacharparenleft}{\isacharbraceleft}G{\isacharcomma}F{\isacharbraceright}\ {\isasymunion}\ Wo{\isacharparenright}\ {\isasymor}\ sat\ {\isacharparenleft}{\isacharbraceleft}H{\isacharcomma}F{\isacharbraceright}\ {\isasymunion}\ Wo{\isacharparenright}{\isachardoublequoteclose}\isanewline
\ \ \ \ \ \ \ \ \ \ \isacommand{by}\isamarkupfalse%
\ {\isacharparenleft}simp\ only{\isacharcolon}\ {\isacartoucheopen}G\ {\isacharequal}\ \isactrlbold {\isasymnot}\ G{\isadigit{1}}{\isacartoucheclose}\ {\isacartoucheopen}H\ {\isacharequal}\ \isactrlbold {\isasymnot}\ H{\isadigit{1}}{\isacartoucheclose}{\isacharparenright}\isanewline
\ \ \ \ \ \ \isacommand{next}\isamarkupfalse%
\isanewline
\ \ \ \ \ \ \ \ \isacommand{assume}\isamarkupfalse%
\ {\isachardoublequoteopen}F\ {\isacharequal}\ \isactrlbold {\isasymnot}\ {\isacharparenleft}\isactrlbold {\isasymnot}\ G{\isacharparenright}\ {\isasymand}\ H\ {\isacharequal}\ G{\isachardoublequoteclose}\isanewline
\ \ \ \ \ \ \ \ \isacommand{then}\isamarkupfalse%
\ \isacommand{have}\isamarkupfalse%
\ {\isachardoublequoteopen}F\ {\isacharequal}\ \isactrlbold {\isasymnot}\ {\isacharparenleft}\isactrlbold {\isasymnot}\ G{\isacharparenright}{\isachardoublequoteclose}\isanewline
\ \ \ \ \ \ \ \ \ \ \isacommand{by}\isamarkupfalse%
\ {\isacharparenleft}rule\ conjunct{\isadigit{1}}{\isacharparenright}\isanewline
\ \ \ \ \ \ \ \ \isacommand{have}\isamarkupfalse%
\ {\isachardoublequoteopen}sat\ {\isacharparenleft}{\isacharbraceleft}G{\isacharcomma}F{\isacharbraceright}\ {\isasymunion}\ Wo{\isacharparenright}{\isachardoublequoteclose}\isanewline
\ \ \ \ \ \ \ \ \ \ \isacommand{using}\isamarkupfalse%
\ assms{\isacharparenleft}{\isadigit{1}}{\isacharparenright}\ {\isacartoucheopen}F\ {\isacharequal}\ \isactrlbold {\isasymnot}\ {\isacharparenleft}\isactrlbold {\isasymnot}\ G{\isacharparenright}{\isacartoucheclose}\ assms{\isacharparenleft}{\isadigit{3}}{\isacharcomma}{\isadigit{4}}{\isacharcomma}{\isadigit{5}}{\isacharparenright}\ \isacommand{by}\isamarkupfalse%
\ {\isacharparenleft}rule\ pcp{\isacharunderscore}colecComp{\isacharunderscore}DIS{\isacharunderscore}sat{\isadigit{4}}{\isacharparenright}\isanewline
\ \ \ \ \ \ \ \ \isacommand{thus}\isamarkupfalse%
\ {\isachardoublequoteopen}sat\ {\isacharparenleft}{\isacharbraceleft}G{\isacharcomma}F{\isacharbraceright}\ {\isasymunion}\ Wo{\isacharparenright}\ {\isasymor}\ sat\ {\isacharparenleft}{\isacharbraceleft}H{\isacharcomma}F{\isacharbraceright}\ {\isasymunion}\ Wo{\isacharparenright}{\isachardoublequoteclose}\isanewline
\ \ \ \ \ \ \ \ \ \ \isacommand{by}\isamarkupfalse%
\ {\isacharparenleft}rule\ disjI{\isadigit{1}}{\isacharparenright}\isanewline
\ \ \ \ \ \ \isacommand{qed}\isamarkupfalse%
\isanewline
\ \ \ \ \isacommand{qed}\isamarkupfalse%
\isanewline
\ \ \isacommand{qed}\isamarkupfalse%
\isanewline
\isacommand{qed}\isamarkupfalse%
%
\endisatagproof
{\isafoldproof}%
%
\isadelimproof
%
\endisadelimproof
%
\begin{isamarkuptext}%
Finalmente, con los lemas auxiliares anteriores podemos demostrar detalladamente la cuarta 
  condición del lema \isa{{\isadigit{2}}{\isachardot}{\isadigit{0}}{\isachardot}{\isadigit{2}}}: dados \isa{W\ {\isasymin}\ C} y \isa{F} una fórmula de tipo \isa{{\isasymbeta}} con componentes \isa{{\isasymbeta}\isactrlsub {\isadigit{1}}} y 
  \isa{{\isasymbeta}\isactrlsub {\isadigit{2}}} tal que \isa{F\ {\isasymin}\ W}, se tiene que o bien \isa{{\isacharbraceleft}{\isasymbeta}\isactrlsub {\isadigit{1}}{\isacharbraceright}\ {\isasymunion}\ W\ {\isasymin}\ C} o bien\\ \isa{{\isacharbraceleft}{\isasymbeta}\isactrlsub {\isadigit{2}}{\isacharbraceright}\ {\isasymunion}\ W\ {\isasymin}\ C}.%
\end{isamarkuptext}\isamarkuptrue%
\isacommand{lemma}\isamarkupfalse%
\ pcp{\isacharunderscore}colecComp{\isacharunderscore}DIS{\isacharcolon}\isanewline
\ \ \isakeyword{assumes}\ {\isachardoublequoteopen}W\ {\isasymin}\ colecComp{\isachardoublequoteclose}\isanewline
\ \ \isakeyword{shows}\ {\isachardoublequoteopen}{\isasymforall}F\ G\ H{\isachardot}\ Dis\ F\ G\ H\ {\isasymlongrightarrow}\ F\ {\isasymin}\ W\ {\isasymlongrightarrow}\ {\isacharbraceleft}G{\isacharbraceright}\ {\isasymunion}\ W\ {\isasymin}\ colecComp\ {\isasymor}\ {\isacharbraceleft}H{\isacharbraceright}\ {\isasymunion}\ W\ {\isasymin}\ colecComp{\isachardoublequoteclose}\isanewline
%
\isadelimproof
%
\endisadelimproof
%
\isatagproof
\isacommand{proof}\isamarkupfalse%
\ {\isacharparenleft}rule\ allI{\isacharparenright}{\isacharplus}\isanewline
\ \ \isacommand{fix}\isamarkupfalse%
\ F\ G\ H\isanewline
\ \ \isacommand{show}\isamarkupfalse%
\ {\isachardoublequoteopen}Dis\ F\ G\ H\ {\isasymlongrightarrow}\ F\ {\isasymin}\ W\ {\isasymlongrightarrow}\ {\isacharbraceleft}G{\isacharbraceright}\ {\isasymunion}\ W\ {\isasymin}\ colecComp\ {\isasymor}\ {\isacharbraceleft}H{\isacharbraceright}\ {\isasymunion}\ W\ {\isasymin}\ colecComp{\isachardoublequoteclose}\isanewline
\ \ \isacommand{proof}\isamarkupfalse%
\ {\isacharparenleft}rule\ impI{\isacharparenright}{\isacharplus}\isanewline
\ \ \ \ \isacommand{assume}\isamarkupfalse%
\ {\isachardoublequoteopen}Dis\ F\ G\ H{\isachardoublequoteclose}\isanewline
\ \ \ \ \isacommand{assume}\isamarkupfalse%
\ {\isachardoublequoteopen}F\ {\isasymin}\ W{\isachardoublequoteclose}\isanewline
\ \ \ \ \isacommand{show}\isamarkupfalse%
\ {\isachardoublequoteopen}{\isacharbraceleft}G{\isacharbraceright}\ {\isasymunion}\ W\ {\isasymin}\ colecComp\ {\isasymor}\ {\isacharbraceleft}H{\isacharbraceright}\ {\isasymunion}\ W\ {\isasymin}\ colecComp{\isachardoublequoteclose}\isanewline
\ \ \ \ \isacommand{proof}\isamarkupfalse%
\ {\isacharparenleft}rule\ ccontr{\isacharparenright}\isanewline
\ \ \ \ \ \ \isacommand{assume}\isamarkupfalse%
\ {\isachardoublequoteopen}{\isasymnot}{\isacharparenleft}{\isacharbraceleft}G{\isacharbraceright}\ {\isasymunion}\ W\ {\isasymin}\ colecComp\ {\isasymor}\ {\isacharbraceleft}H{\isacharbraceright}\ {\isasymunion}\ W\ {\isasymin}\ colecComp{\isacharparenright}{\isachardoublequoteclose}\isanewline
\ \ \ \ \ \ \isacommand{then}\isamarkupfalse%
\ \isacommand{have}\isamarkupfalse%
\ C{\isacharcolon}{\isachardoublequoteopen}{\isacharbraceleft}G{\isacharbraceright}\ {\isasymunion}\ W\ {\isasymnotin}\ colecComp\ {\isasymand}\ {\isacharbraceleft}H{\isacharbraceright}\ {\isasymunion}\ W\ {\isasymnotin}\ colecComp{\isachardoublequoteclose}\isanewline
\ \ \ \ \ \ \ \ \isacommand{by}\isamarkupfalse%
\ {\isacharparenleft}simp\ only{\isacharcolon}\ de{\isacharunderscore}Morgan{\isacharunderscore}disj\ simp{\isacharunderscore}thms{\isacharparenleft}{\isadigit{8}}{\isacharparenright}{\isacharparenright}\isanewline
\ \ \ \ \ \ \isacommand{then}\isamarkupfalse%
\ \isacommand{have}\isamarkupfalse%
\ {\isachardoublequoteopen}{\isacharbraceleft}G{\isacharbraceright}\ {\isasymunion}\ W\ {\isasymnotin}\ colecComp{\isachardoublequoteclose}\isanewline
\ \ \ \ \ \ \ \ \isacommand{by}\isamarkupfalse%
\ {\isacharparenleft}rule\ conjunct{\isadigit{1}}{\isacharparenright}\isanewline
\ \ \ \ \ \ \isacommand{have}\isamarkupfalse%
\ Ex{\isadigit{1}}{\isacharcolon}{\isachardoublequoteopen}{\isasymexists}Wo\ {\isasymsubseteq}\ W{\isachardot}\ finite\ Wo\ {\isasymand}\ {\isasymnot}{\isacharparenleft}sat\ {\isacharparenleft}{\isacharbraceleft}G{\isacharbraceright}\ {\isasymunion}\ Wo{\isacharparenright}{\isacharparenright}{\isachardoublequoteclose}\isanewline
\ \ \ \ \ \ \ \ \isacommand{using}\isamarkupfalse%
\ assms\ {\isacartoucheopen}{\isacharbraceleft}G{\isacharbraceright}\ {\isasymunion}\ W\ {\isasymnotin}\ colecComp{\isacartoucheclose}\ \isacommand{by}\isamarkupfalse%
\ {\isacharparenleft}rule\ not{\isacharunderscore}colecComp{\isacharparenright}\isanewline
\ \ \ \ \ \ \isacommand{obtain}\isamarkupfalse%
\ W{\isadigit{1}}\ \isakeyword{where}\ {\isachardoublequoteopen}W{\isadigit{1}}\ {\isasymsubseteq}\ W{\isachardoublequoteclose}\ \isakeyword{and}\ C{\isadigit{1}}{\isacharcolon}{\isachardoublequoteopen}finite\ W{\isadigit{1}}\ {\isasymand}\ {\isasymnot}{\isacharparenleft}sat\ {\isacharparenleft}{\isacharbraceleft}G{\isacharbraceright}\ {\isasymunion}\ W{\isadigit{1}}{\isacharparenright}{\isacharparenright}{\isachardoublequoteclose}\isanewline
\ \ \ \ \ \ \ \ \isacommand{using}\isamarkupfalse%
\ Ex{\isadigit{1}}\ \isacommand{by}\isamarkupfalse%
\ {\isacharparenleft}rule\ subexE{\isacharparenright}\isanewline
\ \ \ \ \ \ \isacommand{have}\isamarkupfalse%
\ {\isachardoublequoteopen}finite\ W{\isadigit{1}}{\isachardoublequoteclose}\isanewline
\ \ \ \ \ \ \ \ \isacommand{using}\isamarkupfalse%
\ C{\isadigit{1}}\ \isacommand{by}\isamarkupfalse%
\ {\isacharparenleft}rule\ conjunct{\isadigit{1}}{\isacharparenright}\isanewline
\ \ \ \ \ \ \isacommand{have}\isamarkupfalse%
\ {\isachardoublequoteopen}{\isasymnot}{\isacharparenleft}sat\ {\isacharparenleft}{\isacharbraceleft}G{\isacharbraceright}\ {\isasymunion}\ W{\isadigit{1}}{\isacharparenright}{\isacharparenright}{\isachardoublequoteclose}\isanewline
\ \ \ \ \ \ \ \ \isacommand{using}\isamarkupfalse%
\ C{\isadigit{1}}\ \isacommand{by}\isamarkupfalse%
\ {\isacharparenleft}rule\ conjunct{\isadigit{2}}{\isacharparenright}\isanewline
\ \ \ \ \ \ \isacommand{have}\isamarkupfalse%
\ {\isachardoublequoteopen}{\isacharbraceleft}H{\isacharbraceright}\ {\isasymunion}\ W\ {\isasymnotin}\ colecComp{\isachardoublequoteclose}\isanewline
\ \ \ \ \ \ \ \ \isacommand{using}\isamarkupfalse%
\ C\ \isacommand{by}\isamarkupfalse%
\ {\isacharparenleft}rule\ conjunct{\isadigit{2}}{\isacharparenright}\ \isanewline
\ \ \ \ \ \ \isacommand{have}\isamarkupfalse%
\ Ex{\isadigit{2}}{\isacharcolon}{\isachardoublequoteopen}{\isasymexists}Wo\ {\isasymsubseteq}\ W{\isachardot}\ finite\ Wo\ {\isasymand}\ {\isasymnot}{\isacharparenleft}sat\ {\isacharparenleft}{\isacharbraceleft}H{\isacharbraceright}\ {\isasymunion}\ Wo{\isacharparenright}{\isacharparenright}{\isachardoublequoteclose}\isanewline
\ \ \ \ \ \ \ \ \isacommand{using}\isamarkupfalse%
\ assms\ {\isacartoucheopen}{\isacharbraceleft}H{\isacharbraceright}\ {\isasymunion}\ W\ {\isasymnotin}\ colecComp{\isacartoucheclose}\ \isacommand{by}\isamarkupfalse%
\ {\isacharparenleft}rule\ not{\isacharunderscore}colecComp{\isacharparenright}\isanewline
\ \ \ \ \ \ \isacommand{obtain}\isamarkupfalse%
\ W{\isadigit{2}}\ \isakeyword{where}\ {\isachardoublequoteopen}W{\isadigit{2}}\ {\isasymsubseteq}\ W{\isachardoublequoteclose}\ \isakeyword{and}\ C{\isadigit{2}}{\isacharcolon}{\isachardoublequoteopen}finite\ W{\isadigit{2}}\ {\isasymand}\ {\isasymnot}{\isacharparenleft}sat\ {\isacharparenleft}{\isacharbraceleft}H{\isacharbraceright}\ {\isasymunion}\ W{\isadigit{2}}{\isacharparenright}{\isacharparenright}{\isachardoublequoteclose}\isanewline
\ \ \ \ \ \ \ \ \isacommand{using}\isamarkupfalse%
\ Ex{\isadigit{2}}\ \isacommand{by}\isamarkupfalse%
\ {\isacharparenleft}rule\ subexE{\isacharparenright}\isanewline
\ \ \ \ \ \ \isacommand{have}\isamarkupfalse%
\ {\isachardoublequoteopen}finite\ W{\isadigit{2}}{\isachardoublequoteclose}\isanewline
\ \ \ \ \ \ \ \ \isacommand{using}\isamarkupfalse%
\ C{\isadigit{2}}\ \isacommand{by}\isamarkupfalse%
\ {\isacharparenleft}rule\ conjunct{\isadigit{1}}{\isacharparenright}\isanewline
\ \ \ \ \ \ \isacommand{have}\isamarkupfalse%
\ {\isachardoublequoteopen}{\isasymnot}{\isacharparenleft}sat\ {\isacharparenleft}{\isacharbraceleft}H{\isacharbraceright}\ {\isasymunion}\ W{\isadigit{2}}{\isacharparenright}{\isacharparenright}{\isachardoublequoteclose}\isanewline
\ \ \ \ \ \ \ \ \isacommand{using}\isamarkupfalse%
\ C{\isadigit{2}}\ \isacommand{by}\isamarkupfalse%
\ {\isacharparenleft}rule\ conjunct{\isadigit{2}}{\isacharparenright}\isanewline
\ \ \ \ \ \ \isacommand{let}\isamarkupfalse%
\ {\isacharquery}Wo\ {\isacharequal}\ {\isachardoublequoteopen}W{\isadigit{1}}\ {\isasymunion}\ W{\isadigit{2}}{\isachardoublequoteclose}\isanewline
\ \ \ \ \ \ \isacommand{have}\isamarkupfalse%
\ {\isachardoublequoteopen}{\isacharquery}Wo\ {\isasymsubseteq}\ W{\isachardoublequoteclose}\isanewline
\ \ \ \ \ \ \ \ \isacommand{using}\isamarkupfalse%
\ {\isacartoucheopen}W{\isadigit{1}}\ {\isasymsubseteq}\ W{\isacartoucheclose}\ {\isacartoucheopen}W{\isadigit{2}}\ {\isasymsubseteq}\ W{\isacartoucheclose}\ \isacommand{by}\isamarkupfalse%
\ {\isacharparenleft}simp\ only{\isacharcolon}\ Un{\isacharunderscore}least{\isacharparenright}\isanewline
\ \ \ \ \ \ \isacommand{have}\isamarkupfalse%
\ {\isachardoublequoteopen}finite\ {\isacharquery}Wo{\isachardoublequoteclose}\isanewline
\ \ \ \ \ \ \ \ \isacommand{using}\isamarkupfalse%
\ {\isacartoucheopen}finite\ W{\isadigit{1}}{\isacartoucheclose}\ {\isacartoucheopen}finite\ W{\isadigit{2}}{\isacartoucheclose}\ \isacommand{by}\isamarkupfalse%
\ {\isacharparenleft}simp\ only{\isacharcolon}\ finite{\isacharunderscore}Un{\isacharparenright}\isanewline
\ \ \ \ \ \ \isacommand{have}\isamarkupfalse%
\ {\isachardoublequoteopen}{\isacharbraceleft}G{\isacharbraceright}\ {\isasymunion}\ W{\isadigit{1}}\ {\isasymsubseteq}\ {\isacharparenleft}{\isacharbraceleft}G{\isacharbraceright}\ {\isasymunion}\ W{\isadigit{1}}{\isacharparenright}\ {\isasymunion}\ W{\isadigit{2}}{\isachardoublequoteclose}\isanewline
\ \ \ \ \ \ \ \ \isacommand{by}\isamarkupfalse%
\ {\isacharparenleft}simp\ only{\isacharcolon}\ Un{\isacharunderscore}upper{\isadigit{1}}{\isacharparenright}\isanewline
\ \ \ \ \ \ \isacommand{then}\isamarkupfalse%
\ \isacommand{have}\isamarkupfalse%
\ {\isachardoublequoteopen}{\isacharbraceleft}G{\isacharbraceright}\ {\isasymunion}\ W{\isadigit{1}}\ {\isasymsubseteq}\ {\isacharbraceleft}G{\isacharbraceright}\ {\isasymunion}\ {\isacharquery}Wo{\isachardoublequoteclose}\isanewline
\ \ \ \ \ \ \ \ \isacommand{by}\isamarkupfalse%
\ {\isacharparenleft}simp\ only{\isacharcolon}\ Un{\isacharunderscore}assoc{\isacharparenright}\isanewline
\ \ \ \ \ \ \isacommand{then}\isamarkupfalse%
\ \isacommand{have}\isamarkupfalse%
\ {\isachardoublequoteopen}{\isacharbraceleft}G{\isacharbraceright}\ {\isasymunion}\ W{\isadigit{1}}\ {\isasymsubseteq}\ {\isacharbraceleft}G{\isacharcomma}F{\isacharbraceright}\ {\isasymunion}\ {\isacharquery}Wo{\isachardoublequoteclose}\isanewline
\ \ \ \ \ \ \ \ \isacommand{by}\isamarkupfalse%
\ blast\isanewline
\ \ \ \ \ \ \isacommand{then}\isamarkupfalse%
\ \isacommand{have}\isamarkupfalse%
\ {\isadigit{1}}{\isacharcolon}{\isachardoublequoteopen}{\isasymnot}{\isacharparenleft}sat{\isacharparenleft}{\isacharbraceleft}G{\isacharcomma}F{\isacharbraceright}\ {\isasymunion}\ {\isacharquery}Wo{\isacharparenright}{\isacharparenright}{\isachardoublequoteclose}\isanewline
\ \ \ \ \ \ \ \ \isacommand{using}\isamarkupfalse%
\ {\isacartoucheopen}{\isasymnot}sat\ {\isacharparenleft}{\isacharbraceleft}G{\isacharbraceright}\ {\isasymunion}\ W{\isadigit{1}}{\isacharparenright}{\isacartoucheclose}\ \isacommand{by}\isamarkupfalse%
\ {\isacharparenleft}rule\ sat{\isacharunderscore}subset{\isacharunderscore}ccontr{\isacharparenright}\isanewline
\ \ \ \ \ \ \isacommand{have}\isamarkupfalse%
\ {\isachardoublequoteopen}{\isacharbraceleft}H{\isacharbraceright}\ {\isasymunion}\ W{\isadigit{2}}\ {\isasymsubseteq}\ {\isacharparenleft}{\isacharbraceleft}H{\isacharbraceright}\ {\isasymunion}\ W{\isadigit{2}}{\isacharparenright}\ {\isasymunion}\ W{\isadigit{1}}{\isachardoublequoteclose}\isanewline
\ \ \ \ \ \ \ \ \isacommand{by}\isamarkupfalse%
\ {\isacharparenleft}simp\ only{\isacharcolon}\ Un{\isacharunderscore}upper{\isadigit{1}}{\isacharparenright}\isanewline
\ \ \ \ \ \ \isacommand{then}\isamarkupfalse%
\ \isacommand{have}\isamarkupfalse%
\ {\isachardoublequoteopen}{\isacharbraceleft}H{\isacharbraceright}\ {\isasymunion}\ W{\isadigit{2}}\ {\isasymsubseteq}\ {\isacharbraceleft}H{\isacharbraceright}\ {\isasymunion}\ {\isacharparenleft}W{\isadigit{2}}\ {\isasymunion}\ W{\isadigit{1}}{\isacharparenright}{\isachardoublequoteclose}\isanewline
\ \ \ \ \ \ \ \ \isacommand{by}\isamarkupfalse%
\ {\isacharparenleft}simp\ only{\isacharcolon}\ Un{\isacharunderscore}assoc{\isacharparenright}\ \isanewline
\ \ \ \ \ \ \isacommand{then}\isamarkupfalse%
\ \isacommand{have}\isamarkupfalse%
\ {\isachardoublequoteopen}{\isacharbraceleft}H{\isacharbraceright}\ {\isasymunion}\ W{\isadigit{2}}\ {\isasymsubseteq}\ {\isacharbraceleft}H{\isacharbraceright}\ {\isasymunion}\ {\isacharquery}Wo{\isachardoublequoteclose}\isanewline
\ \ \ \ \ \ \ \ \isacommand{by}\isamarkupfalse%
\ {\isacharparenleft}simp\ only{\isacharcolon}\ Un{\isacharunderscore}commute{\isacharparenright}\isanewline
\ \ \ \ \ \ \isacommand{then}\isamarkupfalse%
\ \isacommand{have}\isamarkupfalse%
\ {\isachardoublequoteopen}{\isacharbraceleft}H{\isacharbraceright}\ {\isasymunion}\ W{\isadigit{2}}\ {\isasymsubseteq}\ {\isacharbraceleft}H{\isacharcomma}F{\isacharbraceright}\ {\isasymunion}\ {\isacharquery}Wo{\isachardoublequoteclose}\isanewline
\ \ \ \ \ \ \ \ \isacommand{by}\isamarkupfalse%
\ blast\isanewline
\ \ \ \ \ \ \isacommand{then}\isamarkupfalse%
\ \isacommand{have}\isamarkupfalse%
\ {\isadigit{2}}{\isacharcolon}{\isachardoublequoteopen}{\isasymnot}{\isacharparenleft}sat{\isacharparenleft}{\isacharbraceleft}H{\isacharcomma}F{\isacharbraceright}\ {\isasymunion}\ {\isacharquery}Wo{\isacharparenright}{\isacharparenright}{\isachardoublequoteclose}\isanewline
\ \ \ \ \ \ \ \ \isacommand{using}\isamarkupfalse%
\ {\isacartoucheopen}{\isasymnot}sat\ {\isacharparenleft}{\isacharbraceleft}H{\isacharbraceright}\ {\isasymunion}\ W{\isadigit{2}}{\isacharparenright}{\isacartoucheclose}\ \isacommand{by}\isamarkupfalse%
\ {\isacharparenleft}rule\ sat{\isacharunderscore}subset{\isacharunderscore}ccontr{\isacharparenright}\isanewline
\ \ \ \ \ \ \isacommand{have}\isamarkupfalse%
\ {\isachardoublequoteopen}{\isasymnot}\ sat\ {\isacharparenleft}{\isacharbraceleft}G{\isacharcomma}F{\isacharbraceright}\ {\isasymunion}\ {\isacharquery}Wo{\isacharparenright}\ {\isasymand}\ {\isasymnot}\ sat\ {\isacharparenleft}{\isacharbraceleft}H{\isacharcomma}F{\isacharbraceright}\ {\isasymunion}\ {\isacharquery}Wo{\isacharparenright}{\isachardoublequoteclose}\isanewline
\ \ \ \ \ \ \ \ \isacommand{using}\isamarkupfalse%
\ {\isadigit{1}}\ {\isadigit{2}}\ \isacommand{by}\isamarkupfalse%
\ {\isacharparenleft}rule\ conjI{\isacharparenright}\isanewline
\ \ \ \ \ \ \isacommand{have}\isamarkupfalse%
\ {\isachardoublequoteopen}sat\ {\isacharparenleft}{\isacharbraceleft}G{\isacharcomma}F{\isacharbraceright}\ {\isasymunion}\ {\isacharquery}Wo{\isacharparenright}\ {\isasymor}\ sat\ {\isacharparenleft}{\isacharbraceleft}H{\isacharcomma}F{\isacharbraceright}\ {\isasymunion}\ {\isacharquery}Wo{\isacharparenright}{\isachardoublequoteclose}\isanewline
\ \ \ \ \ \ \ \ \isacommand{using}\isamarkupfalse%
\ assms{\isacharparenleft}{\isadigit{1}}{\isacharparenright}\ {\isacartoucheopen}Dis\ F\ G\ H{\isacartoucheclose}\ {\isacartoucheopen}F\ {\isasymin}\ W{\isacartoucheclose}\ {\isacartoucheopen}finite\ {\isacharquery}Wo{\isacartoucheclose}\ {\isacartoucheopen}{\isacharquery}Wo\ {\isasymsubseteq}\ W{\isacartoucheclose}\ \isacommand{by}\isamarkupfalse%
\ {\isacharparenleft}rule\ pcp{\isacharunderscore}colecComp{\isacharunderscore}DIS{\isacharunderscore}sat{\isacharparenright}\isanewline
\ \ \ \ \ \ \isacommand{then}\isamarkupfalse%
\ \isacommand{have}\isamarkupfalse%
\ {\isachardoublequoteopen}{\isasymnot}{\isasymnot}{\isacharparenleft}sat\ {\isacharparenleft}{\isacharbraceleft}G{\isacharcomma}F{\isacharbraceright}\ {\isasymunion}\ {\isacharquery}Wo{\isacharparenright}\ {\isasymor}\ sat\ {\isacharparenleft}{\isacharbraceleft}H{\isacharcomma}F{\isacharbraceright}\ {\isasymunion}\ {\isacharquery}Wo{\isacharparenright}{\isacharparenright}{\isachardoublequoteclose}\isanewline
\ \ \ \ \ \ \ \ \isacommand{by}\isamarkupfalse%
\ {\isacharparenleft}simp\ only{\isacharcolon}\ not{\isacharunderscore}not{\isacharparenright}\isanewline
\ \ \ \ \ \ \isacommand{then}\isamarkupfalse%
\ \isacommand{have}\isamarkupfalse%
\ {\isachardoublequoteopen}{\isasymnot}{\isacharparenleft}{\isasymnot}{\isacharparenleft}sat\ {\isacharparenleft}{\isacharbraceleft}G{\isacharcomma}F{\isacharbraceright}\ {\isasymunion}\ {\isacharquery}Wo{\isacharparenright}{\isacharparenright}\ {\isasymand}\ {\isasymnot}{\isacharparenleft}sat\ {\isacharparenleft}{\isacharbraceleft}H{\isacharcomma}F{\isacharbraceright}\ {\isasymunion}\ {\isacharquery}Wo{\isacharparenright}{\isacharparenright}{\isacharparenright}{\isachardoublequoteclose}\isanewline
\ \ \ \ \ \ \ \ \isacommand{by}\isamarkupfalse%
\ {\isacharparenleft}simp\ only{\isacharcolon}\ de{\isacharunderscore}Morgan{\isacharunderscore}disj\ simp{\isacharunderscore}thms{\isacharparenleft}{\isadigit{8}}{\isacharparenright}{\isacharparenright}\isanewline
\ \ \ \ \ \ \isacommand{thus}\isamarkupfalse%
\ {\isachardoublequoteopen}False{\isachardoublequoteclose}\isanewline
\ \ \ \ \ \ \ \ \isacommand{using}\isamarkupfalse%
\ {\isacartoucheopen}{\isasymnot}{\isacharparenleft}sat\ {\isacharparenleft}{\isacharbraceleft}G{\isacharcomma}F{\isacharbraceright}\ {\isasymunion}\ {\isacharquery}Wo{\isacharparenright}{\isacharparenright}\ {\isasymand}\ {\isasymnot}{\isacharparenleft}sat\ {\isacharparenleft}{\isacharbraceleft}H{\isacharcomma}F{\isacharbraceright}\ {\isasymunion}\ {\isacharquery}Wo{\isacharparenright}{\isacharparenright}{\isacartoucheclose}\ \isacommand{by}\isamarkupfalse%
\ {\isacharparenleft}rule\ notE{\isacharparenright}\isanewline
\ \ \ \ \isacommand{qed}\isamarkupfalse%
\isanewline
\ \ \isacommand{qed}\isamarkupfalse%
\isanewline
\isacommand{qed}\isamarkupfalse%
%
\endisatagproof
{\isafoldproof}%
%
\isadelimproof
%
\endisadelimproof
%
\begin{isamarkuptext}%
En resumen, con los lemas \isa{pcp{\isacharunderscore}colecComp{\isacharunderscore}bot}, \isa{pcp{\isacharunderscore}colecComp{\isacharunderscore}atoms}, \isa{pcp{\isacharunderscore}colecComp{\isacharunderscore}CON} y
  \isa{pcp{\isacharunderscore}colecComp{\isacharunderscore}DIS} podemos probar de manera detallada que la colección \isa{C} verifica la propiedad 
  de consistencia proposicional.%
\end{isamarkuptext}\isamarkuptrue%
\isacommand{lemma}\isamarkupfalse%
\ pcp{\isacharunderscore}colecComp{\isacharcolon}\ {\isachardoublequoteopen}pcp\ colecComp{\isachardoublequoteclose}\isanewline
%
\isadelimproof
%
\endisadelimproof
%
\isatagproof
\isacommand{proof}\isamarkupfalse%
\ {\isacharparenleft}rule\ pcp{\isacharunderscore}alt{\isadigit{2}}{\isacharparenright}\isanewline
\ \ \isacommand{show}\isamarkupfalse%
\ {\isachardoublequoteopen}{\isasymforall}W\ {\isasymin}\ colecComp{\isachardot}\ {\isasymbottom}\ {\isasymnotin}\ W\isanewline
\ \ \ \ \ \ \ \ {\isasymand}\ {\isacharparenleft}{\isasymforall}k{\isachardot}\ Atom\ k\ {\isasymin}\ W\ {\isasymlongrightarrow}\ \isactrlbold {\isasymnot}\ {\isacharparenleft}Atom\ k{\isacharparenright}\ {\isasymin}\ W\ {\isasymlongrightarrow}\ False{\isacharparenright}\isanewline
\ \ \ \ \ \ \ \ {\isasymand}\ {\isacharparenleft}{\isasymforall}F\ G\ H{\isachardot}\ Con\ F\ G\ H\ {\isasymlongrightarrow}\ F\ {\isasymin}\ W\ {\isasymlongrightarrow}\ {\isacharbraceleft}G{\isacharcomma}H{\isacharbraceright}\ {\isasymunion}\ W\ {\isasymin}\ colecComp{\isacharparenright}\isanewline
\ \ \ \ \ \ \ \ {\isasymand}\ {\isacharparenleft}{\isasymforall}F\ G\ H{\isachardot}\ Dis\ F\ G\ H\ {\isasymlongrightarrow}\ F\ {\isasymin}\ W\ {\isasymlongrightarrow}\ {\isacharbraceleft}G{\isacharbraceright}\ {\isasymunion}\ W\ {\isasymin}\ colecComp\ {\isasymor}\ {\isacharbraceleft}H{\isacharbraceright}\ {\isasymunion}\ W\ {\isasymin}\ colecComp{\isacharparenright}{\isachardoublequoteclose}\isanewline
\ \ \isacommand{proof}\isamarkupfalse%
\ {\isacharparenleft}rule\ ballI{\isacharparenright}\isanewline
\ \ \ \ \isacommand{fix}\isamarkupfalse%
\ W\isanewline
\ \ \ \ \isacommand{assume}\isamarkupfalse%
\ H{\isacharcolon}{\isachardoublequoteopen}W\ {\isasymin}\ colecComp{\isachardoublequoteclose}\isanewline
\ \ \ \ \isacommand{have}\isamarkupfalse%
\ C{\isadigit{1}}{\isacharcolon}{\isachardoublequoteopen}{\isasymbottom}\ {\isasymnotin}\ W{\isachardoublequoteclose}\isanewline
\ \ \ \ \ \ \isacommand{using}\isamarkupfalse%
\ H\ \isacommand{by}\isamarkupfalse%
\ {\isacharparenleft}rule\ pcp{\isacharunderscore}colecComp{\isacharunderscore}bot{\isacharparenright}\isanewline
\ \ \ \ \isacommand{have}\isamarkupfalse%
\ C{\isadigit{2}}{\isacharcolon}{\isachardoublequoteopen}{\isasymforall}k{\isachardot}\ Atom\ k\ {\isasymin}\ W\ {\isasymlongrightarrow}\ \isactrlbold {\isasymnot}\ {\isacharparenleft}Atom\ k{\isacharparenright}\ {\isasymin}\ W\ {\isasymlongrightarrow}\ False{\isachardoublequoteclose}\isanewline
\ \ \ \ \ \ \isacommand{using}\isamarkupfalse%
\ H\ \isacommand{by}\isamarkupfalse%
\ {\isacharparenleft}rule\ pcp{\isacharunderscore}colecComp{\isacharunderscore}atoms{\isacharparenright}\isanewline
\ \ \ \ \isacommand{have}\isamarkupfalse%
\ C{\isadigit{3}}{\isacharcolon}{\isachardoublequoteopen}{\isasymforall}F\ G\ H{\isachardot}\ Con\ F\ G\ H\ {\isasymlongrightarrow}\ F\ {\isasymin}\ W\ {\isasymlongrightarrow}\ {\isacharbraceleft}G{\isacharcomma}H{\isacharbraceright}\ {\isasymunion}\ W\ {\isasymin}\ colecComp{\isachardoublequoteclose}\isanewline
\ \ \ \ \ \ \isacommand{using}\isamarkupfalse%
\ H\ \isacommand{by}\isamarkupfalse%
\ {\isacharparenleft}rule\ pcp{\isacharunderscore}colecComp{\isacharunderscore}CON{\isacharparenright}\isanewline
\ \ \ \ \isacommand{have}\isamarkupfalse%
\ C{\isadigit{4}}{\isacharcolon}{\isachardoublequoteopen}{\isasymforall}F\ G\ H{\isachardot}\ Dis\ F\ G\ H\ {\isasymlongrightarrow}\ F\ {\isasymin}\ W\ {\isasymlongrightarrow}\ {\isacharbraceleft}G{\isacharbraceright}\ {\isasymunion}\ W\ {\isasymin}\ colecComp\ {\isasymor}\ {\isacharbraceleft}H{\isacharbraceright}\ {\isasymunion}\ W\ {\isasymin}\ colecComp{\isachardoublequoteclose}\isanewline
\ \ \ \ \ \ \isacommand{using}\isamarkupfalse%
\ H\ \isacommand{by}\isamarkupfalse%
\ {\isacharparenleft}rule\ pcp{\isacharunderscore}colecComp{\isacharunderscore}DIS{\isacharparenright}\isanewline
\ \ \ \ \isacommand{show}\isamarkupfalse%
\ {\isachardoublequoteopen}{\isasymbottom}\ {\isasymnotin}\ W\isanewline
\ \ \ \ \ \ \ \ \ \ {\isasymand}\ {\isacharparenleft}{\isasymforall}k{\isachardot}\ Atom\ k\ {\isasymin}\ W\ {\isasymlongrightarrow}\ \isactrlbold {\isasymnot}\ {\isacharparenleft}Atom\ k{\isacharparenright}\ {\isasymin}\ W\ {\isasymlongrightarrow}\ False{\isacharparenright}\isanewline
\ \ \ \ \ \ \ \ \ \ {\isasymand}\ {\isacharparenleft}{\isasymforall}F\ G\ H{\isachardot}\ Con\ F\ G\ H\ {\isasymlongrightarrow}\ F\ {\isasymin}\ W\ {\isasymlongrightarrow}\ {\isacharbraceleft}G{\isacharcomma}H{\isacharbraceright}\ {\isasymunion}\ W\ {\isasymin}\ colecComp{\isacharparenright}\isanewline
\ \ \ \ \ \ \ \ \ \ {\isasymand}\ {\isacharparenleft}{\isasymforall}F\ G\ H{\isachardot}\ Dis\ F\ G\ H\ {\isasymlongrightarrow}\ F\ {\isasymin}\ W\ {\isasymlongrightarrow}\ {\isacharbraceleft}G{\isacharbraceright}\ {\isasymunion}\ W\ {\isasymin}\ colecComp\ {\isasymor}\ {\isacharbraceleft}H{\isacharbraceright}\ {\isasymunion}\ W\ {\isasymin}\ colecComp{\isacharparenright}{\isachardoublequoteclose}\isanewline
\ \ \ \ \ \ \isacommand{using}\isamarkupfalse%
\ C{\isadigit{1}}\ C{\isadigit{2}}\ C{\isadigit{3}}\ C{\isadigit{4}}\ \isacommand{by}\isamarkupfalse%
\ {\isacharparenleft}iprover\ intro{\isacharcolon}\ conjI{\isacharparenright}\isanewline
\ \ \isacommand{qed}\isamarkupfalse%
\isanewline
\isacommand{qed}\isamarkupfalse%
%
\endisatagproof
{\isafoldproof}%
%
\isadelimproof
%
\endisadelimproof
%
\begin{isamarkuptext}%
Finalmente, mostremos la demostración del \isa{Teorema\ de\ Compacidad}.%
\end{isamarkuptext}\isamarkuptrue%
\isacommand{theorem}\isamarkupfalse%
\ prop{\isacharunderscore}Compactness{\isacharcolon}\isanewline
\ \ \isakeyword{fixes}\ W\ {\isacharcolon}{\isacharcolon}\ {\isachardoublequoteopen}{\isacharprime}a\ {\isacharcolon}{\isacharcolon}\ countable\ formula\ set{\isachardoublequoteclose}\isanewline
\ \ \isakeyword{assumes}\ {\isachardoublequoteopen}fin{\isacharunderscore}sat\ W{\isachardoublequoteclose}\isanewline
\ \ \isakeyword{shows}\ {\isachardoublequoteopen}sat\ W{\isachardoublequoteclose}\isanewline
%
\isadelimproof
%
\endisadelimproof
%
\isatagproof
\isacommand{proof}\isamarkupfalse%
\ {\isacharparenleft}rule\ pcp{\isacharunderscore}sat{\isacharparenright}\isanewline
\ \ \isacommand{show}\isamarkupfalse%
\ {\isachardoublequoteopen}W\ {\isasymin}\ colecComp{\isachardoublequoteclose}\isanewline
\ \ \ \ \isacommand{unfolding}\isamarkupfalse%
\ colecComp\ \isacommand{using}\isamarkupfalse%
\ assms\ \isacommand{unfolding}\isamarkupfalse%
\ fin{\isacharunderscore}sat{\isacharunderscore}def\ \isacommand{by}\isamarkupfalse%
\ {\isacharparenleft}rule\ CollectI{\isacharparenright}\isanewline
\ \ \isacommand{show}\isamarkupfalse%
\ {\isachardoublequoteopen}pcp\ colecComp{\isachardoublequoteclose}\isanewline
\ \ \ \ \isacommand{by}\isamarkupfalse%
\ {\isacharparenleft}simp\ only{\isacharcolon}\ pcp{\isacharunderscore}colecComp{\isacharparenright}\isanewline
\isacommand{qed}\isamarkupfalse%
\isanewline
%
\endisatagproof
{\isafoldproof}%
%
\isadelimproof
%
\endisadelimproof
%
\isadelimtheory
%
\endisadelimtheory
%
\isatagtheory
%
\endisatagtheory
{\isafoldtheory}%
%
\isadelimtheory
%
\endisadelimtheory
%
\end{isabellebody}%
\endinput
%:%file=~/TFM/TFM/TeoremaEx.thy%:%
%:%19=13%:%
%:%28=15%:%
%:%40=17%:%
%:%41=18%:%
%:%42=19%:%
%:%43=20%:%
%:%44=21%:%
%:%45=22%:%
%:%46=23%:%
%:%47=24%:%
%:%48=25%:%
%:%49=26%:%
%:%50=27%:%
%:%51=28%:%
%:%52=29%:%
%:%53=30%:%
%:%54=31%:%
%:%55=32%:%
%:%56=33%:%
%:%57=34%:%
%:%58=35%:%
%:%58=36%:%
%:%59=37%:%
%:%60=38%:%
%:%61=39%:%
%:%62=40%:%
%:%63=41%:%
%:%64=42%:%
%:%65=43%:%
%:%66=44%:%
%:%67=45%:%
%:%68=46%:%
%:%69=47%:%
%:%70=48%:%
%:%71=49%:%
%:%72=50%:%
%:%73=51%:%
%:%74=52%:%
%:%75=53%:%
%:%76=54%:%
%:%77=55%:%
%:%78=56%:%
%:%79=57%:%
%:%81=59%:%
%:%82=59%:%
%:%83=60%:%
%:%84=61%:%
%:%87=64%:%
%:%88=65%:%
%:%89=66%:%
%:%90=67%:%
%:%91=68%:%
%:%92=69%:%
%:%93=70%:%
%:%94=71%:%
%:%95=72%:%
%:%96=73%:%
%:%97=74%:%
%:%98=75%:%
%:%99=76%:%
%:%100=77%:%
%:%101=78%:%
%:%102=79%:%
%:%103=80%:%
%:%104=81%:%
%:%105=82%:%
%:%106=83%:%
%:%107=84%:%
%:%108=85%:%
%:%109=86%:%
%:%111=88%:%
%:%112=88%:%
%:%113=89%:%
%:%114=90%:%
%:%115=91%:%
%:%122=92%:%
%:%123=92%:%
%:%124=93%:%
%:%125=93%:%
%:%126=94%:%
%:%127=94%:%
%:%128=95%:%
%:%129=95%:%
%:%130=96%:%
%:%131=96%:%
%:%132=97%:%
%:%133=97%:%
%:%134=98%:%
%:%135=98%:%
%:%136=99%:%
%:%137=100%:%
%:%138=100%:%
%:%139=101%:%
%:%140=101%:%
%:%141=101%:%
%:%142=102%:%
%:%143=103%:%
%:%144=103%:%
%:%145=104%:%
%:%146=104%:%
%:%147=105%:%
%:%148=105%:%
%:%149=106%:%
%:%150=106%:%
%:%151=107%:%
%:%152=107%:%
%:%153=108%:%
%:%154=108%:%
%:%155=108%:%
%:%156=109%:%
%:%157=109%:%
%:%158=110%:%
%:%159=110%:%
%:%160=111%:%
%:%161=111%:%
%:%162=112%:%
%:%163=112%:%
%:%164=113%:%
%:%165=113%:%
%:%166=114%:%
%:%167=114%:%
%:%168=114%:%
%:%169=115%:%
%:%170=115%:%
%:%171=116%:%
%:%172=116%:%
%:%173=117%:%
%:%174=117%:%
%:%175=118%:%
%:%185=120%:%
%:%187=122%:%
%:%188=122%:%
%:%195=123%:%
%:%196=123%:%
%:%197=124%:%
%:%198=124%:%
%:%199=125%:%
%:%200=125%:%
%:%201=125%:%
%:%202=126%:%
%:%203=126%:%
%:%204=126%:%
%:%205=127%:%
%:%206=127%:%
%:%215=129%:%
%:%216=130%:%
%:%217=131%:%
%:%218=132%:%
%:%219=133%:%
%:%220=134%:%
%:%221=135%:%
%:%222=136%:%
%:%224=138%:%
%:%225=138%:%
%:%228=139%:%
%:%232=139%:%
%:%242=141%:%
%:%243=142%:%
%:%244=143%:%
%:%245=144%:%
%:%246=145%:%
%:%247=146%:%
%:%248=147%:%
%:%249=148%:%
%:%250=149%:%
%:%251=150%:%
%:%252=151%:%
%:%253=152%:%
%:%254=153%:%
%:%256=155%:%
%:%257=155%:%
%:%264=156%:%
%:%265=156%:%
%:%266=157%:%
%:%267=157%:%
%:%268=158%:%
%:%269=159%:%
%:%270=159%:%
%:%271=160%:%
%:%272=160%:%
%:%273=160%:%
%:%274=161%:%
%:%275=162%:%
%:%276=162%:%
%:%277=163%:%
%:%278=163%:%
%:%279=164%:%
%:%280=164%:%
%:%281=165%:%
%:%282=165%:%
%:%283=166%:%
%:%284=166%:%
%:%285=167%:%
%:%286=167%:%
%:%287=167%:%
%:%288=168%:%
%:%289=168%:%
%:%290=169%:%
%:%291=169%:%
%:%292=170%:%
%:%293=170%:%
%:%294=171%:%
%:%295=171%:%
%:%296=172%:%
%:%297=172%:%
%:%298=173%:%
%:%299=173%:%
%:%300=173%:%
%:%301=174%:%
%:%302=174%:%
%:%303=175%:%
%:%304=175%:%
%:%305=176%:%
%:%306=176%:%
%:%307=177%:%
%:%317=179%:%
%:%319=181%:%
%:%320=181%:%
%:%323=182%:%
%:%327=182%:%
%:%328=182%:%
%:%337=184%:%
%:%338=185%:%
%:%340=187%:%
%:%341=187%:%
%:%342=188%:%
%:%343=189%:%
%:%346=190%:%
%:%350=190%:%
%:%351=190%:%
%:%352=190%:%
%:%361=192%:%
%:%362=193%:%
%:%363=194%:%
%:%364=195%:%
%:%365=196%:%
%:%366=197%:%
%:%367=198%:%
%:%368=199%:%
%:%369=200%:%
%:%370=201%:%
%:%371=202%:%
%:%372=203%:%
%:%373=204%:%
%:%374=205%:%
%:%375=206%:%
%:%376=207%:%
%:%377=208%:%
%:%378=209%:%
%:%379=210%:%
%:%380=211%:%
%:%381=212%:%
%:%382=213%:%
%:%383=214%:%
%:%384=215%:%
%:%385=216%:%
%:%386=217%:%
%:%387=218%:%
%:%388=219%:%
%:%389=220%:%
%:%390=221%:%
%:%391=222%:%
%:%392=223%:%
%:%393=224%:%
%:%394=225%:%
%:%395=226%:%
%:%396=227%:%
%:%397=228%:%
%:%398=229%:%
%:%399=230%:%
%:%400=231%:%
%:%401=232%:%
%:%402=233%:%
%:%403=234%:%
%:%404=235%:%
%:%405=236%:%
%:%407=238%:%
%:%408=238%:%
%:%415=239%:%
%:%416=239%:%
%:%417=240%:%
%:%418=240%:%
%:%419=241%:%
%:%420=241%:%
%:%421=242%:%
%:%422=242%:%
%:%423=242%:%
%:%424=243%:%
%:%425=243%:%
%:%426=244%:%
%:%427=244%:%
%:%428=244%:%
%:%429=245%:%
%:%430=245%:%
%:%431=246%:%
%:%432=246%:%
%:%433=246%:%
%:%434=247%:%
%:%435=247%:%
%:%436=248%:%
%:%437=248%:%
%:%438=248%:%
%:%439=249%:%
%:%440=249%:%
%:%441=250%:%
%:%442=250%:%
%:%443=251%:%
%:%444=251%:%
%:%445=252%:%
%:%446=252%:%
%:%447=253%:%
%:%448=253%:%
%:%449=254%:%
%:%450=254%:%
%:%451=255%:%
%:%452=255%:%
%:%453=255%:%
%:%454=256%:%
%:%455=256%:%
%:%456=257%:%
%:%457=257%:%
%:%458=258%:%
%:%459=258%:%
%:%460=259%:%
%:%461=259%:%
%:%462=259%:%
%:%463=260%:%
%:%464=260%:%
%:%465=261%:%
%:%466=261%:%
%:%467=261%:%
%:%468=262%:%
%:%469=262%:%
%:%470=263%:%
%:%471=263%:%
%:%472=263%:%
%:%473=264%:%
%:%474=264%:%
%:%475=265%:%
%:%476=265%:%
%:%477=265%:%
%:%478=266%:%
%:%479=266%:%
%:%480=267%:%
%:%481=267%:%
%:%482=267%:%
%:%483=268%:%
%:%484=268%:%
%:%485=269%:%
%:%486=269%:%
%:%487=269%:%
%:%488=270%:%
%:%489=270%:%
%:%490=270%:%
%:%491=271%:%
%:%492=271%:%
%:%493=271%:%
%:%494=272%:%
%:%495=272%:%
%:%496=273%:%
%:%506=275%:%
%:%508=277%:%
%:%509=277%:%
%:%516=278%:%
%:%517=278%:%
%:%518=279%:%
%:%519=279%:%
%:%520=280%:%
%:%521=280%:%
%:%522=281%:%
%:%523=281%:%
%:%524=281%:%
%:%525=282%:%
%:%526=282%:%
%:%527=283%:%
%:%528=283%:%
%:%529=284%:%
%:%530=284%:%
%:%531=285%:%
%:%532=285%:%
%:%533=286%:%
%:%534=286%:%
%:%535=286%:%
%:%536=286%:%
%:%537=287%:%
%:%538=287%:%
%:%547=289%:%
%:%548=290%:%
%:%549=291%:%
%:%550=292%:%
%:%551=293%:%
%:%552=294%:%
%:%553=295%:%
%:%554=296%:%
%:%555=297%:%
%:%557=299%:%
%:%558=299%:%
%:%560=301%:%
%:%561=302%:%
%:%562=303%:%
%:%563=304%:%
%:%564=305%:%
%:%565=306%:%
%:%566=307%:%
%:%567=308%:%
%:%568=309%:%
%:%569=310%:%
%:%570=311%:%
%:%571=312%:%
%:%572=313%:%
%:%573=314%:%
%:%574=315%:%
%:%575=316%:%
%:%577=318%:%
%:%578=318%:%
%:%581=319%:%
%:%585=319%:%
%:%586=319%:%
%:%587=320%:%
%:%588=320%:%
%:%589=321%:%
%:%590=321%:%
%:%591=322%:%
%:%592=322%:%
%:%593=323%:%
%:%594=323%:%
%:%595=323%:%
%:%596=324%:%
%:%597=324%:%
%:%598=325%:%
%:%599=325%:%
%:%600=325%:%
%:%601=326%:%
%:%602=326%:%
%:%603=327%:%
%:%604=327%:%
%:%605=328%:%
%:%606=328%:%
%:%607=329%:%
%:%617=331%:%
%:%619=333%:%
%:%620=333%:%
%:%623=334%:%
%:%627=334%:%
%:%628=334%:%
%:%629=334%:%
%:%638=336%:%
%:%639=337%:%
%:%640=338%:%
%:%641=339%:%
%:%642=340%:%
%:%643=341%:%
%:%644=342%:%
%:%645=343%:%
%:%646=344%:%
%:%647=345%:%
%:%648=346%:%
%:%649=347%:%
%:%650=348%:%
%:%651=349%:%
%:%652=350%:%
%:%653=351%:%
%:%655=353%:%
%:%656=353%:%
%:%657=354%:%
%:%658=355%:%
%:%665=356%:%
%:%666=356%:%
%:%667=357%:%
%:%668=357%:%
%:%669=358%:%
%:%670=358%:%
%:%671=358%:%
%:%672=359%:%
%:%673=359%:%
%:%674=359%:%
%:%675=360%:%
%:%676=360%:%
%:%677=361%:%
%:%678=361%:%
%:%679=361%:%
%:%680=362%:%
%:%681=362%:%
%:%682=363%:%
%:%683=363%:%
%:%684=364%:%
%:%685=364%:%
%:%686=365%:%
%:%696=367%:%
%:%698=369%:%
%:%699=369%:%
%:%700=370%:%
%:%703=371%:%
%:%707=371%:%
%:%708=371%:%
%:%709=371%:%
%:%718=373%:%
%:%719=374%:%
%:%720=375%:%
%:%721=376%:%
%:%722=377%:%
%:%723=378%:%
%:%724=379%:%
%:%725=380%:%
%:%726=381%:%
%:%727=382%:%
%:%728=383%:%
%:%729=384%:%
%:%730=385%:%
%:%731=386%:%
%:%732=387%:%
%:%733=388%:%
%:%734=389%:%
%:%735=390%:%
%:%736=391%:%
%:%737=392%:%
%:%738=393%:%
%:%739=394%:%
%:%740=395%:%
%:%741=396%:%
%:%742=397%:%
%:%743=398%:%
%:%744=399%:%
%:%745=400%:%
%:%746=401%:%
%:%747=402%:%
%:%748=403%:%
%:%749=404%:%
%:%750=405%:%
%:%751=406%:%
%:%752=407%:%
%:%754=409%:%
%:%755=409%:%
%:%756=410%:%
%:%757=411%:%
%:%758=412%:%
%:%761=413%:%
%:%765=413%:%
%:%766=413%:%
%:%767=414%:%
%:%768=414%:%
%:%769=415%:%
%:%770=415%:%
%:%771=416%:%
%:%772=416%:%
%:%773=417%:%
%:%774=417%:%
%:%775=418%:%
%:%776=418%:%
%:%777=419%:%
%:%778=419%:%
%:%779=420%:%
%:%780=420%:%
%:%781=420%:%
%:%782=421%:%
%:%783=421%:%
%:%784=422%:%
%:%785=422%:%
%:%786=422%:%
%:%787=423%:%
%:%788=423%:%
%:%789=424%:%
%:%790=424%:%
%:%791=425%:%
%:%792=425%:%
%:%793=426%:%
%:%794=426%:%
%:%795=426%:%
%:%796=427%:%
%:%797=427%:%
%:%798=428%:%
%:%799=428%:%
%:%800=428%:%
%:%801=429%:%
%:%802=429%:%
%:%803=430%:%
%:%804=430%:%
%:%805=430%:%
%:%806=431%:%
%:%807=431%:%
%:%808=432%:%
%:%809=432%:%
%:%810=432%:%
%:%811=433%:%
%:%812=433%:%
%:%813=434%:%
%:%814=434%:%
%:%815=435%:%
%:%816=435%:%
%:%817=435%:%
%:%818=436%:%
%:%819=436%:%
%:%820=437%:%
%:%821=437%:%
%:%822=437%:%
%:%823=438%:%
%:%824=438%:%
%:%825=438%:%
%:%826=439%:%
%:%827=439%:%
%:%828=440%:%
%:%829=440%:%
%:%830=441%:%
%:%831=441%:%
%:%832=441%:%
%:%833=442%:%
%:%834=442%:%
%:%835=443%:%
%:%836=443%:%
%:%837=444%:%
%:%838=444%:%
%:%839=444%:%
%:%840=445%:%
%:%841=445%:%
%:%842=446%:%
%:%843=446%:%
%:%844=447%:%
%:%845=447%:%
%:%846=448%:%
%:%847=448%:%
%:%848=448%:%
%:%849=449%:%
%:%850=449%:%
%:%851=450%:%
%:%852=450%:%
%:%853=451%:%
%:%854=451%:%
%:%855=451%:%
%:%856=452%:%
%:%857=452%:%
%:%858=453%:%
%:%859=453%:%
%:%860=453%:%
%:%861=454%:%
%:%862=454%:%
%:%863=454%:%
%:%864=455%:%
%:%865=455%:%
%:%866=455%:%
%:%867=456%:%
%:%868=456%:%
%:%869=457%:%
%:%870=457%:%
%:%871=458%:%
%:%881=460%:%
%:%883=462%:%
%:%884=462%:%
%:%885=463%:%
%:%886=464%:%
%:%887=465%:%
%:%890=466%:%
%:%894=466%:%
%:%895=466%:%
%:%896=467%:%
%:%897=467%:%
%:%898=468%:%
%:%899=468%:%
%:%900=469%:%
%:%901=469%:%
%:%902=469%:%
%:%903=470%:%
%:%904=470%:%
%:%905=470%:%
%:%906=470%:%
%:%907=471%:%
%:%908=471%:%
%:%909=471%:%
%:%910=472%:%
%:%911=472%:%
%:%912=473%:%
%:%913=473%:%
%:%914=473%:%
%:%915=474%:%
%:%916=474%:%
%:%917=475%:%
%:%918=475%:%
%:%919=475%:%
%:%920=476%:%
%:%921=476%:%
%:%935=478%:%
%:%947=480%:%
%:%948=481%:%
%:%949=482%:%
%:%950=483%:%
%:%951=484%:%
%:%952=485%:%
%:%953=486%:%
%:%954=487%:%
%:%955=488%:%
%:%956=489%:%
%:%957=490%:%
%:%958=491%:%
%:%959=492%:%
%:%960=493%:%
%:%961=494%:%
%:%962=495%:%
%:%963=496%:%
%:%964=497%:%
%:%965=498%:%
%:%966=499%:%
%:%967=500%:%
%:%968=501%:%
%:%969=502%:%
%:%970=503%:%
%:%971=504%:%
%:%972=505%:%
%:%973=506%:%
%:%974=507%:%
%:%975=508%:%
%:%976=509%:%
%:%977=510%:%
%:%978=511%:%
%:%979=512%:%
%:%980=513%:%
%:%981=514%:%
%:%982=515%:%
%:%983=516%:%
%:%984=517%:%
%:%985=518%:%
%:%986=519%:%
%:%987=520%:%
%:%988=521%:%
%:%989=522%:%
%:%990=523%:%
%:%991=524%:%
%:%992=525%:%
%:%994=527%:%
%:%995=527%:%
%:%996=528%:%
%:%997=529%:%
%:%998=530%:%
%:%999=531%:%
%:%1000=532%:%
%:%1007=533%:%
%:%1008=533%:%
%:%1009=534%:%
%:%1010=534%:%
%:%1011=535%:%
%:%1012=535%:%
%:%1013=535%:%
%:%1014=535%:%
%:%1015=536%:%
%:%1016=536%:%
%:%1017=536%:%
%:%1018=537%:%
%:%1019=537%:%
%:%1020=538%:%
%:%1021=538%:%
%:%1022=539%:%
%:%1023=539%:%
%:%1024=539%:%
%:%1025=539%:%
%:%1026=540%:%
%:%1027=540%:%
%:%1028=541%:%
%:%1029=541%:%
%:%1030=542%:%
%:%1031=542%:%
%:%1032=543%:%
%:%1033=543%:%
%:%1034=544%:%
%:%1035=544%:%
%:%1036=545%:%
%:%1037=545%:%
%:%1038=546%:%
%:%1039=546%:%
%:%1040=547%:%
%:%1041=547%:%
%:%1042=547%:%
%:%1043=548%:%
%:%1044=548%:%
%:%1045=548%:%
%:%1046=549%:%
%:%1047=549%:%
%:%1048=550%:%
%:%1049=550%:%
%:%1050=550%:%
%:%1051=551%:%
%:%1052=551%:%
%:%1053=552%:%
%:%1054=552%:%
%:%1055=552%:%
%:%1056=553%:%
%:%1057=553%:%
%:%1058=554%:%
%:%1059=554%:%
%:%1060=554%:%
%:%1061=555%:%
%:%1062=555%:%
%:%1063=556%:%
%:%1064=556%:%
%:%1065=556%:%
%:%1066=557%:%
%:%1067=557%:%
%:%1068=558%:%
%:%1069=558%:%
%:%1070=559%:%
%:%1071=559%:%
%:%1072=560%:%
%:%1073=560%:%
%:%1074=560%:%
%:%1075=561%:%
%:%1085=563%:%
%:%1087=565%:%
%:%1088=565%:%
%:%1089=566%:%
%:%1090=567%:%
%:%1091=568%:%
%:%1092=569%:%
%:%1093=570%:%
%:%1100=571%:%
%:%1101=571%:%
%:%1102=572%:%
%:%1103=572%:%
%:%1104=572%:%
%:%1105=572%:%
%:%1106=573%:%
%:%1107=573%:%
%:%1108=573%:%
%:%1109=573%:%
%:%1110=574%:%
%:%1111=574%:%
%:%1112=575%:%
%:%1113=575%:%
%:%1114=576%:%
%:%1115=576%:%
%:%1116=577%:%
%:%1117=577%:%
%:%1118=578%:%
%:%1119=578%:%
%:%1120=578%:%
%:%1121=579%:%
%:%1122=579%:%
%:%1123=580%:%
%:%1124=580%:%
%:%1125=581%:%
%:%1126=581%:%
%:%1127=582%:%
%:%1128=582%:%
%:%1129=583%:%
%:%1130=583%:%
%:%1131=583%:%
%:%1132=584%:%
%:%1133=584%:%
%:%1134=584%:%
%:%1135=584%:%
%:%1136=585%:%
%:%1137=585%:%
%:%1138=585%:%
%:%1139=586%:%
%:%1140=586%:%
%:%1141=587%:%
%:%1142=587%:%
%:%1143=587%:%
%:%1144=588%:%
%:%1145=588%:%
%:%1146=589%:%
%:%1147=589%:%
%:%1148=589%:%
%:%1149=590%:%
%:%1150=590%:%
%:%1151=591%:%
%:%1152=591%:%
%:%1153=592%:%
%:%1154=592%:%
%:%1155=592%:%
%:%1156=592%:%
%:%1157=593%:%
%:%1158=593%:%
%:%1159=594%:%
%:%1160=594%:%
%:%1161=594%:%
%:%1162=594%:%
%:%1163=594%:%
%:%1164=595%:%
%:%1174=597%:%
%:%1175=598%:%
%:%1176=599%:%
%:%1177=600%:%
%:%1178=601%:%
%:%1179=602%:%
%:%1180=603%:%
%:%1181=604%:%
%:%1182=605%:%
%:%1183=606%:%
%:%1184=607%:%
%:%1185=608%:%
%:%1186=609%:%
%:%1187=610%:%
%:%1188=611%:%
%:%1189=612%:%
%:%1190=613%:%
%:%1191=614%:%
%:%1192=615%:%
%:%1193=616%:%
%:%1194=617%:%
%:%1195=618%:%
%:%1196=619%:%
%:%1197=620%:%
%:%1198=621%:%
%:%1199=622%:%
%:%1200=623%:%
%:%1201=624%:%
%:%1202=625%:%
%:%1203=626%:%
%:%1204=627%:%
%:%1205=628%:%
%:%1206=629%:%
%:%1207=630%:%
%:%1209=632%:%
%:%1210=632%:%
%:%1211=633%:%
%:%1212=634%:%
%:%1213=635%:%
%:%1214=636%:%
%:%1215=637%:%
%:%1222=638%:%
%:%1223=638%:%
%:%1224=639%:%
%:%1225=639%:%
%:%1226=640%:%
%:%1227=640%:%
%:%1228=641%:%
%:%1229=641%:%
%:%1230=641%:%
%:%1231=642%:%
%:%1232=642%:%
%:%1233=643%:%
%:%1234=643%:%
%:%1235=644%:%
%:%1236=644%:%
%:%1237=644%:%
%:%1238=645%:%
%:%1239=645%:%
%:%1240=645%:%
%:%1241=646%:%
%:%1242=646%:%
%:%1243=646%:%
%:%1244=647%:%
%:%1245=647%:%
%:%1246=648%:%
%:%1247=648%:%
%:%1248=648%:%
%:%1249=649%:%
%:%1250=649%:%
%:%1251=650%:%
%:%1252=650%:%
%:%1253=651%:%
%:%1254=651%:%
%:%1255=651%:%
%:%1256=652%:%
%:%1257=652%:%
%:%1258=653%:%
%:%1259=653%:%
%:%1260=653%:%
%:%1261=654%:%
%:%1262=654%:%
%:%1263=655%:%
%:%1264=655%:%
%:%1265=655%:%
%:%1266=656%:%
%:%1267=656%:%
%:%1268=657%:%
%:%1269=657%:%
%:%1270=658%:%
%:%1271=658%:%
%:%1272=658%:%
%:%1273=659%:%
%:%1274=659%:%
%:%1275=660%:%
%:%1276=660%:%
%:%1277=660%:%
%:%1278=661%:%
%:%1279=661%:%
%:%1280=661%:%
%:%1281=662%:%
%:%1282=662%:%
%:%1283=663%:%
%:%1284=663%:%
%:%1285=664%:%
%:%1286=664%:%
%:%1287=665%:%
%:%1288=665%:%
%:%1289=665%:%
%:%1290=666%:%
%:%1291=666%:%
%:%1292=667%:%
%:%1293=667%:%
%:%1294=667%:%
%:%1295=668%:%
%:%1296=668%:%
%:%1297=669%:%
%:%1298=669%:%
%:%1299=670%:%
%:%1300=671%:%
%:%1301=671%:%
%:%1302=672%:%
%:%1303=672%:%
%:%1304=672%:%
%:%1305=673%:%
%:%1306=674%:%
%:%1307=674%:%
%:%1308=675%:%
%:%1309=675%:%
%:%1310=675%:%
%:%1311=676%:%
%:%1312=676%:%
%:%1313=676%:%
%:%1314=677%:%
%:%1315=677%:%
%:%1316=677%:%
%:%1317=678%:%
%:%1318=678%:%
%:%1319=679%:%
%:%1320=679%:%
%:%1321=679%:%
%:%1322=680%:%
%:%1323=680%:%
%:%1324=681%:%
%:%1325=681%:%
%:%1326=682%:%
%:%1327=682%:%
%:%1328=682%:%
%:%1329=683%:%
%:%1330=683%:%
%:%1331=684%:%
%:%1332=684%:%
%:%1333=685%:%
%:%1334=685%:%
%:%1335=686%:%
%:%1336=686%:%
%:%1337=686%:%
%:%1338=687%:%
%:%1339=687%:%
%:%1340=687%:%
%:%1341=688%:%
%:%1342=688%:%
%:%1343=688%:%
%:%1344=689%:%
%:%1345=689%:%
%:%1346=689%:%
%:%1347=690%:%
%:%1348=690%:%
%:%1349=691%:%
%:%1350=691%:%
%:%1351=691%:%
%:%1352=692%:%
%:%1353=692%:%
%:%1354=692%:%
%:%1355=693%:%
%:%1356=693%:%
%:%1357=693%:%
%:%1358=694%:%
%:%1359=694%:%
%:%1360=694%:%
%:%1361=695%:%
%:%1362=695%:%
%:%1363=695%:%
%:%1364=696%:%
%:%1365=696%:%
%:%1366=697%:%
%:%1367=697%:%
%:%1368=697%:%
%:%1369=698%:%
%:%1379=700%:%
%:%1381=702%:%
%:%1382=702%:%
%:%1383=703%:%
%:%1384=704%:%
%:%1385=705%:%
%:%1386=706%:%
%:%1387=707%:%
%:%1394=708%:%
%:%1395=708%:%
%:%1396=709%:%
%:%1397=709%:%
%:%1398=710%:%
%:%1399=710%:%
%:%1400=710%:%
%:%1401=710%:%
%:%1402=711%:%
%:%1403=711%:%
%:%1404=711%:%
%:%1405=711%:%
%:%1406=712%:%
%:%1407=712%:%
%:%1408=712%:%
%:%1409=712%:%
%:%1410=712%:%
%:%1411=713%:%
%:%1412=713%:%
%:%1413=713%:%
%:%1414=714%:%
%:%1415=714%:%
%:%1416=714%:%
%:%1417=714%:%
%:%1418=715%:%
%:%1419=715%:%
%:%1420=715%:%
%:%1421=716%:%
%:%1422=716%:%
%:%1423=716%:%
%:%1424=716%:%
%:%1425=717%:%
%:%1426=717%:%
%:%1427=717%:%
%:%1428=717%:%
%:%1429=718%:%
%:%1439=720%:%
%:%1440=721%:%
%:%1441=722%:%
%:%1442=723%:%
%:%1443=724%:%
%:%1444=725%:%
%:%1445=726%:%
%:%1446=727%:%
%:%1447=728%:%
%:%1448=729%:%
%:%1449=730%:%
%:%1450=731%:%
%:%1451=732%:%
%:%1452=733%:%
%:%1453=734%:%
%:%1454=735%:%
%:%1455=736%:%
%:%1456=737%:%
%:%1457=738%:%
%:%1458=739%:%
%:%1459=740%:%
%:%1460=741%:%
%:%1462=743%:%
%:%1463=743%:%
%:%1464=744%:%
%:%1465=745%:%
%:%1466=746%:%
%:%1473=747%:%
%:%1474=747%:%
%:%1475=748%:%
%:%1476=748%:%
%:%1477=749%:%
%:%1478=749%:%
%:%1479=750%:%
%:%1480=750%:%
%:%1481=751%:%
%:%1482=751%:%
%:%1483=752%:%
%:%1484=752%:%
%:%1485=752%:%
%:%1486=753%:%
%:%1487=753%:%
%:%1488=753%:%
%:%1489=754%:%
%:%1490=754%:%
%:%1491=755%:%
%:%1492=755%:%
%:%1493=755%:%
%:%1494=756%:%
%:%1495=756%:%
%:%1496=757%:%
%:%1497=757%:%
%:%1498=758%:%
%:%1499=758%:%
%:%1500=759%:%
%:%1510=761%:%
%:%1512=763%:%
%:%1513=763%:%
%:%1514=764%:%
%:%1515=765%:%
%:%1516=766%:%
%:%1519=767%:%
%:%1523=767%:%
%:%1524=767%:%
%:%1525=767%:%
%:%1534=769%:%
%:%1535=770%:%
%:%1536=771%:%
%:%1537=772%:%
%:%1538=773%:%
%:%1539=774%:%
%:%1540=775%:%
%:%1541=776%:%
%:%1542=777%:%
%:%1543=778%:%
%:%1544=779%:%
%:%1545=780%:%
%:%1546=781%:%
%:%1547=782%:%
%:%1548=783%:%
%:%1549=784%:%
%:%1550=785%:%
%:%1551=786%:%
%:%1552=787%:%
%:%1553=788%:%
%:%1554=789%:%
%:%1555=790%:%
%:%1556=791%:%
%:%1557=792%:%
%:%1558=793%:%
%:%1559=794%:%
%:%1560=795%:%
%:%1561=796%:%
%:%1562=797%:%
%:%1563=798%:%
%:%1564=799%:%
%:%1565=800%:%
%:%1566=801%:%
%:%1567=802%:%
%:%1568=803%:%
%:%1569=804%:%
%:%1570=805%:%
%:%1571=806%:%
%:%1572=807%:%
%:%1573=808%:%
%:%1574=809%:%
%:%1575=810%:%
%:%1576=811%:%
%:%1577=812%:%
%:%1578=813%:%
%:%1579=814%:%
%:%1580=815%:%
%:%1581=816%:%
%:%1582=817%:%
%:%1583=818%:%
%:%1584=819%:%
%:%1585=820%:%
%:%1586=821%:%
%:%1587=822%:%
%:%1588=823%:%
%:%1589=824%:%
%:%1590=825%:%
%:%1591=826%:%
%:%1592=827%:%
%:%1593=828%:%
%:%1594=829%:%
%:%1595=830%:%
%:%1596=831%:%
%:%1597=832%:%
%:%1598=833%:%
%:%1599=834%:%
%:%1600=835%:%
%:%1601=836%:%
%:%1602=837%:%
%:%1603=838%:%
%:%1604=839%:%
%:%1605=840%:%
%:%1607=842%:%
%:%1608=842%:%
%:%1609=843%:%
%:%1610=844%:%
%:%1611=845%:%
%:%1612=846%:%
%:%1613=847%:%
%:%1620=848%:%
%:%1621=848%:%
%:%1622=849%:%
%:%1623=849%:%
%:%1624=850%:%
%:%1625=850%:%
%:%1626=851%:%
%:%1627=851%:%
%:%1628=851%:%
%:%1629=852%:%
%:%1630=852%:%
%:%1634=856%:%
%:%1635=857%:%
%:%1636=857%:%
%:%1637=857%:%
%:%1638=858%:%
%:%1639=858%:%
%:%1640=858%:%
%:%1643=861%:%
%:%1644=862%:%
%:%1645=862%:%
%:%1646=862%:%
%:%1647=863%:%
%:%1648=863%:%
%:%1649=863%:%
%:%1650=864%:%
%:%1651=864%:%
%:%1652=865%:%
%:%1653=865%:%
%:%1654=866%:%
%:%1655=866%:%
%:%1656=866%:%
%:%1657=867%:%
%:%1658=867%:%
%:%1659=868%:%
%:%1660=868%:%
%:%1661=868%:%
%:%1662=869%:%
%:%1663=869%:%
%:%1664=870%:%
%:%1665=870%:%
%:%1666=871%:%
%:%1667=871%:%
%:%1668=872%:%
%:%1669=872%:%
%:%1670=873%:%
%:%1671=873%:%
%:%1672=874%:%
%:%1673=874%:%
%:%1674=875%:%
%:%1675=875%:%
%:%1676=876%:%
%:%1677=876%:%
%:%1678=877%:%
%:%1679=877%:%
%:%1680=877%:%
%:%1681=878%:%
%:%1682=878%:%
%:%1683=878%:%
%:%1684=879%:%
%:%1685=879%:%
%:%1686=879%:%
%:%1687=880%:%
%:%1688=880%:%
%:%1689=880%:%
%:%1690=881%:%
%:%1691=881%:%
%:%1692=881%:%
%:%1693=882%:%
%:%1694=882%:%
%:%1695=883%:%
%:%1696=883%:%
%:%1697=884%:%
%:%1698=884%:%
%:%1699=884%:%
%:%1700=885%:%
%:%1701=885%:%
%:%1702=885%:%
%:%1703=886%:%
%:%1704=886%:%
%:%1705=887%:%
%:%1706=887%:%
%:%1707=888%:%
%:%1708=888%:%
%:%1709=888%:%
%:%1710=889%:%
%:%1711=889%:%
%:%1712=890%:%
%:%1713=890%:%
%:%1714=891%:%
%:%1715=891%:%
%:%1716=891%:%
%:%1717=892%:%
%:%1718=892%:%
%:%1719=892%:%
%:%1720=893%:%
%:%1721=893%:%
%:%1722=894%:%
%:%1723=894%:%
%:%1724=894%:%
%:%1725=895%:%
%:%1726=895%:%
%:%1727=896%:%
%:%1728=896%:%
%:%1729=896%:%
%:%1730=897%:%
%:%1731=897%:%
%:%1732=898%:%
%:%1733=898%:%
%:%1734=899%:%
%:%1735=899%:%
%:%1736=899%:%
%:%1737=900%:%
%:%1738=900%:%
%:%1739=900%:%
%:%1740=901%:%
%:%1741=901%:%
%:%1742=902%:%
%:%1743=902%:%
%:%1744=902%:%
%:%1745=903%:%
%:%1746=903%:%
%:%1747=904%:%
%:%1748=904%:%
%:%1749=905%:%
%:%1750=905%:%
%:%1751=905%:%
%:%1752=906%:%
%:%1753=906%:%
%:%1754=907%:%
%:%1755=907%:%
%:%1756=908%:%
%:%1757=908%:%
%:%1758=909%:%
%:%1759=909%:%
%:%1760=909%:%
%:%1761=910%:%
%:%1762=910%:%
%:%1763=911%:%
%:%1764=911%:%
%:%1765=912%:%
%:%1766=912%:%
%:%1767=913%:%
%:%1768=913%:%
%:%1769=914%:%
%:%1770=914%:%
%:%1771=915%:%
%:%1772=915%:%
%:%1773=916%:%
%:%1774=916%:%
%:%1775=917%:%
%:%1776=917%:%
%:%1777=918%:%
%:%1778=918%:%
%:%1779=918%:%
%:%1780=919%:%
%:%1781=919%:%
%:%1782=919%:%
%:%1783=920%:%
%:%1784=920%:%
%:%1785=920%:%
%:%1786=921%:%
%:%1787=921%:%
%:%1788=921%:%
%:%1789=922%:%
%:%1790=922%:%
%:%1791=922%:%
%:%1792=923%:%
%:%1793=923%:%
%:%1794=924%:%
%:%1795=924%:%
%:%1796=925%:%
%:%1797=925%:%
%:%1798=926%:%
%:%1799=926%:%
%:%1800=927%:%
%:%1801=927%:%
%:%1802=928%:%
%:%1803=928%:%
%:%1804=929%:%
%:%1805=929%:%
%:%1806=929%:%
%:%1807=930%:%
%:%1808=930%:%
%:%1809=931%:%
%:%1810=931%:%
%:%1811=931%:%
%:%1812=932%:%
%:%1813=932%:%
%:%1814=932%:%
%:%1815=933%:%
%:%1816=933%:%
%:%1817=934%:%
%:%1818=934%:%
%:%1819=935%:%
%:%1820=935%:%
%:%1821=936%:%
%:%1822=936%:%
%:%1823=937%:%
%:%1824=937%:%
%:%1825=938%:%
%:%1826=938%:%
%:%1827=939%:%
%:%1828=939%:%
%:%1829=940%:%
%:%1830=940%:%
%:%1831=941%:%
%:%1832=941%:%
%:%1833=941%:%
%:%1834=942%:%
%:%1835=942%:%
%:%1836=943%:%
%:%1837=943%:%
%:%1838=943%:%
%:%1839=944%:%
%:%1840=944%:%
%:%1841=944%:%
%:%1842=945%:%
%:%1843=945%:%
%:%1844=946%:%
%:%1845=946%:%
%:%1846=947%:%
%:%1847=947%:%
%:%1848=948%:%
%:%1849=948%:%
%:%1850=949%:%
%:%1851=949%:%
%:%1852=950%:%
%:%1853=950%:%
%:%1854=951%:%
%:%1855=951%:%
%:%1858=954%:%
%:%1859=955%:%
%:%1860=955%:%
%:%1861=955%:%
%:%1862=956%:%
%:%1872=958%:%
%:%1874=960%:%
%:%1875=960%:%
%:%1876=961%:%
%:%1877=962%:%
%:%1878=963%:%
%:%1879=964%:%
%:%1880=965%:%
%:%1887=966%:%
%:%1888=966%:%
%:%1889=967%:%
%:%1890=967%:%
%:%1891=968%:%
%:%1892=968%:%
%:%1893=968%:%
%:%1894=968%:%
%:%1895=969%:%
%:%1896=969%:%
%:%1897=970%:%
%:%1898=970%:%
%:%1899=971%:%
%:%1900=972%:%
%:%1901=973%:%
%:%1902=974%:%
%:%1903=974%:%
%:%1904=975%:%
%:%1905=975%:%
%:%1906=975%:%
%:%1907=975%:%
%:%1908=976%:%
%:%1909=976%:%
%:%1910=976%:%
%:%1911=977%:%
%:%1921=979%:%
%:%1922=980%:%
%:%1923=981%:%
%:%1924=982%:%
%:%1925=983%:%
%:%1926=984%:%
%:%1927=985%:%
%:%1928=986%:%
%:%1929=987%:%
%:%1930=988%:%
%:%1931=989%:%
%:%1932=990%:%
%:%1933=991%:%
%:%1934=992%:%
%:%1935=993%:%
%:%1936=994%:%
%:%1938=996%:%
%:%1939=996%:%
%:%1940=997%:%
%:%1941=998%:%
%:%1942=999%:%
%:%1945=1000%:%
%:%1949=1000%:%
%:%1950=1000%:%
%:%1951=1001%:%
%:%1952=1001%:%
%:%1953=1002%:%
%:%1954=1002%:%
%:%1955=1003%:%
%:%1956=1003%:%
%:%1957=1003%:%
%:%1958=1004%:%
%:%1959=1004%:%
%:%1960=1005%:%
%:%1961=1005%:%
%:%1962=1005%:%
%:%1963=1006%:%
%:%1964=1006%:%
%:%1965=1007%:%
%:%1966=1007%:%
%:%1967=1008%:%
%:%1968=1008%:%
%:%1969=1009%:%
%:%1970=1009%:%
%:%1971=1010%:%
%:%1972=1010%:%
%:%1973=1011%:%
%:%1974=1011%:%
%:%1975=1011%:%
%:%1976=1012%:%
%:%1977=1012%:%
%:%1978=1012%:%
%:%1979=1013%:%
%:%1980=1013%:%
%:%1981=1013%:%
%:%1982=1014%:%
%:%1983=1014%:%
%:%1984=1015%:%
%:%1985=1015%:%
%:%1986=1015%:%
%:%1987=1016%:%
%:%1988=1016%:%
%:%1989=1017%:%
%:%1990=1017%:%
%:%1991=1018%:%
%:%1992=1018%:%
%:%1993=1019%:%
%:%2003=1021%:%
%:%2004=1022%:%
%:%2005=1023%:%
%:%2006=1024%:%
%:%2007=1025%:%
%:%2008=1026%:%
%:%2009=1027%:%
%:%2010=1028%:%
%:%2011=1029%:%
%:%2012=1030%:%
%:%2013=1031%:%
%:%2014=1032%:%
%:%2015=1033%:%
%:%2016=1034%:%
%:%2017=1035%:%
%:%2018=1036%:%
%:%2019=1037%:%
%:%2020=1038%:%
%:%2021=1039%:%
%:%2022=1040%:%
%:%2023=1041%:%
%:%2024=1042%:%
%:%2025=1043%:%
%:%2026=1044%:%
%:%2027=1045%:%
%:%2028=1046%:%
%:%2029=1047%:%
%:%2030=1048%:%
%:%2031=1049%:%
%:%2032=1050%:%
%:%2033=1051%:%
%:%2034=1052%:%
%:%2035=1053%:%
%:%2036=1054%:%
%:%2038=1056%:%
%:%2039=1056%:%
%:%2040=1057%:%
%:%2041=1058%:%
%:%2042=1059%:%
%:%2043=1060%:%
%:%2050=1061%:%
%:%2051=1061%:%
%:%2052=1062%:%
%:%2053=1062%:%
%:%2054=1063%:%
%:%2055=1063%:%
%:%2056=1064%:%
%:%2057=1064%:%
%:%2058=1064%:%
%:%2059=1065%:%
%:%2060=1065%:%
%:%2061=1066%:%
%:%2062=1066%:%
%:%2063=1067%:%
%:%2064=1067%:%
%:%2065=1067%:%
%:%2066=1068%:%
%:%2067=1068%:%
%:%2068=1069%:%
%:%2069=1069%:%
%:%2070=1069%:%
%:%2071=1070%:%
%:%2072=1070%:%
%:%2073=1071%:%
%:%2074=1071%:%
%:%2075=1071%:%
%:%2076=1072%:%
%:%2077=1072%:%
%:%2078=1073%:%
%:%2079=1073%:%
%:%2080=1073%:%
%:%2081=1074%:%
%:%2082=1074%:%
%:%2083=1075%:%
%:%2084=1075%:%
%:%2085=1075%:%
%:%2086=1076%:%
%:%2087=1076%:%
%:%2088=1077%:%
%:%2089=1077%:%
%:%2090=1077%:%
%:%2091=1078%:%
%:%2092=1078%:%
%:%2093=1079%:%
%:%2094=1079%:%
%:%2095=1079%:%
%:%2096=1080%:%
%:%2097=1080%:%
%:%2098=1080%:%
%:%2099=1081%:%
%:%2100=1081%:%
%:%2101=1081%:%
%:%2102=1082%:%
%:%2103=1082%:%
%:%2104=1083%:%
%:%2105=1083%:%
%:%2106=1083%:%
%:%2107=1084%:%
%:%2108=1084%:%
%:%2109=1085%:%
%:%2110=1085%:%
%:%2111=1085%:%
%:%2112=1086%:%
%:%2113=1086%:%
%:%2114=1086%:%
%:%2115=1087%:%
%:%2116=1087%:%
%:%2117=1088%:%
%:%2118=1088%:%
%:%2119=1089%:%
%:%2120=1089%:%
%:%2121=1089%:%
%:%2122=1090%:%
%:%2123=1090%:%
%:%2124=1090%:%
%:%2125=1091%:%
%:%2126=1091%:%
%:%2127=1091%:%
%:%2128=1092%:%
%:%2129=1092%:%
%:%2130=1093%:%
%:%2131=1093%:%
%:%2132=1093%:%
%:%2133=1094%:%
%:%2134=1094%:%
%:%2135=1094%:%
%:%2136=1095%:%
%:%2137=1095%:%
%:%2138=1096%:%
%:%2139=1096%:%
%:%2140=1097%:%
%:%2141=1097%:%
%:%2142=1098%:%
%:%2143=1098%:%
%:%2144=1099%:%
%:%2145=1099%:%
%:%2146=1100%:%
%:%2147=1100%:%
%:%2148=1100%:%
%:%2149=1101%:%
%:%2150=1101%:%
%:%2151=1102%:%
%:%2152=1102%:%
%:%2153=1103%:%
%:%2154=1103%:%
%:%2155=1103%:%
%:%2156=1104%:%
%:%2166=1106%:%
%:%2168=1108%:%
%:%2169=1108%:%
%:%2170=1109%:%
%:%2171=1110%:%
%:%2172=1111%:%
%:%2173=1112%:%
%:%2180=1113%:%
%:%2181=1113%:%
%:%2182=1114%:%
%:%2183=1114%:%
%:%2184=1114%:%
%:%2185=1115%:%
%:%2186=1116%:%
%:%2187=1116%:%
%:%2188=1117%:%
%:%2189=1117%:%
%:%2190=1118%:%
%:%2191=1118%:%
%:%2192=1118%:%
%:%2193=1118%:%
%:%2194=1119%:%
%:%2195=1119%:%
%:%2196=1120%:%
%:%2197=1120%:%
%:%2198=1120%:%
%:%2199=1121%:%
%:%2200=1121%:%
%:%2201=1121%:%
%:%2202=1121%:%
%:%2203=1122%:%
%:%2204=1122%:%
%:%2205=1122%:%
%:%2206=1123%:%
%:%2207=1123%:%
%:%2208=1123%:%
%:%2209=1124%:%
%:%2210=1124%:%
%:%2211=1124%:%
%:%2212=1124%:%
%:%2213=1124%:%
%:%2214=1125%:%
%:%2229=1127%:%
%:%2241=1129%:%
%:%2242=1130%:%
%:%2243=1131%:%
%:%2244=1132%:%
%:%2245=1133%:%
%:%2246=1134%:%
%:%2247=1135%:%
%:%2248=1136%:%
%:%2249=1137%:%
%:%2250=1138%:%
%:%2251=1139%:%
%:%2252=1140%:%
%:%2253=1141%:%
%:%2254=1142%:%
%:%2255=1143%:%
%:%2256=1144%:%
%:%2257=1145%:%
%:%2258=1146%:%
%:%2259=1147%:%
%:%2260=1148%:%
%:%2261=1149%:%
%:%2262=1150%:%
%:%2263=1151%:%
%:%2264=1152%:%
%:%2265=1153%:%
%:%2266=1154%:%
%:%2267=1155%:%
%:%2268=1156%:%
%:%2269=1157%:%
%:%2270=1158%:%
%:%2271=1159%:%
%:%2272=1160%:%
%:%2273=1161%:%
%:%2274=1162%:%
%:%2276=1164%:%
%:%2277=1164%:%
%:%2280=1165%:%
%:%2284=1165%:%
%:%2285=1165%:%
%:%2294=1167%:%
%:%2296=1169%:%
%:%2297=1169%:%
%:%2298=1170%:%
%:%2299=1171%:%
%:%2300=1172%:%
%:%2307=1173%:%
%:%2308=1173%:%
%:%2309=1174%:%
%:%2310=1174%:%
%:%2311=1175%:%
%:%2312=1175%:%
%:%2313=1175%:%
%:%2314=1176%:%
%:%2315=1176%:%
%:%2316=1177%:%
%:%2317=1177%:%
%:%2318=1177%:%
%:%2319=1178%:%
%:%2320=1178%:%
%:%2321=1179%:%
%:%2322=1179%:%
%:%2323=1180%:%
%:%2324=1180%:%
%:%2325=1180%:%
%:%2326=1181%:%
%:%2327=1181%:%
%:%2328=1182%:%
%:%2329=1182%:%
%:%2330=1182%:%
%:%2331=1183%:%
%:%2332=1183%:%
%:%2333=1184%:%
%:%2334=1184%:%
%:%2335=1184%:%
%:%2336=1185%:%
%:%2337=1185%:%
%:%2338=1185%:%
%:%2339=1185%:%
%:%2340=1186%:%
%:%2341=1186%:%
%:%2342=1187%:%
%:%2343=1187%:%
%:%2344=1188%:%
%:%2345=1188%:%
%:%2346=1188%:%
%:%2347=1189%:%
%:%2348=1189%:%
%:%2349=1189%:%
%:%2350=1190%:%
%:%2351=1190%:%
%:%2352=1191%:%
%:%2353=1191%:%
%:%2354=1192%:%
%:%2355=1192%:%
%:%2356=1193%:%
%:%2357=1193%:%
%:%2358=1193%:%
%:%2359=1194%:%
%:%2360=1194%:%
%:%2361=1195%:%
%:%2362=1195%:%
%:%2363=1195%:%
%:%2364=1196%:%
%:%2374=1198%:%
%:%2375=1199%:%
%:%2376=1200%:%
%:%2377=1201%:%
%:%2378=1202%:%
%:%2379=1203%:%
%:%2380=1204%:%
%:%2381=1205%:%
%:%2382=1206%:%
%:%2383=1207%:%
%:%2384=1208%:%
%:%2385=1209%:%
%:%2386=1210%:%
%:%2387=1211%:%
%:%2388=1212%:%
%:%2389=1213%:%
%:%2390=1214%:%
%:%2391=1215%:%
%:%2392=1216%:%
%:%2393=1217%:%
%:%2394=1218%:%
%:%2395=1219%:%
%:%2396=1220%:%
%:%2397=1221%:%
%:%2398=1222%:%
%:%2399=1223%:%
%:%2400=1224%:%
%:%2401=1225%:%
%:%2402=1226%:%
%:%2403=1227%:%
%:%2404=1228%:%
%:%2405=1229%:%
%:%2406=1230%:%
%:%2407=1231%:%
%:%2408=1232%:%
%:%2409=1233%:%
%:%2411=1235%:%
%:%2412=1235%:%
%:%2413=1236%:%
%:%2414=1237%:%
%:%2415=1238%:%
%:%2422=1239%:%
%:%2423=1239%:%
%:%2424=1240%:%
%:%2425=1240%:%
%:%2426=1241%:%
%:%2427=1241%:%
%:%2428=1242%:%
%:%2429=1242%:%
%:%2430=1243%:%
%:%2431=1243%:%
%:%2432=1244%:%
%:%2433=1244%:%
%:%2434=1245%:%
%:%2435=1245%:%
%:%2436=1246%:%
%:%2437=1246%:%
%:%2438=1247%:%
%:%2439=1247%:%
%:%2440=1248%:%
%:%2441=1248%:%
%:%2442=1248%:%
%:%2443=1249%:%
%:%2444=1249%:%
%:%2445=1250%:%
%:%2446=1250%:%
%:%2447=1250%:%
%:%2448=1251%:%
%:%2449=1251%:%
%:%2450=1251%:%
%:%2451=1252%:%
%:%2452=1252%:%
%:%2453=1253%:%
%:%2454=1253%:%
%:%2455=1253%:%
%:%2456=1254%:%
%:%2457=1254%:%
%:%2458=1255%:%
%:%2459=1255%:%
%:%2460=1256%:%
%:%2461=1256%:%
%:%2462=1256%:%
%:%2463=1257%:%
%:%2464=1257%:%
%:%2465=1257%:%
%:%2466=1258%:%
%:%2467=1258%:%
%:%2468=1258%:%
%:%2469=1259%:%
%:%2470=1259%:%
%:%2471=1260%:%
%:%2472=1260%:%
%:%2473=1260%:%
%:%2474=1261%:%
%:%2475=1261%:%
%:%2476=1262%:%
%:%2477=1262%:%
%:%2478=1263%:%
%:%2479=1263%:%
%:%2480=1263%:%
%:%2481=1264%:%
%:%2482=1264%:%
%:%2483=1264%:%
%:%2484=1265%:%
%:%2485=1265%:%
%:%2486=1266%:%
%:%2487=1266%:%
%:%2488=1266%:%
%:%2489=1267%:%
%:%2490=1267%:%
%:%2491=1267%:%
%:%2492=1268%:%
%:%2493=1268%:%
%:%2494=1269%:%
%:%2495=1269%:%
%:%2496=1269%:%
%:%2497=1270%:%
%:%2498=1270%:%
%:%2499=1271%:%
%:%2500=1271%:%
%:%2501=1272%:%
%:%2502=1272%:%
%:%2503=1272%:%
%:%2504=1273%:%
%:%2505=1273%:%
%:%2506=1273%:%
%:%2507=1274%:%
%:%2508=1274%:%
%:%2509=1274%:%
%:%2510=1275%:%
%:%2511=1275%:%
%:%2512=1276%:%
%:%2513=1276%:%
%:%2514=1276%:%
%:%2515=1277%:%
%:%2516=1277%:%
%:%2517=1278%:%
%:%2518=1278%:%
%:%2519=1279%:%
%:%2520=1279%:%
%:%2521=1280%:%
%:%2522=1280%:%
%:%2523=1281%:%
%:%2524=1281%:%
%:%2525=1282%:%
%:%2526=1282%:%
%:%2527=1283%:%
%:%2528=1283%:%
%:%2529=1284%:%
%:%2530=1284%:%
%:%2531=1284%:%
%:%2532=1285%:%
%:%2533=1285%:%
%:%2534=1286%:%
%:%2535=1286%:%
%:%2536=1286%:%
%:%2537=1287%:%
%:%2538=1287%:%
%:%2539=1287%:%
%:%2540=1288%:%
%:%2541=1288%:%
%:%2542=1289%:%
%:%2543=1289%:%
%:%2544=1289%:%
%:%2545=1290%:%
%:%2546=1290%:%
%:%2547=1291%:%
%:%2548=1291%:%
%:%2549=1292%:%
%:%2550=1292%:%
%:%2551=1292%:%
%:%2552=1293%:%
%:%2553=1293%:%
%:%2554=1293%:%
%:%2555=1294%:%
%:%2556=1294%:%
%:%2557=1294%:%
%:%2558=1295%:%
%:%2559=1295%:%
%:%2560=1296%:%
%:%2561=1296%:%
%:%2562=1296%:%
%:%2563=1297%:%
%:%2564=1297%:%
%:%2565=1298%:%
%:%2566=1298%:%
%:%2567=1299%:%
%:%2568=1299%:%
%:%2569=1299%:%
%:%2570=1300%:%
%:%2571=1300%:%
%:%2572=1300%:%
%:%2573=1301%:%
%:%2574=1301%:%
%:%2575=1302%:%
%:%2576=1302%:%
%:%2577=1302%:%
%:%2578=1303%:%
%:%2579=1303%:%
%:%2580=1303%:%
%:%2581=1304%:%
%:%2582=1304%:%
%:%2583=1305%:%
%:%2584=1305%:%
%:%2585=1305%:%
%:%2586=1306%:%
%:%2587=1306%:%
%:%2588=1307%:%
%:%2589=1307%:%
%:%2590=1308%:%
%:%2591=1308%:%
%:%2592=1308%:%
%:%2593=1309%:%
%:%2594=1309%:%
%:%2595=1309%:%
%:%2596=1310%:%
%:%2597=1310%:%
%:%2598=1310%:%
%:%2599=1311%:%
%:%2600=1311%:%
%:%2601=1312%:%
%:%2602=1312%:%
%:%2603=1312%:%
%:%2604=1313%:%
%:%2605=1313%:%
%:%2606=1314%:%
%:%2607=1314%:%
%:%2608=1315%:%
%:%2609=1315%:%
%:%2610=1316%:%
%:%2611=1316%:%
%:%2612=1317%:%
%:%2613=1317%:%
%:%2614=1317%:%
%:%2615=1318%:%
%:%2616=1318%:%
%:%2617=1319%:%
%:%2618=1319%:%
%:%2619=1319%:%
%:%2620=1320%:%
%:%2621=1320%:%
%:%2622=1320%:%
%:%2623=1321%:%
%:%2624=1321%:%
%:%2625=1322%:%
%:%2626=1322%:%
%:%2627=1323%:%
%:%2628=1323%:%
%:%2629=1324%:%
%:%2630=1324%:%
%:%2631=1324%:%
%:%2632=1325%:%
%:%2633=1325%:%
%:%2634=1326%:%
%:%2635=1326%:%
%:%2636=1326%:%
%:%2637=1327%:%
%:%2647=1329%:%
%:%2647=1330%:%
%:%2648=1331%:%
%:%2649=1332%:%
%:%2650=1333%:%
%:%2651=1334%:%
%:%2652=1335%:%
%:%2653=1336%:%
%:%2654=1337%:%
%:%2655=1338%:%
%:%2656=1339%:%
%:%2657=1340%:%
%:%2658=1341%:%
%:%2659=1342%:%
%:%2660=1343%:%
%:%2661=1344%:%
%:%2662=1345%:%
%:%2663=1346%:%
%:%2664=1347%:%
%:%2665=1348%:%
%:%2666=1349%:%
%:%2667=1350%:%
%:%2668=1351%:%
%:%2669=1352%:%
%:%2670=1353%:%
%:%2671=1354%:%
%:%2672=1355%:%
%:%2673=1356%:%
%:%2674=1357%:%
%:%2675=1358%:%
%:%2676=1359%:%
%:%2677=1360%:%
%:%2678=1361%:%
%:%2679=1362%:%
%:%2680=1363%:%
%:%2681=1364%:%
%:%2682=1365%:%
%:%2683=1366%:%
%:%2684=1367%:%
%:%2685=1368%:%
%:%2686=1369%:%
%:%2687=1370%:%
%:%2688=1371%:%
%:%2689=1372%:%
%:%2690=1373%:%
%:%2691=1374%:%
%:%2692=1375%:%
%:%2693=1376%:%
%:%2694=1377%:%
%:%2695=1378%:%
%:%2696=1379%:%
%:%2697=1380%:%
%:%2698=1381%:%
%:%2699=1382%:%
%:%2700=1383%:%
%:%2701=1384%:%
%:%2702=1385%:%
%:%2703=1386%:%
%:%2704=1387%:%
%:%2705=1388%:%
%:%2706=1389%:%
%:%2707=1390%:%
%:%2708=1391%:%
%:%2709=1392%:%
%:%2710=1393%:%
%:%2711=1394%:%
%:%2712=1395%:%
%:%2713=1396%:%
%:%2714=1397%:%
%:%2715=1398%:%
%:%2716=1399%:%
%:%2717=1400%:%
%:%2718=1401%:%
%:%2719=1402%:%
%:%2720=1403%:%
%:%2721=1404%:%
%:%2722=1405%:%
%:%2723=1406%:%
%:%2724=1407%:%
%:%2725=1408%:%
%:%2726=1409%:%
%:%2727=1410%:%
%:%2728=1411%:%
%:%2729=1412%:%
%:%2730=1413%:%
%:%2731=1414%:%
%:%2732=1415%:%
%:%2733=1416%:%
%:%2734=1417%:%
%:%2735=1418%:%
%:%2736=1419%:%
%:%2737=1420%:%
%:%2738=1421%:%
%:%2739=1422%:%
%:%2740=1423%:%
%:%2741=1424%:%
%:%2742=1425%:%
%:%2743=1426%:%
%:%2744=1427%:%
%:%2745=1428%:%
%:%2746=1429%:%
%:%2747=1430%:%
%:%2748=1431%:%
%:%2749=1432%:%
%:%2750=1433%:%
%:%2751=1434%:%
%:%2752=1435%:%
%:%2753=1436%:%
%:%2754=1437%:%
%:%2755=1438%:%
%:%2756=1439%:%
%:%2757=1440%:%
%:%2758=1441%:%
%:%2759=1442%:%
%:%2760=1443%:%
%:%2761=1444%:%
%:%2762=1445%:%
%:%2763=1446%:%
%:%2764=1447%:%
%:%2765=1448%:%
%:%2766=1449%:%
%:%2767=1450%:%
%:%2768=1451%:%
%:%2769=1452%:%
%:%2770=1453%:%
%:%2771=1454%:%
%:%2772=1455%:%
%:%2773=1456%:%
%:%2774=1457%:%
%:%2775=1458%:%
%:%2776=1459%:%
%:%2777=1460%:%
%:%2778=1461%:%
%:%2779=1462%:%
%:%2780=1463%:%
%:%2781=1464%:%
%:%2782=1465%:%
%:%2783=1466%:%
%:%2784=1467%:%
%:%2785=1468%:%
%:%2786=1469%:%
%:%2787=1470%:%
%:%2788=1471%:%
%:%2789=1472%:%
%:%2790=1473%:%
%:%2791=1474%:%
%:%2792=1475%:%
%:%2793=1476%:%
%:%2794=1477%:%
%:%2795=1478%:%
%:%2796=1479%:%
%:%2797=1480%:%
%:%2798=1481%:%
%:%2799=1482%:%
%:%2800=1483%:%
%:%2801=1484%:%
%:%2802=1485%:%
%:%2803=1486%:%
%:%2804=1487%:%
%:%2805=1488%:%
%:%2806=1489%:%
%:%2807=1490%:%
%:%2808=1491%:%
%:%2809=1492%:%
%:%2810=1493%:%
%:%2811=1494%:%
%:%2812=1495%:%
%:%2813=1496%:%
%:%2814=1497%:%
%:%2815=1498%:%
%:%2816=1499%:%
%:%2817=1500%:%
%:%2818=1501%:%
%:%2819=1502%:%
%:%2820=1503%:%
%:%2821=1504%:%
%:%2822=1505%:%
%:%2823=1506%:%
%:%2824=1507%:%
%:%2825=1508%:%
%:%2826=1509%:%
%:%2827=1510%:%
%:%2828=1511%:%
%:%2829=1512%:%
%:%2830=1513%:%
%:%2831=1514%:%
%:%2832=1515%:%
%:%2834=1517%:%
%:%2835=1517%:%
%:%2836=1518%:%
%:%2838=1520%:%
%:%2839=1521%:%
%:%2841=1523%:%
%:%2842=1523%:%
%:%2843=1524%:%
%:%2844=1525%:%
%:%2845=1526%:%
%:%2846=1527%:%
%:%2853=1528%:%
%:%2854=1528%:%
%:%2855=1529%:%
%:%2856=1529%:%
%:%2857=1530%:%
%:%2858=1530%:%
%:%2859=1530%:%
%:%2860=1530%:%
%:%2861=1531%:%
%:%2862=1531%:%
%:%2863=1531%:%
%:%2864=1532%:%
%:%2865=1532%:%
%:%2866=1532%:%
%:%2867=1533%:%
%:%2868=1533%:%
%:%2869=1534%:%
%:%2870=1534%:%
%:%2871=1534%:%
%:%2872=1535%:%
%:%2882=1537%:%
%:%2883=1538%:%
%:%2884=1539%:%
%:%2885=1540%:%
%:%2886=1541%:%
%:%2887=1542%:%
%:%2888=1543%:%
%:%2889=1544%:%
%:%2890=1545%:%
%:%2891=1546%:%
%:%2892=1547%:%
%:%2893=1548%:%
%:%2894=1549%:%
%:%2895=1550%:%
%:%2896=1551%:%
%:%2897=1552%:%
%:%2898=1553%:%
%:%2899=1554%:%
%:%2900=1555%:%
%:%2901=1556%:%
%:%2902=1557%:%
%:%2903=1558%:%
%:%2904=1559%:%
%:%2905=1560%:%
%:%2906=1561%:%
%:%2907=1562%:%
%:%2908=1563%:%
%:%2909=1564%:%
%:%2910=1565%:%
%:%2911=1566%:%
%:%2912=1567%:%
%:%2913=1568%:%
%:%2914=1569%:%
%:%2915=1570%:%
%:%2916=1571%:%
%:%2917=1572%:%
%:%2918=1573%:%
%:%2919=1574%:%
%:%2919=1575%:%
%:%2920=1576%:%
%:%2921=1577%:%
%:%2922=1578%:%
%:%2923=1579%:%
%:%2924=1580%:%
%:%2925=1581%:%
%:%2926=1582%:%
%:%2927=1583%:%
%:%2928=1584%:%
%:%2929=1585%:%
%:%2930=1586%:%
%:%2931=1587%:%
%:%2932=1588%:%
%:%2933=1589%:%
%:%2934=1590%:%
%:%2935=1591%:%
%:%2936=1592%:%
%:%2937=1593%:%
%:%2938=1594%:%
%:%2939=1595%:%
%:%2940=1596%:%
%:%2941=1597%:%
%:%2942=1598%:%
%:%2943=1599%:%
%:%2944=1600%:%
%:%2945=1601%:%
%:%2946=1602%:%
%:%2947=1603%:%
%:%2948=1604%:%
%:%2949=1605%:%
%:%2950=1606%:%
%:%2951=1607%:%
%:%2952=1608%:%
%:%2953=1609%:%
%:%2954=1610%:%
%:%2955=1611%:%
%:%2956=1612%:%
%:%2957=1613%:%
%:%2958=1614%:%
%:%2959=1615%:%
%:%2960=1616%:%
%:%2961=1617%:%
%:%2962=1618%:%
%:%2963=1619%:%
%:%2964=1620%:%
%:%2965=1621%:%
%:%2966=1622%:%
%:%2967=1623%:%
%:%2968=1624%:%
%:%2969=1625%:%
%:%2970=1626%:%
%:%2971=1627%:%
%:%2972=1628%:%
%:%2973=1629%:%
%:%2975=1631%:%
%:%2976=1631%:%
%:%2983=1632%:%
%:%2984=1632%:%
%:%2985=1633%:%
%:%2986=1633%:%
%:%2987=1634%:%
%:%2988=1634%:%
%:%2989=1634%:%
%:%2990=1635%:%
%:%2991=1635%:%
%:%2992=1636%:%
%:%2993=1636%:%
%:%2994=1636%:%
%:%2995=1637%:%
%:%2996=1637%:%
%:%2997=1638%:%
%:%2998=1638%:%
%:%2999=1639%:%
%:%3000=1639%:%
%:%3001=1639%:%
%:%3002=1640%:%
%:%3003=1640%:%
%:%3004=1641%:%
%:%3005=1641%:%
%:%3006=1642%:%
%:%3007=1642%:%
%:%3008=1643%:%
%:%3009=1643%:%
%:%3010=1643%:%
%:%3011=1644%:%
%:%3012=1644%:%
%:%3013=1645%:%
%:%3014=1645%:%
%:%3015=1646%:%
%:%3025=1648%:%
%:%3027=1650%:%
%:%3028=1650%:%
%:%3029=1651%:%
%:%3030=1652%:%
%:%3037=1653%:%
%:%3038=1653%:%
%:%3039=1654%:%
%:%3040=1654%:%
%:%3041=1655%:%
%:%3042=1655%:%
%:%3043=1655%:%
%:%3044=1656%:%
%:%3045=1656%:%
%:%3046=1657%:%
%:%3047=1657%:%
%:%3048=1658%:%
%:%3049=1658%:%
%:%3050=1659%:%
%:%3051=1659%:%
%:%3052=1660%:%
%:%3053=1660%:%
%:%3054=1660%:%
%:%3055=1661%:%
%:%3056=1661%:%
%:%3057=1661%:%
%:%3058=1662%:%
%:%3059=1662%:%
%:%3060=1663%:%
%:%3061=1663%:%
%:%3062=1664%:%
%:%3063=1664%:%
%:%3064=1665%:%
%:%3065=1665%:%
%:%3066=1666%:%
%:%3067=1666%:%
%:%3068=1666%:%
%:%3069=1667%:%
%:%3070=1667%:%
%:%3071=1668%:%
%:%3072=1668%:%
%:%3073=1669%:%
%:%3074=1669%:%
%:%3075=1669%:%
%:%3076=1670%:%
%:%3077=1670%:%
%:%3078=1670%:%
%:%3079=1671%:%
%:%3089=1673%:%
%:%3090=1674%:%
%:%3091=1675%:%
%:%3092=1676%:%
%:%3094=1678%:%
%:%3095=1678%:%
%:%3102=1679%:%
%:%3103=1679%:%
%:%3104=1680%:%
%:%3105=1680%:%
%:%3106=1681%:%
%:%3107=1681%:%
%:%3108=1681%:%
%:%3109=1682%:%
%:%3110=1682%:%
%:%3111=1683%:%
%:%3112=1683%:%
%:%3113=1683%:%
%:%3114=1684%:%
%:%3115=1684%:%
%:%3116=1685%:%
%:%3117=1685%:%
%:%3118=1686%:%
%:%3119=1686%:%
%:%3120=1686%:%
%:%3121=1687%:%
%:%3122=1687%:%
%:%3123=1688%:%
%:%3124=1688%:%
%:%3125=1689%:%
%:%3126=1689%:%
%:%3127=1690%:%
%:%3128=1690%:%
%:%3129=1690%:%
%:%3130=1691%:%
%:%3131=1691%:%
%:%3132=1692%:%
%:%3133=1692%:%
%:%3134=1693%:%
%:%3135=1693%:%
%:%3136=1694%:%
%:%3137=1694%:%
%:%3138=1694%:%
%:%3139=1695%:%
%:%3140=1695%:%
%:%3141=1695%:%
%:%3142=1696%:%
%:%3143=1696%:%
%:%3144=1697%:%
%:%3145=1697%:%
%:%3146=1698%:%
%:%3147=1698%:%
%:%3148=1698%:%
%:%3149=1699%:%
%:%3159=1701%:%
%:%3160=1702%:%
%:%3162=1704%:%
%:%3163=1704%:%
%:%3164=1705%:%
%:%3165=1706%:%
%:%3172=1707%:%
%:%3173=1707%:%
%:%3174=1708%:%
%:%3175=1708%:%
%:%3176=1709%:%
%:%3177=1709%:%
%:%3178=1710%:%
%:%3179=1710%:%
%:%3180=1711%:%
%:%3181=1711%:%
%:%3182=1712%:%
%:%3183=1712%:%
%:%3184=1713%:%
%:%3185=1713%:%
%:%3186=1714%:%
%:%3187=1714%:%
%:%3188=1715%:%
%:%3189=1715%:%
%:%3190=1716%:%
%:%3191=1716%:%
%:%3192=1716%:%
%:%3193=1717%:%
%:%3194=1717%:%
%:%3195=1717%:%
%:%3196=1718%:%
%:%3197=1718%:%
%:%3198=1719%:%
%:%3199=1719%:%
%:%3200=1720%:%
%:%3201=1720%:%
%:%3202=1720%:%
%:%3203=1721%:%
%:%3204=1721%:%
%:%3205=1721%:%
%:%3206=1722%:%
%:%3207=1722%:%
%:%3208=1723%:%
%:%3209=1723%:%
%:%3210=1724%:%
%:%3211=1724%:%
%:%3212=1724%:%
%:%3213=1725%:%
%:%3214=1725%:%
%:%3215=1725%:%
%:%3216=1726%:%
%:%3217=1726%:%
%:%3218=1727%:%
%:%3219=1727%:%
%:%3220=1728%:%
%:%3221=1728%:%
%:%3222=1729%:%
%:%3223=1729%:%
%:%3224=1730%:%
%:%3225=1730%:%
%:%3226=1730%:%
%:%3227=1731%:%
%:%3228=1731%:%
%:%3229=1732%:%
%:%3230=1732%:%
%:%3231=1733%:%
%:%3232=1733%:%
%:%3233=1734%:%
%:%3234=1734%:%
%:%3235=1734%:%
%:%3236=1735%:%
%:%3237=1735%:%
%:%3238=1736%:%
%:%3248=1738%:%
%:%3249=1739%:%
%:%3250=1740%:%
%:%3251=1741%:%
%:%3252=1742%:%
%:%3253=1743%:%
%:%3254=1744%:%
%:%3256=1746%:%
%:%3257=1746%:%
%:%3258=1747%:%
%:%3259=1748%:%
%:%3260=1749%:%
%:%3261=1750%:%
%:%3262=1751%:%
%:%3269=1752%:%
%:%3270=1752%:%
%:%3271=1753%:%
%:%3272=1753%:%
%:%3273=1754%:%
%:%3274=1754%:%
%:%3275=1755%:%
%:%3276=1755%:%
%:%3277=1756%:%
%:%3278=1756%:%
%:%3279=1756%:%
%:%3280=1757%:%
%:%3281=1757%:%
%:%3282=1757%:%
%:%3283=1758%:%
%:%3284=1758%:%
%:%3285=1759%:%
%:%3286=1759%:%
%:%3287=1760%:%
%:%3288=1760%:%
%:%3289=1760%:%
%:%3290=1761%:%
%:%3291=1761%:%
%:%3292=1761%:%
%:%3293=1762%:%
%:%3294=1762%:%
%:%3295=1763%:%
%:%3296=1763%:%
%:%3297=1763%:%
%:%3298=1764%:%
%:%3299=1764%:%
%:%3300=1765%:%
%:%3301=1765%:%
%:%3302=1766%:%
%:%3303=1766%:%
%:%3304=1766%:%
%:%3305=1767%:%
%:%3315=1769%:%
%:%3316=1770%:%
%:%3317=1771%:%
%:%3318=1772%:%
%:%3320=1774%:%
%:%3321=1774%:%
%:%3322=1775%:%
%:%3323=1776%:%
%:%3324=1777%:%
%:%3327=1778%:%
%:%3331=1778%:%
%:%3332=1778%:%
%:%3333=1778%:%
%:%3342=1780%:%
%:%3343=1781%:%
%:%3344=1782%:%
%:%3346=1784%:%
%:%3347=1784%:%
%:%3348=1785%:%
%:%3349=1786%:%
%:%3350=1787%:%
%:%3351=1788%:%
%:%3352=1789%:%
%:%3353=1790%:%
%:%3360=1791%:%
%:%3361=1791%:%
%:%3362=1792%:%
%:%3363=1792%:%
%:%3364=1793%:%
%:%3365=1793%:%
%:%3366=1793%:%
%:%3367=1794%:%
%:%3368=1794%:%
%:%3369=1795%:%
%:%3370=1795%:%
%:%3371=1796%:%
%:%3372=1796%:%
%:%3373=1797%:%
%:%3374=1797%:%
%:%3375=1797%:%
%:%3376=1798%:%
%:%3377=1798%:%
%:%3378=1799%:%
%:%3379=1799%:%
%:%3380=1799%:%
%:%3381=1800%:%
%:%3382=1800%:%
%:%3383=1801%:%
%:%3384=1801%:%
%:%3385=1801%:%
%:%3386=1802%:%
%:%3387=1802%:%
%:%3388=1802%:%
%:%3389=1803%:%
%:%3390=1803%:%
%:%3391=1803%:%
%:%3392=1804%:%
%:%3393=1804%:%
%:%3394=1804%:%
%:%3395=1805%:%
%:%3396=1805%:%
%:%3397=1806%:%
%:%3398=1806%:%
%:%3399=1806%:%
%:%3400=1807%:%
%:%3401=1807%:%
%:%3402=1808%:%
%:%3403=1808%:%
%:%3404=1808%:%
%:%3405=1809%:%
%:%3406=1809%:%
%:%3407=1810%:%
%:%3408=1810%:%
%:%3409=1811%:%
%:%3410=1811%:%
%:%3411=1811%:%
%:%3412=1812%:%
%:%3413=1812%:%
%:%3414=1812%:%
%:%3415=1813%:%
%:%3416=1813%:%
%:%3417=1814%:%
%:%3418=1814%:%
%:%3419=1815%:%
%:%3420=1815%:%
%:%3421=1815%:%
%:%3422=1816%:%
%:%3423=1816%:%
%:%3424=1816%:%
%:%3425=1817%:%
%:%3426=1817%:%
%:%3427=1818%:%
%:%3428=1818%:%
%:%3429=1818%:%
%:%3430=1819%:%
%:%3431=1819%:%
%:%3432=1820%:%
%:%3433=1820%:%
%:%3434=1821%:%
%:%3435=1821%:%
%:%3436=1822%:%
%:%3446=1824%:%
%:%3447=1825%:%
%:%3448=1826%:%
%:%3450=1828%:%
%:%3451=1828%:%
%:%3452=1829%:%
%:%3453=1830%:%
%:%3454=1831%:%
%:%3455=1832%:%
%:%3456=1833%:%
%:%3457=1834%:%
%:%3464=1835%:%
%:%3465=1835%:%
%:%3466=1836%:%
%:%3467=1836%:%
%:%3468=1837%:%
%:%3469=1837%:%
%:%3470=1837%:%
%:%3471=1838%:%
%:%3472=1838%:%
%:%3473=1839%:%
%:%3474=1839%:%
%:%3475=1840%:%
%:%3476=1840%:%
%:%3477=1841%:%
%:%3478=1841%:%
%:%3479=1841%:%
%:%3480=1842%:%
%:%3481=1842%:%
%:%3482=1843%:%
%:%3483=1843%:%
%:%3484=1843%:%
%:%3485=1844%:%
%:%3486=1844%:%
%:%3487=1845%:%
%:%3488=1845%:%
%:%3489=1845%:%
%:%3490=1846%:%
%:%3491=1846%:%
%:%3492=1846%:%
%:%3493=1847%:%
%:%3494=1847%:%
%:%3495=1847%:%
%:%3496=1848%:%
%:%3497=1848%:%
%:%3498=1848%:%
%:%3499=1849%:%
%:%3500=1849%:%
%:%3501=1850%:%
%:%3502=1850%:%
%:%3503=1850%:%
%:%3504=1851%:%
%:%3505=1851%:%
%:%3506=1852%:%
%:%3507=1852%:%
%:%3508=1852%:%
%:%3509=1853%:%
%:%3510=1853%:%
%:%3511=1854%:%
%:%3512=1854%:%
%:%3513=1854%:%
%:%3514=1855%:%
%:%3515=1855%:%
%:%3516=1856%:%
%:%3517=1856%:%
%:%3518=1856%:%
%:%3519=1857%:%
%:%3520=1857%:%
%:%3521=1858%:%
%:%3522=1858%:%
%:%3523=1858%:%
%:%3524=1859%:%
%:%3525=1859%:%
%:%3526=1860%:%
%:%3527=1860%:%
%:%3528=1861%:%
%:%3529=1861%:%
%:%3530=1861%:%
%:%3531=1862%:%
%:%3532=1862%:%
%:%3533=1862%:%
%:%3534=1863%:%
%:%3535=1863%:%
%:%3536=1864%:%
%:%3537=1864%:%
%:%3538=1865%:%
%:%3539=1865%:%
%:%3540=1865%:%
%:%3541=1866%:%
%:%3542=1866%:%
%:%3543=1866%:%
%:%3544=1867%:%
%:%3545=1867%:%
%:%3546=1868%:%
%:%3547=1868%:%
%:%3548=1868%:%
%:%3549=1869%:%
%:%3550=1869%:%
%:%3551=1870%:%
%:%3552=1870%:%
%:%3553=1871%:%
%:%3554=1871%:%
%:%3555=1872%:%
%:%3565=1874%:%
%:%3566=1875%:%
%:%3567=1876%:%
%:%3569=1878%:%
%:%3570=1878%:%
%:%3571=1879%:%
%:%3572=1880%:%
%:%3573=1881%:%
%:%3574=1882%:%
%:%3575=1883%:%
%:%3576=1884%:%
%:%3583=1885%:%
%:%3584=1885%:%
%:%3585=1886%:%
%:%3586=1886%:%
%:%3587=1887%:%
%:%3588=1887%:%
%:%3589=1887%:%
%:%3590=1888%:%
%:%3591=1888%:%
%:%3592=1889%:%
%:%3593=1889%:%
%:%3594=1890%:%
%:%3595=1890%:%
%:%3596=1891%:%
%:%3597=1891%:%
%:%3598=1891%:%
%:%3599=1892%:%
%:%3600=1892%:%
%:%3601=1893%:%
%:%3602=1893%:%
%:%3603=1893%:%
%:%3604=1894%:%
%:%3605=1894%:%
%:%3606=1895%:%
%:%3607=1895%:%
%:%3608=1895%:%
%:%3609=1896%:%
%:%3610=1896%:%
%:%3611=1896%:%
%:%3612=1897%:%
%:%3613=1897%:%
%:%3614=1897%:%
%:%3615=1898%:%
%:%3616=1898%:%
%:%3617=1898%:%
%:%3618=1899%:%
%:%3619=1899%:%
%:%3620=1900%:%
%:%3621=1900%:%
%:%3622=1900%:%
%:%3623=1901%:%
%:%3624=1901%:%
%:%3625=1902%:%
%:%3626=1902%:%
%:%3627=1902%:%
%:%3628=1903%:%
%:%3629=1903%:%
%:%3630=1904%:%
%:%3631=1904%:%
%:%3632=1904%:%
%:%3633=1905%:%
%:%3634=1905%:%
%:%3635=1906%:%
%:%3636=1906%:%
%:%3637=1906%:%
%:%3638=1907%:%
%:%3639=1907%:%
%:%3640=1908%:%
%:%3641=1908%:%
%:%3642=1908%:%
%:%3643=1909%:%
%:%3644=1909%:%
%:%3645=1910%:%
%:%3646=1910%:%
%:%3647=1911%:%
%:%3648=1911%:%
%:%3649=1911%:%
%:%3650=1912%:%
%:%3651=1912%:%
%:%3652=1912%:%
%:%3653=1913%:%
%:%3654=1913%:%
%:%3655=1914%:%
%:%3656=1914%:%
%:%3657=1915%:%
%:%3658=1915%:%
%:%3659=1915%:%
%:%3660=1916%:%
%:%3661=1916%:%
%:%3662=1916%:%
%:%3663=1917%:%
%:%3664=1917%:%
%:%3665=1918%:%
%:%3666=1918%:%
%:%3667=1918%:%
%:%3668=1919%:%
%:%3669=1919%:%
%:%3670=1920%:%
%:%3671=1920%:%
%:%3672=1921%:%
%:%3673=1921%:%
%:%3674=1922%:%
%:%3684=1924%:%
%:%3685=1925%:%
%:%3687=1927%:%
%:%3688=1927%:%
%:%3689=1928%:%
%:%3690=1929%:%
%:%3691=1930%:%
%:%3692=1931%:%
%:%3693=1932%:%
%:%3694=1933%:%
%:%3701=1934%:%
%:%3702=1934%:%
%:%3703=1935%:%
%:%3704=1935%:%
%:%3705=1936%:%
%:%3706=1936%:%
%:%3707=1936%:%
%:%3708=1937%:%
%:%3709=1937%:%
%:%3710=1938%:%
%:%3711=1938%:%
%:%3712=1939%:%
%:%3713=1939%:%
%:%3714=1940%:%
%:%3715=1940%:%
%:%3716=1940%:%
%:%3717=1941%:%
%:%3718=1941%:%
%:%3719=1942%:%
%:%3720=1942%:%
%:%3721=1942%:%
%:%3722=1943%:%
%:%3723=1943%:%
%:%3724=1944%:%
%:%3725=1944%:%
%:%3726=1944%:%
%:%3727=1945%:%
%:%3728=1945%:%
%:%3729=1945%:%
%:%3730=1946%:%
%:%3731=1946%:%
%:%3732=1946%:%
%:%3733=1947%:%
%:%3734=1947%:%
%:%3735=1947%:%
%:%3736=1948%:%
%:%3737=1948%:%
%:%3738=1949%:%
%:%3739=1949%:%
%:%3740=1949%:%
%:%3741=1950%:%
%:%3742=1950%:%
%:%3743=1951%:%
%:%3744=1951%:%
%:%3745=1951%:%
%:%3746=1952%:%
%:%3747=1952%:%
%:%3748=1953%:%
%:%3749=1953%:%
%:%3750=1953%:%
%:%3751=1954%:%
%:%3752=1954%:%
%:%3753=1955%:%
%:%3754=1955%:%
%:%3755=1956%:%
%:%3756=1956%:%
%:%3757=1956%:%
%:%3758=1957%:%
%:%3759=1957%:%
%:%3760=1957%:%
%:%3761=1958%:%
%:%3762=1958%:%
%:%3763=1959%:%
%:%3764=1959%:%
%:%3765=1959%:%
%:%3766=1960%:%
%:%3767=1960%:%
%:%3768=1961%:%
%:%3769=1961%:%
%:%3770=1962%:%
%:%3771=1962%:%
%:%3772=1963%:%
%:%3782=1965%:%
%:%3783=1966%:%
%:%3784=1967%:%
%:%3786=1969%:%
%:%3787=1969%:%
%:%3788=1970%:%
%:%3789=1971%:%
%:%3790=1972%:%
%:%3791=1973%:%
%:%3792=1974%:%
%:%3793=1975%:%
%:%3800=1976%:%
%:%3801=1976%:%
%:%3802=1977%:%
%:%3803=1977%:%
%:%3804=1978%:%
%:%3805=1978%:%
%:%3806=1979%:%
%:%3807=1979%:%
%:%3810=1982%:%
%:%3811=1983%:%
%:%3812=1983%:%
%:%3813=1983%:%
%:%3814=1984%:%
%:%3815=1984%:%
%:%3816=1985%:%
%:%3817=1985%:%
%:%3818=1986%:%
%:%3819=1986%:%
%:%3820=1987%:%
%:%3821=1987%:%
%:%3822=1988%:%
%:%3823=1988%:%
%:%3824=1988%:%
%:%3825=1989%:%
%:%3826=1989%:%
%:%3827=1990%:%
%:%3828=1990%:%
%:%3830=1992%:%
%:%3831=1993%:%
%:%3832=1993%:%
%:%3833=1994%:%
%:%3834=1994%:%
%:%3835=1995%:%
%:%3836=1995%:%
%:%3837=1996%:%
%:%3838=1996%:%
%:%3839=1997%:%
%:%3840=1997%:%
%:%3841=1997%:%
%:%3842=1998%:%
%:%3843=1998%:%
%:%3844=1999%:%
%:%3845=1999%:%
%:%3846=1999%:%
%:%3847=2000%:%
%:%3848=2000%:%
%:%3849=2001%:%
%:%3850=2001%:%
%:%3851=2001%:%
%:%3852=2002%:%
%:%3853=2002%:%
%:%3854=2003%:%
%:%3855=2003%:%
%:%3856=2003%:%
%:%3857=2004%:%
%:%3858=2004%:%
%:%3859=2005%:%
%:%3860=2005%:%
%:%3861=2005%:%
%:%3862=2006%:%
%:%3863=2006%:%
%:%3864=2007%:%
%:%3865=2007%:%
%:%3866=2008%:%
%:%3867=2008%:%
%:%3868=2009%:%
%:%3869=2009%:%
%:%3870=2010%:%
%:%3871=2011%:%
%:%3872=2011%:%
%:%3873=2012%:%
%:%3874=2012%:%
%:%3875=2013%:%
%:%3876=2013%:%
%:%3877=2014%:%
%:%3878=2014%:%
%:%3879=2015%:%
%:%3880=2015%:%
%:%3881=2015%:%
%:%3882=2016%:%
%:%3883=2016%:%
%:%3884=2017%:%
%:%3885=2017%:%
%:%3886=2017%:%
%:%3887=2018%:%
%:%3888=2018%:%
%:%3889=2019%:%
%:%3890=2019%:%
%:%3891=2019%:%
%:%3892=2020%:%
%:%3893=2020%:%
%:%3894=2021%:%
%:%3895=2021%:%
%:%3896=2021%:%
%:%3897=2022%:%
%:%3898=2022%:%
%:%3899=2023%:%
%:%3900=2023%:%
%:%3901=2024%:%
%:%3902=2024%:%
%:%3903=2025%:%
%:%3904=2025%:%
%:%3905=2026%:%
%:%3906=2026%:%
%:%3907=2026%:%
%:%3908=2027%:%
%:%3909=2027%:%
%:%3910=2028%:%
%:%3911=2028%:%
%:%3912=2028%:%
%:%3913=2029%:%
%:%3914=2029%:%
%:%3915=2030%:%
%:%3916=2030%:%
%:%3917=2031%:%
%:%3918=2031%:%
%:%3919=2031%:%
%:%3920=2032%:%
%:%3921=2032%:%
%:%3922=2033%:%
%:%3923=2033%:%
%:%3924=2033%:%
%:%3925=2034%:%
%:%3926=2034%:%
%:%3927=2035%:%
%:%3928=2035%:%
%:%3929=2036%:%
%:%3930=2036%:%
%:%3931=2037%:%
%:%3932=2037%:%
%:%3933=2038%:%
%:%3934=2038%:%
%:%3935=2039%:%
%:%3945=2041%:%
%:%3946=2042%:%
%:%3947=2043%:%
%:%3949=2045%:%
%:%3950=2045%:%
%:%3951=2046%:%
%:%3952=2047%:%
%:%3959=2048%:%
%:%3960=2048%:%
%:%3961=2049%:%
%:%3962=2049%:%
%:%3963=2050%:%
%:%3964=2050%:%
%:%3965=2051%:%
%:%3966=2051%:%
%:%3967=2052%:%
%:%3968=2052%:%
%:%3969=2053%:%
%:%3970=2053%:%
%:%3971=2054%:%
%:%3972=2054%:%
%:%3973=2055%:%
%:%3974=2055%:%
%:%3975=2056%:%
%:%3976=2056%:%
%:%3977=2057%:%
%:%3978=2057%:%
%:%3979=2058%:%
%:%3980=2058%:%
%:%3981=2059%:%
%:%3982=2059%:%
%:%3983=2060%:%
%:%3984=2060%:%
%:%3985=2061%:%
%:%3986=2061%:%
%:%3987=2061%:%
%:%3988=2062%:%
%:%3989=2062%:%
%:%3990=2063%:%
%:%3991=2063%:%
%:%3992=2064%:%
%:%3993=2064%:%
%:%3994=2065%:%
%:%3995=2065%:%
%:%3996=2066%:%
%:%3997=2066%:%
%:%3998=2067%:%
%:%3999=2067%:%
%:%4000=2067%:%
%:%4001=2068%:%
%:%4002=2068%:%
%:%4003=2069%:%
%:%4004=2069%:%
%:%4005=2069%:%
%:%4006=2070%:%
%:%4007=2070%:%
%:%4008=2071%:%
%:%4009=2071%:%
%:%4010=2071%:%
%:%4011=2072%:%
%:%4012=2072%:%
%:%4013=2073%:%
%:%4014=2073%:%
%:%4015=2073%:%
%:%4016=2074%:%
%:%4017=2074%:%
%:%4018=2075%:%
%:%4019=2075%:%
%:%4020=2075%:%
%:%4021=2076%:%
%:%4022=2076%:%
%:%4023=2077%:%
%:%4024=2077%:%
%:%4025=2078%:%
%:%4026=2078%:%
%:%4027=2079%:%
%:%4028=2079%:%
%:%4029=2079%:%
%:%4030=2080%:%
%:%4031=2080%:%
%:%4032=2081%:%
%:%4033=2081%:%
%:%4034=2082%:%
%:%4035=2082%:%
%:%4036=2082%:%
%:%4037=2083%:%
%:%4038=2083%:%
%:%4039=2084%:%
%:%4040=2084%:%
%:%4041=2085%:%
%:%4042=2085%:%
%:%4043=2086%:%
%:%4044=2086%:%
%:%4045=2087%:%
%:%4046=2087%:%
%:%4047=2088%:%
%:%4048=2088%:%
%:%4049=2088%:%
%:%4050=2089%:%
%:%4051=2089%:%
%:%4052=2090%:%
%:%4053=2090%:%
%:%4054=2091%:%
%:%4055=2091%:%
%:%4056=2091%:%
%:%4057=2092%:%
%:%4058=2092%:%
%:%4059=2093%:%
%:%4060=2093%:%
%:%4061=2094%:%
%:%4062=2094%:%
%:%4063=2095%:%
%:%4064=2095%:%
%:%4065=2096%:%
%:%4066=2096%:%
%:%4067=2097%:%
%:%4068=2097%:%
%:%4069=2097%:%
%:%4070=2098%:%
%:%4071=2098%:%
%:%4072=2099%:%
%:%4073=2099%:%
%:%4074=2100%:%
%:%4075=2100%:%
%:%4076=2100%:%
%:%4077=2101%:%
%:%4078=2101%:%
%:%4079=2102%:%
%:%4080=2102%:%
%:%4081=2103%:%
%:%4082=2103%:%
%:%4083=2103%:%
%:%4084=2104%:%
%:%4085=2104%:%
%:%4086=2105%:%
%:%4087=2105%:%
%:%4088=2106%:%
%:%4089=2106%:%
%:%4090=2106%:%
%:%4091=2107%:%
%:%4092=2107%:%
%:%4093=2108%:%
%:%4094=2108%:%
%:%4095=2109%:%
%:%4096=2109%:%
%:%4097=2110%:%
%:%4098=2110%:%
%:%4099=2111%:%
%:%4100=2111%:%
%:%4101=2112%:%
%:%4102=2112%:%
%:%4103=2113%:%
%:%4104=2113%:%
%:%4105=2114%:%
%:%4115=2116%:%
%:%4116=2117%:%
%:%4117=2118%:%
%:%4118=2119%:%
%:%4119=2120%:%
%:%4120=2121%:%
%:%4121=2122%:%
%:%4122=2123%:%
%:%4123=2124%:%
%:%4125=2126%:%
%:%4126=2126%:%
%:%4127=2127%:%
%:%4128=2128%:%
%:%4129=2129%:%
%:%4136=2130%:%
%:%4137=2130%:%
%:%4138=2131%:%
%:%4139=2131%:%
%:%4140=2132%:%
%:%4141=2132%:%
%:%4142=2132%:%
%:%4143=2133%:%
%:%4144=2133%:%
%:%4145=2133%:%
%:%4146=2134%:%
%:%4147=2134%:%
%:%4148=2134%:%
%:%4149=2135%:%
%:%4150=2135%:%
%:%4151=2135%:%
%:%4152=2136%:%
%:%4153=2136%:%
%:%4154=2137%:%
%:%4155=2137%:%
%:%4156=2138%:%
%:%4157=2138%:%
%:%4158=2139%:%
%:%4159=2139%:%
%:%4160=2140%:%
%:%4161=2140%:%
%:%4162=2141%:%
%:%4163=2141%:%
%:%4164=2141%:%
%:%4165=2142%:%
%:%4166=2142%:%
%:%4167=2143%:%
%:%4168=2143%:%
%:%4169=2144%:%
%:%4170=2144%:%
%:%4171=2145%:%
%:%4172=2145%:%
%:%4173=2146%:%
%:%4174=2146%:%
%:%4175=2147%:%
%:%4185=2149%:%
%:%4186=2150%:%
%:%4187=2151%:%
%:%4189=2153%:%
%:%4190=2153%:%
%:%4191=2154%:%
%:%4192=2155%:%
%:%4193=2156%:%
%:%4200=2157%:%
%:%4201=2157%:%
%:%4202=2158%:%
%:%4203=2158%:%
%:%4204=2159%:%
%:%4205=2159%:%
%:%4206=2159%:%
%:%4207=2159%:%
%:%4208=2160%:%
%:%4209=2160%:%
%:%4210=2161%:%
%:%4211=2161%:%
%:%4212=2161%:%
%:%4213=2161%:%
%:%4214=2162%:%
%:%4215=2162%:%
%:%4216=2162%:%
%:%4217=2163%:%
%:%4218=2163%:%
%:%4219=2164%:%
%:%4220=2164%:%
%:%4221=2164%:%
%:%4222=2165%:%
%:%4223=2165%:%
%:%4224=2166%:%
%:%4225=2166%:%
%:%4226=2167%:%
%:%4227=2167%:%
%:%4228=2167%:%
%:%4229=2168%:%
%:%4230=2168%:%
%:%4231=2169%:%
%:%4232=2169%:%
%:%4233=2169%:%
%:%4234=2170%:%
%:%4235=2170%:%
%:%4236=2171%:%
%:%4237=2171%:%
%:%4238=2171%:%
%:%4239=2172%:%
%:%4240=2172%:%
%:%4241=2173%:%
%:%4242=2173%:%
%:%4243=2173%:%
%:%4244=2174%:%
%:%4245=2174%:%
%:%4246=2175%:%
%:%4247=2175%:%
%:%4248=2175%:%
%:%4249=2176%:%
%:%4250=2176%:%
%:%4251=2177%:%
%:%4252=2177%:%
%:%4253=2177%:%
%:%4254=2178%:%
%:%4255=2178%:%
%:%4256=2179%:%
%:%4257=2179%:%
%:%4258=2179%:%
%:%4259=2180%:%
%:%4260=2180%:%
%:%4261=2181%:%
%:%4262=2181%:%
%:%4263=2181%:%
%:%4264=2182%:%
%:%4265=2182%:%
%:%4266=2183%:%
%:%4267=2183%:%
%:%4268=2184%:%
%:%4269=2184%:%
%:%4270=2185%:%
%:%4271=2185%:%
%:%4272=2185%:%
%:%4273=2186%:%
%:%4274=2186%:%
%:%4275=2186%:%
%:%4276=2187%:%
%:%4277=2187%:%
%:%4278=2188%:%
%:%4279=2188%:%
%:%4280=2188%:%
%:%4281=2189%:%
%:%4282=2189%:%
%:%4283=2190%:%
%:%4284=2190%:%
%:%4285=2190%:%
%:%4286=2191%:%
%:%4287=2191%:%
%:%4288=2192%:%
%:%4289=2192%:%
%:%4290=2193%:%
%:%4291=2193%:%
%:%4292=2193%:%
%:%4293=2194%:%
%:%4294=2194%:%
%:%4295=2194%:%
%:%4296=2195%:%
%:%4297=2195%:%
%:%4298=2196%:%
%:%4299=2196%:%
%:%4300=2197%:%
%:%4301=2197%:%
%:%4302=2197%:%
%:%4303=2198%:%
%:%4304=2198%:%
%:%4305=2198%:%
%:%4306=2199%:%
%:%4307=2199%:%
%:%4308=2200%:%
%:%4309=2200%:%
%:%4310=2200%:%
%:%4311=2201%:%
%:%4312=2201%:%
%:%4313=2202%:%
%:%4314=2202%:%
%:%4315=2202%:%
%:%4316=2203%:%
%:%4317=2203%:%
%:%4318=2204%:%
%:%4328=2206%:%
%:%4329=2207%:%
%:%4330=2208%:%
%:%4331=2209%:%
%:%4332=2210%:%
%:%4333=2211%:%
%:%4334=2212%:%
%:%4335=2213%:%
%:%4336=2214%:%
%:%4338=2216%:%
%:%4339=2216%:%
%:%4340=2217%:%
%:%4341=2218%:%
%:%4342=2219%:%
%:%4343=2220%:%
%:%4344=2221%:%
%:%4345=2222%:%
%:%4352=2223%:%
%:%4353=2223%:%
%:%4354=2224%:%
%:%4355=2224%:%
%:%4356=2225%:%
%:%4357=2225%:%
%:%4358=2225%:%
%:%4359=2226%:%
%:%4360=2226%:%
%:%4361=2227%:%
%:%4362=2227%:%
%:%4363=2228%:%
%:%4364=2228%:%
%:%4365=2229%:%
%:%4366=2229%:%
%:%4367=2229%:%
%:%4368=2230%:%
%:%4369=2230%:%
%:%4370=2231%:%
%:%4371=2231%:%
%:%4372=2231%:%
%:%4373=2232%:%
%:%4374=2232%:%
%:%4375=2233%:%
%:%4376=2233%:%
%:%4377=2233%:%
%:%4378=2234%:%
%:%4379=2234%:%
%:%4380=2234%:%
%:%4381=2235%:%
%:%4382=2235%:%
%:%4383=2235%:%
%:%4384=2236%:%
%:%4385=2236%:%
%:%4386=2236%:%
%:%4387=2237%:%
%:%4388=2237%:%
%:%4389=2238%:%
%:%4390=2238%:%
%:%4391=2239%:%
%:%4392=2239%:%
%:%4393=2240%:%
%:%4394=2240%:%
%:%4395=2241%:%
%:%4396=2241%:%
%:%4397=2241%:%
%:%4398=2242%:%
%:%4399=2242%:%
%:%4400=2243%:%
%:%4401=2243%:%
%:%4402=2243%:%
%:%4403=2244%:%
%:%4404=2244%:%
%:%4405=2244%:%
%:%4406=2245%:%
%:%4407=2245%:%
%:%4408=2245%:%
%:%4409=2246%:%
%:%4410=2246%:%
%:%4411=2247%:%
%:%4412=2247%:%
%:%4413=2247%:%
%:%4414=2248%:%
%:%4415=2248%:%
%:%4416=2249%:%
%:%4417=2249%:%
%:%4418=2249%:%
%:%4419=2250%:%
%:%4420=2250%:%
%:%4421=2251%:%
%:%4422=2251%:%
%:%4423=2252%:%
%:%4424=2252%:%
%:%4425=2253%:%
%:%4426=2253%:%
%:%4427=2254%:%
%:%4428=2254%:%
%:%4429=2255%:%
%:%4430=2255%:%
%:%4431=2255%:%
%:%4432=2256%:%
%:%4433=2256%:%
%:%4434=2257%:%
%:%4435=2257%:%
%:%4436=2257%:%
%:%4437=2258%:%
%:%4438=2258%:%
%:%4439=2258%:%
%:%4440=2259%:%
%:%4441=2259%:%
%:%4442=2259%:%
%:%4443=2260%:%
%:%4444=2260%:%
%:%4445=2261%:%
%:%4446=2261%:%
%:%4447=2261%:%
%:%4448=2262%:%
%:%4449=2262%:%
%:%4450=2263%:%
%:%4451=2263%:%
%:%4452=2263%:%
%:%4453=2264%:%
%:%4454=2264%:%
%:%4455=2265%:%
%:%4456=2265%:%
%:%4457=2266%:%
%:%4458=2266%:%
%:%4459=2267%:%
%:%4460=2267%:%
%:%4461=2268%:%
%:%4471=2270%:%
%:%4472=2271%:%
%:%4473=2272%:%
%:%4475=2274%:%
%:%4476=2274%:%
%:%4477=2275%:%
%:%4478=2276%:%
%:%4479=2277%:%
%:%4480=2278%:%
%:%4481=2279%:%
%:%4482=2280%:%
%:%4489=2281%:%
%:%4490=2281%:%
%:%4491=2282%:%
%:%4492=2282%:%
%:%4493=2283%:%
%:%4494=2283%:%
%:%4495=2283%:%
%:%4496=2284%:%
%:%4497=2284%:%
%:%4498=2285%:%
%:%4499=2285%:%
%:%4500=2286%:%
%:%4501=2286%:%
%:%4502=2287%:%
%:%4503=2287%:%
%:%4504=2287%:%
%:%4505=2288%:%
%:%4506=2288%:%
%:%4507=2289%:%
%:%4508=2289%:%
%:%4509=2289%:%
%:%4510=2290%:%
%:%4511=2290%:%
%:%4512=2291%:%
%:%4513=2291%:%
%:%4514=2291%:%
%:%4515=2292%:%
%:%4516=2292%:%
%:%4517=2292%:%
%:%4518=2293%:%
%:%4519=2293%:%
%:%4520=2293%:%
%:%4521=2294%:%
%:%4522=2294%:%
%:%4523=2294%:%
%:%4524=2295%:%
%:%4525=2295%:%
%:%4526=2296%:%
%:%4527=2296%:%
%:%4528=2296%:%
%:%4529=2297%:%
%:%4530=2297%:%
%:%4531=2298%:%
%:%4532=2298%:%
%:%4533=2298%:%
%:%4534=2299%:%
%:%4535=2299%:%
%:%4536=2300%:%
%:%4537=2300%:%
%:%4538=2301%:%
%:%4539=2301%:%
%:%4540=2302%:%
%:%4541=2302%:%
%:%4542=2303%:%
%:%4543=2303%:%
%:%4544=2303%:%
%:%4545=2304%:%
%:%4546=2304%:%
%:%4547=2305%:%
%:%4548=2305%:%
%:%4549=2305%:%
%:%4550=2306%:%
%:%4551=2306%:%
%:%4552=2307%:%
%:%4553=2307%:%
%:%4554=2307%:%
%:%4555=2308%:%
%:%4556=2308%:%
%:%4557=2308%:%
%:%4558=2309%:%
%:%4559=2309%:%
%:%4560=2309%:%
%:%4561=2310%:%
%:%4562=2310%:%
%:%4563=2311%:%
%:%4564=2311%:%
%:%4565=2311%:%
%:%4566=2312%:%
%:%4567=2312%:%
%:%4568=2313%:%
%:%4569=2313%:%
%:%4570=2313%:%
%:%4571=2314%:%
%:%4572=2314%:%
%:%4573=2315%:%
%:%4574=2315%:%
%:%4575=2316%:%
%:%4576=2316%:%
%:%4577=2317%:%
%:%4578=2317%:%
%:%4579=2318%:%
%:%4580=2318%:%
%:%4581=2319%:%
%:%4582=2319%:%
%:%4583=2319%:%
%:%4584=2320%:%
%:%4585=2320%:%
%:%4586=2321%:%
%:%4587=2321%:%
%:%4588=2321%:%
%:%4589=2322%:%
%:%4590=2322%:%
%:%4591=2322%:%
%:%4592=2323%:%
%:%4593=2323%:%
%:%4594=2323%:%
%:%4595=2324%:%
%:%4596=2324%:%
%:%4597=2325%:%
%:%4598=2325%:%
%:%4599=2325%:%
%:%4600=2326%:%
%:%4601=2326%:%
%:%4602=2327%:%
%:%4603=2327%:%
%:%4604=2327%:%
%:%4605=2328%:%
%:%4606=2328%:%
%:%4607=2329%:%
%:%4608=2329%:%
%:%4609=2330%:%
%:%4610=2330%:%
%:%4611=2331%:%
%:%4612=2331%:%
%:%4613=2332%:%
%:%4623=2334%:%
%:%4624=2335%:%
%:%4625=2336%:%
%:%4627=2338%:%
%:%4628=2338%:%
%:%4629=2339%:%
%:%4630=2340%:%
%:%4631=2341%:%
%:%4632=2342%:%
%:%4633=2343%:%
%:%4634=2344%:%
%:%4641=2345%:%
%:%4642=2345%:%
%:%4643=2346%:%
%:%4644=2346%:%
%:%4645=2347%:%
%:%4646=2347%:%
%:%4647=2347%:%
%:%4648=2348%:%
%:%4649=2348%:%
%:%4650=2349%:%
%:%4651=2349%:%
%:%4652=2350%:%
%:%4653=2350%:%
%:%4654=2351%:%
%:%4655=2351%:%
%:%4656=2351%:%
%:%4657=2352%:%
%:%4658=2352%:%
%:%4659=2353%:%
%:%4660=2353%:%
%:%4661=2353%:%
%:%4662=2354%:%
%:%4663=2354%:%
%:%4664=2355%:%
%:%4665=2355%:%
%:%4666=2355%:%
%:%4667=2356%:%
%:%4668=2356%:%
%:%4669=2356%:%
%:%4670=2357%:%
%:%4671=2357%:%
%:%4672=2357%:%
%:%4673=2358%:%
%:%4674=2358%:%
%:%4675=2358%:%
%:%4676=2359%:%
%:%4677=2359%:%
%:%4678=2360%:%
%:%4679=2360%:%
%:%4680=2360%:%
%:%4681=2361%:%
%:%4682=2361%:%
%:%4683=2362%:%
%:%4684=2362%:%
%:%4685=2362%:%
%:%4686=2363%:%
%:%4687=2363%:%
%:%4688=2364%:%
%:%4689=2364%:%
%:%4690=2365%:%
%:%4691=2365%:%
%:%4692=2366%:%
%:%4693=2366%:%
%:%4694=2367%:%
%:%4695=2367%:%
%:%4696=2367%:%
%:%4697=2368%:%
%:%4698=2368%:%
%:%4699=2369%:%
%:%4700=2369%:%
%:%4701=2369%:%
%:%4702=2370%:%
%:%4703=2370%:%
%:%4704=2371%:%
%:%4705=2371%:%
%:%4706=2371%:%
%:%4707=2372%:%
%:%4708=2372%:%
%:%4709=2372%:%
%:%4710=2373%:%
%:%4711=2373%:%
%:%4712=2373%:%
%:%4713=2374%:%
%:%4714=2374%:%
%:%4715=2375%:%
%:%4716=2375%:%
%:%4717=2375%:%
%:%4718=2376%:%
%:%4719=2376%:%
%:%4720=2377%:%
%:%4721=2377%:%
%:%4722=2377%:%
%:%4723=2378%:%
%:%4724=2378%:%
%:%4725=2379%:%
%:%4726=2379%:%
%:%4727=2380%:%
%:%4728=2380%:%
%:%4729=2381%:%
%:%4730=2381%:%
%:%4731=2382%:%
%:%4732=2382%:%
%:%4733=2383%:%
%:%4734=2383%:%
%:%4735=2383%:%
%:%4736=2384%:%
%:%4737=2384%:%
%:%4738=2385%:%
%:%4739=2385%:%
%:%4740=2385%:%
%:%4741=2386%:%
%:%4742=2386%:%
%:%4743=2387%:%
%:%4744=2387%:%
%:%4745=2387%:%
%:%4746=2388%:%
%:%4747=2388%:%
%:%4748=2388%:%
%:%4749=2389%:%
%:%4750=2389%:%
%:%4751=2389%:%
%:%4752=2390%:%
%:%4753=2390%:%
%:%4754=2391%:%
%:%4755=2391%:%
%:%4756=2391%:%
%:%4757=2392%:%
%:%4758=2392%:%
%:%4759=2393%:%
%:%4760=2393%:%
%:%4761=2393%:%
%:%4762=2394%:%
%:%4763=2394%:%
%:%4764=2395%:%
%:%4765=2395%:%
%:%4766=2396%:%
%:%4767=2396%:%
%:%4768=2397%:%
%:%4769=2397%:%
%:%4770=2398%:%
%:%4780=2400%:%
%:%4781=2401%:%
%:%4782=2402%:%
%:%4784=2404%:%
%:%4785=2404%:%
%:%4786=2405%:%
%:%4787=2406%:%
%:%4788=2407%:%
%:%4789=2408%:%
%:%4790=2409%:%
%:%4791=2410%:%
%:%4798=2411%:%
%:%4799=2411%:%
%:%4800=2412%:%
%:%4801=2412%:%
%:%4802=2413%:%
%:%4803=2413%:%
%:%4804=2413%:%
%:%4805=2414%:%
%:%4806=2414%:%
%:%4807=2415%:%
%:%4808=2415%:%
%:%4809=2416%:%
%:%4810=2416%:%
%:%4811=2417%:%
%:%4812=2417%:%
%:%4813=2417%:%
%:%4814=2418%:%
%:%4815=2418%:%
%:%4816=2419%:%
%:%4817=2419%:%
%:%4818=2419%:%
%:%4819=2420%:%
%:%4820=2420%:%
%:%4821=2421%:%
%:%4822=2421%:%
%:%4823=2421%:%
%:%4824=2422%:%
%:%4825=2422%:%
%:%4826=2422%:%
%:%4827=2423%:%
%:%4828=2423%:%
%:%4829=2423%:%
%:%4830=2424%:%
%:%4831=2424%:%
%:%4832=2424%:%
%:%4833=2425%:%
%:%4834=2425%:%
%:%4835=2426%:%
%:%4836=2426%:%
%:%4837=2426%:%
%:%4838=2427%:%
%:%4839=2427%:%
%:%4840=2428%:%
%:%4841=2428%:%
%:%4842=2428%:%
%:%4843=2429%:%
%:%4844=2429%:%
%:%4845=2430%:%
%:%4846=2430%:%
%:%4847=2430%:%
%:%4848=2431%:%
%:%4849=2431%:%
%:%4850=2432%:%
%:%4851=2432%:%
%:%4852=2432%:%
%:%4853=2433%:%
%:%4854=2433%:%
%:%4855=2433%:%
%:%4856=2434%:%
%:%4857=2434%:%
%:%4858=2434%:%
%:%4859=2435%:%
%:%4860=2435%:%
%:%4861=2436%:%
%:%4862=2436%:%
%:%4863=2436%:%
%:%4864=2437%:%
%:%4865=2437%:%
%:%4866=2438%:%
%:%4867=2438%:%
%:%4868=2439%:%
%:%4869=2439%:%
%:%4870=2440%:%
%:%4880=2442%:%
%:%4881=2443%:%
%:%4882=2444%:%
%:%4883=2445%:%
%:%4885=2447%:%
%:%4886=2447%:%
%:%4887=2448%:%
%:%4888=2449%:%
%:%4889=2450%:%
%:%4890=2451%:%
%:%4891=2452%:%
%:%4892=2453%:%
%:%4899=2454%:%
%:%4900=2454%:%
%:%4901=2455%:%
%:%4902=2455%:%
%:%4905=2458%:%
%:%4906=2459%:%
%:%4907=2459%:%
%:%4908=2459%:%
%:%4909=2460%:%
%:%4910=2460%:%
%:%4911=2461%:%
%:%4912=2461%:%
%:%4913=2462%:%
%:%4914=2462%:%
%:%4915=2463%:%
%:%4916=2463%:%
%:%4917=2464%:%
%:%4918=2464%:%
%:%4919=2464%:%
%:%4920=2465%:%
%:%4921=2465%:%
%:%4922=2466%:%
%:%4923=2466%:%
%:%4925=2468%:%
%:%4926=2469%:%
%:%4927=2469%:%
%:%4928=2470%:%
%:%4929=2470%:%
%:%4930=2471%:%
%:%4931=2471%:%
%:%4932=2472%:%
%:%4933=2472%:%
%:%4934=2473%:%
%:%4935=2473%:%
%:%4936=2473%:%
%:%4937=2474%:%
%:%4938=2474%:%
%:%4939=2475%:%
%:%4940=2475%:%
%:%4941=2475%:%
%:%4942=2476%:%
%:%4943=2476%:%
%:%4944=2477%:%
%:%4945=2477%:%
%:%4946=2477%:%
%:%4947=2478%:%
%:%4948=2478%:%
%:%4949=2479%:%
%:%4950=2479%:%
%:%4951=2479%:%
%:%4952=2480%:%
%:%4953=2480%:%
%:%4954=2481%:%
%:%4955=2481%:%
%:%4956=2481%:%
%:%4957=2482%:%
%:%4958=2482%:%
%:%4959=2483%:%
%:%4960=2483%:%
%:%4961=2484%:%
%:%4962=2484%:%
%:%4963=2485%:%
%:%4964=2485%:%
%:%4965=2486%:%
%:%4966=2487%:%
%:%4967=2487%:%
%:%4968=2488%:%
%:%4969=2488%:%
%:%4970=2489%:%
%:%4971=2489%:%
%:%4972=2490%:%
%:%4973=2490%:%
%:%4974=2491%:%
%:%4975=2491%:%
%:%4976=2491%:%
%:%4977=2492%:%
%:%4978=2492%:%
%:%4979=2493%:%
%:%4980=2493%:%
%:%4981=2493%:%
%:%4982=2494%:%
%:%4983=2494%:%
%:%4984=2495%:%
%:%4985=2495%:%
%:%4986=2495%:%
%:%4987=2496%:%
%:%4988=2496%:%
%:%4989=2497%:%
%:%4990=2497%:%
%:%4991=2497%:%
%:%4992=2498%:%
%:%4993=2498%:%
%:%4994=2499%:%
%:%4995=2499%:%
%:%4996=2499%:%
%:%4997=2500%:%
%:%4998=2500%:%
%:%4999=2501%:%
%:%5000=2501%:%
%:%5001=2502%:%
%:%5002=2502%:%
%:%5003=2503%:%
%:%5004=2503%:%
%:%5005=2504%:%
%:%5006=2504%:%
%:%5007=2504%:%
%:%5008=2505%:%
%:%5009=2505%:%
%:%5010=2506%:%
%:%5011=2506%:%
%:%5012=2507%:%
%:%5013=2507%:%
%:%5014=2507%:%
%:%5015=2508%:%
%:%5016=2508%:%
%:%5017=2509%:%
%:%5018=2509%:%
%:%5019=2510%:%
%:%5020=2510%:%
%:%5021=2511%:%
%:%5022=2511%:%
%:%5023=2512%:%
%:%5024=2512%:%
%:%5025=2513%:%
%:%5035=2515%:%
%:%5036=2516%:%
%:%5037=2517%:%
%:%5039=2519%:%
%:%5040=2519%:%
%:%5041=2520%:%
%:%5042=2521%:%
%:%5049=2522%:%
%:%5050=2522%:%
%:%5051=2523%:%
%:%5052=2523%:%
%:%5053=2524%:%
%:%5054=2524%:%
%:%5055=2525%:%
%:%5056=2525%:%
%:%5057=2526%:%
%:%5058=2526%:%
%:%5059=2527%:%
%:%5060=2527%:%
%:%5061=2528%:%
%:%5062=2528%:%
%:%5063=2529%:%
%:%5064=2529%:%
%:%5065=2530%:%
%:%5066=2530%:%
%:%5067=2531%:%
%:%5068=2531%:%
%:%5069=2531%:%
%:%5070=2532%:%
%:%5071=2532%:%
%:%5072=2533%:%
%:%5073=2533%:%
%:%5074=2533%:%
%:%5075=2534%:%
%:%5076=2534%:%
%:%5077=2535%:%
%:%5078=2535%:%
%:%5079=2536%:%
%:%5080=2536%:%
%:%5081=2536%:%
%:%5082=2537%:%
%:%5083=2537%:%
%:%5084=2538%:%
%:%5085=2538%:%
%:%5086=2538%:%
%:%5087=2539%:%
%:%5088=2539%:%
%:%5089=2540%:%
%:%5090=2540%:%
%:%5091=2540%:%
%:%5092=2541%:%
%:%5093=2541%:%
%:%5094=2542%:%
%:%5095=2542%:%
%:%5096=2542%:%
%:%5097=2543%:%
%:%5098=2543%:%
%:%5099=2544%:%
%:%5100=2544%:%
%:%5101=2544%:%
%:%5102=2545%:%
%:%5103=2545%:%
%:%5104=2546%:%
%:%5105=2546%:%
%:%5106=2546%:%
%:%5107=2547%:%
%:%5108=2547%:%
%:%5109=2548%:%
%:%5110=2548%:%
%:%5111=2548%:%
%:%5112=2549%:%
%:%5113=2549%:%
%:%5114=2550%:%
%:%5115=2550%:%
%:%5116=2550%:%
%:%5117=2551%:%
%:%5118=2551%:%
%:%5119=2552%:%
%:%5120=2552%:%
%:%5121=2552%:%
%:%5122=2553%:%
%:%5123=2553%:%
%:%5124=2554%:%
%:%5125=2554%:%
%:%5126=2555%:%
%:%5127=2555%:%
%:%5128=2555%:%
%:%5129=2556%:%
%:%5130=2556%:%
%:%5131=2557%:%
%:%5132=2557%:%
%:%5133=2557%:%
%:%5134=2558%:%
%:%5135=2558%:%
%:%5136=2559%:%
%:%5137=2559%:%
%:%5138=2560%:%
%:%5139=2560%:%
%:%5140=2560%:%
%:%5141=2561%:%
%:%5142=2561%:%
%:%5143=2562%:%
%:%5144=2562%:%
%:%5145=2562%:%
%:%5146=2563%:%
%:%5147=2563%:%
%:%5148=2564%:%
%:%5149=2564%:%
%:%5150=2564%:%
%:%5151=2565%:%
%:%5152=2565%:%
%:%5153=2565%:%
%:%5154=2566%:%
%:%5155=2566%:%
%:%5156=2567%:%
%:%5157=2567%:%
%:%5158=2568%:%
%:%5159=2568%:%
%:%5160=2568%:%
%:%5161=2569%:%
%:%5162=2569%:%
%:%5163=2570%:%
%:%5164=2570%:%
%:%5165=2570%:%
%:%5166=2571%:%
%:%5167=2571%:%
%:%5168=2572%:%
%:%5169=2572%:%
%:%5170=2572%:%
%:%5171=2573%:%
%:%5172=2573%:%
%:%5173=2574%:%
%:%5174=2574%:%
%:%5175=2574%:%
%:%5176=2575%:%
%:%5177=2575%:%
%:%5178=2575%:%
%:%5179=2576%:%
%:%5180=2576%:%
%:%5181=2577%:%
%:%5182=2577%:%
%:%5183=2577%:%
%:%5184=2578%:%
%:%5185=2578%:%
%:%5186=2579%:%
%:%5187=2579%:%
%:%5188=2579%:%
%:%5189=2580%:%
%:%5190=2580%:%
%:%5191=2580%:%
%:%5192=2581%:%
%:%5193=2581%:%
%:%5194=2582%:%
%:%5195=2582%:%
%:%5196=2582%:%
%:%5197=2583%:%
%:%5198=2583%:%
%:%5199=2584%:%
%:%5200=2584%:%
%:%5201=2585%:%
%:%5202=2585%:%
%:%5203=2585%:%
%:%5204=2586%:%
%:%5205=2586%:%
%:%5206=2587%:%
%:%5207=2587%:%
%:%5208=2588%:%
%:%5218=2590%:%
%:%5219=2591%:%
%:%5220=2592%:%
%:%5222=2594%:%
%:%5223=2594%:%
%:%5230=2595%:%
%:%5231=2595%:%
%:%5232=2596%:%
%:%5233=2596%:%
%:%5236=2599%:%
%:%5237=2600%:%
%:%5238=2600%:%
%:%5239=2601%:%
%:%5240=2601%:%
%:%5241=2602%:%
%:%5242=2602%:%
%:%5243=2603%:%
%:%5244=2603%:%
%:%5245=2604%:%
%:%5246=2604%:%
%:%5247=2604%:%
%:%5248=2605%:%
%:%5249=2605%:%
%:%5250=2606%:%
%:%5251=2606%:%
%:%5252=2606%:%
%:%5253=2607%:%
%:%5254=2607%:%
%:%5255=2608%:%
%:%5256=2608%:%
%:%5257=2608%:%
%:%5258=2609%:%
%:%5259=2609%:%
%:%5260=2610%:%
%:%5261=2610%:%
%:%5262=2610%:%
%:%5263=2611%:%
%:%5264=2611%:%
%:%5267=2614%:%
%:%5268=2615%:%
%:%5269=2615%:%
%:%5270=2615%:%
%:%5271=2616%:%
%:%5272=2616%:%
%:%5273=2617%:%
%:%5283=2619%:%
%:%5285=2621%:%
%:%5286=2621%:%
%:%5287=2622%:%
%:%5288=2623%:%
%:%5289=2624%:%
%:%5296=2625%:%
%:%5297=2625%:%
%:%5298=2626%:%
%:%5299=2626%:%
%:%5300=2627%:%
%:%5301=2627%:%
%:%5302=2627%:%
%:%5303=2627%:%
%:%5304=2627%:%
%:%5305=2628%:%
%:%5306=2628%:%
%:%5307=2629%:%
%:%5308=2629%:%
%:%5309=2630%:%
%:%5310=2630%:%