%
\begin{isabellebody}%
\setisabellecontext{TeoremaEx}%
%
\isadelimtheory
%
\endisadelimtheory
%
\isatagtheory
%
\endisatagtheory
{\isafoldtheory}%
%
\isadelimtheory
%
\endisadelimtheory
%
\begin{isamarkuptext}%
En este capítulo se demuestra el \isa{Teorema\ de\ Existencia\ de\ Modelo} para la Lógica 
  Proposicional y el \isa{Teorema\ de\ Compacidad}. Para ello, definiremos ciertas sucesiones monótonas 
  de conjuntos de fórmulas a partir de una colección y un conjunto perteneciente a ella.%
\end{isamarkuptext}\isamarkuptrue%
%
\isadelimdocument
%
\endisadelimdocument
%
\isatagdocument
%
\isamarkupsection{Sucesiones de conjuntos%
}
\isamarkuptrue%
%
\endisatagdocument
{\isafolddocument}%
%
\isadelimdocument
%
\endisadelimdocument
%
\begin{isamarkuptext}%
A partir de una colección \isa{C} y un conjunto \isa{S\ {\isasymin}\ C}, vamos 
 a definir una sucesión monótona \isa{{\isacharbraceleft}S\isactrlsub n{\isacharbraceright}} de conjuntos de fórmulas
 proposicionales. Se demostrarán las siguientes propiedades:
  \begin{itemize}
   \item Si \isa{C} verifica la propiedad de consistencia proposicional, 
   entonces todo elemento de la suceción \isa{{\isacharbraceleft}S\isactrlsub n{\isacharbraceright}} pertenece a \isa{C}. 
   \item Para todo \isa{n{\isacharcomma}m\ {\isasymin}\ {\isasymnat}} se verifica $\bigcup_{n \leq m} S_{n} = S_{m}$ 
   \item Definiremos el límite de dichas sucesiones, probando que todo 
    conjunto de \isa{{\isacharbraceleft}S\isactrlsub n{\isacharbraceright}} está contenido en el límite. 
  \item Si una fórmula pertenece al límite de la sucesión \isa{{\isacharbraceleft}S\isactrlsub n{\isacharbraceright}}, 
   entonces pertenece a algún conjunto de dicha sucesión. 
  \item Si \isa{S{\isacharprime}} es un subconjunto finito del límite de la sucesión \isa{{\isacharbraceleft}S\isactrlsub n{\isacharbraceright}}, 
   entonces existe un\\ \isa{k\ {\isasymin}\ {\isasymnat}} tal que \isa{S{\isacharprime}\ {\isasymsubseteq}\ S\isactrlsub k}.
\end{itemize}

  Recordemos que el conjunto de las fórmulas proposicionales se define recursivamente a partir 
  de un alfabeto numerable de variables proposicionales. Por lo tanto, el conjunto de fórmulas 
  proposicionales es igualmente numerable, de modo que es posible enumerar sus elementos. Una vez 
  realizada esta observación, veamos la definición de sucesión de conjuntos de fórmulas 
  proposicionales a partir de una colección y un conjunto de la misma.

\begin{definicion}
  Sea \isa{C} una colección de conjuntos de fórmulas proposicionales, \isa{S\ {\isasymin}\ C} y \isa{F\isactrlsub {\isadigit{1}}{\isacharcomma}\ F\isactrlsub {\isadigit{2}}{\isacharcomma}\ F\isactrlsub {\isadigit{3}}\ {\isasymdots}} una 
  enumeración de las fórmulas proposicionales. Se define la \isa{sucesión\ de\ conjuntos\ de\ C\ a\ partir\ de\ S} como sigue:

  $S_{0} = S$

  $S_{n+1} = \left\{ \begin{array}{lcc} S_{n} \cup \{F_{n}\} &  si  & S_{n} \cup \{F_{n}\} \in C \\ \\ S_{n} & c.c \end{array} \right.$ 
\end{definicion}

  Para su formalización en Isabelle se ha introducido una instancia en la teoría de \isa{Sintaxis} que 
  indica explícitamente que el conjunto de las fórmulas proposicionales es numerable, probado
  mediante el método \isa{countable{\isacharunderscore}datatype} de Isabelle.

  \isa{instance\ formula\ {\isacharcolon}{\isacharcolon}\ {\isacharparenleft}countable{\isacharparenright}\ countable\ by\ countable{\isacharunderscore}datatype}

  De esta manera se genera en Isabelle una enumeración predeterminada de los elementos del conjunto,
  junto con herramientas para probar propiedades referentes a la numerabilidad. En particular, en la 
  formalización de la definición \isa{{\isadigit{4}}{\isachardot}{\isadigit{1}}{\isachardot}{\isadigit{1}}} se utilizará la función \isa{from{\isacharunderscore}nat} que, al aplicarla a un 
  número natural \isa{n}, nos devuelve la \isa{n}-ésima fórmula proposicional según una enumeración 
  predeterminada en Isabelle. 

  Puesto que la definición de las sucesiones en \isa{{\isadigit{4}}{\isachardot}{\isadigit{1}}{\isachardot}{\isadigit{1}}} se trata de una definición 
  recursiva sobre la estructura recursiva de los números naturales, se formalizará en Isabelle
  mediante el tipo de funciones primitivas recursivas de la siguiente manera.%
\end{isamarkuptext}\isamarkuptrue%
\isacommand{primrec}\isamarkupfalse%
\ pcp{\isacharunderscore}seq\ \isakeyword{where}\isanewline
{\isachardoublequoteopen}pcp{\isacharunderscore}seq\ C\ S\ {\isadigit{0}}\ {\isacharequal}\ S{\isachardoublequoteclose}\ {\isacharbar}\isanewline
{\isachardoublequoteopen}pcp{\isacharunderscore}seq\ C\ S\ {\isacharparenleft}Suc\ n{\isacharparenright}\ {\isacharequal}\ {\isacharparenleft}let\ Sn\ {\isacharequal}\ pcp{\isacharunderscore}seq\ C\ S\ n{\isacharsemicolon}\ Sn{\isadigit{1}}\ {\isacharequal}\ insert\ {\isacharparenleft}from{\isacharunderscore}nat\ n{\isacharparenright}\ Sn\ in\isanewline
\ \ \ \ \ \ \ \ \ \ \ \ \ \ \ \ \ \ \ \ \ \ \ \ if\ Sn{\isadigit{1}}\ {\isasymin}\ C\ then\ Sn{\isadigit{1}}\ else\ Sn{\isacharparenright}{\isachardoublequoteclose}%
\begin{isamarkuptext}%
Veamos el primer resultado sobre dichas sucesiones.

  \begin{lema}
    Sea \isa{C} una colección de conjuntos con la propiedad de consistencia proposicional,\\ \isa{S\ {\isasymin}\ C} y 
    \isa{{\isacharbraceleft}S\isactrlsub n{\isacharbraceright}} la sucesión de conjuntos de \isa{C} a partir de \isa{S} construida según la definición \isa{{\isadigit{4}}{\isachardot}{\isadigit{1}}{\isachardot}{\isadigit{1}}}. 
    Entonces, para todo \isa{n\ {\isasymin}\ {\isasymnat}} se verifica que \isa{S\isactrlsub n\ {\isasymin}\ C}.
  \end{lema}

  Procedamos con su demostración.

  \begin{demostracion}
    El resultado se prueba por inducción en los números naturales que conforman los subíndices de la 
    sucesión.

    En primer lugar, tenemos que \isa{S\isactrlsub {\isadigit{0}}\ {\isacharequal}\ S} pertenece a \isa{C} por hipótesis.

    Por otro lado, supongamos que \isa{S\isactrlsub n\ {\isasymin}\ C}. Probemos que \isa{S\isactrlsub n\isactrlsub {\isacharplus}\isactrlsub {\isadigit{1}}\ {\isasymin}\ C}. Si suponemos que \isa{S\isactrlsub n\ {\isasymunion}\ {\isacharbraceleft}F\isactrlsub n{\isacharbraceright}\ {\isasymin}\ C},
    por definición tenemos que \isa{S\isactrlsub n\isactrlsub {\isacharplus}\isactrlsub {\isadigit{1}}\ {\isacharequal}\ S\isactrlsub n\ {\isasymunion}\ {\isacharbraceleft}F\isactrlsub n{\isacharbraceright}}, luego pertenece a \isa{C}. En caso contrario, si
    suponemos que \isa{S\isactrlsub n\ {\isasymunion}\ {\isacharbraceleft}F\isactrlsub n{\isacharbraceright}\ {\isasymnotin}\ C}, por definición tenemos que \isa{S\isactrlsub n\isactrlsub {\isacharplus}\isactrlsub {\isadigit{1}}\ {\isacharequal}\ S\isactrlsub n}, que pertenece igualmente
    a \isa{C} por hipótesis de inducción. Por tanto, queda probado el resultado.
  \end{demostracion}

  La formalización y demostración detallada del lema en Isabelle son las siguientes.%
\end{isamarkuptext}\isamarkuptrue%
\isacommand{lemma}\isamarkupfalse%
\ \isanewline
\ \ \isakeyword{assumes}\ {\isachardoublequoteopen}pcp\ C{\isachardoublequoteclose}\ \isanewline
\ \ \ \ \ \ \ \ \ \ {\isachardoublequoteopen}S\ {\isasymin}\ C{\isachardoublequoteclose}\isanewline
\ \ \ \ \ \ \ \ \isakeyword{shows}\ {\isachardoublequoteopen}pcp{\isacharunderscore}seq\ C\ S\ n\ {\isasymin}\ C{\isachardoublequoteclose}\isanewline
%
\isadelimproof
%
\endisadelimproof
%
\isatagproof
\isacommand{proof}\isamarkupfalse%
\ {\isacharparenleft}induction\ n{\isacharparenright}\isanewline
\ \ \isacommand{show}\isamarkupfalse%
\ {\isachardoublequoteopen}pcp{\isacharunderscore}seq\ C\ S\ {\isadigit{0}}\ {\isasymin}\ C{\isachardoublequoteclose}\isanewline
\ \ \ \ \isacommand{by}\isamarkupfalse%
\ {\isacharparenleft}simp\ only{\isacharcolon}\ pcp{\isacharunderscore}seq{\isachardot}simps{\isacharparenleft}{\isadigit{1}}{\isacharparenright}\ {\isacartoucheopen}S\ {\isasymin}\ C{\isacartoucheclose}{\isacharparenright}\isanewline
\isacommand{next}\isamarkupfalse%
\isanewline
\ \ \isacommand{fix}\isamarkupfalse%
\ n\isanewline
\ \ \isacommand{assume}\isamarkupfalse%
\ HI{\isacharcolon}{\isachardoublequoteopen}pcp{\isacharunderscore}seq\ C\ S\ n\ {\isasymin}\ C{\isachardoublequoteclose}\isanewline
\ \ \isacommand{have}\isamarkupfalse%
\ {\isachardoublequoteopen}pcp{\isacharunderscore}seq\ C\ S\ {\isacharparenleft}Suc\ n{\isacharparenright}\ {\isacharequal}\ {\isacharparenleft}let\ Sn\ {\isacharequal}\ pcp{\isacharunderscore}seq\ C\ S\ n{\isacharsemicolon}\ Sn{\isadigit{1}}\ {\isacharequal}\ insert\ {\isacharparenleft}from{\isacharunderscore}nat\ n{\isacharparenright}\ Sn\ in\isanewline
\ \ \ \ \ \ \ \ \ \ \ \ \ \ \ \ \ \ \ \ \ \ \ \ if\ Sn{\isadigit{1}}\ {\isasymin}\ C\ then\ Sn{\isadigit{1}}\ else\ Sn{\isacharparenright}{\isachardoublequoteclose}\ \isanewline
\ \ \ \ \isacommand{by}\isamarkupfalse%
\ {\isacharparenleft}simp\ only{\isacharcolon}\ pcp{\isacharunderscore}seq{\isachardot}simps{\isacharparenleft}{\isadigit{2}}{\isacharparenright}{\isacharparenright}\isanewline
\ \ \isacommand{then}\isamarkupfalse%
\ \isacommand{have}\isamarkupfalse%
\ SucDef{\isacharcolon}{\isachardoublequoteopen}pcp{\isacharunderscore}seq\ C\ S\ {\isacharparenleft}Suc\ n{\isacharparenright}\ {\isacharequal}\ {\isacharparenleft}if\ insert\ {\isacharparenleft}from{\isacharunderscore}nat\ n{\isacharparenright}\ {\isacharparenleft}pcp{\isacharunderscore}seq\ C\ S\ n{\isacharparenright}\ {\isasymin}\ C\ then\ \isanewline
\ \ \ \ \ \ \ \ \ \ \ \ \ \ \ \ \ \ \ \ insert\ {\isacharparenleft}from{\isacharunderscore}nat\ n{\isacharparenright}\ {\isacharparenleft}pcp{\isacharunderscore}seq\ C\ S\ n{\isacharparenright}\ else\ pcp{\isacharunderscore}seq\ C\ S\ n{\isacharparenright}{\isachardoublequoteclose}\ \isanewline
\ \ \ \ \isacommand{by}\isamarkupfalse%
\ {\isacharparenleft}simp\ only{\isacharcolon}\ Let{\isacharunderscore}def{\isacharparenright}\isanewline
\ \ \isacommand{show}\isamarkupfalse%
\ {\isachardoublequoteopen}pcp{\isacharunderscore}seq\ C\ S\ {\isacharparenleft}Suc\ n{\isacharparenright}\ {\isasymin}\ C{\isachardoublequoteclose}\isanewline
\ \ \isacommand{proof}\isamarkupfalse%
\ {\isacharparenleft}cases{\isacharparenright}\isanewline
\ \ \ \ \isacommand{assume}\isamarkupfalse%
\ {\isadigit{1}}{\isacharcolon}{\isachardoublequoteopen}insert\ {\isacharparenleft}from{\isacharunderscore}nat\ n{\isacharparenright}\ {\isacharparenleft}pcp{\isacharunderscore}seq\ C\ S\ n{\isacharparenright}\ {\isasymin}\ C{\isachardoublequoteclose}\isanewline
\ \ \ \ \isacommand{have}\isamarkupfalse%
\ {\isachardoublequoteopen}pcp{\isacharunderscore}seq\ C\ S\ {\isacharparenleft}Suc\ n{\isacharparenright}\ {\isacharequal}\ insert\ {\isacharparenleft}from{\isacharunderscore}nat\ n{\isacharparenright}\ {\isacharparenleft}pcp{\isacharunderscore}seq\ C\ S\ n{\isacharparenright}{\isachardoublequoteclose}\isanewline
\ \ \ \ \ \ \isacommand{using}\isamarkupfalse%
\ SucDef\ {\isadigit{1}}\ \isacommand{by}\isamarkupfalse%
\ {\isacharparenleft}simp\ only{\isacharcolon}\ if{\isacharunderscore}True{\isacharparenright}\isanewline
\ \ \ \ \isacommand{thus}\isamarkupfalse%
\ {\isachardoublequoteopen}pcp{\isacharunderscore}seq\ C\ S\ {\isacharparenleft}Suc\ n{\isacharparenright}\ {\isasymin}\ C{\isachardoublequoteclose}\isanewline
\ \ \ \ \ \ \isacommand{by}\isamarkupfalse%
\ {\isacharparenleft}simp\ only{\isacharcolon}\ {\isadigit{1}}{\isacharparenright}\isanewline
\ \ \isacommand{next}\isamarkupfalse%
\isanewline
\ \ \ \ \isacommand{assume}\isamarkupfalse%
\ {\isadigit{2}}{\isacharcolon}{\isachardoublequoteopen}insert\ {\isacharparenleft}from{\isacharunderscore}nat\ n{\isacharparenright}\ {\isacharparenleft}pcp{\isacharunderscore}seq\ C\ S\ n{\isacharparenright}\ {\isasymnotin}\ C{\isachardoublequoteclose}\isanewline
\ \ \ \ \isacommand{have}\isamarkupfalse%
\ {\isachardoublequoteopen}pcp{\isacharunderscore}seq\ C\ S\ {\isacharparenleft}Suc\ n{\isacharparenright}\ {\isacharequal}\ pcp{\isacharunderscore}seq\ C\ S\ n{\isachardoublequoteclose}\isanewline
\ \ \ \ \ \ \isacommand{using}\isamarkupfalse%
\ SucDef\ {\isadigit{2}}\ \isacommand{by}\isamarkupfalse%
\ {\isacharparenleft}simp\ only{\isacharcolon}\ if{\isacharunderscore}False{\isacharparenright}\isanewline
\ \ \ \ \isacommand{thus}\isamarkupfalse%
\ {\isachardoublequoteopen}pcp{\isacharunderscore}seq\ C\ S\ {\isacharparenleft}Suc\ n{\isacharparenright}\ {\isasymin}\ C{\isachardoublequoteclose}\isanewline
\ \ \ \ \ \ \isacommand{by}\isamarkupfalse%
\ {\isacharparenleft}simp\ only{\isacharcolon}\ HI{\isacharparenright}\isanewline
\ \ \isacommand{qed}\isamarkupfalse%
\isanewline
\isacommand{qed}\isamarkupfalse%
%
\endisatagproof
{\isafoldproof}%
%
\isadelimproof
%
\endisadelimproof
%
\begin{isamarkuptext}%
Del mismo modo, podemos probar el lema de manera automática en Isabelle.%
\end{isamarkuptext}\isamarkuptrue%
\isacommand{lemma}\isamarkupfalse%
\ pcp{\isacharunderscore}seq{\isacharunderscore}in{\isacharcolon}\ {\isachardoublequoteopen}pcp\ C\ {\isasymLongrightarrow}\ S\ {\isasymin}\ C\ {\isasymLongrightarrow}\ pcp{\isacharunderscore}seq\ C\ S\ n\ {\isasymin}\ C{\isachardoublequoteclose}\isanewline
%
\isadelimproof
%
\endisadelimproof
%
\isatagproof
\isacommand{proof}\isamarkupfalse%
{\isacharparenleft}induction\ n{\isacharparenright}\isanewline
\ \ \isacommand{case}\isamarkupfalse%
\ {\isacharparenleft}Suc\ n{\isacharparenright}\ \ \isanewline
\ \ \isacommand{hence}\isamarkupfalse%
\ {\isachardoublequoteopen}pcp{\isacharunderscore}seq\ C\ S\ n\ {\isasymin}\ C{\isachardoublequoteclose}\ \isacommand{by}\isamarkupfalse%
\ simp\isanewline
\ \ \isacommand{thus}\isamarkupfalse%
\ {\isacharquery}case\ \isacommand{by}\isamarkupfalse%
\ {\isacharparenleft}simp\ add{\isacharcolon}\ Let{\isacharunderscore}def{\isacharparenright}\isanewline
\isacommand{qed}\isamarkupfalse%
\ simp%
\endisatagproof
{\isafoldproof}%
%
\isadelimproof
%
\endisadelimproof
%
\begin{isamarkuptext}%
Por otro lado, veamos la monotonía de dichas sucesiones.

  \begin{lema}
    Toda sucesión de conjuntos construida a partir de una colección y un conjunto según la
    definición \isa{{\isadigit{4}}{\isachardot}{\isadigit{1}}{\isachardot}{\isadigit{1}}} es monótona.
  \end{lema}

  En Isabelle, se formaliza de la siguiente forma.%
\end{isamarkuptext}\isamarkuptrue%
\isacommand{lemma}\isamarkupfalse%
\ {\isachardoublequoteopen}pcp{\isacharunderscore}seq\ C\ S\ n\ {\isasymsubseteq}\ pcp{\isacharunderscore}seq\ C\ S\ {\isacharparenleft}Suc\ n{\isacharparenright}{\isachardoublequoteclose}\isanewline
%
\isadelimproof
\ \ %
\endisadelimproof
%
\isatagproof
\isacommand{oops}\isamarkupfalse%
%
\endisatagproof
{\isafoldproof}%
%
\isadelimproof
%
\endisadelimproof
%
\begin{isamarkuptext}%
Procedamos con la demostración del lema.

  \begin{demostracion}
    Sea una colección de conjuntos \isa{C}, \isa{S\ {\isasymin}\ C} y \isa{{\isacharbraceleft}S\isactrlsub n{\isacharbraceright}} la sucesión de conjuntos de \isa{C} a partir de 
    \isa{S} según la definición \isa{{\isadigit{4}}{\isachardot}{\isadigit{1}}{\isachardot}{\isadigit{1}}}. Para probar que \isa{{\isacharbraceleft}S\isactrlsub n{\isacharbraceright}} es monótona, basta probar que \isa{S\isactrlsub n\ {\isasymsubseteq}\ S\isactrlsub n\isactrlsub {\isacharplus}\isactrlsub {\isadigit{1}}} 
    para todo \isa{n\ {\isasymin}\ {\isasymnat}}. En efecto, el resultado es inmediato al considerar dos casos para todo 
    \isa{n\ {\isasymin}\ {\isasymnat}}: \isa{S\isactrlsub n\ {\isasymunion}\ {\isacharbraceleft}F\isactrlsub n{\isacharbraceright}\ {\isasymin}\ C} o \isa{S\isactrlsub n\ {\isasymunion}\ {\isacharbraceleft}F\isactrlsub n{\isacharbraceright}\ {\isasymnotin}\ C}. Si suponemos que\\ \isa{S\isactrlsub n\ {\isasymunion}\ {\isacharbraceleft}F\isactrlsub n{\isacharbraceright}\ {\isasymin}\ C}, por definición 
    tenemos que \isa{S\isactrlsub n\isactrlsub {\isacharplus}\isactrlsub {\isadigit{1}}\ {\isacharequal}\ S\isactrlsub n\ {\isasymunion}\ {\isacharbraceleft}F\isactrlsub n{\isacharbraceright}}, luego es claro que\\ \isa{S\isactrlsub n\ {\isasymsubseteq}\ S\isactrlsub n\isactrlsub {\isacharplus}\isactrlsub {\isadigit{1}}}. En caso contrario, si 
    \isa{S\isactrlsub n\ {\isasymunion}\ {\isacharbraceleft}F\isactrlsub n{\isacharbraceright}\ {\isasymnotin}\ C}, por definición se tiene que \isa{S\isactrlsub n\isactrlsub {\isacharplus}\isactrlsub {\isadigit{1}}\ {\isacharequal}\ S\isactrlsub n}, obteniéndose igualmente el resultado
    por la propiedad reflexiva de la contención de conjuntos. 
  \end{demostracion}

  La prueba detallada en Isabelle se muestra a continuación.%
\end{isamarkuptext}\isamarkuptrue%
\isacommand{lemma}\isamarkupfalse%
\ {\isachardoublequoteopen}pcp{\isacharunderscore}seq\ C\ S\ n\ {\isasymsubseteq}\ pcp{\isacharunderscore}seq\ C\ S\ {\isacharparenleft}Suc\ n{\isacharparenright}{\isachardoublequoteclose}\isanewline
%
\isadelimproof
%
\endisadelimproof
%
\isatagproof
\isacommand{proof}\isamarkupfalse%
\ {\isacharminus}\isanewline
\ \ \isacommand{have}\isamarkupfalse%
\ {\isachardoublequoteopen}pcp{\isacharunderscore}seq\ C\ S\ {\isacharparenleft}Suc\ n{\isacharparenright}\ {\isacharequal}\ {\isacharparenleft}let\ Sn\ {\isacharequal}\ pcp{\isacharunderscore}seq\ C\ S\ n{\isacharsemicolon}\ Sn{\isadigit{1}}\ {\isacharequal}\ insert\ {\isacharparenleft}from{\isacharunderscore}nat\ n{\isacharparenright}\ Sn\ in\isanewline
\ \ \ \ \ \ \ \ \ \ \ \ \ \ \ \ \ \ \ \ \ \ \ \ if\ Sn{\isadigit{1}}\ {\isasymin}\ C\ then\ Sn{\isadigit{1}}\ else\ Sn{\isacharparenright}{\isachardoublequoteclose}\ \isanewline
\ \ \ \ \isacommand{by}\isamarkupfalse%
\ {\isacharparenleft}simp\ only{\isacharcolon}\ pcp{\isacharunderscore}seq{\isachardot}simps{\isacharparenleft}{\isadigit{2}}{\isacharparenright}{\isacharparenright}\isanewline
\ \ \isacommand{then}\isamarkupfalse%
\ \isacommand{have}\isamarkupfalse%
\ SucDef{\isacharcolon}{\isachardoublequoteopen}pcp{\isacharunderscore}seq\ C\ S\ {\isacharparenleft}Suc\ n{\isacharparenright}\ {\isacharequal}\ {\isacharparenleft}if\ insert\ {\isacharparenleft}from{\isacharunderscore}nat\ n{\isacharparenright}\ {\isacharparenleft}pcp{\isacharunderscore}seq\ C\ S\ n{\isacharparenright}\ {\isasymin}\ C\ then\ \isanewline
\ \ \ \ \ \ \ \ \ \ \ \ \ \ \ \ \ \ \ \ insert\ {\isacharparenleft}from{\isacharunderscore}nat\ n{\isacharparenright}\ {\isacharparenleft}pcp{\isacharunderscore}seq\ C\ S\ n{\isacharparenright}\ else\ pcp{\isacharunderscore}seq\ C\ S\ n{\isacharparenright}{\isachardoublequoteclose}\ \isanewline
\ \ \ \ \isacommand{by}\isamarkupfalse%
\ {\isacharparenleft}simp\ only{\isacharcolon}\ Let{\isacharunderscore}def{\isacharparenright}\isanewline
\ \ \isacommand{thus}\isamarkupfalse%
\ {\isachardoublequoteopen}pcp{\isacharunderscore}seq\ C\ S\ n\ {\isasymsubseteq}\ pcp{\isacharunderscore}seq\ C\ S\ {\isacharparenleft}Suc\ n{\isacharparenright}{\isachardoublequoteclose}\isanewline
\ \ \isacommand{proof}\isamarkupfalse%
\ {\isacharparenleft}cases{\isacharparenright}\isanewline
\ \ \ \ \isacommand{assume}\isamarkupfalse%
\ {\isadigit{1}}{\isacharcolon}{\isachardoublequoteopen}insert\ {\isacharparenleft}from{\isacharunderscore}nat\ n{\isacharparenright}\ {\isacharparenleft}pcp{\isacharunderscore}seq\ C\ S\ n{\isacharparenright}\ {\isasymin}\ C{\isachardoublequoteclose}\isanewline
\ \ \ \ \isacommand{have}\isamarkupfalse%
\ {\isachardoublequoteopen}pcp{\isacharunderscore}seq\ C\ S\ {\isacharparenleft}Suc\ n{\isacharparenright}\ {\isacharequal}\ insert\ {\isacharparenleft}from{\isacharunderscore}nat\ n{\isacharparenright}\ {\isacharparenleft}pcp{\isacharunderscore}seq\ C\ S\ n{\isacharparenright}{\isachardoublequoteclose}\isanewline
\ \ \ \ \ \ \isacommand{using}\isamarkupfalse%
\ SucDef\ {\isadigit{1}}\ \isacommand{by}\isamarkupfalse%
\ {\isacharparenleft}simp\ only{\isacharcolon}\ if{\isacharunderscore}True{\isacharparenright}\isanewline
\ \ \ \ \isacommand{thus}\isamarkupfalse%
\ {\isachardoublequoteopen}pcp{\isacharunderscore}seq\ C\ S\ n\ {\isasymsubseteq}\ pcp{\isacharunderscore}seq\ C\ S\ {\isacharparenleft}Suc\ n{\isacharparenright}{\isachardoublequoteclose}\isanewline
\ \ \ \ \ \ \isacommand{by}\isamarkupfalse%
\ {\isacharparenleft}simp\ only{\isacharcolon}\ subset{\isacharunderscore}insertI{\isacharparenright}\isanewline
\ \ \isacommand{next}\isamarkupfalse%
\isanewline
\ \ \ \ \isacommand{assume}\isamarkupfalse%
\ {\isadigit{2}}{\isacharcolon}{\isachardoublequoteopen}insert\ {\isacharparenleft}from{\isacharunderscore}nat\ n{\isacharparenright}\ {\isacharparenleft}pcp{\isacharunderscore}seq\ C\ S\ n{\isacharparenright}\ {\isasymnotin}\ C{\isachardoublequoteclose}\isanewline
\ \ \ \ \isacommand{have}\isamarkupfalse%
\ {\isachardoublequoteopen}pcp{\isacharunderscore}seq\ C\ S\ {\isacharparenleft}Suc\ n{\isacharparenright}\ {\isacharequal}\ pcp{\isacharunderscore}seq\ C\ S\ n{\isachardoublequoteclose}\isanewline
\ \ \ \ \ \ \isacommand{using}\isamarkupfalse%
\ SucDef\ {\isadigit{2}}\ \isacommand{by}\isamarkupfalse%
\ {\isacharparenleft}simp\ only{\isacharcolon}\ if{\isacharunderscore}False{\isacharparenright}\isanewline
\ \ \ \ \isacommand{thus}\isamarkupfalse%
\ {\isachardoublequoteopen}pcp{\isacharunderscore}seq\ C\ S\ n\ {\isasymsubseteq}\ pcp{\isacharunderscore}seq\ C\ S\ {\isacharparenleft}Suc\ n{\isacharparenright}{\isachardoublequoteclose}\isanewline
\ \ \ \ \ \ \isacommand{by}\isamarkupfalse%
\ {\isacharparenleft}simp\ only{\isacharcolon}\ subset{\isacharunderscore}refl{\isacharparenright}\isanewline
\ \ \isacommand{qed}\isamarkupfalse%
\isanewline
\isacommand{qed}\isamarkupfalse%
%
\endisatagproof
{\isafoldproof}%
%
\isadelimproof
%
\endisadelimproof
%
\begin{isamarkuptext}%
Del mismo modo, se puede probar automáticamente en Isabelle/HOL.%
\end{isamarkuptext}\isamarkuptrue%
\isacommand{lemma}\isamarkupfalse%
\ pcp{\isacharunderscore}seq{\isacharunderscore}monotonicity{\isacharcolon}{\isachardoublequoteopen}pcp{\isacharunderscore}seq\ C\ S\ n\ {\isasymsubseteq}\ pcp{\isacharunderscore}seq\ C\ S\ {\isacharparenleft}Suc\ n{\isacharparenright}{\isachardoublequoteclose}\isanewline
%
\isadelimproof
\ \ %
\endisadelimproof
%
\isatagproof
\isacommand{by}\isamarkupfalse%
\ {\isacharparenleft}smt\ eq{\isacharunderscore}iff\ pcp{\isacharunderscore}seq{\isachardot}simps{\isacharparenleft}{\isadigit{2}}{\isacharparenright}\ subset{\isacharunderscore}insertI{\isacharparenright}%
\endisatagproof
{\isafoldproof}%
%
\isadelimproof
%
\endisadelimproof
%
\begin{isamarkuptext}%
Por otra lado, para facilitar posteriores demostraciones en Isabelle/HOL, vamos a formalizar 
  el lema anterior empleando la siguiente definición generalizada de monotonía.%
\end{isamarkuptext}\isamarkuptrue%
\isacommand{lemma}\isamarkupfalse%
\ pcp{\isacharunderscore}seq{\isacharunderscore}mono{\isacharcolon}\isanewline
\ \ \isakeyword{assumes}\ {\isachardoublequoteopen}n\ {\isasymle}\ m{\isachardoublequoteclose}\ \isanewline
\ \ \isakeyword{shows}\ {\isachardoublequoteopen}pcp{\isacharunderscore}seq\ C\ S\ n\ {\isasymsubseteq}\ pcp{\isacharunderscore}seq\ C\ S\ m{\isachardoublequoteclose}\isanewline
%
\isadelimproof
\ \ %
\endisadelimproof
%
\isatagproof
\isacommand{using}\isamarkupfalse%
\ pcp{\isacharunderscore}seq{\isacharunderscore}monotonicity\ assms\ \isacommand{by}\isamarkupfalse%
\ {\isacharparenleft}rule\ lift{\isacharunderscore}Suc{\isacharunderscore}mono{\isacharunderscore}le{\isacharparenright}%
\endisatagproof
{\isafoldproof}%
%
\isadelimproof
%
\endisadelimproof
%
\begin{isamarkuptext}%
A continuación daremos un lema que permite caracterizar un elemento de la sucesión en función 
  de los anteriores.

\begin{lema}
  Sea \isa{C} una colección de conjuntos, \isa{S\ {\isasymin}\ C} y \isa{{\isacharbraceleft}S\isactrlsub n{\isacharbraceright}} la sucesión de conjuntos de \isa{C} a partir de 
  \isa{S} construida según la definición \isa{{\isadigit{4}}{\isachardot}{\isadigit{1}}{\isachardot}{\isadigit{1}}}. Entonces, para todos \isa{n}, \isa{m\ {\isasymin}\ {\isasymnat}} 
  se verifica $\bigcup_{n \leq m} S_{n} = S_{m}$
\end{lema}

\begin{demostracion}
  En las condiciones del enunciado, la prueba se realiza por inducción en \isa{m\ {\isasymin}\ {\isasymnat}}.

  En primer lugar, consideremos el caso base \isa{m\ {\isacharequal}\ {\isadigit{0}}}. El resultado se obtiene directamente:

  $\bigcup_{n \leq 0} S_{n} = \bigcup_{n = 0} S_{n} = S_{0} = S_{m}$

  Por otro lado, supongamos por hipótesis de inducción que $\bigcup_{n \leq m} S_{n} = S_{m}$.
  Veamos que se verifica $\bigcup_{n \leq m + 1} S_{n} = S_{m + 1}$. Observemos que si \isa{n\ {\isasymle}\ m\ {\isacharplus}\ {\isadigit{1}}},
  entonces se tiene que, o bien \isa{n\ {\isasymle}\ m}, o bien \isa{n\ {\isacharequal}\ m\ {\isacharplus}\ {\isadigit{1}}}. De este modo, aplicando la 
  hipótesis de inducción, deducimos lo siguiente.

  $\bigcup_{n \leq m + 1} S_{n} = \bigcup_{n \leq m} S_{n} \cup \bigcup_{n = m + 1} S_{n} = \bigcup_{n \leq m} S_{n} \cup S_{m + 1} = S_{m} \cup S_{m + 1}$

  Por la monotonía de la sucesión, se tiene que \isa{S\isactrlsub m\ {\isasymsubseteq}\ S\isactrlsub m\isactrlsub {\isacharplus}\isactrlsub {\isadigit{1}}}. Luego, se verifica:

  $\bigcup_{n \leq m + 1} S_{n} = S_{m} \cup S_{m + 1} = S_{m + 1}$

  Lo que prueba el resultado.
\end{demostracion}

  Para formalizar dicho resultado y su demostración en Isabelle, hay que tener en cuenta cómo está
  formalizada la \isa{Teoría\ de\ Conjuntos} \href{https://bit.ly/3ibCuje}{Set.thy}. Los conjuntos están 
  formalizados como predicados, junto con la función \isa{Collect} y el predicado \isa{member}, verificando 
  los siguientes axiomas:

  \isa{mem{\isacharunderscore}Collect{\isacharunderscore}eq{\isacharcolon}}  "\isa{member\ a\ {\isacharparenleft}Collect\ P{\isacharparenright}\ {\isacharequal}\ P\ a}"

  \isa{Collect{\isacharunderscore}mem{\isacharunderscore}eq{\isacharcolon}} "\isa{Collect\ {\isacharparenleft}{\isasymlambda}x{\isachardot}\ member\ x\ A{\isacharparenright}\ {\isacharequal}\ A}"

  Se demuestra también que el tipo de los conjuntos constituye un álgebra de \isa{Boole}, en el que el 
  supremo de dos conjuntos es la unión y el ínfimo es la intersección. De esta forma, podemos usar 
  la unión generalizada, definida en la la teoría de retículos completos de Isabelle 
  \href{https://n9.cl/gtf5x}{Complete-Lattices.thy}.

  Veamos la prueba detallada del resultado en Isabelle/HOL.%
\end{isamarkuptext}\isamarkuptrue%
\isacommand{lemma}\isamarkupfalse%
\ {\isachardoublequoteopen}{\isasymUnion}{\isacharbraceleft}pcp{\isacharunderscore}seq\ C\ S\ n{\isacharbar}n{\isachardot}\ n\ {\isasymle}\ m{\isacharbraceright}\ {\isacharequal}\ pcp{\isacharunderscore}seq\ C\ S\ m{\isachardoublequoteclose}\isanewline
%
\isadelimproof
%
\endisadelimproof
%
\isatagproof
\isacommand{proof}\isamarkupfalse%
\ {\isacharparenleft}induct\ m{\isacharparenright}\isanewline
\ \ \isacommand{have}\isamarkupfalse%
\ \ {\isachardoublequoteopen}{\isasymUnion}{\isacharbraceleft}pcp{\isacharunderscore}seq\ C\ S\ n{\isacharbar}n{\isachardot}\ n\ {\isasymle}\ {\isadigit{0}}{\isacharbraceright}\ {\isacharequal}\ {\isasymUnion}{\isacharbraceleft}pcp{\isacharunderscore}seq\ C\ S\ n{\isacharbar}n{\isachardot}\ n\ {\isacharequal}\ {\isadigit{0}}{\isacharbraceright}{\isachardoublequoteclose}\isanewline
\ \ \ \ \isacommand{by}\isamarkupfalse%
\ {\isacharparenleft}simp\ only{\isacharcolon}\ le{\isacharunderscore}zero{\isacharunderscore}eq{\isacharparenright}\isanewline
\ \ \isacommand{also}\isamarkupfalse%
\ \isacommand{have}\isamarkupfalse%
\ {\isachardoublequoteopen}{\isasymdots}\ {\isacharequal}\ {\isasymUnion}{\isacharparenleft}{\isacharparenleft}pcp{\isacharunderscore}seq\ C\ S{\isacharparenright}{\isacharbackquote}{\isacharbraceleft}n{\isachardot}\ n\ {\isacharequal}\ {\isadigit{0}}{\isacharbraceright}{\isacharparenright}{\isachardoublequoteclose}\isanewline
\ \ \ \ \isacommand{by}\isamarkupfalse%
\ {\isacharparenleft}simp\ only{\isacharcolon}\ image{\isacharunderscore}Collect{\isacharparenright}\isanewline
\ \ \isacommand{also}\isamarkupfalse%
\ \isacommand{have}\isamarkupfalse%
\ {\isachardoublequoteopen}{\isasymdots}\ {\isacharequal}\ {\isasymUnion}{\isacharbraceleft}pcp{\isacharunderscore}seq\ C\ S\ {\isadigit{0}}{\isacharbraceright}{\isachardoublequoteclose}\isanewline
\ \ \ \ \isacommand{by}\isamarkupfalse%
\ {\isacharparenleft}simp\ only{\isacharcolon}\ singleton{\isacharunderscore}conv\ image{\isacharunderscore}insert\ image{\isacharunderscore}empty{\isacharparenright}\isanewline
\ \ \isacommand{also}\isamarkupfalse%
\ \isacommand{have}\isamarkupfalse%
\ {\isachardoublequoteopen}{\isasymdots}\ {\isacharequal}\ pcp{\isacharunderscore}seq\ C\ S\ {\isadigit{0}}{\isachardoublequoteclose}\ \isanewline
\ \ \ \ \isacommand{by}\isamarkupfalse%
\ \ {\isacharparenleft}simp\ only{\isacharcolon}cSup{\isacharunderscore}singleton{\isacharparenright}\isanewline
\ \ \isacommand{finally}\isamarkupfalse%
\ \isacommand{show}\isamarkupfalse%
\ {\isachardoublequoteopen}{\isasymUnion}{\isacharbraceleft}pcp{\isacharunderscore}seq\ C\ S\ n{\isacharbar}n{\isachardot}\ n\ {\isasymle}\ {\isadigit{0}}{\isacharbraceright}\ {\isacharequal}\ pcp{\isacharunderscore}seq\ C\ S\ {\isadigit{0}}{\isachardoublequoteclose}\ \isanewline
\ \ \ \ \isacommand{by}\isamarkupfalse%
\ this\isanewline
\isacommand{next}\isamarkupfalse%
\isanewline
\ \ \isacommand{fix}\isamarkupfalse%
\ m\isanewline
\ \ \isacommand{assume}\isamarkupfalse%
\ HI{\isacharcolon}{\isachardoublequoteopen}{\isasymUnion}{\isacharbraceleft}pcp{\isacharunderscore}seq\ C\ S\ n{\isacharbar}n{\isachardot}\ n\ {\isasymle}\ m{\isacharbraceright}\ {\isacharequal}\ pcp{\isacharunderscore}seq\ C\ S\ m{\isachardoublequoteclose}\isanewline
\ \ \isacommand{have}\isamarkupfalse%
\ {\isachardoublequoteopen}m\ {\isasymle}\ Suc\ m{\isachardoublequoteclose}\ \isanewline
\ \ \ \ \isacommand{by}\isamarkupfalse%
\ {\isacharparenleft}simp\ only{\isacharcolon}\ add{\isacharunderscore}{\isadigit{0}}{\isacharunderscore}right{\isacharparenright}\isanewline
\ \ \isacommand{then}\isamarkupfalse%
\ \isacommand{have}\isamarkupfalse%
\ Mon{\isacharcolon}{\isachardoublequoteopen}pcp{\isacharunderscore}seq\ C\ S\ m\ {\isasymsubseteq}\ pcp{\isacharunderscore}seq\ C\ S\ {\isacharparenleft}Suc\ m{\isacharparenright}{\isachardoublequoteclose}\isanewline
\ \ \ \ \isacommand{by}\isamarkupfalse%
\ {\isacharparenleft}rule\ pcp{\isacharunderscore}seq{\isacharunderscore}mono{\isacharparenright}\isanewline
\ \ \isacommand{have}\isamarkupfalse%
\ {\isachardoublequoteopen}{\isasymUnion}{\isacharbraceleft}pcp{\isacharunderscore}seq\ C\ S\ n\ {\isacharbar}\ n{\isachardot}\ n\ {\isasymle}\ Suc\ m{\isacharbraceright}\ {\isacharequal}\ {\isasymUnion}{\isacharparenleft}{\isacharparenleft}pcp{\isacharunderscore}seq\ C\ S{\isacharparenright}{\isacharbackquote}{\isacharparenleft}{\isacharbraceleft}n{\isachardot}\ n\ {\isasymle}\ Suc\ m{\isacharbraceright}{\isacharparenright}{\isacharparenright}{\isachardoublequoteclose}\isanewline
\ \ \ \ \isacommand{by}\isamarkupfalse%
\ {\isacharparenleft}simp\ only{\isacharcolon}\ image{\isacharunderscore}Collect{\isacharparenright}\isanewline
\ \ \isacommand{also}\isamarkupfalse%
\ \isacommand{have}\isamarkupfalse%
\ {\isachardoublequoteopen}{\isasymdots}\ {\isacharequal}\ {\isasymUnion}{\isacharparenleft}{\isacharparenleft}pcp{\isacharunderscore}seq\ C\ S{\isacharparenright}{\isacharbackquote}{\isacharparenleft}{\isacharbraceleft}Suc\ m{\isacharbraceright}\ {\isasymunion}\ {\isacharbraceleft}n{\isachardot}\ n\ {\isasymle}\ m{\isacharbraceright}{\isacharparenright}{\isacharparenright}{\isachardoublequoteclose}\isanewline
\ \ \ \ \isacommand{by}\isamarkupfalse%
\ {\isacharparenleft}simp\ only{\isacharcolon}\ le{\isacharunderscore}Suc{\isacharunderscore}eq\ Collect{\isacharunderscore}disj{\isacharunderscore}eq\ Un{\isacharunderscore}commute\ singleton{\isacharunderscore}conv{\isacharparenright}\isanewline
\ \ \isacommand{also}\isamarkupfalse%
\ \isacommand{have}\isamarkupfalse%
\ {\isachardoublequoteopen}{\isasymdots}\ {\isacharequal}\ {\isasymUnion}{\isacharparenleft}{\isacharbraceleft}pcp{\isacharunderscore}seq\ C\ S\ {\isacharparenleft}Suc\ m{\isacharparenright}{\isacharbraceright}\ {\isasymunion}\ {\isacharbraceleft}pcp{\isacharunderscore}seq\ C\ S\ n\ {\isacharbar}\ n{\isachardot}\ n\ {\isasymle}\ m{\isacharbraceright}{\isacharparenright}{\isachardoublequoteclose}\isanewline
\ \ \ \ \isacommand{by}\isamarkupfalse%
\ {\isacharparenleft}simp\ only{\isacharcolon}\ image{\isacharunderscore}Un\ image{\isacharunderscore}insert\ image{\isacharunderscore}empty\ image{\isacharunderscore}Collect{\isacharparenright}\isanewline
\ \ \isacommand{also}\isamarkupfalse%
\ \isacommand{have}\isamarkupfalse%
\ {\isachardoublequoteopen}{\isasymdots}\ {\isacharequal}\ {\isasymUnion}{\isacharbraceleft}pcp{\isacharunderscore}seq\ C\ S\ {\isacharparenleft}Suc\ m{\isacharparenright}{\isacharbraceright}\ {\isasymunion}\ {\isasymUnion}{\isacharbraceleft}pcp{\isacharunderscore}seq\ C\ S\ n\ {\isacharbar}\ n{\isachardot}\ n\ {\isasymle}\ m{\isacharbraceright}{\isachardoublequoteclose}\isanewline
\ \ \ \ \isacommand{by}\isamarkupfalse%
\ {\isacharparenleft}simp\ only{\isacharcolon}\ Union{\isacharunderscore}Un{\isacharunderscore}distrib{\isacharparenright}\isanewline
\ \ \isacommand{also}\isamarkupfalse%
\ \isacommand{have}\isamarkupfalse%
\ {\isachardoublequoteopen}{\isasymdots}\ {\isacharequal}\ {\isacharparenleft}pcp{\isacharunderscore}seq\ C\ S\ {\isacharparenleft}Suc\ m{\isacharparenright}{\isacharparenright}\ {\isasymunion}\ {\isasymUnion}{\isacharbraceleft}pcp{\isacharunderscore}seq\ C\ S\ n\ {\isacharbar}\ n{\isachardot}\ n\ {\isasymle}\ m{\isacharbraceright}{\isachardoublequoteclose}\isanewline
\ \ \ \ \isacommand{by}\isamarkupfalse%
\ {\isacharparenleft}simp\ only{\isacharcolon}\ cSup{\isacharunderscore}singleton{\isacharparenright}\isanewline
\ \ \isacommand{also}\isamarkupfalse%
\ \isacommand{have}\isamarkupfalse%
\ {\isachardoublequoteopen}{\isasymdots}\ {\isacharequal}\ {\isacharparenleft}pcp{\isacharunderscore}seq\ C\ S\ {\isacharparenleft}Suc\ m{\isacharparenright}{\isacharparenright}\ {\isasymunion}\ {\isacharparenleft}pcp{\isacharunderscore}seq\ C\ S\ m{\isacharparenright}{\isachardoublequoteclose}\isanewline
\ \ \ \ \isacommand{by}\isamarkupfalse%
\ {\isacharparenleft}simp\ only{\isacharcolon}\ HI{\isacharparenright}\isanewline
\ \ \isacommand{also}\isamarkupfalse%
\ \isacommand{have}\isamarkupfalse%
\ {\isachardoublequoteopen}{\isasymdots}\ {\isacharequal}\ pcp{\isacharunderscore}seq\ C\ S\ {\isacharparenleft}Suc\ m{\isacharparenright}{\isachardoublequoteclose}\isanewline
\ \ \ \ \isacommand{using}\isamarkupfalse%
\ Mon\ \isacommand{by}\isamarkupfalse%
\ {\isacharparenleft}simp\ only{\isacharcolon}\ Un{\isacharunderscore}absorb{\isadigit{2}}{\isacharparenright}\isanewline
\ \ \isacommand{finally}\isamarkupfalse%
\ \isacommand{show}\isamarkupfalse%
\ {\isachardoublequoteopen}{\isasymUnion}{\isacharbraceleft}pcp{\isacharunderscore}seq\ C\ S\ n{\isacharbar}n{\isachardot}\ n\ {\isasymle}\ {\isacharparenleft}Suc\ m{\isacharparenright}{\isacharbraceright}\ {\isacharequal}\ pcp{\isacharunderscore}seq\ C\ S\ {\isacharparenleft}Suc\ m{\isacharparenright}{\isachardoublequoteclose}\isanewline
\ \ \ \ \isacommand{by}\isamarkupfalse%
\ this\isanewline
\isacommand{qed}\isamarkupfalse%
%
\endisatagproof
{\isafoldproof}%
%
\isadelimproof
%
\endisadelimproof
%
\begin{isamarkuptext}%
Análogamente, podemos dar una prueba automática.%
\end{isamarkuptext}\isamarkuptrue%
\isacommand{lemma}\isamarkupfalse%
\ pcp{\isacharunderscore}seq{\isacharunderscore}UN{\isacharcolon}\ {\isachardoublequoteopen}{\isasymUnion}{\isacharbraceleft}pcp{\isacharunderscore}seq\ C\ S\ n{\isacharbar}n{\isachardot}\ n\ {\isasymle}\ m{\isacharbraceright}\ {\isacharequal}\ pcp{\isacharunderscore}seq\ C\ S\ m{\isachardoublequoteclose}\isanewline
%
\isadelimproof
%
\endisadelimproof
%
\isatagproof
\isacommand{proof}\isamarkupfalse%
{\isacharparenleft}induction\ m{\isacharparenright}\isanewline
\ \ \isacommand{case}\isamarkupfalse%
\ {\isacharparenleft}Suc\ m{\isacharparenright}\isanewline
\ \ \isacommand{have}\isamarkupfalse%
\ {\isachardoublequoteopen}{\isacharbraceleft}f\ n\ {\isacharbar}n{\isachardot}\ n\ {\isasymle}\ Suc\ m{\isacharbraceright}\ {\isacharequal}\ insert\ {\isacharparenleft}f\ {\isacharparenleft}Suc\ m{\isacharparenright}{\isacharparenright}\ {\isacharbraceleft}f\ n\ {\isacharbar}n{\isachardot}\ n\ {\isasymle}\ m{\isacharbraceright}{\isachardoublequoteclose}\ \isanewline
\ \ \ \ \isakeyword{for}\ f\ \isacommand{using}\isamarkupfalse%
\ le{\isacharunderscore}Suc{\isacharunderscore}eq\ \isacommand{by}\isamarkupfalse%
\ auto\isanewline
\ \ \isacommand{hence}\isamarkupfalse%
\ {\isachardoublequoteopen}{\isacharbraceleft}pcp{\isacharunderscore}seq\ C\ S\ n\ {\isacharbar}n{\isachardot}\ n\ {\isasymle}\ Suc\ m{\isacharbraceright}\ {\isacharequal}\ \isanewline
\ \ \ \ \ \ \ \ \ \ insert\ {\isacharparenleft}pcp{\isacharunderscore}seq\ C\ S\ {\isacharparenleft}Suc\ m{\isacharparenright}{\isacharparenright}\ {\isacharbraceleft}pcp{\isacharunderscore}seq\ C\ S\ n\ {\isacharbar}n{\isachardot}\ n\ {\isasymle}\ m{\isacharbraceright}{\isachardoublequoteclose}\ \isacommand{{\isachardot}}\isamarkupfalse%
\isanewline
\ \ \isacommand{hence}\isamarkupfalse%
\ {\isachardoublequoteopen}{\isasymUnion}{\isacharbraceleft}pcp{\isacharunderscore}seq\ C\ S\ n\ {\isacharbar}n{\isachardot}\ n\ {\isasymle}\ Suc\ m{\isacharbraceright}\ {\isacharequal}\ \isanewline
\ \ \ \ \ \ \ \ \ {\isasymUnion}{\isacharbraceleft}pcp{\isacharunderscore}seq\ C\ S\ n\ {\isacharbar}n{\isachardot}\ n\ {\isasymle}\ m{\isacharbraceright}\ {\isasymunion}\ pcp{\isacharunderscore}seq\ C\ S\ {\isacharparenleft}Suc\ m{\isacharparenright}{\isachardoublequoteclose}\ \isacommand{by}\isamarkupfalse%
\ auto\isanewline
\ \ \isacommand{thus}\isamarkupfalse%
\ {\isacharquery}case\ \isacommand{using}\isamarkupfalse%
\ Suc\ pcp{\isacharunderscore}seq{\isacharunderscore}mono\ \isacommand{by}\isamarkupfalse%
\ blast\isanewline
\isacommand{qed}\isamarkupfalse%
\ simp%
\endisatagproof
{\isafoldproof}%
%
\isadelimproof
%
\endisadelimproof
%
\begin{isamarkuptext}%
Finalmente, definamos el límite de las sucesiones presentadas en la definición \isa{{\isadigit{4}}{\isachardot}{\isadigit{1}}{\isachardot}{\isadigit{1}}}.

 \begin{definicion}
  Sea \isa{C} una colección, \isa{S\ {\isasymin}\ C} y \isa{{\isacharbraceleft}S\isactrlsub n{\isacharbraceright}} la sucesión de conjuntos de \isa{C} a partir de \isa{S} según la
  definición \isa{{\isadigit{4}}{\isachardot}{\isadigit{1}}{\isachardot}{\isadigit{1}}}. Se define el \isa{límite\ de\ la\ sucesión\ de\ conjuntos\ de\ C\ a\ partir\ de\ S} como 
  $\bigcup_{n = 0}^{\infty} S_{n}$
 \end{definicion}

  La definición del límite se formaliza utilizando la unión generalizada de Isabelle como sigue.%
\end{isamarkuptext}\isamarkuptrue%
\isacommand{definition}\isamarkupfalse%
\ {\isachardoublequoteopen}pcp{\isacharunderscore}lim\ C\ S\ {\isasymequiv}\ {\isasymUnion}{\isacharbraceleft}pcp{\isacharunderscore}seq\ C\ S\ n{\isacharbar}n{\isachardot}\ True{\isacharbraceright}{\isachardoublequoteclose}%
\begin{isamarkuptext}%
Veamos el primer resultado sobre el límite.

\begin{lema}
  Sea \isa{C} una colección de conjuntos, \isa{S\ {\isasymin}\ C} y \isa{{\isacharbraceleft}S\isactrlsub n{\isacharbraceright}} la sucesión de conjuntos de \isa{C} a partir de
  \isa{S} según la definición \isa{{\isadigit{4}}{\isachardot}{\isadigit{1}}{\isachardot}{\isadigit{1}}}. Entonces, para todo \isa{n\ {\isasymin}\ {\isasymnat}}, se verifica:

  $S_{n} \subseteq \bigcup_{n = 0}^{\infty} S_{n}$
\end{lema}

\begin{demostracion}
  El resultado se obtiene de manera inmediata ya que, para todo \isa{n\ {\isasymin}\ {\isasymnat}}, se verifica que 
  $S_{n} \in \{S_{n}\}_{n = 0}^{\infty}$. Por tanto, es claro que 
  $S_{n} \subseteq \bigcup_{n = 0}^{\infty} S_{n}$.
\end{demostracion}

  Su formalización y demostración detallada en Isabelle se muestran a continuación.%
\end{isamarkuptext}\isamarkuptrue%
\isacommand{lemma}\isamarkupfalse%
\ {\isachardoublequoteopen}pcp{\isacharunderscore}seq\ C\ S\ n\ {\isasymsubseteq}\ pcp{\isacharunderscore}lim\ C\ S{\isachardoublequoteclose}\isanewline
%
\isadelimproof
\ \ %
\endisadelimproof
%
\isatagproof
\isacommand{unfolding}\isamarkupfalse%
\ pcp{\isacharunderscore}lim{\isacharunderscore}def\isanewline
\isacommand{proof}\isamarkupfalse%
\ {\isacharminus}\isanewline
\ \ \isacommand{have}\isamarkupfalse%
\ {\isachardoublequoteopen}n\ {\isasymin}\ {\isacharbraceleft}n\ {\isacharbar}\ n{\isachardot}\ True{\isacharbraceright}{\isachardoublequoteclose}\ \isanewline
\ \ \ \ \isacommand{by}\isamarkupfalse%
\ {\isacharparenleft}simp\ only{\isacharcolon}\ simp{\isacharunderscore}thms{\isacharparenleft}{\isadigit{2}}{\isadigit{1}}{\isacharcomma}{\isadigit{3}}{\isadigit{8}}{\isacharparenright}\ Collect{\isacharunderscore}const\ if{\isacharunderscore}True\ UNIV{\isacharunderscore}I{\isacharparenright}\ \isanewline
\ \ \isacommand{then}\isamarkupfalse%
\ \isacommand{have}\isamarkupfalse%
\ {\isachardoublequoteopen}pcp{\isacharunderscore}seq\ C\ S\ n\ {\isasymin}\ {\isacharparenleft}pcp{\isacharunderscore}seq\ C\ S{\isacharparenright}{\isacharbackquote}{\isacharbraceleft}n\ {\isacharbar}\ n{\isachardot}\ True{\isacharbraceright}{\isachardoublequoteclose}\isanewline
\ \ \ \ \isacommand{by}\isamarkupfalse%
\ {\isacharparenleft}simp\ only{\isacharcolon}\ imageI{\isacharparenright}\isanewline
\ \ \isacommand{then}\isamarkupfalse%
\ \isacommand{have}\isamarkupfalse%
\ {\isachardoublequoteopen}pcp{\isacharunderscore}seq\ C\ S\ n\ {\isasymin}\ {\isacharbraceleft}pcp{\isacharunderscore}seq\ C\ S\ n\ {\isacharbar}\ n{\isachardot}\ True{\isacharbraceright}{\isachardoublequoteclose}\isanewline
\ \ \ \ \isacommand{by}\isamarkupfalse%
\ {\isacharparenleft}simp\ only{\isacharcolon}\ image{\isacharunderscore}Collect\ simp{\isacharunderscore}thms{\isacharparenleft}{\isadigit{4}}{\isadigit{0}}{\isacharparenright}{\isacharparenright}\isanewline
\ \ \isacommand{thus}\isamarkupfalse%
\ {\isachardoublequoteopen}pcp{\isacharunderscore}seq\ C\ S\ n\ {\isasymsubseteq}\ {\isasymUnion}{\isacharbraceleft}pcp{\isacharunderscore}seq\ C\ S\ n\ {\isacharbar}\ n{\isachardot}\ True{\isacharbraceright}{\isachardoublequoteclose}\isanewline
\ \ \ \ \isacommand{by}\isamarkupfalse%
\ {\isacharparenleft}simp\ only{\isacharcolon}\ Union{\isacharunderscore}upper{\isacharparenright}\isanewline
\isacommand{qed}\isamarkupfalse%
%
\endisatagproof
{\isafoldproof}%
%
\isadelimproof
%
\endisadelimproof
%
\begin{isamarkuptext}%
Podemos probarlo de manera automática como sigue.%
\end{isamarkuptext}\isamarkuptrue%
\isacommand{lemma}\isamarkupfalse%
\ pcp{\isacharunderscore}seq{\isacharunderscore}sub{\isacharcolon}\ {\isachardoublequoteopen}pcp{\isacharunderscore}seq\ C\ S\ n\ {\isasymsubseteq}\ pcp{\isacharunderscore}lim\ C\ S{\isachardoublequoteclose}\ \isanewline
%
\isadelimproof
\ \ %
\endisadelimproof
%
\isatagproof
\isacommand{unfolding}\isamarkupfalse%
\ pcp{\isacharunderscore}lim{\isacharunderscore}def\ \isacommand{by}\isamarkupfalse%
\ blast%
\endisatagproof
{\isafoldproof}%
%
\isadelimproof
%
\endisadelimproof
%
\begin{isamarkuptext}%
Mostremos otro resultado. 

  \begin{lema}
    Sea \isa{C} una colección de conjuntos de fórmulas proposicionales, \isa{S\ {\isasymin}\ C} y \isa{{\isacharbraceleft}S\isactrlsub n{\isacharbraceright}} la sucesión de 
    conjuntos de \isa{C} a partir de \isa{S} según la definición \isa{{\isadigit{4}}{\isachardot}{\isadigit{1}}{\isachardot}{\isadigit{1}}}. Si \isa{F} es una fórmula tal que
    $F \in \bigcup_{n = 0}^{\infty} S_{n}$, entonces existe un \isa{k\ {\isasymin}\ {\isasymnat}} tal que \isa{F\ {\isasymin}\ S\isactrlsub k}. 
  \end{lema}

  \begin{demostracion}
    La prueba es inmediata de la definición del límite de la sucesión de conjuntos \isa{{\isacharbraceleft}S\isactrlsub n{\isacharbraceright}}: si
    \isa{F} pertenece a la unión generalizada $\bigcup_{n = 0}^{\infty} S_{n}$, entonces existe algún
    conjunto \isa{S\isactrlsub k} tal que \isa{F\ {\isasymin}\ S\isactrlsub k}. Es decir, existe \isa{k\ {\isasymin}\ {\isasymnat}} tal que \isa{F\ {\isasymin}\ S\isactrlsub k}, como queríamos
    demostrar.
  \end{demostracion} 

  Su prueba detallada en Isabelle/HOL es la siguiente.%
\end{isamarkuptext}\isamarkuptrue%
\isacommand{lemma}\isamarkupfalse%
\ \isanewline
\ \ \isakeyword{assumes}\ {\isachardoublequoteopen}F\ {\isasymin}\ pcp{\isacharunderscore}lim\ C\ S{\isachardoublequoteclose}\isanewline
\ \ \isakeyword{shows}\ {\isachardoublequoteopen}{\isasymexists}k{\isachardot}\ F\ {\isasymin}\ pcp{\isacharunderscore}seq\ C\ S\ k{\isachardoublequoteclose}\ \isanewline
%
\isadelimproof
%
\endisadelimproof
%
\isatagproof
\isacommand{proof}\isamarkupfalse%
\ {\isacharminus}\isanewline
\ \ \isacommand{have}\isamarkupfalse%
\ {\isachardoublequoteopen}F\ {\isasymin}\ {\isasymUnion}{\isacharparenleft}{\isacharparenleft}pcp{\isacharunderscore}seq\ C\ S{\isacharparenright}\ {\isacharbackquote}\ {\isacharbraceleft}n\ {\isacharbar}\ n{\isachardot}\ True{\isacharbraceright}{\isacharparenright}{\isachardoublequoteclose}\isanewline
\ \ \ \ \isacommand{using}\isamarkupfalse%
\ assms\ \isacommand{by}\isamarkupfalse%
\ {\isacharparenleft}simp\ only{\isacharcolon}\ pcp{\isacharunderscore}lim{\isacharunderscore}def\ image{\isacharunderscore}Collect\ simp{\isacharunderscore}thms{\isacharparenleft}{\isadigit{4}}{\isadigit{0}}{\isacharparenright}{\isacharparenright}\isanewline
\ \ \isacommand{then}\isamarkupfalse%
\ \isacommand{have}\isamarkupfalse%
\ {\isachardoublequoteopen}{\isasymexists}k\ {\isasymin}\ {\isacharbraceleft}n{\isachardot}\ True{\isacharbraceright}{\isachardot}\ F\ {\isasymin}\ pcp{\isacharunderscore}seq\ C\ S\ k{\isachardoublequoteclose}\isanewline
\ \ \ \ \isacommand{by}\isamarkupfalse%
\ {\isacharparenleft}simp\ only{\isacharcolon}\ UN{\isacharunderscore}iff\ simp{\isacharunderscore}thms{\isacharparenleft}{\isadigit{4}}{\isadigit{0}}{\isacharparenright}{\isacharparenright}\isanewline
\ \ \isacommand{then}\isamarkupfalse%
\ \isacommand{have}\isamarkupfalse%
\ {\isachardoublequoteopen}{\isasymexists}k\ {\isasymin}\ UNIV{\isachardot}\ F\ {\isasymin}\ pcp{\isacharunderscore}seq\ C\ S\ k{\isachardoublequoteclose}\ \isanewline
\ \ \ \ \isacommand{by}\isamarkupfalse%
\ {\isacharparenleft}simp\ only{\isacharcolon}\ UNIV{\isacharunderscore}def{\isacharparenright}\isanewline
\ \ \isacommand{thus}\isamarkupfalse%
\ {\isachardoublequoteopen}{\isasymexists}k{\isachardot}\ F\ {\isasymin}\ pcp{\isacharunderscore}seq\ C\ S\ k{\isachardoublequoteclose}\ \isanewline
\ \ \ \ \isacommand{by}\isamarkupfalse%
\ {\isacharparenleft}simp\ only{\isacharcolon}\ bex{\isacharunderscore}UNIV{\isacharparenright}\isanewline
\isacommand{qed}\isamarkupfalse%
%
\endisatagproof
{\isafoldproof}%
%
\isadelimproof
%
\endisadelimproof
%
\begin{isamarkuptext}%
Mostremos, a continuación, la demostración automática del resultado.%
\end{isamarkuptext}\isamarkuptrue%
\isacommand{lemma}\isamarkupfalse%
\ pcp{\isacharunderscore}lim{\isacharunderscore}inserted{\isacharunderscore}at{\isacharunderscore}ex{\isacharcolon}\ \isanewline
\ \ \ \ {\isachardoublequoteopen}S{\isacharprime}\ {\isasymin}\ pcp{\isacharunderscore}lim\ C\ S\ {\isasymLongrightarrow}\ {\isasymexists}k{\isachardot}\ S{\isacharprime}\ {\isasymin}\ pcp{\isacharunderscore}seq\ C\ S\ k{\isachardoublequoteclose}\isanewline
%
\isadelimproof
\ \ %
\endisadelimproof
%
\isatagproof
\isacommand{unfolding}\isamarkupfalse%
\ pcp{\isacharunderscore}lim{\isacharunderscore}def\ \isacommand{by}\isamarkupfalse%
\ blast%
\endisatagproof
{\isafoldproof}%
%
\isadelimproof
%
\endisadelimproof
%
\begin{isamarkuptext}%
Por último, veamos la siguiente propiedad sobre conjuntos finitos contenidos en el límite de 
  las sucesiones definido en \isa{{\isadigit{4}}{\isachardot}{\isadigit{1}}{\isachardot}{\isadigit{5}}}.

\begin{lema}
  Sea \isa{C} una colección, \isa{S\ {\isasymin}\ C} y \isa{{\isacharbraceleft}S\isactrlsub n{\isacharbraceright}} la sucesión de conjuntos de \isa{C} a partir de \isa{S} según la
  definición \isa{{\isadigit{4}}{\isachardot}{\isadigit{1}}{\isachardot}{\isadigit{1}}}. Si \isa{S{\isacharprime}} es un conjunto finito tal que \isa{S{\isacharprime}\ {\isasymsubseteq}} $\bigcup_{n = 0}^{\infty} S_{n}$, 
  entonces existe un\\ \isa{k\ {\isasymin}\ {\isasymnat}} tal que \isa{S{\isacharprime}\ {\isasymsubseteq}\ S\isactrlsub k}.
\end{lema}

\begin{demostracion}
  La prueba del resultado se realiza por inducción sobre la estructura recursiva de los conjuntos 
  finitos.

  En primer lugar, probemos el caso base correspondiente al conjunto vacío. Supongamos que \isa{{\isacharbraceleft}{\isacharbraceright}} está 
  contenido en el límite de la sucesión de conjuntos de \isa{C} a partir de \isa{S}. Como \isa{{\isacharbraceleft}{\isacharbraceright}} es 
  subconjunto de todo conjunto, en particular lo es de \isa{S\ {\isacharequal}\ S\isactrlsub {\isadigit{0}}}, probando así el primer caso.

  Por otra parte, probemos el paso de inducción. Sea \isa{S{\isacharprime}} un conjunto finito verificando la 
  hipótesis de inducción: si \isa{S{\isacharprime}} está contenido en el límite de la sucesión de conjuntos de 
  \isa{C} a partir de \isa{S}, entonces también está contenido en algún \isa{S\isactrlsub k\isactrlsub {\isacharprime}} para cierto \isa{k{\isacharprime}\ {\isasymin}\ {\isasymnat}}. Sea 
  \isa{F} una fórmula tal que \isa{F\ {\isasymnotin}\ S{\isacharprime}}. Vamos a probar que si \isa{{\isacharbraceleft}F{\isacharbraceright}\ {\isasymunion}\ S{\isacharprime}} está contenido en el límite, 
  entonces está contenido en \isa{S\isactrlsub k} para algún \isa{k\ {\isasymin}\ {\isasymnat}}. 

  Como hemos supuesto que \isa{{\isacharbraceleft}F{\isacharbraceright}\ {\isasymunion}\ S{\isacharprime}} está contenido en el límite, entonces se verifica que \isa{F}
  pertenece al límite y \isa{S{\isacharprime}} está contenido en él. Por el lema \isa{{\isadigit{4}}{\isachardot}{\isadigit{1}}{\isachardot}{\isadigit{7}}}, como \isa{F} pertenece al 
  límite, deducimos que existe un \isa{k\ {\isasymin}\ {\isasymnat}} tal que \isa{F\ {\isasymin}\ S\isactrlsub k}. Por otro lado, como \isa{S{\isacharprime}} está contenido
  en el límite, por hipótesis de inducción existe algún \isa{k{\isacharprime}\ {\isasymin}\ {\isasymnat}} tal que \isa{S{\isacharprime}\ {\isasymsubseteq}\ S\isactrlsub k\isactrlsub {\isacharprime}}. El resultado 
  se obtiene considerando el máximo entre \isa{k} y \isa{k{\isacharprime}}, que notaremos por \isa{k{\isacharprime}{\isacharprime}}. En efecto, por la 
  monotonía de la sucesión, se verifica que tanto \isa{S\isactrlsub k} como \isa{S\isactrlsub k\isactrlsub {\isacharprime}} están contenidos en \isa{S\isactrlsub k\isactrlsub {\isacharprime}\isactrlsub {\isacharprime}}. De este 
  modo, como \isa{S{\isacharprime}\ {\isasymsubseteq}\ S\isactrlsub k\isactrlsub {\isacharprime}}, por la transitividad de la contención de conjuntos se tiene que 
  \isa{S{\isacharprime}\ {\isasymsubseteq}\ S\isactrlsub k\isactrlsub {\isacharprime}\isactrlsub {\isacharprime}}. Además, como \isa{F\ {\isasymin}\ S\isactrlsub k}, se tiene que \isa{F\ {\isasymin}\ S\isactrlsub k\isactrlsub {\isacharprime}\isactrlsub {\isacharprime}}. Por lo tanto, \isa{{\isacharbraceleft}F{\isacharbraceright}\ {\isasymunion}\ S{\isacharprime}\ {\isasymsubseteq}\ S\isactrlsub k\isactrlsub {\isacharprime}\isactrlsub {\isacharprime}}, como 
  queríamos demostrar. 
\end{demostracion}

  Procedamos con la demostración detallada en Isabelle.%
\end{isamarkuptext}\isamarkuptrue%
\isacommand{lemma}\isamarkupfalse%
\ \isanewline
\ \ \isakeyword{assumes}\ {\isachardoublequoteopen}finite\ S{\isacharprime}{\isachardoublequoteclose}\isanewline
\ \ \ \ \ \ \ \ \ \ {\isachardoublequoteopen}S{\isacharprime}\ {\isasymsubseteq}\ pcp{\isacharunderscore}lim\ C\ S{\isachardoublequoteclose}\isanewline
\ \ \ \ \ \ \ \ \isakeyword{shows}\ {\isachardoublequoteopen}{\isasymexists}k{\isachardot}\ S{\isacharprime}\ {\isasymsubseteq}\ pcp{\isacharunderscore}seq\ C\ S\ k{\isachardoublequoteclose}\isanewline
%
\isadelimproof
\ \ %
\endisadelimproof
%
\isatagproof
\isacommand{using}\isamarkupfalse%
\ assms\isanewline
\isacommand{proof}\isamarkupfalse%
\ {\isacharparenleft}induction\ S{\isacharprime}\ rule{\isacharcolon}\ finite{\isacharunderscore}induct{\isacharparenright}\isanewline
\ \ \isacommand{case}\isamarkupfalse%
\ empty\isanewline
\ \ \isacommand{have}\isamarkupfalse%
\ {\isachardoublequoteopen}pcp{\isacharunderscore}seq\ C\ S\ {\isadigit{0}}\ {\isacharequal}\ S{\isachardoublequoteclose}\isanewline
\ \ \ \ \isacommand{by}\isamarkupfalse%
\ {\isacharparenleft}simp\ only{\isacharcolon}\ pcp{\isacharunderscore}seq{\isachardot}simps{\isacharparenleft}{\isadigit{1}}{\isacharparenright}{\isacharparenright}\isanewline
\ \ \isacommand{have}\isamarkupfalse%
\ {\isachardoublequoteopen}{\isacharbraceleft}{\isacharbraceright}\ {\isasymsubseteq}\ S{\isachardoublequoteclose}\isanewline
\ \ \ \ \isacommand{by}\isamarkupfalse%
\ {\isacharparenleft}rule\ order{\isacharunderscore}bot{\isacharunderscore}class{\isachardot}bot{\isachardot}extremum{\isacharparenright}\isanewline
\ \ \isacommand{then}\isamarkupfalse%
\ \isacommand{have}\isamarkupfalse%
\ {\isachardoublequoteopen}{\isacharbraceleft}{\isacharbraceright}\ {\isasymsubseteq}\ pcp{\isacharunderscore}seq\ C\ S\ {\isadigit{0}}{\isachardoublequoteclose}\isanewline
\ \ \ \ \isacommand{by}\isamarkupfalse%
\ {\isacharparenleft}simp\ only{\isacharcolon}\ {\isacartoucheopen}pcp{\isacharunderscore}seq\ C\ S\ {\isadigit{0}}\ {\isacharequal}\ S{\isacartoucheclose}{\isacharparenright}\isanewline
\ \ \isacommand{then}\isamarkupfalse%
\ \isacommand{show}\isamarkupfalse%
\ {\isacharquery}case\ \isanewline
\ \ \ \ \isacommand{by}\isamarkupfalse%
\ {\isacharparenleft}rule\ exI{\isacharparenright}\isanewline
\isacommand{next}\isamarkupfalse%
\isanewline
\ \ \isacommand{case}\isamarkupfalse%
\ {\isacharparenleft}insert\ F\ S{\isacharprime}{\isacharparenright}\isanewline
\ \ \isacommand{then}\isamarkupfalse%
\ \isacommand{have}\isamarkupfalse%
\ {\isachardoublequoteopen}insert\ F\ S{\isacharprime}\ {\isasymsubseteq}\ pcp{\isacharunderscore}lim\ C\ S{\isachardoublequoteclose}\isanewline
\ \ \ \ \isacommand{by}\isamarkupfalse%
\ {\isacharparenleft}simp\ only{\isacharcolon}\ insert{\isachardot}prems{\isacharparenright}\isanewline
\ \ \isacommand{then}\isamarkupfalse%
\ \isacommand{have}\isamarkupfalse%
\ C{\isacharcolon}{\isachardoublequoteopen}F\ {\isasymin}\ {\isacharparenleft}pcp{\isacharunderscore}lim\ C\ S{\isacharparenright}\ {\isasymand}\ S{\isacharprime}\ {\isasymsubseteq}\ pcp{\isacharunderscore}lim\ C\ S{\isachardoublequoteclose}\isanewline
\ \ \ \ \isacommand{by}\isamarkupfalse%
\ {\isacharparenleft}simp\ only{\isacharcolon}\ insert{\isacharunderscore}subset{\isacharparenright}\ \isanewline
\ \ \isacommand{then}\isamarkupfalse%
\ \isacommand{have}\isamarkupfalse%
\ {\isachardoublequoteopen}S{\isacharprime}\ {\isasymsubseteq}\ pcp{\isacharunderscore}lim\ C\ S{\isachardoublequoteclose}\isanewline
\ \ \ \ \isacommand{by}\isamarkupfalse%
\ {\isacharparenleft}rule\ conjunct{\isadigit{2}}{\isacharparenright}\isanewline
\ \ \isacommand{then}\isamarkupfalse%
\ \isacommand{have}\isamarkupfalse%
\ EX{\isadigit{1}}{\isacharcolon}{\isachardoublequoteopen}{\isasymexists}k{\isachardot}\ S{\isacharprime}\ {\isasymsubseteq}\ pcp{\isacharunderscore}seq\ C\ S\ k{\isachardoublequoteclose}\isanewline
\ \ \ \ \isacommand{by}\isamarkupfalse%
\ {\isacharparenleft}simp\ only{\isacharcolon}\ insert{\isachardot}IH{\isacharparenright}\isanewline
\ \ \isacommand{obtain}\isamarkupfalse%
\ k{\isadigit{1}}\ \isakeyword{where}\ {\isachardoublequoteopen}S{\isacharprime}\ {\isasymsubseteq}\ pcp{\isacharunderscore}seq\ C\ S\ k{\isadigit{1}}{\isachardoublequoteclose}\isanewline
\ \ \ \ \isacommand{using}\isamarkupfalse%
\ EX{\isadigit{1}}\ \isacommand{by}\isamarkupfalse%
\ {\isacharparenleft}rule\ exE{\isacharparenright}\isanewline
\ \ \isacommand{have}\isamarkupfalse%
\ {\isachardoublequoteopen}F\ {\isasymin}\ pcp{\isacharunderscore}lim\ C\ S{\isachardoublequoteclose}\isanewline
\ \ \ \ \isacommand{using}\isamarkupfalse%
\ C\ \isacommand{by}\isamarkupfalse%
\ {\isacharparenleft}rule\ conjunct{\isadigit{1}}{\isacharparenright}\isanewline
\ \ \isacommand{then}\isamarkupfalse%
\ \isacommand{have}\isamarkupfalse%
\ EX{\isadigit{2}}{\isacharcolon}{\isachardoublequoteopen}{\isasymexists}k{\isachardot}\ F\ {\isasymin}\ pcp{\isacharunderscore}seq\ C\ S\ k{\isachardoublequoteclose}\isanewline
\ \ \ \ \isacommand{by}\isamarkupfalse%
\ {\isacharparenleft}rule\ pcp{\isacharunderscore}lim{\isacharunderscore}inserted{\isacharunderscore}at{\isacharunderscore}ex{\isacharparenright}\isanewline
\ \ \isacommand{obtain}\isamarkupfalse%
\ k{\isadigit{2}}\ \isakeyword{where}\ {\isachardoublequoteopen}F\ {\isasymin}\ pcp{\isacharunderscore}seq\ C\ S\ k{\isadigit{2}}{\isachardoublequoteclose}\ \isanewline
\ \ \ \ \isacommand{using}\isamarkupfalse%
\ EX{\isadigit{2}}\ \isacommand{by}\isamarkupfalse%
\ {\isacharparenleft}rule\ exE{\isacharparenright}\isanewline
\ \ \isacommand{have}\isamarkupfalse%
\ {\isachardoublequoteopen}k{\isadigit{1}}\ {\isasymle}\ max\ k{\isadigit{1}}\ k{\isadigit{2}}{\isachardoublequoteclose}\isanewline
\ \ \ \ \isacommand{by}\isamarkupfalse%
\ {\isacharparenleft}simp\ only{\isacharcolon}\ linorder{\isacharunderscore}class{\isachardot}max{\isachardot}cobounded{\isadigit{1}}{\isacharparenright}\isanewline
\ \ \isacommand{then}\isamarkupfalse%
\ \isacommand{have}\isamarkupfalse%
\ {\isachardoublequoteopen}pcp{\isacharunderscore}seq\ C\ S\ k{\isadigit{1}}\ {\isasymsubseteq}\ pcp{\isacharunderscore}seq\ C\ S\ {\isacharparenleft}max\ k{\isadigit{1}}\ k{\isadigit{2}}{\isacharparenright}{\isachardoublequoteclose}\isanewline
\ \ \ \ \isacommand{by}\isamarkupfalse%
\ {\isacharparenleft}rule\ pcp{\isacharunderscore}seq{\isacharunderscore}mono{\isacharparenright}\isanewline
\ \ \isacommand{have}\isamarkupfalse%
\ {\isachardoublequoteopen}k{\isadigit{2}}\ {\isasymle}\ max\ k{\isadigit{1}}\ k{\isadigit{2}}{\isachardoublequoteclose}\isanewline
\ \ \ \ \isacommand{by}\isamarkupfalse%
\ {\isacharparenleft}simp\ only{\isacharcolon}\ linorder{\isacharunderscore}class{\isachardot}max{\isachardot}cobounded{\isadigit{2}}{\isacharparenright}\isanewline
\ \ \isacommand{then}\isamarkupfalse%
\ \isacommand{have}\isamarkupfalse%
\ {\isachardoublequoteopen}pcp{\isacharunderscore}seq\ C\ S\ k{\isadigit{2}}\ {\isasymsubseteq}\ pcp{\isacharunderscore}seq\ C\ S\ {\isacharparenleft}max\ k{\isadigit{1}}\ k{\isadigit{2}}{\isacharparenright}{\isachardoublequoteclose}\isanewline
\ \ \ \ \isacommand{by}\isamarkupfalse%
\ {\isacharparenleft}rule\ pcp{\isacharunderscore}seq{\isacharunderscore}mono{\isacharparenright}\isanewline
\ \ \isacommand{have}\isamarkupfalse%
\ {\isachardoublequoteopen}S{\isacharprime}\ {\isasymsubseteq}\ pcp{\isacharunderscore}seq\ C\ S\ {\isacharparenleft}max\ k{\isadigit{1}}\ k{\isadigit{2}}{\isacharparenright}{\isachardoublequoteclose}\isanewline
\ \ \ \ \isacommand{using}\isamarkupfalse%
\ {\isacartoucheopen}S{\isacharprime}\ {\isasymsubseteq}\ pcp{\isacharunderscore}seq\ C\ S\ k{\isadigit{1}}{\isacartoucheclose}\ {\isacartoucheopen}pcp{\isacharunderscore}seq\ C\ S\ k{\isadigit{1}}\ {\isasymsubseteq}\ pcp{\isacharunderscore}seq\ C\ S\ {\isacharparenleft}max\ k{\isadigit{1}}\ k{\isadigit{2}}{\isacharparenright}{\isacartoucheclose}\ \isacommand{by}\isamarkupfalse%
\ {\isacharparenleft}rule\ subset{\isacharunderscore}trans{\isacharparenright}\isanewline
\ \ \isacommand{have}\isamarkupfalse%
\ {\isachardoublequoteopen}F\ {\isasymin}\ pcp{\isacharunderscore}seq\ C\ S\ {\isacharparenleft}max\ k{\isadigit{1}}\ k{\isadigit{2}}{\isacharparenright}{\isachardoublequoteclose}\isanewline
\ \ \ \ \isacommand{using}\isamarkupfalse%
\ {\isacartoucheopen}F\ {\isasymin}\ pcp{\isacharunderscore}seq\ C\ S\ k{\isadigit{2}}{\isacartoucheclose}\ {\isacartoucheopen}pcp{\isacharunderscore}seq\ C\ S\ k{\isadigit{2}}\ {\isasymsubseteq}\ pcp{\isacharunderscore}seq\ C\ S\ {\isacharparenleft}max\ k{\isadigit{1}}\ k{\isadigit{2}}{\isacharparenright}{\isacartoucheclose}\ \isacommand{by}\isamarkupfalse%
\ {\isacharparenleft}rule\ rev{\isacharunderscore}subsetD{\isacharparenright}\isanewline
\ \ \isacommand{then}\isamarkupfalse%
\ \isacommand{have}\isamarkupfalse%
\ {\isadigit{1}}{\isacharcolon}{\isachardoublequoteopen}insert\ F\ S{\isacharprime}\ {\isasymsubseteq}\ pcp{\isacharunderscore}seq\ C\ S\ {\isacharparenleft}max\ k{\isadigit{1}}\ k{\isadigit{2}}{\isacharparenright}{\isachardoublequoteclose}\isanewline
\ \ \ \ \isacommand{using}\isamarkupfalse%
\ {\isacartoucheopen}S{\isacharprime}\ {\isasymsubseteq}\ pcp{\isacharunderscore}seq\ C\ S\ {\isacharparenleft}max\ k{\isadigit{1}}\ k{\isadigit{2}}{\isacharparenright}{\isacartoucheclose}\ \isacommand{by}\isamarkupfalse%
\ {\isacharparenleft}simp\ only{\isacharcolon}\ insert{\isacharunderscore}subset{\isacharparenright}\isanewline
\ \ \isacommand{thus}\isamarkupfalse%
\ {\isacharquery}case\isanewline
\ \ \ \ \isacommand{by}\isamarkupfalse%
\ {\isacharparenleft}rule\ exI{\isacharparenright}\isanewline
\isacommand{qed}\isamarkupfalse%
%
\endisatagproof
{\isafoldproof}%
%
\isadelimproof
%
\endisadelimproof
%
\begin{isamarkuptext}%
Finalmente, su demostración automática en Isabelle/HOL es la siguiente.%
\end{isamarkuptext}\isamarkuptrue%
\isacommand{lemma}\isamarkupfalse%
\ finite{\isacharunderscore}pcp{\isacharunderscore}lim{\isacharunderscore}EX{\isacharcolon}\isanewline
\ \ \isakeyword{assumes}\ {\isachardoublequoteopen}finite\ S{\isacharprime}{\isachardoublequoteclose}\isanewline
\ \ \ \ \ \ \ \ \ \ {\isachardoublequoteopen}S{\isacharprime}\ {\isasymsubseteq}\ pcp{\isacharunderscore}lim\ C\ S{\isachardoublequoteclose}\isanewline
\ \ \ \ \ \ \ \ \isakeyword{shows}\ {\isachardoublequoteopen}{\isasymexists}k{\isachardot}\ S{\isacharprime}\ {\isasymsubseteq}\ pcp{\isacharunderscore}seq\ C\ S\ k{\isachardoublequoteclose}\isanewline
%
\isadelimproof
\ \ %
\endisadelimproof
%
\isatagproof
\isacommand{using}\isamarkupfalse%
\ assms\isanewline
\isacommand{proof}\isamarkupfalse%
{\isacharparenleft}induction\ S{\isacharprime}\ rule{\isacharcolon}\ finite{\isacharunderscore}induct{\isacharparenright}\ \isanewline
\ \ \isacommand{case}\isamarkupfalse%
\ {\isacharparenleft}insert\ F\ S{\isacharprime}{\isacharparenright}\isanewline
\ \ \isacommand{hence}\isamarkupfalse%
\ {\isachardoublequoteopen}{\isasymexists}k{\isachardot}\ S{\isacharprime}\ {\isasymsubseteq}\ pcp{\isacharunderscore}seq\ C\ S\ k{\isachardoublequoteclose}\ \isacommand{by}\isamarkupfalse%
\ fast\isanewline
\ \ \isacommand{then}\isamarkupfalse%
\ \isacommand{guess}\isamarkupfalse%
\ k{\isadigit{1}}\ \isacommand{{\isachardot}{\isachardot}}\isamarkupfalse%
\isanewline
\ \ \isacommand{moreover}\isamarkupfalse%
\ \isacommand{obtain}\isamarkupfalse%
\ k{\isadigit{2}}\ \isakeyword{where}\ {\isachardoublequoteopen}F\ {\isasymin}\ pcp{\isacharunderscore}seq\ C\ S\ k{\isadigit{2}}{\isachardoublequoteclose}\isanewline
\ \ \ \ \isacommand{by}\isamarkupfalse%
\ {\isacharparenleft}meson\ pcp{\isacharunderscore}lim{\isacharunderscore}inserted{\isacharunderscore}at{\isacharunderscore}ex\ insert{\isachardot}prems\ insert{\isacharunderscore}subset{\isacharparenright}\isanewline
\ \ \isacommand{ultimately}\isamarkupfalse%
\ \isacommand{have}\isamarkupfalse%
\ {\isachardoublequoteopen}insert\ F\ S{\isacharprime}\ {\isasymsubseteq}\ pcp{\isacharunderscore}seq\ C\ S\ {\isacharparenleft}max\ k{\isadigit{1}}\ k{\isadigit{2}}{\isacharparenright}{\isachardoublequoteclose}\isanewline
\ \ \ \ \isacommand{by}\isamarkupfalse%
\ {\isacharparenleft}meson\ pcp{\isacharunderscore}seq{\isacharunderscore}mono\ dual{\isacharunderscore}order{\isachardot}trans\ insert{\isacharunderscore}subset\ max{\isachardot}bounded{\isacharunderscore}iff\ order{\isacharunderscore}refl\ subsetCE{\isacharparenright}\isanewline
\ \ \isacommand{thus}\isamarkupfalse%
\ {\isacharquery}case\ \isacommand{by}\isamarkupfalse%
\ blast\isanewline
\isacommand{qed}\isamarkupfalse%
\ simp%
\endisatagproof
{\isafoldproof}%
%
\isadelimproof
%
\endisadelimproof
%
\isadelimdocument
%
\endisadelimdocument
%
\isatagdocument
%
\isamarkupsection{El Teorema de Existencia de Modelo%
}
\isamarkuptrue%
%
\endisatagdocument
{\isafolddocument}%
%
\isadelimdocument
%
\endisadelimdocument
%
\begin{isamarkuptext}%
En esta sección demostraremos el \isa{Teorema\ de\ Existencia\ de\ Modelo}, que prueba 
  que todo conjunto de fórmulas \isa{S} perteneciente a una colección \isa{C} que verifique la propiedad de 
  consistencia proposicional es satisfacible. Para probarlo, extenderemos la colección \isa{C} a otra \isa{C{\isacharprime}} 
  que tenga la propiedad de consistencia proposicional, sea cerrada bajo subconjuntos y sea de 
  carácter finito. Pare ello, introduciremos distintos resultados sobre colecciones \isa{C{\isacharprime}} con 
  las características anteriores. En primer lugar, probaremos que el límite de la sucesión definida
  según la definición \isa{{\isadigit{4}}{\isachardot}{\isadigit{1}}{\isachardot}{\isadigit{1}}} a partir de \isa{C{\isacharprime}} pertenece a \isa{C{\isacharprime}}. De hecho, probaremos que 
  dicho límite es un elemento maximal de la colección que lo define si esta es cerrada bajo 
  subconjuntos y verifica la propiedad de consistencia proposicional. Por otra parte demostraremos 
  que el límite es un conjunto de \isa{Hintikka} si está definido a partir de una colección \isa{C{\isacharprime}} con las 
  propiedades descritas. Por tanto, por el \isa{Teorema\ de\ Hintikka}, en las condiciones anteriores el 
  límite es un conjunto satisfacible. Finalmente, como \isa{S\ {\isasymin}\ C} pertenece también a la extensión 
  \isa{C{\isacharprime}}, se verifica que está contenido en el límite de la sucesión definida según la definición 
  \isa{{\isadigit{4}}{\isachardot}{\isadigit{1}}{\isachardot}{\isadigit{1}}} a partir de \isa{C{\isacharprime}} y \isa{S\ {\isasymin}\ C{\isacharprime}}. Por tanto, quedará demostrada la satisfacibilidad del 
  conjunto \isa{S} al heredarla por contención del límite, lo que prueba el \isa{Teorema\ de\ Existencia\ de\ Modelo}.

  En primer lugar, probemos que si \isa{C} es una colección que verifica la propiedad de 
  consistencia proposicional, es cerrada bajo subconjuntos y es de carácter finito, entonces el 
  límite de toda sucesión de conjuntos de \isa{C} según la definición \isa{{\isadigit{4}}{\isachardot}{\isadigit{1}}{\isachardot}{\isadigit{1}}} pertenece a \isa{C}.

  \begin{lema}
    Sea \isa{C} una colección de conjuntos que verifica la propiedad de consistencia proposicional, es 
    cerrada bajo subconjuntos y es de carácter finito. Sea \isa{S\ {\isasymin}\ C} y \isa{{\isacharbraceleft}S\isactrlsub n{\isacharbraceright}} la sucesión de conjuntos
    de \isa{C} a partir de \isa{S} según la definición \isa{{\isadigit{4}}{\isachardot}{\isadigit{1}}{\isachardot}{\isadigit{1}}}. Entonces, el límite de la sucesión está en
    \isa{C}.
  \end{lema}

  \begin{demostracion}
    Por definición, como \isa{C} es de carácter finito, para todo conjunto son equivalentes:
    \begin{enumerate}
      \item El conjunto pertenece a \isa{C}.
      \item Todo subconjunto finito suyo pertenece a \isa{C}.
    \end{enumerate}

    De este modo, para demostrar que el límite de la sucesión \isa{{\isacharbraceleft}S\isactrlsub n{\isacharbraceright}} pertenece a \isa{C}, basta probar
    que todo subconjunto finito suyo está en \isa{C}.

    Sea \isa{S{\isacharprime}} un subconjunto finito del límite de la sucesión. Por el lema \isa{{\isadigit{1}}{\isachardot}{\isadigit{4}}{\isachardot}{\isadigit{8}}}, existe un
    \isa{k\ {\isasymin}\ {\isasymnat}} tal que \isa{S{\isacharprime}\ {\isasymsubseteq}\ S\isactrlsub k}. Por tanto, como \isa{S\isactrlsub k\ {\isasymin}\ C} para todo \isa{k\ {\isasymin}\ {\isasymnat}} y \isa{C} es cerrada bajo
    subconjuntos, por definición se tiene que \isa{S{\isacharprime}\ {\isasymin}\ C}, como queríamos demostrar.
  \end{demostracion}

  En Isabelle se formaliza y demuestra detalladamente como sigue.%
\end{isamarkuptext}\isamarkuptrue%
\isacommand{lemma}\isamarkupfalse%
\isanewline
\ \ \isakeyword{assumes}\ {\isachardoublequoteopen}pcp\ C{\isachardoublequoteclose}\isanewline
\ \ \ \ \ \ \ \ \ \ {\isachardoublequoteopen}S\ {\isasymin}\ C{\isachardoublequoteclose}\isanewline
\ \ \ \ \ \ \ \ \ \ {\isachardoublequoteopen}subset{\isacharunderscore}closed\ C{\isachardoublequoteclose}\isanewline
\ \ \ \ \ \ \ \ \ \ {\isachardoublequoteopen}finite{\isacharunderscore}character\ C{\isachardoublequoteclose}\isanewline
\ \ \isakeyword{shows}\ {\isachardoublequoteopen}pcp{\isacharunderscore}lim\ C\ S\ {\isasymin}\ C{\isachardoublequoteclose}\ \isanewline
%
\isadelimproof
%
\endisadelimproof
%
\isatagproof
\isacommand{proof}\isamarkupfalse%
\ {\isacharminus}\isanewline
\ \ \isacommand{have}\isamarkupfalse%
\ {\isachardoublequoteopen}{\isasymforall}S{\isachardot}\ S\ {\isasymin}\ C\ {\isasymlongleftrightarrow}\ {\isacharparenleft}{\isasymforall}S{\isacharprime}\ {\isasymsubseteq}\ S{\isachardot}\ finite\ S{\isacharprime}\ {\isasymlongrightarrow}\ S{\isacharprime}\ {\isasymin}\ C{\isacharparenright}{\isachardoublequoteclose}\isanewline
\ \ \ \ \isacommand{using}\isamarkupfalse%
\ assms{\isacharparenleft}{\isadigit{4}}{\isacharparenright}\ \isacommand{unfolding}\isamarkupfalse%
\ finite{\isacharunderscore}character{\isacharunderscore}def\ \isacommand{by}\isamarkupfalse%
\ this\isanewline
\ \ \isacommand{then}\isamarkupfalse%
\ \isacommand{have}\isamarkupfalse%
\ FC{\isadigit{1}}{\isacharcolon}{\isachardoublequoteopen}pcp{\isacharunderscore}lim\ C\ S\ {\isasymin}\ C\ {\isasymlongleftrightarrow}\ {\isacharparenleft}{\isasymforall}S{\isacharprime}\ {\isasymsubseteq}\ {\isacharparenleft}pcp{\isacharunderscore}lim\ C\ S{\isacharparenright}{\isachardot}\ finite\ S{\isacharprime}\ {\isasymlongrightarrow}\ S{\isacharprime}\ {\isasymin}\ C{\isacharparenright}{\isachardoublequoteclose}\isanewline
\ \ \ \ \isacommand{by}\isamarkupfalse%
\ {\isacharparenleft}rule\ allE{\isacharparenright}\isanewline
\ \ \isacommand{have}\isamarkupfalse%
\ SC{\isacharcolon}{\isachardoublequoteopen}{\isasymforall}S\ {\isasymin}\ C{\isachardot}\ {\isasymforall}S{\isacharprime}{\isasymsubseteq}S{\isachardot}\ S{\isacharprime}\ {\isasymin}\ C{\isachardoublequoteclose}\isanewline
\ \ \ \ \isacommand{using}\isamarkupfalse%
\ assms{\isacharparenleft}{\isadigit{3}}{\isacharparenright}\ \isacommand{unfolding}\isamarkupfalse%
\ subset{\isacharunderscore}closed{\isacharunderscore}def\ \isacommand{by}\isamarkupfalse%
\ this\isanewline
\ \ \isacommand{have}\isamarkupfalse%
\ FC{\isadigit{2}}{\isacharcolon}{\isachardoublequoteopen}{\isasymforall}S{\isacharprime}\ {\isasymsubseteq}\ pcp{\isacharunderscore}lim\ C\ S{\isachardot}\ finite\ S{\isacharprime}\ {\isasymlongrightarrow}\ S{\isacharprime}\ {\isasymin}\ C{\isachardoublequoteclose}\isanewline
\ \ \isacommand{proof}\isamarkupfalse%
\ {\isacharparenleft}rule\ sallI{\isacharparenright}\isanewline
\ \ \ \ \isacommand{fix}\isamarkupfalse%
\ S{\isacharprime}\ {\isacharcolon}{\isacharcolon}\ {\isachardoublequoteopen}{\isacharprime}a\ formula\ set{\isachardoublequoteclose}\isanewline
\ \ \ \ \isacommand{assume}\isamarkupfalse%
\ {\isachardoublequoteopen}S{\isacharprime}\ {\isasymsubseteq}\ pcp{\isacharunderscore}lim\ C\ S{\isachardoublequoteclose}\isanewline
\ \ \ \ \isacommand{show}\isamarkupfalse%
\ {\isachardoublequoteopen}finite\ S{\isacharprime}\ {\isasymlongrightarrow}\ S{\isacharprime}\ {\isasymin}\ C{\isachardoublequoteclose}\isanewline
\ \ \ \ \isacommand{proof}\isamarkupfalse%
\ {\isacharparenleft}rule\ impI{\isacharparenright}\isanewline
\ \ \ \ \ \ \isacommand{assume}\isamarkupfalse%
\ {\isachardoublequoteopen}finite\ S{\isacharprime}{\isachardoublequoteclose}\isanewline
\ \ \ \ \ \ \isacommand{then}\isamarkupfalse%
\ \isacommand{have}\isamarkupfalse%
\ EX{\isacharcolon}{\isachardoublequoteopen}{\isasymexists}k{\isachardot}\ S{\isacharprime}\ {\isasymsubseteq}\ pcp{\isacharunderscore}seq\ C\ S\ k{\isachardoublequoteclose}\ \isanewline
\ \ \ \ \ \ \ \ \isacommand{using}\isamarkupfalse%
\ {\isacartoucheopen}S{\isacharprime}\ {\isasymsubseteq}\ pcp{\isacharunderscore}lim\ C\ S{\isacartoucheclose}\ \isacommand{by}\isamarkupfalse%
\ {\isacharparenleft}rule\ finite{\isacharunderscore}pcp{\isacharunderscore}lim{\isacharunderscore}EX{\isacharparenright}\isanewline
\ \ \ \ \ \ \isacommand{obtain}\isamarkupfalse%
\ k\ \isakeyword{where}\ {\isachardoublequoteopen}S{\isacharprime}\ {\isasymsubseteq}\ pcp{\isacharunderscore}seq\ C\ S\ k{\isachardoublequoteclose}\isanewline
\ \ \ \ \ \ \ \ \isacommand{using}\isamarkupfalse%
\ EX\ \isacommand{by}\isamarkupfalse%
\ {\isacharparenleft}rule\ exE{\isacharparenright}\isanewline
\ \ \ \ \ \ \isacommand{have}\isamarkupfalse%
\ {\isachardoublequoteopen}pcp{\isacharunderscore}seq\ C\ S\ k\ {\isasymin}\ C{\isachardoublequoteclose}\isanewline
\ \ \ \ \ \ \ \ \isacommand{using}\isamarkupfalse%
\ assms{\isacharparenleft}{\isadigit{1}}{\isacharparenright}\ assms{\isacharparenleft}{\isadigit{2}}{\isacharparenright}\ \isacommand{by}\isamarkupfalse%
\ {\isacharparenleft}rule\ pcp{\isacharunderscore}seq{\isacharunderscore}in{\isacharparenright}\isanewline
\ \ \ \ \ \ \isacommand{have}\isamarkupfalse%
\ {\isachardoublequoteopen}{\isasymforall}S{\isacharprime}\ {\isasymsubseteq}\ {\isacharparenleft}pcp{\isacharunderscore}seq\ C\ S\ k{\isacharparenright}{\isachardot}\ S{\isacharprime}\ {\isasymin}\ C{\isachardoublequoteclose}\isanewline
\ \ \ \ \ \ \ \ \isacommand{using}\isamarkupfalse%
\ SC\ {\isacartoucheopen}pcp{\isacharunderscore}seq\ C\ S\ k\ {\isasymin}\ C{\isacartoucheclose}\ \isacommand{by}\isamarkupfalse%
\ {\isacharparenleft}rule\ bspec{\isacharparenright}\isanewline
\ \ \ \ \ \ \isacommand{thus}\isamarkupfalse%
\ {\isachardoublequoteopen}S{\isacharprime}\ {\isasymin}\ C{\isachardoublequoteclose}\isanewline
\ \ \ \ \ \ \ \ \isacommand{using}\isamarkupfalse%
\ {\isacartoucheopen}S{\isacharprime}\ {\isasymsubseteq}\ pcp{\isacharunderscore}seq\ C\ S\ k{\isacartoucheclose}\ \isacommand{by}\isamarkupfalse%
\ {\isacharparenleft}rule\ sspec{\isacharparenright}\isanewline
\ \ \ \ \isacommand{qed}\isamarkupfalse%
\isanewline
\ \ \isacommand{qed}\isamarkupfalse%
\isanewline
\ \ \isacommand{show}\isamarkupfalse%
\ {\isachardoublequoteopen}pcp{\isacharunderscore}lim\ C\ S\ {\isasymin}\ C{\isachardoublequoteclose}\ \isanewline
\ \ \ \ \isacommand{using}\isamarkupfalse%
\ FC{\isadigit{1}}\ FC{\isadigit{2}}\ \isacommand{by}\isamarkupfalse%
\ {\isacharparenleft}rule\ forw{\isacharunderscore}subst{\isacharparenright}\isanewline
\isacommand{qed}\isamarkupfalse%
%
\endisatagproof
{\isafoldproof}%
%
\isadelimproof
%
\endisadelimproof
%
\begin{isamarkuptext}%
Por otra parte, podemos dar una prueba automática del resultado.%
\end{isamarkuptext}\isamarkuptrue%
\isacommand{lemma}\isamarkupfalse%
\ pcp{\isacharunderscore}lim{\isacharunderscore}in{\isacharcolon}\isanewline
\ \ \isakeyword{assumes}\ c{\isacharcolon}\ {\isachardoublequoteopen}pcp\ C{\isachardoublequoteclose}\isanewline
\ \ \isakeyword{and}\ el{\isacharcolon}\ {\isachardoublequoteopen}S\ {\isasymin}\ C{\isachardoublequoteclose}\isanewline
\ \ \isakeyword{and}\ sc{\isacharcolon}\ {\isachardoublequoteopen}subset{\isacharunderscore}closed\ C{\isachardoublequoteclose}\isanewline
\ \ \isakeyword{and}\ fc{\isacharcolon}\ {\isachardoublequoteopen}finite{\isacharunderscore}character\ C{\isachardoublequoteclose}\isanewline
\ \ \isakeyword{shows}\ {\isachardoublequoteopen}pcp{\isacharunderscore}lim\ C\ S\ {\isasymin}\ C{\isachardoublequoteclose}\ {\isacharparenleft}\isakeyword{is}\ {\isachardoublequoteopen}{\isacharquery}cl\ {\isasymin}\ C{\isachardoublequoteclose}{\isacharparenright}\isanewline
%
\isadelimproof
%
\endisadelimproof
%
\isatagproof
\isacommand{proof}\isamarkupfalse%
\ {\isacharminus}\isanewline
\ \ \isacommand{from}\isamarkupfalse%
\ pcp{\isacharunderscore}seq{\isacharunderscore}in{\isacharbrackleft}OF\ c\ el{\isacharcomma}\ THEN\ allI{\isacharbrackright}\ \isacommand{have}\isamarkupfalse%
\ {\isachardoublequoteopen}{\isasymforall}n{\isachardot}\ pcp{\isacharunderscore}seq\ C\ S\ n\ {\isasymin}\ C{\isachardoublequoteclose}\ \isacommand{{\isachardot}}\isamarkupfalse%
\isanewline
\ \ \isacommand{hence}\isamarkupfalse%
\ {\isachardoublequoteopen}{\isasymforall}m{\isachardot}\ {\isasymUnion}{\isacharbraceleft}pcp{\isacharunderscore}seq\ C\ S\ n{\isacharbar}n{\isachardot}\ n\ {\isasymle}\ m{\isacharbraceright}\ {\isasymin}\ C{\isachardoublequoteclose}\ \isacommand{unfolding}\isamarkupfalse%
\ pcp{\isacharunderscore}seq{\isacharunderscore}UN\ \isacommand{{\isachardot}}\isamarkupfalse%
\isanewline
\ \ \isacommand{have}\isamarkupfalse%
\ {\isachardoublequoteopen}{\isasymforall}S{\isacharprime}\ {\isasymsubseteq}\ {\isacharquery}cl{\isachardot}\ finite\ S{\isacharprime}\ {\isasymlongrightarrow}\ S{\isacharprime}\ {\isasymin}\ C{\isachardoublequoteclose}\isanewline
\ \ \isacommand{proof}\isamarkupfalse%
\ safe\isanewline
\ \ \ \ \isacommand{fix}\isamarkupfalse%
\ S{\isacharprime}\ {\isacharcolon}{\isacharcolon}\ {\isachardoublequoteopen}{\isacharprime}a\ formula\ set{\isachardoublequoteclose}\isanewline
\ \ \ \ \isacommand{have}\isamarkupfalse%
\ {\isachardoublequoteopen}pcp{\isacharunderscore}seq\ C\ S\ {\isacharparenleft}Suc\ {\isacharparenleft}Max\ {\isacharparenleft}to{\isacharunderscore}nat\ {\isacharbackquote}\ S{\isacharprime}{\isacharparenright}{\isacharparenright}{\isacharparenright}\ {\isasymsubseteq}\ pcp{\isacharunderscore}lim\ C\ S{\isachardoublequoteclose}\ \isanewline
\ \ \ \ \ \ \isacommand{using}\isamarkupfalse%
\ pcp{\isacharunderscore}seq{\isacharunderscore}sub\ \isacommand{by}\isamarkupfalse%
\ blast\isanewline
\ \ \ \ \isacommand{assume}\isamarkupfalse%
\ {\isacartoucheopen}finite\ S{\isacharprime}{\isacartoucheclose}\ {\isacartoucheopen}S{\isacharprime}\ {\isasymsubseteq}\ pcp{\isacharunderscore}lim\ C\ S{\isacartoucheclose}\isanewline
\ \ \ \ \isacommand{hence}\isamarkupfalse%
\ {\isachardoublequoteopen}{\isasymexists}k{\isachardot}\ S{\isacharprime}\ {\isasymsubseteq}\ pcp{\isacharunderscore}seq\ C\ S\ k{\isachardoublequoteclose}\ \isanewline
\ \ \ \ \isacommand{proof}\isamarkupfalse%
{\isacharparenleft}induction\ S{\isacharprime}\ rule{\isacharcolon}\ finite{\isacharunderscore}induct{\isacharparenright}\ \isanewline
\ \ \ \ \ \ \isacommand{case}\isamarkupfalse%
\ {\isacharparenleft}insert\ x\ S{\isacharprime}{\isacharparenright}\isanewline
\ \ \ \ \ \ \isacommand{hence}\isamarkupfalse%
\ {\isachardoublequoteopen}{\isasymexists}k{\isachardot}\ S{\isacharprime}\ {\isasymsubseteq}\ pcp{\isacharunderscore}seq\ C\ S\ k{\isachardoublequoteclose}\ \isacommand{by}\isamarkupfalse%
\ fast\isanewline
\ \ \ \ \ \ \isacommand{then}\isamarkupfalse%
\ \isacommand{guess}\isamarkupfalse%
\ k{\isadigit{1}}\ \isacommand{{\isachardot}{\isachardot}}\isamarkupfalse%
\isanewline
\ \ \ \ \ \ \isacommand{moreover}\isamarkupfalse%
\ \isacommand{obtain}\isamarkupfalse%
\ k{\isadigit{2}}\ \isakeyword{where}\ {\isachardoublequoteopen}x\ {\isasymin}\ pcp{\isacharunderscore}seq\ C\ S\ k{\isadigit{2}}{\isachardoublequoteclose}\isanewline
\ \ \ \ \ \ \ \ \isacommand{by}\isamarkupfalse%
\ {\isacharparenleft}meson\ pcp{\isacharunderscore}lim{\isacharunderscore}inserted{\isacharunderscore}at{\isacharunderscore}ex\ insert{\isachardot}prems\ insert{\isacharunderscore}subset{\isacharparenright}\isanewline
\ \ \ \ \ \ \isacommand{ultimately}\isamarkupfalse%
\ \isacommand{have}\isamarkupfalse%
\ {\isachardoublequoteopen}insert\ x\ S{\isacharprime}\ {\isasymsubseteq}\ pcp{\isacharunderscore}seq\ C\ S\ {\isacharparenleft}max\ k{\isadigit{1}}\ k{\isadigit{2}}{\isacharparenright}{\isachardoublequoteclose}\isanewline
\ \ \ \ \ \ \ \ \isacommand{by}\isamarkupfalse%
\ {\isacharparenleft}meson\ pcp{\isacharunderscore}seq{\isacharunderscore}mono\ dual{\isacharunderscore}order{\isachardot}trans\ insert{\isacharunderscore}subset\ max{\isachardot}bounded{\isacharunderscore}iff\ order{\isacharunderscore}refl\ subsetCE{\isacharparenright}\isanewline
\ \ \ \ \ \ \isacommand{thus}\isamarkupfalse%
\ {\isacharquery}case\ \isacommand{by}\isamarkupfalse%
\ blast\isanewline
\ \ \ \ \isacommand{qed}\isamarkupfalse%
\ simp\isanewline
\ \ \ \ \isacommand{with}\isamarkupfalse%
\ pcp{\isacharunderscore}seq{\isacharunderscore}in{\isacharbrackleft}OF\ c\ el{\isacharbrackright}\ sc\isanewline
\ \ \ \ \isacommand{show}\isamarkupfalse%
\ {\isachardoublequoteopen}S{\isacharprime}\ {\isasymin}\ C{\isachardoublequoteclose}\ \isacommand{unfolding}\isamarkupfalse%
\ subset{\isacharunderscore}closed{\isacharunderscore}def\ \isacommand{by}\isamarkupfalse%
\ blast\isanewline
\ \ \isacommand{qed}\isamarkupfalse%
\isanewline
\ \ \isacommand{thus}\isamarkupfalse%
\ {\isachardoublequoteopen}{\isacharquery}cl\ {\isasymin}\ C{\isachardoublequoteclose}\ \isacommand{using}\isamarkupfalse%
\ fc\ \isacommand{unfolding}\isamarkupfalse%
\ finite{\isacharunderscore}character{\isacharunderscore}def\ \isacommand{by}\isamarkupfalse%
\ blast\isanewline
\isacommand{qed}\isamarkupfalse%
%
\endisatagproof
{\isafoldproof}%
%
\isadelimproof
%
\endisadelimproof
%
\begin{isamarkuptext}%
Probemos que, además, el límite de las sucesión definida en \isa{{\isadigit{4}}{\isachardot}{\isadigit{1}}{\isachardot}{\isadigit{1}}} se trata de un elemento 
  maximal de la colección que lo define si esta verifica la propiedad de consistencia proposicional
  y es cerrada bajo subconjuntos.

  \begin{lema}
    Sea \isa{C} una colección de conjuntos que verifica la propiedad de consistencia proposicional y
    es cerrada bajo subconjuntos, \isa{S} un conjunto y \isa{{\isacharbraceleft}S\isactrlsub n{\isacharbraceright}} la sucesión de conjuntos de \isa{C} a partir 
    de \isa{S} según la definición \isa{{\isadigit{4}}{\isachardot}{\isadigit{1}}{\isachardot}{\isadigit{1}}}. Entonces, el límite de la sucesión \isa{{\isacharbraceleft}S\isactrlsub n{\isacharbraceright}} es un elemento 
    maximal de \isa{C}.
  \end{lema}

  \begin{demostracion}
    Por definición de elemento maximal, basta probar que para cualquier conjunto \isa{K\ {\isasymin}\ C} que
    contenga al límite de la sucesión se tiene que \isa{K} y el límite coinciden.

    La demostración se realizará por reducción al absurdo. Consideremos un conjunto \isa{K\ {\isasymin}\ C} que 
    contenga estrictamente al límite de la sucesión \isa{{\isacharbraceleft}S\isactrlsub n{\isacharbraceright}}. De este modo, existe una fórmula \isa{F} tal 
    que \isa{F\ {\isasymin}\ K} y \isa{F} no está en el límite. Supongamos que \isa{F} es la \isa{n}-ésima fórmula según la 
    enumeración de la definición \isa{{\isadigit{4}}{\isachardot}{\isadigit{1}}{\isachardot}{\isadigit{1}}} utilizada para construir la sucesión. 

    Por un lado, hemos probado que todo elemento de la sucesión está contenido en el límite, luego 
    en particular obtenemos que \isa{S\isactrlsub n\isactrlsub {\isacharplus}\isactrlsub {\isadigit{1}}} está contenido en el límite. De este modo, como \isa{F} no 
    pertenece al límite, es claro que \isa{F\ {\isasymnotin}\ S\isactrlsub n\isactrlsub {\isacharplus}\isactrlsub {\isadigit{1}}}. Además, \isa{{\isacharbraceleft}F{\isacharbraceright}\ {\isasymunion}\ S\isactrlsub n\ {\isasymnotin}\ C} ya que, en caso contrario, 
    por la definición \isa{{\isadigit{4}}{\isachardot}{\isadigit{1}}{\isachardot}{\isadigit{1}}} de la sucesión obtendríamos que\\ \isa{S\isactrlsub n\isactrlsub {\isacharplus}\isactrlsub {\isadigit{1}}\ {\isacharequal}\ {\isacharbraceleft}F{\isacharbraceright}\ {\isasymunion}\ S\isactrlsub n}, lo que contradice 
    que \isa{F\ {\isasymnotin}\ S\isactrlsub n\isactrlsub {\isacharplus}\isactrlsub {\isadigit{1}}}. 

    Por otro lado, como \isa{S\isactrlsub n} también está contenida en el límite que, a su vez, está contenido en 
    \isa{K}, se obtiene por transitividad que \isa{S\isactrlsub n\ {\isasymsubseteq}\ K}. Además, como \isa{F\ {\isasymin}\ K}, se verifica que 
    \isa{{\isacharbraceleft}F{\isacharbraceright}\ {\isasymunion}\ S\isactrlsub n\ {\isasymsubseteq}\ K}. Como \isa{C} es una colección cerrada bajo subconjuntos por hipótesis y \isa{K\ {\isasymin}\ C}, 
    por definición se tiene que \isa{{\isacharbraceleft}F{\isacharbraceright}\ {\isasymunion}\ S\isactrlsub n\ {\isasymin}\ C}, llegando así a una contradicción con lo demostrado 
    anteriormente.
  \end{demostracion}

  Su formalización y prueba detallada en Isabelle/HOL se muestran a continuación.%
\end{isamarkuptext}\isamarkuptrue%
\isacommand{lemma}\isamarkupfalse%
\isanewline
\ \ \isakeyword{assumes}\ {\isachardoublequoteopen}pcp\ C{\isachardoublequoteclose}\isanewline
\ \ \ \ \ \ \ \ \ \ {\isachardoublequoteopen}subset{\isacharunderscore}closed\ C{\isachardoublequoteclose}\isanewline
\ \ \ \ \ \ \ \ \ \ {\isachardoublequoteopen}K\ {\isasymin}\ C{\isachardoublequoteclose}\isanewline
\ \ \ \ \ \ \ \ \ \ {\isachardoublequoteopen}pcp{\isacharunderscore}lim\ C\ S\ {\isasymsubseteq}\ K{\isachardoublequoteclose}\isanewline
\ \ \isakeyword{shows}\ {\isachardoublequoteopen}pcp{\isacharunderscore}lim\ C\ S\ {\isacharequal}\ K{\isachardoublequoteclose}\isanewline
%
\isadelimproof
%
\endisadelimproof
%
\isatagproof
\isacommand{proof}\isamarkupfalse%
\ {\isacharparenleft}rule\ ccontr{\isacharparenright}\isanewline
\ \ \isacommand{assume}\isamarkupfalse%
\ H{\isacharcolon}{\isachardoublequoteopen}{\isasymnot}{\isacharparenleft}pcp{\isacharunderscore}lim\ C\ S\ {\isacharequal}\ K{\isacharparenright}{\isachardoublequoteclose}\isanewline
\ \ \isacommand{have}\isamarkupfalse%
\ CE{\isacharcolon}{\isachardoublequoteopen}pcp{\isacharunderscore}lim\ C\ S\ {\isasymsubseteq}\ K\ {\isasymand}\ pcp{\isacharunderscore}lim\ C\ S\ {\isasymnoteq}\ K{\isachardoublequoteclose}\isanewline
\ \ \ \ \isacommand{using}\isamarkupfalse%
\ assms{\isacharparenleft}{\isadigit{4}}{\isacharparenright}\ H\ \isacommand{by}\isamarkupfalse%
\ {\isacharparenleft}rule\ conjI{\isacharparenright}\isanewline
\ \ \isacommand{have}\isamarkupfalse%
\ {\isachardoublequoteopen}pcp{\isacharunderscore}lim\ C\ S\ {\isasymsubseteq}\ K\ {\isasymand}\ pcp{\isacharunderscore}lim\ C\ S\ {\isasymnoteq}\ K\ {\isasymlongleftrightarrow}\ pcp{\isacharunderscore}lim\ C\ S\ {\isasymsubset}\ K{\isachardoublequoteclose}\isanewline
\ \ \ \ \isacommand{by}\isamarkupfalse%
\ {\isacharparenleft}simp\ only{\isacharcolon}\ psubset{\isacharunderscore}eq{\isacharparenright}\isanewline
\ \ \isacommand{then}\isamarkupfalse%
\ \isacommand{have}\isamarkupfalse%
\ {\isachardoublequoteopen}pcp{\isacharunderscore}lim\ C\ S\ {\isasymsubset}\ K{\isachardoublequoteclose}\ \isanewline
\ \ \ \ \isacommand{using}\isamarkupfalse%
\ CE\ \isacommand{by}\isamarkupfalse%
\ {\isacharparenleft}rule\ iffD{\isadigit{1}}{\isacharparenright}\isanewline
\ \ \isacommand{then}\isamarkupfalse%
\ \isacommand{have}\isamarkupfalse%
\ {\isachardoublequoteopen}{\isasymexists}F{\isachardot}\ F\ {\isasymin}\ {\isacharparenleft}K\ {\isacharminus}\ {\isacharparenleft}pcp{\isacharunderscore}lim\ C\ S{\isacharparenright}{\isacharparenright}{\isachardoublequoteclose}\isanewline
\ \ \ \ \isacommand{by}\isamarkupfalse%
\ {\isacharparenleft}simp\ only{\isacharcolon}\ psubset{\isacharunderscore}imp{\isacharunderscore}ex{\isacharunderscore}mem{\isacharparenright}\ \isanewline
\ \ \isacommand{then}\isamarkupfalse%
\ \isacommand{have}\isamarkupfalse%
\ E{\isacharcolon}{\isachardoublequoteopen}{\isasymexists}F{\isachardot}\ F\ {\isasymin}\ K\ {\isasymand}\ F\ {\isasymnotin}\ {\isacharparenleft}pcp{\isacharunderscore}lim\ C\ S{\isacharparenright}{\isachardoublequoteclose}\isanewline
\ \ \ \ \isacommand{by}\isamarkupfalse%
\ {\isacharparenleft}simp\ only{\isacharcolon}\ Diff{\isacharunderscore}iff{\isacharparenright}\isanewline
\ \ \isacommand{obtain}\isamarkupfalse%
\ F\ \isakeyword{where}\ F{\isacharcolon}{\isachardoublequoteopen}F\ {\isasymin}\ K\ {\isasymand}\ F\ {\isasymnotin}\ pcp{\isacharunderscore}lim\ C\ S{\isachardoublequoteclose}\ \isanewline
\ \ \ \ \isacommand{using}\isamarkupfalse%
\ E\ \isacommand{by}\isamarkupfalse%
\ {\isacharparenleft}rule\ exE{\isacharparenright}\isanewline
\ \ \isacommand{have}\isamarkupfalse%
\ {\isachardoublequoteopen}F\ {\isasymin}\ K{\isachardoublequoteclose}\ \isanewline
\ \ \ \ \isacommand{using}\isamarkupfalse%
\ F\ \isacommand{by}\isamarkupfalse%
\ {\isacharparenleft}rule\ conjunct{\isadigit{1}}{\isacharparenright}\isanewline
\ \ \isacommand{have}\isamarkupfalse%
\ {\isachardoublequoteopen}F\ {\isasymnotin}\ pcp{\isacharunderscore}lim\ C\ S{\isachardoublequoteclose}\isanewline
\ \ \ \ \isacommand{using}\isamarkupfalse%
\ F\ \isacommand{by}\isamarkupfalse%
\ {\isacharparenleft}rule\ conjunct{\isadigit{2}}{\isacharparenright}\isanewline
\ \ \isacommand{have}\isamarkupfalse%
\ {\isachardoublequoteopen}pcp{\isacharunderscore}seq\ C\ S\ {\isacharparenleft}Suc\ {\isacharparenleft}to{\isacharunderscore}nat\ F{\isacharparenright}{\isacharparenright}\ {\isasymsubseteq}\ pcp{\isacharunderscore}lim\ C\ S{\isachardoublequoteclose}\isanewline
\ \ \ \ \isacommand{by}\isamarkupfalse%
\ {\isacharparenleft}rule\ pcp{\isacharunderscore}seq{\isacharunderscore}sub{\isacharparenright}\isanewline
\ \ \isacommand{then}\isamarkupfalse%
\ \isacommand{have}\isamarkupfalse%
\ {\isachardoublequoteopen}F\ {\isasymin}\ pcp{\isacharunderscore}seq\ C\ S\ {\isacharparenleft}Suc\ {\isacharparenleft}to{\isacharunderscore}nat\ F{\isacharparenright}{\isacharparenright}\ {\isasymlongrightarrow}\ F\ {\isasymin}\ pcp{\isacharunderscore}lim\ C\ S{\isachardoublequoteclose}\isanewline
\ \ \ \ \isacommand{by}\isamarkupfalse%
\ {\isacharparenleft}rule\ in{\isacharunderscore}mono{\isacharparenright}\isanewline
\ \ \isacommand{then}\isamarkupfalse%
\ \isacommand{have}\isamarkupfalse%
\ {\isadigit{1}}{\isacharcolon}{\isachardoublequoteopen}F\ {\isasymnotin}\ pcp{\isacharunderscore}seq\ C\ S\ {\isacharparenleft}Suc\ {\isacharparenleft}to{\isacharunderscore}nat\ F{\isacharparenright}{\isacharparenright}{\isachardoublequoteclose}\isanewline
\ \ \ \ \isacommand{using}\isamarkupfalse%
\ {\isacartoucheopen}F\ {\isasymnotin}\ pcp{\isacharunderscore}lim\ C\ S{\isacartoucheclose}\ \isacommand{by}\isamarkupfalse%
\ {\isacharparenleft}rule\ mt{\isacharparenright}\isanewline
\ \ \isacommand{have}\isamarkupfalse%
\ {\isadigit{2}}{\isacharcolon}\ {\isachardoublequoteopen}insert\ F\ {\isacharparenleft}pcp{\isacharunderscore}seq\ C\ S\ {\isacharparenleft}to{\isacharunderscore}nat\ F{\isacharparenright}{\isacharparenright}\ {\isasymnotin}\ C{\isachardoublequoteclose}\ \isanewline
\ \ \isacommand{proof}\isamarkupfalse%
\ {\isacharparenleft}rule\ ccontr{\isacharparenright}\isanewline
\ \ \ \ \isacommand{assume}\isamarkupfalse%
\ {\isachardoublequoteopen}{\isasymnot}{\isacharparenleft}insert\ F\ {\isacharparenleft}pcp{\isacharunderscore}seq\ C\ S\ {\isacharparenleft}to{\isacharunderscore}nat\ F{\isacharparenright}{\isacharparenright}\ {\isasymnotin}\ C{\isacharparenright}{\isachardoublequoteclose}\isanewline
\ \ \ \ \isacommand{then}\isamarkupfalse%
\ \isacommand{have}\isamarkupfalse%
\ {\isachardoublequoteopen}insert\ F\ {\isacharparenleft}pcp{\isacharunderscore}seq\ C\ S\ {\isacharparenleft}to{\isacharunderscore}nat\ F{\isacharparenright}{\isacharparenright}\ {\isasymin}\ C{\isachardoublequoteclose}\isanewline
\ \ \ \ \ \ \isacommand{by}\isamarkupfalse%
\ {\isacharparenleft}rule\ notnotD{\isacharparenright}\isanewline
\ \ \ \ \isacommand{then}\isamarkupfalse%
\ \isacommand{have}\isamarkupfalse%
\ C{\isacharcolon}{\isachardoublequoteopen}insert\ {\isacharparenleft}from{\isacharunderscore}nat\ {\isacharparenleft}to{\isacharunderscore}nat\ F{\isacharparenright}{\isacharparenright}\ {\isacharparenleft}pcp{\isacharunderscore}seq\ C\ S\ {\isacharparenleft}to{\isacharunderscore}nat\ F{\isacharparenright}{\isacharparenright}\ {\isasymin}\ C{\isachardoublequoteclose}\isanewline
\ \ \ \ \ \ \isacommand{by}\isamarkupfalse%
\ {\isacharparenleft}simp\ only{\isacharcolon}\ from{\isacharunderscore}nat{\isacharunderscore}to{\isacharunderscore}nat{\isacharparenright}\isanewline
\ \ \ \ \isacommand{have}\isamarkupfalse%
\ {\isachardoublequoteopen}pcp{\isacharunderscore}seq\ C\ S\ {\isacharparenleft}Suc\ {\isacharparenleft}to{\isacharunderscore}nat\ F{\isacharparenright}{\isacharparenright}\ {\isacharequal}\ {\isacharparenleft}let\ Sn\ {\isacharequal}\ pcp{\isacharunderscore}seq\ C\ S\ {\isacharparenleft}to{\isacharunderscore}nat\ F{\isacharparenright}{\isacharsemicolon}\ \isanewline
\ \ \ \ \ \ \ \ \ \ Sn{\isadigit{1}}\ {\isacharequal}\ insert\ {\isacharparenleft}from{\isacharunderscore}nat\ {\isacharparenleft}to{\isacharunderscore}nat\ F{\isacharparenright}{\isacharparenright}\ Sn\ in\ if\ Sn{\isadigit{1}}\ {\isasymin}\ C\ then\ Sn{\isadigit{1}}\ else\ Sn{\isacharparenright}{\isachardoublequoteclose}\ \isanewline
\ \ \ \ \ \ \isacommand{by}\isamarkupfalse%
\ {\isacharparenleft}simp\ only{\isacharcolon}\ pcp{\isacharunderscore}seq{\isachardot}simps{\isacharparenleft}{\isadigit{2}}{\isacharparenright}{\isacharparenright}\isanewline
\ \ \ \ \isacommand{then}\isamarkupfalse%
\ \isacommand{have}\isamarkupfalse%
\ SucDef{\isacharcolon}{\isachardoublequoteopen}pcp{\isacharunderscore}seq\ C\ S\ {\isacharparenleft}Suc\ {\isacharparenleft}to{\isacharunderscore}nat\ F{\isacharparenright}{\isacharparenright}\ {\isacharequal}\ {\isacharparenleft}if\ insert\ {\isacharparenleft}from{\isacharunderscore}nat\ {\isacharparenleft}to{\isacharunderscore}nat\ F{\isacharparenright}{\isacharparenright}\ {\isacharparenleft}pcp{\isacharunderscore}seq\ C\ S\ {\isacharparenleft}to{\isacharunderscore}nat\ F{\isacharparenright}{\isacharparenright}\ {\isasymin}\ C\ \isanewline
\ \ \ \ \ \ \ \ \ \ then\ insert\ {\isacharparenleft}from{\isacharunderscore}nat\ {\isacharparenleft}to{\isacharunderscore}nat\ F{\isacharparenright}{\isacharparenright}\ {\isacharparenleft}pcp{\isacharunderscore}seq\ C\ S\ {\isacharparenleft}to{\isacharunderscore}nat\ F{\isacharparenright}{\isacharparenright}\ else\ pcp{\isacharunderscore}seq\ C\ S\ {\isacharparenleft}to{\isacharunderscore}nat\ F{\isacharparenright}{\isacharparenright}{\isachardoublequoteclose}\ \isanewline
\ \ \ \ \ \ \isacommand{by}\isamarkupfalse%
\ {\isacharparenleft}simp\ only{\isacharcolon}\ Let{\isacharunderscore}def{\isacharparenright}\isanewline
\ \ \ \ \isacommand{then}\isamarkupfalse%
\ \isacommand{have}\isamarkupfalse%
\ {\isachardoublequoteopen}pcp{\isacharunderscore}seq\ C\ S\ {\isacharparenleft}Suc\ {\isacharparenleft}to{\isacharunderscore}nat\ F{\isacharparenright}{\isacharparenright}\ {\isacharequal}\ insert\ {\isacharparenleft}from{\isacharunderscore}nat\ {\isacharparenleft}to{\isacharunderscore}nat\ F{\isacharparenright}{\isacharparenright}\ {\isacharparenleft}pcp{\isacharunderscore}seq\ C\ S\ {\isacharparenleft}to{\isacharunderscore}nat\ F{\isacharparenright}{\isacharparenright}{\isachardoublequoteclose}\ \isanewline
\ \ \ \ \ \ \isacommand{using}\isamarkupfalse%
\ C\ \isacommand{by}\isamarkupfalse%
\ {\isacharparenleft}simp\ only{\isacharcolon}\ if{\isacharunderscore}True{\isacharparenright}\isanewline
\ \ \ \ \isacommand{then}\isamarkupfalse%
\ \isacommand{have}\isamarkupfalse%
\ {\isachardoublequoteopen}pcp{\isacharunderscore}seq\ C\ S\ {\isacharparenleft}Suc\ {\isacharparenleft}to{\isacharunderscore}nat\ F{\isacharparenright}{\isacharparenright}\ {\isacharequal}\ insert\ F\ {\isacharparenleft}pcp{\isacharunderscore}seq\ C\ S\ {\isacharparenleft}to{\isacharunderscore}nat\ F{\isacharparenright}{\isacharparenright}{\isachardoublequoteclose}\isanewline
\ \ \ \ \ \ \isacommand{by}\isamarkupfalse%
\ {\isacharparenleft}simp\ only{\isacharcolon}\ from{\isacharunderscore}nat{\isacharunderscore}to{\isacharunderscore}nat{\isacharparenright}\isanewline
\ \ \ \ \isacommand{then}\isamarkupfalse%
\ \isacommand{have}\isamarkupfalse%
\ {\isachardoublequoteopen}F\ {\isasymin}\ pcp{\isacharunderscore}seq\ C\ S\ {\isacharparenleft}Suc\ {\isacharparenleft}to{\isacharunderscore}nat\ F{\isacharparenright}{\isacharparenright}{\isachardoublequoteclose}\isanewline
\ \ \ \ \ \ \isacommand{by}\isamarkupfalse%
\ {\isacharparenleft}simp\ only{\isacharcolon}\ insertI{\isadigit{1}}{\isacharparenright}\isanewline
\ \ \ \ \isacommand{show}\isamarkupfalse%
\ {\isachardoublequoteopen}False{\isachardoublequoteclose}\isanewline
\ \ \ \ \ \ \isacommand{using}\isamarkupfalse%
\ {\isacartoucheopen}F\ {\isasymnotin}\ pcp{\isacharunderscore}seq\ C\ S\ {\isacharparenleft}Suc\ {\isacharparenleft}to{\isacharunderscore}nat\ F{\isacharparenright}{\isacharparenright}{\isacartoucheclose}\ {\isacartoucheopen}F\ {\isasymin}\ pcp{\isacharunderscore}seq\ C\ S\ {\isacharparenleft}Suc\ {\isacharparenleft}to{\isacharunderscore}nat\ F{\isacharparenright}{\isacharparenright}{\isacartoucheclose}\ \isacommand{by}\isamarkupfalse%
\ {\isacharparenleft}rule\ notE{\isacharparenright}\isanewline
\ \ \isacommand{qed}\isamarkupfalse%
\isanewline
\ \ \isacommand{have}\isamarkupfalse%
\ {\isachardoublequoteopen}pcp{\isacharunderscore}seq\ C\ S\ {\isacharparenleft}to{\isacharunderscore}nat\ F{\isacharparenright}\ {\isasymsubseteq}\ pcp{\isacharunderscore}lim\ C\ S{\isachardoublequoteclose}\isanewline
\ \ \ \ \isacommand{by}\isamarkupfalse%
\ {\isacharparenleft}rule\ pcp{\isacharunderscore}seq{\isacharunderscore}sub{\isacharparenright}\isanewline
\ \ \isacommand{then}\isamarkupfalse%
\ \isacommand{have}\isamarkupfalse%
\ {\isachardoublequoteopen}pcp{\isacharunderscore}seq\ C\ S\ {\isacharparenleft}to{\isacharunderscore}nat\ F{\isacharparenright}\ {\isasymsubseteq}\ K{\isachardoublequoteclose}\isanewline
\ \ \ \ \isacommand{using}\isamarkupfalse%
\ assms{\isacharparenleft}{\isadigit{4}}{\isacharparenright}\ \isacommand{by}\isamarkupfalse%
\ {\isacharparenleft}rule\ subset{\isacharunderscore}trans{\isacharparenright}\isanewline
\ \ \isacommand{then}\isamarkupfalse%
\ \isacommand{have}\isamarkupfalse%
\ {\isachardoublequoteopen}insert\ F\ {\isacharparenleft}pcp{\isacharunderscore}seq\ C\ S\ {\isacharparenleft}to{\isacharunderscore}nat\ F{\isacharparenright}{\isacharparenright}\ {\isasymsubseteq}\ K{\isachardoublequoteclose}\ \isanewline
\ \ \ \ \isacommand{using}\isamarkupfalse%
\ {\isacartoucheopen}F\ {\isasymin}\ K{\isacartoucheclose}\ \isacommand{by}\isamarkupfalse%
\ {\isacharparenleft}simp\ only{\isacharcolon}\ insert{\isacharunderscore}subset{\isacharparenright}\isanewline
\ \ \isacommand{have}\isamarkupfalse%
\ {\isachardoublequoteopen}{\isasymforall}S\ {\isasymin}\ C{\isachardot}\ {\isasymforall}s{\isasymsubseteq}S{\isachardot}\ s\ {\isasymin}\ C{\isachardoublequoteclose}\isanewline
\ \ \ \ \isacommand{using}\isamarkupfalse%
\ assms{\isacharparenleft}{\isadigit{2}}{\isacharparenright}\ \isacommand{by}\isamarkupfalse%
\ {\isacharparenleft}simp\ only{\isacharcolon}\ subset{\isacharunderscore}closed{\isacharunderscore}def{\isacharparenright}\isanewline
\ \ \isacommand{then}\isamarkupfalse%
\ \isacommand{have}\isamarkupfalse%
\ {\isachardoublequoteopen}{\isasymforall}s\ {\isasymsubseteq}\ K{\isachardot}\ s\ {\isasymin}\ C{\isachardoublequoteclose}\isanewline
\ \ \ \ \isacommand{using}\isamarkupfalse%
\ assms{\isacharparenleft}{\isadigit{3}}{\isacharparenright}\ \isacommand{by}\isamarkupfalse%
\ {\isacharparenleft}rule\ bspec{\isacharparenright}\isanewline
\ \ \isacommand{then}\isamarkupfalse%
\ \isacommand{have}\isamarkupfalse%
\ {\isadigit{3}}{\isacharcolon}{\isachardoublequoteopen}insert\ F\ {\isacharparenleft}pcp{\isacharunderscore}seq\ C\ S\ {\isacharparenleft}to{\isacharunderscore}nat\ F{\isacharparenright}{\isacharparenright}\ {\isasymin}\ C{\isachardoublequoteclose}\ \isanewline
\ \ \ \ \isacommand{using}\isamarkupfalse%
\ {\isacartoucheopen}insert\ F\ {\isacharparenleft}pcp{\isacharunderscore}seq\ C\ S\ {\isacharparenleft}to{\isacharunderscore}nat\ F{\isacharparenright}{\isacharparenright}\ {\isasymsubseteq}\ K{\isacartoucheclose}\ \isacommand{by}\isamarkupfalse%
\ {\isacharparenleft}rule\ sspec{\isacharparenright}\isanewline
\ \ \isacommand{show}\isamarkupfalse%
\ {\isachardoublequoteopen}False{\isachardoublequoteclose}\isanewline
\ \ \ \ \isacommand{using}\isamarkupfalse%
\ {\isadigit{2}}\ {\isadigit{3}}\ \isacommand{by}\isamarkupfalse%
\ {\isacharparenleft}rule\ notE{\isacharparenright}\isanewline
\isacommand{qed}\isamarkupfalse%
%
\endisatagproof
{\isafoldproof}%
%
\isadelimproof
%
\endisadelimproof
%
\begin{isamarkuptext}%
Análogamente a resultados anteriores, veamos su prueba automática.%
\end{isamarkuptext}\isamarkuptrue%
\isacommand{lemma}\isamarkupfalse%
\ cl{\isacharunderscore}max{\isacharcolon}\isanewline
\ \ \isakeyword{assumes}\ c{\isacharcolon}\ {\isachardoublequoteopen}pcp\ C{\isachardoublequoteclose}\isanewline
\ \ \isakeyword{assumes}\ sc{\isacharcolon}\ {\isachardoublequoteopen}subset{\isacharunderscore}closed\ C{\isachardoublequoteclose}\isanewline
\ \ \isakeyword{assumes}\ el{\isacharcolon}\ {\isachardoublequoteopen}K\ {\isasymin}\ C{\isachardoublequoteclose}\isanewline
\ \ \isakeyword{assumes}\ su{\isacharcolon}\ {\isachardoublequoteopen}pcp{\isacharunderscore}lim\ C\ S\ {\isasymsubseteq}\ K{\isachardoublequoteclose}\isanewline
\ \ \isakeyword{shows}\ {\isachardoublequoteopen}pcp{\isacharunderscore}lim\ C\ S\ {\isacharequal}\ K{\isachardoublequoteclose}\ {\isacharparenleft}\isakeyword{is}\ {\isacharquery}e{\isacharparenright}\isanewline
%
\isadelimproof
%
\endisadelimproof
%
\isatagproof
\isacommand{proof}\isamarkupfalse%
\ {\isacharparenleft}rule\ ccontr{\isacharparenright}\isanewline
\ \ \isacommand{assume}\isamarkupfalse%
\ {\isacartoucheopen}{\isasymnot}{\isacharquery}e{\isacartoucheclose}\isanewline
\ \ \isacommand{with}\isamarkupfalse%
\ su\ \isacommand{have}\isamarkupfalse%
\ {\isachardoublequoteopen}pcp{\isacharunderscore}lim\ C\ S\ {\isasymsubset}\ K{\isachardoublequoteclose}\ \isacommand{by}\isamarkupfalse%
\ simp\isanewline
\ \ \isacommand{then}\isamarkupfalse%
\ \isacommand{obtain}\isamarkupfalse%
\ F\ \isakeyword{where}\ e{\isacharcolon}\ {\isachardoublequoteopen}F\ {\isasymin}\ K{\isachardoublequoteclose}\ \isakeyword{and}\ ne{\isacharcolon}\ {\isachardoublequoteopen}F\ {\isasymnotin}\ pcp{\isacharunderscore}lim\ C\ S{\isachardoublequoteclose}\ \isacommand{by}\isamarkupfalse%
\ blast\isanewline
\ \ \isacommand{from}\isamarkupfalse%
\ ne\ \isacommand{have}\isamarkupfalse%
\ {\isachardoublequoteopen}F\ {\isasymnotin}\ pcp{\isacharunderscore}seq\ C\ S\ {\isacharparenleft}Suc\ {\isacharparenleft}to{\isacharunderscore}nat\ F{\isacharparenright}{\isacharparenright}{\isachardoublequoteclose}\ \isacommand{using}\isamarkupfalse%
\ pcp{\isacharunderscore}seq{\isacharunderscore}sub\ \isacommand{by}\isamarkupfalse%
\ fast\isanewline
\ \ \isacommand{hence}\isamarkupfalse%
\ {\isadigit{1}}{\isacharcolon}\ {\isachardoublequoteopen}insert\ F\ {\isacharparenleft}pcp{\isacharunderscore}seq\ C\ S\ {\isacharparenleft}to{\isacharunderscore}nat\ F{\isacharparenright}{\isacharparenright}\ {\isasymnotin}\ C{\isachardoublequoteclose}\ \isacommand{by}\isamarkupfalse%
\ {\isacharparenleft}simp\ add{\isacharcolon}\ Let{\isacharunderscore}def\ split{\isacharcolon}\ if{\isacharunderscore}splits{\isacharparenright}\isanewline
\ \ \isacommand{have}\isamarkupfalse%
\ {\isachardoublequoteopen}insert\ F\ {\isacharparenleft}pcp{\isacharunderscore}seq\ C\ S\ {\isacharparenleft}to{\isacharunderscore}nat\ F{\isacharparenright}{\isacharparenright}\ {\isasymsubseteq}\ K{\isachardoublequoteclose}\ \isacommand{using}\isamarkupfalse%
\ pcp{\isacharunderscore}seq{\isacharunderscore}sub\ e\ su\ \isacommand{by}\isamarkupfalse%
\ blast\isanewline
\ \ \isacommand{hence}\isamarkupfalse%
\ {\isachardoublequoteopen}insert\ F\ {\isacharparenleft}pcp{\isacharunderscore}seq\ C\ S\ {\isacharparenleft}to{\isacharunderscore}nat\ F{\isacharparenright}{\isacharparenright}\ {\isasymin}\ C{\isachardoublequoteclose}\ \isacommand{using}\isamarkupfalse%
\ sc\ \isanewline
\ \ \ \ \isacommand{unfolding}\isamarkupfalse%
\ subset{\isacharunderscore}closed{\isacharunderscore}def\ \isacommand{using}\isamarkupfalse%
\ el\ \isacommand{by}\isamarkupfalse%
\ blast\isanewline
\ \ \isacommand{with}\isamarkupfalse%
\ {\isadigit{1}}\ \isacommand{show}\isamarkupfalse%
\ False\ \isacommand{{\isachardot}{\isachardot}}\isamarkupfalse%
\isanewline
\isacommand{qed}\isamarkupfalse%
%
\endisatagproof
{\isafoldproof}%
%
\isadelimproof
%
\endisadelimproof
%
\begin{isamarkuptext}%
A continuación mostremos un resultado sobre el límite de la sucesión de \isa{{\isadigit{4}}{\isachardot}{\isadigit{1}}{\isachardot}{\isadigit{1}}} que es 
  consecuencia de que dicho límite sea un elemento maximal de la colección que lo define si esta
  verifica la propiedad de consistencia proposicional y es cerrada bajo subconjuntos.
  
  \begin{corolario}
    Sea \isa{C} una colección de conjuntos que verifica la propiedad de consistencia proposicional y
    es cerrada bajo subconjuntos, \isa{S} un conjunto, \isa{{\isacharbraceleft}S\isactrlsub n{\isacharbraceright}} la sucesión de conjuntos de \isa{C} a partir 
    de \isa{S} según la definición \isa{{\isadigit{4}}{\isachardot}{\isadigit{1}}{\isachardot}{\isadigit{1}}} y \isa{F} una fórmula proposicional. Entonces, si\\
    $\{F\} \cup \bigcup_{n = 0}^{\infty} S_{n} \in C$, se verifica que 
    $F \in \bigcup_{n = 0}^{\infty} S_{n}$. 
  \end{corolario}

  \begin{demostracion}
    Como \isa{C} es una colección que verifica la propiedad de consistencia proposicional y es cerrada 
    bajo subconjuntos, se tiene que el límite $\bigcup_{n = 0}^{\infty} S_{n}$ es maximal en \isa{C}. Por 
    lo tanto, si suponemos que $\{F\} \cup \bigcup_{n = 0}^{\infty} S_{n} \in C$, como el límite 
    está contenido en dicho conjunto, se cumple que 
    $\{F\} \cup \bigcup_{n = 0}^{\infty} S_{n} = \bigcup_{n = 0}^{\infty} S_{n}$, luego \isa{F} 
    pertenece al límite, como queríamos demostrar.
  \end{demostracion}

  Veamos su formalización y prueba detallada en Isabelle/HOL.%
\end{isamarkuptext}\isamarkuptrue%
\isacommand{lemma}\isamarkupfalse%
\isanewline
\ \ \isakeyword{assumes}\ {\isachardoublequoteopen}pcp\ C{\isachardoublequoteclose}\isanewline
\ \ \isakeyword{assumes}\ {\isachardoublequoteopen}subset{\isacharunderscore}closed\ C{\isachardoublequoteclose}\isanewline
\ \ \isakeyword{shows}\ {\isachardoublequoteopen}insert\ F\ {\isacharparenleft}pcp{\isacharunderscore}lim\ C\ S{\isacharparenright}\ {\isasymin}\ C\ {\isasymLongrightarrow}\ F\ {\isasymin}\ pcp{\isacharunderscore}lim\ C\ S{\isachardoublequoteclose}\isanewline
%
\isadelimproof
%
\endisadelimproof
%
\isatagproof
\isacommand{proof}\isamarkupfalse%
\ {\isacharminus}\isanewline
\ \ \isacommand{assume}\isamarkupfalse%
\ {\isachardoublequoteopen}insert\ F\ {\isacharparenleft}pcp{\isacharunderscore}lim\ C\ S{\isacharparenright}\ {\isasymin}\ C{\isachardoublequoteclose}\isanewline
\ \ \isacommand{have}\isamarkupfalse%
\ {\isachardoublequoteopen}pcp{\isacharunderscore}lim\ C\ S\ {\isasymsubseteq}\ insert\ F\ {\isacharparenleft}pcp{\isacharunderscore}lim\ C\ S{\isacharparenright}{\isachardoublequoteclose}\isanewline
\ \ \ \ \isacommand{by}\isamarkupfalse%
\ {\isacharparenleft}rule\ subset{\isacharunderscore}insertI{\isacharparenright}\ \isanewline
\ \ \isacommand{have}\isamarkupfalse%
\ {\isachardoublequoteopen}pcp{\isacharunderscore}lim\ C\ S\ {\isacharequal}\ insert\ F\ {\isacharparenleft}pcp{\isacharunderscore}lim\ C\ S{\isacharparenright}{\isachardoublequoteclose}\isanewline
\ \ \ \ \isacommand{using}\isamarkupfalse%
\ assms{\isacharparenleft}{\isadigit{1}}{\isacharparenright}\ assms{\isacharparenleft}{\isadigit{2}}{\isacharparenright}\ {\isacartoucheopen}insert\ F\ {\isacharparenleft}pcp{\isacharunderscore}lim\ C\ S{\isacharparenright}\ {\isasymin}\ C{\isacartoucheclose}\ {\isacartoucheopen}pcp{\isacharunderscore}lim\ C\ S\ {\isasymsubseteq}\ insert\ F\ {\isacharparenleft}pcp{\isacharunderscore}lim\ C\ S{\isacharparenright}{\isacartoucheclose}\ \isacommand{by}\isamarkupfalse%
\ {\isacharparenleft}rule\ cl{\isacharunderscore}max{\isacharparenright}\isanewline
\ \ \isacommand{then}\isamarkupfalse%
\ \isacommand{have}\isamarkupfalse%
\ {\isachardoublequoteopen}insert\ F\ {\isacharparenleft}pcp{\isacharunderscore}lim\ C\ S{\isacharparenright}\ {\isasymsubseteq}\ pcp{\isacharunderscore}lim\ C\ S{\isachardoublequoteclose}\isanewline
\ \ \ \ \isacommand{by}\isamarkupfalse%
\ {\isacharparenleft}rule\ equalityD{\isadigit{2}}{\isacharparenright}\isanewline
\ \ \isacommand{then}\isamarkupfalse%
\ \isacommand{have}\isamarkupfalse%
\ {\isachardoublequoteopen}F\ {\isasymin}\ pcp{\isacharunderscore}lim\ C\ S\ {\isasymand}\ pcp{\isacharunderscore}lim\ C\ S\ {\isasymsubseteq}\ pcp{\isacharunderscore}lim\ C\ S{\isachardoublequoteclose}\isanewline
\ \ \ \ \isacommand{by}\isamarkupfalse%
\ {\isacharparenleft}simp\ only{\isacharcolon}\ insert{\isacharunderscore}subset{\isacharparenright}\isanewline
\ \ \isacommand{thus}\isamarkupfalse%
\ {\isachardoublequoteopen}F\ {\isasymin}\ pcp{\isacharunderscore}lim\ C\ S{\isachardoublequoteclose}\isanewline
\ \ \ \ \isacommand{by}\isamarkupfalse%
\ {\isacharparenleft}rule\ conjunct{\isadigit{1}}{\isacharparenright}\isanewline
\isacommand{qed}\isamarkupfalse%
%
\endisatagproof
{\isafoldproof}%
%
\isadelimproof
%
\endisadelimproof
%
\begin{isamarkuptext}%
Mostremos su demostración automática.%
\end{isamarkuptext}\isamarkuptrue%
\isacommand{lemma}\isamarkupfalse%
\ cl{\isacharunderscore}max{\isacharprime}{\isacharcolon}\isanewline
\ \ \isakeyword{assumes}\ c{\isacharcolon}\ {\isachardoublequoteopen}pcp\ C{\isachardoublequoteclose}\isanewline
\ \ \isakeyword{assumes}\ sc{\isacharcolon}\ {\isachardoublequoteopen}subset{\isacharunderscore}closed\ C{\isachardoublequoteclose}\isanewline
\ \ \isakeyword{shows}\ {\isachardoublequoteopen}insert\ F\ {\isacharparenleft}pcp{\isacharunderscore}lim\ C\ S{\isacharparenright}\ {\isasymin}\ C\ {\isasymLongrightarrow}\ F\ {\isasymin}\ pcp{\isacharunderscore}lim\ C\ S{\isachardoublequoteclose}\isanewline
%
\isadelimproof
\ \ %
\endisadelimproof
%
\isatagproof
\isacommand{using}\isamarkupfalse%
\ cl{\isacharunderscore}max{\isacharbrackleft}OF\ assms{\isacharbrackright}\ \isacommand{by}\isamarkupfalse%
\ blast{\isacharplus}%
\endisatagproof
{\isafoldproof}%
%
\isadelimproof
%
\endisadelimproof
%
\begin{isamarkuptext}%
El siguiente resultado prueba que el límite de la sucesión definida en \isa{{\isadigit{4}}{\isachardot}{\isadigit{1}}{\isachardot}{\isadigit{1}}} es un conjunto
  de Hintikka si la colección que lo define verifica la propiedad de consistencia proposicional, es
  es cerrada bajo subconjuntos y es de carácter finito. Como consecuencia del \isa{teorema\ de\ Hintikka},
  se trata en particular de un conjunto satisfacible. 

  \begin{lema}
    Sea \isa{C} una colección de conjuntos que verifica la propiedad de consistencia proposicional, es
    es cerrada bajo subconjuntos y es de carácter finito. Sea \isa{S\ {\isasymin}\ C} y \isa{{\isacharbraceleft}S\isactrlsub n{\isacharbraceright}} la sucesión de
    conjuntos de \isa{C} a partir de \isa{S} según la definición \isa{{\isadigit{4}}{\isachardot}{\isadigit{1}}{\isachardot}{\isadigit{1}}}. Entonces, el límite de la sucesión
    \isa{{\isacharbraceleft}S\isactrlsub n{\isacharbraceright}} es un conjunto de Hintikka.
  \end{lema}

  \begin{demostracion}
    Para facilitar la lectura, vamos a notar por \isa{L\isactrlsub S\isactrlsub C} al límite de la sucesión \isa{{\isacharbraceleft}S\isactrlsub n{\isacharbraceright}} descrita 
    en el enunciado.

    Por resultados anteriores, como \isa{C} verifica la propiedad de consistencia proposicional, es
    es cerrada bajo subconjuntos y es de carácter finito, se tiene que \isa{L\isactrlsub S\isactrlsub C\ {\isasymin}\ C}. En particular, por 
    verificar la propiedad de consistencia proposicional, por el lema de\\ caracterización de dicha
    propiedad mediante notación uniforme, se cumplen las siguientes condiciones para \isa{L\isactrlsub S\isactrlsub C}:

    \begin{itemize}
      \item \isa{{\isasymbottom}\ {\isasymnotin}\ L\isactrlsub S\isactrlsub C}.
      \item Dada \isa{p} una fórmula atómica cualquiera, no se tiene 
      simultáneamente que\\ \isa{p\ {\isasymin}\ L\isactrlsub S\isactrlsub C} y \isa{{\isasymnot}\ p\ {\isasymin}\ L\isactrlsub S\isactrlsub C}.
      \item Para toda fórmula de tipo \isa{{\isasymalpha}} con componentes \isa{{\isasymalpha}\isactrlsub {\isadigit{1}}} y \isa{{\isasymalpha}\isactrlsub {\isadigit{2}}} tal que \isa{{\isasymalpha}}
      pertenece a \isa{L\isactrlsub S\isactrlsub C}, se tiene que \isa{{\isacharbraceleft}{\isasymalpha}\isactrlsub {\isadigit{1}}{\isacharcomma}{\isasymalpha}\isactrlsub {\isadigit{2}}{\isacharbraceright}\ {\isasymunion}\ L\isactrlsub S\isactrlsub C} pertenece a \isa{C}.
      \item Para toda fórmula de tipo \isa{{\isasymbeta}} con componentes \isa{{\isasymbeta}\isactrlsub {\isadigit{1}}} y \isa{{\isasymbeta}\isactrlsub {\isadigit{2}}} tal que \isa{{\isasymbeta}}
      pertenece a \isa{L\isactrlsub S\isactrlsub C}, se tiene que o bien \isa{{\isacharbraceleft}{\isasymbeta}\isactrlsub {\isadigit{1}}{\isacharbraceright}\ {\isasymunion}\ L\isactrlsub S\isactrlsub C} pertenece a \isa{C} o 
      bien \isa{{\isacharbraceleft}{\isasymbeta}\isactrlsub {\isadigit{2}}{\isacharbraceright}\ {\isasymunion}\ L\isactrlsub S\isactrlsub C} pertenece a \isa{C}.
    \end{itemize}

    Veamos que \isa{L\isactrlsub S\isactrlsub C} es un conjunto de Hintikka probando que cumple las condiciones del
    lema de caracterización de los conjuntos de Hintikka mediante notación uniforme, es decir,
    probaremos que \isa{L\isactrlsub S\isactrlsub C} verifica:

    \begin{itemize}
      \item \isa{{\isasymbottom}\ {\isasymnotin}\ L\isactrlsub S\isactrlsub C}.
      \item Dada \isa{p} una fórmula atómica cualquiera, no se tiene 
      simultáneamente que\\ \isa{p\ {\isasymin}\ L\isactrlsub S\isactrlsub C} y \isa{{\isasymnot}\ p\ {\isasymin}\ L\isactrlsub S\isactrlsub C}.
      \item Para toda fórmula de tipo \isa{{\isasymalpha}} con componentes \isa{{\isasymalpha}\isactrlsub {\isadigit{1}}} y \isa{{\isasymalpha}\isactrlsub {\isadigit{2}}} se verifica 
      que si la fórmula pertenece a \isa{L\isactrlsub S\isactrlsub C}, entonces \isa{{\isasymalpha}\isactrlsub {\isadigit{1}}} y \isa{{\isasymalpha}\isactrlsub {\isadigit{2}}} también.
      \item Para toda fórmula de tipo \isa{{\isasymbeta}} con componentes \isa{{\isasymbeta}\isactrlsub {\isadigit{1}}} y \isa{{\isasymbeta}\isactrlsub {\isadigit{2}}} se verifica 
      que si la fórmula pertenece a \isa{L\isactrlsub S\isactrlsub C}, entonces o bien \isa{{\isasymbeta}\isactrlsub {\isadigit{1}}} pertenece
      a \isa{L\isactrlsub S\isactrlsub C} o bien \isa{{\isasymbeta}\isactrlsub {\isadigit{2}}} pertenece a \isa{L\isactrlsub S\isactrlsub C}.
    \end{itemize} 

    Observemos que las dos primeras condiciones coinciden con las obtenidas anteriormente para \isa{L\isactrlsub S\isactrlsub C} 
    por el lema de caracterización de la propiedad de consistencia proposicional mediante notación
    uniforme. Veamos que, en efecto, se cumplen el resto de condiciones.

    En primer lugar, probemos que para una fórmula \isa{F} de tipo \isa{{\isasymalpha}} y componentes \isa{{\isasymalpha}\isactrlsub {\isadigit{1}}} y \isa{{\isasymalpha}\isactrlsub {\isadigit{2}}} tal que 
    \isa{F\ {\isasymin}\ L\isactrlsub S\isactrlsub C} se verifica que tanto \isa{{\isasymalpha}\isactrlsub {\isadigit{1}}} como \isa{{\isasymalpha}\isactrlsub {\isadigit{2}}} pertenecen a \isa{L\isactrlsub S\isactrlsub C}. Por la tercera condición 
    obtenida anteriormente para \isa{L\isactrlsub S\isactrlsub C} por el lema de caracterización de la propiedad de consistencia 
    proposicional mediante notación uniforme, se cumple que\\ \isa{{\isacharbraceleft}{\isasymalpha}\isactrlsub {\isadigit{1}}{\isacharcomma}{\isasymalpha}\isactrlsub {\isadigit{2}}{\isacharbraceright}\ {\isasymunion}\ L\isactrlsub S\isactrlsub C\ {\isasymin}\ C}. Observemos que
    se verifica \isa{L\isactrlsub S\isactrlsub C\ {\isasymsubseteq}\ {\isacharbraceleft}{\isasymalpha}\isactrlsub {\isadigit{1}}{\isacharcomma}{\isasymalpha}\isactrlsub {\isadigit{2}}{\isacharbraceright}\ {\isasymunion}\ L\isactrlsub S\isactrlsub C}. De este modo, como \isa{C} es una colección con la propiedad de 
    consistencia proposicional y cerrada bajo subconjuntos, por el lema \isa{{\isadigit{4}}{\isachardot}{\isadigit{2}}{\isachardot}{\isadigit{2}}} se tiene que 
    los conjuntos \isa{L\isactrlsub S\isactrlsub C} y \isa{{\isacharbraceleft}{\isasymalpha}\isactrlsub {\isadigit{1}}{\isacharcomma}{\isasymalpha}\isactrlsub {\isadigit{2}}{\isacharbraceright}\ {\isasymunion}\ L\isactrlsub S\isactrlsub C} coinciden. Por tanto, es claro que \isa{{\isasymalpha}\isactrlsub {\isadigit{1}}\ {\isasymin}\ L\isactrlsub S\isactrlsub C} y \isa{{\isasymalpha}\isactrlsub {\isadigit{2}}\ {\isasymin}\ L\isactrlsub S\isactrlsub C}, 
    como queríamos demostrar.

    Por último, demostremos que para una fórmula \isa{F} de tipo \isa{{\isasymbeta}} y componentes \isa{{\isasymbeta}\isactrlsub {\isadigit{1}}} y \isa{{\isasymbeta}\isactrlsub {\isadigit{2}}} tal que
    \isa{F\ {\isasymin}\ L\isactrlsub S\isactrlsub C} se verifica que o bien \isa{{\isasymbeta}\isactrlsub {\isadigit{1}}\ {\isasymin}\ L\isactrlsub S\isactrlsub C} o bien \isa{{\isasymbeta}\isactrlsub {\isadigit{2}}\ {\isasymin}\ L\isactrlsub S\isactrlsub C}. Por la cuarta condición obtenida 
    anteriormente para \isa{L\isactrlsub S\isactrlsub C} por el lema de caracterización de la propiedad de consistencia 
    proposicional mediante notación uniforme, se cumple que o bien\\ \isa{{\isacharbraceleft}{\isasymbeta}\isactrlsub {\isadigit{1}}{\isacharbraceright}\ {\isasymunion}\ L\isactrlsub S\isactrlsub C\ {\isasymin}\ C} o bien 
    \isa{{\isacharbraceleft}{\isasymbeta}\isactrlsub {\isadigit{2}}{\isacharbraceright}\ {\isasymunion}\ L\isactrlsub S\isactrlsub C\ {\isasymin}\ C}. De este modo, si suponemos que \isa{{\isacharbraceleft}{\isasymbeta}\isactrlsub {\isadigit{1}}{\isacharbraceright}\ {\isasymunion}\ L\isactrlsub S\isactrlsub C\ {\isasymin}\ C}, como \isa{C} tiene la propiedad de 
    consistencia proposicional y es cerrada bajo subconjuntos, por el corolario \isa{{\isadigit{4}}{\isachardot}{\isadigit{2}}{\isachardot}{\isadigit{3}}} se tiene 
    que \isa{{\isasymbeta}\isactrlsub {\isadigit{1}}\ {\isasymin}\ L\isactrlsub S\isactrlsub C}. Por tanto, se cumple que o bien \isa{{\isasymbeta}\isactrlsub {\isadigit{1}}\ {\isasymin}\ L\isactrlsub S\isactrlsub C} o bien \isa{{\isasymbeta}\isactrlsub {\isadigit{2}}\ {\isasymin}\ L\isactrlsub S\isactrlsub C}. Si suponemos que 
    \isa{{\isacharbraceleft}{\isasymbeta}\isactrlsub {\isadigit{2}}{\isacharbraceright}\ {\isasymunion}\ L\isactrlsub S\isactrlsub C\ {\isasymin}\ C}, se observa fácilmente que llegamos a la misma conclusión de manera análoga. 
    Por lo tanto, queda probado el resultado.
  \end{demostracion}

  Veamos su formalización y prueba detallada en Isabelle.%
\end{isamarkuptext}\isamarkuptrue%
\isacommand{lemma}\isamarkupfalse%
\isanewline
\ \ \isakeyword{assumes}\ {\isachardoublequoteopen}pcp\ C{\isachardoublequoteclose}\isanewline
\ \ \isakeyword{assumes}\ {\isachardoublequoteopen}subset{\isacharunderscore}closed\ C{\isachardoublequoteclose}\isanewline
\ \ \isakeyword{assumes}\ {\isachardoublequoteopen}finite{\isacharunderscore}character\ C{\isachardoublequoteclose}\isanewline
\ \ \isakeyword{assumes}\ {\isachardoublequoteopen}S\ {\isasymin}\ C{\isachardoublequoteclose}\isanewline
\ \ \isakeyword{shows}\ {\isachardoublequoteopen}Hintikka\ {\isacharparenleft}pcp{\isacharunderscore}lim\ C\ S{\isacharparenright}{\isachardoublequoteclose}\isanewline
%
\isadelimproof
%
\endisadelimproof
%
\isatagproof
\isacommand{proof}\isamarkupfalse%
\ {\isacharparenleft}rule\ Hintikka{\isacharunderscore}alt{\isadigit{2}}{\isacharparenright}\isanewline
\ \ \isacommand{let}\isamarkupfalse%
\ {\isacharquery}cl\ {\isacharequal}\ {\isachardoublequoteopen}pcp{\isacharunderscore}lim\ C\ S{\isachardoublequoteclose}\isanewline
\ \ \isacommand{have}\isamarkupfalse%
\ {\isachardoublequoteopen}{\isacharquery}cl\ {\isasymin}\ C{\isachardoublequoteclose}\isanewline
\ \ \ \ \isacommand{using}\isamarkupfalse%
\ assms{\isacharparenleft}{\isadigit{1}}{\isacharparenright}\ assms{\isacharparenleft}{\isadigit{4}}{\isacharparenright}\ assms{\isacharparenleft}{\isadigit{2}}{\isacharparenright}\ assms{\isacharparenleft}{\isadigit{3}}{\isacharparenright}\ \isacommand{by}\isamarkupfalse%
\ {\isacharparenleft}rule\ pcp{\isacharunderscore}lim{\isacharunderscore}in{\isacharparenright}\isanewline
\ \ \isacommand{have}\isamarkupfalse%
\ {\isachardoublequoteopen}{\isacharparenleft}{\isasymforall}S\ {\isasymin}\ C{\isachardot}\isanewline
\ \ {\isasymbottom}\ {\isasymnotin}\ S\isanewline
{\isasymand}\ {\isacharparenleft}{\isasymforall}k{\isachardot}\ Atom\ k\ {\isasymin}\ S\ {\isasymlongrightarrow}\ \isactrlbold {\isasymnot}\ {\isacharparenleft}Atom\ k{\isacharparenright}\ {\isasymin}\ S\ {\isasymlongrightarrow}\ False{\isacharparenright}\isanewline
{\isasymand}\ {\isacharparenleft}{\isasymforall}F\ G\ H{\isachardot}\ Con\ F\ G\ H\ {\isasymlongrightarrow}\ F\ {\isasymin}\ S\ {\isasymlongrightarrow}\ {\isacharbraceleft}G{\isacharcomma}H{\isacharbraceright}\ {\isasymunion}\ S\ {\isasymin}\ C{\isacharparenright}\isanewline
{\isasymand}\ {\isacharparenleft}{\isasymforall}F\ G\ H{\isachardot}\ Dis\ F\ G\ H\ {\isasymlongrightarrow}\ F\ {\isasymin}\ S\ {\isasymlongrightarrow}\ {\isacharbraceleft}G{\isacharbraceright}\ {\isasymunion}\ S\ {\isasymin}\ C\ {\isasymor}\ {\isacharbraceleft}H{\isacharbraceright}\ {\isasymunion}\ S\ {\isasymin}\ C{\isacharparenright}{\isacharparenright}{\isachardoublequoteclose}\isanewline
\ \ \ \ \isacommand{using}\isamarkupfalse%
\ assms{\isacharparenleft}{\isadigit{1}}{\isacharparenright}\ \isacommand{by}\isamarkupfalse%
\ {\isacharparenleft}rule\ pcp{\isacharunderscore}alt{\isadigit{1}}{\isacharparenright}\isanewline
\ \ \isacommand{then}\isamarkupfalse%
\ \isacommand{have}\isamarkupfalse%
\ d{\isacharcolon}{\isachardoublequoteopen}{\isasymbottom}\ {\isasymnotin}\ {\isacharquery}cl\isanewline
{\isasymand}\ {\isacharparenleft}{\isasymforall}k{\isachardot}\ Atom\ k\ {\isasymin}\ {\isacharquery}cl\ {\isasymlongrightarrow}\ \isactrlbold {\isasymnot}\ {\isacharparenleft}Atom\ k{\isacharparenright}\ {\isasymin}\ {\isacharquery}cl\ {\isasymlongrightarrow}\ False{\isacharparenright}\isanewline
{\isasymand}\ {\isacharparenleft}{\isasymforall}F\ G\ H{\isachardot}\ Con\ F\ G\ H\ {\isasymlongrightarrow}\ F\ {\isasymin}\ {\isacharquery}cl\ {\isasymlongrightarrow}\ {\isacharbraceleft}G{\isacharcomma}H{\isacharbraceright}\ {\isasymunion}\ {\isacharquery}cl\ {\isasymin}\ C{\isacharparenright}\isanewline
{\isasymand}\ {\isacharparenleft}{\isasymforall}F\ G\ H{\isachardot}\ Dis\ F\ G\ H\ {\isasymlongrightarrow}\ F\ {\isasymin}\ {\isacharquery}cl\ {\isasymlongrightarrow}\ {\isacharbraceleft}G{\isacharbraceright}\ {\isasymunion}\ {\isacharquery}cl\ {\isasymin}\ C\ {\isasymor}\ {\isacharbraceleft}H{\isacharbraceright}\ {\isasymunion}\ {\isacharquery}cl\ {\isasymin}\ C{\isacharparenright}{\isachardoublequoteclose}\isanewline
\ \ \ \ \isacommand{using}\isamarkupfalse%
\ {\isacartoucheopen}{\isacharquery}cl\ {\isasymin}\ C{\isacartoucheclose}\ \isacommand{by}\isamarkupfalse%
\ {\isacharparenleft}rule\ bspec{\isacharparenright}\isanewline
\ \ \isacommand{then}\isamarkupfalse%
\ \isacommand{have}\isamarkupfalse%
\ H{\isadigit{1}}{\isacharcolon}{\isachardoublequoteopen}{\isasymbottom}\ {\isasymnotin}\ {\isacharquery}cl{\isachardoublequoteclose}\isanewline
\ \ \ \ \isacommand{by}\isamarkupfalse%
\ {\isacharparenleft}rule\ conjunct{\isadigit{1}}{\isacharparenright}\isanewline
\ \ \isacommand{have}\isamarkupfalse%
\ H{\isadigit{2}}{\isacharcolon}{\isachardoublequoteopen}{\isasymforall}k{\isachardot}\ Atom\ k\ {\isasymin}\ {\isacharquery}cl\ {\isasymlongrightarrow}\ \isactrlbold {\isasymnot}\ {\isacharparenleft}Atom\ k{\isacharparenright}\ {\isasymin}\ {\isacharquery}cl\ {\isasymlongrightarrow}\ False{\isachardoublequoteclose}\isanewline
\ \ \ \ \isacommand{using}\isamarkupfalse%
\ d\ \isacommand{by}\isamarkupfalse%
\ {\isacharparenleft}iprover\ elim{\isacharcolon}\ conjunct{\isadigit{2}}\ conjunct{\isadigit{1}}{\isacharparenright}\isanewline
\ \ \isacommand{have}\isamarkupfalse%
\ Con{\isacharcolon}{\isachardoublequoteopen}{\isasymforall}F\ G\ H{\isachardot}\ Con\ F\ G\ H\ {\isasymlongrightarrow}\ F\ {\isasymin}\ {\isacharquery}cl\ {\isasymlongrightarrow}\ {\isacharbraceleft}G{\isacharcomma}H{\isacharbraceright}\ {\isasymunion}\ {\isacharquery}cl\ {\isasymin}\ C{\isachardoublequoteclose}\isanewline
\ \ \ \ \isacommand{using}\isamarkupfalse%
\ d\ \isacommand{by}\isamarkupfalse%
\ {\isacharparenleft}iprover\ elim{\isacharcolon}\ conjunct{\isadigit{2}}\ conjunct{\isadigit{1}}{\isacharparenright}\isanewline
\ \ \isacommand{have}\isamarkupfalse%
\ H{\isadigit{3}}{\isacharcolon}{\isachardoublequoteopen}{\isasymforall}F\ G\ H{\isachardot}\ Con\ F\ G\ H\ {\isasymlongrightarrow}\ F\ {\isasymin}\ {\isacharquery}cl\ {\isasymlongrightarrow}\ G\ {\isasymin}\ {\isacharquery}cl\ {\isasymand}\ H\ {\isasymin}\ {\isacharquery}cl{\isachardoublequoteclose}\isanewline
\ \ \isacommand{proof}\isamarkupfalse%
\ {\isacharparenleft}rule\ allI{\isacharparenright}{\isacharplus}\isanewline
\ \ \ \ \isacommand{fix}\isamarkupfalse%
\ F\ G\ H\isanewline
\ \ \ \ \isacommand{show}\isamarkupfalse%
\ {\isachardoublequoteopen}Con\ F\ G\ H\ {\isasymlongrightarrow}\ F\ {\isasymin}\ {\isacharquery}cl\ {\isasymlongrightarrow}\ G\ {\isasymin}\ {\isacharquery}cl\ {\isasymand}\ H\ {\isasymin}\ {\isacharquery}cl{\isachardoublequoteclose}\isanewline
\ \ \ \ \isacommand{proof}\isamarkupfalse%
\ {\isacharparenleft}rule\ impI{\isacharparenright}{\isacharplus}\isanewline
\ \ \ \ \ \ \isacommand{assume}\isamarkupfalse%
\ {\isachardoublequoteopen}Con\ F\ G\ H{\isachardoublequoteclose}\isanewline
\ \ \ \ \ \ \isacommand{assume}\isamarkupfalse%
\ {\isachardoublequoteopen}F\ {\isasymin}\ {\isacharquery}cl{\isachardoublequoteclose}\isanewline
\ \ \ \ \ \ \isacommand{have}\isamarkupfalse%
\ {\isachardoublequoteopen}Con\ F\ G\ H\ {\isasymlongrightarrow}\ F\ {\isasymin}\ {\isacharquery}cl\ {\isasymlongrightarrow}\ {\isacharbraceleft}G{\isacharcomma}H{\isacharbraceright}\ {\isasymunion}\ {\isacharquery}cl\ {\isasymin}\ C{\isachardoublequoteclose}\isanewline
\ \ \ \ \ \ \ \ \isacommand{using}\isamarkupfalse%
\ Con\ \isacommand{by}\isamarkupfalse%
\ {\isacharparenleft}iprover\ elim{\isacharcolon}\ allE{\isacharparenright}\isanewline
\ \ \ \ \ \ \isacommand{then}\isamarkupfalse%
\ \isacommand{have}\isamarkupfalse%
\ {\isachardoublequoteopen}F\ {\isasymin}\ {\isacharquery}cl\ {\isasymlongrightarrow}\ {\isacharbraceleft}G{\isacharcomma}H{\isacharbraceright}\ {\isasymunion}\ {\isacharquery}cl\ {\isasymin}\ C{\isachardoublequoteclose}\isanewline
\ \ \ \ \ \ \ \ \isacommand{using}\isamarkupfalse%
\ {\isacartoucheopen}Con\ F\ G\ H{\isacartoucheclose}\ \isacommand{by}\isamarkupfalse%
\ {\isacharparenleft}rule\ mp{\isacharparenright}\isanewline
\ \ \ \ \ \ \isacommand{then}\isamarkupfalse%
\ \isacommand{have}\isamarkupfalse%
\ {\isachardoublequoteopen}{\isacharbraceleft}G{\isacharcomma}H{\isacharbraceright}\ {\isasymunion}\ {\isacharquery}cl\ {\isasymin}\ C{\isachardoublequoteclose}\isanewline
\ \ \ \ \ \ \ \ \isacommand{using}\isamarkupfalse%
\ {\isacartoucheopen}F\ {\isasymin}\ {\isacharquery}cl{\isacartoucheclose}\ \isacommand{by}\isamarkupfalse%
\ {\isacharparenleft}rule\ mp{\isacharparenright}\isanewline
\ \ \ \ \ \ \isacommand{have}\isamarkupfalse%
\ {\isachardoublequoteopen}{\isacharparenleft}insert\ G\ {\isacharparenleft}insert\ H\ {\isacharquery}cl{\isacharparenright}{\isacharparenright}\ {\isacharequal}\ {\isacharbraceleft}G{\isacharcomma}H{\isacharbraceright}\ {\isasymunion}\ {\isacharquery}cl{\isachardoublequoteclose}\isanewline
\ \ \ \ \ \ \ \ \isacommand{by}\isamarkupfalse%
\ {\isacharparenleft}rule\ insertSetElem{\isacharparenright}\isanewline
\ \ \ \ \ \ \isacommand{then}\isamarkupfalse%
\ \isacommand{have}\isamarkupfalse%
\ {\isachardoublequoteopen}{\isacharparenleft}insert\ G\ {\isacharparenleft}insert\ H\ {\isacharquery}cl{\isacharparenright}{\isacharparenright}\ {\isasymin}\ C{\isachardoublequoteclose}\isanewline
\ \ \ \ \ \ \ \ \isacommand{using}\isamarkupfalse%
\ {\isacartoucheopen}{\isacharbraceleft}G{\isacharcomma}H{\isacharbraceright}\ {\isasymunion}\ {\isacharquery}cl\ {\isasymin}\ C{\isacartoucheclose}\ \isacommand{by}\isamarkupfalse%
\ {\isacharparenleft}simp\ only{\isacharcolon}\ {\isacartoucheopen}{\isacharparenleft}insert\ G\ {\isacharparenleft}insert\ H\ {\isacharquery}cl{\isacharparenright}{\isacharparenright}\ {\isacharequal}\ {\isacharbraceleft}G{\isacharcomma}H{\isacharbraceright}\ {\isasymunion}\ {\isacharquery}cl{\isacartoucheclose}{\isacharparenright}\isanewline
\ \ \ \ \ \ \isacommand{have}\isamarkupfalse%
\ {\isachardoublequoteopen}{\isacharquery}cl\ {\isasymsubseteq}\ insert\ H\ {\isacharquery}cl{\isachardoublequoteclose}\isanewline
\ \ \ \ \ \ \ \ \isacommand{by}\isamarkupfalse%
\ {\isacharparenleft}rule\ subset{\isacharunderscore}insertI{\isacharparenright}\isanewline
\ \ \ \ \ \ \isacommand{then}\isamarkupfalse%
\ \isacommand{have}\isamarkupfalse%
\ {\isachardoublequoteopen}{\isacharquery}cl\ {\isasymsubseteq}\ insert\ G\ {\isacharparenleft}insert\ H\ {\isacharquery}cl{\isacharparenright}{\isachardoublequoteclose}\isanewline
\ \ \ \ \ \ \ \ \isacommand{by}\isamarkupfalse%
\ {\isacharparenleft}rule\ subset{\isacharunderscore}insertI{\isadigit{2}}{\isacharparenright}\isanewline
\ \ \ \ \ \ \isacommand{have}\isamarkupfalse%
\ {\isachardoublequoteopen}{\isacharquery}cl\ {\isacharequal}\ insert\ G\ {\isacharparenleft}insert\ H\ {\isacharquery}cl{\isacharparenright}{\isachardoublequoteclose}\ \isanewline
\ \ \ \ \ \ \ \ \isacommand{using}\isamarkupfalse%
\ assms{\isacharparenleft}{\isadigit{1}}{\isacharparenright}\ assms{\isacharparenleft}{\isadigit{2}}{\isacharparenright}\ {\isacartoucheopen}insert\ G\ {\isacharparenleft}insert\ H\ {\isacharquery}cl{\isacharparenright}\ {\isasymin}\ C{\isacartoucheclose}\ {\isacartoucheopen}{\isacharquery}cl\ {\isasymsubseteq}\ insert\ G\ {\isacharparenleft}insert\ H\ {\isacharquery}cl{\isacharparenright}{\isacartoucheclose}\ \isacommand{by}\isamarkupfalse%
\ {\isacharparenleft}rule\ cl{\isacharunderscore}max{\isacharparenright}\isanewline
\ \ \ \ \ \ \isacommand{then}\isamarkupfalse%
\ \isacommand{have}\isamarkupfalse%
\ {\isachardoublequoteopen}insert\ G\ {\isacharparenleft}insert\ H\ {\isacharquery}cl{\isacharparenright}\ {\isasymsubseteq}\ {\isacharquery}cl{\isachardoublequoteclose}\isanewline
\ \ \ \ \ \ \ \ \isacommand{by}\isamarkupfalse%
\ {\isacharparenleft}simp\ only{\isacharcolon}\ equalityD{\isadigit{2}}{\isacharparenright}\isanewline
\ \ \ \ \ \ \isacommand{then}\isamarkupfalse%
\ \isacommand{have}\isamarkupfalse%
\ {\isachardoublequoteopen}G\ {\isasymin}\ {\isacharquery}cl\ {\isasymand}\ insert\ H\ {\isacharquery}cl\ {\isasymsubseteq}\ {\isacharquery}cl{\isachardoublequoteclose}\isanewline
\ \ \ \ \ \ \ \ \isacommand{by}\isamarkupfalse%
\ {\isacharparenleft}simp\ only{\isacharcolon}\ insert{\isacharunderscore}subset{\isacharparenright}\isanewline
\ \ \ \ \ \ \isacommand{then}\isamarkupfalse%
\ \isacommand{have}\isamarkupfalse%
\ {\isachardoublequoteopen}G\ {\isasymin}\ {\isacharquery}cl{\isachardoublequoteclose}\isanewline
\ \ \ \ \ \ \ \ \isacommand{by}\isamarkupfalse%
\ {\isacharparenleft}rule\ conjunct{\isadigit{1}}{\isacharparenright}\isanewline
\ \ \ \ \ \ \isacommand{have}\isamarkupfalse%
\ {\isachardoublequoteopen}insert\ H\ {\isacharquery}cl\ {\isasymsubseteq}\ {\isacharquery}cl{\isachardoublequoteclose}\isanewline
\ \ \ \ \ \ \ \ \isacommand{using}\isamarkupfalse%
\ {\isacartoucheopen}G\ {\isasymin}\ {\isacharquery}cl\ {\isasymand}\ insert\ H\ {\isacharquery}cl\ {\isasymsubseteq}\ {\isacharquery}cl{\isacartoucheclose}\ \isacommand{by}\isamarkupfalse%
\ {\isacharparenleft}rule\ conjunct{\isadigit{2}}{\isacharparenright}\isanewline
\ \ \ \ \ \ \isacommand{then}\isamarkupfalse%
\ \isacommand{have}\isamarkupfalse%
\ {\isachardoublequoteopen}H\ {\isasymin}\ {\isacharquery}cl\ {\isasymand}\ {\isacharquery}cl\ {\isasymsubseteq}\ {\isacharquery}cl{\isachardoublequoteclose}\isanewline
\ \ \ \ \ \ \ \ \isacommand{by}\isamarkupfalse%
\ {\isacharparenleft}simp\ only{\isacharcolon}\ insert{\isacharunderscore}subset{\isacharparenright}\isanewline
\ \ \ \ \ \ \isacommand{then}\isamarkupfalse%
\ \isacommand{have}\isamarkupfalse%
\ {\isachardoublequoteopen}H\ {\isasymin}\ {\isacharquery}cl{\isachardoublequoteclose}\isanewline
\ \ \ \ \ \ \ \ \isacommand{by}\isamarkupfalse%
\ {\isacharparenleft}rule\ conjunct{\isadigit{1}}{\isacharparenright}\isanewline
\ \ \ \ \ \ \isacommand{show}\isamarkupfalse%
\ {\isachardoublequoteopen}G\ {\isasymin}\ {\isacharquery}cl\ {\isasymand}\ H\ {\isasymin}\ {\isacharquery}cl{\isachardoublequoteclose}\isanewline
\ \ \ \ \ \ \ \ \isacommand{using}\isamarkupfalse%
\ {\isacartoucheopen}G\ {\isasymin}\ {\isacharquery}cl{\isacartoucheclose}\ {\isacartoucheopen}H\ {\isasymin}\ {\isacharquery}cl{\isacartoucheclose}\ \isacommand{by}\isamarkupfalse%
\ {\isacharparenleft}rule\ conjI{\isacharparenright}\isanewline
\ \ \ \ \isacommand{qed}\isamarkupfalse%
\isanewline
\ \ \isacommand{qed}\isamarkupfalse%
\isanewline
\ \ \isacommand{have}\isamarkupfalse%
\ Dis{\isacharcolon}{\isachardoublequoteopen}{\isasymforall}F\ G\ H{\isachardot}\ Dis\ F\ G\ H\ {\isasymlongrightarrow}\ F\ {\isasymin}\ {\isacharquery}cl\ {\isasymlongrightarrow}\ {\isacharbraceleft}G{\isacharbraceright}\ {\isasymunion}\ {\isacharquery}cl\ {\isasymin}\ C\ {\isasymor}\ {\isacharbraceleft}H{\isacharbraceright}\ {\isasymunion}\ {\isacharquery}cl\ {\isasymin}\ C{\isachardoublequoteclose}\isanewline
\ \ \ \ \isacommand{using}\isamarkupfalse%
\ d\ \isacommand{by}\isamarkupfalse%
\ {\isacharparenleft}iprover\ elim{\isacharcolon}\ conjunct{\isadigit{2}}\ conjunct{\isadigit{1}}{\isacharparenright}\isanewline
\ \ \isacommand{have}\isamarkupfalse%
\ H{\isadigit{4}}{\isacharcolon}{\isachardoublequoteopen}{\isasymforall}F\ G\ H{\isachardot}\ Dis\ F\ G\ H\ {\isasymlongrightarrow}\ F\ {\isasymin}\ {\isacharquery}cl\ {\isasymlongrightarrow}\ G\ {\isasymin}\ {\isacharquery}cl\ {\isasymor}\ H\ {\isasymin}\ {\isacharquery}cl{\isachardoublequoteclose}\isanewline
\ \ \isacommand{proof}\isamarkupfalse%
\ {\isacharparenleft}rule\ allI{\isacharparenright}{\isacharplus}\isanewline
\ \ \ \ \isacommand{fix}\isamarkupfalse%
\ F\ G\ H\isanewline
\ \ \ \ \isacommand{show}\isamarkupfalse%
\ {\isachardoublequoteopen}Dis\ F\ G\ H\ {\isasymlongrightarrow}\ F\ {\isasymin}\ {\isacharquery}cl\ {\isasymlongrightarrow}\ G\ {\isasymin}\ {\isacharquery}cl\ {\isasymor}\ H\ {\isasymin}\ {\isacharquery}cl{\isachardoublequoteclose}\isanewline
\ \ \ \ \isacommand{proof}\isamarkupfalse%
\ {\isacharparenleft}rule\ impI{\isacharparenright}{\isacharplus}\isanewline
\ \ \ \ \ \ \isacommand{assume}\isamarkupfalse%
\ {\isachardoublequoteopen}Dis\ F\ G\ H{\isachardoublequoteclose}\isanewline
\ \ \ \ \ \ \isacommand{assume}\isamarkupfalse%
\ {\isachardoublequoteopen}F\ {\isasymin}\ {\isacharquery}cl{\isachardoublequoteclose}\isanewline
\ \ \ \ \ \ \isacommand{have}\isamarkupfalse%
\ {\isachardoublequoteopen}Dis\ F\ G\ H\ {\isasymlongrightarrow}\ F\ {\isasymin}\ {\isacharquery}cl\ {\isasymlongrightarrow}\ {\isacharbraceleft}G{\isacharbraceright}\ {\isasymunion}\ {\isacharquery}cl\ {\isasymin}\ C\ {\isasymor}\ {\isacharbraceleft}H{\isacharbraceright}\ {\isasymunion}\ {\isacharquery}cl\ {\isasymin}\ C{\isachardoublequoteclose}\isanewline
\ \ \ \ \ \ \ \ \isacommand{using}\isamarkupfalse%
\ Dis\ \isacommand{by}\isamarkupfalse%
\ {\isacharparenleft}iprover\ elim{\isacharcolon}\ allE{\isacharparenright}\isanewline
\ \ \ \ \ \ \isacommand{then}\isamarkupfalse%
\ \isacommand{have}\isamarkupfalse%
\ {\isachardoublequoteopen}F\ {\isasymin}\ {\isacharquery}cl\ {\isasymlongrightarrow}\ {\isacharbraceleft}G{\isacharbraceright}\ {\isasymunion}\ {\isacharquery}cl\ {\isasymin}\ C\ {\isasymor}\ {\isacharbraceleft}H{\isacharbraceright}\ {\isasymunion}\ {\isacharquery}cl\ {\isasymin}\ C{\isachardoublequoteclose}\isanewline
\ \ \ \ \ \ \ \ \isacommand{using}\isamarkupfalse%
\ {\isacartoucheopen}Dis\ F\ G\ H{\isacartoucheclose}\ \isacommand{by}\isamarkupfalse%
\ {\isacharparenleft}rule\ mp{\isacharparenright}\isanewline
\ \ \ \ \ \ \isacommand{then}\isamarkupfalse%
\ \isacommand{have}\isamarkupfalse%
\ {\isachardoublequoteopen}{\isacharbraceleft}G{\isacharbraceright}\ {\isasymunion}\ {\isacharquery}cl\ {\isasymin}\ C\ {\isasymor}\ {\isacharbraceleft}H{\isacharbraceright}\ {\isasymunion}\ {\isacharquery}cl\ {\isasymin}\ C{\isachardoublequoteclose}\isanewline
\ \ \ \ \ \ \ \ \isacommand{using}\isamarkupfalse%
\ {\isacartoucheopen}F\ {\isasymin}\ {\isacharquery}cl{\isacartoucheclose}\ \isacommand{by}\isamarkupfalse%
\ {\isacharparenleft}rule\ mp{\isacharparenright}\isanewline
\ \ \ \ \ \ \isacommand{thus}\isamarkupfalse%
\ {\isachardoublequoteopen}G\ {\isasymin}\ {\isacharquery}cl\ {\isasymor}\ H\ {\isasymin}\ {\isacharquery}cl{\isachardoublequoteclose}\isanewline
\ \ \ \ \ \ \isacommand{proof}\isamarkupfalse%
\ {\isacharparenleft}rule\ disjE{\isacharparenright}\isanewline
\ \ \ \ \ \ \ \ \isacommand{assume}\isamarkupfalse%
\ {\isachardoublequoteopen}{\isacharbraceleft}G{\isacharbraceright}\ {\isasymunion}\ {\isacharquery}cl\ {\isasymin}\ C{\isachardoublequoteclose}\isanewline
\ \ \ \ \ \ \ \ \isacommand{have}\isamarkupfalse%
\ {\isachardoublequoteopen}insert\ G\ {\isacharquery}cl\ {\isacharequal}\ {\isacharbraceleft}G{\isacharbraceright}\ {\isasymunion}\ {\isacharquery}cl{\isachardoublequoteclose}\isanewline
\ \ \ \ \ \ \ \ \ \ \isacommand{by}\isamarkupfalse%
\ {\isacharparenleft}rule\ insert{\isacharunderscore}is{\isacharunderscore}Un{\isacharparenright}\isanewline
\ \ \ \ \ \ \ \ \isacommand{have}\isamarkupfalse%
\ {\isachardoublequoteopen}insert\ G\ {\isacharquery}cl\ {\isasymin}\ C{\isachardoublequoteclose}\isanewline
\ \ \ \ \ \ \ \ \ \ \isacommand{using}\isamarkupfalse%
\ {\isacartoucheopen}{\isacharbraceleft}G{\isacharbraceright}\ {\isasymunion}\ {\isacharquery}cl\ {\isasymin}\ C{\isacartoucheclose}\ \isacommand{by}\isamarkupfalse%
\ {\isacharparenleft}simp\ only{\isacharcolon}\ {\isacartoucheopen}insert\ G\ {\isacharquery}cl\ {\isacharequal}\ {\isacharbraceleft}G{\isacharbraceright}\ {\isasymunion}\ {\isacharquery}cl{\isacartoucheclose}{\isacharparenright}\isanewline
\ \ \ \ \ \ \ \ \isacommand{have}\isamarkupfalse%
\ {\isachardoublequoteopen}insert\ G\ {\isacharquery}cl\ {\isasymin}\ C\ {\isasymLongrightarrow}\ G\ {\isasymin}\ {\isacharquery}cl{\isachardoublequoteclose}\isanewline
\ \ \ \ \ \ \ \ \ \ \isacommand{using}\isamarkupfalse%
\ assms{\isacharparenleft}{\isadigit{1}}{\isacharparenright}\ assms{\isacharparenleft}{\isadigit{2}}{\isacharparenright}\ \isacommand{by}\isamarkupfalse%
\ {\isacharparenleft}rule\ cl{\isacharunderscore}max{\isacharprime}{\isacharparenright}\isanewline
\ \ \ \ \ \ \ \ \isacommand{then}\isamarkupfalse%
\ \isacommand{have}\isamarkupfalse%
\ {\isachardoublequoteopen}G\ {\isasymin}\ {\isacharquery}cl{\isachardoublequoteclose}\isanewline
\ \ \ \ \ \ \ \ \ \ \isacommand{by}\isamarkupfalse%
\ {\isacharparenleft}simp\ only{\isacharcolon}\ {\isacartoucheopen}insert\ G\ {\isacharquery}cl\ {\isasymin}\ C{\isacartoucheclose}{\isacharparenright}\isanewline
\ \ \ \ \ \ \ \ \isacommand{thus}\isamarkupfalse%
\ {\isachardoublequoteopen}G\ {\isasymin}\ {\isacharquery}cl\ {\isasymor}\ H\ {\isasymin}\ {\isacharquery}cl{\isachardoublequoteclose}\isanewline
\ \ \ \ \ \ \ \ \ \ \isacommand{by}\isamarkupfalse%
\ {\isacharparenleft}rule\ disjI{\isadigit{1}}{\isacharparenright}\isanewline
\ \ \ \ \ \ \isacommand{next}\isamarkupfalse%
\isanewline
\ \ \ \ \ \ \ \ \isacommand{assume}\isamarkupfalse%
\ {\isachardoublequoteopen}{\isacharbraceleft}H{\isacharbraceright}\ {\isasymunion}\ {\isacharquery}cl\ {\isasymin}\ C{\isachardoublequoteclose}\isanewline
\ \ \ \ \ \ \ \ \isacommand{have}\isamarkupfalse%
\ {\isachardoublequoteopen}insert\ H\ {\isacharquery}cl\ {\isacharequal}\ {\isacharbraceleft}H{\isacharbraceright}\ {\isasymunion}\ {\isacharquery}cl{\isachardoublequoteclose}\isanewline
\ \ \ \ \ \ \ \ \ \ \isacommand{by}\isamarkupfalse%
\ {\isacharparenleft}rule\ insert{\isacharunderscore}is{\isacharunderscore}Un{\isacharparenright}\isanewline
\ \ \ \ \ \ \ \ \isacommand{have}\isamarkupfalse%
\ {\isachardoublequoteopen}insert\ H\ {\isacharquery}cl\ {\isasymin}\ C{\isachardoublequoteclose}\isanewline
\ \ \ \ \ \ \ \ \ \ \isacommand{using}\isamarkupfalse%
\ {\isacartoucheopen}{\isacharbraceleft}H{\isacharbraceright}\ {\isasymunion}\ {\isacharquery}cl\ {\isasymin}\ C{\isacartoucheclose}\ \isacommand{by}\isamarkupfalse%
\ {\isacharparenleft}simp\ only{\isacharcolon}\ {\isacartoucheopen}insert\ H\ {\isacharquery}cl\ {\isacharequal}\ {\isacharbraceleft}H{\isacharbraceright}\ {\isasymunion}\ {\isacharquery}cl{\isacartoucheclose}{\isacharparenright}\isanewline
\ \ \ \ \ \ \ \ \isacommand{have}\isamarkupfalse%
\ {\isachardoublequoteopen}insert\ H\ {\isacharquery}cl\ {\isasymin}\ C\ {\isasymLongrightarrow}\ H\ {\isasymin}\ {\isacharquery}cl{\isachardoublequoteclose}\isanewline
\ \ \ \ \ \ \ \ \ \ \isacommand{using}\isamarkupfalse%
\ assms{\isacharparenleft}{\isadigit{1}}{\isacharparenright}\ assms{\isacharparenleft}{\isadigit{2}}{\isacharparenright}\ \isacommand{by}\isamarkupfalse%
\ {\isacharparenleft}rule\ cl{\isacharunderscore}max{\isacharprime}{\isacharparenright}\isanewline
\ \ \ \ \ \ \ \ \isacommand{then}\isamarkupfalse%
\ \isacommand{have}\isamarkupfalse%
\ {\isachardoublequoteopen}H\ {\isasymin}\ {\isacharquery}cl{\isachardoublequoteclose}\isanewline
\ \ \ \ \ \ \ \ \ \ \isacommand{by}\isamarkupfalse%
\ {\isacharparenleft}simp\ only{\isacharcolon}\ {\isacartoucheopen}insert\ H\ {\isacharquery}cl\ {\isasymin}\ C{\isacartoucheclose}{\isacharparenright}\isanewline
\ \ \ \ \ \ \ \ \isacommand{thus}\isamarkupfalse%
\ {\isachardoublequoteopen}G\ {\isasymin}\ {\isacharquery}cl\ {\isasymor}\ H\ {\isasymin}\ {\isacharquery}cl{\isachardoublequoteclose}\isanewline
\ \ \ \ \ \ \ \ \ \ \isacommand{by}\isamarkupfalse%
\ {\isacharparenleft}rule\ disjI{\isadigit{2}}{\isacharparenright}\isanewline
\ \ \ \ \ \ \isacommand{qed}\isamarkupfalse%
\isanewline
\ \ \ \ \isacommand{qed}\isamarkupfalse%
\isanewline
\ \ \isacommand{qed}\isamarkupfalse%
\isanewline
\ \ \isacommand{show}\isamarkupfalse%
\ {\isachardoublequoteopen}{\isasymbottom}\ {\isasymnotin}\ {\isacharquery}cl\ {\isasymand}\isanewline
\ \ \ \ {\isacharparenleft}{\isasymforall}k{\isachardot}\ Atom\ k\ {\isasymin}\ {\isacharquery}cl\ {\isasymlongrightarrow}\ \isactrlbold {\isasymnot}\ {\isacharparenleft}Atom\ k{\isacharparenright}\ {\isasymin}\ {\isacharquery}cl\ {\isasymlongrightarrow}\ False{\isacharparenright}\ {\isasymand}\isanewline
\ \ \ \ {\isacharparenleft}{\isasymforall}F\ G\ H{\isachardot}\ Con\ F\ G\ H\ {\isasymlongrightarrow}\ F\ {\isasymin}\ {\isacharquery}cl\ {\isasymlongrightarrow}\ G\ {\isasymin}\ {\isacharquery}cl\ {\isasymand}\ H\ {\isasymin}\ {\isacharquery}cl{\isacharparenright}\ {\isasymand}\isanewline
\ \ \ \ {\isacharparenleft}{\isasymforall}F\ G\ H{\isachardot}\ Dis\ F\ G\ H\ {\isasymlongrightarrow}\ F\ {\isasymin}\ {\isacharquery}cl\ {\isasymlongrightarrow}\ G\ {\isasymin}\ {\isacharquery}cl\ {\isasymor}\ H\ {\isasymin}\ {\isacharquery}cl{\isacharparenright}{\isachardoublequoteclose}\isanewline
\ \ \ \ \isacommand{using}\isamarkupfalse%
\ H{\isadigit{1}}\ H{\isadigit{2}}\ H{\isadigit{3}}\ H{\isadigit{4}}\ \isacommand{by}\isamarkupfalse%
\ {\isacharparenleft}iprover\ intro{\isacharcolon}\ conjI{\isacharparenright}\isanewline
\isacommand{qed}\isamarkupfalse%
%
\endisatagproof
{\isafoldproof}%
%
\isadelimproof
%
\endisadelimproof
%
\begin{isamarkuptext}%
Del mismo modo, podemos probar el resultado de manera automática como sigue.%
\end{isamarkuptext}\isamarkuptrue%
\isacommand{lemma}\isamarkupfalse%
\ pcp{\isacharunderscore}lim{\isacharunderscore}Hintikka{\isacharcolon}\isanewline
\ \ \isakeyword{assumes}\ c{\isacharcolon}\ {\isachardoublequoteopen}pcp\ C{\isachardoublequoteclose}\isanewline
\ \ \isakeyword{assumes}\ sc{\isacharcolon}\ {\isachardoublequoteopen}subset{\isacharunderscore}closed\ C{\isachardoublequoteclose}\isanewline
\ \ \isakeyword{assumes}\ fc{\isacharcolon}\ {\isachardoublequoteopen}finite{\isacharunderscore}character\ C{\isachardoublequoteclose}\isanewline
\ \ \isakeyword{assumes}\ el{\isacharcolon}\ {\isachardoublequoteopen}S\ {\isasymin}\ C{\isachardoublequoteclose}\isanewline
\ \ \isakeyword{shows}\ {\isachardoublequoteopen}Hintikka\ {\isacharparenleft}pcp{\isacharunderscore}lim\ C\ S{\isacharparenright}{\isachardoublequoteclose}\isanewline
%
\isadelimproof
%
\endisadelimproof
%
\isatagproof
\isacommand{proof}\isamarkupfalse%
\ {\isacharminus}\isanewline
\ \ \isacommand{let}\isamarkupfalse%
\ {\isacharquery}cl\ {\isacharequal}\ {\isachardoublequoteopen}pcp{\isacharunderscore}lim\ C\ S{\isachardoublequoteclose}\isanewline
\ \ \isacommand{have}\isamarkupfalse%
\ {\isachardoublequoteopen}{\isacharquery}cl\ {\isasymin}\ C{\isachardoublequoteclose}\ \isacommand{using}\isamarkupfalse%
\ pcp{\isacharunderscore}lim{\isacharunderscore}in{\isacharbrackleft}OF\ c\ el\ sc\ fc{\isacharbrackright}\ \isacommand{{\isachardot}}\isamarkupfalse%
\isanewline
\ \ \isacommand{from}\isamarkupfalse%
\ c{\isacharbrackleft}unfolded\ pcp{\isacharunderscore}alt{\isacharcomma}\ THEN\ bspec{\isacharcomma}\ OF\ this{\isacharbrackright}\isanewline
\ \ \isacommand{have}\isamarkupfalse%
\ d{\isacharcolon}\ {\isachardoublequoteopen}{\isasymbottom}\ {\isasymnotin}\ {\isacharquery}cl{\isachardoublequoteclose}\isanewline
\ \ \ \ {\isachardoublequoteopen}Atom\ k\ {\isasymin}\ {\isacharquery}cl\ {\isasymLongrightarrow}\ \isactrlbold {\isasymnot}\ {\isacharparenleft}Atom\ k{\isacharparenright}\ {\isasymin}\ {\isacharquery}cl\ {\isasymLongrightarrow}\ False{\isachardoublequoteclose}\isanewline
\ \ \ \ {\isachardoublequoteopen}Con\ F\ G\ H\ {\isasymLongrightarrow}\ F\ {\isasymin}\ {\isacharquery}cl\ {\isasymLongrightarrow}\ insert\ G\ {\isacharparenleft}insert\ H\ {\isacharquery}cl{\isacharparenright}\ {\isasymin}\ C{\isachardoublequoteclose}\isanewline
\ \ \ \ {\isachardoublequoteopen}Dis\ F\ G\ H\ {\isasymLongrightarrow}\ F\ {\isasymin}\ {\isacharquery}cl\ {\isasymLongrightarrow}\ insert\ G\ {\isacharquery}cl\ {\isasymin}\ C\ {\isasymor}\ insert\ H\ {\isacharquery}cl\ {\isasymin}\ C{\isachardoublequoteclose}\isanewline
\ \ \ \ \isakeyword{for}\ k\ F\ G\ H\ \isacommand{by}\isamarkupfalse%
\ force{\isacharplus}\isanewline
\ \ \isacommand{with}\isamarkupfalse%
\ d{\isacharparenleft}{\isadigit{1}}{\isacharcomma}{\isadigit{2}}{\isacharparenright}\ \isacommand{show}\isamarkupfalse%
\ {\isachardoublequoteopen}Hintikka\ {\isacharquery}cl{\isachardoublequoteclose}\ \isacommand{unfolding}\isamarkupfalse%
\ Hintikka{\isacharunderscore}alt\ \isanewline
\ \ \ \ \isacommand{using}\isamarkupfalse%
\ c\ cl{\isacharunderscore}max\ cl{\isacharunderscore}max{\isacharprime}\ d{\isacharparenleft}{\isadigit{4}}{\isacharparenright}\ sc\ \isacommand{by}\isamarkupfalse%
\ blast\isanewline
\isacommand{qed}\isamarkupfalse%
%
\endisatagproof
{\isafoldproof}%
%
\isadelimproof
%
\endisadelimproof
%
\begin{isamarkuptext}%
Finalmente, vamos a demostrar el \isa{teorema\ de\ existencia\ de\ modelo}. Para ello precisaremos de
  un resultado que indica que la satisfacibilidad de conjuntos de fórmulas se hereda por la 
  contención.

  \begin{lema}
    Todo subconjunto de un conjunto de fórmulas satisfacible es satisfacible.
  \end{lema}

  \begin{demostracion}
    Sea \isa{B} un conjunto de fórmulas satisfacible y \isa{A\ {\isasymsubseteq}\ B}. Veamos que \isa{A} es satisfacible.
    Por definición, como \isa{B} es satisfacible, existe una interpretación \isa{{\isasymA}} que es modelo de cada 
    fórmula de \isa{B}. Como \isa{A\ {\isasymsubseteq}\ B}, en particular \isa{{\isasymA}} es modelo de toda fórmula de \isa{A}. Por tanto, 
    \isa{A} es satisfacible, ya que existe una interpretación que es modelo de todas sus fórmulas.
  \end{demostracion}

  Su prueba detallada en Isabelle/HOL es la siguiente.%
\end{isamarkuptext}\isamarkuptrue%
\isacommand{lemma}\isamarkupfalse%
\ sat{\isacharunderscore}mono{\isacharcolon}\isanewline
\ \ \isakeyword{assumes}\ {\isachardoublequoteopen}A\ {\isasymsubseteq}\ B{\isachardoublequoteclose}\isanewline
\ \ \ \ \ \ \ \ \ \ {\isachardoublequoteopen}sat\ B{\isachardoublequoteclose}\isanewline
\ \ \ \ \ \ \ \ \isakeyword{shows}\ {\isachardoublequoteopen}sat\ A{\isachardoublequoteclose}\isanewline
%
\isadelimproof
\ \ %
\endisadelimproof
%
\isatagproof
\isacommand{unfolding}\isamarkupfalse%
\ sat{\isacharunderscore}def\isanewline
\isacommand{proof}\isamarkupfalse%
\ {\isacharminus}\isanewline
\ \isacommand{have}\isamarkupfalse%
\ satB{\isacharcolon}{\isachardoublequoteopen}{\isasymexists}{\isasymA}{\isachardot}\ {\isasymforall}F\ {\isasymin}\ B{\isachardot}\ {\isasymA}\ {\isasymTurnstile}\ F{\isachardoublequoteclose}\isanewline
\ \ \ \isacommand{using}\isamarkupfalse%
\ assms{\isacharparenleft}{\isadigit{2}}{\isacharparenright}\ \isacommand{by}\isamarkupfalse%
\ {\isacharparenleft}simp\ only{\isacharcolon}\ sat{\isacharunderscore}def{\isacharparenright}\isanewline
\ \isacommand{obtain}\isamarkupfalse%
\ {\isasymA}\ \isakeyword{where}\ {\isachardoublequoteopen}{\isasymforall}F\ {\isasymin}\ B{\isachardot}\ {\isasymA}\ {\isasymTurnstile}\ F{\isachardoublequoteclose}\isanewline
\ \ \ \ \isacommand{using}\isamarkupfalse%
\ satB\ \isacommand{by}\isamarkupfalse%
\ {\isacharparenleft}rule\ exE{\isacharparenright}\isanewline
\ \isacommand{have}\isamarkupfalse%
\ {\isachardoublequoteopen}{\isasymforall}F\ {\isasymin}\ A{\isachardot}\ {\isasymA}\ {\isasymTurnstile}\ F{\isachardoublequoteclose}\isanewline
\ \ \isacommand{proof}\isamarkupfalse%
\ {\isacharparenleft}rule\ ballI{\isacharparenright}\isanewline
\ \ \ \ \isacommand{fix}\isamarkupfalse%
\ F\isanewline
\ \ \ \ \isacommand{assume}\isamarkupfalse%
\ {\isachardoublequoteopen}F\ {\isasymin}\ A{\isachardoublequoteclose}\isanewline
\ \ \ \ \isacommand{have}\isamarkupfalse%
\ {\isachardoublequoteopen}F\ {\isasymin}\ A\ {\isasymlongrightarrow}\ F\ {\isasymin}\ B{\isachardoublequoteclose}\isanewline
\ \ \ \ \ \ \isacommand{using}\isamarkupfalse%
\ assms{\isacharparenleft}{\isadigit{1}}{\isacharparenright}\ \isacommand{by}\isamarkupfalse%
\ {\isacharparenleft}rule\ in{\isacharunderscore}mono{\isacharparenright}\isanewline
\ \ \ \ \isacommand{then}\isamarkupfalse%
\ \isacommand{have}\isamarkupfalse%
\ {\isachardoublequoteopen}F\ {\isasymin}\ B{\isachardoublequoteclose}\isanewline
\ \ \ \ \ \ \isacommand{using}\isamarkupfalse%
\ {\isacartoucheopen}F\ {\isasymin}\ A{\isacartoucheclose}\ \isacommand{by}\isamarkupfalse%
\ {\isacharparenleft}rule\ mp{\isacharparenright}\isanewline
\ \ \ \ \isacommand{show}\isamarkupfalse%
\ {\isachardoublequoteopen}{\isasymA}\ {\isasymTurnstile}\ F{\isachardoublequoteclose}\isanewline
\ \ \ \ \ \ \isacommand{using}\isamarkupfalse%
\ {\isacartoucheopen}{\isasymforall}F\ {\isasymin}\ B{\isachardot}\ {\isasymA}\ {\isasymTurnstile}\ F{\isacartoucheclose}\ {\isacartoucheopen}F\ {\isasymin}\ B{\isacartoucheclose}\ \isacommand{by}\isamarkupfalse%
\ {\isacharparenleft}rule\ bspec{\isacharparenright}\isanewline
\ \ \isacommand{qed}\isamarkupfalse%
\isanewline
\ \ \isacommand{thus}\isamarkupfalse%
\ {\isachardoublequoteopen}{\isasymexists}{\isasymA}{\isachardot}\ {\isasymforall}F\ {\isasymin}\ A{\isachardot}\ {\isasymA}\ {\isasymTurnstile}\ F{\isachardoublequoteclose}\isanewline
\ \ \ \ \isacommand{by}\isamarkupfalse%
\ {\isacharparenleft}simp\ only{\isacharcolon}\ exI{\isacharparenright}\isanewline
\isacommand{qed}\isamarkupfalse%
%
\endisatagproof
{\isafoldproof}%
%
\isadelimproof
%
\endisadelimproof
%
\begin{isamarkuptext}%
De este modo, procedamos finalmente con la demostración del teorema.

  \begin{teorema}[Teorema de Existencia de Modelo]
    Todo conjunto de fórmulas perteneciente a una colección que verifique la propiedad de
    consistencia proposicional es satisfacible. 
  \end{teorema}

  \begin{demostracion}
    Sea \isa{C} una colección de conjuntos de fórmulas proposicionales que verifique la propiedad de 
    consistencia proposicional y \isa{S\ {\isasymin}\ C}. Vamos a probar que \isa{S} es satisfacible.

    En primer lugar, como \isa{C} verifica la propiedad de consistencia proposicional, por el lema 
    \isa{{\isadigit{3}}{\isachardot}{\isadigit{0}}{\isachardot}{\isadigit{3}}} podemos extenderla a una colección \isa{C{\isacharprime}} que también verifique la propiedad y
    sea cerrada bajo subconjuntos. A su vez, por el lema \isa{{\isadigit{3}}{\isachardot}{\isadigit{0}}{\isachardot}{\isadigit{5}}}, como la extensión 
    \isa{C{\isacharprime}} es una colección con la propiedad de consistencia proposicional y cerrada bajo 
    subconjuntos, podemos extenderla a otra colección \isa{C{\isacharprime}{\isacharprime}} que también verifica la propiedad de
    consistencia proposicional y sea de carácter finito. De este modo, por la transitividad de la 
    contención, es claro que \isa{C{\isacharprime}{\isacharprime}} es una extensión de \isa{C}, luego \isa{S\ {\isasymin}\ C{\isacharprime}{\isacharprime}} por hipótesis. 
    Por otro lado, por el lema \isa{{\isadigit{3}}{\isachardot}{\isadigit{0}}{\isachardot}{\isadigit{4}}}, como \isa{C{\isacharprime}{\isacharprime}} es de carácter finito, se tiene que es 
    cerrada bajo subconjuntos. 

    En suma, \isa{C{\isacharprime}{\isacharprime}} es una extensión de \isa{C} que verifica la propiedad de consistencia proposicional, 
    es cerrada bajo subconjuntos y es de carácter finito. Luego, por el lema \isa{{\isadigit{4}}{\isachardot}{\isadigit{2}}{\isachardot}{\isadigit{4}}}, el límite de 
    la sucesión \isa{{\isacharbraceleft}S\isactrlsub n{\isacharbraceright}} de conjuntos de \isa{C{\isacharprime}{\isacharprime}} a partir de \isa{S} según la definición \isa{{\isadigit{4}}{\isachardot}{\isadigit{1}}{\isachardot}{\isadigit{1}}} es un 
    conjunto de Hintikka. Por tanto, por el \isa{teorema\ de\ Hintikka}, se trata de un conjunto 
    satisfacible. 

    Finalmente, puesto que para todo \isa{n\ {\isasymin}\ {\isasymnat}} se tiene que \isa{S\isactrlsub n} está contenido en el límite, en 
    particular el conjunto \isa{S\isactrlsub {\isadigit{0}}} está contenido en él. Por definición de la sucesión, dicho conjunto 
    coincide con \isa{S}. Por tanto, como \isa{S} está contenido en el límite que es un conjunto 
    satisfacible, queda demostrada la satisfacibilidad de \isa{S}.
  \end{demostracion}

  Mostremos su formalización y demostración detallada en Isabelle.%
\end{isamarkuptext}\isamarkuptrue%
\isacommand{theorem}\isamarkupfalse%
\isanewline
\ \ \isakeyword{fixes}\ S\ {\isacharcolon}{\isacharcolon}\ {\isachardoublequoteopen}{\isacharprime}a\ {\isacharcolon}{\isacharcolon}\ countable\ formula\ set{\isachardoublequoteclose}\isanewline
\ \ \isakeyword{assumes}\ {\isachardoublequoteopen}pcp\ C{\isachardoublequoteclose}\isanewline
\ \ \isakeyword{assumes}\ {\isachardoublequoteopen}S\ {\isasymin}\ C{\isachardoublequoteclose}\isanewline
\ \ \isakeyword{shows}\ {\isachardoublequoteopen}sat\ S{\isachardoublequoteclose}\isanewline
%
\isadelimproof
%
\endisadelimproof
%
\isatagproof
\isacommand{proof}\isamarkupfalse%
\ {\isacharminus}\isanewline
\ \ \isacommand{have}\isamarkupfalse%
\ {\isachardoublequoteopen}pcp\ C\ {\isasymLongrightarrow}\ {\isasymexists}C{\isacharprime}{\isachardot}\ C\ {\isasymsubseteq}\ C{\isacharprime}\ {\isasymand}\ pcp\ C{\isacharprime}\ {\isasymand}\ subset{\isacharunderscore}closed\ C{\isacharprime}{\isachardoublequoteclose}\isanewline
\ \ \ \ \isacommand{by}\isamarkupfalse%
\ {\isacharparenleft}rule\ ex{\isadigit{1}}{\isacharparenright}\isanewline
\ \ \isacommand{then}\isamarkupfalse%
\ \isacommand{have}\isamarkupfalse%
\ E{\isadigit{1}}{\isacharcolon}{\isachardoublequoteopen}{\isasymexists}C{\isacharprime}{\isachardot}\ C\ {\isasymsubseteq}\ C{\isacharprime}\ {\isasymand}\ pcp\ C{\isacharprime}\ {\isasymand}\ subset{\isacharunderscore}closed\ C{\isacharprime}{\isachardoublequoteclose}\isanewline
\ \ \ \ \isacommand{by}\isamarkupfalse%
\ {\isacharparenleft}simp\ only{\isacharcolon}\ assms{\isacharparenleft}{\isadigit{1}}{\isacharparenright}{\isacharparenright}\isanewline
\ \ \isacommand{obtain}\isamarkupfalse%
\ Ce{\isacharprime}\ \isakeyword{where}\ H{\isadigit{1}}{\isacharcolon}{\isachardoublequoteopen}C\ {\isasymsubseteq}\ Ce{\isacharprime}\ {\isasymand}\ pcp\ Ce{\isacharprime}\ {\isasymand}\ subset{\isacharunderscore}closed\ Ce{\isacharprime}{\isachardoublequoteclose}\isanewline
\ \ \ \ \isacommand{using}\isamarkupfalse%
\ E{\isadigit{1}}\ \isacommand{by}\isamarkupfalse%
\ {\isacharparenleft}rule\ exE{\isacharparenright}\isanewline
\ \ \isacommand{have}\isamarkupfalse%
\ {\isachardoublequoteopen}C\ {\isasymsubseteq}\ Ce{\isacharprime}{\isachardoublequoteclose}\isanewline
\ \ \ \ \isacommand{using}\isamarkupfalse%
\ H{\isadigit{1}}\ \isacommand{by}\isamarkupfalse%
\ {\isacharparenleft}rule\ conjunct{\isadigit{1}}{\isacharparenright}\isanewline
\ \ \isacommand{have}\isamarkupfalse%
\ {\isachardoublequoteopen}pcp\ Ce{\isacharprime}{\isachardoublequoteclose}\isanewline
\ \ \ \ \isacommand{using}\isamarkupfalse%
\ H{\isadigit{1}}\ \isacommand{by}\isamarkupfalse%
\ {\isacharparenleft}iprover\ elim{\isacharcolon}\ conjunct{\isadigit{2}}\ conjunct{\isadigit{1}}{\isacharparenright}\isanewline
\ \ \isacommand{have}\isamarkupfalse%
\ {\isachardoublequoteopen}subset{\isacharunderscore}closed\ Ce{\isacharprime}{\isachardoublequoteclose}\isanewline
\ \ \ \ \isacommand{using}\isamarkupfalse%
\ H{\isadigit{1}}\ \isacommand{by}\isamarkupfalse%
\ {\isacharparenleft}iprover\ elim{\isacharcolon}\ conjunct{\isadigit{2}}\ conjunct{\isadigit{1}}{\isacharparenright}\isanewline
\ \ \isacommand{have}\isamarkupfalse%
\ E{\isadigit{2}}{\isacharcolon}{\isachardoublequoteopen}{\isasymexists}Ce{\isachardot}\ Ce{\isacharprime}\ {\isasymsubseteq}\ Ce\ {\isasymand}\ pcp\ Ce\ {\isasymand}\ finite{\isacharunderscore}character\ Ce{\isachardoublequoteclose}\isanewline
\ \ \ \ \isacommand{using}\isamarkupfalse%
\ {\isacartoucheopen}pcp\ Ce{\isacharprime}{\isacartoucheclose}\ {\isacartoucheopen}subset{\isacharunderscore}closed\ Ce{\isacharprime}{\isacartoucheclose}\ \isacommand{by}\isamarkupfalse%
\ {\isacharparenleft}rule\ ex{\isadigit{3}}{\isacharparenright}\isanewline
\ \ \isacommand{obtain}\isamarkupfalse%
\ Ce\ \isakeyword{where}\ H{\isadigit{2}}{\isacharcolon}{\isachardoublequoteopen}Ce{\isacharprime}\ {\isasymsubseteq}\ Ce\ {\isasymand}\ pcp\ Ce\ {\isasymand}\ finite{\isacharunderscore}character\ Ce{\isachardoublequoteclose}\isanewline
\ \ \ \ \isacommand{using}\isamarkupfalse%
\ E{\isadigit{2}}\ \isacommand{by}\isamarkupfalse%
\ {\isacharparenleft}rule\ exE{\isacharparenright}\isanewline
\ \ \isacommand{have}\isamarkupfalse%
\ {\isachardoublequoteopen}Ce{\isacharprime}\ {\isasymsubseteq}\ Ce{\isachardoublequoteclose}\isanewline
\ \ \ \ \isacommand{using}\isamarkupfalse%
\ H{\isadigit{2}}\ \isacommand{by}\isamarkupfalse%
\ {\isacharparenleft}rule\ conjunct{\isadigit{1}}{\isacharparenright}\isanewline
\ \ \isacommand{then}\isamarkupfalse%
\ \isacommand{have}\isamarkupfalse%
\ Subset{\isacharcolon}{\isachardoublequoteopen}C\ {\isasymsubseteq}\ Ce{\isachardoublequoteclose}\isanewline
\ \ \ \ \isacommand{using}\isamarkupfalse%
\ {\isacartoucheopen}C\ {\isasymsubseteq}\ Ce{\isacharprime}{\isacartoucheclose}\ \isacommand{by}\isamarkupfalse%
\ {\isacharparenleft}simp\ only{\isacharcolon}\ subset{\isacharunderscore}trans{\isacharparenright}\isanewline
\ \ \isacommand{have}\isamarkupfalse%
\ Pcp{\isacharcolon}{\isachardoublequoteopen}pcp\ Ce{\isachardoublequoteclose}\isanewline
\ \ \ \ \isacommand{using}\isamarkupfalse%
\ H{\isadigit{2}}\ \isacommand{by}\isamarkupfalse%
\ {\isacharparenleft}iprover\ elim{\isacharcolon}\ conjunct{\isadigit{2}}\ conjunct{\isadigit{1}}{\isacharparenright}\isanewline
\ \ \isacommand{have}\isamarkupfalse%
\ FC{\isacharcolon}{\isachardoublequoteopen}finite{\isacharunderscore}character\ Ce{\isachardoublequoteclose}\isanewline
\ \ \ \ \isacommand{using}\isamarkupfalse%
\ H{\isadigit{2}}\ \isacommand{by}\isamarkupfalse%
\ {\isacharparenleft}iprover\ elim{\isacharcolon}\ conjunct{\isadigit{2}}\ conjunct{\isadigit{1}}{\isacharparenright}\isanewline
\ \ \isacommand{then}\isamarkupfalse%
\ \isacommand{have}\isamarkupfalse%
\ SC{\isacharcolon}{\isachardoublequoteopen}subset{\isacharunderscore}closed\ Ce{\isachardoublequoteclose}\isanewline
\ \ \ \ \isacommand{by}\isamarkupfalse%
\ {\isacharparenleft}rule\ ex{\isadigit{2}}{\isacharparenright}\isanewline
\ \ \isacommand{have}\isamarkupfalse%
\ {\isachardoublequoteopen}S\ {\isasymin}\ C\ {\isasymlongrightarrow}\ S\ {\isasymin}\ Ce{\isachardoublequoteclose}\isanewline
\ \ \ \ \isacommand{using}\isamarkupfalse%
\ {\isacartoucheopen}C\ {\isasymsubseteq}\ Ce{\isacartoucheclose}\ \isacommand{by}\isamarkupfalse%
\ {\isacharparenleft}rule\ in{\isacharunderscore}mono{\isacharparenright}\isanewline
\ \ \isacommand{then}\isamarkupfalse%
\ \isacommand{have}\isamarkupfalse%
\ {\isachardoublequoteopen}S\ {\isasymin}\ Ce{\isachardoublequoteclose}\ \isanewline
\ \ \ \ \isacommand{using}\isamarkupfalse%
\ assms{\isacharparenleft}{\isadigit{2}}{\isacharparenright}\ \isacommand{by}\isamarkupfalse%
\ {\isacharparenleft}rule\ mp{\isacharparenright}\isanewline
\ \ \isacommand{have}\isamarkupfalse%
\ {\isachardoublequoteopen}Hintikka\ {\isacharparenleft}pcp{\isacharunderscore}lim\ Ce\ S{\isacharparenright}{\isachardoublequoteclose}\isanewline
\ \ \ \ \isacommand{using}\isamarkupfalse%
\ Pcp\ SC\ FC\ {\isacartoucheopen}S\ {\isasymin}\ Ce{\isacartoucheclose}\ \isacommand{by}\isamarkupfalse%
\ {\isacharparenleft}rule\ pcp{\isacharunderscore}lim{\isacharunderscore}Hintikka{\isacharparenright}\isanewline
\ \ \isacommand{then}\isamarkupfalse%
\ \isacommand{have}\isamarkupfalse%
\ {\isachardoublequoteopen}sat\ {\isacharparenleft}pcp{\isacharunderscore}lim\ Ce\ S{\isacharparenright}{\isachardoublequoteclose}\isanewline
\ \ \ \ \isacommand{by}\isamarkupfalse%
\ {\isacharparenleft}rule\ Hintikkaslemma{\isacharparenright}\isanewline
\ \ \isacommand{have}\isamarkupfalse%
\ {\isachardoublequoteopen}pcp{\isacharunderscore}seq\ Ce\ S\ {\isadigit{0}}\ {\isacharequal}\ S{\isachardoublequoteclose}\isanewline
\ \ \ \ \isacommand{by}\isamarkupfalse%
\ {\isacharparenleft}simp\ only{\isacharcolon}\ pcp{\isacharunderscore}seq{\isachardot}simps{\isacharparenleft}{\isadigit{1}}{\isacharparenright}{\isacharparenright}\isanewline
\ \ \isacommand{have}\isamarkupfalse%
\ {\isachardoublequoteopen}pcp{\isacharunderscore}seq\ Ce\ S\ {\isadigit{0}}\ {\isasymsubseteq}\ pcp{\isacharunderscore}lim\ Ce\ S{\isachardoublequoteclose}\isanewline
\ \ \ \ \isacommand{by}\isamarkupfalse%
\ {\isacharparenleft}rule\ pcp{\isacharunderscore}seq{\isacharunderscore}sub{\isacharparenright}\isanewline
\ \ \isacommand{then}\isamarkupfalse%
\ \isacommand{have}\isamarkupfalse%
\ {\isachardoublequoteopen}S\ {\isasymsubseteq}\ pcp{\isacharunderscore}lim\ Ce\ S{\isachardoublequoteclose}\isanewline
\ \ \ \ \isacommand{by}\isamarkupfalse%
\ {\isacharparenleft}simp\ only{\isacharcolon}\ {\isacartoucheopen}pcp{\isacharunderscore}seq\ Ce\ S\ {\isadigit{0}}\ {\isacharequal}\ S{\isacartoucheclose}{\isacharparenright}\isanewline
\ \ \isacommand{thus}\isamarkupfalse%
\ {\isachardoublequoteopen}sat\ S{\isachardoublequoteclose}\isanewline
\ \ \ \ \isacommand{using}\isamarkupfalse%
\ {\isacartoucheopen}sat\ {\isacharparenleft}pcp{\isacharunderscore}lim\ Ce\ S{\isacharparenright}{\isacartoucheclose}\ \isacommand{by}\isamarkupfalse%
\ {\isacharparenleft}rule\ sat{\isacharunderscore}mono{\isacharparenright}\isanewline
\isacommand{qed}\isamarkupfalse%
%
\endisatagproof
{\isafoldproof}%
%
\isadelimproof
%
\endisadelimproof
%
\begin{isamarkuptext}%
Finalmente, demostremos el teorema de manera automática.%
\end{isamarkuptext}\isamarkuptrue%
\isacommand{theorem}\isamarkupfalse%
\ pcp{\isacharunderscore}sat{\isacharcolon}\isanewline
\ \ \isakeyword{fixes}\ S\ {\isacharcolon}{\isacharcolon}\ {\isachardoublequoteopen}{\isacharprime}a\ {\isacharcolon}{\isacharcolon}\ countable\ formula\ set{\isachardoublequoteclose}\isanewline
\ \ \isakeyword{assumes}\ c{\isacharcolon}\ {\isachardoublequoteopen}pcp\ C{\isachardoublequoteclose}\isanewline
\ \ \isakeyword{assumes}\ el{\isacharcolon}\ {\isachardoublequoteopen}S\ {\isasymin}\ C{\isachardoublequoteclose}\isanewline
\ \ \isakeyword{shows}\ {\isachardoublequoteopen}sat\ S{\isachardoublequoteclose}\isanewline
%
\isadelimproof
%
\endisadelimproof
%
\isatagproof
\isacommand{proof}\isamarkupfalse%
\ {\isacharminus}\isanewline
\ \ \isacommand{from}\isamarkupfalse%
\ c\ \isacommand{obtain}\isamarkupfalse%
\ Ce\ \isakeyword{where}\ \isanewline
\ \ \ \ \ \ {\isachardoublequoteopen}C\ {\isasymsubseteq}\ Ce{\isachardoublequoteclose}\ {\isachardoublequoteopen}pcp\ Ce{\isachardoublequoteclose}\ {\isachardoublequoteopen}subset{\isacharunderscore}closed\ Ce{\isachardoublequoteclose}\ {\isachardoublequoteopen}finite{\isacharunderscore}character\ Ce{\isachardoublequoteclose}\ \isanewline
\ \ \ \ \ \ \isacommand{using}\isamarkupfalse%
\ ex{\isadigit{1}}{\isacharbrackleft}\isakeyword{where}\ {\isacharprime}a{\isacharequal}{\isacharprime}a{\isacharbrackright}\ ex{\isadigit{2}}{\isacharbrackleft}\isakeyword{where}\ {\isacharprime}a{\isacharequal}{\isacharprime}a{\isacharbrackright}\ ex{\isadigit{3}}{\isacharbrackleft}\isakeyword{where}\ {\isacharprime}a{\isacharequal}{\isacharprime}a{\isacharbrackright}\isanewline
\ \ \ \ \isacommand{by}\isamarkupfalse%
\ {\isacharparenleft}meson\ dual{\isacharunderscore}order{\isachardot}trans\ ex{\isadigit{2}}{\isacharparenright}\isanewline
\ \ \isacommand{have}\isamarkupfalse%
\ {\isachardoublequoteopen}S\ {\isasymin}\ Ce{\isachardoublequoteclose}\ \isacommand{using}\isamarkupfalse%
\ {\isacartoucheopen}C\ {\isasymsubseteq}\ Ce{\isacartoucheclose}\ el\ \isacommand{{\isachardot}{\isachardot}}\isamarkupfalse%
\isanewline
\ \ \isacommand{with}\isamarkupfalse%
\ pcp{\isacharunderscore}lim{\isacharunderscore}Hintikka\ {\isacartoucheopen}pcp\ Ce{\isacartoucheclose}\ {\isacartoucheopen}subset{\isacharunderscore}closed\ Ce{\isacartoucheclose}\ {\isacartoucheopen}finite{\isacharunderscore}character\ Ce{\isacartoucheclose}\isanewline
\ \ \isacommand{have}\isamarkupfalse%
\ \ {\isachardoublequoteopen}Hintikka\ {\isacharparenleft}pcp{\isacharunderscore}lim\ Ce\ S{\isacharparenright}{\isachardoublequoteclose}\ \isacommand{{\isachardot}}\isamarkupfalse%
\isanewline
\ \ \isacommand{with}\isamarkupfalse%
\ Hintikkaslemma\ \isacommand{have}\isamarkupfalse%
\ {\isachardoublequoteopen}sat\ {\isacharparenleft}pcp{\isacharunderscore}lim\ Ce\ S{\isacharparenright}{\isachardoublequoteclose}\ \isacommand{{\isachardot}}\isamarkupfalse%
\isanewline
\ \ \isacommand{moreover}\isamarkupfalse%
\ \isacommand{have}\isamarkupfalse%
\ {\isachardoublequoteopen}S\ {\isasymsubseteq}\ pcp{\isacharunderscore}lim\ Ce\ S{\isachardoublequoteclose}\ \isanewline
\ \ \ \ \isacommand{using}\isamarkupfalse%
\ pcp{\isacharunderscore}seq{\isachardot}simps{\isacharparenleft}{\isadigit{1}}{\isacharparenright}\ pcp{\isacharunderscore}seq{\isacharunderscore}sub\ \isacommand{by}\isamarkupfalse%
\ fast\isanewline
\ \ \isacommand{ultimately}\isamarkupfalse%
\ \isacommand{show}\isamarkupfalse%
\ {\isacharquery}thesis\ \isacommand{unfolding}\isamarkupfalse%
\ sat{\isacharunderscore}def\ \isacommand{by}\isamarkupfalse%
\ fast\isanewline
\isacommand{qed}\isamarkupfalse%
%
\endisatagproof
{\isafoldproof}%
%
\isadelimproof
%
\endisadelimproof
%
\isadelimdocument
%
\endisadelimdocument
%
\isatagdocument
%
\isamarkupsection{Teorema de Compacidad%
}
\isamarkuptrue%
%
\endisatagdocument
{\isafolddocument}%
%
\isadelimdocument
%
\endisadelimdocument
%
\begin{isamarkuptext}%
En esta sección vamos demostrar el \isa{Teorema\ de\ Compacidad} para la lógica proposicional
  como consecuencia del \isa{Teorema\ de\ Existencia\ de\ Modelo}.

  \begin{teorema}[Teorema de Compacidad]
    Todo conjunto de fórmulas finitamente satisfacible es satisfacible.
  \end{teorema}

  Para su demostración consideraremos la colección formada por los conjuntos de fórmulas finitamente 
  satisfacibles. Probaremos que dicha colección verifica la propiedad de consistencia proposicional
  y, por el \isa{Teorema\ de\ Existencia\ de\ Modelo}, todo conjunto perteneciente a ella será
  satisfacible, demostrando así el teorema.

  Mostremos previamente dos resultados sobre subconjuntos finitos que emplearemos en la 
  demostración del teorema.

  \begin{lema}
    Sea \isa{S} un conjunto finito tal que \isa{S\ {\isasymsubseteq}\ {\isacharbraceleft}a{\isacharbraceright}\ {\isasymunion}\ B}. Entonces, existe un conjunto finito \isa{S{\isacharprime}\ {\isasymsubseteq}\ B} 
    tal que o bien \isa{S\ {\isacharequal}\ {\isacharbraceleft}a{\isacharbraceright}\ {\isasymunion}\ S{\isacharprime}} o bien \isa{S\ {\isacharequal}\ S{\isacharprime}}.
  \end{lema}

  \begin{demostracion}
   La prueba se realiza considerando dos casos: \isa{a\ {\isasymin}\ S} o \isa{a\ {\isasymnotin}\ S}.

   Por un lado, si consideramos que \isa{a\ {\isasymin}\ S}, basta tomar el conjunto \isa{S{\isacharprime}\ {\isacharequal}\ S\ {\isacharminus}\ {\isacharbraceleft}a{\isacharbraceright}}. 
   De este modo, como \isa{S\ {\isasymsubseteq}\ {\isacharbraceleft}a{\isacharbraceright}\ {\isasymunion}\ B}, es claro que \isa{S{\isacharprime}\ {\isasymsubseteq}\ B}. Además, puesto que \isa{S} es finito,
   se tiene que \isa{S{\isacharprime}} también lo es. Finalmente, se observa fácilmente que cumple que \isa{S\ {\isacharequal}\ S{\isacharprime}\ {\isasymunion}\ {\isacharbraceleft}a{\isacharbraceright}}.

   Por otro lado, supongamos que \isa{a\ {\isasymnotin}\ S}. En este caso, si tomamos \isa{S{\isacharprime}\ {\isacharequal}\ S} se verifica que
   \isa{S{\isacharprime}\ {\isasymsubseteq}\ B} ya que \isa{S\ {\isasymsubseteq}\ {\isacharbraceleft}a{\isacharbraceright}\ {\isasymunion}\ B} y \isa{a\ {\isasymnotin}\ S}. Además, \isa{S{\isacharprime}} es finito por serlo \isa{S}. Luego cumple
   las condiciones del resultado, como queríamos demostrar.
  \end{demostracion}

  Procedamos con la prueba detallada y formalización en Isabelle. Para ello, hemos utilizado el
  siguiente lema auxiliar.%
\end{isamarkuptext}\isamarkuptrue%
\isacommand{lemma}\isamarkupfalse%
\ subexI\ {\isacharbrackleft}intro{\isacharbrackright}{\isacharcolon}\ {\isachardoublequoteopen}P\ A\ {\isasymLongrightarrow}\ A\ {\isasymsubseteq}\ B\ {\isasymLongrightarrow}\ {\isasymexists}A{\isasymsubseteq}B{\isachardot}\ P\ A{\isachardoublequoteclose}\isanewline
%
\isadelimproof
\ \ %
\endisadelimproof
%
\isatagproof
\isacommand{by}\isamarkupfalse%
\ blast%
\endisatagproof
{\isafoldproof}%
%
\isadelimproof
%
\endisadelimproof
%
\begin{isamarkuptext}%
De este modo, probemos detalladamente el resultado.%
\end{isamarkuptext}\isamarkuptrue%
\isacommand{lemma}\isamarkupfalse%
\ finite{\isacharunderscore}subset{\isacharunderscore}insert{\isadigit{1}}{\isacharcolon}\isanewline
\ \ \isakeyword{assumes}\ {\isachardoublequoteopen}finite\ S{\isachardoublequoteclose}\isanewline
\ \ \ \ \ \ \ \ \ \ {\isachardoublequoteopen}S\ {\isasymsubseteq}\ insert\ a\ B{\isachardoublequoteclose}\isanewline
\ \ \isakeyword{shows}\ {\isachardoublequoteopen}{\isasymexists}S{\isacharprime}\ {\isasymsubseteq}\ B{\isachardot}\ finite\ S{\isacharprime}\ {\isasymand}\ {\isacharparenleft}S\ {\isacharequal}\ insert\ a\ S{\isacharprime}\ {\isasymor}\ S\ {\isacharequal}\ S{\isacharprime}{\isacharparenright}{\isachardoublequoteclose}\isanewline
%
\isadelimproof
%
\endisadelimproof
%
\isatagproof
\isacommand{proof}\isamarkupfalse%
\ {\isacharparenleft}cases\ {\isachardoublequoteopen}a\ {\isasymin}\ S{\isachardoublequoteclose}{\isacharparenright}\isanewline
\ \ \isacommand{assume}\isamarkupfalse%
\ {\isachardoublequoteopen}a\ {\isasymin}\ S{\isachardoublequoteclose}\isanewline
\ \ \isacommand{then}\isamarkupfalse%
\ \isacommand{have}\isamarkupfalse%
\ {\isachardoublequoteopen}S\ {\isacharequal}\ insert\ a\ {\isacharparenleft}S\ {\isacharminus}\ {\isacharbraceleft}a{\isacharbraceright}{\isacharparenright}{\isachardoublequoteclose}\ \isanewline
\ \ \ \ \isacommand{by}\isamarkupfalse%
\ {\isacharparenleft}simp\ only{\isacharcolon}\ insert{\isacharunderscore}Diff{\isacharbrackleft}THEN\ sym{\isacharbrackright}{\isacharparenright}\isanewline
\ \ \isacommand{then}\isamarkupfalse%
\ \isacommand{have}\isamarkupfalse%
\ {\isadigit{1}}{\isacharcolon}{\isachardoublequoteopen}S\ {\isacharequal}\ insert\ a\ {\isacharparenleft}S\ {\isacharminus}\ {\isacharbraceleft}a{\isacharbraceright}{\isacharparenright}\ {\isasymor}\ S\ {\isacharequal}\ S\ {\isacharminus}\ {\isacharbraceleft}a{\isacharbraceright}{\isachardoublequoteclose}\isanewline
\ \ \ \ \isacommand{by}\isamarkupfalse%
\ {\isacharparenleft}rule\ disjI{\isadigit{1}}{\isacharparenright}\isanewline
\ \ \isacommand{have}\isamarkupfalse%
\ {\isadigit{2}}{\isacharcolon}{\isachardoublequoteopen}finite\ {\isacharparenleft}S\ {\isacharminus}\ {\isacharbraceleft}a{\isacharbraceright}{\isacharparenright}{\isachardoublequoteclose}\ \isanewline
\ \ \ \ \isacommand{using}\isamarkupfalse%
\ assms{\isacharparenleft}{\isadigit{1}}{\isacharparenright}\ \isacommand{by}\isamarkupfalse%
\ {\isacharparenleft}rule\ finite{\isacharunderscore}Diff{\isacharparenright}\isanewline
\ \ \isacommand{have}\isamarkupfalse%
\ {\isadigit{3}}{\isacharcolon}{\isachardoublequoteopen}{\isacharparenleft}S\ {\isacharminus}\ {\isacharbraceleft}a{\isacharbraceright}{\isacharparenright}\ {\isasymsubseteq}\ B{\isachardoublequoteclose}\ \isanewline
\ \ \ \ \isacommand{using}\isamarkupfalse%
\ assms{\isacharparenleft}{\isadigit{2}}{\isacharparenright}\ \isacommand{by}\isamarkupfalse%
\ {\isacharparenleft}simp\ add{\isacharcolon}\ Diff{\isacharunderscore}subset{\isacharunderscore}conv{\isacharparenright}\isanewline
\ \ \isacommand{have}\isamarkupfalse%
\ {\isachardoublequoteopen}finite\ {\isacharparenleft}S\ {\isacharminus}\ {\isacharbraceleft}a{\isacharbraceright}{\isacharparenright}\ {\isasymand}\ {\isacharparenleft}S\ {\isacharequal}\ insert\ a\ {\isacharparenleft}S\ {\isacharminus}\ {\isacharbraceleft}a{\isacharbraceright}{\isacharparenright}\ {\isasymor}\ S\ {\isacharequal}\ S\ {\isacharminus}\ {\isacharbraceleft}a{\isacharbraceright}{\isacharparenright}{\isachardoublequoteclose}\isanewline
\ \ \ \ \isacommand{using}\isamarkupfalse%
\ {\isadigit{2}}\ {\isadigit{1}}\ \isacommand{by}\isamarkupfalse%
\ {\isacharparenleft}rule\ conjI{\isacharparenright}\isanewline
\ \ \isacommand{thus}\isamarkupfalse%
\ {\isacharquery}thesis\ \isacommand{using}\isamarkupfalse%
\ {\isadigit{3}}\ \isacommand{by}\isamarkupfalse%
\ {\isacharparenleft}rule\ subexI{\isacharparenright}\isanewline
\isacommand{next}\isamarkupfalse%
\isanewline
\ \ \isacommand{assume}\isamarkupfalse%
\ {\isachardoublequoteopen}a\ {\isasymnotin}\ S{\isachardoublequoteclose}\isanewline
\ \ \isacommand{then}\isamarkupfalse%
\ \isacommand{have}\isamarkupfalse%
\ {\isadigit{1}}{\isacharcolon}{\isachardoublequoteopen}S\ {\isasymsubseteq}\ B{\isachardoublequoteclose}\ \isanewline
\ \ \ \ \isacommand{using}\isamarkupfalse%
\ assms{\isacharparenleft}{\isadigit{2}}{\isacharparenright}\ \isacommand{by}\isamarkupfalse%
\ blast\ \isanewline
\ \ \isacommand{have}\isamarkupfalse%
\ {\isadigit{2}}{\isacharcolon}{\isachardoublequoteopen}S\ {\isacharequal}\ insert\ a\ S\ {\isasymor}\ S\ {\isacharequal}\ S{\isachardoublequoteclose}\isanewline
\ \ \ \ \isacommand{by}\isamarkupfalse%
\ {\isacharparenleft}simp\ only{\isacharcolon}\ disjI{\isadigit{1}}\ simp{\isacharunderscore}thms{\isacharparenright}\isanewline
\ \ \isacommand{have}\isamarkupfalse%
\ {\isachardoublequoteopen}finite\ S\ {\isasymand}\ {\isacharparenleft}S\ {\isacharequal}\ insert\ a\ S\ {\isasymor}\ S\ {\isacharequal}\ S{\isacharparenright}{\isachardoublequoteclose}\isanewline
\ \ \ \ \isacommand{using}\isamarkupfalse%
\ assms{\isacharparenleft}{\isadigit{1}}{\isacharparenright}\ {\isadigit{2}}\ \isacommand{by}\isamarkupfalse%
\ {\isacharparenleft}rule\ conjI{\isacharparenright}\isanewline
\ \ \isacommand{thus}\isamarkupfalse%
\ {\isacharquery}thesis\ \isanewline
\ \ \ \ \isacommand{using}\isamarkupfalse%
\ {\isadigit{1}}\ \isacommand{by}\isamarkupfalse%
\ {\isacharparenleft}rule\ subexI{\isacharparenright}\isanewline
\isacommand{qed}\isamarkupfalse%
%
\endisatagproof
{\isafoldproof}%
%
\isadelimproof
%
\endisadelimproof
%
\begin{isamarkuptext}%
Veamos otro resultado sobre subconjuntos finitos que emplearemos en las demostraciones
  posteriores.

\begin{lema}
  Sea \isa{S} un conjunto finito tal que \isa{S\ {\isasymsubseteq}\ {\isacharbraceleft}a{\isacharcomma}b{\isacharbraceright}\ {\isasymunion}\ B}. Entonces, existe un conjunto finito \isa{S{\isacharprime}\ {\isasymsubseteq}\ B} 
  tal que se verifica una de las condiciones siguientes: \isa{S\ {\isacharequal}\ {\isacharbraceleft}a{\isacharcomma}b{\isacharbraceright}\ {\isasymunion}\ S{\isacharprime}} o \isa{S\ {\isacharequal}\ {\isacharbraceleft}a{\isacharbraceright}\ {\isasymunion}\ S{\isacharprime}} o 
  \isa{S\ {\isacharequal}\ {\isacharbraceleft}b{\isacharbraceright}\ {\isasymunion}\ S{\isacharprime}} o \isa{S\ {\isacharequal}\ S{\isacharprime}}.
\end{lema}

\begin{demostracion}
  La prueba se realiza considerando cuatro casos:
  \begin{enumerate}
    \item \isa{a\ {\isasymin}\ S} y \isa{b\ {\isasymin}\ S}. 
    \item \isa{a\ {\isasymin}\ S} y \isa{b\ {\isasymnotin}\ S}. 
    \item \isa{a\ {\isasymnotin}\ S} y \isa{b\ {\isasymin}\ S}.
    \item \isa{a\ {\isasymnotin}\ S} y \isa{b\ {\isasymnotin}\ S}. 
  \end{enumerate}

  En primer lugar, supongamos que \isa{a} y \isa{b} están en \isa{S}. En este caso, basta tomar el conjunto
  \isa{S{\isacharprime}\ {\isacharequal}\ S\ {\isacharminus}\ {\isacharbraceleft}a{\isacharcomma}b{\isacharbraceright}}. Como \isa{S\ {\isasymsubseteq}\ {\isacharbraceleft}a{\isacharcomma}b{\isacharbraceright}\ {\isasymunion}\ B}, es claro que \isa{S{\isacharprime}\ {\isasymsubseteq}\ B}. Además, como \isa{S} es finito, se
  tiene que \isa{S{\isacharprime}} también es finito. Por último, verifica que \isa{S\ {\isacharequal}\ {\isacharbraceleft}a{\isacharcomma}b{\isacharbraceright}\ {\isasymunion}\ S{\isacharprime}}, lo que prueba el
  resultado para este caso.

  Supongamos ahora que \isa{a\ {\isasymin}\ S} y \isa{b\ {\isasymnotin}\ S}: considerando el conjunto \isa{S{\isacharprime}\ {\isacharequal}\ S\ {\isacharminus}\ {\isacharbraceleft}a{\isacharbraceright}} se tiene el
  resultado. Como \isa{S\ \ {\isasymsubseteq}\ {\isacharbraceleft}a{\isacharcomma}b{\isacharbraceright}\ {\isasymunion}\ B} y \isa{a\ {\isasymin}\ S} pero \isa{b\ {\isasymnotin}\ S}, es claro que \isa{S{\isacharprime}\ {\isasymsubseteq}\ B}. Como \isa{S} es
  finito, \isa{S{\isacharprime}} también lo es. Por último, se tiene que \isa{S\ {\isacharequal}\ {\isacharbraceleft}a{\isacharbraceright}\ {\isasymunion}\ S{\isacharprime}}, lo que verifica el resultado.

  Consideremos el caso \isa{a\ {\isasymnotin}\ S} y \isa{b\ {\isasymin}\ S}. Basta tomar el conjunto \isa{S{\isacharprime}\ {\isacharequal}\ S\ {\isacharminus}\ {\isacharbraceleft}b{\isacharbraceright}} y, de manera 
  completamente análoga al caso anterior, se demuestra el resultado para dicho caso.

  Finalmente, supongamos que \isa{a\ {\isasymnotin}\ S} y \isa{b\ {\isasymnotin}\ S}. Veamos que tomando el conjunto \isa{S{\isacharprime}\ {\isacharequal}\ S} se cumple
  el resultado. Como \isa{S\ {\isasymsubseteq}\ {\isacharbraceleft}a{\isacharcomma}b{\isacharbraceright}\ {\isasymunion}\ B} y ni \isa{a} ni \isa{b} están en \isa{S}, es claro que \isa{S{\isacharprime}\ {\isasymsubseteq}\ B}.
  Finalmente, como \isa{S} es finito, es claro que \isa{S{\isacharprime}} también lo es. Por lo tanto, queda probado el
  resultado.
\end{demostracion}

  Su formalización y prueba detallada en Isabelle/HOL son las siguientes.%
\end{isamarkuptext}\isamarkuptrue%
\isacommand{lemma}\isamarkupfalse%
\ finite{\isacharunderscore}subset{\isacharunderscore}insert{\isadigit{2}}{\isacharcolon}\isanewline
\ \ \isakeyword{assumes}\ {\isachardoublequoteopen}finite\ S{\isachardoublequoteclose}\isanewline
\ \ \ \ \ \ \ \ \ \ {\isachardoublequoteopen}S\ {\isasymsubseteq}\ {\isacharbraceleft}a{\isacharcomma}b{\isacharbraceright}\ {\isasymunion}\ B{\isachardoublequoteclose}\isanewline
\ \ \ \ \ \ \ \ \isakeyword{shows}\ {\isachardoublequoteopen}{\isasymexists}S{\isacharprime}\ {\isasymsubseteq}\ B{\isachardot}\ finite\ S{\isacharprime}\ {\isasymand}\ {\isacharparenleft}S\ {\isacharequal}\ {\isacharbraceleft}a{\isacharcomma}b{\isacharbraceright}\ {\isasymunion}\ S{\isacharprime}\ {\isasymor}\ S\ {\isacharequal}\ {\isacharbraceleft}a{\isacharbraceright}\ {\isasymunion}\ S{\isacharprime}\ {\isasymor}\ S\ {\isacharequal}\ {\isacharbraceleft}b{\isacharbraceright}\ {\isasymunion}\ S{\isacharprime}\ {\isasymor}\ S\ {\isacharequal}\ S{\isacharprime}{\isacharparenright}{\isachardoublequoteclose}\isanewline
%
\isadelimproof
%
\endisadelimproof
%
\isatagproof
\isacommand{proof}\isamarkupfalse%
\ {\isacharparenleft}cases\ {\isachardoublequoteopen}a\ {\isasymin}\ S\ {\isasymor}\ b\ {\isasymin}\ S{\isachardoublequoteclose}{\isacharparenright}\isanewline
\ \ \isacommand{assume}\isamarkupfalse%
\ {\isachardoublequoteopen}a\ {\isasymin}\ S\ {\isasymor}\ b\ {\isasymin}\ S{\isachardoublequoteclose}\isanewline
\ \ \isacommand{thus}\isamarkupfalse%
\ {\isacharquery}thesis\isanewline
\ \ \isacommand{proof}\isamarkupfalse%
\ {\isacharparenleft}rule\ disjE{\isacharparenright}\isanewline
\ \ \ \ \isacommand{assume}\isamarkupfalse%
\ {\isachardoublequoteopen}a\ {\isasymin}\ S{\isachardoublequoteclose}\isanewline
\ \ \ \ \isacommand{show}\isamarkupfalse%
\ {\isacharquery}thesis\isanewline
\ \ \ \ \isacommand{proof}\isamarkupfalse%
\ {\isacharparenleft}cases\ {\isachardoublequoteopen}b\ {\isasymin}\ S{\isachardoublequoteclose}{\isacharparenright}\isanewline
\ \ \ \ \ \ \isacommand{assume}\isamarkupfalse%
\ {\isachardoublequoteopen}b\ {\isasymin}\ S{\isachardoublequoteclose}\isanewline
\ \ \ \ \ \ \isacommand{have}\isamarkupfalse%
\ {\isadigit{1}}{\isacharcolon}{\isachardoublequoteopen}S\ {\isacharminus}\ {\isacharbraceleft}a{\isacharcomma}b{\isacharbraceright}\ {\isasymsubseteq}\ B{\isachardoublequoteclose}\isanewline
\ \ \ \ \ \ \ \ \isacommand{using}\isamarkupfalse%
\ assms{\isacharparenleft}{\isadigit{2}}{\isacharparenright}\ \isacommand{by}\isamarkupfalse%
\ blast\isanewline
\ \ \ \ \ \ \isacommand{have}\isamarkupfalse%
\ {\isachardoublequoteopen}{\isacharbraceleft}a{\isacharcomma}b{\isacharbraceright}\ {\isasymunion}\ S\ {\isacharequal}\ S{\isachardoublequoteclose}\isanewline
\ \ \ \ \ \ \ \ \isacommand{using}\isamarkupfalse%
\ {\isacartoucheopen}a\ {\isasymin}\ S{\isacartoucheclose}\ {\isacartoucheopen}b\ {\isasymin}\ S{\isacartoucheclose}\ \isacommand{by}\isamarkupfalse%
\ blast\isanewline
\ \ \ \ \ \ \isacommand{then}\isamarkupfalse%
\ \isacommand{have}\isamarkupfalse%
\ {\isachardoublequoteopen}S\ {\isacharequal}\ {\isacharbraceleft}a{\isacharcomma}b{\isacharbraceright}\ {\isasymunion}\ {\isacharparenleft}S\ {\isacharminus}\ {\isacharbraceleft}a{\isacharcomma}b{\isacharbraceright}{\isacharparenright}{\isachardoublequoteclose}\ \isanewline
\ \ \ \ \ \ \ \ \isacommand{by}\isamarkupfalse%
\ blast\isanewline
\ \ \ \ \ \ \isacommand{then}\isamarkupfalse%
\ \isacommand{have}\isamarkupfalse%
\ {\isadigit{2}}{\isacharcolon}{\isachardoublequoteopen}S\ {\isacharequal}\ {\isacharbraceleft}a{\isacharcomma}b{\isacharbraceright}\ {\isasymunion}\ {\isacharparenleft}S\ {\isacharminus}\ {\isacharbraceleft}a{\isacharcomma}b{\isacharbraceright}{\isacharparenright}\ {\isasymor}\ S\ {\isacharequal}\ {\isacharbraceleft}a{\isacharbraceright}\ {\isasymunion}\ {\isacharparenleft}S\ {\isacharminus}\ {\isacharbraceleft}a{\isacharcomma}b{\isacharbraceright}{\isacharparenright}\ {\isasymor}\ S\ {\isacharequal}\ {\isacharbraceleft}b{\isacharbraceright}\ {\isasymunion}\ {\isacharparenleft}S\ {\isacharminus}\ {\isacharbraceleft}a{\isacharcomma}b{\isacharbraceright}{\isacharparenright}\ {\isasymor}\ S\ {\isacharequal}\ {\isacharparenleft}S\ {\isacharminus}\ {\isacharbraceleft}a{\isacharcomma}b{\isacharbraceright}{\isacharparenright}{\isachardoublequoteclose}\isanewline
\ \ \ \ \ \ \ \ \isacommand{by}\isamarkupfalse%
\ {\isacharparenleft}iprover\ intro{\isacharcolon}\ disjI{\isadigit{1}}{\isacharparenright}\isanewline
\ \ \ \ \ \ \isacommand{have}\isamarkupfalse%
\ {\isachardoublequoteopen}finite\ {\isacharparenleft}S\ {\isacharminus}\ {\isacharbraceleft}a{\isacharcomma}b{\isacharbraceright}{\isacharparenright}{\isachardoublequoteclose}\isanewline
\ \ \ \ \ \ \ \ \isacommand{using}\isamarkupfalse%
\ assms{\isacharparenleft}{\isadigit{1}}{\isacharparenright}\ \isacommand{by}\isamarkupfalse%
\ {\isacharparenleft}rule\ finite{\isacharunderscore}Diff{\isacharparenright}\isanewline
\ \ \ \ \ \ \isacommand{then}\isamarkupfalse%
\ \isacommand{have}\isamarkupfalse%
\ {\isachardoublequoteopen}finite\ {\isacharparenleft}S\ {\isacharminus}\ {\isacharbraceleft}a{\isacharcomma}b{\isacharbraceright}{\isacharparenright}\ {\isasymand}\ {\isacharparenleft}S\ {\isacharequal}\ {\isacharbraceleft}a{\isacharcomma}b{\isacharbraceright}\ {\isasymunion}\ {\isacharparenleft}S\ {\isacharminus}\ {\isacharbraceleft}a{\isacharcomma}b{\isacharbraceright}{\isacharparenright}\ {\isasymor}\ S\ {\isacharequal}\ {\isacharbraceleft}a{\isacharbraceright}\ {\isasymunion}\ {\isacharparenleft}S\ {\isacharminus}\ {\isacharbraceleft}a{\isacharcomma}b{\isacharbraceright}{\isacharparenright}\ {\isasymor}\ S\ {\isacharequal}\ {\isacharbraceleft}b{\isacharbraceright}\ {\isasymunion}\ {\isacharparenleft}S\ {\isacharminus}\ {\isacharbraceleft}a{\isacharcomma}b{\isacharbraceright}{\isacharparenright}\ {\isasymor}\ S\ {\isacharequal}\ {\isacharparenleft}S\ {\isacharminus}\ {\isacharbraceleft}a{\isacharcomma}b{\isacharbraceright}{\isacharparenright}{\isacharparenright}{\isachardoublequoteclose}\isanewline
\ \ \ \ \ \ \ \ \isacommand{using}\isamarkupfalse%
\ {\isadigit{2}}\ \isacommand{by}\isamarkupfalse%
\ {\isacharparenleft}rule\ conjI{\isacharparenright}\isanewline
\ \ \ \ \ \ \isacommand{thus}\isamarkupfalse%
\ {\isacharquery}thesis\isanewline
\ \ \ \ \ \ \ \ \isacommand{using}\isamarkupfalse%
\ {\isadigit{1}}\ \isacommand{by}\isamarkupfalse%
\ {\isacharparenleft}rule\ subexI{\isacharparenright}\isanewline
\ \ \ \ \isacommand{next}\isamarkupfalse%
\isanewline
\ \ \ \ \ \ \isacommand{assume}\isamarkupfalse%
\ {\isachardoublequoteopen}b\ {\isasymnotin}\ S{\isachardoublequoteclose}\isanewline
\ \ \ \ \ \ \isacommand{then}\isamarkupfalse%
\ \isacommand{have}\isamarkupfalse%
\ {\isadigit{1}}{\isacharcolon}{\isachardoublequoteopen}S\ {\isacharminus}\ {\isacharbraceleft}a{\isacharbraceright}\ {\isasymsubseteq}\ B{\isachardoublequoteclose}\isanewline
\ \ \ \ \ \ \ \ \isacommand{using}\isamarkupfalse%
\ assms{\isacharparenleft}{\isadigit{2}}{\isacharparenright}\ \isacommand{by}\isamarkupfalse%
\ blast\isanewline
\ \ \ \ \ \ \isacommand{have}\isamarkupfalse%
\ {\isachardoublequoteopen}{\isacharbraceleft}a{\isacharbraceright}\ {\isasymunion}\ S\ {\isacharequal}\ S{\isachardoublequoteclose}\isanewline
\ \ \ \ \ \ \ \ \isacommand{using}\isamarkupfalse%
\ {\isacartoucheopen}a\ {\isasymin}\ S{\isacartoucheclose}\ \isacommand{by}\isamarkupfalse%
\ blast\isanewline
\ \ \ \ \ \ \isacommand{then}\isamarkupfalse%
\ \isacommand{have}\isamarkupfalse%
\ {\isachardoublequoteopen}S\ {\isacharequal}\ {\isacharbraceleft}a{\isacharbraceright}\ {\isasymunion}\ {\isacharparenleft}S\ {\isacharminus}\ {\isacharbraceleft}a{\isacharbraceright}{\isacharparenright}{\isachardoublequoteclose}\ \isanewline
\ \ \ \ \ \ \ \ \isacommand{by}\isamarkupfalse%
\ blast\isanewline
\ \ \ \ \ \ \isacommand{then}\isamarkupfalse%
\ \isacommand{have}\isamarkupfalse%
\ {\isadigit{2}}{\isacharcolon}{\isachardoublequoteopen}S\ {\isacharequal}\ {\isacharbraceleft}a{\isacharcomma}b{\isacharbraceright}\ {\isasymunion}\ {\isacharparenleft}S\ {\isacharminus}\ {\isacharbraceleft}a{\isacharbraceright}{\isacharparenright}\ {\isasymor}\ S\ {\isacharequal}\ {\isacharbraceleft}a{\isacharbraceright}\ {\isasymunion}\ {\isacharparenleft}S\ {\isacharminus}\ {\isacharbraceleft}a{\isacharbraceright}{\isacharparenright}\ {\isasymor}\ S\ {\isacharequal}\ {\isacharbraceleft}b{\isacharbraceright}\ {\isasymunion}\ {\isacharparenleft}S\ {\isacharminus}\ {\isacharbraceleft}a{\isacharbraceright}{\isacharparenright}\ {\isasymor}\ S\ {\isacharequal}\ {\isacharparenleft}S\ {\isacharminus}\ {\isacharbraceleft}a{\isacharbraceright}{\isacharparenright}{\isachardoublequoteclose}\isanewline
\ \ \ \ \ \ \ \ \isacommand{by}\isamarkupfalse%
\ {\isacharparenleft}iprover\ intro{\isacharcolon}\ disjI{\isadigit{1}}{\isacharparenright}\isanewline
\ \ \ \ \ \ \isacommand{have}\isamarkupfalse%
\ {\isachardoublequoteopen}finite\ {\isacharparenleft}S\ {\isacharminus}\ {\isacharbraceleft}a{\isacharbraceright}{\isacharparenright}{\isachardoublequoteclose}\isanewline
\ \ \ \ \ \ \ \ \isacommand{using}\isamarkupfalse%
\ assms{\isacharparenleft}{\isadigit{1}}{\isacharparenright}\ \isacommand{by}\isamarkupfalse%
\ {\isacharparenleft}rule\ finite{\isacharunderscore}Diff{\isacharparenright}\isanewline
\ \ \ \ \ \ \isacommand{then}\isamarkupfalse%
\ \isacommand{have}\isamarkupfalse%
\ {\isachardoublequoteopen}finite\ {\isacharparenleft}S\ {\isacharminus}\ {\isacharbraceleft}a{\isacharbraceright}{\isacharparenright}\ {\isasymand}\ {\isacharparenleft}S\ {\isacharequal}\ {\isacharbraceleft}a{\isacharcomma}b{\isacharbraceright}\ {\isasymunion}\ {\isacharparenleft}S\ {\isacharminus}\ {\isacharbraceleft}a{\isacharbraceright}{\isacharparenright}\ {\isasymor}\ S\ {\isacharequal}\ {\isacharbraceleft}a{\isacharbraceright}\ {\isasymunion}\ {\isacharparenleft}S\ {\isacharminus}\ {\isacharbraceleft}a{\isacharbraceright}{\isacharparenright}\ {\isasymor}\ S\ {\isacharequal}\ {\isacharbraceleft}b{\isacharbraceright}\ {\isasymunion}\ {\isacharparenleft}S\ {\isacharminus}\ {\isacharbraceleft}a{\isacharbraceright}{\isacharparenright}\ {\isasymor}\ S\ {\isacharequal}\ {\isacharparenleft}S\ {\isacharminus}\ {\isacharbraceleft}a{\isacharbraceright}{\isacharparenright}{\isacharparenright}{\isachardoublequoteclose}\isanewline
\ \ \ \ \ \ \ \ \isacommand{using}\isamarkupfalse%
\ {\isadigit{2}}\ \isacommand{by}\isamarkupfalse%
\ {\isacharparenleft}rule\ conjI{\isacharparenright}\isanewline
\ \ \ \ \ \ \isacommand{thus}\isamarkupfalse%
\ {\isacharquery}thesis\isanewline
\ \ \ \ \ \ \ \ \isacommand{using}\isamarkupfalse%
\ {\isadigit{1}}\ \isacommand{by}\isamarkupfalse%
\ {\isacharparenleft}rule\ subexI{\isacharparenright}\isanewline
\ \ \ \ \isacommand{qed}\isamarkupfalse%
\isanewline
\ \ \isacommand{next}\isamarkupfalse%
\isanewline
\ \ \ \ \isacommand{assume}\isamarkupfalse%
\ {\isachardoublequoteopen}b\ {\isasymin}\ S{\isachardoublequoteclose}\isanewline
\ \ \ \ \isacommand{show}\isamarkupfalse%
\ {\isacharquery}thesis\isanewline
\ \ \ \ \isacommand{proof}\isamarkupfalse%
\ {\isacharparenleft}cases\ {\isachardoublequoteopen}a\ {\isasymin}\ S{\isachardoublequoteclose}{\isacharparenright}\isanewline
\ \ \ \ \ \ \isacommand{assume}\isamarkupfalse%
\ {\isachardoublequoteopen}a\ {\isasymin}\ S{\isachardoublequoteclose}\isanewline
\ \ \ \ \ \ \isacommand{have}\isamarkupfalse%
\ {\isadigit{1}}{\isacharcolon}{\isachardoublequoteopen}S\ {\isacharminus}\ {\isacharbraceleft}a{\isacharcomma}b{\isacharbraceright}\ {\isasymsubseteq}\ B{\isachardoublequoteclose}\isanewline
\ \ \ \ \ \ \ \ \isacommand{using}\isamarkupfalse%
\ assms{\isacharparenleft}{\isadigit{2}}{\isacharparenright}\ \isacommand{by}\isamarkupfalse%
\ blast\isanewline
\ \ \ \ \ \ \isacommand{have}\isamarkupfalse%
\ {\isachardoublequoteopen}{\isacharbraceleft}a{\isacharcomma}b{\isacharbraceright}\ {\isasymunion}\ S\ {\isacharequal}\ S{\isachardoublequoteclose}\isanewline
\ \ \ \ \ \ \ \ \isacommand{using}\isamarkupfalse%
\ {\isacartoucheopen}a\ {\isasymin}\ S{\isacartoucheclose}\ {\isacartoucheopen}b\ {\isasymin}\ S{\isacartoucheclose}\ \isacommand{by}\isamarkupfalse%
\ blast\isanewline
\ \ \ \ \ \ \isacommand{then}\isamarkupfalse%
\ \isacommand{have}\isamarkupfalse%
\ {\isachardoublequoteopen}S\ {\isacharequal}\ {\isacharbraceleft}a{\isacharcomma}b{\isacharbraceright}\ {\isasymunion}\ {\isacharparenleft}S\ {\isacharminus}\ {\isacharbraceleft}a{\isacharcomma}b{\isacharbraceright}{\isacharparenright}{\isachardoublequoteclose}\ \isanewline
\ \ \ \ \ \ \ \ \isacommand{by}\isamarkupfalse%
\ blast\isanewline
\ \ \ \ \ \ \isacommand{then}\isamarkupfalse%
\ \isacommand{have}\isamarkupfalse%
\ {\isadigit{2}}{\isacharcolon}{\isachardoublequoteopen}S\ {\isacharequal}\ {\isacharbraceleft}a{\isacharcomma}b{\isacharbraceright}\ {\isasymunion}\ {\isacharparenleft}S\ {\isacharminus}\ {\isacharbraceleft}a{\isacharcomma}b{\isacharbraceright}{\isacharparenright}\ {\isasymor}\ S\ {\isacharequal}\ {\isacharbraceleft}a{\isacharbraceright}\ {\isasymunion}\ {\isacharparenleft}S\ {\isacharminus}\ {\isacharbraceleft}a{\isacharcomma}b{\isacharbraceright}{\isacharparenright}\ {\isasymor}\ S\ {\isacharequal}\ {\isacharbraceleft}b{\isacharbraceright}\ {\isasymunion}\ {\isacharparenleft}S\ {\isacharminus}\ {\isacharbraceleft}a{\isacharcomma}b{\isacharbraceright}{\isacharparenright}\ {\isasymor}\ S\ {\isacharequal}\ {\isacharparenleft}S\ {\isacharminus}\ {\isacharbraceleft}a{\isacharcomma}b{\isacharbraceright}{\isacharparenright}{\isachardoublequoteclose}\isanewline
\ \ \ \ \ \ \ \ \isacommand{by}\isamarkupfalse%
\ {\isacharparenleft}iprover\ intro{\isacharcolon}\ disjI{\isadigit{1}}{\isacharparenright}\isanewline
\ \ \ \ \ \ \isacommand{have}\isamarkupfalse%
\ {\isachardoublequoteopen}finite\ {\isacharparenleft}S\ {\isacharminus}\ {\isacharbraceleft}a{\isacharcomma}b{\isacharbraceright}{\isacharparenright}{\isachardoublequoteclose}\isanewline
\ \ \ \ \ \ \ \ \isacommand{using}\isamarkupfalse%
\ assms{\isacharparenleft}{\isadigit{1}}{\isacharparenright}\ \isacommand{by}\isamarkupfalse%
\ {\isacharparenleft}rule\ finite{\isacharunderscore}Diff{\isacharparenright}\isanewline
\ \ \ \ \ \ \isacommand{then}\isamarkupfalse%
\ \isacommand{have}\isamarkupfalse%
\ {\isachardoublequoteopen}finite\ {\isacharparenleft}S\ {\isacharminus}\ {\isacharbraceleft}a{\isacharcomma}b{\isacharbraceright}{\isacharparenright}\ {\isasymand}\ {\isacharparenleft}S\ {\isacharequal}\ {\isacharbraceleft}a{\isacharcomma}b{\isacharbraceright}\ {\isasymunion}\ {\isacharparenleft}S\ {\isacharminus}\ {\isacharbraceleft}a{\isacharcomma}b{\isacharbraceright}{\isacharparenright}\ {\isasymor}\ S\ {\isacharequal}\ {\isacharbraceleft}a{\isacharbraceright}\ {\isasymunion}\ {\isacharparenleft}S\ {\isacharminus}\ {\isacharbraceleft}a{\isacharcomma}b{\isacharbraceright}{\isacharparenright}\ {\isasymor}\ S\ {\isacharequal}\ {\isacharbraceleft}b{\isacharbraceright}\ {\isasymunion}\ {\isacharparenleft}S\ {\isacharminus}\ {\isacharbraceleft}a{\isacharcomma}b{\isacharbraceright}{\isacharparenright}\ {\isasymor}\ S\ {\isacharequal}\ {\isacharparenleft}S\ {\isacharminus}\ {\isacharbraceleft}a{\isacharcomma}b{\isacharbraceright}{\isacharparenright}{\isacharparenright}{\isachardoublequoteclose}\isanewline
\ \ \ \ \ \ \ \ \isacommand{using}\isamarkupfalse%
\ {\isadigit{2}}\ \isacommand{by}\isamarkupfalse%
\ {\isacharparenleft}rule\ conjI{\isacharparenright}\isanewline
\ \ \ \ \ \ \isacommand{thus}\isamarkupfalse%
\ {\isacharquery}thesis\isanewline
\ \ \ \ \ \ \ \ \isacommand{using}\isamarkupfalse%
\ {\isadigit{1}}\ \isacommand{by}\isamarkupfalse%
\ {\isacharparenleft}rule\ subexI{\isacharparenright}\isanewline
\ \ \ \ \isacommand{next}\isamarkupfalse%
\isanewline
\ \ \ \ \ \ \isacommand{assume}\isamarkupfalse%
\ {\isachardoublequoteopen}a\ {\isasymnotin}\ S{\isachardoublequoteclose}\isanewline
\ \ \ \ \ \ \isacommand{then}\isamarkupfalse%
\ \isacommand{have}\isamarkupfalse%
\ {\isadigit{1}}{\isacharcolon}{\isachardoublequoteopen}S\ {\isacharminus}\ {\isacharbraceleft}b{\isacharbraceright}\ {\isasymsubseteq}\ B{\isachardoublequoteclose}\isanewline
\ \ \ \ \ \ \ \ \isacommand{using}\isamarkupfalse%
\ assms{\isacharparenleft}{\isadigit{2}}{\isacharparenright}\ \isacommand{by}\isamarkupfalse%
\ blast\isanewline
\ \ \ \ \ \ \isacommand{have}\isamarkupfalse%
\ {\isachardoublequoteopen}{\isacharbraceleft}b{\isacharbraceright}\ {\isasymunion}\ S\ {\isacharequal}\ S{\isachardoublequoteclose}\isanewline
\ \ \ \ \ \ \ \ \isacommand{using}\isamarkupfalse%
\ {\isacartoucheopen}b\ {\isasymin}\ S{\isacartoucheclose}\ \isacommand{by}\isamarkupfalse%
\ blast\isanewline
\ \ \ \ \ \ \isacommand{then}\isamarkupfalse%
\ \isacommand{have}\isamarkupfalse%
\ {\isachardoublequoteopen}S\ {\isacharequal}\ {\isacharbraceleft}b{\isacharbraceright}\ {\isasymunion}\ {\isacharparenleft}S\ {\isacharminus}\ {\isacharbraceleft}b{\isacharbraceright}{\isacharparenright}{\isachardoublequoteclose}\ \isanewline
\ \ \ \ \ \ \ \ \isacommand{by}\isamarkupfalse%
\ blast\isanewline
\ \ \ \ \ \ \isacommand{then}\isamarkupfalse%
\ \isacommand{have}\isamarkupfalse%
\ {\isadigit{2}}{\isacharcolon}{\isachardoublequoteopen}S\ {\isacharequal}\ {\isacharbraceleft}a{\isacharcomma}b{\isacharbraceright}\ {\isasymunion}\ {\isacharparenleft}S\ {\isacharminus}\ {\isacharbraceleft}b{\isacharbraceright}{\isacharparenright}\ {\isasymor}\ S\ {\isacharequal}\ {\isacharbraceleft}a{\isacharbraceright}\ {\isasymunion}\ {\isacharparenleft}S\ {\isacharminus}\ {\isacharbraceleft}b{\isacharbraceright}{\isacharparenright}\ {\isasymor}\ S\ {\isacharequal}\ {\isacharbraceleft}b{\isacharbraceright}\ {\isasymunion}\ {\isacharparenleft}S\ {\isacharminus}\ {\isacharbraceleft}b{\isacharbraceright}{\isacharparenright}\ {\isasymor}\ S\ {\isacharequal}\ {\isacharparenleft}S\ {\isacharminus}\ {\isacharbraceleft}b{\isacharbraceright}{\isacharparenright}{\isachardoublequoteclose}\isanewline
\ \ \ \ \ \ \ \ \isacommand{by}\isamarkupfalse%
\ {\isacharparenleft}iprover\ intro{\isacharcolon}\ disjI{\isadigit{1}}{\isacharparenright}\isanewline
\ \ \ \ \ \ \isacommand{have}\isamarkupfalse%
\ {\isachardoublequoteopen}finite\ {\isacharparenleft}S\ {\isacharminus}\ {\isacharbraceleft}b{\isacharbraceright}{\isacharparenright}{\isachardoublequoteclose}\isanewline
\ \ \ \ \ \ \ \ \isacommand{using}\isamarkupfalse%
\ assms{\isacharparenleft}{\isadigit{1}}{\isacharparenright}\ \isacommand{by}\isamarkupfalse%
\ {\isacharparenleft}rule\ finite{\isacharunderscore}Diff{\isacharparenright}\isanewline
\ \ \ \ \ \ \isacommand{then}\isamarkupfalse%
\ \isacommand{have}\isamarkupfalse%
\ {\isachardoublequoteopen}finite\ {\isacharparenleft}S\ {\isacharminus}\ {\isacharbraceleft}b{\isacharbraceright}{\isacharparenright}\ {\isasymand}\ {\isacharparenleft}S\ {\isacharequal}\ {\isacharbraceleft}a{\isacharcomma}b{\isacharbraceright}\ {\isasymunion}\ {\isacharparenleft}S\ {\isacharminus}\ {\isacharbraceleft}b{\isacharbraceright}{\isacharparenright}\ {\isasymor}\ S\ {\isacharequal}\ {\isacharbraceleft}a{\isacharbraceright}\ {\isasymunion}\ {\isacharparenleft}S\ {\isacharminus}\ {\isacharbraceleft}b{\isacharbraceright}{\isacharparenright}\ {\isasymor}\ S\ {\isacharequal}\ {\isacharbraceleft}b{\isacharbraceright}\ {\isasymunion}\ {\isacharparenleft}S\ {\isacharminus}\ {\isacharbraceleft}b{\isacharbraceright}{\isacharparenright}\ {\isasymor}\ S\ {\isacharequal}\ {\isacharparenleft}S\ {\isacharminus}\ {\isacharbraceleft}b{\isacharbraceright}{\isacharparenright}{\isacharparenright}{\isachardoublequoteclose}\isanewline
\ \ \ \ \ \ \ \ \isacommand{using}\isamarkupfalse%
\ {\isadigit{2}}\ \isacommand{by}\isamarkupfalse%
\ {\isacharparenleft}rule\ conjI{\isacharparenright}\isanewline
\ \ \ \ \ \ \isacommand{thus}\isamarkupfalse%
\ {\isacharquery}thesis\isanewline
\ \ \ \ \ \ \ \ \isacommand{using}\isamarkupfalse%
\ {\isadigit{1}}\ \isacommand{by}\isamarkupfalse%
\ {\isacharparenleft}rule\ subexI{\isacharparenright}\isanewline
\ \ \ \ \isacommand{qed}\isamarkupfalse%
\isanewline
\ \ \isacommand{qed}\isamarkupfalse%
\isanewline
\isacommand{next}\isamarkupfalse%
\isanewline
\ \ \isacommand{assume}\isamarkupfalse%
\ {\isachardoublequoteopen}{\isasymnot}{\isacharparenleft}a\ {\isasymin}\ S\ {\isasymor}\ b\ {\isasymin}\ S{\isacharparenright}{\isachardoublequoteclose}\isanewline
\ \ \isacommand{then}\isamarkupfalse%
\ \isacommand{have}\isamarkupfalse%
\ {\isachardoublequoteopen}a\ {\isasymnotin}\ S\ {\isasymand}\ b\ {\isasymnotin}\ S{\isachardoublequoteclose}\isanewline
\ \ \ \ \isacommand{by}\isamarkupfalse%
\ {\isacharparenleft}simp\ only{\isacharcolon}\ de{\isacharunderscore}Morgan{\isacharunderscore}disj\ simp{\isacharunderscore}thms{\isacharparenleft}{\isadigit{8}}{\isacharparenright}{\isacharparenright}\isanewline
\ \ \isacommand{then}\isamarkupfalse%
\ \isacommand{have}\isamarkupfalse%
\ {\isadigit{1}}{\isacharcolon}{\isachardoublequoteopen}S\ {\isasymsubseteq}\ B{\isachardoublequoteclose}\isanewline
\ \ \ \ \isacommand{using}\isamarkupfalse%
\ assms{\isacharparenleft}{\isadigit{2}}{\isacharparenright}\ \isacommand{by}\isamarkupfalse%
\ blast\isanewline
\ \ \isacommand{have}\isamarkupfalse%
\ {\isadigit{2}}{\isacharcolon}{\isachardoublequoteopen}S\ {\isacharequal}\ {\isacharbraceleft}a{\isacharcomma}b{\isacharbraceright}\ {\isasymunion}\ S\ {\isasymor}\ S\ {\isacharequal}\ {\isacharbraceleft}a{\isacharbraceright}\ {\isasymunion}\ S\ {\isasymor}\ S\ {\isacharequal}\ {\isacharbraceleft}b{\isacharbraceright}\ {\isasymunion}\ S\ {\isasymor}\ S\ {\isacharequal}\ S{\isachardoublequoteclose}\isanewline
\ \ \ \ \isacommand{by}\isamarkupfalse%
\ {\isacharparenleft}iprover\ intro{\isacharcolon}\ disjI{\isadigit{1}}\ simp{\isacharunderscore}thms{\isacharparenright}\isanewline
\ \ \isacommand{have}\isamarkupfalse%
\ {\isachardoublequoteopen}finite\ S\ {\isasymand}\ {\isacharparenleft}S\ {\isacharequal}\ {\isacharbraceleft}a{\isacharcomma}b{\isacharbraceright}\ {\isasymunion}\ S\ {\isasymor}\ S\ {\isacharequal}\ {\isacharbraceleft}a{\isacharbraceright}\ {\isasymunion}\ S\ {\isasymor}\ S\ {\isacharequal}\ {\isacharbraceleft}b{\isacharbraceright}\ {\isasymunion}\ S\ {\isasymor}\ S\ {\isacharequal}\ S{\isacharparenright}{\isachardoublequoteclose}\isanewline
\ \ \ \ \isacommand{using}\isamarkupfalse%
\ assms{\isacharparenleft}{\isadigit{1}}{\isacharparenright}\ {\isadigit{2}}\ \isacommand{by}\isamarkupfalse%
\ {\isacharparenleft}rule\ conjI{\isacharparenright}\isanewline
\ \ \isacommand{thus}\isamarkupfalse%
\ {\isacharquery}thesis\isanewline
\ \ \ \ \isacommand{using}\isamarkupfalse%
\ {\isadigit{1}}\ \isacommand{by}\isamarkupfalse%
\ {\isacharparenleft}rule\ subexI{\isacharparenright}\isanewline
\isacommand{qed}\isamarkupfalse%
%
\endisatagproof
{\isafoldproof}%
%
\isadelimproof
%
\endisadelimproof
%
\begin{isamarkuptext}%
Una vez introducidos los resultados anteriores, procedamos con la prueba del \isa{Teorema\ de\ Compacidad}.

  \begin{demostracion}
    Consideremos la colección \isa{C} formada por los conjuntos de fórmulas finitamente satisfacibles.
    Recordemos que un conjunto de fórmulas es finitamente satisfacible si todo subconjunto finito 
    suyo es satisfacible. Vamos a probar que dicha colección verifica la propiedad de consistencia 
    proposicional y, por el \isa{Teorema\ de\ Existencia\ de\ Modelo}, quedará probado que todo conjunto de 
    \isa{C} es satisfacible, lo que demuestra el teorema.

    Para probar que \isa{C} verifica la propiedad de consistencia proposicional, por el lema \isa{{\isadigit{2}}{\isachardot}{\isadigit{0}}{\isachardot}{\isadigit{2}}} de 
    caracterización mediante notación uniforme, basta demostrar que se verifican las siguientes 
    condiciones para todo conjunto \isa{W\ {\isasymin}\ C}:
    \begin{itemize}
     \item \isa{{\isasymbottom}\ {\isasymnotin}\ W}.
     \item Dada \isa{p} una fórmula atómica cualquiera, no se tiene 
      simultáneamente que\\ \isa{p\ {\isasymin}\ W} y \isa{{\isasymnot}\ p\ {\isasymin}\ W}.
     \item Para toda fórmula de tipo \isa{{\isasymalpha}} con componentes \isa{{\isasymalpha}\isactrlsub {\isadigit{1}}} y \isa{{\isasymalpha}\isactrlsub {\isadigit{2}}} tal que \isa{{\isasymalpha}}
      pertenece a \isa{W}, se tiene que \isa{{\isacharbraceleft}{\isasymalpha}\isactrlsub {\isadigit{1}}{\isacharcomma}{\isasymalpha}\isactrlsub {\isadigit{2}}{\isacharbraceright}\ {\isasymunion}\ W} pertenece a \isa{C}.
     \item Para toda fórmula de tipo \isa{{\isasymbeta}} con componentes \isa{{\isasymbeta}\isactrlsub {\isadigit{1}}} y \isa{{\isasymbeta}\isactrlsub {\isadigit{2}}} tal que \isa{{\isasymbeta}}
      pertenece a \isa{W}, se tiene que o bien \isa{{\isacharbraceleft}{\isasymbeta}\isactrlsub {\isadigit{1}}{\isacharbraceright}\ {\isasymunion}\ W} pertenece a \isa{C} o 
      bien \isa{{\isacharbraceleft}{\isasymbeta}\isactrlsub {\isadigit{2}}{\isacharbraceright}\ {\isasymunion}\ W} pertenece a \isa{C}.
    \end{itemize}

    De este modo, consideremos un conjunto cualquiera \isa{W\ {\isasymin}\ C} y procedamos a probar cada una de las
    condiciones anteriores.

    La primera condición se demuestra por reducción al absurdo. En efecto, si suponemos que 
    \isa{{\isasymbottom}\ {\isasymin}\ W}, es claro que \isa{{\isacharbraceleft}{\isasymbottom}{\isacharbraceright}} es un subconjunto finito de \isa{W}. Como \isa{W} es un conjunto
    finitamente satisfacible por pertenecer a \isa{C}, se tiene por lo anterior que \isa{{\isacharbraceleft}{\isasymbottom}{\isacharbraceright}} es 
    satisfacible. De este modo, llegamos a una contradicción pues, por definición, no existe ninguna 
    interpretación que sea modelo de \isa{{\isasymbottom}}.

    A continuación probaremos que, si \isa{W\ {\isasymin}\ C}, entonces dada \isa{p} una fórmula atómica cualquiera, no 
    se tiene simultáneamente que \isa{p\ {\isasymin}\ W} y \isa{{\isasymnot}\ p\ {\isasymin}\ W}. Veamos dicho resultado por reducción al 
    absurdo, suponiendo que tanto \isa{p} como \isa{{\isasymnot}\ p} están en \isa{W}. En este caso, \isa{{\isacharbraceleft}p{\isacharcomma}{\isasymnot}\ p{\isacharbraceright}} sería un
    subconjunto finito de \isa{W} y, por ser \isa{W} finitamente satisfacible ya que\\ \isa{W\ {\isasymin}\ C}, obtendríamos 
    que \isa{{\isacharbraceleft}p{\isacharcomma}{\isasymnot}\ p{\isacharbraceright}} es satisfacible. Sin embargo esto no es cierto ya que, en ese caso, existiría
    una interpretación que sería modelo tanto de \isa{p} como de \isa{{\isasymnot}\ p}, llegando así a una 
    contradicción.

    Probemos ahora que dada una fórmula \isa{F} de tipo \isa{{\isasymalpha}} con componentes \isa{{\isasymalpha}\isactrlsub {\isadigit{1}}} y \isa{{\isasymalpha}\isactrlsub {\isadigit{2}}} tal que \isa{F\ {\isasymin}\ W},
    se tiene que \isa{{\isacharbraceleft}{\isasymalpha}\isactrlsub {\isadigit{1}}{\isacharcomma}{\isasymalpha}\isactrlsub {\isadigit{2}}{\isacharbraceright}\ {\isasymunion}\ W} pertenece a \isa{C}. Por definición de la colección, basta probar que 
    \isa{{\isacharbraceleft}{\isasymalpha}\isactrlsub {\isadigit{1}}{\isacharcomma}{\isasymalpha}\isactrlsub {\isadigit{2}}{\isacharbraceright}\ {\isasymunion}\ W} es finitamente satisfacible, es decir, que todo subconjunto finito suyo es
    satisfacible. Consideremos un subconjunto finito \isa{S} de \isa{{\isacharbraceleft}{\isasymalpha}\isactrlsub {\isadigit{1}}{\isacharcomma}{\isasymalpha}\isactrlsub {\isadigit{2}}{\isacharbraceright}\ {\isasymunion}\ W}. En estas condiciones,
    por el lema \isa{{\isadigit{4}}{\isachardot}{\isadigit{3}}{\isachardot}{\isadigit{3}}}, existe un subconjunto finito \isa{W\isactrlsub {\isadigit{0}}} de \isa{W} tal que\\ \isa{S\ {\isacharequal}\ {\isacharbraceleft}{\isasymalpha}\isactrlsub {\isadigit{1}}{\isacharcomma}{\isasymalpha}\isactrlsub {\isadigit{2}}{\isacharbraceright}\ {\isasymunion}\ W\isactrlsub {\isadigit{0}}},
    \isa{S\ {\isacharequal}\ {\isacharbraceleft}{\isasymalpha}\isactrlsub {\isadigit{1}}{\isacharbraceright}\ {\isasymunion}\ W\isactrlsub {\isadigit{0}}}, \isa{S\ {\isacharequal}\ {\isacharbraceleft}{\isasymalpha}\isactrlsub {\isadigit{2}}{\isacharbraceright}\ {\isasymunion}\ W\isactrlsub {\isadigit{0}}} o \isa{S\ {\isacharequal}\ W\isactrlsub {\isadigit{0}}}. Para probar que \isa{S} es satisfacible en cada uno de 
    estos posibles casos, basta demostrar que el conjunto\\ \isa{{\isacharbraceleft}{\isasymalpha}\isactrlsub {\isadigit{1}}{\isacharcomma}{\isasymalpha}\isactrlsub {\isadigit{2}}{\isacharcomma}F{\isacharbraceright}\ {\isasymunion}\ W\isactrlsub {\isadigit{0}}} es satisfacible. De este
    modo, puesto que todas las opciones posibles de \isa{S} están contenidas en dicho conjunto, se
    tiene la satisfacibilidad de cada una de ellas.

    Para probar que el conjunto \isa{{\isacharbraceleft}{\isasymalpha}\isactrlsub {\isadigit{1}}{\isacharcomma}{\isasymalpha}\isactrlsub {\isadigit{2}}{\isacharcomma}F{\isacharbraceright}\ {\isasymunion}\ W\isactrlsub {\isadigit{0}}} es satisfacible en estas condiciones, demostremos 
    que se verifica para cada caso de la fórmula \isa{F} de tipo \isa{{\isasymalpha}}:

      $\textbf{\isa{{\isasymone}{\isacharparenright}\ F\ {\isacharequal}\ G\ {\isasymand}\ H{\isacharcomma}\ para\ ciertas\ fórmulas\ G\ y\ H}:}$ Observemos que, como \isa{W\isactrlsub {\isadigit{0}}} es un subconjunto 
      finito de \isa{W} y \isa{F\ {\isasymin}\ W} por hipótesis, tenemos que \isa{{\isacharbraceleft}F{\isacharbraceright}\ {\isasymunion}\ W\isactrlsub {\isadigit{0}}} es un subconjunto finito de \isa{W}. 
      Como \isa{W} es finitamente satisfacible ya que pertenece a \isa{C}, se tiene que \isa{{\isacharbraceleft}F{\isacharbraceright}\ {\isasymunion}\ W\isactrlsub {\isadigit{0}}} es 
      satisfacible. Luego, por definición, existe una interpretación \isa{{\isasymA}} que es modelo de todas sus 
      fórmulas y, en particular, \isa{{\isasymA}} es modelo de \isa{F}. Como \isa{F\ {\isacharequal}\ G\ {\isasymand}\ H}, obtenemos por definición 
      del valor de una fórmula en una interpretación que \isa{{\isasymA}} es modelo de \isa{G} y de \isa{H}. En este caso,
      las componentes conjuntivas son \isa{{\isasymalpha}\isactrlsub {\isadigit{1}}\ {\isacharequal}\ G} y \isa{{\isasymalpha}\isactrlsub {\isadigit{2}}\ {\isacharequal}\ H}, luego \isa{{\isasymA}} es modelo de ambas componentes.
      Por lo tanto, \isa{{\isasymA}} es modelo de todas las fórmulas del conjunto \isa{{\isacharbraceleft}{\isasymalpha}\isactrlsub {\isadigit{1}}{\isacharcomma}{\isasymalpha}\isactrlsub {\isadigit{2}}{\isacharcomma}F{\isacharbraceright}\ {\isasymunion}\ W\isactrlsub {\isadigit{0}}}, lo que prueba 
      que se trata de un conjunto satisfacible.

      $\textbf{\isa{{\isasymtwo}{\isacharparenright}\ F\ {\isacharequal}\ {\isasymnot}{\isacharparenleft}G\ {\isasymor}\ H{\isacharparenright}{\isacharcomma}\ para\ ciertas\ fórmulas\ G\ y\ H}:}$ Análogamente al caso anterior, obtenemos 
      que el conjunto \isa{{\isacharbraceleft}F{\isacharbraceright}\ {\isasymunion}\ W\isactrlsub {\isadigit{0}}} es satisfacible. Luego, por definición, existe una interpretación 
      \isa{{\isasymA}} que es modelo de todas sus fórmulas y, en particular, de \isa{F}. Por definición del valor de 
      una fórmula en una interpretación, como \isa{F\ {\isacharequal}\ {\isasymnot}{\isacharparenleft}G\ {\isasymor}\ H{\isacharparenright}}, obtenemos que no es cierto que \isa{{\isasymA}} 
      sea modelo de \isa{G\ {\isasymor}\ H}. Aplicando de nuevo la definición del valor de una fórmula en una 
      interpretación, se obtiene que no es cierto que \isa{{\isasymA}} se modelo de \isa{G} o de \isa{H}. Por las leyes 
      de \isa{Morgan}, obtenemos equivalentemente que \isa{{\isasymA}} no es modelo de \isa{G} y \isa{{\isasymA}} no es modelo de \isa{H}. 
      Por lo tanto, por el valor de una fórmula en una interpretación, obtenemos que \isa{{\isasymA}} es 
      modelo de \isa{{\isasymnot}\ G} y \isa{{\isasymA}} es modelo de \isa{{\isasymnot}\ H}. Como las componentes conjuntivas en este caso son 
      \isa{{\isasymalpha}\isactrlsub {\isadigit{1}}\ {\isacharequal}\ {\isasymnot}\ G} y \isa{{\isasymalpha}\isactrlsub {\isadigit{2}}\ {\isacharequal}\ {\isasymnot}\ H}, es claro que \isa{{\isasymA}} es modelo de \isa{{\isasymalpha}\isactrlsub {\isadigit{1}}} y de \isa{{\isasymalpha}\isactrlsub {\isadigit{2}}}. Por lo tanto, la 
      interpretación \isa{{\isasymA}} es modelo de todas las fórmulas del conjunto \isa{{\isacharbraceleft}{\isasymalpha}\isactrlsub {\isadigit{1}}{\isacharcomma}{\isasymalpha}\isactrlsub {\isadigit{2}}{\isacharcomma}F{\isacharbraceright}\ {\isasymunion}\ W\isactrlsub {\isadigit{0}}}, lo que 
      prueba por definición que se trata de un conjunto satisfacible. 

      $\textbf{\isa{{\isasymthree}{\isacharparenright}\ F\ {\isacharequal}\ {\isasymnot}{\isacharparenleft}G\ {\isasymlongrightarrow}\ H{\isacharparenright}{\isacharcomma}\ para\ ciertas\ fórmulas\ G\ y\ H}:}$ Como hemos visto que \isa{{\isacharbraceleft}F{\isacharbraceright}\ {\isasymunion}\ W\isactrlsub {\isadigit{0}}} 
      es un conjunto satisfacible, existe una interpretación \isa{{\isasymA}} que es modelo de todas sus 
      fórmulas. En particular, \isa{{\isasymA}} es modelo de \isa{F} luego, por definición del valor de una fórmula 
      en una interpretación, es claro que \isa{{\isasymA}} no es modelo de \isa{G\ {\isasymlongrightarrow}\ H}. De nuevo por el valor de 
      una fórmula en una interpretación, obtenemos que no es cierto que si \isa{{\isasymA}} es modelo de \isa{G}, 
      entonces sea modelo de \isa{H}. Equivalentemente, \isa{{\isasymA}} es modelo de \isa{G} y no es modelo de \isa{H}. Por 
      lo tanto, por la definición del valor de una fórmula en una interpretación, se obtiene que 
      \isa{{\isasymA}} es modelo de \isa{G} y de \isa{{\isasymnot}\ H}. Como en este caso las componentes conjuntivas son \isa{{\isasymalpha}\isactrlsub {\isadigit{1}}\ {\isacharequal}\ G} y
      \isa{{\isasymalpha}\isactrlsub {\isadigit{2}}\ {\isacharequal}\ {\isasymnot}\ H}, es claro que \isa{{\isasymA}} es modelo de \isa{{\isasymalpha}\isactrlsub {\isadigit{1}}} y de \isa{{\isasymalpha}\isactrlsub {\isadigit{2}}}. Por lo tanto, \isa{{\isasymA}} es modelo de 
      todas las fórmulas del conjunto  \isa{{\isacharbraceleft}{\isasymalpha}\isactrlsub {\isadigit{1}}{\isacharcomma}{\isasymalpha}\isactrlsub {\isadigit{2}}{\isacharcomma}F{\isacharbraceright}\ {\isasymunion}\ W\isactrlsub {\isadigit{0}}}, probando así su satisfacibilidad.

      $\textbf{\isa{{\isasymfour}{\isacharparenright}\ F\ {\isacharequal}\ {\isasymnot}{\isacharparenleft}{\isasymnot}\ G{\isacharparenright}{\isacharcomma}\ para\ cierta\ fórmula\ G}:}$ Análogamente a los casos anteriores, se 
      prueba que existe una interpretación \isa{{\isasymA}} que es modelo de todas las fórmulas del conjunto\\ 
      \isa{{\isacharbraceleft}F{\isacharbraceright}\ {\isasymunion}\ W\isactrlsub {\isadigit{0}}} por ser este satisfacible. En particular, \isa{{\isasymA}} es modelo de \isa{F} luego, por 
      definición del valor de una fórmula en una interpretación, obtenemos que no es cierto que \isa{{\isasymA}} 
      no es modelo de \isa{G}. Es decir, \isa{{\isasymA}} es modelo de \isa{G} y, como en este caso ambas componentes 
      disyuntivas son \isa{G}, es claro que \isa{{\isasymA}} es modelo de \isa{{\isasymalpha}\isactrlsub {\isadigit{1}}} y de \isa{{\isasymalpha}\isactrlsub {\isadigit{2}}}. Por tanto, \isa{{\isasymA}} es modelo 
      de todas las fórmulas del conjunto \isa{{\isacharbraceleft}{\isasymalpha}\isactrlsub {\isadigit{1}}{\isacharcomma}{\isasymalpha}\isactrlsub {\isadigit{2}}{\isacharcomma}F{\isacharbraceright}\ {\isasymunion}\ W\isactrlsub {\isadigit{0}}}, lo que prueba su satisfacibilidad.

    Por lo tanto, \isa{{\isacharbraceleft}{\isasymalpha}\isactrlsub {\isadigit{1}}{\isacharcomma}{\isasymalpha}\isactrlsub {\isadigit{2}}{\isacharcomma}F{\isacharbraceright}\ {\isasymunion}\ W\isactrlsub {\isadigit{0}}} es un conjunto satisfacible para todos los casos de la fórmula
    \isa{F} de tipo \isa{{\isasymalpha}}. De este modo, como el subconjunto finito \isa{S} de \isa{{\isacharbraceleft}{\isasymalpha}\isactrlsub {\isadigit{1}}{\isacharcomma}{\isasymalpha}\isactrlsub {\isadigit{2}}{\isacharbraceright}\ {\isasymunion}\ W} es de la forma
    \isa{S\ {\isacharequal}\ {\isacharbraceleft}{\isasymalpha}\isactrlsub {\isadigit{1}}{\isacharcomma}{\isasymalpha}\isactrlsub {\isadigit{2}}{\isacharbraceright}\ {\isasymunion}\ W\isactrlsub {\isadigit{0}}}, \isa{S\ {\isacharequal}\ {\isacharbraceleft}{\isasymalpha}\isactrlsub {\isadigit{1}}{\isacharbraceright}\ {\isasymunion}\ W\isactrlsub {\isadigit{0}}}, \isa{S\ {\isacharequal}\ {\isacharbraceleft}{\isasymalpha}\isactrlsub {\isadigit{2}}{\isacharbraceright}\ {\isasymunion}\ W\isactrlsub {\isadigit{0}}} o \isa{S\ {\isacharequal}\ W\isactrlsub {\isadigit{0}}}, se prueba la satisfacibilidad
    de \isa{S} para cada uno de los casos por estar contenidos en el conjunto satisfacible
    \isa{{\isacharbraceleft}{\isasymalpha}\isactrlsub {\isadigit{1}}{\isacharcomma}{\isasymalpha}\isactrlsub {\isadigit{2}}{\isacharcomma}F{\isacharbraceright}\ {\isasymunion}\ W\isactrlsub {\isadigit{0}}}. Por lo tanto, \isa{{\isacharbraceleft}{\isasymalpha}\isactrlsub {\isadigit{1}}{\isacharcomma}{\isasymalpha}\isactrlsub {\isadigit{2}}{\isacharbraceright}\ {\isasymunion}\ W} es finitamente satisfacible y, por definición de 
    la colección \isa{C}, pertenece a ella como queríamos demostrar.

    Finalmente probemos que para toda fórmula \isa{F} de tipo \isa{{\isasymbeta}} con componentes \isa{{\isasymbeta}\isactrlsub {\isadigit{1}}} y \isa{{\isasymbeta}\isactrlsub {\isadigit{2}}} tal que 
    \isa{F\ {\isasymin}\ W}, se tiene que o bien \isa{{\isacharbraceleft}{\isasymbeta}\isactrlsub {\isadigit{1}}{\isacharbraceright}\ {\isasymunion}\ W\ {\isasymin}\ C} o bien \isa{{\isacharbraceleft}{\isasymbeta}\isactrlsub {\isadigit{2}}{\isacharbraceright}\ {\isasymunion}\ W\ {\isasymin}\ C}. La
    demostración se realizará por reducción al absurdo, luego supongamos en estas condiciones que\\
    \isa{{\isacharbraceleft}{\isasymbeta}\isactrlsub {\isadigit{1}}{\isacharbraceright}\ {\isasymunion}\ W\ {\isasymnotin}\ C} y \isa{{\isacharbraceleft}{\isasymbeta}\isactrlsub {\isadigit{2}}{\isacharbraceright}\ {\isasymunion}\ W\ {\isasymnotin}\ C}. 

    En primer lugar, veamos que si \isa{{\isacharbraceleft}{\isasymbeta}\isactrlsub i{\isacharbraceright}\ {\isasymunion}\ W\ {\isasymnotin}\ C}, entonces existe un subconjunto finito \isa{W\isactrlsub i} de 
    \isa{W} tal que el conjunto \isa{{\isacharbraceleft}{\isasymbeta}\isactrlsub i{\isacharbraceright}\ {\isasymunion}\ W\isactrlsub i} no es satisfacible. En efecto, si \isa{{\isacharbraceleft}{\isasymbeta}\isactrlsub i{\isacharbraceright}\ {\isasymunion}\ W\ {\isasymnotin}\ C}, por 
    definición de la colección \isa{C} tenemos que \isa{{\isacharbraceleft}{\isasymbeta}\isactrlsub i{\isacharbraceright}\ {\isasymunion}\ W} no es finitamente satisfacible. Por lo 
    tanto, existe un subconjunto finito \isa{W\isactrlsub i{\isacharprime}} de \isa{{\isacharbraceleft}{\isasymbeta}\isactrlsub i{\isacharbraceright}\ {\isasymunion}\ W} que no es satisfacible. Por el lema 
    \isa{{\isadigit{4}}{\isachardot}{\isadigit{3}}{\isachardot}{\isadigit{2}}}, obtenemos que existe un subconjunto finito \isa{W\isactrlsub i} de \isa{W} tal que o bien\\ \isa{W\isactrlsub i{\isacharprime}\ {\isacharequal}\ {\isacharbraceleft}{\isasymbeta}\isactrlsub i{\isacharbraceright}\ {\isasymunion}\ W\isactrlsub i} 
    o bien \isa{W\isactrlsub i{\isacharprime}\ {\isacharequal}\ W\isactrlsub i}. En efecto, si \isa{W\isactrlsub i{\isacharprime}\ {\isacharequal}\ {\isacharbraceleft}{\isasymbeta}\isactrlsub i{\isacharbraceright}\ {\isasymunion}\ W\isactrlsub i}, como \isa{W\isactrlsub i{\isacharprime}} no es satisfacible, se obtiene el
    resultado para \isa{W\isactrlsub i}. Por otro lado, supongamos que\\ \isa{W\isactrlsub i{\isacharprime}\ {\isacharequal}\ W\isactrlsub i}. Como \isa{W\isactrlsub i{\isacharprime}} no es satisfacible, 
    entonces \isa{{\isacharbraceleft}{\isasymbeta}\isactrlsub i{\isacharbraceright}\ {\isasymunion}\ W\isactrlsub i} tampoco es satisfacible ya que, en caso contrario, obtendríamos que
    \isa{W\isactrlsub i\ {\isacharequal}\ W\isactrlsub i{\isacharprime}} es satisfacible. Luego se verifica también el resultado para \isa{W\isactrlsub i}.

    De este modo, como \isa{{\isacharbraceleft}{\isasymbeta}\isactrlsub {\isadigit{1}}{\isacharbraceright}\ {\isasymunion}\ W\ {\isasymnotin}\ C} y \isa{{\isacharbraceleft}{\isasymbeta}\isactrlsub {\isadigit{2}}{\isacharbraceright}\ {\isasymunion}\ W\ {\isasymnotin}\ C}, obtenemos que existen subconjuntos finitos 
    \isa{W\isactrlsub {\isadigit{1}}} y \isa{W\isactrlsub {\isadigit{2}}} de \isa{W} tales que los conjunto \isa{{\isacharbraceleft}{\isasymbeta}\isactrlsub {\isadigit{1}}{\isacharbraceright}\ {\isasymunion}\ W\isactrlsub {\isadigit{1}}} y \isa{{\isacharbraceleft}{\isasymbeta}\isactrlsub {\isadigit{2}}{\isacharbraceright}\ {\isasymunion}\ W\isactrlsub {\isadigit{2}}} no son satisfacibles. 
    Consideremos el conjunto \isa{W\isactrlsub o\ {\isacharequal}\ W\isactrlsub {\isadigit{1}}\ {\isasymunion}\ W\isactrlsub {\isadigit{2}}}. Es claro que se tiene que\\ \isa{{\isacharbraceleft}{\isasymbeta}\isactrlsub {\isadigit{1}}{\isacharbraceright}\ {\isasymunion}\ W\isactrlsub {\isadigit{1}}\ {\isasymsubseteq}\ {\isacharbraceleft}{\isasymbeta}\isactrlsub {\isadigit{1}}{\isacharcomma}F{\isacharbraceright}\ {\isasymunion}\ W\isactrlsub o} y 
    que \isa{{\isacharbraceleft}{\isasymbeta}\isactrlsub {\isadigit{2}}{\isacharbraceright}\ {\isasymunion}\ W\isactrlsub {\isadigit{2}}\ {\isasymsubseteq}\ {\isacharbraceleft}{\isasymbeta}\isactrlsub {\isadigit{2}}{\isacharcomma}F{\isacharbraceright}\ {\isasymunion}\ W\isactrlsub {\isadigit{0}}}. Por lo tanto, los conjuntos \isa{{\isacharbraceleft}{\isasymbeta}\isactrlsub {\isadigit{1}}{\isacharcomma}F{\isacharbraceright}\ {\isasymunion}\ W\isactrlsub o} y \isa{{\isacharbraceleft}{\isasymbeta}\isactrlsub {\isadigit{2}}{\isacharcomma}F{\isacharbraceright}\ {\isasymunion}\ W\isactrlsub o} no son 
    satisfacibles ya que, en caso contrario, \isa{{\isacharbraceleft}{\isasymbeta}\isactrlsub {\isadigit{1}}{\isacharbraceright}\ {\isasymunion}\ W\isactrlsub {\isadigit{1}}} y \isa{{\isacharbraceleft}{\isasymbeta}\isactrlsub {\isadigit{2}}{\isacharbraceright}\ {\isasymunion}\ W\isactrlsub {\isadigit{2}}} serían satisfacibles. Para 
    llegar a la contradicción, basta probar que o bien \isa{{\isacharbraceleft}{\isasymbeta}\isactrlsub {\isadigit{1}}{\isacharcomma}F{\isacharbraceright}\ {\isasymunion}\ W\isactrlsub o} es satisfacible o bien 
    \isa{{\isacharbraceleft}{\isasymbeta}\isactrlsub {\isadigit{2}}{\isacharcomma}F{\isacharbraceright}\ {\isasymunion}\ W\isactrlsub o} es satisfacible. Para ello, veamos que se verifica el resultado para cada uno de 
    los casos posibles fórmula de tipo \isa{{\isasymbeta}} para \isa{F}.

      $\textbf{\isa{{\isasymone}{\isacharparenright}\ F\ {\isacharequal}\ G\ {\isasymor}\ H{\isacharcomma}\ para\ ciertas\ fórmulas\ G\ y\ H}:}$ Observemos que \isa{W\isactrlsub {\isadigit{0}}\ {\isacharequal}\ W\isactrlsub {\isadigit{1}}\ {\isasymunion}\ W\isactrlsub {\isadigit{2}}} es un 
      subconjunto finito de \isa{W} por ser \isa{W\isactrlsub {\isadigit{1}}} y \isa{W\isactrlsub {\isadigit{2}}} subconjuntos finitos de \isa{W}. Además, como 
      \isa{F\ {\isasymin}\ W} por hipótesis, tenemos que \isa{{\isacharbraceleft}F{\isacharbraceright}\ {\isasymunion}\ W\isactrlsub {\isadigit{0}}} es un subconjunto finito de \isa{W}. Como \isa{W} es 
      finitamente satisfacible ya que pertenece a \isa{C}, se tiene que \isa{{\isacharbraceleft}F{\isacharbraceright}\ {\isasymunion}\ W\isactrlsub {\isadigit{0}}} es satisfacible. 
      Luego, por definición, existe una interpretación \isa{{\isasymA}} que es modelo de todas sus fórmulas y, 
      en particular, \isa{{\isasymA}} es modelo de \isa{F}. Por definición del valor de una fórmula en una
      interpretación, obtenemos que o bien \isa{{\isasymA}} es modelo de \isa{G} o bien \isa{{\isasymA}} es modelo de \isa{H}.
      Como en este caso las componentes disyuntivas son \isa{{\isasymbeta}\isactrlsub {\isadigit{1}}\ {\isacharequal}\ G} y \isa{{\isasymbeta}\isactrlsub {\isadigit{2}}\ {\isacharequal}\ H}, se tiene que o bien \isa{{\isasymA}}
      es modelo de \isa{{\isasymbeta}\isactrlsub {\isadigit{1}}} o bien \isa{{\isasymA}} es modelo de \isa{{\isasymbeta}\isactrlsub {\isadigit{2}}}. Por lo tanto, es claro que o bien \isa{{\isasymA}} es
      modelo de todas las fórmulas del conjunto \isa{{\isacharbraceleft}{\isasymbeta}\isactrlsub {\isadigit{1}}{\isacharcomma}F{\isacharbraceright}\ {\isasymunion}\ W\isactrlsub {\isadigit{0}}} o bien es modelo de todas las fórmulas
      de \isa{{\isacharbraceleft}{\isasymbeta}\isactrlsub {\isadigit{2}}{\isacharcomma}F{\isacharbraceright}\ {\isasymunion}\ W\isactrlsub {\isadigit{0}}}. Luego, por definición de conjunto satisfacible tenemos que o bien\\ 
      \isa{{\isacharbraceleft}{\isasymbeta}\isactrlsub {\isadigit{1}}{\isacharcomma}F{\isacharbraceright}\ {\isasymunion}\ W\isactrlsub {\isadigit{0}}} es satisfacible o bien \isa{{\isacharbraceleft}{\isasymbeta}\isactrlsub {\isadigit{2}}{\isacharcomma}F{\isacharbraceright}\ {\isasymunion}\ W\isactrlsub {\isadigit{0}}} es satisfacible, como queríamos demostrar.

      $\textbf{\isa{{\isasymtwo}{\isacharparenright}\ F\ {\isacharequal}\ G\ {\isasymlongrightarrow}\ H{\isacharcomma}\ para\ ciertas\ fórmulas\ G\ y\ H}:}$ Análogamente se tiene que el 
      conjunto \isa{{\isacharbraceleft}F{\isacharbraceright}\ {\isasymunion}\ W\isactrlsub {\isadigit{0}}} es satisfacible, luego existe una interpretación \isa{{\isasymA}} que es modelo de 
      todas sus fórmulas. En particular, \isa{{\isasymA}} es modelo de \isa{F} y, por definición del valor de una 
      fórmula en una interpretación, se obtiene que si \isa{{\isasymA}} es modelo de \isa{G}, entonces es modelo de 
      \isa{H}. Equivalentemente, tenemos que \isa{{\isasymA}} no es modelo de \isa{G} o \isa{{\isasymA}} es modelo de \isa{H}. Por un 
      lado, si suponemos que \isa{{\isasymA}} no es modelo de \isa{G}, por definición obtenemos que \isa{{\isasymA}} es modelo de 
      \isa{{\isasymnot}\ G}. Como en este caso tenemos que \isa{{\isasymbeta}\isactrlsub {\isadigit{1}}\ {\isacharequal}\ {\isasymnot}\ G}, es claro que \isa{{\isasymA}} es modelo de \isa{{\isasymbeta}\isactrlsub {\isadigit{1}}}. Por 
      tanto, es modelo de todas las fórmulas de \isa{{\isacharbraceleft}{\isasymbeta}\isactrlsub {\isadigit{1}}{\isacharcomma}F{\isacharbraceright}\ {\isasymunion}\ W\isactrlsub {\isadigit{0}}}, luego es un conjunto satisfacible y 
      se verifica el resultado para este caso. Por otro lado, si suponemos que \isa{{\isasymA}} es modelo de \isa{H}, 
      como \isa{{\isasymbeta}\isactrlsub {\isadigit{2}}\ {\isacharequal}\ H}, obtenemos que \isa{{\isasymA}} es modelo de \isa{{\isasymbeta}\isactrlsub {\isadigit{2}}}. Luego, análogamente, \isa{{\isasymA}} es modelo de toda
      fórmula de \isa{{\isacharbraceleft}{\isasymbeta}\isactrlsub {\isadigit{2}}{\isacharcomma}F{\isacharbraceright}\ {\isasymunion}\ W\isactrlsub {\isadigit{0}}}, lo que prueba que se trata de un conjunto satisfacible por
      definición, probando el resultado. 

      $\textbf{\isa{{\isasymthree}{\isacharparenright}\ F\ {\isacharequal}\ {\isasymnot}{\isacharparenleft}G\ {\isasymand}\ H{\isacharparenright}{\isacharcomma}\ para\ ciertas\ fórmulas\ G\ y\ H}:}$ Como \isa{{\isacharbraceleft}F{\isacharbraceright}\ {\isasymunion}\ W\isactrlsub {\isadigit{0}}} es satisfacible,
      existe una interpretación \isa{{\isasymA}} que es modelo de todas sus fórmulas y, en particular, de \isa{F}.
      Luego, por definición del valor de una fórmula en una interpretación, obtenemos que \isa{{\isasymA}} no
      es modelo de \isa{G\ {\isasymand}\ H}. De nuevo por definición, esto implica que no es cierto que \isa{{\isasymA}} sea 
      modelo de \isa{G} y de \isa{H}. Es decir, o bien \isa{{\isasymA}} no es modelo de \isa{G} o bien \isa{{\isasymA}} no es modelo de
      \isa{H}. Si suponemos que no es modelo de \isa{G}, por definición se obtiene que \isa{{\isasymA}} es modelo de\\
      \isa{{\isasymnot}\ G}. Como en este caso la componente disyuntiva \isa{{\isasymbeta}\isactrlsub {\isadigit{1}}} es \isa{{\isasymnot}\ G}, se deduce que \isa{{\isasymA}} es modelo
      de \isa{{\isasymbeta}\isactrlsub {\isadigit{1}}}. Por tanto, \isa{{\isasymA}} es modelo de todas las fórmulas del conjunto \isa{{\isacharbraceleft}{\isasymbeta}\isactrlsub {\isadigit{1}}{\isacharcomma}F{\isacharbraceright}\ {\isasymunion}\ W\isactrlsub {\isadigit{0}}}, por lo que
      se demuestra que dicho conjunto es satisfacible, probando el resultado. Por otro lado, si
      suponemos que \isa{{\isasymA}} no es modelo de \isa{H}, se tiene que sí lo es de \isa{{\isasymnot}\ H}. Como \isa{{\isasymbeta}\isactrlsub {\isadigit{2}}} es \isa{{\isasymnot}\ H}
      en este caso, obtenemos que \isa{{\isasymA}} es modelo de \isa{{\isasymbeta}\isactrlsub {\isadigit{2}}}. Luego \isa{{\isasymA}} es modelo de todas las fórmulas
      de \isa{{\isacharbraceleft}{\isasymbeta}\isactrlsub {\isadigit{2}}{\isacharcomma}F{\isacharbraceright}\ {\isasymunion}\ W\isactrlsub {\isadigit{0}}}, demostrando así que es un conjunto satisfacible. Por tanto, se demuestra
      el resultado en ambos casos.

      $\textbf{\isa{{\isasymfour}{\isacharparenright}\ F\ {\isacharequal}\ {\isasymnot}{\isacharparenleft}{\isasymnot}\ G{\isacharparenright}{\isacharcomma}\ para\ cierta\ fórmula\ G}:}$ Puesto que \isa{{\isacharbraceleft}F{\isacharbraceright}\ {\isasymunion}\ W\isactrlsub {\isadigit{0}}} es satisfacible, 
      existe una interpretación \isa{{\isasymA}} modelo de todas sus fórmulas y, en particular, modelo de \isa{F}. 
      Luego, por definición del valor de una fórmula en una interpretación obtenemos que no es 
      cierto que \isa{{\isasymA}} no sea modelo de \isa{G}, es decir, \isa{{\isasymA}} es modelo de \isa{G}. Como las componentes 
      \isa{{\isasymbeta}\isactrlsub {\isadigit{1}}} y \isa{{\isasymbeta}\isactrlsub {\isadigit{2}}} son ambas \isa{G} en este caso, se obtiene que \isa{{\isasymA}} es modelo suyo. En particular, lo 
      es de \isa{{\isasymbeta}\isactrlsub {\isadigit{1}}}, de modo que \isa{{\isasymA}} es modelo de todas las fórmulas de \isa{{\isacharbraceleft}{\isasymbeta}\isactrlsub {\isadigit{1}}{\isacharcomma}F{\isacharbraceright}\ {\isasymunion}\ W\isactrlsub {\isadigit{0}}}, probando así que 
      es satisfacible. Por lo tanto, se verifica el resultado.
    
    En conclusión, hemos probado que o bien \isa{{\isacharbraceleft}{\isasymbeta}\isactrlsub {\isadigit{1}}{\isacharcomma}F{\isacharbraceright}\ {\isasymunion}\ W\isactrlsub o} es satisfacible o bien\\ \isa{{\isacharbraceleft}{\isasymbeta}\isactrlsub {\isadigit{2}}{\isacharcomma}F{\isacharbraceright}\ {\isasymunion}\ W\isactrlsub o} es 
    satisfacible. Por lo tanto, se tiene que no es cierto que ninguno de los dos conjuntos sea
    insatisfacible. Esto contradice lo demostrado anteriormente, llegando así a una contradicción
    que prueba por reducción al absurdo la última condición del lema \isa{{\isadigit{2}}{\isachardot}{\isadigit{0}}{\isachardot}{\isadigit{2}}}. De este modo, queda
    probado que la colección formada por los conjuntos de fórmulas finitamente satisfacibles 
    verifica la propiedad de consistencia proposicional y, por el \isa{Teorema\ de\ Existencia\ de\ Modelo}, 
    todo conjunto perteneciente a ella es satisfacible, lo que demuestra el teorema.
  \end{demostracion}

  Procedamos con la demostración detallada del \isa{Teorema\ de\ Compacidad} en Isabelle/HOL. Para ello, 
  definamos la colección de conjuntos finitamente satisfacibles en Isabelle/HOL. En adelante
  notaremos por \isa{C} a dicha colección.%
\end{isamarkuptext}\isamarkuptrue%
\isacommand{definition}\isamarkupfalse%
\ colecComp\ {\isacharcolon}{\isacharcolon}\ {\isachardoublequoteopen}{\isacharparenleft}{\isacharprime}a\ formula\ set{\isacharparenright}\ set{\isachardoublequoteclose}\isanewline
\ \ \isakeyword{where}\ colecComp{\isacharcolon}\ {\isachardoublequoteopen}colecComp\ {\isacharequal}\ {\isacharbraceleft}W{\isachardot}\ fin{\isacharunderscore}sat\ W{\isacharbraceright}{\isachardoublequoteclose}%
\begin{isamarkuptext}%
Para facilitar la demostración introduciremos el siguiente lema auxiliar que prueba que
  todo subconjunto finito de un conjunto perteneciente a la colección anterior es satisfacible.%
\end{isamarkuptext}\isamarkuptrue%
\isacommand{lemma}\isamarkupfalse%
\ colecComp{\isacharunderscore}subset{\isacharunderscore}finite{\isacharcolon}\ \isanewline
\ \ \isakeyword{assumes}\ {\isachardoublequoteopen}W\ {\isasymin}\ colecComp{\isachardoublequoteclose}\isanewline
\ \ \ \ \ \ \ \ \ \ {\isachardoublequoteopen}Wo\ {\isasymsubseteq}\ W{\isachardoublequoteclose}\isanewline
\ \ \ \ \ \ \ \ \ \ {\isachardoublequoteopen}finite\ Wo{\isachardoublequoteclose}\isanewline
\ \ \isakeyword{shows}\ {\isachardoublequoteopen}sat\ Wo{\isachardoublequoteclose}\ \isanewline
%
\isadelimproof
%
\endisadelimproof
%
\isatagproof
\isacommand{proof}\isamarkupfalse%
\ {\isacharminus}\isanewline
\ \ \isacommand{have}\isamarkupfalse%
\ {\isachardoublequoteopen}{\isasymforall}Wo\ {\isasymsubseteq}\ W{\isachardot}\ finite\ Wo\ {\isasymlongrightarrow}\ sat\ Wo{\isachardoublequoteclose}\isanewline
\ \ \ \ \isacommand{using}\isamarkupfalse%
\ assms{\isacharparenleft}{\isadigit{1}}{\isacharparenright}\ \isacommand{unfolding}\isamarkupfalse%
\ colecComp\ fin{\isacharunderscore}sat{\isacharunderscore}def\ \isacommand{by}\isamarkupfalse%
\ {\isacharparenleft}rule\ CollectD{\isacharparenright}\isanewline
\ \ \isacommand{then}\isamarkupfalse%
\ \isacommand{have}\isamarkupfalse%
\ {\isachardoublequoteopen}finite\ Wo\ {\isasymlongrightarrow}\ sat\ Wo{\isachardoublequoteclose}\isanewline
\ \ \ \ \isacommand{using}\isamarkupfalse%
\ {\isacartoucheopen}Wo\ {\isasymsubseteq}\ W{\isacartoucheclose}\ \isacommand{by}\isamarkupfalse%
\ {\isacharparenleft}rule\ sspec{\isacharparenright}\isanewline
\ \ \isacommand{thus}\isamarkupfalse%
\ {\isachardoublequoteopen}sat\ Wo{\isachardoublequoteclose}\isanewline
\ \ \ \ \isacommand{using}\isamarkupfalse%
\ {\isacartoucheopen}finite\ Wo{\isacartoucheclose}\ \isacommand{by}\isamarkupfalse%
\ {\isacharparenleft}rule\ mp{\isacharparenright}\isanewline
\isacommand{qed}\isamarkupfalse%
%
\endisatagproof
{\isafoldproof}%
%
\isadelimproof
%
\endisadelimproof
%
\begin{isamarkuptext}%
Para facilitar la comprensión de la demostración, mostraremos a continuación un grafo que 
  estructura las relaciones de necesidad de los lemas auxiliares empleados.

\begin{tikzpicture}
  [
    grow                    = down,
    level distance          = 1.3cm,
    level 1/.style          = {sibling distance=1cm},
    level 2/.style          = {sibling distance=4.3cm},
    level 3/.style          = {sibling distance=3cm},
    level 4/.style          = {sibling distance=4cm},
    level 5/.style          = {sibling distance=5cm},
    level 6/.style          = {sibling distance=5cm},
    level 7/.style          = {sibling distance=5cm};
    edge from parent/.style = {draw},
    every node/.style       = {font=\tiny},
    sloped
  ]
\raggedright
  \node [root] {\isa{prop{\isacharunderscore}Compactness}\\ \isa{{\isacharparenleft}Teorema\ de\ Compacidad\ {\isacharparenleft}{\isadigit{4}}{\isachardot}{\isadigit{3}}{\isachardot}{\isadigit{1}}{\isacharparenright}{\isacharparenright}}}
    child { node [env] {\isa{pcp{\isacharunderscore}colecComp}\\ \isa{{\isacharparenleft}C\ tiene\ la\ propiedad\ de\ consistencia\ proposicional{\isacharparenright}}}
          child { node [env] {\isa{pcp{\isacharunderscore}colecComp{\isacharunderscore}bot}\\ \isa{{\isacharparenleft}{\isasymbottom}\ {\isasymnotin}\ W{\isacharparenright}}}
              child { node [env] {\isa{not{\isacharunderscore}sat{\isacharunderscore}bot}\\ \isa{{\isacharparenleft}{\isacharbraceleft}{\isasymbottom}{\isacharbraceright}\ es\ insatisfacible{\isacharparenright}}}}}
          child { node [env] {\isa{pcp{\isacharunderscore}colecComp{\isacharunderscore}atoms}\\ \isa{{\isacharparenleft}Cond{\isachardot}\ fórmulas\ atómicas{\isacharparenright}}}
              child { node [env] {\isa{not{\isacharunderscore}sat{\isacharunderscore}atoms}\\ \isa{{\isacharparenleft}{\isacharbraceleft}p{\isacharcomma}{\isasymnot}\ p{\isacharbraceright}\ es\ insatisfacible{\isacharparenright}}}}}
      		child { node [env] {\isa{pcp{\isacharunderscore}colecComp{\isacharunderscore}CON}\\ \isa{{\isacharparenleft}Cond{\isachardot}\ fórmulas\ de\ tipo\ {\isasymalpha}{\isacharparenright}}}
        			child { node [env] {\isa{pcp{\isacharunderscore}colecComp{\isacharunderscore}CON{\isacharunderscore}sat}\\ \isa{{\isacharparenleft}Resultado\ {\isasymone}{\isacharparenright}}}
                      child { node [env] {\isa{pcp{\isacharunderscore}colecComp{\isacharunderscore}CON{\isacharunderscore}sat{\isadigit{1}}}\\\isa{pcp{\isacharunderscore}colecComp{\isacharunderscore}CON{\isacharunderscore}sat{\isadigit{2}}}\\\isa{pcp{\isacharunderscore}colecComp{\isacharunderscore}CON{\isacharunderscore}sat{\isadigit{3}}}\\\isa{pcp{\isacharunderscore}colecComp{\isacharunderscore}CON{\isacharunderscore}sat{\isadigit{4}}}}}}}
        			child { node [env] {\isa{pcp{\isacharunderscore}colecComp{\isacharunderscore}DIS}\\ \isa{{\isacharparenleft}Cond{\isachardot}\ fórmulas\ de\ tipo\ {\isasymbeta}{\isacharparenright}}}
                      child { node [env] {\isa{not{\isacharunderscore}colecComp}\\ \isa{{\isacharparenleft}Resultado\ {\isasymtwo}{\isacharparenright}}}
                            child { node [env] {\isa{sat{\isacharunderscore}subset{\isacharunderscore}ccontr}\\ \isa{{\isacharparenleft}Los\ conjuntos\ que}\\ \isa{contienen\ algún}\\ \isa{subconjunto\ insatisfacible}\\ \isa{son\ insatisfacibles{\isacharparenright}}}}}
                                  child { node [env] {\isa{pcp{\isacharunderscore}colecComp{\isacharunderscore}DIS{\isacharunderscore}sat}\\ \isa{{\isacharparenleft}Resultado\ {\isasymthree}{\isacharparenright}}}
                                  child { node [env] {\isa{pcp{\isacharunderscore}colecComp{\isacharunderscore}DIS{\isacharunderscore}sat{\isadigit{1}}}\\\isa{pcp{\isacharunderscore}colecComp{\isacharunderscore}DIS{\isacharunderscore}sat{\isadigit{2}}}\\\isa{pcp{\isacharunderscore}colecComp{\isacharunderscore}DIS{\isacharunderscore}sat{\isadigit{3}}}\\\isa{pcp{\isacharunderscore}colecComp{\isacharunderscore}DIS{\isacharunderscore}sat{\isadigit{4}}}}}}}};
\end{tikzpicture}

  De este modo, el \isa{Teorema\ de\ Compacidad} se demuestra aplicando el \isa{Teorema\ de}\\ \isa{Existencia\ de\ Modelo} a la colección \isa{C}. Por tanto, basta probar que la colección \isa{C} verifica la propiedad de 
  consistencia proposicional (formalizado como \isa{pcp{\isacharunderscore}colecComp}), de modo que todo \isa{W\ {\isasymin}\ C} será
  satisfacible. Para ello, por el lema \isa{{\isadigit{2}}{\isachardot}{\isadigit{0}}{\isachardot}{\isadigit{2}}}, es suficiente probar las siguientes condiciones dado 
  un conjunto \isa{W\ {\isasymin}\ C} cualquiera:

    \begin{enumerate}
     \item \isa{{\isasymbottom}\ {\isasymnotin}\ W}. (\isa{{\isasymLongrightarrow}} formalizado como \isa{pcp{\isacharunderscore}colecComp{\isacharunderscore}sat})
     \item Dada \isa{p} una fórmula atómica cualquiera, no se tiene 
      simultáneamente que\\ \isa{p\ {\isasymin}\ W} y \isa{{\isasymnot}\ p\ {\isasymin}\ W}. (\isa{{\isasymLongrightarrow}} formalizado como \isa{pcp{\isacharunderscore}colecComp{\isacharunderscore}atoms})
     \item Para toda fórmula de tipo \isa{{\isasymalpha}} con componentes \isa{{\isasymalpha}\isactrlsub {\isadigit{1}}} y \isa{{\isasymalpha}\isactrlsub {\isadigit{2}}} tal que \isa{{\isasymalpha}}
      pertenece a \isa{W}, se tiene que \isa{{\isacharbraceleft}{\isasymalpha}\isactrlsub {\isadigit{1}}{\isacharcomma}{\isasymalpha}\isactrlsub {\isadigit{2}}{\isacharbraceright}\ {\isasymunion}\ W\ {\isasymin}\ C}. (\isa{{\isasymLongrightarrow}} formalizado como 
      \isa{pcp{\isacharunderscore}colecComp{\isacharunderscore}CON})
     \item Para toda fórmula de tipo \isa{{\isasymbeta}} con componentes \isa{{\isasymbeta}\isactrlsub {\isadigit{1}}} y \isa{{\isasymbeta}\isactrlsub {\isadigit{2}}} tal que \isa{{\isasymbeta}}
      pertenece a \isa{W}, se tiene que o bien \isa{{\isacharbraceleft}{\isasymbeta}\isactrlsub {\isadigit{1}}{\isacharbraceright}\ {\isasymunion}\ W\ {\isasymin}\ C} o 
      bien \isa{{\isacharbraceleft}{\isasymbeta}\isactrlsub {\isadigit{2}}{\isacharbraceright}\ {\isasymunion}\ W\ {\isasymin}\ C}.\\ (\isa{{\isasymLongrightarrow}} formalizado como \isa{pcp{\isacharunderscore}colecComp{\isacharunderscore}DIS})
    \end{enumerate}
  A su vez, cada uno de los lemas auxiliares que prueban las condiciones anteriores precisa de los
  siguientes lemas:

  \begin{itemize}
    \item \isa{pcp{\isacharunderscore}colecComp{\isacharunderscore}sat}: Se prueba por reducción al absurdo mediante el lema \isa{not{\isacharunderscore}sat{\isacharunderscore}bot} que
    demuestra la insatisfacibilidad del conjunto \isa{{\isacharbraceleft}{\isasymbottom}{\isacharbraceright}}.
    \item \isa{pcp{\isacharunderscore}colecComp{\isacharunderscore}atoms}: Su demostración es por reducción al absurdo empleando el lema
    \isa{not{\isacharunderscore}sat{\isacharunderscore}atoms} que prueba la insatisfacibilidad del conjunto \isa{{\isacharbraceleft}p{\isacharcomma}{\isasymnot}\ p{\isacharbraceright}} para cualquier fórmula
    atómica \isa{p}.
    \item \isa{pcp{\isacharunderscore}colecComp{\isacharunderscore}CON}: Para su prueba, se precisa del \isa{resultado\ {\isasymone}}, formalizado como 
    \isa{pcp{\isacharunderscore}colecComp{\isacharunderscore}CON{\isacharunderscore}sat}. Este demuestra que dados \isa{W\ {\isasymin}\ C}, \isa{F\ {\isasymin}\ W} una fórmula de tipo 
    \isa{{\isasymalpha}} con componentes \isa{{\isasymalpha}\isactrlsub {\isadigit{1}}} y \isa{{\isasymalpha}\isactrlsub {\isadigit{2}}} y \isa{W\isactrlsub {\isadigit{0}}} un subconjunto finito de \isa{W}, se verifica que 
    \isa{{\isacharbraceleft}{\isasymalpha}\isactrlsub {\isadigit{1}}{\isacharcomma}{\isasymalpha}\isactrlsub {\isadigit{2}}{\isacharcomma}F{\isacharbraceright}\ {\isasymunion}\ W\isactrlsub {\isadigit{0}}} es satisfacible. Para probar dicho resultado se emplean a su vez los lemas
    auxiliares \isa{pcp{\isacharunderscore}colecComp{\isacharunderscore}CON{\isacharunderscore}sat{\isadigit{1}}}, \isa{pcp{\isacharunderscore}colecComp{\isacharunderscore}CON{\isacharunderscore}sat{\isadigit{2}}}, \isa{pcp{\isacharunderscore}colecComp{\isacharunderscore}CON{\isacharunderscore}sat{\isadigit{3}}} y 
    \isa{pcp{\isacharunderscore}colecComp{\isacharunderscore}CON{\isacharunderscore}sat{\isadigit{4}}} que demuestran el enunciado para cada tipo de fórmula \isa{{\isasymalpha}}.
    \item \isa{pcp{\isacharunderscore}colecComp{\isacharunderscore}DIS}: La prueba se realizará por reducción al absurdo. Para ello
    precisa de dos resultados.
    \begin{itemize}
      \item \isa{Resultado\ {\isasymtwo}}: Dados \isa{W\ {\isasymin}\ C} y \isa{{\isasymbeta}\isactrlsub i} una fórmula cualquiera tal que\\ \isa{{\isacharbraceleft}{\isasymbeta}\isactrlsub i{\isacharbraceright}\ {\isasymunion}\ W\ {\isasymnotin}\ C}, 
      entonces existe un subconjunto finito \isa{W\isactrlsub i} de \isa{W} tal que el conjunto \isa{{\isacharbraceleft}{\isasymbeta}\isactrlsub i{\isacharbraceright}\ {\isasymunion}\ W\isactrlsub i} no es 
      satisfacible. En Isabelle ha sido formalizado como \isa{not{\isacharunderscore}colecComp}. A su vez, ha precisado
      para su prueba del lema auxiliar \isa{sat{\isacharunderscore}subset{\isacharunderscore}ccontr} que demuestra que todo conjunto de 
      fórmulas que tenga un subconjunto insatisfacible es también insatisfacible.
      \item \isa{Resultado\ {\isasymthree}}: Dados \isa{W\ {\isasymin}\ C}, \isa{F} una fórmula de tipo \isa{{\isasymbeta}} con componentes \isa{{\isasymbeta}\isactrlsub {\isadigit{1}}} y \isa{{\isasymbeta}\isactrlsub {\isadigit{2}}} 
      tal que \isa{F\ {\isasymin}\ W} y \isa{W\isactrlsub {\isadigit{0}}} un subconjunto finito de \isa{W}, entonces se tiene que o bien 
      \isa{{\isacharbraceleft}{\isasymbeta}\isactrlsub {\isadigit{1}}{\isacharcomma}F{\isacharbraceright}\ {\isasymunion}\ W\isactrlsub {\isadigit{0}}} es satisfacible o bien \isa{{\isacharbraceleft}{\isasymbeta}\isactrlsub {\isadigit{2}}{\isacharcomma}F{\isacharbraceright}\ {\isasymunion}\ W\isactrlsub {\isadigit{0}}} es satisfacible. En Isabelle se ha
      formalizado como \isa{pcp{\isacharunderscore}colecComp{\isacharunderscore}DIS{\isacharunderscore}sat}. Para su prueba, ha precisado de cuatro lemas
      auxiliares que prueban el resultado para cada caso de fórmula de tipo \isa{{\isasymbeta}}: 
      \isa{pcp{\isacharunderscore}colecComp{\isacharunderscore}DIS{\isacharunderscore}sat{\isadigit{1}}}, \isa{pcp{\isacharunderscore}colecComp{\isacharunderscore}DIS{\isacharunderscore}sat{\isadigit{2}}}, \isa{pcp{\isacharunderscore}colecComp{\isacharunderscore}DIS{\isacharunderscore}sat{\isadigit{3}}},
      \isa{pcp{\isacharunderscore}colecComp{\isacharunderscore}DIS{\isacharunderscore}sat{\isadigit{4}}}.
    \end{itemize}
  \end{itemize}

  Comencemos con las demostraciones de los lemas auxiliares empleados en la demostración del 
  teorema. Para probar que \isa{C} verifica la propiedad de consistencia proposicional, dado un conjunto 
  \isa{W\ {\isasymin}\ C} probaremos por separado que se verifican cada una de las condiciones del lema \isa{{\isadigit{2}}{\isachardot}{\isadigit{0}}{\isachardot}{\isadigit{2}}}.
  
  En primer lugar, veamos que \isa{{\isasymbottom}\ {\isasymnotin}\ W} si \isa{W\ {\isasymin}\ C}. Para ello, precisaremos del siguiente lema 
  auxiliar que prueba que el conjunto \isa{{\isacharbraceleft}{\isasymbottom}{\isacharbraceright}} no es satisfacible.%
\end{isamarkuptext}\isamarkuptrue%
\isacommand{lemma}\isamarkupfalse%
\ not{\isacharunderscore}sat{\isacharunderscore}bot{\isacharcolon}\ {\isachardoublequoteopen}{\isasymnot}\ sat\ {\isacharbraceleft}{\isasymbottom}{\isacharbraceright}{\isachardoublequoteclose}\isanewline
%
\isadelimproof
%
\endisadelimproof
%
\isatagproof
\isacommand{proof}\isamarkupfalse%
\ {\isacharparenleft}rule\ ccontr{\isacharparenright}\isanewline
\ \ \isacommand{assume}\isamarkupfalse%
\ {\isachardoublequoteopen}{\isasymnot}{\isacharparenleft}{\isasymnot}sat{\isacharbraceleft}{\isasymbottom}\ {\isacharcolon}{\isacharcolon}\ {\isacharprime}a\ formula{\isacharbraceright}{\isacharparenright}{\isachardoublequoteclose}\isanewline
\ \ \isacommand{then}\isamarkupfalse%
\ \isacommand{have}\isamarkupfalse%
\ {\isachardoublequoteopen}sat\ {\isacharbraceleft}{\isasymbottom}\ {\isacharcolon}{\isacharcolon}\ {\isacharprime}a\ formula{\isacharbraceright}{\isachardoublequoteclose}\isanewline
\ \ \ \ \isacommand{by}\isamarkupfalse%
\ {\isacharparenleft}rule\ notnotD{\isacharparenright}\isanewline
\ \ \isacommand{then}\isamarkupfalse%
\ \isacommand{have}\isamarkupfalse%
\ Ex{\isacharcolon}{\isachardoublequoteopen}{\isasymexists}{\isasymA}{\isachardot}\ {\isasymforall}F\ {\isasymin}\ {\isacharbraceleft}{\isasymbottom}\ {\isacharcolon}{\isacharcolon}\ {\isacharprime}a\ formula{\isacharbraceright}{\isachardot}\ {\isasymA}\ {\isasymTurnstile}\ F{\isachardoublequoteclose}\isanewline
\ \ \ \ \isacommand{by}\isamarkupfalse%
\ {\isacharparenleft}simp\ only{\isacharcolon}\ sat{\isacharunderscore}def{\isacharparenright}\isanewline
\ \ \isacommand{obtain}\isamarkupfalse%
\ {\isasymA}\ \isakeyword{where}\ {\isadigit{1}}{\isacharcolon}{\isachardoublequoteopen}{\isasymforall}F\ {\isasymin}\ {\isacharbraceleft}{\isasymbottom}\ {\isacharcolon}{\isacharcolon}\ {\isacharprime}a\ formula{\isacharbraceright}{\isachardot}\ {\isasymA}\ {\isasymTurnstile}\ F{\isachardoublequoteclose}\isanewline
\ \ \ \ \isacommand{using}\isamarkupfalse%
\ Ex\ \isacommand{by}\isamarkupfalse%
\ {\isacharparenleft}rule\ exE{\isacharparenright}\isanewline
\ \ \isacommand{have}\isamarkupfalse%
\ {\isadigit{2}}{\isacharcolon}{\isachardoublequoteopen}{\isasymbottom}\ {\isasymin}\ {\isacharbraceleft}{\isasymbottom}{\isacharcolon}{\isacharcolon}\ {\isacharprime}a\ formula{\isacharbraceright}{\isachardoublequoteclose}\isanewline
\ \ \ \ \isacommand{by}\isamarkupfalse%
\ {\isacharparenleft}simp\ only{\isacharcolon}\ singletonI{\isacharparenright}\isanewline
\ \ \isacommand{have}\isamarkupfalse%
\ {\isachardoublequoteopen}{\isasymA}\ {\isasymTurnstile}\ {\isasymbottom}{\isachardoublequoteclose}\isanewline
\ \ \ \ \isacommand{using}\isamarkupfalse%
\ {\isadigit{1}}\ {\isadigit{2}}\ \isacommand{by}\isamarkupfalse%
\ {\isacharparenleft}rule\ bspec{\isacharparenright}\isanewline
\ \ \isacommand{thus}\isamarkupfalse%
\ {\isachardoublequoteopen}False{\isachardoublequoteclose}\isanewline
\ \ \ \ \isacommand{by}\isamarkupfalse%
\ {\isacharparenleft}simp\ only{\isacharcolon}\ formula{\isacharunderscore}semantics{\isachardot}simps{\isacharparenleft}{\isadigit{2}}{\isacharparenright}{\isacharparenright}\isanewline
\isacommand{qed}\isamarkupfalse%
%
\endisatagproof
{\isafoldproof}%
%
\isadelimproof
%
\endisadelimproof
%
\begin{isamarkuptext}%
Por tanto, probemos que si \isa{W\ {\isasymin}\ C}, entonces \isa{{\isasymbottom}\ {\isasymnotin}\ W}.%
\end{isamarkuptext}\isamarkuptrue%
\isacommand{lemma}\isamarkupfalse%
\ pcp{\isacharunderscore}colecComp{\isacharunderscore}bot{\isacharcolon}\isanewline
\ \ \isakeyword{assumes}\ {\isachardoublequoteopen}W\ {\isasymin}\ colecComp{\isachardoublequoteclose}\isanewline
\ \ \isakeyword{shows}\ {\isachardoublequoteopen}{\isasymbottom}\ {\isasymnotin}\ W{\isachardoublequoteclose}\isanewline
%
\isadelimproof
%
\endisadelimproof
%
\isatagproof
\isacommand{proof}\isamarkupfalse%
\ {\isacharparenleft}rule\ ccontr{\isacharparenright}\isanewline
\ \ \isacommand{assume}\isamarkupfalse%
\ {\isachardoublequoteopen}{\isasymnot}{\isacharparenleft}{\isasymbottom}\ {\isasymnotin}\ W{\isacharparenright}{\isachardoublequoteclose}\isanewline
\ \ \isacommand{then}\isamarkupfalse%
\ \isacommand{have}\isamarkupfalse%
\ {\isachardoublequoteopen}{\isasymbottom}\ {\isasymin}\ W{\isachardoublequoteclose}\isanewline
\ \ \ \ \isacommand{by}\isamarkupfalse%
\ {\isacharparenleft}rule\ notnotD{\isacharparenright}\isanewline
\ \ \isacommand{have}\isamarkupfalse%
\ {\isachardoublequoteopen}{\isacharbraceleft}{\isacharbraceright}\ {\isasymsubseteq}\ W{\isachardoublequoteclose}\ \isanewline
\ \ \ \ \isacommand{by}\isamarkupfalse%
\ {\isacharparenleft}simp\ only{\isacharcolon}\ empty{\isacharunderscore}subsetI{\isacharparenright}\ \isanewline
\ \ \isacommand{have}\isamarkupfalse%
\ {\isachardoublequoteopen}{\isasymbottom}\ {\isasymin}\ W\ {\isasymand}\ {\isacharbraceleft}{\isacharbraceright}\ {\isasymsubseteq}\ W{\isachardoublequoteclose}\isanewline
\ \ \ \ \isacommand{using}\isamarkupfalse%
\ {\isacartoucheopen}{\isasymbottom}\ {\isasymin}\ W{\isacartoucheclose}\ {\isacartoucheopen}{\isacharbraceleft}{\isacharbraceright}\ {\isasymsubseteq}\ W{\isacartoucheclose}\ \isacommand{by}\isamarkupfalse%
\ {\isacharparenleft}rule\ conjI{\isacharparenright}\isanewline
\ \ \isacommand{then}\isamarkupfalse%
\ \isacommand{have}\isamarkupfalse%
\ {\isachardoublequoteopen}{\isacharbraceleft}{\isasymbottom}{\isacharbraceright}\ {\isasymsubseteq}\ W{\isachardoublequoteclose}\isanewline
\ \ \ \ \isacommand{by}\isamarkupfalse%
\ {\isacharparenleft}simp\ only{\isacharcolon}\ insert{\isacharunderscore}subset{\isacharparenright}\isanewline
\ \ \isacommand{have}\isamarkupfalse%
\ {\isachardoublequoteopen}finite\ {\isacharbraceleft}{\isasymbottom}{\isacharbraceright}{\isachardoublequoteclose}\ \isanewline
\ \ \ \ \isacommand{by}\isamarkupfalse%
\ {\isacharparenleft}simp\ only{\isacharcolon}\ finite{\isachardot}emptyI\ finite{\isacharunderscore}insert{\isacharparenright}\isanewline
\ \ \isacommand{have}\isamarkupfalse%
\ {\isachardoublequoteopen}sat\ {\isacharbraceleft}{\isasymbottom}\ {\isacharcolon}{\isacharcolon}\ {\isacharprime}a\ formula{\isacharbraceright}{\isachardoublequoteclose}\ \isanewline
\ \ \ \ \isacommand{using}\isamarkupfalse%
\ assms\ {\isacartoucheopen}{\isacharbraceleft}{\isasymbottom}{\isacharbraceright}\ {\isasymsubseteq}\ W{\isacartoucheclose}\ {\isacartoucheopen}finite\ {\isacharbraceleft}{\isasymbottom}{\isacharbraceright}{\isacartoucheclose}\ \isacommand{by}\isamarkupfalse%
\ {\isacharparenleft}rule\ colecComp{\isacharunderscore}subset{\isacharunderscore}finite{\isacharparenright}\isanewline
\ \ \isacommand{have}\isamarkupfalse%
\ {\isachardoublequoteopen}{\isasymnot}\ sat\ {\isacharbraceleft}{\isasymbottom}\ {\isacharcolon}{\isacharcolon}\ {\isacharprime}a\ formula{\isacharbraceright}{\isachardoublequoteclose}\ \isanewline
\ \ \ \ \isacommand{by}\isamarkupfalse%
\ {\isacharparenleft}rule\ not{\isacharunderscore}sat{\isacharunderscore}bot{\isacharparenright}\isanewline
\ \ \isacommand{then}\isamarkupfalse%
\ \isacommand{show}\isamarkupfalse%
\ False\ \isanewline
\ \ \ \ \isacommand{using}\isamarkupfalse%
\ {\isacartoucheopen}sat\ {\isacharbraceleft}{\isasymbottom}\ {\isacharcolon}{\isacharcolon}\ {\isacharprime}a\ formula{\isacharbraceright}{\isacartoucheclose}\ \isacommand{by}\isamarkupfalse%
\ {\isacharparenleft}rule\ notE{\isacharparenright}\isanewline
\isacommand{qed}\isamarkupfalse%
%
\endisatagproof
{\isafoldproof}%
%
\isadelimproof
%
\endisadelimproof
%
\begin{isamarkuptext}%
Por otro lado, vamos a probar que dado un conjunto \isa{W\ {\isasymin}\ C} y \isa{p} una fórmula atómica 
  cualquiera, no se tiene simultáneamente que \isa{p\ {\isasymin}\ W} y \isa{{\isasymnot}\ p\ {\isasymin}\ W}. Para ello, emplearemos el 
  siguiente lema auxiliar que prueba que el conjunto \isa{{\isacharbraceleft}p{\isacharcomma}{\isasymnot}\ p{\isacharbraceright}} es insatisfacible para cualquier 
  fórmula atómica \isa{p}.%
\end{isamarkuptext}\isamarkuptrue%
\isacommand{lemma}\isamarkupfalse%
\ not{\isacharunderscore}sat{\isacharunderscore}atoms{\isacharcolon}\ {\isachardoublequoteopen}{\isasymnot}\ sat{\isacharparenleft}{\isacharbraceleft}Atom\ k{\isacharcomma}\ \isactrlbold {\isasymnot}\ {\isacharparenleft}Atom\ k{\isacharparenright}{\isacharbraceright}{\isacharparenright}{\isachardoublequoteclose}\isanewline
%
\isadelimproof
%
\endisadelimproof
%
\isatagproof
\isacommand{proof}\isamarkupfalse%
\ {\isacharparenleft}rule\ ccontr{\isacharparenright}\isanewline
\ \ \isacommand{assume}\isamarkupfalse%
\ {\isachardoublequoteopen}{\isasymnot}\ {\isasymnot}\ sat{\isacharparenleft}{\isacharbraceleft}Atom\ k{\isacharcomma}\ \isactrlbold {\isasymnot}\ {\isacharparenleft}Atom\ k{\isacharparenright}{\isacharbraceright}{\isacharparenright}{\isachardoublequoteclose}\isanewline
\ \ \isacommand{then}\isamarkupfalse%
\ \isacommand{have}\isamarkupfalse%
\ {\isachardoublequoteopen}sat{\isacharparenleft}{\isacharbraceleft}Atom\ k{\isacharcomma}\ \isactrlbold {\isasymnot}\ {\isacharparenleft}Atom\ k{\isacharparenright}{\isacharbraceright}{\isacharparenright}{\isachardoublequoteclose}\isanewline
\ \ \ \ \isacommand{by}\isamarkupfalse%
\ {\isacharparenleft}rule\ notnotD{\isacharparenright}\isanewline
\ \ \isacommand{then}\isamarkupfalse%
\ \isacommand{have}\isamarkupfalse%
\ Sat{\isacharcolon}{\isachardoublequoteopen}{\isasymexists}{\isasymA}{\isachardot}\ {\isasymforall}F\ {\isasymin}\ {\isacharbraceleft}Atom\ k{\isacharcomma}\ \isactrlbold {\isasymnot}{\isacharparenleft}Atom\ k{\isacharparenright}{\isacharbraceright}{\isachardot}\ {\isasymA}\ {\isasymTurnstile}\ F{\isachardoublequoteclose}\isanewline
\ \ \ \ \isacommand{by}\isamarkupfalse%
\ {\isacharparenleft}simp\ only{\isacharcolon}\ sat{\isacharunderscore}def{\isacharparenright}\isanewline
\ \ \isacommand{obtain}\isamarkupfalse%
\ {\isasymA}\ \isakeyword{where}\ H{\isacharcolon}{\isachardoublequoteopen}{\isasymforall}F\ {\isasymin}\ {\isacharbraceleft}Atom\ k{\isacharcomma}\ \isactrlbold {\isasymnot}{\isacharparenleft}Atom\ k{\isacharparenright}{\isacharbraceright}{\isachardot}\ {\isasymA}\ {\isasymTurnstile}\ F{\isachardoublequoteclose}\isanewline
\ \ \ \ \isacommand{using}\isamarkupfalse%
\ Sat\ \isacommand{by}\isamarkupfalse%
\ {\isacharparenleft}rule\ exE{\isacharparenright}\isanewline
\ \ \isacommand{have}\isamarkupfalse%
\ {\isachardoublequoteopen}Atom\ k\ {\isasymin}\ {\isacharbraceleft}Atom\ k{\isacharcomma}\ \isactrlbold {\isasymnot}{\isacharparenleft}Atom\ k{\isacharparenright}{\isacharbraceright}{\isachardoublequoteclose}\isanewline
\ \ \ \ \isacommand{by}\isamarkupfalse%
\ simp\isanewline
\ \ \isacommand{have}\isamarkupfalse%
\ {\isachardoublequoteopen}{\isasymA}\ {\isasymTurnstile}\ Atom\ k{\isachardoublequoteclose}\isanewline
\ \ \ \ \isacommand{using}\isamarkupfalse%
\ H\ {\isacartoucheopen}Atom\ k\ {\isasymin}\ {\isacharbraceleft}Atom\ k{\isacharcomma}\ \isactrlbold {\isasymnot}{\isacharparenleft}Atom\ k{\isacharparenright}{\isacharbraceright}{\isacartoucheclose}\ \isacommand{by}\isamarkupfalse%
\ {\isacharparenleft}rule\ bspec{\isacharparenright}\isanewline
\ \ \isacommand{have}\isamarkupfalse%
\ {\isachardoublequoteopen}\isactrlbold {\isasymnot}{\isacharparenleft}Atom\ k{\isacharparenright}\ {\isasymin}\ {\isacharbraceleft}Atom\ k{\isacharcomma}\ \isactrlbold {\isasymnot}{\isacharparenleft}Atom\ k{\isacharparenright}{\isacharbraceright}{\isachardoublequoteclose}\isanewline
\ \ \ \ \isacommand{by}\isamarkupfalse%
\ simp\isanewline
\ \ \isacommand{have}\isamarkupfalse%
\ {\isachardoublequoteopen}{\isasymA}\ {\isasymTurnstile}\ \isactrlbold {\isasymnot}{\isacharparenleft}Atom\ k{\isacharparenright}{\isachardoublequoteclose}\isanewline
\ \ \ \ \isacommand{using}\isamarkupfalse%
\ H\ {\isacartoucheopen}\isactrlbold {\isasymnot}{\isacharparenleft}Atom\ k{\isacharparenright}\ {\isasymin}\ {\isacharbraceleft}Atom\ k{\isacharcomma}\ \isactrlbold {\isasymnot}{\isacharparenleft}Atom\ k{\isacharparenright}{\isacharbraceright}{\isacartoucheclose}\ \isacommand{by}\isamarkupfalse%
\ {\isacharparenleft}rule\ bspec{\isacharparenright}\isanewline
\ \ \isacommand{then}\isamarkupfalse%
\ \isacommand{have}\isamarkupfalse%
\ {\isachardoublequoteopen}{\isasymnot}\ {\isasymA}\ {\isasymTurnstile}\ Atom\ k{\isachardoublequoteclose}\ \isanewline
\ \ \ \ \isacommand{by}\isamarkupfalse%
\ {\isacharparenleft}simp\ only{\isacharcolon}\ simp{\isacharunderscore}thms{\isacharparenleft}{\isadigit{8}}{\isacharparenright}\ formula{\isacharunderscore}semantics{\isachardot}simps{\isacharparenleft}{\isadigit{3}}{\isacharparenright}{\isacharparenright}\isanewline
\ \ \isacommand{thus}\isamarkupfalse%
\ {\isachardoublequoteopen}False{\isachardoublequoteclose}\isanewline
\ \ \ \ \isacommand{using}\isamarkupfalse%
\ {\isacartoucheopen}{\isasymA}\ {\isasymTurnstile}\ Atom\ k{\isacartoucheclose}\ \isacommand{by}\isamarkupfalse%
\ {\isacharparenleft}rule\ notE{\isacharparenright}\isanewline
\isacommand{qed}\isamarkupfalse%
%
\endisatagproof
{\isafoldproof}%
%
\isadelimproof
%
\endisadelimproof
%
\begin{isamarkuptext}%
De este modo, podemos demostrar detalladamente la condición: dados \isa{W\ {\isasymin}\ C} y \isa{p} una fórmula
  atómica cualquiera, no se tiene simultáneamente que \isa{p\ {\isasymin}\ W} y \isa{{\isasymnot}\ p\ {\isasymin}\ W}.%
\end{isamarkuptext}\isamarkuptrue%
\isacommand{lemma}\isamarkupfalse%
\ pcp{\isacharunderscore}colecComp{\isacharunderscore}atoms{\isacharcolon}\isanewline
\ \ \isakeyword{assumes}\ {\isachardoublequoteopen}W\ {\isasymin}\ colecComp{\isachardoublequoteclose}\isanewline
\ \ \isakeyword{shows}\ {\isachardoublequoteopen}{\isasymforall}k{\isachardot}\ Atom\ k\ {\isasymin}\ W\ {\isasymlongrightarrow}\ \isactrlbold {\isasymnot}\ {\isacharparenleft}Atom\ k{\isacharparenright}\ {\isasymin}\ W\ {\isasymlongrightarrow}\ False{\isachardoublequoteclose}\isanewline
%
\isadelimproof
%
\endisadelimproof
%
\isatagproof
\isacommand{proof}\isamarkupfalse%
\ {\isacharparenleft}rule\ allI{\isacharparenright}\isanewline
\ \ \isacommand{fix}\isamarkupfalse%
\ k\isanewline
\ \ \isacommand{show}\isamarkupfalse%
\ {\isachardoublequoteopen}Atom\ k\ {\isasymin}\ W\ {\isasymlongrightarrow}\ \isactrlbold {\isasymnot}\ {\isacharparenleft}Atom\ k{\isacharparenright}\ {\isasymin}\ W\ {\isasymlongrightarrow}\ False{\isachardoublequoteclose}\isanewline
\ \ \isacommand{proof}\isamarkupfalse%
\ {\isacharparenleft}rule\ impI{\isacharparenright}{\isacharplus}\isanewline
\ \ \ \ \isacommand{assume}\isamarkupfalse%
\ {\isadigit{1}}{\isacharcolon}{\isachardoublequoteopen}Atom\ k\ {\isasymin}\ W{\isachardoublequoteclose}\isanewline
\ \ \ \ \isacommand{assume}\isamarkupfalse%
\ {\isadigit{2}}{\isacharcolon}{\isachardoublequoteopen}\isactrlbold {\isasymnot}\ {\isacharparenleft}Atom\ k{\isacharparenright}\ {\isasymin}\ W{\isachardoublequoteclose}\isanewline
\ \ \ \ \isacommand{have}\isamarkupfalse%
\ {\isachardoublequoteopen}{\isacharbraceleft}{\isacharbraceright}\ {\isasymsubseteq}\ W{\isachardoublequoteclose}\isanewline
\ \ \ \ \ \ \isacommand{by}\isamarkupfalse%
\ {\isacharparenleft}simp\ only{\isacharcolon}\ empty{\isacharunderscore}subsetI{\isacharparenright}\ \isanewline
\ \ \ \ \isacommand{have}\isamarkupfalse%
\ {\isachardoublequoteopen}Atom\ k\ {\isasymin}\ W\ {\isasymand}\ {\isacharbraceleft}{\isacharbraceright}\ {\isasymsubseteq}\ W{\isachardoublequoteclose}\isanewline
\ \ \ \ \ \ \isacommand{using}\isamarkupfalse%
\ {\isadigit{1}}\ {\isacartoucheopen}{\isacharbraceleft}{\isacharbraceright}\ {\isasymsubseteq}\ W{\isacartoucheclose}\ \isacommand{by}\isamarkupfalse%
\ {\isacharparenleft}rule\ conjI{\isacharparenright}\isanewline
\ \ \ \ \isacommand{then}\isamarkupfalse%
\ \isacommand{have}\isamarkupfalse%
\ {\isachardoublequoteopen}{\isacharbraceleft}Atom\ k{\isacharbraceright}\ {\isasymsubseteq}\ W{\isachardoublequoteclose}\isanewline
\ \ \ \ \ \ \isacommand{by}\isamarkupfalse%
\ {\isacharparenleft}simp\ only{\isacharcolon}\ insert{\isacharunderscore}subset{\isacharparenright}\isanewline
\ \ \ \ \isacommand{have}\isamarkupfalse%
\ {\isachardoublequoteopen}\isactrlbold {\isasymnot}\ {\isacharparenleft}Atom\ k{\isacharparenright}\ {\isasymin}\ W\ {\isasymand}\ {\isacharbraceleft}{\isacharbraceright}\ {\isasymsubseteq}\ W{\isachardoublequoteclose}\isanewline
\ \ \ \ \ \ \isacommand{using}\isamarkupfalse%
\ {\isadigit{2}}\ {\isacartoucheopen}{\isacharbraceleft}{\isacharbraceright}\ {\isasymsubseteq}\ W{\isacartoucheclose}\ \isacommand{by}\isamarkupfalse%
\ {\isacharparenleft}rule\ conjI{\isacharparenright}\isanewline
\ \ \ \ \isacommand{then}\isamarkupfalse%
\ \isacommand{have}\isamarkupfalse%
\ {\isachardoublequoteopen}{\isacharbraceleft}\isactrlbold {\isasymnot}{\isacharparenleft}Atom\ k{\isacharparenright}{\isacharbraceright}\ {\isasymsubseteq}\ W{\isachardoublequoteclose}\isanewline
\ \ \ \ \ \ \isacommand{by}\isamarkupfalse%
\ {\isacharparenleft}simp\ only{\isacharcolon}\ insert{\isacharunderscore}subset{\isacharparenright}\isanewline
\ \ \ \ \isacommand{have}\isamarkupfalse%
\ {\isachardoublequoteopen}{\isacharbraceleft}Atom\ k{\isacharbraceright}\ {\isasymunion}\ {\isacharbraceleft}\isactrlbold {\isasymnot}{\isacharparenleft}Atom\ k{\isacharparenright}{\isacharbraceright}\ {\isasymsubseteq}\ W{\isachardoublequoteclose}\isanewline
\ \ \ \ \ \ \isacommand{using}\isamarkupfalse%
\ {\isacartoucheopen}{\isacharbraceleft}Atom\ k{\isacharbraceright}\ {\isasymsubseteq}\ W{\isacartoucheclose}\ {\isacartoucheopen}{\isacharbraceleft}\isactrlbold {\isasymnot}{\isacharparenleft}Atom\ k{\isacharparenright}{\isacharbraceright}\ {\isasymsubseteq}\ W{\isacartoucheclose}\ \isacommand{by}\isamarkupfalse%
\ {\isacharparenleft}simp\ only{\isacharcolon}\ Un{\isacharunderscore}least{\isacharparenright}\isanewline
\ \ \ \ \isacommand{then}\isamarkupfalse%
\ \isacommand{have}\isamarkupfalse%
\ {\isachardoublequoteopen}{\isacharbraceleft}Atom\ k{\isacharcomma}\ \isactrlbold {\isasymnot}{\isacharparenleft}Atom\ k{\isacharparenright}{\isacharbraceright}\ {\isasymsubseteq}\ W{\isachardoublequoteclose}\isanewline
\ \ \ \ \ \ \isacommand{by}\isamarkupfalse%
\ simp\ \isanewline
\ \ \ \ \isacommand{have}\isamarkupfalse%
\ {\isachardoublequoteopen}finite\ {\isacharbraceleft}Atom\ k{\isacharcomma}\ \isactrlbold {\isasymnot}{\isacharparenleft}Atom\ k{\isacharparenright}{\isacharbraceright}{\isachardoublequoteclose}\isanewline
\ \ \ \ \ \ \isacommand{by}\isamarkupfalse%
\ blast\isanewline
\ \ \ \ \isacommand{have}\isamarkupfalse%
\ {\isachardoublequoteopen}sat\ {\isacharparenleft}{\isacharbraceleft}Atom\ k{\isacharcomma}\ \isactrlbold {\isasymnot}{\isacharparenleft}Atom\ k{\isacharparenright}{\isacharbraceright}{\isacharparenright}{\isachardoublequoteclose}\isanewline
\ \ \ \ \ \ \isacommand{using}\isamarkupfalse%
\ assms\ {\isacartoucheopen}{\isacharbraceleft}Atom\ k{\isacharcomma}\ \isactrlbold {\isasymnot}{\isacharparenleft}Atom\ k{\isacharparenright}{\isacharbraceright}\ {\isasymsubseteq}\ W{\isacartoucheclose}\ {\isacartoucheopen}finite\ {\isacharbraceleft}Atom\ k{\isacharcomma}\ \isactrlbold {\isasymnot}{\isacharparenleft}Atom\ k{\isacharparenright}{\isacharbraceright}{\isacartoucheclose}\ \isacommand{by}\isamarkupfalse%
\ {\isacharparenleft}rule\ colecComp{\isacharunderscore}subset{\isacharunderscore}finite{\isacharparenright}\isanewline
\ \ \ \ \isacommand{have}\isamarkupfalse%
\ {\isachardoublequoteopen}{\isasymnot}\ sat\ {\isacharparenleft}{\isacharbraceleft}Atom\ k{\isacharcomma}\ \isactrlbold {\isasymnot}{\isacharparenleft}Atom\ k{\isacharparenright}{\isacharbraceright}{\isacharparenright}{\isachardoublequoteclose}\isanewline
\ \ \ \ \ \ \isacommand{by}\isamarkupfalse%
\ {\isacharparenleft}rule\ not{\isacharunderscore}sat{\isacharunderscore}atoms{\isacharparenright}\isanewline
\ \ \ \ \isacommand{thus}\isamarkupfalse%
\ {\isachardoublequoteopen}False{\isachardoublequoteclose}\isanewline
\ \ \ \ \ \ \isacommand{using}\isamarkupfalse%
\ {\isacartoucheopen}sat\ {\isacharparenleft}{\isacharbraceleft}Atom\ k{\isacharcomma}\ \isactrlbold {\isasymnot}{\isacharparenleft}Atom\ k{\isacharparenright}{\isacharbraceright}{\isacharparenright}{\isacartoucheclose}\ \isacommand{by}\isamarkupfalse%
\ {\isacharparenleft}rule\ notE{\isacharparenright}\isanewline
\ \ \isacommand{qed}\isamarkupfalse%
\isanewline
\isacommand{qed}\isamarkupfalse%
%
\endisatagproof
{\isafoldproof}%
%
\isadelimproof
%
\endisadelimproof
%
\begin{isamarkuptext}%
Demostremos la tercera condición del lema \isa{{\isadigit{2}}{\isachardot}{\isadigit{0}}{\isachardot}{\isadigit{2}}}: dados \isa{W\ {\isasymin}\ C} y \isa{F} una fórmula de 
  tipo \isa{{\isasymalpha}} con componentes \isa{{\isasymalpha}\isactrlsub {\isadigit{1}}} y \isa{{\isasymalpha}\isactrlsub {\isadigit{2}}} tal que \isa{F\ {\isasymin}\ W}, se tiene que \isa{{\isacharbraceleft}{\isasymalpha}\isactrlsub {\isadigit{1}}{\isacharcomma}{\isasymalpha}\isactrlsub {\isadigit{2}}{\isacharbraceright}\ {\isasymunion}\ W\ {\isasymin}\ C}. Para probar 
  dicho resultado, emplearemos un lema auxiliar que demuestra que dado un subconjunto finito \isa{W\isactrlsub {\isadigit{0}}} de 
  \isa{W} se tiene que \isa{{\isacharbraceleft}{\isasymalpha}\isactrlsub {\isadigit{1}}{\isacharcomma}{\isasymalpha}\isactrlsub {\isadigit{2}}{\isacharcomma}F{\isacharbraceright}\ {\isasymunion}\ W\isactrlsub {\isadigit{0}}} es un conjunto satisfacible. Mostraremos la prueba para cada
  caso de fórmula de tipo \isa{{\isasymalpha}}. Para ello, precisaremos del siguiente lema auxiliar que demuestra que 
  dado un conjunto \isa{W\ {\isasymin}\ C}, \isa{F} una fórmula perteneciente a \isa{W} y \isa{W\isactrlsub {\isadigit{0}}} un subconjunto finito de \isa{W}, 
  entonces \isa{{\isacharbraceleft}F{\isacharbraceright}\ {\isasymunion}\ W\isactrlsub {\isadigit{0}}} es satisfacible.%
\end{isamarkuptext}\isamarkuptrue%
\isacommand{lemma}\isamarkupfalse%
\ pcp{\isacharunderscore}colecComp{\isacharunderscore}elem{\isacharunderscore}sat{\isacharcolon}\isanewline
\ \ \isakeyword{assumes}\ {\isachardoublequoteopen}W\ {\isasymin}\ colecComp{\isachardoublequoteclose}\isanewline
\ \ \ \ \ \ \ \ \ \ {\isachardoublequoteopen}F\ {\isasymin}\ W{\isachardoublequoteclose}\isanewline
\ \ \ \ \ \ \ \ \ \ {\isachardoublequoteopen}finite\ Wo{\isachardoublequoteclose}\isanewline
\ \ \ \ \ \ \ \ \ \ {\isachardoublequoteopen}Wo\ {\isasymsubseteq}\ W{\isachardoublequoteclose}\isanewline
\ \ \ \ \ \ \ \ \isakeyword{shows}\ {\isachardoublequoteopen}sat\ {\isacharparenleft}{\isacharbraceleft}F{\isacharbraceright}\ {\isasymunion}\ Wo{\isacharparenright}{\isachardoublequoteclose}\isanewline
%
\isadelimproof
%
\endisadelimproof
%
\isatagproof
\isacommand{proof}\isamarkupfalse%
\ {\isacharminus}\isanewline
\ \ \isacommand{have}\isamarkupfalse%
\ {\isadigit{1}}{\isacharcolon}{\isachardoublequoteopen}insert\ F\ Wo\ {\isacharequal}\ {\isacharbraceleft}F{\isacharbraceright}\ {\isasymunion}\ Wo{\isachardoublequoteclose}\isanewline
\ \ \ \ \isacommand{by}\isamarkupfalse%
\ {\isacharparenleft}rule\ insert{\isacharunderscore}is{\isacharunderscore}Un{\isacharparenright}\isanewline
\ \ \isacommand{have}\isamarkupfalse%
\ {\isachardoublequoteopen}finite\ {\isacharparenleft}insert\ F\ Wo{\isacharparenright}{\isachardoublequoteclose}\isanewline
\ \ \ \ \isacommand{using}\isamarkupfalse%
\ assms{\isacharparenleft}{\isadigit{3}}{\isacharparenright}\ \isacommand{by}\isamarkupfalse%
\ {\isacharparenleft}simp\ only{\isacharcolon}\ finite{\isacharunderscore}insert{\isacharparenright}\isanewline
\ \ \isacommand{then}\isamarkupfalse%
\ \isacommand{have}\isamarkupfalse%
\ {\isachardoublequoteopen}finite\ {\isacharparenleft}{\isacharbraceleft}F{\isacharbraceright}\ {\isasymunion}\ Wo{\isacharparenright}{\isachardoublequoteclose}\isanewline
\ \ \ \ \isacommand{by}\isamarkupfalse%
\ {\isacharparenleft}simp\ only{\isacharcolon}\ {\isadigit{1}}{\isacharparenright}\ \isanewline
\ \ \isacommand{have}\isamarkupfalse%
\ {\isachardoublequoteopen}F\ {\isasymin}\ W\ {\isasymand}\ Wo\ {\isasymsubseteq}\ W{\isachardoublequoteclose}\isanewline
\ \ \ \ \isacommand{using}\isamarkupfalse%
\ assms{\isacharparenleft}{\isadigit{2}}{\isacharparenright}\ assms{\isacharparenleft}{\isadigit{4}}{\isacharparenright}\ \isacommand{by}\isamarkupfalse%
\ {\isacharparenleft}rule\ conjI{\isacharparenright}\isanewline
\ \ \isacommand{then}\isamarkupfalse%
\ \isacommand{have}\isamarkupfalse%
\ {\isachardoublequoteopen}insert\ F\ Wo\ {\isasymsubseteq}\ W{\isachardoublequoteclose}\isanewline
\ \ \ \ \isacommand{by}\isamarkupfalse%
\ {\isacharparenleft}simp\ only{\isacharcolon}\ insert{\isacharunderscore}subset{\isacharparenright}\isanewline
\ \ \isacommand{then}\isamarkupfalse%
\ \isacommand{have}\isamarkupfalse%
\ {\isachardoublequoteopen}{\isacharbraceleft}F{\isacharbraceright}\ {\isasymunion}\ Wo\ {\isasymsubseteq}\ W{\isachardoublequoteclose}\isanewline
\ \ \ \ \isacommand{by}\isamarkupfalse%
\ {\isacharparenleft}simp\ only{\isacharcolon}\ {\isadigit{1}}{\isacharparenright}\isanewline
\ \ \isacommand{show}\isamarkupfalse%
\ {\isachardoublequoteopen}sat\ {\isacharparenleft}{\isacharbraceleft}F{\isacharbraceright}\ {\isasymunion}\ Wo{\isacharparenright}{\isachardoublequoteclose}\isanewline
\ \ \ \ \isacommand{using}\isamarkupfalse%
\ assms{\isacharparenleft}{\isadigit{1}}{\isacharparenright}\ {\isacartoucheopen}{\isacharbraceleft}F{\isacharbraceright}\ {\isasymunion}\ Wo\ {\isasymsubseteq}\ W{\isacartoucheclose}\ {\isacartoucheopen}finite\ {\isacharparenleft}{\isacharbraceleft}F{\isacharbraceright}\ {\isasymunion}\ Wo{\isacharparenright}{\isacartoucheclose}\ \isacommand{by}\isamarkupfalse%
\ {\isacharparenleft}rule\ colecComp{\isacharunderscore}subset{\isacharunderscore}finite{\isacharparenright}\isanewline
\isacommand{qed}\isamarkupfalse%
%
\endisatagproof
{\isafoldproof}%
%
\isadelimproof
%
\endisadelimproof
%
\begin{isamarkuptext}%
De este modo, vamos a probar para cada caso de fórmula \isa{{\isasymalpha}} que dados \isa{W\ {\isasymin}\ C}, \isa{F} una fórmula 
  de tipo \isa{{\isasymalpha}} con componentes \isa{{\isasymalpha}\isactrlsub {\isadigit{1}}} y \isa{{\isasymalpha}\isactrlsub {\isadigit{2}}} tal que \isa{F\ {\isasymin}\ W} y \isa{W\isactrlsub {\isadigit{0}}} un subconjunto finito de \isa{W}, se 
  verifica que \isa{{\isacharbraceleft}{\isasymalpha}\isactrlsub {\isadigit{1}}{\isacharcomma}{\isasymalpha}\isactrlsub {\isadigit{2}}{\isacharcomma}F{\isacharbraceright}\ {\isasymunion}\ W\isactrlsub {\isadigit{0}}} es satisfacible. Para ello, emplearemos el siguiente lema auxiliar
  en Isabelle.%
\end{isamarkuptext}\isamarkuptrue%
\isacommand{lemma}\isamarkupfalse%
\ ball{\isacharunderscore}Un{\isacharcolon}\ \isanewline
\ \ \isakeyword{assumes}\ {\isachardoublequoteopen}{\isasymforall}x\ {\isasymin}\ A{\isachardot}\ P\ x{\isachardoublequoteclose}\isanewline
\ \ \ \ \ \ \ \ \ \ {\isachardoublequoteopen}{\isasymforall}x\ {\isasymin}\ B{\isachardot}\ P\ x{\isachardoublequoteclose}\isanewline
\ \ \ \ \ \ \ \ \isakeyword{shows}\ {\isachardoublequoteopen}{\isasymforall}x\ {\isasymin}\ {\isacharparenleft}A\ {\isasymunion}\ B{\isacharparenright}{\isachardot}\ P\ x{\isachardoublequoteclose}\ \isanewline
%
\isadelimproof
\ \ %
\endisadelimproof
%
\isatagproof
\isacommand{using}\isamarkupfalse%
\ assms\ \isacommand{by}\isamarkupfalse%
\ blast%
\endisatagproof
{\isafoldproof}%
%
\isadelimproof
%
\endisadelimproof
%
\begin{isamarkuptext}%
En primer lugar, probemos que dados \isa{W\ {\isasymin}\ C}, una fórmula \isa{F\ {\isacharequal}\ G\ {\isasymand}\ H} para ciertas fórmulas \isa{G} 
  y \isa{H} tal que \isa{F\ {\isasymin}\ W} y \isa{W\isactrlsub {\isadigit{0}}} un subconjunto finito de \isa{W}, se verifica que\\ \isa{{\isacharbraceleft}G{\isacharcomma}H{\isacharcomma}F{\isacharbraceright}\ {\isasymunion}\ W\isactrlsub {\isadigit{0}}} es 
  satisfacible.%
\end{isamarkuptext}\isamarkuptrue%
\isacommand{lemma}\isamarkupfalse%
\ pcp{\isacharunderscore}colecComp{\isacharunderscore}CON{\isacharunderscore}sat{\isadigit{1}}{\isacharcolon}\isanewline
\ \ \isakeyword{assumes}\ {\isachardoublequoteopen}W\ {\isasymin}\ colecComp{\isachardoublequoteclose}\isanewline
\ \ \ \ \ \ \ \ \ \ {\isachardoublequoteopen}F\ {\isacharequal}\ G\ \isactrlbold {\isasymand}\ H{\isachardoublequoteclose}\isanewline
\ \ \ \ \ \ \ \ \ \ {\isachardoublequoteopen}F\ {\isasymin}\ W{\isachardoublequoteclose}\isanewline
\ \ \ \ \ \ \ \ \ \ {\isachardoublequoteopen}finite\ Wo{\isachardoublequoteclose}\isanewline
\ \ \ \ \ \ \ \ \ \ {\isachardoublequoteopen}Wo\ {\isasymsubseteq}\ W{\isachardoublequoteclose}\isanewline
\ \ \ \ \ \ \ \ \isakeyword{shows}\ {\isachardoublequoteopen}sat\ {\isacharparenleft}{\isacharbraceleft}G{\isacharcomma}H{\isacharcomma}F{\isacharbraceright}\ {\isasymunion}\ Wo{\isacharparenright}{\isachardoublequoteclose}\isanewline
%
\isadelimproof
%
\endisadelimproof
%
\isatagproof
\isacommand{proof}\isamarkupfalse%
\ {\isacharminus}\isanewline
\ \ \isacommand{have}\isamarkupfalse%
\ {\isachardoublequoteopen}sat\ {\isacharparenleft}{\isacharbraceleft}F{\isacharbraceright}\ {\isasymunion}\ Wo{\isacharparenright}{\isachardoublequoteclose}\isanewline
\ \ \ \ \isacommand{using}\isamarkupfalse%
\ assms{\isacharparenleft}{\isadigit{1}}{\isacharcomma}{\isadigit{3}}{\isacharcomma}{\isadigit{4}}{\isacharcomma}{\isadigit{5}}{\isacharparenright}\ \isacommand{by}\isamarkupfalse%
\ {\isacharparenleft}rule\ pcp{\isacharunderscore}colecComp{\isacharunderscore}elem{\isacharunderscore}sat{\isacharparenright}\isanewline
\ \ \isacommand{have}\isamarkupfalse%
\ {\isachardoublequoteopen}F\ {\isasymin}\ {\isacharbraceleft}F{\isacharbraceright}\ {\isasymunion}\ Wo{\isachardoublequoteclose}\isanewline
\ \ \ \ \isacommand{by}\isamarkupfalse%
\ {\isacharparenleft}simp\ add{\isacharcolon}\ insertI{\isadigit{1}}{\isacharparenright}\isanewline
\ \ \isacommand{have}\isamarkupfalse%
\ Ex{\isadigit{1}}{\isacharcolon}{\isachardoublequoteopen}{\isasymexists}{\isasymA}{\isachardot}\ {\isasymforall}F\ {\isasymin}\ {\isacharparenleft}{\isacharbraceleft}F{\isacharbraceright}\ {\isasymunion}\ Wo{\isacharparenright}{\isachardot}\ {\isasymA}\ {\isasymTurnstile}\ F{\isachardoublequoteclose}\isanewline
\ \ \ \ \isacommand{using}\isamarkupfalse%
\ {\isacartoucheopen}sat\ {\isacharparenleft}{\isacharbraceleft}F{\isacharbraceright}\ {\isasymunion}\ Wo{\isacharparenright}{\isacartoucheclose}\ \isacommand{by}\isamarkupfalse%
\ {\isacharparenleft}simp\ only{\isacharcolon}\ sat{\isacharunderscore}def{\isacharparenright}\isanewline
\ \ \isacommand{obtain}\isamarkupfalse%
\ {\isasymA}\ \isakeyword{where}\ Forall{\isadigit{1}}{\isacharcolon}{\isachardoublequoteopen}{\isasymforall}F\ {\isasymin}\ {\isacharparenleft}{\isacharbraceleft}F{\isacharbraceright}\ {\isasymunion}\ Wo{\isacharparenright}{\isachardot}\ {\isasymA}\ {\isasymTurnstile}\ F{\isachardoublequoteclose}\isanewline
\ \ \ \ \isacommand{using}\isamarkupfalse%
\ Ex{\isadigit{1}}\ \isacommand{by}\isamarkupfalse%
\ {\isacharparenleft}rule\ exE{\isacharparenright}\isanewline
\ \ \isacommand{have}\isamarkupfalse%
\ {\isachardoublequoteopen}{\isasymA}\ {\isasymTurnstile}\ F{\isachardoublequoteclose}\isanewline
\ \ \ \ \isacommand{using}\isamarkupfalse%
\ Forall{\isadigit{1}}\ {\isacartoucheopen}F\ {\isasymin}\ {\isacharbraceleft}F{\isacharbraceright}\ {\isasymunion}\ Wo{\isacartoucheclose}\ \isacommand{by}\isamarkupfalse%
\ {\isacharparenleft}rule\ bspec{\isacharparenright}\isanewline
\ \ \isacommand{then}\isamarkupfalse%
\ \isacommand{have}\isamarkupfalse%
\ {\isachardoublequoteopen}{\isasymA}\ {\isasymTurnstile}\ {\isacharparenleft}G\ \isactrlbold {\isasymand}\ H{\isacharparenright}{\isachardoublequoteclose}\isanewline
\ \ \ \ \isacommand{using}\isamarkupfalse%
\ assms{\isacharparenleft}{\isadigit{2}}{\isacharparenright}\ \isacommand{by}\isamarkupfalse%
\ {\isacharparenleft}simp\ only{\isacharcolon}\ {\isacartoucheopen}{\isasymA}\ {\isasymTurnstile}\ F{\isacartoucheclose}{\isacharparenright}\isanewline
\ \ \isacommand{then}\isamarkupfalse%
\ \isacommand{have}\isamarkupfalse%
\ {\isachardoublequoteopen}{\isasymA}\ {\isasymTurnstile}\ G\ {\isasymand}\ {\isasymA}\ {\isasymTurnstile}\ H{\isachardoublequoteclose}\isanewline
\ \ \ \ \isacommand{by}\isamarkupfalse%
\ {\isacharparenleft}simp\ only{\isacharcolon}\ formula{\isacharunderscore}semantics{\isachardot}simps{\isacharparenleft}{\isadigit{4}}{\isacharparenright}{\isacharparenright}\isanewline
\ \ \isacommand{then}\isamarkupfalse%
\ \isacommand{have}\isamarkupfalse%
\ {\isachardoublequoteopen}{\isasymA}\ {\isasymTurnstile}\ G{\isachardoublequoteclose}\isanewline
\ \ \ \ \isacommand{by}\isamarkupfalse%
\ {\isacharparenleft}rule\ conjunct{\isadigit{1}}{\isacharparenright}\isanewline
\ \ \isacommand{then}\isamarkupfalse%
\ \isacommand{have}\isamarkupfalse%
\ {\isadigit{1}}{\isacharcolon}{\isachardoublequoteopen}{\isasymforall}F\ {\isasymin}\ {\isacharbraceleft}G{\isacharbraceright}{\isachardot}\ {\isasymA}\ {\isasymTurnstile}\ F{\isachardoublequoteclose}\isanewline
\ \ \ \ \isacommand{by}\isamarkupfalse%
\ simp\isanewline
\ \ \isacommand{have}\isamarkupfalse%
\ {\isachardoublequoteopen}{\isasymA}\ {\isasymTurnstile}\ H{\isachardoublequoteclose}\isanewline
\ \ \ \ \isacommand{using}\isamarkupfalse%
\ {\isacartoucheopen}{\isasymA}\ {\isasymTurnstile}\ G\ {\isasymand}\ {\isasymA}\ {\isasymTurnstile}\ H{\isacartoucheclose}\ \isacommand{by}\isamarkupfalse%
\ {\isacharparenleft}rule\ conjunct{\isadigit{2}}{\isacharparenright}\isanewline
\ \ \isacommand{then}\isamarkupfalse%
\ \isacommand{have}\isamarkupfalse%
\ {\isadigit{2}}{\isacharcolon}{\isachardoublequoteopen}{\isasymforall}F\ {\isasymin}\ {\isacharbraceleft}H{\isacharbraceright}{\isachardot}\ {\isasymA}\ {\isasymTurnstile}\ F{\isachardoublequoteclose}\isanewline
\ \ \ \ \isacommand{by}\isamarkupfalse%
\ simp\isanewline
\ \ \isacommand{have}\isamarkupfalse%
\ {\isachardoublequoteopen}{\isasymforall}F\ {\isasymin}\ {\isacharparenleft}{\isacharbraceleft}G{\isacharbraceright}\ {\isasymunion}\ {\isacharbraceleft}H{\isacharbraceright}{\isacharparenright}\ {\isasymunion}\ {\isacharparenleft}{\isacharbraceleft}F{\isacharbraceright}\ {\isasymunion}\ Wo{\isacharparenright}{\isachardot}\ {\isasymA}\ {\isasymTurnstile}\ F{\isachardoublequoteclose}\isanewline
\ \ \ \ \isacommand{using}\isamarkupfalse%
\ Forall{\isadigit{1}}\ {\isadigit{1}}\ {\isadigit{2}}\ \isacommand{by}\isamarkupfalse%
\ {\isacharparenleft}iprover\ intro{\isacharcolon}\ ball{\isacharunderscore}Un{\isacharparenright}\isanewline
\ \ \isacommand{then}\isamarkupfalse%
\ \isacommand{have}\isamarkupfalse%
\ {\isachardoublequoteopen}{\isasymforall}F\ {\isasymin}\ {\isacharparenleft}{\isacharbraceleft}G{\isacharcomma}H{\isacharcomma}F{\isacharbraceright}\ {\isasymunion}\ Wo{\isacharparenright}{\isachardot}\ {\isasymA}\ {\isasymTurnstile}\ F{\isachardoublequoteclose}\isanewline
\ \ \ \ \isacommand{by}\isamarkupfalse%
\ simp\isanewline
\ \ \isacommand{then}\isamarkupfalse%
\ \isacommand{have}\isamarkupfalse%
\ {\isachardoublequoteopen}{\isasymexists}{\isasymA}{\isachardot}\ {\isasymforall}F\ {\isasymin}\ {\isacharparenleft}{\isacharbraceleft}G{\isacharcomma}H{\isacharcomma}F{\isacharbraceright}\ {\isasymunion}\ Wo{\isacharparenright}{\isachardot}\ {\isasymA}\ {\isasymTurnstile}\ F{\isachardoublequoteclose}\isanewline
\ \ \ \ \isacommand{by}\isamarkupfalse%
\ {\isacharparenleft}iprover\ intro{\isacharcolon}\ exI{\isacharparenright}\isanewline
\ \ \isacommand{thus}\isamarkupfalse%
\ {\isachardoublequoteopen}sat\ {\isacharparenleft}{\isacharbraceleft}G{\isacharcomma}H{\isacharcomma}F{\isacharbraceright}\ {\isasymunion}\ Wo{\isacharparenright}{\isachardoublequoteclose}\isanewline
\ \ \ \ \isacommand{by}\isamarkupfalse%
\ {\isacharparenleft}simp\ only{\isacharcolon}\ sat{\isacharunderscore}def{\isacharparenright}\isanewline
\isacommand{qed}\isamarkupfalse%
%
\endisatagproof
{\isafoldproof}%
%
\isadelimproof
%
\endisadelimproof
%
\begin{isamarkuptext}%
A continuación veamos la prueba detallada del resultado para el segundo caso de fórmula de 
  tipo \isa{{\isasymalpha}}: dados \isa{W\ {\isasymin}\ C}, una fórmula \isa{F\ {\isacharequal}\ {\isasymnot}{\isacharparenleft}G\ {\isasymor}\ H{\isacharparenright}} para ciertas fórmulas \isa{G} y \isa{H} tal que 
  \isa{F\ {\isasymin}\ W} y \isa{W\isactrlsub {\isadigit{0}}} un subconjunto finito de \isa{W}, se verifica que \isa{{\isacharbraceleft}{\isasymnot}\ G{\isacharcomma}{\isasymnot}\ H{\isacharcomma}F{\isacharbraceright}\ {\isasymunion}\ W\isactrlsub {\isadigit{0}}} es satisfacible.%
\end{isamarkuptext}\isamarkuptrue%
\isacommand{lemma}\isamarkupfalse%
\ pcp{\isacharunderscore}colecComp{\isacharunderscore}CON{\isacharunderscore}sat{\isadigit{2}}{\isacharcolon}\isanewline
\ \ \isakeyword{assumes}\ {\isachardoublequoteopen}W\ {\isasymin}\ colecComp{\isachardoublequoteclose}\isanewline
\ \ \ \ \ \ \ \ \ \ {\isachardoublequoteopen}F\ {\isacharequal}\ \isactrlbold {\isasymnot}{\isacharparenleft}G\ \isactrlbold {\isasymor}\ H{\isacharparenright}{\isachardoublequoteclose}\isanewline
\ \ \ \ \ \ \ \ \ \ {\isachardoublequoteopen}F\ {\isasymin}\ W{\isachardoublequoteclose}\isanewline
\ \ \ \ \ \ \ \ \ \ {\isachardoublequoteopen}finite\ Wo{\isachardoublequoteclose}\isanewline
\ \ \ \ \ \ \ \ \ \ {\isachardoublequoteopen}Wo\ {\isasymsubseteq}\ W{\isachardoublequoteclose}\isanewline
\ \ \ \ \ \ \ \ \isakeyword{shows}\ {\isachardoublequoteopen}sat\ {\isacharparenleft}{\isacharbraceleft}\isactrlbold {\isasymnot}\ G{\isacharcomma}\isactrlbold {\isasymnot}\ H{\isacharcomma}F{\isacharbraceright}\ {\isasymunion}\ Wo{\isacharparenright}{\isachardoublequoteclose}\isanewline
%
\isadelimproof
%
\endisadelimproof
%
\isatagproof
\isacommand{proof}\isamarkupfalse%
\ {\isacharminus}\isanewline
\ \ \isacommand{have}\isamarkupfalse%
\ {\isachardoublequoteopen}sat\ {\isacharparenleft}{\isacharbraceleft}F{\isacharbraceright}\ {\isasymunion}\ Wo{\isacharparenright}{\isachardoublequoteclose}\isanewline
\ \ \ \ \isacommand{using}\isamarkupfalse%
\ assms{\isacharparenleft}{\isadigit{1}}{\isacharcomma}{\isadigit{3}}{\isacharcomma}{\isadigit{4}}{\isacharcomma}{\isadigit{5}}{\isacharparenright}\ \isacommand{by}\isamarkupfalse%
\ {\isacharparenleft}rule\ pcp{\isacharunderscore}colecComp{\isacharunderscore}elem{\isacharunderscore}sat{\isacharparenright}\isanewline
\ \ \isacommand{have}\isamarkupfalse%
\ {\isachardoublequoteopen}F\ {\isasymin}\ {\isacharbraceleft}F{\isacharbraceright}\ {\isasymunion}\ Wo{\isachardoublequoteclose}\isanewline
\ \ \ \ \isacommand{by}\isamarkupfalse%
\ {\isacharparenleft}simp\ add{\isacharcolon}\ insertI{\isadigit{1}}{\isacharparenright}\isanewline
\ \ \isacommand{have}\isamarkupfalse%
\ Ex{\isadigit{1}}{\isacharcolon}{\isachardoublequoteopen}{\isasymexists}{\isasymA}{\isachardot}\ {\isasymforall}F\ {\isasymin}\ {\isacharparenleft}{\isacharbraceleft}F{\isacharbraceright}\ {\isasymunion}\ Wo{\isacharparenright}{\isachardot}\ {\isasymA}\ {\isasymTurnstile}\ F{\isachardoublequoteclose}\isanewline
\ \ \ \ \isacommand{using}\isamarkupfalse%
\ {\isacartoucheopen}sat\ {\isacharparenleft}{\isacharbraceleft}F{\isacharbraceright}\ {\isasymunion}\ Wo{\isacharparenright}{\isacartoucheclose}\ \isacommand{by}\isamarkupfalse%
\ {\isacharparenleft}simp\ only{\isacharcolon}\ sat{\isacharunderscore}def{\isacharparenright}\isanewline
\ \ \isacommand{obtain}\isamarkupfalse%
\ {\isasymA}\ \isakeyword{where}\ Forall{\isadigit{1}}{\isacharcolon}{\isachardoublequoteopen}{\isasymforall}F\ {\isasymin}\ {\isacharparenleft}{\isacharbraceleft}F{\isacharbraceright}\ {\isasymunion}\ Wo{\isacharparenright}{\isachardot}\ {\isasymA}\ {\isasymTurnstile}\ F{\isachardoublequoteclose}\isanewline
\ \ \ \ \isacommand{using}\isamarkupfalse%
\ Ex{\isadigit{1}}\ \isacommand{by}\isamarkupfalse%
\ {\isacharparenleft}rule\ exE{\isacharparenright}\isanewline
\ \ \isacommand{have}\isamarkupfalse%
\ {\isachardoublequoteopen}{\isasymA}\ {\isasymTurnstile}\ F{\isachardoublequoteclose}\isanewline
\ \ \ \ \isacommand{using}\isamarkupfalse%
\ Forall{\isadigit{1}}\ {\isacartoucheopen}F\ {\isasymin}\ {\isacharbraceleft}F{\isacharbraceright}\ {\isasymunion}\ Wo{\isacartoucheclose}\ \isacommand{by}\isamarkupfalse%
\ {\isacharparenleft}rule\ bspec{\isacharparenright}\isanewline
\ \ \isacommand{then}\isamarkupfalse%
\ \isacommand{have}\isamarkupfalse%
\ {\isachardoublequoteopen}{\isasymA}\ {\isasymTurnstile}\ \isactrlbold {\isasymnot}{\isacharparenleft}G\ \isactrlbold {\isasymor}\ H{\isacharparenright}{\isachardoublequoteclose}\isanewline
\ \ \ \ \isacommand{using}\isamarkupfalse%
\ assms{\isacharparenleft}{\isadigit{2}}{\isacharparenright}\ \isacommand{by}\isamarkupfalse%
\ {\isacharparenleft}simp\ only{\isacharcolon}\ {\isacartoucheopen}{\isasymA}\ {\isasymTurnstile}\ F{\isacartoucheclose}{\isacharparenright}\isanewline
\ \ \isacommand{then}\isamarkupfalse%
\ \isacommand{have}\isamarkupfalse%
\ {\isachardoublequoteopen}{\isasymnot}{\isacharparenleft}{\isasymA}\ {\isasymTurnstile}\ {\isacharparenleft}G\ \isactrlbold {\isasymor}\ H{\isacharparenright}{\isacharparenright}{\isachardoublequoteclose}\isanewline
\ \ \ \ \isacommand{by}\isamarkupfalse%
\ {\isacharparenleft}simp\ only{\isacharcolon}\ formula{\isacharunderscore}semantics{\isachardot}simps{\isacharparenleft}{\isadigit{3}}{\isacharparenright}\ simp{\isacharunderscore}thms{\isacharparenleft}{\isadigit{8}}{\isacharparenright}{\isacharparenright}\isanewline
\ \ \isacommand{then}\isamarkupfalse%
\ \isacommand{have}\isamarkupfalse%
\ {\isachardoublequoteopen}{\isasymnot}{\isacharparenleft}{\isasymA}\ {\isasymTurnstile}\ G\ {\isasymor}\ {\isasymA}\ {\isasymTurnstile}\ H{\isacharparenright}{\isachardoublequoteclose}\isanewline
\ \ \ \ \isacommand{by}\isamarkupfalse%
\ {\isacharparenleft}simp\ only{\isacharcolon}\ formula{\isacharunderscore}semantics{\isachardot}simps{\isacharparenleft}{\isadigit{5}}{\isacharparenright}\ simp{\isacharunderscore}thms{\isacharparenleft}{\isadigit{8}}{\isacharparenright}{\isacharparenright}\isanewline
\ \ \isacommand{then}\isamarkupfalse%
\ \isacommand{have}\isamarkupfalse%
\ {\isachardoublequoteopen}{\isasymnot}\ {\isasymA}\ {\isasymTurnstile}\ G\ {\isasymand}\ {\isasymnot}\ {\isasymA}\ {\isasymTurnstile}\ H{\isachardoublequoteclose}\ \isanewline
\ \ \ \ \isacommand{by}\isamarkupfalse%
\ {\isacharparenleft}simp\ only{\isacharcolon}\ de{\isacharunderscore}Morgan{\isacharunderscore}disj\ simp{\isacharunderscore}thms{\isacharparenleft}{\isadigit{8}}{\isacharparenright}{\isacharparenright}\isanewline
\ \ \isacommand{then}\isamarkupfalse%
\ \isacommand{have}\isamarkupfalse%
\ {\isachardoublequoteopen}{\isasymA}\ {\isasymTurnstile}\ \isactrlbold {\isasymnot}\ G\ {\isasymand}\ {\isasymA}\ {\isasymTurnstile}\ \isactrlbold {\isasymnot}\ H{\isachardoublequoteclose}\isanewline
\ \ \ \ \isacommand{by}\isamarkupfalse%
\ {\isacharparenleft}simp\ only{\isacharcolon}\ formula{\isacharunderscore}semantics{\isachardot}simps{\isacharparenleft}{\isadigit{3}}{\isacharparenright}\ simp{\isacharunderscore}thms{\isacharparenleft}{\isadigit{8}}{\isacharparenright}{\isacharparenright}\ \isanewline
\ \ \isacommand{then}\isamarkupfalse%
\ \isacommand{have}\isamarkupfalse%
\ {\isachardoublequoteopen}{\isasymA}\ {\isasymTurnstile}\ \isactrlbold {\isasymnot}\ G{\isachardoublequoteclose}\isanewline
\ \ \ \ \isacommand{by}\isamarkupfalse%
\ {\isacharparenleft}rule\ conjunct{\isadigit{1}}{\isacharparenright}\isanewline
\ \ \isacommand{then}\isamarkupfalse%
\ \isacommand{have}\isamarkupfalse%
\ {\isadigit{1}}{\isacharcolon}{\isachardoublequoteopen}{\isasymforall}F\ {\isasymin}\ {\isacharbraceleft}\isactrlbold {\isasymnot}\ G{\isacharbraceright}{\isachardot}\ {\isasymA}\ {\isasymTurnstile}\ F{\isachardoublequoteclose}\isanewline
\ \ \ \ \isacommand{by}\isamarkupfalse%
\ simp\isanewline
\ \ \isacommand{have}\isamarkupfalse%
\ {\isachardoublequoteopen}{\isasymA}\ {\isasymTurnstile}\ \isactrlbold {\isasymnot}\ H{\isachardoublequoteclose}\isanewline
\ \ \ \ \isacommand{using}\isamarkupfalse%
\ {\isacartoucheopen}{\isasymA}\ {\isasymTurnstile}\ \isactrlbold {\isasymnot}\ G\ {\isasymand}\ {\isasymA}\ {\isasymTurnstile}\ \isactrlbold {\isasymnot}\ H{\isacartoucheclose}\ \isacommand{by}\isamarkupfalse%
\ {\isacharparenleft}rule\ conjunct{\isadigit{2}}{\isacharparenright}\isanewline
\ \ \isacommand{then}\isamarkupfalse%
\ \isacommand{have}\isamarkupfalse%
\ {\isadigit{2}}{\isacharcolon}{\isachardoublequoteopen}{\isasymforall}F\ {\isasymin}\ {\isacharbraceleft}\isactrlbold {\isasymnot}\ H{\isacharbraceright}{\isachardot}\ {\isasymA}\ {\isasymTurnstile}\ F{\isachardoublequoteclose}\isanewline
\ \ \ \ \isacommand{by}\isamarkupfalse%
\ simp\isanewline
\ \ \isacommand{have}\isamarkupfalse%
\ {\isachardoublequoteopen}{\isasymforall}F\ {\isasymin}\ {\isacharparenleft}{\isacharbraceleft}\isactrlbold {\isasymnot}\ G{\isacharbraceright}\ {\isasymunion}\ {\isacharbraceleft}\isactrlbold {\isasymnot}\ H{\isacharbraceright}{\isacharparenright}\ {\isasymunion}\ {\isacharparenleft}{\isacharbraceleft}F{\isacharbraceright}\ {\isasymunion}\ Wo{\isacharparenright}{\isachardot}\ {\isasymA}\ {\isasymTurnstile}\ F{\isachardoublequoteclose}\isanewline
\ \ \ \ \isacommand{using}\isamarkupfalse%
\ Forall{\isadigit{1}}\ {\isadigit{1}}\ {\isadigit{2}}\ \isacommand{by}\isamarkupfalse%
\ {\isacharparenleft}iprover\ intro{\isacharcolon}\ ball{\isacharunderscore}Un{\isacharparenright}\isanewline
\ \ \isacommand{then}\isamarkupfalse%
\ \isacommand{have}\isamarkupfalse%
\ {\isachardoublequoteopen}{\isasymforall}F\ {\isasymin}\ {\isacharparenleft}{\isacharbraceleft}\isactrlbold {\isasymnot}\ G{\isacharcomma}\isactrlbold {\isasymnot}\ H{\isacharcomma}\ F{\isacharbraceright}\ {\isasymunion}\ Wo{\isacharparenright}{\isachardot}\ {\isasymA}\ {\isasymTurnstile}\ F{\isachardoublequoteclose}\isanewline
\ \ \ \ \isacommand{by}\isamarkupfalse%
\ simp\isanewline
\ \ \isacommand{then}\isamarkupfalse%
\ \isacommand{have}\isamarkupfalse%
\ {\isachardoublequoteopen}{\isasymexists}{\isasymA}{\isachardot}\ {\isasymforall}F\ {\isasymin}\ {\isacharparenleft}{\isacharbraceleft}\isactrlbold {\isasymnot}\ G{\isacharcomma}\isactrlbold {\isasymnot}\ H{\isacharcomma}F{\isacharbraceright}\ {\isasymunion}\ Wo{\isacharparenright}{\isachardot}\ {\isasymA}\ {\isasymTurnstile}\ F{\isachardoublequoteclose}\isanewline
\ \ \ \ \isacommand{by}\isamarkupfalse%
\ {\isacharparenleft}iprover\ intro{\isacharcolon}\ exI{\isacharparenright}\isanewline
\ \ \isacommand{thus}\isamarkupfalse%
\ {\isachardoublequoteopen}sat\ {\isacharparenleft}{\isacharbraceleft}\isactrlbold {\isasymnot}\ G{\isacharcomma}\isactrlbold {\isasymnot}\ H{\isacharcomma}F{\isacharbraceright}\ {\isasymunion}\ Wo{\isacharparenright}{\isachardoublequoteclose}\isanewline
\ \ \ \ \isacommand{by}\isamarkupfalse%
\ {\isacharparenleft}simp\ only{\isacharcolon}\ sat{\isacharunderscore}def{\isacharparenright}\isanewline
\isacommand{qed}\isamarkupfalse%
%
\endisatagproof
{\isafoldproof}%
%
\isadelimproof
%
\endisadelimproof
%
\begin{isamarkuptext}%
Probemos detalladamente el resultado para el tercer caso de fórmula de tipo \isa{{\isasymalpha}}: dados 
  \isa{W\ {\isasymin}\ C}, una fórmula \isa{F\ {\isacharequal}\ {\isasymnot}{\isacharparenleft}G\ {\isasymlongrightarrow}\ H{\isacharparenright}} para ciertas fórmulas \isa{G} y \isa{H} tal que \isa{F\ {\isasymin}\ W} y \isa{W\isactrlsub {\isadigit{0}}} un 
  subconjunto finito de \isa{W}, se verifica que \isa{{\isacharbraceleft}G{\isacharcomma}{\isasymnot}\ H{\isacharcomma}F{\isacharbraceright}\ {\isasymunion}\ W\isactrlsub {\isadigit{0}}} es satisfacible.%
\end{isamarkuptext}\isamarkuptrue%
\isacommand{lemma}\isamarkupfalse%
\ pcp{\isacharunderscore}colecComp{\isacharunderscore}CON{\isacharunderscore}sat{\isadigit{3}}{\isacharcolon}\isanewline
\ \ \isakeyword{assumes}\ {\isachardoublequoteopen}W\ {\isasymin}\ colecComp{\isachardoublequoteclose}\isanewline
\ \ \ \ \ \ \ \ \ \ {\isachardoublequoteopen}F\ {\isacharequal}\ \isactrlbold {\isasymnot}\ {\isacharparenleft}G\ \isactrlbold {\isasymrightarrow}\ H{\isacharparenright}{\isachardoublequoteclose}\isanewline
\ \ \ \ \ \ \ \ \ \ {\isachardoublequoteopen}F\ {\isasymin}\ W{\isachardoublequoteclose}\isanewline
\ \ \ \ \ \ \ \ \ \ {\isachardoublequoteopen}finite\ Wo{\isachardoublequoteclose}\isanewline
\ \ \ \ \ \ \ \ \ \ {\isachardoublequoteopen}Wo\ {\isasymsubseteq}\ W{\isachardoublequoteclose}\isanewline
\ \ \ \ \ \ \ \ \isakeyword{shows}\ {\isachardoublequoteopen}sat\ {\isacharparenleft}{\isacharbraceleft}G{\isacharcomma}\isactrlbold {\isasymnot}\ H{\isacharcomma}F{\isacharbraceright}\ {\isasymunion}\ Wo{\isacharparenright}{\isachardoublequoteclose}\isanewline
%
\isadelimproof
%
\endisadelimproof
%
\isatagproof
\isacommand{proof}\isamarkupfalse%
\ {\isacharminus}\isanewline
\ \ \isacommand{have}\isamarkupfalse%
\ {\isachardoublequoteopen}sat\ {\isacharparenleft}{\isacharbraceleft}F{\isacharbraceright}\ {\isasymunion}\ Wo{\isacharparenright}{\isachardoublequoteclose}\isanewline
\ \ \ \ \isacommand{using}\isamarkupfalse%
\ assms{\isacharparenleft}{\isadigit{1}}{\isacharcomma}{\isadigit{3}}{\isacharcomma}{\isadigit{4}}{\isacharcomma}{\isadigit{5}}{\isacharparenright}\ \isacommand{by}\isamarkupfalse%
\ {\isacharparenleft}rule\ pcp{\isacharunderscore}colecComp{\isacharunderscore}elem{\isacharunderscore}sat{\isacharparenright}\isanewline
\ \ \isacommand{have}\isamarkupfalse%
\ {\isachardoublequoteopen}F\ {\isasymin}\ {\isacharbraceleft}F{\isacharbraceright}\ {\isasymunion}\ Wo{\isachardoublequoteclose}\isanewline
\ \ \ \ \isacommand{by}\isamarkupfalse%
\ {\isacharparenleft}simp\ add{\isacharcolon}\ insertI{\isadigit{1}}{\isacharparenright}\isanewline
\ \ \isacommand{have}\isamarkupfalse%
\ Ex{\isadigit{1}}{\isacharcolon}{\isachardoublequoteopen}{\isasymexists}{\isasymA}{\isachardot}\ {\isasymforall}F\ {\isasymin}\ {\isacharparenleft}{\isacharbraceleft}F{\isacharbraceright}\ {\isasymunion}\ Wo{\isacharparenright}{\isachardot}\ {\isasymA}\ {\isasymTurnstile}\ F{\isachardoublequoteclose}\isanewline
\ \ \ \ \isacommand{using}\isamarkupfalse%
\ {\isacartoucheopen}sat\ {\isacharparenleft}{\isacharbraceleft}F{\isacharbraceright}\ {\isasymunion}\ Wo{\isacharparenright}{\isacartoucheclose}\ \isacommand{by}\isamarkupfalse%
\ {\isacharparenleft}simp\ only{\isacharcolon}\ sat{\isacharunderscore}def{\isacharparenright}\isanewline
\ \ \isacommand{obtain}\isamarkupfalse%
\ {\isasymA}\ \isakeyword{where}\ Forall{\isadigit{1}}{\isacharcolon}{\isachardoublequoteopen}{\isasymforall}F\ {\isasymin}\ {\isacharparenleft}{\isacharbraceleft}F{\isacharbraceright}\ {\isasymunion}\ Wo{\isacharparenright}{\isachardot}\ {\isasymA}\ {\isasymTurnstile}\ F{\isachardoublequoteclose}\isanewline
\ \ \ \ \isacommand{using}\isamarkupfalse%
\ Ex{\isadigit{1}}\ \isacommand{by}\isamarkupfalse%
\ {\isacharparenleft}rule\ exE{\isacharparenright}\isanewline
\ \ \isacommand{have}\isamarkupfalse%
\ {\isachardoublequoteopen}{\isasymA}\ {\isasymTurnstile}\ F{\isachardoublequoteclose}\isanewline
\ \ \ \ \isacommand{using}\isamarkupfalse%
\ Forall{\isadigit{1}}\ {\isacartoucheopen}F\ {\isasymin}\ {\isacharbraceleft}F{\isacharbraceright}\ {\isasymunion}\ Wo{\isacartoucheclose}\ \isacommand{by}\isamarkupfalse%
\ {\isacharparenleft}rule\ bspec{\isacharparenright}\isanewline
\ \ \isacommand{then}\isamarkupfalse%
\ \isacommand{have}\isamarkupfalse%
\ {\isachardoublequoteopen}{\isasymA}\ {\isasymTurnstile}\ \isactrlbold {\isasymnot}{\isacharparenleft}G\ \isactrlbold {\isasymrightarrow}\ H{\isacharparenright}{\isachardoublequoteclose}\isanewline
\ \ \ \ \isacommand{using}\isamarkupfalse%
\ assms{\isacharparenleft}{\isadigit{2}}{\isacharparenright}\ \isacommand{by}\isamarkupfalse%
\ {\isacharparenleft}simp\ only{\isacharcolon}\ {\isacartoucheopen}{\isasymA}\ {\isasymTurnstile}\ F{\isacartoucheclose}{\isacharparenright}\isanewline
\ \ \isacommand{then}\isamarkupfalse%
\ \isacommand{have}\isamarkupfalse%
\ {\isachardoublequoteopen}{\isasymnot}{\isacharparenleft}{\isasymA}\ {\isasymTurnstile}\ {\isacharparenleft}G\ \isactrlbold {\isasymrightarrow}\ H{\isacharparenright}{\isacharparenright}{\isachardoublequoteclose}\isanewline
\ \ \ \ \isacommand{by}\isamarkupfalse%
\ {\isacharparenleft}simp\ only{\isacharcolon}\ formula{\isacharunderscore}semantics{\isachardot}simps{\isacharparenleft}{\isadigit{3}}{\isacharparenright}\ simp{\isacharunderscore}thms{\isacharparenleft}{\isadigit{8}}{\isacharparenright}{\isacharparenright}\isanewline
\ \ \isacommand{then}\isamarkupfalse%
\ \isacommand{have}\isamarkupfalse%
\ {\isachardoublequoteopen}{\isasymnot}{\isacharparenleft}{\isasymA}\ {\isasymTurnstile}\ G\ {\isasymlongrightarrow}\ {\isasymA}\ {\isasymTurnstile}\ H{\isacharparenright}{\isachardoublequoteclose}\isanewline
\ \ \ \ \isacommand{by}\isamarkupfalse%
\ {\isacharparenleft}simp\ only{\isacharcolon}\ formula{\isacharunderscore}semantics{\isachardot}simps{\isacharparenleft}{\isadigit{6}}{\isacharparenright}\ simp{\isacharunderscore}thms{\isacharparenleft}{\isadigit{8}}{\isacharparenright}{\isacharparenright}\isanewline
\ \ \isacommand{then}\isamarkupfalse%
\ \isacommand{have}\isamarkupfalse%
\ {\isachardoublequoteopen}{\isasymA}\ {\isasymTurnstile}\ G\ {\isasymand}\ {\isasymnot}\ {\isasymA}\ {\isasymTurnstile}\ H{\isachardoublequoteclose}\isanewline
\ \ \ \ \isacommand{by}\isamarkupfalse%
\ {\isacharparenleft}simp\ only{\isacharcolon}\ not{\isacharunderscore}imp\ simp{\isacharunderscore}thms{\isacharparenleft}{\isadigit{8}}{\isacharparenright}{\isacharparenright}\isanewline
\ \ \isacommand{then}\isamarkupfalse%
\ \isacommand{have}\isamarkupfalse%
\ {\isachardoublequoteopen}{\isasymA}\ {\isasymTurnstile}\ G\ {\isasymand}\ {\isasymA}\ {\isasymTurnstile}\ \isactrlbold {\isasymnot}\ H{\isachardoublequoteclose}\isanewline
\ \ \ \ \isacommand{by}\isamarkupfalse%
\ {\isacharparenleft}simp\ only{\isacharcolon}\ formula{\isacharunderscore}semantics{\isachardot}simps{\isacharparenleft}{\isadigit{3}}{\isacharparenright}\ simp{\isacharunderscore}thms{\isacharparenleft}{\isadigit{8}}{\isacharparenright}{\isacharparenright}\ \isanewline
\ \ \isacommand{then}\isamarkupfalse%
\ \isacommand{have}\isamarkupfalse%
\ {\isachardoublequoteopen}{\isasymA}\ {\isasymTurnstile}\ G{\isachardoublequoteclose}\isanewline
\ \ \ \ \isacommand{by}\isamarkupfalse%
\ {\isacharparenleft}rule\ conjunct{\isadigit{1}}{\isacharparenright}\isanewline
\ \ \isacommand{then}\isamarkupfalse%
\ \isacommand{have}\isamarkupfalse%
\ {\isadigit{1}}{\isacharcolon}{\isachardoublequoteopen}{\isasymforall}F\ {\isasymin}\ {\isacharbraceleft}G{\isacharbraceright}{\isachardot}\ {\isasymA}\ {\isasymTurnstile}\ F{\isachardoublequoteclose}\isanewline
\ \ \ \ \isacommand{by}\isamarkupfalse%
\ simp\isanewline
\ \ \isacommand{have}\isamarkupfalse%
\ {\isachardoublequoteopen}{\isasymA}\ {\isasymTurnstile}\ \isactrlbold {\isasymnot}\ H{\isachardoublequoteclose}\isanewline
\ \ \ \ \isacommand{using}\isamarkupfalse%
\ {\isacartoucheopen}{\isasymA}\ {\isasymTurnstile}\ G\ {\isasymand}\ {\isasymA}\ {\isasymTurnstile}\ \isactrlbold {\isasymnot}\ H{\isacartoucheclose}\ \isacommand{by}\isamarkupfalse%
\ {\isacharparenleft}rule\ conjunct{\isadigit{2}}{\isacharparenright}\isanewline
\ \ \isacommand{then}\isamarkupfalse%
\ \isacommand{have}\isamarkupfalse%
\ {\isadigit{2}}{\isacharcolon}{\isachardoublequoteopen}{\isasymforall}F\ {\isasymin}\ {\isacharbraceleft}\isactrlbold {\isasymnot}\ H{\isacharbraceright}{\isachardot}\ {\isasymA}\ {\isasymTurnstile}\ F{\isachardoublequoteclose}\isanewline
\ \ \ \ \isacommand{by}\isamarkupfalse%
\ simp\isanewline
\ \ \isacommand{have}\isamarkupfalse%
\ {\isachardoublequoteopen}{\isasymforall}F\ {\isasymin}\ {\isacharparenleft}{\isacharbraceleft}G{\isacharbraceright}\ {\isasymunion}\ {\isacharbraceleft}\isactrlbold {\isasymnot}\ H{\isacharbraceright}{\isacharparenright}\ {\isasymunion}\ {\isacharparenleft}{\isacharbraceleft}F{\isacharbraceright}\ {\isasymunion}\ Wo{\isacharparenright}{\isachardot}\ {\isasymA}\ {\isasymTurnstile}\ F{\isachardoublequoteclose}\isanewline
\ \ \ \ \isacommand{using}\isamarkupfalse%
\ Forall{\isadigit{1}}\ {\isadigit{1}}\ {\isadigit{2}}\ \isacommand{by}\isamarkupfalse%
\ {\isacharparenleft}iprover\ intro{\isacharcolon}\ ball{\isacharunderscore}Un{\isacharparenright}\isanewline
\ \ \isacommand{then}\isamarkupfalse%
\ \isacommand{have}\isamarkupfalse%
\ {\isachardoublequoteopen}{\isasymforall}F\ {\isasymin}\ {\isacharbraceleft}G{\isacharcomma}\isactrlbold {\isasymnot}\ H{\isacharcomma}F{\isacharbraceright}\ {\isasymunion}\ Wo{\isachardot}\ {\isasymA}\ {\isasymTurnstile}\ F{\isachardoublequoteclose}\isanewline
\ \ \ \ \isacommand{by}\isamarkupfalse%
\ simp\isanewline
\ \ \isacommand{then}\isamarkupfalse%
\ \isacommand{have}\isamarkupfalse%
\ {\isachardoublequoteopen}{\isasymexists}{\isasymA}{\isachardot}\ {\isasymforall}F\ {\isasymin}\ {\isacharparenleft}{\isacharbraceleft}G{\isacharcomma}\isactrlbold {\isasymnot}\ H{\isacharcomma}F{\isacharbraceright}\ {\isasymunion}\ Wo{\isacharparenright}{\isachardot}\ {\isasymA}\ {\isasymTurnstile}\ F{\isachardoublequoteclose}\isanewline
\ \ \ \ \isacommand{by}\isamarkupfalse%
\ {\isacharparenleft}iprover\ intro{\isacharcolon}\ exI{\isacharparenright}\isanewline
\ \ \isacommand{thus}\isamarkupfalse%
\ {\isachardoublequoteopen}sat\ {\isacharparenleft}{\isacharbraceleft}G{\isacharcomma}\isactrlbold {\isasymnot}\ H{\isacharcomma}F{\isacharbraceright}\ {\isasymunion}\ Wo{\isacharparenright}{\isachardoublequoteclose}\isanewline
\ \ \ \ \isacommand{by}\isamarkupfalse%
\ {\isacharparenleft}simp\ only{\isacharcolon}\ sat{\isacharunderscore}def{\isacharparenright}\isanewline
\isacommand{qed}\isamarkupfalse%
%
\endisatagproof
{\isafoldproof}%
%
\isadelimproof
%
\endisadelimproof
%
\begin{isamarkuptext}%
Por último probemos que dados \isa{W\ {\isasymin}\ C}, una fórmula \isa{F\ {\isacharequal}\ {\isasymnot}{\isacharparenleft}{\isasymnot}\ G{\isacharparenright}} para cierta fórmula \isa{G} tal 
  que \isa{F\ {\isasymin}\ W} y \isa{W\isactrlsub {\isadigit{0}}} un subconjunto finito de \isa{W}, se verifica que \isa{{\isacharbraceleft}G{\isacharcomma}F{\isacharbraceright}\ {\isasymunion}\ W\isactrlsub {\isadigit{0}}} es satisfacible.%
\end{isamarkuptext}\isamarkuptrue%
\isacommand{lemma}\isamarkupfalse%
\ pcp{\isacharunderscore}colecComp{\isacharunderscore}CON{\isacharunderscore}sat{\isadigit{4}}{\isacharcolon}\isanewline
\ \ \isakeyword{assumes}\ {\isachardoublequoteopen}W\ {\isasymin}\ colecComp{\isachardoublequoteclose}\isanewline
\ \ \ \ \ \ \ \ \ \ {\isachardoublequoteopen}F\ {\isacharequal}\ \isactrlbold {\isasymnot}\ {\isacharparenleft}\isactrlbold {\isasymnot}\ G{\isacharparenright}{\isachardoublequoteclose}\isanewline
\ \ \ \ \ \ \ \ \ \ {\isachardoublequoteopen}F\ {\isasymin}\ W{\isachardoublequoteclose}\isanewline
\ \ \ \ \ \ \ \ \ \ {\isachardoublequoteopen}finite\ Wo{\isachardoublequoteclose}\isanewline
\ \ \ \ \ \ \ \ \ \ {\isachardoublequoteopen}Wo\ {\isasymsubseteq}\ W{\isachardoublequoteclose}\isanewline
\ \ \ \ \ \ \ \ \isakeyword{shows}\ {\isachardoublequoteopen}sat\ {\isacharparenleft}{\isacharbraceleft}G{\isacharcomma}F{\isacharbraceright}\ {\isasymunion}\ Wo{\isacharparenright}{\isachardoublequoteclose}\isanewline
%
\isadelimproof
%
\endisadelimproof
%
\isatagproof
\isacommand{proof}\isamarkupfalse%
\ {\isacharminus}\isanewline
\ \ \isacommand{have}\isamarkupfalse%
\ {\isachardoublequoteopen}sat\ {\isacharparenleft}{\isacharbraceleft}F{\isacharbraceright}\ {\isasymunion}\ Wo{\isacharparenright}{\isachardoublequoteclose}\isanewline
\ \ \ \ \isacommand{using}\isamarkupfalse%
\ assms{\isacharparenleft}{\isadigit{1}}{\isacharcomma}{\isadigit{3}}{\isacharcomma}{\isadigit{4}}{\isacharcomma}{\isadigit{5}}{\isacharparenright}\ \isacommand{by}\isamarkupfalse%
\ {\isacharparenleft}rule\ pcp{\isacharunderscore}colecComp{\isacharunderscore}elem{\isacharunderscore}sat{\isacharparenright}\isanewline
\ \ \isacommand{have}\isamarkupfalse%
\ {\isachardoublequoteopen}F\ {\isasymin}\ {\isacharbraceleft}F{\isacharbraceright}\ {\isasymunion}\ Wo{\isachardoublequoteclose}\isanewline
\ \ \ \ \isacommand{by}\isamarkupfalse%
\ {\isacharparenleft}simp\ add{\isacharcolon}\ insertI{\isadigit{1}}{\isacharparenright}\isanewline
\ \ \isacommand{have}\isamarkupfalse%
\ Ex{\isadigit{1}}{\isacharcolon}{\isachardoublequoteopen}{\isasymexists}{\isasymA}{\isachardot}\ {\isasymforall}F\ {\isasymin}\ {\isacharparenleft}{\isacharbraceleft}F{\isacharbraceright}\ {\isasymunion}\ Wo{\isacharparenright}{\isachardot}\ {\isasymA}\ {\isasymTurnstile}\ F{\isachardoublequoteclose}\isanewline
\ \ \ \ \isacommand{using}\isamarkupfalse%
\ {\isacartoucheopen}sat\ {\isacharparenleft}{\isacharbraceleft}F{\isacharbraceright}\ {\isasymunion}\ Wo{\isacharparenright}{\isacartoucheclose}\ \isacommand{by}\isamarkupfalse%
\ {\isacharparenleft}simp\ only{\isacharcolon}\ sat{\isacharunderscore}def{\isacharparenright}\isanewline
\ \ \isacommand{obtain}\isamarkupfalse%
\ {\isasymA}\ \isakeyword{where}\ Forall{\isadigit{1}}{\isacharcolon}{\isachardoublequoteopen}{\isasymforall}F\ {\isasymin}\ {\isacharparenleft}{\isacharbraceleft}F{\isacharbraceright}\ {\isasymunion}\ Wo{\isacharparenright}{\isachardot}\ {\isasymA}\ {\isasymTurnstile}\ F{\isachardoublequoteclose}\isanewline
\ \ \ \ \isacommand{using}\isamarkupfalse%
\ Ex{\isadigit{1}}\ \isacommand{by}\isamarkupfalse%
\ {\isacharparenleft}rule\ exE{\isacharparenright}\isanewline
\ \ \isacommand{have}\isamarkupfalse%
\ {\isachardoublequoteopen}{\isasymA}\ {\isasymTurnstile}\ F{\isachardoublequoteclose}\isanewline
\ \ \ \ \isacommand{using}\isamarkupfalse%
\ Forall{\isadigit{1}}\ {\isacartoucheopen}F\ {\isasymin}\ {\isacharbraceleft}F{\isacharbraceright}\ {\isasymunion}\ Wo{\isacartoucheclose}\ \isacommand{by}\isamarkupfalse%
\ {\isacharparenleft}rule\ bspec{\isacharparenright}\isanewline
\ \ \isacommand{then}\isamarkupfalse%
\ \isacommand{have}\isamarkupfalse%
\ {\isachardoublequoteopen}{\isasymA}\ {\isasymTurnstile}\ \isactrlbold {\isasymnot}{\isacharparenleft}\isactrlbold {\isasymnot}\ G{\isacharparenright}{\isachardoublequoteclose}\isanewline
\ \ \ \ \isacommand{using}\isamarkupfalse%
\ assms{\isacharparenleft}{\isadigit{2}}{\isacharparenright}\ \isacommand{by}\isamarkupfalse%
\ {\isacharparenleft}simp\ only{\isacharcolon}\ {\isacartoucheopen}{\isasymA}\ {\isasymTurnstile}\ F{\isacartoucheclose}{\isacharparenright}\isanewline
\ \ \isacommand{then}\isamarkupfalse%
\ \isacommand{have}\isamarkupfalse%
\ {\isachardoublequoteopen}{\isasymnot}\ {\isasymA}\ {\isasymTurnstile}\ \isactrlbold {\isasymnot}\ G{\isachardoublequoteclose}\isanewline
\ \ \ \ \isacommand{by}\isamarkupfalse%
\ {\isacharparenleft}simp\ only{\isacharcolon}\ formula{\isacharunderscore}semantics{\isachardot}simps{\isacharparenleft}{\isadigit{3}}{\isacharparenright}\ simp{\isacharunderscore}thms{\isacharparenleft}{\isadigit{8}}{\isacharparenright}{\isacharparenright}\isanewline
\ \ \isacommand{then}\isamarkupfalse%
\ \isacommand{have}\isamarkupfalse%
\ {\isachardoublequoteopen}{\isasymnot}\ {\isasymnot}{\isasymA}\ {\isasymTurnstile}\ G{\isachardoublequoteclose}\isanewline
\ \ \ \ \isacommand{by}\isamarkupfalse%
\ {\isacharparenleft}simp\ only{\isacharcolon}\ formula{\isacharunderscore}semantics{\isachardot}simps{\isacharparenleft}{\isadigit{3}}{\isacharparenright}\ simp{\isacharunderscore}thms{\isacharparenleft}{\isadigit{8}}{\isacharparenright}{\isacharparenright}\isanewline
\ \ \isacommand{then}\isamarkupfalse%
\ \isacommand{have}\isamarkupfalse%
\ {\isachardoublequoteopen}{\isasymA}\ {\isasymTurnstile}\ G{\isachardoublequoteclose}\isanewline
\ \ \ \ \isacommand{by}\isamarkupfalse%
\ {\isacharparenleft}rule\ notnotD{\isacharparenright}\isanewline
\ \ \isacommand{then}\isamarkupfalse%
\ \isacommand{have}\isamarkupfalse%
\ {\isadigit{1}}{\isacharcolon}{\isachardoublequoteopen}{\isasymforall}F\ {\isasymin}\ {\isacharbraceleft}G{\isacharbraceright}{\isachardot}\ {\isasymA}\ {\isasymTurnstile}\ F{\isachardoublequoteclose}\isanewline
\ \ \ \ \isacommand{by}\isamarkupfalse%
\ simp\isanewline
\ \ \isacommand{have}\isamarkupfalse%
\ {\isachardoublequoteopen}{\isasymforall}F\ {\isasymin}\ {\isacharparenleft}{\isacharbraceleft}G{\isacharbraceright}{\isacharparenright}\ {\isasymunion}\ {\isacharparenleft}{\isacharbraceleft}F{\isacharbraceright}\ {\isasymunion}\ Wo{\isacharparenright}{\isachardot}\ {\isasymA}\ {\isasymTurnstile}\ F{\isachardoublequoteclose}\isanewline
\ \ \ \ \isacommand{using}\isamarkupfalse%
\ Forall{\isadigit{1}}\ {\isadigit{1}}\ \isacommand{by}\isamarkupfalse%
\ {\isacharparenleft}iprover\ intro{\isacharcolon}\ ball{\isacharunderscore}Un{\isacharparenright}\isanewline
\ \ \isacommand{then}\isamarkupfalse%
\ \isacommand{have}\isamarkupfalse%
\ {\isachardoublequoteopen}{\isasymforall}F\ {\isasymin}\ {\isacharbraceleft}G{\isacharcomma}F{\isacharbraceright}\ {\isasymunion}\ Wo{\isachardot}\ {\isasymA}\ {\isasymTurnstile}\ F{\isachardoublequoteclose}\isanewline
\ \ \ \ \isacommand{by}\isamarkupfalse%
\ simp\isanewline
\ \ \isacommand{then}\isamarkupfalse%
\ \isacommand{have}\isamarkupfalse%
\ {\isachardoublequoteopen}{\isasymexists}{\isasymA}{\isachardot}\ {\isasymforall}F\ {\isasymin}\ {\isacharparenleft}{\isacharbraceleft}G{\isacharcomma}F{\isacharbraceright}\ {\isasymunion}\ Wo{\isacharparenright}{\isachardot}\ {\isasymA}\ {\isasymTurnstile}\ F{\isachardoublequoteclose}\isanewline
\ \ \ \ \isacommand{by}\isamarkupfalse%
\ {\isacharparenleft}iprover\ intro{\isacharcolon}\ exI{\isacharparenright}\isanewline
\ \ \isacommand{thus}\isamarkupfalse%
\ {\isachardoublequoteopen}sat\ {\isacharparenleft}{\isacharbraceleft}G{\isacharcomma}F{\isacharbraceright}\ {\isasymunion}\ Wo{\isacharparenright}{\isachardoublequoteclose}\isanewline
\ \ \ \ \isacommand{by}\isamarkupfalse%
\ {\isacharparenleft}simp\ only{\isacharcolon}\ sat{\isacharunderscore}def{\isacharparenright}\isanewline
\isacommand{qed}\isamarkupfalse%
%
\endisatagproof
{\isafoldproof}%
%
\isadelimproof
%
\endisadelimproof
%
\begin{isamarkuptext}%
Por tanto, por las pruebas detalladas de los casos anteriores, podemos demostrar que dados 
  \isa{W\ {\isasymin}\ C}, \isa{F\ {\isasymin}\ W} una fórmula de tipo \isa{{\isasymalpha}} con componentes \isa{{\isasymalpha}\isactrlsub {\isadigit{1}}} y \isa{{\isasymalpha}\isactrlsub {\isadigit{2}}} y \isa{W\isactrlsub {\isadigit{0}}} un subconjunto finito 
  de \isa{W}, se verifica que \isa{{\isacharbraceleft}{\isasymalpha}\isactrlsub {\isadigit{1}}{\isacharcomma}{\isasymalpha}\isactrlsub {\isadigit{2}}{\isacharcomma}F{\isacharbraceright}\ {\isasymunion}\ W\isactrlsub {\isadigit{0}}} es satisfacible.%
\end{isamarkuptext}\isamarkuptrue%
\isacommand{lemma}\isamarkupfalse%
\ pcp{\isacharunderscore}colecComp{\isacharunderscore}CON{\isacharunderscore}sat{\isacharcolon}\isanewline
\ \ \isakeyword{assumes}\ {\isachardoublequoteopen}W\ {\isasymin}\ colecComp{\isachardoublequoteclose}\isanewline
\ \ \ \ \ \ \ \ \ \ {\isachardoublequoteopen}Con\ F\ G\ H{\isachardoublequoteclose}\isanewline
\ \ \ \ \ \ \ \ \ \ {\isachardoublequoteopen}F\ {\isasymin}\ W{\isachardoublequoteclose}\isanewline
\ \ \ \ \ \ \ \ \ \ {\isachardoublequoteopen}finite\ Wo{\isachardoublequoteclose}\isanewline
\ \ \ \ \ \ \ \ \ \ {\isachardoublequoteopen}Wo\ {\isasymsubseteq}\ W{\isachardoublequoteclose}\isanewline
\ \ \ \ \ \ \ \ \isakeyword{shows}\ {\isachardoublequoteopen}sat\ {\isacharparenleft}{\isacharbraceleft}G{\isacharcomma}H{\isacharcomma}F{\isacharbraceright}\ {\isasymunion}\ Wo{\isacharparenright}{\isachardoublequoteclose}\isanewline
%
\isadelimproof
%
\endisadelimproof
%
\isatagproof
\isacommand{proof}\isamarkupfalse%
\ {\isacharminus}\isanewline
\ \ \isacommand{have}\isamarkupfalse%
\ {\isachardoublequoteopen}{\isacharbraceleft}G{\isacharcomma}H{\isacharbraceright}\ {\isasymunion}\ Wo\ {\isasymsubseteq}\ {\isacharbraceleft}G{\isacharcomma}H{\isacharcomma}F{\isacharbraceright}\ {\isasymunion}\ Wo{\isachardoublequoteclose}\isanewline
\ \ \ \ \isacommand{by}\isamarkupfalse%
\ blast\isanewline
\ \ \isacommand{have}\isamarkupfalse%
\ {\isachardoublequoteopen}F\ {\isacharequal}\ G\ \isactrlbold {\isasymand}\ H\ {\isasymor}\ \isanewline
\ \ \ \ {\isacharparenleft}{\isasymexists}F{\isadigit{1}}\ G{\isadigit{1}}{\isachardot}\ F\ {\isacharequal}\ \isactrlbold {\isasymnot}\ {\isacharparenleft}F{\isadigit{1}}\ \isactrlbold {\isasymor}\ G{\isadigit{1}}{\isacharparenright}\ {\isasymand}\ G\ {\isacharequal}\ \isactrlbold {\isasymnot}\ F{\isadigit{1}}\ {\isasymand}\ H\ {\isacharequal}\ \isactrlbold {\isasymnot}\ G{\isadigit{1}}{\isacharparenright}\ {\isasymor}\ \isanewline
\ \ \ \ {\isacharparenleft}{\isasymexists}H{\isadigit{1}}{\isachardot}\ F\ {\isacharequal}\ \isactrlbold {\isasymnot}\ {\isacharparenleft}G\ \isactrlbold {\isasymrightarrow}\ H{\isadigit{1}}{\isacharparenright}\ {\isasymand}\ H\ {\isacharequal}\ \isactrlbold {\isasymnot}\ H{\isadigit{1}}{\isacharparenright}\ {\isasymor}\ \isanewline
\ \ \ \ F\ {\isacharequal}\ \isactrlbold {\isasymnot}\ {\isacharparenleft}\isactrlbold {\isasymnot}\ G{\isacharparenright}\ {\isasymand}\ H\ {\isacharequal}\ G{\isachardoublequoteclose}\isanewline
\ \ \ \ \isacommand{using}\isamarkupfalse%
\ assms{\isacharparenleft}{\isadigit{2}}{\isacharparenright}\ \isacommand{by}\isamarkupfalse%
\ {\isacharparenleft}simp\ only{\isacharcolon}\ con{\isacharunderscore}dis{\isacharunderscore}simps{\isacharparenleft}{\isadigit{1}}{\isacharparenright}{\isacharparenright}\isanewline
\ \ \isacommand{thus}\isamarkupfalse%
\ {\isachardoublequoteopen}sat\ {\isacharparenleft}{\isacharbraceleft}G{\isacharcomma}H{\isacharcomma}F{\isacharbraceright}\ {\isasymunion}\ Wo{\isacharparenright}{\isachardoublequoteclose}\isanewline
\ \ \isacommand{proof}\isamarkupfalse%
\ {\isacharparenleft}rule\ disjE{\isacharparenright}\isanewline
\ \ \ \ \isacommand{assume}\isamarkupfalse%
\ {\isachardoublequoteopen}F\ {\isacharequal}\ G\ \isactrlbold {\isasymand}\ H{\isachardoublequoteclose}\isanewline
\ \ \ \ \isacommand{show}\isamarkupfalse%
\ {\isachardoublequoteopen}sat\ {\isacharparenleft}{\isacharbraceleft}G{\isacharcomma}H{\isacharcomma}F{\isacharbraceright}\ {\isasymunion}\ Wo{\isacharparenright}{\isachardoublequoteclose}\isanewline
\ \ \ \ \ \ \isacommand{using}\isamarkupfalse%
\ assms{\isacharparenleft}{\isadigit{1}}{\isacharparenright}\ {\isacartoucheopen}F\ {\isacharequal}\ G\ \isactrlbold {\isasymand}\ H{\isacartoucheclose}\ assms{\isacharparenleft}{\isadigit{3}}{\isacharcomma}{\isadigit{4}}{\isacharcomma}{\isadigit{5}}{\isacharparenright}\ \isacommand{by}\isamarkupfalse%
\ {\isacharparenleft}rule\ pcp{\isacharunderscore}colecComp{\isacharunderscore}CON{\isacharunderscore}sat{\isadigit{1}}{\isacharparenright}\isanewline
\ \ \isacommand{next}\isamarkupfalse%
\isanewline
\ \ \ \ \isacommand{assume}\isamarkupfalse%
\ {\isachardoublequoteopen}{\isacharparenleft}{\isasymexists}F{\isadigit{1}}\ G{\isadigit{1}}{\isachardot}\ F\ {\isacharequal}\ \isactrlbold {\isasymnot}\ {\isacharparenleft}F{\isadigit{1}}\ \isactrlbold {\isasymor}\ G{\isadigit{1}}{\isacharparenright}\ {\isasymand}\ G\ {\isacharequal}\ \isactrlbold {\isasymnot}\ F{\isadigit{1}}\ {\isasymand}\ H\ {\isacharequal}\ \isactrlbold {\isasymnot}\ G{\isadigit{1}}{\isacharparenright}\ {\isasymor}\ \isanewline
\ \ \ \ {\isacharparenleft}{\isasymexists}H{\isadigit{1}}{\isachardot}\ F\ {\isacharequal}\ \isactrlbold {\isasymnot}\ {\isacharparenleft}G\ \isactrlbold {\isasymrightarrow}\ H{\isadigit{1}}{\isacharparenright}\ {\isasymand}\ H\ {\isacharequal}\ \isactrlbold {\isasymnot}\ H{\isadigit{1}}{\isacharparenright}\ {\isasymor}\ \isanewline
\ \ \ \ F\ {\isacharequal}\ \isactrlbold {\isasymnot}\ {\isacharparenleft}\isactrlbold {\isasymnot}\ G{\isacharparenright}\ {\isasymand}\ H\ {\isacharequal}\ G{\isachardoublequoteclose}\isanewline
\ \ \ \ \isacommand{thus}\isamarkupfalse%
\ {\isachardoublequoteopen}sat\ {\isacharparenleft}{\isacharbraceleft}G{\isacharcomma}H{\isacharcomma}F{\isacharbraceright}\ {\isasymunion}\ Wo{\isacharparenright}{\isachardoublequoteclose}\isanewline
\ \ \ \ \isacommand{proof}\isamarkupfalse%
\ {\isacharparenleft}rule\ disjE{\isacharparenright}\isanewline
\ \ \ \ \ \ \isacommand{assume}\isamarkupfalse%
\ Ex{\isadigit{2}}{\isacharcolon}{\isachardoublequoteopen}{\isasymexists}F{\isadigit{1}}\ G{\isadigit{1}}{\isachardot}\ F\ {\isacharequal}\ \isactrlbold {\isasymnot}\ {\isacharparenleft}F{\isadigit{1}}\ \isactrlbold {\isasymor}\ G{\isadigit{1}}{\isacharparenright}\ {\isasymand}\ G\ {\isacharequal}\ \isactrlbold {\isasymnot}\ F{\isadigit{1}}\ {\isasymand}\ H\ {\isacharequal}\ \isactrlbold {\isasymnot}\ G{\isadigit{1}}{\isachardoublequoteclose}\ \isanewline
\ \ \ \ \ \ \isacommand{obtain}\isamarkupfalse%
\ F{\isadigit{1}}\ G{\isadigit{1}}\ \isakeyword{where}\ {\isadigit{2}}{\isacharcolon}{\isachardoublequoteopen}F\ {\isacharequal}\ \isactrlbold {\isasymnot}{\isacharparenleft}F{\isadigit{1}}\ \isactrlbold {\isasymor}\ G{\isadigit{1}}{\isacharparenright}\ {\isasymand}\ G\ {\isacharequal}\ \isactrlbold {\isasymnot}\ F{\isadigit{1}}\ {\isasymand}\ H\ {\isacharequal}\ \isactrlbold {\isasymnot}\ G{\isadigit{1}}{\isachardoublequoteclose}\isanewline
\ \ \ \ \ \ \ \ \isacommand{using}\isamarkupfalse%
\ Ex{\isadigit{2}}\ \isacommand{by}\isamarkupfalse%
\ {\isacharparenleft}iprover\ elim{\isacharcolon}\ exE{\isacharparenright}\isanewline
\ \ \ \ \ \ \isacommand{have}\isamarkupfalse%
\ {\isachardoublequoteopen}G\ {\isacharequal}\ \isactrlbold {\isasymnot}\ F{\isadigit{1}}{\isachardoublequoteclose}\isanewline
\ \ \ \ \ \ \ \ \isacommand{using}\isamarkupfalse%
\ {\isadigit{2}}\ \isacommand{by}\isamarkupfalse%
\ {\isacharparenleft}iprover\ elim{\isacharcolon}\ conjunct{\isadigit{1}}{\isacharparenright}\isanewline
\ \ \ \ \ \ \isacommand{have}\isamarkupfalse%
\ {\isachardoublequoteopen}H\ {\isacharequal}\ \isactrlbold {\isasymnot}\ G{\isadigit{1}}{\isachardoublequoteclose}\isanewline
\ \ \ \ \ \ \ \ \isacommand{using}\isamarkupfalse%
\ {\isadigit{2}}\ \isacommand{by}\isamarkupfalse%
\ {\isacharparenleft}iprover\ elim{\isacharcolon}\ conjunct{\isadigit{2}}{\isacharparenright}\isanewline
\ \ \ \ \ \ \isacommand{have}\isamarkupfalse%
\ {\isachardoublequoteopen}F\ {\isacharequal}\ \isactrlbold {\isasymnot}{\isacharparenleft}F{\isadigit{1}}\ \isactrlbold {\isasymor}\ G{\isadigit{1}}{\isacharparenright}{\isachardoublequoteclose}\isanewline
\ \ \ \ \ \ \ \ \isacommand{using}\isamarkupfalse%
\ {\isadigit{2}}\ \isacommand{by}\isamarkupfalse%
\ {\isacharparenleft}rule\ conjunct{\isadigit{1}}{\isacharparenright}\isanewline
\ \ \ \ \ \ \isacommand{have}\isamarkupfalse%
\ {\isachardoublequoteopen}sat\ {\isacharparenleft}{\isacharbraceleft}\isactrlbold {\isasymnot}\ F{\isadigit{1}}{\isacharcomma}\ \isactrlbold {\isasymnot}\ G{\isadigit{1}}{\isacharcomma}\ F{\isacharbraceright}\ {\isasymunion}\ Wo{\isacharparenright}{\isachardoublequoteclose}\isanewline
\ \ \ \ \ \ \ \ \isacommand{using}\isamarkupfalse%
\ assms{\isacharparenleft}{\isadigit{1}}{\isacharparenright}\ {\isacartoucheopen}F\ {\isacharequal}\ \isactrlbold {\isasymnot}{\isacharparenleft}F{\isadigit{1}}\ \isactrlbold {\isasymor}\ G{\isadigit{1}}{\isacharparenright}{\isacartoucheclose}\ assms{\isacharparenleft}{\isadigit{3}}{\isacharcomma}{\isadigit{4}}{\isacharcomma}{\isadigit{5}}{\isacharparenright}\ \isacommand{by}\isamarkupfalse%
\ {\isacharparenleft}rule\ pcp{\isacharunderscore}colecComp{\isacharunderscore}CON{\isacharunderscore}sat{\isadigit{2}}{\isacharparenright}\isanewline
\ \ \ \ \ \ \isacommand{thus}\isamarkupfalse%
\ {\isachardoublequoteopen}sat\ {\isacharparenleft}{\isacharbraceleft}G{\isacharcomma}H{\isacharcomma}F{\isacharbraceright}\ {\isasymunion}\ Wo{\isacharparenright}{\isachardoublequoteclose}\isanewline
\ \ \ \ \ \ \ \ \isacommand{by}\isamarkupfalse%
\ {\isacharparenleft}simp\ only{\isacharcolon}\ {\isacartoucheopen}G\ {\isacharequal}\ \isactrlbold {\isasymnot}\ F{\isadigit{1}}{\isacartoucheclose}\ {\isacartoucheopen}H\ {\isacharequal}\ \isactrlbold {\isasymnot}\ G{\isadigit{1}}{\isacartoucheclose}{\isacharparenright}\isanewline
\ \ \ \ \isacommand{next}\isamarkupfalse%
\isanewline
\ \ \ \ \ \ \isacommand{assume}\isamarkupfalse%
\ {\isachardoublequoteopen}{\isacharparenleft}{\isasymexists}H{\isadigit{1}}{\isachardot}\ F\ {\isacharequal}\ \isactrlbold {\isasymnot}\ {\isacharparenleft}G\ \isactrlbold {\isasymrightarrow}\ H{\isadigit{1}}{\isacharparenright}\ {\isasymand}\ H\ {\isacharequal}\ \isactrlbold {\isasymnot}\ H{\isadigit{1}}{\isacharparenright}\ {\isasymor}\ \isanewline
\ \ \ \ \ \ \ \ \ \ \ \ \ \ F\ {\isacharequal}\ \isactrlbold {\isasymnot}\ {\isacharparenleft}\isactrlbold {\isasymnot}\ G{\isacharparenright}\ {\isasymand}\ H\ {\isacharequal}\ G{\isachardoublequoteclose}\isanewline
\ \ \ \ \ \ \isacommand{thus}\isamarkupfalse%
\ {\isachardoublequoteopen}sat\ {\isacharparenleft}{\isacharbraceleft}G{\isacharcomma}H{\isacharcomma}F{\isacharbraceright}\ {\isasymunion}\ Wo{\isacharparenright}{\isachardoublequoteclose}\isanewline
\ \ \ \ \ \ \isacommand{proof}\isamarkupfalse%
\ {\isacharparenleft}rule\ disjE{\isacharparenright}\isanewline
\ \ \ \ \ \ \ \ \isacommand{assume}\isamarkupfalse%
\ Ex{\isadigit{3}}{\isacharcolon}{\isachardoublequoteopen}{\isasymexists}H{\isadigit{1}}{\isachardot}\ F\ {\isacharequal}\ \isactrlbold {\isasymnot}\ {\isacharparenleft}G\ \isactrlbold {\isasymrightarrow}\ H{\isadigit{1}}{\isacharparenright}\ {\isasymand}\ H\ {\isacharequal}\ \isactrlbold {\isasymnot}\ H{\isadigit{1}}{\isachardoublequoteclose}\isanewline
\ \ \ \ \ \ \ \ \isacommand{obtain}\isamarkupfalse%
\ H{\isadigit{1}}\ \isakeyword{where}\ {\isadigit{3}}{\isacharcolon}{\isachardoublequoteopen}F\ {\isacharequal}\ \isactrlbold {\isasymnot}{\isacharparenleft}G\ \isactrlbold {\isasymrightarrow}\ H{\isadigit{1}}{\isacharparenright}\ {\isasymand}\ H\ {\isacharequal}\ \isactrlbold {\isasymnot}\ H{\isadigit{1}}{\isachardoublequoteclose}\isanewline
\ \ \ \ \ \ \ \ \ \ \isacommand{using}\isamarkupfalse%
\ Ex{\isadigit{3}}\ \isacommand{by}\isamarkupfalse%
\ {\isacharparenleft}rule\ exE{\isacharparenright}\isanewline
\ \ \ \ \ \ \ \ \isacommand{have}\isamarkupfalse%
\ {\isachardoublequoteopen}H\ {\isacharequal}\ \isactrlbold {\isasymnot}\ H{\isadigit{1}}{\isachardoublequoteclose}\isanewline
\ \ \ \ \ \ \ \ \ \ \isacommand{using}\isamarkupfalse%
\ {\isadigit{3}}\ \isacommand{by}\isamarkupfalse%
\ {\isacharparenleft}rule\ conjunct{\isadigit{2}}{\isacharparenright}\isanewline
\ \ \ \ \ \ \ \ \isacommand{have}\isamarkupfalse%
\ {\isachardoublequoteopen}F\ {\isacharequal}\ \isactrlbold {\isasymnot}{\isacharparenleft}G\ \isactrlbold {\isasymrightarrow}\ H{\isadigit{1}}{\isacharparenright}{\isachardoublequoteclose}\isanewline
\ \ \ \ \ \ \ \ \ \ \isacommand{using}\isamarkupfalse%
\ {\isadigit{3}}\ \isacommand{by}\isamarkupfalse%
\ {\isacharparenleft}rule\ conjunct{\isadigit{1}}{\isacharparenright}\isanewline
\ \ \ \ \ \ \ \ \isacommand{have}\isamarkupfalse%
\ {\isachardoublequoteopen}sat\ {\isacharparenleft}{\isacharbraceleft}G{\isacharcomma}\ \isactrlbold {\isasymnot}\ H{\isadigit{1}}{\isacharcomma}\ F{\isacharbraceright}\ {\isasymunion}\ Wo{\isacharparenright}{\isachardoublequoteclose}\isanewline
\ \ \ \ \ \ \ \ \ \ \isacommand{using}\isamarkupfalse%
\ assms{\isacharparenleft}{\isadigit{1}}{\isacharparenright}\ {\isacartoucheopen}F\ {\isacharequal}\ \isactrlbold {\isasymnot}{\isacharparenleft}G\ \isactrlbold {\isasymrightarrow}\ H{\isadigit{1}}{\isacharparenright}{\isacartoucheclose}\ assms{\isacharparenleft}{\isadigit{3}}{\isacharcomma}{\isadigit{4}}{\isacharcomma}{\isadigit{5}}{\isacharparenright}\ \isacommand{by}\isamarkupfalse%
\ {\isacharparenleft}rule\ pcp{\isacharunderscore}colecComp{\isacharunderscore}CON{\isacharunderscore}sat{\isadigit{3}}{\isacharparenright}\isanewline
\ \ \ \ \ \ \ \ \isacommand{thus}\isamarkupfalse%
\ {\isachardoublequoteopen}sat\ {\isacharparenleft}{\isacharbraceleft}G{\isacharcomma}H{\isacharcomma}F{\isacharbraceright}\ {\isasymunion}\ Wo{\isacharparenright}{\isachardoublequoteclose}\isanewline
\ \ \ \ \ \ \ \ \ \ \isacommand{by}\isamarkupfalse%
\ {\isacharparenleft}simp\ only{\isacharcolon}\ {\isacartoucheopen}H\ {\isacharequal}\ \isactrlbold {\isasymnot}\ H{\isadigit{1}}{\isacartoucheclose}{\isacharparenright}\isanewline
\ \ \ \ \ \ \isacommand{next}\isamarkupfalse%
\isanewline
\ \ \ \ \ \ \ \ \isacommand{assume}\isamarkupfalse%
\ {\isachardoublequoteopen}F\ {\isacharequal}\ \isactrlbold {\isasymnot}\ {\isacharparenleft}\isactrlbold {\isasymnot}\ G{\isacharparenright}\ {\isasymand}\ H\ {\isacharequal}\ G{\isachardoublequoteclose}\isanewline
\ \ \ \ \ \ \ \ \isacommand{then}\isamarkupfalse%
\ \isacommand{have}\isamarkupfalse%
\ {\isachardoublequoteopen}H\ {\isacharequal}\ G{\isachardoublequoteclose}\isanewline
\ \ \ \ \ \ \ \ \ \ \isacommand{by}\isamarkupfalse%
\ {\isacharparenleft}rule\ conjunct{\isadigit{2}}{\isacharparenright}\isanewline
\ \ \ \ \ \ \ \ \isacommand{then}\isamarkupfalse%
\ \isacommand{have}\isamarkupfalse%
\ C{\isadigit{4}}{\isacharcolon}{\isachardoublequoteopen}{\isacharbraceleft}G{\isacharcomma}F{\isacharbraceright}\ {\isasymunion}\ Wo\ {\isacharequal}\ {\isacharbraceleft}G{\isacharcomma}H{\isacharcomma}F{\isacharbraceright}\ {\isasymunion}\ Wo{\isachardoublequoteclose}\isanewline
\ \ \ \ \ \ \ \ \ \ \isacommand{by}\isamarkupfalse%
\ blast\isanewline
\ \ \ \ \ \ \ \ \isacommand{have}\isamarkupfalse%
\ {\isachardoublequoteopen}F\ {\isacharequal}\ \isactrlbold {\isasymnot}\ {\isacharparenleft}\isactrlbold {\isasymnot}\ G{\isacharparenright}{\isachardoublequoteclose}\isanewline
\ \ \ \ \ \ \ \ \ \ \isacommand{using}\isamarkupfalse%
\ {\isacartoucheopen}F\ {\isacharequal}\ \isactrlbold {\isasymnot}\ {\isacharparenleft}\isactrlbold {\isasymnot}\ G{\isacharparenright}\ {\isasymand}\ H\ {\isacharequal}\ G{\isacartoucheclose}\ \isacommand{by}\isamarkupfalse%
\ {\isacharparenleft}rule\ conjunct{\isadigit{1}}{\isacharparenright}\isanewline
\ \ \ \ \ \ \ \ \isacommand{have}\isamarkupfalse%
\ {\isachardoublequoteopen}sat\ {\isacharparenleft}{\isacharbraceleft}G{\isacharcomma}F{\isacharbraceright}\ {\isasymunion}\ Wo{\isacharparenright}{\isachardoublequoteclose}\isanewline
\ \ \ \ \ \ \ \ \ \ \isacommand{using}\isamarkupfalse%
\ assms{\isacharparenleft}{\isadigit{1}}{\isacharparenright}\ {\isacartoucheopen}F\ {\isacharequal}\ \isactrlbold {\isasymnot}{\isacharparenleft}\isactrlbold {\isasymnot}\ G{\isacharparenright}{\isacartoucheclose}\ assms{\isacharparenleft}{\isadigit{3}}{\isacharcomma}{\isadigit{4}}{\isacharcomma}{\isadigit{5}}{\isacharparenright}\ \isacommand{by}\isamarkupfalse%
\ {\isacharparenleft}rule\ pcp{\isacharunderscore}colecComp{\isacharunderscore}CON{\isacharunderscore}sat{\isadigit{4}}{\isacharparenright}\isanewline
\ \ \ \ \ \ \ \ \isacommand{thus}\isamarkupfalse%
\ {\isachardoublequoteopen}sat\ {\isacharparenleft}{\isacharbraceleft}G{\isacharcomma}H{\isacharcomma}F{\isacharbraceright}\ {\isasymunion}\ Wo{\isacharparenright}{\isachardoublequoteclose}\isanewline
\ \ \ \ \ \ \ \ \ \ \isacommand{by}\isamarkupfalse%
\ {\isacharparenleft}simp\ only{\isacharcolon}\ C{\isadigit{4}}{\isacharparenright}\isanewline
\ \ \ \ \ \ \isacommand{qed}\isamarkupfalse%
\isanewline
\ \ \ \ \isacommand{qed}\isamarkupfalse%
\isanewline
\ \ \isacommand{qed}\isamarkupfalse%
\isanewline
\isacommand{qed}\isamarkupfalse%
%
\endisatagproof
{\isafoldproof}%
%
\isadelimproof
%
\endisadelimproof
%
\begin{isamarkuptext}%
Finalmente, con el resultado anterior, podemos probar la tercera condición del lema \isa{{\isadigit{2}}{\isachardot}{\isadigit{0}}{\isachardot}{\isadigit{2}}}: 
  dados \isa{W\ {\isasymin}\ C} y \isa{F} una fórmula de tipo \isa{{\isasymalpha}} con componentes \isa{{\isasymalpha}\isactrlsub {\isadigit{1}}} y \isa{{\isasymalpha}\isactrlsub {\isadigit{2}}} tal que \isa{F\ {\isasymin}\ W}, se tiene 
  que \isa{{\isacharbraceleft}{\isasymalpha}\isactrlsub {\isadigit{1}}{\isacharcomma}{\isasymalpha}\isactrlsub {\isadigit{2}}{\isacharbraceright}\ {\isasymunion}\ W\ {\isasymin}\ C}.%
\end{isamarkuptext}\isamarkuptrue%
\isacommand{lemma}\isamarkupfalse%
\ pcp{\isacharunderscore}colecComp{\isacharunderscore}CON{\isacharcolon}\isanewline
\ \ \isakeyword{assumes}\ {\isachardoublequoteopen}W\ {\isasymin}\ colecComp{\isachardoublequoteclose}\isanewline
\ \ \isakeyword{shows}\ {\isachardoublequoteopen}{\isasymforall}F\ G\ H{\isachardot}\ Con\ F\ G\ H\ {\isasymlongrightarrow}\ F\ {\isasymin}\ W\ {\isasymlongrightarrow}\ {\isacharbraceleft}G{\isacharcomma}H{\isacharbraceright}\ {\isasymunion}\ W\ {\isasymin}\ colecComp{\isachardoublequoteclose}\isanewline
%
\isadelimproof
%
\endisadelimproof
%
\isatagproof
\isacommand{proof}\isamarkupfalse%
\ {\isacharparenleft}rule\ allI{\isacharparenright}{\isacharplus}\isanewline
\ \ \isacommand{fix}\isamarkupfalse%
\ F\ G\ H\isanewline
\ \ \isacommand{show}\isamarkupfalse%
\ {\isachardoublequoteopen}Con\ F\ G\ H\ {\isasymlongrightarrow}\ F\ {\isasymin}\ W\ {\isasymlongrightarrow}\ {\isacharbraceleft}G{\isacharcomma}H{\isacharbraceright}\ {\isasymunion}\ W\ {\isasymin}\ colecComp{\isachardoublequoteclose}\isanewline
\ \ \isacommand{proof}\isamarkupfalse%
\ {\isacharparenleft}rule\ impI{\isacharparenright}{\isacharplus}\isanewline
\ \ \ \ \isacommand{assume}\isamarkupfalse%
\ {\isachardoublequoteopen}Con\ F\ G\ H{\isachardoublequoteclose}\isanewline
\ \ \ \ \isacommand{assume}\isamarkupfalse%
\ {\isachardoublequoteopen}F\ {\isasymin}\ W{\isachardoublequoteclose}\isanewline
\ \ \ \ \isacommand{show}\isamarkupfalse%
\ {\isachardoublequoteopen}{\isacharbraceleft}G{\isacharcomma}H{\isacharbraceright}\ {\isasymunion}\ W\ {\isasymin}\ colecComp{\isachardoublequoteclose}\isanewline
\ \ \ \ \ \ \isacommand{unfolding}\isamarkupfalse%
\ colecComp\ fin{\isacharunderscore}sat{\isacharunderscore}def\isanewline
\ \ \ \ \isacommand{proof}\isamarkupfalse%
\ {\isacharparenleft}rule\ CollectI{\isacharparenright}\isanewline
\ \ \ \ \ \ \isacommand{show}\isamarkupfalse%
\ {\isachardoublequoteopen}{\isasymforall}S\ {\isasymsubseteq}\ {\isacharbraceleft}G{\isacharcomma}H{\isacharbraceright}\ {\isasymunion}\ W{\isachardot}\ finite\ S\ {\isasymlongrightarrow}\ sat\ S{\isachardoublequoteclose}\isanewline
\ \ \ \ \ \ \isacommand{proof}\isamarkupfalse%
\ {\isacharparenleft}rule\ sallI{\isacharparenright}\isanewline
\ \ \ \ \ \ \ \ \isacommand{fix}\isamarkupfalse%
\ S\isanewline
\ \ \ \ \ \ \ \ \isacommand{assume}\isamarkupfalse%
\ {\isachardoublequoteopen}S\ {\isasymsubseteq}\ {\isacharbraceleft}G{\isacharcomma}H{\isacharbraceright}\ {\isasymunion}\ W{\isachardoublequoteclose}\isanewline
\ \ \ \ \ \ \ \ \isacommand{then}\isamarkupfalse%
\ \isacommand{have}\isamarkupfalse%
\ {\isachardoublequoteopen}S\ {\isasymsubseteq}\ {\isacharbraceleft}G{\isacharbraceright}\ {\isasymunion}\ {\isacharparenleft}{\isacharbraceleft}H{\isacharbraceright}\ {\isasymunion}\ W{\isacharparenright}{\isachardoublequoteclose}\isanewline
\ \ \ \ \ \ \ \ \ \ \isacommand{by}\isamarkupfalse%
\ blast\ \isanewline
\ \ \ \ \ \ \ \ \isacommand{show}\isamarkupfalse%
\ {\isachardoublequoteopen}finite\ S\ {\isasymlongrightarrow}\ sat\ S{\isachardoublequoteclose}\isanewline
\ \ \ \ \ \ \ \ \isacommand{proof}\isamarkupfalse%
\ {\isacharparenleft}rule\ impI{\isacharparenright}\isanewline
\ \ \ \ \ \ \ \ \ \ \isacommand{assume}\isamarkupfalse%
\ {\isachardoublequoteopen}finite\ S{\isachardoublequoteclose}\ \isanewline
\ \ \ \ \ \ \ \ \ \ \isacommand{have}\isamarkupfalse%
\ Ex{\isacharcolon}{\isachardoublequoteopen}{\isasymexists}Wo\ {\isasymsubseteq}\ W{\isachardot}\ finite\ Wo\ {\isasymand}\ {\isacharparenleft}S\ {\isacharequal}\ {\isacharbraceleft}G{\isacharcomma}H{\isacharbraceright}\ {\isasymunion}\ Wo\ {\isasymor}\ S\ {\isacharequal}\ {\isacharbraceleft}G{\isacharbraceright}\ {\isasymunion}\ Wo\ {\isasymor}\ S\ {\isacharequal}\ {\isacharbraceleft}H{\isacharbraceright}\ {\isasymunion}\ Wo\ {\isasymor}\ S\ {\isacharequal}\ Wo{\isacharparenright}{\isachardoublequoteclose}\isanewline
\ \ \ \ \ \ \ \ \ \ \ \ \isacommand{using}\isamarkupfalse%
\ {\isacartoucheopen}finite\ S{\isacartoucheclose}\ {\isacartoucheopen}S\ {\isasymsubseteq}\ {\isacharbraceleft}G{\isacharcomma}H{\isacharbraceright}\ {\isasymunion}\ W{\isacartoucheclose}\ \isacommand{by}\isamarkupfalse%
\ {\isacharparenleft}rule\ finite{\isacharunderscore}subset{\isacharunderscore}insert{\isadigit{2}}{\isacharparenright}\isanewline
\ \ \ \ \ \ \ \ \ \ \isacommand{obtain}\isamarkupfalse%
\ Wo\ \isakeyword{where}\ {\isachardoublequoteopen}Wo\ {\isasymsubseteq}\ W{\isachardoublequoteclose}\ \isakeyword{and}\ {\isadigit{1}}{\isacharcolon}{\isachardoublequoteopen}finite\ Wo\ {\isasymand}\ {\isacharparenleft}S\ {\isacharequal}\ {\isacharbraceleft}G{\isacharcomma}H{\isacharbraceright}\ {\isasymunion}\ Wo\ {\isasymor}\ S\ {\isacharequal}\ {\isacharbraceleft}G{\isacharbraceright}\ {\isasymunion}\ Wo\ {\isasymor}\ S\ {\isacharequal}\ {\isacharbraceleft}H{\isacharbraceright}\ {\isasymunion}\ Wo\ {\isasymor}\ S\ {\isacharequal}\ Wo{\isacharparenright}{\isachardoublequoteclose}\isanewline
\ \ \ \ \ \ \ \ \ \ \ \ \isacommand{using}\isamarkupfalse%
\ Ex\ \isacommand{by}\isamarkupfalse%
\ {\isacharparenleft}rule\ subexE{\isacharparenright}\isanewline
\ \ \ \ \ \ \ \ \ \ \isacommand{have}\isamarkupfalse%
\ {\isachardoublequoteopen}finite\ Wo{\isachardoublequoteclose}\isanewline
\ \ \ \ \ \ \ \ \ \ \ \ \isacommand{using}\isamarkupfalse%
\ {\isadigit{1}}\ \isacommand{by}\isamarkupfalse%
\ {\isacharparenleft}rule\ conjunct{\isadigit{1}}{\isacharparenright}\isanewline
\ \ \ \ \ \ \ \ \ \ \ \ \isacommand{have}\isamarkupfalse%
\ {\isachardoublequoteopen}sat\ {\isacharparenleft}{\isacharbraceleft}G{\isacharcomma}H{\isacharcomma}F{\isacharbraceright}\ {\isasymunion}\ Wo{\isacharparenright}{\isachardoublequoteclose}\ \isanewline
\ \ \ \ \ \ \ \ \ \ \ \ \ \ \isacommand{using}\isamarkupfalse%
\ {\isacartoucheopen}W\ {\isasymin}\ colecComp{\isacartoucheclose}\ {\isacartoucheopen}Con\ F\ G\ H{\isacartoucheclose}\ {\isacartoucheopen}F\ {\isasymin}\ W{\isacartoucheclose}\ {\isacartoucheopen}finite\ Wo{\isacartoucheclose}\ {\isacartoucheopen}Wo\ {\isasymsubseteq}\ W{\isacartoucheclose}\ \isacommand{by}\isamarkupfalse%
\ {\isacharparenleft}rule\ pcp{\isacharunderscore}colecComp{\isacharunderscore}CON{\isacharunderscore}sat{\isacharparenright}\isanewline
\ \ \ \ \ \ \ \ \ \ \isacommand{have}\isamarkupfalse%
\ {\isachardoublequoteopen}S\ {\isacharequal}\ {\isacharbraceleft}G{\isacharcomma}H{\isacharbraceright}\ {\isasymunion}\ Wo\ {\isasymor}\ S\ {\isacharequal}\ {\isacharbraceleft}G{\isacharbraceright}\ {\isasymunion}\ Wo\ {\isasymor}\ S\ {\isacharequal}\ {\isacharbraceleft}H{\isacharbraceright}\ {\isasymunion}\ Wo\ {\isasymor}\ S\ {\isacharequal}\ Wo{\isachardoublequoteclose}\isanewline
\ \ \ \ \ \ \ \ \ \ \ \ \isacommand{using}\isamarkupfalse%
\ {\isadigit{1}}\ \isacommand{by}\isamarkupfalse%
\ {\isacharparenleft}rule\ conjunct{\isadigit{2}}{\isacharparenright}\isanewline
\ \ \ \ \ \ \ \ \ \ \isacommand{thus}\isamarkupfalse%
\ {\isachardoublequoteopen}sat\ S{\isachardoublequoteclose}\isanewline
\ \ \ \ \ \ \ \ \ \ \isacommand{proof}\isamarkupfalse%
\ {\isacharparenleft}rule\ disjE{\isacharparenright}\isanewline
\ \ \ \ \ \ \ \ \ \ \ \ \isacommand{assume}\isamarkupfalse%
\ {\isachardoublequoteopen}S\ {\isacharequal}\ {\isacharbraceleft}G{\isacharcomma}H{\isacharbraceright}\ {\isasymunion}\ Wo{\isachardoublequoteclose}\isanewline
\ \ \ \ \ \ \ \ \ \ \ \ \isacommand{then}\isamarkupfalse%
\ \isacommand{have}\isamarkupfalse%
\ {\isachardoublequoteopen}S\ {\isasymsubseteq}\ {\isacharbraceleft}G{\isacharcomma}H{\isacharcomma}F{\isacharbraceright}\ {\isasymunion}\ Wo{\isachardoublequoteclose}\isanewline
\ \ \ \ \ \ \ \ \ \ \ \ \ \ \isacommand{by}\isamarkupfalse%
\ blast\isanewline
\ \ \ \ \ \ \ \ \ \ \ \ \isacommand{show}\isamarkupfalse%
\ {\isachardoublequoteopen}sat\ S{\isachardoublequoteclose}\isanewline
\ \ \ \ \ \ \ \ \ \ \ \ \ \ \isacommand{using}\isamarkupfalse%
\ {\isacartoucheopen}sat{\isacharparenleft}{\isacharbraceleft}G{\isacharcomma}H{\isacharcomma}F{\isacharbraceright}\ {\isasymunion}\ Wo{\isacharparenright}{\isacartoucheclose}\ {\isacartoucheopen}S\ {\isasymsubseteq}\ {\isacharbraceleft}G{\isacharcomma}H{\isacharcomma}F{\isacharbraceright}\ {\isasymunion}\ Wo{\isacartoucheclose}\ \isacommand{by}\isamarkupfalse%
\ {\isacharparenleft}simp\ only{\isacharcolon}\ sat{\isacharunderscore}mono{\isacharparenright}\isanewline
\ \ \ \ \ \ \ \ \ \ \isacommand{next}\isamarkupfalse%
\isanewline
\ \ \ \ \ \ \ \ \ \ \ \ \isacommand{assume}\isamarkupfalse%
\ {\isachardoublequoteopen}S\ {\isacharequal}\ {\isacharbraceleft}G{\isacharbraceright}\ {\isasymunion}\ Wo\ {\isasymor}\ S\ {\isacharequal}\ {\isacharbraceleft}H{\isacharbraceright}\ {\isasymunion}\ Wo\ {\isasymor}\ S\ {\isacharequal}\ Wo{\isachardoublequoteclose}\isanewline
\ \ \ \ \ \ \ \ \ \ \ \ \isacommand{thus}\isamarkupfalse%
\ {\isachardoublequoteopen}sat\ S{\isachardoublequoteclose}\isanewline
\ \ \ \ \ \ \ \ \ \ \ \ \isacommand{proof}\isamarkupfalse%
\ {\isacharparenleft}rule\ disjE{\isacharparenright}\isanewline
\ \ \ \ \ \ \ \ \ \ \ \ \ \ \isacommand{assume}\isamarkupfalse%
\ {\isachardoublequoteopen}S\ {\isacharequal}\ {\isacharbraceleft}G{\isacharbraceright}\ {\isasymunion}\ Wo{\isachardoublequoteclose}\isanewline
\ \ \ \ \ \ \ \ \ \ \ \ \ \ \isacommand{then}\isamarkupfalse%
\ \isacommand{have}\isamarkupfalse%
\ {\isachardoublequoteopen}S\ {\isasymsubseteq}\ {\isacharbraceleft}G{\isacharcomma}H{\isacharcomma}F{\isacharbraceright}\ {\isasymunion}\ Wo{\isachardoublequoteclose}\isanewline
\ \ \ \ \ \ \ \ \ \ \ \ \ \ \ \ \isacommand{by}\isamarkupfalse%
\ blast\ \isanewline
\ \ \ \ \ \ \ \ \ \ \ \ \ \ \isacommand{thus}\isamarkupfalse%
\ {\isachardoublequoteopen}sat\ S{\isachardoublequoteclose}\isanewline
\ \ \ \ \ \ \ \ \ \ \ \ \ \ \ \ \isacommand{using}\isamarkupfalse%
\ {\isacartoucheopen}sat{\isacharparenleft}{\isacharbraceleft}G{\isacharcomma}H{\isacharcomma}F{\isacharbraceright}\ {\isasymunion}\ Wo{\isacharparenright}{\isacartoucheclose}\ \isacommand{by}\isamarkupfalse%
\ {\isacharparenleft}rule\ sat{\isacharunderscore}mono{\isacharparenright}\isanewline
\ \ \ \ \ \ \ \ \ \ \ \ \isacommand{next}\isamarkupfalse%
\isanewline
\ \ \ \ \ \ \ \ \ \ \ \ \ \ \isacommand{assume}\isamarkupfalse%
\ {\isachardoublequoteopen}S\ {\isacharequal}\ {\isacharbraceleft}H{\isacharbraceright}\ {\isasymunion}\ Wo\ {\isasymor}\ S\ {\isacharequal}\ Wo{\isachardoublequoteclose}\isanewline
\ \ \ \ \ \ \ \ \ \ \ \ \ \ \isacommand{thus}\isamarkupfalse%
\ {\isachardoublequoteopen}sat\ S{\isachardoublequoteclose}\isanewline
\ \ \ \ \ \ \ \ \ \ \ \ \ \ \isacommand{proof}\isamarkupfalse%
\ {\isacharparenleft}rule\ disjE{\isacharparenright}\isanewline
\ \ \ \ \ \ \ \ \ \ \ \ \ \ \ \ \isacommand{assume}\isamarkupfalse%
\ {\isachardoublequoteopen}S\ {\isacharequal}\ {\isacharbraceleft}H{\isacharbraceright}\ {\isasymunion}\ Wo{\isachardoublequoteclose}\isanewline
\ \ \ \ \ \ \ \ \ \ \ \ \ \ \ \ \isacommand{then}\isamarkupfalse%
\ \isacommand{have}\isamarkupfalse%
\ {\isachardoublequoteopen}S\ {\isasymsubseteq}\ {\isacharbraceleft}G{\isacharcomma}H{\isacharcomma}F{\isacharbraceright}\ {\isasymunion}\ Wo{\isachardoublequoteclose}\isanewline
\ \ \ \ \ \ \ \ \ \ \ \ \ \ \ \ \ \ \isacommand{by}\isamarkupfalse%
\ blast\ \isanewline
\ \ \ \ \ \ \ \ \ \ \ \ \ \ \ \ \isacommand{thus}\isamarkupfalse%
\ {\isachardoublequoteopen}sat\ S{\isachardoublequoteclose}\isanewline
\ \ \ \ \ \ \ \ \ \ \ \ \ \ \ \ \ \ \isacommand{using}\isamarkupfalse%
\ {\isacartoucheopen}sat{\isacharparenleft}{\isacharbraceleft}G{\isacharcomma}H{\isacharcomma}F{\isacharbraceright}\ {\isasymunion}\ Wo{\isacharparenright}{\isacartoucheclose}\ \isacommand{by}\isamarkupfalse%
\ {\isacharparenleft}rule\ sat{\isacharunderscore}mono{\isacharparenright}\isanewline
\ \ \ \ \ \ \ \ \ \ \ \ \ \ \isacommand{next}\isamarkupfalse%
\isanewline
\ \ \ \ \ \ \ \ \ \ \ \ \ \ \ \ \isacommand{assume}\isamarkupfalse%
\ {\isachardoublequoteopen}S\ {\isacharequal}\ Wo{\isachardoublequoteclose}\isanewline
\ \ \ \ \ \ \ \ \ \ \ \ \ \ \ \ \isacommand{then}\isamarkupfalse%
\ \isacommand{have}\isamarkupfalse%
\ {\isachardoublequoteopen}S\ {\isasymsubseteq}\ {\isacharbraceleft}G{\isacharcomma}H{\isacharcomma}F{\isacharbraceright}\ {\isasymunion}\ Wo{\isachardoublequoteclose}\isanewline
\ \ \ \ \ \ \ \ \ \ \ \ \ \ \ \ \ \ \isacommand{by}\isamarkupfalse%
\ {\isacharparenleft}simp\ only{\isacharcolon}\ Un{\isacharunderscore}upper{\isadigit{2}}{\isacharparenright}\isanewline
\ \ \ \ \ \ \ \ \ \ \ \ \ \ \ \ \isacommand{thus}\isamarkupfalse%
\ {\isachardoublequoteopen}sat\ S{\isachardoublequoteclose}\isanewline
\ \ \ \ \ \ \ \ \ \ \ \ \ \ \ \ \ \ \isacommand{using}\isamarkupfalse%
\ {\isacartoucheopen}sat{\isacharparenleft}{\isacharbraceleft}G{\isacharcomma}H{\isacharcomma}F{\isacharbraceright}\ {\isasymunion}\ Wo{\isacharparenright}{\isacartoucheclose}\ \isacommand{by}\isamarkupfalse%
\ {\isacharparenleft}rule\ sat{\isacharunderscore}mono{\isacharparenright}\isanewline
\ \ \ \ \ \ \ \ \ \ \ \ \ \ \isacommand{qed}\isamarkupfalse%
\isanewline
\ \ \ \ \ \ \ \ \ \ \ \ \isacommand{qed}\isamarkupfalse%
\isanewline
\ \ \ \ \ \ \ \ \ \ \isacommand{qed}\isamarkupfalse%
\isanewline
\ \ \ \ \ \ \ \ \isacommand{qed}\isamarkupfalse%
\isanewline
\ \ \ \ \ \ \isacommand{qed}\isamarkupfalse%
\isanewline
\ \ \ \ \isacommand{qed}\isamarkupfalse%
\isanewline
\ \ \isacommand{qed}\isamarkupfalse%
\isanewline
\isacommand{qed}\isamarkupfalse%
%
\endisatagproof
{\isafoldproof}%
%
\isadelimproof
%
\endisadelimproof
%
\begin{isamarkuptext}%
Por último, probemos la cuarta condición del lema \isa{{\isadigit{2}}{\isachardot}{\isadigit{0}}{\isachardot}{\isadigit{2}}}: dados \isa{W\ {\isasymin}\ C} y \isa{F} una 
  fórmula de tipo \isa{{\isasymbeta}} con componentes \isa{{\isasymbeta}\isactrlsub {\isadigit{1}}} y \isa{{\isasymbeta}\isactrlsub {\isadigit{2}}} tal que \isa{F\ {\isasymin}\ W}, se tiene que o bien\\ \isa{{\isacharbraceleft}{\isasymbeta}\isactrlsub {\isadigit{1}}{\isacharbraceright}\ {\isasymunion}\ W\ {\isasymin}\ C} 
  o bien \isa{{\isacharbraceleft}{\isasymbeta}\isactrlsub {\isadigit{2}}{\isacharbraceright}\ {\isasymunion}\ W\ {\isasymin}\ C}. 
  
  Por un lado, precisaremos para ello de un lema auxiliar que demuestre que dado \isa{W\ {\isasymin}\ C} y \isa{{\isasymbeta}\isactrlsub i} una 
  fórmula cualquiera tal que \isa{{\isacharbraceleft}{\isasymbeta}\isactrlsub i{\isacharbraceright}\ {\isasymunion}\ W\ {\isasymnotin}\ C}, entonces existe un subconjunto finito \isa{W\isactrlsub i} de \isa{W} tal 
  que el conjunto \isa{{\isacharbraceleft}{\isasymbeta}\isactrlsub i{\isacharbraceright}\ {\isasymunion}\ W\isactrlsub i} no es satisfacible. A su vez, para su demostración emplearemos un lema 
  que prueba que todo conjunto que contiene un subconjunto insatisfacible es también 
  insatisfacible.%
\end{isamarkuptext}\isamarkuptrue%
\isacommand{lemma}\isamarkupfalse%
\ sat{\isacharunderscore}subset{\isacharunderscore}ccontr{\isacharcolon}\isanewline
\ \ \isakeyword{assumes}\ {\isachardoublequoteopen}A\ {\isasymsubseteq}\ B{\isachardoublequoteclose}\isanewline
\ \ \ \ \ \ \ \ \ \ {\isachardoublequoteopen}{\isasymnot}\ sat\ A{\isachardoublequoteclose}\isanewline
\ \ \ \ \ \ \ \ \isakeyword{shows}\ {\isachardoublequoteopen}{\isasymnot}\ sat\ B{\isachardoublequoteclose}\isanewline
%
\isadelimproof
%
\endisadelimproof
%
\isatagproof
\isacommand{proof}\isamarkupfalse%
\ {\isacharminus}\isanewline
\ \ \isacommand{have}\isamarkupfalse%
\ {\isachardoublequoteopen}A\ {\isasymsubseteq}\ B\ {\isasymand}\ sat\ B\ {\isasymlongrightarrow}\ sat\ A{\isachardoublequoteclose}\isanewline
\ \ \ \ \isacommand{using}\isamarkupfalse%
\ sat{\isacharunderscore}mono\ \isacommand{by}\isamarkupfalse%
\ blast\isanewline
\ \ \isacommand{then}\isamarkupfalse%
\ \isacommand{have}\isamarkupfalse%
\ {\isachardoublequoteopen}{\isasymnot}{\isacharparenleft}A\ {\isasymsubseteq}\ B\ {\isasymand}\ sat\ B{\isacharparenright}{\isachardoublequoteclose}\isanewline
\ \ \ \ \isacommand{using}\isamarkupfalse%
\ assms{\isacharparenleft}{\isadigit{2}}{\isacharparenright}\ \isacommand{by}\isamarkupfalse%
\ {\isacharparenleft}rule\ mt{\isacharparenright}\isanewline
\ \ \isacommand{then}\isamarkupfalse%
\ \isacommand{have}\isamarkupfalse%
\ {\isachardoublequoteopen}{\isasymnot}{\isacharparenleft}A\ {\isasymsubseteq}\ B{\isacharparenright}\ {\isasymor}\ {\isasymnot}{\isacharparenleft}sat\ B{\isacharparenright}{\isachardoublequoteclose}\isanewline
\ \ \ \ \isacommand{by}\isamarkupfalse%
\ {\isacharparenleft}simp\ only{\isacharcolon}\ de{\isacharunderscore}Morgan{\isacharunderscore}conj{\isacharparenright}\isanewline
\ \ \isacommand{thus}\isamarkupfalse%
\ {\isacharquery}thesis\isanewline
\ \ \isacommand{proof}\isamarkupfalse%
\ {\isacharparenleft}rule\ disjE{\isacharparenright}\isanewline
\ \ \ \ \isacommand{assume}\isamarkupfalse%
\ {\isachardoublequoteopen}{\isasymnot}{\isacharparenleft}A\ {\isasymsubseteq}\ B{\isacharparenright}{\isachardoublequoteclose}\isanewline
\ \ \ \ \isacommand{thus}\isamarkupfalse%
\ {\isacharquery}thesis\isanewline
\ \ \ \ \ \ \isacommand{using}\isamarkupfalse%
\ assms{\isacharparenleft}{\isadigit{1}}{\isacharparenright}\ \isacommand{by}\isamarkupfalse%
\ {\isacharparenleft}rule\ notE{\isacharparenright}\isanewline
\ \ \isacommand{next}\isamarkupfalse%
\isanewline
\ \ \ \ \isacommand{assume}\isamarkupfalse%
\ {\isachardoublequoteopen}{\isasymnot}{\isacharparenleft}sat\ B{\isacharparenright}{\isachardoublequoteclose}\isanewline
\ \ \ \ \isacommand{thus}\isamarkupfalse%
\ {\isacharquery}thesis\isanewline
\ \ \ \ \ \ \isacommand{by}\isamarkupfalse%
\ this\isanewline
\ \ \isacommand{qed}\isamarkupfalse%
\isanewline
\isacommand{qed}\isamarkupfalse%
%
\endisatagproof
{\isafoldproof}%
%
\isadelimproof
%
\endisadelimproof
%
\begin{isamarkuptext}%
De este modo, podemos demostrar que dados \isa{W\ {\isasymin}\ C} y \isa{{\isasymbeta}\isactrlsub i} una fórmula cualquiera tal que 
  \isa{{\isacharbraceleft}{\isasymbeta}\isactrlsub i{\isacharbraceright}\ {\isasymunion}\ W\ {\isasymnotin}\ C}, entonces existe un subconjunto finito \isa{W\isactrlsub i} de \isa{W} tal que el conjunto \isa{{\isacharbraceleft}{\isasymbeta}\isactrlsub i{\isacharbraceright}\ {\isasymunion}\ W\isactrlsub F} 
  no es satisfacible.%
\end{isamarkuptext}\isamarkuptrue%
\isacommand{lemma}\isamarkupfalse%
\ not{\isacharunderscore}colecComp{\isacharcolon}\isanewline
\ \ \isakeyword{assumes}\ {\isachardoublequoteopen}W\ {\isasymin}\ colecComp{\isachardoublequoteclose}\isanewline
\ \ \ \ \ \ \ \ \ \ {\isachardoublequoteopen}{\isacharbraceleft}Gi{\isacharbraceright}\ {\isasymunion}\ W\ {\isasymnotin}\ colecComp{\isachardoublequoteclose}\isanewline
\ \ \ \ \ \ \ \ \isakeyword{shows}\ {\isachardoublequoteopen}{\isasymexists}Wi\ {\isasymsubseteq}\ W{\isachardot}\ finite\ Wi\ {\isasymand}\ {\isasymnot}{\isacharparenleft}sat\ {\isacharparenleft}{\isacharbraceleft}Gi{\isacharbraceright}\ {\isasymunion}\ Wi{\isacharparenright}{\isacharparenright}{\isachardoublequoteclose}\isanewline
%
\isadelimproof
%
\endisadelimproof
%
\isatagproof
\isacommand{proof}\isamarkupfalse%
\ {\isacharminus}\isanewline
\ \ \isacommand{have}\isamarkupfalse%
\ WCol{\isacharcolon}{\isachardoublequoteopen}{\isasymforall}S{\isacharprime}\ {\isasymsubseteq}\ W{\isachardot}\ finite\ S{\isacharprime}\ {\isasymlongrightarrow}\ sat\ S{\isacharprime}{\isachardoublequoteclose}\isanewline
\ \ \ \ \isacommand{using}\isamarkupfalse%
\ assms{\isacharparenleft}{\isadigit{1}}{\isacharparenright}\ \isacommand{unfolding}\isamarkupfalse%
\ colecComp\ fin{\isacharunderscore}sat{\isacharunderscore}def\ \isacommand{by}\isamarkupfalse%
\ {\isacharparenleft}rule\ CollectD{\isacharparenright}\ \isanewline
\ \ \isacommand{have}\isamarkupfalse%
\ {\isachardoublequoteopen}{\isasymnot}{\isacharparenleft}{\isasymforall}Wo\ {\isasymsubseteq}\ {\isacharbraceleft}Gi{\isacharbraceright}\ {\isasymunion}\ W{\isachardot}\ finite\ Wo\ {\isasymlongrightarrow}\ sat\ Wo{\isacharparenright}{\isachardoublequoteclose}\isanewline
\ \ \ \ \isacommand{using}\isamarkupfalse%
\ assms{\isacharparenleft}{\isadigit{2}}{\isacharparenright}\ \isacommand{unfolding}\isamarkupfalse%
\ colecComp\ fin{\isacharunderscore}sat{\isacharunderscore}def\ \isacommand{by}\isamarkupfalse%
\ {\isacharparenleft}simp\ only{\isacharcolon}\ mem{\isacharunderscore}Collect{\isacharunderscore}eq\ simp{\isacharunderscore}thms{\isacharparenleft}{\isadigit{8}}{\isacharparenright}{\isacharparenright}\isanewline
\ \ \isacommand{then}\isamarkupfalse%
\ \isacommand{have}\isamarkupfalse%
\ {\isachardoublequoteopen}{\isasymexists}Wo\ {\isasymsubseteq}\ {\isacharbraceleft}Gi{\isacharbraceright}\ {\isasymunion}\ W{\isachardot}\ {\isasymnot}{\isacharparenleft}finite\ Wo\ {\isasymlongrightarrow}\ sat\ Wo{\isacharparenright}{\isachardoublequoteclose}\isanewline
\ \ \ \ \isacommand{by}\isamarkupfalse%
\ {\isacharparenleft}rule\ sall{\isacharunderscore}simps{\isacharunderscore}not{\isacharunderscore}all{\isacharparenright}\isanewline
\ \ \isacommand{then}\isamarkupfalse%
\ \isacommand{have}\isamarkupfalse%
\ Ex{\isadigit{1}}{\isacharcolon}{\isachardoublequoteopen}{\isasymexists}Wo\ {\isasymsubseteq}\ {\isacharbraceleft}Gi{\isacharbraceright}\ {\isasymunion}\ W{\isachardot}\ finite\ Wo\ {\isasymand}\ {\isasymnot}{\isacharparenleft}sat\ Wo{\isacharparenright}{\isachardoublequoteclose}\isanewline
\ \ \ \ \isacommand{by}\isamarkupfalse%
\ {\isacharparenleft}simp\ only{\isacharcolon}\ not{\isacharunderscore}imp{\isacharparenright}\isanewline
\ \ \isacommand{obtain}\isamarkupfalse%
\ Wo\ \isakeyword{where}\ {\isachardoublequoteopen}Wo\ {\isasymsubseteq}\ {\isacharbraceleft}Gi{\isacharbraceright}\ {\isasymunion}\ W{\isachardoublequoteclose}\ \isakeyword{and}\ C{\isadigit{1}}{\isacharcolon}{\isachardoublequoteopen}finite\ Wo\ {\isasymand}\ {\isasymnot}{\isacharparenleft}sat\ Wo{\isacharparenright}{\isachardoublequoteclose}\isanewline
\ \ \ \ \isacommand{using}\isamarkupfalse%
\ Ex{\isadigit{1}}\ \isacommand{by}\isamarkupfalse%
\ {\isacharparenleft}rule\ subexE{\isacharparenright}\isanewline
\ \ \isacommand{have}\isamarkupfalse%
\ {\isachardoublequoteopen}finite\ Wo{\isachardoublequoteclose}\isanewline
\ \ \ \ \isacommand{using}\isamarkupfalse%
\ C{\isadigit{1}}\ \isacommand{by}\isamarkupfalse%
\ {\isacharparenleft}rule\ conjunct{\isadigit{1}}{\isacharparenright}\isanewline
\ \ \isacommand{have}\isamarkupfalse%
\ {\isachardoublequoteopen}{\isasymnot}{\isacharparenleft}sat\ Wo{\isacharparenright}{\isachardoublequoteclose}\isanewline
\ \ \ \ \isacommand{using}\isamarkupfalse%
\ C{\isadigit{1}}\ \isacommand{by}\isamarkupfalse%
\ {\isacharparenleft}rule\ conjunct{\isadigit{2}}{\isacharparenright}\isanewline
\ \ \isacommand{have}\isamarkupfalse%
\ {\isachardoublequoteopen}Wo\ {\isasymsubseteq}\ insert\ Gi\ W{\isachardoublequoteclose}\isanewline
\ \ \ \ \isacommand{using}\isamarkupfalse%
\ {\isacartoucheopen}Wo\ {\isasymsubseteq}\ {\isacharbraceleft}Gi{\isacharbraceright}\ {\isasymunion}\ W{\isacartoucheclose}\ \isacommand{by}\isamarkupfalse%
\ blast\isanewline
\ \ \isacommand{have}\isamarkupfalse%
\ Ex{\isadigit{2}}{\isacharcolon}{\isachardoublequoteopen}{\isasymexists}Wo{\isacharprime}\ {\isasymsubseteq}\ W{\isachardot}\ finite\ Wo{\isacharprime}\ {\isasymand}\ {\isacharparenleft}Wo\ {\isacharequal}\ insert\ Gi\ Wo{\isacharprime}\ {\isasymor}\ Wo\ {\isacharequal}\ Wo{\isacharprime}{\isacharparenright}{\isachardoublequoteclose}\isanewline
\ \ \ \ \isacommand{using}\isamarkupfalse%
\ {\isacartoucheopen}finite\ Wo{\isacartoucheclose}\ {\isacartoucheopen}Wo\ {\isasymsubseteq}\ insert\ Gi\ W{\isacartoucheclose}\ \isacommand{by}\isamarkupfalse%
\ {\isacharparenleft}rule\ finite{\isacharunderscore}subset{\isacharunderscore}insert{\isadigit{1}}{\isacharparenright}\isanewline
\ \ \isacommand{obtain}\isamarkupfalse%
\ Wo{\isacharprime}\ \isakeyword{where}\ {\isachardoublequoteopen}Wo{\isacharprime}\ {\isasymsubseteq}\ W{\isachardoublequoteclose}\ \isakeyword{and}\ C{\isadigit{2}}{\isacharcolon}{\isachardoublequoteopen}finite\ Wo{\isacharprime}\ {\isasymand}\ {\isacharparenleft}Wo\ {\isacharequal}\ {\isacharbraceleft}Gi{\isacharbraceright}\ {\isasymunion}\ Wo{\isacharprime}\ {\isasymor}\ Wo\ {\isacharequal}\ Wo{\isacharprime}{\isacharparenright}{\isachardoublequoteclose}\isanewline
\ \ \ \ \isacommand{using}\isamarkupfalse%
\ Ex{\isadigit{2}}\ \isacommand{by}\isamarkupfalse%
\ blast\isanewline
\ \ \isacommand{have}\isamarkupfalse%
\ {\isachardoublequoteopen}finite\ Wo{\isacharprime}{\isachardoublequoteclose}\isanewline
\ \ \ \ \isacommand{using}\isamarkupfalse%
\ C{\isadigit{2}}\ \isacommand{by}\isamarkupfalse%
\ {\isacharparenleft}rule\ conjunct{\isadigit{1}}{\isacharparenright}\isanewline
\ \ \isacommand{have}\isamarkupfalse%
\ {\isachardoublequoteopen}Wo\ {\isacharequal}\ {\isacharbraceleft}Gi{\isacharbraceright}\ {\isasymunion}\ Wo{\isacharprime}\ {\isasymor}\ Wo\ {\isacharequal}\ Wo{\isacharprime}{\isachardoublequoteclose}\isanewline
\ \ \ \ \isacommand{using}\isamarkupfalse%
\ C{\isadigit{2}}\ \isacommand{by}\isamarkupfalse%
\ {\isacharparenleft}rule\ conjunct{\isadigit{2}}{\isacharparenright}\isanewline
\ \ \isacommand{thus}\isamarkupfalse%
\ {\isacharquery}thesis\isanewline
\ \ \isacommand{proof}\isamarkupfalse%
\ {\isacharparenleft}rule\ disjE{\isacharparenright}\isanewline
\ \ \ \ \isacommand{assume}\isamarkupfalse%
\ {\isachardoublequoteopen}Wo\ {\isacharequal}\ {\isacharbraceleft}Gi{\isacharbraceright}\ {\isasymunion}\ Wo{\isacharprime}{\isachardoublequoteclose}\isanewline
\ \ \ \ \isacommand{then}\isamarkupfalse%
\ \isacommand{have}\isamarkupfalse%
\ {\isachardoublequoteopen}{\isasymnot}{\isacharparenleft}sat\ {\isacharparenleft}{\isacharbraceleft}Gi{\isacharbraceright}\ {\isasymunion}\ Wo{\isacharprime}{\isacharparenright}{\isacharparenright}{\isachardoublequoteclose}\ \isanewline
\ \ \ \ \ \ \isacommand{using}\isamarkupfalse%
\ {\isacartoucheopen}{\isasymnot}\ sat\ Wo{\isacartoucheclose}\ \isacommand{by}\isamarkupfalse%
\ {\isacharparenleft}simp\ only{\isacharcolon}\ {\isacartoucheopen}Wo\ {\isacharequal}\ {\isacharbraceleft}Gi{\isacharbraceright}\ {\isasymunion}\ Wo{\isacharprime}{\isacartoucheclose}\ simp{\isacharunderscore}thms{\isacharparenleft}{\isadigit{8}}{\isacharparenright}{\isacharparenright}\isanewline
\ \ \ \ \isacommand{have}\isamarkupfalse%
\ {\isachardoublequoteopen}finite\ Wo{\isacharprime}\ {\isasymand}\ {\isasymnot}{\isacharparenleft}sat\ {\isacharparenleft}{\isacharbraceleft}Gi{\isacharbraceright}\ {\isasymunion}\ Wo{\isacharprime}{\isacharparenright}{\isacharparenright}{\isachardoublequoteclose}\isanewline
\ \ \ \ \ \ \isacommand{using}\isamarkupfalse%
\ {\isacartoucheopen}finite\ Wo{\isacharprime}{\isacartoucheclose}\ {\isacartoucheopen}{\isasymnot}{\isacharparenleft}sat\ {\isacharparenleft}{\isacharbraceleft}Gi{\isacharbraceright}\ {\isasymunion}\ Wo{\isacharprime}{\isacharparenright}{\isacharparenright}{\isacartoucheclose}\ \isacommand{by}\isamarkupfalse%
\ {\isacharparenleft}rule\ conjI{\isacharparenright}\isanewline
\ \ \ \ \isacommand{thus}\isamarkupfalse%
\ {\isacharquery}thesis\isanewline
\ \ \ \ \ \ \isacommand{using}\isamarkupfalse%
\ {\isacartoucheopen}Wo{\isacharprime}\ {\isasymsubseteq}\ W{\isacartoucheclose}\ \isacommand{by}\isamarkupfalse%
\ {\isacharparenleft}rule\ subexI{\isacharparenright}\isanewline
\ \ \isacommand{next}\isamarkupfalse%
\isanewline
\ \ \ \ \isacommand{assume}\isamarkupfalse%
\ {\isachardoublequoteopen}Wo\ {\isacharequal}\ Wo{\isacharprime}{\isachardoublequoteclose}\isanewline
\ \ \ \ \isacommand{then}\isamarkupfalse%
\ \isacommand{have}\isamarkupfalse%
\ {\isachardoublequoteopen}{\isasymnot}\ {\isacharparenleft}sat\ Wo{\isacharprime}{\isacharparenright}{\isachardoublequoteclose}\isanewline
\ \ \ \ \ \ \isacommand{using}\isamarkupfalse%
\ {\isacartoucheopen}{\isasymnot}\ sat\ Wo{\isacartoucheclose}\ \isacommand{by}\isamarkupfalse%
\ {\isacharparenleft}simp\ only{\isacharcolon}\ {\isacartoucheopen}Wo\ {\isacharequal}\ Wo{\isacharprime}{\isacartoucheclose}\ simp{\isacharunderscore}thms{\isacharparenleft}{\isadigit{8}}{\isacharparenright}{\isacharparenright}\isanewline
\ \ \ \ \isacommand{have}\isamarkupfalse%
\ {\isachardoublequoteopen}Wo{\isacharprime}\ {\isasymsubseteq}\ {\isacharbraceleft}Gi{\isacharbraceright}\ {\isasymunion}\ Wo{\isacharprime}{\isachardoublequoteclose}\isanewline
\ \ \ \ \ \ \isacommand{by}\isamarkupfalse%
\ blast\isanewline
\ \ \ \ \isacommand{then}\isamarkupfalse%
\ \isacommand{have}\isamarkupfalse%
\ {\isachardoublequoteopen}{\isasymnot}\ {\isacharparenleft}sat\ {\isacharparenleft}{\isacharbraceleft}Gi{\isacharbraceright}\ {\isasymunion}\ Wo{\isacharprime}{\isacharparenright}{\isacharparenright}{\isachardoublequoteclose}\isanewline
\ \ \ \ \ \ \isacommand{using}\isamarkupfalse%
\ {\isacartoucheopen}{\isasymnot}\ {\isacharparenleft}sat\ Wo{\isacharprime}{\isacharparenright}{\isacartoucheclose}\ \isacommand{by}\isamarkupfalse%
\ {\isacharparenleft}rule\ sat{\isacharunderscore}subset{\isacharunderscore}ccontr{\isacharparenright}\isanewline
\ \ \ \ \isacommand{have}\isamarkupfalse%
\ {\isachardoublequoteopen}finite\ Wo{\isacharprime}\ {\isasymand}\ {\isasymnot}{\isacharparenleft}sat\ {\isacharparenleft}{\isacharbraceleft}Gi{\isacharbraceright}\ {\isasymunion}\ Wo{\isacharprime}{\isacharparenright}{\isacharparenright}{\isachardoublequoteclose}\isanewline
\ \ \ \ \ \ \isacommand{using}\isamarkupfalse%
\ {\isacartoucheopen}finite\ Wo{\isacharprime}{\isacartoucheclose}\ {\isacartoucheopen}{\isasymnot}{\isacharparenleft}sat\ {\isacharparenleft}{\isacharbraceleft}Gi{\isacharbraceright}\ {\isasymunion}\ Wo{\isacharprime}{\isacharparenright}{\isacharparenright}{\isacartoucheclose}\ \isacommand{by}\isamarkupfalse%
\ {\isacharparenleft}rule\ conjI{\isacharparenright}\isanewline
\ \ \ \ \isacommand{thus}\isamarkupfalse%
\ {\isacharquery}thesis\isanewline
\ \ \ \ \ \ \isacommand{using}\isamarkupfalse%
\ {\isacartoucheopen}Wo{\isacharprime}\ {\isasymsubseteq}\ W{\isacartoucheclose}\ \isacommand{by}\isamarkupfalse%
\ {\isacharparenleft}rule\ subexI{\isacharparenright}\isanewline
\ \ \isacommand{qed}\isamarkupfalse%
\isanewline
\isacommand{qed}\isamarkupfalse%
%
\endisatagproof
{\isafoldproof}%
%
\isadelimproof
%
\endisadelimproof
%
\begin{isamarkuptext}%
Por otro lado, para demostrar la cuarta condición del lema \isa{{\isadigit{2}}{\isachardot}{\isadigit{0}}{\isachardot}{\isadigit{2}}} que demuestra que \isa{C} 
  verifica la propiedad de consistencia proposicional, precisaremos de un lema auxiliar que prueba 
  que dados \isa{W\ {\isasymin}\ C}, \isa{F} una fórmula de tipo \isa{{\isasymbeta}} y componentes \isa{{\isasymbeta}\isactrlsub {\isadigit{1}}} y \isa{{\isasymbeta}\isactrlsub {\isadigit{2}}} tal que \isa{F\ {\isasymin}\ W} y \isa{W\isactrlsub {\isadigit{0}}} un 
  subconjunto finito de \isa{W}, entonces se tiene que o bien \isa{{\isacharbraceleft}{\isasymbeta}\isactrlsub {\isadigit{1}}{\isacharcomma}F{\isacharbraceright}\ {\isasymunion}\ W\isactrlsub {\isadigit{0}}} es satisfacible o bien 
  \isa{{\isacharbraceleft}{\isasymbeta}\isactrlsub {\isadigit{2}}{\isacharcomma}F{\isacharbraceright}\ {\isasymunion}\ W\isactrlsub {\isadigit{0}}} es satisfacible. Vamos a probar que, en efecto, se tiene el resultado para cada tipo de fórmula \isa{{\isasymbeta}}.

  En primer lugar, probemos que dados \isa{W\ {\isasymin}\ C}, una fórmula \isa{F\ {\isacharequal}\ G\ {\isasymand}\ H} para ciertas fórmulas \isa{G} y 
  \isa{H} tal que \isa{F\ {\isasymin}\ W} y \isa{W\isactrlsub {\isadigit{0}}} un subconjunto finito de \isa{W}, entonces se tiene que o bien 
  \isa{{\isacharbraceleft}G{\isacharcomma}F{\isacharbraceright}\ {\isasymunion}\ W\isactrlsub {\isadigit{0}}} es satisfacible o bien \isa{{\isacharbraceleft}H{\isacharcomma}F{\isacharbraceright}\ {\isasymunion}\ W\isactrlsub {\isadigit{0}}} es satisfacible.%
\end{isamarkuptext}\isamarkuptrue%
\isacommand{lemma}\isamarkupfalse%
\ pcp{\isacharunderscore}colecComp{\isacharunderscore}DIS{\isacharunderscore}sat{\isadigit{1}}{\isacharcolon}\isanewline
\ \ \isakeyword{assumes}\ {\isachardoublequoteopen}W\ {\isasymin}\ colecComp{\isachardoublequoteclose}\isanewline
\ \ \ \ \ \ \ \ \ \ {\isachardoublequoteopen}F\ {\isacharequal}\ G\ \isactrlbold {\isasymor}\ H{\isachardoublequoteclose}\isanewline
\ \ \ \ \ \ \ \ \ \ {\isachardoublequoteopen}F\ {\isasymin}\ W{\isachardoublequoteclose}\isanewline
\ \ \ \ \ \ \ \ \ \ {\isachardoublequoteopen}finite\ Wo{\isachardoublequoteclose}\isanewline
\ \ \ \ \ \ \ \ \ \ {\isachardoublequoteopen}Wo\ {\isasymsubseteq}\ W{\isachardoublequoteclose}\isanewline
\ \ \ \ \ \ \ \ \isakeyword{shows}\ {\isachardoublequoteopen}sat\ {\isacharparenleft}{\isacharbraceleft}G{\isacharcomma}F{\isacharbraceright}\ {\isasymunion}\ Wo{\isacharparenright}\ {\isasymor}\ sat\ {\isacharparenleft}{\isacharbraceleft}H{\isacharcomma}F{\isacharbraceright}\ {\isasymunion}\ Wo{\isacharparenright}{\isachardoublequoteclose}\isanewline
%
\isadelimproof
%
\endisadelimproof
%
\isatagproof
\isacommand{proof}\isamarkupfalse%
\ {\isacharminus}\isanewline
\ \ \isacommand{have}\isamarkupfalse%
\ {\isachardoublequoteopen}sat\ {\isacharparenleft}{\isacharbraceleft}F{\isacharbraceright}\ {\isasymunion}\ Wo{\isacharparenright}{\isachardoublequoteclose}\isanewline
\ \ \ \ \isacommand{using}\isamarkupfalse%
\ assms{\isacharparenleft}{\isadigit{1}}{\isacharcomma}{\isadigit{3}}{\isacharcomma}{\isadigit{4}}{\isacharcomma}{\isadigit{5}}{\isacharparenright}\ \isacommand{by}\isamarkupfalse%
\ {\isacharparenleft}rule\ pcp{\isacharunderscore}colecComp{\isacharunderscore}elem{\isacharunderscore}sat{\isacharparenright}\isanewline
\ \ \isacommand{have}\isamarkupfalse%
\ {\isachardoublequoteopen}F\ {\isasymin}\ {\isacharbraceleft}F{\isacharbraceright}\ {\isasymunion}\ Wo{\isachardoublequoteclose}\isanewline
\ \ \ \ \isacommand{by}\isamarkupfalse%
\ simp\ \isanewline
\ \ \isacommand{have}\isamarkupfalse%
\ Ex{\isadigit{1}}{\isacharcolon}{\isachardoublequoteopen}{\isasymexists}{\isasymA}{\isachardot}\ {\isasymforall}F\ {\isasymin}\ {\isacharparenleft}{\isacharbraceleft}F{\isacharbraceright}\ {\isasymunion}\ Wo{\isacharparenright}{\isachardot}\ {\isasymA}\ {\isasymTurnstile}\ F{\isachardoublequoteclose}\isanewline
\ \ \ \ \isacommand{using}\isamarkupfalse%
\ {\isacartoucheopen}sat\ {\isacharparenleft}{\isacharbraceleft}F{\isacharbraceright}\ {\isasymunion}\ Wo{\isacharparenright}{\isacartoucheclose}\ \isacommand{by}\isamarkupfalse%
\ {\isacharparenleft}simp\ only{\isacharcolon}\ sat{\isacharunderscore}def{\isacharparenright}\isanewline
\ \ \isacommand{obtain}\isamarkupfalse%
\ {\isasymA}\ \isakeyword{where}\ Forall{\isadigit{1}}{\isacharcolon}{\isachardoublequoteopen}{\isasymforall}F\ {\isasymin}\ {\isacharparenleft}{\isacharbraceleft}F{\isacharbraceright}\ {\isasymunion}\ Wo{\isacharparenright}{\isachardot}\ {\isasymA}\ {\isasymTurnstile}\ F{\isachardoublequoteclose}\isanewline
\ \ \ \ \isacommand{using}\isamarkupfalse%
\ Ex{\isadigit{1}}\ \isacommand{by}\isamarkupfalse%
\ {\isacharparenleft}rule\ exE{\isacharparenright}\isanewline
\ \ \isacommand{have}\isamarkupfalse%
\ {\isachardoublequoteopen}{\isasymA}\ {\isasymTurnstile}\ F{\isachardoublequoteclose}\isanewline
\ \ \ \ \isacommand{using}\isamarkupfalse%
\ Forall{\isadigit{1}}\ {\isacartoucheopen}F\ {\isasymin}\ {\isacharbraceleft}F{\isacharbraceright}\ {\isasymunion}\ Wo{\isacartoucheclose}\ \isacommand{by}\isamarkupfalse%
\ {\isacharparenleft}rule\ bspec{\isacharparenright}\isanewline
\ \ \isacommand{then}\isamarkupfalse%
\ \isacommand{have}\isamarkupfalse%
\ {\isachardoublequoteopen}{\isasymA}\ {\isasymTurnstile}\ {\isacharparenleft}G\ \isactrlbold {\isasymor}\ H{\isacharparenright}{\isachardoublequoteclose}\isanewline
\ \ \ \ \isacommand{using}\isamarkupfalse%
\ assms{\isacharparenleft}{\isadigit{2}}{\isacharparenright}\ \isacommand{by}\isamarkupfalse%
\ {\isacharparenleft}simp\ only{\isacharcolon}\ {\isacartoucheopen}{\isasymA}\ {\isasymTurnstile}\ F{\isacartoucheclose}{\isacharparenright}\isanewline
\ \ \isacommand{then}\isamarkupfalse%
\ \isacommand{have}\isamarkupfalse%
\ {\isachardoublequoteopen}{\isasymA}\ {\isasymTurnstile}\ G\ {\isasymor}\ {\isasymA}\ {\isasymTurnstile}\ H{\isachardoublequoteclose}\isanewline
\ \ \ \ \isacommand{by}\isamarkupfalse%
\ {\isacharparenleft}simp\ only{\isacharcolon}\ formula{\isacharunderscore}semantics{\isachardot}simps{\isacharparenleft}{\isadigit{5}}{\isacharparenright}{\isacharparenright}\isanewline
\ \ \isacommand{thus}\isamarkupfalse%
\ {\isacharquery}thesis\isanewline
\ \ \isacommand{proof}\isamarkupfalse%
\ {\isacharparenleft}rule\ disjE{\isacharparenright}\isanewline
\ \ \ \ \isacommand{assume}\isamarkupfalse%
\ {\isachardoublequoteopen}{\isasymA}\ {\isasymTurnstile}\ G{\isachardoublequoteclose}\isanewline
\ \ \ \ \isacommand{then}\isamarkupfalse%
\ \isacommand{have}\isamarkupfalse%
\ {\isachardoublequoteopen}{\isasymforall}F\ {\isasymin}\ {\isacharbraceleft}G{\isacharbraceright}{\isachardot}\ {\isasymA}\ {\isasymTurnstile}\ F{\isachardoublequoteclose}\isanewline
\ \ \ \ \ \ \isacommand{by}\isamarkupfalse%
\ simp\isanewline
\ \ \ \ \isacommand{then}\isamarkupfalse%
\ \isacommand{have}\isamarkupfalse%
\ {\isachardoublequoteopen}{\isasymforall}F\ {\isasymin}\ {\isacharparenleft}{\isacharbraceleft}G{\isacharbraceright}\ {\isasymunion}\ {\isacharparenleft}{\isacharbraceleft}F{\isacharbraceright}\ {\isasymunion}\ Wo{\isacharparenright}{\isacharparenright}{\isachardot}\ {\isasymA}\ {\isasymTurnstile}\ F{\isachardoublequoteclose}\isanewline
\ \ \ \ \ \ \isacommand{using}\isamarkupfalse%
\ Forall{\isadigit{1}}\ \isacommand{by}\isamarkupfalse%
\ {\isacharparenleft}rule\ ball{\isacharunderscore}Un{\isacharparenright}\isanewline
\ \ \ \ \isacommand{then}\isamarkupfalse%
\ \isacommand{have}\isamarkupfalse%
\ {\isachardoublequoteopen}{\isasymforall}F\ {\isasymin}\ {\isacharbraceleft}G{\isacharcomma}F{\isacharbraceright}\ {\isasymunion}\ Wo{\isachardot}\ {\isasymA}\ {\isasymTurnstile}\ F{\isachardoublequoteclose}\isanewline
\ \ \ \ \ \ \isacommand{by}\isamarkupfalse%
\ simp\ \isanewline
\ \ \ \ \isacommand{then}\isamarkupfalse%
\ \isacommand{have}\isamarkupfalse%
\ {\isachardoublequoteopen}{\isasymexists}{\isasymA}{\isachardot}\ {\isasymforall}F\ {\isasymin}\ {\isacharparenleft}{\isacharbraceleft}G{\isacharcomma}F{\isacharbraceright}\ {\isasymunion}\ Wo{\isacharparenright}{\isachardot}\ {\isasymA}\ {\isasymTurnstile}\ F{\isachardoublequoteclose}\isanewline
\ \ \ \ \ \ \isacommand{by}\isamarkupfalse%
\ {\isacharparenleft}iprover\ intro{\isacharcolon}\ exI{\isacharparenright}\isanewline
\ \ \ \ \isacommand{then}\isamarkupfalse%
\ \isacommand{have}\isamarkupfalse%
\ {\isachardoublequoteopen}sat\ {\isacharparenleft}{\isacharbraceleft}G{\isacharcomma}F{\isacharbraceright}\ {\isasymunion}\ Wo{\isacharparenright}{\isachardoublequoteclose}\isanewline
\ \ \ \ \ \ \isacommand{by}\isamarkupfalse%
\ {\isacharparenleft}simp\ only{\isacharcolon}\ sat{\isacharunderscore}def{\isacharparenright}\isanewline
\ \ \ \ \isacommand{thus}\isamarkupfalse%
\ {\isacharquery}thesis\isanewline
\ \ \ \ \ \ \isacommand{by}\isamarkupfalse%
\ {\isacharparenleft}rule\ disjI{\isadigit{1}}{\isacharparenright}\isanewline
\ \ \isacommand{next}\isamarkupfalse%
\isanewline
\ \ \ \ \isacommand{assume}\isamarkupfalse%
\ {\isachardoublequoteopen}{\isasymA}\ {\isasymTurnstile}\ H{\isachardoublequoteclose}\isanewline
\ \ \ \ \isacommand{then}\isamarkupfalse%
\ \isacommand{have}\isamarkupfalse%
\ {\isachardoublequoteopen}{\isasymforall}F\ {\isasymin}\ {\isacharbraceleft}H{\isacharbraceright}{\isachardot}\ {\isasymA}\ {\isasymTurnstile}\ F{\isachardoublequoteclose}\isanewline
\ \ \ \ \ \ \isacommand{by}\isamarkupfalse%
\ simp\isanewline
\ \ \ \ \isacommand{then}\isamarkupfalse%
\ \isacommand{have}\isamarkupfalse%
\ {\isachardoublequoteopen}{\isasymforall}F\ {\isasymin}\ {\isacharparenleft}{\isacharbraceleft}H{\isacharbraceright}\ {\isasymunion}\ {\isacharparenleft}{\isacharbraceleft}F{\isacharbraceright}\ {\isasymunion}\ Wo{\isacharparenright}{\isacharparenright}{\isachardot}\ {\isasymA}\ {\isasymTurnstile}\ F{\isachardoublequoteclose}\isanewline
\ \ \ \ \ \ \isacommand{using}\isamarkupfalse%
\ Forall{\isadigit{1}}\ \isacommand{by}\isamarkupfalse%
\ {\isacharparenleft}rule\ ball{\isacharunderscore}Un{\isacharparenright}\isanewline
\ \ \ \ \isacommand{then}\isamarkupfalse%
\ \isacommand{have}\isamarkupfalse%
\ {\isachardoublequoteopen}{\isasymforall}F\ {\isasymin}\ {\isacharbraceleft}H{\isacharcomma}F{\isacharbraceright}\ {\isasymunion}\ Wo{\isachardot}\ {\isasymA}\ {\isasymTurnstile}\ F{\isachardoublequoteclose}\isanewline
\ \ \ \ \ \ \isacommand{by}\isamarkupfalse%
\ simp\isanewline
\ \ \ \ \isacommand{then}\isamarkupfalse%
\ \isacommand{have}\isamarkupfalse%
\ {\isachardoublequoteopen}{\isasymexists}{\isasymA}{\isachardot}\ {\isasymforall}F\ {\isasymin}\ {\isacharparenleft}{\isacharbraceleft}H{\isacharcomma}F{\isacharbraceright}\ {\isasymunion}\ Wo{\isacharparenright}{\isachardot}\ {\isasymA}\ {\isasymTurnstile}\ F{\isachardoublequoteclose}\isanewline
\ \ \ \ \ \ \isacommand{by}\isamarkupfalse%
\ {\isacharparenleft}iprover\ intro{\isacharcolon}\ exI{\isacharparenright}\isanewline
\ \ \ \ \isacommand{then}\isamarkupfalse%
\ \isacommand{have}\isamarkupfalse%
\ {\isachardoublequoteopen}sat\ {\isacharparenleft}{\isacharbraceleft}H{\isacharcomma}F{\isacharbraceright}\ {\isasymunion}\ Wo{\isacharparenright}{\isachardoublequoteclose}\isanewline
\ \ \ \ \ \ \isacommand{by}\isamarkupfalse%
\ {\isacharparenleft}simp\ only{\isacharcolon}\ sat{\isacharunderscore}def{\isacharparenright}\isanewline
\ \ \ \ \isacommand{thus}\isamarkupfalse%
\ {\isacharquery}thesis\isanewline
\ \ \ \ \ \ \isacommand{by}\isamarkupfalse%
\ {\isacharparenleft}rule\ disjI{\isadigit{2}}{\isacharparenright}\isanewline
\ \ \isacommand{qed}\isamarkupfalse%
\isanewline
\isacommand{qed}\isamarkupfalse%
%
\endisatagproof
{\isafoldproof}%
%
\isadelimproof
%
\endisadelimproof
%
\begin{isamarkuptext}%
El siguiente lema auxiliar demuestra que dados \isa{W\ {\isasymin}\ C}, una fórmula \isa{F\ {\isacharequal}\ G\ {\isasymlongrightarrow}\ H} para ciertas 
  fórmulas \isa{G} y \isa{H} tal que \isa{F\ {\isasymin}\ W} y \isa{W\isactrlsub {\isadigit{0}}} un subconjunto finito de \isa{W}, entonces se tiene que o 
  bien \isa{{\isacharbraceleft}{\isasymnot}\ G{\isacharcomma}F{\isacharbraceright}\ {\isasymunion}\ W\isactrlsub {\isadigit{0}}} es satisfacible o bien \isa{{\isacharbraceleft}H{\isacharcomma}F{\isacharbraceright}\ {\isasymunion}\ W\isactrlsub {\isadigit{0}}} es satisfacible.%
\end{isamarkuptext}\isamarkuptrue%
\isacommand{lemma}\isamarkupfalse%
\ pcp{\isacharunderscore}colecComp{\isacharunderscore}DIS{\isacharunderscore}sat{\isadigit{2}}{\isacharcolon}\isanewline
\ \ \isakeyword{assumes}\ {\isachardoublequoteopen}W\ {\isasymin}\ colecComp{\isachardoublequoteclose}\isanewline
\ \ \ \ \ \ \ \ \ \ {\isachardoublequoteopen}F\ {\isacharequal}\ G\ \isactrlbold {\isasymrightarrow}\ H{\isachardoublequoteclose}\isanewline
\ \ \ \ \ \ \ \ \ \ {\isachardoublequoteopen}F\ {\isasymin}\ W{\isachardoublequoteclose}\isanewline
\ \ \ \ \ \ \ \ \ \ {\isachardoublequoteopen}finite\ Wo{\isachardoublequoteclose}\isanewline
\ \ \ \ \ \ \ \ \ \ {\isachardoublequoteopen}Wo\ {\isasymsubseteq}\ W{\isachardoublequoteclose}\isanewline
\ \ \ \ \ \ \ \ \isakeyword{shows}\ {\isachardoublequoteopen}sat\ {\isacharparenleft}{\isacharbraceleft}\isactrlbold {\isasymnot}\ G{\isacharcomma}F{\isacharbraceright}\ {\isasymunion}\ Wo{\isacharparenright}\ {\isasymor}\ sat\ {\isacharparenleft}{\isacharbraceleft}H{\isacharcomma}F{\isacharbraceright}\ {\isasymunion}\ Wo{\isacharparenright}{\isachardoublequoteclose}\isanewline
%
\isadelimproof
%
\endisadelimproof
%
\isatagproof
\isacommand{proof}\isamarkupfalse%
\ {\isacharminus}\isanewline
\ \ \isacommand{have}\isamarkupfalse%
\ {\isachardoublequoteopen}sat\ {\isacharparenleft}{\isacharbraceleft}F{\isacharbraceright}\ {\isasymunion}\ Wo{\isacharparenright}{\isachardoublequoteclose}\isanewline
\ \ \ \ \isacommand{using}\isamarkupfalse%
\ assms{\isacharparenleft}{\isadigit{1}}{\isacharcomma}{\isadigit{3}}{\isacharcomma}{\isadigit{4}}{\isacharcomma}{\isadigit{5}}{\isacharparenright}\ \isacommand{by}\isamarkupfalse%
\ {\isacharparenleft}rule\ pcp{\isacharunderscore}colecComp{\isacharunderscore}elem{\isacharunderscore}sat{\isacharparenright}\isanewline
\ \ \isacommand{have}\isamarkupfalse%
\ {\isachardoublequoteopen}F\ {\isasymin}\ {\isacharbraceleft}F{\isacharbraceright}\ {\isasymunion}\ Wo{\isachardoublequoteclose}\isanewline
\ \ \ \ \isacommand{by}\isamarkupfalse%
\ simp\isanewline
\ \ \isacommand{have}\isamarkupfalse%
\ Ex{\isadigit{1}}{\isacharcolon}{\isachardoublequoteopen}{\isasymexists}{\isasymA}{\isachardot}\ {\isasymforall}F\ {\isasymin}\ {\isacharparenleft}{\isacharbraceleft}F{\isacharbraceright}\ {\isasymunion}\ Wo{\isacharparenright}{\isachardot}\ {\isasymA}\ {\isasymTurnstile}\ F{\isachardoublequoteclose}\isanewline
\ \ \ \ \isacommand{using}\isamarkupfalse%
\ {\isacartoucheopen}sat\ {\isacharparenleft}{\isacharbraceleft}F{\isacharbraceright}\ {\isasymunion}\ Wo{\isacharparenright}{\isacartoucheclose}\ \isacommand{by}\isamarkupfalse%
\ {\isacharparenleft}simp\ only{\isacharcolon}\ sat{\isacharunderscore}def{\isacharparenright}\isanewline
\ \ \isacommand{obtain}\isamarkupfalse%
\ {\isasymA}\ \isakeyword{where}\ Forall{\isadigit{1}}{\isacharcolon}{\isachardoublequoteopen}{\isasymforall}F\ {\isasymin}\ {\isacharparenleft}{\isacharbraceleft}F{\isacharbraceright}\ {\isasymunion}\ Wo{\isacharparenright}{\isachardot}\ {\isasymA}\ {\isasymTurnstile}\ F{\isachardoublequoteclose}\isanewline
\ \ \ \ \isacommand{using}\isamarkupfalse%
\ Ex{\isadigit{1}}\ \isacommand{by}\isamarkupfalse%
\ {\isacharparenleft}rule\ exE{\isacharparenright}\isanewline
\ \ \isacommand{have}\isamarkupfalse%
\ {\isachardoublequoteopen}{\isasymA}\ {\isasymTurnstile}\ F{\isachardoublequoteclose}\isanewline
\ \ \ \ \isacommand{using}\isamarkupfalse%
\ Forall{\isadigit{1}}\ {\isacartoucheopen}F\ {\isasymin}\ {\isacharbraceleft}F{\isacharbraceright}\ {\isasymunion}\ Wo{\isacartoucheclose}\ \isacommand{by}\isamarkupfalse%
\ {\isacharparenleft}rule\ bspec{\isacharparenright}\isanewline
\ \ \isacommand{then}\isamarkupfalse%
\ \isacommand{have}\isamarkupfalse%
\ {\isachardoublequoteopen}{\isasymA}\ {\isasymTurnstile}\ {\isacharparenleft}G\ \isactrlbold {\isasymrightarrow}\ H{\isacharparenright}{\isachardoublequoteclose}\isanewline
\ \ \ \ \isacommand{using}\isamarkupfalse%
\ assms{\isacharparenleft}{\isadigit{2}}{\isacharparenright}\ \isacommand{by}\isamarkupfalse%
\ {\isacharparenleft}simp\ only{\isacharcolon}\ {\isacartoucheopen}{\isasymA}\ {\isasymTurnstile}\ F{\isacartoucheclose}{\isacharparenright}\isanewline
\ \ \isacommand{then}\isamarkupfalse%
\ \isacommand{have}\isamarkupfalse%
\ {\isachardoublequoteopen}{\isasymA}\ {\isasymTurnstile}\ G\ {\isasymlongrightarrow}\ {\isasymA}\ {\isasymTurnstile}\ H{\isachardoublequoteclose}\isanewline
\ \ \ \ \isacommand{by}\isamarkupfalse%
\ {\isacharparenleft}simp\ only{\isacharcolon}\ formula{\isacharunderscore}semantics{\isachardot}simps{\isacharparenleft}{\isadigit{6}}{\isacharparenright}{\isacharparenright}\isanewline
\ \ \isacommand{then}\isamarkupfalse%
\ \isacommand{have}\isamarkupfalse%
\ {\isachardoublequoteopen}{\isacharparenleft}{\isasymnot}{\isacharparenleft}{\isasymnot}\ {\isasymA}\ {\isasymTurnstile}\ G{\isacharparenright}{\isacharparenright}\ {\isasymlongrightarrow}\ {\isasymA}\ {\isasymTurnstile}\ H{\isachardoublequoteclose}\isanewline
\ \ \ \ \isacommand{by}\isamarkupfalse%
\ {\isacharparenleft}simp\ only{\isacharcolon}\ not{\isacharunderscore}not{\isacharparenright}\isanewline
\ \ \isacommand{then}\isamarkupfalse%
\ \isacommand{have}\isamarkupfalse%
\ {\isachardoublequoteopen}{\isacharparenleft}{\isasymnot}\ {\isasymA}\ {\isasymTurnstile}\ G{\isacharparenright}\ {\isasymor}\ {\isasymA}\ {\isasymTurnstile}\ H{\isachardoublequoteclose}\isanewline
\ \ \ \ \isacommand{by}\isamarkupfalse%
\ {\isacharparenleft}simp\ only{\isacharcolon}\ disj{\isacharunderscore}imp{\isacharparenright}\isanewline
\ \ \isacommand{thus}\isamarkupfalse%
\ {\isacharquery}thesis\isanewline
\ \ \isacommand{proof}\isamarkupfalse%
\ {\isacharparenleft}rule\ disjE{\isacharparenright}\isanewline
\ \ \ \ \isacommand{assume}\isamarkupfalse%
\ {\isachardoublequoteopen}{\isasymnot}\ {\isasymA}\ {\isasymTurnstile}\ G{\isachardoublequoteclose}\isanewline
\ \ \ \ \isacommand{then}\isamarkupfalse%
\ \isacommand{have}\isamarkupfalse%
\ {\isachardoublequoteopen}{\isasymA}\ {\isasymTurnstile}\ {\isacharparenleft}\isactrlbold {\isasymnot}\ G{\isacharparenright}{\isachardoublequoteclose}\isanewline
\ \ \ \ \ \ \isacommand{by}\isamarkupfalse%
\ {\isacharparenleft}simp\ only{\isacharcolon}\ formula{\isacharunderscore}semantics{\isachardot}simps{\isacharparenleft}{\isadigit{3}}{\isacharparenright}\ simp{\isacharunderscore}thms{\isacharparenleft}{\isadigit{8}}{\isacharparenright}{\isacharparenright}\isanewline
\ \ \ \ \isacommand{then}\isamarkupfalse%
\ \isacommand{have}\isamarkupfalse%
\ {\isachardoublequoteopen}{\isasymforall}F\ {\isasymin}\ {\isacharbraceleft}\isactrlbold {\isasymnot}\ G{\isacharbraceright}{\isachardot}\ {\isasymA}\ {\isasymTurnstile}\ F{\isachardoublequoteclose}\isanewline
\ \ \ \ \ \ \isacommand{by}\isamarkupfalse%
\ simp\isanewline
\ \ \ \ \isacommand{then}\isamarkupfalse%
\ \isacommand{have}\isamarkupfalse%
\ {\isachardoublequoteopen}{\isasymforall}F\ {\isasymin}\ {\isacharparenleft}{\isacharbraceleft}\isactrlbold {\isasymnot}\ G{\isacharbraceright}\ {\isasymunion}\ {\isacharparenleft}{\isacharbraceleft}F{\isacharbraceright}\ {\isasymunion}\ Wo{\isacharparenright}{\isacharparenright}{\isachardot}\ {\isasymA}\ {\isasymTurnstile}\ F{\isachardoublequoteclose}\isanewline
\ \ \ \ \ \ \isacommand{using}\isamarkupfalse%
\ Forall{\isadigit{1}}\ \isacommand{by}\isamarkupfalse%
\ {\isacharparenleft}rule\ ball{\isacharunderscore}Un{\isacharparenright}\isanewline
\ \ \ \ \isacommand{then}\isamarkupfalse%
\ \isacommand{have}\isamarkupfalse%
\ {\isachardoublequoteopen}{\isasymforall}F\ {\isasymin}\ {\isacharbraceleft}\isactrlbold {\isasymnot}\ G{\isacharcomma}F{\isacharbraceright}\ {\isasymunion}\ Wo{\isachardot}\ {\isasymA}\ {\isasymTurnstile}\ F{\isachardoublequoteclose}\isanewline
\ \ \ \ \ \ \isacommand{by}\isamarkupfalse%
\ simp\isanewline
\ \ \ \ \isacommand{then}\isamarkupfalse%
\ \isacommand{have}\isamarkupfalse%
\ {\isachardoublequoteopen}{\isasymexists}{\isasymA}{\isachardot}\ {\isasymforall}F\ {\isasymin}\ {\isacharparenleft}{\isacharbraceleft}\isactrlbold {\isasymnot}\ G{\isacharcomma}F{\isacharbraceright}\ {\isasymunion}\ Wo{\isacharparenright}{\isachardot}\ {\isasymA}\ {\isasymTurnstile}\ F{\isachardoublequoteclose}\isanewline
\ \ \ \ \ \ \isacommand{by}\isamarkupfalse%
\ {\isacharparenleft}iprover\ intro{\isacharcolon}\ exI{\isacharparenright}\isanewline
\ \ \ \ \isacommand{then}\isamarkupfalse%
\ \isacommand{have}\isamarkupfalse%
\ {\isachardoublequoteopen}sat\ {\isacharparenleft}{\isacharbraceleft}\isactrlbold {\isasymnot}\ G{\isacharcomma}F{\isacharbraceright}\ {\isasymunion}\ Wo{\isacharparenright}{\isachardoublequoteclose}\isanewline
\ \ \ \ \ \ \isacommand{by}\isamarkupfalse%
\ {\isacharparenleft}simp\ only{\isacharcolon}\ sat{\isacharunderscore}def{\isacharparenright}\isanewline
\ \ \ \ \isacommand{thus}\isamarkupfalse%
\ {\isacharquery}thesis\isanewline
\ \ \ \ \ \ \isacommand{by}\isamarkupfalse%
\ {\isacharparenleft}rule\ disjI{\isadigit{1}}{\isacharparenright}\isanewline
\ \ \isacommand{next}\isamarkupfalse%
\isanewline
\ \ \ \ \isacommand{assume}\isamarkupfalse%
\ {\isachardoublequoteopen}{\isasymA}\ {\isasymTurnstile}\ H{\isachardoublequoteclose}\isanewline
\ \ \ \ \isacommand{then}\isamarkupfalse%
\ \isacommand{have}\isamarkupfalse%
\ {\isachardoublequoteopen}{\isasymforall}F\ {\isasymin}\ {\isacharbraceleft}H{\isacharbraceright}{\isachardot}\ {\isasymA}\ {\isasymTurnstile}\ F{\isachardoublequoteclose}\isanewline
\ \ \ \ \ \ \isacommand{by}\isamarkupfalse%
\ simp\isanewline
\ \ \ \ \isacommand{then}\isamarkupfalse%
\ \isacommand{have}\isamarkupfalse%
\ {\isachardoublequoteopen}{\isasymforall}F\ {\isasymin}\ {\isacharparenleft}{\isacharbraceleft}H{\isacharbraceright}\ {\isasymunion}\ {\isacharparenleft}{\isacharbraceleft}F{\isacharbraceright}\ {\isasymunion}\ Wo{\isacharparenright}{\isacharparenright}{\isachardot}\ {\isasymA}\ {\isasymTurnstile}\ F{\isachardoublequoteclose}\isanewline
\ \ \ \ \ \ \isacommand{using}\isamarkupfalse%
\ Forall{\isadigit{1}}\ \isacommand{by}\isamarkupfalse%
\ {\isacharparenleft}rule\ ball{\isacharunderscore}Un{\isacharparenright}\isanewline
\ \ \ \ \isacommand{then}\isamarkupfalse%
\ \isacommand{have}\isamarkupfalse%
\ {\isachardoublequoteopen}{\isasymforall}F\ {\isasymin}\ {\isacharbraceleft}H{\isacharcomma}F{\isacharbraceright}\ {\isasymunion}\ Wo{\isachardot}\ {\isasymA}\ {\isasymTurnstile}\ F{\isachardoublequoteclose}\isanewline
\ \ \ \ \ \ \isacommand{by}\isamarkupfalse%
\ simp\isanewline
\ \ \ \ \isacommand{then}\isamarkupfalse%
\ \isacommand{have}\isamarkupfalse%
\ {\isachardoublequoteopen}{\isasymexists}{\isasymA}{\isachardot}\ {\isasymforall}F\ {\isasymin}\ {\isacharparenleft}{\isacharbraceleft}H{\isacharcomma}F{\isacharbraceright}\ {\isasymunion}\ Wo{\isacharparenright}{\isachardot}\ {\isasymA}\ {\isasymTurnstile}\ F{\isachardoublequoteclose}\isanewline
\ \ \ \ \ \ \isacommand{by}\isamarkupfalse%
\ {\isacharparenleft}iprover\ intro{\isacharcolon}\ exI{\isacharparenright}\isanewline
\ \ \ \ \isacommand{then}\isamarkupfalse%
\ \isacommand{have}\isamarkupfalse%
\ {\isachardoublequoteopen}sat\ {\isacharparenleft}{\isacharbraceleft}H{\isacharcomma}F{\isacharbraceright}\ {\isasymunion}\ Wo{\isacharparenright}{\isachardoublequoteclose}\isanewline
\ \ \ \ \ \ \isacommand{by}\isamarkupfalse%
\ {\isacharparenleft}simp\ only{\isacharcolon}\ sat{\isacharunderscore}def{\isacharparenright}\isanewline
\ \ \ \ \isacommand{thus}\isamarkupfalse%
\ {\isacharquery}thesis\isanewline
\ \ \ \ \ \ \isacommand{by}\isamarkupfalse%
\ {\isacharparenleft}rule\ disjI{\isadigit{2}}{\isacharparenright}\isanewline
\ \ \isacommand{qed}\isamarkupfalse%
\isanewline
\isacommand{qed}\isamarkupfalse%
%
\endisatagproof
{\isafoldproof}%
%
\isadelimproof
%
\endisadelimproof
%
\begin{isamarkuptext}%
Por otro lado probemos que dados \isa{W\ {\isasymin}\ C}, una fórmula \isa{F\ {\isacharequal}\ {\isasymnot}{\isacharparenleft}G\ {\isasymand}\ H{\isacharparenright}} para ciertas fórmulas 
  \isa{G} y \isa{H} tal que \isa{F\ {\isasymin}\ W} y \isa{W\isactrlsub {\isadigit{0}}} un subconjunto finito de \isa{W}, entonces se tiene que o bien 
  \isa{{\isacharbraceleft}{\isasymnot}\ G{\isacharcomma}F{\isacharbraceright}\ {\isasymunion}\ W\isactrlsub {\isadigit{0}}} es satisfacible o bien \isa{{\isacharbraceleft}{\isasymnot}\ H{\isacharcomma}F{\isacharbraceright}\ {\isasymunion}\ W\isactrlsub {\isadigit{0}}} es satisfacible.%
\end{isamarkuptext}\isamarkuptrue%
\isacommand{lemma}\isamarkupfalse%
\ pcp{\isacharunderscore}colecComp{\isacharunderscore}DIS{\isacharunderscore}sat{\isadigit{3}}{\isacharcolon}\isanewline
\ \ \isakeyword{assumes}\ {\isachardoublequoteopen}W\ {\isasymin}\ colecComp{\isachardoublequoteclose}\isanewline
\ \ \ \ \ \ \ \ \ \ {\isachardoublequoteopen}F\ {\isacharequal}\ \isactrlbold {\isasymnot}\ {\isacharparenleft}G\ \isactrlbold {\isasymand}\ H{\isacharparenright}{\isachardoublequoteclose}\isanewline
\ \ \ \ \ \ \ \ \ \ {\isachardoublequoteopen}F\ {\isasymin}\ W{\isachardoublequoteclose}\isanewline
\ \ \ \ \ \ \ \ \ \ {\isachardoublequoteopen}finite\ Wo{\isachardoublequoteclose}\isanewline
\ \ \ \ \ \ \ \ \ \ {\isachardoublequoteopen}Wo\ {\isasymsubseteq}\ W{\isachardoublequoteclose}\isanewline
\ \ \ \ \ \ \ \ \isakeyword{shows}\ {\isachardoublequoteopen}sat\ {\isacharparenleft}{\isacharbraceleft}\isactrlbold {\isasymnot}\ G{\isacharcomma}F{\isacharbraceright}\ {\isasymunion}\ Wo{\isacharparenright}\ {\isasymor}\ sat\ {\isacharparenleft}{\isacharbraceleft}\isactrlbold {\isasymnot}\ H{\isacharcomma}F{\isacharbraceright}\ {\isasymunion}\ Wo{\isacharparenright}{\isachardoublequoteclose}\isanewline
%
\isadelimproof
%
\endisadelimproof
%
\isatagproof
\isacommand{proof}\isamarkupfalse%
\ {\isacharminus}\isanewline
\ \ \isacommand{have}\isamarkupfalse%
\ {\isachardoublequoteopen}sat\ {\isacharparenleft}{\isacharbraceleft}F{\isacharbraceright}\ {\isasymunion}\ Wo{\isacharparenright}{\isachardoublequoteclose}\isanewline
\ \ \ \ \isacommand{using}\isamarkupfalse%
\ assms{\isacharparenleft}{\isadigit{1}}{\isacharcomma}{\isadigit{3}}{\isacharcomma}{\isadigit{4}}{\isacharcomma}{\isadigit{5}}{\isacharparenright}\ \isacommand{by}\isamarkupfalse%
\ {\isacharparenleft}rule\ pcp{\isacharunderscore}colecComp{\isacharunderscore}elem{\isacharunderscore}sat{\isacharparenright}\isanewline
\ \ \isacommand{have}\isamarkupfalse%
\ {\isachardoublequoteopen}F\ {\isasymin}\ {\isacharbraceleft}F{\isacharbraceright}\ {\isasymunion}\ Wo{\isachardoublequoteclose}\isanewline
\ \ \ \ \isacommand{by}\isamarkupfalse%
\ simp\isanewline
\ \ \isacommand{have}\isamarkupfalse%
\ Ex{\isadigit{1}}{\isacharcolon}{\isachardoublequoteopen}{\isasymexists}{\isasymA}{\isachardot}\ {\isasymforall}F\ {\isasymin}\ {\isacharparenleft}{\isacharbraceleft}F{\isacharbraceright}\ {\isasymunion}\ Wo{\isacharparenright}{\isachardot}\ {\isasymA}\ {\isasymTurnstile}\ F{\isachardoublequoteclose}\isanewline
\ \ \ \ \isacommand{using}\isamarkupfalse%
\ {\isacartoucheopen}sat\ {\isacharparenleft}{\isacharbraceleft}F{\isacharbraceright}\ {\isasymunion}\ Wo{\isacharparenright}{\isacartoucheclose}\ \isacommand{by}\isamarkupfalse%
\ {\isacharparenleft}simp\ only{\isacharcolon}\ sat{\isacharunderscore}def{\isacharparenright}\isanewline
\ \ \isacommand{obtain}\isamarkupfalse%
\ {\isasymA}\ \isakeyword{where}\ Forall{\isadigit{1}}{\isacharcolon}{\isachardoublequoteopen}{\isasymforall}F\ {\isasymin}\ {\isacharparenleft}{\isacharbraceleft}F{\isacharbraceright}\ {\isasymunion}\ Wo{\isacharparenright}{\isachardot}\ {\isasymA}\ {\isasymTurnstile}\ F{\isachardoublequoteclose}\isanewline
\ \ \ \ \isacommand{using}\isamarkupfalse%
\ Ex{\isadigit{1}}\ \isacommand{by}\isamarkupfalse%
\ {\isacharparenleft}rule\ exE{\isacharparenright}\isanewline
\ \ \isacommand{have}\isamarkupfalse%
\ {\isachardoublequoteopen}{\isasymA}\ {\isasymTurnstile}\ F{\isachardoublequoteclose}\isanewline
\ \ \ \ \isacommand{using}\isamarkupfalse%
\ Forall{\isadigit{1}}\ {\isacartoucheopen}F\ {\isasymin}\ {\isacharbraceleft}F{\isacharbraceright}\ {\isasymunion}\ Wo{\isacartoucheclose}\ \isacommand{by}\isamarkupfalse%
\ {\isacharparenleft}rule\ bspec{\isacharparenright}\isanewline
\ \ \isacommand{then}\isamarkupfalse%
\ \isacommand{have}\isamarkupfalse%
\ {\isachardoublequoteopen}{\isasymA}\ {\isasymTurnstile}\ \isactrlbold {\isasymnot}\ {\isacharparenleft}G\ \isactrlbold {\isasymand}\ H{\isacharparenright}{\isachardoublequoteclose}\isanewline
\ \ \ \ \isacommand{using}\isamarkupfalse%
\ assms{\isacharparenleft}{\isadigit{2}}{\isacharparenright}\ \isacommand{by}\isamarkupfalse%
\ {\isacharparenleft}simp\ only{\isacharcolon}\ {\isacartoucheopen}{\isasymA}\ {\isasymTurnstile}\ F{\isacartoucheclose}{\isacharparenright}\isanewline
\ \ \isacommand{then}\isamarkupfalse%
\ \isacommand{have}\isamarkupfalse%
\ {\isachardoublequoteopen}{\isasymnot}\ {\isacharparenleft}{\isasymA}\ {\isasymTurnstile}\ {\isacharparenleft}G\ \isactrlbold {\isasymand}\ H{\isacharparenright}{\isacharparenright}{\isachardoublequoteclose}\isanewline
\ \ \ \ \isacommand{by}\isamarkupfalse%
\ {\isacharparenleft}simp\ only{\isacharcolon}\ formula{\isacharunderscore}semantics{\isachardot}simps{\isacharparenleft}{\isadigit{3}}{\isacharparenright}\ simp{\isacharunderscore}thms{\isacharparenleft}{\isadigit{8}}{\isacharparenright}{\isacharparenright}\isanewline
\ \ \isacommand{then}\isamarkupfalse%
\ \isacommand{have}\isamarkupfalse%
\ {\isachardoublequoteopen}{\isasymnot}{\isacharparenleft}{\isasymA}\ {\isasymTurnstile}\ G\ {\isasymand}\ {\isasymA}\ {\isasymTurnstile}\ H{\isacharparenright}{\isachardoublequoteclose}\isanewline
\ \ \ \ \isacommand{by}\isamarkupfalse%
\ {\isacharparenleft}simp\ only{\isacharcolon}\ formula{\isacharunderscore}semantics{\isachardot}simps{\isacharparenleft}{\isadigit{4}}{\isacharparenright}\ simp{\isacharunderscore}thms{\isacharparenleft}{\isadigit{8}}{\isacharparenright}{\isacharparenright}\isanewline
\ \ \isacommand{then}\isamarkupfalse%
\ \isacommand{have}\isamarkupfalse%
\ {\isachardoublequoteopen}{\isasymnot}\ {\isacharparenleft}{\isasymA}\ {\isasymTurnstile}\ G{\isacharparenright}\ {\isasymor}\ {\isasymnot}\ {\isacharparenleft}{\isasymA}\ {\isasymTurnstile}\ H{\isacharparenright}{\isachardoublequoteclose}\isanewline
\ \ \ \ \isacommand{by}\isamarkupfalse%
\ {\isacharparenleft}simp\ only{\isacharcolon}\ de{\isacharunderscore}Morgan{\isacharunderscore}conj{\isacharparenright}\isanewline
\ \ \isacommand{thus}\isamarkupfalse%
\ {\isacharquery}thesis\isanewline
\ \ \isacommand{proof}\isamarkupfalse%
\ {\isacharparenleft}rule\ disjE{\isacharparenright}\isanewline
\ \ \ \ \isacommand{assume}\isamarkupfalse%
\ {\isachardoublequoteopen}{\isasymnot}\ {\isacharparenleft}{\isasymA}\ {\isasymTurnstile}\ G{\isacharparenright}{\isachardoublequoteclose}\isanewline
\ \ \ \ \isacommand{then}\isamarkupfalse%
\ \isacommand{have}\isamarkupfalse%
\ {\isachardoublequoteopen}{\isasymA}\ {\isasymTurnstile}\ \isactrlbold {\isasymnot}\ G{\isachardoublequoteclose}\isanewline
\ \ \ \ \ \ \isacommand{by}\isamarkupfalse%
\ {\isacharparenleft}simp\ only{\isacharcolon}\ formula{\isacharunderscore}semantics{\isachardot}simps{\isacharparenleft}{\isadigit{3}}{\isacharparenright}\ simp{\isacharunderscore}thms{\isacharparenleft}{\isadigit{8}}{\isacharparenright}{\isacharparenright}\isanewline
\ \ \ \ \isacommand{then}\isamarkupfalse%
\ \isacommand{have}\isamarkupfalse%
\ {\isachardoublequoteopen}{\isasymforall}F\ {\isasymin}\ {\isacharbraceleft}\isactrlbold {\isasymnot}\ G{\isacharbraceright}{\isachardot}\ {\isasymA}\ {\isasymTurnstile}\ F{\isachardoublequoteclose}\isanewline
\ \ \ \ \ \ \isacommand{by}\isamarkupfalse%
\ simp\isanewline
\ \ \ \ \isacommand{then}\isamarkupfalse%
\ \isacommand{have}\isamarkupfalse%
\ {\isachardoublequoteopen}{\isasymforall}F\ {\isasymin}\ {\isacharparenleft}{\isacharbraceleft}\isactrlbold {\isasymnot}\ G{\isacharbraceright}\ {\isasymunion}\ {\isacharparenleft}{\isacharbraceleft}F{\isacharbraceright}\ {\isasymunion}\ Wo{\isacharparenright}{\isacharparenright}{\isachardot}\ {\isasymA}\ {\isasymTurnstile}\ F{\isachardoublequoteclose}\isanewline
\ \ \ \ \ \ \isacommand{using}\isamarkupfalse%
\ Forall{\isadigit{1}}\ \isacommand{by}\isamarkupfalse%
\ {\isacharparenleft}rule\ ball{\isacharunderscore}Un{\isacharparenright}\isanewline
\ \ \ \ \isacommand{then}\isamarkupfalse%
\ \isacommand{have}\isamarkupfalse%
\ {\isachardoublequoteopen}{\isasymforall}F\ {\isasymin}\ {\isacharbraceleft}\isactrlbold {\isasymnot}\ G{\isacharcomma}F{\isacharbraceright}\ {\isasymunion}\ Wo{\isachardot}\ {\isasymA}\ {\isasymTurnstile}\ F{\isachardoublequoteclose}\isanewline
\ \ \ \ \ \ \isacommand{by}\isamarkupfalse%
\ simp\isanewline
\ \ \ \ \isacommand{then}\isamarkupfalse%
\ \isacommand{have}\isamarkupfalse%
\ {\isachardoublequoteopen}{\isasymexists}{\isasymA}{\isachardot}\ {\isasymforall}F\ {\isasymin}\ {\isacharparenleft}{\isacharbraceleft}\isactrlbold {\isasymnot}\ G{\isacharcomma}F{\isacharbraceright}\ {\isasymunion}\ Wo{\isacharparenright}{\isachardot}\ {\isasymA}\ {\isasymTurnstile}\ F{\isachardoublequoteclose}\isanewline
\ \ \ \ \ \ \isacommand{by}\isamarkupfalse%
\ {\isacharparenleft}iprover\ intro{\isacharcolon}\ exI{\isacharparenright}\isanewline
\ \ \ \ \isacommand{then}\isamarkupfalse%
\ \isacommand{have}\isamarkupfalse%
\ {\isachardoublequoteopen}sat\ {\isacharparenleft}{\isacharbraceleft}\isactrlbold {\isasymnot}\ G{\isacharcomma}F{\isacharbraceright}\ {\isasymunion}\ Wo{\isacharparenright}{\isachardoublequoteclose}\isanewline
\ \ \ \ \ \ \isacommand{by}\isamarkupfalse%
\ {\isacharparenleft}simp\ only{\isacharcolon}\ sat{\isacharunderscore}def{\isacharparenright}\isanewline
\ \ \ \ \isacommand{thus}\isamarkupfalse%
\ {\isacharquery}thesis\isanewline
\ \ \ \ \ \ \isacommand{by}\isamarkupfalse%
\ {\isacharparenleft}rule\ disjI{\isadigit{1}}{\isacharparenright}\isanewline
\ \ \isacommand{next}\isamarkupfalse%
\isanewline
\ \ \ \ \isacommand{assume}\isamarkupfalse%
\ {\isachardoublequoteopen}{\isasymnot}\ {\isacharparenleft}{\isasymA}\ {\isasymTurnstile}\ H{\isacharparenright}{\isachardoublequoteclose}\isanewline
\ \ \ \ \isacommand{then}\isamarkupfalse%
\ \isacommand{have}\isamarkupfalse%
\ {\isachardoublequoteopen}{\isasymA}\ {\isasymTurnstile}\ \isactrlbold {\isasymnot}\ H{\isachardoublequoteclose}\isanewline
\ \ \ \ \ \ \isacommand{by}\isamarkupfalse%
\ {\isacharparenleft}simp\ only{\isacharcolon}\ formula{\isacharunderscore}semantics{\isachardot}simps{\isacharparenleft}{\isadigit{3}}{\isacharparenright}\ simp{\isacharunderscore}thms{\isacharparenleft}{\isadigit{8}}{\isacharparenright}{\isacharparenright}\isanewline
\ \ \ \ \isacommand{then}\isamarkupfalse%
\ \isacommand{have}\isamarkupfalse%
\ {\isachardoublequoteopen}{\isasymforall}F\ {\isasymin}\ {\isacharbraceleft}\isactrlbold {\isasymnot}\ H{\isacharbraceright}{\isachardot}\ {\isasymA}\ {\isasymTurnstile}\ F{\isachardoublequoteclose}\isanewline
\ \ \ \ \ \ \isacommand{by}\isamarkupfalse%
\ simp\isanewline
\ \ \ \ \isacommand{then}\isamarkupfalse%
\ \isacommand{have}\isamarkupfalse%
\ {\isachardoublequoteopen}{\isasymforall}F\ {\isasymin}\ {\isacharparenleft}{\isacharbraceleft}\isactrlbold {\isasymnot}\ H{\isacharbraceright}\ {\isasymunion}\ {\isacharparenleft}{\isacharbraceleft}F{\isacharbraceright}\ {\isasymunion}\ Wo{\isacharparenright}{\isacharparenright}{\isachardot}\ {\isasymA}\ {\isasymTurnstile}\ F{\isachardoublequoteclose}\isanewline
\ \ \ \ \ \ \isacommand{using}\isamarkupfalse%
\ Forall{\isadigit{1}}\ \isacommand{by}\isamarkupfalse%
\ {\isacharparenleft}rule\ ball{\isacharunderscore}Un{\isacharparenright}\isanewline
\ \ \ \ \isacommand{then}\isamarkupfalse%
\ \isacommand{have}\isamarkupfalse%
\ {\isachardoublequoteopen}{\isasymforall}F\ {\isasymin}\ {\isacharbraceleft}\isactrlbold {\isasymnot}\ H{\isacharcomma}F{\isacharbraceright}\ {\isasymunion}\ Wo{\isachardot}\ {\isasymA}\ {\isasymTurnstile}\ F{\isachardoublequoteclose}\isanewline
\ \ \ \ \ \ \isacommand{by}\isamarkupfalse%
\ simp\isanewline
\ \ \ \ \isacommand{then}\isamarkupfalse%
\ \isacommand{have}\isamarkupfalse%
\ {\isachardoublequoteopen}{\isasymexists}{\isasymA}{\isachardot}\ {\isasymforall}F\ {\isasymin}\ {\isacharparenleft}{\isacharbraceleft}\isactrlbold {\isasymnot}\ H{\isacharcomma}F{\isacharbraceright}\ {\isasymunion}\ Wo{\isacharparenright}{\isachardot}\ {\isasymA}\ {\isasymTurnstile}\ F{\isachardoublequoteclose}\isanewline
\ \ \ \ \ \ \isacommand{by}\isamarkupfalse%
\ {\isacharparenleft}iprover\ intro{\isacharcolon}\ exI{\isacharparenright}\isanewline
\ \ \ \ \isacommand{then}\isamarkupfalse%
\ \isacommand{have}\isamarkupfalse%
\ {\isachardoublequoteopen}sat\ {\isacharparenleft}{\isacharbraceleft}\isactrlbold {\isasymnot}\ H{\isacharcomma}F{\isacharbraceright}\ {\isasymunion}\ Wo{\isacharparenright}{\isachardoublequoteclose}\isanewline
\ \ \ \ \ \ \isacommand{by}\isamarkupfalse%
\ {\isacharparenleft}simp\ only{\isacharcolon}\ sat{\isacharunderscore}def{\isacharparenright}\isanewline
\ \ \ \ \isacommand{thus}\isamarkupfalse%
\ {\isacharquery}thesis\isanewline
\ \ \ \ \ \ \isacommand{by}\isamarkupfalse%
\ {\isacharparenleft}rule\ disjI{\isadigit{2}}{\isacharparenright}\isanewline
\ \ \isacommand{qed}\isamarkupfalse%
\isanewline
\isacommand{qed}\isamarkupfalse%
%
\endisatagproof
{\isafoldproof}%
%
\isadelimproof
%
\endisadelimproof
%
\begin{isamarkuptext}%
Por último, probemos que dados \isa{W\ {\isasymin}\ C}, una fórmula \isa{F\ {\isacharequal}\ {\isasymnot}\ {\isacharparenleft}{\isasymnot}\ G{\isacharparenright}} para cierta fórmula \isa{G} tal 
  que \isa{F\ {\isasymin}\ W} y \isa{W\isactrlsub {\isadigit{0}}} un subconjunto finito de \isa{W}, entonces se tiene que \isa{{\isacharbraceleft}G{\isacharcomma}F{\isacharbraceright}\ {\isasymunion}\ W\isactrlsub {\isadigit{0}}} es 
  satisfacible.%
\end{isamarkuptext}\isamarkuptrue%
\isacommand{lemma}\isamarkupfalse%
\ pcp{\isacharunderscore}colecComp{\isacharunderscore}DIS{\isacharunderscore}sat{\isadigit{4}}{\isacharcolon}\isanewline
\ \ \isakeyword{assumes}\ {\isachardoublequoteopen}W\ {\isasymin}\ colecComp{\isachardoublequoteclose}\isanewline
\ \ \ \ \ \ \ \ \ \ {\isachardoublequoteopen}F\ {\isacharequal}\ \isactrlbold {\isasymnot}\ {\isacharparenleft}\isactrlbold {\isasymnot}\ G{\isacharparenright}{\isachardoublequoteclose}\isanewline
\ \ \ \ \ \ \ \ \ \ {\isachardoublequoteopen}F\ {\isasymin}\ W{\isachardoublequoteclose}\isanewline
\ \ \ \ \ \ \ \ \ \ {\isachardoublequoteopen}finite\ Wo{\isachardoublequoteclose}\isanewline
\ \ \ \ \ \ \ \ \ \ {\isachardoublequoteopen}Wo\ {\isasymsubseteq}\ W{\isachardoublequoteclose}\isanewline
\ \ \ \ \ \ \ \ \isakeyword{shows}\ {\isachardoublequoteopen}sat\ {\isacharparenleft}{\isacharbraceleft}G{\isacharcomma}F{\isacharbraceright}\ {\isasymunion}\ Wo{\isacharparenright}{\isachardoublequoteclose}\isanewline
%
\isadelimproof
%
\endisadelimproof
%
\isatagproof
\isacommand{proof}\isamarkupfalse%
\ {\isacharminus}\isanewline
\ \ \isacommand{have}\isamarkupfalse%
\ {\isachardoublequoteopen}sat\ {\isacharparenleft}{\isacharbraceleft}F{\isacharbraceright}\ {\isasymunion}\ Wo{\isacharparenright}{\isachardoublequoteclose}\isanewline
\ \ \ \ \isacommand{using}\isamarkupfalse%
\ assms{\isacharparenleft}{\isadigit{1}}{\isacharcomma}{\isadigit{3}}{\isacharcomma}{\isadigit{4}}{\isacharcomma}{\isadigit{5}}{\isacharparenright}\ \isacommand{by}\isamarkupfalse%
\ {\isacharparenleft}rule\ pcp{\isacharunderscore}colecComp{\isacharunderscore}elem{\isacharunderscore}sat{\isacharparenright}\isanewline
\ \ \isacommand{have}\isamarkupfalse%
\ {\isachardoublequoteopen}F\ {\isasymin}\ {\isacharbraceleft}F{\isacharbraceright}\ {\isasymunion}\ Wo{\isachardoublequoteclose}\isanewline
\ \ \ \ \isacommand{by}\isamarkupfalse%
\ simp\ \isanewline
\ \ \isacommand{have}\isamarkupfalse%
\ Ex{\isadigit{1}}{\isacharcolon}{\isachardoublequoteopen}{\isasymexists}{\isasymA}{\isachardot}\ {\isasymforall}F\ {\isasymin}\ {\isacharparenleft}{\isacharbraceleft}F{\isacharbraceright}\ {\isasymunion}\ Wo{\isacharparenright}{\isachardot}\ {\isasymA}\ {\isasymTurnstile}\ F{\isachardoublequoteclose}\isanewline
\ \ \ \ \isacommand{using}\isamarkupfalse%
\ {\isacartoucheopen}sat\ {\isacharparenleft}{\isacharbraceleft}F{\isacharbraceright}\ {\isasymunion}\ Wo{\isacharparenright}{\isacartoucheclose}\ \isacommand{by}\isamarkupfalse%
\ {\isacharparenleft}simp\ only{\isacharcolon}\ sat{\isacharunderscore}def{\isacharparenright}\isanewline
\ \ \isacommand{obtain}\isamarkupfalse%
\ {\isasymA}\ \isakeyword{where}\ Forall{\isadigit{1}}{\isacharcolon}{\isachardoublequoteopen}{\isasymforall}F\ {\isasymin}\ {\isacharparenleft}{\isacharbraceleft}F{\isacharbraceright}\ {\isasymunion}\ Wo{\isacharparenright}{\isachardot}\ {\isasymA}\ {\isasymTurnstile}\ F{\isachardoublequoteclose}\isanewline
\ \ \ \ \isacommand{using}\isamarkupfalse%
\ Ex{\isadigit{1}}\ \isacommand{by}\isamarkupfalse%
\ {\isacharparenleft}rule\ exE{\isacharparenright}\isanewline
\ \ \isacommand{have}\isamarkupfalse%
\ {\isachardoublequoteopen}{\isasymA}\ {\isasymTurnstile}\ F{\isachardoublequoteclose}\isanewline
\ \ \ \ \isacommand{using}\isamarkupfalse%
\ Forall{\isadigit{1}}\ {\isacartoucheopen}F\ {\isasymin}\ {\isacharbraceleft}F{\isacharbraceright}\ {\isasymunion}\ Wo{\isacartoucheclose}\ \isacommand{by}\isamarkupfalse%
\ {\isacharparenleft}rule\ bspec{\isacharparenright}\isanewline
\ \ \isacommand{then}\isamarkupfalse%
\ \isacommand{have}\isamarkupfalse%
\ {\isachardoublequoteopen}{\isasymA}\ {\isasymTurnstile}\ \isactrlbold {\isasymnot}{\isacharparenleft}\isactrlbold {\isasymnot}\ G{\isacharparenright}{\isachardoublequoteclose}\isanewline
\ \ \ \ \isacommand{using}\isamarkupfalse%
\ assms{\isacharparenleft}{\isadigit{2}}{\isacharparenright}\ \isacommand{by}\isamarkupfalse%
\ {\isacharparenleft}simp\ only{\isacharcolon}\ {\isacartoucheopen}{\isasymA}\ {\isasymTurnstile}\ F{\isacartoucheclose}{\isacharparenright}\isanewline
\ \ \isacommand{then}\isamarkupfalse%
\ \isacommand{have}\isamarkupfalse%
\ {\isachardoublequoteopen}{\isasymnot}\ {\isasymA}\ {\isasymTurnstile}\ \isactrlbold {\isasymnot}\ G{\isachardoublequoteclose}\isanewline
\ \ \ \ \isacommand{by}\isamarkupfalse%
\ {\isacharparenleft}simp\ only{\isacharcolon}\ formula{\isacharunderscore}semantics{\isachardot}simps{\isacharparenleft}{\isadigit{3}}{\isacharparenright}\ simp{\isacharunderscore}thms{\isacharparenleft}{\isadigit{8}}{\isacharparenright}{\isacharparenright}\isanewline
\ \ \isacommand{then}\isamarkupfalse%
\ \isacommand{have}\isamarkupfalse%
\ {\isachardoublequoteopen}{\isasymnot}\ {\isasymnot}{\isasymA}\ {\isasymTurnstile}\ G{\isachardoublequoteclose}\isanewline
\ \ \ \ \isacommand{by}\isamarkupfalse%
\ {\isacharparenleft}simp\ only{\isacharcolon}\ formula{\isacharunderscore}semantics{\isachardot}simps{\isacharparenleft}{\isadigit{3}}{\isacharparenright}\ simp{\isacharunderscore}thms{\isacharparenleft}{\isadigit{8}}{\isacharparenright}{\isacharparenright}\isanewline
\ \ \isacommand{then}\isamarkupfalse%
\ \isacommand{have}\isamarkupfalse%
\ {\isachardoublequoteopen}{\isasymA}\ {\isasymTurnstile}\ G{\isachardoublequoteclose}\isanewline
\ \ \ \ \isacommand{by}\isamarkupfalse%
\ {\isacharparenleft}rule\ notnotD{\isacharparenright}\isanewline
\ \ \isacommand{then}\isamarkupfalse%
\ \isacommand{have}\isamarkupfalse%
\ {\isachardoublequoteopen}{\isasymforall}F\ {\isasymin}\ {\isacharbraceleft}G{\isacharbraceright}{\isachardot}\ {\isasymA}\ {\isasymTurnstile}\ F{\isachardoublequoteclose}\isanewline
\ \ \ \ \isacommand{by}\isamarkupfalse%
\ simp\isanewline
\ \ \isacommand{then}\isamarkupfalse%
\ \isacommand{have}\isamarkupfalse%
\ {\isachardoublequoteopen}{\isasymforall}F\ {\isasymin}\ {\isacharparenleft}{\isacharbraceleft}G{\isacharbraceright}\ {\isasymunion}\ {\isacharparenleft}{\isacharbraceleft}F{\isacharbraceright}\ {\isasymunion}\ Wo{\isacharparenright}{\isacharparenright}{\isachardot}\ {\isasymA}\ {\isasymTurnstile}\ F{\isachardoublequoteclose}\isanewline
\ \ \ \ \isacommand{using}\isamarkupfalse%
\ Forall{\isadigit{1}}\ \isacommand{by}\isamarkupfalse%
\ {\isacharparenleft}rule\ ball{\isacharunderscore}Un{\isacharparenright}\isanewline
\ \ \isacommand{then}\isamarkupfalse%
\ \isacommand{have}\isamarkupfalse%
\ {\isachardoublequoteopen}{\isasymforall}F\ {\isasymin}\ {\isacharbraceleft}G{\isacharcomma}F{\isacharbraceright}\ {\isasymunion}\ Wo{\isachardot}\ {\isasymA}\ {\isasymTurnstile}\ F{\isachardoublequoteclose}\isanewline
\ \ \ \ \isacommand{by}\isamarkupfalse%
\ simp\isanewline
\ \ \isacommand{then}\isamarkupfalse%
\ \isacommand{have}\isamarkupfalse%
\ {\isachardoublequoteopen}{\isasymexists}{\isasymA}{\isachardot}\ {\isasymforall}F\ {\isasymin}\ {\isacharparenleft}{\isacharbraceleft}G{\isacharcomma}F{\isacharbraceright}\ {\isasymunion}\ Wo{\isacharparenright}{\isachardot}\ {\isasymA}\ {\isasymTurnstile}\ F{\isachardoublequoteclose}\isanewline
\ \ \ \ \isacommand{by}\isamarkupfalse%
\ {\isacharparenleft}iprover\ intro{\isacharcolon}\ exI{\isacharparenright}\isanewline
\ \ \isacommand{thus}\isamarkupfalse%
\ {\isacharquery}thesis\isanewline
\ \ \ \ \isacommand{by}\isamarkupfalse%
\ {\isacharparenleft}simp\ only{\isacharcolon}\ sat{\isacharunderscore}def{\isacharparenright}\isanewline
\isacommand{qed}\isamarkupfalse%
%
\endisatagproof
{\isafoldproof}%
%
\isadelimproof
%
\endisadelimproof
%
\begin{isamarkuptext}%
De este modo, por los lemas anteriores para los distintos tipos de fórmula \isa{{\isasymbeta}}, se
  demuestra que dados \isa{W\ {\isasymin}\ C}, \isa{F} una fórmula de tipo \isa{{\isasymbeta}} con componentes \isa{{\isasymbeta}\isactrlsub {\isadigit{1}}} y \isa{{\isasymbeta}\isactrlsub {\isadigit{2}}} tal que 
  \isa{F\ {\isasymin}\ W} y \isa{W\isactrlsub {\isadigit{0}}} un subconjunto finito de \isa{W}, entonces se tiene que o bien \isa{{\isacharbraceleft}{\isasymbeta}\isactrlsub {\isadigit{1}}{\isacharcomma}F{\isacharbraceright}\ {\isasymunion}\ W\isactrlsub {\isadigit{0}}} es 
  satisfacible o bien \isa{{\isacharbraceleft}{\isasymbeta}\isactrlsub {\isadigit{2}}{\isacharcomma}F{\isacharbraceright}\ {\isasymunion}\ W\isactrlsub {\isadigit{0}}} es satisfacible.%
\end{isamarkuptext}\isamarkuptrue%
\isacommand{lemma}\isamarkupfalse%
\ pcp{\isacharunderscore}colecComp{\isacharunderscore}DIS{\isacharunderscore}sat{\isacharcolon}\isanewline
\ \ \isakeyword{assumes}\ {\isachardoublequoteopen}W\ {\isasymin}\ colecComp{\isachardoublequoteclose}\isanewline
\ \ \ \ \ \ \ \ \ \ {\isachardoublequoteopen}Dis\ F\ G\ H{\isachardoublequoteclose}\isanewline
\ \ \ \ \ \ \ \ \ \ {\isachardoublequoteopen}F\ {\isasymin}\ W{\isachardoublequoteclose}\isanewline
\ \ \ \ \ \ \ \ \ \ {\isachardoublequoteopen}finite\ Wo{\isachardoublequoteclose}\isanewline
\ \ \ \ \ \ \ \ \ \ {\isachardoublequoteopen}Wo\ {\isasymsubseteq}\ W{\isachardoublequoteclose}\isanewline
\ \ \ \ \ \ \ \ \isakeyword{shows}\ {\isachardoublequoteopen}sat\ {\isacharparenleft}{\isacharbraceleft}G{\isacharcomma}F{\isacharbraceright}\ {\isasymunion}\ Wo{\isacharparenright}\ {\isasymor}\ sat\ {\isacharparenleft}{\isacharbraceleft}H{\isacharcomma}F{\isacharbraceright}\ {\isasymunion}\ Wo{\isacharparenright}{\isachardoublequoteclose}\isanewline
%
\isadelimproof
%
\endisadelimproof
%
\isatagproof
\isacommand{proof}\isamarkupfalse%
\ {\isacharminus}\isanewline
\ \ \isacommand{have}\isamarkupfalse%
\ {\isachardoublequoteopen}{\isacharparenleft}F\ {\isacharequal}\ G\ \isactrlbold {\isasymor}\ H\ {\isasymor}\ \isanewline
\ \ \ \ \ \ \ \ {\isacharparenleft}{\isasymexists}G{\isadigit{1}}\ H{\isadigit{1}}{\isachardot}\ F\ {\isacharequal}\ G{\isadigit{1}}\ \isactrlbold {\isasymrightarrow}\ H{\isadigit{1}}\ {\isasymand}\ G\ {\isacharequal}\ \isactrlbold {\isasymnot}\ G{\isadigit{1}}\ {\isasymand}\ H\ {\isacharequal}\ H{\isadigit{1}}{\isacharparenright}\ {\isasymor}\ \isanewline
\ \ \ \ \ \ \ \ {\isacharparenleft}{\isasymexists}G{\isadigit{1}}\ H{\isadigit{1}}{\isachardot}\ F\ {\isacharequal}\ \isactrlbold {\isasymnot}\ {\isacharparenleft}G{\isadigit{1}}\ \isactrlbold {\isasymand}\ H{\isadigit{1}}{\isacharparenright}\ {\isasymand}\ G\ {\isacharequal}\ \isactrlbold {\isasymnot}\ G{\isadigit{1}}\ {\isasymand}\ H\ {\isacharequal}\ \isactrlbold {\isasymnot}\ H{\isadigit{1}}{\isacharparenright}\ {\isasymor}\ \isanewline
\ \ \ \ \ \ \ \ F\ {\isacharequal}\ \isactrlbold {\isasymnot}\ {\isacharparenleft}\isactrlbold {\isasymnot}\ G{\isacharparenright}\ {\isasymand}\ H\ {\isacharequal}\ G{\isacharparenright}{\isachardoublequoteclose}\isanewline
\ \ \ \ \isacommand{using}\isamarkupfalse%
\ assms{\isacharparenleft}{\isadigit{2}}{\isacharparenright}\ \isacommand{by}\isamarkupfalse%
\ {\isacharparenleft}simp\ only{\isacharcolon}\ con{\isacharunderscore}dis{\isacharunderscore}simps{\isacharparenleft}{\isadigit{2}}{\isacharparenright}{\isacharparenright}\isanewline
\ \ \isacommand{thus}\isamarkupfalse%
\ {\isacharquery}thesis\isanewline
\ \ \isacommand{proof}\isamarkupfalse%
\ {\isacharparenleft}rule\ disjE{\isacharparenright}\isanewline
\ \ \ \ \isacommand{assume}\isamarkupfalse%
\ {\isachardoublequoteopen}F\ {\isacharequal}\ G\ \isactrlbold {\isasymor}\ H{\isachardoublequoteclose}\isanewline
\ \ \ \ \isacommand{show}\isamarkupfalse%
\ {\isachardoublequoteopen}sat\ {\isacharparenleft}{\isacharbraceleft}G{\isacharcomma}F{\isacharbraceright}\ {\isasymunion}\ Wo{\isacharparenright}\ {\isasymor}\ sat\ {\isacharparenleft}{\isacharbraceleft}H{\isacharcomma}F{\isacharbraceright}\ {\isasymunion}\ Wo{\isacharparenright}{\isachardoublequoteclose}\isanewline
\ \ \ \ \ \ \isacommand{using}\isamarkupfalse%
\ assms{\isacharparenleft}{\isadigit{1}}{\isacharparenright}\ {\isacartoucheopen}F\ {\isacharequal}\ G\ \isactrlbold {\isasymor}\ H{\isacartoucheclose}\ assms{\isacharparenleft}{\isadigit{3}}{\isacharcomma}{\isadigit{4}}{\isacharcomma}{\isadigit{5}}{\isacharparenright}\ \isacommand{by}\isamarkupfalse%
\ {\isacharparenleft}rule\ pcp{\isacharunderscore}colecComp{\isacharunderscore}DIS{\isacharunderscore}sat{\isadigit{1}}{\isacharparenright}\isanewline
\ \ \isacommand{next}\isamarkupfalse%
\isanewline
\ \ \ \ \isacommand{assume}\isamarkupfalse%
\ {\isachardoublequoteopen}{\isacharparenleft}{\isasymexists}G{\isadigit{1}}\ H{\isadigit{1}}{\isachardot}\ F\ {\isacharequal}\ G{\isadigit{1}}\ \isactrlbold {\isasymrightarrow}\ H{\isadigit{1}}\ {\isasymand}\ G\ {\isacharequal}\ \isactrlbold {\isasymnot}\ G{\isadigit{1}}\ {\isasymand}\ H\ {\isacharequal}\ H{\isadigit{1}}{\isacharparenright}\ {\isasymor}\ \isanewline
\ \ \ \ \ \ \ \ {\isacharparenleft}{\isasymexists}G{\isadigit{1}}\ H{\isadigit{1}}{\isachardot}\ F\ {\isacharequal}\ \isactrlbold {\isasymnot}\ {\isacharparenleft}G{\isadigit{1}}\ \isactrlbold {\isasymand}\ H{\isadigit{1}}{\isacharparenright}\ {\isasymand}\ G\ {\isacharequal}\ \isactrlbold {\isasymnot}\ G{\isadigit{1}}\ {\isasymand}\ H\ {\isacharequal}\ \isactrlbold {\isasymnot}\ H{\isadigit{1}}{\isacharparenright}\ {\isasymor}\ \isanewline
\ \ \ \ \ \ \ \ F\ {\isacharequal}\ \isactrlbold {\isasymnot}\ {\isacharparenleft}\isactrlbold {\isasymnot}\ G{\isacharparenright}\ {\isasymand}\ H\ {\isacharequal}\ G{\isachardoublequoteclose}\isanewline
\ \ \ \ \isacommand{thus}\isamarkupfalse%
\ {\isacharquery}thesis\isanewline
\ \ \ \ \isacommand{proof}\isamarkupfalse%
\ {\isacharparenleft}rule\ disjE{\isacharparenright}\isanewline
\ \ \ \ \ \ \isacommand{assume}\isamarkupfalse%
\ Ex{\isadigit{1}}{\isacharcolon}{\isachardoublequoteopen}{\isasymexists}G{\isadigit{1}}\ H{\isadigit{1}}{\isachardot}\ F\ {\isacharequal}\ G{\isadigit{1}}\ \isactrlbold {\isasymrightarrow}\ H{\isadigit{1}}\ {\isasymand}\ G\ {\isacharequal}\ \isactrlbold {\isasymnot}\ G{\isadigit{1}}\ {\isasymand}\ H\ {\isacharequal}\ H{\isadigit{1}}{\isachardoublequoteclose}\isanewline
\ \ \ \ \ \ \isacommand{obtain}\isamarkupfalse%
\ G{\isadigit{1}}\ H{\isadigit{1}}\ \isakeyword{where}\ C{\isadigit{1}}{\isacharcolon}{\isachardoublequoteopen}F\ {\isacharequal}\ G{\isadigit{1}}\ \isactrlbold {\isasymrightarrow}\ H{\isadigit{1}}\ {\isasymand}\ G\ {\isacharequal}\ \isactrlbold {\isasymnot}\ G{\isadigit{1}}\ {\isasymand}\ H\ {\isacharequal}\ H{\isadigit{1}}{\isachardoublequoteclose}\isanewline
\ \ \ \ \ \ \ \ \isacommand{using}\isamarkupfalse%
\ Ex{\isadigit{1}}\ \isacommand{by}\isamarkupfalse%
\ {\isacharparenleft}iprover\ elim{\isacharcolon}\ exE{\isacharparenright}\isanewline
\ \ \ \ \ \ \isacommand{have}\isamarkupfalse%
\ {\isachardoublequoteopen}F\ {\isacharequal}\ G{\isadigit{1}}\ \isactrlbold {\isasymrightarrow}\ H{\isadigit{1}}{\isachardoublequoteclose}\isanewline
\ \ \ \ \ \ \ \ \isacommand{using}\isamarkupfalse%
\ C{\isadigit{1}}\ \isacommand{by}\isamarkupfalse%
\ {\isacharparenleft}rule\ conjunct{\isadigit{1}}{\isacharparenright}\isanewline
\ \ \ \ \ \ \isacommand{have}\isamarkupfalse%
\ {\isachardoublequoteopen}G\ {\isacharequal}\ \isactrlbold {\isasymnot}\ G{\isadigit{1}}{\isachardoublequoteclose}\isanewline
\ \ \ \ \ \ \ \ \isacommand{using}\isamarkupfalse%
\ C{\isadigit{1}}\ \isacommand{by}\isamarkupfalse%
\ {\isacharparenleft}iprover\ elim{\isacharcolon}\ conjunct{\isadigit{1}}{\isacharparenright}\isanewline
\ \ \ \ \ \ \isacommand{have}\isamarkupfalse%
\ {\isachardoublequoteopen}H\ {\isacharequal}\ H{\isadigit{1}}{\isachardoublequoteclose}\isanewline
\ \ \ \ \ \ \ \ \isacommand{using}\isamarkupfalse%
\ C{\isadigit{1}}\ \isacommand{by}\isamarkupfalse%
\ {\isacharparenleft}iprover\ elim{\isacharcolon}\ conjunct{\isadigit{2}}{\isacharparenright}\isanewline
\ \ \ \ \ \ \isacommand{have}\isamarkupfalse%
\ {\isachardoublequoteopen}sat\ {\isacharparenleft}{\isacharbraceleft}\isactrlbold {\isasymnot}\ G{\isadigit{1}}{\isacharcomma}F{\isacharbraceright}\ {\isasymunion}\ Wo{\isacharparenright}\ {\isasymor}\ sat\ {\isacharparenleft}{\isacharbraceleft}H{\isadigit{1}}{\isacharcomma}F{\isacharbraceright}\ {\isasymunion}\ Wo{\isacharparenright}{\isachardoublequoteclose}\isanewline
\ \ \ \ \ \ \ \ \isacommand{using}\isamarkupfalse%
\ assms{\isacharparenleft}{\isadigit{1}}{\isacharparenright}\ {\isacartoucheopen}F\ {\isacharequal}\ G{\isadigit{1}}\ \isactrlbold {\isasymrightarrow}\ H{\isadigit{1}}{\isacartoucheclose}\ assms{\isacharparenleft}{\isadigit{3}}{\isacharcomma}{\isadigit{4}}{\isacharcomma}{\isadigit{5}}{\isacharparenright}\ \isacommand{by}\isamarkupfalse%
\ {\isacharparenleft}rule\ pcp{\isacharunderscore}colecComp{\isacharunderscore}DIS{\isacharunderscore}sat{\isadigit{2}}{\isacharparenright}\isanewline
\ \ \ \ \ \ \isacommand{thus}\isamarkupfalse%
\ {\isachardoublequoteopen}sat\ {\isacharparenleft}{\isacharbraceleft}G{\isacharcomma}\ F{\isacharbraceright}\ {\isasymunion}\ Wo{\isacharparenright}\ {\isasymor}\ sat\ {\isacharparenleft}{\isacharbraceleft}H{\isacharcomma}\ F{\isacharbraceright}\ {\isasymunion}\ Wo{\isacharparenright}{\isachardoublequoteclose}\isanewline
\ \ \ \ \ \ \ \ \isacommand{by}\isamarkupfalse%
\ {\isacharparenleft}simp\ only{\isacharcolon}\ {\isacartoucheopen}G\ {\isacharequal}\ \isactrlbold {\isasymnot}\ G{\isadigit{1}}{\isacartoucheclose}\ {\isacartoucheopen}H\ {\isacharequal}\ H{\isadigit{1}}{\isacartoucheclose}{\isacharparenright}\ \isanewline
\ \ \ \ \isacommand{next}\isamarkupfalse%
\isanewline
\ \ \ \ \ \ \isacommand{assume}\isamarkupfalse%
\ {\isachardoublequoteopen}{\isacharparenleft}{\isasymexists}G{\isadigit{1}}\ H{\isadigit{1}}{\isachardot}\ F\ {\isacharequal}\ \isactrlbold {\isasymnot}\ {\isacharparenleft}G{\isadigit{1}}\ \isactrlbold {\isasymand}\ H{\isadigit{1}}{\isacharparenright}\ {\isasymand}\ G\ {\isacharequal}\ \isactrlbold {\isasymnot}\ G{\isadigit{1}}\ {\isasymand}\ H\ {\isacharequal}\ \isactrlbold {\isasymnot}\ H{\isadigit{1}}{\isacharparenright}\ {\isasymor}\ \isanewline
\ \ \ \ \ \ \ \ F\ {\isacharequal}\ \isactrlbold {\isasymnot}\ {\isacharparenleft}\isactrlbold {\isasymnot}\ G{\isacharparenright}\ {\isasymand}\ H\ {\isacharequal}\ G{\isachardoublequoteclose}\isanewline
\ \ \ \ \ \ \isacommand{thus}\isamarkupfalse%
\ {\isacharquery}thesis\isanewline
\ \ \ \ \ \ \isacommand{proof}\isamarkupfalse%
\ {\isacharparenleft}rule\ disjE{\isacharparenright}\isanewline
\ \ \ \ \ \ \ \ \isacommand{assume}\isamarkupfalse%
\ Ex{\isadigit{2}}{\isacharcolon}{\isachardoublequoteopen}{\isasymexists}G{\isadigit{1}}\ H{\isadigit{1}}{\isachardot}\ F\ {\isacharequal}\ \isactrlbold {\isasymnot}\ {\isacharparenleft}G{\isadigit{1}}\ \isactrlbold {\isasymand}\ H{\isadigit{1}}{\isacharparenright}\ {\isasymand}\ G\ {\isacharequal}\ \isactrlbold {\isasymnot}\ G{\isadigit{1}}\ {\isasymand}\ H\ {\isacharequal}\ \isactrlbold {\isasymnot}\ H{\isadigit{1}}{\isachardoublequoteclose}\isanewline
\ \ \ \ \ \ \ \ \isacommand{obtain}\isamarkupfalse%
\ G{\isadigit{1}}\ H{\isadigit{1}}\ \isakeyword{where}\ C{\isadigit{2}}{\isacharcolon}{\isachardoublequoteopen}F\ {\isacharequal}\ \isactrlbold {\isasymnot}\ {\isacharparenleft}G{\isadigit{1}}\ \isactrlbold {\isasymand}\ H{\isadigit{1}}{\isacharparenright}\ {\isasymand}\ G\ {\isacharequal}\ \isactrlbold {\isasymnot}\ G{\isadigit{1}}\ {\isasymand}\ H\ {\isacharequal}\ \isactrlbold {\isasymnot}\ H{\isadigit{1}}{\isachardoublequoteclose}\isanewline
\ \ \ \ \ \ \ \ \ \ \isacommand{using}\isamarkupfalse%
\ Ex{\isadigit{2}}\ \isacommand{by}\isamarkupfalse%
\ {\isacharparenleft}iprover\ elim{\isacharcolon}\ exE{\isacharparenright}\isanewline
\ \ \ \ \ \ \ \ \isacommand{have}\isamarkupfalse%
\ {\isachardoublequoteopen}F\ {\isacharequal}\ \isactrlbold {\isasymnot}\ {\isacharparenleft}G{\isadigit{1}}\ \isactrlbold {\isasymand}\ H{\isadigit{1}}{\isacharparenright}{\isachardoublequoteclose}\isanewline
\ \ \ \ \ \ \ \ \ \ \isacommand{using}\isamarkupfalse%
\ C{\isadigit{2}}\ \isacommand{by}\isamarkupfalse%
\ {\isacharparenleft}rule\ conjunct{\isadigit{1}}{\isacharparenright}\isanewline
\ \ \ \ \ \ \ \ \isacommand{have}\isamarkupfalse%
\ {\isachardoublequoteopen}G\ {\isacharequal}\ \isactrlbold {\isasymnot}\ G{\isadigit{1}}{\isachardoublequoteclose}\isanewline
\ \ \ \ \ \ \ \ \ \ \isacommand{using}\isamarkupfalse%
\ C{\isadigit{2}}\ \isacommand{by}\isamarkupfalse%
\ {\isacharparenleft}iprover\ elim{\isacharcolon}\ conjunct{\isadigit{1}}{\isacharparenright}\isanewline
\ \ \ \ \ \ \ \ \isacommand{have}\isamarkupfalse%
\ {\isachardoublequoteopen}H\ {\isacharequal}\ \isactrlbold {\isasymnot}\ H{\isadigit{1}}{\isachardoublequoteclose}\isanewline
\ \ \ \ \ \ \ \ \ \ \isacommand{using}\isamarkupfalse%
\ C{\isadigit{2}}\ \isacommand{by}\isamarkupfalse%
\ {\isacharparenleft}iprover\ elim{\isacharcolon}\ conjunct{\isadigit{2}}{\isacharparenright}\isanewline
\ \ \ \ \ \ \ \ \isacommand{have}\isamarkupfalse%
\ {\isachardoublequoteopen}sat\ {\isacharparenleft}{\isacharbraceleft}\isactrlbold {\isasymnot}\ G{\isadigit{1}}{\isacharcomma}F{\isacharbraceright}\ {\isasymunion}\ Wo{\isacharparenright}\ {\isasymor}\ sat\ {\isacharparenleft}{\isacharbraceleft}\isactrlbold {\isasymnot}\ H{\isadigit{1}}{\isacharcomma}F{\isacharbraceright}\ {\isasymunion}\ Wo{\isacharparenright}{\isachardoublequoteclose}\isanewline
\ \ \ \ \ \ \ \ \ \ \isacommand{using}\isamarkupfalse%
\ assms{\isacharparenleft}{\isadigit{1}}{\isacharparenright}\ {\isacartoucheopen}F\ {\isacharequal}\ \isactrlbold {\isasymnot}\ {\isacharparenleft}G{\isadigit{1}}\ \isactrlbold {\isasymand}\ H{\isadigit{1}}{\isacharparenright}{\isacartoucheclose}\ assms{\isacharparenleft}{\isadigit{3}}{\isacharcomma}{\isadigit{4}}{\isacharcomma}{\isadigit{5}}{\isacharparenright}\ \isacommand{by}\isamarkupfalse%
\ {\isacharparenleft}rule\ pcp{\isacharunderscore}colecComp{\isacharunderscore}DIS{\isacharunderscore}sat{\isadigit{3}}{\isacharparenright}\isanewline
\ \ \ \ \ \ \ \ \isacommand{thus}\isamarkupfalse%
\ {\isachardoublequoteopen}sat\ {\isacharparenleft}{\isacharbraceleft}G{\isacharcomma}F{\isacharbraceright}\ {\isasymunion}\ Wo{\isacharparenright}\ {\isasymor}\ sat\ {\isacharparenleft}{\isacharbraceleft}H{\isacharcomma}F{\isacharbraceright}\ {\isasymunion}\ Wo{\isacharparenright}{\isachardoublequoteclose}\isanewline
\ \ \ \ \ \ \ \ \ \ \isacommand{by}\isamarkupfalse%
\ {\isacharparenleft}simp\ only{\isacharcolon}\ {\isacartoucheopen}G\ {\isacharequal}\ \isactrlbold {\isasymnot}\ G{\isadigit{1}}{\isacartoucheclose}\ {\isacartoucheopen}H\ {\isacharequal}\ \isactrlbold {\isasymnot}\ H{\isadigit{1}}{\isacartoucheclose}{\isacharparenright}\isanewline
\ \ \ \ \ \ \isacommand{next}\isamarkupfalse%
\isanewline
\ \ \ \ \ \ \ \ \isacommand{assume}\isamarkupfalse%
\ {\isachardoublequoteopen}F\ {\isacharequal}\ \isactrlbold {\isasymnot}\ {\isacharparenleft}\isactrlbold {\isasymnot}\ G{\isacharparenright}\ {\isasymand}\ H\ {\isacharequal}\ G{\isachardoublequoteclose}\isanewline
\ \ \ \ \ \ \ \ \isacommand{then}\isamarkupfalse%
\ \isacommand{have}\isamarkupfalse%
\ {\isachardoublequoteopen}F\ {\isacharequal}\ \isactrlbold {\isasymnot}\ {\isacharparenleft}\isactrlbold {\isasymnot}\ G{\isacharparenright}{\isachardoublequoteclose}\isanewline
\ \ \ \ \ \ \ \ \ \ \isacommand{by}\isamarkupfalse%
\ {\isacharparenleft}rule\ conjunct{\isadigit{1}}{\isacharparenright}\isanewline
\ \ \ \ \ \ \ \ \isacommand{have}\isamarkupfalse%
\ {\isachardoublequoteopen}sat\ {\isacharparenleft}{\isacharbraceleft}G{\isacharcomma}F{\isacharbraceright}\ {\isasymunion}\ Wo{\isacharparenright}{\isachardoublequoteclose}\isanewline
\ \ \ \ \ \ \ \ \ \ \isacommand{using}\isamarkupfalse%
\ assms{\isacharparenleft}{\isadigit{1}}{\isacharparenright}\ {\isacartoucheopen}F\ {\isacharequal}\ \isactrlbold {\isasymnot}\ {\isacharparenleft}\isactrlbold {\isasymnot}\ G{\isacharparenright}{\isacartoucheclose}\ assms{\isacharparenleft}{\isadigit{3}}{\isacharcomma}{\isadigit{4}}{\isacharcomma}{\isadigit{5}}{\isacharparenright}\ \isacommand{by}\isamarkupfalse%
\ {\isacharparenleft}rule\ pcp{\isacharunderscore}colecComp{\isacharunderscore}DIS{\isacharunderscore}sat{\isadigit{4}}{\isacharparenright}\isanewline
\ \ \ \ \ \ \ \ \isacommand{thus}\isamarkupfalse%
\ {\isachardoublequoteopen}sat\ {\isacharparenleft}{\isacharbraceleft}G{\isacharcomma}F{\isacharbraceright}\ {\isasymunion}\ Wo{\isacharparenright}\ {\isasymor}\ sat\ {\isacharparenleft}{\isacharbraceleft}H{\isacharcomma}F{\isacharbraceright}\ {\isasymunion}\ Wo{\isacharparenright}{\isachardoublequoteclose}\isanewline
\ \ \ \ \ \ \ \ \ \ \isacommand{by}\isamarkupfalse%
\ {\isacharparenleft}rule\ disjI{\isadigit{1}}{\isacharparenright}\isanewline
\ \ \ \ \ \ \isacommand{qed}\isamarkupfalse%
\isanewline
\ \ \ \ \isacommand{qed}\isamarkupfalse%
\isanewline
\ \ \isacommand{qed}\isamarkupfalse%
\isanewline
\isacommand{qed}\isamarkupfalse%
%
\endisatagproof
{\isafoldproof}%
%
\isadelimproof
%
\endisadelimproof
%
\begin{isamarkuptext}%
Finalmente, con los lemas auxiliares anteriores podemos demostrar detalladamente la cuarta 
  condición del lema \isa{{\isadigit{2}}{\isachardot}{\isadigit{0}}{\isachardot}{\isadigit{2}}}: dados \isa{W\ {\isasymin}\ C} y \isa{F} una fórmula de tipo \isa{{\isasymbeta}} con componentes \isa{{\isasymbeta}\isactrlsub {\isadigit{1}}} y 
  \isa{{\isasymbeta}\isactrlsub {\isadigit{2}}} tal que \isa{F\ {\isasymin}\ W}, se tiene que o bien \isa{{\isacharbraceleft}{\isasymbeta}\isactrlsub {\isadigit{1}}{\isacharbraceright}\ {\isasymunion}\ W\ {\isasymin}\ C} o bien\\ \isa{{\isacharbraceleft}{\isasymbeta}\isactrlsub {\isadigit{2}}{\isacharbraceright}\ {\isasymunion}\ W\ {\isasymin}\ C}.%
\end{isamarkuptext}\isamarkuptrue%
\isacommand{lemma}\isamarkupfalse%
\ pcp{\isacharunderscore}colecComp{\isacharunderscore}DIS{\isacharcolon}\isanewline
\ \ \isakeyword{assumes}\ {\isachardoublequoteopen}W\ {\isasymin}\ colecComp{\isachardoublequoteclose}\isanewline
\ \ \isakeyword{shows}\ {\isachardoublequoteopen}{\isasymforall}F\ G\ H{\isachardot}\ Dis\ F\ G\ H\ {\isasymlongrightarrow}\ F\ {\isasymin}\ W\ {\isasymlongrightarrow}\ {\isacharbraceleft}G{\isacharbraceright}\ {\isasymunion}\ W\ {\isasymin}\ colecComp\ {\isasymor}\ {\isacharbraceleft}H{\isacharbraceright}\ {\isasymunion}\ W\ {\isasymin}\ colecComp{\isachardoublequoteclose}\isanewline
%
\isadelimproof
%
\endisadelimproof
%
\isatagproof
\isacommand{proof}\isamarkupfalse%
\ {\isacharparenleft}rule\ allI{\isacharparenright}{\isacharplus}\isanewline
\ \ \isacommand{fix}\isamarkupfalse%
\ F\ G\ H\isanewline
\ \ \isacommand{show}\isamarkupfalse%
\ {\isachardoublequoteopen}Dis\ F\ G\ H\ {\isasymlongrightarrow}\ F\ {\isasymin}\ W\ {\isasymlongrightarrow}\ {\isacharbraceleft}G{\isacharbraceright}\ {\isasymunion}\ W\ {\isasymin}\ colecComp\ {\isasymor}\ {\isacharbraceleft}H{\isacharbraceright}\ {\isasymunion}\ W\ {\isasymin}\ colecComp{\isachardoublequoteclose}\isanewline
\ \ \isacommand{proof}\isamarkupfalse%
\ {\isacharparenleft}rule\ impI{\isacharparenright}{\isacharplus}\isanewline
\ \ \ \ \isacommand{assume}\isamarkupfalse%
\ {\isachardoublequoteopen}Dis\ F\ G\ H{\isachardoublequoteclose}\isanewline
\ \ \ \ \isacommand{assume}\isamarkupfalse%
\ {\isachardoublequoteopen}F\ {\isasymin}\ W{\isachardoublequoteclose}\isanewline
\ \ \ \ \isacommand{show}\isamarkupfalse%
\ {\isachardoublequoteopen}{\isacharbraceleft}G{\isacharbraceright}\ {\isasymunion}\ W\ {\isasymin}\ colecComp\ {\isasymor}\ {\isacharbraceleft}H{\isacharbraceright}\ {\isasymunion}\ W\ {\isasymin}\ colecComp{\isachardoublequoteclose}\isanewline
\ \ \ \ \isacommand{proof}\isamarkupfalse%
\ {\isacharparenleft}rule\ ccontr{\isacharparenright}\isanewline
\ \ \ \ \ \ \isacommand{assume}\isamarkupfalse%
\ {\isachardoublequoteopen}{\isasymnot}{\isacharparenleft}{\isacharbraceleft}G{\isacharbraceright}\ {\isasymunion}\ W\ {\isasymin}\ colecComp\ {\isasymor}\ {\isacharbraceleft}H{\isacharbraceright}\ {\isasymunion}\ W\ {\isasymin}\ colecComp{\isacharparenright}{\isachardoublequoteclose}\isanewline
\ \ \ \ \ \ \isacommand{then}\isamarkupfalse%
\ \isacommand{have}\isamarkupfalse%
\ C{\isacharcolon}{\isachardoublequoteopen}{\isacharbraceleft}G{\isacharbraceright}\ {\isasymunion}\ W\ {\isasymnotin}\ colecComp\ {\isasymand}\ {\isacharbraceleft}H{\isacharbraceright}\ {\isasymunion}\ W\ {\isasymnotin}\ colecComp{\isachardoublequoteclose}\isanewline
\ \ \ \ \ \ \ \ \isacommand{by}\isamarkupfalse%
\ {\isacharparenleft}simp\ only{\isacharcolon}\ de{\isacharunderscore}Morgan{\isacharunderscore}disj\ simp{\isacharunderscore}thms{\isacharparenleft}{\isadigit{8}}{\isacharparenright}{\isacharparenright}\isanewline
\ \ \ \ \ \ \isacommand{then}\isamarkupfalse%
\ \isacommand{have}\isamarkupfalse%
\ {\isachardoublequoteopen}{\isacharbraceleft}G{\isacharbraceright}\ {\isasymunion}\ W\ {\isasymnotin}\ colecComp{\isachardoublequoteclose}\isanewline
\ \ \ \ \ \ \ \ \isacommand{by}\isamarkupfalse%
\ {\isacharparenleft}rule\ conjunct{\isadigit{1}}{\isacharparenright}\isanewline
\ \ \ \ \ \ \isacommand{have}\isamarkupfalse%
\ Ex{\isadigit{1}}{\isacharcolon}{\isachardoublequoteopen}{\isasymexists}Wo\ {\isasymsubseteq}\ W{\isachardot}\ finite\ Wo\ {\isasymand}\ {\isasymnot}{\isacharparenleft}sat\ {\isacharparenleft}{\isacharbraceleft}G{\isacharbraceright}\ {\isasymunion}\ Wo{\isacharparenright}{\isacharparenright}{\isachardoublequoteclose}\isanewline
\ \ \ \ \ \ \ \ \isacommand{using}\isamarkupfalse%
\ assms\ {\isacartoucheopen}{\isacharbraceleft}G{\isacharbraceright}\ {\isasymunion}\ W\ {\isasymnotin}\ colecComp{\isacartoucheclose}\ \isacommand{by}\isamarkupfalse%
\ {\isacharparenleft}rule\ not{\isacharunderscore}colecComp{\isacharparenright}\isanewline
\ \ \ \ \ \ \isacommand{obtain}\isamarkupfalse%
\ W{\isadigit{1}}\ \isakeyword{where}\ {\isachardoublequoteopen}W{\isadigit{1}}\ {\isasymsubseteq}\ W{\isachardoublequoteclose}\ \isakeyword{and}\ C{\isadigit{1}}{\isacharcolon}{\isachardoublequoteopen}finite\ W{\isadigit{1}}\ {\isasymand}\ {\isasymnot}{\isacharparenleft}sat\ {\isacharparenleft}{\isacharbraceleft}G{\isacharbraceright}\ {\isasymunion}\ W{\isadigit{1}}{\isacharparenright}{\isacharparenright}{\isachardoublequoteclose}\isanewline
\ \ \ \ \ \ \ \ \isacommand{using}\isamarkupfalse%
\ Ex{\isadigit{1}}\ \isacommand{by}\isamarkupfalse%
\ {\isacharparenleft}rule\ subexE{\isacharparenright}\isanewline
\ \ \ \ \ \ \isacommand{have}\isamarkupfalse%
\ {\isachardoublequoteopen}finite\ W{\isadigit{1}}{\isachardoublequoteclose}\isanewline
\ \ \ \ \ \ \ \ \isacommand{using}\isamarkupfalse%
\ C{\isadigit{1}}\ \isacommand{by}\isamarkupfalse%
\ {\isacharparenleft}rule\ conjunct{\isadigit{1}}{\isacharparenright}\isanewline
\ \ \ \ \ \ \isacommand{have}\isamarkupfalse%
\ {\isachardoublequoteopen}{\isasymnot}{\isacharparenleft}sat\ {\isacharparenleft}{\isacharbraceleft}G{\isacharbraceright}\ {\isasymunion}\ W{\isadigit{1}}{\isacharparenright}{\isacharparenright}{\isachardoublequoteclose}\isanewline
\ \ \ \ \ \ \ \ \isacommand{using}\isamarkupfalse%
\ C{\isadigit{1}}\ \isacommand{by}\isamarkupfalse%
\ {\isacharparenleft}rule\ conjunct{\isadigit{2}}{\isacharparenright}\isanewline
\ \ \ \ \ \ \isacommand{have}\isamarkupfalse%
\ {\isachardoublequoteopen}{\isacharbraceleft}H{\isacharbraceright}\ {\isasymunion}\ W\ {\isasymnotin}\ colecComp{\isachardoublequoteclose}\isanewline
\ \ \ \ \ \ \ \ \isacommand{using}\isamarkupfalse%
\ C\ \isacommand{by}\isamarkupfalse%
\ {\isacharparenleft}rule\ conjunct{\isadigit{2}}{\isacharparenright}\ \isanewline
\ \ \ \ \ \ \isacommand{have}\isamarkupfalse%
\ Ex{\isadigit{2}}{\isacharcolon}{\isachardoublequoteopen}{\isasymexists}Wo\ {\isasymsubseteq}\ W{\isachardot}\ finite\ Wo\ {\isasymand}\ {\isasymnot}{\isacharparenleft}sat\ {\isacharparenleft}{\isacharbraceleft}H{\isacharbraceright}\ {\isasymunion}\ Wo{\isacharparenright}{\isacharparenright}{\isachardoublequoteclose}\isanewline
\ \ \ \ \ \ \ \ \isacommand{using}\isamarkupfalse%
\ assms\ {\isacartoucheopen}{\isacharbraceleft}H{\isacharbraceright}\ {\isasymunion}\ W\ {\isasymnotin}\ colecComp{\isacartoucheclose}\ \isacommand{by}\isamarkupfalse%
\ {\isacharparenleft}rule\ not{\isacharunderscore}colecComp{\isacharparenright}\isanewline
\ \ \ \ \ \ \isacommand{obtain}\isamarkupfalse%
\ W{\isadigit{2}}\ \isakeyword{where}\ {\isachardoublequoteopen}W{\isadigit{2}}\ {\isasymsubseteq}\ W{\isachardoublequoteclose}\ \isakeyword{and}\ C{\isadigit{2}}{\isacharcolon}{\isachardoublequoteopen}finite\ W{\isadigit{2}}\ {\isasymand}\ {\isasymnot}{\isacharparenleft}sat\ {\isacharparenleft}{\isacharbraceleft}H{\isacharbraceright}\ {\isasymunion}\ W{\isadigit{2}}{\isacharparenright}{\isacharparenright}{\isachardoublequoteclose}\isanewline
\ \ \ \ \ \ \ \ \isacommand{using}\isamarkupfalse%
\ Ex{\isadigit{2}}\ \isacommand{by}\isamarkupfalse%
\ {\isacharparenleft}rule\ subexE{\isacharparenright}\isanewline
\ \ \ \ \ \ \isacommand{have}\isamarkupfalse%
\ {\isachardoublequoteopen}finite\ W{\isadigit{2}}{\isachardoublequoteclose}\isanewline
\ \ \ \ \ \ \ \ \isacommand{using}\isamarkupfalse%
\ C{\isadigit{2}}\ \isacommand{by}\isamarkupfalse%
\ {\isacharparenleft}rule\ conjunct{\isadigit{1}}{\isacharparenright}\isanewline
\ \ \ \ \ \ \isacommand{have}\isamarkupfalse%
\ {\isachardoublequoteopen}{\isasymnot}{\isacharparenleft}sat\ {\isacharparenleft}{\isacharbraceleft}H{\isacharbraceright}\ {\isasymunion}\ W{\isadigit{2}}{\isacharparenright}{\isacharparenright}{\isachardoublequoteclose}\isanewline
\ \ \ \ \ \ \ \ \isacommand{using}\isamarkupfalse%
\ C{\isadigit{2}}\ \isacommand{by}\isamarkupfalse%
\ {\isacharparenleft}rule\ conjunct{\isadigit{2}}{\isacharparenright}\isanewline
\ \ \ \ \ \ \isacommand{let}\isamarkupfalse%
\ {\isacharquery}Wo\ {\isacharequal}\ {\isachardoublequoteopen}W{\isadigit{1}}\ {\isasymunion}\ W{\isadigit{2}}{\isachardoublequoteclose}\isanewline
\ \ \ \ \ \ \isacommand{have}\isamarkupfalse%
\ {\isachardoublequoteopen}{\isacharquery}Wo\ {\isasymsubseteq}\ W{\isachardoublequoteclose}\isanewline
\ \ \ \ \ \ \ \ \isacommand{using}\isamarkupfalse%
\ {\isacartoucheopen}W{\isadigit{1}}\ {\isasymsubseteq}\ W{\isacartoucheclose}\ {\isacartoucheopen}W{\isadigit{2}}\ {\isasymsubseteq}\ W{\isacartoucheclose}\ \isacommand{by}\isamarkupfalse%
\ {\isacharparenleft}simp\ only{\isacharcolon}\ Un{\isacharunderscore}least{\isacharparenright}\isanewline
\ \ \ \ \ \ \isacommand{have}\isamarkupfalse%
\ {\isachardoublequoteopen}finite\ {\isacharquery}Wo{\isachardoublequoteclose}\isanewline
\ \ \ \ \ \ \ \ \isacommand{using}\isamarkupfalse%
\ {\isacartoucheopen}finite\ W{\isadigit{1}}{\isacartoucheclose}\ {\isacartoucheopen}finite\ W{\isadigit{2}}{\isacartoucheclose}\ \isacommand{by}\isamarkupfalse%
\ {\isacharparenleft}simp\ only{\isacharcolon}\ finite{\isacharunderscore}Un{\isacharparenright}\isanewline
\ \ \ \ \ \ \isacommand{have}\isamarkupfalse%
\ {\isachardoublequoteopen}{\isacharbraceleft}G{\isacharbraceright}\ {\isasymunion}\ W{\isadigit{1}}\ {\isasymsubseteq}\ {\isacharparenleft}{\isacharbraceleft}G{\isacharbraceright}\ {\isasymunion}\ W{\isadigit{1}}{\isacharparenright}\ {\isasymunion}\ W{\isadigit{2}}{\isachardoublequoteclose}\isanewline
\ \ \ \ \ \ \ \ \isacommand{by}\isamarkupfalse%
\ {\isacharparenleft}simp\ only{\isacharcolon}\ Un{\isacharunderscore}upper{\isadigit{1}}{\isacharparenright}\isanewline
\ \ \ \ \ \ \isacommand{then}\isamarkupfalse%
\ \isacommand{have}\isamarkupfalse%
\ {\isachardoublequoteopen}{\isacharbraceleft}G{\isacharbraceright}\ {\isasymunion}\ W{\isadigit{1}}\ {\isasymsubseteq}\ {\isacharbraceleft}G{\isacharbraceright}\ {\isasymunion}\ {\isacharquery}Wo{\isachardoublequoteclose}\isanewline
\ \ \ \ \ \ \ \ \isacommand{by}\isamarkupfalse%
\ {\isacharparenleft}simp\ only{\isacharcolon}\ Un{\isacharunderscore}assoc{\isacharparenright}\isanewline
\ \ \ \ \ \ \isacommand{then}\isamarkupfalse%
\ \isacommand{have}\isamarkupfalse%
\ {\isachardoublequoteopen}{\isacharbraceleft}G{\isacharbraceright}\ {\isasymunion}\ W{\isadigit{1}}\ {\isasymsubseteq}\ {\isacharbraceleft}G{\isacharcomma}F{\isacharbraceright}\ {\isasymunion}\ {\isacharquery}Wo{\isachardoublequoteclose}\isanewline
\ \ \ \ \ \ \ \ \isacommand{by}\isamarkupfalse%
\ blast\isanewline
\ \ \ \ \ \ \isacommand{then}\isamarkupfalse%
\ \isacommand{have}\isamarkupfalse%
\ {\isadigit{1}}{\isacharcolon}{\isachardoublequoteopen}{\isasymnot}{\isacharparenleft}sat{\isacharparenleft}{\isacharbraceleft}G{\isacharcomma}F{\isacharbraceright}\ {\isasymunion}\ {\isacharquery}Wo{\isacharparenright}{\isacharparenright}{\isachardoublequoteclose}\isanewline
\ \ \ \ \ \ \ \ \isacommand{using}\isamarkupfalse%
\ {\isacartoucheopen}{\isasymnot}sat\ {\isacharparenleft}{\isacharbraceleft}G{\isacharbraceright}\ {\isasymunion}\ W{\isadigit{1}}{\isacharparenright}{\isacartoucheclose}\ \isacommand{by}\isamarkupfalse%
\ {\isacharparenleft}rule\ sat{\isacharunderscore}subset{\isacharunderscore}ccontr{\isacharparenright}\isanewline
\ \ \ \ \ \ \isacommand{have}\isamarkupfalse%
\ {\isachardoublequoteopen}{\isacharbraceleft}H{\isacharbraceright}\ {\isasymunion}\ W{\isadigit{2}}\ {\isasymsubseteq}\ {\isacharparenleft}{\isacharbraceleft}H{\isacharbraceright}\ {\isasymunion}\ W{\isadigit{2}}{\isacharparenright}\ {\isasymunion}\ W{\isadigit{1}}{\isachardoublequoteclose}\isanewline
\ \ \ \ \ \ \ \ \isacommand{by}\isamarkupfalse%
\ {\isacharparenleft}simp\ only{\isacharcolon}\ Un{\isacharunderscore}upper{\isadigit{1}}{\isacharparenright}\isanewline
\ \ \ \ \ \ \isacommand{then}\isamarkupfalse%
\ \isacommand{have}\isamarkupfalse%
\ {\isachardoublequoteopen}{\isacharbraceleft}H{\isacharbraceright}\ {\isasymunion}\ W{\isadigit{2}}\ {\isasymsubseteq}\ {\isacharbraceleft}H{\isacharbraceright}\ {\isasymunion}\ {\isacharparenleft}W{\isadigit{2}}\ {\isasymunion}\ W{\isadigit{1}}{\isacharparenright}{\isachardoublequoteclose}\isanewline
\ \ \ \ \ \ \ \ \isacommand{by}\isamarkupfalse%
\ {\isacharparenleft}simp\ only{\isacharcolon}\ Un{\isacharunderscore}assoc{\isacharparenright}\ \isanewline
\ \ \ \ \ \ \isacommand{then}\isamarkupfalse%
\ \isacommand{have}\isamarkupfalse%
\ {\isachardoublequoteopen}{\isacharbraceleft}H{\isacharbraceright}\ {\isasymunion}\ W{\isadigit{2}}\ {\isasymsubseteq}\ {\isacharbraceleft}H{\isacharbraceright}\ {\isasymunion}\ {\isacharquery}Wo{\isachardoublequoteclose}\isanewline
\ \ \ \ \ \ \ \ \isacommand{by}\isamarkupfalse%
\ {\isacharparenleft}simp\ only{\isacharcolon}\ Un{\isacharunderscore}commute{\isacharparenright}\isanewline
\ \ \ \ \ \ \isacommand{then}\isamarkupfalse%
\ \isacommand{have}\isamarkupfalse%
\ {\isachardoublequoteopen}{\isacharbraceleft}H{\isacharbraceright}\ {\isasymunion}\ W{\isadigit{2}}\ {\isasymsubseteq}\ {\isacharbraceleft}H{\isacharcomma}F{\isacharbraceright}\ {\isasymunion}\ {\isacharquery}Wo{\isachardoublequoteclose}\isanewline
\ \ \ \ \ \ \ \ \isacommand{by}\isamarkupfalse%
\ blast\isanewline
\ \ \ \ \ \ \isacommand{then}\isamarkupfalse%
\ \isacommand{have}\isamarkupfalse%
\ {\isadigit{2}}{\isacharcolon}{\isachardoublequoteopen}{\isasymnot}{\isacharparenleft}sat{\isacharparenleft}{\isacharbraceleft}H{\isacharcomma}F{\isacharbraceright}\ {\isasymunion}\ {\isacharquery}Wo{\isacharparenright}{\isacharparenright}{\isachardoublequoteclose}\isanewline
\ \ \ \ \ \ \ \ \isacommand{using}\isamarkupfalse%
\ {\isacartoucheopen}{\isasymnot}sat\ {\isacharparenleft}{\isacharbraceleft}H{\isacharbraceright}\ {\isasymunion}\ W{\isadigit{2}}{\isacharparenright}{\isacartoucheclose}\ \isacommand{by}\isamarkupfalse%
\ {\isacharparenleft}rule\ sat{\isacharunderscore}subset{\isacharunderscore}ccontr{\isacharparenright}\isanewline
\ \ \ \ \ \ \isacommand{have}\isamarkupfalse%
\ {\isachardoublequoteopen}{\isasymnot}\ sat\ {\isacharparenleft}{\isacharbraceleft}G{\isacharcomma}F{\isacharbraceright}\ {\isasymunion}\ {\isacharquery}Wo{\isacharparenright}\ {\isasymand}\ {\isasymnot}\ sat\ {\isacharparenleft}{\isacharbraceleft}H{\isacharcomma}F{\isacharbraceright}\ {\isasymunion}\ {\isacharquery}Wo{\isacharparenright}{\isachardoublequoteclose}\isanewline
\ \ \ \ \ \ \ \ \isacommand{using}\isamarkupfalse%
\ {\isadigit{1}}\ {\isadigit{2}}\ \isacommand{by}\isamarkupfalse%
\ {\isacharparenleft}rule\ conjI{\isacharparenright}\isanewline
\ \ \ \ \ \ \isacommand{have}\isamarkupfalse%
\ {\isachardoublequoteopen}sat\ {\isacharparenleft}{\isacharbraceleft}G{\isacharcomma}F{\isacharbraceright}\ {\isasymunion}\ {\isacharquery}Wo{\isacharparenright}\ {\isasymor}\ sat\ {\isacharparenleft}{\isacharbraceleft}H{\isacharcomma}F{\isacharbraceright}\ {\isasymunion}\ {\isacharquery}Wo{\isacharparenright}{\isachardoublequoteclose}\isanewline
\ \ \ \ \ \ \ \ \isacommand{using}\isamarkupfalse%
\ assms{\isacharparenleft}{\isadigit{1}}{\isacharparenright}\ {\isacartoucheopen}Dis\ F\ G\ H{\isacartoucheclose}\ {\isacartoucheopen}F\ {\isasymin}\ W{\isacartoucheclose}\ {\isacartoucheopen}finite\ {\isacharquery}Wo{\isacartoucheclose}\ {\isacartoucheopen}{\isacharquery}Wo\ {\isasymsubseteq}\ W{\isacartoucheclose}\ \isacommand{by}\isamarkupfalse%
\ {\isacharparenleft}rule\ pcp{\isacharunderscore}colecComp{\isacharunderscore}DIS{\isacharunderscore}sat{\isacharparenright}\isanewline
\ \ \ \ \ \ \isacommand{then}\isamarkupfalse%
\ \isacommand{have}\isamarkupfalse%
\ {\isachardoublequoteopen}{\isasymnot}{\isasymnot}{\isacharparenleft}sat\ {\isacharparenleft}{\isacharbraceleft}G{\isacharcomma}F{\isacharbraceright}\ {\isasymunion}\ {\isacharquery}Wo{\isacharparenright}\ {\isasymor}\ sat\ {\isacharparenleft}{\isacharbraceleft}H{\isacharcomma}F{\isacharbraceright}\ {\isasymunion}\ {\isacharquery}Wo{\isacharparenright}{\isacharparenright}{\isachardoublequoteclose}\isanewline
\ \ \ \ \ \ \ \ \isacommand{by}\isamarkupfalse%
\ {\isacharparenleft}simp\ only{\isacharcolon}\ not{\isacharunderscore}not{\isacharparenright}\isanewline
\ \ \ \ \ \ \isacommand{then}\isamarkupfalse%
\ \isacommand{have}\isamarkupfalse%
\ {\isachardoublequoteopen}{\isasymnot}{\isacharparenleft}{\isasymnot}{\isacharparenleft}sat\ {\isacharparenleft}{\isacharbraceleft}G{\isacharcomma}F{\isacharbraceright}\ {\isasymunion}\ {\isacharquery}Wo{\isacharparenright}{\isacharparenright}\ {\isasymand}\ {\isasymnot}{\isacharparenleft}sat\ {\isacharparenleft}{\isacharbraceleft}H{\isacharcomma}F{\isacharbraceright}\ {\isasymunion}\ {\isacharquery}Wo{\isacharparenright}{\isacharparenright}{\isacharparenright}{\isachardoublequoteclose}\isanewline
\ \ \ \ \ \ \ \ \isacommand{by}\isamarkupfalse%
\ {\isacharparenleft}simp\ only{\isacharcolon}\ de{\isacharunderscore}Morgan{\isacharunderscore}disj\ simp{\isacharunderscore}thms{\isacharparenleft}{\isadigit{8}}{\isacharparenright}{\isacharparenright}\isanewline
\ \ \ \ \ \ \isacommand{thus}\isamarkupfalse%
\ {\isachardoublequoteopen}False{\isachardoublequoteclose}\isanewline
\ \ \ \ \ \ \ \ \isacommand{using}\isamarkupfalse%
\ {\isacartoucheopen}{\isasymnot}{\isacharparenleft}sat\ {\isacharparenleft}{\isacharbraceleft}G{\isacharcomma}F{\isacharbraceright}\ {\isasymunion}\ {\isacharquery}Wo{\isacharparenright}{\isacharparenright}\ {\isasymand}\ {\isasymnot}{\isacharparenleft}sat\ {\isacharparenleft}{\isacharbraceleft}H{\isacharcomma}F{\isacharbraceright}\ {\isasymunion}\ {\isacharquery}Wo{\isacharparenright}{\isacharparenright}{\isacartoucheclose}\ \isacommand{by}\isamarkupfalse%
\ {\isacharparenleft}rule\ notE{\isacharparenright}\isanewline
\ \ \ \ \isacommand{qed}\isamarkupfalse%
\isanewline
\ \ \isacommand{qed}\isamarkupfalse%
\isanewline
\isacommand{qed}\isamarkupfalse%
%
\endisatagproof
{\isafoldproof}%
%
\isadelimproof
%
\endisadelimproof
%
\begin{isamarkuptext}%
En resumen, con los lemas \isa{pcp{\isacharunderscore}colecComp{\isacharunderscore}bot}, \isa{pcp{\isacharunderscore}colecComp{\isacharunderscore}atoms}, \isa{pcp{\isacharunderscore}colecComp{\isacharunderscore}CON} y
  \isa{pcp{\isacharunderscore}colecComp{\isacharunderscore}DIS} podemos probar de manera detallada que la colección \isa{C} verifica la propiedad 
  de consistencia proposicional.%
\end{isamarkuptext}\isamarkuptrue%
\isacommand{lemma}\isamarkupfalse%
\ pcp{\isacharunderscore}colecComp{\isacharcolon}\ {\isachardoublequoteopen}pcp\ colecComp{\isachardoublequoteclose}\isanewline
%
\isadelimproof
%
\endisadelimproof
%
\isatagproof
\isacommand{proof}\isamarkupfalse%
\ {\isacharparenleft}rule\ pcp{\isacharunderscore}alt{\isadigit{2}}{\isacharparenright}\isanewline
\ \ \isacommand{show}\isamarkupfalse%
\ {\isachardoublequoteopen}{\isasymforall}W\ {\isasymin}\ colecComp{\isachardot}\ {\isasymbottom}\ {\isasymnotin}\ W\isanewline
\ \ \ \ \ \ \ \ {\isasymand}\ {\isacharparenleft}{\isasymforall}k{\isachardot}\ Atom\ k\ {\isasymin}\ W\ {\isasymlongrightarrow}\ \isactrlbold {\isasymnot}\ {\isacharparenleft}Atom\ k{\isacharparenright}\ {\isasymin}\ W\ {\isasymlongrightarrow}\ False{\isacharparenright}\isanewline
\ \ \ \ \ \ \ \ {\isasymand}\ {\isacharparenleft}{\isasymforall}F\ G\ H{\isachardot}\ Con\ F\ G\ H\ {\isasymlongrightarrow}\ F\ {\isasymin}\ W\ {\isasymlongrightarrow}\ {\isacharbraceleft}G{\isacharcomma}H{\isacharbraceright}\ {\isasymunion}\ W\ {\isasymin}\ colecComp{\isacharparenright}\isanewline
\ \ \ \ \ \ \ \ {\isasymand}\ {\isacharparenleft}{\isasymforall}F\ G\ H{\isachardot}\ Dis\ F\ G\ H\ {\isasymlongrightarrow}\ F\ {\isasymin}\ W\ {\isasymlongrightarrow}\ {\isacharbraceleft}G{\isacharbraceright}\ {\isasymunion}\ W\ {\isasymin}\ colecComp\ {\isasymor}\ {\isacharbraceleft}H{\isacharbraceright}\ {\isasymunion}\ W\ {\isasymin}\ colecComp{\isacharparenright}{\isachardoublequoteclose}\isanewline
\ \ \isacommand{proof}\isamarkupfalse%
\ {\isacharparenleft}rule\ ballI{\isacharparenright}\isanewline
\ \ \ \ \isacommand{fix}\isamarkupfalse%
\ W\isanewline
\ \ \ \ \isacommand{assume}\isamarkupfalse%
\ H{\isacharcolon}{\isachardoublequoteopen}W\ {\isasymin}\ colecComp{\isachardoublequoteclose}\isanewline
\ \ \ \ \isacommand{have}\isamarkupfalse%
\ C{\isadigit{1}}{\isacharcolon}{\isachardoublequoteopen}{\isasymbottom}\ {\isasymnotin}\ W{\isachardoublequoteclose}\isanewline
\ \ \ \ \ \ \isacommand{using}\isamarkupfalse%
\ H\ \isacommand{by}\isamarkupfalse%
\ {\isacharparenleft}rule\ pcp{\isacharunderscore}colecComp{\isacharunderscore}bot{\isacharparenright}\isanewline
\ \ \ \ \isacommand{have}\isamarkupfalse%
\ C{\isadigit{2}}{\isacharcolon}{\isachardoublequoteopen}{\isasymforall}k{\isachardot}\ Atom\ k\ {\isasymin}\ W\ {\isasymlongrightarrow}\ \isactrlbold {\isasymnot}\ {\isacharparenleft}Atom\ k{\isacharparenright}\ {\isasymin}\ W\ {\isasymlongrightarrow}\ False{\isachardoublequoteclose}\isanewline
\ \ \ \ \ \ \isacommand{using}\isamarkupfalse%
\ H\ \isacommand{by}\isamarkupfalse%
\ {\isacharparenleft}rule\ pcp{\isacharunderscore}colecComp{\isacharunderscore}atoms{\isacharparenright}\isanewline
\ \ \ \ \isacommand{have}\isamarkupfalse%
\ C{\isadigit{3}}{\isacharcolon}{\isachardoublequoteopen}{\isasymforall}F\ G\ H{\isachardot}\ Con\ F\ G\ H\ {\isasymlongrightarrow}\ F\ {\isasymin}\ W\ {\isasymlongrightarrow}\ {\isacharbraceleft}G{\isacharcomma}H{\isacharbraceright}\ {\isasymunion}\ W\ {\isasymin}\ colecComp{\isachardoublequoteclose}\isanewline
\ \ \ \ \ \ \isacommand{using}\isamarkupfalse%
\ H\ \isacommand{by}\isamarkupfalse%
\ {\isacharparenleft}rule\ pcp{\isacharunderscore}colecComp{\isacharunderscore}CON{\isacharparenright}\isanewline
\ \ \ \ \isacommand{have}\isamarkupfalse%
\ C{\isadigit{4}}{\isacharcolon}{\isachardoublequoteopen}{\isasymforall}F\ G\ H{\isachardot}\ Dis\ F\ G\ H\ {\isasymlongrightarrow}\ F\ {\isasymin}\ W\ {\isasymlongrightarrow}\ {\isacharbraceleft}G{\isacharbraceright}\ {\isasymunion}\ W\ {\isasymin}\ colecComp\ {\isasymor}\ {\isacharbraceleft}H{\isacharbraceright}\ {\isasymunion}\ W\ {\isasymin}\ colecComp{\isachardoublequoteclose}\isanewline
\ \ \ \ \ \ \isacommand{using}\isamarkupfalse%
\ H\ \isacommand{by}\isamarkupfalse%
\ {\isacharparenleft}rule\ pcp{\isacharunderscore}colecComp{\isacharunderscore}DIS{\isacharparenright}\isanewline
\ \ \ \ \isacommand{show}\isamarkupfalse%
\ {\isachardoublequoteopen}{\isasymbottom}\ {\isasymnotin}\ W\isanewline
\ \ \ \ \ \ \ \ \ \ {\isasymand}\ {\isacharparenleft}{\isasymforall}k{\isachardot}\ Atom\ k\ {\isasymin}\ W\ {\isasymlongrightarrow}\ \isactrlbold {\isasymnot}\ {\isacharparenleft}Atom\ k{\isacharparenright}\ {\isasymin}\ W\ {\isasymlongrightarrow}\ False{\isacharparenright}\isanewline
\ \ \ \ \ \ \ \ \ \ {\isasymand}\ {\isacharparenleft}{\isasymforall}F\ G\ H{\isachardot}\ Con\ F\ G\ H\ {\isasymlongrightarrow}\ F\ {\isasymin}\ W\ {\isasymlongrightarrow}\ {\isacharbraceleft}G{\isacharcomma}H{\isacharbraceright}\ {\isasymunion}\ W\ {\isasymin}\ colecComp{\isacharparenright}\isanewline
\ \ \ \ \ \ \ \ \ \ {\isasymand}\ {\isacharparenleft}{\isasymforall}F\ G\ H{\isachardot}\ Dis\ F\ G\ H\ {\isasymlongrightarrow}\ F\ {\isasymin}\ W\ {\isasymlongrightarrow}\ {\isacharbraceleft}G{\isacharbraceright}\ {\isasymunion}\ W\ {\isasymin}\ colecComp\ {\isasymor}\ {\isacharbraceleft}H{\isacharbraceright}\ {\isasymunion}\ W\ {\isasymin}\ colecComp{\isacharparenright}{\isachardoublequoteclose}\isanewline
\ \ \ \ \ \ \isacommand{using}\isamarkupfalse%
\ C{\isadigit{1}}\ C{\isadigit{2}}\ C{\isadigit{3}}\ C{\isadigit{4}}\ \isacommand{by}\isamarkupfalse%
\ {\isacharparenleft}iprover\ intro{\isacharcolon}\ conjI{\isacharparenright}\isanewline
\ \ \isacommand{qed}\isamarkupfalse%
\isanewline
\isacommand{qed}\isamarkupfalse%
%
\endisatagproof
{\isafoldproof}%
%
\isadelimproof
%
\endisadelimproof
%
\begin{isamarkuptext}%
Finalmente, mostremos la demostración del \isa{Teorema\ de\ Compacidad}.%
\end{isamarkuptext}\isamarkuptrue%
\isacommand{theorem}\isamarkupfalse%
\ prop{\isacharunderscore}Compactness{\isacharcolon}\isanewline
\ \ \isakeyword{fixes}\ W\ {\isacharcolon}{\isacharcolon}\ {\isachardoublequoteopen}{\isacharprime}a\ {\isacharcolon}{\isacharcolon}\ countable\ formula\ set{\isachardoublequoteclose}\isanewline
\ \ \isakeyword{assumes}\ {\isachardoublequoteopen}fin{\isacharunderscore}sat\ W{\isachardoublequoteclose}\isanewline
\ \ \isakeyword{shows}\ {\isachardoublequoteopen}sat\ W{\isachardoublequoteclose}\isanewline
%
\isadelimproof
%
\endisadelimproof
%
\isatagproof
\isacommand{proof}\isamarkupfalse%
\ {\isacharparenleft}rule\ pcp{\isacharunderscore}sat{\isacharparenright}\isanewline
\ \ \isacommand{show}\isamarkupfalse%
\ {\isachardoublequoteopen}W\ {\isasymin}\ colecComp{\isachardoublequoteclose}\isanewline
\ \ \ \ \isacommand{unfolding}\isamarkupfalse%
\ colecComp\ \isacommand{using}\isamarkupfalse%
\ assms\ \isacommand{unfolding}\isamarkupfalse%
\ fin{\isacharunderscore}sat{\isacharunderscore}def\ \isacommand{by}\isamarkupfalse%
\ {\isacharparenleft}rule\ CollectI{\isacharparenright}\isanewline
\ \ \isacommand{show}\isamarkupfalse%
\ {\isachardoublequoteopen}pcp\ colecComp{\isachardoublequoteclose}\isanewline
\ \ \ \ \isacommand{by}\isamarkupfalse%
\ {\isacharparenleft}simp\ only{\isacharcolon}\ pcp{\isacharunderscore}colecComp{\isacharparenright}\isanewline
\isacommand{qed}\isamarkupfalse%
\isanewline
%
\endisatagproof
{\isafoldproof}%
%
\isadelimproof
%
\endisadelimproof
%
\isadelimtheory
%
\endisadelimtheory
%
\isatagtheory
%
\endisatagtheory
{\isafoldtheory}%
%
\isadelimtheory
%
\endisadelimtheory
%
\end{isabellebody}%
\endinput
%:%file=~/TFM/TFM/TeoremaEx.thy%:%
%:%19=13%:%
%:%20=14%:%
%:%21=15%:%
%:%30=17%:%
%:%42=19%:%
%:%43=20%:%
%:%44=21%:%
%:%45=22%:%
%:%46=23%:%
%:%47=24%:%
%:%48=25%:%
%:%49=26%:%
%:%50=27%:%
%:%51=28%:%
%:%52=29%:%
%:%53=30%:%
%:%54=31%:%
%:%55=32%:%
%:%56=33%:%
%:%57=34%:%
%:%58=35%:%
%:%59=36%:%
%:%60=37%:%
%:%61=38%:%
%:%62=39%:%
%:%63=40%:%
%:%64=41%:%
%:%65=42%:%
%:%65=43%:%
%:%66=44%:%
%:%67=45%:%
%:%68=46%:%
%:%69=47%:%
%:%70=48%:%
%:%71=49%:%
%:%72=50%:%
%:%73=51%:%
%:%74=52%:%
%:%75=53%:%
%:%76=54%:%
%:%77=55%:%
%:%78=56%:%
%:%79=57%:%
%:%80=58%:%
%:%81=59%:%
%:%82=60%:%
%:%83=61%:%
%:%84=62%:%
%:%85=63%:%
%:%86=64%:%
%:%88=66%:%
%:%89=66%:%
%:%90=67%:%
%:%91=68%:%
%:%94=71%:%
%:%95=72%:%
%:%96=73%:%
%:%97=74%:%
%:%98=75%:%
%:%99=76%:%
%:%100=77%:%
%:%101=78%:%
%:%102=79%:%
%:%103=80%:%
%:%104=81%:%
%:%105=82%:%
%:%106=83%:%
%:%107=84%:%
%:%108=85%:%
%:%109=86%:%
%:%110=87%:%
%:%111=88%:%
%:%112=89%:%
%:%113=90%:%
%:%114=91%:%
%:%115=92%:%
%:%116=93%:%
%:%118=95%:%
%:%119=95%:%
%:%120=96%:%
%:%121=97%:%
%:%122=98%:%
%:%129=99%:%
%:%130=99%:%
%:%131=100%:%
%:%132=100%:%
%:%133=101%:%
%:%134=101%:%
%:%135=102%:%
%:%136=102%:%
%:%137=103%:%
%:%138=103%:%
%:%139=104%:%
%:%140=104%:%
%:%141=105%:%
%:%142=105%:%
%:%143=106%:%
%:%144=107%:%
%:%145=107%:%
%:%146=108%:%
%:%147=108%:%
%:%148=108%:%
%:%149=109%:%
%:%150=110%:%
%:%151=110%:%
%:%152=111%:%
%:%153=111%:%
%:%154=112%:%
%:%155=112%:%
%:%156=113%:%
%:%157=113%:%
%:%158=114%:%
%:%159=114%:%
%:%160=115%:%
%:%161=115%:%
%:%162=115%:%
%:%163=116%:%
%:%164=116%:%
%:%165=117%:%
%:%166=117%:%
%:%167=118%:%
%:%168=118%:%
%:%169=119%:%
%:%170=119%:%
%:%171=120%:%
%:%172=120%:%
%:%173=121%:%
%:%174=121%:%
%:%175=121%:%
%:%176=122%:%
%:%177=122%:%
%:%178=123%:%
%:%179=123%:%
%:%180=124%:%
%:%181=124%:%
%:%182=125%:%
%:%192=127%:%
%:%194=129%:%
%:%195=129%:%
%:%202=130%:%
%:%203=130%:%
%:%204=131%:%
%:%205=131%:%
%:%206=132%:%
%:%207=132%:%
%:%208=132%:%
%:%209=133%:%
%:%210=133%:%
%:%211=133%:%
%:%212=134%:%
%:%213=134%:%
%:%222=136%:%
%:%223=137%:%
%:%224=138%:%
%:%225=139%:%
%:%226=140%:%
%:%227=141%:%
%:%228=142%:%
%:%229=143%:%
%:%231=145%:%
%:%232=145%:%
%:%235=146%:%
%:%239=146%:%
%:%249=148%:%
%:%250=149%:%
%:%251=150%:%
%:%252=151%:%
%:%253=152%:%
%:%254=153%:%
%:%255=154%:%
%:%256=155%:%
%:%257=156%:%
%:%258=157%:%
%:%259=158%:%
%:%260=159%:%
%:%261=160%:%
%:%263=162%:%
%:%264=162%:%
%:%271=163%:%
%:%272=163%:%
%:%273=164%:%
%:%274=164%:%
%:%275=165%:%
%:%276=166%:%
%:%277=166%:%
%:%278=167%:%
%:%279=167%:%
%:%280=167%:%
%:%281=168%:%
%:%282=169%:%
%:%283=169%:%
%:%284=170%:%
%:%285=170%:%
%:%286=171%:%
%:%287=171%:%
%:%288=172%:%
%:%289=172%:%
%:%290=173%:%
%:%291=173%:%
%:%292=174%:%
%:%293=174%:%
%:%294=174%:%
%:%295=175%:%
%:%296=175%:%
%:%297=176%:%
%:%298=176%:%
%:%299=177%:%
%:%300=177%:%
%:%301=178%:%
%:%302=178%:%
%:%303=179%:%
%:%304=179%:%
%:%305=180%:%
%:%306=180%:%
%:%307=180%:%
%:%308=181%:%
%:%309=181%:%
%:%310=182%:%
%:%311=182%:%
%:%312=183%:%
%:%313=183%:%
%:%314=184%:%
%:%324=186%:%
%:%326=188%:%
%:%327=188%:%
%:%330=189%:%
%:%334=189%:%
%:%335=189%:%
%:%344=191%:%
%:%345=192%:%
%:%347=194%:%
%:%348=194%:%
%:%349=195%:%
%:%350=196%:%
%:%353=197%:%
%:%357=197%:%
%:%358=197%:%
%:%359=197%:%
%:%368=199%:%
%:%369=200%:%
%:%370=201%:%
%:%371=202%:%
%:%372=203%:%
%:%373=204%:%
%:%374=205%:%
%:%375=206%:%
%:%376=207%:%
%:%377=208%:%
%:%378=209%:%
%:%379=210%:%
%:%380=211%:%
%:%381=212%:%
%:%382=213%:%
%:%383=214%:%
%:%384=215%:%
%:%385=216%:%
%:%386=217%:%
%:%387=218%:%
%:%388=219%:%
%:%389=220%:%
%:%390=221%:%
%:%391=222%:%
%:%392=223%:%
%:%393=224%:%
%:%394=225%:%
%:%395=226%:%
%:%396=227%:%
%:%397=228%:%
%:%398=229%:%
%:%399=230%:%
%:%400=231%:%
%:%401=232%:%
%:%402=233%:%
%:%403=234%:%
%:%404=235%:%
%:%405=236%:%
%:%406=237%:%
%:%407=238%:%
%:%408=239%:%
%:%409=240%:%
%:%410=241%:%
%:%411=242%:%
%:%412=243%:%
%:%414=245%:%
%:%415=245%:%
%:%422=246%:%
%:%423=246%:%
%:%424=247%:%
%:%425=247%:%
%:%426=248%:%
%:%427=248%:%
%:%428=249%:%
%:%429=249%:%
%:%430=249%:%
%:%431=250%:%
%:%432=250%:%
%:%433=251%:%
%:%434=251%:%
%:%435=251%:%
%:%436=252%:%
%:%437=252%:%
%:%438=253%:%
%:%439=253%:%
%:%440=253%:%
%:%441=254%:%
%:%442=254%:%
%:%443=255%:%
%:%444=255%:%
%:%445=255%:%
%:%446=256%:%
%:%447=256%:%
%:%448=257%:%
%:%449=257%:%
%:%450=258%:%
%:%451=258%:%
%:%452=259%:%
%:%453=259%:%
%:%454=260%:%
%:%455=260%:%
%:%456=261%:%
%:%457=261%:%
%:%458=262%:%
%:%459=262%:%
%:%460=262%:%
%:%461=263%:%
%:%462=263%:%
%:%463=264%:%
%:%464=264%:%
%:%465=265%:%
%:%466=265%:%
%:%467=266%:%
%:%468=266%:%
%:%469=266%:%
%:%470=267%:%
%:%471=267%:%
%:%472=268%:%
%:%473=268%:%
%:%474=268%:%
%:%475=269%:%
%:%476=269%:%
%:%477=270%:%
%:%478=270%:%
%:%479=270%:%
%:%480=271%:%
%:%481=271%:%
%:%482=272%:%
%:%483=272%:%
%:%484=272%:%
%:%485=273%:%
%:%486=273%:%
%:%487=274%:%
%:%488=274%:%
%:%489=274%:%
%:%490=275%:%
%:%491=275%:%
%:%492=276%:%
%:%493=276%:%
%:%494=276%:%
%:%495=277%:%
%:%496=277%:%
%:%497=277%:%
%:%498=278%:%
%:%499=278%:%
%:%500=278%:%
%:%501=279%:%
%:%502=279%:%
%:%503=280%:%
%:%513=282%:%
%:%515=284%:%
%:%516=284%:%
%:%523=285%:%
%:%524=285%:%
%:%525=286%:%
%:%526=286%:%
%:%527=287%:%
%:%528=287%:%
%:%529=288%:%
%:%530=288%:%
%:%531=288%:%
%:%532=289%:%
%:%533=289%:%
%:%534=290%:%
%:%535=290%:%
%:%536=291%:%
%:%537=291%:%
%:%538=292%:%
%:%539=292%:%
%:%540=293%:%
%:%541=293%:%
%:%542=293%:%
%:%543=293%:%
%:%544=294%:%
%:%545=294%:%
%:%554=296%:%
%:%555=297%:%
%:%556=298%:%
%:%557=299%:%
%:%558=300%:%
%:%559=301%:%
%:%560=302%:%
%:%561=303%:%
%:%562=304%:%
%:%564=306%:%
%:%565=306%:%
%:%567=308%:%
%:%568=309%:%
%:%569=310%:%
%:%570=311%:%
%:%571=312%:%
%:%572=313%:%
%:%573=314%:%
%:%574=315%:%
%:%575=316%:%
%:%576=317%:%
%:%577=318%:%
%:%578=319%:%
%:%579=320%:%
%:%580=321%:%
%:%581=322%:%
%:%582=323%:%
%:%584=325%:%
%:%585=325%:%
%:%588=326%:%
%:%592=326%:%
%:%593=326%:%
%:%594=327%:%
%:%595=327%:%
%:%596=328%:%
%:%597=328%:%
%:%598=329%:%
%:%599=329%:%
%:%600=330%:%
%:%601=330%:%
%:%602=330%:%
%:%603=331%:%
%:%604=331%:%
%:%605=332%:%
%:%606=332%:%
%:%607=332%:%
%:%608=333%:%
%:%609=333%:%
%:%610=334%:%
%:%611=334%:%
%:%612=335%:%
%:%613=335%:%
%:%614=336%:%
%:%624=338%:%
%:%626=340%:%
%:%627=340%:%
%:%630=341%:%
%:%634=341%:%
%:%635=341%:%
%:%636=341%:%
%:%645=343%:%
%:%646=344%:%
%:%647=345%:%
%:%648=346%:%
%:%649=347%:%
%:%650=348%:%
%:%651=349%:%
%:%652=350%:%
%:%653=351%:%
%:%654=352%:%
%:%655=353%:%
%:%656=354%:%
%:%657=355%:%
%:%658=356%:%
%:%659=357%:%
%:%660=358%:%
%:%662=360%:%
%:%663=360%:%
%:%664=361%:%
%:%665=362%:%
%:%672=363%:%
%:%673=363%:%
%:%674=364%:%
%:%675=364%:%
%:%676=365%:%
%:%677=365%:%
%:%678=365%:%
%:%679=366%:%
%:%680=366%:%
%:%681=366%:%
%:%682=367%:%
%:%683=367%:%
%:%684=368%:%
%:%685=368%:%
%:%686=368%:%
%:%687=369%:%
%:%688=369%:%
%:%689=370%:%
%:%690=370%:%
%:%691=371%:%
%:%692=371%:%
%:%693=372%:%
%:%703=374%:%
%:%705=376%:%
%:%706=376%:%
%:%707=377%:%
%:%710=378%:%
%:%714=378%:%
%:%715=378%:%
%:%716=378%:%
%:%725=380%:%
%:%726=381%:%
%:%727=382%:%
%:%728=383%:%
%:%729=384%:%
%:%730=385%:%
%:%731=386%:%
%:%732=387%:%
%:%733=388%:%
%:%734=389%:%
%:%735=390%:%
%:%736=391%:%
%:%737=392%:%
%:%738=393%:%
%:%739=394%:%
%:%740=395%:%
%:%741=396%:%
%:%742=397%:%
%:%743=398%:%
%:%744=399%:%
%:%745=400%:%
%:%746=401%:%
%:%747=402%:%
%:%748=403%:%
%:%749=404%:%
%:%750=405%:%
%:%751=406%:%
%:%752=407%:%
%:%753=408%:%
%:%754=409%:%
%:%755=410%:%
%:%756=411%:%
%:%757=412%:%
%:%758=413%:%
%:%759=414%:%
%:%761=416%:%
%:%762=416%:%
%:%763=417%:%
%:%764=418%:%
%:%765=419%:%
%:%768=420%:%
%:%772=420%:%
%:%773=420%:%
%:%774=421%:%
%:%775=421%:%
%:%776=422%:%
%:%777=422%:%
%:%778=423%:%
%:%779=423%:%
%:%780=424%:%
%:%781=424%:%
%:%782=425%:%
%:%783=425%:%
%:%784=426%:%
%:%785=426%:%
%:%786=427%:%
%:%787=427%:%
%:%788=427%:%
%:%789=428%:%
%:%790=428%:%
%:%791=429%:%
%:%792=429%:%
%:%793=429%:%
%:%794=430%:%
%:%795=430%:%
%:%796=431%:%
%:%797=431%:%
%:%798=432%:%
%:%799=432%:%
%:%800=433%:%
%:%801=433%:%
%:%802=433%:%
%:%803=434%:%
%:%804=434%:%
%:%805=435%:%
%:%806=435%:%
%:%807=435%:%
%:%808=436%:%
%:%809=436%:%
%:%810=437%:%
%:%811=437%:%
%:%812=437%:%
%:%813=438%:%
%:%814=438%:%
%:%815=439%:%
%:%816=439%:%
%:%817=439%:%
%:%818=440%:%
%:%819=440%:%
%:%820=441%:%
%:%821=441%:%
%:%822=442%:%
%:%823=442%:%
%:%824=442%:%
%:%825=443%:%
%:%826=443%:%
%:%827=444%:%
%:%828=444%:%
%:%829=444%:%
%:%830=445%:%
%:%831=445%:%
%:%832=445%:%
%:%833=446%:%
%:%834=446%:%
%:%835=447%:%
%:%836=447%:%
%:%837=448%:%
%:%838=448%:%
%:%839=448%:%
%:%840=449%:%
%:%841=449%:%
%:%842=450%:%
%:%843=450%:%
%:%844=451%:%
%:%845=451%:%
%:%846=451%:%
%:%847=452%:%
%:%848=452%:%
%:%849=453%:%
%:%850=453%:%
%:%851=454%:%
%:%852=454%:%
%:%853=455%:%
%:%854=455%:%
%:%855=455%:%
%:%856=456%:%
%:%857=456%:%
%:%858=457%:%
%:%859=457%:%
%:%860=458%:%
%:%861=458%:%
%:%862=458%:%
%:%863=459%:%
%:%864=459%:%
%:%865=460%:%
%:%866=460%:%
%:%867=460%:%
%:%868=461%:%
%:%869=461%:%
%:%870=461%:%
%:%871=462%:%
%:%872=462%:%
%:%873=462%:%
%:%874=463%:%
%:%875=463%:%
%:%876=464%:%
%:%877=464%:%
%:%878=465%:%
%:%888=467%:%
%:%890=469%:%
%:%891=469%:%
%:%892=470%:%
%:%893=471%:%
%:%894=472%:%
%:%897=473%:%
%:%901=473%:%
%:%902=473%:%
%:%903=474%:%
%:%904=474%:%
%:%905=475%:%
%:%906=475%:%
%:%907=476%:%
%:%908=476%:%
%:%909=476%:%
%:%910=477%:%
%:%911=477%:%
%:%912=477%:%
%:%913=477%:%
%:%914=478%:%
%:%915=478%:%
%:%916=478%:%
%:%917=479%:%
%:%918=479%:%
%:%919=480%:%
%:%920=480%:%
%:%921=480%:%
%:%922=481%:%
%:%923=481%:%
%:%924=482%:%
%:%925=482%:%
%:%926=482%:%
%:%927=483%:%
%:%928=483%:%
%:%942=485%:%
%:%954=487%:%
%:%955=488%:%
%:%956=489%:%
%:%957=490%:%
%:%958=491%:%
%:%959=492%:%
%:%960=493%:%
%:%961=494%:%
%:%962=495%:%
%:%963=496%:%
%:%964=497%:%
%:%965=498%:%
%:%966=499%:%
%:%967=500%:%
%:%968=501%:%
%:%968=502%:%
%:%969=503%:%
%:%970=504%:%
%:%971=505%:%
%:%972=506%:%
%:%973=507%:%
%:%974=508%:%
%:%975=509%:%
%:%976=510%:%
%:%977=511%:%
%:%978=512%:%
%:%979=513%:%
%:%980=514%:%
%:%981=515%:%
%:%982=516%:%
%:%983=517%:%
%:%984=518%:%
%:%985=519%:%
%:%986=520%:%
%:%987=521%:%
%:%988=522%:%
%:%989=523%:%
%:%990=524%:%
%:%991=525%:%
%:%992=526%:%
%:%993=527%:%
%:%994=528%:%
%:%995=529%:%
%:%996=530%:%
%:%998=532%:%
%:%999=532%:%
%:%1000=533%:%
%:%1001=534%:%
%:%1002=535%:%
%:%1003=536%:%
%:%1004=537%:%
%:%1011=538%:%
%:%1012=538%:%
%:%1013=539%:%
%:%1014=539%:%
%:%1015=540%:%
%:%1016=540%:%
%:%1017=540%:%
%:%1018=540%:%
%:%1019=541%:%
%:%1020=541%:%
%:%1021=541%:%
%:%1022=542%:%
%:%1023=542%:%
%:%1024=543%:%
%:%1025=543%:%
%:%1026=544%:%
%:%1027=544%:%
%:%1028=544%:%
%:%1029=544%:%
%:%1030=545%:%
%:%1031=545%:%
%:%1032=546%:%
%:%1033=546%:%
%:%1034=547%:%
%:%1035=547%:%
%:%1036=548%:%
%:%1037=548%:%
%:%1038=549%:%
%:%1039=549%:%
%:%1040=550%:%
%:%1041=550%:%
%:%1042=551%:%
%:%1043=551%:%
%:%1044=552%:%
%:%1045=552%:%
%:%1046=552%:%
%:%1047=553%:%
%:%1048=553%:%
%:%1049=553%:%
%:%1050=554%:%
%:%1051=554%:%
%:%1052=555%:%
%:%1053=555%:%
%:%1054=555%:%
%:%1055=556%:%
%:%1056=556%:%
%:%1057=557%:%
%:%1058=557%:%
%:%1059=557%:%
%:%1060=558%:%
%:%1061=558%:%
%:%1062=559%:%
%:%1063=559%:%
%:%1064=559%:%
%:%1065=560%:%
%:%1066=560%:%
%:%1067=561%:%
%:%1068=561%:%
%:%1069=561%:%
%:%1070=562%:%
%:%1071=562%:%
%:%1072=563%:%
%:%1073=563%:%
%:%1074=564%:%
%:%1075=564%:%
%:%1076=565%:%
%:%1077=565%:%
%:%1078=565%:%
%:%1079=566%:%
%:%1089=568%:%
%:%1091=570%:%
%:%1092=570%:%
%:%1093=571%:%
%:%1094=572%:%
%:%1095=573%:%
%:%1096=574%:%
%:%1097=575%:%
%:%1104=576%:%
%:%1105=576%:%
%:%1106=577%:%
%:%1107=577%:%
%:%1108=577%:%
%:%1109=577%:%
%:%1110=578%:%
%:%1111=578%:%
%:%1112=578%:%
%:%1113=578%:%
%:%1114=579%:%
%:%1115=579%:%
%:%1116=580%:%
%:%1117=580%:%
%:%1118=581%:%
%:%1119=581%:%
%:%1120=582%:%
%:%1121=582%:%
%:%1122=583%:%
%:%1123=583%:%
%:%1124=583%:%
%:%1125=584%:%
%:%1126=584%:%
%:%1127=585%:%
%:%1128=585%:%
%:%1129=586%:%
%:%1130=586%:%
%:%1131=587%:%
%:%1132=587%:%
%:%1133=588%:%
%:%1134=588%:%
%:%1135=588%:%
%:%1136=589%:%
%:%1137=589%:%
%:%1138=589%:%
%:%1139=589%:%
%:%1140=590%:%
%:%1141=590%:%
%:%1142=590%:%
%:%1143=591%:%
%:%1144=591%:%
%:%1145=592%:%
%:%1146=592%:%
%:%1147=592%:%
%:%1148=593%:%
%:%1149=593%:%
%:%1150=594%:%
%:%1151=594%:%
%:%1152=594%:%
%:%1153=595%:%
%:%1154=595%:%
%:%1155=596%:%
%:%1156=596%:%
%:%1157=597%:%
%:%1158=597%:%
%:%1159=597%:%
%:%1160=597%:%
%:%1161=598%:%
%:%1162=598%:%
%:%1163=599%:%
%:%1164=599%:%
%:%1165=599%:%
%:%1166=599%:%
%:%1167=599%:%
%:%1168=600%:%
%:%1178=602%:%
%:%1179=603%:%
%:%1180=604%:%
%:%1181=605%:%
%:%1182=606%:%
%:%1183=607%:%
%:%1184=608%:%
%:%1185=609%:%
%:%1186=610%:%
%:%1187=611%:%
%:%1188=612%:%
%:%1189=613%:%
%:%1190=614%:%
%:%1191=615%:%
%:%1192=616%:%
%:%1193=617%:%
%:%1194=618%:%
%:%1195=619%:%
%:%1196=620%:%
%:%1197=621%:%
%:%1198=622%:%
%:%1199=623%:%
%:%1200=624%:%
%:%1201=625%:%
%:%1202=626%:%
%:%1203=627%:%
%:%1204=628%:%
%:%1205=629%:%
%:%1206=630%:%
%:%1207=631%:%
%:%1208=632%:%
%:%1209=633%:%
%:%1210=634%:%
%:%1211=635%:%
%:%1213=637%:%
%:%1214=637%:%
%:%1215=638%:%
%:%1216=639%:%
%:%1217=640%:%
%:%1218=641%:%
%:%1219=642%:%
%:%1226=643%:%
%:%1227=643%:%
%:%1228=644%:%
%:%1229=644%:%
%:%1230=645%:%
%:%1231=645%:%
%:%1232=646%:%
%:%1233=646%:%
%:%1234=646%:%
%:%1235=647%:%
%:%1236=647%:%
%:%1237=648%:%
%:%1238=648%:%
%:%1239=649%:%
%:%1240=649%:%
%:%1241=649%:%
%:%1242=650%:%
%:%1243=650%:%
%:%1244=650%:%
%:%1245=651%:%
%:%1246=651%:%
%:%1247=651%:%
%:%1248=652%:%
%:%1249=652%:%
%:%1250=653%:%
%:%1251=653%:%
%:%1252=653%:%
%:%1253=654%:%
%:%1254=654%:%
%:%1255=655%:%
%:%1256=655%:%
%:%1257=656%:%
%:%1258=656%:%
%:%1259=656%:%
%:%1260=657%:%
%:%1261=657%:%
%:%1262=658%:%
%:%1263=658%:%
%:%1264=658%:%
%:%1265=659%:%
%:%1266=659%:%
%:%1267=660%:%
%:%1268=660%:%
%:%1269=660%:%
%:%1270=661%:%
%:%1271=661%:%
%:%1272=662%:%
%:%1273=662%:%
%:%1274=663%:%
%:%1275=663%:%
%:%1276=663%:%
%:%1277=664%:%
%:%1278=664%:%
%:%1279=665%:%
%:%1280=665%:%
%:%1281=665%:%
%:%1282=666%:%
%:%1283=666%:%
%:%1284=666%:%
%:%1285=667%:%
%:%1286=667%:%
%:%1287=668%:%
%:%1288=668%:%
%:%1289=669%:%
%:%1290=669%:%
%:%1291=670%:%
%:%1292=670%:%
%:%1293=670%:%
%:%1294=671%:%
%:%1295=671%:%
%:%1296=672%:%
%:%1297=672%:%
%:%1298=672%:%
%:%1299=673%:%
%:%1300=673%:%
%:%1301=674%:%
%:%1302=674%:%
%:%1303=675%:%
%:%1304=676%:%
%:%1305=676%:%
%:%1306=677%:%
%:%1307=677%:%
%:%1308=677%:%
%:%1309=678%:%
%:%1310=679%:%
%:%1311=679%:%
%:%1312=680%:%
%:%1313=680%:%
%:%1314=680%:%
%:%1315=681%:%
%:%1316=681%:%
%:%1317=681%:%
%:%1318=682%:%
%:%1319=682%:%
%:%1320=682%:%
%:%1321=683%:%
%:%1322=683%:%
%:%1323=684%:%
%:%1324=684%:%
%:%1325=684%:%
%:%1326=685%:%
%:%1327=685%:%
%:%1328=686%:%
%:%1329=686%:%
%:%1330=687%:%
%:%1331=687%:%
%:%1332=687%:%
%:%1333=688%:%
%:%1334=688%:%
%:%1335=689%:%
%:%1336=689%:%
%:%1337=690%:%
%:%1338=690%:%
%:%1339=691%:%
%:%1340=691%:%
%:%1341=691%:%
%:%1342=692%:%
%:%1343=692%:%
%:%1344=692%:%
%:%1345=693%:%
%:%1346=693%:%
%:%1347=693%:%
%:%1348=694%:%
%:%1349=694%:%
%:%1350=694%:%
%:%1351=695%:%
%:%1352=695%:%
%:%1353=696%:%
%:%1354=696%:%
%:%1355=696%:%
%:%1356=697%:%
%:%1357=697%:%
%:%1358=697%:%
%:%1359=698%:%
%:%1360=698%:%
%:%1361=698%:%
%:%1362=699%:%
%:%1363=699%:%
%:%1364=699%:%
%:%1365=700%:%
%:%1366=700%:%
%:%1367=700%:%
%:%1368=701%:%
%:%1369=701%:%
%:%1370=702%:%
%:%1371=702%:%
%:%1372=702%:%
%:%1373=703%:%
%:%1383=705%:%
%:%1385=707%:%
%:%1386=707%:%
%:%1387=708%:%
%:%1388=709%:%
%:%1389=710%:%
%:%1390=711%:%
%:%1391=712%:%
%:%1398=713%:%
%:%1399=713%:%
%:%1400=714%:%
%:%1401=714%:%
%:%1402=715%:%
%:%1403=715%:%
%:%1404=715%:%
%:%1405=715%:%
%:%1406=716%:%
%:%1407=716%:%
%:%1408=716%:%
%:%1409=716%:%
%:%1410=717%:%
%:%1411=717%:%
%:%1412=717%:%
%:%1413=717%:%
%:%1414=717%:%
%:%1415=718%:%
%:%1416=718%:%
%:%1417=718%:%
%:%1418=719%:%
%:%1419=719%:%
%:%1420=719%:%
%:%1421=719%:%
%:%1422=720%:%
%:%1423=720%:%
%:%1424=720%:%
%:%1425=721%:%
%:%1426=721%:%
%:%1427=721%:%
%:%1428=721%:%
%:%1429=722%:%
%:%1430=722%:%
%:%1431=722%:%
%:%1432=722%:%
%:%1433=723%:%
%:%1443=725%:%
%:%1444=726%:%
%:%1445=727%:%
%:%1446=728%:%
%:%1447=729%:%
%:%1448=730%:%
%:%1449=731%:%
%:%1450=732%:%
%:%1451=733%:%
%:%1452=734%:%
%:%1453=735%:%
%:%1454=736%:%
%:%1455=737%:%
%:%1456=738%:%
%:%1457=739%:%
%:%1458=740%:%
%:%1459=741%:%
%:%1460=742%:%
%:%1461=743%:%
%:%1462=744%:%
%:%1463=745%:%
%:%1464=746%:%
%:%1466=748%:%
%:%1467=748%:%
%:%1468=749%:%
%:%1469=750%:%
%:%1470=751%:%
%:%1477=752%:%
%:%1478=752%:%
%:%1479=753%:%
%:%1480=753%:%
%:%1481=754%:%
%:%1482=754%:%
%:%1483=755%:%
%:%1484=755%:%
%:%1485=756%:%
%:%1486=756%:%
%:%1487=757%:%
%:%1488=757%:%
%:%1489=757%:%
%:%1490=758%:%
%:%1491=758%:%
%:%1492=758%:%
%:%1493=759%:%
%:%1494=759%:%
%:%1495=760%:%
%:%1496=760%:%
%:%1497=760%:%
%:%1498=761%:%
%:%1499=761%:%
%:%1500=762%:%
%:%1501=762%:%
%:%1502=763%:%
%:%1503=763%:%
%:%1504=764%:%
%:%1514=766%:%
%:%1516=768%:%
%:%1517=768%:%
%:%1518=769%:%
%:%1519=770%:%
%:%1520=771%:%
%:%1523=772%:%
%:%1527=772%:%
%:%1528=772%:%
%:%1529=772%:%
%:%1538=774%:%
%:%1539=775%:%
%:%1540=776%:%
%:%1541=777%:%
%:%1542=778%:%
%:%1543=779%:%
%:%1544=780%:%
%:%1545=781%:%
%:%1546=782%:%
%:%1547=783%:%
%:%1548=784%:%
%:%1549=785%:%
%:%1550=786%:%
%:%1551=787%:%
%:%1552=788%:%
%:%1553=789%:%
%:%1554=790%:%
%:%1555=791%:%
%:%1556=792%:%
%:%1557=793%:%
%:%1558=794%:%
%:%1559=795%:%
%:%1560=796%:%
%:%1561=797%:%
%:%1562=798%:%
%:%1563=799%:%
%:%1564=800%:%
%:%1565=801%:%
%:%1566=802%:%
%:%1567=803%:%
%:%1568=804%:%
%:%1569=805%:%
%:%1570=806%:%
%:%1571=807%:%
%:%1572=808%:%
%:%1573=809%:%
%:%1574=810%:%
%:%1575=811%:%
%:%1576=812%:%
%:%1577=813%:%
%:%1578=814%:%
%:%1579=815%:%
%:%1580=816%:%
%:%1581=817%:%
%:%1582=818%:%
%:%1583=819%:%
%:%1584=820%:%
%:%1585=821%:%
%:%1586=822%:%
%:%1587=823%:%
%:%1588=824%:%
%:%1589=825%:%
%:%1590=826%:%
%:%1591=827%:%
%:%1592=828%:%
%:%1593=829%:%
%:%1594=830%:%
%:%1595=831%:%
%:%1596=832%:%
%:%1597=833%:%
%:%1598=834%:%
%:%1599=835%:%
%:%1600=836%:%
%:%1601=837%:%
%:%1602=838%:%
%:%1603=839%:%
%:%1604=840%:%
%:%1605=841%:%
%:%1606=842%:%
%:%1607=843%:%
%:%1608=844%:%
%:%1609=845%:%
%:%1611=847%:%
%:%1612=847%:%
%:%1613=848%:%
%:%1614=849%:%
%:%1615=850%:%
%:%1616=851%:%
%:%1617=852%:%
%:%1624=853%:%
%:%1625=853%:%
%:%1626=854%:%
%:%1627=854%:%
%:%1628=855%:%
%:%1629=855%:%
%:%1630=856%:%
%:%1631=856%:%
%:%1632=856%:%
%:%1633=857%:%
%:%1634=857%:%
%:%1638=861%:%
%:%1639=862%:%
%:%1640=862%:%
%:%1641=862%:%
%:%1642=863%:%
%:%1643=863%:%
%:%1644=863%:%
%:%1647=866%:%
%:%1648=867%:%
%:%1649=867%:%
%:%1650=867%:%
%:%1651=868%:%
%:%1652=868%:%
%:%1653=868%:%
%:%1654=869%:%
%:%1655=869%:%
%:%1656=870%:%
%:%1657=870%:%
%:%1658=871%:%
%:%1659=871%:%
%:%1660=871%:%
%:%1661=872%:%
%:%1662=872%:%
%:%1663=873%:%
%:%1664=873%:%
%:%1665=873%:%
%:%1666=874%:%
%:%1667=874%:%
%:%1668=875%:%
%:%1669=875%:%
%:%1670=876%:%
%:%1671=876%:%
%:%1672=877%:%
%:%1673=877%:%
%:%1674=878%:%
%:%1675=878%:%
%:%1676=879%:%
%:%1677=879%:%
%:%1678=880%:%
%:%1679=880%:%
%:%1680=881%:%
%:%1681=881%:%
%:%1682=882%:%
%:%1683=882%:%
%:%1684=882%:%
%:%1685=883%:%
%:%1686=883%:%
%:%1687=883%:%
%:%1688=884%:%
%:%1689=884%:%
%:%1690=884%:%
%:%1691=885%:%
%:%1692=885%:%
%:%1693=885%:%
%:%1694=886%:%
%:%1695=886%:%
%:%1696=886%:%
%:%1697=887%:%
%:%1698=887%:%
%:%1699=888%:%
%:%1700=888%:%
%:%1701=889%:%
%:%1702=889%:%
%:%1703=889%:%
%:%1704=890%:%
%:%1705=890%:%
%:%1706=890%:%
%:%1707=891%:%
%:%1708=891%:%
%:%1709=892%:%
%:%1710=892%:%
%:%1711=893%:%
%:%1712=893%:%
%:%1713=893%:%
%:%1714=894%:%
%:%1715=894%:%
%:%1716=895%:%
%:%1717=895%:%
%:%1718=896%:%
%:%1719=896%:%
%:%1720=896%:%
%:%1721=897%:%
%:%1722=897%:%
%:%1723=897%:%
%:%1724=898%:%
%:%1725=898%:%
%:%1726=899%:%
%:%1727=899%:%
%:%1728=899%:%
%:%1729=900%:%
%:%1730=900%:%
%:%1731=901%:%
%:%1732=901%:%
%:%1733=901%:%
%:%1734=902%:%
%:%1735=902%:%
%:%1736=903%:%
%:%1737=903%:%
%:%1738=904%:%
%:%1739=904%:%
%:%1740=904%:%
%:%1741=905%:%
%:%1742=905%:%
%:%1743=905%:%
%:%1744=906%:%
%:%1745=906%:%
%:%1746=907%:%
%:%1747=907%:%
%:%1748=907%:%
%:%1749=908%:%
%:%1750=908%:%
%:%1751=909%:%
%:%1752=909%:%
%:%1753=910%:%
%:%1754=910%:%
%:%1755=910%:%
%:%1756=911%:%
%:%1757=911%:%
%:%1758=912%:%
%:%1759=912%:%
%:%1760=913%:%
%:%1761=913%:%
%:%1762=914%:%
%:%1763=914%:%
%:%1764=914%:%
%:%1765=915%:%
%:%1766=915%:%
%:%1767=916%:%
%:%1768=916%:%
%:%1769=917%:%
%:%1770=917%:%
%:%1771=918%:%
%:%1772=918%:%
%:%1773=919%:%
%:%1774=919%:%
%:%1775=920%:%
%:%1776=920%:%
%:%1777=921%:%
%:%1778=921%:%
%:%1779=922%:%
%:%1780=922%:%
%:%1781=923%:%
%:%1782=923%:%
%:%1783=923%:%
%:%1784=924%:%
%:%1785=924%:%
%:%1786=924%:%
%:%1787=925%:%
%:%1788=925%:%
%:%1789=925%:%
%:%1790=926%:%
%:%1791=926%:%
%:%1792=926%:%
%:%1793=927%:%
%:%1794=927%:%
%:%1795=927%:%
%:%1796=928%:%
%:%1797=928%:%
%:%1798=929%:%
%:%1799=929%:%
%:%1800=930%:%
%:%1801=930%:%
%:%1802=931%:%
%:%1803=931%:%
%:%1804=932%:%
%:%1805=932%:%
%:%1806=933%:%
%:%1807=933%:%
%:%1808=934%:%
%:%1809=934%:%
%:%1810=934%:%
%:%1811=935%:%
%:%1812=935%:%
%:%1813=936%:%
%:%1814=936%:%
%:%1815=936%:%
%:%1816=937%:%
%:%1817=937%:%
%:%1818=937%:%
%:%1819=938%:%
%:%1820=938%:%
%:%1821=939%:%
%:%1822=939%:%
%:%1823=940%:%
%:%1824=940%:%
%:%1825=941%:%
%:%1826=941%:%
%:%1827=942%:%
%:%1828=942%:%
%:%1829=943%:%
%:%1830=943%:%
%:%1831=944%:%
%:%1832=944%:%
%:%1833=945%:%
%:%1834=945%:%
%:%1835=946%:%
%:%1836=946%:%
%:%1837=946%:%
%:%1838=947%:%
%:%1839=947%:%
%:%1840=948%:%
%:%1841=948%:%
%:%1842=948%:%
%:%1843=949%:%
%:%1844=949%:%
%:%1845=949%:%
%:%1846=950%:%
%:%1847=950%:%
%:%1848=951%:%
%:%1849=951%:%
%:%1850=952%:%
%:%1851=952%:%
%:%1852=953%:%
%:%1853=953%:%
%:%1854=954%:%
%:%1855=954%:%
%:%1856=955%:%
%:%1857=955%:%
%:%1858=956%:%
%:%1859=956%:%
%:%1862=959%:%
%:%1863=960%:%
%:%1864=960%:%
%:%1865=960%:%
%:%1866=961%:%
%:%1876=963%:%
%:%1878=965%:%
%:%1879=965%:%
%:%1880=966%:%
%:%1881=967%:%
%:%1882=968%:%
%:%1883=969%:%
%:%1884=970%:%
%:%1891=971%:%
%:%1892=971%:%
%:%1893=972%:%
%:%1894=972%:%
%:%1895=973%:%
%:%1896=973%:%
%:%1897=973%:%
%:%1898=973%:%
%:%1899=974%:%
%:%1900=974%:%
%:%1901=975%:%
%:%1902=975%:%
%:%1903=976%:%
%:%1904=977%:%
%:%1905=978%:%
%:%1906=979%:%
%:%1907=979%:%
%:%1908=980%:%
%:%1909=980%:%
%:%1910=980%:%
%:%1911=980%:%
%:%1912=981%:%
%:%1913=981%:%
%:%1914=981%:%
%:%1915=982%:%
%:%1925=984%:%
%:%1926=985%:%
%:%1927=986%:%
%:%1928=987%:%
%:%1929=988%:%
%:%1930=989%:%
%:%1931=990%:%
%:%1932=991%:%
%:%1933=992%:%
%:%1934=993%:%
%:%1935=994%:%
%:%1936=995%:%
%:%1937=996%:%
%:%1938=997%:%
%:%1939=998%:%
%:%1940=999%:%
%:%1942=1001%:%
%:%1943=1001%:%
%:%1944=1002%:%
%:%1945=1003%:%
%:%1946=1004%:%
%:%1949=1005%:%
%:%1953=1005%:%
%:%1954=1005%:%
%:%1955=1006%:%
%:%1956=1006%:%
%:%1957=1007%:%
%:%1958=1007%:%
%:%1959=1008%:%
%:%1960=1008%:%
%:%1961=1008%:%
%:%1962=1009%:%
%:%1963=1009%:%
%:%1964=1010%:%
%:%1965=1010%:%
%:%1966=1010%:%
%:%1967=1011%:%
%:%1968=1011%:%
%:%1969=1012%:%
%:%1970=1012%:%
%:%1971=1013%:%
%:%1972=1013%:%
%:%1973=1014%:%
%:%1974=1014%:%
%:%1975=1015%:%
%:%1976=1015%:%
%:%1977=1016%:%
%:%1978=1016%:%
%:%1979=1016%:%
%:%1980=1017%:%
%:%1981=1017%:%
%:%1982=1017%:%
%:%1983=1018%:%
%:%1984=1018%:%
%:%1985=1018%:%
%:%1986=1019%:%
%:%1987=1019%:%
%:%1988=1020%:%
%:%1989=1020%:%
%:%1990=1020%:%
%:%1991=1021%:%
%:%1992=1021%:%
%:%1993=1022%:%
%:%1994=1022%:%
%:%1995=1023%:%
%:%1996=1023%:%
%:%1997=1024%:%
%:%2007=1026%:%
%:%2008=1027%:%
%:%2009=1028%:%
%:%2010=1029%:%
%:%2011=1030%:%
%:%2012=1031%:%
%:%2013=1032%:%
%:%2014=1033%:%
%:%2015=1034%:%
%:%2016=1035%:%
%:%2017=1036%:%
%:%2018=1037%:%
%:%2019=1038%:%
%:%2020=1039%:%
%:%2021=1040%:%
%:%2022=1041%:%
%:%2023=1042%:%
%:%2024=1043%:%
%:%2025=1044%:%
%:%2026=1045%:%
%:%2027=1046%:%
%:%2028=1047%:%
%:%2029=1048%:%
%:%2030=1049%:%
%:%2031=1050%:%
%:%2032=1051%:%
%:%2033=1052%:%
%:%2034=1053%:%
%:%2035=1054%:%
%:%2036=1055%:%
%:%2037=1056%:%
%:%2038=1057%:%
%:%2039=1058%:%
%:%2040=1059%:%
%:%2042=1061%:%
%:%2043=1061%:%
%:%2044=1062%:%
%:%2045=1063%:%
%:%2046=1064%:%
%:%2047=1065%:%
%:%2054=1066%:%
%:%2055=1066%:%
%:%2056=1067%:%
%:%2057=1067%:%
%:%2058=1068%:%
%:%2059=1068%:%
%:%2060=1069%:%
%:%2061=1069%:%
%:%2062=1069%:%
%:%2063=1070%:%
%:%2064=1070%:%
%:%2065=1071%:%
%:%2066=1071%:%
%:%2067=1072%:%
%:%2068=1072%:%
%:%2069=1072%:%
%:%2070=1073%:%
%:%2071=1073%:%
%:%2072=1074%:%
%:%2073=1074%:%
%:%2074=1074%:%
%:%2075=1075%:%
%:%2076=1075%:%
%:%2077=1076%:%
%:%2078=1076%:%
%:%2079=1076%:%
%:%2080=1077%:%
%:%2081=1077%:%
%:%2082=1078%:%
%:%2083=1078%:%
%:%2084=1078%:%
%:%2085=1079%:%
%:%2086=1079%:%
%:%2087=1080%:%
%:%2088=1080%:%
%:%2089=1080%:%
%:%2090=1081%:%
%:%2091=1081%:%
%:%2092=1082%:%
%:%2093=1082%:%
%:%2094=1082%:%
%:%2095=1083%:%
%:%2096=1083%:%
%:%2097=1084%:%
%:%2098=1084%:%
%:%2099=1084%:%
%:%2100=1085%:%
%:%2101=1085%:%
%:%2102=1085%:%
%:%2103=1086%:%
%:%2104=1086%:%
%:%2105=1086%:%
%:%2106=1087%:%
%:%2107=1087%:%
%:%2108=1088%:%
%:%2109=1088%:%
%:%2110=1088%:%
%:%2111=1089%:%
%:%2112=1089%:%
%:%2113=1090%:%
%:%2114=1090%:%
%:%2115=1090%:%
%:%2116=1091%:%
%:%2117=1091%:%
%:%2118=1091%:%
%:%2119=1092%:%
%:%2120=1092%:%
%:%2121=1093%:%
%:%2122=1093%:%
%:%2123=1094%:%
%:%2124=1094%:%
%:%2125=1094%:%
%:%2126=1095%:%
%:%2127=1095%:%
%:%2128=1095%:%
%:%2129=1096%:%
%:%2130=1096%:%
%:%2131=1096%:%
%:%2132=1097%:%
%:%2133=1097%:%
%:%2134=1098%:%
%:%2135=1098%:%
%:%2136=1098%:%
%:%2137=1099%:%
%:%2138=1099%:%
%:%2139=1099%:%
%:%2140=1100%:%
%:%2141=1100%:%
%:%2142=1101%:%
%:%2143=1101%:%
%:%2144=1102%:%
%:%2145=1102%:%
%:%2146=1103%:%
%:%2147=1103%:%
%:%2148=1104%:%
%:%2149=1104%:%
%:%2150=1105%:%
%:%2151=1105%:%
%:%2152=1105%:%
%:%2153=1106%:%
%:%2154=1106%:%
%:%2155=1107%:%
%:%2156=1107%:%
%:%2157=1108%:%
%:%2158=1108%:%
%:%2159=1108%:%
%:%2160=1109%:%
%:%2170=1111%:%
%:%2172=1113%:%
%:%2173=1113%:%
%:%2174=1114%:%
%:%2175=1115%:%
%:%2176=1116%:%
%:%2177=1117%:%
%:%2184=1118%:%
%:%2185=1118%:%
%:%2186=1119%:%
%:%2187=1119%:%
%:%2188=1119%:%
%:%2189=1120%:%
%:%2190=1121%:%
%:%2191=1121%:%
%:%2192=1122%:%
%:%2193=1122%:%
%:%2194=1123%:%
%:%2195=1123%:%
%:%2196=1123%:%
%:%2197=1123%:%
%:%2198=1124%:%
%:%2199=1124%:%
%:%2200=1125%:%
%:%2201=1125%:%
%:%2202=1125%:%
%:%2203=1126%:%
%:%2204=1126%:%
%:%2205=1126%:%
%:%2206=1126%:%
%:%2207=1127%:%
%:%2208=1127%:%
%:%2209=1127%:%
%:%2210=1128%:%
%:%2211=1128%:%
%:%2212=1128%:%
%:%2213=1129%:%
%:%2214=1129%:%
%:%2215=1129%:%
%:%2216=1129%:%
%:%2217=1129%:%
%:%2218=1130%:%
%:%2233=1132%:%
%:%2245=1134%:%
%:%2246=1135%:%
%:%2247=1136%:%
%:%2248=1137%:%
%:%2249=1138%:%
%:%2250=1139%:%
%:%2251=1140%:%
%:%2252=1141%:%
%:%2253=1142%:%
%:%2254=1143%:%
%:%2255=1144%:%
%:%2256=1145%:%
%:%2257=1146%:%
%:%2258=1147%:%
%:%2259=1148%:%
%:%2260=1149%:%
%:%2261=1150%:%
%:%2262=1151%:%
%:%2263=1152%:%
%:%2264=1153%:%
%:%2265=1154%:%
%:%2266=1155%:%
%:%2267=1156%:%
%:%2268=1157%:%
%:%2269=1158%:%
%:%2270=1159%:%
%:%2271=1160%:%
%:%2272=1161%:%
%:%2273=1162%:%
%:%2274=1163%:%
%:%2275=1164%:%
%:%2276=1165%:%
%:%2277=1166%:%
%:%2278=1167%:%
%:%2280=1169%:%
%:%2281=1169%:%
%:%2284=1170%:%
%:%2288=1170%:%
%:%2289=1170%:%
%:%2298=1172%:%
%:%2300=1174%:%
%:%2301=1174%:%
%:%2302=1175%:%
%:%2303=1176%:%
%:%2304=1177%:%
%:%2311=1178%:%
%:%2312=1178%:%
%:%2313=1179%:%
%:%2314=1179%:%
%:%2315=1180%:%
%:%2316=1180%:%
%:%2317=1180%:%
%:%2318=1181%:%
%:%2319=1181%:%
%:%2320=1182%:%
%:%2321=1182%:%
%:%2322=1182%:%
%:%2323=1183%:%
%:%2324=1183%:%
%:%2325=1184%:%
%:%2326=1184%:%
%:%2327=1185%:%
%:%2328=1185%:%
%:%2329=1185%:%
%:%2330=1186%:%
%:%2331=1186%:%
%:%2332=1187%:%
%:%2333=1187%:%
%:%2334=1187%:%
%:%2335=1188%:%
%:%2336=1188%:%
%:%2337=1189%:%
%:%2338=1189%:%
%:%2339=1189%:%
%:%2340=1190%:%
%:%2341=1190%:%
%:%2342=1190%:%
%:%2343=1190%:%
%:%2344=1191%:%
%:%2345=1191%:%
%:%2346=1192%:%
%:%2347=1192%:%
%:%2348=1193%:%
%:%2349=1193%:%
%:%2350=1193%:%
%:%2351=1194%:%
%:%2352=1194%:%
%:%2353=1194%:%
%:%2354=1195%:%
%:%2355=1195%:%
%:%2356=1196%:%
%:%2357=1196%:%
%:%2358=1197%:%
%:%2359=1197%:%
%:%2360=1198%:%
%:%2361=1198%:%
%:%2362=1198%:%
%:%2363=1199%:%
%:%2364=1199%:%
%:%2365=1200%:%
%:%2366=1200%:%
%:%2367=1200%:%
%:%2368=1201%:%
%:%2378=1203%:%
%:%2379=1204%:%
%:%2380=1205%:%
%:%2381=1206%:%
%:%2382=1207%:%
%:%2383=1208%:%
%:%2384=1209%:%
%:%2385=1210%:%
%:%2386=1211%:%
%:%2387=1212%:%
%:%2388=1213%:%
%:%2389=1214%:%
%:%2390=1215%:%
%:%2391=1216%:%
%:%2392=1217%:%
%:%2393=1218%:%
%:%2394=1219%:%
%:%2395=1220%:%
%:%2396=1221%:%
%:%2397=1222%:%
%:%2398=1223%:%
%:%2399=1224%:%
%:%2400=1225%:%
%:%2401=1226%:%
%:%2402=1227%:%
%:%2403=1228%:%
%:%2404=1229%:%
%:%2405=1230%:%
%:%2406=1231%:%
%:%2407=1232%:%
%:%2408=1233%:%
%:%2409=1234%:%
%:%2410=1235%:%
%:%2411=1236%:%
%:%2412=1237%:%
%:%2413=1238%:%
%:%2414=1239%:%
%:%2416=1241%:%
%:%2417=1241%:%
%:%2418=1242%:%
%:%2419=1243%:%
%:%2420=1244%:%
%:%2427=1245%:%
%:%2428=1245%:%
%:%2429=1246%:%
%:%2430=1246%:%
%:%2431=1247%:%
%:%2432=1247%:%
%:%2433=1248%:%
%:%2434=1248%:%
%:%2435=1249%:%
%:%2436=1249%:%
%:%2437=1250%:%
%:%2438=1250%:%
%:%2439=1251%:%
%:%2440=1251%:%
%:%2441=1252%:%
%:%2442=1252%:%
%:%2443=1253%:%
%:%2444=1253%:%
%:%2445=1254%:%
%:%2446=1254%:%
%:%2447=1254%:%
%:%2448=1255%:%
%:%2449=1255%:%
%:%2450=1256%:%
%:%2451=1256%:%
%:%2452=1256%:%
%:%2453=1257%:%
%:%2454=1257%:%
%:%2455=1257%:%
%:%2456=1258%:%
%:%2457=1258%:%
%:%2458=1259%:%
%:%2459=1259%:%
%:%2460=1259%:%
%:%2461=1260%:%
%:%2462=1260%:%
%:%2463=1261%:%
%:%2464=1261%:%
%:%2465=1262%:%
%:%2466=1262%:%
%:%2467=1262%:%
%:%2468=1263%:%
%:%2469=1263%:%
%:%2470=1263%:%
%:%2471=1264%:%
%:%2472=1264%:%
%:%2473=1264%:%
%:%2474=1265%:%
%:%2475=1265%:%
%:%2476=1266%:%
%:%2477=1266%:%
%:%2478=1266%:%
%:%2479=1267%:%
%:%2480=1267%:%
%:%2481=1268%:%
%:%2482=1268%:%
%:%2483=1269%:%
%:%2484=1269%:%
%:%2485=1269%:%
%:%2486=1270%:%
%:%2487=1270%:%
%:%2488=1270%:%
%:%2489=1271%:%
%:%2490=1271%:%
%:%2491=1272%:%
%:%2492=1272%:%
%:%2493=1272%:%
%:%2494=1273%:%
%:%2495=1273%:%
%:%2496=1273%:%
%:%2497=1274%:%
%:%2498=1274%:%
%:%2499=1275%:%
%:%2500=1275%:%
%:%2501=1275%:%
%:%2502=1276%:%
%:%2503=1276%:%
%:%2504=1277%:%
%:%2505=1277%:%
%:%2506=1278%:%
%:%2507=1278%:%
%:%2508=1278%:%
%:%2509=1279%:%
%:%2510=1279%:%
%:%2511=1279%:%
%:%2512=1280%:%
%:%2513=1280%:%
%:%2514=1280%:%
%:%2515=1281%:%
%:%2516=1281%:%
%:%2517=1282%:%
%:%2518=1282%:%
%:%2519=1282%:%
%:%2520=1283%:%
%:%2521=1283%:%
%:%2522=1284%:%
%:%2523=1284%:%
%:%2524=1285%:%
%:%2525=1285%:%
%:%2526=1286%:%
%:%2527=1286%:%
%:%2528=1287%:%
%:%2529=1287%:%
%:%2530=1288%:%
%:%2531=1288%:%
%:%2532=1289%:%
%:%2533=1289%:%
%:%2534=1290%:%
%:%2535=1290%:%
%:%2536=1290%:%
%:%2537=1291%:%
%:%2538=1291%:%
%:%2539=1292%:%
%:%2540=1292%:%
%:%2541=1292%:%
%:%2542=1293%:%
%:%2543=1293%:%
%:%2544=1293%:%
%:%2545=1294%:%
%:%2546=1294%:%
%:%2547=1295%:%
%:%2548=1295%:%
%:%2549=1295%:%
%:%2550=1296%:%
%:%2551=1296%:%
%:%2552=1297%:%
%:%2553=1297%:%
%:%2554=1298%:%
%:%2555=1298%:%
%:%2556=1298%:%
%:%2557=1299%:%
%:%2558=1299%:%
%:%2559=1299%:%
%:%2560=1300%:%
%:%2561=1300%:%
%:%2562=1300%:%
%:%2563=1301%:%
%:%2564=1301%:%
%:%2565=1302%:%
%:%2566=1302%:%
%:%2567=1302%:%
%:%2568=1303%:%
%:%2569=1303%:%
%:%2570=1304%:%
%:%2571=1304%:%
%:%2572=1305%:%
%:%2573=1305%:%
%:%2574=1305%:%
%:%2575=1306%:%
%:%2576=1306%:%
%:%2577=1306%:%
%:%2578=1307%:%
%:%2579=1307%:%
%:%2580=1308%:%
%:%2581=1308%:%
%:%2582=1308%:%
%:%2583=1309%:%
%:%2584=1309%:%
%:%2585=1309%:%
%:%2586=1310%:%
%:%2587=1310%:%
%:%2588=1311%:%
%:%2589=1311%:%
%:%2590=1311%:%
%:%2591=1312%:%
%:%2592=1312%:%
%:%2593=1313%:%
%:%2594=1313%:%
%:%2595=1314%:%
%:%2596=1314%:%
%:%2597=1314%:%
%:%2598=1315%:%
%:%2599=1315%:%
%:%2600=1315%:%
%:%2601=1316%:%
%:%2602=1316%:%
%:%2603=1316%:%
%:%2604=1317%:%
%:%2605=1317%:%
%:%2606=1318%:%
%:%2607=1318%:%
%:%2608=1318%:%
%:%2609=1319%:%
%:%2610=1319%:%
%:%2611=1320%:%
%:%2612=1320%:%
%:%2613=1321%:%
%:%2614=1321%:%
%:%2615=1322%:%
%:%2616=1322%:%
%:%2617=1323%:%
%:%2618=1323%:%
%:%2619=1323%:%
%:%2620=1324%:%
%:%2621=1324%:%
%:%2622=1325%:%
%:%2623=1325%:%
%:%2624=1325%:%
%:%2625=1326%:%
%:%2626=1326%:%
%:%2627=1326%:%
%:%2628=1327%:%
%:%2629=1327%:%
%:%2630=1328%:%
%:%2631=1328%:%
%:%2632=1329%:%
%:%2633=1329%:%
%:%2634=1330%:%
%:%2635=1330%:%
%:%2636=1330%:%
%:%2637=1331%:%
%:%2638=1331%:%
%:%2639=1332%:%
%:%2640=1332%:%
%:%2641=1332%:%
%:%2642=1333%:%
%:%2652=1335%:%
%:%2652=1336%:%
%:%2653=1337%:%
%:%2654=1338%:%
%:%2655=1339%:%
%:%2656=1340%:%
%:%2657=1341%:%
%:%2658=1342%:%
%:%2659=1343%:%
%:%2660=1344%:%
%:%2661=1345%:%
%:%2662=1346%:%
%:%2663=1347%:%
%:%2664=1348%:%
%:%2665=1349%:%
%:%2666=1350%:%
%:%2667=1351%:%
%:%2668=1352%:%
%:%2669=1353%:%
%:%2670=1354%:%
%:%2671=1355%:%
%:%2672=1356%:%
%:%2673=1357%:%
%:%2674=1358%:%
%:%2675=1359%:%
%:%2676=1360%:%
%:%2677=1361%:%
%:%2678=1362%:%
%:%2679=1363%:%
%:%2680=1364%:%
%:%2681=1365%:%
%:%2682=1366%:%
%:%2683=1367%:%
%:%2684=1368%:%
%:%2685=1369%:%
%:%2686=1370%:%
%:%2687=1371%:%
%:%2688=1372%:%
%:%2689=1373%:%
%:%2690=1374%:%
%:%2691=1375%:%
%:%2692=1376%:%
%:%2693=1377%:%
%:%2694=1378%:%
%:%2695=1379%:%
%:%2696=1380%:%
%:%2697=1381%:%
%:%2698=1382%:%
%:%2699=1383%:%
%:%2700=1384%:%
%:%2701=1385%:%
%:%2702=1386%:%
%:%2703=1387%:%
%:%2704=1388%:%
%:%2705=1389%:%
%:%2706=1390%:%
%:%2707=1391%:%
%:%2708=1392%:%
%:%2709=1393%:%
%:%2710=1394%:%
%:%2711=1395%:%
%:%2712=1396%:%
%:%2713=1397%:%
%:%2714=1398%:%
%:%2715=1399%:%
%:%2716=1400%:%
%:%2717=1401%:%
%:%2718=1402%:%
%:%2719=1403%:%
%:%2720=1404%:%
%:%2721=1405%:%
%:%2722=1406%:%
%:%2723=1407%:%
%:%2724=1408%:%
%:%2725=1409%:%
%:%2726=1410%:%
%:%2727=1411%:%
%:%2728=1412%:%
%:%2729=1413%:%
%:%2730=1414%:%
%:%2731=1415%:%
%:%2732=1416%:%
%:%2733=1417%:%
%:%2734=1418%:%
%:%2735=1419%:%
%:%2736=1420%:%
%:%2737=1421%:%
%:%2738=1422%:%
%:%2739=1423%:%
%:%2740=1424%:%
%:%2741=1425%:%
%:%2742=1426%:%
%:%2743=1427%:%
%:%2744=1428%:%
%:%2745=1429%:%
%:%2746=1430%:%
%:%2747=1431%:%
%:%2748=1432%:%
%:%2749=1433%:%
%:%2750=1434%:%
%:%2751=1435%:%
%:%2752=1436%:%
%:%2753=1437%:%
%:%2754=1438%:%
%:%2755=1439%:%
%:%2756=1440%:%
%:%2757=1441%:%
%:%2758=1442%:%
%:%2759=1443%:%
%:%2760=1444%:%
%:%2761=1445%:%
%:%2762=1446%:%
%:%2763=1447%:%
%:%2764=1448%:%
%:%2765=1449%:%
%:%2766=1450%:%
%:%2767=1451%:%
%:%2768=1452%:%
%:%2769=1453%:%
%:%2770=1454%:%
%:%2771=1455%:%
%:%2772=1456%:%
%:%2773=1457%:%
%:%2774=1458%:%
%:%2775=1459%:%
%:%2776=1460%:%
%:%2777=1461%:%
%:%2778=1462%:%
%:%2779=1463%:%
%:%2780=1464%:%
%:%2781=1465%:%
%:%2782=1466%:%
%:%2783=1467%:%
%:%2784=1468%:%
%:%2785=1469%:%
%:%2786=1470%:%
%:%2787=1471%:%
%:%2788=1472%:%
%:%2789=1473%:%
%:%2790=1474%:%
%:%2791=1475%:%
%:%2792=1476%:%
%:%2793=1477%:%
%:%2794=1478%:%
%:%2795=1479%:%
%:%2796=1480%:%
%:%2797=1481%:%
%:%2798=1482%:%
%:%2799=1483%:%
%:%2800=1484%:%
%:%2801=1485%:%
%:%2802=1486%:%
%:%2803=1487%:%
%:%2804=1488%:%
%:%2805=1489%:%
%:%2806=1490%:%
%:%2807=1491%:%
%:%2808=1492%:%
%:%2809=1493%:%
%:%2810=1494%:%
%:%2811=1495%:%
%:%2812=1496%:%
%:%2813=1497%:%
%:%2814=1498%:%
%:%2815=1499%:%
%:%2816=1500%:%
%:%2817=1501%:%
%:%2818=1502%:%
%:%2819=1503%:%
%:%2820=1504%:%
%:%2821=1505%:%
%:%2822=1506%:%
%:%2823=1507%:%
%:%2824=1508%:%
%:%2825=1509%:%
%:%2826=1510%:%
%:%2827=1511%:%
%:%2828=1512%:%
%:%2829=1513%:%
%:%2830=1514%:%
%:%2831=1515%:%
%:%2832=1516%:%
%:%2833=1517%:%
%:%2834=1518%:%
%:%2835=1519%:%
%:%2836=1520%:%
%:%2837=1521%:%
%:%2839=1523%:%
%:%2840=1523%:%
%:%2841=1524%:%
%:%2843=1526%:%
%:%2844=1527%:%
%:%2846=1529%:%
%:%2847=1529%:%
%:%2848=1530%:%
%:%2849=1531%:%
%:%2850=1532%:%
%:%2851=1533%:%
%:%2858=1534%:%
%:%2859=1534%:%
%:%2860=1535%:%
%:%2861=1535%:%
%:%2862=1536%:%
%:%2863=1536%:%
%:%2864=1536%:%
%:%2865=1536%:%
%:%2866=1537%:%
%:%2867=1537%:%
%:%2868=1537%:%
%:%2869=1538%:%
%:%2870=1538%:%
%:%2871=1538%:%
%:%2872=1539%:%
%:%2873=1539%:%
%:%2874=1540%:%
%:%2875=1540%:%
%:%2876=1540%:%
%:%2877=1541%:%
%:%2887=1543%:%
%:%2888=1544%:%
%:%2889=1545%:%
%:%2890=1546%:%
%:%2891=1547%:%
%:%2892=1548%:%
%:%2893=1549%:%
%:%2894=1550%:%
%:%2895=1551%:%
%:%2896=1552%:%
%:%2897=1553%:%
%:%2898=1554%:%
%:%2899=1555%:%
%:%2900=1556%:%
%:%2901=1557%:%
%:%2902=1558%:%
%:%2903=1559%:%
%:%2904=1560%:%
%:%2905=1561%:%
%:%2906=1562%:%
%:%2907=1563%:%
%:%2908=1564%:%
%:%2909=1565%:%
%:%2910=1566%:%
%:%2911=1567%:%
%:%2912=1568%:%
%:%2913=1569%:%
%:%2914=1570%:%
%:%2915=1571%:%
%:%2916=1572%:%
%:%2917=1573%:%
%:%2918=1574%:%
%:%2919=1575%:%
%:%2920=1576%:%
%:%2921=1577%:%
%:%2922=1578%:%
%:%2922=1579%:%
%:%2923=1580%:%
%:%2924=1581%:%
%:%2925=1582%:%
%:%2926=1583%:%
%:%2927=1584%:%
%:%2928=1585%:%
%:%2929=1586%:%
%:%2930=1587%:%
%:%2931=1588%:%
%:%2932=1589%:%
%:%2933=1590%:%
%:%2934=1591%:%
%:%2935=1592%:%
%:%2936=1593%:%
%:%2937=1594%:%
%:%2938=1595%:%
%:%2939=1596%:%
%:%2940=1597%:%
%:%2941=1598%:%
%:%2942=1599%:%
%:%2943=1600%:%
%:%2944=1601%:%
%:%2945=1602%:%
%:%2946=1603%:%
%:%2947=1604%:%
%:%2948=1605%:%
%:%2949=1606%:%
%:%2950=1607%:%
%:%2951=1608%:%
%:%2952=1609%:%
%:%2953=1610%:%
%:%2954=1611%:%
%:%2955=1612%:%
%:%2956=1613%:%
%:%2957=1614%:%
%:%2958=1615%:%
%:%2959=1616%:%
%:%2960=1617%:%
%:%2961=1618%:%
%:%2962=1619%:%
%:%2963=1620%:%
%:%2964=1621%:%
%:%2965=1622%:%
%:%2966=1623%:%
%:%2967=1624%:%
%:%2968=1625%:%
%:%2969=1626%:%
%:%2970=1627%:%
%:%2971=1628%:%
%:%2972=1629%:%
%:%2973=1630%:%
%:%2974=1631%:%
%:%2975=1632%:%
%:%2976=1633%:%
%:%2978=1635%:%
%:%2979=1635%:%
%:%2986=1636%:%
%:%2987=1636%:%
%:%2988=1637%:%
%:%2989=1637%:%
%:%2990=1638%:%
%:%2991=1638%:%
%:%2992=1638%:%
%:%2993=1639%:%
%:%2994=1639%:%
%:%2995=1640%:%
%:%2996=1640%:%
%:%2997=1640%:%
%:%2998=1641%:%
%:%2999=1641%:%
%:%3000=1642%:%
%:%3001=1642%:%
%:%3002=1643%:%
%:%3003=1643%:%
%:%3004=1643%:%
%:%3005=1644%:%
%:%3006=1644%:%
%:%3007=1645%:%
%:%3008=1645%:%
%:%3009=1646%:%
%:%3010=1646%:%
%:%3011=1647%:%
%:%3012=1647%:%
%:%3013=1647%:%
%:%3014=1648%:%
%:%3015=1648%:%
%:%3016=1649%:%
%:%3017=1649%:%
%:%3018=1650%:%
%:%3028=1652%:%
%:%3030=1654%:%
%:%3031=1654%:%
%:%3032=1655%:%
%:%3033=1656%:%
%:%3040=1657%:%
%:%3041=1657%:%
%:%3042=1658%:%
%:%3043=1658%:%
%:%3044=1659%:%
%:%3045=1659%:%
%:%3046=1659%:%
%:%3047=1660%:%
%:%3048=1660%:%
%:%3049=1661%:%
%:%3050=1661%:%
%:%3051=1662%:%
%:%3052=1662%:%
%:%3053=1663%:%
%:%3054=1663%:%
%:%3055=1664%:%
%:%3056=1664%:%
%:%3057=1664%:%
%:%3058=1665%:%
%:%3059=1665%:%
%:%3060=1665%:%
%:%3061=1666%:%
%:%3062=1666%:%
%:%3063=1667%:%
%:%3064=1667%:%
%:%3065=1668%:%
%:%3066=1668%:%
%:%3067=1669%:%
%:%3068=1669%:%
%:%3069=1670%:%
%:%3070=1670%:%
%:%3071=1670%:%
%:%3072=1671%:%
%:%3073=1671%:%
%:%3074=1672%:%
%:%3075=1672%:%
%:%3076=1673%:%
%:%3077=1673%:%
%:%3078=1673%:%
%:%3079=1674%:%
%:%3080=1674%:%
%:%3081=1674%:%
%:%3082=1675%:%
%:%3092=1677%:%
%:%3093=1678%:%
%:%3094=1679%:%
%:%3095=1680%:%
%:%3097=1682%:%
%:%3098=1682%:%
%:%3105=1683%:%
%:%3106=1683%:%
%:%3107=1684%:%
%:%3108=1684%:%
%:%3109=1685%:%
%:%3110=1685%:%
%:%3111=1685%:%
%:%3112=1686%:%
%:%3113=1686%:%
%:%3114=1687%:%
%:%3115=1687%:%
%:%3116=1687%:%
%:%3117=1688%:%
%:%3118=1688%:%
%:%3119=1689%:%
%:%3120=1689%:%
%:%3121=1690%:%
%:%3122=1690%:%
%:%3123=1690%:%
%:%3124=1691%:%
%:%3125=1691%:%
%:%3126=1692%:%
%:%3127=1692%:%
%:%3128=1693%:%
%:%3129=1693%:%
%:%3130=1694%:%
%:%3131=1694%:%
%:%3132=1694%:%
%:%3133=1695%:%
%:%3134=1695%:%
%:%3135=1696%:%
%:%3136=1696%:%
%:%3137=1697%:%
%:%3138=1697%:%
%:%3139=1698%:%
%:%3140=1698%:%
%:%3141=1698%:%
%:%3142=1699%:%
%:%3143=1699%:%
%:%3144=1699%:%
%:%3145=1700%:%
%:%3146=1700%:%
%:%3147=1701%:%
%:%3148=1701%:%
%:%3149=1702%:%
%:%3150=1702%:%
%:%3151=1702%:%
%:%3152=1703%:%
%:%3162=1705%:%
%:%3163=1706%:%
%:%3165=1708%:%
%:%3166=1708%:%
%:%3167=1709%:%
%:%3168=1710%:%
%:%3175=1711%:%
%:%3176=1711%:%
%:%3177=1712%:%
%:%3178=1712%:%
%:%3179=1713%:%
%:%3180=1713%:%
%:%3181=1714%:%
%:%3182=1714%:%
%:%3183=1715%:%
%:%3184=1715%:%
%:%3185=1716%:%
%:%3186=1716%:%
%:%3187=1717%:%
%:%3188=1717%:%
%:%3189=1718%:%
%:%3190=1718%:%
%:%3191=1719%:%
%:%3192=1719%:%
%:%3193=1720%:%
%:%3194=1720%:%
%:%3195=1720%:%
%:%3196=1721%:%
%:%3197=1721%:%
%:%3198=1721%:%
%:%3199=1722%:%
%:%3200=1722%:%
%:%3201=1723%:%
%:%3202=1723%:%
%:%3203=1724%:%
%:%3204=1724%:%
%:%3205=1724%:%
%:%3206=1725%:%
%:%3207=1725%:%
%:%3208=1725%:%
%:%3209=1726%:%
%:%3210=1726%:%
%:%3211=1727%:%
%:%3212=1727%:%
%:%3213=1728%:%
%:%3214=1728%:%
%:%3215=1728%:%
%:%3216=1729%:%
%:%3217=1729%:%
%:%3218=1729%:%
%:%3219=1730%:%
%:%3220=1730%:%
%:%3221=1731%:%
%:%3222=1731%:%
%:%3223=1732%:%
%:%3224=1732%:%
%:%3225=1733%:%
%:%3226=1733%:%
%:%3227=1734%:%
%:%3228=1734%:%
%:%3229=1734%:%
%:%3230=1735%:%
%:%3231=1735%:%
%:%3232=1736%:%
%:%3233=1736%:%
%:%3234=1737%:%
%:%3235=1737%:%
%:%3236=1738%:%
%:%3237=1738%:%
%:%3238=1738%:%
%:%3239=1739%:%
%:%3240=1739%:%
%:%3241=1740%:%
%:%3251=1742%:%
%:%3252=1743%:%
%:%3253=1744%:%
%:%3254=1745%:%
%:%3255=1746%:%
%:%3256=1747%:%
%:%3257=1748%:%
%:%3259=1750%:%
%:%3260=1750%:%
%:%3261=1751%:%
%:%3262=1752%:%
%:%3263=1753%:%
%:%3264=1754%:%
%:%3265=1755%:%
%:%3272=1756%:%
%:%3273=1756%:%
%:%3274=1757%:%
%:%3275=1757%:%
%:%3276=1758%:%
%:%3277=1758%:%
%:%3278=1759%:%
%:%3279=1759%:%
%:%3280=1760%:%
%:%3281=1760%:%
%:%3282=1760%:%
%:%3283=1761%:%
%:%3284=1761%:%
%:%3285=1761%:%
%:%3286=1762%:%
%:%3287=1762%:%
%:%3288=1763%:%
%:%3289=1763%:%
%:%3290=1764%:%
%:%3291=1764%:%
%:%3292=1764%:%
%:%3293=1765%:%
%:%3294=1765%:%
%:%3295=1765%:%
%:%3296=1766%:%
%:%3297=1766%:%
%:%3298=1767%:%
%:%3299=1767%:%
%:%3300=1767%:%
%:%3301=1768%:%
%:%3302=1768%:%
%:%3303=1769%:%
%:%3304=1769%:%
%:%3305=1770%:%
%:%3306=1770%:%
%:%3307=1770%:%
%:%3308=1771%:%
%:%3318=1773%:%
%:%3319=1774%:%
%:%3320=1775%:%
%:%3321=1776%:%
%:%3323=1778%:%
%:%3324=1778%:%
%:%3325=1779%:%
%:%3326=1780%:%
%:%3327=1781%:%
%:%3330=1782%:%
%:%3334=1782%:%
%:%3335=1782%:%
%:%3336=1782%:%
%:%3345=1784%:%
%:%3346=1785%:%
%:%3347=1786%:%
%:%3349=1788%:%
%:%3350=1788%:%
%:%3351=1789%:%
%:%3352=1790%:%
%:%3353=1791%:%
%:%3354=1792%:%
%:%3355=1793%:%
%:%3356=1794%:%
%:%3363=1795%:%
%:%3364=1795%:%
%:%3365=1796%:%
%:%3366=1796%:%
%:%3367=1797%:%
%:%3368=1797%:%
%:%3369=1797%:%
%:%3370=1798%:%
%:%3371=1798%:%
%:%3372=1799%:%
%:%3373=1799%:%
%:%3374=1800%:%
%:%3375=1800%:%
%:%3376=1801%:%
%:%3377=1801%:%
%:%3378=1801%:%
%:%3379=1802%:%
%:%3380=1802%:%
%:%3381=1803%:%
%:%3382=1803%:%
%:%3383=1803%:%
%:%3384=1804%:%
%:%3385=1804%:%
%:%3386=1805%:%
%:%3387=1805%:%
%:%3388=1805%:%
%:%3389=1806%:%
%:%3390=1806%:%
%:%3391=1806%:%
%:%3392=1807%:%
%:%3393=1807%:%
%:%3394=1807%:%
%:%3395=1808%:%
%:%3396=1808%:%
%:%3397=1808%:%
%:%3398=1809%:%
%:%3399=1809%:%
%:%3400=1810%:%
%:%3401=1810%:%
%:%3402=1810%:%
%:%3403=1811%:%
%:%3404=1811%:%
%:%3405=1812%:%
%:%3406=1812%:%
%:%3407=1812%:%
%:%3408=1813%:%
%:%3409=1813%:%
%:%3410=1814%:%
%:%3411=1814%:%
%:%3412=1815%:%
%:%3413=1815%:%
%:%3414=1815%:%
%:%3415=1816%:%
%:%3416=1816%:%
%:%3417=1816%:%
%:%3418=1817%:%
%:%3419=1817%:%
%:%3420=1818%:%
%:%3421=1818%:%
%:%3422=1819%:%
%:%3423=1819%:%
%:%3424=1819%:%
%:%3425=1820%:%
%:%3426=1820%:%
%:%3427=1820%:%
%:%3428=1821%:%
%:%3429=1821%:%
%:%3430=1822%:%
%:%3431=1822%:%
%:%3432=1822%:%
%:%3433=1823%:%
%:%3434=1823%:%
%:%3435=1824%:%
%:%3436=1824%:%
%:%3437=1825%:%
%:%3438=1825%:%
%:%3439=1826%:%
%:%3449=1828%:%
%:%3450=1829%:%
%:%3451=1830%:%
%:%3453=1832%:%
%:%3454=1832%:%
%:%3455=1833%:%
%:%3456=1834%:%
%:%3457=1835%:%
%:%3458=1836%:%
%:%3459=1837%:%
%:%3460=1838%:%
%:%3467=1839%:%
%:%3468=1839%:%
%:%3469=1840%:%
%:%3470=1840%:%
%:%3471=1841%:%
%:%3472=1841%:%
%:%3473=1841%:%
%:%3474=1842%:%
%:%3475=1842%:%
%:%3476=1843%:%
%:%3477=1843%:%
%:%3478=1844%:%
%:%3479=1844%:%
%:%3480=1845%:%
%:%3481=1845%:%
%:%3482=1845%:%
%:%3483=1846%:%
%:%3484=1846%:%
%:%3485=1847%:%
%:%3486=1847%:%
%:%3487=1847%:%
%:%3488=1848%:%
%:%3489=1848%:%
%:%3490=1849%:%
%:%3491=1849%:%
%:%3492=1849%:%
%:%3493=1850%:%
%:%3494=1850%:%
%:%3495=1850%:%
%:%3496=1851%:%
%:%3497=1851%:%
%:%3498=1851%:%
%:%3499=1852%:%
%:%3500=1852%:%
%:%3501=1852%:%
%:%3502=1853%:%
%:%3503=1853%:%
%:%3504=1854%:%
%:%3505=1854%:%
%:%3506=1854%:%
%:%3507=1855%:%
%:%3508=1855%:%
%:%3509=1856%:%
%:%3510=1856%:%
%:%3511=1856%:%
%:%3512=1857%:%
%:%3513=1857%:%
%:%3514=1858%:%
%:%3515=1858%:%
%:%3516=1858%:%
%:%3517=1859%:%
%:%3518=1859%:%
%:%3519=1860%:%
%:%3520=1860%:%
%:%3521=1860%:%
%:%3522=1861%:%
%:%3523=1861%:%
%:%3524=1862%:%
%:%3525=1862%:%
%:%3526=1862%:%
%:%3527=1863%:%
%:%3528=1863%:%
%:%3529=1864%:%
%:%3530=1864%:%
%:%3531=1865%:%
%:%3532=1865%:%
%:%3533=1865%:%
%:%3534=1866%:%
%:%3535=1866%:%
%:%3536=1866%:%
%:%3537=1867%:%
%:%3538=1867%:%
%:%3539=1868%:%
%:%3540=1868%:%
%:%3541=1869%:%
%:%3542=1869%:%
%:%3543=1869%:%
%:%3544=1870%:%
%:%3545=1870%:%
%:%3546=1870%:%
%:%3547=1871%:%
%:%3548=1871%:%
%:%3549=1872%:%
%:%3550=1872%:%
%:%3551=1872%:%
%:%3552=1873%:%
%:%3553=1873%:%
%:%3554=1874%:%
%:%3555=1874%:%
%:%3556=1875%:%
%:%3557=1875%:%
%:%3558=1876%:%
%:%3568=1878%:%
%:%3569=1879%:%
%:%3570=1880%:%
%:%3572=1882%:%
%:%3573=1882%:%
%:%3574=1883%:%
%:%3575=1884%:%
%:%3576=1885%:%
%:%3577=1886%:%
%:%3578=1887%:%
%:%3579=1888%:%
%:%3586=1889%:%
%:%3587=1889%:%
%:%3588=1890%:%
%:%3589=1890%:%
%:%3590=1891%:%
%:%3591=1891%:%
%:%3592=1891%:%
%:%3593=1892%:%
%:%3594=1892%:%
%:%3595=1893%:%
%:%3596=1893%:%
%:%3597=1894%:%
%:%3598=1894%:%
%:%3599=1895%:%
%:%3600=1895%:%
%:%3601=1895%:%
%:%3602=1896%:%
%:%3603=1896%:%
%:%3604=1897%:%
%:%3605=1897%:%
%:%3606=1897%:%
%:%3607=1898%:%
%:%3608=1898%:%
%:%3609=1899%:%
%:%3610=1899%:%
%:%3611=1899%:%
%:%3612=1900%:%
%:%3613=1900%:%
%:%3614=1900%:%
%:%3615=1901%:%
%:%3616=1901%:%
%:%3617=1901%:%
%:%3618=1902%:%
%:%3619=1902%:%
%:%3620=1902%:%
%:%3621=1903%:%
%:%3622=1903%:%
%:%3623=1904%:%
%:%3624=1904%:%
%:%3625=1904%:%
%:%3626=1905%:%
%:%3627=1905%:%
%:%3628=1906%:%
%:%3629=1906%:%
%:%3630=1906%:%
%:%3631=1907%:%
%:%3632=1907%:%
%:%3633=1908%:%
%:%3634=1908%:%
%:%3635=1908%:%
%:%3636=1909%:%
%:%3637=1909%:%
%:%3638=1910%:%
%:%3639=1910%:%
%:%3640=1910%:%
%:%3641=1911%:%
%:%3642=1911%:%
%:%3643=1912%:%
%:%3644=1912%:%
%:%3645=1912%:%
%:%3646=1913%:%
%:%3647=1913%:%
%:%3648=1914%:%
%:%3649=1914%:%
%:%3650=1915%:%
%:%3651=1915%:%
%:%3652=1915%:%
%:%3653=1916%:%
%:%3654=1916%:%
%:%3655=1916%:%
%:%3656=1917%:%
%:%3657=1917%:%
%:%3658=1918%:%
%:%3659=1918%:%
%:%3660=1919%:%
%:%3661=1919%:%
%:%3662=1919%:%
%:%3663=1920%:%
%:%3664=1920%:%
%:%3665=1920%:%
%:%3666=1921%:%
%:%3667=1921%:%
%:%3668=1922%:%
%:%3669=1922%:%
%:%3670=1922%:%
%:%3671=1923%:%
%:%3672=1923%:%
%:%3673=1924%:%
%:%3674=1924%:%
%:%3675=1925%:%
%:%3676=1925%:%
%:%3677=1926%:%
%:%3687=1928%:%
%:%3688=1929%:%
%:%3690=1931%:%
%:%3691=1931%:%
%:%3692=1932%:%
%:%3693=1933%:%
%:%3694=1934%:%
%:%3695=1935%:%
%:%3696=1936%:%
%:%3697=1937%:%
%:%3704=1938%:%
%:%3705=1938%:%
%:%3706=1939%:%
%:%3707=1939%:%
%:%3708=1940%:%
%:%3709=1940%:%
%:%3710=1940%:%
%:%3711=1941%:%
%:%3712=1941%:%
%:%3713=1942%:%
%:%3714=1942%:%
%:%3715=1943%:%
%:%3716=1943%:%
%:%3717=1944%:%
%:%3718=1944%:%
%:%3719=1944%:%
%:%3720=1945%:%
%:%3721=1945%:%
%:%3722=1946%:%
%:%3723=1946%:%
%:%3724=1946%:%
%:%3725=1947%:%
%:%3726=1947%:%
%:%3727=1948%:%
%:%3728=1948%:%
%:%3729=1948%:%
%:%3730=1949%:%
%:%3731=1949%:%
%:%3732=1949%:%
%:%3733=1950%:%
%:%3734=1950%:%
%:%3735=1950%:%
%:%3736=1951%:%
%:%3737=1951%:%
%:%3738=1951%:%
%:%3739=1952%:%
%:%3740=1952%:%
%:%3741=1953%:%
%:%3742=1953%:%
%:%3743=1953%:%
%:%3744=1954%:%
%:%3745=1954%:%
%:%3746=1955%:%
%:%3747=1955%:%
%:%3748=1955%:%
%:%3749=1956%:%
%:%3750=1956%:%
%:%3751=1957%:%
%:%3752=1957%:%
%:%3753=1957%:%
%:%3754=1958%:%
%:%3755=1958%:%
%:%3756=1959%:%
%:%3757=1959%:%
%:%3758=1960%:%
%:%3759=1960%:%
%:%3760=1960%:%
%:%3761=1961%:%
%:%3762=1961%:%
%:%3763=1961%:%
%:%3764=1962%:%
%:%3765=1962%:%
%:%3766=1963%:%
%:%3767=1963%:%
%:%3768=1963%:%
%:%3769=1964%:%
%:%3770=1964%:%
%:%3771=1965%:%
%:%3772=1965%:%
%:%3773=1966%:%
%:%3774=1966%:%
%:%3775=1967%:%
%:%3785=1969%:%
%:%3786=1970%:%
%:%3787=1971%:%
%:%3789=1973%:%
%:%3790=1973%:%
%:%3791=1974%:%
%:%3792=1975%:%
%:%3793=1976%:%
%:%3794=1977%:%
%:%3795=1978%:%
%:%3796=1979%:%
%:%3803=1980%:%
%:%3804=1980%:%
%:%3805=1981%:%
%:%3806=1981%:%
%:%3807=1982%:%
%:%3808=1982%:%
%:%3809=1983%:%
%:%3810=1983%:%
%:%3813=1986%:%
%:%3814=1987%:%
%:%3815=1987%:%
%:%3816=1987%:%
%:%3817=1988%:%
%:%3818=1988%:%
%:%3819=1989%:%
%:%3820=1989%:%
%:%3821=1990%:%
%:%3822=1990%:%
%:%3823=1991%:%
%:%3824=1991%:%
%:%3825=1992%:%
%:%3826=1992%:%
%:%3827=1992%:%
%:%3828=1993%:%
%:%3829=1993%:%
%:%3830=1994%:%
%:%3831=1994%:%
%:%3833=1996%:%
%:%3834=1997%:%
%:%3835=1997%:%
%:%3836=1998%:%
%:%3837=1998%:%
%:%3838=1999%:%
%:%3839=1999%:%
%:%3840=2000%:%
%:%3841=2000%:%
%:%3842=2001%:%
%:%3843=2001%:%
%:%3844=2001%:%
%:%3845=2002%:%
%:%3846=2002%:%
%:%3847=2003%:%
%:%3848=2003%:%
%:%3849=2003%:%
%:%3850=2004%:%
%:%3851=2004%:%
%:%3852=2005%:%
%:%3853=2005%:%
%:%3854=2005%:%
%:%3855=2006%:%
%:%3856=2006%:%
%:%3857=2007%:%
%:%3858=2007%:%
%:%3859=2007%:%
%:%3860=2008%:%
%:%3861=2008%:%
%:%3862=2009%:%
%:%3863=2009%:%
%:%3864=2009%:%
%:%3865=2010%:%
%:%3866=2010%:%
%:%3867=2011%:%
%:%3868=2011%:%
%:%3869=2012%:%
%:%3870=2012%:%
%:%3871=2013%:%
%:%3872=2013%:%
%:%3873=2014%:%
%:%3874=2015%:%
%:%3875=2015%:%
%:%3876=2016%:%
%:%3877=2016%:%
%:%3878=2017%:%
%:%3879=2017%:%
%:%3880=2018%:%
%:%3881=2018%:%
%:%3882=2019%:%
%:%3883=2019%:%
%:%3884=2019%:%
%:%3885=2020%:%
%:%3886=2020%:%
%:%3887=2021%:%
%:%3888=2021%:%
%:%3889=2021%:%
%:%3890=2022%:%
%:%3891=2022%:%
%:%3892=2023%:%
%:%3893=2023%:%
%:%3894=2023%:%
%:%3895=2024%:%
%:%3896=2024%:%
%:%3897=2025%:%
%:%3898=2025%:%
%:%3899=2025%:%
%:%3900=2026%:%
%:%3901=2026%:%
%:%3902=2027%:%
%:%3903=2027%:%
%:%3904=2028%:%
%:%3905=2028%:%
%:%3906=2029%:%
%:%3907=2029%:%
%:%3908=2030%:%
%:%3909=2030%:%
%:%3910=2030%:%
%:%3911=2031%:%
%:%3912=2031%:%
%:%3913=2032%:%
%:%3914=2032%:%
%:%3915=2032%:%
%:%3916=2033%:%
%:%3917=2033%:%
%:%3918=2034%:%
%:%3919=2034%:%
%:%3920=2035%:%
%:%3921=2035%:%
%:%3922=2035%:%
%:%3923=2036%:%
%:%3924=2036%:%
%:%3925=2037%:%
%:%3926=2037%:%
%:%3927=2037%:%
%:%3928=2038%:%
%:%3929=2038%:%
%:%3930=2039%:%
%:%3931=2039%:%
%:%3932=2040%:%
%:%3933=2040%:%
%:%3934=2041%:%
%:%3935=2041%:%
%:%3936=2042%:%
%:%3937=2042%:%
%:%3938=2043%:%
%:%3948=2045%:%
%:%3949=2046%:%
%:%3950=2047%:%
%:%3952=2049%:%
%:%3953=2049%:%
%:%3954=2050%:%
%:%3955=2051%:%
%:%3962=2052%:%
%:%3963=2052%:%
%:%3964=2053%:%
%:%3965=2053%:%
%:%3966=2054%:%
%:%3967=2054%:%
%:%3968=2055%:%
%:%3969=2055%:%
%:%3970=2056%:%
%:%3971=2056%:%
%:%3972=2057%:%
%:%3973=2057%:%
%:%3974=2058%:%
%:%3975=2058%:%
%:%3976=2059%:%
%:%3977=2059%:%
%:%3978=2060%:%
%:%3979=2060%:%
%:%3980=2061%:%
%:%3981=2061%:%
%:%3982=2062%:%
%:%3983=2062%:%
%:%3984=2063%:%
%:%3985=2063%:%
%:%3986=2064%:%
%:%3987=2064%:%
%:%3988=2065%:%
%:%3989=2065%:%
%:%3990=2065%:%
%:%3991=2066%:%
%:%3992=2066%:%
%:%3993=2067%:%
%:%3994=2067%:%
%:%3995=2068%:%
%:%3996=2068%:%
%:%3997=2069%:%
%:%3998=2069%:%
%:%3999=2070%:%
%:%4000=2070%:%
%:%4001=2071%:%
%:%4002=2071%:%
%:%4003=2071%:%
%:%4004=2072%:%
%:%4005=2072%:%
%:%4006=2073%:%
%:%4007=2073%:%
%:%4008=2073%:%
%:%4009=2074%:%
%:%4010=2074%:%
%:%4011=2075%:%
%:%4012=2075%:%
%:%4013=2075%:%
%:%4014=2076%:%
%:%4015=2076%:%
%:%4016=2077%:%
%:%4017=2077%:%
%:%4018=2077%:%
%:%4019=2078%:%
%:%4020=2078%:%
%:%4021=2079%:%
%:%4022=2079%:%
%:%4023=2079%:%
%:%4024=2080%:%
%:%4025=2080%:%
%:%4026=2081%:%
%:%4027=2081%:%
%:%4028=2082%:%
%:%4029=2082%:%
%:%4030=2083%:%
%:%4031=2083%:%
%:%4032=2083%:%
%:%4033=2084%:%
%:%4034=2084%:%
%:%4035=2085%:%
%:%4036=2085%:%
%:%4037=2086%:%
%:%4038=2086%:%
%:%4039=2086%:%
%:%4040=2087%:%
%:%4041=2087%:%
%:%4042=2088%:%
%:%4043=2088%:%
%:%4044=2089%:%
%:%4045=2089%:%
%:%4046=2090%:%
%:%4047=2090%:%
%:%4048=2091%:%
%:%4049=2091%:%
%:%4050=2092%:%
%:%4051=2092%:%
%:%4052=2092%:%
%:%4053=2093%:%
%:%4054=2093%:%
%:%4055=2094%:%
%:%4056=2094%:%
%:%4057=2095%:%
%:%4058=2095%:%
%:%4059=2095%:%
%:%4060=2096%:%
%:%4061=2096%:%
%:%4062=2097%:%
%:%4063=2097%:%
%:%4064=2098%:%
%:%4065=2098%:%
%:%4066=2099%:%
%:%4067=2099%:%
%:%4068=2100%:%
%:%4069=2100%:%
%:%4070=2101%:%
%:%4071=2101%:%
%:%4072=2101%:%
%:%4073=2102%:%
%:%4074=2102%:%
%:%4075=2103%:%
%:%4076=2103%:%
%:%4077=2104%:%
%:%4078=2104%:%
%:%4079=2104%:%
%:%4080=2105%:%
%:%4081=2105%:%
%:%4082=2106%:%
%:%4083=2106%:%
%:%4084=2107%:%
%:%4085=2107%:%
%:%4086=2107%:%
%:%4087=2108%:%
%:%4088=2108%:%
%:%4089=2109%:%
%:%4090=2109%:%
%:%4091=2110%:%
%:%4092=2110%:%
%:%4093=2110%:%
%:%4094=2111%:%
%:%4095=2111%:%
%:%4096=2112%:%
%:%4097=2112%:%
%:%4098=2113%:%
%:%4099=2113%:%
%:%4100=2114%:%
%:%4101=2114%:%
%:%4102=2115%:%
%:%4103=2115%:%
%:%4104=2116%:%
%:%4105=2116%:%
%:%4106=2117%:%
%:%4107=2117%:%
%:%4108=2118%:%
%:%4118=2120%:%
%:%4119=2121%:%
%:%4120=2122%:%
%:%4121=2123%:%
%:%4122=2124%:%
%:%4123=2125%:%
%:%4124=2126%:%
%:%4125=2127%:%
%:%4126=2128%:%
%:%4128=2130%:%
%:%4129=2130%:%
%:%4130=2131%:%
%:%4131=2132%:%
%:%4132=2133%:%
%:%4139=2134%:%
%:%4140=2134%:%
%:%4141=2135%:%
%:%4142=2135%:%
%:%4143=2136%:%
%:%4144=2136%:%
%:%4145=2136%:%
%:%4146=2137%:%
%:%4147=2137%:%
%:%4148=2137%:%
%:%4149=2138%:%
%:%4150=2138%:%
%:%4151=2138%:%
%:%4152=2139%:%
%:%4153=2139%:%
%:%4154=2139%:%
%:%4155=2140%:%
%:%4156=2140%:%
%:%4157=2141%:%
%:%4158=2141%:%
%:%4159=2142%:%
%:%4160=2142%:%
%:%4161=2143%:%
%:%4162=2143%:%
%:%4163=2144%:%
%:%4164=2144%:%
%:%4165=2145%:%
%:%4166=2145%:%
%:%4167=2145%:%
%:%4168=2146%:%
%:%4169=2146%:%
%:%4170=2147%:%
%:%4171=2147%:%
%:%4172=2148%:%
%:%4173=2148%:%
%:%4174=2149%:%
%:%4175=2149%:%
%:%4176=2150%:%
%:%4177=2150%:%
%:%4178=2151%:%
%:%4188=2153%:%
%:%4189=2154%:%
%:%4190=2155%:%
%:%4192=2157%:%
%:%4193=2157%:%
%:%4194=2158%:%
%:%4195=2159%:%
%:%4196=2160%:%
%:%4203=2161%:%
%:%4204=2161%:%
%:%4205=2162%:%
%:%4206=2162%:%
%:%4207=2163%:%
%:%4208=2163%:%
%:%4209=2163%:%
%:%4210=2163%:%
%:%4211=2164%:%
%:%4212=2164%:%
%:%4213=2165%:%
%:%4214=2165%:%
%:%4215=2165%:%
%:%4216=2165%:%
%:%4217=2166%:%
%:%4218=2166%:%
%:%4219=2166%:%
%:%4220=2167%:%
%:%4221=2167%:%
%:%4222=2168%:%
%:%4223=2168%:%
%:%4224=2168%:%
%:%4225=2169%:%
%:%4226=2169%:%
%:%4227=2170%:%
%:%4228=2170%:%
%:%4229=2171%:%
%:%4230=2171%:%
%:%4231=2171%:%
%:%4232=2172%:%
%:%4233=2172%:%
%:%4234=2173%:%
%:%4235=2173%:%
%:%4236=2173%:%
%:%4237=2174%:%
%:%4238=2174%:%
%:%4239=2175%:%
%:%4240=2175%:%
%:%4241=2175%:%
%:%4242=2176%:%
%:%4243=2176%:%
%:%4244=2177%:%
%:%4245=2177%:%
%:%4246=2177%:%
%:%4247=2178%:%
%:%4248=2178%:%
%:%4249=2179%:%
%:%4250=2179%:%
%:%4251=2179%:%
%:%4252=2180%:%
%:%4253=2180%:%
%:%4254=2181%:%
%:%4255=2181%:%
%:%4256=2181%:%
%:%4257=2182%:%
%:%4258=2182%:%
%:%4259=2183%:%
%:%4260=2183%:%
%:%4261=2183%:%
%:%4262=2184%:%
%:%4263=2184%:%
%:%4264=2185%:%
%:%4265=2185%:%
%:%4266=2185%:%
%:%4267=2186%:%
%:%4268=2186%:%
%:%4269=2187%:%
%:%4270=2187%:%
%:%4271=2188%:%
%:%4272=2188%:%
%:%4273=2189%:%
%:%4274=2189%:%
%:%4275=2189%:%
%:%4276=2190%:%
%:%4277=2190%:%
%:%4278=2190%:%
%:%4279=2191%:%
%:%4280=2191%:%
%:%4281=2192%:%
%:%4282=2192%:%
%:%4283=2192%:%
%:%4284=2193%:%
%:%4285=2193%:%
%:%4286=2194%:%
%:%4287=2194%:%
%:%4288=2194%:%
%:%4289=2195%:%
%:%4290=2195%:%
%:%4291=2196%:%
%:%4292=2196%:%
%:%4293=2197%:%
%:%4294=2197%:%
%:%4295=2197%:%
%:%4296=2198%:%
%:%4297=2198%:%
%:%4298=2198%:%
%:%4299=2199%:%
%:%4300=2199%:%
%:%4301=2200%:%
%:%4302=2200%:%
%:%4303=2201%:%
%:%4304=2201%:%
%:%4305=2201%:%
%:%4306=2202%:%
%:%4307=2202%:%
%:%4308=2202%:%
%:%4309=2203%:%
%:%4310=2203%:%
%:%4311=2204%:%
%:%4312=2204%:%
%:%4313=2204%:%
%:%4314=2205%:%
%:%4315=2205%:%
%:%4316=2206%:%
%:%4317=2206%:%
%:%4318=2206%:%
%:%4319=2207%:%
%:%4320=2207%:%
%:%4321=2208%:%
%:%4331=2210%:%
%:%4332=2211%:%
%:%4333=2212%:%
%:%4334=2213%:%
%:%4335=2214%:%
%:%4336=2215%:%
%:%4337=2216%:%
%:%4338=2217%:%
%:%4339=2218%:%
%:%4341=2220%:%
%:%4342=2220%:%
%:%4343=2221%:%
%:%4344=2222%:%
%:%4345=2223%:%
%:%4346=2224%:%
%:%4347=2225%:%
%:%4348=2226%:%
%:%4355=2227%:%
%:%4356=2227%:%
%:%4357=2228%:%
%:%4358=2228%:%
%:%4359=2229%:%
%:%4360=2229%:%
%:%4361=2229%:%
%:%4362=2230%:%
%:%4363=2230%:%
%:%4364=2231%:%
%:%4365=2231%:%
%:%4366=2232%:%
%:%4367=2232%:%
%:%4368=2233%:%
%:%4369=2233%:%
%:%4370=2233%:%
%:%4371=2234%:%
%:%4372=2234%:%
%:%4373=2235%:%
%:%4374=2235%:%
%:%4375=2235%:%
%:%4376=2236%:%
%:%4377=2236%:%
%:%4378=2237%:%
%:%4379=2237%:%
%:%4380=2237%:%
%:%4381=2238%:%
%:%4382=2238%:%
%:%4383=2238%:%
%:%4384=2239%:%
%:%4385=2239%:%
%:%4386=2239%:%
%:%4387=2240%:%
%:%4388=2240%:%
%:%4389=2240%:%
%:%4390=2241%:%
%:%4391=2241%:%
%:%4392=2242%:%
%:%4393=2242%:%
%:%4394=2243%:%
%:%4395=2243%:%
%:%4396=2244%:%
%:%4397=2244%:%
%:%4398=2245%:%
%:%4399=2245%:%
%:%4400=2245%:%
%:%4401=2246%:%
%:%4402=2246%:%
%:%4403=2247%:%
%:%4404=2247%:%
%:%4405=2247%:%
%:%4406=2248%:%
%:%4407=2248%:%
%:%4408=2248%:%
%:%4409=2249%:%
%:%4410=2249%:%
%:%4411=2249%:%
%:%4412=2250%:%
%:%4413=2250%:%
%:%4414=2251%:%
%:%4415=2251%:%
%:%4416=2251%:%
%:%4417=2252%:%
%:%4418=2252%:%
%:%4419=2253%:%
%:%4420=2253%:%
%:%4421=2253%:%
%:%4422=2254%:%
%:%4423=2254%:%
%:%4424=2255%:%
%:%4425=2255%:%
%:%4426=2256%:%
%:%4427=2256%:%
%:%4428=2257%:%
%:%4429=2257%:%
%:%4430=2258%:%
%:%4431=2258%:%
%:%4432=2259%:%
%:%4433=2259%:%
%:%4434=2259%:%
%:%4435=2260%:%
%:%4436=2260%:%
%:%4437=2261%:%
%:%4438=2261%:%
%:%4439=2261%:%
%:%4440=2262%:%
%:%4441=2262%:%
%:%4442=2262%:%
%:%4443=2263%:%
%:%4444=2263%:%
%:%4445=2263%:%
%:%4446=2264%:%
%:%4447=2264%:%
%:%4448=2265%:%
%:%4449=2265%:%
%:%4450=2265%:%
%:%4451=2266%:%
%:%4452=2266%:%
%:%4453=2267%:%
%:%4454=2267%:%
%:%4455=2267%:%
%:%4456=2268%:%
%:%4457=2268%:%
%:%4458=2269%:%
%:%4459=2269%:%
%:%4460=2270%:%
%:%4461=2270%:%
%:%4462=2271%:%
%:%4463=2271%:%
%:%4464=2272%:%
%:%4474=2274%:%
%:%4475=2275%:%
%:%4476=2276%:%
%:%4478=2278%:%
%:%4479=2278%:%
%:%4480=2279%:%
%:%4481=2280%:%
%:%4482=2281%:%
%:%4483=2282%:%
%:%4484=2283%:%
%:%4485=2284%:%
%:%4492=2285%:%
%:%4493=2285%:%
%:%4494=2286%:%
%:%4495=2286%:%
%:%4496=2287%:%
%:%4497=2287%:%
%:%4498=2287%:%
%:%4499=2288%:%
%:%4500=2288%:%
%:%4501=2289%:%
%:%4502=2289%:%
%:%4503=2290%:%
%:%4504=2290%:%
%:%4505=2291%:%
%:%4506=2291%:%
%:%4507=2291%:%
%:%4508=2292%:%
%:%4509=2292%:%
%:%4510=2293%:%
%:%4511=2293%:%
%:%4512=2293%:%
%:%4513=2294%:%
%:%4514=2294%:%
%:%4515=2295%:%
%:%4516=2295%:%
%:%4517=2295%:%
%:%4518=2296%:%
%:%4519=2296%:%
%:%4520=2296%:%
%:%4521=2297%:%
%:%4522=2297%:%
%:%4523=2297%:%
%:%4524=2298%:%
%:%4525=2298%:%
%:%4526=2298%:%
%:%4527=2299%:%
%:%4528=2299%:%
%:%4529=2300%:%
%:%4530=2300%:%
%:%4531=2300%:%
%:%4532=2301%:%
%:%4533=2301%:%
%:%4534=2302%:%
%:%4535=2302%:%
%:%4536=2302%:%
%:%4537=2303%:%
%:%4538=2303%:%
%:%4539=2304%:%
%:%4540=2304%:%
%:%4541=2305%:%
%:%4542=2305%:%
%:%4543=2306%:%
%:%4544=2306%:%
%:%4545=2307%:%
%:%4546=2307%:%
%:%4547=2307%:%
%:%4548=2308%:%
%:%4549=2308%:%
%:%4550=2309%:%
%:%4551=2309%:%
%:%4552=2309%:%
%:%4553=2310%:%
%:%4554=2310%:%
%:%4555=2311%:%
%:%4556=2311%:%
%:%4557=2311%:%
%:%4558=2312%:%
%:%4559=2312%:%
%:%4560=2312%:%
%:%4561=2313%:%
%:%4562=2313%:%
%:%4563=2313%:%
%:%4564=2314%:%
%:%4565=2314%:%
%:%4566=2315%:%
%:%4567=2315%:%
%:%4568=2315%:%
%:%4569=2316%:%
%:%4570=2316%:%
%:%4571=2317%:%
%:%4572=2317%:%
%:%4573=2317%:%
%:%4574=2318%:%
%:%4575=2318%:%
%:%4576=2319%:%
%:%4577=2319%:%
%:%4578=2320%:%
%:%4579=2320%:%
%:%4580=2321%:%
%:%4581=2321%:%
%:%4582=2322%:%
%:%4583=2322%:%
%:%4584=2323%:%
%:%4585=2323%:%
%:%4586=2323%:%
%:%4587=2324%:%
%:%4588=2324%:%
%:%4589=2325%:%
%:%4590=2325%:%
%:%4591=2325%:%
%:%4592=2326%:%
%:%4593=2326%:%
%:%4594=2326%:%
%:%4595=2327%:%
%:%4596=2327%:%
%:%4597=2327%:%
%:%4598=2328%:%
%:%4599=2328%:%
%:%4600=2329%:%
%:%4601=2329%:%
%:%4602=2329%:%
%:%4603=2330%:%
%:%4604=2330%:%
%:%4605=2331%:%
%:%4606=2331%:%
%:%4607=2331%:%
%:%4608=2332%:%
%:%4609=2332%:%
%:%4610=2333%:%
%:%4611=2333%:%
%:%4612=2334%:%
%:%4613=2334%:%
%:%4614=2335%:%
%:%4615=2335%:%
%:%4616=2336%:%
%:%4626=2338%:%
%:%4627=2339%:%
%:%4628=2340%:%
%:%4630=2342%:%
%:%4631=2342%:%
%:%4632=2343%:%
%:%4633=2344%:%
%:%4634=2345%:%
%:%4635=2346%:%
%:%4636=2347%:%
%:%4637=2348%:%
%:%4644=2349%:%
%:%4645=2349%:%
%:%4646=2350%:%
%:%4647=2350%:%
%:%4648=2351%:%
%:%4649=2351%:%
%:%4650=2351%:%
%:%4651=2352%:%
%:%4652=2352%:%
%:%4653=2353%:%
%:%4654=2353%:%
%:%4655=2354%:%
%:%4656=2354%:%
%:%4657=2355%:%
%:%4658=2355%:%
%:%4659=2355%:%
%:%4660=2356%:%
%:%4661=2356%:%
%:%4662=2357%:%
%:%4663=2357%:%
%:%4664=2357%:%
%:%4665=2358%:%
%:%4666=2358%:%
%:%4667=2359%:%
%:%4668=2359%:%
%:%4669=2359%:%
%:%4670=2360%:%
%:%4671=2360%:%
%:%4672=2360%:%
%:%4673=2361%:%
%:%4674=2361%:%
%:%4675=2361%:%
%:%4676=2362%:%
%:%4677=2362%:%
%:%4678=2362%:%
%:%4679=2363%:%
%:%4680=2363%:%
%:%4681=2364%:%
%:%4682=2364%:%
%:%4683=2364%:%
%:%4684=2365%:%
%:%4685=2365%:%
%:%4686=2366%:%
%:%4687=2366%:%
%:%4688=2366%:%
%:%4689=2367%:%
%:%4690=2367%:%
%:%4691=2368%:%
%:%4692=2368%:%
%:%4693=2369%:%
%:%4694=2369%:%
%:%4695=2370%:%
%:%4696=2370%:%
%:%4697=2371%:%
%:%4698=2371%:%
%:%4699=2371%:%
%:%4700=2372%:%
%:%4701=2372%:%
%:%4702=2373%:%
%:%4703=2373%:%
%:%4704=2373%:%
%:%4705=2374%:%
%:%4706=2374%:%
%:%4707=2375%:%
%:%4708=2375%:%
%:%4709=2375%:%
%:%4710=2376%:%
%:%4711=2376%:%
%:%4712=2376%:%
%:%4713=2377%:%
%:%4714=2377%:%
%:%4715=2377%:%
%:%4716=2378%:%
%:%4717=2378%:%
%:%4718=2379%:%
%:%4719=2379%:%
%:%4720=2379%:%
%:%4721=2380%:%
%:%4722=2380%:%
%:%4723=2381%:%
%:%4724=2381%:%
%:%4725=2381%:%
%:%4726=2382%:%
%:%4727=2382%:%
%:%4728=2383%:%
%:%4729=2383%:%
%:%4730=2384%:%
%:%4731=2384%:%
%:%4732=2385%:%
%:%4733=2385%:%
%:%4734=2386%:%
%:%4735=2386%:%
%:%4736=2387%:%
%:%4737=2387%:%
%:%4738=2387%:%
%:%4739=2388%:%
%:%4740=2388%:%
%:%4741=2389%:%
%:%4742=2389%:%
%:%4743=2389%:%
%:%4744=2390%:%
%:%4745=2390%:%
%:%4746=2391%:%
%:%4747=2391%:%
%:%4748=2391%:%
%:%4749=2392%:%
%:%4750=2392%:%
%:%4751=2392%:%
%:%4752=2393%:%
%:%4753=2393%:%
%:%4754=2393%:%
%:%4755=2394%:%
%:%4756=2394%:%
%:%4757=2395%:%
%:%4758=2395%:%
%:%4759=2395%:%
%:%4760=2396%:%
%:%4761=2396%:%
%:%4762=2397%:%
%:%4763=2397%:%
%:%4764=2397%:%
%:%4765=2398%:%
%:%4766=2398%:%
%:%4767=2399%:%
%:%4768=2399%:%
%:%4769=2400%:%
%:%4770=2400%:%
%:%4771=2401%:%
%:%4772=2401%:%
%:%4773=2402%:%
%:%4783=2404%:%
%:%4784=2405%:%
%:%4785=2406%:%
%:%4787=2408%:%
%:%4788=2408%:%
%:%4789=2409%:%
%:%4790=2410%:%
%:%4791=2411%:%
%:%4792=2412%:%
%:%4793=2413%:%
%:%4794=2414%:%
%:%4801=2415%:%
%:%4802=2415%:%
%:%4803=2416%:%
%:%4804=2416%:%
%:%4805=2417%:%
%:%4806=2417%:%
%:%4807=2417%:%
%:%4808=2418%:%
%:%4809=2418%:%
%:%4810=2419%:%
%:%4811=2419%:%
%:%4812=2420%:%
%:%4813=2420%:%
%:%4814=2421%:%
%:%4815=2421%:%
%:%4816=2421%:%
%:%4817=2422%:%
%:%4818=2422%:%
%:%4819=2423%:%
%:%4820=2423%:%
%:%4821=2423%:%
%:%4822=2424%:%
%:%4823=2424%:%
%:%4824=2425%:%
%:%4825=2425%:%
%:%4826=2425%:%
%:%4827=2426%:%
%:%4828=2426%:%
%:%4829=2426%:%
%:%4830=2427%:%
%:%4831=2427%:%
%:%4832=2427%:%
%:%4833=2428%:%
%:%4834=2428%:%
%:%4835=2428%:%
%:%4836=2429%:%
%:%4837=2429%:%
%:%4838=2430%:%
%:%4839=2430%:%
%:%4840=2430%:%
%:%4841=2431%:%
%:%4842=2431%:%
%:%4843=2432%:%
%:%4844=2432%:%
%:%4845=2432%:%
%:%4846=2433%:%
%:%4847=2433%:%
%:%4848=2434%:%
%:%4849=2434%:%
%:%4850=2434%:%
%:%4851=2435%:%
%:%4852=2435%:%
%:%4853=2436%:%
%:%4854=2436%:%
%:%4855=2436%:%
%:%4856=2437%:%
%:%4857=2437%:%
%:%4858=2437%:%
%:%4859=2438%:%
%:%4860=2438%:%
%:%4861=2438%:%
%:%4862=2439%:%
%:%4863=2439%:%
%:%4864=2440%:%
%:%4865=2440%:%
%:%4866=2440%:%
%:%4867=2441%:%
%:%4868=2441%:%
%:%4869=2442%:%
%:%4870=2442%:%
%:%4871=2443%:%
%:%4872=2443%:%
%:%4873=2444%:%
%:%4883=2446%:%
%:%4884=2447%:%
%:%4885=2448%:%
%:%4886=2449%:%
%:%4888=2451%:%
%:%4889=2451%:%
%:%4890=2452%:%
%:%4891=2453%:%
%:%4892=2454%:%
%:%4893=2455%:%
%:%4894=2456%:%
%:%4895=2457%:%
%:%4902=2458%:%
%:%4903=2458%:%
%:%4904=2459%:%
%:%4905=2459%:%
%:%4908=2462%:%
%:%4909=2463%:%
%:%4910=2463%:%
%:%4911=2463%:%
%:%4912=2464%:%
%:%4913=2464%:%
%:%4914=2465%:%
%:%4915=2465%:%
%:%4916=2466%:%
%:%4917=2466%:%
%:%4918=2467%:%
%:%4919=2467%:%
%:%4920=2468%:%
%:%4921=2468%:%
%:%4922=2468%:%
%:%4923=2469%:%
%:%4924=2469%:%
%:%4925=2470%:%
%:%4926=2470%:%
%:%4928=2472%:%
%:%4929=2473%:%
%:%4930=2473%:%
%:%4931=2474%:%
%:%4932=2474%:%
%:%4933=2475%:%
%:%4934=2475%:%
%:%4935=2476%:%
%:%4936=2476%:%
%:%4937=2477%:%
%:%4938=2477%:%
%:%4939=2477%:%
%:%4940=2478%:%
%:%4941=2478%:%
%:%4942=2479%:%
%:%4943=2479%:%
%:%4944=2479%:%
%:%4945=2480%:%
%:%4946=2480%:%
%:%4947=2481%:%
%:%4948=2481%:%
%:%4949=2481%:%
%:%4950=2482%:%
%:%4951=2482%:%
%:%4952=2483%:%
%:%4953=2483%:%
%:%4954=2483%:%
%:%4955=2484%:%
%:%4956=2484%:%
%:%4957=2485%:%
%:%4958=2485%:%
%:%4959=2485%:%
%:%4960=2486%:%
%:%4961=2486%:%
%:%4962=2487%:%
%:%4963=2487%:%
%:%4964=2488%:%
%:%4965=2488%:%
%:%4966=2489%:%
%:%4967=2489%:%
%:%4968=2490%:%
%:%4969=2491%:%
%:%4970=2491%:%
%:%4971=2492%:%
%:%4972=2492%:%
%:%4973=2493%:%
%:%4974=2493%:%
%:%4975=2494%:%
%:%4976=2494%:%
%:%4977=2495%:%
%:%4978=2495%:%
%:%4979=2495%:%
%:%4980=2496%:%
%:%4981=2496%:%
%:%4982=2497%:%
%:%4983=2497%:%
%:%4984=2497%:%
%:%4985=2498%:%
%:%4986=2498%:%
%:%4987=2499%:%
%:%4988=2499%:%
%:%4989=2499%:%
%:%4990=2500%:%
%:%4991=2500%:%
%:%4992=2501%:%
%:%4993=2501%:%
%:%4994=2501%:%
%:%4995=2502%:%
%:%4996=2502%:%
%:%4997=2503%:%
%:%4998=2503%:%
%:%4999=2503%:%
%:%5000=2504%:%
%:%5001=2504%:%
%:%5002=2505%:%
%:%5003=2505%:%
%:%5004=2506%:%
%:%5005=2506%:%
%:%5006=2507%:%
%:%5007=2507%:%
%:%5008=2508%:%
%:%5009=2508%:%
%:%5010=2508%:%
%:%5011=2509%:%
%:%5012=2509%:%
%:%5013=2510%:%
%:%5014=2510%:%
%:%5015=2511%:%
%:%5016=2511%:%
%:%5017=2511%:%
%:%5018=2512%:%
%:%5019=2512%:%
%:%5020=2513%:%
%:%5021=2513%:%
%:%5022=2514%:%
%:%5023=2514%:%
%:%5024=2515%:%
%:%5025=2515%:%
%:%5026=2516%:%
%:%5027=2516%:%
%:%5028=2517%:%
%:%5038=2519%:%
%:%5039=2520%:%
%:%5040=2521%:%
%:%5042=2523%:%
%:%5043=2523%:%
%:%5044=2524%:%
%:%5045=2525%:%
%:%5052=2526%:%
%:%5053=2526%:%
%:%5054=2527%:%
%:%5055=2527%:%
%:%5056=2528%:%
%:%5057=2528%:%
%:%5058=2529%:%
%:%5059=2529%:%
%:%5060=2530%:%
%:%5061=2530%:%
%:%5062=2531%:%
%:%5063=2531%:%
%:%5064=2532%:%
%:%5065=2532%:%
%:%5066=2533%:%
%:%5067=2533%:%
%:%5068=2534%:%
%:%5069=2534%:%
%:%5070=2535%:%
%:%5071=2535%:%
%:%5072=2535%:%
%:%5073=2536%:%
%:%5074=2536%:%
%:%5075=2537%:%
%:%5076=2537%:%
%:%5077=2537%:%
%:%5078=2538%:%
%:%5079=2538%:%
%:%5080=2539%:%
%:%5081=2539%:%
%:%5082=2540%:%
%:%5083=2540%:%
%:%5084=2540%:%
%:%5085=2541%:%
%:%5086=2541%:%
%:%5087=2542%:%
%:%5088=2542%:%
%:%5089=2542%:%
%:%5090=2543%:%
%:%5091=2543%:%
%:%5092=2544%:%
%:%5093=2544%:%
%:%5094=2544%:%
%:%5095=2545%:%
%:%5096=2545%:%
%:%5097=2546%:%
%:%5098=2546%:%
%:%5099=2546%:%
%:%5100=2547%:%
%:%5101=2547%:%
%:%5102=2548%:%
%:%5103=2548%:%
%:%5104=2548%:%
%:%5105=2549%:%
%:%5106=2549%:%
%:%5107=2550%:%
%:%5108=2550%:%
%:%5109=2550%:%
%:%5110=2551%:%
%:%5111=2551%:%
%:%5112=2552%:%
%:%5113=2552%:%
%:%5114=2552%:%
%:%5115=2553%:%
%:%5116=2553%:%
%:%5117=2554%:%
%:%5118=2554%:%
%:%5119=2554%:%
%:%5120=2555%:%
%:%5121=2555%:%
%:%5122=2556%:%
%:%5123=2556%:%
%:%5124=2556%:%
%:%5125=2557%:%
%:%5126=2557%:%
%:%5127=2558%:%
%:%5128=2558%:%
%:%5129=2559%:%
%:%5130=2559%:%
%:%5131=2559%:%
%:%5132=2560%:%
%:%5133=2560%:%
%:%5134=2561%:%
%:%5135=2561%:%
%:%5136=2561%:%
%:%5137=2562%:%
%:%5138=2562%:%
%:%5139=2563%:%
%:%5140=2563%:%
%:%5141=2564%:%
%:%5142=2564%:%
%:%5143=2564%:%
%:%5144=2565%:%
%:%5145=2565%:%
%:%5146=2566%:%
%:%5147=2566%:%
%:%5148=2566%:%
%:%5149=2567%:%
%:%5150=2567%:%
%:%5151=2568%:%
%:%5152=2568%:%
%:%5153=2568%:%
%:%5154=2569%:%
%:%5155=2569%:%
%:%5156=2569%:%
%:%5157=2570%:%
%:%5158=2570%:%
%:%5159=2571%:%
%:%5160=2571%:%
%:%5161=2572%:%
%:%5162=2572%:%
%:%5163=2572%:%
%:%5164=2573%:%
%:%5165=2573%:%
%:%5166=2574%:%
%:%5167=2574%:%
%:%5168=2574%:%
%:%5169=2575%:%
%:%5170=2575%:%
%:%5171=2576%:%
%:%5172=2576%:%
%:%5173=2576%:%
%:%5174=2577%:%
%:%5175=2577%:%
%:%5176=2578%:%
%:%5177=2578%:%
%:%5178=2578%:%
%:%5179=2579%:%
%:%5180=2579%:%
%:%5181=2579%:%
%:%5182=2580%:%
%:%5183=2580%:%
%:%5184=2581%:%
%:%5185=2581%:%
%:%5186=2581%:%
%:%5187=2582%:%
%:%5188=2582%:%
%:%5189=2583%:%
%:%5190=2583%:%
%:%5191=2583%:%
%:%5192=2584%:%
%:%5193=2584%:%
%:%5194=2584%:%
%:%5195=2585%:%
%:%5196=2585%:%
%:%5197=2586%:%
%:%5198=2586%:%
%:%5199=2586%:%
%:%5200=2587%:%
%:%5201=2587%:%
%:%5202=2588%:%
%:%5203=2588%:%
%:%5204=2589%:%
%:%5205=2589%:%
%:%5206=2589%:%
%:%5207=2590%:%
%:%5208=2590%:%
%:%5209=2591%:%
%:%5210=2591%:%
%:%5211=2592%:%
%:%5221=2594%:%
%:%5222=2595%:%
%:%5223=2596%:%
%:%5225=2598%:%
%:%5226=2598%:%
%:%5233=2599%:%
%:%5234=2599%:%
%:%5235=2600%:%
%:%5236=2600%:%
%:%5239=2603%:%
%:%5240=2604%:%
%:%5241=2604%:%
%:%5242=2605%:%
%:%5243=2605%:%
%:%5244=2606%:%
%:%5245=2606%:%
%:%5246=2607%:%
%:%5247=2607%:%
%:%5248=2608%:%
%:%5249=2608%:%
%:%5250=2608%:%
%:%5251=2609%:%
%:%5252=2609%:%
%:%5253=2610%:%
%:%5254=2610%:%
%:%5255=2610%:%
%:%5256=2611%:%
%:%5257=2611%:%
%:%5258=2612%:%
%:%5259=2612%:%
%:%5260=2612%:%
%:%5261=2613%:%
%:%5262=2613%:%
%:%5263=2614%:%
%:%5264=2614%:%
%:%5265=2614%:%
%:%5266=2615%:%
%:%5267=2615%:%
%:%5270=2618%:%
%:%5271=2619%:%
%:%5272=2619%:%
%:%5273=2619%:%
%:%5274=2620%:%
%:%5275=2620%:%
%:%5276=2621%:%
%:%5286=2623%:%
%:%5288=2625%:%
%:%5289=2625%:%
%:%5290=2626%:%
%:%5291=2627%:%
%:%5292=2628%:%
%:%5299=2629%:%
%:%5300=2629%:%
%:%5301=2630%:%
%:%5302=2630%:%
%:%5303=2631%:%
%:%5304=2631%:%
%:%5305=2631%:%
%:%5306=2631%:%
%:%5307=2631%:%
%:%5308=2632%:%
%:%5309=2632%:%
%:%5310=2633%:%
%:%5311=2633%:%
%:%5312=2634%:%
%:%5313=2634%:%