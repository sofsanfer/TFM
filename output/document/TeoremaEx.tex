%
\begin{isabellebody}%
\setisabellecontext{TeoremaEx}%
%
\isadelimtheory
%
\endisadelimtheory
%
\isatagtheory
%
\endisatagtheory
{\isafoldtheory}%
%
\isadelimtheory
%
\endisadelimtheory
%
\begin{isamarkuptext}%
\comentario{Añadir introducción.}%
\end{isamarkuptext}\isamarkuptrue%
%
\begin{isamarkuptext}%
\comentario{Cambiar referencias de los lemas tras el cambio de índice.}%
\end{isamarkuptext}\isamarkuptrue%
%
\isadelimdocument
%
\endisadelimdocument
%
\isatagdocument
%
\isamarkupsection{Sucesiones de conjuntos%
}
\isamarkuptrue%
%
\endisatagdocument
{\isafolddocument}%
%
\isadelimdocument
%
\endisadelimdocument
%
\begin{isamarkuptext}%
En este apartado daremos una introducción sobre sucesiones de conjuntos de fórmulas a 
  partir de una colección y un conjunto de la misma. De este modo, se mostrarán distintas 
  características sobre las sucesiones y se definirá su límite. En la siguiente sección 
  probaremos que dicho límite constituye un conjunto satisfacible por el lema de Hintikka.

\comentario{Revisar el párrafo anterior al final}

  Recordemos que el conjunto de las fórmulas proposicionales se define recursivamente a partir 
  de un alfabeto numerable de variables proposicionales. Por lo tanto, el conjunto de fórmulas 
  proposicionales es igualmente numerable, de modo que es posible enumerar sus elementos. Una vez 
  realizada esta observación, veamos la definición de sucesión de conjuntos de fórmulas 
  proposicionales a partir de una colección y un conjunto de la misma.

\begin{definicion}
  Sea \isa{C} una colección de conjuntos de fórmulas proposicionales, \isa{S\ {\isasymin}\ C} y \isa{F\isactrlsub {\isadigit{1}}{\isacharcomma}\ F\isactrlsub {\isadigit{2}}{\isacharcomma}\ F\isactrlsub {\isadigit{3}}\ {\isasymdots}} una 
  enumeración de las fórmulas proposicionales. Se define la \isa{sucesión\ de\ conjuntos\ de\ C\ a\ partir\ de\ S} como sigue:

  $S_{0} = S$

  $S_{n+1} = \left\{ \begin{array}{lcc} S_{n} \cup \{F_{n}\} &  si  & S_{n} \cup \{F_{n}\} \in C \\ \\ S_{n} & c.c \end{array} \right.$ 
\end{definicion}

  Para su formalización en Isabelle se ha introducido una instancia en la teoría de \isa{Sintaxis} que 
  indica explícitamente que el conjunto de las fórmulas proposicionales es numerable, probado
  mediante el método \isa{countable{\isacharunderscore}datatype} de Isabelle.

  \isa{instance\ formula\ {\isacharcolon}{\isacharcolon}\ {\isacharparenleft}countable{\isacharparenright}\ countable\ by\ countable{\isacharunderscore}datatype}

  De esta manera se genera en Isabelle una enumeración predeterminada de los elementos del conjunto,
  junto con herramientas para probar propiedades referentes a la numerabilidad. En particular, en la 
  formalización de la definición \isa{{\isadigit{4}}{\isachardot}{\isadigit{1}}{\isachardot}{\isadigit{1}}} se utilizará la función \isa{from{\isacharunderscore}nat} que, al aplicarla a un 
  número natural \isa{n}, nos devuelve la \isa{n}-ésima fórmula proposicional según una enumeración 
  predeterminada en Isabelle. 

  Puesto que la definición de las sucesiones en \isa{{\isadigit{4}}{\isachardot}{\isadigit{1}}{\isachardot}{\isadigit{1}}} se trata de una definición 
  recursiva sobre la estructura recursiva de los números naturales, se formalizará en Isabelle
  mediante el tipo de funciones primitivas recursivas de la siguiente manera.%
\end{isamarkuptext}\isamarkuptrue%
\isacommand{primrec}\isamarkupfalse%
\ pcp{\isacharunderscore}seq\ \isakeyword{where}\isanewline
{\isachardoublequoteopen}pcp{\isacharunderscore}seq\ C\ S\ {\isadigit{0}}\ {\isacharequal}\ S{\isachardoublequoteclose}\ {\isacharbar}\isanewline
{\isachardoublequoteopen}pcp{\isacharunderscore}seq\ C\ S\ {\isacharparenleft}Suc\ n{\isacharparenright}\ {\isacharequal}\ {\isacharparenleft}let\ Sn\ {\isacharequal}\ pcp{\isacharunderscore}seq\ C\ S\ n{\isacharsemicolon}\ Sn{\isadigit{1}}\ {\isacharequal}\ insert\ {\isacharparenleft}from{\isacharunderscore}nat\ n{\isacharparenright}\ Sn\ in\isanewline
\ \ \ \ \ \ \ \ \ \ \ \ \ \ \ \ \ \ \ \ \ \ \ \ if\ Sn{\isadigit{1}}\ {\isasymin}\ C\ then\ Sn{\isadigit{1}}\ else\ Sn{\isacharparenright}{\isachardoublequoteclose}%
\begin{isamarkuptext}%
Veamos el primer resultado sobre dichas sucesiones.

  \begin{lema}
    Sea \isa{C} una colección de conjuntos con la propiedad de consistencia proposicional,\\ \isa{S\ {\isasymin}\ C} y 
    \isa{{\isacharbraceleft}S\isactrlsub n{\isacharbraceright}} la sucesión de conjuntos de \isa{C} a partir de \isa{S} construida según la definición \isa{{\isadigit{4}}{\isachardot}{\isadigit{1}}{\isachardot}{\isadigit{1}}}. 
    Entonces, para todo \isa{n\ {\isasymin}\ {\isasymnat}} se verifica que \isa{S\isactrlsub n\ {\isasymin}\ C}.
  \end{lema}

  Procedamos con su demostración.

  \begin{demostracion}
    El resultado se prueba por inducción en los números naturales que conforman los subíndices de la 
    sucesión.

    En primer lugar, tenemos que \isa{S\isactrlsub {\isadigit{0}}\ {\isacharequal}\ S} pertenece a \isa{C} por hipótesis.

    Por otro lado, supongamos que \isa{S\isactrlsub n\ {\isasymin}\ C}. Probemos que \isa{S\isactrlsub n\isactrlsub {\isacharplus}\isactrlsub {\isadigit{1}}\ {\isasymin}\ C}. Si suponemos que \isa{S\isactrlsub n\ {\isasymunion}\ {\isacharbraceleft}F\isactrlsub n{\isacharbraceright}\ {\isasymin}\ C},
    por definición tenemos que \isa{S\isactrlsub n\isactrlsub {\isacharplus}\isactrlsub {\isadigit{1}}\ {\isacharequal}\ S\isactrlsub n\ {\isasymunion}\ {\isacharbraceleft}F\isactrlsub n{\isacharbraceright}}, luego pertenece a \isa{C}. En caso contrario, si
    suponemos que \isa{S\isactrlsub n\ {\isasymunion}\ {\isacharbraceleft}F\isactrlsub n{\isacharbraceright}\ {\isasymnotin}\ C}, por definición tenemos que \isa{S\isactrlsub n\isactrlsub {\isacharplus}\isactrlsub {\isadigit{1}}\ {\isacharequal}\ S\isactrlsub n}, que pertenece igualmente
    a \isa{C} por hipótesis de inducción. Por tanto, queda probado el resultado.
  \end{demostracion}

  La formalización y demostración detallada del lema en Isabelle son las siguientes.%
\end{isamarkuptext}\isamarkuptrue%
\isacommand{lemma}\isamarkupfalse%
\ \isanewline
\ \ \isakeyword{assumes}\ {\isachardoublequoteopen}pcp\ C{\isachardoublequoteclose}\ \isanewline
\ \ \ \ \ \ \ \ \ \ {\isachardoublequoteopen}S\ {\isasymin}\ C{\isachardoublequoteclose}\isanewline
\ \ \ \ \ \ \ \ \isakeyword{shows}\ {\isachardoublequoteopen}pcp{\isacharunderscore}seq\ C\ S\ n\ {\isasymin}\ C{\isachardoublequoteclose}\isanewline
%
\isadelimproof
%
\endisadelimproof
%
\isatagproof
\isacommand{proof}\isamarkupfalse%
\ {\isacharparenleft}induction\ n{\isacharparenright}\isanewline
\ \ \isacommand{show}\isamarkupfalse%
\ {\isachardoublequoteopen}pcp{\isacharunderscore}seq\ C\ S\ {\isadigit{0}}\ {\isasymin}\ C{\isachardoublequoteclose}\isanewline
\ \ \ \ \isacommand{by}\isamarkupfalse%
\ {\isacharparenleft}simp\ only{\isacharcolon}\ pcp{\isacharunderscore}seq{\isachardot}simps{\isacharparenleft}{\isadigit{1}}{\isacharparenright}\ {\isacartoucheopen}S\ {\isasymin}\ C{\isacartoucheclose}{\isacharparenright}\isanewline
\isacommand{next}\isamarkupfalse%
\isanewline
\ \ \isacommand{fix}\isamarkupfalse%
\ n\isanewline
\ \ \isacommand{assume}\isamarkupfalse%
\ HI{\isacharcolon}{\isachardoublequoteopen}pcp{\isacharunderscore}seq\ C\ S\ n\ {\isasymin}\ C{\isachardoublequoteclose}\isanewline
\ \ \isacommand{have}\isamarkupfalse%
\ {\isachardoublequoteopen}pcp{\isacharunderscore}seq\ C\ S\ {\isacharparenleft}Suc\ n{\isacharparenright}\ {\isacharequal}\ {\isacharparenleft}let\ Sn\ {\isacharequal}\ pcp{\isacharunderscore}seq\ C\ S\ n{\isacharsemicolon}\ Sn{\isadigit{1}}\ {\isacharequal}\ insert\ {\isacharparenleft}from{\isacharunderscore}nat\ n{\isacharparenright}\ Sn\ in\isanewline
\ \ \ \ \ \ \ \ \ \ \ \ \ \ \ \ \ \ \ \ \ \ \ \ if\ Sn{\isadigit{1}}\ {\isasymin}\ C\ then\ Sn{\isadigit{1}}\ else\ Sn{\isacharparenright}{\isachardoublequoteclose}\ \isanewline
\ \ \ \ \isacommand{by}\isamarkupfalse%
\ {\isacharparenleft}simp\ only{\isacharcolon}\ pcp{\isacharunderscore}seq{\isachardot}simps{\isacharparenleft}{\isadigit{2}}{\isacharparenright}{\isacharparenright}\isanewline
\ \ \isacommand{then}\isamarkupfalse%
\ \isacommand{have}\isamarkupfalse%
\ SucDef{\isacharcolon}{\isachardoublequoteopen}pcp{\isacharunderscore}seq\ C\ S\ {\isacharparenleft}Suc\ n{\isacharparenright}\ {\isacharequal}\ {\isacharparenleft}if\ insert\ {\isacharparenleft}from{\isacharunderscore}nat\ n{\isacharparenright}\ {\isacharparenleft}pcp{\isacharunderscore}seq\ C\ S\ n{\isacharparenright}\ {\isasymin}\ C\ then\ \isanewline
\ \ \ \ \ \ \ \ \ \ \ \ \ \ \ \ \ \ \ \ insert\ {\isacharparenleft}from{\isacharunderscore}nat\ n{\isacharparenright}\ {\isacharparenleft}pcp{\isacharunderscore}seq\ C\ S\ n{\isacharparenright}\ else\ pcp{\isacharunderscore}seq\ C\ S\ n{\isacharparenright}{\isachardoublequoteclose}\ \isanewline
\ \ \ \ \isacommand{by}\isamarkupfalse%
\ {\isacharparenleft}simp\ only{\isacharcolon}\ Let{\isacharunderscore}def{\isacharparenright}\isanewline
\ \ \isacommand{show}\isamarkupfalse%
\ {\isachardoublequoteopen}pcp{\isacharunderscore}seq\ C\ S\ {\isacharparenleft}Suc\ n{\isacharparenright}\ {\isasymin}\ C{\isachardoublequoteclose}\isanewline
\ \ \isacommand{proof}\isamarkupfalse%
\ {\isacharparenleft}cases{\isacharparenright}\isanewline
\ \ \ \ \isacommand{assume}\isamarkupfalse%
\ {\isadigit{1}}{\isacharcolon}{\isachardoublequoteopen}insert\ {\isacharparenleft}from{\isacharunderscore}nat\ n{\isacharparenright}\ {\isacharparenleft}pcp{\isacharunderscore}seq\ C\ S\ n{\isacharparenright}\ {\isasymin}\ C{\isachardoublequoteclose}\isanewline
\ \ \ \ \isacommand{have}\isamarkupfalse%
\ {\isachardoublequoteopen}pcp{\isacharunderscore}seq\ C\ S\ {\isacharparenleft}Suc\ n{\isacharparenright}\ {\isacharequal}\ insert\ {\isacharparenleft}from{\isacharunderscore}nat\ n{\isacharparenright}\ {\isacharparenleft}pcp{\isacharunderscore}seq\ C\ S\ n{\isacharparenright}{\isachardoublequoteclose}\isanewline
\ \ \ \ \ \ \isacommand{using}\isamarkupfalse%
\ SucDef\ {\isadigit{1}}\ \isacommand{by}\isamarkupfalse%
\ {\isacharparenleft}simp\ only{\isacharcolon}\ if{\isacharunderscore}True{\isacharparenright}\isanewline
\ \ \ \ \isacommand{thus}\isamarkupfalse%
\ {\isachardoublequoteopen}pcp{\isacharunderscore}seq\ C\ S\ {\isacharparenleft}Suc\ n{\isacharparenright}\ {\isasymin}\ C{\isachardoublequoteclose}\isanewline
\ \ \ \ \ \ \isacommand{by}\isamarkupfalse%
\ {\isacharparenleft}simp\ only{\isacharcolon}\ {\isadigit{1}}{\isacharparenright}\isanewline
\ \ \isacommand{next}\isamarkupfalse%
\isanewline
\ \ \ \ \isacommand{assume}\isamarkupfalse%
\ {\isadigit{2}}{\isacharcolon}{\isachardoublequoteopen}insert\ {\isacharparenleft}from{\isacharunderscore}nat\ n{\isacharparenright}\ {\isacharparenleft}pcp{\isacharunderscore}seq\ C\ S\ n{\isacharparenright}\ {\isasymnotin}\ C{\isachardoublequoteclose}\isanewline
\ \ \ \ \isacommand{have}\isamarkupfalse%
\ {\isachardoublequoteopen}pcp{\isacharunderscore}seq\ C\ S\ {\isacharparenleft}Suc\ n{\isacharparenright}\ {\isacharequal}\ pcp{\isacharunderscore}seq\ C\ S\ n{\isachardoublequoteclose}\isanewline
\ \ \ \ \ \ \isacommand{using}\isamarkupfalse%
\ SucDef\ {\isadigit{2}}\ \isacommand{by}\isamarkupfalse%
\ {\isacharparenleft}simp\ only{\isacharcolon}\ if{\isacharunderscore}False{\isacharparenright}\isanewline
\ \ \ \ \isacommand{thus}\isamarkupfalse%
\ {\isachardoublequoteopen}pcp{\isacharunderscore}seq\ C\ S\ {\isacharparenleft}Suc\ n{\isacharparenright}\ {\isasymin}\ C{\isachardoublequoteclose}\isanewline
\ \ \ \ \ \ \isacommand{by}\isamarkupfalse%
\ {\isacharparenleft}simp\ only{\isacharcolon}\ HI{\isacharparenright}\isanewline
\ \ \isacommand{qed}\isamarkupfalse%
\isanewline
\isacommand{qed}\isamarkupfalse%
%
\endisatagproof
{\isafoldproof}%
%
\isadelimproof
%
\endisadelimproof
%
\begin{isamarkuptext}%
Del mismo modo, podemos probar el lema de manera automática en Isabelle.%
\end{isamarkuptext}\isamarkuptrue%
\isacommand{lemma}\isamarkupfalse%
\ pcp{\isacharunderscore}seq{\isacharunderscore}in{\isacharcolon}\ {\isachardoublequoteopen}pcp\ C\ {\isasymLongrightarrow}\ S\ {\isasymin}\ C\ {\isasymLongrightarrow}\ pcp{\isacharunderscore}seq\ C\ S\ n\ {\isasymin}\ C{\isachardoublequoteclose}\isanewline
%
\isadelimproof
%
\endisadelimproof
%
\isatagproof
\isacommand{proof}\isamarkupfalse%
{\isacharparenleft}induction\ n{\isacharparenright}\isanewline
\ \ \isacommand{case}\isamarkupfalse%
\ {\isacharparenleft}Suc\ n{\isacharparenright}\ \ \isanewline
\ \ \isacommand{hence}\isamarkupfalse%
\ {\isachardoublequoteopen}pcp{\isacharunderscore}seq\ C\ S\ n\ {\isasymin}\ C{\isachardoublequoteclose}\ \isacommand{by}\isamarkupfalse%
\ simp\isanewline
\ \ \isacommand{thus}\isamarkupfalse%
\ {\isacharquery}case\ \isacommand{by}\isamarkupfalse%
\ {\isacharparenleft}simp\ add{\isacharcolon}\ Let{\isacharunderscore}def{\isacharparenright}\isanewline
\isacommand{qed}\isamarkupfalse%
\ simp%
\endisatagproof
{\isafoldproof}%
%
\isadelimproof
%
\endisadelimproof
%
\begin{isamarkuptext}%
Por otro lado, veamos la monotonía de dichas sucesiones.

  \begin{lema}
    Toda sucesión de conjuntos construida a partir de una colección y un conjunto según la
    definición \isa{{\isadigit{4}}{\isachardot}{\isadigit{1}}{\isachardot}{\isadigit{1}}} es monótona.
  \end{lema}

  En Isabelle, se formaliza de la siguiente forma.%
\end{isamarkuptext}\isamarkuptrue%
\isacommand{lemma}\isamarkupfalse%
\ {\isachardoublequoteopen}pcp{\isacharunderscore}seq\ C\ S\ n\ {\isasymsubseteq}\ pcp{\isacharunderscore}seq\ C\ S\ {\isacharparenleft}Suc\ n{\isacharparenright}{\isachardoublequoteclose}\isanewline
%
\isadelimproof
\ \ %
\endisadelimproof
%
\isatagproof
\isacommand{oops}\isamarkupfalse%
%
\endisatagproof
{\isafoldproof}%
%
\isadelimproof
%
\endisadelimproof
%
\begin{isamarkuptext}%
Procedamos con la demostración del lema.

  \begin{demostracion}
    Sea una colección de conjuntos \isa{C}, \isa{S\ {\isasymin}\ C} y \isa{{\isacharbraceleft}S\isactrlsub n{\isacharbraceright}} la sucesión de conjuntos de \isa{C} a partir de 
    \isa{S} según la definición \isa{{\isadigit{4}}{\isachardot}{\isadigit{1}}{\isachardot}{\isadigit{1}}}. Para probar que \isa{{\isacharbraceleft}S\isactrlsub n{\isacharbraceright}} es monótona, basta probar que \isa{S\isactrlsub n\ {\isasymsubseteq}\ S\isactrlsub n\isactrlsub {\isacharplus}\isactrlsub {\isadigit{1}}} 
    para todo \isa{n\ {\isasymin}\ {\isasymnat}}. En efecto, el resultado es inmediato al considerar dos casos para todo 
    \isa{n\ {\isasymin}\ {\isasymnat}}: \isa{S\isactrlsub n\ {\isasymunion}\ {\isacharbraceleft}F\isactrlsub n{\isacharbraceright}\ {\isasymin}\ C} o \isa{S\isactrlsub n\ {\isasymunion}\ {\isacharbraceleft}F\isactrlsub n{\isacharbraceright}\ {\isasymnotin}\ C}. Si suponemos que\\ \isa{S\isactrlsub n\ {\isasymunion}\ {\isacharbraceleft}F\isactrlsub n{\isacharbraceright}\ {\isasymin}\ C}, por definición 
    tenemos que \isa{S\isactrlsub n\isactrlsub {\isacharplus}\isactrlsub {\isadigit{1}}\ {\isacharequal}\ S\isactrlsub n\ {\isasymunion}\ {\isacharbraceleft}F\isactrlsub n{\isacharbraceright}}, luego es claro que\\ \isa{S\isactrlsub n\ {\isasymsubseteq}\ S\isactrlsub n\isactrlsub {\isacharplus}\isactrlsub {\isadigit{1}}}. En caso contrario, si 
    \isa{S\isactrlsub n\ {\isasymunion}\ {\isacharbraceleft}F\isactrlsub n{\isacharbraceright}\ {\isasymnotin}\ C}, por definición se tiene que \isa{S\isactrlsub n\isactrlsub {\isacharplus}\isactrlsub {\isadigit{1}}\ {\isacharequal}\ S\isactrlsub n}, obteniéndose igualmente el resultado
    por la propiedad reflexiva de la contención de conjuntos. 
  \end{demostracion}

  La prueba detallada en Isabelle se muestra a continuación.%
\end{isamarkuptext}\isamarkuptrue%
\isacommand{lemma}\isamarkupfalse%
\ {\isachardoublequoteopen}pcp{\isacharunderscore}seq\ C\ S\ n\ {\isasymsubseteq}\ pcp{\isacharunderscore}seq\ C\ S\ {\isacharparenleft}Suc\ n{\isacharparenright}{\isachardoublequoteclose}\isanewline
%
\isadelimproof
%
\endisadelimproof
%
\isatagproof
\isacommand{proof}\isamarkupfalse%
\ {\isacharminus}\isanewline
\ \ \isacommand{have}\isamarkupfalse%
\ {\isachardoublequoteopen}pcp{\isacharunderscore}seq\ C\ S\ {\isacharparenleft}Suc\ n{\isacharparenright}\ {\isacharequal}\ {\isacharparenleft}let\ Sn\ {\isacharequal}\ pcp{\isacharunderscore}seq\ C\ S\ n{\isacharsemicolon}\ Sn{\isadigit{1}}\ {\isacharequal}\ insert\ {\isacharparenleft}from{\isacharunderscore}nat\ n{\isacharparenright}\ Sn\ in\isanewline
\ \ \ \ \ \ \ \ \ \ \ \ \ \ \ \ \ \ \ \ \ \ \ \ if\ Sn{\isadigit{1}}\ {\isasymin}\ C\ then\ Sn{\isadigit{1}}\ else\ Sn{\isacharparenright}{\isachardoublequoteclose}\ \isanewline
\ \ \ \ \isacommand{by}\isamarkupfalse%
\ {\isacharparenleft}simp\ only{\isacharcolon}\ pcp{\isacharunderscore}seq{\isachardot}simps{\isacharparenleft}{\isadigit{2}}{\isacharparenright}{\isacharparenright}\isanewline
\ \ \isacommand{then}\isamarkupfalse%
\ \isacommand{have}\isamarkupfalse%
\ SucDef{\isacharcolon}{\isachardoublequoteopen}pcp{\isacharunderscore}seq\ C\ S\ {\isacharparenleft}Suc\ n{\isacharparenright}\ {\isacharequal}\ {\isacharparenleft}if\ insert\ {\isacharparenleft}from{\isacharunderscore}nat\ n{\isacharparenright}\ {\isacharparenleft}pcp{\isacharunderscore}seq\ C\ S\ n{\isacharparenright}\ {\isasymin}\ C\ then\ \isanewline
\ \ \ \ \ \ \ \ \ \ \ \ \ \ \ \ \ \ \ \ insert\ {\isacharparenleft}from{\isacharunderscore}nat\ n{\isacharparenright}\ {\isacharparenleft}pcp{\isacharunderscore}seq\ C\ S\ n{\isacharparenright}\ else\ pcp{\isacharunderscore}seq\ C\ S\ n{\isacharparenright}{\isachardoublequoteclose}\ \isanewline
\ \ \ \ \isacommand{by}\isamarkupfalse%
\ {\isacharparenleft}simp\ only{\isacharcolon}\ Let{\isacharunderscore}def{\isacharparenright}\isanewline
\ \ \isacommand{thus}\isamarkupfalse%
\ {\isachardoublequoteopen}pcp{\isacharunderscore}seq\ C\ S\ n\ {\isasymsubseteq}\ pcp{\isacharunderscore}seq\ C\ S\ {\isacharparenleft}Suc\ n{\isacharparenright}{\isachardoublequoteclose}\isanewline
\ \ \isacommand{proof}\isamarkupfalse%
\ {\isacharparenleft}cases{\isacharparenright}\isanewline
\ \ \ \ \isacommand{assume}\isamarkupfalse%
\ {\isadigit{1}}{\isacharcolon}{\isachardoublequoteopen}insert\ {\isacharparenleft}from{\isacharunderscore}nat\ n{\isacharparenright}\ {\isacharparenleft}pcp{\isacharunderscore}seq\ C\ S\ n{\isacharparenright}\ {\isasymin}\ C{\isachardoublequoteclose}\isanewline
\ \ \ \ \isacommand{have}\isamarkupfalse%
\ {\isachardoublequoteopen}pcp{\isacharunderscore}seq\ C\ S\ {\isacharparenleft}Suc\ n{\isacharparenright}\ {\isacharequal}\ insert\ {\isacharparenleft}from{\isacharunderscore}nat\ n{\isacharparenright}\ {\isacharparenleft}pcp{\isacharunderscore}seq\ C\ S\ n{\isacharparenright}{\isachardoublequoteclose}\isanewline
\ \ \ \ \ \ \isacommand{using}\isamarkupfalse%
\ SucDef\ {\isadigit{1}}\ \isacommand{by}\isamarkupfalse%
\ {\isacharparenleft}simp\ only{\isacharcolon}\ if{\isacharunderscore}True{\isacharparenright}\isanewline
\ \ \ \ \isacommand{thus}\isamarkupfalse%
\ {\isachardoublequoteopen}pcp{\isacharunderscore}seq\ C\ S\ n\ {\isasymsubseteq}\ pcp{\isacharunderscore}seq\ C\ S\ {\isacharparenleft}Suc\ n{\isacharparenright}{\isachardoublequoteclose}\isanewline
\ \ \ \ \ \ \isacommand{by}\isamarkupfalse%
\ {\isacharparenleft}simp\ only{\isacharcolon}\ subset{\isacharunderscore}insertI{\isacharparenright}\isanewline
\ \ \isacommand{next}\isamarkupfalse%
\isanewline
\ \ \ \ \isacommand{assume}\isamarkupfalse%
\ {\isadigit{2}}{\isacharcolon}{\isachardoublequoteopen}insert\ {\isacharparenleft}from{\isacharunderscore}nat\ n{\isacharparenright}\ {\isacharparenleft}pcp{\isacharunderscore}seq\ C\ S\ n{\isacharparenright}\ {\isasymnotin}\ C{\isachardoublequoteclose}\isanewline
\ \ \ \ \isacommand{have}\isamarkupfalse%
\ {\isachardoublequoteopen}pcp{\isacharunderscore}seq\ C\ S\ {\isacharparenleft}Suc\ n{\isacharparenright}\ {\isacharequal}\ pcp{\isacharunderscore}seq\ C\ S\ n{\isachardoublequoteclose}\isanewline
\ \ \ \ \ \ \isacommand{using}\isamarkupfalse%
\ SucDef\ {\isadigit{2}}\ \isacommand{by}\isamarkupfalse%
\ {\isacharparenleft}simp\ only{\isacharcolon}\ if{\isacharunderscore}False{\isacharparenright}\isanewline
\ \ \ \ \isacommand{thus}\isamarkupfalse%
\ {\isachardoublequoteopen}pcp{\isacharunderscore}seq\ C\ S\ n\ {\isasymsubseteq}\ pcp{\isacharunderscore}seq\ C\ S\ {\isacharparenleft}Suc\ n{\isacharparenright}{\isachardoublequoteclose}\isanewline
\ \ \ \ \ \ \isacommand{by}\isamarkupfalse%
\ {\isacharparenleft}simp\ only{\isacharcolon}\ subset{\isacharunderscore}refl{\isacharparenright}\isanewline
\ \ \isacommand{qed}\isamarkupfalse%
\isanewline
\isacommand{qed}\isamarkupfalse%
%
\endisatagproof
{\isafoldproof}%
%
\isadelimproof
%
\endisadelimproof
%
\begin{isamarkuptext}%
Del mismo modo, se puede probar automáticamente en Isabelle/HOL.%
\end{isamarkuptext}\isamarkuptrue%
\isacommand{lemma}\isamarkupfalse%
\ pcp{\isacharunderscore}seq{\isacharunderscore}monotonicity{\isacharcolon}{\isachardoublequoteopen}pcp{\isacharunderscore}seq\ C\ S\ n\ {\isasymsubseteq}\ pcp{\isacharunderscore}seq\ C\ S\ {\isacharparenleft}Suc\ n{\isacharparenright}{\isachardoublequoteclose}\isanewline
%
\isadelimproof
\ \ %
\endisadelimproof
%
\isatagproof
\isacommand{by}\isamarkupfalse%
\ {\isacharparenleft}smt\ eq{\isacharunderscore}iff\ pcp{\isacharunderscore}seq{\isachardot}simps{\isacharparenleft}{\isadigit{2}}{\isacharparenright}\ subset{\isacharunderscore}insertI{\isacharparenright}%
\endisatagproof
{\isafoldproof}%
%
\isadelimproof
%
\endisadelimproof
%
\begin{isamarkuptext}%
Por otra lado, para facilitar posteriores demostraciones en Isabelle/HOL, vamos a formalizar 
  el lema anterior empleando la siguiente definición generalizada de monotonía.%
\end{isamarkuptext}\isamarkuptrue%
\isacommand{lemma}\isamarkupfalse%
\ pcp{\isacharunderscore}seq{\isacharunderscore}mono{\isacharcolon}\isanewline
\ \ \isakeyword{assumes}\ {\isachardoublequoteopen}n\ {\isasymle}\ m{\isachardoublequoteclose}\ \isanewline
\ \ \isakeyword{shows}\ {\isachardoublequoteopen}pcp{\isacharunderscore}seq\ C\ S\ n\ {\isasymsubseteq}\ pcp{\isacharunderscore}seq\ C\ S\ m{\isachardoublequoteclose}\isanewline
%
\isadelimproof
\ \ %
\endisadelimproof
%
\isatagproof
\isacommand{using}\isamarkupfalse%
\ pcp{\isacharunderscore}seq{\isacharunderscore}monotonicity\ assms\ \isacommand{by}\isamarkupfalse%
\ {\isacharparenleft}rule\ lift{\isacharunderscore}Suc{\isacharunderscore}mono{\isacharunderscore}le{\isacharparenright}%
\endisatagproof
{\isafoldproof}%
%
\isadelimproof
%
\endisadelimproof
%
\begin{isamarkuptext}%
A continuación daremos un lema que permite caracterizar un elemento de la sucesión en función 
  de los anteriores.

\begin{lema}
  Sea \isa{C} una colección de conjuntos, \isa{S\ {\isasymin}\ C} y \isa{{\isacharbraceleft}S\isactrlsub n{\isacharbraceright}} la sucesión de conjuntos de \isa{C} a partir de 
  \isa{S} construida según la definición \isa{{\isadigit{4}}{\isachardot}{\isadigit{1}}{\isachardot}{\isadigit{1}}}. Entonces, para todos \isa{n}, \isa{m\ {\isasymin}\ {\isasymnat}} 
  se verifica $\bigcup_{n \leq m} S_{n} = S_{m}$
\end{lema}

\begin{demostracion}
  En las condiciones del enunciado, la prueba se realiza por inducción en \isa{m\ {\isasymin}\ {\isasymnat}}.

  En primer lugar, consideremos el caso base \isa{m\ {\isacharequal}\ {\isadigit{0}}}. El resultado se obtiene directamente:

  $\bigcup_{n \leq 0} S_{n} = \bigcup_{n = 0} S_{n} = S_{0} = S_{m}$

  Por otro lado, supongamos por hipótesis de inducción que $\bigcup_{n \leq m} S_{n} = S_{m}$.
  Veamos que se verifica $\bigcup_{n \leq m + 1} S_{n} = S_{m + 1}$. Observemos que si \isa{n\ {\isasymle}\ m\ {\isacharplus}\ {\isadigit{1}}},
  entonces se tiene que, o bien \isa{n\ {\isasymle}\ m}, o bien \isa{n\ {\isacharequal}\ m\ {\isacharplus}\ {\isadigit{1}}}. De este modo, aplicando la 
  hipótesis de inducción, deducimos lo siguiente.

  $\bigcup_{n \leq m + 1} S_{n} = \bigcup_{n \leq m} S_{n} \cup \bigcup_{n = m + 1} S_{n} = \bigcup_{n \leq m} S_{n} \cup S_{m + 1} = S_{m} \cup S_{m + 1}$

  Por la monotonía de la sucesión, se tiene que \isa{S\isactrlsub m\ {\isasymsubseteq}\ S\isactrlsub m\isactrlsub {\isacharplus}\isactrlsub {\isadigit{1}}}. Luego, se verifica:

  $\bigcup_{n \leq m + 1} S_{n} = S_{m} \cup S_{m + 1} = S_{m + 1}$

  Lo que prueba el resultado.
\end{demostracion}

  Procedamos a su formalización y demostración detallada. Para ello, emplearemos la unión 
  generalizada en Isabelle/HOL perteneciente a la teoría 
  \href{https://n9.cl/gtf5x}{Complete-Lattices.thy}, junto con distintas propiedades sobre la misma
  definidas en dicha teoría. El uso de teoría de retículos en este caso se debe a que, en Isabelle,
  los conjuntos se han formalizado como predicados según la teoría 
  \href{https://bit.ly/3ibCuje}{Set.thy}. De esta manera, un elemento pertenece a un conjunto si 
  verifica el predicado que lo caracteriza. Además, en dicha teoría se instancia que el tipo de los 
  conjuntos es un álgebra de \isa{Boole} acotada, es decir, es un retículo distributivo para las 
  operaciones unión e intersección que tiene un supremo y un ínfimo. En consecuencia, la unión 
  generalizada de conjuntos se formaliza en Isabelle como el supremo del retículo completo que 
  conforman.

  Veamos la prueba detallada del resultado en Isabelle/HOL.%
\end{isamarkuptext}\isamarkuptrue%
\isacommand{lemma}\isamarkupfalse%
\ {\isachardoublequoteopen}{\isasymUnion}{\isacharbraceleft}pcp{\isacharunderscore}seq\ C\ S\ n{\isacharbar}n{\isachardot}\ n\ {\isasymle}\ m{\isacharbraceright}\ {\isacharequal}\ pcp{\isacharunderscore}seq\ C\ S\ m{\isachardoublequoteclose}\isanewline
%
\isadelimproof
%
\endisadelimproof
%
\isatagproof
\isacommand{proof}\isamarkupfalse%
\ {\isacharparenleft}induct\ m{\isacharparenright}\isanewline
\ \ \isacommand{have}\isamarkupfalse%
\ \ {\isachardoublequoteopen}{\isasymUnion}{\isacharbraceleft}pcp{\isacharunderscore}seq\ C\ S\ n{\isacharbar}n{\isachardot}\ n\ {\isasymle}\ {\isadigit{0}}{\isacharbraceright}\ {\isacharequal}\ {\isasymUnion}{\isacharbraceleft}pcp{\isacharunderscore}seq\ C\ S\ n{\isacharbar}n{\isachardot}\ n\ {\isacharequal}\ {\isadigit{0}}{\isacharbraceright}{\isachardoublequoteclose}\isanewline
\ \ \ \ \isacommand{by}\isamarkupfalse%
\ {\isacharparenleft}simp\ only{\isacharcolon}\ le{\isacharunderscore}zero{\isacharunderscore}eq{\isacharparenright}\isanewline
\ \ \isacommand{also}\isamarkupfalse%
\ \isacommand{have}\isamarkupfalse%
\ {\isachardoublequoteopen}{\isasymdots}\ {\isacharequal}\ {\isasymUnion}{\isacharparenleft}{\isacharparenleft}pcp{\isacharunderscore}seq\ C\ S{\isacharparenright}{\isacharbackquote}{\isacharbraceleft}n{\isachardot}\ n\ {\isacharequal}\ {\isadigit{0}}{\isacharbraceright}{\isacharparenright}{\isachardoublequoteclose}\isanewline
\ \ \ \ \isacommand{by}\isamarkupfalse%
\ {\isacharparenleft}simp\ only{\isacharcolon}\ image{\isacharunderscore}Collect{\isacharparenright}\isanewline
\ \ \isacommand{also}\isamarkupfalse%
\ \isacommand{have}\isamarkupfalse%
\ {\isachardoublequoteopen}{\isasymdots}\ {\isacharequal}\ {\isasymUnion}{\isacharbraceleft}pcp{\isacharunderscore}seq\ C\ S\ {\isadigit{0}}{\isacharbraceright}{\isachardoublequoteclose}\isanewline
\ \ \ \ \isacommand{by}\isamarkupfalse%
\ {\isacharparenleft}simp\ only{\isacharcolon}\ singleton{\isacharunderscore}conv\ image{\isacharunderscore}insert\ image{\isacharunderscore}empty{\isacharparenright}\isanewline
\ \ \isacommand{also}\isamarkupfalse%
\ \isacommand{have}\isamarkupfalse%
\ {\isachardoublequoteopen}{\isasymdots}\ {\isacharequal}\ pcp{\isacharunderscore}seq\ C\ S\ {\isadigit{0}}{\isachardoublequoteclose}\ \isanewline
\ \ \ \ \isacommand{by}\isamarkupfalse%
\ \ {\isacharparenleft}simp\ only{\isacharcolon}cSup{\isacharunderscore}singleton{\isacharparenright}\isanewline
\ \ \isacommand{finally}\isamarkupfalse%
\ \isacommand{show}\isamarkupfalse%
\ {\isachardoublequoteopen}{\isasymUnion}{\isacharbraceleft}pcp{\isacharunderscore}seq\ C\ S\ n{\isacharbar}n{\isachardot}\ n\ {\isasymle}\ {\isadigit{0}}{\isacharbraceright}\ {\isacharequal}\ pcp{\isacharunderscore}seq\ C\ S\ {\isadigit{0}}{\isachardoublequoteclose}\ \isanewline
\ \ \ \ \isacommand{by}\isamarkupfalse%
\ this\isanewline
\isacommand{next}\isamarkupfalse%
\isanewline
\ \ \isacommand{fix}\isamarkupfalse%
\ m\isanewline
\ \ \isacommand{assume}\isamarkupfalse%
\ HI{\isacharcolon}{\isachardoublequoteopen}{\isasymUnion}{\isacharbraceleft}pcp{\isacharunderscore}seq\ C\ S\ n{\isacharbar}n{\isachardot}\ n\ {\isasymle}\ m{\isacharbraceright}\ {\isacharequal}\ pcp{\isacharunderscore}seq\ C\ S\ m{\isachardoublequoteclose}\isanewline
\ \ \isacommand{have}\isamarkupfalse%
\ {\isachardoublequoteopen}m\ {\isasymle}\ Suc\ m{\isachardoublequoteclose}\ \isanewline
\ \ \ \ \isacommand{by}\isamarkupfalse%
\ {\isacharparenleft}simp\ only{\isacharcolon}\ add{\isacharunderscore}{\isadigit{0}}{\isacharunderscore}right{\isacharparenright}\isanewline
\ \ \isacommand{then}\isamarkupfalse%
\ \isacommand{have}\isamarkupfalse%
\ Mon{\isacharcolon}{\isachardoublequoteopen}pcp{\isacharunderscore}seq\ C\ S\ m\ {\isasymsubseteq}\ pcp{\isacharunderscore}seq\ C\ S\ {\isacharparenleft}Suc\ m{\isacharparenright}{\isachardoublequoteclose}\isanewline
\ \ \ \ \isacommand{by}\isamarkupfalse%
\ {\isacharparenleft}rule\ pcp{\isacharunderscore}seq{\isacharunderscore}mono{\isacharparenright}\isanewline
\ \ \isacommand{have}\isamarkupfalse%
\ {\isachardoublequoteopen}{\isasymUnion}{\isacharbraceleft}pcp{\isacharunderscore}seq\ C\ S\ n\ {\isacharbar}\ n{\isachardot}\ n\ {\isasymle}\ Suc\ m{\isacharbraceright}\ {\isacharequal}\ {\isasymUnion}{\isacharparenleft}{\isacharparenleft}pcp{\isacharunderscore}seq\ C\ S{\isacharparenright}{\isacharbackquote}{\isacharparenleft}{\isacharbraceleft}n{\isachardot}\ n\ {\isasymle}\ Suc\ m{\isacharbraceright}{\isacharparenright}{\isacharparenright}{\isachardoublequoteclose}\isanewline
\ \ \ \ \isacommand{by}\isamarkupfalse%
\ {\isacharparenleft}simp\ only{\isacharcolon}\ image{\isacharunderscore}Collect{\isacharparenright}\isanewline
\ \ \isacommand{also}\isamarkupfalse%
\ \isacommand{have}\isamarkupfalse%
\ {\isachardoublequoteopen}{\isasymdots}\ {\isacharequal}\ {\isasymUnion}{\isacharparenleft}{\isacharparenleft}pcp{\isacharunderscore}seq\ C\ S{\isacharparenright}{\isacharbackquote}{\isacharparenleft}{\isacharbraceleft}Suc\ m{\isacharbraceright}\ {\isasymunion}\ {\isacharbraceleft}n{\isachardot}\ n\ {\isasymle}\ m{\isacharbraceright}{\isacharparenright}{\isacharparenright}{\isachardoublequoteclose}\isanewline
\ \ \ \ \isacommand{by}\isamarkupfalse%
\ {\isacharparenleft}simp\ only{\isacharcolon}\ le{\isacharunderscore}Suc{\isacharunderscore}eq\ Collect{\isacharunderscore}disj{\isacharunderscore}eq\ Un{\isacharunderscore}commute\ singleton{\isacharunderscore}conv{\isacharparenright}\isanewline
\ \ \isacommand{also}\isamarkupfalse%
\ \isacommand{have}\isamarkupfalse%
\ {\isachardoublequoteopen}{\isasymdots}\ {\isacharequal}\ {\isasymUnion}{\isacharparenleft}{\isacharbraceleft}pcp{\isacharunderscore}seq\ C\ S\ {\isacharparenleft}Suc\ m{\isacharparenright}{\isacharbraceright}\ {\isasymunion}\ {\isacharbraceleft}pcp{\isacharunderscore}seq\ C\ S\ n\ {\isacharbar}\ n{\isachardot}\ n\ {\isasymle}\ m{\isacharbraceright}{\isacharparenright}{\isachardoublequoteclose}\isanewline
\ \ \ \ \isacommand{by}\isamarkupfalse%
\ {\isacharparenleft}simp\ only{\isacharcolon}\ image{\isacharunderscore}Un\ image{\isacharunderscore}insert\ image{\isacharunderscore}empty\ image{\isacharunderscore}Collect{\isacharparenright}\isanewline
\ \ \isacommand{also}\isamarkupfalse%
\ \isacommand{have}\isamarkupfalse%
\ {\isachardoublequoteopen}{\isasymdots}\ {\isacharequal}\ {\isasymUnion}{\isacharbraceleft}pcp{\isacharunderscore}seq\ C\ S\ {\isacharparenleft}Suc\ m{\isacharparenright}{\isacharbraceright}\ {\isasymunion}\ {\isasymUnion}{\isacharbraceleft}pcp{\isacharunderscore}seq\ C\ S\ n\ {\isacharbar}\ n{\isachardot}\ n\ {\isasymle}\ m{\isacharbraceright}{\isachardoublequoteclose}\isanewline
\ \ \ \ \isacommand{by}\isamarkupfalse%
\ {\isacharparenleft}simp\ only{\isacharcolon}\ Union{\isacharunderscore}Un{\isacharunderscore}distrib{\isacharparenright}\isanewline
\ \ \isacommand{also}\isamarkupfalse%
\ \isacommand{have}\isamarkupfalse%
\ {\isachardoublequoteopen}{\isasymdots}\ {\isacharequal}\ {\isacharparenleft}pcp{\isacharunderscore}seq\ C\ S\ {\isacharparenleft}Suc\ m{\isacharparenright}{\isacharparenright}\ {\isasymunion}\ {\isasymUnion}{\isacharbraceleft}pcp{\isacharunderscore}seq\ C\ S\ n\ {\isacharbar}\ n{\isachardot}\ n\ {\isasymle}\ m{\isacharbraceright}{\isachardoublequoteclose}\isanewline
\ \ \ \ \isacommand{by}\isamarkupfalse%
\ {\isacharparenleft}simp\ only{\isacharcolon}\ cSup{\isacharunderscore}singleton{\isacharparenright}\isanewline
\ \ \isacommand{also}\isamarkupfalse%
\ \isacommand{have}\isamarkupfalse%
\ {\isachardoublequoteopen}{\isasymdots}\ {\isacharequal}\ {\isacharparenleft}pcp{\isacharunderscore}seq\ C\ S\ {\isacharparenleft}Suc\ m{\isacharparenright}{\isacharparenright}\ {\isasymunion}\ {\isacharparenleft}pcp{\isacharunderscore}seq\ C\ S\ m{\isacharparenright}{\isachardoublequoteclose}\isanewline
\ \ \ \ \isacommand{by}\isamarkupfalse%
\ {\isacharparenleft}simp\ only{\isacharcolon}\ HI{\isacharparenright}\isanewline
\ \ \isacommand{also}\isamarkupfalse%
\ \isacommand{have}\isamarkupfalse%
\ {\isachardoublequoteopen}{\isasymdots}\ {\isacharequal}\ pcp{\isacharunderscore}seq\ C\ S\ {\isacharparenleft}Suc\ m{\isacharparenright}{\isachardoublequoteclose}\isanewline
\ \ \ \ \isacommand{using}\isamarkupfalse%
\ Mon\ \isacommand{by}\isamarkupfalse%
\ {\isacharparenleft}simp\ only{\isacharcolon}\ Un{\isacharunderscore}absorb{\isadigit{2}}{\isacharparenright}\isanewline
\ \ \isacommand{finally}\isamarkupfalse%
\ \isacommand{show}\isamarkupfalse%
\ {\isachardoublequoteopen}{\isasymUnion}{\isacharbraceleft}pcp{\isacharunderscore}seq\ C\ S\ n{\isacharbar}n{\isachardot}\ n\ {\isasymle}\ {\isacharparenleft}Suc\ m{\isacharparenright}{\isacharbraceright}\ {\isacharequal}\ pcp{\isacharunderscore}seq\ C\ S\ {\isacharparenleft}Suc\ m{\isacharparenright}{\isachardoublequoteclose}\isanewline
\ \ \ \ \isacommand{by}\isamarkupfalse%
\ this\isanewline
\isacommand{qed}\isamarkupfalse%
%
\endisatagproof
{\isafoldproof}%
%
\isadelimproof
%
\endisadelimproof
%
\begin{isamarkuptext}%
Análogamente, podemos dar una prueba automática.%
\end{isamarkuptext}\isamarkuptrue%
\isacommand{lemma}\isamarkupfalse%
\ pcp{\isacharunderscore}seq{\isacharunderscore}UN{\isacharcolon}\ {\isachardoublequoteopen}{\isasymUnion}{\isacharbraceleft}pcp{\isacharunderscore}seq\ C\ S\ n{\isacharbar}n{\isachardot}\ n\ {\isasymle}\ m{\isacharbraceright}\ {\isacharequal}\ pcp{\isacharunderscore}seq\ C\ S\ m{\isachardoublequoteclose}\isanewline
%
\isadelimproof
%
\endisadelimproof
%
\isatagproof
\isacommand{proof}\isamarkupfalse%
{\isacharparenleft}induction\ m{\isacharparenright}\isanewline
\ \ \isacommand{case}\isamarkupfalse%
\ {\isacharparenleft}Suc\ m{\isacharparenright}\isanewline
\ \ \isacommand{have}\isamarkupfalse%
\ {\isachardoublequoteopen}{\isacharbraceleft}f\ n\ {\isacharbar}n{\isachardot}\ n\ {\isasymle}\ Suc\ m{\isacharbraceright}\ {\isacharequal}\ insert\ {\isacharparenleft}f\ {\isacharparenleft}Suc\ m{\isacharparenright}{\isacharparenright}\ {\isacharbraceleft}f\ n\ {\isacharbar}n{\isachardot}\ n\ {\isasymle}\ m{\isacharbraceright}{\isachardoublequoteclose}\ \isanewline
\ \ \ \ \isakeyword{for}\ f\ \isacommand{using}\isamarkupfalse%
\ le{\isacharunderscore}Suc{\isacharunderscore}eq\ \isacommand{by}\isamarkupfalse%
\ auto\isanewline
\ \ \isacommand{hence}\isamarkupfalse%
\ {\isachardoublequoteopen}{\isacharbraceleft}pcp{\isacharunderscore}seq\ C\ S\ n\ {\isacharbar}n{\isachardot}\ n\ {\isasymle}\ Suc\ m{\isacharbraceright}\ {\isacharequal}\ \isanewline
\ \ \ \ \ \ \ \ \ \ insert\ {\isacharparenleft}pcp{\isacharunderscore}seq\ C\ S\ {\isacharparenleft}Suc\ m{\isacharparenright}{\isacharparenright}\ {\isacharbraceleft}pcp{\isacharunderscore}seq\ C\ S\ n\ {\isacharbar}n{\isachardot}\ n\ {\isasymle}\ m{\isacharbraceright}{\isachardoublequoteclose}\ \isacommand{{\isachardot}}\isamarkupfalse%
\isanewline
\ \ \isacommand{hence}\isamarkupfalse%
\ {\isachardoublequoteopen}{\isasymUnion}{\isacharbraceleft}pcp{\isacharunderscore}seq\ C\ S\ n\ {\isacharbar}n{\isachardot}\ n\ {\isasymle}\ Suc\ m{\isacharbraceright}\ {\isacharequal}\ \isanewline
\ \ \ \ \ \ \ \ \ {\isasymUnion}{\isacharbraceleft}pcp{\isacharunderscore}seq\ C\ S\ n\ {\isacharbar}n{\isachardot}\ n\ {\isasymle}\ m{\isacharbraceright}\ {\isasymunion}\ pcp{\isacharunderscore}seq\ C\ S\ {\isacharparenleft}Suc\ m{\isacharparenright}{\isachardoublequoteclose}\ \isacommand{by}\isamarkupfalse%
\ auto\isanewline
\ \ \isacommand{thus}\isamarkupfalse%
\ {\isacharquery}case\ \isacommand{using}\isamarkupfalse%
\ Suc\ pcp{\isacharunderscore}seq{\isacharunderscore}mono\ \isacommand{by}\isamarkupfalse%
\ blast\isanewline
\isacommand{qed}\isamarkupfalse%
\ simp%
\endisatagproof
{\isafoldproof}%
%
\isadelimproof
%
\endisadelimproof
%
\begin{isamarkuptext}%
Finalmente, definamos el límite de las sucesiones presentadas en la definición \isa{{\isadigit{4}}{\isachardot}{\isadigit{1}}{\isachardot}{\isadigit{1}}}.

 \begin{definicion}
  Sea \isa{C} una colección, \isa{S\ {\isasymin}\ C} y \isa{{\isacharbraceleft}S\isactrlsub n{\isacharbraceright}} la sucesión de conjuntos de \isa{C} a partir de \isa{S} según la
  definición \isa{{\isadigit{4}}{\isachardot}{\isadigit{1}}{\isachardot}{\isadigit{1}}}. Se define el \isa{límite\ de\ la\ sucesión\ de\ conjuntos\ de\ C\ a\ partir\ de\ S} como 
  $\bigcup_{n = 0}^{\infty} S_{n}$
 \end{definicion}

  La definición del límite se formaliza utilizando la unión generalizada de Isabelle como sigue.%
\end{isamarkuptext}\isamarkuptrue%
\isacommand{definition}\isamarkupfalse%
\ {\isachardoublequoteopen}pcp{\isacharunderscore}lim\ C\ S\ {\isasymequiv}\ {\isasymUnion}{\isacharbraceleft}pcp{\isacharunderscore}seq\ C\ S\ n{\isacharbar}n{\isachardot}\ True{\isacharbraceright}{\isachardoublequoteclose}%
\begin{isamarkuptext}%
Veamos el primer resultado sobre el límite.

\begin{lema}
  Sea \isa{C} una colección de conjuntos, \isa{S\ {\isasymin}\ C} y \isa{{\isacharbraceleft}S\isactrlsub n{\isacharbraceright}} la sucesión de conjuntos de \isa{C} a partir de
  \isa{S} según la definición \isa{{\isadigit{4}}{\isachardot}{\isadigit{1}}{\isachardot}{\isadigit{1}}}. Entonces, para todo \isa{n\ {\isasymin}\ {\isasymnat}}, se verifica:

  $S_{n} \subseteq \bigcup_{n = 0}^{\infty} S_{n}$
\end{lema}

\begin{demostracion}
  El resultado se obtiene de manera inmediata ya que, para todo \isa{n\ {\isasymin}\ {\isasymnat}}, se verifica que 
  $S_{n} \in \{S_{n}\}_{n = 0}^{\infty}$. Por tanto, es claro que 
  $S_{n} \subseteq \bigcup_{n = 0}^{\infty} S_{n}$.
\end{demostracion}

  Su formalización y demostración detallada en Isabelle se muestran a continuación.%
\end{isamarkuptext}\isamarkuptrue%
\isacommand{lemma}\isamarkupfalse%
\ {\isachardoublequoteopen}pcp{\isacharunderscore}seq\ C\ S\ n\ {\isasymsubseteq}\ pcp{\isacharunderscore}lim\ C\ S{\isachardoublequoteclose}\isanewline
%
\isadelimproof
\ \ %
\endisadelimproof
%
\isatagproof
\isacommand{unfolding}\isamarkupfalse%
\ pcp{\isacharunderscore}lim{\isacharunderscore}def\isanewline
\isacommand{proof}\isamarkupfalse%
\ {\isacharminus}\isanewline
\ \ \isacommand{have}\isamarkupfalse%
\ {\isachardoublequoteopen}n\ {\isasymin}\ {\isacharbraceleft}n\ {\isacharbar}\ n{\isachardot}\ True{\isacharbraceright}{\isachardoublequoteclose}\ \isanewline
\ \ \ \ \isacommand{by}\isamarkupfalse%
\ {\isacharparenleft}simp\ only{\isacharcolon}\ simp{\isacharunderscore}thms{\isacharparenleft}{\isadigit{2}}{\isadigit{1}}{\isacharcomma}{\isadigit{3}}{\isadigit{8}}{\isacharparenright}\ Collect{\isacharunderscore}const\ if{\isacharunderscore}True\ UNIV{\isacharunderscore}I{\isacharparenright}\ \isanewline
\ \ \isacommand{then}\isamarkupfalse%
\ \isacommand{have}\isamarkupfalse%
\ {\isachardoublequoteopen}pcp{\isacharunderscore}seq\ C\ S\ n\ {\isasymin}\ {\isacharparenleft}pcp{\isacharunderscore}seq\ C\ S{\isacharparenright}{\isacharbackquote}{\isacharbraceleft}n\ {\isacharbar}\ n{\isachardot}\ True{\isacharbraceright}{\isachardoublequoteclose}\isanewline
\ \ \ \ \isacommand{by}\isamarkupfalse%
\ {\isacharparenleft}simp\ only{\isacharcolon}\ imageI{\isacharparenright}\isanewline
\ \ \isacommand{then}\isamarkupfalse%
\ \isacommand{have}\isamarkupfalse%
\ {\isachardoublequoteopen}pcp{\isacharunderscore}seq\ C\ S\ n\ {\isasymin}\ {\isacharbraceleft}pcp{\isacharunderscore}seq\ C\ S\ n\ {\isacharbar}\ n{\isachardot}\ True{\isacharbraceright}{\isachardoublequoteclose}\isanewline
\ \ \ \ \isacommand{by}\isamarkupfalse%
\ {\isacharparenleft}simp\ only{\isacharcolon}\ image{\isacharunderscore}Collect\ simp{\isacharunderscore}thms{\isacharparenleft}{\isadigit{4}}{\isadigit{0}}{\isacharparenright}{\isacharparenright}\isanewline
\ \ \isacommand{thus}\isamarkupfalse%
\ {\isachardoublequoteopen}pcp{\isacharunderscore}seq\ C\ S\ n\ {\isasymsubseteq}\ {\isasymUnion}{\isacharbraceleft}pcp{\isacharunderscore}seq\ C\ S\ n\ {\isacharbar}\ n{\isachardot}\ True{\isacharbraceright}{\isachardoublequoteclose}\isanewline
\ \ \ \ \isacommand{by}\isamarkupfalse%
\ {\isacharparenleft}simp\ only{\isacharcolon}\ Union{\isacharunderscore}upper{\isacharparenright}\isanewline
\isacommand{qed}\isamarkupfalse%
%
\endisatagproof
{\isafoldproof}%
%
\isadelimproof
%
\endisadelimproof
%
\begin{isamarkuptext}%
Podemos probarlo de manera automática como sigue.%
\end{isamarkuptext}\isamarkuptrue%
\isacommand{lemma}\isamarkupfalse%
\ pcp{\isacharunderscore}seq{\isacharunderscore}sub{\isacharcolon}\ {\isachardoublequoteopen}pcp{\isacharunderscore}seq\ C\ S\ n\ {\isasymsubseteq}\ pcp{\isacharunderscore}lim\ C\ S{\isachardoublequoteclose}\ \isanewline
%
\isadelimproof
\ \ %
\endisadelimproof
%
\isatagproof
\isacommand{unfolding}\isamarkupfalse%
\ pcp{\isacharunderscore}lim{\isacharunderscore}def\ \isacommand{by}\isamarkupfalse%
\ blast%
\endisatagproof
{\isafoldproof}%
%
\isadelimproof
%
\endisadelimproof
%
\begin{isamarkuptext}%
Mostremos otro resultado. 

  \begin{lema}
    Sea \isa{C} una colección de conjuntos de fórmulas proposicionales, \isa{S\ {\isasymin}\ C} y \isa{{\isacharbraceleft}S\isactrlsub n{\isacharbraceright}} la sucesión de 
    conjuntos de \isa{C} a partir de \isa{S} según la definición \isa{{\isadigit{4}}{\isachardot}{\isadigit{1}}{\isachardot}{\isadigit{1}}}. Si \isa{F} es una fórmula tal que
    $F \in \bigcup_{n = 0}^{\infty} S_{n}$, entonces existe un \isa{k\ {\isasymin}\ {\isasymnat}} tal que \isa{F\ {\isasymin}\ S\isactrlsub k}. 
  \end{lema}

  \begin{demostracion}
    La prueba es inmediata de la definición del límite de la sucesión de conjuntos \isa{{\isacharbraceleft}S\isactrlsub n{\isacharbraceright}}: si
    \isa{F} pertenece a la unión generalizada $\bigcup_{n = 0}^{\infty} S_{n}$, entonces existe algún
    conjunto \isa{S\isactrlsub k} tal que \isa{F\ {\isasymin}\ S\isactrlsub k}. Es decir, existe \isa{k\ {\isasymin}\ {\isasymnat}} tal que \isa{F\ {\isasymin}\ S\isactrlsub k}, como queríamos
    demostrar.
  \end{demostracion} 

  Su prueba detallada en Isabelle/HOL es la siguiente.%
\end{isamarkuptext}\isamarkuptrue%
\isacommand{lemma}\isamarkupfalse%
\ \isanewline
\ \ \isakeyword{assumes}\ {\isachardoublequoteopen}F\ {\isasymin}\ pcp{\isacharunderscore}lim\ C\ S{\isachardoublequoteclose}\isanewline
\ \ \isakeyword{shows}\ {\isachardoublequoteopen}{\isasymexists}k{\isachardot}\ F\ {\isasymin}\ pcp{\isacharunderscore}seq\ C\ S\ k{\isachardoublequoteclose}\ \isanewline
%
\isadelimproof
%
\endisadelimproof
%
\isatagproof
\isacommand{proof}\isamarkupfalse%
\ {\isacharminus}\isanewline
\ \ \isacommand{have}\isamarkupfalse%
\ {\isachardoublequoteopen}F\ {\isasymin}\ {\isasymUnion}{\isacharparenleft}{\isacharparenleft}pcp{\isacharunderscore}seq\ C\ S{\isacharparenright}\ {\isacharbackquote}\ {\isacharbraceleft}n\ {\isacharbar}\ n{\isachardot}\ True{\isacharbraceright}{\isacharparenright}{\isachardoublequoteclose}\isanewline
\ \ \ \ \isacommand{using}\isamarkupfalse%
\ assms\ \isacommand{by}\isamarkupfalse%
\ {\isacharparenleft}simp\ only{\isacharcolon}\ pcp{\isacharunderscore}lim{\isacharunderscore}def\ image{\isacharunderscore}Collect\ simp{\isacharunderscore}thms{\isacharparenleft}{\isadigit{4}}{\isadigit{0}}{\isacharparenright}{\isacharparenright}\isanewline
\ \ \isacommand{then}\isamarkupfalse%
\ \isacommand{have}\isamarkupfalse%
\ {\isachardoublequoteopen}{\isasymexists}k\ {\isasymin}\ {\isacharbraceleft}n{\isachardot}\ True{\isacharbraceright}{\isachardot}\ F\ {\isasymin}\ pcp{\isacharunderscore}seq\ C\ S\ k{\isachardoublequoteclose}\isanewline
\ \ \ \ \isacommand{by}\isamarkupfalse%
\ {\isacharparenleft}simp\ only{\isacharcolon}\ UN{\isacharunderscore}iff\ simp{\isacharunderscore}thms{\isacharparenleft}{\isadigit{4}}{\isadigit{0}}{\isacharparenright}{\isacharparenright}\isanewline
\ \ \isacommand{then}\isamarkupfalse%
\ \isacommand{have}\isamarkupfalse%
\ {\isachardoublequoteopen}{\isasymexists}k\ {\isasymin}\ UNIV{\isachardot}\ F\ {\isasymin}\ pcp{\isacharunderscore}seq\ C\ S\ k{\isachardoublequoteclose}\ \isanewline
\ \ \ \ \isacommand{by}\isamarkupfalse%
\ {\isacharparenleft}simp\ only{\isacharcolon}\ UNIV{\isacharunderscore}def{\isacharparenright}\isanewline
\ \ \isacommand{thus}\isamarkupfalse%
\ {\isachardoublequoteopen}{\isasymexists}k{\isachardot}\ F\ {\isasymin}\ pcp{\isacharunderscore}seq\ C\ S\ k{\isachardoublequoteclose}\ \isanewline
\ \ \ \ \isacommand{by}\isamarkupfalse%
\ {\isacharparenleft}simp\ only{\isacharcolon}\ bex{\isacharunderscore}UNIV{\isacharparenright}\isanewline
\isacommand{qed}\isamarkupfalse%
%
\endisatagproof
{\isafoldproof}%
%
\isadelimproof
%
\endisadelimproof
%
\begin{isamarkuptext}%
Mostremos, a continuación, la demostración automática del resultado.%
\end{isamarkuptext}\isamarkuptrue%
\isacommand{lemma}\isamarkupfalse%
\ pcp{\isacharunderscore}lim{\isacharunderscore}inserted{\isacharunderscore}at{\isacharunderscore}ex{\isacharcolon}\ \isanewline
\ \ \ \ {\isachardoublequoteopen}S{\isacharprime}\ {\isasymin}\ pcp{\isacharunderscore}lim\ C\ S\ {\isasymLongrightarrow}\ {\isasymexists}k{\isachardot}\ S{\isacharprime}\ {\isasymin}\ pcp{\isacharunderscore}seq\ C\ S\ k{\isachardoublequoteclose}\isanewline
%
\isadelimproof
\ \ %
\endisadelimproof
%
\isatagproof
\isacommand{unfolding}\isamarkupfalse%
\ pcp{\isacharunderscore}lim{\isacharunderscore}def\ \isacommand{by}\isamarkupfalse%
\ blast%
\endisatagproof
{\isafoldproof}%
%
\isadelimproof
%
\endisadelimproof
%
\begin{isamarkuptext}%
Por último, veamos la siguiente propiedad sobre conjuntos finitos contenidos en el límite de 
  las sucesiones definido en \isa{{\isadigit{4}}{\isachardot}{\isadigit{1}}{\isachardot}{\isadigit{5}}}.

\begin{lema}
  Sea \isa{C} una colección, \isa{S\ {\isasymin}\ C} y \isa{{\isacharbraceleft}S\isactrlsub n{\isacharbraceright}} la sucesión de conjuntos de \isa{C} a partir de \isa{S} según la
  definición \isa{{\isadigit{4}}{\isachardot}{\isadigit{1}}{\isachardot}{\isadigit{1}}}. Si \isa{S{\isacharprime}} es un conjunto finito tal que \isa{S{\isacharprime}\ {\isasymsubseteq}} $\bigcup_{n = 0}^{\infty} S_{n}$, 
  entonces existe un\\ \isa{k\ {\isasymin}\ {\isasymnat}} tal que \isa{S{\isacharprime}\ {\isasymsubseteq}\ S\isactrlsub k}.
\end{lema}

\begin{demostracion}
  La prueba del resultado se realiza por inducción sobre la estructura recursiva de los conjuntos 
  finitos.

  En primer lugar, probemos el caso base correspondiente al conjunto vacío. Supongamos que \isa{{\isacharbraceleft}{\isacharbraceright}} está 
  contenido en el límite de la sucesión de conjuntos de \isa{C} a partir de \isa{S}. Como \isa{{\isacharbraceleft}{\isacharbraceright}} es 
  subconjunto de todo conjunto, en particular lo es de \isa{S\ {\isacharequal}\ S\isactrlsub {\isadigit{0}}}, probando así el primer caso.

  Por otra parte, probemos el paso de inducción. Sea \isa{S{\isacharprime}} un conjunto finito verificando la 
  hipótesis de inducción: si \isa{S{\isacharprime}} está contenido en el límite de la sucesión de conjuntos de 
  \isa{C} a partir de \isa{S}, entonces también está contenido en algún \isa{S\isactrlsub k\isactrlsub {\isacharprime}} para cierto \isa{k{\isacharprime}\ {\isasymin}\ {\isasymnat}}. Sea 
  \isa{F} una fórmula tal que \isa{F\ {\isasymnotin}\ S{\isacharprime}}. Vamos a probar que si \isa{{\isacharbraceleft}F{\isacharbraceright}\ {\isasymunion}\ S{\isacharprime}} está contenido en el límite, 
  entonces está contenido en \isa{S\isactrlsub k} para algún \isa{k\ {\isasymin}\ {\isasymnat}}. 

  Como hemos supuesto que \isa{{\isacharbraceleft}F{\isacharbraceright}\ {\isasymunion}\ S{\isacharprime}} está contenido en el límite, entonces se verifica que \isa{F}
  pertenece al límite y \isa{S{\isacharprime}} está contenido en él. Por el lema \isa{{\isadigit{4}}{\isachardot}{\isadigit{1}}{\isachardot}{\isadigit{7}}}, como \isa{F} pertenece al 
  límite, deducimos que existe un \isa{k\ {\isasymin}\ {\isasymnat}} tal que \isa{F\ {\isasymin}\ S\isactrlsub k}. Por otro lado, como \isa{S{\isacharprime}} está contenido
  en el límite, por hipótesis de inducción existe algún \isa{k{\isacharprime}\ {\isasymin}\ {\isasymnat}} tal que \isa{S{\isacharprime}\ {\isasymsubseteq}\ S\isactrlsub k\isactrlsub {\isacharprime}}. El resultado 
  se obtiene considerando el máximo entre \isa{k} y \isa{k{\isacharprime}}, que notaremos por \isa{k{\isacharprime}{\isacharprime}}. En efecto, por la 
  monotonía de la sucesión, se verifica que tanto \isa{S\isactrlsub k} como \isa{S\isactrlsub k\isactrlsub {\isacharprime}} están contenidos en \isa{S\isactrlsub k\isactrlsub {\isacharprime}\isactrlsub {\isacharprime}}. De este 
  modo, como \isa{S{\isacharprime}\ {\isasymsubseteq}\ S\isactrlsub k\isactrlsub {\isacharprime}}, por la transitividad de la contención de conjuntos se tiene que 
  \isa{S{\isacharprime}\ {\isasymsubseteq}\ S\isactrlsub k\isactrlsub {\isacharprime}\isactrlsub {\isacharprime}}. Además, como \isa{F\ {\isasymin}\ S\isactrlsub k}, se tiene que \isa{F\ {\isasymin}\ S\isactrlsub k\isactrlsub {\isacharprime}\isactrlsub {\isacharprime}}. Por lo tanto, \isa{{\isacharbraceleft}F{\isacharbraceright}\ {\isasymunion}\ S{\isacharprime}\ {\isasymsubseteq}\ S\isactrlsub k\isactrlsub {\isacharprime}\isactrlsub {\isacharprime}}, como 
  queríamos demostrar. 
\end{demostracion}

  Procedamos con la demostración detallada en Isabelle.%
\end{isamarkuptext}\isamarkuptrue%
\isacommand{lemma}\isamarkupfalse%
\ \isanewline
\ \ \isakeyword{assumes}\ {\isachardoublequoteopen}finite\ S{\isacharprime}{\isachardoublequoteclose}\isanewline
\ \ \ \ \ \ \ \ \ \ {\isachardoublequoteopen}S{\isacharprime}\ {\isasymsubseteq}\ pcp{\isacharunderscore}lim\ C\ S{\isachardoublequoteclose}\isanewline
\ \ \ \ \ \ \ \ \isakeyword{shows}\ {\isachardoublequoteopen}{\isasymexists}k{\isachardot}\ S{\isacharprime}\ {\isasymsubseteq}\ pcp{\isacharunderscore}seq\ C\ S\ k{\isachardoublequoteclose}\isanewline
%
\isadelimproof
\ \ %
\endisadelimproof
%
\isatagproof
\isacommand{using}\isamarkupfalse%
\ assms\isanewline
\isacommand{proof}\isamarkupfalse%
\ {\isacharparenleft}induction\ S{\isacharprime}\ rule{\isacharcolon}\ finite{\isacharunderscore}induct{\isacharparenright}\isanewline
\ \ \isacommand{case}\isamarkupfalse%
\ empty\isanewline
\ \ \isacommand{have}\isamarkupfalse%
\ {\isachardoublequoteopen}pcp{\isacharunderscore}seq\ C\ S\ {\isadigit{0}}\ {\isacharequal}\ S{\isachardoublequoteclose}\isanewline
\ \ \ \ \isacommand{by}\isamarkupfalse%
\ {\isacharparenleft}simp\ only{\isacharcolon}\ pcp{\isacharunderscore}seq{\isachardot}simps{\isacharparenleft}{\isadigit{1}}{\isacharparenright}{\isacharparenright}\isanewline
\ \ \isacommand{have}\isamarkupfalse%
\ {\isachardoublequoteopen}{\isacharbraceleft}{\isacharbraceright}\ {\isasymsubseteq}\ S{\isachardoublequoteclose}\isanewline
\ \ \ \ \isacommand{by}\isamarkupfalse%
\ {\isacharparenleft}rule\ order{\isacharunderscore}bot{\isacharunderscore}class{\isachardot}bot{\isachardot}extremum{\isacharparenright}\isanewline
\ \ \isacommand{then}\isamarkupfalse%
\ \isacommand{have}\isamarkupfalse%
\ {\isachardoublequoteopen}{\isacharbraceleft}{\isacharbraceright}\ {\isasymsubseteq}\ pcp{\isacharunderscore}seq\ C\ S\ {\isadigit{0}}{\isachardoublequoteclose}\isanewline
\ \ \ \ \isacommand{by}\isamarkupfalse%
\ {\isacharparenleft}simp\ only{\isacharcolon}\ {\isacartoucheopen}pcp{\isacharunderscore}seq\ C\ S\ {\isadigit{0}}\ {\isacharequal}\ S{\isacartoucheclose}{\isacharparenright}\isanewline
\ \ \isacommand{then}\isamarkupfalse%
\ \isacommand{show}\isamarkupfalse%
\ {\isacharquery}case\ \isanewline
\ \ \ \ \isacommand{by}\isamarkupfalse%
\ {\isacharparenleft}rule\ exI{\isacharparenright}\isanewline
\isacommand{next}\isamarkupfalse%
\isanewline
\ \ \isacommand{case}\isamarkupfalse%
\ {\isacharparenleft}insert\ F\ S{\isacharprime}{\isacharparenright}\isanewline
\ \ \isacommand{then}\isamarkupfalse%
\ \isacommand{have}\isamarkupfalse%
\ {\isachardoublequoteopen}insert\ F\ S{\isacharprime}\ {\isasymsubseteq}\ pcp{\isacharunderscore}lim\ C\ S{\isachardoublequoteclose}\isanewline
\ \ \ \ \isacommand{by}\isamarkupfalse%
\ {\isacharparenleft}simp\ only{\isacharcolon}\ insert{\isachardot}prems{\isacharparenright}\isanewline
\ \ \isacommand{then}\isamarkupfalse%
\ \isacommand{have}\isamarkupfalse%
\ C{\isacharcolon}{\isachardoublequoteopen}F\ {\isasymin}\ {\isacharparenleft}pcp{\isacharunderscore}lim\ C\ S{\isacharparenright}\ {\isasymand}\ S{\isacharprime}\ {\isasymsubseteq}\ pcp{\isacharunderscore}lim\ C\ S{\isachardoublequoteclose}\isanewline
\ \ \ \ \isacommand{by}\isamarkupfalse%
\ {\isacharparenleft}simp\ only{\isacharcolon}\ insert{\isacharunderscore}subset{\isacharparenright}\ \isanewline
\ \ \isacommand{then}\isamarkupfalse%
\ \isacommand{have}\isamarkupfalse%
\ {\isachardoublequoteopen}S{\isacharprime}\ {\isasymsubseteq}\ pcp{\isacharunderscore}lim\ C\ S{\isachardoublequoteclose}\isanewline
\ \ \ \ \isacommand{by}\isamarkupfalse%
\ {\isacharparenleft}rule\ conjunct{\isadigit{2}}{\isacharparenright}\isanewline
\ \ \isacommand{then}\isamarkupfalse%
\ \isacommand{have}\isamarkupfalse%
\ EX{\isadigit{1}}{\isacharcolon}{\isachardoublequoteopen}{\isasymexists}k{\isachardot}\ S{\isacharprime}\ {\isasymsubseteq}\ pcp{\isacharunderscore}seq\ C\ S\ k{\isachardoublequoteclose}\isanewline
\ \ \ \ \isacommand{by}\isamarkupfalse%
\ {\isacharparenleft}simp\ only{\isacharcolon}\ insert{\isachardot}IH{\isacharparenright}\isanewline
\ \ \isacommand{obtain}\isamarkupfalse%
\ k{\isadigit{1}}\ \isakeyword{where}\ {\isachardoublequoteopen}S{\isacharprime}\ {\isasymsubseteq}\ pcp{\isacharunderscore}seq\ C\ S\ k{\isadigit{1}}{\isachardoublequoteclose}\isanewline
\ \ \ \ \isacommand{using}\isamarkupfalse%
\ EX{\isadigit{1}}\ \isacommand{by}\isamarkupfalse%
\ {\isacharparenleft}rule\ exE{\isacharparenright}\isanewline
\ \ \isacommand{have}\isamarkupfalse%
\ {\isachardoublequoteopen}F\ {\isasymin}\ pcp{\isacharunderscore}lim\ C\ S{\isachardoublequoteclose}\isanewline
\ \ \ \ \isacommand{using}\isamarkupfalse%
\ C\ \isacommand{by}\isamarkupfalse%
\ {\isacharparenleft}rule\ conjunct{\isadigit{1}}{\isacharparenright}\isanewline
\ \ \isacommand{then}\isamarkupfalse%
\ \isacommand{have}\isamarkupfalse%
\ EX{\isadigit{2}}{\isacharcolon}{\isachardoublequoteopen}{\isasymexists}k{\isachardot}\ F\ {\isasymin}\ pcp{\isacharunderscore}seq\ C\ S\ k{\isachardoublequoteclose}\isanewline
\ \ \ \ \isacommand{by}\isamarkupfalse%
\ {\isacharparenleft}rule\ pcp{\isacharunderscore}lim{\isacharunderscore}inserted{\isacharunderscore}at{\isacharunderscore}ex{\isacharparenright}\isanewline
\ \ \isacommand{obtain}\isamarkupfalse%
\ k{\isadigit{2}}\ \isakeyword{where}\ {\isachardoublequoteopen}F\ {\isasymin}\ pcp{\isacharunderscore}seq\ C\ S\ k{\isadigit{2}}{\isachardoublequoteclose}\ \isanewline
\ \ \ \ \isacommand{using}\isamarkupfalse%
\ EX{\isadigit{2}}\ \isacommand{by}\isamarkupfalse%
\ {\isacharparenleft}rule\ exE{\isacharparenright}\isanewline
\ \ \isacommand{have}\isamarkupfalse%
\ {\isachardoublequoteopen}k{\isadigit{1}}\ {\isasymle}\ max\ k{\isadigit{1}}\ k{\isadigit{2}}{\isachardoublequoteclose}\isanewline
\ \ \ \ \isacommand{by}\isamarkupfalse%
\ {\isacharparenleft}simp\ only{\isacharcolon}\ linorder{\isacharunderscore}class{\isachardot}max{\isachardot}cobounded{\isadigit{1}}{\isacharparenright}\isanewline
\ \ \isacommand{then}\isamarkupfalse%
\ \isacommand{have}\isamarkupfalse%
\ {\isachardoublequoteopen}pcp{\isacharunderscore}seq\ C\ S\ k{\isadigit{1}}\ {\isasymsubseteq}\ pcp{\isacharunderscore}seq\ C\ S\ {\isacharparenleft}max\ k{\isadigit{1}}\ k{\isadigit{2}}{\isacharparenright}{\isachardoublequoteclose}\isanewline
\ \ \ \ \isacommand{by}\isamarkupfalse%
\ {\isacharparenleft}rule\ pcp{\isacharunderscore}seq{\isacharunderscore}mono{\isacharparenright}\isanewline
\ \ \isacommand{have}\isamarkupfalse%
\ {\isachardoublequoteopen}k{\isadigit{2}}\ {\isasymle}\ max\ k{\isadigit{1}}\ k{\isadigit{2}}{\isachardoublequoteclose}\isanewline
\ \ \ \ \isacommand{by}\isamarkupfalse%
\ {\isacharparenleft}simp\ only{\isacharcolon}\ linorder{\isacharunderscore}class{\isachardot}max{\isachardot}cobounded{\isadigit{2}}{\isacharparenright}\isanewline
\ \ \isacommand{then}\isamarkupfalse%
\ \isacommand{have}\isamarkupfalse%
\ {\isachardoublequoteopen}pcp{\isacharunderscore}seq\ C\ S\ k{\isadigit{2}}\ {\isasymsubseteq}\ pcp{\isacharunderscore}seq\ C\ S\ {\isacharparenleft}max\ k{\isadigit{1}}\ k{\isadigit{2}}{\isacharparenright}{\isachardoublequoteclose}\isanewline
\ \ \ \ \isacommand{by}\isamarkupfalse%
\ {\isacharparenleft}rule\ pcp{\isacharunderscore}seq{\isacharunderscore}mono{\isacharparenright}\isanewline
\ \ \isacommand{have}\isamarkupfalse%
\ {\isachardoublequoteopen}S{\isacharprime}\ {\isasymsubseteq}\ pcp{\isacharunderscore}seq\ C\ S\ {\isacharparenleft}max\ k{\isadigit{1}}\ k{\isadigit{2}}{\isacharparenright}{\isachardoublequoteclose}\isanewline
\ \ \ \ \isacommand{using}\isamarkupfalse%
\ {\isacartoucheopen}S{\isacharprime}\ {\isasymsubseteq}\ pcp{\isacharunderscore}seq\ C\ S\ k{\isadigit{1}}{\isacartoucheclose}\ {\isacartoucheopen}pcp{\isacharunderscore}seq\ C\ S\ k{\isadigit{1}}\ {\isasymsubseteq}\ pcp{\isacharunderscore}seq\ C\ S\ {\isacharparenleft}max\ k{\isadigit{1}}\ k{\isadigit{2}}{\isacharparenright}{\isacartoucheclose}\ \isacommand{by}\isamarkupfalse%
\ {\isacharparenleft}rule\ subset{\isacharunderscore}trans{\isacharparenright}\isanewline
\ \ \isacommand{have}\isamarkupfalse%
\ {\isachardoublequoteopen}F\ {\isasymin}\ pcp{\isacharunderscore}seq\ C\ S\ {\isacharparenleft}max\ k{\isadigit{1}}\ k{\isadigit{2}}{\isacharparenright}{\isachardoublequoteclose}\isanewline
\ \ \ \ \isacommand{using}\isamarkupfalse%
\ {\isacartoucheopen}F\ {\isasymin}\ pcp{\isacharunderscore}seq\ C\ S\ k{\isadigit{2}}{\isacartoucheclose}\ {\isacartoucheopen}pcp{\isacharunderscore}seq\ C\ S\ k{\isadigit{2}}\ {\isasymsubseteq}\ pcp{\isacharunderscore}seq\ C\ S\ {\isacharparenleft}max\ k{\isadigit{1}}\ k{\isadigit{2}}{\isacharparenright}{\isacartoucheclose}\ \isacommand{by}\isamarkupfalse%
\ {\isacharparenleft}rule\ rev{\isacharunderscore}subsetD{\isacharparenright}\isanewline
\ \ \isacommand{then}\isamarkupfalse%
\ \isacommand{have}\isamarkupfalse%
\ {\isadigit{1}}{\isacharcolon}{\isachardoublequoteopen}insert\ F\ S{\isacharprime}\ {\isasymsubseteq}\ pcp{\isacharunderscore}seq\ C\ S\ {\isacharparenleft}max\ k{\isadigit{1}}\ k{\isadigit{2}}{\isacharparenright}{\isachardoublequoteclose}\isanewline
\ \ \ \ \isacommand{using}\isamarkupfalse%
\ {\isacartoucheopen}S{\isacharprime}\ {\isasymsubseteq}\ pcp{\isacharunderscore}seq\ C\ S\ {\isacharparenleft}max\ k{\isadigit{1}}\ k{\isadigit{2}}{\isacharparenright}{\isacartoucheclose}\ \isacommand{by}\isamarkupfalse%
\ {\isacharparenleft}simp\ only{\isacharcolon}\ insert{\isacharunderscore}subset{\isacharparenright}\isanewline
\ \ \isacommand{thus}\isamarkupfalse%
\ {\isacharquery}case\isanewline
\ \ \ \ \isacommand{by}\isamarkupfalse%
\ {\isacharparenleft}rule\ exI{\isacharparenright}\isanewline
\isacommand{qed}\isamarkupfalse%
%
\endisatagproof
{\isafoldproof}%
%
\isadelimproof
%
\endisadelimproof
%
\begin{isamarkuptext}%
Finalmente, su demostración automática en Isabelle/HOL es la siguiente.%
\end{isamarkuptext}\isamarkuptrue%
\isacommand{lemma}\isamarkupfalse%
\ finite{\isacharunderscore}pcp{\isacharunderscore}lim{\isacharunderscore}EX{\isacharcolon}\isanewline
\ \ \isakeyword{assumes}\ {\isachardoublequoteopen}finite\ S{\isacharprime}{\isachardoublequoteclose}\isanewline
\ \ \ \ \ \ \ \ \ \ {\isachardoublequoteopen}S{\isacharprime}\ {\isasymsubseteq}\ pcp{\isacharunderscore}lim\ C\ S{\isachardoublequoteclose}\isanewline
\ \ \ \ \ \ \ \ \isakeyword{shows}\ {\isachardoublequoteopen}{\isasymexists}k{\isachardot}\ S{\isacharprime}\ {\isasymsubseteq}\ pcp{\isacharunderscore}seq\ C\ S\ k{\isachardoublequoteclose}\isanewline
%
\isadelimproof
\ \ %
\endisadelimproof
%
\isatagproof
\isacommand{using}\isamarkupfalse%
\ assms\isanewline
\isacommand{proof}\isamarkupfalse%
{\isacharparenleft}induction\ S{\isacharprime}\ rule{\isacharcolon}\ finite{\isacharunderscore}induct{\isacharparenright}\ \isanewline
\ \ \isacommand{case}\isamarkupfalse%
\ {\isacharparenleft}insert\ F\ S{\isacharprime}{\isacharparenright}\isanewline
\ \ \isacommand{hence}\isamarkupfalse%
\ {\isachardoublequoteopen}{\isasymexists}k{\isachardot}\ S{\isacharprime}\ {\isasymsubseteq}\ pcp{\isacharunderscore}seq\ C\ S\ k{\isachardoublequoteclose}\ \isacommand{by}\isamarkupfalse%
\ fast\isanewline
\ \ \isacommand{then}\isamarkupfalse%
\ \isacommand{guess}\isamarkupfalse%
\ k{\isadigit{1}}\ \isacommand{{\isachardot}{\isachardot}}\isamarkupfalse%
\isanewline
\ \ \isacommand{moreover}\isamarkupfalse%
\ \isacommand{obtain}\isamarkupfalse%
\ k{\isadigit{2}}\ \isakeyword{where}\ {\isachardoublequoteopen}F\ {\isasymin}\ pcp{\isacharunderscore}seq\ C\ S\ k{\isadigit{2}}{\isachardoublequoteclose}\isanewline
\ \ \ \ \isacommand{by}\isamarkupfalse%
\ {\isacharparenleft}meson\ pcp{\isacharunderscore}lim{\isacharunderscore}inserted{\isacharunderscore}at{\isacharunderscore}ex\ insert{\isachardot}prems\ insert{\isacharunderscore}subset{\isacharparenright}\isanewline
\ \ \isacommand{ultimately}\isamarkupfalse%
\ \isacommand{have}\isamarkupfalse%
\ {\isachardoublequoteopen}insert\ F\ S{\isacharprime}\ {\isasymsubseteq}\ pcp{\isacharunderscore}seq\ C\ S\ {\isacharparenleft}max\ k{\isadigit{1}}\ k{\isadigit{2}}{\isacharparenright}{\isachardoublequoteclose}\isanewline
\ \ \ \ \isacommand{by}\isamarkupfalse%
\ {\isacharparenleft}meson\ pcp{\isacharunderscore}seq{\isacharunderscore}mono\ dual{\isacharunderscore}order{\isachardot}trans\ insert{\isacharunderscore}subset\ max{\isachardot}bounded{\isacharunderscore}iff\ order{\isacharunderscore}refl\ subsetCE{\isacharparenright}\isanewline
\ \ \isacommand{thus}\isamarkupfalse%
\ {\isacharquery}case\ \isacommand{by}\isamarkupfalse%
\ blast\isanewline
\isacommand{qed}\isamarkupfalse%
\ simp%
\endisatagproof
{\isafoldproof}%
%
\isadelimproof
%
\endisadelimproof
%
\isadelimdocument
%
\endisadelimdocument
%
\isatagdocument
%
\isamarkupsection{El Teorema de Existencia de Modelo%
}
\isamarkuptrue%
%
\endisatagdocument
{\isafolddocument}%
%
\isadelimdocument
%
\endisadelimdocument
%
\begin{isamarkuptext}%
En esta sección demostraremos finalmente el 
  \isa{teorema\ de\ existencia\ de\ modelo}, el cual prueba que todo conjunto de fórmulas perteneciente a 
  una colección que verifique la propiedad de consistencia proposicional es satisfacible. Para ello, 
  considerando una colección \isa{C} cualquiera y \isa{S\ {\isasymin}\ C}, empleando resultados anteriores extenderemos 
  la colección a una colección \isa{C{\isacharprime}{\isacharprime}} que tenga la propiedad de consistencia proposicional, sea
  cerrada bajo subconjuntos y sea de carácter finito. De este modo, en esta sección probaremos que el 
  límite de la sucesión formada a partir de una colección que tenga dichas condiciones y un conjunto
  cualquiera \isa{S} como se indica en la definición \isa{{\isadigit{1}}{\isachardot}{\isadigit{4}}{\isachardot}{\isadigit{1}}} pertenece a la colección. Es más, 
  demostraremos que dicho límite se trata de un conjunto de \isa{Hintikka} luego, por el \isa{teorema\ de\ Hintikka}, es satisfacible. Finalmente, como \isa{S} está contenido en el límite, quedará demostrada 
  la satisfacibilidad del conjunto \isa{S} al heredarla por contención.

  \comentario{Habrá que modificar el párrafo anterior al final.}%
\end{isamarkuptext}\isamarkuptrue%
%
\begin{isamarkuptext}%
En primer lugar, probemos que si \isa{C} es una colección que verifica la propiedad de 
  consistencia proposicional, es cerrada bajo subconjuntos y es de carácter finito, entonces el 
  límite de toda sucesión de conjuntos de \isa{C} según la definición \isa{{\isadigit{4}}{\isachardot}{\isadigit{1}}{\isachardot}{\isadigit{1}}} pertenece a \isa{C}.

  \begin{lema}
    Sea \isa{C} una colección de conjuntos que verifica la propiedad de consistencia proposicional, es 
    cerrada bajo subconjuntos y es de carácter finito. Sea \isa{S\ {\isasymin}\ C} y \isa{{\isacharbraceleft}S\isactrlsub n{\isacharbraceright}} la sucesión de conjuntos
    de \isa{C} a partir de \isa{S} según la definición \isa{{\isadigit{4}}{\isachardot}{\isadigit{1}}{\isachardot}{\isadigit{1}}}. Entonces, el límite de la sucesión está en
    \isa{C}.
  \end{lema}

  \begin{demostracion}
    Por definición, como \isa{C} es de carácter finito, para todo conjunto son equivalentes:
    \begin{enumerate}
      \item El conjunto pertenece a \isa{C}.
      \item Todo subconjunto finito suyo pertenece a \isa{C}.
    \end{enumerate}

    De este modo, para demostrar que el límite de la sucesión \isa{{\isacharbraceleft}S\isactrlsub n{\isacharbraceright}} pertenece a \isa{C}, basta probar
    que todo subconjunto finito suyo está en \isa{C}.

    Sea \isa{S{\isacharprime}} un subconjunto finito del límite de la sucesión. Por el lema \isa{{\isadigit{1}}{\isachardot}{\isadigit{4}}{\isachardot}{\isadigit{8}}}, existe un
    \isa{k\ {\isasymin}\ {\isasymnat}} tal que \isa{S{\isacharprime}\ {\isasymsubseteq}\ S\isactrlsub k}. Por tanto, como \isa{S\isactrlsub k\ {\isasymin}\ C} para todo \isa{k\ {\isasymin}\ {\isasymnat}} y \isa{C} es cerrada bajo
    subconjuntos, por definición se tiene que \isa{S{\isacharprime}\ {\isasymin}\ C}, como queríamos demostrar.
  \end{demostracion}

  En Isabelle se formaliza y demuestra detalladamente como sigue.%
\end{isamarkuptext}\isamarkuptrue%
\isacommand{lemma}\isamarkupfalse%
\isanewline
\ \ \isakeyword{assumes}\ {\isachardoublequoteopen}pcp\ C{\isachardoublequoteclose}\isanewline
\ \ \ \ \ \ \ \ \ \ {\isachardoublequoteopen}S\ {\isasymin}\ C{\isachardoublequoteclose}\isanewline
\ \ \ \ \ \ \ \ \ \ {\isachardoublequoteopen}subset{\isacharunderscore}closed\ C{\isachardoublequoteclose}\isanewline
\ \ \ \ \ \ \ \ \ \ {\isachardoublequoteopen}finite{\isacharunderscore}character\ C{\isachardoublequoteclose}\isanewline
\ \ \isakeyword{shows}\ {\isachardoublequoteopen}pcp{\isacharunderscore}lim\ C\ S\ {\isasymin}\ C{\isachardoublequoteclose}\ \isanewline
%
\isadelimproof
%
\endisadelimproof
%
\isatagproof
\isacommand{proof}\isamarkupfalse%
\ {\isacharminus}\isanewline
\ \ \isacommand{have}\isamarkupfalse%
\ {\isachardoublequoteopen}{\isasymforall}S{\isachardot}\ S\ {\isasymin}\ C\ {\isasymlongleftrightarrow}\ {\isacharparenleft}{\isasymforall}S{\isacharprime}\ {\isasymsubseteq}\ S{\isachardot}\ finite\ S{\isacharprime}\ {\isasymlongrightarrow}\ S{\isacharprime}\ {\isasymin}\ C{\isacharparenright}{\isachardoublequoteclose}\isanewline
\ \ \ \ \isacommand{using}\isamarkupfalse%
\ assms{\isacharparenleft}{\isadigit{4}}{\isacharparenright}\ \isacommand{unfolding}\isamarkupfalse%
\ finite{\isacharunderscore}character{\isacharunderscore}def\ \isacommand{by}\isamarkupfalse%
\ this\isanewline
\ \ \isacommand{then}\isamarkupfalse%
\ \isacommand{have}\isamarkupfalse%
\ FC{\isadigit{1}}{\isacharcolon}{\isachardoublequoteopen}pcp{\isacharunderscore}lim\ C\ S\ {\isasymin}\ C\ {\isasymlongleftrightarrow}\ {\isacharparenleft}{\isasymforall}S{\isacharprime}\ {\isasymsubseteq}\ {\isacharparenleft}pcp{\isacharunderscore}lim\ C\ S{\isacharparenright}{\isachardot}\ finite\ S{\isacharprime}\ {\isasymlongrightarrow}\ S{\isacharprime}\ {\isasymin}\ C{\isacharparenright}{\isachardoublequoteclose}\isanewline
\ \ \ \ \isacommand{by}\isamarkupfalse%
\ {\isacharparenleft}rule\ allE{\isacharparenright}\isanewline
\ \ \isacommand{have}\isamarkupfalse%
\ SC{\isacharcolon}{\isachardoublequoteopen}{\isasymforall}S\ {\isasymin}\ C{\isachardot}\ {\isasymforall}S{\isacharprime}{\isasymsubseteq}S{\isachardot}\ S{\isacharprime}\ {\isasymin}\ C{\isachardoublequoteclose}\isanewline
\ \ \ \ \isacommand{using}\isamarkupfalse%
\ assms{\isacharparenleft}{\isadigit{3}}{\isacharparenright}\ \isacommand{unfolding}\isamarkupfalse%
\ subset{\isacharunderscore}closed{\isacharunderscore}def\ \isacommand{by}\isamarkupfalse%
\ this\isanewline
\ \ \isacommand{have}\isamarkupfalse%
\ FC{\isadigit{2}}{\isacharcolon}{\isachardoublequoteopen}{\isasymforall}S{\isacharprime}\ {\isasymsubseteq}\ pcp{\isacharunderscore}lim\ C\ S{\isachardot}\ finite\ S{\isacharprime}\ {\isasymlongrightarrow}\ S{\isacharprime}\ {\isasymin}\ C{\isachardoublequoteclose}\isanewline
\ \ \isacommand{proof}\isamarkupfalse%
\ {\isacharparenleft}rule\ sallI{\isacharparenright}\isanewline
\ \ \ \ \isacommand{fix}\isamarkupfalse%
\ S{\isacharprime}\ {\isacharcolon}{\isacharcolon}\ {\isachardoublequoteopen}{\isacharprime}a\ formula\ set{\isachardoublequoteclose}\isanewline
\ \ \ \ \isacommand{assume}\isamarkupfalse%
\ {\isachardoublequoteopen}S{\isacharprime}\ {\isasymsubseteq}\ pcp{\isacharunderscore}lim\ C\ S{\isachardoublequoteclose}\isanewline
\ \ \ \ \isacommand{show}\isamarkupfalse%
\ {\isachardoublequoteopen}finite\ S{\isacharprime}\ {\isasymlongrightarrow}\ S{\isacharprime}\ {\isasymin}\ C{\isachardoublequoteclose}\isanewline
\ \ \ \ \isacommand{proof}\isamarkupfalse%
\ {\isacharparenleft}rule\ impI{\isacharparenright}\isanewline
\ \ \ \ \ \ \isacommand{assume}\isamarkupfalse%
\ {\isachardoublequoteopen}finite\ S{\isacharprime}{\isachardoublequoteclose}\isanewline
\ \ \ \ \ \ \isacommand{then}\isamarkupfalse%
\ \isacommand{have}\isamarkupfalse%
\ EX{\isacharcolon}{\isachardoublequoteopen}{\isasymexists}k{\isachardot}\ S{\isacharprime}\ {\isasymsubseteq}\ pcp{\isacharunderscore}seq\ C\ S\ k{\isachardoublequoteclose}\ \isanewline
\ \ \ \ \ \ \ \ \isacommand{using}\isamarkupfalse%
\ {\isacartoucheopen}S{\isacharprime}\ {\isasymsubseteq}\ pcp{\isacharunderscore}lim\ C\ S{\isacartoucheclose}\ \isacommand{by}\isamarkupfalse%
\ {\isacharparenleft}rule\ finite{\isacharunderscore}pcp{\isacharunderscore}lim{\isacharunderscore}EX{\isacharparenright}\isanewline
\ \ \ \ \ \ \isacommand{obtain}\isamarkupfalse%
\ k\ \isakeyword{where}\ {\isachardoublequoteopen}S{\isacharprime}\ {\isasymsubseteq}\ pcp{\isacharunderscore}seq\ C\ S\ k{\isachardoublequoteclose}\isanewline
\ \ \ \ \ \ \ \ \isacommand{using}\isamarkupfalse%
\ EX\ \isacommand{by}\isamarkupfalse%
\ {\isacharparenleft}rule\ exE{\isacharparenright}\isanewline
\ \ \ \ \ \ \isacommand{have}\isamarkupfalse%
\ {\isachardoublequoteopen}pcp{\isacharunderscore}seq\ C\ S\ k\ {\isasymin}\ C{\isachardoublequoteclose}\isanewline
\ \ \ \ \ \ \ \ \isacommand{using}\isamarkupfalse%
\ assms{\isacharparenleft}{\isadigit{1}}{\isacharparenright}\ assms{\isacharparenleft}{\isadigit{2}}{\isacharparenright}\ \isacommand{by}\isamarkupfalse%
\ {\isacharparenleft}rule\ pcp{\isacharunderscore}seq{\isacharunderscore}in{\isacharparenright}\isanewline
\ \ \ \ \ \ \isacommand{have}\isamarkupfalse%
\ {\isachardoublequoteopen}{\isasymforall}S{\isacharprime}\ {\isasymsubseteq}\ {\isacharparenleft}pcp{\isacharunderscore}seq\ C\ S\ k{\isacharparenright}{\isachardot}\ S{\isacharprime}\ {\isasymin}\ C{\isachardoublequoteclose}\isanewline
\ \ \ \ \ \ \ \ \isacommand{using}\isamarkupfalse%
\ SC\ {\isacartoucheopen}pcp{\isacharunderscore}seq\ C\ S\ k\ {\isasymin}\ C{\isacartoucheclose}\ \isacommand{by}\isamarkupfalse%
\ {\isacharparenleft}rule\ bspec{\isacharparenright}\isanewline
\ \ \ \ \ \ \isacommand{thus}\isamarkupfalse%
\ {\isachardoublequoteopen}S{\isacharprime}\ {\isasymin}\ C{\isachardoublequoteclose}\isanewline
\ \ \ \ \ \ \ \ \isacommand{using}\isamarkupfalse%
\ {\isacartoucheopen}S{\isacharprime}\ {\isasymsubseteq}\ pcp{\isacharunderscore}seq\ C\ S\ k{\isacartoucheclose}\ \isacommand{by}\isamarkupfalse%
\ {\isacharparenleft}rule\ sspec{\isacharparenright}\isanewline
\ \ \ \ \isacommand{qed}\isamarkupfalse%
\isanewline
\ \ \isacommand{qed}\isamarkupfalse%
\isanewline
\ \ \isacommand{show}\isamarkupfalse%
\ {\isachardoublequoteopen}pcp{\isacharunderscore}lim\ C\ S\ {\isasymin}\ C{\isachardoublequoteclose}\ \isanewline
\ \ \ \ \isacommand{using}\isamarkupfalse%
\ FC{\isadigit{1}}\ FC{\isadigit{2}}\ \isacommand{by}\isamarkupfalse%
\ {\isacharparenleft}rule\ forw{\isacharunderscore}subst{\isacharparenright}\isanewline
\isacommand{qed}\isamarkupfalse%
%
\endisatagproof
{\isafoldproof}%
%
\isadelimproof
%
\endisadelimproof
%
\begin{isamarkuptext}%
Por otra parte, podemos dar una prueba automática del resultado.%
\end{isamarkuptext}\isamarkuptrue%
\isacommand{lemma}\isamarkupfalse%
\ pcp{\isacharunderscore}lim{\isacharunderscore}in{\isacharcolon}\isanewline
\ \ \isakeyword{assumes}\ c{\isacharcolon}\ {\isachardoublequoteopen}pcp\ C{\isachardoublequoteclose}\isanewline
\ \ \isakeyword{and}\ el{\isacharcolon}\ {\isachardoublequoteopen}S\ {\isasymin}\ C{\isachardoublequoteclose}\isanewline
\ \ \isakeyword{and}\ sc{\isacharcolon}\ {\isachardoublequoteopen}subset{\isacharunderscore}closed\ C{\isachardoublequoteclose}\isanewline
\ \ \isakeyword{and}\ fc{\isacharcolon}\ {\isachardoublequoteopen}finite{\isacharunderscore}character\ C{\isachardoublequoteclose}\isanewline
\ \ \isakeyword{shows}\ {\isachardoublequoteopen}pcp{\isacharunderscore}lim\ C\ S\ {\isasymin}\ C{\isachardoublequoteclose}\ {\isacharparenleft}\isakeyword{is}\ {\isachardoublequoteopen}{\isacharquery}cl\ {\isasymin}\ C{\isachardoublequoteclose}{\isacharparenright}\isanewline
%
\isadelimproof
%
\endisadelimproof
%
\isatagproof
\isacommand{proof}\isamarkupfalse%
\ {\isacharminus}\isanewline
\ \ \isacommand{from}\isamarkupfalse%
\ pcp{\isacharunderscore}seq{\isacharunderscore}in{\isacharbrackleft}OF\ c\ el{\isacharcomma}\ THEN\ allI{\isacharbrackright}\ \isacommand{have}\isamarkupfalse%
\ {\isachardoublequoteopen}{\isasymforall}n{\isachardot}\ pcp{\isacharunderscore}seq\ C\ S\ n\ {\isasymin}\ C{\isachardoublequoteclose}\ \isacommand{{\isachardot}}\isamarkupfalse%
\isanewline
\ \ \isacommand{hence}\isamarkupfalse%
\ {\isachardoublequoteopen}{\isasymforall}m{\isachardot}\ {\isasymUnion}{\isacharbraceleft}pcp{\isacharunderscore}seq\ C\ S\ n{\isacharbar}n{\isachardot}\ n\ {\isasymle}\ m{\isacharbraceright}\ {\isasymin}\ C{\isachardoublequoteclose}\ \isacommand{unfolding}\isamarkupfalse%
\ pcp{\isacharunderscore}seq{\isacharunderscore}UN\ \isacommand{{\isachardot}}\isamarkupfalse%
\isanewline
\ \ \isacommand{have}\isamarkupfalse%
\ {\isachardoublequoteopen}{\isasymforall}S{\isacharprime}\ {\isasymsubseteq}\ {\isacharquery}cl{\isachardot}\ finite\ S{\isacharprime}\ {\isasymlongrightarrow}\ S{\isacharprime}\ {\isasymin}\ C{\isachardoublequoteclose}\isanewline
\ \ \isacommand{proof}\isamarkupfalse%
\ safe\isanewline
\ \ \ \ \isacommand{fix}\isamarkupfalse%
\ S{\isacharprime}\ {\isacharcolon}{\isacharcolon}\ {\isachardoublequoteopen}{\isacharprime}a\ formula\ set{\isachardoublequoteclose}\isanewline
\ \ \ \ \isacommand{have}\isamarkupfalse%
\ {\isachardoublequoteopen}pcp{\isacharunderscore}seq\ C\ S\ {\isacharparenleft}Suc\ {\isacharparenleft}Max\ {\isacharparenleft}to{\isacharunderscore}nat\ {\isacharbackquote}\ S{\isacharprime}{\isacharparenright}{\isacharparenright}{\isacharparenright}\ {\isasymsubseteq}\ pcp{\isacharunderscore}lim\ C\ S{\isachardoublequoteclose}\ \isanewline
\ \ \ \ \ \ \isacommand{using}\isamarkupfalse%
\ pcp{\isacharunderscore}seq{\isacharunderscore}sub\ \isacommand{by}\isamarkupfalse%
\ blast\isanewline
\ \ \ \ \isacommand{assume}\isamarkupfalse%
\ {\isacartoucheopen}finite\ S{\isacharprime}{\isacartoucheclose}\ {\isacartoucheopen}S{\isacharprime}\ {\isasymsubseteq}\ pcp{\isacharunderscore}lim\ C\ S{\isacartoucheclose}\isanewline
\ \ \ \ \isacommand{hence}\isamarkupfalse%
\ {\isachardoublequoteopen}{\isasymexists}k{\isachardot}\ S{\isacharprime}\ {\isasymsubseteq}\ pcp{\isacharunderscore}seq\ C\ S\ k{\isachardoublequoteclose}\ \isanewline
\ \ \ \ \isacommand{proof}\isamarkupfalse%
{\isacharparenleft}induction\ S{\isacharprime}\ rule{\isacharcolon}\ finite{\isacharunderscore}induct{\isacharparenright}\ \isanewline
\ \ \ \ \ \ \isacommand{case}\isamarkupfalse%
\ {\isacharparenleft}insert\ x\ S{\isacharprime}{\isacharparenright}\isanewline
\ \ \ \ \ \ \isacommand{hence}\isamarkupfalse%
\ {\isachardoublequoteopen}{\isasymexists}k{\isachardot}\ S{\isacharprime}\ {\isasymsubseteq}\ pcp{\isacharunderscore}seq\ C\ S\ k{\isachardoublequoteclose}\ \isacommand{by}\isamarkupfalse%
\ fast\isanewline
\ \ \ \ \ \ \isacommand{then}\isamarkupfalse%
\ \isacommand{guess}\isamarkupfalse%
\ k{\isadigit{1}}\ \isacommand{{\isachardot}{\isachardot}}\isamarkupfalse%
\isanewline
\ \ \ \ \ \ \isacommand{moreover}\isamarkupfalse%
\ \isacommand{obtain}\isamarkupfalse%
\ k{\isadigit{2}}\ \isakeyword{where}\ {\isachardoublequoteopen}x\ {\isasymin}\ pcp{\isacharunderscore}seq\ C\ S\ k{\isadigit{2}}{\isachardoublequoteclose}\isanewline
\ \ \ \ \ \ \ \ \isacommand{by}\isamarkupfalse%
\ {\isacharparenleft}meson\ pcp{\isacharunderscore}lim{\isacharunderscore}inserted{\isacharunderscore}at{\isacharunderscore}ex\ insert{\isachardot}prems\ insert{\isacharunderscore}subset{\isacharparenright}\isanewline
\ \ \ \ \ \ \isacommand{ultimately}\isamarkupfalse%
\ \isacommand{have}\isamarkupfalse%
\ {\isachardoublequoteopen}insert\ x\ S{\isacharprime}\ {\isasymsubseteq}\ pcp{\isacharunderscore}seq\ C\ S\ {\isacharparenleft}max\ k{\isadigit{1}}\ k{\isadigit{2}}{\isacharparenright}{\isachardoublequoteclose}\isanewline
\ \ \ \ \ \ \ \ \isacommand{by}\isamarkupfalse%
\ {\isacharparenleft}meson\ pcp{\isacharunderscore}seq{\isacharunderscore}mono\ dual{\isacharunderscore}order{\isachardot}trans\ insert{\isacharunderscore}subset\ max{\isachardot}bounded{\isacharunderscore}iff\ order{\isacharunderscore}refl\ subsetCE{\isacharparenright}\isanewline
\ \ \ \ \ \ \isacommand{thus}\isamarkupfalse%
\ {\isacharquery}case\ \isacommand{by}\isamarkupfalse%
\ blast\isanewline
\ \ \ \ \isacommand{qed}\isamarkupfalse%
\ simp\isanewline
\ \ \ \ \isacommand{with}\isamarkupfalse%
\ pcp{\isacharunderscore}seq{\isacharunderscore}in{\isacharbrackleft}OF\ c\ el{\isacharbrackright}\ sc\isanewline
\ \ \ \ \isacommand{show}\isamarkupfalse%
\ {\isachardoublequoteopen}S{\isacharprime}\ {\isasymin}\ C{\isachardoublequoteclose}\ \isacommand{unfolding}\isamarkupfalse%
\ subset{\isacharunderscore}closed{\isacharunderscore}def\ \isacommand{by}\isamarkupfalse%
\ blast\isanewline
\ \ \isacommand{qed}\isamarkupfalse%
\isanewline
\ \ \isacommand{thus}\isamarkupfalse%
\ {\isachardoublequoteopen}{\isacharquery}cl\ {\isasymin}\ C{\isachardoublequoteclose}\ \isacommand{using}\isamarkupfalse%
\ fc\ \isacommand{unfolding}\isamarkupfalse%
\ finite{\isacharunderscore}character{\isacharunderscore}def\ \isacommand{by}\isamarkupfalse%
\ blast\isanewline
\isacommand{qed}\isamarkupfalse%
%
\endisatagproof
{\isafoldproof}%
%
\isadelimproof
%
\endisadelimproof
%
\begin{isamarkuptext}%
Probemos que, además, el límite de las sucesión definida en \isa{{\isadigit{4}}{\isachardot}{\isadigit{1}}{\isachardot}{\isadigit{1}}} se trata de un elemento 
  maximal de la colección que lo define si esta verifica la propiedad de consistencia proposicional
  y es cerrada bajo subconjuntos.

  \begin{lema}
    Sea \isa{C} una colección de conjuntos que verifica la propiedad de consistencia proposicional y
    es cerrada bajo subconjuntos, \isa{S} un conjunto y \isa{{\isacharbraceleft}S\isactrlsub n{\isacharbraceright}} la sucesión de conjuntos de \isa{C} a partir 
    de \isa{S} según la definición \isa{{\isadigit{4}}{\isachardot}{\isadigit{1}}{\isachardot}{\isadigit{1}}}. Entonces, el límite de la sucesión \isa{{\isacharbraceleft}S\isactrlsub n{\isacharbraceright}} es un elemento 
    maximal de \isa{C}.
  \end{lema}

  \begin{demostracion}
    Por definición de elemento maximal, basta probar que para cualquier conjunto \isa{K\ {\isasymin}\ C} que
    contenga al límite de la sucesión se tiene que \isa{K} y el límite coinciden.

    La demostración se realizará por reducción al absurdo. Consideremos un conjunto \isa{K\ {\isasymin}\ C} que 
    contenga estrictamente al límite de la sucesión \isa{{\isacharbraceleft}S\isactrlsub n{\isacharbraceright}}. De este modo, existe una fórmula \isa{F} tal 
    que \isa{F\ {\isasymin}\ K} y \isa{F} no está en el límite. Supongamos que \isa{F} es la \isa{n}-ésima fórmula según la 
    enumeración de la definición \isa{{\isadigit{4}}{\isachardot}{\isadigit{1}}{\isachardot}{\isadigit{1}}} utilizada para construir la sucesión. 

    Por un lado, hemos probado que todo elemento de la sucesión está contenido en el límite, luego 
    en particular obtenemos que \isa{S\isactrlsub n\isactrlsub {\isacharplus}\isactrlsub {\isadigit{1}}} está contenido en el límite. De este modo, como \isa{F} no 
    pertenece al límite, es claro que \isa{F\ {\isasymnotin}\ S\isactrlsub n\isactrlsub {\isacharplus}\isactrlsub {\isadigit{1}}}. Además, \isa{{\isacharbraceleft}F{\isacharbraceright}\ {\isasymunion}\ S\isactrlsub n\ {\isasymnotin}\ C} ya que, en caso contrario, 
    por la definición \isa{{\isadigit{4}}{\isachardot}{\isadigit{1}}{\isachardot}{\isadigit{1}}} de la sucesión obtendríamos que\\ \isa{S\isactrlsub n\isactrlsub {\isacharplus}\isactrlsub {\isadigit{1}}\ {\isacharequal}\ {\isacharbraceleft}F{\isacharbraceright}\ {\isasymunion}\ S\isactrlsub n}, lo que contradice 
    que \isa{F\ {\isasymnotin}\ S\isactrlsub n\isactrlsub {\isacharplus}\isactrlsub {\isadigit{1}}}. 

    Por otro lado, como \isa{S\isactrlsub n} también está contenida en el límite que, a su vez, está contenido en 
    \isa{K}, se obtiene por transitividad que \isa{S\isactrlsub n\ {\isasymsubseteq}\ K}. Además, como \isa{F\ {\isasymin}\ K}, se verifica que 
    \isa{{\isacharbraceleft}F{\isacharbraceright}\ {\isasymunion}\ S\isactrlsub n\ {\isasymsubseteq}\ K}. Como \isa{C} es una colección cerrada bajo subconjuntos por hipótesis y \isa{K\ {\isasymin}\ C}, 
    por definición se tiene que \isa{{\isacharbraceleft}F{\isacharbraceright}\ {\isasymunion}\ S\isactrlsub n\ {\isasymin}\ C}, llegando así a una contradicción con lo demostrado 
    anteriormente.
  \end{demostracion}

  Su formalización y prueba detallada en Isabelle/HOL se muestran a continuación.%
\end{isamarkuptext}\isamarkuptrue%
\isacommand{lemma}\isamarkupfalse%
\isanewline
\ \ \isakeyword{assumes}\ {\isachardoublequoteopen}pcp\ C{\isachardoublequoteclose}\isanewline
\ \ \ \ \ \ \ \ \ \ {\isachardoublequoteopen}subset{\isacharunderscore}closed\ C{\isachardoublequoteclose}\isanewline
\ \ \ \ \ \ \ \ \ \ {\isachardoublequoteopen}K\ {\isasymin}\ C{\isachardoublequoteclose}\isanewline
\ \ \ \ \ \ \ \ \ \ {\isachardoublequoteopen}pcp{\isacharunderscore}lim\ C\ S\ {\isasymsubseteq}\ K{\isachardoublequoteclose}\isanewline
\ \ \isakeyword{shows}\ {\isachardoublequoteopen}pcp{\isacharunderscore}lim\ C\ S\ {\isacharequal}\ K{\isachardoublequoteclose}\isanewline
%
\isadelimproof
%
\endisadelimproof
%
\isatagproof
\isacommand{proof}\isamarkupfalse%
\ {\isacharparenleft}rule\ ccontr{\isacharparenright}\isanewline
\ \ \isacommand{assume}\isamarkupfalse%
\ H{\isacharcolon}{\isachardoublequoteopen}{\isasymnot}{\isacharparenleft}pcp{\isacharunderscore}lim\ C\ S\ {\isacharequal}\ K{\isacharparenright}{\isachardoublequoteclose}\isanewline
\ \ \isacommand{have}\isamarkupfalse%
\ CE{\isacharcolon}{\isachardoublequoteopen}pcp{\isacharunderscore}lim\ C\ S\ {\isasymsubseteq}\ K\ {\isasymand}\ pcp{\isacharunderscore}lim\ C\ S\ {\isasymnoteq}\ K{\isachardoublequoteclose}\isanewline
\ \ \ \ \isacommand{using}\isamarkupfalse%
\ assms{\isacharparenleft}{\isadigit{4}}{\isacharparenright}\ H\ \isacommand{by}\isamarkupfalse%
\ {\isacharparenleft}rule\ conjI{\isacharparenright}\isanewline
\ \ \isacommand{have}\isamarkupfalse%
\ {\isachardoublequoteopen}pcp{\isacharunderscore}lim\ C\ S\ {\isasymsubseteq}\ K\ {\isasymand}\ pcp{\isacharunderscore}lim\ C\ S\ {\isasymnoteq}\ K\ {\isasymlongleftrightarrow}\ pcp{\isacharunderscore}lim\ C\ S\ {\isasymsubset}\ K{\isachardoublequoteclose}\isanewline
\ \ \ \ \isacommand{by}\isamarkupfalse%
\ {\isacharparenleft}simp\ only{\isacharcolon}\ psubset{\isacharunderscore}eq{\isacharparenright}\isanewline
\ \ \isacommand{then}\isamarkupfalse%
\ \isacommand{have}\isamarkupfalse%
\ {\isachardoublequoteopen}pcp{\isacharunderscore}lim\ C\ S\ {\isasymsubset}\ K{\isachardoublequoteclose}\ \isanewline
\ \ \ \ \isacommand{using}\isamarkupfalse%
\ CE\ \isacommand{by}\isamarkupfalse%
\ {\isacharparenleft}rule\ iffD{\isadigit{1}}{\isacharparenright}\isanewline
\ \ \isacommand{then}\isamarkupfalse%
\ \isacommand{have}\isamarkupfalse%
\ {\isachardoublequoteopen}{\isasymexists}F{\isachardot}\ F\ {\isasymin}\ {\isacharparenleft}K\ {\isacharminus}\ {\isacharparenleft}pcp{\isacharunderscore}lim\ C\ S{\isacharparenright}{\isacharparenright}{\isachardoublequoteclose}\isanewline
\ \ \ \ \isacommand{by}\isamarkupfalse%
\ {\isacharparenleft}simp\ only{\isacharcolon}\ psubset{\isacharunderscore}imp{\isacharunderscore}ex{\isacharunderscore}mem{\isacharparenright}\ \isanewline
\ \ \isacommand{then}\isamarkupfalse%
\ \isacommand{have}\isamarkupfalse%
\ E{\isacharcolon}{\isachardoublequoteopen}{\isasymexists}F{\isachardot}\ F\ {\isasymin}\ K\ {\isasymand}\ F\ {\isasymnotin}\ {\isacharparenleft}pcp{\isacharunderscore}lim\ C\ S{\isacharparenright}{\isachardoublequoteclose}\isanewline
\ \ \ \ \isacommand{by}\isamarkupfalse%
\ {\isacharparenleft}simp\ only{\isacharcolon}\ Diff{\isacharunderscore}iff{\isacharparenright}\isanewline
\ \ \isacommand{obtain}\isamarkupfalse%
\ F\ \isakeyword{where}\ F{\isacharcolon}{\isachardoublequoteopen}F\ {\isasymin}\ K\ {\isasymand}\ F\ {\isasymnotin}\ pcp{\isacharunderscore}lim\ C\ S{\isachardoublequoteclose}\ \isanewline
\ \ \ \ \isacommand{using}\isamarkupfalse%
\ E\ \isacommand{by}\isamarkupfalse%
\ {\isacharparenleft}rule\ exE{\isacharparenright}\isanewline
\ \ \isacommand{have}\isamarkupfalse%
\ {\isachardoublequoteopen}F\ {\isasymin}\ K{\isachardoublequoteclose}\ \isanewline
\ \ \ \ \isacommand{using}\isamarkupfalse%
\ F\ \isacommand{by}\isamarkupfalse%
\ {\isacharparenleft}rule\ conjunct{\isadigit{1}}{\isacharparenright}\isanewline
\ \ \isacommand{have}\isamarkupfalse%
\ {\isachardoublequoteopen}F\ {\isasymnotin}\ pcp{\isacharunderscore}lim\ C\ S{\isachardoublequoteclose}\isanewline
\ \ \ \ \isacommand{using}\isamarkupfalse%
\ F\ \isacommand{by}\isamarkupfalse%
\ {\isacharparenleft}rule\ conjunct{\isadigit{2}}{\isacharparenright}\isanewline
\ \ \isacommand{have}\isamarkupfalse%
\ {\isachardoublequoteopen}pcp{\isacharunderscore}seq\ C\ S\ {\isacharparenleft}Suc\ {\isacharparenleft}to{\isacharunderscore}nat\ F{\isacharparenright}{\isacharparenright}\ {\isasymsubseteq}\ pcp{\isacharunderscore}lim\ C\ S{\isachardoublequoteclose}\isanewline
\ \ \ \ \isacommand{by}\isamarkupfalse%
\ {\isacharparenleft}rule\ pcp{\isacharunderscore}seq{\isacharunderscore}sub{\isacharparenright}\isanewline
\ \ \isacommand{then}\isamarkupfalse%
\ \isacommand{have}\isamarkupfalse%
\ {\isachardoublequoteopen}F\ {\isasymin}\ pcp{\isacharunderscore}seq\ C\ S\ {\isacharparenleft}Suc\ {\isacharparenleft}to{\isacharunderscore}nat\ F{\isacharparenright}{\isacharparenright}\ {\isasymlongrightarrow}\ F\ {\isasymin}\ pcp{\isacharunderscore}lim\ C\ S{\isachardoublequoteclose}\isanewline
\ \ \ \ \isacommand{by}\isamarkupfalse%
\ {\isacharparenleft}rule\ in{\isacharunderscore}mono{\isacharparenright}\isanewline
\ \ \isacommand{then}\isamarkupfalse%
\ \isacommand{have}\isamarkupfalse%
\ {\isadigit{1}}{\isacharcolon}{\isachardoublequoteopen}F\ {\isasymnotin}\ pcp{\isacharunderscore}seq\ C\ S\ {\isacharparenleft}Suc\ {\isacharparenleft}to{\isacharunderscore}nat\ F{\isacharparenright}{\isacharparenright}{\isachardoublequoteclose}\isanewline
\ \ \ \ \isacommand{using}\isamarkupfalse%
\ {\isacartoucheopen}F\ {\isasymnotin}\ pcp{\isacharunderscore}lim\ C\ S{\isacartoucheclose}\ \isacommand{by}\isamarkupfalse%
\ {\isacharparenleft}rule\ mt{\isacharparenright}\isanewline
\ \ \isacommand{have}\isamarkupfalse%
\ {\isadigit{2}}{\isacharcolon}\ {\isachardoublequoteopen}insert\ F\ {\isacharparenleft}pcp{\isacharunderscore}seq\ C\ S\ {\isacharparenleft}to{\isacharunderscore}nat\ F{\isacharparenright}{\isacharparenright}\ {\isasymnotin}\ C{\isachardoublequoteclose}\ \isanewline
\ \ \isacommand{proof}\isamarkupfalse%
\ {\isacharparenleft}rule\ ccontr{\isacharparenright}\isanewline
\ \ \ \ \isacommand{assume}\isamarkupfalse%
\ {\isachardoublequoteopen}{\isasymnot}{\isacharparenleft}insert\ F\ {\isacharparenleft}pcp{\isacharunderscore}seq\ C\ S\ {\isacharparenleft}to{\isacharunderscore}nat\ F{\isacharparenright}{\isacharparenright}\ {\isasymnotin}\ C{\isacharparenright}{\isachardoublequoteclose}\isanewline
\ \ \ \ \isacommand{then}\isamarkupfalse%
\ \isacommand{have}\isamarkupfalse%
\ {\isachardoublequoteopen}insert\ F\ {\isacharparenleft}pcp{\isacharunderscore}seq\ C\ S\ {\isacharparenleft}to{\isacharunderscore}nat\ F{\isacharparenright}{\isacharparenright}\ {\isasymin}\ C{\isachardoublequoteclose}\isanewline
\ \ \ \ \ \ \isacommand{by}\isamarkupfalse%
\ {\isacharparenleft}rule\ notnotD{\isacharparenright}\isanewline
\ \ \ \ \isacommand{then}\isamarkupfalse%
\ \isacommand{have}\isamarkupfalse%
\ C{\isacharcolon}{\isachardoublequoteopen}insert\ {\isacharparenleft}from{\isacharunderscore}nat\ {\isacharparenleft}to{\isacharunderscore}nat\ F{\isacharparenright}{\isacharparenright}\ {\isacharparenleft}pcp{\isacharunderscore}seq\ C\ S\ {\isacharparenleft}to{\isacharunderscore}nat\ F{\isacharparenright}{\isacharparenright}\ {\isasymin}\ C{\isachardoublequoteclose}\isanewline
\ \ \ \ \ \ \isacommand{by}\isamarkupfalse%
\ {\isacharparenleft}simp\ only{\isacharcolon}\ from{\isacharunderscore}nat{\isacharunderscore}to{\isacharunderscore}nat{\isacharparenright}\isanewline
\ \ \ \ \isacommand{have}\isamarkupfalse%
\ {\isachardoublequoteopen}pcp{\isacharunderscore}seq\ C\ S\ {\isacharparenleft}Suc\ {\isacharparenleft}to{\isacharunderscore}nat\ F{\isacharparenright}{\isacharparenright}\ {\isacharequal}\ {\isacharparenleft}let\ Sn\ {\isacharequal}\ pcp{\isacharunderscore}seq\ C\ S\ {\isacharparenleft}to{\isacharunderscore}nat\ F{\isacharparenright}{\isacharsemicolon}\ \isanewline
\ \ \ \ \ \ \ \ \ \ Sn{\isadigit{1}}\ {\isacharequal}\ insert\ {\isacharparenleft}from{\isacharunderscore}nat\ {\isacharparenleft}to{\isacharunderscore}nat\ F{\isacharparenright}{\isacharparenright}\ Sn\ in\ if\ Sn{\isadigit{1}}\ {\isasymin}\ C\ then\ Sn{\isadigit{1}}\ else\ Sn{\isacharparenright}{\isachardoublequoteclose}\ \isanewline
\ \ \ \ \ \ \isacommand{by}\isamarkupfalse%
\ {\isacharparenleft}simp\ only{\isacharcolon}\ pcp{\isacharunderscore}seq{\isachardot}simps{\isacharparenleft}{\isadigit{2}}{\isacharparenright}{\isacharparenright}\isanewline
\ \ \ \ \isacommand{then}\isamarkupfalse%
\ \isacommand{have}\isamarkupfalse%
\ SucDef{\isacharcolon}{\isachardoublequoteopen}pcp{\isacharunderscore}seq\ C\ S\ {\isacharparenleft}Suc\ {\isacharparenleft}to{\isacharunderscore}nat\ F{\isacharparenright}{\isacharparenright}\ {\isacharequal}\ {\isacharparenleft}if\ insert\ {\isacharparenleft}from{\isacharunderscore}nat\ {\isacharparenleft}to{\isacharunderscore}nat\ F{\isacharparenright}{\isacharparenright}\ {\isacharparenleft}pcp{\isacharunderscore}seq\ C\ S\ {\isacharparenleft}to{\isacharunderscore}nat\ F{\isacharparenright}{\isacharparenright}\ {\isasymin}\ C\ \isanewline
\ \ \ \ \ \ \ \ \ \ then\ insert\ {\isacharparenleft}from{\isacharunderscore}nat\ {\isacharparenleft}to{\isacharunderscore}nat\ F{\isacharparenright}{\isacharparenright}\ {\isacharparenleft}pcp{\isacharunderscore}seq\ C\ S\ {\isacharparenleft}to{\isacharunderscore}nat\ F{\isacharparenright}{\isacharparenright}\ else\ pcp{\isacharunderscore}seq\ C\ S\ {\isacharparenleft}to{\isacharunderscore}nat\ F{\isacharparenright}{\isacharparenright}{\isachardoublequoteclose}\ \isanewline
\ \ \ \ \ \ \isacommand{by}\isamarkupfalse%
\ {\isacharparenleft}simp\ only{\isacharcolon}\ Let{\isacharunderscore}def{\isacharparenright}\isanewline
\ \ \ \ \isacommand{then}\isamarkupfalse%
\ \isacommand{have}\isamarkupfalse%
\ {\isachardoublequoteopen}pcp{\isacharunderscore}seq\ C\ S\ {\isacharparenleft}Suc\ {\isacharparenleft}to{\isacharunderscore}nat\ F{\isacharparenright}{\isacharparenright}\ {\isacharequal}\ insert\ {\isacharparenleft}from{\isacharunderscore}nat\ {\isacharparenleft}to{\isacharunderscore}nat\ F{\isacharparenright}{\isacharparenright}\ {\isacharparenleft}pcp{\isacharunderscore}seq\ C\ S\ {\isacharparenleft}to{\isacharunderscore}nat\ F{\isacharparenright}{\isacharparenright}{\isachardoublequoteclose}\ \isanewline
\ \ \ \ \ \ \isacommand{using}\isamarkupfalse%
\ C\ \isacommand{by}\isamarkupfalse%
\ {\isacharparenleft}simp\ only{\isacharcolon}\ if{\isacharunderscore}True{\isacharparenright}\isanewline
\ \ \ \ \isacommand{then}\isamarkupfalse%
\ \isacommand{have}\isamarkupfalse%
\ {\isachardoublequoteopen}pcp{\isacharunderscore}seq\ C\ S\ {\isacharparenleft}Suc\ {\isacharparenleft}to{\isacharunderscore}nat\ F{\isacharparenright}{\isacharparenright}\ {\isacharequal}\ insert\ F\ {\isacharparenleft}pcp{\isacharunderscore}seq\ C\ S\ {\isacharparenleft}to{\isacharunderscore}nat\ F{\isacharparenright}{\isacharparenright}{\isachardoublequoteclose}\isanewline
\ \ \ \ \ \ \isacommand{by}\isamarkupfalse%
\ {\isacharparenleft}simp\ only{\isacharcolon}\ from{\isacharunderscore}nat{\isacharunderscore}to{\isacharunderscore}nat{\isacharparenright}\isanewline
\ \ \ \ \isacommand{then}\isamarkupfalse%
\ \isacommand{have}\isamarkupfalse%
\ {\isachardoublequoteopen}F\ {\isasymin}\ pcp{\isacharunderscore}seq\ C\ S\ {\isacharparenleft}Suc\ {\isacharparenleft}to{\isacharunderscore}nat\ F{\isacharparenright}{\isacharparenright}{\isachardoublequoteclose}\isanewline
\ \ \ \ \ \ \isacommand{by}\isamarkupfalse%
\ {\isacharparenleft}simp\ only{\isacharcolon}\ insertI{\isadigit{1}}{\isacharparenright}\isanewline
\ \ \ \ \isacommand{show}\isamarkupfalse%
\ {\isachardoublequoteopen}False{\isachardoublequoteclose}\isanewline
\ \ \ \ \ \ \isacommand{using}\isamarkupfalse%
\ {\isacartoucheopen}F\ {\isasymnotin}\ pcp{\isacharunderscore}seq\ C\ S\ {\isacharparenleft}Suc\ {\isacharparenleft}to{\isacharunderscore}nat\ F{\isacharparenright}{\isacharparenright}{\isacartoucheclose}\ {\isacartoucheopen}F\ {\isasymin}\ pcp{\isacharunderscore}seq\ C\ S\ {\isacharparenleft}Suc\ {\isacharparenleft}to{\isacharunderscore}nat\ F{\isacharparenright}{\isacharparenright}{\isacartoucheclose}\ \isacommand{by}\isamarkupfalse%
\ {\isacharparenleft}rule\ notE{\isacharparenright}\isanewline
\ \ \isacommand{qed}\isamarkupfalse%
\isanewline
\ \ \isacommand{have}\isamarkupfalse%
\ {\isachardoublequoteopen}pcp{\isacharunderscore}seq\ C\ S\ {\isacharparenleft}to{\isacharunderscore}nat\ F{\isacharparenright}\ {\isasymsubseteq}\ pcp{\isacharunderscore}lim\ C\ S{\isachardoublequoteclose}\isanewline
\ \ \ \ \isacommand{by}\isamarkupfalse%
\ {\isacharparenleft}rule\ pcp{\isacharunderscore}seq{\isacharunderscore}sub{\isacharparenright}\isanewline
\ \ \isacommand{then}\isamarkupfalse%
\ \isacommand{have}\isamarkupfalse%
\ {\isachardoublequoteopen}pcp{\isacharunderscore}seq\ C\ S\ {\isacharparenleft}to{\isacharunderscore}nat\ F{\isacharparenright}\ {\isasymsubseteq}\ K{\isachardoublequoteclose}\isanewline
\ \ \ \ \isacommand{using}\isamarkupfalse%
\ assms{\isacharparenleft}{\isadigit{4}}{\isacharparenright}\ \isacommand{by}\isamarkupfalse%
\ {\isacharparenleft}rule\ subset{\isacharunderscore}trans{\isacharparenright}\isanewline
\ \ \isacommand{then}\isamarkupfalse%
\ \isacommand{have}\isamarkupfalse%
\ {\isachardoublequoteopen}insert\ F\ {\isacharparenleft}pcp{\isacharunderscore}seq\ C\ S\ {\isacharparenleft}to{\isacharunderscore}nat\ F{\isacharparenright}{\isacharparenright}\ {\isasymsubseteq}\ K{\isachardoublequoteclose}\ \isanewline
\ \ \ \ \isacommand{using}\isamarkupfalse%
\ {\isacartoucheopen}F\ {\isasymin}\ K{\isacartoucheclose}\ \isacommand{by}\isamarkupfalse%
\ {\isacharparenleft}simp\ only{\isacharcolon}\ insert{\isacharunderscore}subset{\isacharparenright}\isanewline
\ \ \isacommand{have}\isamarkupfalse%
\ {\isachardoublequoteopen}{\isasymforall}S\ {\isasymin}\ C{\isachardot}\ {\isasymforall}s{\isasymsubseteq}S{\isachardot}\ s\ {\isasymin}\ C{\isachardoublequoteclose}\isanewline
\ \ \ \ \isacommand{using}\isamarkupfalse%
\ assms{\isacharparenleft}{\isadigit{2}}{\isacharparenright}\ \isacommand{by}\isamarkupfalse%
\ {\isacharparenleft}simp\ only{\isacharcolon}\ subset{\isacharunderscore}closed{\isacharunderscore}def{\isacharparenright}\isanewline
\ \ \isacommand{then}\isamarkupfalse%
\ \isacommand{have}\isamarkupfalse%
\ {\isachardoublequoteopen}{\isasymforall}s\ {\isasymsubseteq}\ K{\isachardot}\ s\ {\isasymin}\ C{\isachardoublequoteclose}\isanewline
\ \ \ \ \isacommand{using}\isamarkupfalse%
\ assms{\isacharparenleft}{\isadigit{3}}{\isacharparenright}\ \isacommand{by}\isamarkupfalse%
\ {\isacharparenleft}rule\ bspec{\isacharparenright}\isanewline
\ \ \isacommand{then}\isamarkupfalse%
\ \isacommand{have}\isamarkupfalse%
\ {\isadigit{3}}{\isacharcolon}{\isachardoublequoteopen}insert\ F\ {\isacharparenleft}pcp{\isacharunderscore}seq\ C\ S\ {\isacharparenleft}to{\isacharunderscore}nat\ F{\isacharparenright}{\isacharparenright}\ {\isasymin}\ C{\isachardoublequoteclose}\ \isanewline
\ \ \ \ \isacommand{using}\isamarkupfalse%
\ {\isacartoucheopen}insert\ F\ {\isacharparenleft}pcp{\isacharunderscore}seq\ C\ S\ {\isacharparenleft}to{\isacharunderscore}nat\ F{\isacharparenright}{\isacharparenright}\ {\isasymsubseteq}\ K{\isacartoucheclose}\ \isacommand{by}\isamarkupfalse%
\ {\isacharparenleft}rule\ sspec{\isacharparenright}\isanewline
\ \ \isacommand{show}\isamarkupfalse%
\ {\isachardoublequoteopen}False{\isachardoublequoteclose}\isanewline
\ \ \ \ \isacommand{using}\isamarkupfalse%
\ {\isadigit{2}}\ {\isadigit{3}}\ \isacommand{by}\isamarkupfalse%
\ {\isacharparenleft}rule\ notE{\isacharparenright}\isanewline
\isacommand{qed}\isamarkupfalse%
%
\endisatagproof
{\isafoldproof}%
%
\isadelimproof
%
\endisadelimproof
%
\begin{isamarkuptext}%
Análogamente a resultados anteriores, veamos su prueba automática.%
\end{isamarkuptext}\isamarkuptrue%
\isacommand{lemma}\isamarkupfalse%
\ cl{\isacharunderscore}max{\isacharcolon}\isanewline
\ \ \isakeyword{assumes}\ c{\isacharcolon}\ {\isachardoublequoteopen}pcp\ C{\isachardoublequoteclose}\isanewline
\ \ \isakeyword{assumes}\ sc{\isacharcolon}\ {\isachardoublequoteopen}subset{\isacharunderscore}closed\ C{\isachardoublequoteclose}\isanewline
\ \ \isakeyword{assumes}\ el{\isacharcolon}\ {\isachardoublequoteopen}K\ {\isasymin}\ C{\isachardoublequoteclose}\isanewline
\ \ \isakeyword{assumes}\ su{\isacharcolon}\ {\isachardoublequoteopen}pcp{\isacharunderscore}lim\ C\ S\ {\isasymsubseteq}\ K{\isachardoublequoteclose}\isanewline
\ \ \isakeyword{shows}\ {\isachardoublequoteopen}pcp{\isacharunderscore}lim\ C\ S\ {\isacharequal}\ K{\isachardoublequoteclose}\ {\isacharparenleft}\isakeyword{is}\ {\isacharquery}e{\isacharparenright}\isanewline
%
\isadelimproof
%
\endisadelimproof
%
\isatagproof
\isacommand{proof}\isamarkupfalse%
\ {\isacharparenleft}rule\ ccontr{\isacharparenright}\isanewline
\ \ \isacommand{assume}\isamarkupfalse%
\ {\isacartoucheopen}{\isasymnot}{\isacharquery}e{\isacartoucheclose}\isanewline
\ \ \isacommand{with}\isamarkupfalse%
\ su\ \isacommand{have}\isamarkupfalse%
\ {\isachardoublequoteopen}pcp{\isacharunderscore}lim\ C\ S\ {\isasymsubset}\ K{\isachardoublequoteclose}\ \isacommand{by}\isamarkupfalse%
\ simp\isanewline
\ \ \isacommand{then}\isamarkupfalse%
\ \isacommand{obtain}\isamarkupfalse%
\ F\ \isakeyword{where}\ e{\isacharcolon}\ {\isachardoublequoteopen}F\ {\isasymin}\ K{\isachardoublequoteclose}\ \isakeyword{and}\ ne{\isacharcolon}\ {\isachardoublequoteopen}F\ {\isasymnotin}\ pcp{\isacharunderscore}lim\ C\ S{\isachardoublequoteclose}\ \isacommand{by}\isamarkupfalse%
\ blast\isanewline
\ \ \isacommand{from}\isamarkupfalse%
\ ne\ \isacommand{have}\isamarkupfalse%
\ {\isachardoublequoteopen}F\ {\isasymnotin}\ pcp{\isacharunderscore}seq\ C\ S\ {\isacharparenleft}Suc\ {\isacharparenleft}to{\isacharunderscore}nat\ F{\isacharparenright}{\isacharparenright}{\isachardoublequoteclose}\ \isacommand{using}\isamarkupfalse%
\ pcp{\isacharunderscore}seq{\isacharunderscore}sub\ \isacommand{by}\isamarkupfalse%
\ fast\isanewline
\ \ \isacommand{hence}\isamarkupfalse%
\ {\isadigit{1}}{\isacharcolon}\ {\isachardoublequoteopen}insert\ F\ {\isacharparenleft}pcp{\isacharunderscore}seq\ C\ S\ {\isacharparenleft}to{\isacharunderscore}nat\ F{\isacharparenright}{\isacharparenright}\ {\isasymnotin}\ C{\isachardoublequoteclose}\ \isacommand{by}\isamarkupfalse%
\ {\isacharparenleft}simp\ add{\isacharcolon}\ Let{\isacharunderscore}def\ split{\isacharcolon}\ if{\isacharunderscore}splits{\isacharparenright}\isanewline
\ \ \isacommand{have}\isamarkupfalse%
\ {\isachardoublequoteopen}insert\ F\ {\isacharparenleft}pcp{\isacharunderscore}seq\ C\ S\ {\isacharparenleft}to{\isacharunderscore}nat\ F{\isacharparenright}{\isacharparenright}\ {\isasymsubseteq}\ K{\isachardoublequoteclose}\ \isacommand{using}\isamarkupfalse%
\ pcp{\isacharunderscore}seq{\isacharunderscore}sub\ e\ su\ \isacommand{by}\isamarkupfalse%
\ blast\isanewline
\ \ \isacommand{hence}\isamarkupfalse%
\ {\isachardoublequoteopen}insert\ F\ {\isacharparenleft}pcp{\isacharunderscore}seq\ C\ S\ {\isacharparenleft}to{\isacharunderscore}nat\ F{\isacharparenright}{\isacharparenright}\ {\isasymin}\ C{\isachardoublequoteclose}\ \isacommand{using}\isamarkupfalse%
\ sc\ \isanewline
\ \ \ \ \isacommand{unfolding}\isamarkupfalse%
\ subset{\isacharunderscore}closed{\isacharunderscore}def\ \isacommand{using}\isamarkupfalse%
\ el\ \isacommand{by}\isamarkupfalse%
\ blast\isanewline
\ \ \isacommand{with}\isamarkupfalse%
\ {\isadigit{1}}\ \isacommand{show}\isamarkupfalse%
\ False\ \isacommand{{\isachardot}{\isachardot}}\isamarkupfalse%
\isanewline
\isacommand{qed}\isamarkupfalse%
%
\endisatagproof
{\isafoldproof}%
%
\isadelimproof
%
\endisadelimproof
%
\begin{isamarkuptext}%
A continuación mostremos un resultado sobre el límite de la sucesión de \isa{{\isadigit{4}}{\isachardot}{\isadigit{1}}{\isachardot}{\isadigit{1}}} que es 
  consecuencia de que dicho límite sea un elemento maximal de la colección que lo define si esta
  verifica la propiedad de consistencia proposicional y es cerrada bajo subconjuntos.
  
  \begin{corolario}
    Sea \isa{C} una colección de conjuntos que verifica la propiedad de consistencia proposicional y
    es cerrada bajo subconjuntos, \isa{S} un conjunto, \isa{{\isacharbraceleft}S\isactrlsub n{\isacharbraceright}} la sucesión de conjuntos de \isa{C} a partir 
    de \isa{S} según la definición \isa{{\isadigit{4}}{\isachardot}{\isadigit{1}}{\isachardot}{\isadigit{1}}} y \isa{F} una fórmula proposicional. Entonces, si\\
    $\{F\} \cup \bigcup_{n = 0}^{\infty} S_{n} \in C$, se verifica que 
    $F \in \bigcup_{n = 0}^{\infty} S_{n}$. 
  \end{corolario}

  \begin{demostracion}
    Como \isa{C} es una colección que verifica la propiedad de consistencia proposicional y es cerrada 
    bajo subconjuntos, se tiene que el límite $\bigcup_{n = 0}^{\infty} S_{n}$ es maximal en \isa{C}. Por 
    lo tanto, si suponemos que $\{F\} \cup \bigcup_{n = 0}^{\infty} S_{n} \in C$, como el límite 
    está contenido en dicho conjunto, se cumple que 
    $\{F\} \cup \bigcup_{n = 0}^{\infty} S_{n} = \bigcup_{n = 0}^{\infty} S_{n}$, luego \isa{F} 
    pertenece al límite, como queríamos demostrar.
  \end{demostracion}

  Veamos su formalización y prueba detallada en Isabelle/HOL.%
\end{isamarkuptext}\isamarkuptrue%
\isacommand{lemma}\isamarkupfalse%
\isanewline
\ \ \isakeyword{assumes}\ {\isachardoublequoteopen}pcp\ C{\isachardoublequoteclose}\isanewline
\ \ \isakeyword{assumes}\ {\isachardoublequoteopen}subset{\isacharunderscore}closed\ C{\isachardoublequoteclose}\isanewline
\ \ \isakeyword{shows}\ {\isachardoublequoteopen}insert\ F\ {\isacharparenleft}pcp{\isacharunderscore}lim\ C\ S{\isacharparenright}\ {\isasymin}\ C\ {\isasymLongrightarrow}\ F\ {\isasymin}\ pcp{\isacharunderscore}lim\ C\ S{\isachardoublequoteclose}\isanewline
%
\isadelimproof
%
\endisadelimproof
%
\isatagproof
\isacommand{proof}\isamarkupfalse%
\ {\isacharminus}\isanewline
\ \ \isacommand{assume}\isamarkupfalse%
\ {\isachardoublequoteopen}insert\ F\ {\isacharparenleft}pcp{\isacharunderscore}lim\ C\ S{\isacharparenright}\ {\isasymin}\ C{\isachardoublequoteclose}\isanewline
\ \ \isacommand{have}\isamarkupfalse%
\ {\isachardoublequoteopen}pcp{\isacharunderscore}lim\ C\ S\ {\isasymsubseteq}\ insert\ F\ {\isacharparenleft}pcp{\isacharunderscore}lim\ C\ S{\isacharparenright}{\isachardoublequoteclose}\isanewline
\ \ \ \ \isacommand{by}\isamarkupfalse%
\ {\isacharparenleft}rule\ subset{\isacharunderscore}insertI{\isacharparenright}\ \isanewline
\ \ \isacommand{have}\isamarkupfalse%
\ {\isachardoublequoteopen}pcp{\isacharunderscore}lim\ C\ S\ {\isacharequal}\ insert\ F\ {\isacharparenleft}pcp{\isacharunderscore}lim\ C\ S{\isacharparenright}{\isachardoublequoteclose}\isanewline
\ \ \ \ \isacommand{using}\isamarkupfalse%
\ assms{\isacharparenleft}{\isadigit{1}}{\isacharparenright}\ assms{\isacharparenleft}{\isadigit{2}}{\isacharparenright}\ {\isacartoucheopen}insert\ F\ {\isacharparenleft}pcp{\isacharunderscore}lim\ C\ S{\isacharparenright}\ {\isasymin}\ C{\isacartoucheclose}\ {\isacartoucheopen}pcp{\isacharunderscore}lim\ C\ S\ {\isasymsubseteq}\ insert\ F\ {\isacharparenleft}pcp{\isacharunderscore}lim\ C\ S{\isacharparenright}{\isacartoucheclose}\ \isacommand{by}\isamarkupfalse%
\ {\isacharparenleft}rule\ cl{\isacharunderscore}max{\isacharparenright}\isanewline
\ \ \isacommand{then}\isamarkupfalse%
\ \isacommand{have}\isamarkupfalse%
\ {\isachardoublequoteopen}insert\ F\ {\isacharparenleft}pcp{\isacharunderscore}lim\ C\ S{\isacharparenright}\ {\isasymsubseteq}\ pcp{\isacharunderscore}lim\ C\ S{\isachardoublequoteclose}\isanewline
\ \ \ \ \isacommand{by}\isamarkupfalse%
\ {\isacharparenleft}rule\ equalityD{\isadigit{2}}{\isacharparenright}\isanewline
\ \ \isacommand{then}\isamarkupfalse%
\ \isacommand{have}\isamarkupfalse%
\ {\isachardoublequoteopen}F\ {\isasymin}\ pcp{\isacharunderscore}lim\ C\ S\ {\isasymand}\ pcp{\isacharunderscore}lim\ C\ S\ {\isasymsubseteq}\ pcp{\isacharunderscore}lim\ C\ S{\isachardoublequoteclose}\isanewline
\ \ \ \ \isacommand{by}\isamarkupfalse%
\ {\isacharparenleft}simp\ only{\isacharcolon}\ insert{\isacharunderscore}subset{\isacharparenright}\isanewline
\ \ \isacommand{thus}\isamarkupfalse%
\ {\isachardoublequoteopen}F\ {\isasymin}\ pcp{\isacharunderscore}lim\ C\ S{\isachardoublequoteclose}\isanewline
\ \ \ \ \isacommand{by}\isamarkupfalse%
\ {\isacharparenleft}rule\ conjunct{\isadigit{1}}{\isacharparenright}\isanewline
\isacommand{qed}\isamarkupfalse%
%
\endisatagproof
{\isafoldproof}%
%
\isadelimproof
%
\endisadelimproof
%
\begin{isamarkuptext}%
Mostremos su demostración automática.%
\end{isamarkuptext}\isamarkuptrue%
\isacommand{lemma}\isamarkupfalse%
\ cl{\isacharunderscore}max{\isacharprime}{\isacharcolon}\isanewline
\ \ \isakeyword{assumes}\ c{\isacharcolon}\ {\isachardoublequoteopen}pcp\ C{\isachardoublequoteclose}\isanewline
\ \ \isakeyword{assumes}\ sc{\isacharcolon}\ {\isachardoublequoteopen}subset{\isacharunderscore}closed\ C{\isachardoublequoteclose}\isanewline
\ \ \isakeyword{shows}\ {\isachardoublequoteopen}insert\ F\ {\isacharparenleft}pcp{\isacharunderscore}lim\ C\ S{\isacharparenright}\ {\isasymin}\ C\ {\isasymLongrightarrow}\ F\ {\isasymin}\ pcp{\isacharunderscore}lim\ C\ S{\isachardoublequoteclose}\isanewline
%
\isadelimproof
\ \ %
\endisadelimproof
%
\isatagproof
\isacommand{using}\isamarkupfalse%
\ cl{\isacharunderscore}max{\isacharbrackleft}OF\ assms{\isacharbrackright}\ \isacommand{by}\isamarkupfalse%
\ blast{\isacharplus}%
\endisatagproof
{\isafoldproof}%
%
\isadelimproof
%
\endisadelimproof
%
\begin{isamarkuptext}%
El siguiente resultado prueba que el límite de la sucesión definida en \isa{{\isadigit{4}}{\isachardot}{\isadigit{1}}{\isachardot}{\isadigit{1}}} es un conjunto
  de Hintikka si la colección que lo define verifica la propiedad de consistencia proposicional, es
  es cerrada bajo subconjuntos y es de carácter finito. Como consecuencia del \isa{teorema\ de\ Hintikka},
  se trata en particular de un conjunto satisfacible. 

  \begin{lema}
    Sea \isa{C} una colección de conjuntos que verifica la propiedad de consistencia proposicional, es
    es cerrada bajo subconjuntos y es de carácter finito. Sea \isa{S\ {\isasymin}\ C} y \isa{{\isacharbraceleft}S\isactrlsub n{\isacharbraceright}} la sucesión de
    conjuntos de \isa{C} a partir de \isa{S} según la definición \isa{{\isadigit{4}}{\isachardot}{\isadigit{1}}{\isachardot}{\isadigit{1}}}. Entonces, el límite de la sucesión
    \isa{{\isacharbraceleft}S\isactrlsub n{\isacharbraceright}} es un conjunto de Hintikka.
  \end{lema}

  \begin{demostracion}
    Para facilitar la lectura, vamos a notar por \isa{L\isactrlsub S\isactrlsub C} al límite de la sucesión \isa{{\isacharbraceleft}S\isactrlsub n{\isacharbraceright}} descrita 
    en el enunciado.

    Por resultados anteriores, como \isa{C} verifica la propiedad de consistencia proposicional, es
    es cerrada bajo subconjuntos y es de carácter finito, se tiene que \isa{L\isactrlsub S\isactrlsub C\ {\isasymin}\ C}. En particular, por 
    verificar la propiedad de consistencia proposicional, por el lema de\\ caracterización de dicha
    propiedad mediante notación uniforme, se cumplen las siguientes condiciones para \isa{L\isactrlsub S\isactrlsub C}:

    \begin{itemize}
      \item \isa{{\isasymbottom}\ {\isasymnotin}\ L\isactrlsub S\isactrlsub C}.
      \item Dada \isa{p} una fórmula atómica cualquiera, no se tiene 
      simultáneamente que\\ \isa{p\ {\isasymin}\ L\isactrlsub S\isactrlsub C} y \isa{{\isasymnot}\ p\ {\isasymin}\ L\isactrlsub S\isactrlsub C}.
      \item Para toda fórmula de tipo \isa{{\isasymalpha}} con componentes \isa{{\isasymalpha}\isactrlsub {\isadigit{1}}} y \isa{{\isasymalpha}\isactrlsub {\isadigit{2}}} tal que \isa{{\isasymalpha}}
      pertenece a \isa{L\isactrlsub S\isactrlsub C}, se tiene que \isa{{\isacharbraceleft}{\isasymalpha}\isactrlsub {\isadigit{1}}{\isacharcomma}{\isasymalpha}\isactrlsub {\isadigit{2}}{\isacharbraceright}\ {\isasymunion}\ L\isactrlsub S\isactrlsub C} pertenece a \isa{C}.
      \item Para toda fórmula de tipo \isa{{\isasymbeta}} con componentes \isa{{\isasymbeta}\isactrlsub {\isadigit{1}}} y \isa{{\isasymbeta}\isactrlsub {\isadigit{2}}} tal que \isa{{\isasymbeta}}
      pertenece a \isa{L\isactrlsub S\isactrlsub C}, se tiene que o bien \isa{{\isacharbraceleft}{\isasymbeta}\isactrlsub {\isadigit{1}}{\isacharbraceright}\ {\isasymunion}\ L\isactrlsub S\isactrlsub C} pertenece a \isa{C} o 
      bien \isa{{\isacharbraceleft}{\isasymbeta}\isactrlsub {\isadigit{2}}{\isacharbraceright}\ {\isasymunion}\ L\isactrlsub S\isactrlsub C} pertenece a \isa{C}.
    \end{itemize}

    Veamos que \isa{L\isactrlsub S\isactrlsub C} es un conjunto de Hintikka probando que cumple las condiciones del
    lema de caracterización de los conjuntos de Hintikka mediante notación uniforme, es decir,
    probaremos que \isa{L\isactrlsub S\isactrlsub C} verifica:

    \begin{itemize}
      \item \isa{{\isasymbottom}\ {\isasymnotin}\ L\isactrlsub S\isactrlsub C}.
      \item Dada \isa{p} una fórmula atómica cualquiera, no se tiene 
      simultáneamente que\\ \isa{p\ {\isasymin}\ L\isactrlsub S\isactrlsub C} y \isa{{\isasymnot}\ p\ {\isasymin}\ L\isactrlsub S\isactrlsub C}.
      \item Para toda fórmula de tipo \isa{{\isasymalpha}} con componentes \isa{{\isasymalpha}\isactrlsub {\isadigit{1}}} y \isa{{\isasymalpha}\isactrlsub {\isadigit{2}}} se verifica 
      que si la fórmula pertenece a \isa{L\isactrlsub S\isactrlsub C}, entonces \isa{{\isasymalpha}\isactrlsub {\isadigit{1}}} y \isa{{\isasymalpha}\isactrlsub {\isadigit{2}}} también.
      \item Para toda fórmula de tipo \isa{{\isasymbeta}} con componentes \isa{{\isasymbeta}\isactrlsub {\isadigit{1}}} y \isa{{\isasymbeta}\isactrlsub {\isadigit{2}}} se verifica 
      que si la fórmula pertenece a \isa{L\isactrlsub S\isactrlsub C}, entonces o bien \isa{{\isasymbeta}\isactrlsub {\isadigit{1}}} pertenece
      a \isa{L\isactrlsub S\isactrlsub C} o bien \isa{{\isasymbeta}\isactrlsub {\isadigit{2}}} pertenece a \isa{L\isactrlsub S\isactrlsub C}.
    \end{itemize} 

    Observemos que las dos primeras condiciones coinciden con las obtenidas anteriormente para \isa{L\isactrlsub S\isactrlsub C} 
    por el lema de caracterización de la propiedad de consistencia proposicional mediante notación
    uniforme. Veamos que, en efecto, se cumplen el resto de condiciones.

    En primer lugar, probemos que para una fórmula \isa{F} de tipo \isa{{\isasymalpha}} y componentes \isa{{\isasymalpha}\isactrlsub {\isadigit{1}}} y \isa{{\isasymalpha}\isactrlsub {\isadigit{2}}} tal que 
    \isa{F\ {\isasymin}\ L\isactrlsub S\isactrlsub C} se verifica que tanto \isa{{\isasymalpha}\isactrlsub {\isadigit{1}}} como \isa{{\isasymalpha}\isactrlsub {\isadigit{2}}} pertenecen a \isa{L\isactrlsub S\isactrlsub C}. Por la tercera condición 
    obtenida anteriormente para \isa{L\isactrlsub S\isactrlsub C} por el lema de caracterización de la propiedad de consistencia 
    proposicional mediante notación uniforme, se cumple que\\ \isa{{\isacharbraceleft}{\isasymalpha}\isactrlsub {\isadigit{1}}{\isacharcomma}{\isasymalpha}\isactrlsub {\isadigit{2}}{\isacharbraceright}\ {\isasymunion}\ L\isactrlsub S\isactrlsub C\ {\isasymin}\ C}. Observemos que
    se verifica \isa{L\isactrlsub S\isactrlsub C\ {\isasymsubseteq}\ {\isacharbraceleft}{\isasymalpha}\isactrlsub {\isadigit{1}}{\isacharcomma}{\isasymalpha}\isactrlsub {\isadigit{2}}{\isacharbraceright}\ {\isasymunion}\ L\isactrlsub S\isactrlsub C}. De este modo, como \isa{C} es una colección con la propiedad de 
    consistencia proposicional y cerrada bajo subconjuntos, por el lema \isa{{\isadigit{4}}{\isachardot}{\isadigit{2}}{\isachardot}{\isadigit{2}}} se tiene que 
    los conjuntos \isa{L\isactrlsub S\isactrlsub C} y \isa{{\isacharbraceleft}{\isasymalpha}\isactrlsub {\isadigit{1}}{\isacharcomma}{\isasymalpha}\isactrlsub {\isadigit{2}}{\isacharbraceright}\ {\isasymunion}\ L\isactrlsub S\isactrlsub C} coinciden. Por tanto, es claro que \isa{{\isasymalpha}\isactrlsub {\isadigit{1}}\ {\isasymin}\ L\isactrlsub S\isactrlsub C} y \isa{{\isasymalpha}\isactrlsub {\isadigit{2}}\ {\isasymin}\ L\isactrlsub S\isactrlsub C}, 
    como queríamos demostrar.

    Por último, demostremos que para una fórmula \isa{F} de tipo \isa{{\isasymbeta}} y componentes \isa{{\isasymbeta}\isactrlsub {\isadigit{1}}} y \isa{{\isasymbeta}\isactrlsub {\isadigit{2}}} tal que
    \isa{F\ {\isasymin}\ L\isactrlsub S\isactrlsub C} se verifica que o bien \isa{{\isasymbeta}\isactrlsub {\isadigit{1}}\ {\isasymin}\ L\isactrlsub S\isactrlsub C} o bien \isa{{\isasymbeta}\isactrlsub {\isadigit{2}}\ {\isasymin}\ L\isactrlsub S\isactrlsub C}. Por la cuarta condición obtenida 
    anteriormente para \isa{L\isactrlsub S\isactrlsub C} por el lema de caracterización de la propiedad de consistencia 
    proposicional mediante notación uniforme, se cumple que o bien\\ \isa{{\isacharbraceleft}{\isasymbeta}\isactrlsub {\isadigit{1}}{\isacharbraceright}\ {\isasymunion}\ L\isactrlsub S\isactrlsub C\ {\isasymin}\ C} o bien 
    \isa{{\isacharbraceleft}{\isasymbeta}\isactrlsub {\isadigit{2}}{\isacharbraceright}\ {\isasymunion}\ L\isactrlsub S\isactrlsub C\ {\isasymin}\ C}. De este modo, si suponemos que \isa{{\isacharbraceleft}{\isasymbeta}\isactrlsub {\isadigit{1}}{\isacharbraceright}\ {\isasymunion}\ L\isactrlsub S\isactrlsub C\ {\isasymin}\ C}, como \isa{C} tiene la propiedad de 
    consistencia proposicional y es cerrada bajo subconjuntos, por el corolario \isa{{\isadigit{4}}{\isachardot}{\isadigit{2}}{\isachardot}{\isadigit{3}}} se tiene 
    que \isa{{\isasymbeta}\isactrlsub {\isadigit{1}}\ {\isasymin}\ L\isactrlsub S\isactrlsub C}. Por tanto, se cumple que o bien \isa{{\isasymbeta}\isactrlsub {\isadigit{1}}\ {\isasymin}\ L\isactrlsub S\isactrlsub C} o bien \isa{{\isasymbeta}\isactrlsub {\isadigit{2}}\ {\isasymin}\ L\isactrlsub S\isactrlsub C}. Si suponemos que 
    \isa{{\isacharbraceleft}{\isasymbeta}\isactrlsub {\isadigit{2}}{\isacharbraceright}\ {\isasymunion}\ L\isactrlsub S\isactrlsub C\ {\isasymin}\ C}, se observa fácilmente que llegamos a la misma conclusión de manera análoga. 
    Por lo tanto, queda probado el resultado.
  \end{demostracion}

  Veamos su formalización y prueba detallada en Isabelle.%
\end{isamarkuptext}\isamarkuptrue%
\isacommand{lemma}\isamarkupfalse%
\isanewline
\ \ \isakeyword{assumes}\ {\isachardoublequoteopen}pcp\ C{\isachardoublequoteclose}\isanewline
\ \ \isakeyword{assumes}\ {\isachardoublequoteopen}subset{\isacharunderscore}closed\ C{\isachardoublequoteclose}\isanewline
\ \ \isakeyword{assumes}\ {\isachardoublequoteopen}finite{\isacharunderscore}character\ C{\isachardoublequoteclose}\isanewline
\ \ \isakeyword{assumes}\ {\isachardoublequoteopen}S\ {\isasymin}\ C{\isachardoublequoteclose}\isanewline
\ \ \isakeyword{shows}\ {\isachardoublequoteopen}Hintikka\ {\isacharparenleft}pcp{\isacharunderscore}lim\ C\ S{\isacharparenright}{\isachardoublequoteclose}\isanewline
%
\isadelimproof
%
\endisadelimproof
%
\isatagproof
\isacommand{proof}\isamarkupfalse%
\ {\isacharparenleft}rule\ Hintikka{\isacharunderscore}alt{\isadigit{2}}{\isacharparenright}\isanewline
\ \ \isacommand{let}\isamarkupfalse%
\ {\isacharquery}cl\ {\isacharequal}\ {\isachardoublequoteopen}pcp{\isacharunderscore}lim\ C\ S{\isachardoublequoteclose}\isanewline
\ \ \isacommand{have}\isamarkupfalse%
\ {\isachardoublequoteopen}{\isacharquery}cl\ {\isasymin}\ C{\isachardoublequoteclose}\isanewline
\ \ \ \ \isacommand{using}\isamarkupfalse%
\ assms{\isacharparenleft}{\isadigit{1}}{\isacharparenright}\ assms{\isacharparenleft}{\isadigit{4}}{\isacharparenright}\ assms{\isacharparenleft}{\isadigit{2}}{\isacharparenright}\ assms{\isacharparenleft}{\isadigit{3}}{\isacharparenright}\ \isacommand{by}\isamarkupfalse%
\ {\isacharparenleft}rule\ pcp{\isacharunderscore}lim{\isacharunderscore}in{\isacharparenright}\isanewline
\ \ \isacommand{have}\isamarkupfalse%
\ {\isachardoublequoteopen}{\isacharparenleft}{\isasymforall}S\ {\isasymin}\ C{\isachardot}\isanewline
\ \ {\isasymbottom}\ {\isasymnotin}\ S\isanewline
{\isasymand}\ {\isacharparenleft}{\isasymforall}k{\isachardot}\ Atom\ k\ {\isasymin}\ S\ {\isasymlongrightarrow}\ \isactrlbold {\isasymnot}\ {\isacharparenleft}Atom\ k{\isacharparenright}\ {\isasymin}\ S\ {\isasymlongrightarrow}\ False{\isacharparenright}\isanewline
{\isasymand}\ {\isacharparenleft}{\isasymforall}F\ G\ H{\isachardot}\ Con\ F\ G\ H\ {\isasymlongrightarrow}\ F\ {\isasymin}\ S\ {\isasymlongrightarrow}\ {\isacharbraceleft}G{\isacharcomma}H{\isacharbraceright}\ {\isasymunion}\ S\ {\isasymin}\ C{\isacharparenright}\isanewline
{\isasymand}\ {\isacharparenleft}{\isasymforall}F\ G\ H{\isachardot}\ Dis\ F\ G\ H\ {\isasymlongrightarrow}\ F\ {\isasymin}\ S\ {\isasymlongrightarrow}\ {\isacharbraceleft}G{\isacharbraceright}\ {\isasymunion}\ S\ {\isasymin}\ C\ {\isasymor}\ {\isacharbraceleft}H{\isacharbraceright}\ {\isasymunion}\ S\ {\isasymin}\ C{\isacharparenright}{\isacharparenright}{\isachardoublequoteclose}\isanewline
\ \ \ \ \isacommand{using}\isamarkupfalse%
\ assms{\isacharparenleft}{\isadigit{1}}{\isacharparenright}\ \isacommand{by}\isamarkupfalse%
\ {\isacharparenleft}rule\ pcp{\isacharunderscore}alt{\isadigit{1}}{\isacharparenright}\isanewline
\ \ \isacommand{then}\isamarkupfalse%
\ \isacommand{have}\isamarkupfalse%
\ d{\isacharcolon}{\isachardoublequoteopen}{\isasymbottom}\ {\isasymnotin}\ {\isacharquery}cl\isanewline
{\isasymand}\ {\isacharparenleft}{\isasymforall}k{\isachardot}\ Atom\ k\ {\isasymin}\ {\isacharquery}cl\ {\isasymlongrightarrow}\ \isactrlbold {\isasymnot}\ {\isacharparenleft}Atom\ k{\isacharparenright}\ {\isasymin}\ {\isacharquery}cl\ {\isasymlongrightarrow}\ False{\isacharparenright}\isanewline
{\isasymand}\ {\isacharparenleft}{\isasymforall}F\ G\ H{\isachardot}\ Con\ F\ G\ H\ {\isasymlongrightarrow}\ F\ {\isasymin}\ {\isacharquery}cl\ {\isasymlongrightarrow}\ {\isacharbraceleft}G{\isacharcomma}H{\isacharbraceright}\ {\isasymunion}\ {\isacharquery}cl\ {\isasymin}\ C{\isacharparenright}\isanewline
{\isasymand}\ {\isacharparenleft}{\isasymforall}F\ G\ H{\isachardot}\ Dis\ F\ G\ H\ {\isasymlongrightarrow}\ F\ {\isasymin}\ {\isacharquery}cl\ {\isasymlongrightarrow}\ {\isacharbraceleft}G{\isacharbraceright}\ {\isasymunion}\ {\isacharquery}cl\ {\isasymin}\ C\ {\isasymor}\ {\isacharbraceleft}H{\isacharbraceright}\ {\isasymunion}\ {\isacharquery}cl\ {\isasymin}\ C{\isacharparenright}{\isachardoublequoteclose}\isanewline
\ \ \ \ \isacommand{using}\isamarkupfalse%
\ {\isacartoucheopen}{\isacharquery}cl\ {\isasymin}\ C{\isacartoucheclose}\ \isacommand{by}\isamarkupfalse%
\ {\isacharparenleft}rule\ bspec{\isacharparenright}\isanewline
\ \ \isacommand{then}\isamarkupfalse%
\ \isacommand{have}\isamarkupfalse%
\ H{\isadigit{1}}{\isacharcolon}{\isachardoublequoteopen}{\isasymbottom}\ {\isasymnotin}\ {\isacharquery}cl{\isachardoublequoteclose}\isanewline
\ \ \ \ \isacommand{by}\isamarkupfalse%
\ {\isacharparenleft}rule\ conjunct{\isadigit{1}}{\isacharparenright}\isanewline
\ \ \isacommand{have}\isamarkupfalse%
\ H{\isadigit{2}}{\isacharcolon}{\isachardoublequoteopen}{\isasymforall}k{\isachardot}\ Atom\ k\ {\isasymin}\ {\isacharquery}cl\ {\isasymlongrightarrow}\ \isactrlbold {\isasymnot}\ {\isacharparenleft}Atom\ k{\isacharparenright}\ {\isasymin}\ {\isacharquery}cl\ {\isasymlongrightarrow}\ False{\isachardoublequoteclose}\isanewline
\ \ \ \ \isacommand{using}\isamarkupfalse%
\ d\ \isacommand{by}\isamarkupfalse%
\ {\isacharparenleft}iprover\ elim{\isacharcolon}\ conjunct{\isadigit{2}}\ conjunct{\isadigit{1}}{\isacharparenright}\isanewline
\ \ \isacommand{have}\isamarkupfalse%
\ Con{\isacharcolon}{\isachardoublequoteopen}{\isasymforall}F\ G\ H{\isachardot}\ Con\ F\ G\ H\ {\isasymlongrightarrow}\ F\ {\isasymin}\ {\isacharquery}cl\ {\isasymlongrightarrow}\ {\isacharbraceleft}G{\isacharcomma}H{\isacharbraceright}\ {\isasymunion}\ {\isacharquery}cl\ {\isasymin}\ C{\isachardoublequoteclose}\isanewline
\ \ \ \ \isacommand{using}\isamarkupfalse%
\ d\ \isacommand{by}\isamarkupfalse%
\ {\isacharparenleft}iprover\ elim{\isacharcolon}\ conjunct{\isadigit{2}}\ conjunct{\isadigit{1}}{\isacharparenright}\isanewline
\ \ \isacommand{have}\isamarkupfalse%
\ H{\isadigit{3}}{\isacharcolon}{\isachardoublequoteopen}{\isasymforall}F\ G\ H{\isachardot}\ Con\ F\ G\ H\ {\isasymlongrightarrow}\ F\ {\isasymin}\ {\isacharquery}cl\ {\isasymlongrightarrow}\ G\ {\isasymin}\ {\isacharquery}cl\ {\isasymand}\ H\ {\isasymin}\ {\isacharquery}cl{\isachardoublequoteclose}\isanewline
\ \ \isacommand{proof}\isamarkupfalse%
\ {\isacharparenleft}rule\ allI{\isacharparenright}{\isacharplus}\isanewline
\ \ \ \ \isacommand{fix}\isamarkupfalse%
\ F\ G\ H\isanewline
\ \ \ \ \isacommand{show}\isamarkupfalse%
\ {\isachardoublequoteopen}Con\ F\ G\ H\ {\isasymlongrightarrow}\ F\ {\isasymin}\ {\isacharquery}cl\ {\isasymlongrightarrow}\ G\ {\isasymin}\ {\isacharquery}cl\ {\isasymand}\ H\ {\isasymin}\ {\isacharquery}cl{\isachardoublequoteclose}\isanewline
\ \ \ \ \isacommand{proof}\isamarkupfalse%
\ {\isacharparenleft}rule\ impI{\isacharparenright}{\isacharplus}\isanewline
\ \ \ \ \ \ \isacommand{assume}\isamarkupfalse%
\ {\isachardoublequoteopen}Con\ F\ G\ H{\isachardoublequoteclose}\isanewline
\ \ \ \ \ \ \isacommand{assume}\isamarkupfalse%
\ {\isachardoublequoteopen}F\ {\isasymin}\ {\isacharquery}cl{\isachardoublequoteclose}\isanewline
\ \ \ \ \ \ \isacommand{have}\isamarkupfalse%
\ {\isachardoublequoteopen}Con\ F\ G\ H\ {\isasymlongrightarrow}\ F\ {\isasymin}\ {\isacharquery}cl\ {\isasymlongrightarrow}\ {\isacharbraceleft}G{\isacharcomma}H{\isacharbraceright}\ {\isasymunion}\ {\isacharquery}cl\ {\isasymin}\ C{\isachardoublequoteclose}\isanewline
\ \ \ \ \ \ \ \ \isacommand{using}\isamarkupfalse%
\ Con\ \isacommand{by}\isamarkupfalse%
\ {\isacharparenleft}iprover\ elim{\isacharcolon}\ allE{\isacharparenright}\isanewline
\ \ \ \ \ \ \isacommand{then}\isamarkupfalse%
\ \isacommand{have}\isamarkupfalse%
\ {\isachardoublequoteopen}F\ {\isasymin}\ {\isacharquery}cl\ {\isasymlongrightarrow}\ {\isacharbraceleft}G{\isacharcomma}H{\isacharbraceright}\ {\isasymunion}\ {\isacharquery}cl\ {\isasymin}\ C{\isachardoublequoteclose}\isanewline
\ \ \ \ \ \ \ \ \isacommand{using}\isamarkupfalse%
\ {\isacartoucheopen}Con\ F\ G\ H{\isacartoucheclose}\ \isacommand{by}\isamarkupfalse%
\ {\isacharparenleft}rule\ mp{\isacharparenright}\isanewline
\ \ \ \ \ \ \isacommand{then}\isamarkupfalse%
\ \isacommand{have}\isamarkupfalse%
\ {\isachardoublequoteopen}{\isacharbraceleft}G{\isacharcomma}H{\isacharbraceright}\ {\isasymunion}\ {\isacharquery}cl\ {\isasymin}\ C{\isachardoublequoteclose}\isanewline
\ \ \ \ \ \ \ \ \isacommand{using}\isamarkupfalse%
\ {\isacartoucheopen}F\ {\isasymin}\ {\isacharquery}cl{\isacartoucheclose}\ \isacommand{by}\isamarkupfalse%
\ {\isacharparenleft}rule\ mp{\isacharparenright}\isanewline
\ \ \ \ \ \ \isacommand{have}\isamarkupfalse%
\ {\isachardoublequoteopen}{\isacharparenleft}insert\ G\ {\isacharparenleft}insert\ H\ {\isacharquery}cl{\isacharparenright}{\isacharparenright}\ {\isacharequal}\ {\isacharbraceleft}G{\isacharcomma}H{\isacharbraceright}\ {\isasymunion}\ {\isacharquery}cl{\isachardoublequoteclose}\isanewline
\ \ \ \ \ \ \ \ \isacommand{by}\isamarkupfalse%
\ {\isacharparenleft}rule\ insertSetElem{\isacharparenright}\isanewline
\ \ \ \ \ \ \isacommand{then}\isamarkupfalse%
\ \isacommand{have}\isamarkupfalse%
\ {\isachardoublequoteopen}{\isacharparenleft}insert\ G\ {\isacharparenleft}insert\ H\ {\isacharquery}cl{\isacharparenright}{\isacharparenright}\ {\isasymin}\ C{\isachardoublequoteclose}\isanewline
\ \ \ \ \ \ \ \ \isacommand{using}\isamarkupfalse%
\ {\isacartoucheopen}{\isacharbraceleft}G{\isacharcomma}H{\isacharbraceright}\ {\isasymunion}\ {\isacharquery}cl\ {\isasymin}\ C{\isacartoucheclose}\ \isacommand{by}\isamarkupfalse%
\ {\isacharparenleft}simp\ only{\isacharcolon}\ {\isacartoucheopen}{\isacharparenleft}insert\ G\ {\isacharparenleft}insert\ H\ {\isacharquery}cl{\isacharparenright}{\isacharparenright}\ {\isacharequal}\ {\isacharbraceleft}G{\isacharcomma}H{\isacharbraceright}\ {\isasymunion}\ {\isacharquery}cl{\isacartoucheclose}{\isacharparenright}\isanewline
\ \ \ \ \ \ \isacommand{have}\isamarkupfalse%
\ {\isachardoublequoteopen}{\isacharquery}cl\ {\isasymsubseteq}\ insert\ H\ {\isacharquery}cl{\isachardoublequoteclose}\isanewline
\ \ \ \ \ \ \ \ \isacommand{by}\isamarkupfalse%
\ {\isacharparenleft}rule\ subset{\isacharunderscore}insertI{\isacharparenright}\isanewline
\ \ \ \ \ \ \isacommand{then}\isamarkupfalse%
\ \isacommand{have}\isamarkupfalse%
\ {\isachardoublequoteopen}{\isacharquery}cl\ {\isasymsubseteq}\ insert\ G\ {\isacharparenleft}insert\ H\ {\isacharquery}cl{\isacharparenright}{\isachardoublequoteclose}\isanewline
\ \ \ \ \ \ \ \ \isacommand{by}\isamarkupfalse%
\ {\isacharparenleft}rule\ subset{\isacharunderscore}insertI{\isadigit{2}}{\isacharparenright}\isanewline
\ \ \ \ \ \ \isacommand{have}\isamarkupfalse%
\ {\isachardoublequoteopen}{\isacharquery}cl\ {\isacharequal}\ insert\ G\ {\isacharparenleft}insert\ H\ {\isacharquery}cl{\isacharparenright}{\isachardoublequoteclose}\ \isanewline
\ \ \ \ \ \ \ \ \isacommand{using}\isamarkupfalse%
\ assms{\isacharparenleft}{\isadigit{1}}{\isacharparenright}\ assms{\isacharparenleft}{\isadigit{2}}{\isacharparenright}\ {\isacartoucheopen}insert\ G\ {\isacharparenleft}insert\ H\ {\isacharquery}cl{\isacharparenright}\ {\isasymin}\ C{\isacartoucheclose}\ {\isacartoucheopen}{\isacharquery}cl\ {\isasymsubseteq}\ insert\ G\ {\isacharparenleft}insert\ H\ {\isacharquery}cl{\isacharparenright}{\isacartoucheclose}\ \isacommand{by}\isamarkupfalse%
\ {\isacharparenleft}rule\ cl{\isacharunderscore}max{\isacharparenright}\isanewline
\ \ \ \ \ \ \isacommand{then}\isamarkupfalse%
\ \isacommand{have}\isamarkupfalse%
\ {\isachardoublequoteopen}insert\ G\ {\isacharparenleft}insert\ H\ {\isacharquery}cl{\isacharparenright}\ {\isasymsubseteq}\ {\isacharquery}cl{\isachardoublequoteclose}\isanewline
\ \ \ \ \ \ \ \ \isacommand{by}\isamarkupfalse%
\ {\isacharparenleft}simp\ only{\isacharcolon}\ equalityD{\isadigit{2}}{\isacharparenright}\isanewline
\ \ \ \ \ \ \isacommand{then}\isamarkupfalse%
\ \isacommand{have}\isamarkupfalse%
\ {\isachardoublequoteopen}G\ {\isasymin}\ {\isacharquery}cl\ {\isasymand}\ insert\ H\ {\isacharquery}cl\ {\isasymsubseteq}\ {\isacharquery}cl{\isachardoublequoteclose}\isanewline
\ \ \ \ \ \ \ \ \isacommand{by}\isamarkupfalse%
\ {\isacharparenleft}simp\ only{\isacharcolon}\ insert{\isacharunderscore}subset{\isacharparenright}\isanewline
\ \ \ \ \ \ \isacommand{then}\isamarkupfalse%
\ \isacommand{have}\isamarkupfalse%
\ {\isachardoublequoteopen}G\ {\isasymin}\ {\isacharquery}cl{\isachardoublequoteclose}\isanewline
\ \ \ \ \ \ \ \ \isacommand{by}\isamarkupfalse%
\ {\isacharparenleft}rule\ conjunct{\isadigit{1}}{\isacharparenright}\isanewline
\ \ \ \ \ \ \isacommand{have}\isamarkupfalse%
\ {\isachardoublequoteopen}insert\ H\ {\isacharquery}cl\ {\isasymsubseteq}\ {\isacharquery}cl{\isachardoublequoteclose}\isanewline
\ \ \ \ \ \ \ \ \isacommand{using}\isamarkupfalse%
\ {\isacartoucheopen}G\ {\isasymin}\ {\isacharquery}cl\ {\isasymand}\ insert\ H\ {\isacharquery}cl\ {\isasymsubseteq}\ {\isacharquery}cl{\isacartoucheclose}\ \isacommand{by}\isamarkupfalse%
\ {\isacharparenleft}rule\ conjunct{\isadigit{2}}{\isacharparenright}\isanewline
\ \ \ \ \ \ \isacommand{then}\isamarkupfalse%
\ \isacommand{have}\isamarkupfalse%
\ {\isachardoublequoteopen}H\ {\isasymin}\ {\isacharquery}cl\ {\isasymand}\ {\isacharquery}cl\ {\isasymsubseteq}\ {\isacharquery}cl{\isachardoublequoteclose}\isanewline
\ \ \ \ \ \ \ \ \isacommand{by}\isamarkupfalse%
\ {\isacharparenleft}simp\ only{\isacharcolon}\ insert{\isacharunderscore}subset{\isacharparenright}\isanewline
\ \ \ \ \ \ \isacommand{then}\isamarkupfalse%
\ \isacommand{have}\isamarkupfalse%
\ {\isachardoublequoteopen}H\ {\isasymin}\ {\isacharquery}cl{\isachardoublequoteclose}\isanewline
\ \ \ \ \ \ \ \ \isacommand{by}\isamarkupfalse%
\ {\isacharparenleft}rule\ conjunct{\isadigit{1}}{\isacharparenright}\isanewline
\ \ \ \ \ \ \isacommand{show}\isamarkupfalse%
\ {\isachardoublequoteopen}G\ {\isasymin}\ {\isacharquery}cl\ {\isasymand}\ H\ {\isasymin}\ {\isacharquery}cl{\isachardoublequoteclose}\isanewline
\ \ \ \ \ \ \ \ \isacommand{using}\isamarkupfalse%
\ {\isacartoucheopen}G\ {\isasymin}\ {\isacharquery}cl{\isacartoucheclose}\ {\isacartoucheopen}H\ {\isasymin}\ {\isacharquery}cl{\isacartoucheclose}\ \isacommand{by}\isamarkupfalse%
\ {\isacharparenleft}rule\ conjI{\isacharparenright}\isanewline
\ \ \ \ \isacommand{qed}\isamarkupfalse%
\isanewline
\ \ \isacommand{qed}\isamarkupfalse%
\isanewline
\ \ \isacommand{have}\isamarkupfalse%
\ Dis{\isacharcolon}{\isachardoublequoteopen}{\isasymforall}F\ G\ H{\isachardot}\ Dis\ F\ G\ H\ {\isasymlongrightarrow}\ F\ {\isasymin}\ {\isacharquery}cl\ {\isasymlongrightarrow}\ {\isacharbraceleft}G{\isacharbraceright}\ {\isasymunion}\ {\isacharquery}cl\ {\isasymin}\ C\ {\isasymor}\ {\isacharbraceleft}H{\isacharbraceright}\ {\isasymunion}\ {\isacharquery}cl\ {\isasymin}\ C{\isachardoublequoteclose}\isanewline
\ \ \ \ \isacommand{using}\isamarkupfalse%
\ d\ \isacommand{by}\isamarkupfalse%
\ {\isacharparenleft}iprover\ elim{\isacharcolon}\ conjunct{\isadigit{2}}\ conjunct{\isadigit{1}}{\isacharparenright}\isanewline
\ \ \isacommand{have}\isamarkupfalse%
\ H{\isadigit{4}}{\isacharcolon}{\isachardoublequoteopen}{\isasymforall}F\ G\ H{\isachardot}\ Dis\ F\ G\ H\ {\isasymlongrightarrow}\ F\ {\isasymin}\ {\isacharquery}cl\ {\isasymlongrightarrow}\ G\ {\isasymin}\ {\isacharquery}cl\ {\isasymor}\ H\ {\isasymin}\ {\isacharquery}cl{\isachardoublequoteclose}\isanewline
\ \ \isacommand{proof}\isamarkupfalse%
\ {\isacharparenleft}rule\ allI{\isacharparenright}{\isacharplus}\isanewline
\ \ \ \ \isacommand{fix}\isamarkupfalse%
\ F\ G\ H\isanewline
\ \ \ \ \isacommand{show}\isamarkupfalse%
\ {\isachardoublequoteopen}Dis\ F\ G\ H\ {\isasymlongrightarrow}\ F\ {\isasymin}\ {\isacharquery}cl\ {\isasymlongrightarrow}\ G\ {\isasymin}\ {\isacharquery}cl\ {\isasymor}\ H\ {\isasymin}\ {\isacharquery}cl{\isachardoublequoteclose}\isanewline
\ \ \ \ \isacommand{proof}\isamarkupfalse%
\ {\isacharparenleft}rule\ impI{\isacharparenright}{\isacharplus}\isanewline
\ \ \ \ \ \ \isacommand{assume}\isamarkupfalse%
\ {\isachardoublequoteopen}Dis\ F\ G\ H{\isachardoublequoteclose}\isanewline
\ \ \ \ \ \ \isacommand{assume}\isamarkupfalse%
\ {\isachardoublequoteopen}F\ {\isasymin}\ {\isacharquery}cl{\isachardoublequoteclose}\isanewline
\ \ \ \ \ \ \isacommand{have}\isamarkupfalse%
\ {\isachardoublequoteopen}Dis\ F\ G\ H\ {\isasymlongrightarrow}\ F\ {\isasymin}\ {\isacharquery}cl\ {\isasymlongrightarrow}\ {\isacharbraceleft}G{\isacharbraceright}\ {\isasymunion}\ {\isacharquery}cl\ {\isasymin}\ C\ {\isasymor}\ {\isacharbraceleft}H{\isacharbraceright}\ {\isasymunion}\ {\isacharquery}cl\ {\isasymin}\ C{\isachardoublequoteclose}\isanewline
\ \ \ \ \ \ \ \ \isacommand{using}\isamarkupfalse%
\ Dis\ \isacommand{by}\isamarkupfalse%
\ {\isacharparenleft}iprover\ elim{\isacharcolon}\ allE{\isacharparenright}\isanewline
\ \ \ \ \ \ \isacommand{then}\isamarkupfalse%
\ \isacommand{have}\isamarkupfalse%
\ {\isachardoublequoteopen}F\ {\isasymin}\ {\isacharquery}cl\ {\isasymlongrightarrow}\ {\isacharbraceleft}G{\isacharbraceright}\ {\isasymunion}\ {\isacharquery}cl\ {\isasymin}\ C\ {\isasymor}\ {\isacharbraceleft}H{\isacharbraceright}\ {\isasymunion}\ {\isacharquery}cl\ {\isasymin}\ C{\isachardoublequoteclose}\isanewline
\ \ \ \ \ \ \ \ \isacommand{using}\isamarkupfalse%
\ {\isacartoucheopen}Dis\ F\ G\ H{\isacartoucheclose}\ \isacommand{by}\isamarkupfalse%
\ {\isacharparenleft}rule\ mp{\isacharparenright}\isanewline
\ \ \ \ \ \ \isacommand{then}\isamarkupfalse%
\ \isacommand{have}\isamarkupfalse%
\ {\isachardoublequoteopen}{\isacharbraceleft}G{\isacharbraceright}\ {\isasymunion}\ {\isacharquery}cl\ {\isasymin}\ C\ {\isasymor}\ {\isacharbraceleft}H{\isacharbraceright}\ {\isasymunion}\ {\isacharquery}cl\ {\isasymin}\ C{\isachardoublequoteclose}\isanewline
\ \ \ \ \ \ \ \ \isacommand{using}\isamarkupfalse%
\ {\isacartoucheopen}F\ {\isasymin}\ {\isacharquery}cl{\isacartoucheclose}\ \isacommand{by}\isamarkupfalse%
\ {\isacharparenleft}rule\ mp{\isacharparenright}\isanewline
\ \ \ \ \ \ \isacommand{thus}\isamarkupfalse%
\ {\isachardoublequoteopen}G\ {\isasymin}\ {\isacharquery}cl\ {\isasymor}\ H\ {\isasymin}\ {\isacharquery}cl{\isachardoublequoteclose}\isanewline
\ \ \ \ \ \ \isacommand{proof}\isamarkupfalse%
\ {\isacharparenleft}rule\ disjE{\isacharparenright}\isanewline
\ \ \ \ \ \ \ \ \isacommand{assume}\isamarkupfalse%
\ {\isachardoublequoteopen}{\isacharbraceleft}G{\isacharbraceright}\ {\isasymunion}\ {\isacharquery}cl\ {\isasymin}\ C{\isachardoublequoteclose}\isanewline
\ \ \ \ \ \ \ \ \isacommand{have}\isamarkupfalse%
\ {\isachardoublequoteopen}insert\ G\ {\isacharquery}cl\ {\isacharequal}\ {\isacharbraceleft}G{\isacharbraceright}\ {\isasymunion}\ {\isacharquery}cl{\isachardoublequoteclose}\isanewline
\ \ \ \ \ \ \ \ \ \ \isacommand{by}\isamarkupfalse%
\ {\isacharparenleft}rule\ insert{\isacharunderscore}is{\isacharunderscore}Un{\isacharparenright}\isanewline
\ \ \ \ \ \ \ \ \isacommand{have}\isamarkupfalse%
\ {\isachardoublequoteopen}insert\ G\ {\isacharquery}cl\ {\isasymin}\ C{\isachardoublequoteclose}\isanewline
\ \ \ \ \ \ \ \ \ \ \isacommand{using}\isamarkupfalse%
\ {\isacartoucheopen}{\isacharbraceleft}G{\isacharbraceright}\ {\isasymunion}\ {\isacharquery}cl\ {\isasymin}\ C{\isacartoucheclose}\ \isacommand{by}\isamarkupfalse%
\ {\isacharparenleft}simp\ only{\isacharcolon}\ {\isacartoucheopen}insert\ G\ {\isacharquery}cl\ {\isacharequal}\ {\isacharbraceleft}G{\isacharbraceright}\ {\isasymunion}\ {\isacharquery}cl{\isacartoucheclose}{\isacharparenright}\isanewline
\ \ \ \ \ \ \ \ \isacommand{have}\isamarkupfalse%
\ {\isachardoublequoteopen}insert\ G\ {\isacharquery}cl\ {\isasymin}\ C\ {\isasymLongrightarrow}\ G\ {\isasymin}\ {\isacharquery}cl{\isachardoublequoteclose}\isanewline
\ \ \ \ \ \ \ \ \ \ \isacommand{using}\isamarkupfalse%
\ assms{\isacharparenleft}{\isadigit{1}}{\isacharparenright}\ assms{\isacharparenleft}{\isadigit{2}}{\isacharparenright}\ \isacommand{by}\isamarkupfalse%
\ {\isacharparenleft}rule\ cl{\isacharunderscore}max{\isacharprime}{\isacharparenright}\isanewline
\ \ \ \ \ \ \ \ \isacommand{then}\isamarkupfalse%
\ \isacommand{have}\isamarkupfalse%
\ {\isachardoublequoteopen}G\ {\isasymin}\ {\isacharquery}cl{\isachardoublequoteclose}\isanewline
\ \ \ \ \ \ \ \ \ \ \isacommand{by}\isamarkupfalse%
\ {\isacharparenleft}simp\ only{\isacharcolon}\ {\isacartoucheopen}insert\ G\ {\isacharquery}cl\ {\isasymin}\ C{\isacartoucheclose}{\isacharparenright}\isanewline
\ \ \ \ \ \ \ \ \isacommand{thus}\isamarkupfalse%
\ {\isachardoublequoteopen}G\ {\isasymin}\ {\isacharquery}cl\ {\isasymor}\ H\ {\isasymin}\ {\isacharquery}cl{\isachardoublequoteclose}\isanewline
\ \ \ \ \ \ \ \ \ \ \isacommand{by}\isamarkupfalse%
\ {\isacharparenleft}rule\ disjI{\isadigit{1}}{\isacharparenright}\isanewline
\ \ \ \ \ \ \isacommand{next}\isamarkupfalse%
\isanewline
\ \ \ \ \ \ \ \ \isacommand{assume}\isamarkupfalse%
\ {\isachardoublequoteopen}{\isacharbraceleft}H{\isacharbraceright}\ {\isasymunion}\ {\isacharquery}cl\ {\isasymin}\ C{\isachardoublequoteclose}\isanewline
\ \ \ \ \ \ \ \ \isacommand{have}\isamarkupfalse%
\ {\isachardoublequoteopen}insert\ H\ {\isacharquery}cl\ {\isacharequal}\ {\isacharbraceleft}H{\isacharbraceright}\ {\isasymunion}\ {\isacharquery}cl{\isachardoublequoteclose}\isanewline
\ \ \ \ \ \ \ \ \ \ \isacommand{by}\isamarkupfalse%
\ {\isacharparenleft}rule\ insert{\isacharunderscore}is{\isacharunderscore}Un{\isacharparenright}\isanewline
\ \ \ \ \ \ \ \ \isacommand{have}\isamarkupfalse%
\ {\isachardoublequoteopen}insert\ H\ {\isacharquery}cl\ {\isasymin}\ C{\isachardoublequoteclose}\isanewline
\ \ \ \ \ \ \ \ \ \ \isacommand{using}\isamarkupfalse%
\ {\isacartoucheopen}{\isacharbraceleft}H{\isacharbraceright}\ {\isasymunion}\ {\isacharquery}cl\ {\isasymin}\ C{\isacartoucheclose}\ \isacommand{by}\isamarkupfalse%
\ {\isacharparenleft}simp\ only{\isacharcolon}\ {\isacartoucheopen}insert\ H\ {\isacharquery}cl\ {\isacharequal}\ {\isacharbraceleft}H{\isacharbraceright}\ {\isasymunion}\ {\isacharquery}cl{\isacartoucheclose}{\isacharparenright}\isanewline
\ \ \ \ \ \ \ \ \isacommand{have}\isamarkupfalse%
\ {\isachardoublequoteopen}insert\ H\ {\isacharquery}cl\ {\isasymin}\ C\ {\isasymLongrightarrow}\ H\ {\isasymin}\ {\isacharquery}cl{\isachardoublequoteclose}\isanewline
\ \ \ \ \ \ \ \ \ \ \isacommand{using}\isamarkupfalse%
\ assms{\isacharparenleft}{\isadigit{1}}{\isacharparenright}\ assms{\isacharparenleft}{\isadigit{2}}{\isacharparenright}\ \isacommand{by}\isamarkupfalse%
\ {\isacharparenleft}rule\ cl{\isacharunderscore}max{\isacharprime}{\isacharparenright}\isanewline
\ \ \ \ \ \ \ \ \isacommand{then}\isamarkupfalse%
\ \isacommand{have}\isamarkupfalse%
\ {\isachardoublequoteopen}H\ {\isasymin}\ {\isacharquery}cl{\isachardoublequoteclose}\isanewline
\ \ \ \ \ \ \ \ \ \ \isacommand{by}\isamarkupfalse%
\ {\isacharparenleft}simp\ only{\isacharcolon}\ {\isacartoucheopen}insert\ H\ {\isacharquery}cl\ {\isasymin}\ C{\isacartoucheclose}{\isacharparenright}\isanewline
\ \ \ \ \ \ \ \ \isacommand{thus}\isamarkupfalse%
\ {\isachardoublequoteopen}G\ {\isasymin}\ {\isacharquery}cl\ {\isasymor}\ H\ {\isasymin}\ {\isacharquery}cl{\isachardoublequoteclose}\isanewline
\ \ \ \ \ \ \ \ \ \ \isacommand{by}\isamarkupfalse%
\ {\isacharparenleft}rule\ disjI{\isadigit{2}}{\isacharparenright}\isanewline
\ \ \ \ \ \ \isacommand{qed}\isamarkupfalse%
\isanewline
\ \ \ \ \isacommand{qed}\isamarkupfalse%
\isanewline
\ \ \isacommand{qed}\isamarkupfalse%
\isanewline
\ \ \isacommand{show}\isamarkupfalse%
\ {\isachardoublequoteopen}{\isasymbottom}\ {\isasymnotin}\ {\isacharquery}cl\ {\isasymand}\isanewline
\ \ \ \ {\isacharparenleft}{\isasymforall}k{\isachardot}\ Atom\ k\ {\isasymin}\ {\isacharquery}cl\ {\isasymlongrightarrow}\ \isactrlbold {\isasymnot}\ {\isacharparenleft}Atom\ k{\isacharparenright}\ {\isasymin}\ {\isacharquery}cl\ {\isasymlongrightarrow}\ False{\isacharparenright}\ {\isasymand}\isanewline
\ \ \ \ {\isacharparenleft}{\isasymforall}F\ G\ H{\isachardot}\ Con\ F\ G\ H\ {\isasymlongrightarrow}\ F\ {\isasymin}\ {\isacharquery}cl\ {\isasymlongrightarrow}\ G\ {\isasymin}\ {\isacharquery}cl\ {\isasymand}\ H\ {\isasymin}\ {\isacharquery}cl{\isacharparenright}\ {\isasymand}\isanewline
\ \ \ \ {\isacharparenleft}{\isasymforall}F\ G\ H{\isachardot}\ Dis\ F\ G\ H\ {\isasymlongrightarrow}\ F\ {\isasymin}\ {\isacharquery}cl\ {\isasymlongrightarrow}\ G\ {\isasymin}\ {\isacharquery}cl\ {\isasymor}\ H\ {\isasymin}\ {\isacharquery}cl{\isacharparenright}{\isachardoublequoteclose}\isanewline
\ \ \ \ \isacommand{using}\isamarkupfalse%
\ H{\isadigit{1}}\ H{\isadigit{2}}\ H{\isadigit{3}}\ H{\isadigit{4}}\ \isacommand{by}\isamarkupfalse%
\ {\isacharparenleft}iprover\ intro{\isacharcolon}\ conjI{\isacharparenright}\isanewline
\isacommand{qed}\isamarkupfalse%
%
\endisatagproof
{\isafoldproof}%
%
\isadelimproof
%
\endisadelimproof
%
\begin{isamarkuptext}%
Del mismo modo, podemos probar el resultado de manera automática como sigue.%
\end{isamarkuptext}\isamarkuptrue%
\isacommand{lemma}\isamarkupfalse%
\ pcp{\isacharunderscore}lim{\isacharunderscore}Hintikka{\isacharcolon}\isanewline
\ \ \isakeyword{assumes}\ c{\isacharcolon}\ {\isachardoublequoteopen}pcp\ C{\isachardoublequoteclose}\isanewline
\ \ \isakeyword{assumes}\ sc{\isacharcolon}\ {\isachardoublequoteopen}subset{\isacharunderscore}closed\ C{\isachardoublequoteclose}\isanewline
\ \ \isakeyword{assumes}\ fc{\isacharcolon}\ {\isachardoublequoteopen}finite{\isacharunderscore}character\ C{\isachardoublequoteclose}\isanewline
\ \ \isakeyword{assumes}\ el{\isacharcolon}\ {\isachardoublequoteopen}S\ {\isasymin}\ C{\isachardoublequoteclose}\isanewline
\ \ \isakeyword{shows}\ {\isachardoublequoteopen}Hintikka\ {\isacharparenleft}pcp{\isacharunderscore}lim\ C\ S{\isacharparenright}{\isachardoublequoteclose}\isanewline
%
\isadelimproof
%
\endisadelimproof
%
\isatagproof
\isacommand{proof}\isamarkupfalse%
\ {\isacharminus}\isanewline
\ \ \isacommand{let}\isamarkupfalse%
\ {\isacharquery}cl\ {\isacharequal}\ {\isachardoublequoteopen}pcp{\isacharunderscore}lim\ C\ S{\isachardoublequoteclose}\isanewline
\ \ \isacommand{have}\isamarkupfalse%
\ {\isachardoublequoteopen}{\isacharquery}cl\ {\isasymin}\ C{\isachardoublequoteclose}\ \isacommand{using}\isamarkupfalse%
\ pcp{\isacharunderscore}lim{\isacharunderscore}in{\isacharbrackleft}OF\ c\ el\ sc\ fc{\isacharbrackright}\ \isacommand{{\isachardot}}\isamarkupfalse%
\isanewline
\ \ \isacommand{from}\isamarkupfalse%
\ c{\isacharbrackleft}unfolded\ pcp{\isacharunderscore}alt{\isacharcomma}\ THEN\ bspec{\isacharcomma}\ OF\ this{\isacharbrackright}\isanewline
\ \ \isacommand{have}\isamarkupfalse%
\ d{\isacharcolon}\ {\isachardoublequoteopen}{\isasymbottom}\ {\isasymnotin}\ {\isacharquery}cl{\isachardoublequoteclose}\isanewline
\ \ \ \ {\isachardoublequoteopen}Atom\ k\ {\isasymin}\ {\isacharquery}cl\ {\isasymLongrightarrow}\ \isactrlbold {\isasymnot}\ {\isacharparenleft}Atom\ k{\isacharparenright}\ {\isasymin}\ {\isacharquery}cl\ {\isasymLongrightarrow}\ False{\isachardoublequoteclose}\isanewline
\ \ \ \ {\isachardoublequoteopen}Con\ F\ G\ H\ {\isasymLongrightarrow}\ F\ {\isasymin}\ {\isacharquery}cl\ {\isasymLongrightarrow}\ insert\ G\ {\isacharparenleft}insert\ H\ {\isacharquery}cl{\isacharparenright}\ {\isasymin}\ C{\isachardoublequoteclose}\isanewline
\ \ \ \ {\isachardoublequoteopen}Dis\ F\ G\ H\ {\isasymLongrightarrow}\ F\ {\isasymin}\ {\isacharquery}cl\ {\isasymLongrightarrow}\ insert\ G\ {\isacharquery}cl\ {\isasymin}\ C\ {\isasymor}\ insert\ H\ {\isacharquery}cl\ {\isasymin}\ C{\isachardoublequoteclose}\isanewline
\ \ \ \ \isakeyword{for}\ k\ F\ G\ H\ \isacommand{by}\isamarkupfalse%
\ force{\isacharplus}\isanewline
\ \ \isacommand{with}\isamarkupfalse%
\ d{\isacharparenleft}{\isadigit{1}}{\isacharcomma}{\isadigit{2}}{\isacharparenright}\ \isacommand{show}\isamarkupfalse%
\ {\isachardoublequoteopen}Hintikka\ {\isacharquery}cl{\isachardoublequoteclose}\ \isacommand{unfolding}\isamarkupfalse%
\ Hintikka{\isacharunderscore}alt\ \isanewline
\ \ \ \ \isacommand{using}\isamarkupfalse%
\ c\ cl{\isacharunderscore}max\ cl{\isacharunderscore}max{\isacharprime}\ d{\isacharparenleft}{\isadigit{4}}{\isacharparenright}\ sc\ \isacommand{by}\isamarkupfalse%
\ blast\isanewline
\isacommand{qed}\isamarkupfalse%
%
\endisatagproof
{\isafoldproof}%
%
\isadelimproof
%
\endisadelimproof
%
\begin{isamarkuptext}%
Finalmente, vamos a demostrar el \isa{teorema\ de\ existencia\ de\ modelo}. Para ello precisaremos de
  un resultado que indica que la satisfacibilidad de conjuntos de fórmulas se hereda por la 
  contención.

  \begin{lema}
    Todo subconjunto de un conjunto de fórmulas satisfacible es satisfacible.
  \end{lema}

  \begin{demostracion}
    Sea \isa{B} un conjunto de fórmulas satisfacible y \isa{A\ {\isasymsubseteq}\ B}. Veamos que \isa{A} es satisfacible.
    Por definición, como \isa{B} es satisfacible, existe una interpretación \isa{{\isasymA}} que es modelo de cada 
    fórmula de \isa{B}. Como \isa{A\ {\isasymsubseteq}\ B}, en particular \isa{{\isasymA}} es modelo de toda fórmula de \isa{A}. Por tanto, 
    \isa{A} es satisfacible, ya que existe una interpretación que es modelo de todas sus fórmulas.
  \end{demostracion}

  Su prueba detallada en Isabelle/HOL es la siguiente.%
\end{isamarkuptext}\isamarkuptrue%
\isacommand{lemma}\isamarkupfalse%
\ sat{\isacharunderscore}mono{\isacharcolon}\isanewline
\ \ \isakeyword{assumes}\ {\isachardoublequoteopen}A\ {\isasymsubseteq}\ B{\isachardoublequoteclose}\isanewline
\ \ \ \ \ \ \ \ \ \ {\isachardoublequoteopen}sat\ B{\isachardoublequoteclose}\isanewline
\ \ \ \ \ \ \ \ \isakeyword{shows}\ {\isachardoublequoteopen}sat\ A{\isachardoublequoteclose}\isanewline
%
\isadelimproof
\ \ %
\endisadelimproof
%
\isatagproof
\isacommand{unfolding}\isamarkupfalse%
\ sat{\isacharunderscore}def\isanewline
\isacommand{proof}\isamarkupfalse%
\ {\isacharminus}\isanewline
\ \isacommand{have}\isamarkupfalse%
\ satB{\isacharcolon}{\isachardoublequoteopen}{\isasymexists}{\isasymA}{\isachardot}\ {\isasymforall}F\ {\isasymin}\ B{\isachardot}\ {\isasymA}\ {\isasymTurnstile}\ F{\isachardoublequoteclose}\isanewline
\ \ \ \isacommand{using}\isamarkupfalse%
\ assms{\isacharparenleft}{\isadigit{2}}{\isacharparenright}\ \isacommand{by}\isamarkupfalse%
\ {\isacharparenleft}simp\ only{\isacharcolon}\ sat{\isacharunderscore}def{\isacharparenright}\isanewline
\ \isacommand{obtain}\isamarkupfalse%
\ {\isasymA}\ \isakeyword{where}\ {\isachardoublequoteopen}{\isasymforall}F\ {\isasymin}\ B{\isachardot}\ {\isasymA}\ {\isasymTurnstile}\ F{\isachardoublequoteclose}\isanewline
\ \ \ \ \isacommand{using}\isamarkupfalse%
\ satB\ \isacommand{by}\isamarkupfalse%
\ {\isacharparenleft}rule\ exE{\isacharparenright}\isanewline
\ \isacommand{have}\isamarkupfalse%
\ {\isachardoublequoteopen}{\isasymforall}F\ {\isasymin}\ A{\isachardot}\ {\isasymA}\ {\isasymTurnstile}\ F{\isachardoublequoteclose}\isanewline
\ \ \isacommand{proof}\isamarkupfalse%
\ {\isacharparenleft}rule\ ballI{\isacharparenright}\isanewline
\ \ \ \ \isacommand{fix}\isamarkupfalse%
\ F\isanewline
\ \ \ \ \isacommand{assume}\isamarkupfalse%
\ {\isachardoublequoteopen}F\ {\isasymin}\ A{\isachardoublequoteclose}\isanewline
\ \ \ \ \isacommand{have}\isamarkupfalse%
\ {\isachardoublequoteopen}F\ {\isasymin}\ A\ {\isasymlongrightarrow}\ F\ {\isasymin}\ B{\isachardoublequoteclose}\isanewline
\ \ \ \ \ \ \isacommand{using}\isamarkupfalse%
\ assms{\isacharparenleft}{\isadigit{1}}{\isacharparenright}\ \isacommand{by}\isamarkupfalse%
\ {\isacharparenleft}rule\ in{\isacharunderscore}mono{\isacharparenright}\isanewline
\ \ \ \ \isacommand{then}\isamarkupfalse%
\ \isacommand{have}\isamarkupfalse%
\ {\isachardoublequoteopen}F\ {\isasymin}\ B{\isachardoublequoteclose}\isanewline
\ \ \ \ \ \ \isacommand{using}\isamarkupfalse%
\ {\isacartoucheopen}F\ {\isasymin}\ A{\isacartoucheclose}\ \isacommand{by}\isamarkupfalse%
\ {\isacharparenleft}rule\ mp{\isacharparenright}\isanewline
\ \ \ \ \isacommand{show}\isamarkupfalse%
\ {\isachardoublequoteopen}{\isasymA}\ {\isasymTurnstile}\ F{\isachardoublequoteclose}\isanewline
\ \ \ \ \ \ \isacommand{using}\isamarkupfalse%
\ {\isacartoucheopen}{\isasymforall}F\ {\isasymin}\ B{\isachardot}\ {\isasymA}\ {\isasymTurnstile}\ F{\isacartoucheclose}\ {\isacartoucheopen}F\ {\isasymin}\ B{\isacartoucheclose}\ \isacommand{by}\isamarkupfalse%
\ {\isacharparenleft}rule\ bspec{\isacharparenright}\isanewline
\ \ \isacommand{qed}\isamarkupfalse%
\isanewline
\ \ \isacommand{thus}\isamarkupfalse%
\ {\isachardoublequoteopen}{\isasymexists}{\isasymA}{\isachardot}\ {\isasymforall}F\ {\isasymin}\ A{\isachardot}\ {\isasymA}\ {\isasymTurnstile}\ F{\isachardoublequoteclose}\isanewline
\ \ \ \ \isacommand{by}\isamarkupfalse%
\ {\isacharparenleft}simp\ only{\isacharcolon}\ exI{\isacharparenright}\isanewline
\isacommand{qed}\isamarkupfalse%
%
\endisatagproof
{\isafoldproof}%
%
\isadelimproof
%
\endisadelimproof
%
\begin{isamarkuptext}%
De este modo, procedamos finalmente con la demostración del teorema.

  \begin{teorema}[Teorema de Existencia de Modelo]
    Todo conjunto de fórmulas perteneciente a una colección que verifique la propiedad de
    consistencia proposicional es satisfacible. 
  \end{teorema}

  \begin{demostracion}
    Sea \isa{C} una colección de conjuntos de fórmulas proposicionales que verifique la propiedad de 
    consistencia proposicional y \isa{S\ {\isasymin}\ C}. Vamos a probar que \isa{S} es satisfacible.

    En primer lugar, como \isa{C} verifica la propiedad de consistencia proposicional, por el lema 
    \isa{{\isadigit{3}}{\isachardot}{\isadigit{0}}{\isachardot}{\isadigit{3}}} podemos extenderla a una colección \isa{C{\isacharprime}} que también verifique la propiedad y
    sea cerrada bajo subconjuntos. A su vez, por el lema \isa{{\isadigit{3}}{\isachardot}{\isadigit{0}}{\isachardot}{\isadigit{5}}}, como la extensión 
    \isa{C{\isacharprime}} es una colección con la propiedad de consistencia proposicional y cerrada bajo 
    subconjuntos, podemos extenderla a otra colección \isa{C{\isacharprime}{\isacharprime}} que también verifica la propiedad de
    consistencia proposicional y sea de carácter finito. De este modo, por la transitividad de la 
    contención, es claro que \isa{C{\isacharprime}{\isacharprime}} es una extensión de \isa{C}, luego \isa{S\ {\isasymin}\ C{\isacharprime}{\isacharprime}} por hipótesis. 
    Por otro lado, por el lema \isa{{\isadigit{3}}{\isachardot}{\isadigit{0}}{\isachardot}{\isadigit{4}}}, como \isa{C{\isacharprime}{\isacharprime}} es de carácter finito, se tiene que es 
    cerrada bajo subconjuntos. 

    En suma, \isa{C{\isacharprime}{\isacharprime}} es una extensión de \isa{C} que verifica la propiedad de consistencia proposicional, 
    es cerrada bajo subconjuntos y es de carácter finito. Luego, por el lema \isa{{\isadigit{4}}{\isachardot}{\isadigit{2}}{\isachardot}{\isadigit{4}}}, el límite de 
    la sucesión \isa{{\isacharbraceleft}S\isactrlsub n{\isacharbraceright}} de conjuntos de \isa{C{\isacharprime}{\isacharprime}} a partir de \isa{S} según la definición \isa{{\isadigit{4}}{\isachardot}{\isadigit{1}}{\isachardot}{\isadigit{1}}} es un 
    conjunto de Hintikka. Por tanto, por el \isa{teorema\ de\ Hintikka}, se trata de un conjunto 
    satisfacible. 

    Finalmente, puesto que para todo \isa{n\ {\isasymin}\ {\isasymnat}} se tiene que \isa{S\isactrlsub n} está contenido en el límite, en 
    particular el conjunto \isa{S\isactrlsub {\isadigit{0}}} está contenido en él. Por definición de la sucesión, dicho conjunto 
    coincide con \isa{S}. Por tanto, como \isa{S} está contenido en el límite que es un conjunto 
    satisfacible, queda demostrada la satisfacibilidad de \isa{S}.
  \end{demostracion}

  \comentario{Tal vez sería buena idea hacer un grafo similar al de ex3.}

  Mostremos su formalización y demostración detallada en Isabelle.%
\end{isamarkuptext}\isamarkuptrue%
\isacommand{theorem}\isamarkupfalse%
\isanewline
\ \ \isakeyword{fixes}\ S\ {\isacharcolon}{\isacharcolon}\ {\isachardoublequoteopen}{\isacharprime}a\ {\isacharcolon}{\isacharcolon}\ countable\ formula\ set{\isachardoublequoteclose}\isanewline
\ \ \isakeyword{assumes}\ {\isachardoublequoteopen}pcp\ C{\isachardoublequoteclose}\isanewline
\ \ \isakeyword{assumes}\ {\isachardoublequoteopen}S\ {\isasymin}\ C{\isachardoublequoteclose}\isanewline
\ \ \isakeyword{shows}\ {\isachardoublequoteopen}sat\ S{\isachardoublequoteclose}\isanewline
%
\isadelimproof
%
\endisadelimproof
%
\isatagproof
\isacommand{proof}\isamarkupfalse%
\ {\isacharminus}\isanewline
\ \ \isacommand{have}\isamarkupfalse%
\ {\isachardoublequoteopen}pcp\ C\ {\isasymLongrightarrow}\ {\isasymexists}C{\isacharprime}{\isachardot}\ C\ {\isasymsubseteq}\ C{\isacharprime}\ {\isasymand}\ pcp\ C{\isacharprime}\ {\isasymand}\ subset{\isacharunderscore}closed\ C{\isacharprime}{\isachardoublequoteclose}\isanewline
\ \ \ \ \isacommand{by}\isamarkupfalse%
\ {\isacharparenleft}rule\ ex{\isadigit{1}}{\isacharparenright}\isanewline
\ \ \isacommand{then}\isamarkupfalse%
\ \isacommand{have}\isamarkupfalse%
\ E{\isadigit{1}}{\isacharcolon}{\isachardoublequoteopen}{\isasymexists}C{\isacharprime}{\isachardot}\ C\ {\isasymsubseteq}\ C{\isacharprime}\ {\isasymand}\ pcp\ C{\isacharprime}\ {\isasymand}\ subset{\isacharunderscore}closed\ C{\isacharprime}{\isachardoublequoteclose}\isanewline
\ \ \ \ \isacommand{by}\isamarkupfalse%
\ {\isacharparenleft}simp\ only{\isacharcolon}\ assms{\isacharparenleft}{\isadigit{1}}{\isacharparenright}{\isacharparenright}\isanewline
\ \ \isacommand{obtain}\isamarkupfalse%
\ Ce{\isacharprime}\ \isakeyword{where}\ H{\isadigit{1}}{\isacharcolon}{\isachardoublequoteopen}C\ {\isasymsubseteq}\ Ce{\isacharprime}\ {\isasymand}\ pcp\ Ce{\isacharprime}\ {\isasymand}\ subset{\isacharunderscore}closed\ Ce{\isacharprime}{\isachardoublequoteclose}\isanewline
\ \ \ \ \isacommand{using}\isamarkupfalse%
\ E{\isadigit{1}}\ \isacommand{by}\isamarkupfalse%
\ {\isacharparenleft}rule\ exE{\isacharparenright}\isanewline
\ \ \isacommand{have}\isamarkupfalse%
\ {\isachardoublequoteopen}C\ {\isasymsubseteq}\ Ce{\isacharprime}{\isachardoublequoteclose}\isanewline
\ \ \ \ \isacommand{using}\isamarkupfalse%
\ H{\isadigit{1}}\ \isacommand{by}\isamarkupfalse%
\ {\isacharparenleft}rule\ conjunct{\isadigit{1}}{\isacharparenright}\isanewline
\ \ \isacommand{have}\isamarkupfalse%
\ {\isachardoublequoteopen}pcp\ Ce{\isacharprime}{\isachardoublequoteclose}\isanewline
\ \ \ \ \isacommand{using}\isamarkupfalse%
\ H{\isadigit{1}}\ \isacommand{by}\isamarkupfalse%
\ {\isacharparenleft}iprover\ elim{\isacharcolon}\ conjunct{\isadigit{2}}\ conjunct{\isadigit{1}}{\isacharparenright}\isanewline
\ \ \isacommand{have}\isamarkupfalse%
\ {\isachardoublequoteopen}subset{\isacharunderscore}closed\ Ce{\isacharprime}{\isachardoublequoteclose}\isanewline
\ \ \ \ \isacommand{using}\isamarkupfalse%
\ H{\isadigit{1}}\ \isacommand{by}\isamarkupfalse%
\ {\isacharparenleft}iprover\ elim{\isacharcolon}\ conjunct{\isadigit{2}}\ conjunct{\isadigit{1}}{\isacharparenright}\isanewline
\ \ \isacommand{have}\isamarkupfalse%
\ E{\isadigit{2}}{\isacharcolon}{\isachardoublequoteopen}{\isasymexists}Ce{\isachardot}\ Ce{\isacharprime}\ {\isasymsubseteq}\ Ce\ {\isasymand}\ pcp\ Ce\ {\isasymand}\ finite{\isacharunderscore}character\ Ce{\isachardoublequoteclose}\isanewline
\ \ \ \ \isacommand{using}\isamarkupfalse%
\ {\isacartoucheopen}pcp\ Ce{\isacharprime}{\isacartoucheclose}\ {\isacartoucheopen}subset{\isacharunderscore}closed\ Ce{\isacharprime}{\isacartoucheclose}\ \isacommand{by}\isamarkupfalse%
\ {\isacharparenleft}rule\ ex{\isadigit{3}}{\isacharparenright}\isanewline
\ \ \isacommand{obtain}\isamarkupfalse%
\ Ce\ \isakeyword{where}\ H{\isadigit{2}}{\isacharcolon}{\isachardoublequoteopen}Ce{\isacharprime}\ {\isasymsubseteq}\ Ce\ {\isasymand}\ pcp\ Ce\ {\isasymand}\ finite{\isacharunderscore}character\ Ce{\isachardoublequoteclose}\isanewline
\ \ \ \ \isacommand{using}\isamarkupfalse%
\ E{\isadigit{2}}\ \isacommand{by}\isamarkupfalse%
\ {\isacharparenleft}rule\ exE{\isacharparenright}\isanewline
\ \ \isacommand{have}\isamarkupfalse%
\ {\isachardoublequoteopen}Ce{\isacharprime}\ {\isasymsubseteq}\ Ce{\isachardoublequoteclose}\isanewline
\ \ \ \ \isacommand{using}\isamarkupfalse%
\ H{\isadigit{2}}\ \isacommand{by}\isamarkupfalse%
\ {\isacharparenleft}rule\ conjunct{\isadigit{1}}{\isacharparenright}\isanewline
\ \ \isacommand{then}\isamarkupfalse%
\ \isacommand{have}\isamarkupfalse%
\ Subset{\isacharcolon}{\isachardoublequoteopen}C\ {\isasymsubseteq}\ Ce{\isachardoublequoteclose}\isanewline
\ \ \ \ \isacommand{using}\isamarkupfalse%
\ {\isacartoucheopen}C\ {\isasymsubseteq}\ Ce{\isacharprime}{\isacartoucheclose}\ \isacommand{by}\isamarkupfalse%
\ {\isacharparenleft}simp\ only{\isacharcolon}\ subset{\isacharunderscore}trans{\isacharparenright}\isanewline
\ \ \isacommand{have}\isamarkupfalse%
\ Pcp{\isacharcolon}{\isachardoublequoteopen}pcp\ Ce{\isachardoublequoteclose}\isanewline
\ \ \ \ \isacommand{using}\isamarkupfalse%
\ H{\isadigit{2}}\ \isacommand{by}\isamarkupfalse%
\ {\isacharparenleft}iprover\ elim{\isacharcolon}\ conjunct{\isadigit{2}}\ conjunct{\isadigit{1}}{\isacharparenright}\isanewline
\ \ \isacommand{have}\isamarkupfalse%
\ FC{\isacharcolon}{\isachardoublequoteopen}finite{\isacharunderscore}character\ Ce{\isachardoublequoteclose}\isanewline
\ \ \ \ \isacommand{using}\isamarkupfalse%
\ H{\isadigit{2}}\ \isacommand{by}\isamarkupfalse%
\ {\isacharparenleft}iprover\ elim{\isacharcolon}\ conjunct{\isadigit{2}}\ conjunct{\isadigit{1}}{\isacharparenright}\isanewline
\ \ \isacommand{then}\isamarkupfalse%
\ \isacommand{have}\isamarkupfalse%
\ SC{\isacharcolon}{\isachardoublequoteopen}subset{\isacharunderscore}closed\ Ce{\isachardoublequoteclose}\isanewline
\ \ \ \ \isacommand{by}\isamarkupfalse%
\ {\isacharparenleft}rule\ ex{\isadigit{2}}{\isacharparenright}\isanewline
\ \ \isacommand{have}\isamarkupfalse%
\ {\isachardoublequoteopen}S\ {\isasymin}\ C\ {\isasymlongrightarrow}\ S\ {\isasymin}\ Ce{\isachardoublequoteclose}\isanewline
\ \ \ \ \isacommand{using}\isamarkupfalse%
\ {\isacartoucheopen}C\ {\isasymsubseteq}\ Ce{\isacartoucheclose}\ \isacommand{by}\isamarkupfalse%
\ {\isacharparenleft}rule\ in{\isacharunderscore}mono{\isacharparenright}\isanewline
\ \ \isacommand{then}\isamarkupfalse%
\ \isacommand{have}\isamarkupfalse%
\ {\isachardoublequoteopen}S\ {\isasymin}\ Ce{\isachardoublequoteclose}\ \isanewline
\ \ \ \ \isacommand{using}\isamarkupfalse%
\ assms{\isacharparenleft}{\isadigit{2}}{\isacharparenright}\ \isacommand{by}\isamarkupfalse%
\ {\isacharparenleft}rule\ mp{\isacharparenright}\isanewline
\ \ \isacommand{have}\isamarkupfalse%
\ {\isachardoublequoteopen}Hintikka\ {\isacharparenleft}pcp{\isacharunderscore}lim\ Ce\ S{\isacharparenright}{\isachardoublequoteclose}\isanewline
\ \ \ \ \isacommand{using}\isamarkupfalse%
\ Pcp\ SC\ FC\ {\isacartoucheopen}S\ {\isasymin}\ Ce{\isacartoucheclose}\ \isacommand{by}\isamarkupfalse%
\ {\isacharparenleft}rule\ pcp{\isacharunderscore}lim{\isacharunderscore}Hintikka{\isacharparenright}\isanewline
\ \ \isacommand{then}\isamarkupfalse%
\ \isacommand{have}\isamarkupfalse%
\ {\isachardoublequoteopen}sat\ {\isacharparenleft}pcp{\isacharunderscore}lim\ Ce\ S{\isacharparenright}{\isachardoublequoteclose}\isanewline
\ \ \ \ \isacommand{by}\isamarkupfalse%
\ {\isacharparenleft}rule\ Hintikkaslemma{\isacharparenright}\isanewline
\ \ \isacommand{have}\isamarkupfalse%
\ {\isachardoublequoteopen}pcp{\isacharunderscore}seq\ Ce\ S\ {\isadigit{0}}\ {\isacharequal}\ S{\isachardoublequoteclose}\isanewline
\ \ \ \ \isacommand{by}\isamarkupfalse%
\ {\isacharparenleft}simp\ only{\isacharcolon}\ pcp{\isacharunderscore}seq{\isachardot}simps{\isacharparenleft}{\isadigit{1}}{\isacharparenright}{\isacharparenright}\isanewline
\ \ \isacommand{have}\isamarkupfalse%
\ {\isachardoublequoteopen}pcp{\isacharunderscore}seq\ Ce\ S\ {\isadigit{0}}\ {\isasymsubseteq}\ pcp{\isacharunderscore}lim\ Ce\ S{\isachardoublequoteclose}\isanewline
\ \ \ \ \isacommand{by}\isamarkupfalse%
\ {\isacharparenleft}rule\ pcp{\isacharunderscore}seq{\isacharunderscore}sub{\isacharparenright}\isanewline
\ \ \isacommand{then}\isamarkupfalse%
\ \isacommand{have}\isamarkupfalse%
\ {\isachardoublequoteopen}S\ {\isasymsubseteq}\ pcp{\isacharunderscore}lim\ Ce\ S{\isachardoublequoteclose}\isanewline
\ \ \ \ \isacommand{by}\isamarkupfalse%
\ {\isacharparenleft}simp\ only{\isacharcolon}\ {\isacartoucheopen}pcp{\isacharunderscore}seq\ Ce\ S\ {\isadigit{0}}\ {\isacharequal}\ S{\isacartoucheclose}{\isacharparenright}\isanewline
\ \ \isacommand{thus}\isamarkupfalse%
\ {\isachardoublequoteopen}sat\ S{\isachardoublequoteclose}\isanewline
\ \ \ \ \isacommand{using}\isamarkupfalse%
\ {\isacartoucheopen}sat\ {\isacharparenleft}pcp{\isacharunderscore}lim\ Ce\ S{\isacharparenright}{\isacartoucheclose}\ \isacommand{by}\isamarkupfalse%
\ {\isacharparenleft}rule\ sat{\isacharunderscore}mono{\isacharparenright}\isanewline
\isacommand{qed}\isamarkupfalse%
%
\endisatagproof
{\isafoldproof}%
%
\isadelimproof
%
\endisadelimproof
%
\begin{isamarkuptext}%
Finalmente, demostremos el teorema de manera automática.%
\end{isamarkuptext}\isamarkuptrue%
\isacommand{theorem}\isamarkupfalse%
\ pcp{\isacharunderscore}sat{\isacharcolon}\isanewline
\ \ \isakeyword{fixes}\ S\ {\isacharcolon}{\isacharcolon}\ {\isachardoublequoteopen}{\isacharprime}a\ {\isacharcolon}{\isacharcolon}\ countable\ formula\ set{\isachardoublequoteclose}\isanewline
\ \ \isakeyword{assumes}\ c{\isacharcolon}\ {\isachardoublequoteopen}pcp\ C{\isachardoublequoteclose}\isanewline
\ \ \isakeyword{assumes}\ el{\isacharcolon}\ {\isachardoublequoteopen}S\ {\isasymin}\ C{\isachardoublequoteclose}\isanewline
\ \ \isakeyword{shows}\ {\isachardoublequoteopen}sat\ S{\isachardoublequoteclose}\isanewline
%
\isadelimproof
%
\endisadelimproof
%
\isatagproof
\isacommand{proof}\isamarkupfalse%
\ {\isacharminus}\isanewline
\ \ \isacommand{from}\isamarkupfalse%
\ c\ \isacommand{obtain}\isamarkupfalse%
\ Ce\ \isakeyword{where}\ \isanewline
\ \ \ \ \ \ {\isachardoublequoteopen}C\ {\isasymsubseteq}\ Ce{\isachardoublequoteclose}\ {\isachardoublequoteopen}pcp\ Ce{\isachardoublequoteclose}\ {\isachardoublequoteopen}subset{\isacharunderscore}closed\ Ce{\isachardoublequoteclose}\ {\isachardoublequoteopen}finite{\isacharunderscore}character\ Ce{\isachardoublequoteclose}\ \isanewline
\ \ \ \ \ \ \isacommand{using}\isamarkupfalse%
\ ex{\isadigit{1}}{\isacharbrackleft}\isakeyword{where}\ {\isacharprime}a{\isacharequal}{\isacharprime}a{\isacharbrackright}\ ex{\isadigit{2}}{\isacharbrackleft}\isakeyword{where}\ {\isacharprime}a{\isacharequal}{\isacharprime}a{\isacharbrackright}\ ex{\isadigit{3}}{\isacharbrackleft}\isakeyword{where}\ {\isacharprime}a{\isacharequal}{\isacharprime}a{\isacharbrackright}\isanewline
\ \ \ \ \isacommand{by}\isamarkupfalse%
\ {\isacharparenleft}meson\ dual{\isacharunderscore}order{\isachardot}trans\ ex{\isadigit{2}}{\isacharparenright}\isanewline
\ \ \isacommand{have}\isamarkupfalse%
\ {\isachardoublequoteopen}S\ {\isasymin}\ Ce{\isachardoublequoteclose}\ \isacommand{using}\isamarkupfalse%
\ {\isacartoucheopen}C\ {\isasymsubseteq}\ Ce{\isacartoucheclose}\ el\ \isacommand{{\isachardot}{\isachardot}}\isamarkupfalse%
\isanewline
\ \ \isacommand{with}\isamarkupfalse%
\ pcp{\isacharunderscore}lim{\isacharunderscore}Hintikka\ {\isacartoucheopen}pcp\ Ce{\isacartoucheclose}\ {\isacartoucheopen}subset{\isacharunderscore}closed\ Ce{\isacartoucheclose}\ {\isacartoucheopen}finite{\isacharunderscore}character\ Ce{\isacartoucheclose}\isanewline
\ \ \isacommand{have}\isamarkupfalse%
\ \ {\isachardoublequoteopen}Hintikka\ {\isacharparenleft}pcp{\isacharunderscore}lim\ Ce\ S{\isacharparenright}{\isachardoublequoteclose}\ \isacommand{{\isachardot}}\isamarkupfalse%
\isanewline
\ \ \isacommand{with}\isamarkupfalse%
\ Hintikkaslemma\ \isacommand{have}\isamarkupfalse%
\ {\isachardoublequoteopen}sat\ {\isacharparenleft}pcp{\isacharunderscore}lim\ Ce\ S{\isacharparenright}{\isachardoublequoteclose}\ \isacommand{{\isachardot}}\isamarkupfalse%
\isanewline
\ \ \isacommand{moreover}\isamarkupfalse%
\ \isacommand{have}\isamarkupfalse%
\ {\isachardoublequoteopen}S\ {\isasymsubseteq}\ pcp{\isacharunderscore}lim\ Ce\ S{\isachardoublequoteclose}\ \isanewline
\ \ \ \ \isacommand{using}\isamarkupfalse%
\ pcp{\isacharunderscore}seq{\isachardot}simps{\isacharparenleft}{\isadigit{1}}{\isacharparenright}\ pcp{\isacharunderscore}seq{\isacharunderscore}sub\ \isacommand{by}\isamarkupfalse%
\ fast\isanewline
\ \ \isacommand{ultimately}\isamarkupfalse%
\ \isacommand{show}\isamarkupfalse%
\ {\isacharquery}thesis\ \isacommand{unfolding}\isamarkupfalse%
\ sat{\isacharunderscore}def\ \isacommand{by}\isamarkupfalse%
\ fast\isanewline
\isacommand{qed}\isamarkupfalse%
%
\endisatagproof
{\isafoldproof}%
%
\isadelimproof
%
\endisadelimproof
%
\isadelimdocument
%
\endisadelimdocument
%
\isatagdocument
%
\isamarkupsection{Teorema de Compacidad%
}
\isamarkuptrue%
%
\endisatagdocument
{\isafolddocument}%
%
\isadelimdocument
%
\endisadelimdocument
%
\begin{isamarkuptext}%
En esta sección vamos demostrar el \isa{Teorema\ de\ Compacidad} para la lógica proposicional
  como consecuencia del \isa{Teorema\ de\ Existencia\ de\ Modelo}.

  \begin{teorema}[Teorema de Compacidad]
    Todo conjunto de fórmulas finitamente satisfacible es satisfacible.
  \end{teorema}

  Para su demostración consideraremos la colección formada por los conjuntos de fórmulas finitamente 
  satisfacibles. Probaremos que dicha colección verifica la propiedad de consistencia proposicional
  y, por el \isa{Teorema\ de\ Existencia\ de\ Modelo}, todo conjunto perteneciente a ella será
  satisfacible, demostrando así el teorema.

  Mostremos previamente dos resultados sobre subconjuntos finitos que emplearemos en la 
  demostración del teorema.

  \begin{lema}
    Sea un conjunto de la forma \isa{{\isacharbraceleft}a{\isacharbraceright}\ {\isasymunion}\ B} y \isa{S} un subconjunto finito suyo. Entonces,
    existe un subconjunto finito \isa{S{\isacharprime}} de \isa{B} tal que o bien \isa{S\ {\isacharequal}\ {\isacharbraceleft}a{\isacharbraceright}\ {\isasymunion}\ S{\isacharprime}} o bien \isa{S\ {\isacharequal}\ S{\isacharprime}}.
  \end{lema}

  \begin{demostracion}
    La prueba del resultado se realiza por inducción en la estructura recursiva de los conjuntos 
    finitos.

    En primer lugar, probemos el caso base correspondiente al conjunto vacío. Si consideramos 
    \isa{S\ {\isacharequal}\ {\isacharbraceleft}{\isacharbraceright}}, tomando también \isa{S{\isacharprime}\ {\isacharequal}\ {\isacharbraceleft}{\isacharbraceright}} es claro que verifica que es subconjunto finito de \isa{B}
    y o bien \isa{S\ {\isacharequal}\ {\isacharbraceleft}a{\isacharbraceright}\ {\isasymunion}\ S{\isacharprime}} o bien \isa{S\ {\isacharequal}\ S{\isacharprime}}.

    Veamos el caso de inducción. Sea \isa{S} un conjunto finito verificando la hipótesis de inducción:
    
    \isa{{\isacharparenleft}HI{\isacharparenright}{\isacharcolon}\ Si\ S\ {\isasymsubseteq}\ {\isacharbraceleft}a{\isacharbraceright}\ {\isasymunion}\ B{\isacharcomma}\ entonces\ existe\ un\ subconjunto\ finito\ S{\isacharprime}\ de\ B\ tal\ que\ o\ bien}

    \hspace{1cm}\isa{S\ {\isacharequal}\ {\isacharbraceleft}a{\isacharbraceright}\ {\isasymunion}\ S{\isacharprime}\ o\ bien\ S\ {\isacharequal}\ S{\isacharprime}{\isachardot}}

    Sea un elemento cualquiera \isa{x\ {\isasymnotin}\ S}. Vamos a probar que se verifica el resultado para el conjunto
    \isa{{\isacharbraceleft}x{\isacharbraceright}\ {\isasymunion}\ S}. Es decir, si \isa{{\isacharbraceleft}x{\isacharbraceright}\ {\isasymunion}\ S\ {\isasymsubseteq}\ {\isacharbraceleft}a{\isacharbraceright}\ {\isasymunion}\ B}, vamos a encontrar un subconjunto finito \isa{S{\isacharprime}{\isacharprime}} de 
    \isa{B} tal que o bien \isa{{\isacharbraceleft}x{\isacharbraceright}\ {\isasymunion}\ S\ {\isacharequal}\ {\isacharbraceleft}a{\isacharbraceright}\ {\isasymunion}\ S{\isacharprime}{\isacharprime}} o bien \isa{{\isacharbraceleft}x{\isacharbraceright}\ {\isasymunion}\ S\ {\isacharequal}\ S{\isacharprime}{\isacharprime}}.

    Supongamos, pues, que \isa{{\isacharbraceleft}x{\isacharbraceright}\ {\isasymunion}\ S\ {\isasymsubseteq}\ {\isacharbraceleft}a{\isacharbraceright}\ {\isasymunion}\ B}. En este caso, es claro que se verifica que
    \isa{x\ {\isasymin}\ {\isacharbraceleft}a{\isacharbraceright}\ {\isasymunion}\ B} y \isa{S\ {\isasymsubseteq}\ {\isacharbraceleft}a{\isacharbraceright}\ {\isasymunion}\ B}. Por lo último, aplicando hipótesis de inducción, podemos hallar 
    un subconjunto finito \isa{S{\isacharprime}} de \isa{B} tal que o bien \isa{S\ {\isacharequal}\ {\isacharbraceleft}a{\isacharbraceright}\ {\isasymunion}\ S{\isacharprime}} o bien \isa{S\ {\isacharequal}\ S{\isacharprime}}. Por otro lado,
    como \isa{x\ {\isasymin}\ {\isacharbraceleft}a{\isacharbraceright}\ {\isasymunion}\ B}, deducimos que o bien \isa{x\ {\isacharequal}\ a} o bien \isa{x\ {\isasymin}\ B}. En efecto, probemos que se 
    verifica el resultado para ambos casos de la última disyunción.

    En primer lugar, supongamos que \isa{x\ {\isacharequal}\ a}. En este caso, veremos que el conjunto \isa{S{\isacharprime}{\isacharprime}} que 
    verifica el resultado es el propio \isa{S{\isacharprime}}. Se observa fácilmente ya que, si \isa{S\ {\isacharequal}\ {\isacharbraceleft}a{\isacharbraceright}\ {\isasymunion}\ S{\isacharprime}}, como 
    \isa{x\ {\isacharequal}\ a} se obtiene que \isa{{\isacharbraceleft}x{\isacharbraceright}\ {\isasymunion}\ S\ {\isacharequal}\ {\isacharbraceleft}a{\isacharbraceright}\ {\isasymunion}\ S{\isacharprime}}, de modo que \isa{S{\isacharprime}{\isacharprime}\ {\isacharequal}\ S{\isacharprime}} es un subconjunto finito de 
    \isa{B} tal que o bien \isa{{\isacharbraceleft}x{\isacharbraceright}\ {\isasymunion}\ S\ {\isacharequal}\ {\isacharbraceleft}a{\isacharbraceright}\ {\isasymunion}\ S{\isacharprime}{\isacharprime}} o bien \isa{{\isacharbraceleft}x{\isacharbraceright}\ {\isasymunion}\ S\ {\isacharequal}\ S{\isacharprime}{\isacharprime}}. Por otro lado, suponiendo que 
    \isa{S\ {\isacharequal}\ S{\isacharprime}}, se deduce análogamente que \isa{{\isacharbraceleft}x{\isacharbraceright}\ {\isasymunion}\ S\ {\isacharequal}\ {\isacharbraceleft}a{\isacharbraceright}\ {\isasymunion}\ S{\isacharprime}} pues se tiene que \isa{x\ {\isacharequal}\ a}, llegando
    a la misma conclusión.

    Por otra parte, supongamos que \isa{x\ {\isasymin}\ B}. En este caso, el conjunto \isa{S{\isacharprime}{\isacharprime}} que verifica el
    resultado es el conjunto \isa{{\isacharbraceleft}x{\isacharbraceright}\ {\isasymunion}\ S{\isacharprime}}. Observemos que se trata de un subconjunto finito de \isa{B} ya 
    que \isa{S{\isacharprime}} es un subconjunto finito de \isa{B} y \isa{x\ {\isasymin}\ B}. Además, en efecto si \isa{S\ {\isacharequal}\ {\isacharbraceleft}a{\isacharbraceright}\ {\isasymunion}\ S{\isacharprime}}, se 
    deduce que \isa{{\isacharbraceleft}x{\isacharbraceright}\ {\isasymunion}\ S\ {\isacharequal}\ {\isacharbraceleft}x{\isacharcomma}a{\isacharbraceright}\ {\isasymunion}\ S{\isacharprime}\ {\isacharequal}\ {\isacharbraceleft}a{\isacharbraceright}\ {\isasymunion}\ S{\isacharprime}{\isacharprime}}, luego cumple que o bien\\ \isa{{\isacharbraceleft}x{\isacharbraceright}\ {\isasymunion}\ S\ {\isacharequal}\ {\isacharbraceleft}a{\isacharbraceright}\ {\isasymunion}\ S{\isacharprime}{\isacharprime}} o 
    bien \isa{{\isacharbraceleft}x{\isacharbraceright}\ {\isasymunion}\ S\ {\isacharequal}\ S{\isacharprime}{\isacharprime}}. Por otro lado, en el caso en que \isa{S\ {\isacharequal}\ S{\isacharprime}}, es claro que \isa{{\isacharbraceleft}x{\isacharbraceright}\ {\isasymunion}\ S\ {\isacharequal}\ S{\isacharprime}{\isacharprime}} 
    por la elección de \isa{S{\isacharprime}{\isacharprime}}, llegando la misma conclusión.
  \end{demostracion}

  Procedamos con la prueba detallada y formalización en Isabelle. Para ello, hemos utilizado el
  siguiente lema auxiliar.%
\end{isamarkuptext}\isamarkuptrue%
\isacommand{lemma}\isamarkupfalse%
\ subexI\ {\isacharbrackleft}intro{\isacharbrackright}{\isacharcolon}\ {\isachardoublequoteopen}P\ A\ {\isasymLongrightarrow}\ A\ {\isasymsubseteq}\ B\ {\isasymLongrightarrow}\ {\isasymexists}A{\isasymsubseteq}B{\isachardot}\ P\ A{\isachardoublequoteclose}\isanewline
%
\isadelimproof
\ \ %
\endisadelimproof
%
\isatagproof
\isacommand{by}\isamarkupfalse%
\ blast%
\endisatagproof
{\isafoldproof}%
%
\isadelimproof
%
\endisadelimproof
%
\begin{isamarkuptext}%
De este modo, probemos detalladamente el resultado.%
\end{isamarkuptext}\isamarkuptrue%
\isacommand{lemma}\isamarkupfalse%
\ finite{\isacharunderscore}subset{\isacharunderscore}insert{\isadigit{1}}{\isacharcolon}\isanewline
\ \ {\isachardoublequoteopen}{\isasymlbrakk}finite\ S{\isacharsemicolon}\ S\ {\isasymsubseteq}\ {\isacharbraceleft}a{\isacharbraceright}\ {\isasymunion}\ B\ {\isasymrbrakk}\ {\isasymLongrightarrow}\isanewline
\ \ \ \ \ {\isasymexists}S{\isacharprime}\ {\isasymsubseteq}\ B{\isachardot}\ finite\ S{\isacharprime}\ {\isasymand}\ {\isacharparenleft}S\ {\isacharequal}\ {\isacharbraceleft}a{\isacharbraceright}\ {\isasymunion}\ S{\isacharprime}\ {\isasymor}\ S\ {\isacharequal}\ S{\isacharprime}{\isacharparenright}{\isachardoublequoteclose}\isanewline
%
\isadelimproof
%
\endisadelimproof
%
\isatagproof
\isacommand{proof}\isamarkupfalse%
\ {\isacharparenleft}induct\ rule{\isacharcolon}\ finite{\isacharunderscore}induct{\isacharparenright}\isanewline
\ \ \isacommand{case}\isamarkupfalse%
\ empty\isanewline
\ \ \isacommand{have}\isamarkupfalse%
\ {\isachardoublequoteopen}{\isacharbraceleft}{\isacharbraceright}\ {\isasymsubseteq}\ B{\isachardoublequoteclose}\isanewline
\ \ \ \ \isacommand{by}\isamarkupfalse%
\ {\isacharparenleft}rule\ empty{\isacharunderscore}subsetI{\isacharparenright}\isanewline
\ \ \isacommand{have}\isamarkupfalse%
\ {\isadigit{1}}{\isacharcolon}{\isachardoublequoteopen}finite\ {\isacharbraceleft}{\isacharbraceright}{\isachardoublequoteclose}\isanewline
\ \ \ \ \isacommand{by}\isamarkupfalse%
\ {\isacharparenleft}simp\ only{\isacharcolon}\ finite{\isachardot}emptyI{\isacharparenright}\isanewline
\ \ \isacommand{have}\isamarkupfalse%
\ {\isachardoublequoteopen}{\isacharbraceleft}{\isacharbraceright}\ {\isacharequal}\ {\isacharbraceleft}{\isacharbraceright}{\isachardoublequoteclose}\isanewline
\ \ \ \ \isacommand{by}\isamarkupfalse%
\ {\isacharparenleft}simp\ only{\isacharcolon}\ simp{\isacharunderscore}thms{\isacharparenleft}{\isadigit{6}}{\isacharparenright}{\isacharparenright}\isanewline
\ \ \isacommand{then}\isamarkupfalse%
\ \isacommand{have}\isamarkupfalse%
\ {\isadigit{2}}{\isacharcolon}{\isachardoublequoteopen}{\isacharbraceleft}{\isacharbraceright}\ {\isacharequal}\ {\isacharbraceleft}a{\isacharbraceright}\ {\isasymunion}\ {\isacharbraceleft}{\isacharbraceright}\ {\isasymor}\ {\isacharbraceleft}{\isacharbraceright}\ {\isacharequal}\ {\isacharbraceleft}{\isacharbraceright}{\isachardoublequoteclose}\isanewline
\ \ \ \ \isacommand{by}\isamarkupfalse%
\ {\isacharparenleft}rule\ disjI{\isadigit{2}}{\isacharparenright}\isanewline
\ \ \isacommand{have}\isamarkupfalse%
\ {\isachardoublequoteopen}finite\ {\isacharbraceleft}{\isacharbraceright}\ {\isasymand}\ {\isacharparenleft}{\isacharbraceleft}{\isacharbraceright}\ {\isacharequal}\ {\isacharbraceleft}a{\isacharbraceright}\ {\isasymunion}\ {\isacharbraceleft}{\isacharbraceright}\ {\isasymor}\ {\isacharbraceleft}{\isacharbraceright}\ {\isacharequal}\ {\isacharbraceleft}{\isacharbraceright}{\isacharparenright}{\isachardoublequoteclose}\isanewline
\ \ \ \ \isacommand{using}\isamarkupfalse%
\ {\isadigit{1}}\ {\isadigit{2}}\ \isacommand{by}\isamarkupfalse%
\ {\isacharparenleft}rule\ conjI{\isacharparenright}\isanewline
\ \ \isacommand{thus}\isamarkupfalse%
\ {\isachardoublequoteopen}{\isasymexists}S{\isacharprime}\ {\isasymsubseteq}\ B{\isachardot}\ finite\ S{\isacharprime}\ {\isasymand}\ {\isacharparenleft}{\isacharbraceleft}{\isacharbraceright}\ {\isacharequal}\ {\isacharbraceleft}a{\isacharbraceright}\ {\isasymunion}\ S{\isacharprime}\ {\isasymor}\ {\isacharbraceleft}{\isacharbraceright}\ {\isacharequal}\ S{\isacharprime}{\isacharparenright}{\isachardoublequoteclose}\isanewline
\ \ \ \ \isacommand{using}\isamarkupfalse%
\ {\isacartoucheopen}{\isacharbraceleft}{\isacharbraceright}\ {\isasymsubseteq}\ B{\isacartoucheclose}\ \isacommand{by}\isamarkupfalse%
\ {\isacharparenleft}rule\ subexI{\isacharparenright}\isanewline
\isacommand{next}\isamarkupfalse%
\isanewline
\ \ \isacommand{case}\isamarkupfalse%
\ {\isacharparenleft}insert\ x\ S{\isacharparenright}\isanewline
\ \ \isacommand{assume}\isamarkupfalse%
\ {\isachardoublequoteopen}finite\ S{\isachardoublequoteclose}\isanewline
\ \ \isacommand{assume}\isamarkupfalse%
\ {\isachardoublequoteopen}x\ {\isasymnotin}\ S{\isachardoublequoteclose}\isanewline
\ \ \isacommand{assume}\isamarkupfalse%
\ HI{\isacharcolon}{\isachardoublequoteopen}S\ {\isasymsubseteq}\ {\isacharbraceleft}a{\isacharbraceright}\ {\isasymunion}\ B\ {\isasymLongrightarrow}\ {\isasymexists}S{\isacharprime}{\isasymsubseteq}B{\isachardot}\ finite\ S{\isacharprime}\ {\isasymand}\ {\isacharparenleft}S\ {\isacharequal}\ {\isacharbraceleft}a{\isacharbraceright}\ {\isasymunion}\ S{\isacharprime}\ {\isasymor}\ S\ {\isacharequal}\ S{\isacharprime}{\isacharparenright}{\isachardoublequoteclose}\isanewline
\ \ \isacommand{show}\isamarkupfalse%
\ {\isachardoublequoteopen}insert\ x\ S\ {\isasymsubseteq}\ {\isacharbraceleft}a{\isacharbraceright}\ {\isasymunion}\ B\ {\isasymLongrightarrow}\ {\isasymexists}S{\isacharprime}{\isacharprime}{\isasymsubseteq}B{\isachardot}\ finite\ S{\isacharprime}{\isacharprime}\ {\isasymand}\ {\isacharparenleft}insert\ x\ S\ {\isacharequal}\ {\isacharbraceleft}a{\isacharbraceright}\ {\isasymunion}\ S{\isacharprime}{\isacharprime}\ {\isasymor}\ insert\ x\ S\ {\isacharequal}\ S{\isacharprime}{\isacharprime}{\isacharparenright}{\isachardoublequoteclose}\isanewline
\ \ \isacommand{proof}\isamarkupfalse%
\ {\isacharminus}\isanewline
\ \ \ \ \isacommand{assume}\isamarkupfalse%
\ {\isachardoublequoteopen}insert\ x\ S\ {\isasymsubseteq}\ {\isacharbraceleft}a{\isacharbraceright}\ {\isasymunion}\ B{\isachardoublequoteclose}\ \isanewline
\ \ \ \ \isacommand{then}\isamarkupfalse%
\ \isacommand{have}\isamarkupfalse%
\ C{\isacharcolon}{\isachardoublequoteopen}x\ {\isasymin}\ {\isacharbraceleft}a{\isacharbraceright}\ {\isasymunion}\ B\ {\isasymand}\ S\ {\isasymsubseteq}\ {\isacharbraceleft}a{\isacharbraceright}\ {\isasymunion}\ B{\isachardoublequoteclose}\isanewline
\ \ \ \ \ \ \isacommand{by}\isamarkupfalse%
\ {\isacharparenleft}simp\ only{\isacharcolon}\ insert{\isacharunderscore}subset{\isacharparenright}\isanewline
\ \ \ \ \isacommand{then}\isamarkupfalse%
\ \isacommand{have}\isamarkupfalse%
\ {\isachardoublequoteopen}S\ {\isasymsubseteq}\ {\isacharbraceleft}a{\isacharbraceright}\ {\isasymunion}\ B{\isachardoublequoteclose}\isanewline
\ \ \ \ \ \ \isacommand{by}\isamarkupfalse%
\ {\isacharparenleft}rule\ conjunct{\isadigit{2}}{\isacharparenright}\isanewline
\ \ \ \ \isacommand{have}\isamarkupfalse%
\ Ex{\isadigit{1}}{\isacharcolon}{\isachardoublequoteopen}{\isasymexists}S{\isacharprime}{\isasymsubseteq}B{\isachardot}\ finite\ S{\isacharprime}\ {\isasymand}\ {\isacharparenleft}S\ {\isacharequal}\ {\isacharbraceleft}a{\isacharbraceright}\ {\isasymunion}\ S{\isacharprime}\ {\isasymor}\ S\ {\isacharequal}\ S{\isacharprime}{\isacharparenright}{\isachardoublequoteclose}\isanewline
\ \ \ \ \ \ \isacommand{using}\isamarkupfalse%
\ {\isacartoucheopen}S\ {\isasymsubseteq}\ {\isacharbraceleft}a{\isacharbraceright}\ {\isasymunion}\ B{\isacartoucheclose}\ \isacommand{by}\isamarkupfalse%
\ {\isacharparenleft}rule\ HI{\isacharparenright}\isanewline
\ \ \ \ \isacommand{obtain}\isamarkupfalse%
\ S{\isacharprime}\ \isakeyword{where}\ {\isachardoublequoteopen}S{\isacharprime}\ {\isasymsubseteq}\ B{\isachardoublequoteclose}\ \isakeyword{and}\ C{\isadigit{1}}{\isacharcolon}{\isachardoublequoteopen}finite\ S{\isacharprime}\ {\isasymand}\ {\isacharparenleft}S\ {\isacharequal}\ {\isacharbraceleft}a{\isacharbraceright}\ {\isasymunion}\ S{\isacharprime}\ {\isasymor}\ S\ {\isacharequal}\ S{\isacharprime}{\isacharparenright}{\isachardoublequoteclose}\isanewline
\ \ \ \ \ \ \isacommand{using}\isamarkupfalse%
\ Ex{\isadigit{1}}\ \isacommand{by}\isamarkupfalse%
\ {\isacharparenleft}rule\ subexE{\isacharparenright}\isanewline
\ \ \ \ \isacommand{have}\isamarkupfalse%
\ {\isachardoublequoteopen}finite\ S{\isacharprime}{\isachardoublequoteclose}\isanewline
\ \ \ \ \ \ \isacommand{using}\isamarkupfalse%
\ C{\isadigit{1}}\ \isacommand{by}\isamarkupfalse%
\ {\isacharparenleft}rule\ conjunct{\isadigit{1}}{\isacharparenright}\isanewline
\ \ \ \ \isacommand{then}\isamarkupfalse%
\ \isacommand{have}\isamarkupfalse%
\ {\isachardoublequoteopen}finite\ {\isacharparenleft}insert\ x\ S{\isacharprime}{\isacharparenright}{\isachardoublequoteclose}\isanewline
\ \ \ \ \ \ \isacommand{by}\isamarkupfalse%
\ {\isacharparenleft}simp\ only{\isacharcolon}\ finite{\isachardot}insertI{\isacharparenright}\isanewline
\ \ \ \ \isacommand{have}\isamarkupfalse%
\ {\isachardoublequoteopen}x\ {\isasymin}\ {\isacharbraceleft}a{\isacharbraceright}\ {\isasymunion}\ B{\isachardoublequoteclose}\isanewline
\ \ \ \ \ \ \isacommand{using}\isamarkupfalse%
\ C\ \isacommand{by}\isamarkupfalse%
\ {\isacharparenleft}rule\ conjunct{\isadigit{1}}{\isacharparenright}\isanewline
\ \ \ \ \isacommand{then}\isamarkupfalse%
\ \isacommand{have}\isamarkupfalse%
\ {\isachardoublequoteopen}x\ {\isasymin}\ {\isacharbraceleft}a{\isacharbraceright}\ {\isasymor}\ x\ {\isasymin}\ B{\isachardoublequoteclose}\isanewline
\ \ \ \ \ \ \isacommand{by}\isamarkupfalse%
\ {\isacharparenleft}simp\ only{\isacharcolon}\ Un{\isacharunderscore}iff{\isacharparenright}\isanewline
\ \ \ \ \isacommand{then}\isamarkupfalse%
\ \isacommand{have}\isamarkupfalse%
\ {\isachardoublequoteopen}x\ {\isacharequal}\ a\ {\isasymor}\ x\ {\isasymin}\ B{\isachardoublequoteclose}\isanewline
\ \ \ \ \ \ \isacommand{by}\isamarkupfalse%
\ {\isacharparenleft}simp\ only{\isacharcolon}\ singleton{\isacharunderscore}iff{\isacharparenright}\isanewline
\ \ \ \ \isacommand{thus}\isamarkupfalse%
\ {\isachardoublequoteopen}{\isasymexists}S{\isacharprime}{\isacharprime}{\isasymsubseteq}B{\isachardot}\ finite\ S{\isacharprime}{\isacharprime}\ {\isasymand}\ {\isacharparenleft}insert\ x\ S\ {\isacharequal}\ {\isacharbraceleft}a{\isacharbraceright}\ {\isasymunion}\ S{\isacharprime}{\isacharprime}\ {\isasymor}\ insert\ x\ S\ {\isacharequal}\ S{\isacharprime}{\isacharprime}{\isacharparenright}{\isachardoublequoteclose}\isanewline
\ \ \ \ \isacommand{proof}\isamarkupfalse%
\ {\isacharparenleft}rule\ disjE{\isacharparenright}\isanewline
\ \ \ \ \ \ \isacommand{assume}\isamarkupfalse%
\ {\isachardoublequoteopen}x\ {\isacharequal}\ a{\isachardoublequoteclose}\isanewline
\ \ \ \ \ \ \isacommand{have}\isamarkupfalse%
\ {\isachardoublequoteopen}S\ {\isacharequal}\ {\isacharbraceleft}a{\isacharbraceright}\ {\isasymunion}\ S{\isacharprime}\ {\isasymor}\ S\ {\isacharequal}\ S{\isacharprime}{\isachardoublequoteclose}\isanewline
\ \ \ \ \ \ \ \ \isacommand{using}\isamarkupfalse%
\ C{\isadigit{1}}\ \isacommand{by}\isamarkupfalse%
\ {\isacharparenleft}rule\ conjunct{\isadigit{2}}{\isacharparenright}\isanewline
\ \ \ \ \ \ \isacommand{thus}\isamarkupfalse%
\ {\isacharquery}thesis\isanewline
\ \ \ \ \ \ \isacommand{proof}\isamarkupfalse%
\ {\isacharparenleft}rule\ disjE{\isacharparenright}\isanewline
\ \ \ \ \ \ \ \ \isacommand{assume}\isamarkupfalse%
\ {\isachardoublequoteopen}S\ {\isacharequal}\ {\isacharbraceleft}a{\isacharbraceright}\ {\isasymunion}\ S{\isacharprime}{\isachardoublequoteclose}\isanewline
\ \ \ \ \ \ \ \ \isacommand{have}\isamarkupfalse%
\ {\isachardoublequoteopen}x\ {\isasymin}\ {\isacharbraceleft}a{\isacharbraceright}{\isachardoublequoteclose}\isanewline
\ \ \ \ \ \ \ \ \ \ \isacommand{using}\isamarkupfalse%
\ {\isacartoucheopen}x\ {\isacharequal}\ a{\isacartoucheclose}\ \isacommand{by}\isamarkupfalse%
\ {\isacharparenleft}simp\ only{\isacharcolon}\ singleton{\isacharunderscore}iff{\isacharparenright}\isanewline
\ \ \ \ \ \ \ \ \isacommand{then}\isamarkupfalse%
\ \isacommand{have}\isamarkupfalse%
\ {\isachardoublequoteopen}x\ {\isasymin}\ {\isacharbraceleft}a{\isacharbraceright}\ {\isasymunion}\ S{\isacharprime}{\isachardoublequoteclose}\ \isanewline
\ \ \ \ \ \ \ \ \ \ \isacommand{by}\isamarkupfalse%
\ {\isacharparenleft}simp\ only{\isacharcolon}\ UnI{\isadigit{1}}{\isacharparenright}\isanewline
\ \ \ \ \ \ \ \ \isacommand{then}\isamarkupfalse%
\ \isacommand{have}\isamarkupfalse%
\ {\isachardoublequoteopen}insert\ x\ {\isacharparenleft}{\isacharbraceleft}a{\isacharbraceright}\ {\isasymunion}\ S{\isacharprime}{\isacharparenright}\ {\isacharequal}\ {\isacharbraceleft}a{\isacharbraceright}\ {\isasymunion}\ S{\isacharprime}{\isachardoublequoteclose}\isanewline
\ \ \ \ \ \ \ \ \ \ \isacommand{by}\isamarkupfalse%
\ {\isacharparenleft}rule\ insert{\isacharunderscore}absorb{\isacharparenright}\isanewline
\ \ \ \ \ \ \ \ \isacommand{have}\isamarkupfalse%
\ {\isachardoublequoteopen}insert\ x\ S\ {\isacharequal}\ insert\ x\ {\isacharparenleft}{\isacharbraceleft}a{\isacharbraceright}\ {\isasymunion}\ S{\isacharprime}{\isacharparenright}{\isachardoublequoteclose}\isanewline
\ \ \ \ \ \ \ \ \ \ \isacommand{by}\isamarkupfalse%
\ {\isacharparenleft}simp\ only{\isacharcolon}\ {\isacartoucheopen}S\ {\isacharequal}\ {\isacharbraceleft}a{\isacharbraceright}\ {\isasymunion}\ S{\isacharprime}{\isacartoucheclose}{\isacharparenright}\isanewline
\ \ \ \ \ \ \ \ \isacommand{then}\isamarkupfalse%
\ \isacommand{have}\isamarkupfalse%
\ {\isachardoublequoteopen}insert\ x\ S\ {\isacharequal}\ {\isacharbraceleft}a{\isacharbraceright}\ {\isasymunion}\ S{\isacharprime}{\isachardoublequoteclose}\isanewline
\ \ \ \ \ \ \ \ \ \ \isacommand{by}\isamarkupfalse%
\ {\isacharparenleft}simp\ only{\isacharcolon}\ {\isacartoucheopen}insert\ x\ {\isacharparenleft}{\isacharbraceleft}a{\isacharbraceright}\ {\isasymunion}\ S{\isacharprime}{\isacharparenright}\ {\isacharequal}\ {\isacharbraceleft}a{\isacharbraceright}\ {\isasymunion}\ S{\isacharprime}{\isacartoucheclose}{\isacharparenright}\isanewline
\ \ \ \ \ \ \ \ \isacommand{then}\isamarkupfalse%
\ \isacommand{have}\isamarkupfalse%
\ {\isadigit{1}}{\isacharcolon}{\isachardoublequoteopen}insert\ x\ S\ {\isacharequal}\ {\isacharbraceleft}a{\isacharbraceright}\ {\isasymunion}\ S{\isacharprime}\ {\isasymor}\ insert\ x\ S\ {\isacharequal}\ S{\isacharprime}{\isachardoublequoteclose}\isanewline
\ \ \ \ \ \ \ \ \ \ \isacommand{by}\isamarkupfalse%
\ {\isacharparenleft}rule\ disjI{\isadigit{1}}{\isacharparenright}\isanewline
\ \ \ \ \ \ \ \ \isacommand{have}\isamarkupfalse%
\ {\isachardoublequoteopen}finite\ S{\isacharprime}\ {\isasymand}\ {\isacharparenleft}insert\ x\ S\ {\isacharequal}\ {\isacharbraceleft}a{\isacharbraceright}\ {\isasymunion}\ S{\isacharprime}\ {\isasymor}\ insert\ x\ S\ {\isacharequal}\ S{\isacharprime}{\isacharparenright}{\isachardoublequoteclose}\isanewline
\ \ \ \ \ \ \ \ \ \ \isacommand{using}\isamarkupfalse%
\ {\isacartoucheopen}finite\ S{\isacharprime}{\isacartoucheclose}\ {\isadigit{1}}\ \isacommand{by}\isamarkupfalse%
\ {\isacharparenleft}rule\ conjI{\isacharparenright}\isanewline
\ \ \ \ \ \ \ \ \isacommand{thus}\isamarkupfalse%
\ {\isacharquery}thesis\isanewline
\ \ \ \ \ \ \ \ \ \ \isacommand{using}\isamarkupfalse%
\ {\isacartoucheopen}S{\isacharprime}\ {\isasymsubseteq}\ B{\isacartoucheclose}\ \isacommand{by}\isamarkupfalse%
\ {\isacharparenleft}rule\ subexI{\isacharparenright}\isanewline
\ \ \ \ \ \ \isacommand{next}\isamarkupfalse%
\isanewline
\ \ \ \ \ \ \ \ \isacommand{assume}\isamarkupfalse%
\ {\isachardoublequoteopen}S\ {\isacharequal}\ S{\isacharprime}{\isachardoublequoteclose}\isanewline
\ \ \ \ \ \ \ \ \isacommand{have}\isamarkupfalse%
\ {\isachardoublequoteopen}insert\ x\ S\ {\isacharequal}\ {\isacharbraceleft}x{\isacharbraceright}\ {\isasymunion}\ S{\isachardoublequoteclose}\isanewline
\ \ \ \ \ \ \ \ \ \ \isacommand{by}\isamarkupfalse%
\ {\isacharparenleft}rule\ insert{\isacharunderscore}is{\isacharunderscore}Un{\isacharparenright}\isanewline
\ \ \ \ \ \ \ \ \isacommand{then}\isamarkupfalse%
\ \isacommand{have}\isamarkupfalse%
\ {\isachardoublequoteopen}insert\ x\ S\ {\isacharequal}\ {\isacharbraceleft}a{\isacharbraceright}\ {\isasymunion}\ S{\isacharprime}{\isachardoublequoteclose}\isanewline
\ \ \ \ \ \ \ \ \ \ \isacommand{by}\isamarkupfalse%
\ {\isacharparenleft}simp\ only{\isacharcolon}\ {\isacartoucheopen}x\ {\isacharequal}\ a{\isacartoucheclose}\ {\isacartoucheopen}S\ {\isacharequal}\ S{\isacharprime}{\isacartoucheclose}{\isacharparenright}\isanewline
\ \ \ \ \ \ \ \ \isacommand{then}\isamarkupfalse%
\ \isacommand{have}\isamarkupfalse%
\ {\isadigit{1}}{\isacharcolon}{\isachardoublequoteopen}insert\ x\ S\ {\isacharequal}\ {\isacharbraceleft}a{\isacharbraceright}\ {\isasymunion}\ S{\isacharprime}\ {\isasymor}\ insert\ x\ S\ {\isacharequal}\ S{\isacharprime}{\isachardoublequoteclose}\isanewline
\ \ \ \ \ \ \ \ \ \ \isacommand{by}\isamarkupfalse%
\ {\isacharparenleft}rule\ disjI{\isadigit{1}}{\isacharparenright}\isanewline
\ \ \ \ \ \ \ \ \isacommand{have}\isamarkupfalse%
\ {\isachardoublequoteopen}finite\ S{\isacharprime}\ {\isasymand}\ {\isacharparenleft}insert\ x\ S\ {\isacharequal}\ {\isacharbraceleft}a{\isacharbraceright}\ {\isasymunion}\ S{\isacharprime}\ {\isasymor}\ insert\ x\ S\ {\isacharequal}\ S{\isacharprime}{\isacharparenright}{\isachardoublequoteclose}\isanewline
\ \ \ \ \ \ \ \ \ \ \isacommand{using}\isamarkupfalse%
\ {\isacartoucheopen}finite\ S{\isacharprime}{\isacartoucheclose}\ {\isadigit{1}}\ \isacommand{by}\isamarkupfalse%
\ {\isacharparenleft}rule\ conjI{\isacharparenright}\isanewline
\ \ \ \ \ \ \ \ \isacommand{thus}\isamarkupfalse%
\ {\isacharquery}thesis\isanewline
\ \ \ \ \ \ \ \ \ \ \isacommand{using}\isamarkupfalse%
\ {\isacartoucheopen}S{\isacharprime}\ {\isasymsubseteq}\ B{\isacartoucheclose}\ \isacommand{by}\isamarkupfalse%
\ {\isacharparenleft}rule\ subexI{\isacharparenright}\isanewline
\ \ \ \ \ \ \isacommand{qed}\isamarkupfalse%
\isanewline
\ \ \ \ \isacommand{next}\isamarkupfalse%
\isanewline
\ \ \ \ \ \ \isacommand{assume}\isamarkupfalse%
\ {\isachardoublequoteopen}x\ {\isasymin}\ B{\isachardoublequoteclose}\isanewline
\ \ \ \ \ \ \isacommand{have}\isamarkupfalse%
\ {\isachardoublequoteopen}x\ {\isasymin}\ B\ {\isasymand}\ S{\isacharprime}\ {\isasymsubseteq}\ B{\isachardoublequoteclose}\isanewline
\ \ \ \ \ \ \ \ \isacommand{using}\isamarkupfalse%
\ {\isacartoucheopen}x\ {\isasymin}\ B{\isacartoucheclose}\ {\isacartoucheopen}S{\isacharprime}\ {\isasymsubseteq}\ B{\isacartoucheclose}\ \isacommand{by}\isamarkupfalse%
\ {\isacharparenleft}rule\ conjI{\isacharparenright}\isanewline
\ \ \ \ \ \ \isacommand{then}\isamarkupfalse%
\ \isacommand{have}\isamarkupfalse%
\ {\isachardoublequoteopen}insert\ x\ S{\isacharprime}\ {\isasymsubseteq}\ B{\isachardoublequoteclose}\isanewline
\ \ \ \ \ \ \ \ \isacommand{by}\isamarkupfalse%
\ {\isacharparenleft}simp\ only{\isacharcolon}\ insert{\isacharunderscore}subset{\isacharparenright}\isanewline
\ \ \ \ \ \ \isacommand{have}\isamarkupfalse%
\ {\isachardoublequoteopen}finite\ {\isacharparenleft}insert\ x\ S{\isacharprime}{\isacharparenright}{\isachardoublequoteclose}\isanewline
\ \ \ \ \ \ \ \ \isacommand{using}\isamarkupfalse%
\ {\isacartoucheopen}finite\ S{\isacharprime}{\isacartoucheclose}\ \isacommand{by}\isamarkupfalse%
\ {\isacharparenleft}simp\ only{\isacharcolon}\ finite{\isachardot}insertI{\isacharparenright}\isanewline
\ \ \ \ \ \ \isacommand{have}\isamarkupfalse%
\ {\isachardoublequoteopen}S\ {\isacharequal}\ {\isacharbraceleft}a{\isacharbraceright}\ {\isasymunion}\ S{\isacharprime}\ {\isasymor}\ S\ {\isacharequal}\ S{\isacharprime}{\isachardoublequoteclose}\isanewline
\ \ \ \ \ \ \ \ \isacommand{using}\isamarkupfalse%
\ C{\isadigit{1}}\ \isacommand{by}\isamarkupfalse%
\ {\isacharparenleft}rule\ conjunct{\isadigit{2}}{\isacharparenright}\isanewline
\ \ \ \ \ \ \isacommand{thus}\isamarkupfalse%
\ {\isacharquery}thesis\isanewline
\ \ \ \ \ \ \isacommand{proof}\isamarkupfalse%
\ {\isacharparenleft}rule\ disjE{\isacharparenright}\isanewline
\ \ \ \ \ \ \ \ \isacommand{assume}\isamarkupfalse%
\ {\isachardoublequoteopen}S\ {\isacharequal}\ {\isacharbraceleft}a{\isacharbraceright}\ {\isasymunion}\ S{\isacharprime}{\isachardoublequoteclose}\isanewline
\ \ \ \ \ \ \ \ \isacommand{have}\isamarkupfalse%
\ {\isachardoublequoteopen}insert\ x\ S\ {\isacharequal}\ insert\ x\ {\isacharparenleft}{\isacharbraceleft}a{\isacharbraceright}\ {\isasymunion}\ S{\isacharprime}{\isacharparenright}{\isachardoublequoteclose}\isanewline
\ \ \ \ \ \ \ \ \ \ \isacommand{by}\isamarkupfalse%
\ {\isacharparenleft}simp\ only{\isacharcolon}\ {\isacartoucheopen}S\ {\isacharequal}\ {\isacharbraceleft}a{\isacharbraceright}\ {\isasymunion}\ S{\isacharprime}{\isacartoucheclose}{\isacharparenright}\isanewline
\ \ \ \ \ \ \ \ \isacommand{then}\isamarkupfalse%
\ \isacommand{have}\isamarkupfalse%
\ {\isachardoublequoteopen}insert\ x\ S\ {\isacharequal}\ {\isacharbraceleft}a{\isacharbraceright}\ {\isasymunion}\ {\isacharparenleft}insert\ x\ S{\isacharprime}{\isacharparenright}{\isachardoublequoteclose}\isanewline
\ \ \ \ \ \ \ \ \ \ \isacommand{by}\isamarkupfalse%
\ blast\isanewline
\ \ \ \ \ \ \ \ \isacommand{then}\isamarkupfalse%
\ \isacommand{have}\isamarkupfalse%
\ {\isadigit{1}}{\isacharcolon}{\isachardoublequoteopen}insert\ x\ S\ {\isacharequal}\ {\isacharbraceleft}a{\isacharbraceright}\ {\isasymunion}\ {\isacharparenleft}insert\ x\ S{\isacharprime}{\isacharparenright}\ {\isasymor}\ insert\ x\ S\ {\isacharequal}\ insert\ x\ S{\isacharprime}{\isachardoublequoteclose}\isanewline
\ \ \ \ \ \ \ \ \ \ \isacommand{by}\isamarkupfalse%
\ {\isacharparenleft}rule\ disjI{\isadigit{1}}{\isacharparenright}\isanewline
\ \ \ \ \ \ \ \ \isacommand{have}\isamarkupfalse%
\ {\isachardoublequoteopen}finite\ {\isacharparenleft}insert\ x\ S{\isacharprime}{\isacharparenright}\ {\isasymand}\ {\isacharparenleft}insert\ x\ S\ {\isacharequal}\ {\isacharbraceleft}a{\isacharbraceright}\ {\isasymunion}\ {\isacharparenleft}insert\ x\ S{\isacharprime}{\isacharparenright}\ {\isasymor}\ insert\ x\ S\ {\isacharequal}\ insert\ x\ S{\isacharprime}{\isacharparenright}{\isachardoublequoteclose}\isanewline
\ \ \ \ \ \ \ \ \ \ \isacommand{using}\isamarkupfalse%
\ {\isacartoucheopen}finite\ {\isacharparenleft}insert\ x\ S{\isacharprime}{\isacharparenright}{\isacartoucheclose}\ {\isadigit{1}}\ \isacommand{by}\isamarkupfalse%
\ {\isacharparenleft}rule\ conjI{\isacharparenright}\isanewline
\ \ \ \ \ \ \ \ \isacommand{thus}\isamarkupfalse%
\ {\isacharquery}thesis\isanewline
\ \ \ \ \ \ \ \ \ \ \isacommand{using}\isamarkupfalse%
\ {\isacartoucheopen}insert\ x\ S{\isacharprime}\ {\isasymsubseteq}\ B{\isacartoucheclose}\ \isacommand{by}\isamarkupfalse%
\ {\isacharparenleft}rule\ subexI{\isacharparenright}\isanewline
\ \ \ \ \ \ \isacommand{next}\isamarkupfalse%
\isanewline
\ \ \ \ \ \ \ \ \isacommand{assume}\isamarkupfalse%
\ {\isachardoublequoteopen}S\ {\isacharequal}\ S{\isacharprime}{\isachardoublequoteclose}\isanewline
\ \ \ \ \ \ \ \ \isacommand{have}\isamarkupfalse%
\ {\isachardoublequoteopen}insert\ x\ S\ {\isacharequal}\ insert\ x\ S{\isacharprime}{\isachardoublequoteclose}\isanewline
\ \ \ \ \ \ \ \ \ \ \isacommand{by}\isamarkupfalse%
\ {\isacharparenleft}simp\ only{\isacharcolon}\ {\isacartoucheopen}S\ {\isacharequal}\ S{\isacharprime}{\isacartoucheclose}{\isacharparenright}\isanewline
\ \ \ \ \ \ \ \ \isacommand{then}\isamarkupfalse%
\ \isacommand{have}\isamarkupfalse%
\ {\isadigit{1}}{\isacharcolon}{\isachardoublequoteopen}insert\ x\ S\ {\isacharequal}\ {\isacharbraceleft}a{\isacharbraceright}\ {\isasymunion}\ {\isacharparenleft}insert\ x\ S{\isacharprime}{\isacharparenright}\ {\isasymor}\ insert\ x\ S\ {\isacharequal}\ insert\ x\ S{\isacharprime}{\isachardoublequoteclose}\isanewline
\ \ \ \ \ \ \ \ \ \ \isacommand{by}\isamarkupfalse%
\ {\isacharparenleft}rule\ disjI{\isadigit{2}}{\isacharparenright}\isanewline
\ \ \ \ \ \ \ \ \isacommand{have}\isamarkupfalse%
\ {\isachardoublequoteopen}finite\ {\isacharparenleft}insert\ x\ S{\isacharprime}{\isacharparenright}\ {\isasymand}\ {\isacharparenleft}insert\ x\ S\ {\isacharequal}\ {\isacharbraceleft}a{\isacharbraceright}\ {\isasymunion}\ {\isacharparenleft}insert\ x\ S{\isacharprime}{\isacharparenright}\ {\isasymor}\ insert\ x\ S\ {\isacharequal}\ insert\ x\ S{\isacharprime}{\isacharparenright}{\isachardoublequoteclose}\isanewline
\ \ \ \ \ \ \ \ \ \ \isacommand{using}\isamarkupfalse%
\ {\isacartoucheopen}finite\ {\isacharparenleft}insert\ x\ S{\isacharprime}{\isacharparenright}{\isacartoucheclose}\ {\isadigit{1}}\ \isacommand{by}\isamarkupfalse%
\ {\isacharparenleft}rule\ conjI{\isacharparenright}\isanewline
\ \ \ \ \ \ \ \ \isacommand{thus}\isamarkupfalse%
\ {\isacharquery}thesis\isanewline
\ \ \ \ \ \ \ \ \ \ \isacommand{using}\isamarkupfalse%
\ {\isacartoucheopen}insert\ x\ S{\isacharprime}\ {\isasymsubseteq}\ B{\isacartoucheclose}\ \isacommand{by}\isamarkupfalse%
\ {\isacharparenleft}rule\ subexI{\isacharparenright}\isanewline
\ \ \ \ \ \ \isacommand{qed}\isamarkupfalse%
\isanewline
\ \ \ \ \isacommand{qed}\isamarkupfalse%
\isanewline
\ \ \isacommand{qed}\isamarkupfalse%
\isanewline
\isacommand{qed}\isamarkupfalse%
%
\endisatagproof
{\isafoldproof}%
%
\isadelimproof
%
\endisadelimproof
%
\begin{isamarkuptext}%
El segundo resultado sobre subconjuntos finitos es consecuencia del anterior.

\begin{lema}
  Sea un conjunto de la forma \isa{{\isacharbraceleft}a{\isacharcomma}b{\isacharbraceright}\ {\isasymunion}\ B} y \isa{S} un subconjunto finito suyo. Entonces, existe un
  subconjunto finito \isa{S{\isacharprime}} de \isa{B} tal que se cumple \isa{S\ {\isacharequal}\ {\isacharbraceleft}a{\isacharcomma}b{\isacharbraceright}\ {\isasymunion}\ S{\isacharprime}}, \isa{S\ {\isacharequal}\ {\isacharbraceleft}a{\isacharbraceright}\ {\isasymunion}\ S{\isacharprime}},\\ \isa{S\ {\isacharequal}\ {\isacharbraceleft}b{\isacharbraceright}\ {\isasymunion}\ S{\isacharprime}} 
  o \isa{S\ {\isacharequal}\ S{\isacharprime}}.
\end{lema}

\begin{demostracion}
  En particular, \isa{S} es un subconjunto finito de \isa{{\isacharbraceleft}a{\isacharbraceright}\ {\isasymunion}\ {\isacharparenleft}{\isacharbraceleft}b{\isacharbraceright}\ {\isasymunion}\ B{\isacharparenright}} luego, aplicando el lema \isa{{\isadigit{4}}{\isachardot}{\isadigit{3}}{\isachardot}{\isadigit{2}}}
  anterior, podemos hallar un subconjunto finito \isa{S\isactrlsub {\isadigit{1}}} de \isa{{\isacharbraceleft}b{\isacharbraceright}\ {\isasymunion}\ B} tal que o bien \isa{S\ {\isacharequal}\ {\isacharbraceleft}a{\isacharbraceright}\ {\isasymunion}\ S\isactrlsub {\isadigit{1}}} o 
  bien \isa{S\ {\isacharequal}\ S\isactrlsub {\isadigit{1}}}. A su vez, podemos aplicar dicho resultado para el subconjunto finito \isa{S\isactrlsub {\isadigit{1}}} de 
  \isa{{\isacharbraceleft}b{\isacharbraceright}\ {\isasymunion}\ B}, obteniendo un subconjunto finito \isa{S\isactrlsub {\isadigit{2}}} de \isa{B} tal que o bien \isa{S\isactrlsub {\isadigit{1}}\ {\isacharequal}\ {\isacharbraceleft}b{\isacharbraceright}\ {\isasymunion}\ S\isactrlsub {\isadigit{2}}} o bien 
  \isa{S\isactrlsub {\isadigit{1}}\ {\isacharequal}\ S\isactrlsub {\isadigit{2}}}. Veamos que el lema se verifica en ambas opciones posibles de \isa{S\isactrlsub {\isadigit{1}}} para el conjunto 
  \isa{S{\isacharprime}\ {\isacharequal}\ S\isactrlsub {\isadigit{2}}}.

  En primer lugar, supongamos que \isa{S\isactrlsub {\isadigit{1}}\ {\isacharequal}\ {\isacharbraceleft}b{\isacharbraceright}\ {\isasymunion}\ S\isactrlsub {\isadigit{2}}}. De este modo, se verifica el resultado tanto para
  \isa{S\ {\isacharequal}\ {\isacharbraceleft}a{\isacharbraceright}\ {\isasymunion}\ S\isactrlsub {\isadigit{1}}} como para \isa{S\ {\isacharequal}\ S\isactrlsub {\isadigit{1}}}. En efecto, en la primera opción, por elección de \isa{S\isactrlsub {\isadigit{1}}}, es claro
  que \isa{S\ {\isacharequal}\ {\isacharbraceleft}a{\isacharcomma}b{\isacharbraceright}\ {\isasymunion}\ S\isactrlsub {\isadigit{2}}}. Finalmente, para \isa{S\ {\isacharequal}\ S\isactrlsub {\isadigit{1}}}, obtenemos que \isa{S\ {\isacharequal}\ {\isacharbraceleft}b{\isacharbraceright}\ {\isasymunion}\ S\isactrlsub {\isadigit{2}}}, lo que prueba
  igualmente el lema para \isa{S{\isacharprime}\ {\isacharequal}\ S\isactrlsub {\isadigit{2}}}.

  Por último, supongamos que \isa{S\isactrlsub {\isadigit{1}}\ {\isacharequal}\ S\isactrlsub {\isadigit{2}}}. Análogamente, el resultado es inmediato pues si 
  \isa{S\ {\isacharequal}\ {\isacharbraceleft}a{\isacharbraceright}\ {\isasymunion}\ S\isactrlsub {\isadigit{1}}} obtenemos que \isa{S\ {\isacharequal}\ {\isacharbraceleft}a{\isacharbraceright}\ {\isasymunion}\ S\isactrlsub {\isadigit{2}}}, y si suponemos \isa{S\ {\isacharequal}\ S\isactrlsub {\isadigit{1}}} obtenemos\\ \isa{S\ {\isacharequal}\ S\isactrlsub {\isadigit{2}}}, probando
  así el lema.
\end{demostracion}

  Su formalización y prueba detallada en Isabelle/HOL son las siguientes.%
\end{isamarkuptext}\isamarkuptrue%
\isacommand{lemma}\isamarkupfalse%
\ finite{\isacharunderscore}subset{\isacharunderscore}insert{\isadigit{2}}{\isacharcolon}\isanewline
\ \ \isakeyword{assumes}\ {\isachardoublequoteopen}finite\ S{\isachardoublequoteclose}\isanewline
\ \ \ \ \ \ \ \ \ \ {\isachardoublequoteopen}S\ {\isasymsubseteq}\ {\isacharbraceleft}a{\isacharcomma}b{\isacharbraceright}\ {\isasymunion}\ B{\isachardoublequoteclose}\isanewline
\ \ \ \ \ \ \ \ \isakeyword{shows}\ {\isachardoublequoteopen}{\isasymexists}S{\isacharprime}\ {\isasymsubseteq}\ B{\isachardot}\ finite\ S{\isacharprime}\ {\isasymand}\ {\isacharparenleft}S\ {\isacharequal}\ {\isacharbraceleft}a{\isacharcomma}b{\isacharbraceright}\ {\isasymunion}\ S{\isacharprime}\ {\isasymor}\ S\ {\isacharequal}\ {\isacharbraceleft}a{\isacharbraceright}\ {\isasymunion}\ S{\isacharprime}\ {\isasymor}\ S\ {\isacharequal}\ {\isacharbraceleft}b{\isacharbraceright}\ {\isasymunion}\ S{\isacharprime}\ {\isasymor}\ S\ {\isacharequal}\ S{\isacharprime}{\isacharparenright}{\isachardoublequoteclose}\isanewline
%
\isadelimproof
%
\endisadelimproof
%
\isatagproof
\isacommand{proof}\isamarkupfalse%
\ {\isacharminus}\isanewline
\ \ \isacommand{have}\isamarkupfalse%
\ {\isachardoublequoteopen}S\ {\isasymsubseteq}\ {\isacharbraceleft}a{\isacharbraceright}\ {\isasymunion}\ {\isacharparenleft}{\isacharbraceleft}b{\isacharbraceright}\ {\isasymunion}\ B{\isacharparenright}{\isachardoublequoteclose}\isanewline
\ \ \ \ \isacommand{using}\isamarkupfalse%
\ assms{\isacharparenleft}{\isadigit{2}}{\isacharparenright}\ \isacommand{by}\isamarkupfalse%
\ blast\isanewline
\ \ \isacommand{then}\isamarkupfalse%
\ \isacommand{have}\isamarkupfalse%
\ Ex{\isadigit{1}}{\isacharcolon}{\isachardoublequoteopen}{\isasymexists}S{\isadigit{1}}\ {\isasymsubseteq}\ {\isacharparenleft}{\isacharbraceleft}b{\isacharbraceright}\ {\isasymunion}\ B{\isacharparenright}{\isachardot}\ finite\ S{\isadigit{1}}\ {\isasymand}\ {\isacharparenleft}S\ {\isacharequal}\ {\isacharbraceleft}a{\isacharbraceright}\ {\isasymunion}\ S{\isadigit{1}}\ {\isasymor}\ S\ {\isacharequal}\ S{\isadigit{1}}{\isacharparenright}{\isachardoublequoteclose}\isanewline
\ \ \ \ \isacommand{using}\isamarkupfalse%
\ assms{\isacharparenleft}{\isadigit{1}}{\isacharparenright}\ \isacommand{by}\isamarkupfalse%
\ {\isacharparenleft}simp\ only{\isacharcolon}\ finite{\isacharunderscore}subset{\isacharunderscore}insert{\isadigit{1}}{\isacharparenright}\isanewline
\ \ \isacommand{obtain}\isamarkupfalse%
\ S{\isadigit{1}}\ \isakeyword{where}\ {\isachardoublequoteopen}S{\isadigit{1}}\ {\isasymsubseteq}\ {\isacharbraceleft}b{\isacharbraceright}\ {\isasymunion}\ B{\isachardoublequoteclose}\ \isakeyword{and}\ {\isadigit{1}}{\isacharcolon}{\isachardoublequoteopen}finite\ S{\isadigit{1}}\ {\isasymand}\ {\isacharparenleft}S\ {\isacharequal}\ {\isacharbraceleft}a{\isacharbraceright}\ {\isasymunion}\ S{\isadigit{1}}\ {\isasymor}\ S\ {\isacharequal}\ S{\isadigit{1}}{\isacharparenright}{\isachardoublequoteclose}\isanewline
\ \ \ \ \isacommand{using}\isamarkupfalse%
\ Ex{\isadigit{1}}\ \isacommand{by}\isamarkupfalse%
\ {\isacharparenleft}rule\ subexE{\isacharparenright}\isanewline
\ \ \isacommand{have}\isamarkupfalse%
\ {\isachardoublequoteopen}finite\ S{\isadigit{1}}{\isachardoublequoteclose}\isanewline
\ \ \ \ \isacommand{using}\isamarkupfalse%
\ {\isadigit{1}}\ \isacommand{by}\isamarkupfalse%
\ {\isacharparenleft}rule\ conjunct{\isadigit{1}}{\isacharparenright}\isanewline
\ \ \isacommand{have}\isamarkupfalse%
\ Ex{\isadigit{2}}{\isacharcolon}{\isachardoublequoteopen}{\isasymexists}S{\isadigit{2}}\ {\isasymsubseteq}\ B{\isachardot}\ finite\ S{\isadigit{2}}\ {\isasymand}\ {\isacharparenleft}S{\isadigit{1}}\ {\isacharequal}\ {\isacharbraceleft}b{\isacharbraceright}\ {\isasymunion}\ S{\isadigit{2}}\ {\isasymor}\ S{\isadigit{1}}\ {\isacharequal}\ S{\isadigit{2}}{\isacharparenright}{\isachardoublequoteclose}\isanewline
\ \ \ \ \isacommand{using}\isamarkupfalse%
\ {\isacartoucheopen}finite\ S{\isadigit{1}}{\isacartoucheclose}\ {\isacartoucheopen}S{\isadigit{1}}\ {\isasymsubseteq}\ {\isacharbraceleft}b{\isacharbraceright}\ {\isasymunion}\ B{\isacartoucheclose}\ \isacommand{by}\isamarkupfalse%
\ {\isacharparenleft}rule\ finite{\isacharunderscore}subset{\isacharunderscore}insert{\isadigit{1}}{\isacharparenright}\isanewline
\ \ \isacommand{obtain}\isamarkupfalse%
\ S{\isadigit{2}}\ \isakeyword{where}\ {\isachardoublequoteopen}S{\isadigit{2}}\ {\isasymsubseteq}\ B{\isachardoublequoteclose}\ \isakeyword{and}\ {\isadigit{2}}{\isacharcolon}{\isachardoublequoteopen}finite\ S{\isadigit{2}}\ {\isasymand}\ {\isacharparenleft}S{\isadigit{1}}\ {\isacharequal}\ {\isacharbraceleft}b{\isacharbraceright}\ {\isasymunion}\ S{\isadigit{2}}\ {\isasymor}\ S{\isadigit{1}}\ {\isacharequal}\ S{\isadigit{2}}{\isacharparenright}{\isachardoublequoteclose}\isanewline
\ \ \ \ \isacommand{using}\isamarkupfalse%
\ Ex{\isadigit{2}}\ \isacommand{by}\isamarkupfalse%
\ {\isacharparenleft}rule\ subexE{\isacharparenright}\isanewline
\ \ \isacommand{have}\isamarkupfalse%
\ {\isachardoublequoteopen}finite\ S{\isadigit{2}}{\isachardoublequoteclose}\isanewline
\ \ \ \ \isacommand{using}\isamarkupfalse%
\ {\isadigit{2}}\ \isacommand{by}\isamarkupfalse%
\ {\isacharparenleft}rule\ conjunct{\isadigit{1}}{\isacharparenright}\isanewline
\ \ \isacommand{have}\isamarkupfalse%
\ {\isachardoublequoteopen}S{\isadigit{1}}\ {\isacharequal}\ {\isacharbraceleft}b{\isacharbraceright}\ {\isasymunion}\ S{\isadigit{2}}\ {\isasymor}\ S{\isadigit{1}}\ {\isacharequal}\ S{\isadigit{2}}{\isachardoublequoteclose}\isanewline
\ \ \ \ \isacommand{using}\isamarkupfalse%
\ {\isadigit{2}}\ \isacommand{by}\isamarkupfalse%
\ {\isacharparenleft}rule\ conjunct{\isadigit{2}}{\isacharparenright}\isanewline
\ \ \isacommand{thus}\isamarkupfalse%
\ {\isacharquery}thesis\isanewline
\ \ \isacommand{proof}\isamarkupfalse%
\ {\isacharparenleft}rule\ disjE{\isacharparenright}\isanewline
\ \ \ \ \isacommand{assume}\isamarkupfalse%
\ {\isachardoublequoteopen}S{\isadigit{1}}\ {\isacharequal}\ {\isacharbraceleft}b{\isacharbraceright}\ {\isasymunion}\ S{\isadigit{2}}{\isachardoublequoteclose}\isanewline
\ \ \ \ \isacommand{have}\isamarkupfalse%
\ {\isachardoublequoteopen}S\ {\isacharequal}\ {\isacharbraceleft}a{\isacharbraceright}\ {\isasymunion}\ S{\isadigit{1}}\ {\isasymor}\ S\ {\isacharequal}\ S{\isadigit{1}}{\isachardoublequoteclose}\isanewline
\ \ \ \ \ \ \isacommand{using}\isamarkupfalse%
\ {\isadigit{1}}\ \isacommand{by}\isamarkupfalse%
\ {\isacharparenleft}rule\ conjunct{\isadigit{2}}{\isacharparenright}\isanewline
\ \ \ \ \isacommand{thus}\isamarkupfalse%
\ {\isacharquery}thesis\isanewline
\ \ \ \ \isacommand{proof}\isamarkupfalse%
\ {\isacharparenleft}rule\ disjE{\isacharparenright}\isanewline
\ \ \ \ \ \ \isacommand{assume}\isamarkupfalse%
\ {\isachardoublequoteopen}S\ {\isacharequal}\ {\isacharbraceleft}a{\isacharbraceright}\ {\isasymunion}\ S{\isadigit{1}}{\isachardoublequoteclose}\isanewline
\ \ \ \ \ \ \isacommand{then}\isamarkupfalse%
\ \isacommand{have}\isamarkupfalse%
\ {\isachardoublequoteopen}S\ {\isacharequal}\ {\isacharbraceleft}a{\isacharbraceright}\ {\isasymunion}\ {\isacharbraceleft}b{\isacharbraceright}\ {\isasymunion}\ S{\isadigit{2}}{\isachardoublequoteclose}\isanewline
\ \ \ \ \ \ \ \ \isacommand{by}\isamarkupfalse%
\ {\isacharparenleft}simp\ add{\isacharcolon}\ {\isacartoucheopen}S{\isadigit{1}}\ {\isacharequal}\ {\isacharbraceleft}b{\isacharbraceright}\ {\isasymunion}\ S{\isadigit{2}}{\isacartoucheclose}{\isacharparenright}\isanewline
\ \ \ \ \ \ \isacommand{then}\isamarkupfalse%
\ \isacommand{have}\isamarkupfalse%
\ {\isachardoublequoteopen}S\ {\isacharequal}\ {\isacharbraceleft}a{\isacharcomma}b{\isacharbraceright}\ {\isasymunion}\ S{\isadigit{2}}{\isachardoublequoteclose}\isanewline
\ \ \ \ \ \ \ \ \isacommand{by}\isamarkupfalse%
\ blast\isanewline
\ \ \ \ \ \ \isacommand{then}\isamarkupfalse%
\ \isacommand{have}\isamarkupfalse%
\ {\isachardoublequoteopen}S\ {\isacharequal}\ {\isacharbraceleft}a{\isacharcomma}b{\isacharbraceright}\ {\isasymunion}\ S{\isadigit{2}}\ {\isasymor}\ S\ {\isacharequal}\ {\isacharbraceleft}a{\isacharbraceright}\ {\isasymunion}\ S{\isadigit{2}}\ {\isasymor}\ S\ {\isacharequal}\ {\isacharbraceleft}b{\isacharbraceright}\ {\isasymunion}\ S{\isadigit{2}}\ {\isasymor}\ S\ {\isacharequal}\ S{\isadigit{2}}{\isachardoublequoteclose}\isanewline
\ \ \ \ \ \ \ \ \isacommand{by}\isamarkupfalse%
\ {\isacharparenleft}iprover\ intro{\isacharcolon}\ disjI{\isadigit{1}}{\isacharparenright}\isanewline
\ \ \ \ \ \ \isacommand{then}\isamarkupfalse%
\ \isacommand{have}\isamarkupfalse%
\ {\isachardoublequoteopen}finite\ S{\isadigit{2}}\ {\isasymand}\ {\isacharparenleft}S\ {\isacharequal}\ {\isacharbraceleft}a{\isacharcomma}b{\isacharbraceright}\ {\isasymunion}\ S{\isadigit{2}}\ {\isasymor}\ S\ {\isacharequal}\ {\isacharbraceleft}a{\isacharbraceright}\ {\isasymunion}\ S{\isadigit{2}}\ {\isasymor}\ S\ {\isacharequal}\ {\isacharbraceleft}b{\isacharbraceright}\ {\isasymunion}\ S{\isadigit{2}}\ {\isasymor}\ S\ {\isacharequal}\ S{\isadigit{2}}{\isacharparenright}{\isachardoublequoteclose}\isanewline
\ \ \ \ \ \ \ \ \isacommand{using}\isamarkupfalse%
\ {\isacartoucheopen}finite\ S{\isadigit{2}}{\isacartoucheclose}\ \isacommand{by}\isamarkupfalse%
\ {\isacharparenleft}iprover\ intro{\isacharcolon}\ conjI{\isacharparenright}\isanewline
\ \ \ \ \ \ \isacommand{thus}\isamarkupfalse%
\ {\isacharquery}thesis\isanewline
\ \ \ \ \ \ \ \ \isacommand{using}\isamarkupfalse%
\ {\isacartoucheopen}S{\isadigit{2}}\ {\isasymsubseteq}\ B{\isacartoucheclose}\ \isacommand{by}\isamarkupfalse%
\ {\isacharparenleft}rule\ subexI{\isacharparenright}\isanewline
\ \ \ \ \isacommand{next}\isamarkupfalse%
\isanewline
\ \ \ \ \ \ \isacommand{assume}\isamarkupfalse%
\ {\isachardoublequoteopen}S\ {\isacharequal}\ S{\isadigit{1}}{\isachardoublequoteclose}\isanewline
\ \ \ \ \ \ \isacommand{then}\isamarkupfalse%
\ \isacommand{have}\isamarkupfalse%
\ {\isachardoublequoteopen}S\ {\isacharequal}\ {\isacharbraceleft}b{\isacharbraceright}\ {\isasymunion}\ S{\isadigit{2}}{\isachardoublequoteclose}\isanewline
\ \ \ \ \ \ \ \ \isacommand{by}\isamarkupfalse%
\ {\isacharparenleft}simp\ add{\isacharcolon}\ {\isacartoucheopen}S{\isadigit{1}}\ {\isacharequal}\ {\isacharbraceleft}b{\isacharbraceright}\ {\isasymunion}\ S{\isadigit{2}}{\isacartoucheclose}{\isacharparenright}\isanewline
\ \ \ \ \ \ \isacommand{then}\isamarkupfalse%
\ \isacommand{have}\isamarkupfalse%
\ {\isachardoublequoteopen}S\ {\isacharequal}\ {\isacharbraceleft}a{\isacharcomma}b{\isacharbraceright}\ {\isasymunion}\ S{\isadigit{2}}\ {\isasymor}\ S\ {\isacharequal}\ {\isacharbraceleft}a{\isacharbraceright}\ {\isasymunion}\ S{\isadigit{2}}\ {\isasymor}\ S\ {\isacharequal}\ {\isacharbraceleft}b{\isacharbraceright}\ {\isasymunion}\ S{\isadigit{2}}\ {\isasymor}\ S\ {\isacharequal}\ S{\isadigit{2}}{\isachardoublequoteclose}\isanewline
\ \ \ \ \ \ \ \ \isacommand{by}\isamarkupfalse%
\ {\isacharparenleft}iprover\ intro{\isacharcolon}\ disjI{\isadigit{1}}{\isacharparenright}\isanewline
\ \ \ \ \ \ \isacommand{then}\isamarkupfalse%
\ \isacommand{have}\isamarkupfalse%
\ {\isachardoublequoteopen}finite\ S{\isadigit{2}}\ {\isasymand}\ {\isacharparenleft}S\ {\isacharequal}\ {\isacharbraceleft}a{\isacharcomma}b{\isacharbraceright}\ {\isasymunion}\ S{\isadigit{2}}\ {\isasymor}\ S\ {\isacharequal}\ {\isacharbraceleft}a{\isacharbraceright}\ {\isasymunion}\ S{\isadigit{2}}\ {\isasymor}\ S\ {\isacharequal}\ {\isacharbraceleft}b{\isacharbraceright}\ {\isasymunion}\ S{\isadigit{2}}\ {\isasymor}\ S\ {\isacharequal}\ S{\isadigit{2}}{\isacharparenright}{\isachardoublequoteclose}\isanewline
\ \ \ \ \ \ \ \ \isacommand{using}\isamarkupfalse%
\ {\isacartoucheopen}finite\ S{\isadigit{2}}{\isacartoucheclose}\ \isacommand{by}\isamarkupfalse%
\ {\isacharparenleft}iprover\ intro{\isacharcolon}\ conjI{\isacharparenright}\isanewline
\ \ \ \ \ \ \isacommand{thus}\isamarkupfalse%
\ {\isacharquery}thesis\isanewline
\ \ \ \ \ \ \ \ \isacommand{using}\isamarkupfalse%
\ {\isacartoucheopen}S{\isadigit{2}}\ {\isasymsubseteq}\ B{\isacartoucheclose}\ \isacommand{by}\isamarkupfalse%
\ {\isacharparenleft}rule\ subexI{\isacharparenright}\isanewline
\ \ \ \ \isacommand{qed}\isamarkupfalse%
\isanewline
\ \ \isacommand{next}\isamarkupfalse%
\isanewline
\ \ \ \ \isacommand{assume}\isamarkupfalse%
\ {\isachardoublequoteopen}S{\isadigit{1}}\ {\isacharequal}\ S{\isadigit{2}}{\isachardoublequoteclose}\isanewline
\ \ \ \ \isacommand{have}\isamarkupfalse%
\ {\isachardoublequoteopen}S\ {\isacharequal}\ {\isacharbraceleft}a{\isacharbraceright}\ {\isasymunion}\ S{\isadigit{1}}\ {\isasymor}\ S\ {\isacharequal}\ S{\isadigit{1}}{\isachardoublequoteclose}\isanewline
\ \ \ \ \ \ \isacommand{using}\isamarkupfalse%
\ {\isadigit{1}}\ \isacommand{by}\isamarkupfalse%
\ {\isacharparenleft}rule\ conjunct{\isadigit{2}}{\isacharparenright}\isanewline
\ \ \ \ \isacommand{thus}\isamarkupfalse%
\ {\isacharquery}thesis\isanewline
\ \ \ \ \isacommand{proof}\isamarkupfalse%
\ {\isacharparenleft}rule\ disjE{\isacharparenright}\isanewline
\ \ \ \ \ \ \isacommand{assume}\isamarkupfalse%
\ {\isachardoublequoteopen}S\ {\isacharequal}\ {\isacharbraceleft}a{\isacharbraceright}\ {\isasymunion}\ S{\isadigit{1}}{\isachardoublequoteclose}\isanewline
\ \ \ \ \ \ \isacommand{then}\isamarkupfalse%
\ \isacommand{have}\isamarkupfalse%
\ {\isachardoublequoteopen}S\ {\isacharequal}\ {\isacharbraceleft}a{\isacharbraceright}\ {\isasymunion}\ S{\isadigit{2}}{\isachardoublequoteclose}\isanewline
\ \ \ \ \ \ \ \ \isacommand{by}\isamarkupfalse%
\ {\isacharparenleft}simp\ only{\isacharcolon}\ {\isacartoucheopen}S{\isadigit{1}}\ {\isacharequal}\ S{\isadigit{2}}{\isacartoucheclose}{\isacharparenright}\isanewline
\ \ \ \ \ \ \isacommand{then}\isamarkupfalse%
\ \isacommand{have}\isamarkupfalse%
\ {\isachardoublequoteopen}S\ {\isacharequal}\ {\isacharbraceleft}a{\isacharcomma}b{\isacharbraceright}\ {\isasymunion}\ S{\isadigit{2}}\ {\isasymor}\ S\ {\isacharequal}\ {\isacharbraceleft}a{\isacharbraceright}\ {\isasymunion}\ S{\isadigit{2}}\ {\isasymor}\ S\ {\isacharequal}\ {\isacharbraceleft}b{\isacharbraceright}\ {\isasymunion}\ S{\isadigit{2}}\ {\isasymor}\ S\ {\isacharequal}\ S{\isadigit{2}}{\isachardoublequoteclose}\isanewline
\ \ \ \ \ \ \ \ \isacommand{by}\isamarkupfalse%
\ {\isacharparenleft}iprover\ intro{\isacharcolon}\ disjI{\isadigit{1}}{\isacharparenright}\isanewline
\ \ \ \ \ \ \isacommand{then}\isamarkupfalse%
\ \isacommand{have}\isamarkupfalse%
\ {\isachardoublequoteopen}finite\ S{\isadigit{2}}\ {\isasymand}\ {\isacharparenleft}S\ {\isacharequal}\ {\isacharbraceleft}a{\isacharcomma}b{\isacharbraceright}\ {\isasymunion}\ S{\isadigit{2}}\ {\isasymor}\ S\ {\isacharequal}\ {\isacharbraceleft}a{\isacharbraceright}\ {\isasymunion}\ S{\isadigit{2}}\ {\isasymor}\ S\ {\isacharequal}\ {\isacharbraceleft}b{\isacharbraceright}\ {\isasymunion}\ S{\isadigit{2}}\ {\isasymor}\ S\ {\isacharequal}\ S{\isadigit{2}}{\isacharparenright}{\isachardoublequoteclose}\isanewline
\ \ \ \ \ \ \ \ \isacommand{using}\isamarkupfalse%
\ {\isacartoucheopen}finite\ S{\isadigit{2}}{\isacartoucheclose}\ \isacommand{by}\isamarkupfalse%
\ {\isacharparenleft}iprover\ intro{\isacharcolon}\ conjI{\isacharparenright}\isanewline
\ \ \ \ \ \ \isacommand{thus}\isamarkupfalse%
\ {\isacharquery}thesis\isanewline
\ \ \ \ \ \ \ \ \isacommand{using}\isamarkupfalse%
\ {\isacartoucheopen}S{\isadigit{2}}\ {\isasymsubseteq}\ B{\isacartoucheclose}\ \isacommand{by}\isamarkupfalse%
\ {\isacharparenleft}rule\ subexI{\isacharparenright}\isanewline
\ \ \ \ \isacommand{next}\isamarkupfalse%
\isanewline
\ \ \ \ \ \ \isacommand{assume}\isamarkupfalse%
\ {\isachardoublequoteopen}S\ {\isacharequal}\ S{\isadigit{1}}{\isachardoublequoteclose}\isanewline
\ \ \ \ \ \ \isacommand{then}\isamarkupfalse%
\ \isacommand{have}\isamarkupfalse%
\ {\isachardoublequoteopen}S\ {\isacharequal}\ S{\isadigit{2}}{\isachardoublequoteclose}\isanewline
\ \ \ \ \ \ \ \ \isacommand{by}\isamarkupfalse%
\ {\isacharparenleft}simp\ only{\isacharcolon}\ {\isacartoucheopen}S{\isadigit{1}}\ {\isacharequal}\ S{\isadigit{2}}{\isacartoucheclose}{\isacharparenright}\isanewline
\ \ \ \ \ \ \isacommand{then}\isamarkupfalse%
\ \isacommand{have}\isamarkupfalse%
\ {\isachardoublequoteopen}S\ {\isacharequal}\ {\isacharbraceleft}a{\isacharcomma}b{\isacharbraceright}\ {\isasymunion}\ S{\isadigit{2}}\ {\isasymor}\ S\ {\isacharequal}\ {\isacharbraceleft}a{\isacharbraceright}\ {\isasymunion}\ S{\isadigit{2}}\ {\isasymor}\ S\ {\isacharequal}\ {\isacharbraceleft}b{\isacharbraceright}\ {\isasymunion}\ S{\isadigit{2}}\ {\isasymor}\ S\ {\isacharequal}\ S{\isadigit{2}}{\isachardoublequoteclose}\isanewline
\ \ \ \ \ \ \ \ \isacommand{by}\isamarkupfalse%
\ {\isacharparenleft}iprover\ intro{\isacharcolon}\ disjI{\isadigit{1}}{\isacharparenright}\isanewline
\ \ \ \ \ \ \isacommand{then}\isamarkupfalse%
\ \isacommand{have}\isamarkupfalse%
\ {\isachardoublequoteopen}finite\ S{\isadigit{2}}\ {\isasymand}\ {\isacharparenleft}S\ {\isacharequal}\ {\isacharbraceleft}a{\isacharcomma}b{\isacharbraceright}\ {\isasymunion}\ S{\isadigit{2}}\ {\isasymor}\ S\ {\isacharequal}\ {\isacharbraceleft}a{\isacharbraceright}\ {\isasymunion}\ S{\isadigit{2}}\ {\isasymor}\ S\ {\isacharequal}\ {\isacharbraceleft}b{\isacharbraceright}\ {\isasymunion}\ S{\isadigit{2}}\ {\isasymor}\ S\ {\isacharequal}\ S{\isadigit{2}}{\isacharparenright}{\isachardoublequoteclose}\isanewline
\ \ \ \ \ \ \ \ \isacommand{using}\isamarkupfalse%
\ {\isacartoucheopen}finite\ S{\isadigit{2}}{\isacartoucheclose}\ \isacommand{by}\isamarkupfalse%
\ {\isacharparenleft}iprover\ intro{\isacharcolon}\ conjI{\isacharparenright}\isanewline
\ \ \ \ \ \ \isacommand{thus}\isamarkupfalse%
\ {\isacharquery}thesis\isanewline
\ \ \ \ \ \ \ \ \isacommand{using}\isamarkupfalse%
\ {\isacartoucheopen}S{\isadigit{2}}\ {\isasymsubseteq}\ B{\isacartoucheclose}\ \isacommand{by}\isamarkupfalse%
\ {\isacharparenleft}rule\ subexI{\isacharparenright}\isanewline
\ \ \ \ \isacommand{qed}\isamarkupfalse%
\isanewline
\ \ \isacommand{qed}\isamarkupfalse%
\isanewline
\isacommand{qed}\isamarkupfalse%
%
\endisatagproof
{\isafoldproof}%
%
\isadelimproof
%
\endisadelimproof
%
\begin{isamarkuptext}%
Una vez introducidos los resultados anteriores, procedamos con la prueba del \isa{Teorema\ de\ Compacidad}.

  \begin{demostracion}
    Consideremos la colección \isa{C} formada por los conjuntos de fórmulas finitamente satisfacibles.
    Recordemos que un conjunto de fórmulas es finitamente satisfacible si todo subconjunto finito 
    suyo es satisfacible. Vamos a probar que dicho conjunto verifica la propiedad de consistencia 
    proposicional y, por el \isa{Teorema\ de\ Existencia\ de\ Modelo}, quedará probado que todo conjunto de 
    \isa{C} es satisfacible, lo que demuestra el teorema.

    Para probar que \isa{C} verifica la propiedad de consistencia proposicional, por el lema \isa{{\isadigit{2}}{\isachardot}{\isadigit{0}}{\isachardot}{\isadigit{2}}} de 
    caracterización mediante notación uniforme, basta demostrar que se verifican las siguientes 
    condiciones para todo conjunto \isa{W\ {\isasymin}\ C}:
    \begin{itemize}
     \item \isa{{\isasymbottom}\ {\isasymnotin}\ W}.
     \item Dada \isa{p} una fórmula atómica cualquiera, no se tiene 
      simultáneamente que\\ \isa{p\ {\isasymin}\ W} y \isa{{\isasymnot}\ p\ {\isasymin}\ W}.
     \item Para toda fórmula de tipo \isa{{\isasymalpha}} con componentes \isa{{\isasymalpha}\isactrlsub {\isadigit{1}}} y \isa{{\isasymalpha}\isactrlsub {\isadigit{2}}} tal que \isa{{\isasymalpha}}
      pertenece a \isa{W}, se tiene que \isa{{\isacharbraceleft}{\isasymalpha}\isactrlsub {\isadigit{1}}{\isacharcomma}{\isasymalpha}\isactrlsub {\isadigit{2}}{\isacharbraceright}\ {\isasymunion}\ W} pertenece a \isa{C}.
     \item Para toda fórmula de tipo \isa{{\isasymbeta}} con componentes \isa{{\isasymbeta}\isactrlsub {\isadigit{1}}} y \isa{{\isasymbeta}\isactrlsub {\isadigit{2}}} tal que \isa{{\isasymbeta}}
      pertenece a \isa{W}, se tiene que o bien \isa{{\isacharbraceleft}{\isasymbeta}\isactrlsub {\isadigit{1}}{\isacharbraceright}\ {\isasymunion}\ W} pertenece a \isa{C} o 
      bien \isa{{\isacharbraceleft}{\isasymbeta}\isactrlsub {\isadigit{2}}{\isacharbraceright}\ {\isasymunion}\ W} pertenece a \isa{C}.
    \end{itemize}

    De este modo, consideremos un conjunto cualquiera \isa{W\ {\isasymin}\ C} y procedamos a probar cada una de las
    condiciones anteriores.

    La primera condición se demuestra por reducción al absurdo. En efecto, si suponemos que 
    \isa{{\isasymbottom}\ {\isasymin}\ W}, es claro que \isa{{\isacharbraceleft}{\isasymbottom}{\isacharbraceright}} es un subconjunto finito de \isa{W}. Como \isa{W} es un conjunto
    finitamente satisfacible por pertenecer a \isa{C}, se tiene por lo anterior que \isa{{\isacharbraceleft}{\isasymbottom}{\isacharbraceright}} es 
    satisfacible. De este modo, llegamos a una contradicción pues, por definición, no existe ninguna 
    interpretación que sea modelo de \isa{{\isasymbottom}}.

    A continuación probaremos que, si \isa{W\ {\isasymin}\ C}, entonces dada \isa{p} una fórmula atómica cualquiera, no 
    se tiene simultáneamente que \isa{p\ {\isasymin}\ W} y \isa{{\isasymnot}\ p\ {\isasymin}\ W}. Veamos dicho resultado por reducción al 
    absurdo, suponiendo que tanto \isa{p} como \isa{{\isasymnot}\ p} están en \isa{W}. En este caso, \isa{{\isacharbraceleft}p{\isacharcomma}{\isasymnot}\ p{\isacharbraceright}} sería un
    subconjunto finito de \isa{W} y, por ser \isa{W} finitamente satisfacible ya que\\ \isa{W\ {\isasymin}\ C}, obtendríamos 
    que \isa{{\isacharbraceleft}p{\isacharcomma}{\isasymnot}\ p{\isacharbraceright}} es satisfacible. Sin embargo esto no es cierto ya que, en ese caso, existiría
    una interpretación que sería modelo tanto de \isa{p} como de \isa{{\isasymnot}\ p}, llegando así a una 
    contradicción.

    Probemos ahora que dada una fórmula \isa{F} de tipo \isa{{\isasymalpha}} con componentes \isa{{\isasymalpha}\isactrlsub {\isadigit{1}}} y \isa{{\isasymalpha}\isactrlsub {\isadigit{2}}} tal que \isa{F\ {\isasymin}\ W},
    se tiene que \isa{{\isacharbraceleft}{\isasymalpha}\isactrlsub {\isadigit{1}}{\isacharcomma}{\isasymalpha}\isactrlsub {\isadigit{2}}{\isacharbraceright}\ {\isasymunion}\ W} pertenece a \isa{C}. Por definición de la colección, basta probar que 
    \isa{{\isacharbraceleft}{\isasymalpha}\isactrlsub {\isadigit{1}}{\isacharcomma}{\isasymalpha}\isactrlsub {\isadigit{2}}{\isacharbraceright}\ {\isasymunion}\ W} es finitamente satisfacible, es decir, que todo subconjunto finito suyo es
    satisfacible. Consideremos un subconjunto finito \isa{S} de \isa{{\isacharbraceleft}{\isasymalpha}\isactrlsub {\isadigit{1}}{\isacharcomma}{\isasymalpha}\isactrlsub {\isadigit{2}}{\isacharbraceright}\ {\isasymunion}\ W}. En estas condiciones,
    por el lema \isa{{\isadigit{4}}{\isachardot}{\isadigit{3}}{\isachardot}{\isadigit{3}}}, existe un subconjunto finito \isa{W\isactrlsub {\isadigit{0}}} de \isa{W} tal que\\ \isa{S\ {\isacharequal}\ {\isacharbraceleft}{\isasymalpha}\isactrlsub {\isadigit{1}}{\isacharcomma}{\isasymalpha}\isactrlsub {\isadigit{2}}{\isacharbraceright}\ {\isasymunion}\ W\isactrlsub {\isadigit{0}}},
    \isa{S\ {\isacharequal}\ {\isacharbraceleft}{\isasymalpha}\isactrlsub {\isadigit{1}}{\isacharbraceright}\ {\isasymunion}\ W\isactrlsub {\isadigit{0}}}, \isa{S\ {\isacharequal}\ {\isacharbraceleft}{\isasymalpha}\isactrlsub {\isadigit{2}}{\isacharbraceright}\ {\isasymunion}\ W\isactrlsub {\isadigit{0}}} o \isa{S\ {\isacharequal}\ W\isactrlsub {\isadigit{0}}}. Para probar que \isa{S} es satisfacible en cada uno de 
    estos posibles casos, basta demostrar que el conjunto\\ \isa{{\isacharbraceleft}{\isasymalpha}\isactrlsub {\isadigit{1}}{\isacharcomma}{\isasymalpha}\isactrlsub {\isadigit{2}}{\isacharcomma}F{\isacharbraceright}\ {\isasymunion}\ W\isactrlsub {\isadigit{0}}} es satisfacible. De este
    modo, puesto que todas las opciones posibles de \isa{S} están contenidas en dicho conjunto, se
    tiene la satisfacibilidad de cada una de ellas.

    Para probar que el conjunto \isa{{\isacharbraceleft}{\isasymalpha}\isactrlsub {\isadigit{1}}{\isacharcomma}{\isasymalpha}\isactrlsub {\isadigit{2}}{\isacharcomma}F{\isacharbraceright}\ {\isasymunion}\ W\isactrlsub {\isadigit{0}}} es satisfacible en estas condiciones, demostremos 
    que se verifica para cada caso de la fórmula \isa{F} de tipo \isa{{\isasymalpha}}:

      $\textbf{\isa{{\isasymone}{\isacharparenright}\ F\ {\isacharequal}\ G\ {\isasymand}\ H{\isacharcomma}\ para\ ciertas\ fórmulas\ G\ y\ H}:}$ Observemos que, como \isa{W\isactrlsub {\isadigit{0}}} es un subconjunto 
      finito de \isa{W} y \isa{F\ {\isasymin}\ W} por hipótesis, tenemos que \isa{{\isacharbraceleft}F{\isacharbraceright}\ {\isasymunion}\ W\isactrlsub {\isadigit{0}}} es un subconjunto finito de \isa{W}. 
      Como \isa{W} es finitamente satisfacible ya que pertenece a \isa{C}, se tiene que \isa{{\isacharbraceleft}F{\isacharbraceright}\ {\isasymunion}\ W\isactrlsub {\isadigit{0}}} es 
      satisfacible. Luego, por definición, existe una interpretación \isa{{\isasymA}} que es modelo de todas sus 
      fórmulas y, en particular, \isa{{\isasymA}} es modelo de \isa{F}. Como \isa{F\ {\isacharequal}\ G\ {\isasymand}\ H}, obtenemos por definición 
      del valor de una fórmula en una interpretación que \isa{{\isasymA}} es modelo de \isa{G} y de \isa{H}. En este caso,
      las componentes conjuntivas son \isa{{\isasymalpha}\isactrlsub {\isadigit{1}}\ {\isacharequal}\ G} y \isa{{\isasymalpha}\isactrlsub {\isadigit{2}}\ {\isacharequal}\ H}, luego \isa{{\isasymA}} es modelo de ambas componentes.
      Por lo tanto, \isa{{\isasymA}} es modelo de todas las fórmulas del conjunto \isa{{\isacharbraceleft}{\isasymalpha}\isactrlsub {\isadigit{1}}{\isacharcomma}{\isasymalpha}\isactrlsub {\isadigit{2}}{\isacharcomma}F{\isacharbraceright}\ {\isasymunion}\ W\isactrlsub {\isadigit{0}}}, lo que prueba 
      que se trata de un conjunto satisfacible.

      $\textbf{\isa{{\isasymtwo}{\isacharparenright}\ F\ {\isacharequal}\ {\isasymnot}{\isacharparenleft}G\ {\isasymor}\ H{\isacharparenright}{\isacharcomma}\ para\ ciertas\ fórmulas\ G\ y\ H}:}$ Análogamente al caso anterior, obtenemos 
      que el conjunto \isa{{\isacharbraceleft}F{\isacharbraceright}\ {\isasymunion}\ W\isactrlsub {\isadigit{0}}} es satisfacible. Luego, por definición, existe una interpretación 
      \isa{{\isasymA}} que es modelo de todas sus fórmulas y, en particular, de \isa{F}. Por definición del valor de 
      una fórmula en una interpretación, como \isa{F\ {\isacharequal}\ {\isasymnot}{\isacharparenleft}G\ {\isasymor}\ H{\isacharparenright}}, obtenemos que no es cierto que \isa{{\isasymA}} 
      sea modelo de \isa{G\ {\isasymor}\ H}. Aplicando de nuevo la definición del valor de una fórmula en una 
      interpretación, se obtiene que no es cierto que \isa{{\isasymA}} se modelo de \isa{G} o de \isa{H}. Por las leyes 
      de \isa{Morgan}, obtenemos equivalentemente que \isa{{\isasymA}} no es modelo de \isa{G} y \isa{{\isasymA}} no es modelo de \isa{H}. 
      Por lo tanto, por el valor de una fórmula en una interpretación, obtenemos que \isa{{\isasymA}} es 
      modelo de \isa{{\isasymnot}\ G} y \isa{{\isasymA}} es modelo de \isa{{\isasymnot}\ H}. Como las componentes conjuntivas en este caso son 
      \isa{{\isasymalpha}\isactrlsub {\isadigit{1}}\ {\isacharequal}\ {\isasymnot}\ G} y \isa{{\isasymalpha}\isactrlsub {\isadigit{2}}\ {\isacharequal}\ {\isasymnot}\ H}, es claro que \isa{{\isasymA}} es modelo de \isa{{\isasymalpha}\isactrlsub {\isadigit{1}}} y de \isa{{\isasymalpha}\isactrlsub {\isadigit{2}}}. Por lo tanto, la 
      interpretación \isa{{\isasymA}} es modelo de todas las fórmulas del conjunto \isa{{\isacharbraceleft}{\isasymalpha}\isactrlsub {\isadigit{1}}{\isacharcomma}{\isasymalpha}\isactrlsub {\isadigit{2}}{\isacharcomma}F{\isacharbraceright}\ {\isasymunion}\ W\isactrlsub {\isadigit{0}}}, lo que 
      prueba por definición que se trata de un conjunto satisfacible. 

      $\textbf{\isa{{\isasymthree}{\isacharparenright}\ F\ {\isacharequal}\ {\isasymnot}{\isacharparenleft}G\ {\isasymlongrightarrow}\ H{\isacharparenright}{\isacharcomma}\ para\ ciertas\ fórmulas\ G\ y\ H}:}$ Como hemos visto que \isa{{\isacharbraceleft}F{\isacharbraceright}\ {\isasymunion}\ W\isactrlsub {\isadigit{0}}} 
      es un conjunto satisfacible, existe una interpretación \isa{{\isasymA}} que es modelo de todas sus 
      fórmulas. En particular, \isa{{\isasymA}} es modelo de \isa{F} luego, por definición del valor de una fórmula 
      en una interpretación, es claro que \isa{{\isasymA}} no es modelo de \isa{G\ {\isasymlongrightarrow}\ H}. De nuevo por el valor de 
      una fórmula en una interpretación, obtenemos que no es cierto que si \isa{{\isasymA}} es modelo de \isa{G}, 
      entonces sea modelo de \isa{H}. Equivalentemente, \isa{{\isasymA}} es modelo de \isa{G} y no es modelo de \isa{H}. Por 
      lo tanto, por la definición del valor de una fórmula en una interpretación, se obtiene que 
      \isa{{\isasymA}} es modelo de \isa{G} y de \isa{{\isasymnot}\ H}. Como en este caso las componentes conjuntivas son \isa{{\isasymalpha}\isactrlsub {\isadigit{1}}\ {\isacharequal}\ G} y
      \isa{{\isasymalpha}\isactrlsub {\isadigit{2}}\ {\isacharequal}\ {\isasymnot}\ H}, es claro que \isa{{\isasymA}} es modelo de \isa{{\isasymalpha}\isactrlsub {\isadigit{1}}} y de \isa{{\isasymalpha}\isactrlsub {\isadigit{2}}}. Por lo tanto, \isa{{\isasymA}} es modelo de 
      todas las fórmulas del conjunto  \isa{{\isacharbraceleft}{\isasymalpha}\isactrlsub {\isadigit{1}}{\isacharcomma}{\isasymalpha}\isactrlsub {\isadigit{2}}{\isacharcomma}F{\isacharbraceright}\ {\isasymunion}\ W\isactrlsub {\isadigit{0}}}, probando así su satisfacibilidad.

      $\textbf{\isa{{\isasymfour}{\isacharparenright}\ F\ {\isacharequal}\ {\isasymnot}{\isacharparenleft}{\isasymnot}\ G{\isacharparenright}{\isacharcomma}\ para\ cierta\ fórmula\ G}:}$ Análogamente a los casos anteriores, se 
      prueba que existe una interpretación \isa{{\isasymA}} que es modelo de todas las fórmulas del conjunto\\ 
      \isa{{\isacharbraceleft}F{\isacharbraceright}\ {\isasymunion}\ W\isactrlsub {\isadigit{0}}} por ser este satisfacible. En particular, \isa{{\isasymA}} es modelo de \isa{F} luego, por 
      definición del valor de una fórmula en una interpretación, obtenemos que no es cierto que \isa{{\isasymA}} 
      no es modelo de \isa{G}. Es decir, \isa{{\isasymA}} es modelo de \isa{G} y, como en este caso ambas componentes 
      disyuntivas son \isa{G}, es claro que \isa{{\isasymA}} es modelo de \isa{{\isasymalpha}\isactrlsub {\isadigit{1}}} y de \isa{{\isasymalpha}\isactrlsub {\isadigit{2}}}. Por tanto, \isa{{\isasymA}} es modelo 
      de todas las fórmulas del conjunto \isa{{\isacharbraceleft}{\isasymalpha}\isactrlsub {\isadigit{1}}{\isacharcomma}{\isasymalpha}\isactrlsub {\isadigit{2}}{\isacharcomma}F{\isacharbraceright}\ {\isasymunion}\ W\isactrlsub {\isadigit{0}}}, lo que prueba su satisfacibilidad.

    Por lo tanto, \isa{{\isacharbraceleft}{\isasymalpha}\isactrlsub {\isadigit{1}}{\isacharcomma}{\isasymalpha}\isactrlsub {\isadigit{2}}{\isacharcomma}F{\isacharbraceright}\ {\isasymunion}\ W\isactrlsub {\isadigit{0}}} es un conjunto satisfacible para todos los casos de la fórmula
    \isa{F} de tipo \isa{{\isasymalpha}}. De este modo, como el subconjunto finito \isa{S} de \isa{{\isacharbraceleft}{\isasymalpha}\isactrlsub {\isadigit{1}}{\isacharcomma}{\isasymalpha}\isactrlsub {\isadigit{2}}{\isacharbraceright}\ {\isasymunion}\ W} es de la forma
    \isa{S\ {\isacharequal}\ {\isacharbraceleft}{\isasymalpha}\isactrlsub {\isadigit{1}}{\isacharcomma}{\isasymalpha}\isactrlsub {\isadigit{2}}{\isacharbraceright}\ {\isasymunion}\ W\isactrlsub {\isadigit{0}}}, \isa{S\ {\isacharequal}\ {\isacharbraceleft}{\isasymalpha}\isactrlsub {\isadigit{1}}{\isacharbraceright}\ {\isasymunion}\ W\isactrlsub {\isadigit{0}}}, \isa{S\ {\isacharequal}\ {\isacharbraceleft}{\isasymalpha}\isactrlsub {\isadigit{2}}{\isacharbraceright}\ {\isasymunion}\ W\isactrlsub {\isadigit{0}}} o \isa{S\ {\isacharequal}\ W\isactrlsub {\isadigit{0}}}, se prueba la satisfacibilidad
    de \isa{S} para cada uno de los casos por estar contenidos en el conjunto satisfacible
    \isa{{\isacharbraceleft}{\isasymalpha}\isactrlsub {\isadigit{1}}{\isacharcomma}{\isasymalpha}\isactrlsub {\isadigit{2}}{\isacharcomma}F{\isacharbraceright}\ {\isasymunion}\ W\isactrlsub {\isadigit{0}}}. Por lo tanto, \isa{{\isacharbraceleft}{\isasymalpha}\isactrlsub {\isadigit{1}}{\isacharcomma}{\isasymalpha}\isactrlsub {\isadigit{2}}{\isacharbraceright}\ {\isasymunion}\ W} es finitamente satisfacible y, por definición de 
    la colección \isa{C}, pertenece a ella como queríamos demostrar.

    Finalmente probemos que para toda fórmula \isa{F} de tipo \isa{{\isasymbeta}} con componentes \isa{{\isasymbeta}\isactrlsub {\isadigit{1}}} y \isa{{\isasymbeta}\isactrlsub {\isadigit{2}}} tal que 
    \isa{F\ {\isasymin}\ W}, se tiene que o bien \isa{{\isacharbraceleft}{\isasymbeta}\isactrlsub {\isadigit{1}}{\isacharbraceright}\ {\isasymunion}\ W\ {\isasymin}\ C} o bien \isa{{\isacharbraceleft}{\isasymbeta}\isactrlsub {\isadigit{2}}{\isacharbraceright}\ {\isasymunion}\ W\ {\isasymin}\ C}. La
    demostración se realizará por reducción al absurdo, luego supongamos en estas condiciones que\\
    \isa{{\isacharbraceleft}{\isasymbeta}\isactrlsub {\isadigit{1}}{\isacharbraceright}\ {\isasymunion}\ W\ {\isasymnotin}\ C} y \isa{{\isacharbraceleft}{\isasymbeta}\isactrlsub {\isadigit{2}}{\isacharbraceright}\ {\isasymunion}\ W\ {\isasymnotin}\ C}. 

    En primer lugar, veamos que si \isa{{\isacharbraceleft}{\isasymbeta}\isactrlsub i{\isacharbraceright}\ {\isasymunion}\ W\ {\isasymnotin}\ C}, entonces existe un subconjunto finito \isa{W\isactrlsub i} de 
    \isa{W} tal que el conjunto \isa{{\isacharbraceleft}{\isasymbeta}\isactrlsub i{\isacharbraceright}\ {\isasymunion}\ W\isactrlsub i} no es satisfacible. En efecto, si \isa{{\isacharbraceleft}{\isasymbeta}\isactrlsub i{\isacharbraceright}\ {\isasymunion}\ W\ {\isasymnotin}\ C}, por 
    definición de la colección \isa{C} tenemos que \isa{{\isacharbraceleft}{\isasymbeta}\isactrlsub i{\isacharbraceright}\ {\isasymunion}\ W} no es finitamente satisfacible. Por lo 
    tanto, existe un subconjunto finito \isa{W\isactrlsub i{\isacharprime}} de \isa{{\isacharbraceleft}{\isasymbeta}\isactrlsub i{\isacharbraceright}\ {\isasymunion}\ W} que no es satisfacible. Por el lema 
    \isa{{\isadigit{4}}{\isachardot}{\isadigit{3}}{\isachardot}{\isadigit{2}}}, obtenemos que existe un subconjunto finito \isa{W\isactrlsub i} de \isa{W} tal que o bien\\ \isa{W\isactrlsub i{\isacharprime}\ {\isacharequal}\ {\isacharbraceleft}{\isasymbeta}\isactrlsub i{\isacharbraceright}\ {\isasymunion}\ W\isactrlsub i} 
    o bien \isa{W\isactrlsub i{\isacharprime}\ {\isacharequal}\ W\isactrlsub i}. En efecto, si \isa{W\isactrlsub i{\isacharprime}\ {\isacharequal}\ {\isacharbraceleft}{\isasymbeta}\isactrlsub i{\isacharbraceright}\ {\isasymunion}\ W\isactrlsub i}, como \isa{W\isactrlsub i{\isacharprime}} no es satisfacible, se obtiene el
    resultado para \isa{W\isactrlsub i}. Por otro lado, supongamos que\\ \isa{W\isactrlsub i{\isacharprime}\ {\isacharequal}\ W\isactrlsub i}. Como \isa{W\isactrlsub i{\isacharprime}} no es satisfacible, 
    entonces \isa{{\isacharbraceleft}{\isasymbeta}\isactrlsub i{\isacharbraceright}\ {\isasymunion}\ W\isactrlsub i} tampoco es satisfacible ya que, en caso contrario, obtendríamos que
    \isa{W\isactrlsub i\ {\isacharequal}\ W\isactrlsub i{\isacharprime}} es satisfacible. Luego se verifica también el resultado para \isa{W\isactrlsub i}.

    De este modo, como \isa{{\isacharbraceleft}{\isasymbeta}\isactrlsub {\isadigit{1}}{\isacharbraceright}\ {\isasymunion}\ W\ {\isasymnotin}\ C} y \isa{{\isacharbraceleft}{\isasymbeta}\isactrlsub {\isadigit{2}}{\isacharbraceright}\ {\isasymunion}\ W\ {\isasymnotin}\ C}, obtenemos que existen subconjuntos finitos 
    \isa{W\isactrlsub {\isadigit{1}}} y \isa{W\isactrlsub {\isadigit{2}}} de \isa{W} tales que los conjunto \isa{{\isacharbraceleft}{\isasymbeta}\isactrlsub {\isadigit{1}}{\isacharbraceright}\ {\isasymunion}\ W\isactrlsub {\isadigit{1}}} y \isa{{\isacharbraceleft}{\isasymbeta}\isactrlsub {\isadigit{2}}{\isacharbraceright}\ {\isasymunion}\ W\isactrlsub {\isadigit{2}}} no son satisfacibles. 
    Consideremos el conjunto \isa{W\isactrlsub o\ {\isacharequal}\ W\isactrlsub {\isadigit{1}}\ {\isasymunion}\ W\isactrlsub {\isadigit{2}}}. Es claro que se tiene que\\ \isa{{\isacharbraceleft}{\isasymbeta}\isactrlsub {\isadigit{1}}{\isacharbraceright}\ {\isasymunion}\ W\isactrlsub {\isadigit{1}}\ {\isasymsubseteq}\ {\isacharbraceleft}{\isasymbeta}\isactrlsub {\isadigit{1}}{\isacharcomma}F{\isacharbraceright}\ {\isasymunion}\ W\isactrlsub o} y 
    que \isa{{\isacharbraceleft}{\isasymbeta}\isactrlsub {\isadigit{2}}{\isacharbraceright}\ {\isasymunion}\ W\isactrlsub {\isadigit{2}}\ {\isasymsubseteq}\ {\isacharbraceleft}{\isasymbeta}\isactrlsub {\isadigit{2}}{\isacharcomma}F{\isacharbraceright}\ {\isasymunion}\ W\isactrlsub {\isadigit{0}}}. Por lo tanto, los conjuntos \isa{{\isacharbraceleft}{\isasymbeta}\isactrlsub {\isadigit{1}}{\isacharcomma}F{\isacharbraceright}\ {\isasymunion}\ W\isactrlsub o} y \isa{{\isacharbraceleft}{\isasymbeta}\isactrlsub {\isadigit{2}}{\isacharcomma}F{\isacharbraceright}\ {\isasymunion}\ W\isactrlsub o} no son 
    satisfacibles ya que, en caso contrario, \isa{{\isacharbraceleft}{\isasymbeta}\isactrlsub {\isadigit{1}}{\isacharbraceright}\ {\isasymunion}\ W\isactrlsub {\isadigit{1}}} y \isa{{\isacharbraceleft}{\isasymbeta}\isactrlsub {\isadigit{2}}{\isacharbraceright}\ {\isasymunion}\ W\isactrlsub {\isadigit{2}}} serían satisfacibles. Para 
    llegar a la contradicción, basta probar que o bien \isa{{\isacharbraceleft}{\isasymbeta}\isactrlsub {\isadigit{1}}{\isacharcomma}F{\isacharbraceright}\ {\isasymunion}\ W\isactrlsub o} es satisfacible o bien 
    \isa{{\isacharbraceleft}{\isasymbeta}\isactrlsub {\isadigit{2}}{\isacharcomma}F{\isacharbraceright}\ {\isasymunion}\ W\isactrlsub o} es satisfacible. Para ello, veamos que se verifica el resultado para cada uno de 
    los casos posibles fórmula de tipo \isa{{\isasymbeta}} para \isa{F}.

      $\textbf{\isa{{\isasymone}{\isacharparenright}\ F\ {\isacharequal}\ G\ {\isasymor}\ H{\isacharcomma}\ para\ ciertas\ fórmulas\ G\ y\ H}:}$ Observemos que \isa{W\isactrlsub {\isadigit{0}}\ {\isacharequal}\ W\isactrlsub {\isadigit{1}}\ {\isasymunion}\ W\isactrlsub {\isadigit{2}}} es un 
      subconjunto finito de \isa{W} por ser \isa{W\isactrlsub {\isadigit{1}}} y \isa{W\isactrlsub {\isadigit{2}}} subconjuntos finitos de \isa{W}. Además, como 
      \isa{F\ {\isasymin}\ W} por hipótesis, tenemos que \isa{{\isacharbraceleft}F{\isacharbraceright}\ {\isasymunion}\ W\isactrlsub {\isadigit{0}}} es un subconjunto finito de \isa{W}. Como \isa{W} es 
      finitamente satisfacible ya que pertenece a \isa{C}, se tiene que \isa{{\isacharbraceleft}F{\isacharbraceright}\ {\isasymunion}\ W\isactrlsub {\isadigit{0}}} es satisfacible. 
      Luego, por definición, existe una interpretación \isa{{\isasymA}} que es modelo de todas sus fórmulas y, 
      en particular, \isa{{\isasymA}} es modelo de \isa{F}. Por definición del valor de una fórmula en una
      interpretación, obtenemos que o bien \isa{{\isasymA}} es modelo de \isa{G} o bien \isa{{\isasymA}} es modelo de \isa{H}.
      Como en este caso las componentes disyuntivas son \isa{{\isasymbeta}\isactrlsub {\isadigit{1}}\ {\isacharequal}\ G} y \isa{{\isasymbeta}\isactrlsub {\isadigit{2}}\ {\isacharequal}\ H}, se tiene que o bien \isa{{\isasymA}}
      es modelo de \isa{{\isasymbeta}\isactrlsub {\isadigit{1}}} o bien \isa{{\isasymA}} es modelo de \isa{{\isasymbeta}\isactrlsub {\isadigit{2}}}. Por lo tanto, es claro que o bien \isa{{\isasymA}} es
      modelo de todas las fórmulas del conjunto \isa{{\isacharbraceleft}{\isasymbeta}\isactrlsub {\isadigit{1}}{\isacharcomma}F{\isacharbraceright}\ {\isasymunion}\ W\isactrlsub {\isadigit{0}}} o bien es modelo de todas las fórmulas
      de \isa{{\isacharbraceleft}{\isasymbeta}\isactrlsub {\isadigit{2}}{\isacharcomma}F{\isacharbraceright}\ {\isasymunion}\ W\isactrlsub {\isadigit{0}}}. Luego, por definición de conjunto satisfacible tenemos que o bien\\ 
      \isa{{\isacharbraceleft}{\isasymbeta}\isactrlsub {\isadigit{1}}{\isacharcomma}F{\isacharbraceright}\ {\isasymunion}\ W\isactrlsub {\isadigit{0}}} es satisfacible o bien \isa{{\isacharbraceleft}{\isasymbeta}\isactrlsub {\isadigit{2}}{\isacharcomma}F{\isacharbraceright}\ {\isasymunion}\ W\isactrlsub {\isadigit{0}}} es satisfacible, como queríamos demostrar.

      $\textbf{\isa{{\isasymtwo}{\isacharparenright}\ F\ {\isacharequal}\ G\ {\isasymlongrightarrow}\ H{\isacharcomma}\ para\ ciertas\ fórmulas\ G\ y\ H}:}$ Análogamente se tiene que el 
      conjunto \isa{{\isacharbraceleft}F{\isacharbraceright}\ {\isasymunion}\ W\isactrlsub {\isadigit{0}}} es satisfacible, luego existe una interpretación \isa{{\isasymA}} que es modelo de 
      todas sus fórmulas. En particular, \isa{{\isasymA}} es modelo de \isa{F} y, por definición del valor de una 
      fórmula en una interpretación, se obtiene que si \isa{{\isasymA}} es modelo de \isa{G}, entonces es modelo de 
      \isa{H}. Equivalentemente, tenemos que \isa{{\isasymA}} no es modelo de \isa{G} o \isa{{\isasymA}} es modelo de \isa{H}. Por un 
      lado, si suponemos que \isa{{\isasymA}} no es modelo de \isa{G}, por definición obtenemos que \isa{{\isasymA}} es modelo de 
      \isa{{\isasymnot}\ G}. Como en este caso tenemos que \isa{{\isasymbeta}\isactrlsub {\isadigit{1}}\ {\isacharequal}\ {\isasymnot}\ G}, es claro que \isa{{\isasymA}} es modelo de \isa{{\isasymbeta}\isactrlsub {\isadigit{1}}}. Por 
      tanto, es modelo de todas las fórmulas de \isa{{\isacharbraceleft}{\isasymbeta}\isactrlsub {\isadigit{1}}{\isacharcomma}F{\isacharbraceright}\ {\isasymunion}\ W\isactrlsub {\isadigit{0}}}, luego es un conjunto satisfacible y 
      se verifica el resultado para este caso. Por otro lado, si suponemos que \isa{{\isasymA}} es modelo de \isa{H}, 
      como \isa{{\isasymbeta}\isactrlsub {\isadigit{2}}\ {\isacharequal}\ H}, obtenemos que \isa{{\isasymA}} es modelo de \isa{{\isasymbeta}\isactrlsub {\isadigit{2}}}. Luego, análogamente, \isa{{\isasymA}} es modelo de toda
      fórmula de \isa{{\isacharbraceleft}{\isasymbeta}\isactrlsub {\isadigit{2}}{\isacharcomma}F{\isacharbraceright}\ {\isasymunion}\ W\isactrlsub {\isadigit{0}}}, lo que prueba que se trata de un conjunto satisfacible por
      definición, probando el resultado. 

      $\textbf{\isa{{\isasymthree}{\isacharparenright}\ F\ {\isacharequal}\ {\isasymnot}{\isacharparenleft}G\ {\isasymand}\ H{\isacharparenright}{\isacharcomma}\ para\ ciertas\ fórmulas\ G\ y\ H}:}$ Como \isa{{\isacharbraceleft}F{\isacharbraceright}\ {\isasymunion}\ W\isactrlsub {\isadigit{0}}} es satisfacible,
      existe una interpretación \isa{{\isasymA}} que es modelo de todas sus fórmulas y, en particular, de \isa{F}.
      Luego, por definición del valor de una fórmula en una interpretación, obtenemos que \isa{{\isasymA}} no
      es modelo de \isa{G\ {\isasymand}\ H}. De nuevo por definición, esto implica que no es cierto que \isa{{\isasymA}} sea 
      modelo de \isa{G} y de \isa{H}. Es decir, o bien \isa{{\isasymA}} no es modelo de \isa{G} o bien \isa{{\isasymA}} no es modelo de
      \isa{H}. Si suponemos que no es modelo de \isa{G}, por definición se obtiene que \isa{{\isasymA}} es modelo de\\
      \isa{{\isasymnot}\ G}. Como en este caso la componente disyuntiva \isa{{\isasymbeta}\isactrlsub {\isadigit{1}}} es \isa{{\isasymnot}\ G}, se deduce que \isa{{\isasymA}} es modelo
      de \isa{{\isasymbeta}\isactrlsub {\isadigit{1}}}. Por tanto, \isa{{\isasymA}} es modelo de todas las fórmulas del conjunto \isa{{\isacharbraceleft}{\isasymbeta}\isactrlsub {\isadigit{1}}{\isacharcomma}F{\isacharbraceright}\ {\isasymunion}\ W\isactrlsub {\isadigit{0}}}, por lo que
      se demuestra que dicho conjunto es satisfacible, probando el resultado. Por otro lado, si
      suponemos que \isa{{\isasymA}} no es modelo de \isa{H}, se tiene que sí lo es de \isa{{\isasymnot}\ H}. Como \isa{{\isasymbeta}\isactrlsub {\isadigit{2}}} es \isa{{\isasymnot}\ H}
      en este caso, obtenemos que \isa{{\isasymA}} es modelo de \isa{{\isasymbeta}\isactrlsub {\isadigit{2}}}. Luego \isa{{\isasymA}} es modelo de todas las fórmulas
      de \isa{{\isacharbraceleft}{\isasymbeta}\isactrlsub {\isadigit{2}}{\isacharcomma}F{\isacharbraceright}\ {\isasymunion}\ W\isactrlsub {\isadigit{0}}}, demostrando así que es un conjunto satisfacible. Por tanto, se demuestra
      el resultado en ambos casos.

      $\textbf{\isa{{\isasymfour}{\isacharparenright}\ F\ {\isacharequal}\ {\isasymnot}{\isacharparenleft}{\isasymnot}\ G{\isacharparenright}{\isacharcomma}\ para\ cierta\ fórmula\ G}:}$ Puesto que \isa{{\isacharbraceleft}F{\isacharbraceright}\ {\isasymunion}\ W\isactrlsub {\isadigit{0}}} es satisfacible, 
      existe una interpretación \isa{{\isasymA}} modelo de todas sus fórmulas y, en particular, modelo de \isa{F}. 
      Luego, por definición del valor de una fórmula en una interpretación obtenemos que no es 
      cierto que \isa{{\isasymA}} no sea modelo de \isa{G}, es decir, \isa{{\isasymA}} es modelo de \isa{G}. Como las componentes 
      \isa{{\isasymbeta}\isactrlsub {\isadigit{1}}} y \isa{{\isasymbeta}\isactrlsub {\isadigit{2}}} son ambas \isa{G} en este caso, se obtiene que \isa{{\isasymA}} es modelo suyo. En particular, lo 
      es de \isa{{\isasymbeta}\isactrlsub {\isadigit{1}}}, de modo que \isa{{\isasymA}} es modelo de todas las fórmulas de \isa{{\isacharbraceleft}{\isasymbeta}\isactrlsub {\isadigit{1}}{\isacharcomma}F{\isacharbraceright}\ {\isasymunion}\ W\isactrlsub {\isadigit{0}}}, probando así que 
      es satisfacible. Por lo tanto, se verifica el resultado.
    
    En conclusión, hemos probado que o bien \isa{{\isacharbraceleft}{\isasymbeta}\isactrlsub {\isadigit{1}}{\isacharcomma}F{\isacharbraceright}\ {\isasymunion}\ W\isactrlsub o} es satisfacible o bien\\ \isa{{\isacharbraceleft}{\isasymbeta}\isactrlsub {\isadigit{2}}{\isacharcomma}F{\isacharbraceright}\ {\isasymunion}\ W\isactrlsub o} es 
    satisfacible. Por lo tanto, se tiene que no es cierto que ninguno de los dos conjuntos sea
    insatisfacible. Esto contradice lo demostrado anteriormente, llegando así a una contradicción
    que prueba por reducción al absurdo la última condición del lema \isa{{\isadigit{2}}{\isachardot}{\isadigit{0}}{\isachardot}{\isadigit{2}}}. De este modo, queda
    probado que la colección formada por los conjuntos de fórmulas finitamente satisfacibles 
    verifica la propiedad de consistencia proposicional y, por el \isa{Teorema\ de\ Existencia\ de\ Modelo}, 
    todo conjunto perteneciente a ella es satisfacible, lo que demuestra el teorema.
  \end{demostracion}

  Procedamos con la demostración detallada del \isa{Teorema\ de\ Compacidad} en Isabelle/HOL. Para ello, 
  definamos la colección de conjuntos finitamente satisfacibles en Isabelle/HOL. En adelante
  notaremos por \isa{C} a dicha colección.%
\end{isamarkuptext}\isamarkuptrue%
\isacommand{definition}\isamarkupfalse%
\ colecComp\ {\isacharcolon}{\isacharcolon}\ {\isachardoublequoteopen}{\isacharparenleft}{\isacharprime}a\ formula\ set{\isacharparenright}\ set{\isachardoublequoteclose}\isanewline
\ \ \isakeyword{where}\ colecComp{\isacharcolon}\ {\isachardoublequoteopen}colecComp\ {\isacharequal}\ {\isacharbraceleft}W{\isachardot}\ fin{\isacharunderscore}sat\ W{\isacharbraceright}{\isachardoublequoteclose}%
\begin{isamarkuptext}%
Para facilitar la demostración introduciremos el siguiente lema auxiliar que prueba que
  todo subconjunto finito de un conjunto perteneciente a la colección anterior es satisfacible.%
\end{isamarkuptext}\isamarkuptrue%
\isacommand{lemma}\isamarkupfalse%
\ colecComp{\isacharunderscore}subset{\isacharunderscore}finite{\isacharcolon}\ \isanewline
\ \ \isakeyword{assumes}\ {\isachardoublequoteopen}W\ {\isasymin}\ colecComp{\isachardoublequoteclose}\isanewline
\ \ \ \ \ \ \ \ \ \ {\isachardoublequoteopen}Wo\ {\isasymsubseteq}\ W{\isachardoublequoteclose}\isanewline
\ \ \ \ \ \ \ \ \ \ {\isachardoublequoteopen}finite\ Wo{\isachardoublequoteclose}\isanewline
\ \ \isakeyword{shows}\ {\isachardoublequoteopen}sat\ Wo{\isachardoublequoteclose}\ \isanewline
%
\isadelimproof
%
\endisadelimproof
%
\isatagproof
\isacommand{proof}\isamarkupfalse%
\ {\isacharminus}\isanewline
\ \ \isacommand{have}\isamarkupfalse%
\ {\isachardoublequoteopen}{\isasymforall}Wo\ {\isasymsubseteq}\ W{\isachardot}\ finite\ Wo\ {\isasymlongrightarrow}\ sat\ Wo{\isachardoublequoteclose}\isanewline
\ \ \ \ \isacommand{using}\isamarkupfalse%
\ assms{\isacharparenleft}{\isadigit{1}}{\isacharparenright}\ \isacommand{unfolding}\isamarkupfalse%
\ colecComp\ fin{\isacharunderscore}sat{\isacharunderscore}def\ \isacommand{by}\isamarkupfalse%
\ {\isacharparenleft}rule\ CollectD{\isacharparenright}\isanewline
\ \ \isacommand{then}\isamarkupfalse%
\ \isacommand{have}\isamarkupfalse%
\ {\isachardoublequoteopen}finite\ Wo\ {\isasymlongrightarrow}\ sat\ Wo{\isachardoublequoteclose}\isanewline
\ \ \ \ \isacommand{using}\isamarkupfalse%
\ {\isacartoucheopen}Wo\ {\isasymsubseteq}\ W{\isacartoucheclose}\ \isacommand{by}\isamarkupfalse%
\ {\isacharparenleft}rule\ sspec{\isacharparenright}\isanewline
\ \ \isacommand{thus}\isamarkupfalse%
\ {\isachardoublequoteopen}sat\ Wo{\isachardoublequoteclose}\isanewline
\ \ \ \ \isacommand{using}\isamarkupfalse%
\ {\isacartoucheopen}finite\ Wo{\isacartoucheclose}\ \isacommand{by}\isamarkupfalse%
\ {\isacharparenleft}rule\ mp{\isacharparenright}\isanewline
\isacommand{qed}\isamarkupfalse%
%
\endisatagproof
{\isafoldproof}%
%
\isadelimproof
%
\endisadelimproof
%
\begin{isamarkuptext}%
Para facilitar la comprensión de la demostración, mostraremos a continuación un grafo que 
  estructura las relaciones de necesidad de los lemas auxiliares empleados.

\comentario{Poner grafo bien.}

\begin{tikzpicture}
  [
    grow                    = down,
    level 1/.style          = {sibling distance=6cm},
    level 2/.style          = {sibling distance=4.5cm},
    level 3/.style          = {sibling distance=8cm}
    level 4/.style          = {sibling distance=4cm}
    level 5/.style          = {sibling distance=5cm}
    level 6/.style          = {sibling distance=5cm}
    level 7/.style          = {sibling distance=5cm};
    level distance          = 4.5cm,
    edge from parent/.style = {draw},
    every node/.style       = {font=\tiny},
    sloped
  ]
\raggedright
  \node [root] {\isa{prop{\isacharunderscore}Compactness}\\ \isa{{\isacharparenleft}Teorema\ de\ Compacidad\ {\isacharparenleft}{\isadigit{4}}{\isachardot}{\isadigit{3}}{\isachardot}{\isadigit{1}}{\isacharparenright}{\isacharparenright}}}
    child { node [env] {\isa{set{\isacharunderscore}in{\isacharunderscore}colecComp}\\ \isa{{\isacharparenleft}W\ {\isasymin}\ C{\isacharparenright}}}}
    child { node [env] {\isa{pcp{\isacharunderscore}colecComp}\\ \isa{{\isacharparenleft}C\ tiene\ la\ propiedad\ de\ consistencia\ proposicional{\isacharparenright}}}
          child { node [env] {\isa{pcp{\isacharunderscore}colecComp{\isacharunderscore}bot}\\ \isa{{\isacharparenleft}{\isasymbottom}\ {\isasymnotin}\ W{\isacharparenright}}}
              child { node [env] {\isa{not{\isacharunderscore}sat{\isacharunderscore}bot}\\ \isa{{\isacharparenleft}{\isacharbraceleft}{\isasymbottom}{\isacharbraceright}\ es\ insatisfacible{\isacharparenright}}}}}
          child { node [env] {\isa{pcp{\isacharunderscore}colecComp{\isacharunderscore}atoms}\\ \isa{{\isacharparenleft}Cond{\isachardot}\ fórmulas\ atómicas{\isacharparenright}}}
              child { node [env] {\isa{not{\isacharunderscore}sat{\isacharunderscore}atoms}\\ \isa{{\isacharparenleft}{\isacharbraceleft}p{\isacharcomma}{\isasymnot}\ p{\isacharbraceright}\ es\ insatisfacible{\isacharparenright}}}}}
      		child { node [env] {\isa{pcp{\isacharunderscore}colecComp{\isacharunderscore}CON}\\ \isa{{\isacharparenleft}Cond{\isachardot}\ fórmulas\ de\ tipo\ {\isasymalpha}{\isacharparenright}}}
        			child { node [env] {\isa{pcp{\isacharunderscore}colecComp{\isacharunderscore}CON{\isacharunderscore}sat}\\ \isa{{\isacharparenleft}Resultado\ {\isasymone}{\isacharparenright}}}
                      child { node [env] {\isa{pcp{\isacharunderscore}colecComp{\isacharunderscore}CON{\isacharunderscore}sat{\isadigit{1}}}\\\isa{pcp{\isacharunderscore}colecComp{\isacharunderscore}CON{\isacharunderscore}sat{\isadigit{2}}}\\\isa{pcp{\isacharunderscore}colecComp{\isacharunderscore}CON{\isacharunderscore}sat{\isadigit{3}}}\\\isa{pcp{\isacharunderscore}colecComp{\isacharunderscore}CON{\isacharunderscore}sat{\isadigit{4}}}}}}}
        			child { node [env] {\isa{pcp{\isacharunderscore}colecComp{\isacharunderscore}DIS}\\ \isa{{\isacharparenleft}Cond{\isachardot}\ fórmulas\ de\ tipo\ {\isasymbeta}{\isacharparenright}}}
                      child { node [env] {\isa{not{\isacharunderscore}colecComp}\\ \isa{{\isacharparenleft}Resultado\ {\isasymtwo}{\isacharparenright}}}
                            child { node [env] {\isa{sat{\isacharunderscore}subset{\isacharunderscore}ccontr}\\ \isa{{\isacharparenleft}Los\ conjuntos\ que}\\ \isa{contienen\ algún}\\ \isa{subconjunto\ insatisfacible}\\ \isa{son\ insatisfacibles{\isacharparenright}}}}}
                                  child { node [env] {\isa{pcp{\isacharunderscore}colecComp{\isacharunderscore}DIS{\isacharunderscore}sat}\\ \isa{{\isacharparenleft}Resultado\ {\isasymthree}{\isacharparenright}}}
                                  child { node [env] {\isa{pcp{\isacharunderscore}colecComp{\isacharunderscore}DIS{\isacharunderscore}sat{\isadigit{1}}}\\\isa{pcp{\isacharunderscore}colecComp{\isacharunderscore}DIS{\isacharunderscore}sat{\isadigit{2}}}\\\isa{pcp{\isacharunderscore}colecComp{\isacharunderscore}DIS{\isacharunderscore}sat{\isadigit{3}}}\\\isa{pcp{\isacharunderscore}colecComp{\isacharunderscore}DIS{\isacharunderscore}sat{\isadigit{4}}}}}}}};
\end{tikzpicture}

  De este modo, el \isa{Teorema\ de\ Compacidad} se demuestra aplicando el \isa{Teorema\ de}\\ \isa{Existencia\ de\ Modelo} a la colección \isa{C}. Por tanto, basta probar que dado un conjunto finitamente satisfacible 
  \isa{W} se tiene que \isa{W\ {\isasymin}\ C} (formalizado mediante el lema auxiliar \isa{set{\isacharunderscore}in{\isacharunderscore}colecComp}) y que \isa{C} 
  verifica la propiedad de consistencia proposicional (formalizado como \isa{pcp{\isacharunderscore}colecComp}). Por el 
  lema \isa{{\isadigit{2}}{\isachardot}{\isadigit{0}}{\isachardot}{\isadigit{2}}}, es suficiente probar las siguientes condiciones dado un conjunto \isa{W\ {\isasymin}\ C} cualquiera:
    \begin{enumerate}
     \item \isa{{\isasymbottom}\ {\isasymnotin}\ W}. (\isa{{\isasymLongrightarrow}} formalizado como \isa{pcp{\isacharunderscore}colecComp{\isacharunderscore}sat})
     \item Dada \isa{p} una fórmula atómica cualquiera, no se tiene 
      simultáneamente que\\ \isa{p\ {\isasymin}\ W} y \isa{{\isasymnot}\ p\ {\isasymin}\ W}. (\isa{{\isasymLongrightarrow}} formalizado como \isa{pcp{\isacharunderscore}colecComp{\isacharunderscore}atoms})
     \item Para toda fórmula de tipo \isa{{\isasymalpha}} con componentes \isa{{\isasymalpha}\isactrlsub {\isadigit{1}}} y \isa{{\isasymalpha}\isactrlsub {\isadigit{2}}} tal que \isa{{\isasymalpha}}
      pertenece a \isa{W}, se tiene que \isa{{\isacharbraceleft}{\isasymalpha}\isactrlsub {\isadigit{1}}{\isacharcomma}{\isasymalpha}\isactrlsub {\isadigit{2}}{\isacharbraceright}\ {\isasymunion}\ W\ {\isasymin}\ C}. (\isa{{\isasymLongrightarrow}} formalizado como 
      \isa{pcp{\isacharunderscore}colecComp{\isacharunderscore}CON})
     \item Para toda fórmula de tipo \isa{{\isasymbeta}} con componentes \isa{{\isasymbeta}\isactrlsub {\isadigit{1}}} y \isa{{\isasymbeta}\isactrlsub {\isadigit{2}}} tal que \isa{{\isasymbeta}}
      pertenece a \isa{W}, se tiene que o bien \isa{{\isacharbraceleft}{\isasymbeta}\isactrlsub {\isadigit{1}}{\isacharbraceright}\ {\isasymunion}\ W\ {\isasymin}\ C} o 
      bien \isa{{\isacharbraceleft}{\isasymbeta}\isactrlsub {\isadigit{2}}{\isacharbraceright}\ {\isasymunion}\ W\ {\isasymin}\ C}.\\ (\isa{{\isasymLongrightarrow}} formalizado como \isa{pcp{\isacharunderscore}colecComp{\isacharunderscore}DIS})
    \end{enumerate}
  A su vez, cada uno de los lemas auxiliares que prueban las condiciones anteriores precisa de los
  siguientes lemas:
  \begin{itemize}
    \item \isa{pcp{\isacharunderscore}colecComp{\isacharunderscore}sat}: Se prueba por reducción al absurdo mediante el lema \isa{not{\isacharunderscore}sat{\isacharunderscore}bot} que
    demuestra la insatisfacibilidad del conjunto \isa{{\isacharbraceleft}{\isasymbottom}{\isacharbraceright}}.
    \item \isa{pcp{\isacharunderscore}colecComp{\isacharunderscore}atoms}: Su demostración es por reducción al absurdo empleando el lema
    \isa{not{\isacharunderscore}sat{\isacharunderscore}atoms} que prueba la insatisfacibilidad del conjunto \isa{{\isacharbraceleft}p{\isacharcomma}{\isasymnot}\ p{\isacharbraceright}} para cualquier fórmula
    atómica \isa{p}.
    \item \isa{pcp{\isacharunderscore}colecComp{\isacharunderscore}CON}: Para su prueba, se precisa del \isa{resultado\ {\isasymone}}, formalizado como 
    \isa{pcp{\isacharunderscore}colecComp{\isacharunderscore}CON{\isacharunderscore}sat}. Este demuestra que dados \isa{W\ {\isasymin}\ C}, \isa{F\ {\isasymin}\ W} una fórmula de tipo 
    \isa{{\isasymalpha}} con componentes \isa{{\isasymalpha}\isactrlsub {\isadigit{1}}} y \isa{{\isasymalpha}\isactrlsub {\isadigit{2}}} y \isa{W\isactrlsub {\isadigit{0}}} un subconjunto finito de \isa{W}, se verifica que 
    \isa{{\isacharbraceleft}{\isasymalpha}\isactrlsub {\isadigit{1}}{\isacharcomma}{\isasymalpha}\isactrlsub {\isadigit{2}}{\isacharcomma}F{\isacharbraceright}\ {\isasymunion}\ W\isactrlsub {\isadigit{0}}} es satisfacible. Para probar dicho resultado se emplean a su vez los lemas
    auxiliares \isa{pcp{\isacharunderscore}colecComp{\isacharunderscore}CON{\isacharunderscore}sat{\isadigit{1}}}, \isa{pcp{\isacharunderscore}colecComp{\isacharunderscore}CON{\isacharunderscore}sat{\isadigit{2}}}, \isa{pcp{\isacharunderscore}colecComp{\isacharunderscore}CON{\isacharunderscore}sat{\isadigit{3}}} y 
    \isa{pcp{\isacharunderscore}colecComp{\isacharunderscore}CON{\isacharunderscore}sat{\isadigit{4}}} que demuestran el enunciado para cada tipo de fórmula \isa{{\isasymalpha}}.
    \item \isa{pcp{\isacharunderscore}colecComp{\isacharunderscore}DIS}: La prueba se realizará por reducción al absurdo. Para ello
    precisa de dos resultados.
    \begin{itemize}
      \item \isa{Resultado\ {\isasymtwo}}: Dados \isa{W\ {\isasymin}\ C} y \isa{{\isasymbeta}\isactrlsub i} una fórmula cualquiera tal que\\ \isa{{\isacharbraceleft}{\isasymbeta}\isactrlsub i{\isacharbraceright}\ {\isasymunion}\ W\ {\isasymnotin}\ C}, 
      entonces existe un subconjunto finito \isa{W\isactrlsub i} de \isa{W} tal que el conjunto \isa{{\isacharbraceleft}{\isasymbeta}\isactrlsub i{\isacharbraceright}\ {\isasymunion}\ W\isactrlsub i} no es 
      satisfacible. En Isabelle ha sido formalizado como \isa{not{\isacharunderscore}colecComp}. A su vez, ha precisado
      para su prueba del lema auxiliar \isa{sat{\isacharunderscore}subset{\isacharunderscore}ccontr} que demuestra que todo conjunto de 
      fórmulas que tenga un subconjunto insatisfacible es también insatisfacible.
      \item \isa{Resultado\ {\isasymthree}}: Dados \isa{W\ {\isasymin}\ C}, \isa{F} una fórmula de tipo \isa{{\isasymbeta}} con componentes \isa{{\isasymbeta}\isactrlsub {\isadigit{1}}} y \isa{{\isasymbeta}\isactrlsub {\isadigit{2}}} 
      tal que \isa{F\ {\isasymin}\ W} y \isa{W\isactrlsub {\isadigit{0}}} un subconjunto finito de \isa{W}, entonces se tiene que o bien 
      \isa{{\isacharbraceleft}{\isasymbeta}\isactrlsub {\isadigit{1}}{\isacharcomma}F{\isacharbraceright}\ {\isasymunion}\ W\isactrlsub {\isadigit{0}}} es satisfacible o bien \isa{{\isacharbraceleft}{\isasymbeta}\isactrlsub {\isadigit{2}}{\isacharcomma}F{\isacharbraceright}\ {\isasymunion}\ W\isactrlsub {\isadigit{0}}} es satisfacible. En Isabelle se ha
      formalizado como \isa{pcp{\isacharunderscore}colecComp{\isacharunderscore}DIS{\isacharunderscore}sat}. Para su prueba, ha precisado de cuatro lemas
      auxiliares que prueban el resultado para cada caso de fórmula de tipo \isa{{\isasymbeta}}: 
      \isa{pcp{\isacharunderscore}colecComp{\isacharunderscore}DIS{\isacharunderscore}sat{\isadigit{1}}}, \isa{pcp{\isacharunderscore}colecComp{\isacharunderscore}DIS{\isacharunderscore}sat{\isadigit{2}}}, \isa{pcp{\isacharunderscore}colecComp{\isacharunderscore}DIS{\isacharunderscore}sat{\isadigit{3}}},
      \isa{pcp{\isacharunderscore}colecComp{\isacharunderscore}DIS{\isacharunderscore}sat{\isadigit{4}}}.
    \end{itemize}
  \end{itemize}

  Comencemos con las demostraciones de los lemas auxiliares empleados en la prueba del teorema.
  El siguiente lema prueba que si un conjunto es finitamente satisfacible, entonces pertenece a \isa{C}.%
\end{isamarkuptext}\isamarkuptrue%
\isacommand{lemma}\isamarkupfalse%
\ set{\isacharunderscore}in{\isacharunderscore}colecComp{\isacharcolon}\ \isanewline
\ \ \isakeyword{assumes}\ {\isachardoublequoteopen}fin{\isacharunderscore}sat\ S{\isachardoublequoteclose}\isanewline
\ \ \isakeyword{shows}\ {\isachardoublequoteopen}S\ {\isasymin}\ colecComp{\isachardoublequoteclose}\isanewline
%
\isadelimproof
\ \ %
\endisadelimproof
%
\isatagproof
\isacommand{unfolding}\isamarkupfalse%
\ colecComp\ \isacommand{using}\isamarkupfalse%
\ assms\ \isacommand{unfolding}\isamarkupfalse%
\ fin{\isacharunderscore}sat{\isacharunderscore}def\ \isacommand{by}\isamarkupfalse%
\ {\isacharparenleft}rule\ CollectI{\isacharparenright}%
\endisatagproof
{\isafoldproof}%
%
\isadelimproof
%
\endisadelimproof
%
\begin{isamarkuptext}%
Probemos ahora que \isa{C} verifica la propiedad de consistencia proposicional. Para ello, dado un 
  conjunto \isa{W\ {\isasymin}\ C}, probaremos por separado que se verifican cada una de las condiciones del 
  lema \isa{{\isadigit{2}}{\isachardot}{\isadigit{0}}{\isachardot}{\isadigit{2}}}.
  
  En primer lugar, veamos que \isa{{\isasymbottom}\ {\isasymnotin}\ W} si \isa{W\ {\isasymin}\ C}. Para ello, precisaremos del siguiente lema 
  auxiliar que prueba que el conjunto \isa{{\isacharbraceleft}{\isasymbottom}{\isacharbraceright}} no es satisfacible.%
\end{isamarkuptext}\isamarkuptrue%
\isacommand{lemma}\isamarkupfalse%
\ not{\isacharunderscore}sat{\isacharunderscore}bot{\isacharcolon}\ {\isachardoublequoteopen}{\isasymnot}\ sat\ {\isacharbraceleft}{\isasymbottom}{\isacharbraceright}{\isachardoublequoteclose}\isanewline
%
\isadelimproof
%
\endisadelimproof
%
\isatagproof
\isacommand{proof}\isamarkupfalse%
\ {\isacharparenleft}rule\ ccontr{\isacharparenright}\isanewline
\ \ \isacommand{assume}\isamarkupfalse%
\ {\isachardoublequoteopen}{\isasymnot}{\isacharparenleft}{\isasymnot}sat{\isacharbraceleft}{\isasymbottom}\ {\isacharcolon}{\isacharcolon}\ {\isacharprime}a\ formula{\isacharbraceright}{\isacharparenright}{\isachardoublequoteclose}\isanewline
\ \ \isacommand{then}\isamarkupfalse%
\ \isacommand{have}\isamarkupfalse%
\ {\isachardoublequoteopen}sat\ {\isacharbraceleft}{\isasymbottom}\ {\isacharcolon}{\isacharcolon}\ {\isacharprime}a\ formula{\isacharbraceright}{\isachardoublequoteclose}\isanewline
\ \ \ \ \isacommand{by}\isamarkupfalse%
\ {\isacharparenleft}rule\ notnotD{\isacharparenright}\isanewline
\ \ \isacommand{then}\isamarkupfalse%
\ \isacommand{have}\isamarkupfalse%
\ Ex{\isacharcolon}{\isachardoublequoteopen}{\isasymexists}{\isasymA}{\isachardot}\ {\isasymforall}F\ {\isasymin}\ {\isacharbraceleft}{\isasymbottom}\ {\isacharcolon}{\isacharcolon}\ {\isacharprime}a\ formula{\isacharbraceright}{\isachardot}\ {\isasymA}\ {\isasymTurnstile}\ F{\isachardoublequoteclose}\isanewline
\ \ \ \ \isacommand{by}\isamarkupfalse%
\ {\isacharparenleft}simp\ only{\isacharcolon}\ sat{\isacharunderscore}def{\isacharparenright}\isanewline
\ \ \isacommand{obtain}\isamarkupfalse%
\ {\isasymA}\ \isakeyword{where}\ {\isadigit{1}}{\isacharcolon}{\isachardoublequoteopen}{\isasymforall}F\ {\isasymin}\ {\isacharbraceleft}{\isasymbottom}\ {\isacharcolon}{\isacharcolon}\ {\isacharprime}a\ formula{\isacharbraceright}{\isachardot}\ {\isasymA}\ {\isasymTurnstile}\ F{\isachardoublequoteclose}\isanewline
\ \ \ \ \isacommand{using}\isamarkupfalse%
\ Ex\ \isacommand{by}\isamarkupfalse%
\ {\isacharparenleft}rule\ exE{\isacharparenright}\isanewline
\ \ \isacommand{have}\isamarkupfalse%
\ {\isadigit{2}}{\isacharcolon}{\isachardoublequoteopen}{\isasymbottom}\ {\isasymin}\ {\isacharbraceleft}{\isasymbottom}{\isacharcolon}{\isacharcolon}\ {\isacharprime}a\ formula{\isacharbraceright}{\isachardoublequoteclose}\isanewline
\ \ \ \ \isacommand{by}\isamarkupfalse%
\ {\isacharparenleft}simp\ only{\isacharcolon}\ singletonI{\isacharparenright}\isanewline
\ \ \isacommand{have}\isamarkupfalse%
\ {\isachardoublequoteopen}{\isasymA}\ {\isasymTurnstile}\ {\isasymbottom}{\isachardoublequoteclose}\isanewline
\ \ \ \ \isacommand{using}\isamarkupfalse%
\ {\isadigit{1}}\ {\isadigit{2}}\ \isacommand{by}\isamarkupfalse%
\ {\isacharparenleft}rule\ bspec{\isacharparenright}\isanewline
\ \ \isacommand{thus}\isamarkupfalse%
\ {\isachardoublequoteopen}False{\isachardoublequoteclose}\isanewline
\ \ \ \ \isacommand{by}\isamarkupfalse%
\ {\isacharparenleft}simp\ only{\isacharcolon}\ formula{\isacharunderscore}semantics{\isachardot}simps{\isacharparenleft}{\isadigit{2}}{\isacharparenright}{\isacharparenright}\isanewline
\isacommand{qed}\isamarkupfalse%
%
\endisatagproof
{\isafoldproof}%
%
\isadelimproof
%
\endisadelimproof
%
\begin{isamarkuptext}%
Por tanto, probemos que si \isa{W\ {\isasymin}\ C}, entonces \isa{{\isasymbottom}\ {\isasymnotin}\ W}.%
\end{isamarkuptext}\isamarkuptrue%
\isacommand{lemma}\isamarkupfalse%
\ pcp{\isacharunderscore}colecComp{\isacharunderscore}bot{\isacharcolon}\isanewline
\ \ \isakeyword{assumes}\ {\isachardoublequoteopen}W\ {\isasymin}\ colecComp{\isachardoublequoteclose}\isanewline
\ \ \isakeyword{shows}\ {\isachardoublequoteopen}{\isasymbottom}\ {\isasymnotin}\ W{\isachardoublequoteclose}\isanewline
%
\isadelimproof
%
\endisadelimproof
%
\isatagproof
\isacommand{proof}\isamarkupfalse%
\ {\isacharparenleft}rule\ ccontr{\isacharparenright}\isanewline
\ \ \isacommand{assume}\isamarkupfalse%
\ {\isachardoublequoteopen}{\isasymnot}{\isacharparenleft}{\isasymbottom}\ {\isasymnotin}\ W{\isacharparenright}{\isachardoublequoteclose}\isanewline
\ \ \isacommand{then}\isamarkupfalse%
\ \isacommand{have}\isamarkupfalse%
\ {\isachardoublequoteopen}{\isasymbottom}\ {\isasymin}\ W{\isachardoublequoteclose}\isanewline
\ \ \ \ \isacommand{by}\isamarkupfalse%
\ {\isacharparenleft}rule\ notnotD{\isacharparenright}\isanewline
\ \ \isacommand{have}\isamarkupfalse%
\ {\isachardoublequoteopen}{\isacharbraceleft}{\isacharbraceright}\ {\isasymsubseteq}\ W{\isachardoublequoteclose}\ \isanewline
\ \ \ \ \isacommand{by}\isamarkupfalse%
\ {\isacharparenleft}simp\ only{\isacharcolon}\ empty{\isacharunderscore}subsetI{\isacharparenright}\ \isanewline
\ \ \isacommand{have}\isamarkupfalse%
\ {\isachardoublequoteopen}{\isasymbottom}\ {\isasymin}\ W\ {\isasymand}\ {\isacharbraceleft}{\isacharbraceright}\ {\isasymsubseteq}\ W{\isachardoublequoteclose}\isanewline
\ \ \ \ \isacommand{using}\isamarkupfalse%
\ {\isacartoucheopen}{\isasymbottom}\ {\isasymin}\ W{\isacartoucheclose}\ {\isacartoucheopen}{\isacharbraceleft}{\isacharbraceright}\ {\isasymsubseteq}\ W{\isacartoucheclose}\ \isacommand{by}\isamarkupfalse%
\ {\isacharparenleft}rule\ conjI{\isacharparenright}\isanewline
\ \ \isacommand{then}\isamarkupfalse%
\ \isacommand{have}\isamarkupfalse%
\ {\isachardoublequoteopen}{\isacharbraceleft}{\isasymbottom}{\isacharbraceright}\ {\isasymsubseteq}\ W{\isachardoublequoteclose}\isanewline
\ \ \ \ \isacommand{by}\isamarkupfalse%
\ {\isacharparenleft}simp\ only{\isacharcolon}\ insert{\isacharunderscore}subset{\isacharparenright}\isanewline
\ \ \isacommand{have}\isamarkupfalse%
\ {\isachardoublequoteopen}finite\ {\isacharbraceleft}{\isasymbottom}{\isacharbraceright}{\isachardoublequoteclose}\ \isanewline
\ \ \ \ \isacommand{by}\isamarkupfalse%
\ {\isacharparenleft}simp\ only{\isacharcolon}\ finite{\isachardot}emptyI\ finite{\isacharunderscore}insert{\isacharparenright}\isanewline
\ \ \isacommand{have}\isamarkupfalse%
\ {\isachardoublequoteopen}sat\ {\isacharbraceleft}{\isasymbottom}\ {\isacharcolon}{\isacharcolon}\ {\isacharprime}a\ formula{\isacharbraceright}{\isachardoublequoteclose}\ \isanewline
\ \ \ \ \isacommand{using}\isamarkupfalse%
\ assms\ {\isacartoucheopen}{\isacharbraceleft}{\isasymbottom}{\isacharbraceright}\ {\isasymsubseteq}\ W{\isacartoucheclose}\ {\isacartoucheopen}finite\ {\isacharbraceleft}{\isasymbottom}{\isacharbraceright}{\isacartoucheclose}\ \isacommand{by}\isamarkupfalse%
\ {\isacharparenleft}rule\ colecComp{\isacharunderscore}subset{\isacharunderscore}finite{\isacharparenright}\isanewline
\ \ \isacommand{have}\isamarkupfalse%
\ {\isachardoublequoteopen}{\isasymnot}\ sat\ {\isacharbraceleft}{\isasymbottom}\ {\isacharcolon}{\isacharcolon}\ {\isacharprime}a\ formula{\isacharbraceright}{\isachardoublequoteclose}\ \isanewline
\ \ \ \ \isacommand{by}\isamarkupfalse%
\ {\isacharparenleft}rule\ not{\isacharunderscore}sat{\isacharunderscore}bot{\isacharparenright}\isanewline
\ \ \isacommand{then}\isamarkupfalse%
\ \isacommand{show}\isamarkupfalse%
\ False\ \isanewline
\ \ \ \ \isacommand{using}\isamarkupfalse%
\ {\isacartoucheopen}sat\ {\isacharbraceleft}{\isasymbottom}\ {\isacharcolon}{\isacharcolon}\ {\isacharprime}a\ formula{\isacharbraceright}{\isacartoucheclose}\ \isacommand{by}\isamarkupfalse%
\ {\isacharparenleft}rule\ notE{\isacharparenright}\isanewline
\isacommand{qed}\isamarkupfalse%
%
\endisatagproof
{\isafoldproof}%
%
\isadelimproof
%
\endisadelimproof
%
\begin{isamarkuptext}%
Por otro lado, vamos a probar que dado un conjunto \isa{W\ {\isasymin}\ C} y \isa{p} una fórmula atómica 
  cualquiera, no se tiene simultáneamente que \isa{p\ {\isasymin}\ W} y \isa{{\isasymnot}\ p\ {\isasymin}\ W}. Para ello, emplearemos el 
  siguiente lema auxiliar que prueba que el conjunto \isa{{\isacharbraceleft}p{\isacharcomma}{\isasymnot}\ p{\isacharbraceright}} es insatisfacible para cualquier 
  fórmula atómica \isa{p}.%
\end{isamarkuptext}\isamarkuptrue%
\isacommand{lemma}\isamarkupfalse%
\ not{\isacharunderscore}sat{\isacharunderscore}atoms{\isacharcolon}\ {\isachardoublequoteopen}{\isasymnot}\ sat{\isacharparenleft}{\isacharbraceleft}Atom\ k{\isacharcomma}\ \isactrlbold {\isasymnot}\ {\isacharparenleft}Atom\ k{\isacharparenright}{\isacharbraceright}{\isacharparenright}{\isachardoublequoteclose}\isanewline
%
\isadelimproof
%
\endisadelimproof
%
\isatagproof
\isacommand{proof}\isamarkupfalse%
\ {\isacharparenleft}rule\ ccontr{\isacharparenright}\isanewline
\ \ \isacommand{assume}\isamarkupfalse%
\ {\isachardoublequoteopen}{\isasymnot}\ {\isasymnot}\ sat{\isacharparenleft}{\isacharbraceleft}Atom\ k{\isacharcomma}\ \isactrlbold {\isasymnot}\ {\isacharparenleft}Atom\ k{\isacharparenright}{\isacharbraceright}{\isacharparenright}{\isachardoublequoteclose}\isanewline
\ \ \isacommand{then}\isamarkupfalse%
\ \isacommand{have}\isamarkupfalse%
\ {\isachardoublequoteopen}sat{\isacharparenleft}{\isacharbraceleft}Atom\ k{\isacharcomma}\ \isactrlbold {\isasymnot}\ {\isacharparenleft}Atom\ k{\isacharparenright}{\isacharbraceright}{\isacharparenright}{\isachardoublequoteclose}\isanewline
\ \ \ \ \isacommand{by}\isamarkupfalse%
\ {\isacharparenleft}rule\ notnotD{\isacharparenright}\isanewline
\ \ \isacommand{then}\isamarkupfalse%
\ \isacommand{have}\isamarkupfalse%
\ Sat{\isacharcolon}{\isachardoublequoteopen}{\isasymexists}{\isasymA}{\isachardot}\ {\isasymforall}F\ {\isasymin}\ {\isacharbraceleft}Atom\ k{\isacharcomma}\ \isactrlbold {\isasymnot}{\isacharparenleft}Atom\ k{\isacharparenright}{\isacharbraceright}{\isachardot}\ {\isasymA}\ {\isasymTurnstile}\ F{\isachardoublequoteclose}\isanewline
\ \ \ \ \isacommand{by}\isamarkupfalse%
\ {\isacharparenleft}simp\ only{\isacharcolon}\ sat{\isacharunderscore}def{\isacharparenright}\isanewline
\ \ \isacommand{obtain}\isamarkupfalse%
\ {\isasymA}\ \isakeyword{where}\ H{\isacharcolon}{\isachardoublequoteopen}{\isasymforall}F\ {\isasymin}\ {\isacharbraceleft}Atom\ k{\isacharcomma}\ \isactrlbold {\isasymnot}{\isacharparenleft}Atom\ k{\isacharparenright}{\isacharbraceright}{\isachardot}\ {\isasymA}\ {\isasymTurnstile}\ F{\isachardoublequoteclose}\isanewline
\ \ \ \ \isacommand{using}\isamarkupfalse%
\ Sat\ \isacommand{by}\isamarkupfalse%
\ {\isacharparenleft}rule\ exE{\isacharparenright}\isanewline
\ \ \isacommand{have}\isamarkupfalse%
\ {\isachardoublequoteopen}Atom\ k\ {\isasymin}\ {\isacharbraceleft}Atom\ k{\isacharcomma}\ \isactrlbold {\isasymnot}{\isacharparenleft}Atom\ k{\isacharparenright}{\isacharbraceright}{\isachardoublequoteclose}\isanewline
\ \ \ \ \isacommand{by}\isamarkupfalse%
\ simp\isanewline
\ \ \isacommand{have}\isamarkupfalse%
\ {\isachardoublequoteopen}{\isasymA}\ {\isasymTurnstile}\ Atom\ k{\isachardoublequoteclose}\isanewline
\ \ \ \ \isacommand{using}\isamarkupfalse%
\ H\ {\isacartoucheopen}Atom\ k\ {\isasymin}\ {\isacharbraceleft}Atom\ k{\isacharcomma}\ \isactrlbold {\isasymnot}{\isacharparenleft}Atom\ k{\isacharparenright}{\isacharbraceright}{\isacartoucheclose}\ \isacommand{by}\isamarkupfalse%
\ {\isacharparenleft}rule\ bspec{\isacharparenright}\isanewline
\ \ \isacommand{have}\isamarkupfalse%
\ {\isachardoublequoteopen}\isactrlbold {\isasymnot}{\isacharparenleft}Atom\ k{\isacharparenright}\ {\isasymin}\ {\isacharbraceleft}Atom\ k{\isacharcomma}\ \isactrlbold {\isasymnot}{\isacharparenleft}Atom\ k{\isacharparenright}{\isacharbraceright}{\isachardoublequoteclose}\isanewline
\ \ \ \ \isacommand{by}\isamarkupfalse%
\ simp\isanewline
\ \ \isacommand{have}\isamarkupfalse%
\ {\isachardoublequoteopen}{\isasymA}\ {\isasymTurnstile}\ \isactrlbold {\isasymnot}{\isacharparenleft}Atom\ k{\isacharparenright}{\isachardoublequoteclose}\isanewline
\ \ \ \ \isacommand{using}\isamarkupfalse%
\ H\ {\isacartoucheopen}\isactrlbold {\isasymnot}{\isacharparenleft}Atom\ k{\isacharparenright}\ {\isasymin}\ {\isacharbraceleft}Atom\ k{\isacharcomma}\ \isactrlbold {\isasymnot}{\isacharparenleft}Atom\ k{\isacharparenright}{\isacharbraceright}{\isacartoucheclose}\ \isacommand{by}\isamarkupfalse%
\ {\isacharparenleft}rule\ bspec{\isacharparenright}\isanewline
\ \ \isacommand{then}\isamarkupfalse%
\ \isacommand{have}\isamarkupfalse%
\ {\isachardoublequoteopen}{\isasymnot}\ {\isasymA}\ {\isasymTurnstile}\ Atom\ k{\isachardoublequoteclose}\ \isanewline
\ \ \ \ \isacommand{by}\isamarkupfalse%
\ {\isacharparenleft}simp\ only{\isacharcolon}\ simp{\isacharunderscore}thms{\isacharparenleft}{\isadigit{8}}{\isacharparenright}\ formula{\isacharunderscore}semantics{\isachardot}simps{\isacharparenleft}{\isadigit{3}}{\isacharparenright}{\isacharparenright}\isanewline
\ \ \isacommand{thus}\isamarkupfalse%
\ {\isachardoublequoteopen}False{\isachardoublequoteclose}\isanewline
\ \ \ \ \isacommand{using}\isamarkupfalse%
\ {\isacartoucheopen}{\isasymA}\ {\isasymTurnstile}\ Atom\ k{\isacartoucheclose}\ \isacommand{by}\isamarkupfalse%
\ {\isacharparenleft}rule\ notE{\isacharparenright}\isanewline
\isacommand{qed}\isamarkupfalse%
%
\endisatagproof
{\isafoldproof}%
%
\isadelimproof
%
\endisadelimproof
%
\begin{isamarkuptext}%
De este modo, podemos demostrar detalladamente la condición: dados \isa{W\ {\isasymin}\ C} y \isa{p} una fórmula
  atómica cualquiera, no se tiene simultáneamente que \isa{p\ {\isasymin}\ W} y \isa{{\isasymnot}\ p\ {\isasymin}\ W}.%
\end{isamarkuptext}\isamarkuptrue%
\isacommand{lemma}\isamarkupfalse%
\ pcp{\isacharunderscore}colecComp{\isacharunderscore}atoms{\isacharcolon}\isanewline
\ \ \isakeyword{assumes}\ {\isachardoublequoteopen}W\ {\isasymin}\ colecComp{\isachardoublequoteclose}\isanewline
\ \ \isakeyword{shows}\ {\isachardoublequoteopen}{\isasymforall}k{\isachardot}\ Atom\ k\ {\isasymin}\ W\ {\isasymlongrightarrow}\ \isactrlbold {\isasymnot}\ {\isacharparenleft}Atom\ k{\isacharparenright}\ {\isasymin}\ W\ {\isasymlongrightarrow}\ False{\isachardoublequoteclose}\isanewline
%
\isadelimproof
%
\endisadelimproof
%
\isatagproof
\isacommand{proof}\isamarkupfalse%
\ {\isacharparenleft}rule\ allI{\isacharparenright}\isanewline
\ \ \isacommand{fix}\isamarkupfalse%
\ k\isanewline
\ \ \isacommand{show}\isamarkupfalse%
\ {\isachardoublequoteopen}Atom\ k\ {\isasymin}\ W\ {\isasymlongrightarrow}\ \isactrlbold {\isasymnot}\ {\isacharparenleft}Atom\ k{\isacharparenright}\ {\isasymin}\ W\ {\isasymlongrightarrow}\ False{\isachardoublequoteclose}\isanewline
\ \ \isacommand{proof}\isamarkupfalse%
\ {\isacharparenleft}rule\ impI{\isacharparenright}{\isacharplus}\isanewline
\ \ \ \ \isacommand{assume}\isamarkupfalse%
\ {\isadigit{1}}{\isacharcolon}{\isachardoublequoteopen}Atom\ k\ {\isasymin}\ W{\isachardoublequoteclose}\isanewline
\ \ \ \ \isacommand{assume}\isamarkupfalse%
\ {\isadigit{2}}{\isacharcolon}{\isachardoublequoteopen}\isactrlbold {\isasymnot}\ {\isacharparenleft}Atom\ k{\isacharparenright}\ {\isasymin}\ W{\isachardoublequoteclose}\isanewline
\ \ \ \ \isacommand{have}\isamarkupfalse%
\ {\isachardoublequoteopen}{\isacharbraceleft}{\isacharbraceright}\ {\isasymsubseteq}\ W{\isachardoublequoteclose}\isanewline
\ \ \ \ \ \ \isacommand{by}\isamarkupfalse%
\ {\isacharparenleft}simp\ only{\isacharcolon}\ empty{\isacharunderscore}subsetI{\isacharparenright}\ \isanewline
\ \ \ \ \isacommand{have}\isamarkupfalse%
\ {\isachardoublequoteopen}Atom\ k\ {\isasymin}\ W\ {\isasymand}\ {\isacharbraceleft}{\isacharbraceright}\ {\isasymsubseteq}\ W{\isachardoublequoteclose}\isanewline
\ \ \ \ \ \ \isacommand{using}\isamarkupfalse%
\ {\isadigit{1}}\ {\isacartoucheopen}{\isacharbraceleft}{\isacharbraceright}\ {\isasymsubseteq}\ W{\isacartoucheclose}\ \isacommand{by}\isamarkupfalse%
\ {\isacharparenleft}rule\ conjI{\isacharparenright}\isanewline
\ \ \ \ \isacommand{then}\isamarkupfalse%
\ \isacommand{have}\isamarkupfalse%
\ {\isachardoublequoteopen}{\isacharbraceleft}Atom\ k{\isacharbraceright}\ {\isasymsubseteq}\ W{\isachardoublequoteclose}\isanewline
\ \ \ \ \ \ \isacommand{by}\isamarkupfalse%
\ {\isacharparenleft}simp\ only{\isacharcolon}\ insert{\isacharunderscore}subset{\isacharparenright}\isanewline
\ \ \ \ \isacommand{have}\isamarkupfalse%
\ {\isachardoublequoteopen}\isactrlbold {\isasymnot}\ {\isacharparenleft}Atom\ k{\isacharparenright}\ {\isasymin}\ W\ {\isasymand}\ {\isacharbraceleft}{\isacharbraceright}\ {\isasymsubseteq}\ W{\isachardoublequoteclose}\isanewline
\ \ \ \ \ \ \isacommand{using}\isamarkupfalse%
\ {\isadigit{2}}\ {\isacartoucheopen}{\isacharbraceleft}{\isacharbraceright}\ {\isasymsubseteq}\ W{\isacartoucheclose}\ \isacommand{by}\isamarkupfalse%
\ {\isacharparenleft}rule\ conjI{\isacharparenright}\isanewline
\ \ \ \ \isacommand{then}\isamarkupfalse%
\ \isacommand{have}\isamarkupfalse%
\ {\isachardoublequoteopen}{\isacharbraceleft}\isactrlbold {\isasymnot}{\isacharparenleft}Atom\ k{\isacharparenright}{\isacharbraceright}\ {\isasymsubseteq}\ W{\isachardoublequoteclose}\isanewline
\ \ \ \ \ \ \isacommand{by}\isamarkupfalse%
\ {\isacharparenleft}simp\ only{\isacharcolon}\ insert{\isacharunderscore}subset{\isacharparenright}\isanewline
\ \ \ \ \isacommand{have}\isamarkupfalse%
\ {\isachardoublequoteopen}{\isacharbraceleft}Atom\ k{\isacharbraceright}\ {\isasymunion}\ {\isacharbraceleft}\isactrlbold {\isasymnot}{\isacharparenleft}Atom\ k{\isacharparenright}{\isacharbraceright}\ {\isasymsubseteq}\ W{\isachardoublequoteclose}\isanewline
\ \ \ \ \ \ \isacommand{using}\isamarkupfalse%
\ {\isacartoucheopen}{\isacharbraceleft}Atom\ k{\isacharbraceright}\ {\isasymsubseteq}\ W{\isacartoucheclose}\ {\isacartoucheopen}{\isacharbraceleft}\isactrlbold {\isasymnot}{\isacharparenleft}Atom\ k{\isacharparenright}{\isacharbraceright}\ {\isasymsubseteq}\ W{\isacartoucheclose}\ \isacommand{by}\isamarkupfalse%
\ {\isacharparenleft}simp\ only{\isacharcolon}\ Un{\isacharunderscore}least{\isacharparenright}\isanewline
\ \ \ \ \isacommand{then}\isamarkupfalse%
\ \isacommand{have}\isamarkupfalse%
\ {\isachardoublequoteopen}{\isacharbraceleft}Atom\ k{\isacharcomma}\ \isactrlbold {\isasymnot}{\isacharparenleft}Atom\ k{\isacharparenright}{\isacharbraceright}\ {\isasymsubseteq}\ W{\isachardoublequoteclose}\isanewline
\ \ \ \ \ \ \isacommand{by}\isamarkupfalse%
\ simp\ \isanewline
\ \ \ \ \isacommand{have}\isamarkupfalse%
\ {\isachardoublequoteopen}finite\ {\isacharbraceleft}Atom\ k{\isacharcomma}\ \isactrlbold {\isasymnot}{\isacharparenleft}Atom\ k{\isacharparenright}{\isacharbraceright}{\isachardoublequoteclose}\isanewline
\ \ \ \ \ \ \isacommand{by}\isamarkupfalse%
\ blast\isanewline
\ \ \ \ \isacommand{have}\isamarkupfalse%
\ {\isachardoublequoteopen}sat\ {\isacharparenleft}{\isacharbraceleft}Atom\ k{\isacharcomma}\ \isactrlbold {\isasymnot}{\isacharparenleft}Atom\ k{\isacharparenright}{\isacharbraceright}{\isacharparenright}{\isachardoublequoteclose}\isanewline
\ \ \ \ \ \ \isacommand{using}\isamarkupfalse%
\ assms\ {\isacartoucheopen}{\isacharbraceleft}Atom\ k{\isacharcomma}\ \isactrlbold {\isasymnot}{\isacharparenleft}Atom\ k{\isacharparenright}{\isacharbraceright}\ {\isasymsubseteq}\ W{\isacartoucheclose}\ {\isacartoucheopen}finite\ {\isacharbraceleft}Atom\ k{\isacharcomma}\ \isactrlbold {\isasymnot}{\isacharparenleft}Atom\ k{\isacharparenright}{\isacharbraceright}{\isacartoucheclose}\ \isacommand{by}\isamarkupfalse%
\ {\isacharparenleft}rule\ colecComp{\isacharunderscore}subset{\isacharunderscore}finite{\isacharparenright}\isanewline
\ \ \ \ \isacommand{have}\isamarkupfalse%
\ {\isachardoublequoteopen}{\isasymnot}\ sat\ {\isacharparenleft}{\isacharbraceleft}Atom\ k{\isacharcomma}\ \isactrlbold {\isasymnot}{\isacharparenleft}Atom\ k{\isacharparenright}{\isacharbraceright}{\isacharparenright}{\isachardoublequoteclose}\isanewline
\ \ \ \ \ \ \isacommand{by}\isamarkupfalse%
\ {\isacharparenleft}rule\ not{\isacharunderscore}sat{\isacharunderscore}atoms{\isacharparenright}\isanewline
\ \ \ \ \isacommand{thus}\isamarkupfalse%
\ {\isachardoublequoteopen}False{\isachardoublequoteclose}\isanewline
\ \ \ \ \ \ \isacommand{using}\isamarkupfalse%
\ {\isacartoucheopen}sat\ {\isacharparenleft}{\isacharbraceleft}Atom\ k{\isacharcomma}\ \isactrlbold {\isasymnot}{\isacharparenleft}Atom\ k{\isacharparenright}{\isacharbraceright}{\isacharparenright}{\isacartoucheclose}\ \isacommand{by}\isamarkupfalse%
\ {\isacharparenleft}rule\ notE{\isacharparenright}\isanewline
\ \ \isacommand{qed}\isamarkupfalse%
\isanewline
\isacommand{qed}\isamarkupfalse%
%
\endisatagproof
{\isafoldproof}%
%
\isadelimproof
%
\endisadelimproof
%
\begin{isamarkuptext}%
Demostremos la tercera condición del lema \isa{{\isadigit{2}}{\isachardot}{\isadigit{0}}{\isachardot}{\isadigit{2}}}: dados \isa{W\ {\isasymin}\ C} y \isa{F} una fórmula de 
  tipo \isa{{\isasymalpha}} con componentes \isa{{\isasymalpha}\isactrlsub {\isadigit{1}}} y \isa{{\isasymalpha}\isactrlsub {\isadigit{2}}} tal que \isa{F\ {\isasymin}\ W}, se tiene que \isa{{\isacharbraceleft}{\isasymalpha}\isactrlsub {\isadigit{1}}{\isacharcomma}{\isasymalpha}\isactrlsub {\isadigit{2}}{\isacharbraceright}\ {\isasymunion}\ W\ {\isasymin}\ C}. Para probar 
  dicho resultado, emplearemos un lema auxiliar que demuestra que dado un subconjunto finito \isa{W\isactrlsub {\isadigit{0}}} de 
  \isa{W} se tiene que \isa{{\isacharbraceleft}{\isasymalpha}\isactrlsub {\isadigit{1}}{\isacharcomma}{\isasymalpha}\isactrlsub {\isadigit{2}}{\isacharcomma}F{\isacharbraceright}\ {\isasymunion}\ W\isactrlsub {\isadigit{0}}} es un conjunto satisfacible. Mostraremos la prueba para cada
  caso de fórmula de tipo \isa{{\isasymalpha}}. Para ello, precisaremos del siguiente lema auxiliar que demuestra que 
  dado un conjunto \isa{W\ {\isasymin}\ C}, \isa{F} una fórmula perteneciente a \isa{W} y \isa{W\isactrlsub {\isadigit{0}}} un subconjunto finito de \isa{W}, 
  entonces \isa{{\isacharbraceleft}F{\isacharbraceright}\ {\isasymunion}\ W\isactrlsub {\isadigit{0}}} es satisfacible.%
\end{isamarkuptext}\isamarkuptrue%
\isacommand{lemma}\isamarkupfalse%
\ pcp{\isacharunderscore}colecComp{\isacharunderscore}elem{\isacharunderscore}sat{\isacharcolon}\isanewline
\ \ \isakeyword{assumes}\ {\isachardoublequoteopen}W\ {\isasymin}\ colecComp{\isachardoublequoteclose}\isanewline
\ \ \ \ \ \ \ \ \ \ {\isachardoublequoteopen}F\ {\isasymin}\ W{\isachardoublequoteclose}\isanewline
\ \ \ \ \ \ \ \ \ \ {\isachardoublequoteopen}finite\ Wo{\isachardoublequoteclose}\isanewline
\ \ \ \ \ \ \ \ \ \ {\isachardoublequoteopen}Wo\ {\isasymsubseteq}\ W{\isachardoublequoteclose}\isanewline
\ \ \ \ \ \ \ \ \isakeyword{shows}\ {\isachardoublequoteopen}sat\ {\isacharparenleft}{\isacharbraceleft}F{\isacharbraceright}\ {\isasymunion}\ Wo{\isacharparenright}{\isachardoublequoteclose}\isanewline
%
\isadelimproof
%
\endisadelimproof
%
\isatagproof
\isacommand{proof}\isamarkupfalse%
\ {\isacharminus}\isanewline
\ \ \isacommand{have}\isamarkupfalse%
\ {\isadigit{1}}{\isacharcolon}{\isachardoublequoteopen}insert\ F\ Wo\ {\isacharequal}\ {\isacharbraceleft}F{\isacharbraceright}\ {\isasymunion}\ Wo{\isachardoublequoteclose}\isanewline
\ \ \ \ \isacommand{by}\isamarkupfalse%
\ {\isacharparenleft}rule\ insert{\isacharunderscore}is{\isacharunderscore}Un{\isacharparenright}\isanewline
\ \ \isacommand{have}\isamarkupfalse%
\ {\isachardoublequoteopen}finite\ {\isacharparenleft}insert\ F\ Wo{\isacharparenright}{\isachardoublequoteclose}\isanewline
\ \ \ \ \isacommand{using}\isamarkupfalse%
\ assms{\isacharparenleft}{\isadigit{3}}{\isacharparenright}\ \isacommand{by}\isamarkupfalse%
\ {\isacharparenleft}simp\ only{\isacharcolon}\ finite{\isacharunderscore}insert{\isacharparenright}\isanewline
\ \ \isacommand{then}\isamarkupfalse%
\ \isacommand{have}\isamarkupfalse%
\ {\isachardoublequoteopen}finite\ {\isacharparenleft}{\isacharbraceleft}F{\isacharbraceright}\ {\isasymunion}\ Wo{\isacharparenright}{\isachardoublequoteclose}\isanewline
\ \ \ \ \isacommand{by}\isamarkupfalse%
\ {\isacharparenleft}simp\ only{\isacharcolon}\ {\isadigit{1}}{\isacharparenright}\ \isanewline
\ \ \isacommand{have}\isamarkupfalse%
\ {\isachardoublequoteopen}F\ {\isasymin}\ W\ {\isasymand}\ Wo\ {\isasymsubseteq}\ W{\isachardoublequoteclose}\isanewline
\ \ \ \ \isacommand{using}\isamarkupfalse%
\ assms{\isacharparenleft}{\isadigit{2}}{\isacharparenright}\ assms{\isacharparenleft}{\isadigit{4}}{\isacharparenright}\ \isacommand{by}\isamarkupfalse%
\ {\isacharparenleft}rule\ conjI{\isacharparenright}\isanewline
\ \ \isacommand{then}\isamarkupfalse%
\ \isacommand{have}\isamarkupfalse%
\ {\isachardoublequoteopen}insert\ F\ Wo\ {\isasymsubseteq}\ W{\isachardoublequoteclose}\isanewline
\ \ \ \ \isacommand{by}\isamarkupfalse%
\ {\isacharparenleft}simp\ only{\isacharcolon}\ insert{\isacharunderscore}subset{\isacharparenright}\isanewline
\ \ \isacommand{then}\isamarkupfalse%
\ \isacommand{have}\isamarkupfalse%
\ {\isachardoublequoteopen}{\isacharbraceleft}F{\isacharbraceright}\ {\isasymunion}\ Wo\ {\isasymsubseteq}\ W{\isachardoublequoteclose}\isanewline
\ \ \ \ \isacommand{by}\isamarkupfalse%
\ {\isacharparenleft}simp\ only{\isacharcolon}\ {\isadigit{1}}{\isacharparenright}\isanewline
\ \ \isacommand{show}\isamarkupfalse%
\ {\isachardoublequoteopen}sat\ {\isacharparenleft}{\isacharbraceleft}F{\isacharbraceright}\ {\isasymunion}\ Wo{\isacharparenright}{\isachardoublequoteclose}\isanewline
\ \ \ \ \isacommand{using}\isamarkupfalse%
\ assms{\isacharparenleft}{\isadigit{1}}{\isacharparenright}\ {\isacartoucheopen}{\isacharbraceleft}F{\isacharbraceright}\ {\isasymunion}\ Wo\ {\isasymsubseteq}\ W{\isacartoucheclose}\ {\isacartoucheopen}finite\ {\isacharparenleft}{\isacharbraceleft}F{\isacharbraceright}\ {\isasymunion}\ Wo{\isacharparenright}{\isacartoucheclose}\ \isacommand{by}\isamarkupfalse%
\ {\isacharparenleft}rule\ colecComp{\isacharunderscore}subset{\isacharunderscore}finite{\isacharparenright}\isanewline
\isacommand{qed}\isamarkupfalse%
%
\endisatagproof
{\isafoldproof}%
%
\isadelimproof
%
\endisadelimproof
%
\begin{isamarkuptext}%
De este modo, vamos a probar para cada caso de fórmula \isa{{\isasymalpha}} que dados \isa{W\ {\isasymin}\ C}, \isa{F} una fórmula 
  de tipo \isa{{\isasymalpha}} con componentes \isa{{\isasymalpha}\isactrlsub {\isadigit{1}}} y \isa{{\isasymalpha}\isactrlsub {\isadigit{2}}} tal que \isa{F\ {\isasymin}\ W} y \isa{W\isactrlsub {\isadigit{0}}} un subconjunto finito de \isa{W}, se 
  verifica que \isa{{\isacharbraceleft}{\isasymalpha}\isactrlsub {\isadigit{1}}{\isacharcomma}{\isasymalpha}\isactrlsub {\isadigit{2}}{\isacharcomma}F{\isacharbraceright}\ {\isasymunion}\ W\isactrlsub {\isadigit{0}}} es satisfacible. Para ello, emplearemos el siguiente lema auxiliar
  en Isabelle.%
\end{isamarkuptext}\isamarkuptrue%
\isacommand{lemma}\isamarkupfalse%
\ ball{\isacharunderscore}Un{\isacharcolon}\ \isanewline
\ \ \isakeyword{assumes}\ {\isachardoublequoteopen}{\isasymforall}x\ {\isasymin}\ A{\isachardot}\ P\ x{\isachardoublequoteclose}\isanewline
\ \ \ \ \ \ \ \ \ \ {\isachardoublequoteopen}{\isasymforall}x\ {\isasymin}\ B{\isachardot}\ P\ x{\isachardoublequoteclose}\isanewline
\ \ \ \ \ \ \ \ \isakeyword{shows}\ {\isachardoublequoteopen}{\isasymforall}x\ {\isasymin}\ {\isacharparenleft}A\ {\isasymunion}\ B{\isacharparenright}{\isachardot}\ P\ x{\isachardoublequoteclose}\ \isanewline
%
\isadelimproof
\ \ %
\endisadelimproof
%
\isatagproof
\isacommand{using}\isamarkupfalse%
\ assms\ \isacommand{by}\isamarkupfalse%
\ blast%
\endisatagproof
{\isafoldproof}%
%
\isadelimproof
%
\endisadelimproof
%
\begin{isamarkuptext}%
En primer lugar, probemos que dados \isa{W\ {\isasymin}\ C}, una fórmula \isa{F\ {\isacharequal}\ G\ {\isasymand}\ H} para ciertas fórmulas \isa{G} 
  y \isa{H} tal que \isa{F\ {\isasymin}\ W} y \isa{W\isactrlsub {\isadigit{0}}} un subconjunto finito de \isa{W}, se verifica que\\ \isa{{\isacharbraceleft}G{\isacharcomma}H{\isacharcomma}F{\isacharbraceright}\ {\isasymunion}\ W\isactrlsub {\isadigit{0}}} es 
  satisfacible.%
\end{isamarkuptext}\isamarkuptrue%
\isacommand{lemma}\isamarkupfalse%
\ pcp{\isacharunderscore}colecComp{\isacharunderscore}CON{\isacharunderscore}sat{\isadigit{1}}{\isacharcolon}\isanewline
\ \ \isakeyword{assumes}\ {\isachardoublequoteopen}W\ {\isasymin}\ colecComp{\isachardoublequoteclose}\isanewline
\ \ \ \ \ \ \ \ \ \ {\isachardoublequoteopen}F\ {\isacharequal}\ G\ \isactrlbold {\isasymand}\ H{\isachardoublequoteclose}\isanewline
\ \ \ \ \ \ \ \ \ \ {\isachardoublequoteopen}F\ {\isasymin}\ W{\isachardoublequoteclose}\isanewline
\ \ \ \ \ \ \ \ \ \ {\isachardoublequoteopen}finite\ Wo{\isachardoublequoteclose}\isanewline
\ \ \ \ \ \ \ \ \ \ {\isachardoublequoteopen}Wo\ {\isasymsubseteq}\ W{\isachardoublequoteclose}\isanewline
\ \ \ \ \ \ \ \ \isakeyword{shows}\ {\isachardoublequoteopen}sat\ {\isacharparenleft}{\isacharbraceleft}G{\isacharcomma}H{\isacharcomma}F{\isacharbraceright}\ {\isasymunion}\ Wo{\isacharparenright}{\isachardoublequoteclose}\isanewline
%
\isadelimproof
%
\endisadelimproof
%
\isatagproof
\isacommand{proof}\isamarkupfalse%
\ {\isacharminus}\isanewline
\ \ \isacommand{have}\isamarkupfalse%
\ {\isachardoublequoteopen}sat\ {\isacharparenleft}{\isacharbraceleft}F{\isacharbraceright}\ {\isasymunion}\ Wo{\isacharparenright}{\isachardoublequoteclose}\isanewline
\ \ \ \ \isacommand{using}\isamarkupfalse%
\ assms{\isacharparenleft}{\isadigit{1}}{\isacharcomma}{\isadigit{3}}{\isacharcomma}{\isadigit{4}}{\isacharcomma}{\isadigit{5}}{\isacharparenright}\ \isacommand{by}\isamarkupfalse%
\ {\isacharparenleft}rule\ pcp{\isacharunderscore}colecComp{\isacharunderscore}elem{\isacharunderscore}sat{\isacharparenright}\isanewline
\ \ \isacommand{have}\isamarkupfalse%
\ {\isachardoublequoteopen}F\ {\isasymin}\ {\isacharbraceleft}F{\isacharbraceright}\ {\isasymunion}\ Wo{\isachardoublequoteclose}\isanewline
\ \ \ \ \isacommand{by}\isamarkupfalse%
\ {\isacharparenleft}simp\ add{\isacharcolon}\ insertI{\isadigit{1}}{\isacharparenright}\isanewline
\ \ \isacommand{have}\isamarkupfalse%
\ Ex{\isadigit{1}}{\isacharcolon}{\isachardoublequoteopen}{\isasymexists}{\isasymA}{\isachardot}\ {\isasymforall}F\ {\isasymin}\ {\isacharparenleft}{\isacharbraceleft}F{\isacharbraceright}\ {\isasymunion}\ Wo{\isacharparenright}{\isachardot}\ {\isasymA}\ {\isasymTurnstile}\ F{\isachardoublequoteclose}\isanewline
\ \ \ \ \isacommand{using}\isamarkupfalse%
\ {\isacartoucheopen}sat\ {\isacharparenleft}{\isacharbraceleft}F{\isacharbraceright}\ {\isasymunion}\ Wo{\isacharparenright}{\isacartoucheclose}\ \isacommand{by}\isamarkupfalse%
\ {\isacharparenleft}simp\ only{\isacharcolon}\ sat{\isacharunderscore}def{\isacharparenright}\isanewline
\ \ \isacommand{obtain}\isamarkupfalse%
\ {\isasymA}\ \isakeyword{where}\ Forall{\isadigit{1}}{\isacharcolon}{\isachardoublequoteopen}{\isasymforall}F\ {\isasymin}\ {\isacharparenleft}{\isacharbraceleft}F{\isacharbraceright}\ {\isasymunion}\ Wo{\isacharparenright}{\isachardot}\ {\isasymA}\ {\isasymTurnstile}\ F{\isachardoublequoteclose}\isanewline
\ \ \ \ \isacommand{using}\isamarkupfalse%
\ Ex{\isadigit{1}}\ \isacommand{by}\isamarkupfalse%
\ {\isacharparenleft}rule\ exE{\isacharparenright}\isanewline
\ \ \isacommand{have}\isamarkupfalse%
\ {\isachardoublequoteopen}{\isasymA}\ {\isasymTurnstile}\ F{\isachardoublequoteclose}\isanewline
\ \ \ \ \isacommand{using}\isamarkupfalse%
\ Forall{\isadigit{1}}\ {\isacartoucheopen}F\ {\isasymin}\ {\isacharbraceleft}F{\isacharbraceright}\ {\isasymunion}\ Wo{\isacartoucheclose}\ \isacommand{by}\isamarkupfalse%
\ {\isacharparenleft}rule\ bspec{\isacharparenright}\isanewline
\ \ \isacommand{then}\isamarkupfalse%
\ \isacommand{have}\isamarkupfalse%
\ {\isachardoublequoteopen}{\isasymA}\ {\isasymTurnstile}\ {\isacharparenleft}G\ \isactrlbold {\isasymand}\ H{\isacharparenright}{\isachardoublequoteclose}\isanewline
\ \ \ \ \isacommand{using}\isamarkupfalse%
\ assms{\isacharparenleft}{\isadigit{2}}{\isacharparenright}\ \isacommand{by}\isamarkupfalse%
\ {\isacharparenleft}simp\ only{\isacharcolon}\ {\isacartoucheopen}{\isasymA}\ {\isasymTurnstile}\ F{\isacartoucheclose}{\isacharparenright}\isanewline
\ \ \isacommand{then}\isamarkupfalse%
\ \isacommand{have}\isamarkupfalse%
\ {\isachardoublequoteopen}{\isasymA}\ {\isasymTurnstile}\ G\ {\isasymand}\ {\isasymA}\ {\isasymTurnstile}\ H{\isachardoublequoteclose}\isanewline
\ \ \ \ \isacommand{by}\isamarkupfalse%
\ {\isacharparenleft}simp\ only{\isacharcolon}\ formula{\isacharunderscore}semantics{\isachardot}simps{\isacharparenleft}{\isadigit{4}}{\isacharparenright}{\isacharparenright}\isanewline
\ \ \isacommand{then}\isamarkupfalse%
\ \isacommand{have}\isamarkupfalse%
\ {\isachardoublequoteopen}{\isasymA}\ {\isasymTurnstile}\ G{\isachardoublequoteclose}\isanewline
\ \ \ \ \isacommand{by}\isamarkupfalse%
\ {\isacharparenleft}rule\ conjunct{\isadigit{1}}{\isacharparenright}\isanewline
\ \ \isacommand{then}\isamarkupfalse%
\ \isacommand{have}\isamarkupfalse%
\ {\isadigit{1}}{\isacharcolon}{\isachardoublequoteopen}{\isasymforall}F\ {\isasymin}\ {\isacharbraceleft}G{\isacharbraceright}{\isachardot}\ {\isasymA}\ {\isasymTurnstile}\ F{\isachardoublequoteclose}\isanewline
\ \ \ \ \isacommand{by}\isamarkupfalse%
\ simp\isanewline
\ \ \isacommand{have}\isamarkupfalse%
\ {\isachardoublequoteopen}{\isasymA}\ {\isasymTurnstile}\ H{\isachardoublequoteclose}\isanewline
\ \ \ \ \isacommand{using}\isamarkupfalse%
\ {\isacartoucheopen}{\isasymA}\ {\isasymTurnstile}\ G\ {\isasymand}\ {\isasymA}\ {\isasymTurnstile}\ H{\isacartoucheclose}\ \isacommand{by}\isamarkupfalse%
\ {\isacharparenleft}rule\ conjunct{\isadigit{2}}{\isacharparenright}\isanewline
\ \ \isacommand{then}\isamarkupfalse%
\ \isacommand{have}\isamarkupfalse%
\ {\isadigit{2}}{\isacharcolon}{\isachardoublequoteopen}{\isasymforall}F\ {\isasymin}\ {\isacharbraceleft}H{\isacharbraceright}{\isachardot}\ {\isasymA}\ {\isasymTurnstile}\ F{\isachardoublequoteclose}\isanewline
\ \ \ \ \isacommand{by}\isamarkupfalse%
\ simp\isanewline
\ \ \isacommand{have}\isamarkupfalse%
\ {\isachardoublequoteopen}{\isasymforall}F\ {\isasymin}\ {\isacharparenleft}{\isacharbraceleft}G{\isacharbraceright}\ {\isasymunion}\ {\isacharbraceleft}H{\isacharbraceright}{\isacharparenright}\ {\isasymunion}\ {\isacharparenleft}{\isacharbraceleft}F{\isacharbraceright}\ {\isasymunion}\ Wo{\isacharparenright}{\isachardot}\ {\isasymA}\ {\isasymTurnstile}\ F{\isachardoublequoteclose}\isanewline
\ \ \ \ \isacommand{using}\isamarkupfalse%
\ Forall{\isadigit{1}}\ {\isadigit{1}}\ {\isadigit{2}}\ \isacommand{by}\isamarkupfalse%
\ {\isacharparenleft}iprover\ intro{\isacharcolon}\ ball{\isacharunderscore}Un{\isacharparenright}\isanewline
\ \ \isacommand{then}\isamarkupfalse%
\ \isacommand{have}\isamarkupfalse%
\ {\isachardoublequoteopen}{\isasymforall}F\ {\isasymin}\ {\isacharparenleft}{\isacharbraceleft}G{\isacharcomma}H{\isacharcomma}F{\isacharbraceright}\ {\isasymunion}\ Wo{\isacharparenright}{\isachardot}\ {\isasymA}\ {\isasymTurnstile}\ F{\isachardoublequoteclose}\isanewline
\ \ \ \ \isacommand{by}\isamarkupfalse%
\ simp\isanewline
\ \ \isacommand{then}\isamarkupfalse%
\ \isacommand{have}\isamarkupfalse%
\ {\isachardoublequoteopen}{\isasymexists}{\isasymA}{\isachardot}\ {\isasymforall}F\ {\isasymin}\ {\isacharparenleft}{\isacharbraceleft}G{\isacharcomma}H{\isacharcomma}F{\isacharbraceright}\ {\isasymunion}\ Wo{\isacharparenright}{\isachardot}\ {\isasymA}\ {\isasymTurnstile}\ F{\isachardoublequoteclose}\isanewline
\ \ \ \ \isacommand{by}\isamarkupfalse%
\ {\isacharparenleft}iprover\ intro{\isacharcolon}\ exI{\isacharparenright}\isanewline
\ \ \isacommand{thus}\isamarkupfalse%
\ {\isachardoublequoteopen}sat\ {\isacharparenleft}{\isacharbraceleft}G{\isacharcomma}H{\isacharcomma}F{\isacharbraceright}\ {\isasymunion}\ Wo{\isacharparenright}{\isachardoublequoteclose}\isanewline
\ \ \ \ \isacommand{by}\isamarkupfalse%
\ {\isacharparenleft}simp\ only{\isacharcolon}\ sat{\isacharunderscore}def{\isacharparenright}\isanewline
\isacommand{qed}\isamarkupfalse%
%
\endisatagproof
{\isafoldproof}%
%
\isadelimproof
%
\endisadelimproof
%
\begin{isamarkuptext}%
A continuación veamos la prueba detallada del resultado para el segundo caso de fórmula de 
  tipo \isa{{\isasymalpha}}: dados \isa{W\ {\isasymin}\ C}, una fórmula \isa{F\ {\isacharequal}\ {\isasymnot}{\isacharparenleft}G\ {\isasymor}\ H{\isacharparenright}} para ciertas fórmulas \isa{G} y \isa{H} tal que 
  \isa{F\ {\isasymin}\ W} y \isa{W\isactrlsub {\isadigit{0}}} un subconjunto finito de \isa{W}, se verifica que \isa{{\isacharbraceleft}{\isasymnot}\ G{\isacharcomma}{\isasymnot}\ H{\isacharcomma}F{\isacharbraceright}\ {\isasymunion}\ W\isactrlsub {\isadigit{0}}} es satisfacible.%
\end{isamarkuptext}\isamarkuptrue%
\isacommand{lemma}\isamarkupfalse%
\ pcp{\isacharunderscore}colecComp{\isacharunderscore}CON{\isacharunderscore}sat{\isadigit{2}}{\isacharcolon}\isanewline
\ \ \isakeyword{assumes}\ {\isachardoublequoteopen}W\ {\isasymin}\ colecComp{\isachardoublequoteclose}\isanewline
\ \ \ \ \ \ \ \ \ \ {\isachardoublequoteopen}F\ {\isacharequal}\ \isactrlbold {\isasymnot}{\isacharparenleft}G\ \isactrlbold {\isasymor}\ H{\isacharparenright}{\isachardoublequoteclose}\isanewline
\ \ \ \ \ \ \ \ \ \ {\isachardoublequoteopen}F\ {\isasymin}\ W{\isachardoublequoteclose}\isanewline
\ \ \ \ \ \ \ \ \ \ {\isachardoublequoteopen}finite\ Wo{\isachardoublequoteclose}\isanewline
\ \ \ \ \ \ \ \ \ \ {\isachardoublequoteopen}Wo\ {\isasymsubseteq}\ W{\isachardoublequoteclose}\isanewline
\ \ \ \ \ \ \ \ \isakeyword{shows}\ {\isachardoublequoteopen}sat\ {\isacharparenleft}{\isacharbraceleft}\isactrlbold {\isasymnot}\ G{\isacharcomma}\isactrlbold {\isasymnot}\ H{\isacharcomma}F{\isacharbraceright}\ {\isasymunion}\ Wo{\isacharparenright}{\isachardoublequoteclose}\isanewline
%
\isadelimproof
%
\endisadelimproof
%
\isatagproof
\isacommand{proof}\isamarkupfalse%
\ {\isacharminus}\isanewline
\ \ \isacommand{have}\isamarkupfalse%
\ {\isachardoublequoteopen}sat\ {\isacharparenleft}{\isacharbraceleft}F{\isacharbraceright}\ {\isasymunion}\ Wo{\isacharparenright}{\isachardoublequoteclose}\isanewline
\ \ \ \ \isacommand{using}\isamarkupfalse%
\ assms{\isacharparenleft}{\isadigit{1}}{\isacharcomma}{\isadigit{3}}{\isacharcomma}{\isadigit{4}}{\isacharcomma}{\isadigit{5}}{\isacharparenright}\ \isacommand{by}\isamarkupfalse%
\ {\isacharparenleft}rule\ pcp{\isacharunderscore}colecComp{\isacharunderscore}elem{\isacharunderscore}sat{\isacharparenright}\isanewline
\ \ \isacommand{have}\isamarkupfalse%
\ {\isachardoublequoteopen}F\ {\isasymin}\ {\isacharbraceleft}F{\isacharbraceright}\ {\isasymunion}\ Wo{\isachardoublequoteclose}\isanewline
\ \ \ \ \isacommand{by}\isamarkupfalse%
\ {\isacharparenleft}simp\ add{\isacharcolon}\ insertI{\isadigit{1}}{\isacharparenright}\isanewline
\ \ \isacommand{have}\isamarkupfalse%
\ Ex{\isadigit{1}}{\isacharcolon}{\isachardoublequoteopen}{\isasymexists}{\isasymA}{\isachardot}\ {\isasymforall}F\ {\isasymin}\ {\isacharparenleft}{\isacharbraceleft}F{\isacharbraceright}\ {\isasymunion}\ Wo{\isacharparenright}{\isachardot}\ {\isasymA}\ {\isasymTurnstile}\ F{\isachardoublequoteclose}\isanewline
\ \ \ \ \isacommand{using}\isamarkupfalse%
\ {\isacartoucheopen}sat\ {\isacharparenleft}{\isacharbraceleft}F{\isacharbraceright}\ {\isasymunion}\ Wo{\isacharparenright}{\isacartoucheclose}\ \isacommand{by}\isamarkupfalse%
\ {\isacharparenleft}simp\ only{\isacharcolon}\ sat{\isacharunderscore}def{\isacharparenright}\isanewline
\ \ \isacommand{obtain}\isamarkupfalse%
\ {\isasymA}\ \isakeyword{where}\ Forall{\isadigit{1}}{\isacharcolon}{\isachardoublequoteopen}{\isasymforall}F\ {\isasymin}\ {\isacharparenleft}{\isacharbraceleft}F{\isacharbraceright}\ {\isasymunion}\ Wo{\isacharparenright}{\isachardot}\ {\isasymA}\ {\isasymTurnstile}\ F{\isachardoublequoteclose}\isanewline
\ \ \ \ \isacommand{using}\isamarkupfalse%
\ Ex{\isadigit{1}}\ \isacommand{by}\isamarkupfalse%
\ {\isacharparenleft}rule\ exE{\isacharparenright}\isanewline
\ \ \isacommand{have}\isamarkupfalse%
\ {\isachardoublequoteopen}{\isasymA}\ {\isasymTurnstile}\ F{\isachardoublequoteclose}\isanewline
\ \ \ \ \isacommand{using}\isamarkupfalse%
\ Forall{\isadigit{1}}\ {\isacartoucheopen}F\ {\isasymin}\ {\isacharbraceleft}F{\isacharbraceright}\ {\isasymunion}\ Wo{\isacartoucheclose}\ \isacommand{by}\isamarkupfalse%
\ {\isacharparenleft}rule\ bspec{\isacharparenright}\isanewline
\ \ \isacommand{then}\isamarkupfalse%
\ \isacommand{have}\isamarkupfalse%
\ {\isachardoublequoteopen}{\isasymA}\ {\isasymTurnstile}\ \isactrlbold {\isasymnot}{\isacharparenleft}G\ \isactrlbold {\isasymor}\ H{\isacharparenright}{\isachardoublequoteclose}\isanewline
\ \ \ \ \isacommand{using}\isamarkupfalse%
\ assms{\isacharparenleft}{\isadigit{2}}{\isacharparenright}\ \isacommand{by}\isamarkupfalse%
\ {\isacharparenleft}simp\ only{\isacharcolon}\ {\isacartoucheopen}{\isasymA}\ {\isasymTurnstile}\ F{\isacartoucheclose}{\isacharparenright}\isanewline
\ \ \isacommand{then}\isamarkupfalse%
\ \isacommand{have}\isamarkupfalse%
\ {\isachardoublequoteopen}{\isasymnot}{\isacharparenleft}{\isasymA}\ {\isasymTurnstile}\ {\isacharparenleft}G\ \isactrlbold {\isasymor}\ H{\isacharparenright}{\isacharparenright}{\isachardoublequoteclose}\isanewline
\ \ \ \ \isacommand{by}\isamarkupfalse%
\ {\isacharparenleft}simp\ only{\isacharcolon}\ formula{\isacharunderscore}semantics{\isachardot}simps{\isacharparenleft}{\isadigit{3}}{\isacharparenright}\ simp{\isacharunderscore}thms{\isacharparenleft}{\isadigit{8}}{\isacharparenright}{\isacharparenright}\isanewline
\ \ \isacommand{then}\isamarkupfalse%
\ \isacommand{have}\isamarkupfalse%
\ {\isachardoublequoteopen}{\isasymnot}{\isacharparenleft}{\isasymA}\ {\isasymTurnstile}\ G\ {\isasymor}\ {\isasymA}\ {\isasymTurnstile}\ H{\isacharparenright}{\isachardoublequoteclose}\isanewline
\ \ \ \ \isacommand{by}\isamarkupfalse%
\ {\isacharparenleft}simp\ only{\isacharcolon}\ formula{\isacharunderscore}semantics{\isachardot}simps{\isacharparenleft}{\isadigit{5}}{\isacharparenright}\ simp{\isacharunderscore}thms{\isacharparenleft}{\isadigit{8}}{\isacharparenright}{\isacharparenright}\isanewline
\ \ \isacommand{then}\isamarkupfalse%
\ \isacommand{have}\isamarkupfalse%
\ {\isachardoublequoteopen}{\isasymnot}\ {\isasymA}\ {\isasymTurnstile}\ G\ {\isasymand}\ {\isasymnot}\ {\isasymA}\ {\isasymTurnstile}\ H{\isachardoublequoteclose}\ \isanewline
\ \ \ \ \isacommand{by}\isamarkupfalse%
\ {\isacharparenleft}simp\ only{\isacharcolon}\ de{\isacharunderscore}Morgan{\isacharunderscore}disj\ simp{\isacharunderscore}thms{\isacharparenleft}{\isadigit{8}}{\isacharparenright}{\isacharparenright}\isanewline
\ \ \isacommand{then}\isamarkupfalse%
\ \isacommand{have}\isamarkupfalse%
\ {\isachardoublequoteopen}{\isasymA}\ {\isasymTurnstile}\ \isactrlbold {\isasymnot}\ G\ {\isasymand}\ {\isasymA}\ {\isasymTurnstile}\ \isactrlbold {\isasymnot}\ H{\isachardoublequoteclose}\isanewline
\ \ \ \ \isacommand{by}\isamarkupfalse%
\ {\isacharparenleft}simp\ only{\isacharcolon}\ formula{\isacharunderscore}semantics{\isachardot}simps{\isacharparenleft}{\isadigit{3}}{\isacharparenright}\ simp{\isacharunderscore}thms{\isacharparenleft}{\isadigit{8}}{\isacharparenright}{\isacharparenright}\ \isanewline
\ \ \isacommand{then}\isamarkupfalse%
\ \isacommand{have}\isamarkupfalse%
\ {\isachardoublequoteopen}{\isasymA}\ {\isasymTurnstile}\ \isactrlbold {\isasymnot}\ G{\isachardoublequoteclose}\isanewline
\ \ \ \ \isacommand{by}\isamarkupfalse%
\ {\isacharparenleft}rule\ conjunct{\isadigit{1}}{\isacharparenright}\isanewline
\ \ \isacommand{then}\isamarkupfalse%
\ \isacommand{have}\isamarkupfalse%
\ {\isadigit{1}}{\isacharcolon}{\isachardoublequoteopen}{\isasymforall}F\ {\isasymin}\ {\isacharbraceleft}\isactrlbold {\isasymnot}\ G{\isacharbraceright}{\isachardot}\ {\isasymA}\ {\isasymTurnstile}\ F{\isachardoublequoteclose}\isanewline
\ \ \ \ \isacommand{by}\isamarkupfalse%
\ simp\isanewline
\ \ \isacommand{have}\isamarkupfalse%
\ {\isachardoublequoteopen}{\isasymA}\ {\isasymTurnstile}\ \isactrlbold {\isasymnot}\ H{\isachardoublequoteclose}\isanewline
\ \ \ \ \isacommand{using}\isamarkupfalse%
\ {\isacartoucheopen}{\isasymA}\ {\isasymTurnstile}\ \isactrlbold {\isasymnot}\ G\ {\isasymand}\ {\isasymA}\ {\isasymTurnstile}\ \isactrlbold {\isasymnot}\ H{\isacartoucheclose}\ \isacommand{by}\isamarkupfalse%
\ {\isacharparenleft}rule\ conjunct{\isadigit{2}}{\isacharparenright}\isanewline
\ \ \isacommand{then}\isamarkupfalse%
\ \isacommand{have}\isamarkupfalse%
\ {\isadigit{2}}{\isacharcolon}{\isachardoublequoteopen}{\isasymforall}F\ {\isasymin}\ {\isacharbraceleft}\isactrlbold {\isasymnot}\ H{\isacharbraceright}{\isachardot}\ {\isasymA}\ {\isasymTurnstile}\ F{\isachardoublequoteclose}\isanewline
\ \ \ \ \isacommand{by}\isamarkupfalse%
\ simp\isanewline
\ \ \isacommand{have}\isamarkupfalse%
\ {\isachardoublequoteopen}{\isasymforall}F\ {\isasymin}\ {\isacharparenleft}{\isacharbraceleft}\isactrlbold {\isasymnot}\ G{\isacharbraceright}\ {\isasymunion}\ {\isacharbraceleft}\isactrlbold {\isasymnot}\ H{\isacharbraceright}{\isacharparenright}\ {\isasymunion}\ {\isacharparenleft}{\isacharbraceleft}F{\isacharbraceright}\ {\isasymunion}\ Wo{\isacharparenright}{\isachardot}\ {\isasymA}\ {\isasymTurnstile}\ F{\isachardoublequoteclose}\isanewline
\ \ \ \ \isacommand{using}\isamarkupfalse%
\ Forall{\isadigit{1}}\ {\isadigit{1}}\ {\isadigit{2}}\ \isacommand{by}\isamarkupfalse%
\ {\isacharparenleft}iprover\ intro{\isacharcolon}\ ball{\isacharunderscore}Un{\isacharparenright}\isanewline
\ \ \isacommand{then}\isamarkupfalse%
\ \isacommand{have}\isamarkupfalse%
\ {\isachardoublequoteopen}{\isasymforall}F\ {\isasymin}\ {\isacharparenleft}{\isacharbraceleft}\isactrlbold {\isasymnot}\ G{\isacharcomma}\isactrlbold {\isasymnot}\ H{\isacharcomma}\ F{\isacharbraceright}\ {\isasymunion}\ Wo{\isacharparenright}{\isachardot}\ {\isasymA}\ {\isasymTurnstile}\ F{\isachardoublequoteclose}\isanewline
\ \ \ \ \isacommand{by}\isamarkupfalse%
\ simp\isanewline
\ \ \isacommand{then}\isamarkupfalse%
\ \isacommand{have}\isamarkupfalse%
\ {\isachardoublequoteopen}{\isasymexists}{\isasymA}{\isachardot}\ {\isasymforall}F\ {\isasymin}\ {\isacharparenleft}{\isacharbraceleft}\isactrlbold {\isasymnot}\ G{\isacharcomma}\isactrlbold {\isasymnot}\ H{\isacharcomma}F{\isacharbraceright}\ {\isasymunion}\ Wo{\isacharparenright}{\isachardot}\ {\isasymA}\ {\isasymTurnstile}\ F{\isachardoublequoteclose}\isanewline
\ \ \ \ \isacommand{by}\isamarkupfalse%
\ {\isacharparenleft}iprover\ intro{\isacharcolon}\ exI{\isacharparenright}\isanewline
\ \ \isacommand{thus}\isamarkupfalse%
\ {\isachardoublequoteopen}sat\ {\isacharparenleft}{\isacharbraceleft}\isactrlbold {\isasymnot}\ G{\isacharcomma}\isactrlbold {\isasymnot}\ H{\isacharcomma}F{\isacharbraceright}\ {\isasymunion}\ Wo{\isacharparenright}{\isachardoublequoteclose}\isanewline
\ \ \ \ \isacommand{by}\isamarkupfalse%
\ {\isacharparenleft}simp\ only{\isacharcolon}\ sat{\isacharunderscore}def{\isacharparenright}\isanewline
\isacommand{qed}\isamarkupfalse%
%
\endisatagproof
{\isafoldproof}%
%
\isadelimproof
%
\endisadelimproof
%
\begin{isamarkuptext}%
Probemos detalladamente el resultado para el tercer caso de fórmula de tipo \isa{{\isasymalpha}}: dados 
  \isa{W\ {\isasymin}\ C}, una fórmula \isa{F\ {\isacharequal}\ {\isasymnot}{\isacharparenleft}G\ {\isasymlongrightarrow}\ H{\isacharparenright}} para ciertas fórmulas \isa{G} y \isa{H} tal que \isa{F\ {\isasymin}\ W} y \isa{W\isactrlsub {\isadigit{0}}} un 
  subconjunto finito de \isa{W}, se verifica que \isa{{\isacharbraceleft}G{\isacharcomma}{\isasymnot}\ H{\isacharcomma}F{\isacharbraceright}\ {\isasymunion}\ W\isactrlsub {\isadigit{0}}} es satisfacible.%
\end{isamarkuptext}\isamarkuptrue%
\isacommand{lemma}\isamarkupfalse%
\ pcp{\isacharunderscore}colecComp{\isacharunderscore}CON{\isacharunderscore}sat{\isadigit{3}}{\isacharcolon}\isanewline
\ \ \isakeyword{assumes}\ {\isachardoublequoteopen}W\ {\isasymin}\ colecComp{\isachardoublequoteclose}\isanewline
\ \ \ \ \ \ \ \ \ \ {\isachardoublequoteopen}F\ {\isacharequal}\ \isactrlbold {\isasymnot}\ {\isacharparenleft}G\ \isactrlbold {\isasymrightarrow}\ H{\isacharparenright}{\isachardoublequoteclose}\isanewline
\ \ \ \ \ \ \ \ \ \ {\isachardoublequoteopen}F\ {\isasymin}\ W{\isachardoublequoteclose}\isanewline
\ \ \ \ \ \ \ \ \ \ {\isachardoublequoteopen}finite\ Wo{\isachardoublequoteclose}\isanewline
\ \ \ \ \ \ \ \ \ \ {\isachardoublequoteopen}Wo\ {\isasymsubseteq}\ W{\isachardoublequoteclose}\isanewline
\ \ \ \ \ \ \ \ \isakeyword{shows}\ {\isachardoublequoteopen}sat\ {\isacharparenleft}{\isacharbraceleft}G{\isacharcomma}\isactrlbold {\isasymnot}\ H{\isacharcomma}F{\isacharbraceright}\ {\isasymunion}\ Wo{\isacharparenright}{\isachardoublequoteclose}\isanewline
%
\isadelimproof
%
\endisadelimproof
%
\isatagproof
\isacommand{proof}\isamarkupfalse%
\ {\isacharminus}\isanewline
\ \ \isacommand{have}\isamarkupfalse%
\ {\isachardoublequoteopen}sat\ {\isacharparenleft}{\isacharbraceleft}F{\isacharbraceright}\ {\isasymunion}\ Wo{\isacharparenright}{\isachardoublequoteclose}\isanewline
\ \ \ \ \isacommand{using}\isamarkupfalse%
\ assms{\isacharparenleft}{\isadigit{1}}{\isacharcomma}{\isadigit{3}}{\isacharcomma}{\isadigit{4}}{\isacharcomma}{\isadigit{5}}{\isacharparenright}\ \isacommand{by}\isamarkupfalse%
\ {\isacharparenleft}rule\ pcp{\isacharunderscore}colecComp{\isacharunderscore}elem{\isacharunderscore}sat{\isacharparenright}\isanewline
\ \ \isacommand{have}\isamarkupfalse%
\ {\isachardoublequoteopen}F\ {\isasymin}\ {\isacharbraceleft}F{\isacharbraceright}\ {\isasymunion}\ Wo{\isachardoublequoteclose}\isanewline
\ \ \ \ \isacommand{by}\isamarkupfalse%
\ {\isacharparenleft}simp\ add{\isacharcolon}\ insertI{\isadigit{1}}{\isacharparenright}\isanewline
\ \ \isacommand{have}\isamarkupfalse%
\ Ex{\isadigit{1}}{\isacharcolon}{\isachardoublequoteopen}{\isasymexists}{\isasymA}{\isachardot}\ {\isasymforall}F\ {\isasymin}\ {\isacharparenleft}{\isacharbraceleft}F{\isacharbraceright}\ {\isasymunion}\ Wo{\isacharparenright}{\isachardot}\ {\isasymA}\ {\isasymTurnstile}\ F{\isachardoublequoteclose}\isanewline
\ \ \ \ \isacommand{using}\isamarkupfalse%
\ {\isacartoucheopen}sat\ {\isacharparenleft}{\isacharbraceleft}F{\isacharbraceright}\ {\isasymunion}\ Wo{\isacharparenright}{\isacartoucheclose}\ \isacommand{by}\isamarkupfalse%
\ {\isacharparenleft}simp\ only{\isacharcolon}\ sat{\isacharunderscore}def{\isacharparenright}\isanewline
\ \ \isacommand{obtain}\isamarkupfalse%
\ {\isasymA}\ \isakeyword{where}\ Forall{\isadigit{1}}{\isacharcolon}{\isachardoublequoteopen}{\isasymforall}F\ {\isasymin}\ {\isacharparenleft}{\isacharbraceleft}F{\isacharbraceright}\ {\isasymunion}\ Wo{\isacharparenright}{\isachardot}\ {\isasymA}\ {\isasymTurnstile}\ F{\isachardoublequoteclose}\isanewline
\ \ \ \ \isacommand{using}\isamarkupfalse%
\ Ex{\isadigit{1}}\ \isacommand{by}\isamarkupfalse%
\ {\isacharparenleft}rule\ exE{\isacharparenright}\isanewline
\ \ \isacommand{have}\isamarkupfalse%
\ {\isachardoublequoteopen}{\isasymA}\ {\isasymTurnstile}\ F{\isachardoublequoteclose}\isanewline
\ \ \ \ \isacommand{using}\isamarkupfalse%
\ Forall{\isadigit{1}}\ {\isacartoucheopen}F\ {\isasymin}\ {\isacharbraceleft}F{\isacharbraceright}\ {\isasymunion}\ Wo{\isacartoucheclose}\ \isacommand{by}\isamarkupfalse%
\ {\isacharparenleft}rule\ bspec{\isacharparenright}\isanewline
\ \ \isacommand{then}\isamarkupfalse%
\ \isacommand{have}\isamarkupfalse%
\ {\isachardoublequoteopen}{\isasymA}\ {\isasymTurnstile}\ \isactrlbold {\isasymnot}{\isacharparenleft}G\ \isactrlbold {\isasymrightarrow}\ H{\isacharparenright}{\isachardoublequoteclose}\isanewline
\ \ \ \ \isacommand{using}\isamarkupfalse%
\ assms{\isacharparenleft}{\isadigit{2}}{\isacharparenright}\ \isacommand{by}\isamarkupfalse%
\ {\isacharparenleft}simp\ only{\isacharcolon}\ {\isacartoucheopen}{\isasymA}\ {\isasymTurnstile}\ F{\isacartoucheclose}{\isacharparenright}\isanewline
\ \ \isacommand{then}\isamarkupfalse%
\ \isacommand{have}\isamarkupfalse%
\ {\isachardoublequoteopen}{\isasymnot}{\isacharparenleft}{\isasymA}\ {\isasymTurnstile}\ {\isacharparenleft}G\ \isactrlbold {\isasymrightarrow}\ H{\isacharparenright}{\isacharparenright}{\isachardoublequoteclose}\isanewline
\ \ \ \ \isacommand{by}\isamarkupfalse%
\ {\isacharparenleft}simp\ only{\isacharcolon}\ formula{\isacharunderscore}semantics{\isachardot}simps{\isacharparenleft}{\isadigit{3}}{\isacharparenright}\ simp{\isacharunderscore}thms{\isacharparenleft}{\isadigit{8}}{\isacharparenright}{\isacharparenright}\isanewline
\ \ \isacommand{then}\isamarkupfalse%
\ \isacommand{have}\isamarkupfalse%
\ {\isachardoublequoteopen}{\isasymnot}{\isacharparenleft}{\isasymA}\ {\isasymTurnstile}\ G\ {\isasymlongrightarrow}\ {\isasymA}\ {\isasymTurnstile}\ H{\isacharparenright}{\isachardoublequoteclose}\isanewline
\ \ \ \ \isacommand{by}\isamarkupfalse%
\ {\isacharparenleft}simp\ only{\isacharcolon}\ formula{\isacharunderscore}semantics{\isachardot}simps{\isacharparenleft}{\isadigit{6}}{\isacharparenright}\ simp{\isacharunderscore}thms{\isacharparenleft}{\isadigit{8}}{\isacharparenright}{\isacharparenright}\isanewline
\ \ \isacommand{then}\isamarkupfalse%
\ \isacommand{have}\isamarkupfalse%
\ {\isachardoublequoteopen}{\isasymA}\ {\isasymTurnstile}\ G\ {\isasymand}\ {\isasymnot}\ {\isasymA}\ {\isasymTurnstile}\ H{\isachardoublequoteclose}\isanewline
\ \ \ \ \isacommand{by}\isamarkupfalse%
\ {\isacharparenleft}simp\ only{\isacharcolon}\ not{\isacharunderscore}imp\ simp{\isacharunderscore}thms{\isacharparenleft}{\isadigit{8}}{\isacharparenright}{\isacharparenright}\isanewline
\ \ \isacommand{then}\isamarkupfalse%
\ \isacommand{have}\isamarkupfalse%
\ {\isachardoublequoteopen}{\isasymA}\ {\isasymTurnstile}\ G\ {\isasymand}\ {\isasymA}\ {\isasymTurnstile}\ \isactrlbold {\isasymnot}\ H{\isachardoublequoteclose}\isanewline
\ \ \ \ \isacommand{by}\isamarkupfalse%
\ {\isacharparenleft}simp\ only{\isacharcolon}\ formula{\isacharunderscore}semantics{\isachardot}simps{\isacharparenleft}{\isadigit{3}}{\isacharparenright}\ simp{\isacharunderscore}thms{\isacharparenleft}{\isadigit{8}}{\isacharparenright}{\isacharparenright}\ \isanewline
\ \ \isacommand{then}\isamarkupfalse%
\ \isacommand{have}\isamarkupfalse%
\ {\isachardoublequoteopen}{\isasymA}\ {\isasymTurnstile}\ G{\isachardoublequoteclose}\isanewline
\ \ \ \ \isacommand{by}\isamarkupfalse%
\ {\isacharparenleft}rule\ conjunct{\isadigit{1}}{\isacharparenright}\isanewline
\ \ \isacommand{then}\isamarkupfalse%
\ \isacommand{have}\isamarkupfalse%
\ {\isadigit{1}}{\isacharcolon}{\isachardoublequoteopen}{\isasymforall}F\ {\isasymin}\ {\isacharbraceleft}G{\isacharbraceright}{\isachardot}\ {\isasymA}\ {\isasymTurnstile}\ F{\isachardoublequoteclose}\isanewline
\ \ \ \ \isacommand{by}\isamarkupfalse%
\ simp\isanewline
\ \ \isacommand{have}\isamarkupfalse%
\ {\isachardoublequoteopen}{\isasymA}\ {\isasymTurnstile}\ \isactrlbold {\isasymnot}\ H{\isachardoublequoteclose}\isanewline
\ \ \ \ \isacommand{using}\isamarkupfalse%
\ {\isacartoucheopen}{\isasymA}\ {\isasymTurnstile}\ G\ {\isasymand}\ {\isasymA}\ {\isasymTurnstile}\ \isactrlbold {\isasymnot}\ H{\isacartoucheclose}\ \isacommand{by}\isamarkupfalse%
\ {\isacharparenleft}rule\ conjunct{\isadigit{2}}{\isacharparenright}\isanewline
\ \ \isacommand{then}\isamarkupfalse%
\ \isacommand{have}\isamarkupfalse%
\ {\isadigit{2}}{\isacharcolon}{\isachardoublequoteopen}{\isasymforall}F\ {\isasymin}\ {\isacharbraceleft}\isactrlbold {\isasymnot}\ H{\isacharbraceright}{\isachardot}\ {\isasymA}\ {\isasymTurnstile}\ F{\isachardoublequoteclose}\isanewline
\ \ \ \ \isacommand{by}\isamarkupfalse%
\ simp\isanewline
\ \ \isacommand{have}\isamarkupfalse%
\ {\isachardoublequoteopen}{\isasymforall}F\ {\isasymin}\ {\isacharparenleft}{\isacharbraceleft}G{\isacharbraceright}\ {\isasymunion}\ {\isacharbraceleft}\isactrlbold {\isasymnot}\ H{\isacharbraceright}{\isacharparenright}\ {\isasymunion}\ {\isacharparenleft}{\isacharbraceleft}F{\isacharbraceright}\ {\isasymunion}\ Wo{\isacharparenright}{\isachardot}\ {\isasymA}\ {\isasymTurnstile}\ F{\isachardoublequoteclose}\isanewline
\ \ \ \ \isacommand{using}\isamarkupfalse%
\ Forall{\isadigit{1}}\ {\isadigit{1}}\ {\isadigit{2}}\ \isacommand{by}\isamarkupfalse%
\ {\isacharparenleft}iprover\ intro{\isacharcolon}\ ball{\isacharunderscore}Un{\isacharparenright}\isanewline
\ \ \isacommand{then}\isamarkupfalse%
\ \isacommand{have}\isamarkupfalse%
\ {\isachardoublequoteopen}{\isasymforall}F\ {\isasymin}\ {\isacharbraceleft}G{\isacharcomma}\isactrlbold {\isasymnot}\ H{\isacharcomma}F{\isacharbraceright}\ {\isasymunion}\ Wo{\isachardot}\ {\isasymA}\ {\isasymTurnstile}\ F{\isachardoublequoteclose}\isanewline
\ \ \ \ \isacommand{by}\isamarkupfalse%
\ simp\isanewline
\ \ \isacommand{then}\isamarkupfalse%
\ \isacommand{have}\isamarkupfalse%
\ {\isachardoublequoteopen}{\isasymexists}{\isasymA}{\isachardot}\ {\isasymforall}F\ {\isasymin}\ {\isacharparenleft}{\isacharbraceleft}G{\isacharcomma}\isactrlbold {\isasymnot}\ H{\isacharcomma}F{\isacharbraceright}\ {\isasymunion}\ Wo{\isacharparenright}{\isachardot}\ {\isasymA}\ {\isasymTurnstile}\ F{\isachardoublequoteclose}\isanewline
\ \ \ \ \isacommand{by}\isamarkupfalse%
\ {\isacharparenleft}iprover\ intro{\isacharcolon}\ exI{\isacharparenright}\isanewline
\ \ \isacommand{thus}\isamarkupfalse%
\ {\isachardoublequoteopen}sat\ {\isacharparenleft}{\isacharbraceleft}G{\isacharcomma}\isactrlbold {\isasymnot}\ H{\isacharcomma}F{\isacharbraceright}\ {\isasymunion}\ Wo{\isacharparenright}{\isachardoublequoteclose}\isanewline
\ \ \ \ \isacommand{by}\isamarkupfalse%
\ {\isacharparenleft}simp\ only{\isacharcolon}\ sat{\isacharunderscore}def{\isacharparenright}\isanewline
\isacommand{qed}\isamarkupfalse%
%
\endisatagproof
{\isafoldproof}%
%
\isadelimproof
%
\endisadelimproof
%
\begin{isamarkuptext}%
Por último, la prueba detallada del resultado para el cuarto caso de fórmula de tipo \isa{{\isasymalpha}}: 
  dados \isa{W\ {\isasymin}\ C}, una fórmula \isa{F\ {\isacharequal}\ {\isasymnot}{\isacharparenleft}{\isasymnot}\ G{\isacharparenright}} para cierta fórmula \isa{G} tal que \isa{F\ {\isasymin}\ W}, \isa{H\ {\isacharequal}\ G} y \isa{W\isactrlsub {\isadigit{0}}} 
  un subconjunto finito de \isa{W}, se verifica que \isa{{\isacharbraceleft}G{\isacharcomma}H{\isacharcomma}F{\isacharbraceright}\ {\isasymunion}\ W\isactrlsub {\isadigit{0}}} es satisfacible.%
\end{isamarkuptext}\isamarkuptrue%
\isacommand{lemma}\isamarkupfalse%
\ pcp{\isacharunderscore}colecComp{\isacharunderscore}CON{\isacharunderscore}sat{\isadigit{4}}{\isacharcolon}\isanewline
\ \ \isakeyword{assumes}\ {\isachardoublequoteopen}W\ {\isasymin}\ colecComp{\isachardoublequoteclose}\isanewline
\ \ \ \ \ \ \ \ \ \ {\isachardoublequoteopen}F\ {\isacharequal}\ \isactrlbold {\isasymnot}\ {\isacharparenleft}\isactrlbold {\isasymnot}\ G{\isacharparenright}{\isachardoublequoteclose}\isanewline
\ \ \ \ \ \ \ \ \ \ {\isachardoublequoteopen}H\ {\isacharequal}\ G{\isachardoublequoteclose}\isanewline
\ \ \ \ \ \ \ \ \ \ {\isachardoublequoteopen}F\ {\isasymin}\ W{\isachardoublequoteclose}\isanewline
\ \ \ \ \ \ \ \ \ \ {\isachardoublequoteopen}finite\ Wo{\isachardoublequoteclose}\isanewline
\ \ \ \ \ \ \ \ \ \ {\isachardoublequoteopen}Wo\ {\isasymsubseteq}\ W{\isachardoublequoteclose}\isanewline
\ \ \ \ \ \ \ \ \isakeyword{shows}\ {\isachardoublequoteopen}sat\ {\isacharparenleft}{\isacharbraceleft}G{\isacharcomma}H{\isacharcomma}F{\isacharbraceright}\ {\isasymunion}\ Wo{\isacharparenright}{\isachardoublequoteclose}\isanewline
%
\isadelimproof
%
\endisadelimproof
%
\isatagproof
\isacommand{proof}\isamarkupfalse%
\ {\isacharminus}\isanewline
\ \ \isacommand{have}\isamarkupfalse%
\ {\isachardoublequoteopen}sat\ {\isacharparenleft}{\isacharbraceleft}F{\isacharbraceright}\ {\isasymunion}\ Wo{\isacharparenright}{\isachardoublequoteclose}\isanewline
\ \ \ \ \isacommand{using}\isamarkupfalse%
\ assms{\isacharparenleft}{\isadigit{1}}{\isacharcomma}{\isadigit{4}}{\isacharcomma}{\isadigit{5}}{\isacharcomma}{\isadigit{6}}{\isacharparenright}\ \isacommand{by}\isamarkupfalse%
\ {\isacharparenleft}rule\ pcp{\isacharunderscore}colecComp{\isacharunderscore}elem{\isacharunderscore}sat{\isacharparenright}\isanewline
\ \ \isacommand{have}\isamarkupfalse%
\ {\isachardoublequoteopen}F\ {\isasymin}\ {\isacharbraceleft}F{\isacharbraceright}\ {\isasymunion}\ Wo{\isachardoublequoteclose}\isanewline
\ \ \ \ \isacommand{by}\isamarkupfalse%
\ {\isacharparenleft}simp\ add{\isacharcolon}\ insertI{\isadigit{1}}{\isacharparenright}\isanewline
\ \ \isacommand{have}\isamarkupfalse%
\ Ex{\isadigit{1}}{\isacharcolon}{\isachardoublequoteopen}{\isasymexists}{\isasymA}{\isachardot}\ {\isasymforall}F\ {\isasymin}\ {\isacharparenleft}{\isacharbraceleft}F{\isacharbraceright}\ {\isasymunion}\ Wo{\isacharparenright}{\isachardot}\ {\isasymA}\ {\isasymTurnstile}\ F{\isachardoublequoteclose}\isanewline
\ \ \ \ \isacommand{using}\isamarkupfalse%
\ {\isacartoucheopen}sat\ {\isacharparenleft}{\isacharbraceleft}F{\isacharbraceright}\ {\isasymunion}\ Wo{\isacharparenright}{\isacartoucheclose}\ \isacommand{by}\isamarkupfalse%
\ {\isacharparenleft}simp\ only{\isacharcolon}\ sat{\isacharunderscore}def{\isacharparenright}\isanewline
\ \ \isacommand{obtain}\isamarkupfalse%
\ {\isasymA}\ \isakeyword{where}\ Forall{\isadigit{1}}{\isacharcolon}{\isachardoublequoteopen}{\isasymforall}F\ {\isasymin}\ {\isacharparenleft}{\isacharbraceleft}F{\isacharbraceright}\ {\isasymunion}\ Wo{\isacharparenright}{\isachardot}\ {\isasymA}\ {\isasymTurnstile}\ F{\isachardoublequoteclose}\isanewline
\ \ \ \ \isacommand{using}\isamarkupfalse%
\ Ex{\isadigit{1}}\ \isacommand{by}\isamarkupfalse%
\ {\isacharparenleft}rule\ exE{\isacharparenright}\isanewline
\ \ \isacommand{have}\isamarkupfalse%
\ {\isachardoublequoteopen}{\isasymA}\ {\isasymTurnstile}\ F{\isachardoublequoteclose}\isanewline
\ \ \ \ \isacommand{using}\isamarkupfalse%
\ Forall{\isadigit{1}}\ {\isacartoucheopen}F\ {\isasymin}\ {\isacharbraceleft}F{\isacharbraceright}\ {\isasymunion}\ Wo{\isacartoucheclose}\ \isacommand{by}\isamarkupfalse%
\ {\isacharparenleft}rule\ bspec{\isacharparenright}\isanewline
\ \ \isacommand{then}\isamarkupfalse%
\ \isacommand{have}\isamarkupfalse%
\ {\isachardoublequoteopen}{\isasymA}\ {\isasymTurnstile}\ \isactrlbold {\isasymnot}{\isacharparenleft}\isactrlbold {\isasymnot}\ G{\isacharparenright}{\isachardoublequoteclose}\isanewline
\ \ \ \ \isacommand{using}\isamarkupfalse%
\ assms{\isacharparenleft}{\isadigit{2}}{\isacharparenright}\ \isacommand{by}\isamarkupfalse%
\ {\isacharparenleft}simp\ only{\isacharcolon}\ {\isacartoucheopen}{\isasymA}\ {\isasymTurnstile}\ F{\isacartoucheclose}{\isacharparenright}\isanewline
\ \ \isacommand{then}\isamarkupfalse%
\ \isacommand{have}\isamarkupfalse%
\ {\isachardoublequoteopen}{\isasymnot}\ {\isasymA}\ {\isasymTurnstile}\ \isactrlbold {\isasymnot}\ G{\isachardoublequoteclose}\isanewline
\ \ \ \ \isacommand{by}\isamarkupfalse%
\ {\isacharparenleft}simp\ only{\isacharcolon}\ formula{\isacharunderscore}semantics{\isachardot}simps{\isacharparenleft}{\isadigit{3}}{\isacharparenright}\ simp{\isacharunderscore}thms{\isacharparenleft}{\isadigit{8}}{\isacharparenright}{\isacharparenright}\isanewline
\ \ \isacommand{then}\isamarkupfalse%
\ \isacommand{have}\isamarkupfalse%
\ {\isachardoublequoteopen}{\isasymnot}\ {\isasymnot}{\isasymA}\ {\isasymTurnstile}\ G{\isachardoublequoteclose}\isanewline
\ \ \ \ \isacommand{by}\isamarkupfalse%
\ {\isacharparenleft}simp\ only{\isacharcolon}\ formula{\isacharunderscore}semantics{\isachardot}simps{\isacharparenleft}{\isadigit{3}}{\isacharparenright}\ simp{\isacharunderscore}thms{\isacharparenleft}{\isadigit{8}}{\isacharparenright}{\isacharparenright}\isanewline
\ \ \isacommand{then}\isamarkupfalse%
\ \isacommand{have}\isamarkupfalse%
\ {\isachardoublequoteopen}{\isasymA}\ {\isasymTurnstile}\ G{\isachardoublequoteclose}\isanewline
\ \ \ \ \isacommand{by}\isamarkupfalse%
\ {\isacharparenleft}rule\ notnotD{\isacharparenright}\isanewline
\ \ \isacommand{then}\isamarkupfalse%
\ \isacommand{have}\isamarkupfalse%
\ {\isadigit{1}}{\isacharcolon}{\isachardoublequoteopen}{\isasymforall}F\ {\isasymin}\ {\isacharbraceleft}G{\isacharbraceright}{\isachardot}\ {\isasymA}\ {\isasymTurnstile}\ F{\isachardoublequoteclose}\isanewline
\ \ \ \ \isacommand{by}\isamarkupfalse%
\ simp\isanewline
\ \ \isacommand{have}\isamarkupfalse%
\ {\isachardoublequoteopen}{\isasymA}\ {\isasymTurnstile}\ H{\isachardoublequoteclose}\isanewline
\ \ \ \ \isacommand{using}\isamarkupfalse%
\ {\isacartoucheopen}{\isasymA}\ {\isasymTurnstile}\ G{\isacartoucheclose}\ \isacommand{by}\isamarkupfalse%
\ {\isacharparenleft}simp\ only{\isacharcolon}\ {\isacartoucheopen}H\ {\isacharequal}\ G{\isacartoucheclose}{\isacharparenright}\isanewline
\ \ \isacommand{then}\isamarkupfalse%
\ \isacommand{have}\isamarkupfalse%
\ {\isadigit{2}}{\isacharcolon}{\isachardoublequoteopen}{\isasymforall}F\ {\isasymin}\ {\isacharbraceleft}H{\isacharbraceright}{\isachardot}\ {\isasymA}\ {\isasymTurnstile}\ F{\isachardoublequoteclose}\isanewline
\ \ \ \ \isacommand{by}\isamarkupfalse%
\ simp\isanewline
\ \ \isacommand{have}\isamarkupfalse%
\ {\isachardoublequoteopen}{\isasymforall}F\ {\isasymin}\ {\isacharparenleft}{\isacharbraceleft}G{\isacharbraceright}\ {\isasymunion}\ {\isacharbraceleft}H{\isacharbraceright}{\isacharparenright}\ {\isasymunion}\ {\isacharparenleft}{\isacharbraceleft}F{\isacharbraceright}\ {\isasymunion}\ Wo{\isacharparenright}{\isachardot}\ {\isasymA}\ {\isasymTurnstile}\ F{\isachardoublequoteclose}\isanewline
\ \ \ \ \isacommand{using}\isamarkupfalse%
\ Forall{\isadigit{1}}\ {\isadigit{1}}\ {\isadigit{2}}\ \isacommand{by}\isamarkupfalse%
\ {\isacharparenleft}iprover\ intro{\isacharcolon}\ ball{\isacharunderscore}Un{\isacharparenright}\isanewline
\ \ \isacommand{then}\isamarkupfalse%
\ \isacommand{have}\isamarkupfalse%
\ {\isachardoublequoteopen}{\isasymforall}F\ {\isasymin}\ {\isacharbraceleft}G{\isacharcomma}H{\isacharcomma}F{\isacharbraceright}\ {\isasymunion}\ Wo{\isachardot}\ {\isasymA}\ {\isasymTurnstile}\ F{\isachardoublequoteclose}\isanewline
\ \ \ \ \isacommand{by}\isamarkupfalse%
\ simp\isanewline
\ \ \isacommand{then}\isamarkupfalse%
\ \isacommand{have}\isamarkupfalse%
\ {\isachardoublequoteopen}{\isasymexists}{\isasymA}{\isachardot}\ {\isasymforall}F\ {\isasymin}\ {\isacharparenleft}{\isacharbraceleft}G{\isacharcomma}H{\isacharcomma}F{\isacharbraceright}\ {\isasymunion}\ Wo{\isacharparenright}{\isachardot}\ {\isasymA}\ {\isasymTurnstile}\ F{\isachardoublequoteclose}\isanewline
\ \ \ \ \isacommand{by}\isamarkupfalse%
\ {\isacharparenleft}iprover\ intro{\isacharcolon}\ exI{\isacharparenright}\isanewline
\ \ \isacommand{thus}\isamarkupfalse%
\ {\isachardoublequoteopen}sat\ {\isacharparenleft}{\isacharbraceleft}G{\isacharcomma}H{\isacharcomma}F{\isacharbraceright}\ {\isasymunion}\ Wo{\isacharparenright}{\isachardoublequoteclose}\isanewline
\ \ \ \ \isacommand{by}\isamarkupfalse%
\ {\isacharparenleft}simp\ only{\isacharcolon}\ sat{\isacharunderscore}def{\isacharparenright}\isanewline
\isacommand{qed}\isamarkupfalse%
%
\endisatagproof
{\isafoldproof}%
%
\isadelimproof
%
\endisadelimproof
%
\begin{isamarkuptext}%
Por tanto, por las pruebas detalladas de los casos anteriores, podemos demostrar que dados 
  \isa{W\ {\isasymin}\ C}, \isa{F\ {\isasymin}\ W} una fórmula de tipo \isa{{\isasymalpha}} con componentes \isa{{\isasymalpha}\isactrlsub {\isadigit{1}}} y \isa{{\isasymalpha}\isactrlsub {\isadigit{2}}} y \isa{W\isactrlsub {\isadigit{0}}} un subconjunto finito 
  de \isa{W}, se verifica que \isa{{\isacharbraceleft}{\isasymalpha}\isactrlsub {\isadigit{1}}{\isacharcomma}{\isasymalpha}\isactrlsub {\isadigit{2}}{\isacharcomma}F{\isacharbraceright}\ {\isasymunion}\ W\isactrlsub {\isadigit{0}}} es satisfacible.%
\end{isamarkuptext}\isamarkuptrue%
\isacommand{lemma}\isamarkupfalse%
\ pcp{\isacharunderscore}colecComp{\isacharunderscore}CON{\isacharunderscore}sat{\isacharcolon}\isanewline
\ \ \isakeyword{assumes}\ {\isachardoublequoteopen}W\ {\isasymin}\ colecComp{\isachardoublequoteclose}\isanewline
\ \ \ \ \ \ \ \ \ \ {\isachardoublequoteopen}Con\ F\ G\ H{\isachardoublequoteclose}\isanewline
\ \ \ \ \ \ \ \ \ \ {\isachardoublequoteopen}F\ {\isasymin}\ W{\isachardoublequoteclose}\isanewline
\ \ \ \ \ \ \ \ \ \ {\isachardoublequoteopen}finite\ Wo{\isachardoublequoteclose}\isanewline
\ \ \ \ \ \ \ \ \ \ {\isachardoublequoteopen}Wo\ {\isasymsubseteq}\ W{\isachardoublequoteclose}\isanewline
\ \ \ \ \ \ \ \ \isakeyword{shows}\ {\isachardoublequoteopen}sat\ {\isacharparenleft}{\isacharbraceleft}G{\isacharcomma}H{\isacharcomma}F{\isacharbraceright}\ {\isasymunion}\ Wo{\isacharparenright}{\isachardoublequoteclose}\isanewline
%
\isadelimproof
%
\endisadelimproof
%
\isatagproof
\isacommand{proof}\isamarkupfalse%
\ {\isacharminus}\isanewline
\ \ \isacommand{have}\isamarkupfalse%
\ {\isachardoublequoteopen}{\isacharbraceleft}G{\isacharcomma}H{\isacharbraceright}\ {\isasymunion}\ Wo\ {\isasymsubseteq}\ {\isacharbraceleft}G{\isacharcomma}H{\isacharcomma}F{\isacharbraceright}\ {\isasymunion}\ Wo{\isachardoublequoteclose}\isanewline
\ \ \ \ \isacommand{by}\isamarkupfalse%
\ blast\isanewline
\ \ \isacommand{have}\isamarkupfalse%
\ {\isachardoublequoteopen}F\ {\isacharequal}\ G\ \isactrlbold {\isasymand}\ H\ {\isasymor}\ \isanewline
\ \ \ \ {\isacharparenleft}{\isasymexists}F{\isadigit{1}}\ G{\isadigit{1}}{\isachardot}\ F\ {\isacharequal}\ \isactrlbold {\isasymnot}\ {\isacharparenleft}F{\isadigit{1}}\ \isactrlbold {\isasymor}\ G{\isadigit{1}}{\isacharparenright}\ {\isasymand}\ G\ {\isacharequal}\ \isactrlbold {\isasymnot}\ F{\isadigit{1}}\ {\isasymand}\ H\ {\isacharequal}\ \isactrlbold {\isasymnot}\ G{\isadigit{1}}{\isacharparenright}\ {\isasymor}\ \isanewline
\ \ \ \ {\isacharparenleft}{\isasymexists}H{\isadigit{1}}{\isachardot}\ F\ {\isacharequal}\ \isactrlbold {\isasymnot}\ {\isacharparenleft}G\ \isactrlbold {\isasymrightarrow}\ H{\isadigit{1}}{\isacharparenright}\ {\isasymand}\ H\ {\isacharequal}\ \isactrlbold {\isasymnot}\ H{\isadigit{1}}{\isacharparenright}\ {\isasymor}\ \isanewline
\ \ \ \ F\ {\isacharequal}\ \isactrlbold {\isasymnot}\ {\isacharparenleft}\isactrlbold {\isasymnot}\ G{\isacharparenright}\ {\isasymand}\ H\ {\isacharequal}\ G{\isachardoublequoteclose}\isanewline
\ \ \ \ \isacommand{using}\isamarkupfalse%
\ assms{\isacharparenleft}{\isadigit{2}}{\isacharparenright}\ \isacommand{by}\isamarkupfalse%
\ {\isacharparenleft}simp\ only{\isacharcolon}\ con{\isacharunderscore}dis{\isacharunderscore}simps{\isacharparenleft}{\isadigit{1}}{\isacharparenright}{\isacharparenright}\isanewline
\ \ \isacommand{thus}\isamarkupfalse%
\ {\isachardoublequoteopen}sat\ {\isacharparenleft}{\isacharbraceleft}G{\isacharcomma}H{\isacharcomma}F{\isacharbraceright}\ {\isasymunion}\ Wo{\isacharparenright}{\isachardoublequoteclose}\isanewline
\ \ \isacommand{proof}\isamarkupfalse%
\ {\isacharparenleft}rule\ disjE{\isacharparenright}\isanewline
\ \ \ \ \isacommand{assume}\isamarkupfalse%
\ {\isachardoublequoteopen}F\ {\isacharequal}\ G\ \isactrlbold {\isasymand}\ H{\isachardoublequoteclose}\isanewline
\ \ \ \ \isacommand{show}\isamarkupfalse%
\ {\isachardoublequoteopen}sat\ {\isacharparenleft}{\isacharbraceleft}G{\isacharcomma}H{\isacharcomma}F{\isacharbraceright}\ {\isasymunion}\ Wo{\isacharparenright}{\isachardoublequoteclose}\isanewline
\ \ \ \ \ \ \isacommand{using}\isamarkupfalse%
\ assms{\isacharparenleft}{\isadigit{1}}{\isacharparenright}\ {\isacartoucheopen}F\ {\isacharequal}\ G\ \isactrlbold {\isasymand}\ H{\isacartoucheclose}\ assms{\isacharparenleft}{\isadigit{3}}{\isacharcomma}{\isadigit{4}}{\isacharcomma}{\isadigit{5}}{\isacharparenright}\ \isacommand{by}\isamarkupfalse%
\ {\isacharparenleft}rule\ pcp{\isacharunderscore}colecComp{\isacharunderscore}CON{\isacharunderscore}sat{\isadigit{1}}{\isacharparenright}\isanewline
\ \ \isacommand{next}\isamarkupfalse%
\isanewline
\ \ \ \ \isacommand{assume}\isamarkupfalse%
\ {\isachardoublequoteopen}{\isacharparenleft}{\isasymexists}F{\isadigit{1}}\ G{\isadigit{1}}{\isachardot}\ F\ {\isacharequal}\ \isactrlbold {\isasymnot}\ {\isacharparenleft}F{\isadigit{1}}\ \isactrlbold {\isasymor}\ G{\isadigit{1}}{\isacharparenright}\ {\isasymand}\ G\ {\isacharequal}\ \isactrlbold {\isasymnot}\ F{\isadigit{1}}\ {\isasymand}\ H\ {\isacharequal}\ \isactrlbold {\isasymnot}\ G{\isadigit{1}}{\isacharparenright}\ {\isasymor}\ \isanewline
\ \ \ \ {\isacharparenleft}{\isasymexists}H{\isadigit{1}}{\isachardot}\ F\ {\isacharequal}\ \isactrlbold {\isasymnot}\ {\isacharparenleft}G\ \isactrlbold {\isasymrightarrow}\ H{\isadigit{1}}{\isacharparenright}\ {\isasymand}\ H\ {\isacharequal}\ \isactrlbold {\isasymnot}\ H{\isadigit{1}}{\isacharparenright}\ {\isasymor}\ \isanewline
\ \ \ \ F\ {\isacharequal}\ \isactrlbold {\isasymnot}\ {\isacharparenleft}\isactrlbold {\isasymnot}\ G{\isacharparenright}\ {\isasymand}\ H\ {\isacharequal}\ G{\isachardoublequoteclose}\isanewline
\ \ \ \ \isacommand{thus}\isamarkupfalse%
\ {\isachardoublequoteopen}sat\ {\isacharparenleft}{\isacharbraceleft}G{\isacharcomma}H{\isacharcomma}F{\isacharbraceright}\ {\isasymunion}\ Wo{\isacharparenright}{\isachardoublequoteclose}\isanewline
\ \ \ \ \isacommand{proof}\isamarkupfalse%
\ {\isacharparenleft}rule\ disjE{\isacharparenright}\isanewline
\ \ \ \ \ \ \isacommand{assume}\isamarkupfalse%
\ Ex{\isadigit{2}}{\isacharcolon}{\isachardoublequoteopen}{\isasymexists}F{\isadigit{1}}\ G{\isadigit{1}}{\isachardot}\ F\ {\isacharequal}\ \isactrlbold {\isasymnot}\ {\isacharparenleft}F{\isadigit{1}}\ \isactrlbold {\isasymor}\ G{\isadigit{1}}{\isacharparenright}\ {\isasymand}\ G\ {\isacharequal}\ \isactrlbold {\isasymnot}\ F{\isadigit{1}}\ {\isasymand}\ H\ {\isacharequal}\ \isactrlbold {\isasymnot}\ G{\isadigit{1}}{\isachardoublequoteclose}\ \isanewline
\ \ \ \ \ \ \isacommand{obtain}\isamarkupfalse%
\ F{\isadigit{1}}\ G{\isadigit{1}}\ \isakeyword{where}\ {\isadigit{2}}{\isacharcolon}{\isachardoublequoteopen}F\ {\isacharequal}\ \isactrlbold {\isasymnot}{\isacharparenleft}F{\isadigit{1}}\ \isactrlbold {\isasymor}\ G{\isadigit{1}}{\isacharparenright}\ {\isasymand}\ G\ {\isacharequal}\ \isactrlbold {\isasymnot}\ F{\isadigit{1}}\ {\isasymand}\ H\ {\isacharequal}\ \isactrlbold {\isasymnot}\ G{\isadigit{1}}{\isachardoublequoteclose}\isanewline
\ \ \ \ \ \ \ \ \isacommand{using}\isamarkupfalse%
\ Ex{\isadigit{2}}\ \isacommand{by}\isamarkupfalse%
\ {\isacharparenleft}iprover\ elim{\isacharcolon}\ exE{\isacharparenright}\isanewline
\ \ \ \ \ \ \isacommand{have}\isamarkupfalse%
\ {\isachardoublequoteopen}G\ {\isacharequal}\ \isactrlbold {\isasymnot}\ F{\isadigit{1}}{\isachardoublequoteclose}\isanewline
\ \ \ \ \ \ \ \ \isacommand{using}\isamarkupfalse%
\ {\isadigit{2}}\ \isacommand{by}\isamarkupfalse%
\ {\isacharparenleft}iprover\ elim{\isacharcolon}\ conjunct{\isadigit{1}}{\isacharparenright}\isanewline
\ \ \ \ \ \ \isacommand{have}\isamarkupfalse%
\ {\isachardoublequoteopen}H\ {\isacharequal}\ \isactrlbold {\isasymnot}\ G{\isadigit{1}}{\isachardoublequoteclose}\isanewline
\ \ \ \ \ \ \ \ \isacommand{using}\isamarkupfalse%
\ {\isadigit{2}}\ \isacommand{by}\isamarkupfalse%
\ {\isacharparenleft}iprover\ elim{\isacharcolon}\ conjunct{\isadigit{2}}{\isacharparenright}\isanewline
\ \ \ \ \ \ \isacommand{have}\isamarkupfalse%
\ {\isachardoublequoteopen}F\ {\isacharequal}\ \isactrlbold {\isasymnot}{\isacharparenleft}F{\isadigit{1}}\ \isactrlbold {\isasymor}\ G{\isadigit{1}}{\isacharparenright}{\isachardoublequoteclose}\isanewline
\ \ \ \ \ \ \ \ \isacommand{using}\isamarkupfalse%
\ {\isadigit{2}}\ \isacommand{by}\isamarkupfalse%
\ {\isacharparenleft}rule\ conjunct{\isadigit{1}}{\isacharparenright}\isanewline
\ \ \ \ \ \ \isacommand{have}\isamarkupfalse%
\ {\isachardoublequoteopen}sat\ {\isacharparenleft}{\isacharbraceleft}\isactrlbold {\isasymnot}\ F{\isadigit{1}}{\isacharcomma}\ \isactrlbold {\isasymnot}\ G{\isadigit{1}}{\isacharcomma}\ F{\isacharbraceright}\ {\isasymunion}\ Wo{\isacharparenright}{\isachardoublequoteclose}\isanewline
\ \ \ \ \ \ \ \ \isacommand{using}\isamarkupfalse%
\ assms{\isacharparenleft}{\isadigit{1}}{\isacharparenright}\ {\isacartoucheopen}F\ {\isacharequal}\ \isactrlbold {\isasymnot}{\isacharparenleft}F{\isadigit{1}}\ \isactrlbold {\isasymor}\ G{\isadigit{1}}{\isacharparenright}{\isacartoucheclose}\ assms{\isacharparenleft}{\isadigit{3}}{\isacharcomma}{\isadigit{4}}{\isacharcomma}{\isadigit{5}}{\isacharparenright}\ \isacommand{by}\isamarkupfalse%
\ {\isacharparenleft}rule\ pcp{\isacharunderscore}colecComp{\isacharunderscore}CON{\isacharunderscore}sat{\isadigit{2}}{\isacharparenright}\isanewline
\ \ \ \ \ \ \isacommand{thus}\isamarkupfalse%
\ {\isachardoublequoteopen}sat\ {\isacharparenleft}{\isacharbraceleft}G{\isacharcomma}H{\isacharcomma}F{\isacharbraceright}\ {\isasymunion}\ Wo{\isacharparenright}{\isachardoublequoteclose}\isanewline
\ \ \ \ \ \ \ \ \isacommand{by}\isamarkupfalse%
\ {\isacharparenleft}simp\ only{\isacharcolon}\ {\isacartoucheopen}G\ {\isacharequal}\ \isactrlbold {\isasymnot}\ F{\isadigit{1}}{\isacartoucheclose}\ {\isacartoucheopen}H\ {\isacharequal}\ \isactrlbold {\isasymnot}\ G{\isadigit{1}}{\isacartoucheclose}{\isacharparenright}\isanewline
\ \ \ \ \isacommand{next}\isamarkupfalse%
\isanewline
\ \ \ \ \ \ \isacommand{assume}\isamarkupfalse%
\ {\isachardoublequoteopen}{\isacharparenleft}{\isasymexists}H{\isadigit{1}}{\isachardot}\ F\ {\isacharequal}\ \isactrlbold {\isasymnot}\ {\isacharparenleft}G\ \isactrlbold {\isasymrightarrow}\ H{\isadigit{1}}{\isacharparenright}\ {\isasymand}\ H\ {\isacharequal}\ \isactrlbold {\isasymnot}\ H{\isadigit{1}}{\isacharparenright}\ {\isasymor}\ \isanewline
\ \ \ \ \ \ \ \ \ \ \ \ \ \ F\ {\isacharequal}\ \isactrlbold {\isasymnot}\ {\isacharparenleft}\isactrlbold {\isasymnot}\ G{\isacharparenright}\ {\isasymand}\ H\ {\isacharequal}\ G{\isachardoublequoteclose}\isanewline
\ \ \ \ \ \ \isacommand{thus}\isamarkupfalse%
\ {\isachardoublequoteopen}sat\ {\isacharparenleft}{\isacharbraceleft}G{\isacharcomma}H{\isacharcomma}F{\isacharbraceright}\ {\isasymunion}\ Wo{\isacharparenright}{\isachardoublequoteclose}\isanewline
\ \ \ \ \ \ \isacommand{proof}\isamarkupfalse%
\ {\isacharparenleft}rule\ disjE{\isacharparenright}\isanewline
\ \ \ \ \ \ \ \ \isacommand{assume}\isamarkupfalse%
\ Ex{\isadigit{3}}{\isacharcolon}{\isachardoublequoteopen}{\isasymexists}H{\isadigit{1}}{\isachardot}\ F\ {\isacharequal}\ \isactrlbold {\isasymnot}\ {\isacharparenleft}G\ \isactrlbold {\isasymrightarrow}\ H{\isadigit{1}}{\isacharparenright}\ {\isasymand}\ H\ {\isacharequal}\ \isactrlbold {\isasymnot}\ H{\isadigit{1}}{\isachardoublequoteclose}\isanewline
\ \ \ \ \ \ \ \ \isacommand{obtain}\isamarkupfalse%
\ H{\isadigit{1}}\ \isakeyword{where}\ {\isadigit{3}}{\isacharcolon}{\isachardoublequoteopen}F\ {\isacharequal}\ \isactrlbold {\isasymnot}{\isacharparenleft}G\ \isactrlbold {\isasymrightarrow}\ H{\isadigit{1}}{\isacharparenright}\ {\isasymand}\ H\ {\isacharequal}\ \isactrlbold {\isasymnot}\ H{\isadigit{1}}{\isachardoublequoteclose}\isanewline
\ \ \ \ \ \ \ \ \ \ \isacommand{using}\isamarkupfalse%
\ Ex{\isadigit{3}}\ \isacommand{by}\isamarkupfalse%
\ {\isacharparenleft}rule\ exE{\isacharparenright}\isanewline
\ \ \ \ \ \ \ \ \isacommand{have}\isamarkupfalse%
\ {\isachardoublequoteopen}H\ {\isacharequal}\ \isactrlbold {\isasymnot}\ H{\isadigit{1}}{\isachardoublequoteclose}\isanewline
\ \ \ \ \ \ \ \ \ \ \isacommand{using}\isamarkupfalse%
\ {\isadigit{3}}\ \isacommand{by}\isamarkupfalse%
\ {\isacharparenleft}rule\ conjunct{\isadigit{2}}{\isacharparenright}\isanewline
\ \ \ \ \ \ \ \ \isacommand{have}\isamarkupfalse%
\ {\isachardoublequoteopen}F\ {\isacharequal}\ \isactrlbold {\isasymnot}{\isacharparenleft}G\ \isactrlbold {\isasymrightarrow}\ H{\isadigit{1}}{\isacharparenright}{\isachardoublequoteclose}\isanewline
\ \ \ \ \ \ \ \ \ \ \isacommand{using}\isamarkupfalse%
\ {\isadigit{3}}\ \isacommand{by}\isamarkupfalse%
\ {\isacharparenleft}rule\ conjunct{\isadigit{1}}{\isacharparenright}\isanewline
\ \ \ \ \ \ \ \ \isacommand{have}\isamarkupfalse%
\ {\isachardoublequoteopen}sat\ {\isacharparenleft}{\isacharbraceleft}G{\isacharcomma}\ \isactrlbold {\isasymnot}\ H{\isadigit{1}}{\isacharcomma}\ F{\isacharbraceright}\ {\isasymunion}\ Wo{\isacharparenright}{\isachardoublequoteclose}\isanewline
\ \ \ \ \ \ \ \ \ \ \isacommand{using}\isamarkupfalse%
\ assms{\isacharparenleft}{\isadigit{1}}{\isacharparenright}\ {\isacartoucheopen}F\ {\isacharequal}\ \isactrlbold {\isasymnot}{\isacharparenleft}G\ \isactrlbold {\isasymrightarrow}\ H{\isadigit{1}}{\isacharparenright}{\isacartoucheclose}\ assms{\isacharparenleft}{\isadigit{3}}{\isacharcomma}{\isadigit{4}}{\isacharcomma}{\isadigit{5}}{\isacharparenright}\ \isacommand{by}\isamarkupfalse%
\ {\isacharparenleft}rule\ pcp{\isacharunderscore}colecComp{\isacharunderscore}CON{\isacharunderscore}sat{\isadigit{3}}{\isacharparenright}\isanewline
\ \ \ \ \ \ \ \ \isacommand{thus}\isamarkupfalse%
\ {\isachardoublequoteopen}sat\ {\isacharparenleft}{\isacharbraceleft}G{\isacharcomma}H{\isacharcomma}F{\isacharbraceright}\ {\isasymunion}\ Wo{\isacharparenright}{\isachardoublequoteclose}\isanewline
\ \ \ \ \ \ \ \ \ \ \isacommand{by}\isamarkupfalse%
\ {\isacharparenleft}simp\ only{\isacharcolon}\ {\isacartoucheopen}H\ {\isacharequal}\ \isactrlbold {\isasymnot}\ H{\isadigit{1}}{\isacartoucheclose}{\isacharparenright}\isanewline
\ \ \ \ \ \ \isacommand{next}\isamarkupfalse%
\isanewline
\ \ \ \ \ \ \ \ \isacommand{assume}\isamarkupfalse%
\ {\isachardoublequoteopen}F\ {\isacharequal}\ \isactrlbold {\isasymnot}\ {\isacharparenleft}\isactrlbold {\isasymnot}\ G{\isacharparenright}\ {\isasymand}\ H\ {\isacharequal}\ G{\isachardoublequoteclose}\isanewline
\ \ \ \ \ \ \ \ \isacommand{then}\isamarkupfalse%
\ \isacommand{have}\isamarkupfalse%
\ {\isachardoublequoteopen}H\ {\isacharequal}\ G{\isachardoublequoteclose}\isanewline
\ \ \ \ \ \ \ \ \ \ \isacommand{by}\isamarkupfalse%
\ {\isacharparenleft}rule\ conjunct{\isadigit{2}}{\isacharparenright}\isanewline
\ \ \ \ \ \ \ \ \isacommand{have}\isamarkupfalse%
\ {\isachardoublequoteopen}F\ {\isacharequal}\ \isactrlbold {\isasymnot}\ {\isacharparenleft}\isactrlbold {\isasymnot}\ G{\isacharparenright}{\isachardoublequoteclose}\isanewline
\ \ \ \ \ \ \ \ \ \ \isacommand{using}\isamarkupfalse%
\ {\isacartoucheopen}F\ {\isacharequal}\ \isactrlbold {\isasymnot}\ {\isacharparenleft}\isactrlbold {\isasymnot}\ G{\isacharparenright}\ {\isasymand}\ H\ {\isacharequal}\ G{\isacartoucheclose}\ \isacommand{by}\isamarkupfalse%
\ {\isacharparenleft}rule\ conjunct{\isadigit{1}}{\isacharparenright}\isanewline
\ \ \ \ \ \ \ \ \isacommand{show}\isamarkupfalse%
\ {\isachardoublequoteopen}sat\ {\isacharparenleft}{\isacharbraceleft}G{\isacharcomma}\ H{\isacharcomma}\ F{\isacharbraceright}\ {\isasymunion}\ Wo{\isacharparenright}{\isachardoublequoteclose}\isanewline
\ \ \ \ \ \ \ \ \ \ \isacommand{using}\isamarkupfalse%
\ assms{\isacharparenleft}{\isadigit{1}}{\isacharparenright}\ {\isacartoucheopen}F\ {\isacharequal}\ \isactrlbold {\isasymnot}{\isacharparenleft}\isactrlbold {\isasymnot}\ G{\isacharparenright}{\isacartoucheclose}\ {\isacartoucheopen}H\ {\isacharequal}\ G{\isacartoucheclose}\ assms{\isacharparenleft}{\isadigit{3}}{\isacharcomma}{\isadigit{4}}{\isacharcomma}{\isadigit{5}}{\isacharparenright}\ \isacommand{by}\isamarkupfalse%
\ {\isacharparenleft}rule\ pcp{\isacharunderscore}colecComp{\isacharunderscore}CON{\isacharunderscore}sat{\isadigit{4}}{\isacharparenright}\isanewline
\ \ \ \ \ \ \isacommand{qed}\isamarkupfalse%
\isanewline
\ \ \ \ \isacommand{qed}\isamarkupfalse%
\isanewline
\ \ \isacommand{qed}\isamarkupfalse%
\isanewline
\isacommand{qed}\isamarkupfalse%
%
\endisatagproof
{\isafoldproof}%
%
\isadelimproof
%
\endisadelimproof
%
\begin{isamarkuptext}%
Finalmente, con el resultado anterior, podemos probar la tercera condición del lema \isa{{\isadigit{2}}{\isachardot}{\isadigit{0}}{\isachardot}{\isadigit{2}}}: 
  dados \isa{W\ {\isasymin}\ C} y \isa{F} una fórmula de tipo \isa{{\isasymalpha}} con componentes \isa{{\isasymalpha}\isactrlsub {\isadigit{1}}} y \isa{{\isasymalpha}\isactrlsub {\isadigit{2}}} tal que \isa{F\ {\isasymin}\ W}, se tiene 
  que \isa{{\isacharbraceleft}{\isasymalpha}\isactrlsub {\isadigit{1}}{\isacharcomma}{\isasymalpha}\isactrlsub {\isadigit{2}}{\isacharbraceright}\ {\isasymunion}\ W\ {\isasymin}\ C}.%
\end{isamarkuptext}\isamarkuptrue%
\isacommand{lemma}\isamarkupfalse%
\ pcp{\isacharunderscore}colecComp{\isacharunderscore}CON{\isacharcolon}\isanewline
\ \ \isakeyword{assumes}\ {\isachardoublequoteopen}W\ {\isasymin}\ colecComp{\isachardoublequoteclose}\isanewline
\ \ \isakeyword{shows}\ {\isachardoublequoteopen}{\isasymforall}F\ G\ H{\isachardot}\ Con\ F\ G\ H\ {\isasymlongrightarrow}\ F\ {\isasymin}\ W\ {\isasymlongrightarrow}\ {\isacharbraceleft}G{\isacharcomma}H{\isacharbraceright}\ {\isasymunion}\ W\ {\isasymin}\ colecComp{\isachardoublequoteclose}\isanewline
%
\isadelimproof
%
\endisadelimproof
%
\isatagproof
\isacommand{proof}\isamarkupfalse%
\ {\isacharparenleft}rule\ allI{\isacharparenright}{\isacharplus}\isanewline
\ \ \isacommand{fix}\isamarkupfalse%
\ F\ G\ H\isanewline
\ \ \isacommand{show}\isamarkupfalse%
\ {\isachardoublequoteopen}Con\ F\ G\ H\ {\isasymlongrightarrow}\ F\ {\isasymin}\ W\ {\isasymlongrightarrow}\ {\isacharbraceleft}G{\isacharcomma}H{\isacharbraceright}\ {\isasymunion}\ W\ {\isasymin}\ colecComp{\isachardoublequoteclose}\isanewline
\ \ \isacommand{proof}\isamarkupfalse%
\ {\isacharparenleft}rule\ impI{\isacharparenright}{\isacharplus}\isanewline
\ \ \ \ \isacommand{assume}\isamarkupfalse%
\ {\isachardoublequoteopen}Con\ F\ G\ H{\isachardoublequoteclose}\isanewline
\ \ \ \ \isacommand{assume}\isamarkupfalse%
\ {\isachardoublequoteopen}F\ {\isasymin}\ W{\isachardoublequoteclose}\isanewline
\ \ \ \ \isacommand{show}\isamarkupfalse%
\ {\isachardoublequoteopen}{\isacharbraceleft}G{\isacharcomma}H{\isacharbraceright}\ {\isasymunion}\ W\ {\isasymin}\ colecComp{\isachardoublequoteclose}\isanewline
\ \ \ \ \ \ \isacommand{unfolding}\isamarkupfalse%
\ colecComp\ fin{\isacharunderscore}sat{\isacharunderscore}def\isanewline
\ \ \ \ \isacommand{proof}\isamarkupfalse%
\ {\isacharparenleft}rule\ CollectI{\isacharparenright}\isanewline
\ \ \ \ \ \ \isacommand{show}\isamarkupfalse%
\ {\isachardoublequoteopen}{\isasymforall}S\ {\isasymsubseteq}\ {\isacharbraceleft}G{\isacharcomma}H{\isacharbraceright}\ {\isasymunion}\ W{\isachardot}\ finite\ S\ {\isasymlongrightarrow}\ sat\ S{\isachardoublequoteclose}\isanewline
\ \ \ \ \ \ \isacommand{proof}\isamarkupfalse%
\ {\isacharparenleft}rule\ sallI{\isacharparenright}\isanewline
\ \ \ \ \ \ \ \ \isacommand{fix}\isamarkupfalse%
\ S\isanewline
\ \ \ \ \ \ \ \ \isacommand{assume}\isamarkupfalse%
\ {\isachardoublequoteopen}S\ {\isasymsubseteq}\ {\isacharbraceleft}G{\isacharcomma}H{\isacharbraceright}\ {\isasymunion}\ W{\isachardoublequoteclose}\isanewline
\ \ \ \ \ \ \ \ \isacommand{then}\isamarkupfalse%
\ \isacommand{have}\isamarkupfalse%
\ {\isachardoublequoteopen}S\ {\isasymsubseteq}\ {\isacharbraceleft}G{\isacharbraceright}\ {\isasymunion}\ {\isacharparenleft}{\isacharbraceleft}H{\isacharbraceright}\ {\isasymunion}\ W{\isacharparenright}{\isachardoublequoteclose}\isanewline
\ \ \ \ \ \ \ \ \ \ \isacommand{by}\isamarkupfalse%
\ blast\ \isanewline
\ \ \ \ \ \ \ \ \isacommand{show}\isamarkupfalse%
\ {\isachardoublequoteopen}finite\ S\ {\isasymlongrightarrow}\ sat\ S{\isachardoublequoteclose}\isanewline
\ \ \ \ \ \ \ \ \isacommand{proof}\isamarkupfalse%
\ {\isacharparenleft}rule\ impI{\isacharparenright}\isanewline
\ \ \ \ \ \ \ \ \ \ \isacommand{assume}\isamarkupfalse%
\ {\isachardoublequoteopen}finite\ S{\isachardoublequoteclose}\ \isanewline
\ \ \ \ \ \ \ \ \ \ \isacommand{have}\isamarkupfalse%
\ Ex{\isacharcolon}{\isachardoublequoteopen}{\isasymexists}Wo\ {\isasymsubseteq}\ W{\isachardot}\ finite\ Wo\ {\isasymand}\ {\isacharparenleft}S\ {\isacharequal}\ {\isacharbraceleft}G{\isacharcomma}H{\isacharbraceright}\ {\isasymunion}\ Wo\ {\isasymor}\ S\ {\isacharequal}\ {\isacharbraceleft}G{\isacharbraceright}\ {\isasymunion}\ Wo\ {\isasymor}\ S\ {\isacharequal}\ {\isacharbraceleft}H{\isacharbraceright}\ {\isasymunion}\ Wo\ {\isasymor}\ S\ {\isacharequal}\ Wo{\isacharparenright}{\isachardoublequoteclose}\isanewline
\ \ \ \ \ \ \ \ \ \ \ \ \isacommand{using}\isamarkupfalse%
\ {\isacartoucheopen}finite\ S{\isacartoucheclose}\ {\isacartoucheopen}S\ {\isasymsubseteq}\ {\isacharbraceleft}G{\isacharcomma}H{\isacharbraceright}\ {\isasymunion}\ W{\isacartoucheclose}\ \isacommand{by}\isamarkupfalse%
\ {\isacharparenleft}rule\ finite{\isacharunderscore}subset{\isacharunderscore}insert{\isadigit{2}}{\isacharparenright}\isanewline
\ \ \ \ \ \ \ \ \ \ \isacommand{obtain}\isamarkupfalse%
\ Wo\ \isakeyword{where}\ {\isachardoublequoteopen}Wo\ {\isasymsubseteq}\ W{\isachardoublequoteclose}\ \isakeyword{and}\ {\isadigit{1}}{\isacharcolon}{\isachardoublequoteopen}finite\ Wo\ {\isasymand}\ {\isacharparenleft}S\ {\isacharequal}\ {\isacharbraceleft}G{\isacharcomma}H{\isacharbraceright}\ {\isasymunion}\ Wo\ {\isasymor}\ S\ {\isacharequal}\ {\isacharbraceleft}G{\isacharbraceright}\ {\isasymunion}\ Wo\ {\isasymor}\ S\ {\isacharequal}\ {\isacharbraceleft}H{\isacharbraceright}\ {\isasymunion}\ Wo\ {\isasymor}\ S\ {\isacharequal}\ Wo{\isacharparenright}{\isachardoublequoteclose}\isanewline
\ \ \ \ \ \ \ \ \ \ \ \ \isacommand{using}\isamarkupfalse%
\ Ex\ \isacommand{by}\isamarkupfalse%
\ {\isacharparenleft}rule\ subexE{\isacharparenright}\isanewline
\ \ \ \ \ \ \ \ \ \ \isacommand{have}\isamarkupfalse%
\ {\isachardoublequoteopen}finite\ Wo{\isachardoublequoteclose}\isanewline
\ \ \ \ \ \ \ \ \ \ \ \ \isacommand{using}\isamarkupfalse%
\ {\isadigit{1}}\ \isacommand{by}\isamarkupfalse%
\ {\isacharparenleft}rule\ conjunct{\isadigit{1}}{\isacharparenright}\isanewline
\ \ \ \ \ \ \ \ \ \ \ \ \isacommand{have}\isamarkupfalse%
\ {\isachardoublequoteopen}sat\ {\isacharparenleft}{\isacharbraceleft}G{\isacharcomma}H{\isacharcomma}F{\isacharbraceright}\ {\isasymunion}\ Wo{\isacharparenright}{\isachardoublequoteclose}\ \isanewline
\ \ \ \ \ \ \ \ \ \ \ \ \ \ \isacommand{using}\isamarkupfalse%
\ {\isacartoucheopen}W\ {\isasymin}\ colecComp{\isacartoucheclose}\ {\isacartoucheopen}Con\ F\ G\ H{\isacartoucheclose}\ {\isacartoucheopen}F\ {\isasymin}\ W{\isacartoucheclose}\ {\isacartoucheopen}finite\ Wo{\isacartoucheclose}\ {\isacartoucheopen}Wo\ {\isasymsubseteq}\ W{\isacartoucheclose}\ \isacommand{by}\isamarkupfalse%
\ {\isacharparenleft}rule\ pcp{\isacharunderscore}colecComp{\isacharunderscore}CON{\isacharunderscore}sat{\isacharparenright}\isanewline
\ \ \ \ \ \ \ \ \ \ \isacommand{have}\isamarkupfalse%
\ {\isachardoublequoteopen}S\ {\isacharequal}\ {\isacharbraceleft}G{\isacharcomma}H{\isacharbraceright}\ {\isasymunion}\ Wo\ {\isasymor}\ S\ {\isacharequal}\ {\isacharbraceleft}G{\isacharbraceright}\ {\isasymunion}\ Wo\ {\isasymor}\ S\ {\isacharequal}\ {\isacharbraceleft}H{\isacharbraceright}\ {\isasymunion}\ Wo\ {\isasymor}\ S\ {\isacharequal}\ Wo{\isachardoublequoteclose}\isanewline
\ \ \ \ \ \ \ \ \ \ \ \ \isacommand{using}\isamarkupfalse%
\ {\isadigit{1}}\ \isacommand{by}\isamarkupfalse%
\ {\isacharparenleft}rule\ conjunct{\isadigit{2}}{\isacharparenright}\isanewline
\ \ \ \ \ \ \ \ \ \ \isacommand{thus}\isamarkupfalse%
\ {\isachardoublequoteopen}sat\ S{\isachardoublequoteclose}\isanewline
\ \ \ \ \ \ \ \ \ \ \isacommand{proof}\isamarkupfalse%
\ {\isacharparenleft}rule\ disjE{\isacharparenright}\isanewline
\ \ \ \ \ \ \ \ \ \ \ \ \isacommand{assume}\isamarkupfalse%
\ {\isachardoublequoteopen}S\ {\isacharequal}\ {\isacharbraceleft}G{\isacharcomma}H{\isacharbraceright}\ {\isasymunion}\ Wo{\isachardoublequoteclose}\isanewline
\ \ \ \ \ \ \ \ \ \ \ \ \isacommand{then}\isamarkupfalse%
\ \isacommand{have}\isamarkupfalse%
\ {\isachardoublequoteopen}S\ {\isasymsubseteq}\ {\isacharbraceleft}G{\isacharcomma}H{\isacharcomma}F{\isacharbraceright}\ {\isasymunion}\ Wo{\isachardoublequoteclose}\isanewline
\ \ \ \ \ \ \ \ \ \ \ \ \ \ \isacommand{by}\isamarkupfalse%
\ blast\isanewline
\ \ \ \ \ \ \ \ \ \ \ \ \isacommand{show}\isamarkupfalse%
\ {\isachardoublequoteopen}sat\ S{\isachardoublequoteclose}\isanewline
\ \ \ \ \ \ \ \ \ \ \ \ \ \ \isacommand{using}\isamarkupfalse%
\ {\isacartoucheopen}sat{\isacharparenleft}{\isacharbraceleft}G{\isacharcomma}H{\isacharcomma}F{\isacharbraceright}\ {\isasymunion}\ Wo{\isacharparenright}{\isacartoucheclose}\ {\isacartoucheopen}S\ {\isasymsubseteq}\ {\isacharbraceleft}G{\isacharcomma}H{\isacharcomma}F{\isacharbraceright}\ {\isasymunion}\ Wo{\isacartoucheclose}\ \isacommand{by}\isamarkupfalse%
\ {\isacharparenleft}simp\ only{\isacharcolon}\ sat{\isacharunderscore}mono{\isacharparenright}\isanewline
\ \ \ \ \ \ \ \ \ \ \isacommand{next}\isamarkupfalse%
\isanewline
\ \ \ \ \ \ \ \ \ \ \ \ \isacommand{assume}\isamarkupfalse%
\ {\isachardoublequoteopen}S\ {\isacharequal}\ {\isacharbraceleft}G{\isacharbraceright}\ {\isasymunion}\ Wo\ {\isasymor}\ S\ {\isacharequal}\ {\isacharbraceleft}H{\isacharbraceright}\ {\isasymunion}\ Wo\ {\isasymor}\ S\ {\isacharequal}\ Wo{\isachardoublequoteclose}\isanewline
\ \ \ \ \ \ \ \ \ \ \ \ \isacommand{thus}\isamarkupfalse%
\ {\isachardoublequoteopen}sat\ S{\isachardoublequoteclose}\isanewline
\ \ \ \ \ \ \ \ \ \ \ \ \isacommand{proof}\isamarkupfalse%
\ {\isacharparenleft}rule\ disjE{\isacharparenright}\isanewline
\ \ \ \ \ \ \ \ \ \ \ \ \ \ \isacommand{assume}\isamarkupfalse%
\ {\isachardoublequoteopen}S\ {\isacharequal}\ {\isacharbraceleft}G{\isacharbraceright}\ {\isasymunion}\ Wo{\isachardoublequoteclose}\isanewline
\ \ \ \ \ \ \ \ \ \ \ \ \ \ \isacommand{then}\isamarkupfalse%
\ \isacommand{have}\isamarkupfalse%
\ {\isachardoublequoteopen}S\ {\isasymsubseteq}\ {\isacharbraceleft}G{\isacharcomma}H{\isacharcomma}F{\isacharbraceright}\ {\isasymunion}\ Wo{\isachardoublequoteclose}\isanewline
\ \ \ \ \ \ \ \ \ \ \ \ \ \ \ \ \isacommand{by}\isamarkupfalse%
\ blast\ \isanewline
\ \ \ \ \ \ \ \ \ \ \ \ \ \ \isacommand{thus}\isamarkupfalse%
\ {\isachardoublequoteopen}sat\ S{\isachardoublequoteclose}\isanewline
\ \ \ \ \ \ \ \ \ \ \ \ \ \ \ \ \isacommand{using}\isamarkupfalse%
\ {\isacartoucheopen}sat{\isacharparenleft}{\isacharbraceleft}G{\isacharcomma}H{\isacharcomma}F{\isacharbraceright}\ {\isasymunion}\ Wo{\isacharparenright}{\isacartoucheclose}\ \isacommand{by}\isamarkupfalse%
\ {\isacharparenleft}rule\ sat{\isacharunderscore}mono{\isacharparenright}\isanewline
\ \ \ \ \ \ \ \ \ \ \ \ \isacommand{next}\isamarkupfalse%
\isanewline
\ \ \ \ \ \ \ \ \ \ \ \ \ \ \isacommand{assume}\isamarkupfalse%
\ {\isachardoublequoteopen}S\ {\isacharequal}\ {\isacharbraceleft}H{\isacharbraceright}\ {\isasymunion}\ Wo\ {\isasymor}\ S\ {\isacharequal}\ Wo{\isachardoublequoteclose}\isanewline
\ \ \ \ \ \ \ \ \ \ \ \ \ \ \isacommand{thus}\isamarkupfalse%
\ {\isachardoublequoteopen}sat\ S{\isachardoublequoteclose}\isanewline
\ \ \ \ \ \ \ \ \ \ \ \ \ \ \isacommand{proof}\isamarkupfalse%
\ {\isacharparenleft}rule\ disjE{\isacharparenright}\isanewline
\ \ \ \ \ \ \ \ \ \ \ \ \ \ \ \ \isacommand{assume}\isamarkupfalse%
\ {\isachardoublequoteopen}S\ {\isacharequal}\ {\isacharbraceleft}H{\isacharbraceright}\ {\isasymunion}\ Wo{\isachardoublequoteclose}\isanewline
\ \ \ \ \ \ \ \ \ \ \ \ \ \ \ \ \isacommand{then}\isamarkupfalse%
\ \isacommand{have}\isamarkupfalse%
\ {\isachardoublequoteopen}S\ {\isasymsubseteq}\ {\isacharbraceleft}G{\isacharcomma}H{\isacharcomma}F{\isacharbraceright}\ {\isasymunion}\ Wo{\isachardoublequoteclose}\isanewline
\ \ \ \ \ \ \ \ \ \ \ \ \ \ \ \ \ \ \isacommand{by}\isamarkupfalse%
\ blast\ \isanewline
\ \ \ \ \ \ \ \ \ \ \ \ \ \ \ \ \isacommand{thus}\isamarkupfalse%
\ {\isachardoublequoteopen}sat\ S{\isachardoublequoteclose}\isanewline
\ \ \ \ \ \ \ \ \ \ \ \ \ \ \ \ \ \ \isacommand{using}\isamarkupfalse%
\ {\isacartoucheopen}sat{\isacharparenleft}{\isacharbraceleft}G{\isacharcomma}H{\isacharcomma}F{\isacharbraceright}\ {\isasymunion}\ Wo{\isacharparenright}{\isacartoucheclose}\ \isacommand{by}\isamarkupfalse%
\ {\isacharparenleft}rule\ sat{\isacharunderscore}mono{\isacharparenright}\isanewline
\ \ \ \ \ \ \ \ \ \ \ \ \ \ \isacommand{next}\isamarkupfalse%
\isanewline
\ \ \ \ \ \ \ \ \ \ \ \ \ \ \ \ \isacommand{assume}\isamarkupfalse%
\ {\isachardoublequoteopen}S\ {\isacharequal}\ Wo{\isachardoublequoteclose}\isanewline
\ \ \ \ \ \ \ \ \ \ \ \ \ \ \ \ \isacommand{then}\isamarkupfalse%
\ \isacommand{have}\isamarkupfalse%
\ {\isachardoublequoteopen}S\ {\isasymsubseteq}\ {\isacharbraceleft}G{\isacharcomma}H{\isacharcomma}F{\isacharbraceright}\ {\isasymunion}\ Wo{\isachardoublequoteclose}\isanewline
\ \ \ \ \ \ \ \ \ \ \ \ \ \ \ \ \ \ \isacommand{by}\isamarkupfalse%
\ {\isacharparenleft}simp\ only{\isacharcolon}\ Un{\isacharunderscore}upper{\isadigit{2}}{\isacharparenright}\isanewline
\ \ \ \ \ \ \ \ \ \ \ \ \ \ \ \ \isacommand{thus}\isamarkupfalse%
\ {\isachardoublequoteopen}sat\ S{\isachardoublequoteclose}\isanewline
\ \ \ \ \ \ \ \ \ \ \ \ \ \ \ \ \ \ \isacommand{using}\isamarkupfalse%
\ {\isacartoucheopen}sat{\isacharparenleft}{\isacharbraceleft}G{\isacharcomma}H{\isacharcomma}F{\isacharbraceright}\ {\isasymunion}\ Wo{\isacharparenright}{\isacartoucheclose}\ \isacommand{by}\isamarkupfalse%
\ {\isacharparenleft}rule\ sat{\isacharunderscore}mono{\isacharparenright}\isanewline
\ \ \ \ \ \ \ \ \ \ \ \ \ \ \isacommand{qed}\isamarkupfalse%
\isanewline
\ \ \ \ \ \ \ \ \ \ \ \ \isacommand{qed}\isamarkupfalse%
\isanewline
\ \ \ \ \ \ \ \ \ \ \isacommand{qed}\isamarkupfalse%
\isanewline
\ \ \ \ \ \ \ \ \isacommand{qed}\isamarkupfalse%
\isanewline
\ \ \ \ \ \ \isacommand{qed}\isamarkupfalse%
\isanewline
\ \ \ \ \isacommand{qed}\isamarkupfalse%
\isanewline
\ \ \isacommand{qed}\isamarkupfalse%
\isanewline
\isacommand{qed}\isamarkupfalse%
%
\endisatagproof
{\isafoldproof}%
%
\isadelimproof
%
\endisadelimproof
%
\begin{isamarkuptext}%
Por último, probemos la cuarta condición del lema \isa{{\isadigit{2}}{\isachardot}{\isadigit{0}}{\isachardot}{\isadigit{2}}}: dados \isa{W\ {\isasymin}\ C} y \isa{F} una 
  fórmula de tipo \isa{{\isasymbeta}} con componentes \isa{{\isasymbeta}\isactrlsub {\isadigit{1}}} y \isa{{\isasymbeta}\isactrlsub {\isadigit{2}}} tal que \isa{F\ {\isasymin}\ W}, se tiene que o bien\\ \isa{{\isacharbraceleft}{\isasymbeta}\isactrlsub {\isadigit{1}}{\isacharbraceright}\ {\isasymunion}\ W\ {\isasymin}\ C} 
  o bien \isa{{\isacharbraceleft}{\isasymbeta}\isactrlsub {\isadigit{2}}{\isacharbraceright}\ {\isasymunion}\ W\ {\isasymin}\ C}. 
  
  Por un lado, precisaremos para ello de un lema auxiliar que demuestre que dado \isa{W\ {\isasymin}\ C} y \isa{{\isasymbeta}\isactrlsub i} una 
  fórmula cualquiera tal que \isa{{\isacharbraceleft}{\isasymbeta}\isactrlsub i{\isacharbraceright}\ {\isasymunion}\ W\ {\isasymnotin}\ C}, entonces existe un subconjunto finito \isa{W\isactrlsub i} de \isa{W} tal 
  que el conjunto \isa{{\isacharbraceleft}{\isasymbeta}\isactrlsub i{\isacharbraceright}\ {\isasymunion}\ W\isactrlsub i} no es satisfacible. A su vez, para su demostración emplearemos un lema 
  que prueba que todo conjunto que contiene un subconjunto insatisfacible es también 
  insatisfacible.%
\end{isamarkuptext}\isamarkuptrue%
\isacommand{lemma}\isamarkupfalse%
\ sat{\isacharunderscore}subset{\isacharunderscore}ccontr{\isacharcolon}\isanewline
\ \ \isakeyword{assumes}\ {\isachardoublequoteopen}A\ {\isasymsubseteq}\ B{\isachardoublequoteclose}\isanewline
\ \ \ \ \ \ \ \ \ \ {\isachardoublequoteopen}{\isasymnot}\ sat\ A{\isachardoublequoteclose}\isanewline
\ \ \ \ \ \ \ \ \isakeyword{shows}\ {\isachardoublequoteopen}{\isasymnot}\ sat\ B{\isachardoublequoteclose}\isanewline
%
\isadelimproof
%
\endisadelimproof
%
\isatagproof
\isacommand{proof}\isamarkupfalse%
\ {\isacharminus}\isanewline
\ \ \isacommand{have}\isamarkupfalse%
\ {\isachardoublequoteopen}A\ {\isasymsubseteq}\ B\ {\isasymand}\ sat\ B\ {\isasymlongrightarrow}\ sat\ A{\isachardoublequoteclose}\isanewline
\ \ \ \ \isacommand{using}\isamarkupfalse%
\ sat{\isacharunderscore}mono\ \isacommand{by}\isamarkupfalse%
\ blast\isanewline
\ \ \isacommand{then}\isamarkupfalse%
\ \isacommand{have}\isamarkupfalse%
\ {\isachardoublequoteopen}{\isasymnot}{\isacharparenleft}A\ {\isasymsubseteq}\ B\ {\isasymand}\ sat\ B{\isacharparenright}{\isachardoublequoteclose}\isanewline
\ \ \ \ \isacommand{using}\isamarkupfalse%
\ assms{\isacharparenleft}{\isadigit{2}}{\isacharparenright}\ \isacommand{by}\isamarkupfalse%
\ {\isacharparenleft}rule\ mt{\isacharparenright}\isanewline
\ \ \isacommand{then}\isamarkupfalse%
\ \isacommand{have}\isamarkupfalse%
\ {\isachardoublequoteopen}{\isasymnot}{\isacharparenleft}A\ {\isasymsubseteq}\ B{\isacharparenright}\ {\isasymor}\ {\isasymnot}{\isacharparenleft}sat\ B{\isacharparenright}{\isachardoublequoteclose}\isanewline
\ \ \ \ \isacommand{by}\isamarkupfalse%
\ {\isacharparenleft}simp\ only{\isacharcolon}\ de{\isacharunderscore}Morgan{\isacharunderscore}conj{\isacharparenright}\isanewline
\ \ \isacommand{thus}\isamarkupfalse%
\ {\isacharquery}thesis\isanewline
\ \ \isacommand{proof}\isamarkupfalse%
\ {\isacharparenleft}rule\ disjE{\isacharparenright}\isanewline
\ \ \ \ \isacommand{assume}\isamarkupfalse%
\ {\isachardoublequoteopen}{\isasymnot}{\isacharparenleft}A\ {\isasymsubseteq}\ B{\isacharparenright}{\isachardoublequoteclose}\isanewline
\ \ \ \ \isacommand{thus}\isamarkupfalse%
\ {\isacharquery}thesis\isanewline
\ \ \ \ \ \ \isacommand{using}\isamarkupfalse%
\ assms{\isacharparenleft}{\isadigit{1}}{\isacharparenright}\ \isacommand{by}\isamarkupfalse%
\ {\isacharparenleft}rule\ notE{\isacharparenright}\isanewline
\ \ \isacommand{next}\isamarkupfalse%
\isanewline
\ \ \ \ \isacommand{assume}\isamarkupfalse%
\ {\isachardoublequoteopen}{\isasymnot}{\isacharparenleft}sat\ B{\isacharparenright}{\isachardoublequoteclose}\isanewline
\ \ \ \ \isacommand{thus}\isamarkupfalse%
\ {\isacharquery}thesis\isanewline
\ \ \ \ \ \ \isacommand{by}\isamarkupfalse%
\ this\isanewline
\ \ \isacommand{qed}\isamarkupfalse%
\isanewline
\isacommand{qed}\isamarkupfalse%
%
\endisatagproof
{\isafoldproof}%
%
\isadelimproof
%
\endisadelimproof
%
\begin{isamarkuptext}%
De este modo, podemos demostrar que dados \isa{W\ {\isasymin}\ C} y \isa{{\isasymbeta}\isactrlsub i} una fórmula cualquiera tal que 
  \isa{{\isacharbraceleft}{\isasymbeta}\isactrlsub i{\isacharbraceright}\ {\isasymunion}\ W\ {\isasymnotin}\ C}, entonces existe un subconjunto finito \isa{W\isactrlsub i} de \isa{W} tal que el conjunto \isa{{\isacharbraceleft}{\isasymbeta}\isactrlsub i{\isacharbraceright}\ {\isasymunion}\ W\isactrlsub F} 
  no es satisfacible.%
\end{isamarkuptext}\isamarkuptrue%
\isacommand{lemma}\isamarkupfalse%
\ not{\isacharunderscore}colecComp{\isacharcolon}\isanewline
\ \ \isakeyword{assumes}\ {\isachardoublequoteopen}W\ {\isasymin}\ colecComp{\isachardoublequoteclose}\isanewline
\ \ \ \ \ \ \ \ \ \ {\isachardoublequoteopen}{\isacharbraceleft}Gi{\isacharbraceright}\ {\isasymunion}\ W\ {\isasymnotin}\ colecComp{\isachardoublequoteclose}\isanewline
\ \ \ \ \ \ \ \ \isakeyword{shows}\ {\isachardoublequoteopen}{\isasymexists}Wi\ {\isasymsubseteq}\ W{\isachardot}\ finite\ Wi\ {\isasymand}\ {\isasymnot}{\isacharparenleft}sat\ {\isacharparenleft}{\isacharbraceleft}Gi{\isacharbraceright}\ {\isasymunion}\ Wi{\isacharparenright}{\isacharparenright}{\isachardoublequoteclose}\isanewline
%
\isadelimproof
%
\endisadelimproof
%
\isatagproof
\isacommand{proof}\isamarkupfalse%
\ {\isacharminus}\isanewline
\ \ \isacommand{have}\isamarkupfalse%
\ WCol{\isacharcolon}{\isachardoublequoteopen}{\isasymforall}S{\isacharprime}\ {\isasymsubseteq}\ W{\isachardot}\ finite\ S{\isacharprime}\ {\isasymlongrightarrow}\ sat\ S{\isacharprime}{\isachardoublequoteclose}\isanewline
\ \ \ \ \isacommand{using}\isamarkupfalse%
\ assms{\isacharparenleft}{\isadigit{1}}{\isacharparenright}\ \isacommand{unfolding}\isamarkupfalse%
\ colecComp\ fin{\isacharunderscore}sat{\isacharunderscore}def\ \isacommand{by}\isamarkupfalse%
\ {\isacharparenleft}rule\ CollectD{\isacharparenright}\ \isanewline
\ \ \isacommand{have}\isamarkupfalse%
\ {\isachardoublequoteopen}{\isasymnot}{\isacharparenleft}{\isasymforall}Wo\ {\isasymsubseteq}\ {\isacharbraceleft}Gi{\isacharbraceright}\ {\isasymunion}\ W{\isachardot}\ finite\ Wo\ {\isasymlongrightarrow}\ sat\ Wo{\isacharparenright}{\isachardoublequoteclose}\isanewline
\ \ \ \ \isacommand{using}\isamarkupfalse%
\ assms{\isacharparenleft}{\isadigit{2}}{\isacharparenright}\ \isacommand{unfolding}\isamarkupfalse%
\ colecComp\ fin{\isacharunderscore}sat{\isacharunderscore}def\ \isacommand{by}\isamarkupfalse%
\ {\isacharparenleft}simp\ only{\isacharcolon}\ mem{\isacharunderscore}Collect{\isacharunderscore}eq\ simp{\isacharunderscore}thms{\isacharparenleft}{\isadigit{8}}{\isacharparenright}{\isacharparenright}\isanewline
\ \ \isacommand{then}\isamarkupfalse%
\ \isacommand{have}\isamarkupfalse%
\ {\isachardoublequoteopen}{\isasymexists}Wo\ {\isasymsubseteq}\ {\isacharbraceleft}Gi{\isacharbraceright}\ {\isasymunion}\ W{\isachardot}\ {\isasymnot}{\isacharparenleft}finite\ Wo\ {\isasymlongrightarrow}\ sat\ Wo{\isacharparenright}{\isachardoublequoteclose}\isanewline
\ \ \ \ \isacommand{by}\isamarkupfalse%
\ {\isacharparenleft}rule\ sall{\isacharunderscore}simps{\isacharunderscore}not{\isacharunderscore}all{\isacharparenright}\isanewline
\ \ \isacommand{then}\isamarkupfalse%
\ \isacommand{have}\isamarkupfalse%
\ Ex{\isadigit{1}}{\isacharcolon}{\isachardoublequoteopen}{\isasymexists}Wo\ {\isasymsubseteq}\ {\isacharbraceleft}Gi{\isacharbraceright}\ {\isasymunion}\ W{\isachardot}\ finite\ Wo\ {\isasymand}\ {\isasymnot}{\isacharparenleft}sat\ Wo{\isacharparenright}{\isachardoublequoteclose}\isanewline
\ \ \ \ \isacommand{by}\isamarkupfalse%
\ {\isacharparenleft}simp\ only{\isacharcolon}\ not{\isacharunderscore}imp{\isacharparenright}\isanewline
\ \ \isacommand{obtain}\isamarkupfalse%
\ Wo\ \isakeyword{where}\ {\isachardoublequoteopen}Wo\ {\isasymsubseteq}\ {\isacharbraceleft}Gi{\isacharbraceright}\ {\isasymunion}\ W{\isachardoublequoteclose}\ \isakeyword{and}\ C{\isadigit{1}}{\isacharcolon}{\isachardoublequoteopen}finite\ Wo\ {\isasymand}\ {\isasymnot}{\isacharparenleft}sat\ Wo{\isacharparenright}{\isachardoublequoteclose}\isanewline
\ \ \ \ \isacommand{using}\isamarkupfalse%
\ Ex{\isadigit{1}}\ \isacommand{by}\isamarkupfalse%
\ {\isacharparenleft}rule\ subexE{\isacharparenright}\isanewline
\ \ \isacommand{have}\isamarkupfalse%
\ {\isachardoublequoteopen}finite\ Wo{\isachardoublequoteclose}\isanewline
\ \ \ \ \isacommand{using}\isamarkupfalse%
\ C{\isadigit{1}}\ \isacommand{by}\isamarkupfalse%
\ {\isacharparenleft}rule\ conjunct{\isadigit{1}}{\isacharparenright}\isanewline
\ \ \isacommand{have}\isamarkupfalse%
\ {\isachardoublequoteopen}{\isasymnot}{\isacharparenleft}sat\ Wo{\isacharparenright}{\isachardoublequoteclose}\isanewline
\ \ \ \ \isacommand{using}\isamarkupfalse%
\ C{\isadigit{1}}\ \isacommand{by}\isamarkupfalse%
\ {\isacharparenleft}rule\ conjunct{\isadigit{2}}{\isacharparenright}\isanewline
\ \ \isacommand{have}\isamarkupfalse%
\ Ex{\isadigit{2}}{\isacharcolon}{\isachardoublequoteopen}{\isasymexists}Wo{\isacharprime}\ {\isasymsubseteq}\ W{\isachardot}\ finite\ Wo{\isacharprime}\ {\isasymand}\ {\isacharparenleft}Wo\ {\isacharequal}\ {\isacharbraceleft}Gi{\isacharbraceright}\ {\isasymunion}\ Wo{\isacharprime}\ {\isasymor}\ Wo\ {\isacharequal}\ Wo{\isacharprime}{\isacharparenright}{\isachardoublequoteclose}\isanewline
\ \ \ \ \isacommand{using}\isamarkupfalse%
\ {\isacartoucheopen}finite\ Wo{\isacartoucheclose}\ {\isacartoucheopen}Wo\ {\isasymsubseteq}\ {\isacharbraceleft}Gi{\isacharbraceright}\ {\isasymunion}\ W{\isacartoucheclose}\ \isacommand{by}\isamarkupfalse%
\ {\isacharparenleft}rule\ finite{\isacharunderscore}subset{\isacharunderscore}insert{\isadigit{1}}{\isacharparenright}\isanewline
\ \ \isacommand{obtain}\isamarkupfalse%
\ Wo{\isacharprime}\ \isakeyword{where}\ {\isachardoublequoteopen}Wo{\isacharprime}\ {\isasymsubseteq}\ W{\isachardoublequoteclose}\ \isakeyword{and}\ C{\isadigit{2}}{\isacharcolon}{\isachardoublequoteopen}finite\ Wo{\isacharprime}\ {\isasymand}\ {\isacharparenleft}Wo\ {\isacharequal}\ {\isacharbraceleft}Gi{\isacharbraceright}\ {\isasymunion}\ Wo{\isacharprime}\ {\isasymor}\ Wo\ {\isacharequal}\ Wo{\isacharprime}{\isacharparenright}{\isachardoublequoteclose}\isanewline
\ \ \ \ \isacommand{using}\isamarkupfalse%
\ Ex{\isadigit{2}}\ \isacommand{by}\isamarkupfalse%
\ blast\isanewline
\ \ \isacommand{have}\isamarkupfalse%
\ {\isachardoublequoteopen}finite\ Wo{\isacharprime}{\isachardoublequoteclose}\isanewline
\ \ \ \ \isacommand{using}\isamarkupfalse%
\ C{\isadigit{2}}\ \isacommand{by}\isamarkupfalse%
\ {\isacharparenleft}rule\ conjunct{\isadigit{1}}{\isacharparenright}\isanewline
\ \ \isacommand{have}\isamarkupfalse%
\ {\isachardoublequoteopen}Wo\ {\isacharequal}\ {\isacharbraceleft}Gi{\isacharbraceright}\ {\isasymunion}\ Wo{\isacharprime}\ {\isasymor}\ Wo\ {\isacharequal}\ Wo{\isacharprime}{\isachardoublequoteclose}\isanewline
\ \ \ \ \isacommand{using}\isamarkupfalse%
\ C{\isadigit{2}}\ \isacommand{by}\isamarkupfalse%
\ {\isacharparenleft}rule\ conjunct{\isadigit{2}}{\isacharparenright}\isanewline
\ \ \isacommand{thus}\isamarkupfalse%
\ {\isacharquery}thesis\isanewline
\ \ \isacommand{proof}\isamarkupfalse%
\ {\isacharparenleft}rule\ disjE{\isacharparenright}\isanewline
\ \ \ \ \isacommand{assume}\isamarkupfalse%
\ {\isachardoublequoteopen}Wo\ {\isacharequal}\ {\isacharbraceleft}Gi{\isacharbraceright}\ {\isasymunion}\ Wo{\isacharprime}{\isachardoublequoteclose}\isanewline
\ \ \ \ \isacommand{then}\isamarkupfalse%
\ \isacommand{have}\isamarkupfalse%
\ {\isachardoublequoteopen}{\isasymnot}{\isacharparenleft}sat\ {\isacharparenleft}{\isacharbraceleft}Gi{\isacharbraceright}\ {\isasymunion}\ Wo{\isacharprime}{\isacharparenright}{\isacharparenright}{\isachardoublequoteclose}\ \isanewline
\ \ \ \ \ \ \isacommand{using}\isamarkupfalse%
\ {\isacartoucheopen}{\isasymnot}\ sat\ Wo{\isacartoucheclose}\ \isacommand{by}\isamarkupfalse%
\ {\isacharparenleft}simp\ only{\isacharcolon}\ {\isacartoucheopen}Wo\ {\isacharequal}\ {\isacharbraceleft}Gi{\isacharbraceright}\ {\isasymunion}\ Wo{\isacharprime}{\isacartoucheclose}\ simp{\isacharunderscore}thms{\isacharparenleft}{\isadigit{8}}{\isacharparenright}{\isacharparenright}\isanewline
\ \ \ \ \isacommand{have}\isamarkupfalse%
\ {\isachardoublequoteopen}finite\ Wo{\isacharprime}\ {\isasymand}\ {\isasymnot}{\isacharparenleft}sat\ {\isacharparenleft}{\isacharbraceleft}Gi{\isacharbraceright}\ {\isasymunion}\ Wo{\isacharprime}{\isacharparenright}{\isacharparenright}{\isachardoublequoteclose}\isanewline
\ \ \ \ \ \ \isacommand{using}\isamarkupfalse%
\ {\isacartoucheopen}finite\ Wo{\isacharprime}{\isacartoucheclose}\ {\isacartoucheopen}{\isasymnot}{\isacharparenleft}sat\ {\isacharparenleft}{\isacharbraceleft}Gi{\isacharbraceright}\ {\isasymunion}\ Wo{\isacharprime}{\isacharparenright}{\isacharparenright}{\isacartoucheclose}\ \isacommand{by}\isamarkupfalse%
\ {\isacharparenleft}rule\ conjI{\isacharparenright}\isanewline
\ \ \ \ \isacommand{thus}\isamarkupfalse%
\ {\isacharquery}thesis\isanewline
\ \ \ \ \ \ \isacommand{using}\isamarkupfalse%
\ {\isacartoucheopen}Wo{\isacharprime}\ {\isasymsubseteq}\ W{\isacartoucheclose}\ \isacommand{by}\isamarkupfalse%
\ {\isacharparenleft}rule\ subexI{\isacharparenright}\isanewline
\ \ \isacommand{next}\isamarkupfalse%
\isanewline
\ \ \ \ \isacommand{assume}\isamarkupfalse%
\ {\isachardoublequoteopen}Wo\ {\isacharequal}\ Wo{\isacharprime}{\isachardoublequoteclose}\isanewline
\ \ \ \ \isacommand{then}\isamarkupfalse%
\ \isacommand{have}\isamarkupfalse%
\ {\isachardoublequoteopen}{\isasymnot}\ {\isacharparenleft}sat\ Wo{\isacharprime}{\isacharparenright}{\isachardoublequoteclose}\isanewline
\ \ \ \ \ \ \isacommand{using}\isamarkupfalse%
\ {\isacartoucheopen}{\isasymnot}\ sat\ Wo{\isacartoucheclose}\ \isacommand{by}\isamarkupfalse%
\ {\isacharparenleft}simp\ only{\isacharcolon}\ {\isacartoucheopen}Wo\ {\isacharequal}\ Wo{\isacharprime}{\isacartoucheclose}\ simp{\isacharunderscore}thms{\isacharparenleft}{\isadigit{8}}{\isacharparenright}{\isacharparenright}\isanewline
\ \ \ \ \isacommand{have}\isamarkupfalse%
\ {\isachardoublequoteopen}Wo{\isacharprime}\ {\isasymsubseteq}\ {\isacharbraceleft}Gi{\isacharbraceright}\ {\isasymunion}\ Wo{\isacharprime}{\isachardoublequoteclose}\isanewline
\ \ \ \ \ \ \isacommand{by}\isamarkupfalse%
\ blast\isanewline
\ \ \ \ \isacommand{then}\isamarkupfalse%
\ \isacommand{have}\isamarkupfalse%
\ {\isachardoublequoteopen}{\isasymnot}\ {\isacharparenleft}sat\ {\isacharparenleft}{\isacharbraceleft}Gi{\isacharbraceright}\ {\isasymunion}\ Wo{\isacharprime}{\isacharparenright}{\isacharparenright}{\isachardoublequoteclose}\isanewline
\ \ \ \ \ \ \isacommand{using}\isamarkupfalse%
\ {\isacartoucheopen}{\isasymnot}\ {\isacharparenleft}sat\ Wo{\isacharprime}{\isacharparenright}{\isacartoucheclose}\ \isacommand{by}\isamarkupfalse%
\ {\isacharparenleft}rule\ sat{\isacharunderscore}subset{\isacharunderscore}ccontr{\isacharparenright}\isanewline
\ \ \ \ \isacommand{have}\isamarkupfalse%
\ {\isachardoublequoteopen}finite\ Wo{\isacharprime}\ {\isasymand}\ {\isasymnot}{\isacharparenleft}sat\ {\isacharparenleft}{\isacharbraceleft}Gi{\isacharbraceright}\ {\isasymunion}\ Wo{\isacharprime}{\isacharparenright}{\isacharparenright}{\isachardoublequoteclose}\isanewline
\ \ \ \ \ \ \isacommand{using}\isamarkupfalse%
\ {\isacartoucheopen}finite\ Wo{\isacharprime}{\isacartoucheclose}\ {\isacartoucheopen}{\isasymnot}{\isacharparenleft}sat\ {\isacharparenleft}{\isacharbraceleft}Gi{\isacharbraceright}\ {\isasymunion}\ Wo{\isacharprime}{\isacharparenright}{\isacharparenright}{\isacartoucheclose}\ \isacommand{by}\isamarkupfalse%
\ {\isacharparenleft}rule\ conjI{\isacharparenright}\isanewline
\ \ \ \ \isacommand{thus}\isamarkupfalse%
\ {\isacharquery}thesis\isanewline
\ \ \ \ \ \ \isacommand{using}\isamarkupfalse%
\ {\isacartoucheopen}Wo{\isacharprime}\ {\isasymsubseteq}\ W{\isacartoucheclose}\ \isacommand{by}\isamarkupfalse%
\ {\isacharparenleft}rule\ subexI{\isacharparenright}\isanewline
\ \ \isacommand{qed}\isamarkupfalse%
\isanewline
\isacommand{qed}\isamarkupfalse%
%
\endisatagproof
{\isafoldproof}%
%
\isadelimproof
%
\endisadelimproof
%
\begin{isamarkuptext}%
Por otro lado, para demostrar la cuarta condición del lema \isa{{\isadigit{2}}{\isachardot}{\isadigit{0}}{\isachardot}{\isadigit{2}}} que demuestra que \isa{C} 
  verifica la propiedad de consistencia proposicional, precisaremos de un lema auxiliar que prueba 
  que dados \isa{W\ {\isasymin}\ C}, \isa{F} una fórmula de tipo \isa{{\isasymbeta}} y componentes \isa{{\isasymbeta}\isactrlsub {\isadigit{1}}} y \isa{{\isasymbeta}\isactrlsub {\isadigit{2}}} tal que \isa{F\ {\isasymin}\ W} y \isa{W\isactrlsub {\isadigit{0}}} un 
  subconjunto finito de \isa{W}, entonces se tiene que o bien \isa{{\isacharbraceleft}{\isasymbeta}\isactrlsub {\isadigit{1}}{\isacharcomma}F{\isacharbraceright}\ {\isasymunion}\ W\isactrlsub {\isadigit{0}}} es satisfacible o bien 
  \isa{{\isacharbraceleft}{\isasymbeta}\isactrlsub {\isadigit{2}}{\isacharcomma}F{\isacharbraceright}\ {\isasymunion}\ W\isactrlsub {\isadigit{0}}} es satisfacible. Vamos a probar que, en efecto, se tiene el resultado para cada tipo de fórmula \isa{{\isasymbeta}}.

  En primer lugar, probemos que dados \isa{W\ {\isasymin}\ C}, una fórmula \isa{F\ {\isacharequal}\ G\ {\isasymand}\ H} para ciertas fórmulas \isa{G} y 
  \isa{H} tal que \isa{F\ {\isasymin}\ W} y \isa{W\isactrlsub {\isadigit{0}}} un subconjunto finito de \isa{W}, entonces se tiene que o bien 
  \isa{{\isacharbraceleft}G{\isacharcomma}F{\isacharbraceright}\ {\isasymunion}\ W\isactrlsub {\isadigit{0}}} es satisfacible o bien \isa{{\isacharbraceleft}H{\isacharcomma}F{\isacharbraceright}\ {\isasymunion}\ W\isactrlsub {\isadigit{0}}} es satisfacible.%
\end{isamarkuptext}\isamarkuptrue%
\isacommand{lemma}\isamarkupfalse%
\ pcp{\isacharunderscore}colecComp{\isacharunderscore}DIS{\isacharunderscore}sat{\isadigit{1}}{\isacharcolon}\isanewline
\ \ \isakeyword{assumes}\ {\isachardoublequoteopen}W\ {\isasymin}\ colecComp{\isachardoublequoteclose}\isanewline
\ \ \ \ \ \ \ \ \ \ {\isachardoublequoteopen}F\ {\isacharequal}\ G\ \isactrlbold {\isasymor}\ H{\isachardoublequoteclose}\isanewline
\ \ \ \ \ \ \ \ \ \ {\isachardoublequoteopen}F\ {\isasymin}\ W{\isachardoublequoteclose}\isanewline
\ \ \ \ \ \ \ \ \ \ {\isachardoublequoteopen}finite\ Wo{\isachardoublequoteclose}\isanewline
\ \ \ \ \ \ \ \ \ \ {\isachardoublequoteopen}Wo\ {\isasymsubseteq}\ W{\isachardoublequoteclose}\isanewline
\ \ \ \ \ \ \ \ \isakeyword{shows}\ {\isachardoublequoteopen}sat\ {\isacharparenleft}{\isacharbraceleft}G{\isacharcomma}F{\isacharbraceright}\ {\isasymunion}\ Wo{\isacharparenright}\ {\isasymor}\ sat\ {\isacharparenleft}{\isacharbraceleft}H{\isacharcomma}F{\isacharbraceright}\ {\isasymunion}\ Wo{\isacharparenright}{\isachardoublequoteclose}\isanewline
%
\isadelimproof
%
\endisadelimproof
%
\isatagproof
\isacommand{proof}\isamarkupfalse%
\ {\isacharminus}\isanewline
\ \ \isacommand{have}\isamarkupfalse%
\ {\isachardoublequoteopen}sat\ {\isacharparenleft}{\isacharbraceleft}F{\isacharbraceright}\ {\isasymunion}\ Wo{\isacharparenright}{\isachardoublequoteclose}\isanewline
\ \ \ \ \isacommand{using}\isamarkupfalse%
\ assms{\isacharparenleft}{\isadigit{1}}{\isacharcomma}{\isadigit{3}}{\isacharcomma}{\isadigit{4}}{\isacharcomma}{\isadigit{5}}{\isacharparenright}\ \isacommand{by}\isamarkupfalse%
\ {\isacharparenleft}rule\ pcp{\isacharunderscore}colecComp{\isacharunderscore}elem{\isacharunderscore}sat{\isacharparenright}\isanewline
\ \ \isacommand{have}\isamarkupfalse%
\ {\isachardoublequoteopen}F\ {\isasymin}\ {\isacharbraceleft}F{\isacharbraceright}\ {\isasymunion}\ Wo{\isachardoublequoteclose}\isanewline
\ \ \ \ \isacommand{by}\isamarkupfalse%
\ simp\ \isanewline
\ \ \isacommand{have}\isamarkupfalse%
\ Ex{\isadigit{1}}{\isacharcolon}{\isachardoublequoteopen}{\isasymexists}{\isasymA}{\isachardot}\ {\isasymforall}F\ {\isasymin}\ {\isacharparenleft}{\isacharbraceleft}F{\isacharbraceright}\ {\isasymunion}\ Wo{\isacharparenright}{\isachardot}\ {\isasymA}\ {\isasymTurnstile}\ F{\isachardoublequoteclose}\isanewline
\ \ \ \ \isacommand{using}\isamarkupfalse%
\ {\isacartoucheopen}sat\ {\isacharparenleft}{\isacharbraceleft}F{\isacharbraceright}\ {\isasymunion}\ Wo{\isacharparenright}{\isacartoucheclose}\ \isacommand{by}\isamarkupfalse%
\ {\isacharparenleft}simp\ only{\isacharcolon}\ sat{\isacharunderscore}def{\isacharparenright}\isanewline
\ \ \isacommand{obtain}\isamarkupfalse%
\ {\isasymA}\ \isakeyword{where}\ Forall{\isadigit{1}}{\isacharcolon}{\isachardoublequoteopen}{\isasymforall}F\ {\isasymin}\ {\isacharparenleft}{\isacharbraceleft}F{\isacharbraceright}\ {\isasymunion}\ Wo{\isacharparenright}{\isachardot}\ {\isasymA}\ {\isasymTurnstile}\ F{\isachardoublequoteclose}\isanewline
\ \ \ \ \isacommand{using}\isamarkupfalse%
\ Ex{\isadigit{1}}\ \isacommand{by}\isamarkupfalse%
\ {\isacharparenleft}rule\ exE{\isacharparenright}\isanewline
\ \ \isacommand{have}\isamarkupfalse%
\ {\isachardoublequoteopen}{\isasymA}\ {\isasymTurnstile}\ F{\isachardoublequoteclose}\isanewline
\ \ \ \ \isacommand{using}\isamarkupfalse%
\ Forall{\isadigit{1}}\ {\isacartoucheopen}F\ {\isasymin}\ {\isacharbraceleft}F{\isacharbraceright}\ {\isasymunion}\ Wo{\isacartoucheclose}\ \isacommand{by}\isamarkupfalse%
\ {\isacharparenleft}rule\ bspec{\isacharparenright}\isanewline
\ \ \isacommand{then}\isamarkupfalse%
\ \isacommand{have}\isamarkupfalse%
\ {\isachardoublequoteopen}{\isasymA}\ {\isasymTurnstile}\ {\isacharparenleft}G\ \isactrlbold {\isasymor}\ H{\isacharparenright}{\isachardoublequoteclose}\isanewline
\ \ \ \ \isacommand{using}\isamarkupfalse%
\ assms{\isacharparenleft}{\isadigit{2}}{\isacharparenright}\ \isacommand{by}\isamarkupfalse%
\ {\isacharparenleft}simp\ only{\isacharcolon}\ {\isacartoucheopen}{\isasymA}\ {\isasymTurnstile}\ F{\isacartoucheclose}{\isacharparenright}\isanewline
\ \ \isacommand{then}\isamarkupfalse%
\ \isacommand{have}\isamarkupfalse%
\ {\isachardoublequoteopen}{\isasymA}\ {\isasymTurnstile}\ G\ {\isasymor}\ {\isasymA}\ {\isasymTurnstile}\ H{\isachardoublequoteclose}\isanewline
\ \ \ \ \isacommand{by}\isamarkupfalse%
\ {\isacharparenleft}simp\ only{\isacharcolon}\ formula{\isacharunderscore}semantics{\isachardot}simps{\isacharparenleft}{\isadigit{5}}{\isacharparenright}{\isacharparenright}\isanewline
\ \ \isacommand{thus}\isamarkupfalse%
\ {\isacharquery}thesis\isanewline
\ \ \isacommand{proof}\isamarkupfalse%
\ {\isacharparenleft}rule\ disjE{\isacharparenright}\isanewline
\ \ \ \ \isacommand{assume}\isamarkupfalse%
\ {\isachardoublequoteopen}{\isasymA}\ {\isasymTurnstile}\ G{\isachardoublequoteclose}\isanewline
\ \ \ \ \isacommand{then}\isamarkupfalse%
\ \isacommand{have}\isamarkupfalse%
\ {\isachardoublequoteopen}{\isasymforall}F\ {\isasymin}\ {\isacharbraceleft}G{\isacharbraceright}{\isachardot}\ {\isasymA}\ {\isasymTurnstile}\ F{\isachardoublequoteclose}\isanewline
\ \ \ \ \ \ \isacommand{by}\isamarkupfalse%
\ simp\isanewline
\ \ \ \ \isacommand{then}\isamarkupfalse%
\ \isacommand{have}\isamarkupfalse%
\ {\isachardoublequoteopen}{\isasymforall}F\ {\isasymin}\ {\isacharparenleft}{\isacharbraceleft}G{\isacharbraceright}\ {\isasymunion}\ {\isacharparenleft}{\isacharbraceleft}F{\isacharbraceright}\ {\isasymunion}\ Wo{\isacharparenright}{\isacharparenright}{\isachardot}\ {\isasymA}\ {\isasymTurnstile}\ F{\isachardoublequoteclose}\isanewline
\ \ \ \ \ \ \isacommand{using}\isamarkupfalse%
\ Forall{\isadigit{1}}\ \isacommand{by}\isamarkupfalse%
\ {\isacharparenleft}rule\ ball{\isacharunderscore}Un{\isacharparenright}\isanewline
\ \ \ \ \isacommand{then}\isamarkupfalse%
\ \isacommand{have}\isamarkupfalse%
\ {\isachardoublequoteopen}{\isasymforall}F\ {\isasymin}\ {\isacharbraceleft}G{\isacharcomma}F{\isacharbraceright}\ {\isasymunion}\ Wo{\isachardot}\ {\isasymA}\ {\isasymTurnstile}\ F{\isachardoublequoteclose}\isanewline
\ \ \ \ \ \ \isacommand{by}\isamarkupfalse%
\ simp\ \isanewline
\ \ \ \ \isacommand{then}\isamarkupfalse%
\ \isacommand{have}\isamarkupfalse%
\ {\isachardoublequoteopen}{\isasymexists}{\isasymA}{\isachardot}\ {\isasymforall}F\ {\isasymin}\ {\isacharparenleft}{\isacharbraceleft}G{\isacharcomma}F{\isacharbraceright}\ {\isasymunion}\ Wo{\isacharparenright}{\isachardot}\ {\isasymA}\ {\isasymTurnstile}\ F{\isachardoublequoteclose}\isanewline
\ \ \ \ \ \ \isacommand{by}\isamarkupfalse%
\ {\isacharparenleft}iprover\ intro{\isacharcolon}\ exI{\isacharparenright}\isanewline
\ \ \ \ \isacommand{then}\isamarkupfalse%
\ \isacommand{have}\isamarkupfalse%
\ {\isachardoublequoteopen}sat\ {\isacharparenleft}{\isacharbraceleft}G{\isacharcomma}F{\isacharbraceright}\ {\isasymunion}\ Wo{\isacharparenright}{\isachardoublequoteclose}\isanewline
\ \ \ \ \ \ \isacommand{by}\isamarkupfalse%
\ {\isacharparenleft}simp\ only{\isacharcolon}\ sat{\isacharunderscore}def{\isacharparenright}\isanewline
\ \ \ \ \isacommand{thus}\isamarkupfalse%
\ {\isacharquery}thesis\isanewline
\ \ \ \ \ \ \isacommand{by}\isamarkupfalse%
\ {\isacharparenleft}rule\ disjI{\isadigit{1}}{\isacharparenright}\isanewline
\ \ \isacommand{next}\isamarkupfalse%
\isanewline
\ \ \ \ \isacommand{assume}\isamarkupfalse%
\ {\isachardoublequoteopen}{\isasymA}\ {\isasymTurnstile}\ H{\isachardoublequoteclose}\isanewline
\ \ \ \ \isacommand{then}\isamarkupfalse%
\ \isacommand{have}\isamarkupfalse%
\ {\isachardoublequoteopen}{\isasymforall}F\ {\isasymin}\ {\isacharbraceleft}H{\isacharbraceright}{\isachardot}\ {\isasymA}\ {\isasymTurnstile}\ F{\isachardoublequoteclose}\isanewline
\ \ \ \ \ \ \isacommand{by}\isamarkupfalse%
\ simp\isanewline
\ \ \ \ \isacommand{then}\isamarkupfalse%
\ \isacommand{have}\isamarkupfalse%
\ {\isachardoublequoteopen}{\isasymforall}F\ {\isasymin}\ {\isacharparenleft}{\isacharbraceleft}H{\isacharbraceright}\ {\isasymunion}\ {\isacharparenleft}{\isacharbraceleft}F{\isacharbraceright}\ {\isasymunion}\ Wo{\isacharparenright}{\isacharparenright}{\isachardot}\ {\isasymA}\ {\isasymTurnstile}\ F{\isachardoublequoteclose}\isanewline
\ \ \ \ \ \ \isacommand{using}\isamarkupfalse%
\ Forall{\isadigit{1}}\ \isacommand{by}\isamarkupfalse%
\ {\isacharparenleft}rule\ ball{\isacharunderscore}Un{\isacharparenright}\isanewline
\ \ \ \ \isacommand{then}\isamarkupfalse%
\ \isacommand{have}\isamarkupfalse%
\ {\isachardoublequoteopen}{\isasymforall}F\ {\isasymin}\ {\isacharbraceleft}H{\isacharcomma}F{\isacharbraceright}\ {\isasymunion}\ Wo{\isachardot}\ {\isasymA}\ {\isasymTurnstile}\ F{\isachardoublequoteclose}\isanewline
\ \ \ \ \ \ \isacommand{by}\isamarkupfalse%
\ simp\isanewline
\ \ \ \ \isacommand{then}\isamarkupfalse%
\ \isacommand{have}\isamarkupfalse%
\ {\isachardoublequoteopen}{\isasymexists}{\isasymA}{\isachardot}\ {\isasymforall}F\ {\isasymin}\ {\isacharparenleft}{\isacharbraceleft}H{\isacharcomma}F{\isacharbraceright}\ {\isasymunion}\ Wo{\isacharparenright}{\isachardot}\ {\isasymA}\ {\isasymTurnstile}\ F{\isachardoublequoteclose}\isanewline
\ \ \ \ \ \ \isacommand{by}\isamarkupfalse%
\ {\isacharparenleft}iprover\ intro{\isacharcolon}\ exI{\isacharparenright}\isanewline
\ \ \ \ \isacommand{then}\isamarkupfalse%
\ \isacommand{have}\isamarkupfalse%
\ {\isachardoublequoteopen}sat\ {\isacharparenleft}{\isacharbraceleft}H{\isacharcomma}F{\isacharbraceright}\ {\isasymunion}\ Wo{\isacharparenright}{\isachardoublequoteclose}\isanewline
\ \ \ \ \ \ \isacommand{by}\isamarkupfalse%
\ {\isacharparenleft}simp\ only{\isacharcolon}\ sat{\isacharunderscore}def{\isacharparenright}\isanewline
\ \ \ \ \isacommand{thus}\isamarkupfalse%
\ {\isacharquery}thesis\isanewline
\ \ \ \ \ \ \isacommand{by}\isamarkupfalse%
\ {\isacharparenleft}rule\ disjI{\isadigit{2}}{\isacharparenright}\isanewline
\ \ \isacommand{qed}\isamarkupfalse%
\isanewline
\isacommand{qed}\isamarkupfalse%
%
\endisatagproof
{\isafoldproof}%
%
\isadelimproof
%
\endisadelimproof
%
\begin{isamarkuptext}%
El siguiente lema auxiliar demuestra que dados \isa{W\ {\isasymin}\ C}, una fórmula \isa{F\ {\isacharequal}\ G\ {\isasymlongrightarrow}\ H} para ciertas 
  fórmulas \isa{G} y \isa{H} tal que \isa{F\ {\isasymin}\ W} y \isa{W\isactrlsub {\isadigit{0}}} un subconjunto finito de \isa{W}, entonces se tiene que o 
  bien \isa{{\isacharbraceleft}{\isasymnot}\ G{\isacharcomma}F{\isacharbraceright}\ {\isasymunion}\ W\isactrlsub {\isadigit{0}}} es satisfacible o bien \isa{{\isacharbraceleft}H{\isacharcomma}F{\isacharbraceright}\ {\isasymunion}\ W\isactrlsub {\isadigit{0}}} es satisfacible.%
\end{isamarkuptext}\isamarkuptrue%
\isacommand{lemma}\isamarkupfalse%
\ pcp{\isacharunderscore}colecComp{\isacharunderscore}DIS{\isacharunderscore}sat{\isadigit{2}}{\isacharcolon}\isanewline
\ \ \isakeyword{assumes}\ {\isachardoublequoteopen}W\ {\isasymin}\ colecComp{\isachardoublequoteclose}\isanewline
\ \ \ \ \ \ \ \ \ \ {\isachardoublequoteopen}F\ {\isacharequal}\ G\ \isactrlbold {\isasymrightarrow}\ H{\isachardoublequoteclose}\isanewline
\ \ \ \ \ \ \ \ \ \ {\isachardoublequoteopen}F\ {\isasymin}\ W{\isachardoublequoteclose}\isanewline
\ \ \ \ \ \ \ \ \ \ {\isachardoublequoteopen}finite\ Wo{\isachardoublequoteclose}\isanewline
\ \ \ \ \ \ \ \ \ \ {\isachardoublequoteopen}Wo\ {\isasymsubseteq}\ W{\isachardoublequoteclose}\isanewline
\ \ \ \ \ \ \ \ \isakeyword{shows}\ {\isachardoublequoteopen}sat\ {\isacharparenleft}{\isacharbraceleft}\isactrlbold {\isasymnot}\ G{\isacharcomma}F{\isacharbraceright}\ {\isasymunion}\ Wo{\isacharparenright}\ {\isasymor}\ sat\ {\isacharparenleft}{\isacharbraceleft}H{\isacharcomma}F{\isacharbraceright}\ {\isasymunion}\ Wo{\isacharparenright}{\isachardoublequoteclose}\isanewline
%
\isadelimproof
%
\endisadelimproof
%
\isatagproof
\isacommand{proof}\isamarkupfalse%
\ {\isacharminus}\isanewline
\ \ \isacommand{have}\isamarkupfalse%
\ {\isachardoublequoteopen}sat\ {\isacharparenleft}{\isacharbraceleft}F{\isacharbraceright}\ {\isasymunion}\ Wo{\isacharparenright}{\isachardoublequoteclose}\isanewline
\ \ \ \ \isacommand{using}\isamarkupfalse%
\ assms{\isacharparenleft}{\isadigit{1}}{\isacharcomma}{\isadigit{3}}{\isacharcomma}{\isadigit{4}}{\isacharcomma}{\isadigit{5}}{\isacharparenright}\ \isacommand{by}\isamarkupfalse%
\ {\isacharparenleft}rule\ pcp{\isacharunderscore}colecComp{\isacharunderscore}elem{\isacharunderscore}sat{\isacharparenright}\isanewline
\ \ \isacommand{have}\isamarkupfalse%
\ {\isachardoublequoteopen}F\ {\isasymin}\ {\isacharbraceleft}F{\isacharbraceright}\ {\isasymunion}\ Wo{\isachardoublequoteclose}\isanewline
\ \ \ \ \isacommand{by}\isamarkupfalse%
\ simp\isanewline
\ \ \isacommand{have}\isamarkupfalse%
\ Ex{\isadigit{1}}{\isacharcolon}{\isachardoublequoteopen}{\isasymexists}{\isasymA}{\isachardot}\ {\isasymforall}F\ {\isasymin}\ {\isacharparenleft}{\isacharbraceleft}F{\isacharbraceright}\ {\isasymunion}\ Wo{\isacharparenright}{\isachardot}\ {\isasymA}\ {\isasymTurnstile}\ F{\isachardoublequoteclose}\isanewline
\ \ \ \ \isacommand{using}\isamarkupfalse%
\ {\isacartoucheopen}sat\ {\isacharparenleft}{\isacharbraceleft}F{\isacharbraceright}\ {\isasymunion}\ Wo{\isacharparenright}{\isacartoucheclose}\ \isacommand{by}\isamarkupfalse%
\ {\isacharparenleft}simp\ only{\isacharcolon}\ sat{\isacharunderscore}def{\isacharparenright}\isanewline
\ \ \isacommand{obtain}\isamarkupfalse%
\ {\isasymA}\ \isakeyword{where}\ Forall{\isadigit{1}}{\isacharcolon}{\isachardoublequoteopen}{\isasymforall}F\ {\isasymin}\ {\isacharparenleft}{\isacharbraceleft}F{\isacharbraceright}\ {\isasymunion}\ Wo{\isacharparenright}{\isachardot}\ {\isasymA}\ {\isasymTurnstile}\ F{\isachardoublequoteclose}\isanewline
\ \ \ \ \isacommand{using}\isamarkupfalse%
\ Ex{\isadigit{1}}\ \isacommand{by}\isamarkupfalse%
\ {\isacharparenleft}rule\ exE{\isacharparenright}\isanewline
\ \ \isacommand{have}\isamarkupfalse%
\ {\isachardoublequoteopen}{\isasymA}\ {\isasymTurnstile}\ F{\isachardoublequoteclose}\isanewline
\ \ \ \ \isacommand{using}\isamarkupfalse%
\ Forall{\isadigit{1}}\ {\isacartoucheopen}F\ {\isasymin}\ {\isacharbraceleft}F{\isacharbraceright}\ {\isasymunion}\ Wo{\isacartoucheclose}\ \isacommand{by}\isamarkupfalse%
\ {\isacharparenleft}rule\ bspec{\isacharparenright}\isanewline
\ \ \isacommand{then}\isamarkupfalse%
\ \isacommand{have}\isamarkupfalse%
\ {\isachardoublequoteopen}{\isasymA}\ {\isasymTurnstile}\ {\isacharparenleft}G\ \isactrlbold {\isasymrightarrow}\ H{\isacharparenright}{\isachardoublequoteclose}\isanewline
\ \ \ \ \isacommand{using}\isamarkupfalse%
\ assms{\isacharparenleft}{\isadigit{2}}{\isacharparenright}\ \isacommand{by}\isamarkupfalse%
\ {\isacharparenleft}simp\ only{\isacharcolon}\ {\isacartoucheopen}{\isasymA}\ {\isasymTurnstile}\ F{\isacartoucheclose}{\isacharparenright}\isanewline
\ \ \isacommand{then}\isamarkupfalse%
\ \isacommand{have}\isamarkupfalse%
\ {\isachardoublequoteopen}{\isasymA}\ {\isasymTurnstile}\ G\ {\isasymlongrightarrow}\ {\isasymA}\ {\isasymTurnstile}\ H{\isachardoublequoteclose}\isanewline
\ \ \ \ \isacommand{by}\isamarkupfalse%
\ {\isacharparenleft}simp\ only{\isacharcolon}\ formula{\isacharunderscore}semantics{\isachardot}simps{\isacharparenleft}{\isadigit{6}}{\isacharparenright}{\isacharparenright}\isanewline
\ \ \isacommand{then}\isamarkupfalse%
\ \isacommand{have}\isamarkupfalse%
\ {\isachardoublequoteopen}{\isacharparenleft}{\isasymnot}{\isacharparenleft}{\isasymnot}\ {\isasymA}\ {\isasymTurnstile}\ G{\isacharparenright}{\isacharparenright}\ {\isasymlongrightarrow}\ {\isasymA}\ {\isasymTurnstile}\ H{\isachardoublequoteclose}\isanewline
\ \ \ \ \isacommand{by}\isamarkupfalse%
\ {\isacharparenleft}simp\ only{\isacharcolon}\ not{\isacharunderscore}not{\isacharparenright}\isanewline
\ \ \isacommand{then}\isamarkupfalse%
\ \isacommand{have}\isamarkupfalse%
\ {\isachardoublequoteopen}{\isacharparenleft}{\isasymnot}\ {\isasymA}\ {\isasymTurnstile}\ G{\isacharparenright}\ {\isasymor}\ {\isasymA}\ {\isasymTurnstile}\ H{\isachardoublequoteclose}\isanewline
\ \ \ \ \isacommand{by}\isamarkupfalse%
\ {\isacharparenleft}simp\ only{\isacharcolon}\ disj{\isacharunderscore}imp{\isacharparenright}\isanewline
\ \ \isacommand{thus}\isamarkupfalse%
\ {\isacharquery}thesis\isanewline
\ \ \isacommand{proof}\isamarkupfalse%
\ {\isacharparenleft}rule\ disjE{\isacharparenright}\isanewline
\ \ \ \ \isacommand{assume}\isamarkupfalse%
\ {\isachardoublequoteopen}{\isasymnot}\ {\isasymA}\ {\isasymTurnstile}\ G{\isachardoublequoteclose}\isanewline
\ \ \ \ \isacommand{then}\isamarkupfalse%
\ \isacommand{have}\isamarkupfalse%
\ {\isachardoublequoteopen}{\isasymA}\ {\isasymTurnstile}\ {\isacharparenleft}\isactrlbold {\isasymnot}\ G{\isacharparenright}{\isachardoublequoteclose}\isanewline
\ \ \ \ \ \ \isacommand{by}\isamarkupfalse%
\ {\isacharparenleft}simp\ only{\isacharcolon}\ formula{\isacharunderscore}semantics{\isachardot}simps{\isacharparenleft}{\isadigit{3}}{\isacharparenright}\ simp{\isacharunderscore}thms{\isacharparenleft}{\isadigit{8}}{\isacharparenright}{\isacharparenright}\isanewline
\ \ \ \ \isacommand{then}\isamarkupfalse%
\ \isacommand{have}\isamarkupfalse%
\ {\isachardoublequoteopen}{\isasymforall}F\ {\isasymin}\ {\isacharbraceleft}\isactrlbold {\isasymnot}\ G{\isacharbraceright}{\isachardot}\ {\isasymA}\ {\isasymTurnstile}\ F{\isachardoublequoteclose}\isanewline
\ \ \ \ \ \ \isacommand{by}\isamarkupfalse%
\ simp\isanewline
\ \ \ \ \isacommand{then}\isamarkupfalse%
\ \isacommand{have}\isamarkupfalse%
\ {\isachardoublequoteopen}{\isasymforall}F\ {\isasymin}\ {\isacharparenleft}{\isacharbraceleft}\isactrlbold {\isasymnot}\ G{\isacharbraceright}\ {\isasymunion}\ {\isacharparenleft}{\isacharbraceleft}F{\isacharbraceright}\ {\isasymunion}\ Wo{\isacharparenright}{\isacharparenright}{\isachardot}\ {\isasymA}\ {\isasymTurnstile}\ F{\isachardoublequoteclose}\isanewline
\ \ \ \ \ \ \isacommand{using}\isamarkupfalse%
\ Forall{\isadigit{1}}\ \isacommand{by}\isamarkupfalse%
\ {\isacharparenleft}rule\ ball{\isacharunderscore}Un{\isacharparenright}\isanewline
\ \ \ \ \isacommand{then}\isamarkupfalse%
\ \isacommand{have}\isamarkupfalse%
\ {\isachardoublequoteopen}{\isasymforall}F\ {\isasymin}\ {\isacharbraceleft}\isactrlbold {\isasymnot}\ G{\isacharcomma}F{\isacharbraceright}\ {\isasymunion}\ Wo{\isachardot}\ {\isasymA}\ {\isasymTurnstile}\ F{\isachardoublequoteclose}\isanewline
\ \ \ \ \ \ \isacommand{by}\isamarkupfalse%
\ simp\isanewline
\ \ \ \ \isacommand{then}\isamarkupfalse%
\ \isacommand{have}\isamarkupfalse%
\ {\isachardoublequoteopen}{\isasymexists}{\isasymA}{\isachardot}\ {\isasymforall}F\ {\isasymin}\ {\isacharparenleft}{\isacharbraceleft}\isactrlbold {\isasymnot}\ G{\isacharcomma}F{\isacharbraceright}\ {\isasymunion}\ Wo{\isacharparenright}{\isachardot}\ {\isasymA}\ {\isasymTurnstile}\ F{\isachardoublequoteclose}\isanewline
\ \ \ \ \ \ \isacommand{by}\isamarkupfalse%
\ {\isacharparenleft}iprover\ intro{\isacharcolon}\ exI{\isacharparenright}\isanewline
\ \ \ \ \isacommand{then}\isamarkupfalse%
\ \isacommand{have}\isamarkupfalse%
\ {\isachardoublequoteopen}sat\ {\isacharparenleft}{\isacharbraceleft}\isactrlbold {\isasymnot}\ G{\isacharcomma}F{\isacharbraceright}\ {\isasymunion}\ Wo{\isacharparenright}{\isachardoublequoteclose}\isanewline
\ \ \ \ \ \ \isacommand{by}\isamarkupfalse%
\ {\isacharparenleft}simp\ only{\isacharcolon}\ sat{\isacharunderscore}def{\isacharparenright}\isanewline
\ \ \ \ \isacommand{thus}\isamarkupfalse%
\ {\isacharquery}thesis\isanewline
\ \ \ \ \ \ \isacommand{by}\isamarkupfalse%
\ {\isacharparenleft}rule\ disjI{\isadigit{1}}{\isacharparenright}\isanewline
\ \ \isacommand{next}\isamarkupfalse%
\isanewline
\ \ \ \ \isacommand{assume}\isamarkupfalse%
\ {\isachardoublequoteopen}{\isasymA}\ {\isasymTurnstile}\ H{\isachardoublequoteclose}\isanewline
\ \ \ \ \isacommand{then}\isamarkupfalse%
\ \isacommand{have}\isamarkupfalse%
\ {\isachardoublequoteopen}{\isasymforall}F\ {\isasymin}\ {\isacharbraceleft}H{\isacharbraceright}{\isachardot}\ {\isasymA}\ {\isasymTurnstile}\ F{\isachardoublequoteclose}\isanewline
\ \ \ \ \ \ \isacommand{by}\isamarkupfalse%
\ simp\isanewline
\ \ \ \ \isacommand{then}\isamarkupfalse%
\ \isacommand{have}\isamarkupfalse%
\ {\isachardoublequoteopen}{\isasymforall}F\ {\isasymin}\ {\isacharparenleft}{\isacharbraceleft}H{\isacharbraceright}\ {\isasymunion}\ {\isacharparenleft}{\isacharbraceleft}F{\isacharbraceright}\ {\isasymunion}\ Wo{\isacharparenright}{\isacharparenright}{\isachardot}\ {\isasymA}\ {\isasymTurnstile}\ F{\isachardoublequoteclose}\isanewline
\ \ \ \ \ \ \isacommand{using}\isamarkupfalse%
\ Forall{\isadigit{1}}\ \isacommand{by}\isamarkupfalse%
\ {\isacharparenleft}rule\ ball{\isacharunderscore}Un{\isacharparenright}\isanewline
\ \ \ \ \isacommand{then}\isamarkupfalse%
\ \isacommand{have}\isamarkupfalse%
\ {\isachardoublequoteopen}{\isasymforall}F\ {\isasymin}\ {\isacharbraceleft}H{\isacharcomma}F{\isacharbraceright}\ {\isasymunion}\ Wo{\isachardot}\ {\isasymA}\ {\isasymTurnstile}\ F{\isachardoublequoteclose}\isanewline
\ \ \ \ \ \ \isacommand{by}\isamarkupfalse%
\ simp\isanewline
\ \ \ \ \isacommand{then}\isamarkupfalse%
\ \isacommand{have}\isamarkupfalse%
\ {\isachardoublequoteopen}{\isasymexists}{\isasymA}{\isachardot}\ {\isasymforall}F\ {\isasymin}\ {\isacharparenleft}{\isacharbraceleft}H{\isacharcomma}F{\isacharbraceright}\ {\isasymunion}\ Wo{\isacharparenright}{\isachardot}\ {\isasymA}\ {\isasymTurnstile}\ F{\isachardoublequoteclose}\isanewline
\ \ \ \ \ \ \isacommand{by}\isamarkupfalse%
\ {\isacharparenleft}iprover\ intro{\isacharcolon}\ exI{\isacharparenright}\isanewline
\ \ \ \ \isacommand{then}\isamarkupfalse%
\ \isacommand{have}\isamarkupfalse%
\ {\isachardoublequoteopen}sat\ {\isacharparenleft}{\isacharbraceleft}H{\isacharcomma}F{\isacharbraceright}\ {\isasymunion}\ Wo{\isacharparenright}{\isachardoublequoteclose}\isanewline
\ \ \ \ \ \ \isacommand{by}\isamarkupfalse%
\ {\isacharparenleft}simp\ only{\isacharcolon}\ sat{\isacharunderscore}def{\isacharparenright}\isanewline
\ \ \ \ \isacommand{thus}\isamarkupfalse%
\ {\isacharquery}thesis\isanewline
\ \ \ \ \ \ \isacommand{by}\isamarkupfalse%
\ {\isacharparenleft}rule\ disjI{\isadigit{2}}{\isacharparenright}\isanewline
\ \ \isacommand{qed}\isamarkupfalse%
\isanewline
\isacommand{qed}\isamarkupfalse%
%
\endisatagproof
{\isafoldproof}%
%
\isadelimproof
%
\endisadelimproof
%
\begin{isamarkuptext}%
Por otro lado probemos que dados \isa{W\ {\isasymin}\ C}, una fórmula \isa{F\ {\isacharequal}\ {\isasymnot}{\isacharparenleft}G\ {\isasymand}\ H{\isacharparenright}} para ciertas fórmulas 
  \isa{G} y \isa{H} tal que \isa{F\ {\isasymin}\ W} y \isa{W\isactrlsub {\isadigit{0}}} un subconjunto finito de \isa{W}, entonces se tiene que o bien 
  \isa{{\isacharbraceleft}{\isasymnot}\ G{\isacharcomma}F{\isacharbraceright}\ {\isasymunion}\ W\isactrlsub {\isadigit{0}}} es satisfacible o bien \isa{{\isacharbraceleft}{\isasymnot}\ H{\isacharcomma}F{\isacharbraceright}\ {\isasymunion}\ W\isactrlsub {\isadigit{0}}} es satisfacible.%
\end{isamarkuptext}\isamarkuptrue%
\isacommand{lemma}\isamarkupfalse%
\ pcp{\isacharunderscore}colecComp{\isacharunderscore}DIS{\isacharunderscore}sat{\isadigit{3}}{\isacharcolon}\isanewline
\ \ \isakeyword{assumes}\ {\isachardoublequoteopen}W\ {\isasymin}\ colecComp{\isachardoublequoteclose}\isanewline
\ \ \ \ \ \ \ \ \ \ {\isachardoublequoteopen}F\ {\isacharequal}\ \isactrlbold {\isasymnot}\ {\isacharparenleft}G\ \isactrlbold {\isasymand}\ H{\isacharparenright}{\isachardoublequoteclose}\isanewline
\ \ \ \ \ \ \ \ \ \ {\isachardoublequoteopen}F\ {\isasymin}\ W{\isachardoublequoteclose}\isanewline
\ \ \ \ \ \ \ \ \ \ {\isachardoublequoteopen}finite\ Wo{\isachardoublequoteclose}\isanewline
\ \ \ \ \ \ \ \ \ \ {\isachardoublequoteopen}Wo\ {\isasymsubseteq}\ W{\isachardoublequoteclose}\isanewline
\ \ \ \ \ \ \ \ \isakeyword{shows}\ {\isachardoublequoteopen}sat\ {\isacharparenleft}{\isacharbraceleft}\isactrlbold {\isasymnot}\ G{\isacharcomma}F{\isacharbraceright}\ {\isasymunion}\ Wo{\isacharparenright}\ {\isasymor}\ sat\ {\isacharparenleft}{\isacharbraceleft}\isactrlbold {\isasymnot}\ H{\isacharcomma}F{\isacharbraceright}\ {\isasymunion}\ Wo{\isacharparenright}{\isachardoublequoteclose}\isanewline
%
\isadelimproof
%
\endisadelimproof
%
\isatagproof
\isacommand{proof}\isamarkupfalse%
\ {\isacharminus}\isanewline
\ \ \isacommand{have}\isamarkupfalse%
\ {\isachardoublequoteopen}sat\ {\isacharparenleft}{\isacharbraceleft}F{\isacharbraceright}\ {\isasymunion}\ Wo{\isacharparenright}{\isachardoublequoteclose}\isanewline
\ \ \ \ \isacommand{using}\isamarkupfalse%
\ assms{\isacharparenleft}{\isadigit{1}}{\isacharcomma}{\isadigit{3}}{\isacharcomma}{\isadigit{4}}{\isacharcomma}{\isadigit{5}}{\isacharparenright}\ \isacommand{by}\isamarkupfalse%
\ {\isacharparenleft}rule\ pcp{\isacharunderscore}colecComp{\isacharunderscore}elem{\isacharunderscore}sat{\isacharparenright}\isanewline
\ \ \isacommand{have}\isamarkupfalse%
\ {\isachardoublequoteopen}F\ {\isasymin}\ {\isacharbraceleft}F{\isacharbraceright}\ {\isasymunion}\ Wo{\isachardoublequoteclose}\isanewline
\ \ \ \ \isacommand{by}\isamarkupfalse%
\ simp\isanewline
\ \ \isacommand{have}\isamarkupfalse%
\ Ex{\isadigit{1}}{\isacharcolon}{\isachardoublequoteopen}{\isasymexists}{\isasymA}{\isachardot}\ {\isasymforall}F\ {\isasymin}\ {\isacharparenleft}{\isacharbraceleft}F{\isacharbraceright}\ {\isasymunion}\ Wo{\isacharparenright}{\isachardot}\ {\isasymA}\ {\isasymTurnstile}\ F{\isachardoublequoteclose}\isanewline
\ \ \ \ \isacommand{using}\isamarkupfalse%
\ {\isacartoucheopen}sat\ {\isacharparenleft}{\isacharbraceleft}F{\isacharbraceright}\ {\isasymunion}\ Wo{\isacharparenright}{\isacartoucheclose}\ \isacommand{by}\isamarkupfalse%
\ {\isacharparenleft}simp\ only{\isacharcolon}\ sat{\isacharunderscore}def{\isacharparenright}\isanewline
\ \ \isacommand{obtain}\isamarkupfalse%
\ {\isasymA}\ \isakeyword{where}\ Forall{\isadigit{1}}{\isacharcolon}{\isachardoublequoteopen}{\isasymforall}F\ {\isasymin}\ {\isacharparenleft}{\isacharbraceleft}F{\isacharbraceright}\ {\isasymunion}\ Wo{\isacharparenright}{\isachardot}\ {\isasymA}\ {\isasymTurnstile}\ F{\isachardoublequoteclose}\isanewline
\ \ \ \ \isacommand{using}\isamarkupfalse%
\ Ex{\isadigit{1}}\ \isacommand{by}\isamarkupfalse%
\ {\isacharparenleft}rule\ exE{\isacharparenright}\isanewline
\ \ \isacommand{have}\isamarkupfalse%
\ {\isachardoublequoteopen}{\isasymA}\ {\isasymTurnstile}\ F{\isachardoublequoteclose}\isanewline
\ \ \ \ \isacommand{using}\isamarkupfalse%
\ Forall{\isadigit{1}}\ {\isacartoucheopen}F\ {\isasymin}\ {\isacharbraceleft}F{\isacharbraceright}\ {\isasymunion}\ Wo{\isacartoucheclose}\ \isacommand{by}\isamarkupfalse%
\ {\isacharparenleft}rule\ bspec{\isacharparenright}\isanewline
\ \ \isacommand{then}\isamarkupfalse%
\ \isacommand{have}\isamarkupfalse%
\ {\isachardoublequoteopen}{\isasymA}\ {\isasymTurnstile}\ \isactrlbold {\isasymnot}\ {\isacharparenleft}G\ \isactrlbold {\isasymand}\ H{\isacharparenright}{\isachardoublequoteclose}\isanewline
\ \ \ \ \isacommand{using}\isamarkupfalse%
\ assms{\isacharparenleft}{\isadigit{2}}{\isacharparenright}\ \isacommand{by}\isamarkupfalse%
\ {\isacharparenleft}simp\ only{\isacharcolon}\ {\isacartoucheopen}{\isasymA}\ {\isasymTurnstile}\ F{\isacartoucheclose}{\isacharparenright}\isanewline
\ \ \isacommand{then}\isamarkupfalse%
\ \isacommand{have}\isamarkupfalse%
\ {\isachardoublequoteopen}{\isasymnot}\ {\isacharparenleft}{\isasymA}\ {\isasymTurnstile}\ {\isacharparenleft}G\ \isactrlbold {\isasymand}\ H{\isacharparenright}{\isacharparenright}{\isachardoublequoteclose}\isanewline
\ \ \ \ \isacommand{by}\isamarkupfalse%
\ {\isacharparenleft}simp\ only{\isacharcolon}\ formula{\isacharunderscore}semantics{\isachardot}simps{\isacharparenleft}{\isadigit{3}}{\isacharparenright}\ simp{\isacharunderscore}thms{\isacharparenleft}{\isadigit{8}}{\isacharparenright}{\isacharparenright}\isanewline
\ \ \isacommand{then}\isamarkupfalse%
\ \isacommand{have}\isamarkupfalse%
\ {\isachardoublequoteopen}{\isasymnot}{\isacharparenleft}{\isasymA}\ {\isasymTurnstile}\ G\ {\isasymand}\ {\isasymA}\ {\isasymTurnstile}\ H{\isacharparenright}{\isachardoublequoteclose}\isanewline
\ \ \ \ \isacommand{by}\isamarkupfalse%
\ {\isacharparenleft}simp\ only{\isacharcolon}\ formula{\isacharunderscore}semantics{\isachardot}simps{\isacharparenleft}{\isadigit{4}}{\isacharparenright}\ simp{\isacharunderscore}thms{\isacharparenleft}{\isadigit{8}}{\isacharparenright}{\isacharparenright}\isanewline
\ \ \isacommand{then}\isamarkupfalse%
\ \isacommand{have}\isamarkupfalse%
\ {\isachardoublequoteopen}{\isasymnot}\ {\isacharparenleft}{\isasymA}\ {\isasymTurnstile}\ G{\isacharparenright}\ {\isasymor}\ {\isasymnot}\ {\isacharparenleft}{\isasymA}\ {\isasymTurnstile}\ H{\isacharparenright}{\isachardoublequoteclose}\isanewline
\ \ \ \ \isacommand{by}\isamarkupfalse%
\ {\isacharparenleft}simp\ only{\isacharcolon}\ de{\isacharunderscore}Morgan{\isacharunderscore}conj{\isacharparenright}\isanewline
\ \ \isacommand{thus}\isamarkupfalse%
\ {\isacharquery}thesis\isanewline
\ \ \isacommand{proof}\isamarkupfalse%
\ {\isacharparenleft}rule\ disjE{\isacharparenright}\isanewline
\ \ \ \ \isacommand{assume}\isamarkupfalse%
\ {\isachardoublequoteopen}{\isasymnot}\ {\isacharparenleft}{\isasymA}\ {\isasymTurnstile}\ G{\isacharparenright}{\isachardoublequoteclose}\isanewline
\ \ \ \ \isacommand{then}\isamarkupfalse%
\ \isacommand{have}\isamarkupfalse%
\ {\isachardoublequoteopen}{\isasymA}\ {\isasymTurnstile}\ \isactrlbold {\isasymnot}\ G{\isachardoublequoteclose}\isanewline
\ \ \ \ \ \ \isacommand{by}\isamarkupfalse%
\ {\isacharparenleft}simp\ only{\isacharcolon}\ formula{\isacharunderscore}semantics{\isachardot}simps{\isacharparenleft}{\isadigit{3}}{\isacharparenright}\ simp{\isacharunderscore}thms{\isacharparenleft}{\isadigit{8}}{\isacharparenright}{\isacharparenright}\isanewline
\ \ \ \ \isacommand{then}\isamarkupfalse%
\ \isacommand{have}\isamarkupfalse%
\ {\isachardoublequoteopen}{\isasymforall}F\ {\isasymin}\ {\isacharbraceleft}\isactrlbold {\isasymnot}\ G{\isacharbraceright}{\isachardot}\ {\isasymA}\ {\isasymTurnstile}\ F{\isachardoublequoteclose}\isanewline
\ \ \ \ \ \ \isacommand{by}\isamarkupfalse%
\ simp\isanewline
\ \ \ \ \isacommand{then}\isamarkupfalse%
\ \isacommand{have}\isamarkupfalse%
\ {\isachardoublequoteopen}{\isasymforall}F\ {\isasymin}\ {\isacharparenleft}{\isacharbraceleft}\isactrlbold {\isasymnot}\ G{\isacharbraceright}\ {\isasymunion}\ {\isacharparenleft}{\isacharbraceleft}F{\isacharbraceright}\ {\isasymunion}\ Wo{\isacharparenright}{\isacharparenright}{\isachardot}\ {\isasymA}\ {\isasymTurnstile}\ F{\isachardoublequoteclose}\isanewline
\ \ \ \ \ \ \isacommand{using}\isamarkupfalse%
\ Forall{\isadigit{1}}\ \isacommand{by}\isamarkupfalse%
\ {\isacharparenleft}rule\ ball{\isacharunderscore}Un{\isacharparenright}\isanewline
\ \ \ \ \isacommand{then}\isamarkupfalse%
\ \isacommand{have}\isamarkupfalse%
\ {\isachardoublequoteopen}{\isasymforall}F\ {\isasymin}\ {\isacharbraceleft}\isactrlbold {\isasymnot}\ G{\isacharcomma}F{\isacharbraceright}\ {\isasymunion}\ Wo{\isachardot}\ {\isasymA}\ {\isasymTurnstile}\ F{\isachardoublequoteclose}\isanewline
\ \ \ \ \ \ \isacommand{by}\isamarkupfalse%
\ simp\isanewline
\ \ \ \ \isacommand{then}\isamarkupfalse%
\ \isacommand{have}\isamarkupfalse%
\ {\isachardoublequoteopen}{\isasymexists}{\isasymA}{\isachardot}\ {\isasymforall}F\ {\isasymin}\ {\isacharparenleft}{\isacharbraceleft}\isactrlbold {\isasymnot}\ G{\isacharcomma}F{\isacharbraceright}\ {\isasymunion}\ Wo{\isacharparenright}{\isachardot}\ {\isasymA}\ {\isasymTurnstile}\ F{\isachardoublequoteclose}\isanewline
\ \ \ \ \ \ \isacommand{by}\isamarkupfalse%
\ {\isacharparenleft}iprover\ intro{\isacharcolon}\ exI{\isacharparenright}\isanewline
\ \ \ \ \isacommand{then}\isamarkupfalse%
\ \isacommand{have}\isamarkupfalse%
\ {\isachardoublequoteopen}sat\ {\isacharparenleft}{\isacharbraceleft}\isactrlbold {\isasymnot}\ G{\isacharcomma}F{\isacharbraceright}\ {\isasymunion}\ Wo{\isacharparenright}{\isachardoublequoteclose}\isanewline
\ \ \ \ \ \ \isacommand{by}\isamarkupfalse%
\ {\isacharparenleft}simp\ only{\isacharcolon}\ sat{\isacharunderscore}def{\isacharparenright}\isanewline
\ \ \ \ \isacommand{thus}\isamarkupfalse%
\ {\isacharquery}thesis\isanewline
\ \ \ \ \ \ \isacommand{by}\isamarkupfalse%
\ {\isacharparenleft}rule\ disjI{\isadigit{1}}{\isacharparenright}\isanewline
\ \ \isacommand{next}\isamarkupfalse%
\isanewline
\ \ \ \ \isacommand{assume}\isamarkupfalse%
\ {\isachardoublequoteopen}{\isasymnot}\ {\isacharparenleft}{\isasymA}\ {\isasymTurnstile}\ H{\isacharparenright}{\isachardoublequoteclose}\isanewline
\ \ \ \ \isacommand{then}\isamarkupfalse%
\ \isacommand{have}\isamarkupfalse%
\ {\isachardoublequoteopen}{\isasymA}\ {\isasymTurnstile}\ \isactrlbold {\isasymnot}\ H{\isachardoublequoteclose}\isanewline
\ \ \ \ \ \ \isacommand{by}\isamarkupfalse%
\ {\isacharparenleft}simp\ only{\isacharcolon}\ formula{\isacharunderscore}semantics{\isachardot}simps{\isacharparenleft}{\isadigit{3}}{\isacharparenright}\ simp{\isacharunderscore}thms{\isacharparenleft}{\isadigit{8}}{\isacharparenright}{\isacharparenright}\isanewline
\ \ \ \ \isacommand{then}\isamarkupfalse%
\ \isacommand{have}\isamarkupfalse%
\ {\isachardoublequoteopen}{\isasymforall}F\ {\isasymin}\ {\isacharbraceleft}\isactrlbold {\isasymnot}\ H{\isacharbraceright}{\isachardot}\ {\isasymA}\ {\isasymTurnstile}\ F{\isachardoublequoteclose}\isanewline
\ \ \ \ \ \ \isacommand{by}\isamarkupfalse%
\ simp\isanewline
\ \ \ \ \isacommand{then}\isamarkupfalse%
\ \isacommand{have}\isamarkupfalse%
\ {\isachardoublequoteopen}{\isasymforall}F\ {\isasymin}\ {\isacharparenleft}{\isacharbraceleft}\isactrlbold {\isasymnot}\ H{\isacharbraceright}\ {\isasymunion}\ {\isacharparenleft}{\isacharbraceleft}F{\isacharbraceright}\ {\isasymunion}\ Wo{\isacharparenright}{\isacharparenright}{\isachardot}\ {\isasymA}\ {\isasymTurnstile}\ F{\isachardoublequoteclose}\isanewline
\ \ \ \ \ \ \isacommand{using}\isamarkupfalse%
\ Forall{\isadigit{1}}\ \isacommand{by}\isamarkupfalse%
\ {\isacharparenleft}rule\ ball{\isacharunderscore}Un{\isacharparenright}\isanewline
\ \ \ \ \isacommand{then}\isamarkupfalse%
\ \isacommand{have}\isamarkupfalse%
\ {\isachardoublequoteopen}{\isasymforall}F\ {\isasymin}\ {\isacharbraceleft}\isactrlbold {\isasymnot}\ H{\isacharcomma}F{\isacharbraceright}\ {\isasymunion}\ Wo{\isachardot}\ {\isasymA}\ {\isasymTurnstile}\ F{\isachardoublequoteclose}\isanewline
\ \ \ \ \ \ \isacommand{by}\isamarkupfalse%
\ simp\isanewline
\ \ \ \ \isacommand{then}\isamarkupfalse%
\ \isacommand{have}\isamarkupfalse%
\ {\isachardoublequoteopen}{\isasymexists}{\isasymA}{\isachardot}\ {\isasymforall}F\ {\isasymin}\ {\isacharparenleft}{\isacharbraceleft}\isactrlbold {\isasymnot}\ H{\isacharcomma}F{\isacharbraceright}\ {\isasymunion}\ Wo{\isacharparenright}{\isachardot}\ {\isasymA}\ {\isasymTurnstile}\ F{\isachardoublequoteclose}\isanewline
\ \ \ \ \ \ \isacommand{by}\isamarkupfalse%
\ {\isacharparenleft}iprover\ intro{\isacharcolon}\ exI{\isacharparenright}\isanewline
\ \ \ \ \isacommand{then}\isamarkupfalse%
\ \isacommand{have}\isamarkupfalse%
\ {\isachardoublequoteopen}sat\ {\isacharparenleft}{\isacharbraceleft}\isactrlbold {\isasymnot}\ H{\isacharcomma}F{\isacharbraceright}\ {\isasymunion}\ Wo{\isacharparenright}{\isachardoublequoteclose}\isanewline
\ \ \ \ \ \ \isacommand{by}\isamarkupfalse%
\ {\isacharparenleft}simp\ only{\isacharcolon}\ sat{\isacharunderscore}def{\isacharparenright}\isanewline
\ \ \ \ \isacommand{thus}\isamarkupfalse%
\ {\isacharquery}thesis\isanewline
\ \ \ \ \ \ \isacommand{by}\isamarkupfalse%
\ {\isacharparenleft}rule\ disjI{\isadigit{2}}{\isacharparenright}\isanewline
\ \ \isacommand{qed}\isamarkupfalse%
\isanewline
\isacommand{qed}\isamarkupfalse%
%
\endisatagproof
{\isafoldproof}%
%
\isadelimproof
%
\endisadelimproof
%
\begin{isamarkuptext}%
Por último, probemos que dados \isa{W\ {\isasymin}\ C}, una fórmula \isa{F\ {\isacharequal}\ {\isasymnot}\ {\isacharparenleft}{\isasymnot}\ G{\isacharparenright}} para cierta fórmula \isa{G} tal 
  que \isa{F\ {\isasymin}\ W}, \isa{H\ {\isacharequal}\ G} y \isa{W\isactrlsub {\isadigit{0}}} un subconjunto finito de \isa{W}, entonces se tiene que o bien 
  \isa{{\isacharbraceleft}G{\isacharcomma}F{\isacharbraceright}\ {\isasymunion}\ W\isactrlsub {\isadigit{0}}} es satisfacible o bien \isa{{\isacharbraceleft}H{\isacharcomma}F{\isacharbraceright}\ {\isasymunion}\ W\isactrlsub {\isadigit{0}}} es satisfacible.%
\end{isamarkuptext}\isamarkuptrue%
\isacommand{lemma}\isamarkupfalse%
\ pcp{\isacharunderscore}colecComp{\isacharunderscore}DIS{\isacharunderscore}sat{\isadigit{4}}{\isacharcolon}\isanewline
\ \ \isakeyword{assumes}\ {\isachardoublequoteopen}W\ {\isasymin}\ colecComp{\isachardoublequoteclose}\isanewline
\ \ \ \ \ \ \ \ \ \ {\isachardoublequoteopen}F\ {\isacharequal}\ \isactrlbold {\isasymnot}\ {\isacharparenleft}\isactrlbold {\isasymnot}\ G{\isacharparenright}{\isachardoublequoteclose}\isanewline
\ \ \ \ \ \ \ \ \ \ {\isachardoublequoteopen}H\ {\isacharequal}\ G{\isachardoublequoteclose}\isanewline
\ \ \ \ \ \ \ \ \ \ {\isachardoublequoteopen}F\ {\isasymin}\ W{\isachardoublequoteclose}\isanewline
\ \ \ \ \ \ \ \ \ \ {\isachardoublequoteopen}finite\ Wo{\isachardoublequoteclose}\isanewline
\ \ \ \ \ \ \ \ \ \ {\isachardoublequoteopen}Wo\ {\isasymsubseteq}\ W{\isachardoublequoteclose}\isanewline
\ \ \ \ \ \ \ \ \isakeyword{shows}\ {\isachardoublequoteopen}sat\ {\isacharparenleft}{\isacharbraceleft}G{\isacharcomma}F{\isacharbraceright}\ {\isasymunion}\ Wo{\isacharparenright}\ {\isasymor}\ sat\ {\isacharparenleft}{\isacharbraceleft}H{\isacharcomma}F{\isacharbraceright}\ {\isasymunion}\ Wo{\isacharparenright}{\isachardoublequoteclose}\isanewline
%
\isadelimproof
%
\endisadelimproof
%
\isatagproof
\isacommand{proof}\isamarkupfalse%
\ {\isacharminus}\isanewline
\ \ \isacommand{have}\isamarkupfalse%
\ {\isachardoublequoteopen}sat\ {\isacharparenleft}{\isacharbraceleft}F{\isacharbraceright}\ {\isasymunion}\ Wo{\isacharparenright}{\isachardoublequoteclose}\isanewline
\ \ \ \ \isacommand{using}\isamarkupfalse%
\ assms{\isacharparenleft}{\isadigit{1}}{\isacharcomma}{\isadigit{4}}{\isacharcomma}{\isadigit{5}}{\isacharcomma}{\isadigit{6}}{\isacharparenright}\ \isacommand{by}\isamarkupfalse%
\ {\isacharparenleft}rule\ pcp{\isacharunderscore}colecComp{\isacharunderscore}elem{\isacharunderscore}sat{\isacharparenright}\isanewline
\ \ \isacommand{have}\isamarkupfalse%
\ {\isachardoublequoteopen}F\ {\isasymin}\ {\isacharbraceleft}F{\isacharbraceright}\ {\isasymunion}\ Wo{\isachardoublequoteclose}\isanewline
\ \ \ \ \isacommand{by}\isamarkupfalse%
\ simp\ \isanewline
\ \ \isacommand{have}\isamarkupfalse%
\ Ex{\isadigit{1}}{\isacharcolon}{\isachardoublequoteopen}{\isasymexists}{\isasymA}{\isachardot}\ {\isasymforall}F\ {\isasymin}\ {\isacharparenleft}{\isacharbraceleft}F{\isacharbraceright}\ {\isasymunion}\ Wo{\isacharparenright}{\isachardot}\ {\isasymA}\ {\isasymTurnstile}\ F{\isachardoublequoteclose}\isanewline
\ \ \ \ \isacommand{using}\isamarkupfalse%
\ {\isacartoucheopen}sat\ {\isacharparenleft}{\isacharbraceleft}F{\isacharbraceright}\ {\isasymunion}\ Wo{\isacharparenright}{\isacartoucheclose}\ \isacommand{by}\isamarkupfalse%
\ {\isacharparenleft}simp\ only{\isacharcolon}\ sat{\isacharunderscore}def{\isacharparenright}\isanewline
\ \ \isacommand{obtain}\isamarkupfalse%
\ {\isasymA}\ \isakeyword{where}\ Forall{\isadigit{1}}{\isacharcolon}{\isachardoublequoteopen}{\isasymforall}F\ {\isasymin}\ {\isacharparenleft}{\isacharbraceleft}F{\isacharbraceright}\ {\isasymunion}\ Wo{\isacharparenright}{\isachardot}\ {\isasymA}\ {\isasymTurnstile}\ F{\isachardoublequoteclose}\isanewline
\ \ \ \ \isacommand{using}\isamarkupfalse%
\ Ex{\isadigit{1}}\ \isacommand{by}\isamarkupfalse%
\ {\isacharparenleft}rule\ exE{\isacharparenright}\isanewline
\ \ \isacommand{have}\isamarkupfalse%
\ {\isachardoublequoteopen}{\isasymA}\ {\isasymTurnstile}\ F{\isachardoublequoteclose}\isanewline
\ \ \ \ \isacommand{using}\isamarkupfalse%
\ Forall{\isadigit{1}}\ {\isacartoucheopen}F\ {\isasymin}\ {\isacharbraceleft}F{\isacharbraceright}\ {\isasymunion}\ Wo{\isacartoucheclose}\ \isacommand{by}\isamarkupfalse%
\ {\isacharparenleft}rule\ bspec{\isacharparenright}\isanewline
\ \ \isacommand{then}\isamarkupfalse%
\ \isacommand{have}\isamarkupfalse%
\ {\isachardoublequoteopen}{\isasymA}\ {\isasymTurnstile}\ \isactrlbold {\isasymnot}{\isacharparenleft}\isactrlbold {\isasymnot}\ G{\isacharparenright}{\isachardoublequoteclose}\isanewline
\ \ \ \ \isacommand{using}\isamarkupfalse%
\ assms{\isacharparenleft}{\isadigit{2}}{\isacharparenright}\ \isacommand{by}\isamarkupfalse%
\ {\isacharparenleft}simp\ only{\isacharcolon}\ {\isacartoucheopen}{\isasymA}\ {\isasymTurnstile}\ F{\isacartoucheclose}{\isacharparenright}\isanewline
\ \ \isacommand{then}\isamarkupfalse%
\ \isacommand{have}\isamarkupfalse%
\ {\isachardoublequoteopen}{\isasymnot}\ {\isasymA}\ {\isasymTurnstile}\ \isactrlbold {\isasymnot}\ G{\isachardoublequoteclose}\isanewline
\ \ \ \ \isacommand{by}\isamarkupfalse%
\ {\isacharparenleft}simp\ only{\isacharcolon}\ formula{\isacharunderscore}semantics{\isachardot}simps{\isacharparenleft}{\isadigit{3}}{\isacharparenright}\ simp{\isacharunderscore}thms{\isacharparenleft}{\isadigit{8}}{\isacharparenright}{\isacharparenright}\isanewline
\ \ \isacommand{then}\isamarkupfalse%
\ \isacommand{have}\isamarkupfalse%
\ {\isachardoublequoteopen}{\isasymnot}\ {\isasymnot}{\isasymA}\ {\isasymTurnstile}\ G{\isachardoublequoteclose}\isanewline
\ \ \ \ \isacommand{by}\isamarkupfalse%
\ {\isacharparenleft}simp\ only{\isacharcolon}\ formula{\isacharunderscore}semantics{\isachardot}simps{\isacharparenleft}{\isadigit{3}}{\isacharparenright}\ simp{\isacharunderscore}thms{\isacharparenleft}{\isadigit{8}}{\isacharparenright}{\isacharparenright}\isanewline
\ \ \isacommand{then}\isamarkupfalse%
\ \isacommand{have}\isamarkupfalse%
\ {\isachardoublequoteopen}{\isasymA}\ {\isasymTurnstile}\ G{\isachardoublequoteclose}\isanewline
\ \ \ \ \isacommand{by}\isamarkupfalse%
\ {\isacharparenleft}rule\ notnotD{\isacharparenright}\isanewline
\ \ \isacommand{then}\isamarkupfalse%
\ \isacommand{have}\isamarkupfalse%
\ {\isachardoublequoteopen}{\isasymforall}F\ {\isasymin}\ {\isacharbraceleft}G{\isacharbraceright}{\isachardot}\ {\isasymA}\ {\isasymTurnstile}\ F{\isachardoublequoteclose}\isanewline
\ \ \ \ \isacommand{by}\isamarkupfalse%
\ simp\isanewline
\ \ \isacommand{then}\isamarkupfalse%
\ \isacommand{have}\isamarkupfalse%
\ {\isachardoublequoteopen}{\isasymforall}F\ {\isasymin}\ {\isacharparenleft}{\isacharbraceleft}G{\isacharbraceright}\ {\isasymunion}\ {\isacharparenleft}{\isacharbraceleft}F{\isacharbraceright}\ {\isasymunion}\ Wo{\isacharparenright}{\isacharparenright}{\isachardot}\ {\isasymA}\ {\isasymTurnstile}\ F{\isachardoublequoteclose}\isanewline
\ \ \ \ \isacommand{using}\isamarkupfalse%
\ Forall{\isadigit{1}}\ \isacommand{by}\isamarkupfalse%
\ {\isacharparenleft}rule\ ball{\isacharunderscore}Un{\isacharparenright}\isanewline
\ \ \isacommand{then}\isamarkupfalse%
\ \isacommand{have}\isamarkupfalse%
\ {\isachardoublequoteopen}{\isasymforall}F\ {\isasymin}\ {\isacharbraceleft}G{\isacharcomma}F{\isacharbraceright}\ {\isasymunion}\ Wo{\isachardot}\ {\isasymA}\ {\isasymTurnstile}\ F{\isachardoublequoteclose}\isanewline
\ \ \ \ \isacommand{by}\isamarkupfalse%
\ simp\isanewline
\ \ \isacommand{then}\isamarkupfalse%
\ \isacommand{have}\isamarkupfalse%
\ {\isachardoublequoteopen}{\isasymexists}{\isasymA}{\isachardot}\ {\isasymforall}F\ {\isasymin}\ {\isacharparenleft}{\isacharbraceleft}G{\isacharcomma}F{\isacharbraceright}\ {\isasymunion}\ Wo{\isacharparenright}{\isachardot}\ {\isasymA}\ {\isasymTurnstile}\ F{\isachardoublequoteclose}\isanewline
\ \ \ \ \isacommand{by}\isamarkupfalse%
\ {\isacharparenleft}iprover\ intro{\isacharcolon}\ exI{\isacharparenright}\isanewline
\ \ \isacommand{then}\isamarkupfalse%
\ \isacommand{have}\isamarkupfalse%
\ {\isachardoublequoteopen}sat\ {\isacharparenleft}{\isacharbraceleft}G{\isacharcomma}F{\isacharbraceright}\ {\isasymunion}\ Wo{\isacharparenright}{\isachardoublequoteclose}\isanewline
\ \ \ \ \isacommand{by}\isamarkupfalse%
\ {\isacharparenleft}simp\ only{\isacharcolon}\ sat{\isacharunderscore}def{\isacharparenright}\isanewline
\ \ \isacommand{thus}\isamarkupfalse%
\ {\isacharquery}thesis\isanewline
\ \ \ \ \isacommand{by}\isamarkupfalse%
\ {\isacharparenleft}rule\ disjI{\isadigit{1}}{\isacharparenright}\isanewline
\isacommand{qed}\isamarkupfalse%
%
\endisatagproof
{\isafoldproof}%
%
\isadelimproof
%
\endisadelimproof
%
\begin{isamarkuptext}%
De este modo, por los lemas anteriores para los distintos tipos de fórmula \isa{{\isasymbeta}}, se
  demuestra que dados \isa{W\ {\isasymin}\ C}, \isa{F} una fórmula de tipo \isa{{\isasymbeta}} con componentes \isa{{\isasymbeta}\isactrlsub {\isadigit{1}}} y \isa{{\isasymbeta}\isactrlsub {\isadigit{2}}} tal que 
  \isa{F\ {\isasymin}\ W} y \isa{W\isactrlsub {\isadigit{0}}} un subconjunto finito de \isa{W}, entonces se tiene que o bien \isa{{\isacharbraceleft}{\isasymbeta}\isactrlsub {\isadigit{1}}{\isacharcomma}F{\isacharbraceright}\ {\isasymunion}\ W\isactrlsub {\isadigit{0}}} es 
  satisfacible o bien \isa{{\isacharbraceleft}{\isasymbeta}\isactrlsub {\isadigit{2}}{\isacharcomma}F{\isacharbraceright}\ {\isasymunion}\ W\isactrlsub {\isadigit{0}}} es satisfacible.%
\end{isamarkuptext}\isamarkuptrue%
\isacommand{lemma}\isamarkupfalse%
\ pcp{\isacharunderscore}colecComp{\isacharunderscore}DIS{\isacharunderscore}sat{\isacharcolon}\isanewline
\ \ \isakeyword{assumes}\ {\isachardoublequoteopen}W\ {\isasymin}\ colecComp{\isachardoublequoteclose}\isanewline
\ \ \ \ \ \ \ \ \ \ {\isachardoublequoteopen}Dis\ F\ G\ H{\isachardoublequoteclose}\isanewline
\ \ \ \ \ \ \ \ \ \ {\isachardoublequoteopen}F\ {\isasymin}\ W{\isachardoublequoteclose}\isanewline
\ \ \ \ \ \ \ \ \ \ {\isachardoublequoteopen}finite\ Wo{\isachardoublequoteclose}\isanewline
\ \ \ \ \ \ \ \ \ \ {\isachardoublequoteopen}Wo\ {\isasymsubseteq}\ W{\isachardoublequoteclose}\isanewline
\ \ \ \ \ \ \ \ \isakeyword{shows}\ {\isachardoublequoteopen}sat\ {\isacharparenleft}{\isacharbraceleft}G{\isacharcomma}F{\isacharbraceright}\ {\isasymunion}\ Wo{\isacharparenright}\ {\isasymor}\ sat\ {\isacharparenleft}{\isacharbraceleft}H{\isacharcomma}F{\isacharbraceright}\ {\isasymunion}\ Wo{\isacharparenright}{\isachardoublequoteclose}\isanewline
%
\isadelimproof
%
\endisadelimproof
%
\isatagproof
\isacommand{proof}\isamarkupfalse%
\ {\isacharminus}\isanewline
\ \ \isacommand{have}\isamarkupfalse%
\ {\isachardoublequoteopen}{\isacharparenleft}F\ {\isacharequal}\ G\ \isactrlbold {\isasymor}\ H\ {\isasymor}\ \isanewline
\ \ \ \ \ \ \ \ {\isacharparenleft}{\isasymexists}G{\isadigit{1}}\ H{\isadigit{1}}{\isachardot}\ F\ {\isacharequal}\ G{\isadigit{1}}\ \isactrlbold {\isasymrightarrow}\ H{\isadigit{1}}\ {\isasymand}\ G\ {\isacharequal}\ \isactrlbold {\isasymnot}\ G{\isadigit{1}}\ {\isasymand}\ H\ {\isacharequal}\ H{\isadigit{1}}{\isacharparenright}\ {\isasymor}\ \isanewline
\ \ \ \ \ \ \ \ {\isacharparenleft}{\isasymexists}G{\isadigit{1}}\ H{\isadigit{1}}{\isachardot}\ F\ {\isacharequal}\ \isactrlbold {\isasymnot}\ {\isacharparenleft}G{\isadigit{1}}\ \isactrlbold {\isasymand}\ H{\isadigit{1}}{\isacharparenright}\ {\isasymand}\ G\ {\isacharequal}\ \isactrlbold {\isasymnot}\ G{\isadigit{1}}\ {\isasymand}\ H\ {\isacharequal}\ \isactrlbold {\isasymnot}\ H{\isadigit{1}}{\isacharparenright}\ {\isasymor}\ \isanewline
\ \ \ \ \ \ \ \ F\ {\isacharequal}\ \isactrlbold {\isasymnot}\ {\isacharparenleft}\isactrlbold {\isasymnot}\ G{\isacharparenright}\ {\isasymand}\ H\ {\isacharequal}\ G{\isacharparenright}{\isachardoublequoteclose}\isanewline
\ \ \ \ \isacommand{using}\isamarkupfalse%
\ assms{\isacharparenleft}{\isadigit{2}}{\isacharparenright}\ \isacommand{by}\isamarkupfalse%
\ {\isacharparenleft}simp\ only{\isacharcolon}\ con{\isacharunderscore}dis{\isacharunderscore}simps{\isacharparenleft}{\isadigit{2}}{\isacharparenright}{\isacharparenright}\isanewline
\ \ \isacommand{thus}\isamarkupfalse%
\ {\isacharquery}thesis\isanewline
\ \ \isacommand{proof}\isamarkupfalse%
\ {\isacharparenleft}rule\ disjE{\isacharparenright}\isanewline
\ \ \ \ \isacommand{assume}\isamarkupfalse%
\ {\isachardoublequoteopen}F\ {\isacharequal}\ G\ \isactrlbold {\isasymor}\ H{\isachardoublequoteclose}\isanewline
\ \ \ \ \isacommand{show}\isamarkupfalse%
\ {\isachardoublequoteopen}sat\ {\isacharparenleft}{\isacharbraceleft}G{\isacharcomma}F{\isacharbraceright}\ {\isasymunion}\ Wo{\isacharparenright}\ {\isasymor}\ sat\ {\isacharparenleft}{\isacharbraceleft}H{\isacharcomma}F{\isacharbraceright}\ {\isasymunion}\ Wo{\isacharparenright}{\isachardoublequoteclose}\isanewline
\ \ \ \ \ \ \isacommand{using}\isamarkupfalse%
\ assms{\isacharparenleft}{\isadigit{1}}{\isacharparenright}\ {\isacartoucheopen}F\ {\isacharequal}\ G\ \isactrlbold {\isasymor}\ H{\isacartoucheclose}\ assms{\isacharparenleft}{\isadigit{3}}{\isacharcomma}{\isadigit{4}}{\isacharcomma}{\isadigit{5}}{\isacharparenright}\ \isacommand{by}\isamarkupfalse%
\ {\isacharparenleft}rule\ pcp{\isacharunderscore}colecComp{\isacharunderscore}DIS{\isacharunderscore}sat{\isadigit{1}}{\isacharparenright}\isanewline
\ \ \isacommand{next}\isamarkupfalse%
\isanewline
\ \ \ \ \isacommand{assume}\isamarkupfalse%
\ {\isachardoublequoteopen}{\isacharparenleft}{\isasymexists}G{\isadigit{1}}\ H{\isadigit{1}}{\isachardot}\ F\ {\isacharequal}\ G{\isadigit{1}}\ \isactrlbold {\isasymrightarrow}\ H{\isadigit{1}}\ {\isasymand}\ G\ {\isacharequal}\ \isactrlbold {\isasymnot}\ G{\isadigit{1}}\ {\isasymand}\ H\ {\isacharequal}\ H{\isadigit{1}}{\isacharparenright}\ {\isasymor}\ \isanewline
\ \ \ \ \ \ \ \ {\isacharparenleft}{\isasymexists}G{\isadigit{1}}\ H{\isadigit{1}}{\isachardot}\ F\ {\isacharequal}\ \isactrlbold {\isasymnot}\ {\isacharparenleft}G{\isadigit{1}}\ \isactrlbold {\isasymand}\ H{\isadigit{1}}{\isacharparenright}\ {\isasymand}\ G\ {\isacharequal}\ \isactrlbold {\isasymnot}\ G{\isadigit{1}}\ {\isasymand}\ H\ {\isacharequal}\ \isactrlbold {\isasymnot}\ H{\isadigit{1}}{\isacharparenright}\ {\isasymor}\ \isanewline
\ \ \ \ \ \ \ \ F\ {\isacharequal}\ \isactrlbold {\isasymnot}\ {\isacharparenleft}\isactrlbold {\isasymnot}\ G{\isacharparenright}\ {\isasymand}\ H\ {\isacharequal}\ G{\isachardoublequoteclose}\isanewline
\ \ \ \ \isacommand{thus}\isamarkupfalse%
\ {\isacharquery}thesis\isanewline
\ \ \ \ \isacommand{proof}\isamarkupfalse%
\ {\isacharparenleft}rule\ disjE{\isacharparenright}\isanewline
\ \ \ \ \ \ \isacommand{assume}\isamarkupfalse%
\ Ex{\isadigit{1}}{\isacharcolon}{\isachardoublequoteopen}{\isasymexists}G{\isadigit{1}}\ H{\isadigit{1}}{\isachardot}\ F\ {\isacharequal}\ G{\isadigit{1}}\ \isactrlbold {\isasymrightarrow}\ H{\isadigit{1}}\ {\isasymand}\ G\ {\isacharequal}\ \isactrlbold {\isasymnot}\ G{\isadigit{1}}\ {\isasymand}\ H\ {\isacharequal}\ H{\isadigit{1}}{\isachardoublequoteclose}\isanewline
\ \ \ \ \ \ \isacommand{obtain}\isamarkupfalse%
\ G{\isadigit{1}}\ H{\isadigit{1}}\ \isakeyword{where}\ C{\isadigit{1}}{\isacharcolon}{\isachardoublequoteopen}F\ {\isacharequal}\ G{\isadigit{1}}\ \isactrlbold {\isasymrightarrow}\ H{\isadigit{1}}\ {\isasymand}\ G\ {\isacharequal}\ \isactrlbold {\isasymnot}\ G{\isadigit{1}}\ {\isasymand}\ H\ {\isacharequal}\ H{\isadigit{1}}{\isachardoublequoteclose}\isanewline
\ \ \ \ \ \ \ \ \isacommand{using}\isamarkupfalse%
\ Ex{\isadigit{1}}\ \isacommand{by}\isamarkupfalse%
\ {\isacharparenleft}iprover\ elim{\isacharcolon}\ exE{\isacharparenright}\isanewline
\ \ \ \ \ \ \isacommand{have}\isamarkupfalse%
\ {\isachardoublequoteopen}F\ {\isacharequal}\ G{\isadigit{1}}\ \isactrlbold {\isasymrightarrow}\ H{\isadigit{1}}{\isachardoublequoteclose}\isanewline
\ \ \ \ \ \ \ \ \isacommand{using}\isamarkupfalse%
\ C{\isadigit{1}}\ \isacommand{by}\isamarkupfalse%
\ {\isacharparenleft}rule\ conjunct{\isadigit{1}}{\isacharparenright}\isanewline
\ \ \ \ \ \ \isacommand{have}\isamarkupfalse%
\ {\isachardoublequoteopen}G\ {\isacharequal}\ \isactrlbold {\isasymnot}\ G{\isadigit{1}}{\isachardoublequoteclose}\isanewline
\ \ \ \ \ \ \ \ \isacommand{using}\isamarkupfalse%
\ C{\isadigit{1}}\ \isacommand{by}\isamarkupfalse%
\ {\isacharparenleft}iprover\ elim{\isacharcolon}\ conjunct{\isadigit{1}}{\isacharparenright}\isanewline
\ \ \ \ \ \ \isacommand{have}\isamarkupfalse%
\ {\isachardoublequoteopen}H\ {\isacharequal}\ H{\isadigit{1}}{\isachardoublequoteclose}\isanewline
\ \ \ \ \ \ \ \ \isacommand{using}\isamarkupfalse%
\ C{\isadigit{1}}\ \isacommand{by}\isamarkupfalse%
\ {\isacharparenleft}iprover\ elim{\isacharcolon}\ conjunct{\isadigit{2}}{\isacharparenright}\isanewline
\ \ \ \ \ \ \isacommand{have}\isamarkupfalse%
\ {\isachardoublequoteopen}sat\ {\isacharparenleft}{\isacharbraceleft}\isactrlbold {\isasymnot}\ G{\isadigit{1}}{\isacharcomma}F{\isacharbraceright}\ {\isasymunion}\ Wo{\isacharparenright}\ {\isasymor}\ sat\ {\isacharparenleft}{\isacharbraceleft}H{\isadigit{1}}{\isacharcomma}F{\isacharbraceright}\ {\isasymunion}\ Wo{\isacharparenright}{\isachardoublequoteclose}\isanewline
\ \ \ \ \ \ \ \ \isacommand{using}\isamarkupfalse%
\ assms{\isacharparenleft}{\isadigit{1}}{\isacharparenright}\ {\isacartoucheopen}F\ {\isacharequal}\ G{\isadigit{1}}\ \isactrlbold {\isasymrightarrow}\ H{\isadigit{1}}{\isacartoucheclose}\ assms{\isacharparenleft}{\isadigit{3}}{\isacharcomma}{\isadigit{4}}{\isacharcomma}{\isadigit{5}}{\isacharparenright}\ \isacommand{by}\isamarkupfalse%
\ {\isacharparenleft}rule\ pcp{\isacharunderscore}colecComp{\isacharunderscore}DIS{\isacharunderscore}sat{\isadigit{2}}{\isacharparenright}\isanewline
\ \ \ \ \ \ \isacommand{thus}\isamarkupfalse%
\ {\isachardoublequoteopen}sat\ {\isacharparenleft}{\isacharbraceleft}G{\isacharcomma}\ F{\isacharbraceright}\ {\isasymunion}\ Wo{\isacharparenright}\ {\isasymor}\ sat\ {\isacharparenleft}{\isacharbraceleft}H{\isacharcomma}\ F{\isacharbraceright}\ {\isasymunion}\ Wo{\isacharparenright}{\isachardoublequoteclose}\isanewline
\ \ \ \ \ \ \ \ \isacommand{by}\isamarkupfalse%
\ {\isacharparenleft}simp\ only{\isacharcolon}\ {\isacartoucheopen}G\ {\isacharequal}\ \isactrlbold {\isasymnot}\ G{\isadigit{1}}{\isacartoucheclose}\ {\isacartoucheopen}H\ {\isacharequal}\ H{\isadigit{1}}{\isacartoucheclose}{\isacharparenright}\ \isanewline
\ \ \ \ \isacommand{next}\isamarkupfalse%
\isanewline
\ \ \ \ \ \ \isacommand{assume}\isamarkupfalse%
\ {\isachardoublequoteopen}{\isacharparenleft}{\isasymexists}G{\isadigit{1}}\ H{\isadigit{1}}{\isachardot}\ F\ {\isacharequal}\ \isactrlbold {\isasymnot}\ {\isacharparenleft}G{\isadigit{1}}\ \isactrlbold {\isasymand}\ H{\isadigit{1}}{\isacharparenright}\ {\isasymand}\ G\ {\isacharequal}\ \isactrlbold {\isasymnot}\ G{\isadigit{1}}\ {\isasymand}\ H\ {\isacharequal}\ \isactrlbold {\isasymnot}\ H{\isadigit{1}}{\isacharparenright}\ {\isasymor}\ \isanewline
\ \ \ \ \ \ \ \ F\ {\isacharequal}\ \isactrlbold {\isasymnot}\ {\isacharparenleft}\isactrlbold {\isasymnot}\ G{\isacharparenright}\ {\isasymand}\ H\ {\isacharequal}\ G{\isachardoublequoteclose}\isanewline
\ \ \ \ \ \ \isacommand{thus}\isamarkupfalse%
\ {\isacharquery}thesis\isanewline
\ \ \ \ \ \ \isacommand{proof}\isamarkupfalse%
\ {\isacharparenleft}rule\ disjE{\isacharparenright}\isanewline
\ \ \ \ \ \ \ \ \isacommand{assume}\isamarkupfalse%
\ Ex{\isadigit{2}}{\isacharcolon}{\isachardoublequoteopen}{\isasymexists}G{\isadigit{1}}\ H{\isadigit{1}}{\isachardot}\ F\ {\isacharequal}\ \isactrlbold {\isasymnot}\ {\isacharparenleft}G{\isadigit{1}}\ \isactrlbold {\isasymand}\ H{\isadigit{1}}{\isacharparenright}\ {\isasymand}\ G\ {\isacharequal}\ \isactrlbold {\isasymnot}\ G{\isadigit{1}}\ {\isasymand}\ H\ {\isacharequal}\ \isactrlbold {\isasymnot}\ H{\isadigit{1}}{\isachardoublequoteclose}\isanewline
\ \ \ \ \ \ \ \ \isacommand{obtain}\isamarkupfalse%
\ G{\isadigit{1}}\ H{\isadigit{1}}\ \isakeyword{where}\ C{\isadigit{2}}{\isacharcolon}{\isachardoublequoteopen}F\ {\isacharequal}\ \isactrlbold {\isasymnot}\ {\isacharparenleft}G{\isadigit{1}}\ \isactrlbold {\isasymand}\ H{\isadigit{1}}{\isacharparenright}\ {\isasymand}\ G\ {\isacharequal}\ \isactrlbold {\isasymnot}\ G{\isadigit{1}}\ {\isasymand}\ H\ {\isacharequal}\ \isactrlbold {\isasymnot}\ H{\isadigit{1}}{\isachardoublequoteclose}\isanewline
\ \ \ \ \ \ \ \ \ \ \isacommand{using}\isamarkupfalse%
\ Ex{\isadigit{2}}\ \isacommand{by}\isamarkupfalse%
\ {\isacharparenleft}iprover\ elim{\isacharcolon}\ exE{\isacharparenright}\isanewline
\ \ \ \ \ \ \ \ \isacommand{have}\isamarkupfalse%
\ {\isachardoublequoteopen}F\ {\isacharequal}\ \isactrlbold {\isasymnot}\ {\isacharparenleft}G{\isadigit{1}}\ \isactrlbold {\isasymand}\ H{\isadigit{1}}{\isacharparenright}{\isachardoublequoteclose}\isanewline
\ \ \ \ \ \ \ \ \ \ \isacommand{using}\isamarkupfalse%
\ C{\isadigit{2}}\ \isacommand{by}\isamarkupfalse%
\ {\isacharparenleft}rule\ conjunct{\isadigit{1}}{\isacharparenright}\isanewline
\ \ \ \ \ \ \ \ \isacommand{have}\isamarkupfalse%
\ {\isachardoublequoteopen}G\ {\isacharequal}\ \isactrlbold {\isasymnot}\ G{\isadigit{1}}{\isachardoublequoteclose}\isanewline
\ \ \ \ \ \ \ \ \ \ \isacommand{using}\isamarkupfalse%
\ C{\isadigit{2}}\ \isacommand{by}\isamarkupfalse%
\ {\isacharparenleft}iprover\ elim{\isacharcolon}\ conjunct{\isadigit{1}}{\isacharparenright}\isanewline
\ \ \ \ \ \ \ \ \isacommand{have}\isamarkupfalse%
\ {\isachardoublequoteopen}H\ {\isacharequal}\ \isactrlbold {\isasymnot}\ H{\isadigit{1}}{\isachardoublequoteclose}\isanewline
\ \ \ \ \ \ \ \ \ \ \isacommand{using}\isamarkupfalse%
\ C{\isadigit{2}}\ \isacommand{by}\isamarkupfalse%
\ {\isacharparenleft}iprover\ elim{\isacharcolon}\ conjunct{\isadigit{2}}{\isacharparenright}\isanewline
\ \ \ \ \ \ \ \ \isacommand{have}\isamarkupfalse%
\ {\isachardoublequoteopen}sat\ {\isacharparenleft}{\isacharbraceleft}\isactrlbold {\isasymnot}\ G{\isadigit{1}}{\isacharcomma}F{\isacharbraceright}\ {\isasymunion}\ Wo{\isacharparenright}\ {\isasymor}\ sat\ {\isacharparenleft}{\isacharbraceleft}\isactrlbold {\isasymnot}\ H{\isadigit{1}}{\isacharcomma}F{\isacharbraceright}\ {\isasymunion}\ Wo{\isacharparenright}{\isachardoublequoteclose}\isanewline
\ \ \ \ \ \ \ \ \ \ \isacommand{using}\isamarkupfalse%
\ assms{\isacharparenleft}{\isadigit{1}}{\isacharparenright}\ {\isacartoucheopen}F\ {\isacharequal}\ \isactrlbold {\isasymnot}\ {\isacharparenleft}G{\isadigit{1}}\ \isactrlbold {\isasymand}\ H{\isadigit{1}}{\isacharparenright}{\isacartoucheclose}\ assms{\isacharparenleft}{\isadigit{3}}{\isacharcomma}{\isadigit{4}}{\isacharcomma}{\isadigit{5}}{\isacharparenright}\ \isacommand{by}\isamarkupfalse%
\ {\isacharparenleft}rule\ pcp{\isacharunderscore}colecComp{\isacharunderscore}DIS{\isacharunderscore}sat{\isadigit{3}}{\isacharparenright}\isanewline
\ \ \ \ \ \ \ \ \isacommand{thus}\isamarkupfalse%
\ {\isachardoublequoteopen}sat\ {\isacharparenleft}{\isacharbraceleft}G{\isacharcomma}F{\isacharbraceright}\ {\isasymunion}\ Wo{\isacharparenright}\ {\isasymor}\ sat\ {\isacharparenleft}{\isacharbraceleft}H{\isacharcomma}F{\isacharbraceright}\ {\isasymunion}\ Wo{\isacharparenright}{\isachardoublequoteclose}\isanewline
\ \ \ \ \ \ \ \ \ \ \isacommand{by}\isamarkupfalse%
\ {\isacharparenleft}simp\ only{\isacharcolon}\ {\isacartoucheopen}G\ {\isacharequal}\ \isactrlbold {\isasymnot}\ G{\isadigit{1}}{\isacartoucheclose}\ {\isacartoucheopen}H\ {\isacharequal}\ \isactrlbold {\isasymnot}\ H{\isadigit{1}}{\isacartoucheclose}{\isacharparenright}\isanewline
\ \ \ \ \ \ \isacommand{next}\isamarkupfalse%
\isanewline
\ \ \ \ \ \ \ \ \isacommand{assume}\isamarkupfalse%
\ {\isachardoublequoteopen}F\ {\isacharequal}\ \isactrlbold {\isasymnot}\ {\isacharparenleft}\isactrlbold {\isasymnot}\ G{\isacharparenright}\ {\isasymand}\ H\ {\isacharequal}\ G{\isachardoublequoteclose}\isanewline
\ \ \ \ \ \ \ \ \isacommand{then}\isamarkupfalse%
\ \isacommand{have}\isamarkupfalse%
\ {\isachardoublequoteopen}F\ {\isacharequal}\ \isactrlbold {\isasymnot}\ {\isacharparenleft}\isactrlbold {\isasymnot}\ G{\isacharparenright}{\isachardoublequoteclose}\isanewline
\ \ \ \ \ \ \ \ \ \ \isacommand{by}\isamarkupfalse%
\ {\isacharparenleft}rule\ conjunct{\isadigit{1}}{\isacharparenright}\isanewline
\ \ \ \ \ \ \ \ \isacommand{have}\isamarkupfalse%
\ {\isachardoublequoteopen}H\ {\isacharequal}\ G{\isachardoublequoteclose}\isanewline
\ \ \ \ \ \ \ \ \ \ \isacommand{using}\isamarkupfalse%
\ {\isacartoucheopen}F\ {\isacharequal}\ \isactrlbold {\isasymnot}\ {\isacharparenleft}\isactrlbold {\isasymnot}\ G{\isacharparenright}\ {\isasymand}\ H\ {\isacharequal}\ G{\isacartoucheclose}\ \isacommand{by}\isamarkupfalse%
\ {\isacharparenleft}rule\ conjunct{\isadigit{2}}{\isacharparenright}\isanewline
\ \ \ \ \ \ \ \ \isacommand{show}\isamarkupfalse%
\ {\isachardoublequoteopen}sat\ {\isacharparenleft}{\isacharbraceleft}G{\isacharcomma}F{\isacharbraceright}\ {\isasymunion}\ Wo{\isacharparenright}\ {\isasymor}\ sat\ {\isacharparenleft}{\isacharbraceleft}H{\isacharcomma}F{\isacharbraceright}\ {\isasymunion}\ Wo{\isacharparenright}{\isachardoublequoteclose}\isanewline
\ \ \ \ \ \ \ \ \ \ \isacommand{using}\isamarkupfalse%
\ assms{\isacharparenleft}{\isadigit{1}}{\isacharparenright}\ {\isacartoucheopen}F\ {\isacharequal}\ \isactrlbold {\isasymnot}\ {\isacharparenleft}\isactrlbold {\isasymnot}\ G{\isacharparenright}{\isacartoucheclose}\ {\isacartoucheopen}H\ {\isacharequal}\ G{\isacartoucheclose}\ assms{\isacharparenleft}{\isadigit{3}}{\isacharcomma}{\isadigit{4}}{\isacharcomma}{\isadigit{5}}{\isacharparenright}\ \isacommand{by}\isamarkupfalse%
\ {\isacharparenleft}rule\ pcp{\isacharunderscore}colecComp{\isacharunderscore}DIS{\isacharunderscore}sat{\isadigit{4}}{\isacharparenright}\isanewline
\ \ \ \ \ \ \isacommand{qed}\isamarkupfalse%
\isanewline
\ \ \ \ \isacommand{qed}\isamarkupfalse%
\isanewline
\ \ \isacommand{qed}\isamarkupfalse%
\isanewline
\isacommand{qed}\isamarkupfalse%
%
\endisatagproof
{\isafoldproof}%
%
\isadelimproof
%
\endisadelimproof
%
\begin{isamarkuptext}%
Finalmente, con los lemas auxiliares anteriores podemos demostrar detalladamente la cuarta 
  condición del lema \isa{{\isadigit{2}}{\isachardot}{\isadigit{0}}{\isachardot}{\isadigit{2}}}: dados \isa{W\ {\isasymin}\ C} y \isa{F} una fórmula de tipo \isa{{\isasymbeta}} con componentes \isa{{\isasymbeta}\isactrlsub {\isadigit{1}}} y 
  \isa{{\isasymbeta}\isactrlsub {\isadigit{2}}} tal que \isa{F\ {\isasymin}\ W}, se tiene que o bien \isa{{\isacharbraceleft}{\isasymbeta}\isactrlsub {\isadigit{1}}{\isacharbraceright}\ {\isasymunion}\ W\ {\isasymin}\ C} o bien\\ \isa{{\isacharbraceleft}{\isasymbeta}\isactrlsub {\isadigit{2}}{\isacharbraceright}\ {\isasymunion}\ W\ {\isasymin}\ C}.%
\end{isamarkuptext}\isamarkuptrue%
\isacommand{lemma}\isamarkupfalse%
\ pcp{\isacharunderscore}colecComp{\isacharunderscore}DIS{\isacharcolon}\isanewline
\ \ \isakeyword{assumes}\ {\isachardoublequoteopen}W\ {\isasymin}\ colecComp{\isachardoublequoteclose}\isanewline
\ \ \isakeyword{shows}\ {\isachardoublequoteopen}{\isasymforall}F\ G\ H{\isachardot}\ Dis\ F\ G\ H\ {\isasymlongrightarrow}\ F\ {\isasymin}\ W\ {\isasymlongrightarrow}\ {\isacharbraceleft}G{\isacharbraceright}\ {\isasymunion}\ W\ {\isasymin}\ colecComp\ {\isasymor}\ {\isacharbraceleft}H{\isacharbraceright}\ {\isasymunion}\ W\ {\isasymin}\ colecComp{\isachardoublequoteclose}\isanewline
%
\isadelimproof
%
\endisadelimproof
%
\isatagproof
\isacommand{proof}\isamarkupfalse%
\ {\isacharparenleft}rule\ allI{\isacharparenright}{\isacharplus}\isanewline
\ \ \isacommand{fix}\isamarkupfalse%
\ F\ G\ H\isanewline
\ \ \isacommand{show}\isamarkupfalse%
\ {\isachardoublequoteopen}Dis\ F\ G\ H\ {\isasymlongrightarrow}\ F\ {\isasymin}\ W\ {\isasymlongrightarrow}\ {\isacharbraceleft}G{\isacharbraceright}\ {\isasymunion}\ W\ {\isasymin}\ colecComp\ {\isasymor}\ {\isacharbraceleft}H{\isacharbraceright}\ {\isasymunion}\ W\ {\isasymin}\ colecComp{\isachardoublequoteclose}\isanewline
\ \ \isacommand{proof}\isamarkupfalse%
\ {\isacharparenleft}rule\ impI{\isacharparenright}{\isacharplus}\isanewline
\ \ \ \ \isacommand{assume}\isamarkupfalse%
\ {\isachardoublequoteopen}Dis\ F\ G\ H{\isachardoublequoteclose}\isanewline
\ \ \ \ \isacommand{assume}\isamarkupfalse%
\ {\isachardoublequoteopen}F\ {\isasymin}\ W{\isachardoublequoteclose}\isanewline
\ \ \ \ \isacommand{show}\isamarkupfalse%
\ {\isachardoublequoteopen}{\isacharbraceleft}G{\isacharbraceright}\ {\isasymunion}\ W\ {\isasymin}\ colecComp\ {\isasymor}\ {\isacharbraceleft}H{\isacharbraceright}\ {\isasymunion}\ W\ {\isasymin}\ colecComp{\isachardoublequoteclose}\isanewline
\ \ \ \ \isacommand{proof}\isamarkupfalse%
\ {\isacharparenleft}rule\ ccontr{\isacharparenright}\isanewline
\ \ \ \ \ \ \isacommand{assume}\isamarkupfalse%
\ {\isachardoublequoteopen}{\isasymnot}{\isacharparenleft}{\isacharbraceleft}G{\isacharbraceright}\ {\isasymunion}\ W\ {\isasymin}\ colecComp\ {\isasymor}\ {\isacharbraceleft}H{\isacharbraceright}\ {\isasymunion}\ W\ {\isasymin}\ colecComp{\isacharparenright}{\isachardoublequoteclose}\isanewline
\ \ \ \ \ \ \isacommand{then}\isamarkupfalse%
\ \isacommand{have}\isamarkupfalse%
\ C{\isacharcolon}{\isachardoublequoteopen}{\isacharbraceleft}G{\isacharbraceright}\ {\isasymunion}\ W\ {\isasymnotin}\ colecComp\ {\isasymand}\ {\isacharbraceleft}H{\isacharbraceright}\ {\isasymunion}\ W\ {\isasymnotin}\ colecComp{\isachardoublequoteclose}\isanewline
\ \ \ \ \ \ \ \ \isacommand{by}\isamarkupfalse%
\ {\isacharparenleft}simp\ only{\isacharcolon}\ de{\isacharunderscore}Morgan{\isacharunderscore}disj\ simp{\isacharunderscore}thms{\isacharparenleft}{\isadigit{8}}{\isacharparenright}{\isacharparenright}\isanewline
\ \ \ \ \ \ \isacommand{then}\isamarkupfalse%
\ \isacommand{have}\isamarkupfalse%
\ {\isachardoublequoteopen}{\isacharbraceleft}G{\isacharbraceright}\ {\isasymunion}\ W\ {\isasymnotin}\ colecComp{\isachardoublequoteclose}\isanewline
\ \ \ \ \ \ \ \ \isacommand{by}\isamarkupfalse%
\ {\isacharparenleft}rule\ conjunct{\isadigit{1}}{\isacharparenright}\isanewline
\ \ \ \ \ \ \isacommand{have}\isamarkupfalse%
\ Ex{\isadigit{1}}{\isacharcolon}{\isachardoublequoteopen}{\isasymexists}Wo\ {\isasymsubseteq}\ W{\isachardot}\ finite\ Wo\ {\isasymand}\ {\isasymnot}{\isacharparenleft}sat\ {\isacharparenleft}{\isacharbraceleft}G{\isacharbraceright}\ {\isasymunion}\ Wo{\isacharparenright}{\isacharparenright}{\isachardoublequoteclose}\isanewline
\ \ \ \ \ \ \ \ \isacommand{using}\isamarkupfalse%
\ assms\ {\isacartoucheopen}{\isacharbraceleft}G{\isacharbraceright}\ {\isasymunion}\ W\ {\isasymnotin}\ colecComp{\isacartoucheclose}\ \isacommand{by}\isamarkupfalse%
\ {\isacharparenleft}rule\ not{\isacharunderscore}colecComp{\isacharparenright}\isanewline
\ \ \ \ \ \ \isacommand{obtain}\isamarkupfalse%
\ W{\isadigit{1}}\ \isakeyword{where}\ {\isachardoublequoteopen}W{\isadigit{1}}\ {\isasymsubseteq}\ W{\isachardoublequoteclose}\ \isakeyword{and}\ C{\isadigit{1}}{\isacharcolon}{\isachardoublequoteopen}finite\ W{\isadigit{1}}\ {\isasymand}\ {\isasymnot}{\isacharparenleft}sat\ {\isacharparenleft}{\isacharbraceleft}G{\isacharbraceright}\ {\isasymunion}\ W{\isadigit{1}}{\isacharparenright}{\isacharparenright}{\isachardoublequoteclose}\isanewline
\ \ \ \ \ \ \ \ \isacommand{using}\isamarkupfalse%
\ Ex{\isadigit{1}}\ \isacommand{by}\isamarkupfalse%
\ {\isacharparenleft}rule\ subexE{\isacharparenright}\isanewline
\ \ \ \ \ \ \isacommand{have}\isamarkupfalse%
\ {\isachardoublequoteopen}finite\ W{\isadigit{1}}{\isachardoublequoteclose}\isanewline
\ \ \ \ \ \ \ \ \isacommand{using}\isamarkupfalse%
\ C{\isadigit{1}}\ \isacommand{by}\isamarkupfalse%
\ {\isacharparenleft}rule\ conjunct{\isadigit{1}}{\isacharparenright}\isanewline
\ \ \ \ \ \ \isacommand{have}\isamarkupfalse%
\ {\isachardoublequoteopen}{\isasymnot}{\isacharparenleft}sat\ {\isacharparenleft}{\isacharbraceleft}G{\isacharbraceright}\ {\isasymunion}\ W{\isadigit{1}}{\isacharparenright}{\isacharparenright}{\isachardoublequoteclose}\isanewline
\ \ \ \ \ \ \ \ \isacommand{using}\isamarkupfalse%
\ C{\isadigit{1}}\ \isacommand{by}\isamarkupfalse%
\ {\isacharparenleft}rule\ conjunct{\isadigit{2}}{\isacharparenright}\isanewline
\ \ \ \ \ \ \isacommand{have}\isamarkupfalse%
\ {\isachardoublequoteopen}{\isacharbraceleft}H{\isacharbraceright}\ {\isasymunion}\ W\ {\isasymnotin}\ colecComp{\isachardoublequoteclose}\isanewline
\ \ \ \ \ \ \ \ \isacommand{using}\isamarkupfalse%
\ C\ \isacommand{by}\isamarkupfalse%
\ {\isacharparenleft}rule\ conjunct{\isadigit{2}}{\isacharparenright}\ \isanewline
\ \ \ \ \ \ \isacommand{have}\isamarkupfalse%
\ Ex{\isadigit{2}}{\isacharcolon}{\isachardoublequoteopen}{\isasymexists}Wo\ {\isasymsubseteq}\ W{\isachardot}\ finite\ Wo\ {\isasymand}\ {\isasymnot}{\isacharparenleft}sat\ {\isacharparenleft}{\isacharbraceleft}H{\isacharbraceright}\ {\isasymunion}\ Wo{\isacharparenright}{\isacharparenright}{\isachardoublequoteclose}\isanewline
\ \ \ \ \ \ \ \ \isacommand{using}\isamarkupfalse%
\ assms\ {\isacartoucheopen}{\isacharbraceleft}H{\isacharbraceright}\ {\isasymunion}\ W\ {\isasymnotin}\ colecComp{\isacartoucheclose}\ \isacommand{by}\isamarkupfalse%
\ {\isacharparenleft}rule\ not{\isacharunderscore}colecComp{\isacharparenright}\isanewline
\ \ \ \ \ \ \isacommand{obtain}\isamarkupfalse%
\ W{\isadigit{2}}\ \isakeyword{where}\ {\isachardoublequoteopen}W{\isadigit{2}}\ {\isasymsubseteq}\ W{\isachardoublequoteclose}\ \isakeyword{and}\ C{\isadigit{2}}{\isacharcolon}{\isachardoublequoteopen}finite\ W{\isadigit{2}}\ {\isasymand}\ {\isasymnot}{\isacharparenleft}sat\ {\isacharparenleft}{\isacharbraceleft}H{\isacharbraceright}\ {\isasymunion}\ W{\isadigit{2}}{\isacharparenright}{\isacharparenright}{\isachardoublequoteclose}\isanewline
\ \ \ \ \ \ \ \ \isacommand{using}\isamarkupfalse%
\ Ex{\isadigit{2}}\ \isacommand{by}\isamarkupfalse%
\ {\isacharparenleft}rule\ subexE{\isacharparenright}\isanewline
\ \ \ \ \ \ \isacommand{have}\isamarkupfalse%
\ {\isachardoublequoteopen}finite\ W{\isadigit{2}}{\isachardoublequoteclose}\isanewline
\ \ \ \ \ \ \ \ \isacommand{using}\isamarkupfalse%
\ C{\isadigit{2}}\ \isacommand{by}\isamarkupfalse%
\ {\isacharparenleft}rule\ conjunct{\isadigit{1}}{\isacharparenright}\isanewline
\ \ \ \ \ \ \isacommand{have}\isamarkupfalse%
\ {\isachardoublequoteopen}{\isasymnot}{\isacharparenleft}sat\ {\isacharparenleft}{\isacharbraceleft}H{\isacharbraceright}\ {\isasymunion}\ W{\isadigit{2}}{\isacharparenright}{\isacharparenright}{\isachardoublequoteclose}\isanewline
\ \ \ \ \ \ \ \ \isacommand{using}\isamarkupfalse%
\ C{\isadigit{2}}\ \isacommand{by}\isamarkupfalse%
\ {\isacharparenleft}rule\ conjunct{\isadigit{2}}{\isacharparenright}\isanewline
\ \ \ \ \ \ \isacommand{let}\isamarkupfalse%
\ {\isacharquery}Wo\ {\isacharequal}\ {\isachardoublequoteopen}W{\isadigit{1}}\ {\isasymunion}\ W{\isadigit{2}}{\isachardoublequoteclose}\isanewline
\ \ \ \ \ \ \isacommand{have}\isamarkupfalse%
\ {\isachardoublequoteopen}{\isacharquery}Wo\ {\isasymsubseteq}\ W{\isachardoublequoteclose}\isanewline
\ \ \ \ \ \ \ \ \isacommand{using}\isamarkupfalse%
\ {\isacartoucheopen}W{\isadigit{1}}\ {\isasymsubseteq}\ W{\isacartoucheclose}\ {\isacartoucheopen}W{\isadigit{2}}\ {\isasymsubseteq}\ W{\isacartoucheclose}\ \isacommand{by}\isamarkupfalse%
\ {\isacharparenleft}simp\ only{\isacharcolon}\ Un{\isacharunderscore}least{\isacharparenright}\isanewline
\ \ \ \ \ \ \isacommand{have}\isamarkupfalse%
\ {\isachardoublequoteopen}finite\ {\isacharquery}Wo{\isachardoublequoteclose}\isanewline
\ \ \ \ \ \ \ \ \isacommand{using}\isamarkupfalse%
\ {\isacartoucheopen}finite\ W{\isadigit{1}}{\isacartoucheclose}\ {\isacartoucheopen}finite\ W{\isadigit{2}}{\isacartoucheclose}\ \isacommand{by}\isamarkupfalse%
\ {\isacharparenleft}simp\ only{\isacharcolon}\ finite{\isacharunderscore}Un{\isacharparenright}\isanewline
\ \ \ \ \ \ \isacommand{have}\isamarkupfalse%
\ {\isachardoublequoteopen}{\isacharbraceleft}G{\isacharbraceright}\ {\isasymunion}\ W{\isadigit{1}}\ {\isasymsubseteq}\ {\isacharparenleft}{\isacharbraceleft}G{\isacharbraceright}\ {\isasymunion}\ W{\isadigit{1}}{\isacharparenright}\ {\isasymunion}\ W{\isadigit{2}}{\isachardoublequoteclose}\isanewline
\ \ \ \ \ \ \ \ \isacommand{by}\isamarkupfalse%
\ {\isacharparenleft}simp\ only{\isacharcolon}\ Un{\isacharunderscore}upper{\isadigit{1}}{\isacharparenright}\isanewline
\ \ \ \ \ \ \isacommand{then}\isamarkupfalse%
\ \isacommand{have}\isamarkupfalse%
\ {\isachardoublequoteopen}{\isacharbraceleft}G{\isacharbraceright}\ {\isasymunion}\ W{\isadigit{1}}\ {\isasymsubseteq}\ {\isacharbraceleft}G{\isacharbraceright}\ {\isasymunion}\ {\isacharquery}Wo{\isachardoublequoteclose}\isanewline
\ \ \ \ \ \ \ \ \isacommand{by}\isamarkupfalse%
\ {\isacharparenleft}simp\ only{\isacharcolon}\ Un{\isacharunderscore}assoc{\isacharparenright}\isanewline
\ \ \ \ \ \ \isacommand{then}\isamarkupfalse%
\ \isacommand{have}\isamarkupfalse%
\ {\isachardoublequoteopen}{\isacharbraceleft}G{\isacharbraceright}\ {\isasymunion}\ W{\isadigit{1}}\ {\isasymsubseteq}\ {\isacharbraceleft}G{\isacharcomma}F{\isacharbraceright}\ {\isasymunion}\ {\isacharquery}Wo{\isachardoublequoteclose}\isanewline
\ \ \ \ \ \ \ \ \isacommand{by}\isamarkupfalse%
\ blast\isanewline
\ \ \ \ \ \ \isacommand{then}\isamarkupfalse%
\ \isacommand{have}\isamarkupfalse%
\ {\isadigit{1}}{\isacharcolon}{\isachardoublequoteopen}{\isasymnot}{\isacharparenleft}sat{\isacharparenleft}{\isacharbraceleft}G{\isacharcomma}F{\isacharbraceright}\ {\isasymunion}\ {\isacharquery}Wo{\isacharparenright}{\isacharparenright}{\isachardoublequoteclose}\isanewline
\ \ \ \ \ \ \ \ \isacommand{using}\isamarkupfalse%
\ {\isacartoucheopen}{\isasymnot}sat\ {\isacharparenleft}{\isacharbraceleft}G{\isacharbraceright}\ {\isasymunion}\ W{\isadigit{1}}{\isacharparenright}{\isacartoucheclose}\ \isacommand{by}\isamarkupfalse%
\ {\isacharparenleft}rule\ sat{\isacharunderscore}subset{\isacharunderscore}ccontr{\isacharparenright}\isanewline
\ \ \ \ \ \ \isacommand{have}\isamarkupfalse%
\ {\isachardoublequoteopen}{\isacharbraceleft}H{\isacharbraceright}\ {\isasymunion}\ W{\isadigit{2}}\ {\isasymsubseteq}\ {\isacharparenleft}{\isacharbraceleft}H{\isacharbraceright}\ {\isasymunion}\ W{\isadigit{2}}{\isacharparenright}\ {\isasymunion}\ W{\isadigit{1}}{\isachardoublequoteclose}\isanewline
\ \ \ \ \ \ \ \ \isacommand{by}\isamarkupfalse%
\ {\isacharparenleft}simp\ only{\isacharcolon}\ Un{\isacharunderscore}upper{\isadigit{1}}{\isacharparenright}\isanewline
\ \ \ \ \ \ \isacommand{then}\isamarkupfalse%
\ \isacommand{have}\isamarkupfalse%
\ {\isachardoublequoteopen}{\isacharbraceleft}H{\isacharbraceright}\ {\isasymunion}\ W{\isadigit{2}}\ {\isasymsubseteq}\ {\isacharbraceleft}H{\isacharbraceright}\ {\isasymunion}\ {\isacharparenleft}W{\isadigit{2}}\ {\isasymunion}\ W{\isadigit{1}}{\isacharparenright}{\isachardoublequoteclose}\isanewline
\ \ \ \ \ \ \ \ \isacommand{by}\isamarkupfalse%
\ {\isacharparenleft}simp\ only{\isacharcolon}\ Un{\isacharunderscore}assoc{\isacharparenright}\ \isanewline
\ \ \ \ \ \ \isacommand{then}\isamarkupfalse%
\ \isacommand{have}\isamarkupfalse%
\ {\isachardoublequoteopen}{\isacharbraceleft}H{\isacharbraceright}\ {\isasymunion}\ W{\isadigit{2}}\ {\isasymsubseteq}\ {\isacharbraceleft}H{\isacharbraceright}\ {\isasymunion}\ {\isacharquery}Wo{\isachardoublequoteclose}\isanewline
\ \ \ \ \ \ \ \ \isacommand{by}\isamarkupfalse%
\ {\isacharparenleft}simp\ only{\isacharcolon}\ Un{\isacharunderscore}commute{\isacharparenright}\isanewline
\ \ \ \ \ \ \isacommand{then}\isamarkupfalse%
\ \isacommand{have}\isamarkupfalse%
\ {\isachardoublequoteopen}{\isacharbraceleft}H{\isacharbraceright}\ {\isasymunion}\ W{\isadigit{2}}\ {\isasymsubseteq}\ {\isacharbraceleft}H{\isacharcomma}F{\isacharbraceright}\ {\isasymunion}\ {\isacharquery}Wo{\isachardoublequoteclose}\isanewline
\ \ \ \ \ \ \ \ \isacommand{by}\isamarkupfalse%
\ blast\isanewline
\ \ \ \ \ \ \isacommand{then}\isamarkupfalse%
\ \isacommand{have}\isamarkupfalse%
\ {\isadigit{2}}{\isacharcolon}{\isachardoublequoteopen}{\isasymnot}{\isacharparenleft}sat{\isacharparenleft}{\isacharbraceleft}H{\isacharcomma}F{\isacharbraceright}\ {\isasymunion}\ {\isacharquery}Wo{\isacharparenright}{\isacharparenright}{\isachardoublequoteclose}\isanewline
\ \ \ \ \ \ \ \ \isacommand{using}\isamarkupfalse%
\ {\isacartoucheopen}{\isasymnot}sat\ {\isacharparenleft}{\isacharbraceleft}H{\isacharbraceright}\ {\isasymunion}\ W{\isadigit{2}}{\isacharparenright}{\isacartoucheclose}\ \isacommand{by}\isamarkupfalse%
\ {\isacharparenleft}rule\ sat{\isacharunderscore}subset{\isacharunderscore}ccontr{\isacharparenright}\isanewline
\ \ \ \ \ \ \isacommand{have}\isamarkupfalse%
\ {\isachardoublequoteopen}{\isasymnot}\ sat\ {\isacharparenleft}{\isacharbraceleft}G{\isacharcomma}F{\isacharbraceright}\ {\isasymunion}\ {\isacharquery}Wo{\isacharparenright}\ {\isasymand}\ {\isasymnot}\ sat\ {\isacharparenleft}{\isacharbraceleft}H{\isacharcomma}F{\isacharbraceright}\ {\isasymunion}\ {\isacharquery}Wo{\isacharparenright}{\isachardoublequoteclose}\isanewline
\ \ \ \ \ \ \ \ \isacommand{using}\isamarkupfalse%
\ {\isadigit{1}}\ {\isadigit{2}}\ \isacommand{by}\isamarkupfalse%
\ {\isacharparenleft}rule\ conjI{\isacharparenright}\isanewline
\ \ \ \ \ \ \isacommand{have}\isamarkupfalse%
\ {\isachardoublequoteopen}sat\ {\isacharparenleft}{\isacharbraceleft}G{\isacharcomma}F{\isacharbraceright}\ {\isasymunion}\ {\isacharquery}Wo{\isacharparenright}\ {\isasymor}\ sat\ {\isacharparenleft}{\isacharbraceleft}H{\isacharcomma}F{\isacharbraceright}\ {\isasymunion}\ {\isacharquery}Wo{\isacharparenright}{\isachardoublequoteclose}\isanewline
\ \ \ \ \ \ \ \ \isacommand{using}\isamarkupfalse%
\ assms{\isacharparenleft}{\isadigit{1}}{\isacharparenright}\ {\isacartoucheopen}Dis\ F\ G\ H{\isacartoucheclose}\ {\isacartoucheopen}F\ {\isasymin}\ W{\isacartoucheclose}\ {\isacartoucheopen}finite\ {\isacharquery}Wo{\isacartoucheclose}\ {\isacartoucheopen}{\isacharquery}Wo\ {\isasymsubseteq}\ W{\isacartoucheclose}\ \isacommand{by}\isamarkupfalse%
\ {\isacharparenleft}rule\ pcp{\isacharunderscore}colecComp{\isacharunderscore}DIS{\isacharunderscore}sat{\isacharparenright}\isanewline
\ \ \ \ \ \ \isacommand{then}\isamarkupfalse%
\ \isacommand{have}\isamarkupfalse%
\ {\isachardoublequoteopen}{\isasymnot}{\isasymnot}{\isacharparenleft}sat\ {\isacharparenleft}{\isacharbraceleft}G{\isacharcomma}F{\isacharbraceright}\ {\isasymunion}\ {\isacharquery}Wo{\isacharparenright}\ {\isasymor}\ sat\ {\isacharparenleft}{\isacharbraceleft}H{\isacharcomma}F{\isacharbraceright}\ {\isasymunion}\ {\isacharquery}Wo{\isacharparenright}{\isacharparenright}{\isachardoublequoteclose}\isanewline
\ \ \ \ \ \ \ \ \isacommand{by}\isamarkupfalse%
\ {\isacharparenleft}simp\ only{\isacharcolon}\ not{\isacharunderscore}not{\isacharparenright}\isanewline
\ \ \ \ \ \ \isacommand{then}\isamarkupfalse%
\ \isacommand{have}\isamarkupfalse%
\ {\isachardoublequoteopen}{\isasymnot}{\isacharparenleft}{\isasymnot}{\isacharparenleft}sat\ {\isacharparenleft}{\isacharbraceleft}G{\isacharcomma}F{\isacharbraceright}\ {\isasymunion}\ {\isacharquery}Wo{\isacharparenright}{\isacharparenright}\ {\isasymand}\ {\isasymnot}{\isacharparenleft}sat\ {\isacharparenleft}{\isacharbraceleft}H{\isacharcomma}F{\isacharbraceright}\ {\isasymunion}\ {\isacharquery}Wo{\isacharparenright}{\isacharparenright}{\isacharparenright}{\isachardoublequoteclose}\isanewline
\ \ \ \ \ \ \ \ \isacommand{by}\isamarkupfalse%
\ {\isacharparenleft}simp\ only{\isacharcolon}\ de{\isacharunderscore}Morgan{\isacharunderscore}disj\ simp{\isacharunderscore}thms{\isacharparenleft}{\isadigit{8}}{\isacharparenright}{\isacharparenright}\isanewline
\ \ \ \ \ \ \isacommand{thus}\isamarkupfalse%
\ {\isachardoublequoteopen}False{\isachardoublequoteclose}\isanewline
\ \ \ \ \ \ \ \ \isacommand{using}\isamarkupfalse%
\ {\isacartoucheopen}{\isasymnot}{\isacharparenleft}sat\ {\isacharparenleft}{\isacharbraceleft}G{\isacharcomma}F{\isacharbraceright}\ {\isasymunion}\ {\isacharquery}Wo{\isacharparenright}{\isacharparenright}\ {\isasymand}\ {\isasymnot}{\isacharparenleft}sat\ {\isacharparenleft}{\isacharbraceleft}H{\isacharcomma}F{\isacharbraceright}\ {\isasymunion}\ {\isacharquery}Wo{\isacharparenright}{\isacharparenright}{\isacartoucheclose}\ \isacommand{by}\isamarkupfalse%
\ {\isacharparenleft}rule\ notE{\isacharparenright}\isanewline
\ \ \ \ \isacommand{qed}\isamarkupfalse%
\isanewline
\ \ \isacommand{qed}\isamarkupfalse%
\isanewline
\isacommand{qed}\isamarkupfalse%
%
\endisatagproof
{\isafoldproof}%
%
\isadelimproof
%
\endisadelimproof
%
\begin{isamarkuptext}%
En resumen, con los lemas \isa{pcp{\isacharunderscore}colecComp{\isacharunderscore}bot}, \isa{pcp{\isacharunderscore}colecComp{\isacharunderscore}atoms}, \isa{pcp{\isacharunderscore}colecComp{\isacharunderscore}CON} y
  \isa{pcp{\isacharunderscore}colecComp{\isacharunderscore}DIS} podemos probar de manera detallada que la colección \isa{C} verifica la propiedad 
  de consistencia proposicional.%
\end{isamarkuptext}\isamarkuptrue%
\isacommand{lemma}\isamarkupfalse%
\ pcp{\isacharunderscore}colecComp{\isacharcolon}\ {\isachardoublequoteopen}pcp\ colecComp{\isachardoublequoteclose}\isanewline
%
\isadelimproof
%
\endisadelimproof
%
\isatagproof
\isacommand{proof}\isamarkupfalse%
\ {\isacharparenleft}rule\ pcp{\isacharunderscore}alt{\isadigit{2}}{\isacharparenright}\isanewline
\ \ \isacommand{show}\isamarkupfalse%
\ {\isachardoublequoteopen}{\isasymforall}W\ {\isasymin}\ colecComp{\isachardot}\ {\isasymbottom}\ {\isasymnotin}\ W\isanewline
\ \ \ \ \ \ \ \ {\isasymand}\ {\isacharparenleft}{\isasymforall}k{\isachardot}\ Atom\ k\ {\isasymin}\ W\ {\isasymlongrightarrow}\ \isactrlbold {\isasymnot}\ {\isacharparenleft}Atom\ k{\isacharparenright}\ {\isasymin}\ W\ {\isasymlongrightarrow}\ False{\isacharparenright}\isanewline
\ \ \ \ \ \ \ \ {\isasymand}\ {\isacharparenleft}{\isasymforall}F\ G\ H{\isachardot}\ Con\ F\ G\ H\ {\isasymlongrightarrow}\ F\ {\isasymin}\ W\ {\isasymlongrightarrow}\ {\isacharbraceleft}G{\isacharcomma}H{\isacharbraceright}\ {\isasymunion}\ W\ {\isasymin}\ colecComp{\isacharparenright}\isanewline
\ \ \ \ \ \ \ \ {\isasymand}\ {\isacharparenleft}{\isasymforall}F\ G\ H{\isachardot}\ Dis\ F\ G\ H\ {\isasymlongrightarrow}\ F\ {\isasymin}\ W\ {\isasymlongrightarrow}\ {\isacharbraceleft}G{\isacharbraceright}\ {\isasymunion}\ W\ {\isasymin}\ colecComp\ {\isasymor}\ {\isacharbraceleft}H{\isacharbraceright}\ {\isasymunion}\ W\ {\isasymin}\ colecComp{\isacharparenright}{\isachardoublequoteclose}\isanewline
\ \ \isacommand{proof}\isamarkupfalse%
\ {\isacharparenleft}rule\ ballI{\isacharparenright}\isanewline
\ \ \ \ \isacommand{fix}\isamarkupfalse%
\ W\isanewline
\ \ \ \ \isacommand{assume}\isamarkupfalse%
\ H{\isacharcolon}{\isachardoublequoteopen}W\ {\isasymin}\ colecComp{\isachardoublequoteclose}\isanewline
\ \ \ \ \isacommand{have}\isamarkupfalse%
\ C{\isadigit{1}}{\isacharcolon}{\isachardoublequoteopen}{\isasymbottom}\ {\isasymnotin}\ W{\isachardoublequoteclose}\isanewline
\ \ \ \ \ \ \isacommand{using}\isamarkupfalse%
\ H\ \isacommand{by}\isamarkupfalse%
\ {\isacharparenleft}rule\ pcp{\isacharunderscore}colecComp{\isacharunderscore}bot{\isacharparenright}\isanewline
\ \ \ \ \isacommand{have}\isamarkupfalse%
\ C{\isadigit{2}}{\isacharcolon}{\isachardoublequoteopen}{\isasymforall}k{\isachardot}\ Atom\ k\ {\isasymin}\ W\ {\isasymlongrightarrow}\ \isactrlbold {\isasymnot}\ {\isacharparenleft}Atom\ k{\isacharparenright}\ {\isasymin}\ W\ {\isasymlongrightarrow}\ False{\isachardoublequoteclose}\isanewline
\ \ \ \ \ \ \isacommand{using}\isamarkupfalse%
\ H\ \isacommand{by}\isamarkupfalse%
\ {\isacharparenleft}rule\ pcp{\isacharunderscore}colecComp{\isacharunderscore}atoms{\isacharparenright}\isanewline
\ \ \ \ \isacommand{have}\isamarkupfalse%
\ C{\isadigit{3}}{\isacharcolon}{\isachardoublequoteopen}{\isasymforall}F\ G\ H{\isachardot}\ Con\ F\ G\ H\ {\isasymlongrightarrow}\ F\ {\isasymin}\ W\ {\isasymlongrightarrow}\ {\isacharbraceleft}G{\isacharcomma}H{\isacharbraceright}\ {\isasymunion}\ W\ {\isasymin}\ colecComp{\isachardoublequoteclose}\isanewline
\ \ \ \ \ \ \isacommand{using}\isamarkupfalse%
\ H\ \isacommand{by}\isamarkupfalse%
\ {\isacharparenleft}rule\ pcp{\isacharunderscore}colecComp{\isacharunderscore}CON{\isacharparenright}\isanewline
\ \ \ \ \isacommand{have}\isamarkupfalse%
\ C{\isadigit{4}}{\isacharcolon}{\isachardoublequoteopen}{\isasymforall}F\ G\ H{\isachardot}\ Dis\ F\ G\ H\ {\isasymlongrightarrow}\ F\ {\isasymin}\ W\ {\isasymlongrightarrow}\ {\isacharbraceleft}G{\isacharbraceright}\ {\isasymunion}\ W\ {\isasymin}\ colecComp\ {\isasymor}\ {\isacharbraceleft}H{\isacharbraceright}\ {\isasymunion}\ W\ {\isasymin}\ colecComp{\isachardoublequoteclose}\isanewline
\ \ \ \ \ \ \isacommand{using}\isamarkupfalse%
\ H\ \isacommand{by}\isamarkupfalse%
\ {\isacharparenleft}rule\ pcp{\isacharunderscore}colecComp{\isacharunderscore}DIS{\isacharparenright}\isanewline
\ \ \ \ \isacommand{show}\isamarkupfalse%
\ {\isachardoublequoteopen}{\isasymbottom}\ {\isasymnotin}\ W\isanewline
\ \ \ \ \ \ \ \ \ \ {\isasymand}\ {\isacharparenleft}{\isasymforall}k{\isachardot}\ Atom\ k\ {\isasymin}\ W\ {\isasymlongrightarrow}\ \isactrlbold {\isasymnot}\ {\isacharparenleft}Atom\ k{\isacharparenright}\ {\isasymin}\ W\ {\isasymlongrightarrow}\ False{\isacharparenright}\isanewline
\ \ \ \ \ \ \ \ \ \ {\isasymand}\ {\isacharparenleft}{\isasymforall}F\ G\ H{\isachardot}\ Con\ F\ G\ H\ {\isasymlongrightarrow}\ F\ {\isasymin}\ W\ {\isasymlongrightarrow}\ {\isacharbraceleft}G{\isacharcomma}H{\isacharbraceright}\ {\isasymunion}\ W\ {\isasymin}\ colecComp{\isacharparenright}\isanewline
\ \ \ \ \ \ \ \ \ \ {\isasymand}\ {\isacharparenleft}{\isasymforall}F\ G\ H{\isachardot}\ Dis\ F\ G\ H\ {\isasymlongrightarrow}\ F\ {\isasymin}\ W\ {\isasymlongrightarrow}\ {\isacharbraceleft}G{\isacharbraceright}\ {\isasymunion}\ W\ {\isasymin}\ colecComp\ {\isasymor}\ {\isacharbraceleft}H{\isacharbraceright}\ {\isasymunion}\ W\ {\isasymin}\ colecComp{\isacharparenright}{\isachardoublequoteclose}\isanewline
\ \ \ \ \ \ \isacommand{using}\isamarkupfalse%
\ C{\isadigit{1}}\ C{\isadigit{2}}\ C{\isadigit{3}}\ C{\isadigit{4}}\ \isacommand{by}\isamarkupfalse%
\ {\isacharparenleft}iprover\ intro{\isacharcolon}\ conjI{\isacharparenright}\isanewline
\ \ \isacommand{qed}\isamarkupfalse%
\isanewline
\isacommand{qed}\isamarkupfalse%
%
\endisatagproof
{\isafoldproof}%
%
\isadelimproof
%
\endisadelimproof
%
\begin{isamarkuptext}%
Finalmente, mostremos la demostración del \isa{Teorema\ de\ Compacidad}.%
\end{isamarkuptext}\isamarkuptrue%
\isacommand{theorem}\isamarkupfalse%
\ prop{\isacharunderscore}Compactness{\isacharcolon}\isanewline
\ \ \isakeyword{fixes}\ W\ {\isacharcolon}{\isacharcolon}\ {\isachardoublequoteopen}{\isacharprime}a\ {\isacharcolon}{\isacharcolon}\ countable\ formula\ set{\isachardoublequoteclose}\isanewline
\ \ \isakeyword{assumes}\ {\isachardoublequoteopen}fin{\isacharunderscore}sat\ W{\isachardoublequoteclose}\isanewline
\ \ \isakeyword{shows}\ {\isachardoublequoteopen}sat\ W{\isachardoublequoteclose}\isanewline
%
\isadelimproof
%
\endisadelimproof
%
\isatagproof
\isacommand{proof}\isamarkupfalse%
\ {\isacharparenleft}rule\ pcp{\isacharunderscore}sat{\isacharparenright}\isanewline
\ \ \isacommand{show}\isamarkupfalse%
\ {\isachardoublequoteopen}W\ {\isasymin}\ colecComp{\isachardoublequoteclose}\isanewline
\ \ \ \ \isacommand{using}\isamarkupfalse%
\ assms\ \isacommand{by}\isamarkupfalse%
\ {\isacharparenleft}simp\ only{\isacharcolon}\ set{\isacharunderscore}in{\isacharunderscore}colecComp{\isacharparenright}\isanewline
\ \ \isacommand{show}\isamarkupfalse%
\ {\isachardoublequoteopen}pcp\ colecComp{\isachardoublequoteclose}\isanewline
\ \ \ \ \isacommand{by}\isamarkupfalse%
\ {\isacharparenleft}simp\ only{\isacharcolon}\ pcp{\isacharunderscore}colecComp{\isacharparenright}\isanewline
\isacommand{qed}\isamarkupfalse%
\isanewline
%
\endisatagproof
{\isafoldproof}%
%
\isadelimproof
%
\endisadelimproof
%
\isadelimtheory
%
\endisadelimtheory
%
\isatagtheory
%
\endisatagtheory
{\isafoldtheory}%
%
\isadelimtheory
%
\endisadelimtheory
%
\end{isabellebody}%
\endinput
%:%file=~/TFM/TFM/TeoremaEx.thy%:%
%:%19=13%:%
%:%23=15%:%
%:%32=17%:%
%:%44=19%:%
%:%45=20%:%
%:%46=21%:%
%:%47=22%:%
%:%48=23%:%
%:%49=24%:%
%:%50=25%:%
%:%51=26%:%
%:%52=27%:%
%:%53=28%:%
%:%54=29%:%
%:%55=30%:%
%:%56=31%:%
%:%57=32%:%
%:%58=33%:%
%:%59=34%:%
%:%59=35%:%
%:%60=36%:%
%:%61=37%:%
%:%62=38%:%
%:%63=39%:%
%:%64=40%:%
%:%65=41%:%
%:%66=42%:%
%:%67=43%:%
%:%68=44%:%
%:%69=45%:%
%:%70=46%:%
%:%71=47%:%
%:%72=48%:%
%:%73=49%:%
%:%74=50%:%
%:%75=51%:%
%:%76=52%:%
%:%77=53%:%
%:%78=54%:%
%:%79=55%:%
%:%80=56%:%
%:%82=58%:%
%:%83=58%:%
%:%84=59%:%
%:%85=60%:%
%:%88=63%:%
%:%89=64%:%
%:%90=65%:%
%:%91=66%:%
%:%92=67%:%
%:%93=68%:%
%:%94=69%:%
%:%95=70%:%
%:%96=71%:%
%:%97=72%:%
%:%98=73%:%
%:%99=74%:%
%:%100=75%:%
%:%101=76%:%
%:%102=77%:%
%:%103=78%:%
%:%104=79%:%
%:%105=80%:%
%:%106=81%:%
%:%107=82%:%
%:%108=83%:%
%:%109=84%:%
%:%110=85%:%
%:%112=87%:%
%:%113=87%:%
%:%114=88%:%
%:%115=89%:%
%:%116=90%:%
%:%123=91%:%
%:%124=91%:%
%:%125=92%:%
%:%126=92%:%
%:%127=93%:%
%:%128=93%:%
%:%129=94%:%
%:%130=94%:%
%:%131=95%:%
%:%132=95%:%
%:%133=96%:%
%:%134=96%:%
%:%135=97%:%
%:%136=97%:%
%:%137=98%:%
%:%138=99%:%
%:%139=99%:%
%:%140=100%:%
%:%141=100%:%
%:%142=100%:%
%:%143=101%:%
%:%144=102%:%
%:%145=102%:%
%:%146=103%:%
%:%147=103%:%
%:%148=104%:%
%:%149=104%:%
%:%150=105%:%
%:%151=105%:%
%:%152=106%:%
%:%153=106%:%
%:%154=107%:%
%:%155=107%:%
%:%156=107%:%
%:%157=108%:%
%:%158=108%:%
%:%159=109%:%
%:%160=109%:%
%:%161=110%:%
%:%162=110%:%
%:%163=111%:%
%:%164=111%:%
%:%165=112%:%
%:%166=112%:%
%:%167=113%:%
%:%168=113%:%
%:%169=113%:%
%:%170=114%:%
%:%171=114%:%
%:%172=115%:%
%:%173=115%:%
%:%174=116%:%
%:%175=116%:%
%:%176=117%:%
%:%186=119%:%
%:%188=121%:%
%:%189=121%:%
%:%196=122%:%
%:%197=122%:%
%:%198=123%:%
%:%199=123%:%
%:%200=124%:%
%:%201=124%:%
%:%202=124%:%
%:%203=125%:%
%:%204=125%:%
%:%205=125%:%
%:%206=126%:%
%:%207=126%:%
%:%216=128%:%
%:%217=129%:%
%:%218=130%:%
%:%219=131%:%
%:%220=132%:%
%:%221=133%:%
%:%222=134%:%
%:%223=135%:%
%:%225=137%:%
%:%226=137%:%
%:%229=138%:%
%:%233=138%:%
%:%243=140%:%
%:%244=141%:%
%:%245=142%:%
%:%246=143%:%
%:%247=144%:%
%:%248=145%:%
%:%249=146%:%
%:%250=147%:%
%:%251=148%:%
%:%252=149%:%
%:%253=150%:%
%:%254=151%:%
%:%255=152%:%
%:%257=154%:%
%:%258=154%:%
%:%265=155%:%
%:%266=155%:%
%:%267=156%:%
%:%268=156%:%
%:%269=157%:%
%:%270=158%:%
%:%271=158%:%
%:%272=159%:%
%:%273=159%:%
%:%274=159%:%
%:%275=160%:%
%:%276=161%:%
%:%277=161%:%
%:%278=162%:%
%:%279=162%:%
%:%280=163%:%
%:%281=163%:%
%:%282=164%:%
%:%283=164%:%
%:%284=165%:%
%:%285=165%:%
%:%286=166%:%
%:%287=166%:%
%:%288=166%:%
%:%289=167%:%
%:%290=167%:%
%:%291=168%:%
%:%292=168%:%
%:%293=169%:%
%:%294=169%:%
%:%295=170%:%
%:%296=170%:%
%:%297=171%:%
%:%298=171%:%
%:%299=172%:%
%:%300=172%:%
%:%301=172%:%
%:%302=173%:%
%:%303=173%:%
%:%304=174%:%
%:%305=174%:%
%:%306=175%:%
%:%307=175%:%
%:%308=176%:%
%:%318=178%:%
%:%320=180%:%
%:%321=180%:%
%:%324=181%:%
%:%328=181%:%
%:%329=181%:%
%:%338=183%:%
%:%339=184%:%
%:%341=186%:%
%:%342=186%:%
%:%343=187%:%
%:%344=188%:%
%:%347=189%:%
%:%351=189%:%
%:%352=189%:%
%:%353=189%:%
%:%362=191%:%
%:%363=192%:%
%:%364=193%:%
%:%365=194%:%
%:%366=195%:%
%:%367=196%:%
%:%368=197%:%
%:%369=198%:%
%:%370=199%:%
%:%371=200%:%
%:%372=201%:%
%:%373=202%:%
%:%374=203%:%
%:%375=204%:%
%:%376=205%:%
%:%377=206%:%
%:%378=207%:%
%:%379=208%:%
%:%380=209%:%
%:%381=210%:%
%:%382=211%:%
%:%383=212%:%
%:%384=213%:%
%:%385=214%:%
%:%386=215%:%
%:%387=216%:%
%:%388=217%:%
%:%389=218%:%
%:%390=219%:%
%:%391=220%:%
%:%392=221%:%
%:%393=222%:%
%:%394=223%:%
%:%395=224%:%
%:%396=225%:%
%:%397=226%:%
%:%398=227%:%
%:%399=228%:%
%:%400=229%:%
%:%401=230%:%
%:%402=231%:%
%:%403=232%:%
%:%404=233%:%
%:%406=235%:%
%:%407=235%:%
%:%414=236%:%
%:%415=236%:%
%:%416=237%:%
%:%417=237%:%
%:%418=238%:%
%:%419=238%:%
%:%420=239%:%
%:%421=239%:%
%:%422=239%:%
%:%423=240%:%
%:%424=240%:%
%:%425=241%:%
%:%426=241%:%
%:%427=241%:%
%:%428=242%:%
%:%429=242%:%
%:%430=243%:%
%:%431=243%:%
%:%432=243%:%
%:%433=244%:%
%:%434=244%:%
%:%435=245%:%
%:%436=245%:%
%:%437=245%:%
%:%438=246%:%
%:%439=246%:%
%:%440=247%:%
%:%441=247%:%
%:%442=248%:%
%:%443=248%:%
%:%444=249%:%
%:%445=249%:%
%:%446=250%:%
%:%447=250%:%
%:%448=251%:%
%:%449=251%:%
%:%450=252%:%
%:%451=252%:%
%:%452=252%:%
%:%453=253%:%
%:%454=253%:%
%:%455=254%:%
%:%456=254%:%
%:%457=255%:%
%:%458=255%:%
%:%459=256%:%
%:%460=256%:%
%:%461=256%:%
%:%462=257%:%
%:%463=257%:%
%:%464=258%:%
%:%465=258%:%
%:%466=258%:%
%:%467=259%:%
%:%468=259%:%
%:%469=260%:%
%:%470=260%:%
%:%471=260%:%
%:%472=261%:%
%:%473=261%:%
%:%474=262%:%
%:%475=262%:%
%:%476=262%:%
%:%477=263%:%
%:%478=263%:%
%:%479=264%:%
%:%480=264%:%
%:%481=264%:%
%:%482=265%:%
%:%483=265%:%
%:%484=266%:%
%:%485=266%:%
%:%486=266%:%
%:%487=267%:%
%:%488=267%:%
%:%489=267%:%
%:%490=268%:%
%:%491=268%:%
%:%492=268%:%
%:%493=269%:%
%:%494=269%:%
%:%495=270%:%
%:%505=272%:%
%:%507=274%:%
%:%508=274%:%
%:%515=275%:%
%:%516=275%:%
%:%517=276%:%
%:%518=276%:%
%:%519=277%:%
%:%520=277%:%
%:%521=278%:%
%:%522=278%:%
%:%523=278%:%
%:%524=279%:%
%:%525=279%:%
%:%526=280%:%
%:%527=280%:%
%:%528=281%:%
%:%529=281%:%
%:%530=282%:%
%:%531=282%:%
%:%532=283%:%
%:%533=283%:%
%:%534=283%:%
%:%535=283%:%
%:%536=284%:%
%:%537=284%:%
%:%546=286%:%
%:%547=287%:%
%:%548=288%:%
%:%549=289%:%
%:%550=290%:%
%:%551=291%:%
%:%552=292%:%
%:%553=293%:%
%:%554=294%:%
%:%556=296%:%
%:%557=296%:%
%:%559=298%:%
%:%560=299%:%
%:%561=300%:%
%:%562=301%:%
%:%563=302%:%
%:%564=303%:%
%:%565=304%:%
%:%566=305%:%
%:%567=306%:%
%:%568=307%:%
%:%569=308%:%
%:%570=309%:%
%:%571=310%:%
%:%572=311%:%
%:%573=312%:%
%:%574=313%:%
%:%576=315%:%
%:%577=315%:%
%:%580=316%:%
%:%584=316%:%
%:%585=316%:%
%:%586=317%:%
%:%587=317%:%
%:%588=318%:%
%:%589=318%:%
%:%590=319%:%
%:%591=319%:%
%:%592=320%:%
%:%593=320%:%
%:%594=320%:%
%:%595=321%:%
%:%596=321%:%
%:%597=322%:%
%:%598=322%:%
%:%599=322%:%
%:%600=323%:%
%:%601=323%:%
%:%602=324%:%
%:%603=324%:%
%:%604=325%:%
%:%605=325%:%
%:%606=326%:%
%:%616=328%:%
%:%618=330%:%
%:%619=330%:%
%:%622=331%:%
%:%626=331%:%
%:%627=331%:%
%:%628=331%:%
%:%637=333%:%
%:%638=334%:%
%:%639=335%:%
%:%640=336%:%
%:%641=337%:%
%:%642=338%:%
%:%643=339%:%
%:%644=340%:%
%:%645=341%:%
%:%646=342%:%
%:%647=343%:%
%:%648=344%:%
%:%649=345%:%
%:%650=346%:%
%:%651=347%:%
%:%652=348%:%
%:%654=350%:%
%:%655=350%:%
%:%656=351%:%
%:%657=352%:%
%:%664=353%:%
%:%665=353%:%
%:%666=354%:%
%:%667=354%:%
%:%668=355%:%
%:%669=355%:%
%:%670=355%:%
%:%671=356%:%
%:%672=356%:%
%:%673=356%:%
%:%674=357%:%
%:%675=357%:%
%:%676=358%:%
%:%677=358%:%
%:%678=358%:%
%:%679=359%:%
%:%680=359%:%
%:%681=360%:%
%:%682=360%:%
%:%683=361%:%
%:%684=361%:%
%:%685=362%:%
%:%695=364%:%
%:%697=366%:%
%:%698=366%:%
%:%699=367%:%
%:%702=368%:%
%:%706=368%:%
%:%707=368%:%
%:%708=368%:%
%:%717=370%:%
%:%718=371%:%
%:%719=372%:%
%:%720=373%:%
%:%721=374%:%
%:%722=375%:%
%:%723=376%:%
%:%724=377%:%
%:%725=378%:%
%:%726=379%:%
%:%727=380%:%
%:%728=381%:%
%:%729=382%:%
%:%730=383%:%
%:%731=384%:%
%:%732=385%:%
%:%733=386%:%
%:%734=387%:%
%:%735=388%:%
%:%736=389%:%
%:%737=390%:%
%:%738=391%:%
%:%739=392%:%
%:%740=393%:%
%:%741=394%:%
%:%742=395%:%
%:%743=396%:%
%:%744=397%:%
%:%745=398%:%
%:%746=399%:%
%:%747=400%:%
%:%748=401%:%
%:%749=402%:%
%:%750=403%:%
%:%751=404%:%
%:%753=406%:%
%:%754=406%:%
%:%755=407%:%
%:%756=408%:%
%:%757=409%:%
%:%760=410%:%
%:%764=410%:%
%:%765=410%:%
%:%766=411%:%
%:%767=411%:%
%:%768=412%:%
%:%769=412%:%
%:%770=413%:%
%:%771=413%:%
%:%772=414%:%
%:%773=414%:%
%:%774=415%:%
%:%775=415%:%
%:%776=416%:%
%:%777=416%:%
%:%778=417%:%
%:%779=417%:%
%:%780=417%:%
%:%781=418%:%
%:%782=418%:%
%:%783=419%:%
%:%784=419%:%
%:%785=419%:%
%:%786=420%:%
%:%787=420%:%
%:%788=421%:%
%:%789=421%:%
%:%790=422%:%
%:%791=422%:%
%:%792=423%:%
%:%793=423%:%
%:%794=423%:%
%:%795=424%:%
%:%796=424%:%
%:%797=425%:%
%:%798=425%:%
%:%799=425%:%
%:%800=426%:%
%:%801=426%:%
%:%802=427%:%
%:%803=427%:%
%:%804=427%:%
%:%805=428%:%
%:%806=428%:%
%:%807=429%:%
%:%808=429%:%
%:%809=429%:%
%:%810=430%:%
%:%811=430%:%
%:%812=431%:%
%:%813=431%:%
%:%814=432%:%
%:%815=432%:%
%:%816=432%:%
%:%817=433%:%
%:%818=433%:%
%:%819=434%:%
%:%820=434%:%
%:%821=434%:%
%:%822=435%:%
%:%823=435%:%
%:%824=435%:%
%:%825=436%:%
%:%826=436%:%
%:%827=437%:%
%:%828=437%:%
%:%829=438%:%
%:%830=438%:%
%:%831=438%:%
%:%832=439%:%
%:%833=439%:%
%:%834=440%:%
%:%835=440%:%
%:%836=441%:%
%:%837=441%:%
%:%838=441%:%
%:%839=442%:%
%:%840=442%:%
%:%841=443%:%
%:%842=443%:%
%:%843=444%:%
%:%844=444%:%
%:%845=445%:%
%:%846=445%:%
%:%847=445%:%
%:%848=446%:%
%:%849=446%:%
%:%850=447%:%
%:%851=447%:%
%:%852=448%:%
%:%853=448%:%
%:%854=448%:%
%:%855=449%:%
%:%856=449%:%
%:%857=450%:%
%:%858=450%:%
%:%859=450%:%
%:%860=451%:%
%:%861=451%:%
%:%862=451%:%
%:%863=452%:%
%:%864=452%:%
%:%865=452%:%
%:%866=453%:%
%:%867=453%:%
%:%868=454%:%
%:%869=454%:%
%:%870=455%:%
%:%880=457%:%
%:%882=459%:%
%:%883=459%:%
%:%884=460%:%
%:%885=461%:%
%:%886=462%:%
%:%889=463%:%
%:%893=463%:%
%:%894=463%:%
%:%895=464%:%
%:%896=464%:%
%:%897=465%:%
%:%898=465%:%
%:%899=466%:%
%:%900=466%:%
%:%901=466%:%
%:%902=467%:%
%:%903=467%:%
%:%904=467%:%
%:%905=467%:%
%:%906=468%:%
%:%907=468%:%
%:%908=468%:%
%:%909=469%:%
%:%910=469%:%
%:%911=470%:%
%:%912=470%:%
%:%913=470%:%
%:%914=471%:%
%:%915=471%:%
%:%916=472%:%
%:%917=472%:%
%:%918=472%:%
%:%919=473%:%
%:%920=473%:%
%:%934=475%:%
%:%946=477%:%
%:%947=478%:%
%:%948=479%:%
%:%949=480%:%
%:%950=481%:%
%:%951=482%:%
%:%952=483%:%
%:%953=484%:%
%:%954=485%:%
%:%954=486%:%
%:%955=487%:%
%:%956=488%:%
%:%957=489%:%
%:%961=491%:%
%:%962=492%:%
%:%963=493%:%
%:%964=494%:%
%:%965=495%:%
%:%966=496%:%
%:%967=497%:%
%:%968=498%:%
%:%969=499%:%
%:%970=500%:%
%:%971=501%:%
%:%972=502%:%
%:%973=503%:%
%:%974=504%:%
%:%975=505%:%
%:%976=506%:%
%:%977=507%:%
%:%978=508%:%
%:%979=509%:%
%:%980=510%:%
%:%981=511%:%
%:%982=512%:%
%:%983=513%:%
%:%984=514%:%
%:%985=515%:%
%:%986=516%:%
%:%987=517%:%
%:%989=519%:%
%:%990=519%:%
%:%991=520%:%
%:%992=521%:%
%:%993=522%:%
%:%994=523%:%
%:%995=524%:%
%:%1002=525%:%
%:%1003=525%:%
%:%1004=526%:%
%:%1005=526%:%
%:%1006=527%:%
%:%1007=527%:%
%:%1008=527%:%
%:%1009=527%:%
%:%1010=528%:%
%:%1011=528%:%
%:%1012=528%:%
%:%1013=529%:%
%:%1014=529%:%
%:%1015=530%:%
%:%1016=530%:%
%:%1017=531%:%
%:%1018=531%:%
%:%1019=531%:%
%:%1020=531%:%
%:%1021=532%:%
%:%1022=532%:%
%:%1023=533%:%
%:%1024=533%:%
%:%1025=534%:%
%:%1026=534%:%
%:%1027=535%:%
%:%1028=535%:%
%:%1029=536%:%
%:%1030=536%:%
%:%1031=537%:%
%:%1032=537%:%
%:%1033=538%:%
%:%1034=538%:%
%:%1035=539%:%
%:%1036=539%:%
%:%1037=539%:%
%:%1038=540%:%
%:%1039=540%:%
%:%1040=540%:%
%:%1041=541%:%
%:%1042=541%:%
%:%1043=542%:%
%:%1044=542%:%
%:%1045=542%:%
%:%1046=543%:%
%:%1047=543%:%
%:%1048=544%:%
%:%1049=544%:%
%:%1050=544%:%
%:%1051=545%:%
%:%1052=545%:%
%:%1053=546%:%
%:%1054=546%:%
%:%1055=546%:%
%:%1056=547%:%
%:%1057=547%:%
%:%1058=548%:%
%:%1059=548%:%
%:%1060=548%:%
%:%1061=549%:%
%:%1062=549%:%
%:%1063=550%:%
%:%1064=550%:%
%:%1065=551%:%
%:%1066=551%:%
%:%1067=552%:%
%:%1068=552%:%
%:%1069=552%:%
%:%1070=553%:%
%:%1080=555%:%
%:%1082=557%:%
%:%1083=557%:%
%:%1084=558%:%
%:%1085=559%:%
%:%1086=560%:%
%:%1087=561%:%
%:%1088=562%:%
%:%1095=563%:%
%:%1096=563%:%
%:%1097=564%:%
%:%1098=564%:%
%:%1099=564%:%
%:%1100=564%:%
%:%1101=565%:%
%:%1102=565%:%
%:%1103=565%:%
%:%1104=565%:%
%:%1105=566%:%
%:%1106=566%:%
%:%1107=567%:%
%:%1108=567%:%
%:%1109=568%:%
%:%1110=568%:%
%:%1111=569%:%
%:%1112=569%:%
%:%1113=570%:%
%:%1114=570%:%
%:%1115=570%:%
%:%1116=571%:%
%:%1117=571%:%
%:%1118=572%:%
%:%1119=572%:%
%:%1120=573%:%
%:%1121=573%:%
%:%1122=574%:%
%:%1123=574%:%
%:%1124=575%:%
%:%1125=575%:%
%:%1126=575%:%
%:%1127=576%:%
%:%1128=576%:%
%:%1129=576%:%
%:%1130=576%:%
%:%1131=577%:%
%:%1132=577%:%
%:%1133=577%:%
%:%1134=578%:%
%:%1135=578%:%
%:%1136=579%:%
%:%1137=579%:%
%:%1138=579%:%
%:%1139=580%:%
%:%1140=580%:%
%:%1141=581%:%
%:%1142=581%:%
%:%1143=581%:%
%:%1144=582%:%
%:%1145=582%:%
%:%1146=583%:%
%:%1147=583%:%
%:%1148=584%:%
%:%1149=584%:%
%:%1150=584%:%
%:%1151=584%:%
%:%1152=585%:%
%:%1153=585%:%
%:%1154=586%:%
%:%1155=586%:%
%:%1156=586%:%
%:%1157=586%:%
%:%1158=586%:%
%:%1159=587%:%
%:%1169=589%:%
%:%1170=590%:%
%:%1171=591%:%
%:%1172=592%:%
%:%1173=593%:%
%:%1174=594%:%
%:%1175=595%:%
%:%1176=596%:%
%:%1177=597%:%
%:%1178=598%:%
%:%1179=599%:%
%:%1180=600%:%
%:%1181=601%:%
%:%1182=602%:%
%:%1183=603%:%
%:%1184=604%:%
%:%1185=605%:%
%:%1186=606%:%
%:%1187=607%:%
%:%1188=608%:%
%:%1189=609%:%
%:%1190=610%:%
%:%1191=611%:%
%:%1192=612%:%
%:%1193=613%:%
%:%1194=614%:%
%:%1195=615%:%
%:%1196=616%:%
%:%1197=617%:%
%:%1198=618%:%
%:%1199=619%:%
%:%1200=620%:%
%:%1201=621%:%
%:%1202=622%:%
%:%1204=624%:%
%:%1205=624%:%
%:%1206=625%:%
%:%1207=626%:%
%:%1208=627%:%
%:%1209=628%:%
%:%1210=629%:%
%:%1217=630%:%
%:%1218=630%:%
%:%1219=631%:%
%:%1220=631%:%
%:%1221=632%:%
%:%1222=632%:%
%:%1223=633%:%
%:%1224=633%:%
%:%1225=633%:%
%:%1226=634%:%
%:%1227=634%:%
%:%1228=635%:%
%:%1229=635%:%
%:%1230=636%:%
%:%1231=636%:%
%:%1232=636%:%
%:%1233=637%:%
%:%1234=637%:%
%:%1235=637%:%
%:%1236=638%:%
%:%1237=638%:%
%:%1238=638%:%
%:%1239=639%:%
%:%1240=639%:%
%:%1241=640%:%
%:%1242=640%:%
%:%1243=640%:%
%:%1244=641%:%
%:%1245=641%:%
%:%1246=642%:%
%:%1247=642%:%
%:%1248=643%:%
%:%1249=643%:%
%:%1250=643%:%
%:%1251=644%:%
%:%1252=644%:%
%:%1253=645%:%
%:%1254=645%:%
%:%1255=645%:%
%:%1256=646%:%
%:%1257=646%:%
%:%1258=647%:%
%:%1259=647%:%
%:%1260=647%:%
%:%1261=648%:%
%:%1262=648%:%
%:%1263=649%:%
%:%1264=649%:%
%:%1265=650%:%
%:%1266=650%:%
%:%1267=650%:%
%:%1268=651%:%
%:%1269=651%:%
%:%1270=652%:%
%:%1271=652%:%
%:%1272=652%:%
%:%1273=653%:%
%:%1274=653%:%
%:%1275=653%:%
%:%1276=654%:%
%:%1277=654%:%
%:%1278=655%:%
%:%1279=655%:%
%:%1280=656%:%
%:%1281=656%:%
%:%1282=657%:%
%:%1283=657%:%
%:%1284=657%:%
%:%1285=658%:%
%:%1286=658%:%
%:%1287=659%:%
%:%1288=659%:%
%:%1289=659%:%
%:%1290=660%:%
%:%1291=660%:%
%:%1292=661%:%
%:%1293=661%:%
%:%1294=662%:%
%:%1295=663%:%
%:%1296=663%:%
%:%1297=664%:%
%:%1298=664%:%
%:%1299=664%:%
%:%1300=665%:%
%:%1301=666%:%
%:%1302=666%:%
%:%1303=667%:%
%:%1304=667%:%
%:%1305=667%:%
%:%1306=668%:%
%:%1307=668%:%
%:%1308=668%:%
%:%1309=669%:%
%:%1310=669%:%
%:%1311=669%:%
%:%1312=670%:%
%:%1313=670%:%
%:%1314=671%:%
%:%1315=671%:%
%:%1316=671%:%
%:%1317=672%:%
%:%1318=672%:%
%:%1319=673%:%
%:%1320=673%:%
%:%1321=674%:%
%:%1322=674%:%
%:%1323=674%:%
%:%1324=675%:%
%:%1325=675%:%
%:%1326=676%:%
%:%1327=676%:%
%:%1328=677%:%
%:%1329=677%:%
%:%1330=678%:%
%:%1331=678%:%
%:%1332=678%:%
%:%1333=679%:%
%:%1334=679%:%
%:%1335=679%:%
%:%1336=680%:%
%:%1337=680%:%
%:%1338=680%:%
%:%1339=681%:%
%:%1340=681%:%
%:%1341=681%:%
%:%1342=682%:%
%:%1343=682%:%
%:%1344=683%:%
%:%1345=683%:%
%:%1346=683%:%
%:%1347=684%:%
%:%1348=684%:%
%:%1349=684%:%
%:%1350=685%:%
%:%1351=685%:%
%:%1352=685%:%
%:%1353=686%:%
%:%1354=686%:%
%:%1355=686%:%
%:%1356=687%:%
%:%1357=687%:%
%:%1358=687%:%
%:%1359=688%:%
%:%1360=688%:%
%:%1361=689%:%
%:%1362=689%:%
%:%1363=689%:%
%:%1364=690%:%
%:%1374=692%:%
%:%1376=694%:%
%:%1377=694%:%
%:%1378=695%:%
%:%1379=696%:%
%:%1380=697%:%
%:%1381=698%:%
%:%1382=699%:%
%:%1389=700%:%
%:%1390=700%:%
%:%1391=701%:%
%:%1392=701%:%
%:%1393=702%:%
%:%1394=702%:%
%:%1395=702%:%
%:%1396=702%:%
%:%1397=703%:%
%:%1398=703%:%
%:%1399=703%:%
%:%1400=703%:%
%:%1401=704%:%
%:%1402=704%:%
%:%1403=704%:%
%:%1404=704%:%
%:%1405=704%:%
%:%1406=705%:%
%:%1407=705%:%
%:%1408=705%:%
%:%1409=706%:%
%:%1410=706%:%
%:%1411=706%:%
%:%1412=706%:%
%:%1413=707%:%
%:%1414=707%:%
%:%1415=707%:%
%:%1416=708%:%
%:%1417=708%:%
%:%1418=708%:%
%:%1419=708%:%
%:%1420=709%:%
%:%1421=709%:%
%:%1422=709%:%
%:%1423=709%:%
%:%1424=710%:%
%:%1434=712%:%
%:%1435=713%:%
%:%1436=714%:%
%:%1437=715%:%
%:%1438=716%:%
%:%1439=717%:%
%:%1440=718%:%
%:%1441=719%:%
%:%1442=720%:%
%:%1443=721%:%
%:%1444=722%:%
%:%1445=723%:%
%:%1446=724%:%
%:%1447=725%:%
%:%1448=726%:%
%:%1449=727%:%
%:%1450=728%:%
%:%1451=729%:%
%:%1452=730%:%
%:%1453=731%:%
%:%1454=732%:%
%:%1455=733%:%
%:%1457=735%:%
%:%1458=735%:%
%:%1459=736%:%
%:%1460=737%:%
%:%1461=738%:%
%:%1468=739%:%
%:%1469=739%:%
%:%1470=740%:%
%:%1471=740%:%
%:%1472=741%:%
%:%1473=741%:%
%:%1474=742%:%
%:%1475=742%:%
%:%1476=743%:%
%:%1477=743%:%
%:%1478=744%:%
%:%1479=744%:%
%:%1480=744%:%
%:%1481=745%:%
%:%1482=745%:%
%:%1483=745%:%
%:%1484=746%:%
%:%1485=746%:%
%:%1486=747%:%
%:%1487=747%:%
%:%1488=747%:%
%:%1489=748%:%
%:%1490=748%:%
%:%1491=749%:%
%:%1492=749%:%
%:%1493=750%:%
%:%1494=750%:%
%:%1495=751%:%
%:%1505=753%:%
%:%1507=755%:%
%:%1508=755%:%
%:%1509=756%:%
%:%1510=757%:%
%:%1511=758%:%
%:%1514=759%:%
%:%1518=759%:%
%:%1519=759%:%
%:%1520=759%:%
%:%1529=761%:%
%:%1530=762%:%
%:%1531=763%:%
%:%1532=764%:%
%:%1533=765%:%
%:%1534=766%:%
%:%1535=767%:%
%:%1536=768%:%
%:%1537=769%:%
%:%1538=770%:%
%:%1539=771%:%
%:%1540=772%:%
%:%1541=773%:%
%:%1542=774%:%
%:%1543=775%:%
%:%1544=776%:%
%:%1545=777%:%
%:%1546=778%:%
%:%1547=779%:%
%:%1548=780%:%
%:%1549=781%:%
%:%1550=782%:%
%:%1551=783%:%
%:%1552=784%:%
%:%1553=785%:%
%:%1554=786%:%
%:%1555=787%:%
%:%1556=788%:%
%:%1557=789%:%
%:%1558=790%:%
%:%1559=791%:%
%:%1560=792%:%
%:%1561=793%:%
%:%1562=794%:%
%:%1563=795%:%
%:%1564=796%:%
%:%1565=797%:%
%:%1566=798%:%
%:%1567=799%:%
%:%1568=800%:%
%:%1569=801%:%
%:%1570=802%:%
%:%1571=803%:%
%:%1572=804%:%
%:%1573=805%:%
%:%1574=806%:%
%:%1575=807%:%
%:%1576=808%:%
%:%1577=809%:%
%:%1578=810%:%
%:%1579=811%:%
%:%1580=812%:%
%:%1581=813%:%
%:%1582=814%:%
%:%1583=815%:%
%:%1584=816%:%
%:%1585=817%:%
%:%1586=818%:%
%:%1587=819%:%
%:%1588=820%:%
%:%1589=821%:%
%:%1590=822%:%
%:%1591=823%:%
%:%1592=824%:%
%:%1593=825%:%
%:%1594=826%:%
%:%1595=827%:%
%:%1596=828%:%
%:%1597=829%:%
%:%1598=830%:%
%:%1599=831%:%
%:%1600=832%:%
%:%1602=834%:%
%:%1603=834%:%
%:%1604=835%:%
%:%1605=836%:%
%:%1606=837%:%
%:%1607=838%:%
%:%1608=839%:%
%:%1615=840%:%
%:%1616=840%:%
%:%1617=841%:%
%:%1618=841%:%
%:%1619=842%:%
%:%1620=842%:%
%:%1621=843%:%
%:%1622=843%:%
%:%1623=843%:%
%:%1624=844%:%
%:%1625=844%:%
%:%1629=848%:%
%:%1630=849%:%
%:%1631=849%:%
%:%1632=849%:%
%:%1633=850%:%
%:%1634=850%:%
%:%1635=850%:%
%:%1638=853%:%
%:%1639=854%:%
%:%1640=854%:%
%:%1641=854%:%
%:%1642=855%:%
%:%1643=855%:%
%:%1644=855%:%
%:%1645=856%:%
%:%1646=856%:%
%:%1647=857%:%
%:%1648=857%:%
%:%1649=858%:%
%:%1650=858%:%
%:%1651=858%:%
%:%1652=859%:%
%:%1653=859%:%
%:%1654=860%:%
%:%1655=860%:%
%:%1656=860%:%
%:%1657=861%:%
%:%1658=861%:%
%:%1659=862%:%
%:%1660=862%:%
%:%1661=863%:%
%:%1662=863%:%
%:%1663=864%:%
%:%1664=864%:%
%:%1665=865%:%
%:%1666=865%:%
%:%1667=866%:%
%:%1668=866%:%
%:%1669=867%:%
%:%1670=867%:%
%:%1671=868%:%
%:%1672=868%:%
%:%1673=869%:%
%:%1674=869%:%
%:%1675=869%:%
%:%1676=870%:%
%:%1677=870%:%
%:%1678=870%:%
%:%1679=871%:%
%:%1680=871%:%
%:%1681=871%:%
%:%1682=872%:%
%:%1683=872%:%
%:%1684=872%:%
%:%1685=873%:%
%:%1686=873%:%
%:%1687=873%:%
%:%1688=874%:%
%:%1689=874%:%
%:%1690=875%:%
%:%1691=875%:%
%:%1692=876%:%
%:%1693=876%:%
%:%1694=876%:%
%:%1695=877%:%
%:%1696=877%:%
%:%1697=877%:%
%:%1698=878%:%
%:%1699=878%:%
%:%1700=879%:%
%:%1701=879%:%
%:%1702=880%:%
%:%1703=880%:%
%:%1704=880%:%
%:%1705=881%:%
%:%1706=881%:%
%:%1707=882%:%
%:%1708=882%:%
%:%1709=883%:%
%:%1710=883%:%
%:%1711=883%:%
%:%1712=884%:%
%:%1713=884%:%
%:%1714=884%:%
%:%1715=885%:%
%:%1716=885%:%
%:%1717=886%:%
%:%1718=886%:%
%:%1719=886%:%
%:%1720=887%:%
%:%1721=887%:%
%:%1722=888%:%
%:%1723=888%:%
%:%1724=888%:%
%:%1725=889%:%
%:%1726=889%:%
%:%1727=890%:%
%:%1728=890%:%
%:%1729=891%:%
%:%1730=891%:%
%:%1731=891%:%
%:%1732=892%:%
%:%1733=892%:%
%:%1734=892%:%
%:%1735=893%:%
%:%1736=893%:%
%:%1737=894%:%
%:%1738=894%:%
%:%1739=894%:%
%:%1740=895%:%
%:%1741=895%:%
%:%1742=896%:%
%:%1743=896%:%
%:%1744=897%:%
%:%1745=897%:%
%:%1746=897%:%
%:%1747=898%:%
%:%1748=898%:%
%:%1749=899%:%
%:%1750=899%:%
%:%1751=900%:%
%:%1752=900%:%
%:%1753=901%:%
%:%1754=901%:%
%:%1755=901%:%
%:%1756=902%:%
%:%1757=902%:%
%:%1758=903%:%
%:%1759=903%:%
%:%1760=904%:%
%:%1761=904%:%
%:%1762=905%:%
%:%1763=905%:%
%:%1764=906%:%
%:%1765=906%:%
%:%1766=907%:%
%:%1767=907%:%
%:%1768=908%:%
%:%1769=908%:%
%:%1770=909%:%
%:%1771=909%:%
%:%1772=910%:%
%:%1773=910%:%
%:%1774=910%:%
%:%1775=911%:%
%:%1776=911%:%
%:%1777=911%:%
%:%1778=912%:%
%:%1779=912%:%
%:%1780=912%:%
%:%1781=913%:%
%:%1782=913%:%
%:%1783=913%:%
%:%1784=914%:%
%:%1785=914%:%
%:%1786=914%:%
%:%1787=915%:%
%:%1788=915%:%
%:%1789=916%:%
%:%1790=916%:%
%:%1791=917%:%
%:%1792=917%:%
%:%1793=918%:%
%:%1794=918%:%
%:%1795=919%:%
%:%1796=919%:%
%:%1797=920%:%
%:%1798=920%:%
%:%1799=921%:%
%:%1800=921%:%
%:%1801=921%:%
%:%1802=922%:%
%:%1803=922%:%
%:%1804=923%:%
%:%1805=923%:%
%:%1806=923%:%
%:%1807=924%:%
%:%1808=924%:%
%:%1809=924%:%
%:%1810=925%:%
%:%1811=925%:%
%:%1812=926%:%
%:%1813=926%:%
%:%1814=927%:%
%:%1815=927%:%
%:%1816=928%:%
%:%1817=928%:%
%:%1818=929%:%
%:%1819=929%:%
%:%1820=930%:%
%:%1821=930%:%
%:%1822=931%:%
%:%1823=931%:%
%:%1824=932%:%
%:%1825=932%:%
%:%1826=933%:%
%:%1827=933%:%
%:%1828=933%:%
%:%1829=934%:%
%:%1830=934%:%
%:%1831=935%:%
%:%1832=935%:%
%:%1833=935%:%
%:%1834=936%:%
%:%1835=936%:%
%:%1836=936%:%
%:%1837=937%:%
%:%1838=937%:%
%:%1839=938%:%
%:%1840=938%:%
%:%1841=939%:%
%:%1842=939%:%
%:%1843=940%:%
%:%1844=940%:%
%:%1845=941%:%
%:%1846=941%:%
%:%1847=942%:%
%:%1848=942%:%
%:%1849=943%:%
%:%1850=943%:%
%:%1853=946%:%
%:%1854=947%:%
%:%1855=947%:%
%:%1856=947%:%
%:%1857=948%:%
%:%1867=950%:%
%:%1869=952%:%
%:%1870=952%:%
%:%1871=953%:%
%:%1872=954%:%
%:%1873=955%:%
%:%1874=956%:%
%:%1875=957%:%
%:%1882=958%:%
%:%1883=958%:%
%:%1884=959%:%
%:%1885=959%:%
%:%1886=960%:%
%:%1887=960%:%
%:%1888=960%:%
%:%1889=960%:%
%:%1890=961%:%
%:%1891=961%:%
%:%1892=962%:%
%:%1893=962%:%
%:%1894=963%:%
%:%1895=964%:%
%:%1896=965%:%
%:%1897=966%:%
%:%1898=966%:%
%:%1899=967%:%
%:%1900=967%:%
%:%1901=967%:%
%:%1902=967%:%
%:%1903=968%:%
%:%1904=968%:%
%:%1905=968%:%
%:%1906=969%:%
%:%1916=971%:%
%:%1917=972%:%
%:%1918=973%:%
%:%1919=974%:%
%:%1920=975%:%
%:%1921=976%:%
%:%1922=977%:%
%:%1923=978%:%
%:%1924=979%:%
%:%1925=980%:%
%:%1926=981%:%
%:%1927=982%:%
%:%1928=983%:%
%:%1929=984%:%
%:%1930=985%:%
%:%1931=986%:%
%:%1933=988%:%
%:%1934=988%:%
%:%1935=989%:%
%:%1936=990%:%
%:%1937=991%:%
%:%1940=992%:%
%:%1944=992%:%
%:%1945=992%:%
%:%1946=993%:%
%:%1947=993%:%
%:%1948=994%:%
%:%1949=994%:%
%:%1950=995%:%
%:%1951=995%:%
%:%1952=995%:%
%:%1953=996%:%
%:%1954=996%:%
%:%1955=997%:%
%:%1956=997%:%
%:%1957=997%:%
%:%1958=998%:%
%:%1959=998%:%
%:%1960=999%:%
%:%1961=999%:%
%:%1962=1000%:%
%:%1963=1000%:%
%:%1964=1001%:%
%:%1965=1001%:%
%:%1966=1002%:%
%:%1967=1002%:%
%:%1968=1003%:%
%:%1969=1003%:%
%:%1970=1003%:%
%:%1971=1004%:%
%:%1972=1004%:%
%:%1973=1004%:%
%:%1974=1005%:%
%:%1975=1005%:%
%:%1976=1005%:%
%:%1977=1006%:%
%:%1978=1006%:%
%:%1979=1007%:%
%:%1980=1007%:%
%:%1981=1007%:%
%:%1982=1008%:%
%:%1983=1008%:%
%:%1984=1009%:%
%:%1985=1009%:%
%:%1986=1010%:%
%:%1987=1010%:%
%:%1988=1011%:%
%:%1998=1013%:%
%:%1999=1014%:%
%:%2000=1015%:%
%:%2001=1016%:%
%:%2002=1017%:%
%:%2003=1018%:%
%:%2004=1019%:%
%:%2005=1020%:%
%:%2006=1021%:%
%:%2007=1022%:%
%:%2008=1023%:%
%:%2009=1024%:%
%:%2010=1025%:%
%:%2011=1026%:%
%:%2012=1027%:%
%:%2013=1028%:%
%:%2014=1029%:%
%:%2015=1030%:%
%:%2016=1031%:%
%:%2017=1032%:%
%:%2018=1033%:%
%:%2019=1034%:%
%:%2020=1035%:%
%:%2021=1036%:%
%:%2022=1037%:%
%:%2023=1038%:%
%:%2024=1039%:%
%:%2025=1040%:%
%:%2026=1041%:%
%:%2027=1042%:%
%:%2028=1043%:%
%:%2029=1044%:%
%:%2030=1045%:%
%:%2031=1046%:%
%:%2032=1047%:%
%:%2033=1048%:%
%:%2035=1050%:%
%:%2036=1050%:%
%:%2037=1051%:%
%:%2038=1052%:%
%:%2039=1053%:%
%:%2040=1054%:%
%:%2047=1055%:%
%:%2048=1055%:%
%:%2049=1056%:%
%:%2050=1056%:%
%:%2051=1057%:%
%:%2052=1057%:%
%:%2053=1058%:%
%:%2054=1058%:%
%:%2055=1058%:%
%:%2056=1059%:%
%:%2057=1059%:%
%:%2058=1060%:%
%:%2059=1060%:%
%:%2060=1061%:%
%:%2061=1061%:%
%:%2062=1061%:%
%:%2063=1062%:%
%:%2064=1062%:%
%:%2065=1063%:%
%:%2066=1063%:%
%:%2067=1063%:%
%:%2068=1064%:%
%:%2069=1064%:%
%:%2070=1065%:%
%:%2071=1065%:%
%:%2072=1065%:%
%:%2073=1066%:%
%:%2074=1066%:%
%:%2075=1067%:%
%:%2076=1067%:%
%:%2077=1067%:%
%:%2078=1068%:%
%:%2079=1068%:%
%:%2080=1069%:%
%:%2081=1069%:%
%:%2082=1069%:%
%:%2083=1070%:%
%:%2084=1070%:%
%:%2085=1071%:%
%:%2086=1071%:%
%:%2087=1071%:%
%:%2088=1072%:%
%:%2089=1072%:%
%:%2090=1073%:%
%:%2091=1073%:%
%:%2092=1073%:%
%:%2093=1074%:%
%:%2094=1074%:%
%:%2095=1074%:%
%:%2096=1075%:%
%:%2097=1075%:%
%:%2098=1075%:%
%:%2099=1076%:%
%:%2100=1076%:%
%:%2101=1077%:%
%:%2102=1077%:%
%:%2103=1077%:%
%:%2104=1078%:%
%:%2105=1078%:%
%:%2106=1079%:%
%:%2107=1079%:%
%:%2108=1079%:%
%:%2109=1080%:%
%:%2110=1080%:%
%:%2111=1080%:%
%:%2112=1081%:%
%:%2113=1081%:%
%:%2114=1082%:%
%:%2115=1082%:%
%:%2116=1083%:%
%:%2117=1083%:%
%:%2118=1083%:%
%:%2119=1084%:%
%:%2120=1084%:%
%:%2121=1084%:%
%:%2122=1085%:%
%:%2123=1085%:%
%:%2124=1085%:%
%:%2125=1086%:%
%:%2126=1086%:%
%:%2127=1087%:%
%:%2128=1087%:%
%:%2129=1087%:%
%:%2130=1088%:%
%:%2131=1088%:%
%:%2132=1088%:%
%:%2133=1089%:%
%:%2134=1089%:%
%:%2135=1090%:%
%:%2136=1090%:%
%:%2137=1091%:%
%:%2138=1091%:%
%:%2139=1092%:%
%:%2140=1092%:%
%:%2141=1093%:%
%:%2142=1093%:%
%:%2143=1094%:%
%:%2144=1094%:%
%:%2145=1094%:%
%:%2146=1095%:%
%:%2147=1095%:%
%:%2148=1096%:%
%:%2149=1096%:%
%:%2150=1097%:%
%:%2151=1097%:%
%:%2152=1097%:%
%:%2153=1098%:%
%:%2163=1100%:%
%:%2165=1102%:%
%:%2166=1102%:%
%:%2167=1103%:%
%:%2168=1104%:%
%:%2169=1105%:%
%:%2170=1106%:%
%:%2177=1107%:%
%:%2178=1107%:%
%:%2179=1108%:%
%:%2180=1108%:%
%:%2181=1108%:%
%:%2182=1109%:%
%:%2183=1110%:%
%:%2184=1110%:%
%:%2185=1111%:%
%:%2186=1111%:%
%:%2187=1112%:%
%:%2188=1112%:%
%:%2189=1112%:%
%:%2190=1112%:%
%:%2191=1113%:%
%:%2192=1113%:%
%:%2193=1114%:%
%:%2194=1114%:%
%:%2195=1114%:%
%:%2196=1115%:%
%:%2197=1115%:%
%:%2198=1115%:%
%:%2199=1115%:%
%:%2200=1116%:%
%:%2201=1116%:%
%:%2202=1116%:%
%:%2203=1117%:%
%:%2204=1117%:%
%:%2205=1117%:%
%:%2206=1118%:%
%:%2207=1118%:%
%:%2208=1118%:%
%:%2209=1118%:%
%:%2210=1118%:%
%:%2211=1119%:%
%:%2226=1121%:%
%:%2238=1123%:%
%:%2239=1124%:%
%:%2240=1125%:%
%:%2241=1126%:%
%:%2242=1127%:%
%:%2243=1128%:%
%:%2244=1129%:%
%:%2245=1130%:%
%:%2246=1131%:%
%:%2247=1132%:%
%:%2248=1133%:%
%:%2249=1134%:%
%:%2250=1135%:%
%:%2251=1136%:%
%:%2252=1137%:%
%:%2253=1138%:%
%:%2254=1139%:%
%:%2255=1140%:%
%:%2256=1141%:%
%:%2257=1142%:%
%:%2258=1143%:%
%:%2259=1144%:%
%:%2260=1145%:%
%:%2261=1146%:%
%:%2262=1147%:%
%:%2263=1148%:%
%:%2264=1149%:%
%:%2265=1150%:%
%:%2266=1151%:%
%:%2267=1152%:%
%:%2268=1153%:%
%:%2269=1154%:%
%:%2270=1155%:%
%:%2271=1156%:%
%:%2272=1157%:%
%:%2273=1158%:%
%:%2274=1159%:%
%:%2275=1160%:%
%:%2276=1161%:%
%:%2277=1162%:%
%:%2278=1163%:%
%:%2279=1164%:%
%:%2280=1165%:%
%:%2281=1166%:%
%:%2282=1167%:%
%:%2283=1168%:%
%:%2284=1169%:%
%:%2285=1170%:%
%:%2286=1171%:%
%:%2287=1172%:%
%:%2288=1173%:%
%:%2289=1174%:%
%:%2290=1175%:%
%:%2291=1176%:%
%:%2292=1177%:%
%:%2293=1178%:%
%:%2294=1179%:%
%:%2295=1180%:%
%:%2296=1181%:%
%:%2297=1182%:%
%:%2298=1183%:%
%:%2300=1185%:%
%:%2301=1185%:%
%:%2304=1186%:%
%:%2308=1186%:%
%:%2309=1186%:%
%:%2318=1188%:%
%:%2320=1190%:%
%:%2321=1190%:%
%:%2322=1191%:%
%:%2323=1192%:%
%:%2330=1193%:%
%:%2331=1193%:%
%:%2332=1194%:%
%:%2333=1194%:%
%:%2334=1195%:%
%:%2335=1195%:%
%:%2336=1196%:%
%:%2337=1196%:%
%:%2338=1197%:%
%:%2339=1197%:%
%:%2340=1198%:%
%:%2341=1198%:%
%:%2342=1199%:%
%:%2343=1199%:%
%:%2344=1200%:%
%:%2345=1200%:%
%:%2346=1201%:%
%:%2347=1201%:%
%:%2348=1201%:%
%:%2349=1202%:%
%:%2350=1202%:%
%:%2351=1203%:%
%:%2352=1203%:%
%:%2353=1204%:%
%:%2354=1204%:%
%:%2355=1204%:%
%:%2356=1205%:%
%:%2357=1205%:%
%:%2358=1206%:%
%:%2359=1206%:%
%:%2360=1206%:%
%:%2361=1207%:%
%:%2362=1207%:%
%:%2363=1208%:%
%:%2364=1208%:%
%:%2365=1209%:%
%:%2366=1209%:%
%:%2367=1210%:%
%:%2368=1210%:%
%:%2369=1211%:%
%:%2370=1211%:%
%:%2371=1212%:%
%:%2372=1212%:%
%:%2373=1213%:%
%:%2374=1213%:%
%:%2375=1214%:%
%:%2376=1214%:%
%:%2377=1215%:%
%:%2378=1215%:%
%:%2379=1215%:%
%:%2380=1216%:%
%:%2381=1216%:%
%:%2382=1217%:%
%:%2383=1217%:%
%:%2384=1217%:%
%:%2385=1218%:%
%:%2386=1218%:%
%:%2387=1219%:%
%:%2388=1219%:%
%:%2389=1220%:%
%:%2390=1220%:%
%:%2391=1220%:%
%:%2392=1221%:%
%:%2393=1221%:%
%:%2394=1222%:%
%:%2395=1222%:%
%:%2396=1222%:%
%:%2397=1223%:%
%:%2398=1223%:%
%:%2399=1224%:%
%:%2400=1224%:%
%:%2401=1224%:%
%:%2402=1225%:%
%:%2403=1225%:%
%:%2404=1225%:%
%:%2405=1226%:%
%:%2406=1226%:%
%:%2407=1227%:%
%:%2408=1227%:%
%:%2409=1228%:%
%:%2410=1228%:%
%:%2411=1228%:%
%:%2412=1229%:%
%:%2413=1229%:%
%:%2414=1229%:%
%:%2415=1230%:%
%:%2416=1230%:%
%:%2417=1231%:%
%:%2418=1231%:%
%:%2419=1231%:%
%:%2420=1232%:%
%:%2421=1232%:%
%:%2422=1233%:%
%:%2423=1233%:%
%:%2424=1234%:%
%:%2425=1234%:%
%:%2426=1235%:%
%:%2427=1235%:%
%:%2428=1236%:%
%:%2429=1236%:%
%:%2430=1237%:%
%:%2431=1237%:%
%:%2432=1237%:%
%:%2433=1238%:%
%:%2434=1238%:%
%:%2435=1239%:%
%:%2436=1239%:%
%:%2437=1240%:%
%:%2438=1240%:%
%:%2439=1241%:%
%:%2440=1241%:%
%:%2441=1242%:%
%:%2442=1242%:%
%:%2443=1242%:%
%:%2444=1243%:%
%:%2445=1243%:%
%:%2446=1243%:%
%:%2447=1244%:%
%:%2448=1244%:%
%:%2449=1245%:%
%:%2450=1245%:%
%:%2451=1245%:%
%:%2452=1246%:%
%:%2453=1246%:%
%:%2454=1247%:%
%:%2455=1247%:%
%:%2456=1248%:%
%:%2457=1248%:%
%:%2458=1249%:%
%:%2459=1249%:%
%:%2460=1249%:%
%:%2461=1250%:%
%:%2462=1250%:%
%:%2463=1251%:%
%:%2464=1251%:%
%:%2465=1251%:%
%:%2466=1252%:%
%:%2467=1252%:%
%:%2468=1253%:%
%:%2469=1253%:%
%:%2470=1254%:%
%:%2471=1254%:%
%:%2472=1254%:%
%:%2473=1255%:%
%:%2474=1255%:%
%:%2475=1256%:%
%:%2476=1256%:%
%:%2477=1256%:%
%:%2478=1257%:%
%:%2479=1257%:%
%:%2480=1258%:%
%:%2481=1258%:%
%:%2482=1259%:%
%:%2483=1259%:%
%:%2484=1260%:%
%:%2485=1260%:%
%:%2486=1261%:%
%:%2487=1261%:%
%:%2488=1261%:%
%:%2489=1262%:%
%:%2490=1262%:%
%:%2491=1263%:%
%:%2492=1263%:%
%:%2493=1263%:%
%:%2494=1264%:%
%:%2495=1264%:%
%:%2496=1265%:%
%:%2497=1265%:%
%:%2498=1266%:%
%:%2499=1266%:%
%:%2500=1266%:%
%:%2501=1267%:%
%:%2502=1267%:%
%:%2503=1268%:%
%:%2504=1268%:%
%:%2505=1268%:%
%:%2506=1269%:%
%:%2507=1269%:%
%:%2508=1270%:%
%:%2509=1270%:%
%:%2510=1271%:%
%:%2511=1271%:%
%:%2512=1272%:%
%:%2513=1272%:%
%:%2514=1273%:%
%:%2515=1273%:%
%:%2516=1273%:%
%:%2517=1274%:%
%:%2518=1274%:%
%:%2519=1274%:%
%:%2520=1275%:%
%:%2521=1275%:%
%:%2522=1276%:%
%:%2523=1276%:%
%:%2524=1277%:%
%:%2525=1277%:%
%:%2526=1277%:%
%:%2527=1278%:%
%:%2528=1278%:%
%:%2529=1279%:%
%:%2530=1279%:%
%:%2531=1279%:%
%:%2532=1280%:%
%:%2533=1280%:%
%:%2534=1281%:%
%:%2535=1281%:%
%:%2536=1282%:%
%:%2537=1282%:%
%:%2538=1283%:%
%:%2539=1283%:%
%:%2540=1284%:%
%:%2541=1284%:%
%:%2542=1285%:%
%:%2543=1285%:%
%:%2544=1285%:%
%:%2545=1286%:%
%:%2546=1286%:%
%:%2547=1287%:%
%:%2548=1287%:%
%:%2549=1287%:%
%:%2550=1288%:%
%:%2551=1288%:%
%:%2552=1289%:%
%:%2553=1289%:%
%:%2554=1290%:%
%:%2555=1290%:%
%:%2556=1290%:%
%:%2557=1291%:%
%:%2558=1291%:%
%:%2559=1292%:%
%:%2560=1292%:%
%:%2561=1292%:%
%:%2562=1293%:%
%:%2563=1293%:%
%:%2564=1294%:%
%:%2565=1294%:%
%:%2566=1295%:%
%:%2567=1295%:%
%:%2568=1296%:%
%:%2569=1296%:%
%:%2570=1297%:%
%:%2571=1297%:%
%:%2572=1297%:%
%:%2573=1298%:%
%:%2574=1298%:%
%:%2575=1299%:%
%:%2576=1299%:%
%:%2577=1300%:%
%:%2578=1300%:%
%:%2579=1300%:%
%:%2580=1301%:%
%:%2581=1301%:%
%:%2582=1302%:%
%:%2583=1302%:%
%:%2584=1302%:%
%:%2585=1303%:%
%:%2586=1303%:%
%:%2587=1304%:%
%:%2588=1304%:%
%:%2589=1305%:%
%:%2590=1305%:%
%:%2591=1306%:%
%:%2601=1308%:%
%:%2602=1309%:%
%:%2603=1310%:%
%:%2604=1311%:%
%:%2605=1312%:%
%:%2606=1313%:%
%:%2607=1314%:%
%:%2608=1315%:%
%:%2609=1316%:%
%:%2610=1317%:%
%:%2611=1318%:%
%:%2612=1319%:%
%:%2613=1320%:%
%:%2614=1321%:%
%:%2615=1322%:%
%:%2616=1323%:%
%:%2617=1324%:%
%:%2618=1325%:%
%:%2619=1326%:%
%:%2620=1327%:%
%:%2621=1328%:%
%:%2622=1329%:%
%:%2623=1330%:%
%:%2624=1331%:%
%:%2625=1332%:%
%:%2626=1333%:%
%:%2627=1334%:%
%:%2629=1336%:%
%:%2630=1336%:%
%:%2631=1337%:%
%:%2632=1338%:%
%:%2633=1339%:%
%:%2640=1340%:%
%:%2641=1340%:%
%:%2642=1341%:%
%:%2643=1341%:%
%:%2644=1342%:%
%:%2645=1342%:%
%:%2646=1342%:%
%:%2647=1343%:%
%:%2648=1343%:%
%:%2649=1343%:%
%:%2650=1344%:%
%:%2651=1344%:%
%:%2652=1344%:%
%:%2653=1345%:%
%:%2654=1345%:%
%:%2655=1346%:%
%:%2656=1346%:%
%:%2657=1346%:%
%:%2658=1347%:%
%:%2659=1347%:%
%:%2660=1348%:%
%:%2661=1348%:%
%:%2662=1348%:%
%:%2663=1349%:%
%:%2664=1349%:%
%:%2665=1350%:%
%:%2666=1350%:%
%:%2667=1350%:%
%:%2668=1351%:%
%:%2669=1351%:%
%:%2670=1352%:%
%:%2671=1352%:%
%:%2672=1352%:%
%:%2673=1353%:%
%:%2674=1353%:%
%:%2675=1354%:%
%:%2676=1354%:%
%:%2677=1354%:%
%:%2678=1355%:%
%:%2679=1355%:%
%:%2680=1356%:%
%:%2681=1356%:%
%:%2682=1356%:%
%:%2683=1357%:%
%:%2684=1357%:%
%:%2685=1358%:%
%:%2686=1358%:%
%:%2687=1359%:%
%:%2688=1359%:%
%:%2689=1360%:%
%:%2690=1360%:%
%:%2691=1361%:%
%:%2692=1361%:%
%:%2693=1361%:%
%:%2694=1362%:%
%:%2695=1362%:%
%:%2696=1363%:%
%:%2697=1363%:%
%:%2698=1364%:%
%:%2699=1364%:%
%:%2700=1365%:%
%:%2701=1365%:%
%:%2702=1365%:%
%:%2703=1366%:%
%:%2704=1366%:%
%:%2705=1367%:%
%:%2706=1367%:%
%:%2707=1367%:%
%:%2708=1368%:%
%:%2709=1368%:%
%:%2710=1369%:%
%:%2711=1369%:%
%:%2712=1369%:%
%:%2713=1370%:%
%:%2714=1370%:%
%:%2715=1371%:%
%:%2716=1371%:%
%:%2717=1371%:%
%:%2718=1372%:%
%:%2719=1372%:%
%:%2720=1372%:%
%:%2721=1373%:%
%:%2722=1373%:%
%:%2723=1374%:%
%:%2724=1374%:%
%:%2725=1374%:%
%:%2726=1375%:%
%:%2727=1375%:%
%:%2728=1376%:%
%:%2729=1376%:%
%:%2730=1377%:%
%:%2731=1377%:%
%:%2732=1377%:%
%:%2733=1378%:%
%:%2734=1378%:%
%:%2735=1379%:%
%:%2736=1379%:%
%:%2737=1379%:%
%:%2738=1380%:%
%:%2739=1380%:%
%:%2740=1381%:%
%:%2741=1381%:%
%:%2742=1381%:%
%:%2743=1382%:%
%:%2744=1382%:%
%:%2745=1382%:%
%:%2746=1383%:%
%:%2747=1383%:%
%:%2748=1384%:%
%:%2749=1384%:%
%:%2750=1384%:%
%:%2751=1385%:%
%:%2752=1385%:%
%:%2753=1386%:%
%:%2754=1386%:%
%:%2755=1387%:%
%:%2756=1387%:%
%:%2757=1388%:%
%:%2758=1388%:%
%:%2759=1389%:%
%:%2760=1389%:%
%:%2761=1389%:%
%:%2762=1390%:%
%:%2763=1390%:%
%:%2764=1391%:%
%:%2765=1391%:%
%:%2766=1392%:%
%:%2767=1392%:%
%:%2768=1393%:%
%:%2769=1393%:%
%:%2770=1393%:%
%:%2771=1394%:%
%:%2772=1394%:%
%:%2773=1395%:%
%:%2774=1395%:%
%:%2775=1395%:%
%:%2776=1396%:%
%:%2777=1396%:%
%:%2778=1397%:%
%:%2779=1397%:%
%:%2780=1397%:%
%:%2781=1398%:%
%:%2782=1398%:%
%:%2783=1398%:%
%:%2784=1399%:%
%:%2785=1399%:%
%:%2786=1400%:%
%:%2787=1400%:%
%:%2788=1400%:%
%:%2789=1401%:%
%:%2790=1401%:%
%:%2791=1402%:%
%:%2792=1402%:%
%:%2793=1403%:%
%:%2794=1403%:%
%:%2795=1403%:%
%:%2796=1404%:%
%:%2797=1404%:%
%:%2798=1405%:%
%:%2799=1405%:%
%:%2800=1405%:%
%:%2801=1406%:%
%:%2802=1406%:%
%:%2803=1407%:%
%:%2804=1407%:%
%:%2805=1407%:%
%:%2806=1408%:%
%:%2807=1408%:%
%:%2808=1408%:%
%:%2809=1409%:%
%:%2810=1409%:%
%:%2811=1410%:%
%:%2812=1410%:%
%:%2813=1410%:%
%:%2814=1411%:%
%:%2815=1411%:%
%:%2816=1412%:%
%:%2817=1412%:%
%:%2818=1413%:%
%:%2828=1415%:%
%:%2828=1416%:%
%:%2829=1417%:%
%:%2830=1418%:%
%:%2831=1419%:%
%:%2832=1420%:%
%:%2833=1421%:%
%:%2834=1422%:%
%:%2835=1423%:%
%:%2836=1424%:%
%:%2837=1425%:%
%:%2838=1426%:%
%:%2839=1427%:%
%:%2840=1428%:%
%:%2841=1429%:%
%:%2842=1430%:%
%:%2843=1431%:%
%:%2844=1432%:%
%:%2845=1433%:%
%:%2846=1434%:%
%:%2847=1435%:%
%:%2848=1436%:%
%:%2849=1437%:%
%:%2850=1438%:%
%:%2851=1439%:%
%:%2852=1440%:%
%:%2853=1441%:%
%:%2854=1442%:%
%:%2855=1443%:%
%:%2856=1444%:%
%:%2857=1445%:%
%:%2858=1446%:%
%:%2859=1447%:%
%:%2860=1448%:%
%:%2861=1449%:%
%:%2862=1450%:%
%:%2863=1451%:%
%:%2864=1452%:%
%:%2865=1453%:%
%:%2866=1454%:%
%:%2867=1455%:%
%:%2868=1456%:%
%:%2869=1457%:%
%:%2870=1458%:%
%:%2871=1459%:%
%:%2872=1460%:%
%:%2873=1461%:%
%:%2874=1462%:%
%:%2875=1463%:%
%:%2876=1464%:%
%:%2877=1465%:%
%:%2878=1466%:%
%:%2879=1467%:%
%:%2880=1468%:%
%:%2881=1469%:%
%:%2882=1470%:%
%:%2883=1471%:%
%:%2884=1472%:%
%:%2885=1473%:%
%:%2886=1474%:%
%:%2887=1475%:%
%:%2888=1476%:%
%:%2889=1477%:%
%:%2890=1478%:%
%:%2891=1479%:%
%:%2892=1480%:%
%:%2893=1481%:%
%:%2894=1482%:%
%:%2895=1483%:%
%:%2896=1484%:%
%:%2897=1485%:%
%:%2898=1486%:%
%:%2899=1487%:%
%:%2900=1488%:%
%:%2901=1489%:%
%:%2902=1490%:%
%:%2903=1491%:%
%:%2904=1492%:%
%:%2905=1493%:%
%:%2906=1494%:%
%:%2907=1495%:%
%:%2908=1496%:%
%:%2909=1497%:%
%:%2910=1498%:%
%:%2911=1499%:%
%:%2912=1500%:%
%:%2913=1501%:%
%:%2914=1502%:%
%:%2915=1503%:%
%:%2916=1504%:%
%:%2917=1505%:%
%:%2918=1506%:%
%:%2919=1507%:%
%:%2920=1508%:%
%:%2921=1509%:%
%:%2922=1510%:%
%:%2923=1511%:%
%:%2924=1512%:%
%:%2925=1513%:%
%:%2926=1514%:%
%:%2927=1515%:%
%:%2928=1516%:%
%:%2929=1517%:%
%:%2930=1518%:%
%:%2931=1519%:%
%:%2932=1520%:%
%:%2933=1521%:%
%:%2934=1522%:%
%:%2935=1523%:%
%:%2936=1524%:%
%:%2937=1525%:%
%:%2938=1526%:%
%:%2939=1527%:%
%:%2940=1528%:%
%:%2941=1529%:%
%:%2942=1530%:%
%:%2943=1531%:%
%:%2944=1532%:%
%:%2945=1533%:%
%:%2946=1534%:%
%:%2947=1535%:%
%:%2948=1536%:%
%:%2949=1537%:%
%:%2950=1538%:%
%:%2951=1539%:%
%:%2952=1540%:%
%:%2953=1541%:%
%:%2954=1542%:%
%:%2955=1543%:%
%:%2956=1544%:%
%:%2957=1545%:%
%:%2958=1546%:%
%:%2959=1547%:%
%:%2960=1548%:%
%:%2961=1549%:%
%:%2962=1550%:%
%:%2963=1551%:%
%:%2964=1552%:%
%:%2965=1553%:%
%:%2966=1554%:%
%:%2967=1555%:%
%:%2968=1556%:%
%:%2969=1557%:%
%:%2970=1558%:%
%:%2971=1559%:%
%:%2972=1560%:%
%:%2973=1561%:%
%:%2974=1562%:%
%:%2975=1563%:%
%:%2976=1564%:%
%:%2977=1565%:%
%:%2978=1566%:%
%:%2979=1567%:%
%:%2980=1568%:%
%:%2981=1569%:%
%:%2982=1570%:%
%:%2983=1571%:%
%:%2984=1572%:%
%:%2985=1573%:%
%:%2986=1574%:%
%:%2987=1575%:%
%:%2988=1576%:%
%:%2989=1577%:%
%:%2990=1578%:%
%:%2991=1579%:%
%:%2992=1580%:%
%:%2993=1581%:%
%:%2994=1582%:%
%:%2995=1583%:%
%:%2996=1584%:%
%:%2997=1585%:%
%:%2998=1586%:%
%:%2999=1587%:%
%:%3000=1588%:%
%:%3001=1589%:%
%:%3002=1590%:%
%:%3003=1591%:%
%:%3004=1592%:%
%:%3005=1593%:%
%:%3006=1594%:%
%:%3007=1595%:%
%:%3008=1596%:%
%:%3009=1597%:%
%:%3010=1598%:%
%:%3011=1599%:%
%:%3012=1600%:%
%:%3013=1601%:%
%:%3015=1603%:%
%:%3016=1603%:%
%:%3017=1604%:%
%:%3019=1606%:%
%:%3020=1607%:%
%:%3022=1609%:%
%:%3023=1609%:%
%:%3024=1610%:%
%:%3025=1611%:%
%:%3026=1612%:%
%:%3027=1613%:%
%:%3034=1614%:%
%:%3035=1614%:%
%:%3036=1615%:%
%:%3037=1615%:%
%:%3038=1616%:%
%:%3039=1616%:%
%:%3040=1616%:%
%:%3041=1616%:%
%:%3042=1617%:%
%:%3043=1617%:%
%:%3044=1617%:%
%:%3045=1618%:%
%:%3046=1618%:%
%:%3047=1618%:%
%:%3048=1619%:%
%:%3049=1619%:%
%:%3050=1620%:%
%:%3051=1620%:%
%:%3052=1620%:%
%:%3053=1621%:%
%:%3063=1623%:%
%:%3064=1624%:%
%:%3065=1625%:%
%:%3066=1626%:%
%:%3067=1627%:%
%:%3068=1628%:%
%:%3069=1629%:%
%:%3070=1630%:%
%:%3071=1631%:%
%:%3072=1632%:%
%:%3073=1633%:%
%:%3074=1634%:%
%:%3075=1635%:%
%:%3076=1636%:%
%:%3077=1637%:%
%:%3078=1638%:%
%:%3079=1639%:%
%:%3080=1640%:%
%:%3081=1641%:%
%:%3082=1642%:%
%:%3083=1643%:%
%:%3084=1644%:%
%:%3085=1645%:%
%:%3086=1646%:%
%:%3087=1647%:%
%:%3088=1648%:%
%:%3089=1649%:%
%:%3090=1650%:%
%:%3091=1651%:%
%:%3092=1652%:%
%:%3093=1653%:%
%:%3094=1654%:%
%:%3095=1655%:%
%:%3096=1656%:%
%:%3097=1657%:%
%:%3098=1658%:%
%:%3099=1659%:%
%:%3100=1660%:%
%:%3101=1661%:%
%:%3101=1662%:%
%:%3102=1663%:%
%:%3103=1664%:%
%:%3104=1665%:%
%:%3105=1666%:%
%:%3106=1667%:%
%:%3107=1668%:%
%:%3108=1669%:%
%:%3109=1670%:%
%:%3110=1671%:%
%:%3111=1672%:%
%:%3112=1673%:%
%:%3113=1674%:%
%:%3114=1675%:%
%:%3115=1676%:%
%:%3116=1677%:%
%:%3117=1678%:%
%:%3118=1679%:%
%:%3119=1680%:%
%:%3120=1681%:%
%:%3121=1682%:%
%:%3122=1683%:%
%:%3123=1684%:%
%:%3124=1685%:%
%:%3125=1686%:%
%:%3126=1687%:%
%:%3127=1688%:%
%:%3128=1689%:%
%:%3129=1690%:%
%:%3130=1691%:%
%:%3131=1692%:%
%:%3132=1693%:%
%:%3133=1694%:%
%:%3134=1695%:%
%:%3135=1696%:%
%:%3136=1697%:%
%:%3137=1698%:%
%:%3138=1699%:%
%:%3139=1700%:%
%:%3140=1701%:%
%:%3141=1702%:%
%:%3142=1703%:%
%:%3143=1704%:%
%:%3144=1705%:%
%:%3145=1706%:%
%:%3146=1707%:%
%:%3147=1708%:%
%:%3148=1709%:%
%:%3149=1710%:%
%:%3151=1712%:%
%:%3152=1712%:%
%:%3153=1713%:%
%:%3154=1714%:%
%:%3157=1715%:%
%:%3161=1715%:%
%:%3162=1715%:%
%:%3163=1715%:%
%:%3164=1715%:%
%:%3165=1715%:%
%:%3174=1717%:%
%:%3175=1718%:%
%:%3176=1719%:%
%:%3177=1720%:%
%:%3178=1721%:%
%:%3179=1722%:%
%:%3181=1724%:%
%:%3182=1724%:%
%:%3189=1725%:%
%:%3190=1725%:%
%:%3191=1726%:%
%:%3192=1726%:%
%:%3193=1727%:%
%:%3194=1727%:%
%:%3195=1727%:%
%:%3196=1728%:%
%:%3197=1728%:%
%:%3198=1729%:%
%:%3199=1729%:%
%:%3200=1729%:%
%:%3201=1730%:%
%:%3202=1730%:%
%:%3203=1731%:%
%:%3204=1731%:%
%:%3205=1732%:%
%:%3206=1732%:%
%:%3207=1732%:%
%:%3208=1733%:%
%:%3209=1733%:%
%:%3210=1734%:%
%:%3211=1734%:%
%:%3212=1735%:%
%:%3213=1735%:%
%:%3214=1736%:%
%:%3215=1736%:%
%:%3216=1736%:%
%:%3217=1737%:%
%:%3218=1737%:%
%:%3219=1738%:%
%:%3220=1738%:%
%:%3221=1739%:%
%:%3231=1741%:%
%:%3233=1743%:%
%:%3234=1743%:%
%:%3235=1744%:%
%:%3236=1745%:%
%:%3243=1746%:%
%:%3244=1746%:%
%:%3245=1747%:%
%:%3246=1747%:%
%:%3247=1748%:%
%:%3248=1748%:%
%:%3249=1748%:%
%:%3250=1749%:%
%:%3251=1749%:%
%:%3252=1750%:%
%:%3253=1750%:%
%:%3254=1751%:%
%:%3255=1751%:%
%:%3256=1752%:%
%:%3257=1752%:%
%:%3258=1753%:%
%:%3259=1753%:%
%:%3260=1753%:%
%:%3261=1754%:%
%:%3262=1754%:%
%:%3263=1754%:%
%:%3264=1755%:%
%:%3265=1755%:%
%:%3266=1756%:%
%:%3267=1756%:%
%:%3268=1757%:%
%:%3269=1757%:%
%:%3270=1758%:%
%:%3271=1758%:%
%:%3272=1759%:%
%:%3273=1759%:%
%:%3274=1759%:%
%:%3275=1760%:%
%:%3276=1760%:%
%:%3277=1761%:%
%:%3278=1761%:%
%:%3279=1762%:%
%:%3280=1762%:%
%:%3281=1762%:%
%:%3282=1763%:%
%:%3283=1763%:%
%:%3284=1763%:%
%:%3285=1764%:%
%:%3295=1766%:%
%:%3296=1767%:%
%:%3297=1768%:%
%:%3298=1769%:%
%:%3300=1771%:%
%:%3301=1771%:%
%:%3308=1772%:%
%:%3309=1772%:%
%:%3310=1773%:%
%:%3311=1773%:%
%:%3312=1774%:%
%:%3313=1774%:%
%:%3314=1774%:%
%:%3315=1775%:%
%:%3316=1775%:%
%:%3317=1776%:%
%:%3318=1776%:%
%:%3319=1776%:%
%:%3320=1777%:%
%:%3321=1777%:%
%:%3322=1778%:%
%:%3323=1778%:%
%:%3324=1779%:%
%:%3325=1779%:%
%:%3326=1779%:%
%:%3327=1780%:%
%:%3328=1780%:%
%:%3329=1781%:%
%:%3330=1781%:%
%:%3331=1782%:%
%:%3332=1782%:%
%:%3333=1783%:%
%:%3334=1783%:%
%:%3335=1783%:%
%:%3336=1784%:%
%:%3337=1784%:%
%:%3338=1785%:%
%:%3339=1785%:%
%:%3340=1786%:%
%:%3341=1786%:%
%:%3342=1787%:%
%:%3343=1787%:%
%:%3344=1787%:%
%:%3345=1788%:%
%:%3346=1788%:%
%:%3347=1788%:%
%:%3348=1789%:%
%:%3349=1789%:%
%:%3350=1790%:%
%:%3351=1790%:%
%:%3352=1791%:%
%:%3353=1791%:%
%:%3354=1791%:%
%:%3355=1792%:%
%:%3365=1794%:%
%:%3366=1795%:%
%:%3368=1797%:%
%:%3369=1797%:%
%:%3370=1798%:%
%:%3371=1799%:%
%:%3378=1800%:%
%:%3379=1800%:%
%:%3380=1801%:%
%:%3381=1801%:%
%:%3382=1802%:%
%:%3383=1802%:%
%:%3384=1803%:%
%:%3385=1803%:%
%:%3386=1804%:%
%:%3387=1804%:%
%:%3388=1805%:%
%:%3389=1805%:%
%:%3390=1806%:%
%:%3391=1806%:%
%:%3392=1807%:%
%:%3393=1807%:%
%:%3394=1808%:%
%:%3395=1808%:%
%:%3396=1809%:%
%:%3397=1809%:%
%:%3398=1809%:%
%:%3399=1810%:%
%:%3400=1810%:%
%:%3401=1810%:%
%:%3402=1811%:%
%:%3403=1811%:%
%:%3404=1812%:%
%:%3405=1812%:%
%:%3406=1813%:%
%:%3407=1813%:%
%:%3408=1813%:%
%:%3409=1814%:%
%:%3410=1814%:%
%:%3411=1814%:%
%:%3412=1815%:%
%:%3413=1815%:%
%:%3414=1816%:%
%:%3415=1816%:%
%:%3416=1817%:%
%:%3417=1817%:%
%:%3418=1817%:%
%:%3419=1818%:%
%:%3420=1818%:%
%:%3421=1818%:%
%:%3422=1819%:%
%:%3423=1819%:%
%:%3424=1820%:%
%:%3425=1820%:%
%:%3426=1821%:%
%:%3427=1821%:%
%:%3428=1822%:%
%:%3429=1822%:%
%:%3430=1823%:%
%:%3431=1823%:%
%:%3432=1823%:%
%:%3433=1824%:%
%:%3434=1824%:%
%:%3435=1825%:%
%:%3436=1825%:%
%:%3437=1826%:%
%:%3438=1826%:%
%:%3439=1827%:%
%:%3440=1827%:%
%:%3441=1827%:%
%:%3442=1828%:%
%:%3443=1828%:%
%:%3444=1829%:%
%:%3454=1831%:%
%:%3455=1832%:%
%:%3456=1833%:%
%:%3457=1834%:%
%:%3458=1835%:%
%:%3459=1836%:%
%:%3460=1837%:%
%:%3462=1839%:%
%:%3463=1839%:%
%:%3464=1840%:%
%:%3465=1841%:%
%:%3466=1842%:%
%:%3467=1843%:%
%:%3468=1844%:%
%:%3475=1845%:%
%:%3476=1845%:%
%:%3477=1846%:%
%:%3478=1846%:%
%:%3479=1847%:%
%:%3480=1847%:%
%:%3481=1848%:%
%:%3482=1848%:%
%:%3483=1849%:%
%:%3484=1849%:%
%:%3485=1849%:%
%:%3486=1850%:%
%:%3487=1850%:%
%:%3488=1850%:%
%:%3489=1851%:%
%:%3490=1851%:%
%:%3491=1852%:%
%:%3492=1852%:%
%:%3493=1853%:%
%:%3494=1853%:%
%:%3495=1853%:%
%:%3496=1854%:%
%:%3497=1854%:%
%:%3498=1854%:%
%:%3499=1855%:%
%:%3500=1855%:%
%:%3501=1856%:%
%:%3502=1856%:%
%:%3503=1856%:%
%:%3504=1857%:%
%:%3505=1857%:%
%:%3506=1858%:%
%:%3507=1858%:%
%:%3508=1859%:%
%:%3509=1859%:%
%:%3510=1859%:%
%:%3511=1860%:%
%:%3521=1862%:%
%:%3522=1863%:%
%:%3523=1864%:%
%:%3524=1865%:%
%:%3526=1867%:%
%:%3527=1867%:%
%:%3528=1868%:%
%:%3529=1869%:%
%:%3530=1870%:%
%:%3533=1871%:%
%:%3537=1871%:%
%:%3538=1871%:%
%:%3539=1871%:%
%:%3548=1873%:%
%:%3549=1874%:%
%:%3550=1875%:%
%:%3552=1877%:%
%:%3553=1877%:%
%:%3554=1878%:%
%:%3555=1879%:%
%:%3556=1880%:%
%:%3557=1881%:%
%:%3558=1882%:%
%:%3559=1883%:%
%:%3566=1884%:%
%:%3567=1884%:%
%:%3568=1885%:%
%:%3569=1885%:%
%:%3570=1886%:%
%:%3571=1886%:%
%:%3572=1886%:%
%:%3573=1887%:%
%:%3574=1887%:%
%:%3575=1888%:%
%:%3576=1888%:%
%:%3577=1889%:%
%:%3578=1889%:%
%:%3579=1890%:%
%:%3580=1890%:%
%:%3581=1890%:%
%:%3582=1891%:%
%:%3583=1891%:%
%:%3584=1892%:%
%:%3585=1892%:%
%:%3586=1892%:%
%:%3587=1893%:%
%:%3588=1893%:%
%:%3589=1894%:%
%:%3590=1894%:%
%:%3591=1894%:%
%:%3592=1895%:%
%:%3593=1895%:%
%:%3594=1895%:%
%:%3595=1896%:%
%:%3596=1896%:%
%:%3597=1896%:%
%:%3598=1897%:%
%:%3599=1897%:%
%:%3600=1897%:%
%:%3601=1898%:%
%:%3602=1898%:%
%:%3603=1899%:%
%:%3604=1899%:%
%:%3605=1899%:%
%:%3606=1900%:%
%:%3607=1900%:%
%:%3608=1901%:%
%:%3609=1901%:%
%:%3610=1901%:%
%:%3611=1902%:%
%:%3612=1902%:%
%:%3613=1903%:%
%:%3614=1903%:%
%:%3615=1904%:%
%:%3616=1904%:%
%:%3617=1904%:%
%:%3618=1905%:%
%:%3619=1905%:%
%:%3620=1905%:%
%:%3621=1906%:%
%:%3622=1906%:%
%:%3623=1907%:%
%:%3624=1907%:%
%:%3625=1908%:%
%:%3626=1908%:%
%:%3627=1908%:%
%:%3628=1909%:%
%:%3629=1909%:%
%:%3630=1909%:%
%:%3631=1910%:%
%:%3632=1910%:%
%:%3633=1911%:%
%:%3634=1911%:%
%:%3635=1911%:%
%:%3636=1912%:%
%:%3637=1912%:%
%:%3638=1913%:%
%:%3639=1913%:%
%:%3640=1914%:%
%:%3641=1914%:%
%:%3642=1915%:%
%:%3652=1917%:%
%:%3653=1918%:%
%:%3654=1919%:%
%:%3656=1921%:%
%:%3657=1921%:%
%:%3658=1922%:%
%:%3659=1923%:%
%:%3660=1924%:%
%:%3661=1925%:%
%:%3662=1926%:%
%:%3663=1927%:%
%:%3670=1928%:%
%:%3671=1928%:%
%:%3672=1929%:%
%:%3673=1929%:%
%:%3674=1930%:%
%:%3675=1930%:%
%:%3676=1930%:%
%:%3677=1931%:%
%:%3678=1931%:%
%:%3679=1932%:%
%:%3680=1932%:%
%:%3681=1933%:%
%:%3682=1933%:%
%:%3683=1934%:%
%:%3684=1934%:%
%:%3685=1934%:%
%:%3686=1935%:%
%:%3687=1935%:%
%:%3688=1936%:%
%:%3689=1936%:%
%:%3690=1936%:%
%:%3691=1937%:%
%:%3692=1937%:%
%:%3693=1938%:%
%:%3694=1938%:%
%:%3695=1938%:%
%:%3696=1939%:%
%:%3697=1939%:%
%:%3698=1939%:%
%:%3699=1940%:%
%:%3700=1940%:%
%:%3701=1940%:%
%:%3702=1941%:%
%:%3703=1941%:%
%:%3704=1941%:%
%:%3705=1942%:%
%:%3706=1942%:%
%:%3707=1943%:%
%:%3708=1943%:%
%:%3709=1943%:%
%:%3710=1944%:%
%:%3711=1944%:%
%:%3712=1945%:%
%:%3713=1945%:%
%:%3714=1945%:%
%:%3715=1946%:%
%:%3716=1946%:%
%:%3717=1947%:%
%:%3718=1947%:%
%:%3719=1947%:%
%:%3720=1948%:%
%:%3721=1948%:%
%:%3722=1949%:%
%:%3723=1949%:%
%:%3724=1949%:%
%:%3725=1950%:%
%:%3726=1950%:%
%:%3727=1951%:%
%:%3728=1951%:%
%:%3729=1951%:%
%:%3730=1952%:%
%:%3731=1952%:%
%:%3732=1953%:%
%:%3733=1953%:%
%:%3734=1954%:%
%:%3735=1954%:%
%:%3736=1954%:%
%:%3737=1955%:%
%:%3738=1955%:%
%:%3739=1955%:%
%:%3740=1956%:%
%:%3741=1956%:%
%:%3742=1957%:%
%:%3743=1957%:%
%:%3744=1958%:%
%:%3745=1958%:%
%:%3746=1958%:%
%:%3747=1959%:%
%:%3748=1959%:%
%:%3749=1959%:%
%:%3750=1960%:%
%:%3751=1960%:%
%:%3752=1961%:%
%:%3753=1961%:%
%:%3754=1961%:%
%:%3755=1962%:%
%:%3756=1962%:%
%:%3757=1963%:%
%:%3758=1963%:%
%:%3759=1964%:%
%:%3760=1964%:%
%:%3761=1965%:%
%:%3771=1967%:%
%:%3772=1968%:%
%:%3773=1969%:%
%:%3775=1971%:%
%:%3776=1971%:%
%:%3777=1972%:%
%:%3778=1973%:%
%:%3779=1974%:%
%:%3780=1975%:%
%:%3781=1976%:%
%:%3782=1977%:%
%:%3789=1978%:%
%:%3790=1978%:%
%:%3791=1979%:%
%:%3792=1979%:%
%:%3793=1980%:%
%:%3794=1980%:%
%:%3795=1980%:%
%:%3796=1981%:%
%:%3797=1981%:%
%:%3798=1982%:%
%:%3799=1982%:%
%:%3800=1983%:%
%:%3801=1983%:%
%:%3802=1984%:%
%:%3803=1984%:%
%:%3804=1984%:%
%:%3805=1985%:%
%:%3806=1985%:%
%:%3807=1986%:%
%:%3808=1986%:%
%:%3809=1986%:%
%:%3810=1987%:%
%:%3811=1987%:%
%:%3812=1988%:%
%:%3813=1988%:%
%:%3814=1988%:%
%:%3815=1989%:%
%:%3816=1989%:%
%:%3817=1989%:%
%:%3818=1990%:%
%:%3819=1990%:%
%:%3820=1990%:%
%:%3821=1991%:%
%:%3822=1991%:%
%:%3823=1991%:%
%:%3824=1992%:%
%:%3825=1992%:%
%:%3826=1993%:%
%:%3827=1993%:%
%:%3828=1993%:%
%:%3829=1994%:%
%:%3830=1994%:%
%:%3831=1995%:%
%:%3832=1995%:%
%:%3833=1995%:%
%:%3834=1996%:%
%:%3835=1996%:%
%:%3836=1997%:%
%:%3837=1997%:%
%:%3838=1997%:%
%:%3839=1998%:%
%:%3840=1998%:%
%:%3841=1999%:%
%:%3842=1999%:%
%:%3843=1999%:%
%:%3844=2000%:%
%:%3845=2000%:%
%:%3846=2001%:%
%:%3847=2001%:%
%:%3848=2001%:%
%:%3849=2002%:%
%:%3850=2002%:%
%:%3851=2003%:%
%:%3852=2003%:%
%:%3853=2004%:%
%:%3854=2004%:%
%:%3855=2004%:%
%:%3856=2005%:%
%:%3857=2005%:%
%:%3858=2005%:%
%:%3859=2006%:%
%:%3860=2006%:%
%:%3861=2007%:%
%:%3862=2007%:%
%:%3863=2008%:%
%:%3864=2008%:%
%:%3865=2008%:%
%:%3866=2009%:%
%:%3867=2009%:%
%:%3868=2009%:%
%:%3869=2010%:%
%:%3870=2010%:%
%:%3871=2011%:%
%:%3872=2011%:%
%:%3873=2011%:%
%:%3874=2012%:%
%:%3875=2012%:%
%:%3876=2013%:%
%:%3877=2013%:%
%:%3878=2014%:%
%:%3879=2014%:%
%:%3880=2015%:%
%:%3890=2017%:%
%:%3891=2018%:%
%:%3892=2019%:%
%:%3894=2021%:%
%:%3895=2021%:%
%:%3896=2022%:%
%:%3897=2023%:%
%:%3898=2024%:%
%:%3899=2025%:%
%:%3900=2026%:%
%:%3901=2027%:%
%:%3902=2028%:%
%:%3909=2029%:%
%:%3910=2029%:%
%:%3911=2030%:%
%:%3912=2030%:%
%:%3913=2031%:%
%:%3914=2031%:%
%:%3915=2031%:%
%:%3916=2032%:%
%:%3917=2032%:%
%:%3918=2033%:%
%:%3919=2033%:%
%:%3920=2034%:%
%:%3921=2034%:%
%:%3922=2035%:%
%:%3923=2035%:%
%:%3924=2035%:%
%:%3925=2036%:%
%:%3926=2036%:%
%:%3927=2037%:%
%:%3928=2037%:%
%:%3929=2037%:%
%:%3930=2038%:%
%:%3931=2038%:%
%:%3932=2039%:%
%:%3933=2039%:%
%:%3934=2039%:%
%:%3935=2040%:%
%:%3936=2040%:%
%:%3937=2040%:%
%:%3938=2041%:%
%:%3939=2041%:%
%:%3940=2041%:%
%:%3941=2042%:%
%:%3942=2042%:%
%:%3943=2042%:%
%:%3944=2043%:%
%:%3945=2043%:%
%:%3946=2044%:%
%:%3947=2044%:%
%:%3948=2044%:%
%:%3949=2045%:%
%:%3950=2045%:%
%:%3951=2046%:%
%:%3952=2046%:%
%:%3953=2046%:%
%:%3954=2047%:%
%:%3955=2047%:%
%:%3956=2048%:%
%:%3957=2048%:%
%:%3958=2048%:%
%:%3959=2049%:%
%:%3960=2049%:%
%:%3961=2050%:%
%:%3962=2050%:%
%:%3963=2051%:%
%:%3964=2051%:%
%:%3965=2051%:%
%:%3966=2052%:%
%:%3967=2052%:%
%:%3968=2052%:%
%:%3969=2053%:%
%:%3970=2053%:%
%:%3971=2054%:%
%:%3972=2054%:%
%:%3973=2055%:%
%:%3974=2055%:%
%:%3975=2055%:%
%:%3976=2056%:%
%:%3977=2056%:%
%:%3978=2056%:%
%:%3979=2057%:%
%:%3980=2057%:%
%:%3981=2058%:%
%:%3982=2058%:%
%:%3983=2058%:%
%:%3984=2059%:%
%:%3985=2059%:%
%:%3986=2060%:%
%:%3987=2060%:%
%:%3988=2061%:%
%:%3989=2061%:%
%:%3990=2062%:%
%:%4000=2064%:%
%:%4001=2065%:%
%:%4002=2066%:%
%:%4004=2068%:%
%:%4005=2068%:%
%:%4006=2069%:%
%:%4007=2070%:%
%:%4008=2071%:%
%:%4009=2072%:%
%:%4010=2073%:%
%:%4011=2074%:%
%:%4018=2075%:%
%:%4019=2075%:%
%:%4020=2076%:%
%:%4021=2076%:%
%:%4022=2077%:%
%:%4023=2077%:%
%:%4024=2078%:%
%:%4025=2078%:%
%:%4028=2081%:%
%:%4029=2082%:%
%:%4030=2082%:%
%:%4031=2082%:%
%:%4032=2083%:%
%:%4033=2083%:%
%:%4034=2084%:%
%:%4035=2084%:%
%:%4036=2085%:%
%:%4037=2085%:%
%:%4038=2086%:%
%:%4039=2086%:%
%:%4040=2087%:%
%:%4041=2087%:%
%:%4042=2087%:%
%:%4043=2088%:%
%:%4044=2088%:%
%:%4045=2089%:%
%:%4046=2089%:%
%:%4048=2091%:%
%:%4049=2092%:%
%:%4050=2092%:%
%:%4051=2093%:%
%:%4052=2093%:%
%:%4053=2094%:%
%:%4054=2094%:%
%:%4055=2095%:%
%:%4056=2095%:%
%:%4057=2096%:%
%:%4058=2096%:%
%:%4059=2096%:%
%:%4060=2097%:%
%:%4061=2097%:%
%:%4062=2098%:%
%:%4063=2098%:%
%:%4064=2098%:%
%:%4065=2099%:%
%:%4066=2099%:%
%:%4067=2100%:%
%:%4068=2100%:%
%:%4069=2100%:%
%:%4070=2101%:%
%:%4071=2101%:%
%:%4072=2102%:%
%:%4073=2102%:%
%:%4074=2102%:%
%:%4075=2103%:%
%:%4076=2103%:%
%:%4077=2104%:%
%:%4078=2104%:%
%:%4079=2104%:%
%:%4080=2105%:%
%:%4081=2105%:%
%:%4082=2106%:%
%:%4083=2106%:%
%:%4084=2107%:%
%:%4085=2107%:%
%:%4086=2108%:%
%:%4087=2108%:%
%:%4088=2109%:%
%:%4089=2110%:%
%:%4090=2110%:%
%:%4091=2111%:%
%:%4092=2111%:%
%:%4093=2112%:%
%:%4094=2112%:%
%:%4095=2113%:%
%:%4096=2113%:%
%:%4097=2114%:%
%:%4098=2114%:%
%:%4099=2114%:%
%:%4100=2115%:%
%:%4101=2115%:%
%:%4102=2116%:%
%:%4103=2116%:%
%:%4104=2116%:%
%:%4105=2117%:%
%:%4106=2117%:%
%:%4107=2118%:%
%:%4108=2118%:%
%:%4109=2118%:%
%:%4110=2119%:%
%:%4111=2119%:%
%:%4112=2120%:%
%:%4113=2120%:%
%:%4114=2120%:%
%:%4115=2121%:%
%:%4116=2121%:%
%:%4117=2122%:%
%:%4118=2122%:%
%:%4119=2123%:%
%:%4120=2123%:%
%:%4121=2124%:%
%:%4122=2124%:%
%:%4123=2125%:%
%:%4124=2125%:%
%:%4125=2125%:%
%:%4126=2126%:%
%:%4127=2126%:%
%:%4128=2127%:%
%:%4129=2127%:%
%:%4130=2128%:%
%:%4131=2128%:%
%:%4132=2128%:%
%:%4133=2129%:%
%:%4134=2129%:%
%:%4135=2130%:%
%:%4136=2130%:%
%:%4137=2130%:%
%:%4138=2131%:%
%:%4139=2131%:%
%:%4140=2132%:%
%:%4141=2132%:%
%:%4142=2133%:%
%:%4143=2133%:%
%:%4144=2134%:%
%:%4154=2136%:%
%:%4155=2137%:%
%:%4156=2138%:%
%:%4158=2140%:%
%:%4159=2140%:%
%:%4160=2141%:%
%:%4161=2142%:%
%:%4168=2143%:%
%:%4169=2143%:%
%:%4170=2144%:%
%:%4171=2144%:%
%:%4172=2145%:%
%:%4173=2145%:%
%:%4174=2146%:%
%:%4175=2146%:%
%:%4176=2147%:%
%:%4177=2147%:%
%:%4178=2148%:%
%:%4179=2148%:%
%:%4180=2149%:%
%:%4181=2149%:%
%:%4182=2150%:%
%:%4183=2150%:%
%:%4184=2151%:%
%:%4185=2151%:%
%:%4186=2152%:%
%:%4187=2152%:%
%:%4188=2153%:%
%:%4189=2153%:%
%:%4190=2154%:%
%:%4191=2154%:%
%:%4192=2155%:%
%:%4193=2155%:%
%:%4194=2156%:%
%:%4195=2156%:%
%:%4196=2156%:%
%:%4197=2157%:%
%:%4198=2157%:%
%:%4199=2158%:%
%:%4200=2158%:%
%:%4201=2159%:%
%:%4202=2159%:%
%:%4203=2160%:%
%:%4204=2160%:%
%:%4205=2161%:%
%:%4206=2161%:%
%:%4207=2162%:%
%:%4208=2162%:%
%:%4209=2162%:%
%:%4210=2163%:%
%:%4211=2163%:%
%:%4212=2164%:%
%:%4213=2164%:%
%:%4214=2164%:%
%:%4215=2165%:%
%:%4216=2165%:%
%:%4217=2166%:%
%:%4218=2166%:%
%:%4219=2166%:%
%:%4220=2167%:%
%:%4221=2167%:%
%:%4222=2168%:%
%:%4223=2168%:%
%:%4224=2168%:%
%:%4225=2169%:%
%:%4226=2169%:%
%:%4227=2170%:%
%:%4228=2170%:%
%:%4229=2170%:%
%:%4230=2171%:%
%:%4231=2171%:%
%:%4232=2172%:%
%:%4233=2172%:%
%:%4234=2173%:%
%:%4235=2173%:%
%:%4236=2174%:%
%:%4237=2174%:%
%:%4238=2174%:%
%:%4239=2175%:%
%:%4240=2175%:%
%:%4241=2176%:%
%:%4242=2176%:%
%:%4243=2177%:%
%:%4244=2177%:%
%:%4245=2177%:%
%:%4246=2178%:%
%:%4247=2178%:%
%:%4248=2179%:%
%:%4249=2179%:%
%:%4250=2180%:%
%:%4251=2180%:%
%:%4252=2181%:%
%:%4253=2181%:%
%:%4254=2182%:%
%:%4255=2182%:%
%:%4256=2183%:%
%:%4257=2183%:%
%:%4258=2183%:%
%:%4259=2184%:%
%:%4260=2184%:%
%:%4261=2185%:%
%:%4262=2185%:%
%:%4263=2186%:%
%:%4264=2186%:%
%:%4265=2186%:%
%:%4266=2187%:%
%:%4267=2187%:%
%:%4268=2188%:%
%:%4269=2188%:%
%:%4270=2189%:%
%:%4271=2189%:%
%:%4272=2190%:%
%:%4273=2190%:%
%:%4274=2191%:%
%:%4275=2191%:%
%:%4276=2192%:%
%:%4277=2192%:%
%:%4278=2192%:%
%:%4279=2193%:%
%:%4280=2193%:%
%:%4281=2194%:%
%:%4282=2194%:%
%:%4283=2195%:%
%:%4284=2195%:%
%:%4285=2195%:%
%:%4286=2196%:%
%:%4287=2196%:%
%:%4288=2197%:%
%:%4289=2197%:%
%:%4290=2198%:%
%:%4291=2198%:%
%:%4292=2198%:%
%:%4293=2199%:%
%:%4294=2199%:%
%:%4295=2200%:%
%:%4296=2200%:%
%:%4297=2201%:%
%:%4298=2201%:%
%:%4299=2201%:%
%:%4300=2202%:%
%:%4301=2202%:%
%:%4302=2203%:%
%:%4303=2203%:%
%:%4304=2204%:%
%:%4305=2204%:%
%:%4306=2205%:%
%:%4307=2205%:%
%:%4308=2206%:%
%:%4309=2206%:%
%:%4310=2207%:%
%:%4311=2207%:%
%:%4312=2208%:%
%:%4313=2208%:%
%:%4314=2209%:%
%:%4324=2211%:%
%:%4325=2212%:%
%:%4326=2213%:%
%:%4327=2214%:%
%:%4328=2215%:%
%:%4329=2216%:%
%:%4330=2217%:%
%:%4331=2218%:%
%:%4332=2219%:%
%:%4334=2221%:%
%:%4335=2221%:%
%:%4336=2222%:%
%:%4337=2223%:%
%:%4338=2224%:%
%:%4345=2225%:%
%:%4346=2225%:%
%:%4347=2226%:%
%:%4348=2226%:%
%:%4349=2227%:%
%:%4350=2227%:%
%:%4351=2227%:%
%:%4352=2228%:%
%:%4353=2228%:%
%:%4354=2228%:%
%:%4355=2229%:%
%:%4356=2229%:%
%:%4357=2229%:%
%:%4358=2230%:%
%:%4359=2230%:%
%:%4360=2230%:%
%:%4361=2231%:%
%:%4362=2231%:%
%:%4363=2232%:%
%:%4364=2232%:%
%:%4365=2233%:%
%:%4366=2233%:%
%:%4367=2234%:%
%:%4368=2234%:%
%:%4369=2235%:%
%:%4370=2235%:%
%:%4371=2236%:%
%:%4372=2236%:%
%:%4373=2236%:%
%:%4374=2237%:%
%:%4375=2237%:%
%:%4376=2238%:%
%:%4377=2238%:%
%:%4378=2239%:%
%:%4379=2239%:%
%:%4380=2240%:%
%:%4381=2240%:%
%:%4382=2241%:%
%:%4383=2241%:%
%:%4384=2242%:%
%:%4394=2244%:%
%:%4395=2245%:%
%:%4396=2246%:%
%:%4398=2248%:%
%:%4399=2248%:%
%:%4400=2249%:%
%:%4401=2250%:%
%:%4402=2251%:%
%:%4409=2252%:%
%:%4410=2252%:%
%:%4411=2253%:%
%:%4412=2253%:%
%:%4413=2254%:%
%:%4414=2254%:%
%:%4415=2254%:%
%:%4416=2254%:%
%:%4417=2255%:%
%:%4418=2255%:%
%:%4419=2256%:%
%:%4420=2256%:%
%:%4421=2256%:%
%:%4422=2256%:%
%:%4423=2257%:%
%:%4424=2257%:%
%:%4425=2257%:%
%:%4426=2258%:%
%:%4427=2258%:%
%:%4428=2259%:%
%:%4429=2259%:%
%:%4430=2259%:%
%:%4431=2260%:%
%:%4432=2260%:%
%:%4433=2261%:%
%:%4434=2261%:%
%:%4435=2262%:%
%:%4436=2262%:%
%:%4437=2262%:%
%:%4438=2263%:%
%:%4439=2263%:%
%:%4440=2264%:%
%:%4441=2264%:%
%:%4442=2264%:%
%:%4443=2265%:%
%:%4444=2265%:%
%:%4445=2266%:%
%:%4446=2266%:%
%:%4447=2266%:%
%:%4448=2267%:%
%:%4449=2267%:%
%:%4450=2268%:%
%:%4451=2268%:%
%:%4452=2268%:%
%:%4453=2269%:%
%:%4454=2269%:%
%:%4455=2270%:%
%:%4456=2270%:%
%:%4457=2270%:%
%:%4458=2271%:%
%:%4459=2271%:%
%:%4460=2272%:%
%:%4461=2272%:%
%:%4462=2272%:%
%:%4463=2273%:%
%:%4464=2273%:%
%:%4465=2274%:%
%:%4466=2274%:%
%:%4467=2274%:%
%:%4468=2275%:%
%:%4469=2275%:%
%:%4470=2276%:%
%:%4471=2276%:%
%:%4472=2277%:%
%:%4473=2277%:%
%:%4474=2278%:%
%:%4475=2278%:%
%:%4476=2278%:%
%:%4477=2279%:%
%:%4478=2279%:%
%:%4479=2279%:%
%:%4480=2280%:%
%:%4481=2280%:%
%:%4482=2281%:%
%:%4483=2281%:%
%:%4484=2281%:%
%:%4485=2282%:%
%:%4486=2282%:%
%:%4487=2283%:%
%:%4488=2283%:%
%:%4489=2283%:%
%:%4490=2284%:%
%:%4491=2284%:%
%:%4492=2285%:%
%:%4493=2285%:%
%:%4494=2286%:%
%:%4495=2286%:%
%:%4496=2286%:%
%:%4497=2287%:%
%:%4498=2287%:%
%:%4499=2287%:%
%:%4500=2288%:%
%:%4501=2288%:%
%:%4502=2289%:%
%:%4503=2289%:%
%:%4504=2290%:%
%:%4505=2290%:%
%:%4506=2290%:%
%:%4507=2291%:%
%:%4508=2291%:%
%:%4509=2291%:%
%:%4510=2292%:%
%:%4511=2292%:%
%:%4512=2293%:%
%:%4513=2293%:%
%:%4514=2293%:%
%:%4515=2294%:%
%:%4516=2294%:%
%:%4517=2295%:%
%:%4518=2295%:%
%:%4519=2295%:%
%:%4520=2296%:%
%:%4521=2296%:%
%:%4522=2297%:%
%:%4532=2299%:%
%:%4533=2300%:%
%:%4534=2301%:%
%:%4535=2302%:%
%:%4536=2303%:%
%:%4537=2304%:%
%:%4538=2305%:%
%:%4539=2306%:%
%:%4540=2307%:%
%:%4542=2309%:%
%:%4543=2309%:%
%:%4544=2310%:%
%:%4545=2311%:%
%:%4546=2312%:%
%:%4547=2313%:%
%:%4548=2314%:%
%:%4549=2315%:%
%:%4556=2316%:%
%:%4557=2316%:%
%:%4558=2317%:%
%:%4559=2317%:%
%:%4560=2318%:%
%:%4561=2318%:%
%:%4562=2318%:%
%:%4563=2319%:%
%:%4564=2319%:%
%:%4565=2320%:%
%:%4566=2320%:%
%:%4567=2321%:%
%:%4568=2321%:%
%:%4569=2322%:%
%:%4570=2322%:%
%:%4571=2322%:%
%:%4572=2323%:%
%:%4573=2323%:%
%:%4574=2324%:%
%:%4575=2324%:%
%:%4576=2324%:%
%:%4577=2325%:%
%:%4578=2325%:%
%:%4579=2326%:%
%:%4580=2326%:%
%:%4581=2326%:%
%:%4582=2327%:%
%:%4583=2327%:%
%:%4584=2327%:%
%:%4585=2328%:%
%:%4586=2328%:%
%:%4587=2328%:%
%:%4588=2329%:%
%:%4589=2329%:%
%:%4590=2329%:%
%:%4591=2330%:%
%:%4592=2330%:%
%:%4593=2331%:%
%:%4594=2331%:%
%:%4595=2332%:%
%:%4596=2332%:%
%:%4597=2333%:%
%:%4598=2333%:%
%:%4599=2334%:%
%:%4600=2334%:%
%:%4601=2334%:%
%:%4602=2335%:%
%:%4603=2335%:%
%:%4604=2336%:%
%:%4605=2336%:%
%:%4606=2336%:%
%:%4607=2337%:%
%:%4608=2337%:%
%:%4609=2337%:%
%:%4610=2338%:%
%:%4611=2338%:%
%:%4612=2338%:%
%:%4613=2339%:%
%:%4614=2339%:%
%:%4615=2340%:%
%:%4616=2340%:%
%:%4617=2340%:%
%:%4618=2341%:%
%:%4619=2341%:%
%:%4620=2342%:%
%:%4621=2342%:%
%:%4622=2342%:%
%:%4623=2343%:%
%:%4624=2343%:%
%:%4625=2344%:%
%:%4626=2344%:%
%:%4627=2345%:%
%:%4628=2345%:%
%:%4629=2346%:%
%:%4630=2346%:%
%:%4631=2347%:%
%:%4632=2347%:%
%:%4633=2348%:%
%:%4634=2348%:%
%:%4635=2348%:%
%:%4636=2349%:%
%:%4637=2349%:%
%:%4638=2350%:%
%:%4639=2350%:%
%:%4640=2350%:%
%:%4641=2351%:%
%:%4642=2351%:%
%:%4643=2351%:%
%:%4644=2352%:%
%:%4645=2352%:%
%:%4646=2352%:%
%:%4647=2353%:%
%:%4648=2353%:%
%:%4649=2354%:%
%:%4650=2354%:%
%:%4651=2354%:%
%:%4652=2355%:%
%:%4653=2355%:%
%:%4654=2356%:%
%:%4655=2356%:%
%:%4656=2356%:%
%:%4657=2357%:%
%:%4658=2357%:%
%:%4659=2358%:%
%:%4660=2358%:%
%:%4661=2359%:%
%:%4662=2359%:%
%:%4663=2360%:%
%:%4664=2360%:%
%:%4665=2361%:%
%:%4675=2363%:%
%:%4676=2364%:%
%:%4677=2365%:%
%:%4679=2367%:%
%:%4680=2367%:%
%:%4681=2368%:%
%:%4682=2369%:%
%:%4683=2370%:%
%:%4684=2371%:%
%:%4685=2372%:%
%:%4686=2373%:%
%:%4693=2374%:%
%:%4694=2374%:%
%:%4695=2375%:%
%:%4696=2375%:%
%:%4697=2376%:%
%:%4698=2376%:%
%:%4699=2376%:%
%:%4700=2377%:%
%:%4701=2377%:%
%:%4702=2378%:%
%:%4703=2378%:%
%:%4704=2379%:%
%:%4705=2379%:%
%:%4706=2380%:%
%:%4707=2380%:%
%:%4708=2380%:%
%:%4709=2381%:%
%:%4710=2381%:%
%:%4711=2382%:%
%:%4712=2382%:%
%:%4713=2382%:%
%:%4714=2383%:%
%:%4715=2383%:%
%:%4716=2384%:%
%:%4717=2384%:%
%:%4718=2384%:%
%:%4719=2385%:%
%:%4720=2385%:%
%:%4721=2385%:%
%:%4722=2386%:%
%:%4723=2386%:%
%:%4724=2386%:%
%:%4725=2387%:%
%:%4726=2387%:%
%:%4727=2387%:%
%:%4728=2388%:%
%:%4729=2388%:%
%:%4730=2389%:%
%:%4731=2389%:%
%:%4732=2389%:%
%:%4733=2390%:%
%:%4734=2390%:%
%:%4735=2391%:%
%:%4736=2391%:%
%:%4737=2391%:%
%:%4738=2392%:%
%:%4739=2392%:%
%:%4740=2393%:%
%:%4741=2393%:%
%:%4742=2394%:%
%:%4743=2394%:%
%:%4744=2395%:%
%:%4745=2395%:%
%:%4746=2396%:%
%:%4747=2396%:%
%:%4748=2396%:%
%:%4749=2397%:%
%:%4750=2397%:%
%:%4751=2398%:%
%:%4752=2398%:%
%:%4753=2398%:%
%:%4754=2399%:%
%:%4755=2399%:%
%:%4756=2400%:%
%:%4757=2400%:%
%:%4758=2400%:%
%:%4759=2401%:%
%:%4760=2401%:%
%:%4761=2401%:%
%:%4762=2402%:%
%:%4763=2402%:%
%:%4764=2402%:%
%:%4765=2403%:%
%:%4766=2403%:%
%:%4767=2404%:%
%:%4768=2404%:%
%:%4769=2404%:%
%:%4770=2405%:%
%:%4771=2405%:%
%:%4772=2406%:%
%:%4773=2406%:%
%:%4774=2406%:%
%:%4775=2407%:%
%:%4776=2407%:%
%:%4777=2408%:%
%:%4778=2408%:%
%:%4779=2409%:%
%:%4780=2409%:%
%:%4781=2410%:%
%:%4782=2410%:%
%:%4783=2411%:%
%:%4784=2411%:%
%:%4785=2412%:%
%:%4786=2412%:%
%:%4787=2412%:%
%:%4788=2413%:%
%:%4789=2413%:%
%:%4790=2414%:%
%:%4791=2414%:%
%:%4792=2414%:%
%:%4793=2415%:%
%:%4794=2415%:%
%:%4795=2415%:%
%:%4796=2416%:%
%:%4797=2416%:%
%:%4798=2416%:%
%:%4799=2417%:%
%:%4800=2417%:%
%:%4801=2418%:%
%:%4802=2418%:%
%:%4803=2418%:%
%:%4804=2419%:%
%:%4805=2419%:%
%:%4806=2420%:%
%:%4807=2420%:%
%:%4808=2420%:%
%:%4809=2421%:%
%:%4810=2421%:%
%:%4811=2422%:%
%:%4812=2422%:%
%:%4813=2423%:%
%:%4814=2423%:%
%:%4815=2424%:%
%:%4816=2424%:%
%:%4817=2425%:%
%:%4827=2427%:%
%:%4828=2428%:%
%:%4829=2429%:%
%:%4831=2431%:%
%:%4832=2431%:%
%:%4833=2432%:%
%:%4834=2433%:%
%:%4835=2434%:%
%:%4836=2435%:%
%:%4837=2436%:%
%:%4838=2437%:%
%:%4845=2438%:%
%:%4846=2438%:%
%:%4847=2439%:%
%:%4848=2439%:%
%:%4849=2440%:%
%:%4850=2440%:%
%:%4851=2440%:%
%:%4852=2441%:%
%:%4853=2441%:%
%:%4854=2442%:%
%:%4855=2442%:%
%:%4856=2443%:%
%:%4857=2443%:%
%:%4858=2444%:%
%:%4859=2444%:%
%:%4860=2444%:%
%:%4861=2445%:%
%:%4862=2445%:%
%:%4863=2446%:%
%:%4864=2446%:%
%:%4865=2446%:%
%:%4866=2447%:%
%:%4867=2447%:%
%:%4868=2448%:%
%:%4869=2448%:%
%:%4870=2448%:%
%:%4871=2449%:%
%:%4872=2449%:%
%:%4873=2449%:%
%:%4874=2450%:%
%:%4875=2450%:%
%:%4876=2450%:%
%:%4877=2451%:%
%:%4878=2451%:%
%:%4879=2451%:%
%:%4880=2452%:%
%:%4881=2452%:%
%:%4882=2453%:%
%:%4883=2453%:%
%:%4884=2453%:%
%:%4885=2454%:%
%:%4886=2454%:%
%:%4887=2455%:%
%:%4888=2455%:%
%:%4889=2455%:%
%:%4890=2456%:%
%:%4891=2456%:%
%:%4892=2457%:%
%:%4893=2457%:%
%:%4894=2458%:%
%:%4895=2458%:%
%:%4896=2459%:%
%:%4897=2459%:%
%:%4898=2460%:%
%:%4899=2460%:%
%:%4900=2460%:%
%:%4901=2461%:%
%:%4902=2461%:%
%:%4903=2462%:%
%:%4904=2462%:%
%:%4905=2462%:%
%:%4906=2463%:%
%:%4907=2463%:%
%:%4908=2464%:%
%:%4909=2464%:%
%:%4910=2464%:%
%:%4911=2465%:%
%:%4912=2465%:%
%:%4913=2465%:%
%:%4914=2466%:%
%:%4915=2466%:%
%:%4916=2466%:%
%:%4917=2467%:%
%:%4918=2467%:%
%:%4919=2468%:%
%:%4920=2468%:%
%:%4921=2468%:%
%:%4922=2469%:%
%:%4923=2469%:%
%:%4924=2470%:%
%:%4925=2470%:%
%:%4926=2470%:%
%:%4927=2471%:%
%:%4928=2471%:%
%:%4929=2472%:%
%:%4930=2472%:%
%:%4931=2473%:%
%:%4932=2473%:%
%:%4933=2474%:%
%:%4934=2474%:%
%:%4935=2475%:%
%:%4936=2475%:%
%:%4937=2476%:%
%:%4938=2476%:%
%:%4939=2476%:%
%:%4940=2477%:%
%:%4941=2477%:%
%:%4942=2478%:%
%:%4943=2478%:%
%:%4944=2478%:%
%:%4945=2479%:%
%:%4946=2479%:%
%:%4947=2480%:%
%:%4948=2480%:%
%:%4949=2480%:%
%:%4950=2481%:%
%:%4951=2481%:%
%:%4952=2481%:%
%:%4953=2482%:%
%:%4954=2482%:%
%:%4955=2482%:%
%:%4956=2483%:%
%:%4957=2483%:%
%:%4958=2484%:%
%:%4959=2484%:%
%:%4960=2484%:%
%:%4961=2485%:%
%:%4962=2485%:%
%:%4963=2486%:%
%:%4964=2486%:%
%:%4965=2486%:%
%:%4966=2487%:%
%:%4967=2487%:%
%:%4968=2488%:%
%:%4969=2488%:%
%:%4970=2489%:%
%:%4971=2489%:%
%:%4972=2490%:%
%:%4973=2490%:%
%:%4974=2491%:%
%:%4984=2493%:%
%:%4985=2494%:%
%:%4986=2495%:%
%:%4988=2497%:%
%:%4989=2497%:%
%:%4990=2498%:%
%:%4991=2499%:%
%:%4992=2500%:%
%:%4993=2501%:%
%:%4994=2502%:%
%:%4995=2503%:%
%:%4996=2504%:%
%:%5003=2505%:%
%:%5004=2505%:%
%:%5005=2506%:%
%:%5006=2506%:%
%:%5007=2507%:%
%:%5008=2507%:%
%:%5009=2507%:%
%:%5010=2508%:%
%:%5011=2508%:%
%:%5012=2509%:%
%:%5013=2509%:%
%:%5014=2510%:%
%:%5015=2510%:%
%:%5016=2511%:%
%:%5017=2511%:%
%:%5018=2511%:%
%:%5019=2512%:%
%:%5020=2512%:%
%:%5021=2513%:%
%:%5022=2513%:%
%:%5023=2513%:%
%:%5024=2514%:%
%:%5025=2514%:%
%:%5026=2515%:%
%:%5027=2515%:%
%:%5028=2515%:%
%:%5029=2516%:%
%:%5030=2516%:%
%:%5031=2516%:%
%:%5032=2517%:%
%:%5033=2517%:%
%:%5034=2517%:%
%:%5035=2518%:%
%:%5036=2518%:%
%:%5037=2518%:%
%:%5038=2519%:%
%:%5039=2519%:%
%:%5040=2520%:%
%:%5041=2520%:%
%:%5042=2520%:%
%:%5043=2521%:%
%:%5044=2521%:%
%:%5045=2522%:%
%:%5046=2522%:%
%:%5047=2522%:%
%:%5048=2523%:%
%:%5049=2523%:%
%:%5050=2524%:%
%:%5051=2524%:%
%:%5052=2524%:%
%:%5053=2525%:%
%:%5054=2525%:%
%:%5055=2526%:%
%:%5056=2526%:%
%:%5057=2526%:%
%:%5058=2527%:%
%:%5059=2527%:%
%:%5060=2527%:%
%:%5061=2528%:%
%:%5062=2528%:%
%:%5063=2528%:%
%:%5064=2529%:%
%:%5065=2529%:%
%:%5066=2530%:%
%:%5067=2530%:%
%:%5068=2530%:%
%:%5069=2531%:%
%:%5070=2531%:%
%:%5071=2532%:%
%:%5072=2532%:%
%:%5073=2532%:%
%:%5074=2533%:%
%:%5075=2533%:%
%:%5076=2534%:%
%:%5077=2534%:%
%:%5078=2535%:%
%:%5079=2535%:%
%:%5080=2536%:%
%:%5090=2538%:%
%:%5091=2539%:%
%:%5092=2540%:%
%:%5093=2541%:%
%:%5095=2543%:%
%:%5096=2543%:%
%:%5097=2544%:%
%:%5098=2545%:%
%:%5099=2546%:%
%:%5100=2547%:%
%:%5101=2548%:%
%:%5102=2549%:%
%:%5109=2550%:%
%:%5110=2550%:%
%:%5111=2551%:%
%:%5112=2551%:%
%:%5115=2554%:%
%:%5116=2555%:%
%:%5117=2555%:%
%:%5118=2555%:%
%:%5119=2556%:%
%:%5120=2556%:%
%:%5121=2557%:%
%:%5122=2557%:%
%:%5123=2558%:%
%:%5124=2558%:%
%:%5125=2559%:%
%:%5126=2559%:%
%:%5127=2560%:%
%:%5128=2560%:%
%:%5129=2560%:%
%:%5130=2561%:%
%:%5131=2561%:%
%:%5132=2562%:%
%:%5133=2562%:%
%:%5135=2564%:%
%:%5136=2565%:%
%:%5137=2565%:%
%:%5138=2566%:%
%:%5139=2566%:%
%:%5140=2567%:%
%:%5141=2567%:%
%:%5142=2568%:%
%:%5143=2568%:%
%:%5144=2569%:%
%:%5145=2569%:%
%:%5146=2569%:%
%:%5147=2570%:%
%:%5148=2570%:%
%:%5149=2571%:%
%:%5150=2571%:%
%:%5151=2571%:%
%:%5152=2572%:%
%:%5153=2572%:%
%:%5154=2573%:%
%:%5155=2573%:%
%:%5156=2573%:%
%:%5157=2574%:%
%:%5158=2574%:%
%:%5159=2575%:%
%:%5160=2575%:%
%:%5161=2575%:%
%:%5162=2576%:%
%:%5163=2576%:%
%:%5164=2577%:%
%:%5165=2577%:%
%:%5166=2577%:%
%:%5167=2578%:%
%:%5168=2578%:%
%:%5169=2579%:%
%:%5170=2579%:%
%:%5171=2580%:%
%:%5172=2580%:%
%:%5173=2581%:%
%:%5174=2581%:%
%:%5175=2582%:%
%:%5176=2583%:%
%:%5177=2583%:%
%:%5178=2584%:%
%:%5179=2584%:%
%:%5180=2585%:%
%:%5181=2585%:%
%:%5182=2586%:%
%:%5183=2586%:%
%:%5184=2587%:%
%:%5185=2587%:%
%:%5186=2587%:%
%:%5187=2588%:%
%:%5188=2588%:%
%:%5189=2589%:%
%:%5190=2589%:%
%:%5191=2589%:%
%:%5192=2590%:%
%:%5193=2590%:%
%:%5194=2591%:%
%:%5195=2591%:%
%:%5196=2591%:%
%:%5197=2592%:%
%:%5198=2592%:%
%:%5199=2593%:%
%:%5200=2593%:%
%:%5201=2593%:%
%:%5202=2594%:%
%:%5203=2594%:%
%:%5204=2595%:%
%:%5205=2595%:%
%:%5206=2595%:%
%:%5207=2596%:%
%:%5208=2596%:%
%:%5209=2597%:%
%:%5210=2597%:%
%:%5211=2598%:%
%:%5212=2598%:%
%:%5213=2599%:%
%:%5214=2599%:%
%:%5215=2600%:%
%:%5216=2600%:%
%:%5217=2600%:%
%:%5218=2601%:%
%:%5219=2601%:%
%:%5220=2602%:%
%:%5221=2602%:%
%:%5222=2603%:%
%:%5223=2603%:%
%:%5224=2603%:%
%:%5225=2604%:%
%:%5226=2604%:%
%:%5227=2605%:%
%:%5228=2605%:%
%:%5229=2605%:%
%:%5230=2606%:%
%:%5231=2606%:%
%:%5232=2607%:%
%:%5233=2607%:%
%:%5234=2608%:%
%:%5235=2608%:%
%:%5236=2609%:%
%:%5246=2611%:%
%:%5247=2612%:%
%:%5248=2613%:%
%:%5250=2615%:%
%:%5251=2615%:%
%:%5252=2616%:%
%:%5253=2617%:%
%:%5260=2618%:%
%:%5261=2618%:%
%:%5262=2619%:%
%:%5263=2619%:%
%:%5264=2620%:%
%:%5265=2620%:%
%:%5266=2621%:%
%:%5267=2621%:%
%:%5268=2622%:%
%:%5269=2622%:%
%:%5270=2623%:%
%:%5271=2623%:%
%:%5272=2624%:%
%:%5273=2624%:%
%:%5274=2625%:%
%:%5275=2625%:%
%:%5276=2626%:%
%:%5277=2626%:%
%:%5278=2627%:%
%:%5279=2627%:%
%:%5280=2627%:%
%:%5281=2628%:%
%:%5282=2628%:%
%:%5283=2629%:%
%:%5284=2629%:%
%:%5285=2629%:%
%:%5286=2630%:%
%:%5287=2630%:%
%:%5288=2631%:%
%:%5289=2631%:%
%:%5290=2632%:%
%:%5291=2632%:%
%:%5292=2632%:%
%:%5293=2633%:%
%:%5294=2633%:%
%:%5295=2634%:%
%:%5296=2634%:%
%:%5297=2634%:%
%:%5298=2635%:%
%:%5299=2635%:%
%:%5300=2636%:%
%:%5301=2636%:%
%:%5302=2636%:%
%:%5303=2637%:%
%:%5304=2637%:%
%:%5305=2638%:%
%:%5306=2638%:%
%:%5307=2638%:%
%:%5308=2639%:%
%:%5309=2639%:%
%:%5310=2640%:%
%:%5311=2640%:%
%:%5312=2640%:%
%:%5313=2641%:%
%:%5314=2641%:%
%:%5315=2642%:%
%:%5316=2642%:%
%:%5317=2642%:%
%:%5318=2643%:%
%:%5319=2643%:%
%:%5320=2644%:%
%:%5321=2644%:%
%:%5322=2644%:%
%:%5323=2645%:%
%:%5324=2645%:%
%:%5325=2646%:%
%:%5326=2646%:%
%:%5327=2646%:%
%:%5328=2647%:%
%:%5329=2647%:%
%:%5330=2648%:%
%:%5331=2648%:%
%:%5332=2648%:%
%:%5333=2649%:%
%:%5334=2649%:%
%:%5335=2650%:%
%:%5336=2650%:%
%:%5337=2651%:%
%:%5338=2651%:%
%:%5339=2651%:%
%:%5340=2652%:%
%:%5341=2652%:%
%:%5342=2653%:%
%:%5343=2653%:%
%:%5344=2653%:%
%:%5345=2654%:%
%:%5346=2654%:%
%:%5347=2655%:%
%:%5348=2655%:%
%:%5349=2656%:%
%:%5350=2656%:%
%:%5351=2656%:%
%:%5352=2657%:%
%:%5353=2657%:%
%:%5354=2658%:%
%:%5355=2658%:%
%:%5356=2658%:%
%:%5357=2659%:%
%:%5358=2659%:%
%:%5359=2660%:%
%:%5360=2660%:%
%:%5361=2660%:%
%:%5362=2661%:%
%:%5363=2661%:%
%:%5364=2661%:%
%:%5365=2662%:%
%:%5366=2662%:%
%:%5367=2663%:%
%:%5368=2663%:%
%:%5369=2664%:%
%:%5370=2664%:%
%:%5371=2664%:%
%:%5372=2665%:%
%:%5373=2665%:%
%:%5374=2666%:%
%:%5375=2666%:%
%:%5376=2666%:%
%:%5377=2667%:%
%:%5378=2667%:%
%:%5379=2668%:%
%:%5380=2668%:%
%:%5381=2668%:%
%:%5382=2669%:%
%:%5383=2669%:%
%:%5384=2670%:%
%:%5385=2670%:%
%:%5386=2670%:%
%:%5387=2671%:%
%:%5388=2671%:%
%:%5389=2671%:%
%:%5390=2672%:%
%:%5391=2672%:%
%:%5392=2673%:%
%:%5393=2673%:%
%:%5394=2673%:%
%:%5395=2674%:%
%:%5396=2674%:%
%:%5397=2675%:%
%:%5398=2675%:%
%:%5399=2675%:%
%:%5400=2676%:%
%:%5401=2676%:%
%:%5402=2676%:%
%:%5403=2677%:%
%:%5404=2677%:%
%:%5405=2678%:%
%:%5406=2678%:%
%:%5407=2678%:%
%:%5408=2679%:%
%:%5409=2679%:%
%:%5410=2680%:%
%:%5411=2680%:%
%:%5412=2681%:%
%:%5413=2681%:%
%:%5414=2681%:%
%:%5415=2682%:%
%:%5416=2682%:%
%:%5417=2683%:%
%:%5418=2683%:%
%:%5419=2684%:%
%:%5429=2686%:%
%:%5430=2687%:%
%:%5431=2688%:%
%:%5433=2690%:%
%:%5434=2690%:%
%:%5441=2691%:%
%:%5442=2691%:%
%:%5443=2692%:%
%:%5444=2692%:%
%:%5447=2695%:%
%:%5448=2696%:%
%:%5449=2696%:%
%:%5450=2697%:%
%:%5451=2697%:%
%:%5452=2698%:%
%:%5453=2698%:%
%:%5454=2699%:%
%:%5455=2699%:%
%:%5456=2700%:%
%:%5457=2700%:%
%:%5458=2700%:%
%:%5459=2701%:%
%:%5460=2701%:%
%:%5461=2702%:%
%:%5462=2702%:%
%:%5463=2702%:%
%:%5464=2703%:%
%:%5465=2703%:%
%:%5466=2704%:%
%:%5467=2704%:%
%:%5468=2704%:%
%:%5469=2705%:%
%:%5470=2705%:%
%:%5471=2706%:%
%:%5472=2706%:%
%:%5473=2706%:%
%:%5474=2707%:%
%:%5475=2707%:%
%:%5478=2710%:%
%:%5479=2711%:%
%:%5480=2711%:%
%:%5481=2711%:%
%:%5482=2712%:%
%:%5483=2712%:%
%:%5484=2713%:%
%:%5494=2715%:%
%:%5496=2717%:%
%:%5497=2717%:%
%:%5498=2718%:%
%:%5499=2719%:%
%:%5500=2720%:%
%:%5507=2721%:%
%:%5508=2721%:%
%:%5509=2722%:%
%:%5510=2722%:%
%:%5511=2723%:%
%:%5512=2723%:%
%:%5513=2723%:%
%:%5514=2724%:%
%:%5515=2724%:%
%:%5516=2725%:%
%:%5517=2725%:%
%:%5518=2726%:%
%:%5519=2726%:%