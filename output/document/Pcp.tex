%
\begin{isabellebody}%
\setisabellecontext{Pcp}%
%
\isadelimtheory
%
\endisadelimtheory
%
\isatagtheory
%
\endisatagtheory
{\isafoldtheory}%
%
\isadelimtheory
%
\endisadelimtheory
%
\begin{isamarkuptext}%
En esta sección vamos a definir la \isa{propiedad\ de\ consistencia\ proposicional} para una 
  colección de conjuntos de fórmulas proposicionales. De este modo, el \isa{Teorema\ de\ Existencia\ de\ Modelo} demuestra la satisfacibilidad de un conjunto de fórmulas si este pertenece a una 
  colección de conjuntos que verifica dicha propiedad. Al final de la sección mostraremos un 
  lema que caracteriza la propiedad de consistencia proposicional mediante la notación uniforme.

  \begin{definicion}
    Sea \isa{C} una colección de conjuntos de fórmulas proposicionales. Decimos que
    \isa{C} verifica la \isa{propiedad\ de\ consistencia\ proposicional} si, para todo
    conjunto \isa{S} perteneciente a la colección, se verifica:
    \begin{enumerate}
      \item \isa{{\isasymbottom}\ {\isasymnotin}\ S}.
      \item Dada \isa{p} una fórmula atómica cualquiera, no se tiene 
        simultáneamente que\\ \isa{p\ {\isasymin}\ S} y \isa{{\isasymnot}\ p\ {\isasymin}\ S}.
      \item Si \isa{F\ {\isasymand}\ G\ {\isasymin}\ S}, entonces el conjunto \isa{{\isacharbraceleft}F{\isacharcomma}G{\isacharbraceright}\ {\isasymunion}\ S} pertenece a \isa{C}.
      \item Si \isa{F\ {\isasymor}\ G\ {\isasymin}\ S}, entonces o bien el conjunto \isa{{\isacharbraceleft}F{\isacharbraceright}\ {\isasymunion}\ S} pertenece a \isa{C}, o bien el 
        conjunto \isa{{\isacharbraceleft}G{\isacharbraceright}\ {\isasymunion}\ S} pertenece a \isa{C}.
      \item Si \isa{F\ {\isasymrightarrow}\ G\ {\isasymin}\ S}, entonces o bien el conjunto \isa{{\isacharbraceleft}{\isasymnot}\ F{\isacharbraceright}\ {\isasymunion}\ S} pertenece a \isa{C}, o bien el 
        conjunto \isa{{\isacharbraceleft}G{\isacharbraceright}\ {\isasymunion}\ S} pertenece a \isa{C}.
      \item Si \isa{{\isasymnot}{\isacharparenleft}{\isasymnot}\ F{\isacharparenright}\ {\isasymin}\ S}, entonces el conjunto \isa{{\isacharbraceleft}F{\isacharbraceright}\ {\isasymunion}\ S} pertenece a \isa{C}.
      \item Si \isa{{\isasymnot}{\isacharparenleft}F\ {\isasymand}\ G{\isacharparenright}\ {\isasymin}\ S}, entonces o bien el conjunto \isa{{\isacharbraceleft}{\isasymnot}\ F{\isacharbraceright}\ {\isasymunion}\ S} pertenece a \isa{C}, o bien el 
        conjunto \isa{{\isacharbraceleft}{\isasymnot}\ G{\isacharbraceright}\ {\isasymunion}\ S} pertenece a \isa{C}.
      \item Si \isa{{\isasymnot}{\isacharparenleft}F\ {\isasymor}\ G{\isacharparenright}\ {\isasymin}\ S}, entonces el conjunto \isa{{\isacharbraceleft}{\isasymnot}\ F{\isacharcomma}\ {\isasymnot}\ G{\isacharbraceright}\ {\isasymunion}\ S} pertenece a \isa{C}.
      \item Si \isa{{\isasymnot}{\isacharparenleft}F\ {\isasymrightarrow}\ G{\isacharparenright}\ {\isasymin}\ S}, entonces el conjunto \isa{{\isacharbraceleft}F{\isacharcomma}\ {\isasymnot}\ G{\isacharbraceright}\ {\isasymunion}\ S} pertenece a \isa{C}.
    \end{enumerate}
  \end{definicion}

  Veamos, a continuación, su formalización en Isabelle mediante el tipo \isa{definition}.%
\end{isamarkuptext}\isamarkuptrue%
\isacommand{definition}\isamarkupfalse%
\ {\isachardoublequoteopen}pcp\ C\ {\isasymequiv}\ {\isacharparenleft}{\isasymforall}S\ {\isasymin}\ C{\isachardot}\isanewline
\ \ {\isasymbottom}\ {\isasymnotin}\ S\isanewline
{\isasymand}\ {\isacharparenleft}{\isasymforall}k{\isachardot}\ Atom\ k\ {\isasymin}\ S\ {\isasymlongrightarrow}\ \isactrlbold {\isasymnot}\ {\isacharparenleft}Atom\ k{\isacharparenright}\ {\isasymin}\ S\ {\isasymlongrightarrow}\ False{\isacharparenright}\isanewline
{\isasymand}\ {\isacharparenleft}{\isasymforall}F\ G{\isachardot}\ F\ \isactrlbold {\isasymand}\ G\ {\isasymin}\ S\ {\isasymlongrightarrow}\ {\isacharbraceleft}F{\isacharcomma}G{\isacharbraceright}\ {\isasymunion}\ S\ {\isasymin}\ C{\isacharparenright}\isanewline
{\isasymand}\ {\isacharparenleft}{\isasymforall}F\ G{\isachardot}\ F\ \isactrlbold {\isasymor}\ G\ {\isasymin}\ S\ {\isasymlongrightarrow}\ {\isacharbraceleft}F{\isacharbraceright}\ {\isasymunion}\ S\ {\isasymin}\ C\ {\isasymor}\ {\isacharbraceleft}G{\isacharbraceright}\ {\isasymunion}\ S\ {\isasymin}\ C{\isacharparenright}\isanewline
{\isasymand}\ {\isacharparenleft}{\isasymforall}F\ G{\isachardot}\ F\ \isactrlbold {\isasymrightarrow}\ G\ {\isasymin}\ S\ {\isasymlongrightarrow}\ {\isacharbraceleft}\isactrlbold {\isasymnot}F{\isacharbraceright}\ {\isasymunion}\ S\ {\isasymin}\ C\ {\isasymor}\ {\isacharbraceleft}G{\isacharbraceright}\ {\isasymunion}\ S\ {\isasymin}\ C{\isacharparenright}\isanewline
{\isasymand}\ {\isacharparenleft}{\isasymforall}F{\isachardot}\ \isactrlbold {\isasymnot}\ {\isacharparenleft}\isactrlbold {\isasymnot}F{\isacharparenright}\ {\isasymin}\ S\ {\isasymlongrightarrow}\ {\isacharbraceleft}F{\isacharbraceright}\ {\isasymunion}\ S\ {\isasymin}\ C{\isacharparenright}\isanewline
{\isasymand}\ {\isacharparenleft}{\isasymforall}F\ G{\isachardot}\ \isactrlbold {\isasymnot}{\isacharparenleft}F\ \isactrlbold {\isasymand}\ G{\isacharparenright}\ {\isasymin}\ S\ {\isasymlongrightarrow}\ {\isacharbraceleft}\isactrlbold {\isasymnot}\ F{\isacharbraceright}\ {\isasymunion}\ S\ {\isasymin}\ C\ {\isasymor}\ {\isacharbraceleft}\isactrlbold {\isasymnot}\ G{\isacharbraceright}\ {\isasymunion}\ S\ {\isasymin}\ C{\isacharparenright}\isanewline
{\isasymand}\ {\isacharparenleft}{\isasymforall}F\ G{\isachardot}\ \isactrlbold {\isasymnot}{\isacharparenleft}F\ \isactrlbold {\isasymor}\ G{\isacharparenright}\ {\isasymin}\ S\ {\isasymlongrightarrow}\ {\isacharbraceleft}\isactrlbold {\isasymnot}\ F{\isacharcomma}\ \isactrlbold {\isasymnot}\ G{\isacharbraceright}\ {\isasymunion}\ S\ {\isasymin}\ C{\isacharparenright}\isanewline
{\isasymand}\ {\isacharparenleft}{\isasymforall}F\ G{\isachardot}\ \isactrlbold {\isasymnot}{\isacharparenleft}F\ \isactrlbold {\isasymrightarrow}\ G{\isacharparenright}\ {\isasymin}\ S\ {\isasymlongrightarrow}\ {\isacharbraceleft}F{\isacharcomma}\isactrlbold {\isasymnot}\ G{\isacharbraceright}\ {\isasymunion}\ S\ {\isasymin}\ C{\isacharparenright}{\isacharparenright}{\isachardoublequoteclose}%
\begin{isamarkuptext}%
Observando la definición anterior, se prueba fácilmente que la colección trivial
  formada por el conjunto vacío de fórmulas verifica la propiedad de consistencia 
  proposicional.%
\end{isamarkuptext}\isamarkuptrue%
\isacommand{lemma}\isamarkupfalse%
\ {\isachardoublequoteopen}pcp\ {\isacharbraceleft}{\isacharbraceleft}{\isacharbraceright}{\isacharbraceright}{\isachardoublequoteclose}\isanewline
%
\isadelimproof
\ \ %
\endisadelimproof
%
\isatagproof
\isacommand{unfolding}\isamarkupfalse%
\ pcp{\isacharunderscore}def\ \isacommand{by}\isamarkupfalse%
\ simp%
\endisatagproof
{\isafoldproof}%
%
\isadelimproof
%
\endisadelimproof
%
\begin{isamarkuptext}%
Del mismo modo, aplicando la definición, se demuestra que los siguientes ejemplos
  de colecciones de conjuntos de fórmulas proposicionales verifican igualmente la 
  propiedad.%
\end{isamarkuptext}\isamarkuptrue%
\isacommand{lemma}\isamarkupfalse%
\ {\isachardoublequoteopen}pcp\ {\isacharbraceleft}{\isacharbraceleft}Atom\ {\isadigit{0}}{\isacharbraceright}{\isacharbraceright}{\isachardoublequoteclose}\isanewline
%
\isadelimproof
\ \ %
\endisadelimproof
%
\isatagproof
\isacommand{unfolding}\isamarkupfalse%
\ pcp{\isacharunderscore}def\ \isacommand{by}\isamarkupfalse%
\ simp%
\endisatagproof
{\isafoldproof}%
%
\isadelimproof
\isanewline
%
\endisadelimproof
\isanewline
\isacommand{lemma}\isamarkupfalse%
\ {\isachardoublequoteopen}pcp\ {\isacharbraceleft}{\isacharbraceleft}{\isacharparenleft}\isactrlbold {\isasymnot}\ {\isacharparenleft}Atom\ {\isadigit{1}}{\isacharparenright}{\isacharparenright}\ \isactrlbold {\isasymrightarrow}\ Atom\ {\isadigit{2}}{\isacharbraceright}{\isacharcomma}\isanewline
\ \ \ {\isacharbraceleft}{\isacharparenleft}{\isacharparenleft}\isactrlbold {\isasymnot}\ {\isacharparenleft}Atom\ {\isadigit{1}}{\isacharparenright}{\isacharparenright}\ \isactrlbold {\isasymrightarrow}\ Atom\ {\isadigit{2}}{\isacharparenright}{\isacharcomma}\ \isactrlbold {\isasymnot}{\isacharparenleft}\isactrlbold {\isasymnot}\ {\isacharparenleft}Atom\ {\isadigit{1}}{\isacharparenright}{\isacharparenright}{\isacharbraceright}{\isacharcomma}\isanewline
\ \ {\isacharbraceleft}{\isacharparenleft}{\isacharparenleft}\isactrlbold {\isasymnot}\ {\isacharparenleft}Atom\ {\isadigit{1}}{\isacharparenright}{\isacharparenright}\ \isactrlbold {\isasymrightarrow}\ Atom\ {\isadigit{2}}{\isacharparenright}{\isacharcomma}\ \isactrlbold {\isasymnot}{\isacharparenleft}\isactrlbold {\isasymnot}\ {\isacharparenleft}Atom\ {\isadigit{1}}{\isacharparenright}{\isacharparenright}{\isacharcomma}\ \ Atom\ {\isadigit{1}}{\isacharbraceright}{\isacharbraceright}{\isachardoublequoteclose}\ \isanewline
%
\isadelimproof
\ \ %
\endisadelimproof
%
\isatagproof
\isacommand{unfolding}\isamarkupfalse%
\ pcp{\isacharunderscore}def\ \isacommand{by}\isamarkupfalse%
\ auto%
\endisatagproof
{\isafoldproof}%
%
\isadelimproof
%
\endisadelimproof
%
\begin{isamarkuptext}%
Por último, en contraposición podemos ilustrar un caso de colección que no verifique la 
  propiedad con la siguiente colección obtenida al modificar el último ejemplo. De
  esta manera, aunque la colección verifique correctamente la quinta condición de la
  definición, no cumplirá la sexta.%
\end{isamarkuptext}\isamarkuptrue%
\isacommand{lemma}\isamarkupfalse%
\ {\isachardoublequoteopen}{\isasymnot}\ pcp\ {\isacharbraceleft}{\isacharbraceleft}{\isacharparenleft}\isactrlbold {\isasymnot}\ {\isacharparenleft}Atom\ {\isadigit{1}}{\isacharparenright}{\isacharparenright}\ \isactrlbold {\isasymrightarrow}\ Atom\ {\isadigit{2}}{\isacharbraceright}{\isacharcomma}\isanewline
\ \ \ {\isacharbraceleft}{\isacharparenleft}{\isacharparenleft}\isactrlbold {\isasymnot}\ {\isacharparenleft}Atom\ {\isadigit{1}}{\isacharparenright}{\isacharparenright}\ \isactrlbold {\isasymrightarrow}\ Atom\ {\isadigit{2}}{\isacharparenright}{\isacharcomma}\ \isactrlbold {\isasymnot}{\isacharparenleft}\isactrlbold {\isasymnot}\ {\isacharparenleft}Atom\ {\isadigit{1}}{\isacharparenright}{\isacharparenright}{\isacharbraceright}{\isacharbraceright}{\isachardoublequoteclose}\ \isanewline
%
\isadelimproof
\ \ %
\endisadelimproof
%
\isatagproof
\isacommand{unfolding}\isamarkupfalse%
\ pcp{\isacharunderscore}def\ \isacommand{by}\isamarkupfalse%
\ auto%
\endisatagproof
{\isafoldproof}%
%
\isadelimproof
%
\endisadelimproof
%
\begin{isamarkuptext}%
Por otra parte, veamos un resultado que permite la caracterización de la 
  propiedad de consistencia proposicional mediante la notación uniforme.

  \begin{lema}[Caracterización de \isa{P{\isachardot}C{\isachardot}P} mediante la notación uniforme]
    Dada una colección \isa{C} de conjuntos de fórmulas proposicionales, son equivalentes:
    \begin{enumerate}
      \item \isa{C} verifica la propiedad de consistencia proposicional.
      \item Para cualquier conjunto de fórmulas \isa{S} de la colección, se verifican las 
      condiciones:
      \begin{itemize}
        \item \isa{{\isasymbottom}} no pertenece a \isa{S}.
        \item Dada \isa{p} una fórmula atómica cualquiera, no se tiene 
        simultáneamente que\\ \isa{p\ {\isasymin}\ S} y \isa{{\isasymnot}\ p\ {\isasymin}\ S}.
        \item Para toda fórmula de tipo \isa{{\isasymalpha}} con componentes \isa{{\isasymalpha}\isactrlsub {\isadigit{1}}} y \isa{{\isasymalpha}\isactrlsub {\isadigit{2}}} tal que \isa{{\isasymalpha}}
        pertenece a \isa{S}, se tiene que \isa{{\isacharbraceleft}{\isasymalpha}\isactrlsub {\isadigit{1}}{\isacharcomma}{\isasymalpha}\isactrlsub {\isadigit{2}}{\isacharbraceright}\ {\isasymunion}\ S} pertenece a \isa{C}.
        \item Para toda fórmula de tipo \isa{{\isasymbeta}} con componentes \isa{{\isasymbeta}\isactrlsub {\isadigit{1}}} y \isa{{\isasymbeta}\isactrlsub {\isadigit{2}}} tal que \isa{{\isasymbeta}}
        pertenece a \isa{S}, se tiene que o bien \isa{{\isacharbraceleft}{\isasymbeta}\isactrlsub {\isadigit{1}}{\isacharbraceright}\ {\isasymunion}\ S} pertenece a \isa{C} o 
        bien \isa{{\isacharbraceleft}{\isasymbeta}\isactrlsub {\isadigit{2}}{\isacharbraceright}\ {\isasymunion}\ S} pertenece a \isa{C}.
      \end{itemize} 
    \end{enumerate}
  \end{lema}

  En Isabelle/HOL se formaliza el resultado como sigue.%
\end{isamarkuptext}\isamarkuptrue%
\isacommand{lemma}\isamarkupfalse%
\ {\isachardoublequoteopen}pcp\ C\ {\isacharequal}\ {\isacharparenleft}{\isasymforall}S\ {\isasymin}\ C{\isachardot}\ {\isasymbottom}\ {\isasymnotin}\ S\isanewline
{\isasymand}\ {\isacharparenleft}{\isasymforall}k{\isachardot}\ Atom\ k\ {\isasymin}\ S\ {\isasymlongrightarrow}\ \isactrlbold {\isasymnot}\ {\isacharparenleft}Atom\ k{\isacharparenright}\ {\isasymin}\ S\ {\isasymlongrightarrow}\ False{\isacharparenright}\isanewline
{\isasymand}\ {\isacharparenleft}{\isasymforall}F\ G\ H{\isachardot}\ Con\ F\ G\ H\ {\isasymlongrightarrow}\ F\ {\isasymin}\ S\ {\isasymlongrightarrow}\ {\isacharbraceleft}G{\isacharcomma}H{\isacharbraceright}\ {\isasymunion}\ S\ {\isasymin}\ C{\isacharparenright}\isanewline
{\isasymand}\ {\isacharparenleft}{\isasymforall}F\ G\ H{\isachardot}\ Dis\ F\ G\ H\ {\isasymlongrightarrow}\ F\ {\isasymin}\ S\ {\isasymlongrightarrow}\ {\isacharbraceleft}G{\isacharbraceright}\ {\isasymunion}\ S\ {\isasymin}\ C\ {\isasymor}\ {\isacharbraceleft}H{\isacharbraceright}\ {\isasymunion}\ S\ {\isasymin}\ C{\isacharparenright}{\isacharparenright}{\isachardoublequoteclose}\isanewline
%
\isadelimproof
\ \ %
\endisadelimproof
%
\isatagproof
\isacommand{oops}\isamarkupfalse%
%
\endisatagproof
{\isafoldproof}%
%
\isadelimproof
%
\endisadelimproof
%
\begin{isamarkuptext}%
En primer lugar, veamos la demostración del lema.

\begin{demostracion}
  Para probar la equivalencia, veamos cada una de las implicaciones por separado.

\textbf{\isa{{\isadigit{1}}{\isacharparenright}\ {\isasymLongrightarrow}\ {\isadigit{2}}{\isacharparenright}}}
  
  Supongamos que \isa{C} es una colección de conjuntos de fórmulas proposicionales que
  verifica la propiedad de consistencia proposicional. Vamos a probar que, en efecto,
  cumple las condiciones de \isa{{\isadigit{2}}{\isacharparenright}}. 

  Consideremos un conjunto de fórmulas \isa{S} perteneciente a la colección \isa{C}.
  Por hipótesis, de la definición de propiedad de consistencia proposicional obtenemos
  que \isa{S} verifica las siguientes condiciones:
 \begin{enumerate}
      \item \isa{{\isasymbottom}\ {\isasymnotin}\ S}.
      \item Dada \isa{p} una fórmula atómica cualquiera, no se tiene 
        simultáneamente que\\ \isa{p\ {\isasymin}\ S} y \isa{{\isasymnot}\ p\ {\isasymin}\ S}.
      \item Si \isa{G\ {\isasymand}\ H\ {\isasymin}\ S}, entonces el conjunto \isa{{\isacharbraceleft}G{\isacharcomma}H{\isacharbraceright}\ {\isasymunion}\ S} pertenece a \isa{C}.
      \item Si \isa{G\ {\isasymor}\ H\ {\isasymin}\ S}, entonces o bien el conjunto \isa{{\isacharbraceleft}G{\isacharbraceright}\ {\isasymunion}\ S} pertenece a \isa{C}, o bien el 
        conjunto \isa{{\isacharbraceleft}H{\isacharbraceright}\ {\isasymunion}\ S} pertenece a \isa{C}.
      \item Si \isa{G\ {\isasymrightarrow}\ H\ {\isasymin}\ S}, entonces o bien el conjunto \isa{{\isacharbraceleft}{\isasymnot}\ G{\isacharbraceright}\ {\isasymunion}\ S} pertenece a \isa{C}, o bien el 
        conjunto \isa{{\isacharbraceleft}H{\isacharbraceright}\ {\isasymunion}\ S} pertenece a \isa{C}.
      \item Si \isa{{\isasymnot}{\isacharparenleft}{\isasymnot}\ G{\isacharparenright}\ {\isasymin}\ S}, entonces el conjunto \isa{{\isacharbraceleft}G{\isacharbraceright}\ {\isasymunion}\ S} pertenece a \isa{C}.
      \item Si \isa{{\isasymnot}{\isacharparenleft}G\ {\isasymand}\ H{\isacharparenright}\ {\isasymin}\ S}, entonces o bien el conjunto \isa{{\isacharbraceleft}{\isasymnot}\ G{\isacharbraceright}\ {\isasymunion}\ S} pertenece a \isa{C}, o bien el 
        conjunto \isa{{\isacharbraceleft}{\isasymnot}\ H{\isacharbraceright}\ {\isasymunion}\ S} pertenece a \isa{C}.
      \item Si \isa{{\isasymnot}{\isacharparenleft}G\ {\isasymor}\ H{\isacharparenright}\ {\isasymin}\ S}, entonces el conjunto \isa{{\isacharbraceleft}{\isasymnot}\ G{\isacharcomma}\ {\isasymnot}\ H{\isacharbraceright}\ {\isasymunion}\ S} pertenece a \isa{C}.
      \item Si \isa{{\isasymnot}{\isacharparenleft}G\ {\isasymrightarrow}\ H{\isacharparenright}\ {\isasymin}\ S}, entonces el conjunto \isa{{\isacharbraceleft}G{\isacharcomma}\ {\isasymnot}\ H{\isacharbraceright}\ {\isasymunion}\ S} pertenece a \isa{C}.
 \end{enumerate}

  Las dos primeras condiciones se corresponden con los dos primeros resultados que queríamos
  demostrar. De este modo, falta probar:
  \begin{itemize}
     \item Para toda fórmula de tipo \isa{{\isasymalpha}} con componentes \isa{{\isasymalpha}\isactrlsub {\isadigit{1}}} y \isa{{\isasymalpha}\isactrlsub {\isadigit{2}}} tal que \isa{{\isasymalpha}}
     pertenece a \isa{S}, se tiene que \isa{{\isacharbraceleft}{\isasymalpha}\isactrlsub {\isadigit{1}}{\isacharcomma}{\isasymalpha}\isactrlsub {\isadigit{2}}{\isacharbraceright}\ {\isasymunion}\ S} pertenece a \isa{C}.
     \item Para toda fórmula de tipo \isa{{\isasymbeta}} con componentes \isa{{\isasymbeta}\isactrlsub {\isadigit{1}}} y \isa{{\isasymbeta}\isactrlsub {\isadigit{2}}} tal que \isa{{\isasymbeta}}
     pertenece a \isa{S}, se tiene que o bien \isa{{\isacharbraceleft}{\isasymbeta}\isactrlsub {\isadigit{1}}{\isacharbraceright}\ {\isasymunion}\ S} pertenece a \isa{C} o 
     bien \isa{{\isacharbraceleft}{\isasymbeta}\isactrlsub {\isadigit{2}}{\isacharbraceright}\ {\isasymunion}\ S} pertenece a \isa{C}.   
  \end{itemize} 

  En primer lugar, vamos a deducir el primer resultado correspondiente a las fórmulas
  de tipo \isa{{\isasymalpha}} de las condiciones tercera, sexta, octava y novena de la definición de 
  propiedad de consistencia proposicional. En efecto, consideremos una fórmula de tipo 
  \isa{{\isasymalpha}} cualquiera con componentes \isa{{\isasymalpha}\isactrlsub {\isadigit{1}}} y \isa{{\isasymalpha}\isactrlsub {\isadigit{2}}} tal que \isa{{\isasymalpha}} pertenece a \isa{S}. Sabemos que 
  la fórmula es de la forma \isa{G\ {\isasymand}\ H}, \isa{{\isasymnot}\ {\isacharparenleft}{\isasymnot}\ G{\isacharparenright}}, \isa{{\isasymnot}\ {\isacharparenleft}G\ {\isasymor}\ H{\isacharparenright}} o 
  \isa{{\isasymnot}{\isacharparenleft}G\ {\isasymlongrightarrow}\ H{\isacharparenright}} para ciertas fórmulas \isa{G} y \isa{H}. Vamos a probar que para cada caso se cumple que 
  \isa{{\isacharbraceleft}{\isasymalpha}\isactrlsub {\isadigit{1}}{\isacharcomma}\ {\isasymalpha}\isactrlsub {\isadigit{2}}{\isacharbraceright}\ {\isasymunion}\ S} pertenece a la colección:

  \isa{{\isasymsqdot}\ Fórmula\ de\ tipo\ G\ {\isasymand}\ H}: En este caso, sus componentes conjuntivas \isa{{\isasymalpha}\isactrlsub {\isadigit{1}}} y \isa{{\isasymalpha}\isactrlsub {\isadigit{2}}} son \isa{G} 
    y \isa{H} respectivamente. Luego tenemos que \isa{{\isacharbraceleft}{\isasymalpha}\isactrlsub {\isadigit{1}}{\isacharcomma}\ {\isasymalpha}\isactrlsub {\isadigit{2}}{\isacharbraceright}\ {\isasymunion}\ S}  pertenece a \isa{C} por
    la tercera condición de la definición de propiedad de consistencia
    proposicional.

  \isa{{\isasymsqdot}\ Fórmula\ de\ tipo\ {\isasymnot}\ {\isacharparenleft}{\isasymnot}\ G{\isacharparenright}}: En este caso, sus componentes conjuntivas \isa{{\isasymalpha}\isactrlsub {\isadigit{1}}} y \isa{{\isasymalpha}\isactrlsub {\isadigit{2}}} son 
    ambas \isa{G}. Como el conjunto \isa{{\isacharbraceleft}{\isasymalpha}\isactrlsub {\isadigit{1}}{\isacharbraceright}\ {\isasymunion}\ S} es equivalente a \isa{{\isacharbraceleft}{\isasymalpha}\isactrlsub {\isadigit{1}}{\isacharcomma}\ {\isasymalpha}\isactrlsub {\isadigit{2}}{\isacharbraceright}\ {\isasymunion}\ S} ya
    que \isa{{\isasymalpha}\isactrlsub {\isadigit{1}}} y \isa{{\isasymalpha}\isactrlsub {\isadigit{2}}} son iguales, tenemos que este último pertenece a \isa{C} por la sexta 
    condición de la definición de propiedad de consistencia proposicional.

  \isa{{\isasymsqdot}\ Fórmula\ de\ tipo\ {\isasymnot}{\isacharparenleft}G\ {\isasymor}\ H{\isacharparenright}}: En este caso, sus componentes conjuntivas \isa{{\isasymalpha}\isactrlsub {\isadigit{1}}} y \isa{{\isasymalpha}\isactrlsub {\isadigit{2}}} son\\ 
    \isa{{\isasymnot}\ G} y \isa{{\isasymnot}\ H} respectivamente. Luego tenemos que \isa{{\isacharbraceleft}{\isasymalpha}\isactrlsub {\isadigit{1}}{\isacharcomma}\ {\isasymalpha}\isactrlsub {\isadigit{2}}{\isacharbraceright}\ {\isasymunion}\ S}  pertenece a \isa{C} por
    la octava condición de la definición de propiedad de consistencia proposicional.

  \isa{{\isasymsqdot}\ Fórmula\ de\ tipo\ {\isasymnot}{\isacharparenleft}G\ {\isasymlongrightarrow}\ H{\isacharparenright}}: En este caso, sus componentes conjuntivas \isa{{\isasymalpha}\isactrlsub {\isadigit{1}}} y \isa{{\isasymalpha}\isactrlsub {\isadigit{2}}} son \isa{G} y 
    \isa{{\isasymnot}\ H} respectivamente. Luego tenemos que \isa{{\isacharbraceleft}{\isasymalpha}\isactrlsub {\isadigit{1}}{\isacharcomma}\ {\isasymalpha}\isactrlsub {\isadigit{2}}{\isacharbraceright}\ {\isasymunion}\ S}  pertenece a \isa{C} por la novena 
    condición de la definición de propiedad de consistencia proposicional.

  Finalmente, el resultado correspondiente a las fórmulas de tipo \isa{{\isasymbeta}} se obtiene de las 
  condiciones cuarta, quinta, sexta y séptima de la definición de propiedad de consistencia 
  proposicional. Para probarlo, consideremos una fórmula cualquiera de tipo \isa{{\isasymbeta}} perteneciente
  al conjunto \isa{S} y cuyas componentes disyuntivas son \isa{{\isasymbeta}\isactrlsub {\isadigit{1}}} y \isa{{\isasymbeta}\isactrlsub {\isadigit{2}}}. Por simplificación, sabemos 
  que dicha fórmula es de la forma \isa{G\ {\isasymor}\ H}, \isa{G\ {\isasymlongrightarrow}\ H}, \isa{{\isasymnot}\ {\isacharparenleft}{\isasymnot}\ G{\isacharparenright}} o \isa{{\isasymnot}{\isacharparenleft}G\ {\isasymand}\ H{\isacharparenright}} para ciertas 
  fórmulas \isa{G} y \isa{H}. Deduzcamos que, en efecto, tenemos que o bien \isa{{\isacharbraceleft}{\isasymbeta}\isactrlsub {\isadigit{1}}{\isacharbraceright}\ {\isasymunion}\ S} está en \isa{C} o bien 
  \isa{{\isacharbraceleft}{\isasymbeta}\isactrlsub {\isadigit{2}}{\isacharbraceright}\ {\isasymunion}\ S} está en \isa{C}.

  \isa{{\isasymsqdot}\ Fórmula\ de\ tipo\ G\ {\isasymor}\ H}: En este caso, sus componentes disyuntivas \isa{{\isasymbeta}\isactrlsub {\isadigit{1}}} y \isa{{\isasymbeta}\isactrlsub {\isadigit{2}}} son \isa{G} y 
    \isa{H} respectivamente. Luego tenemos que o bien \isa{{\isacharbraceleft}{\isasymbeta}\isactrlsub {\isadigit{1}}{\isacharbraceright}\ {\isasymunion}\ S}  pertenece a \isa{C} o bien\\
    \isa{{\isacharbraceleft}{\isasymbeta}\isactrlsub {\isadigit{2}}{\isacharbraceright}\ {\isasymunion}\ S} pertenece a \isa{C} por la cuarta condición de la definición de propiedad de 
    consistencia proposicional.

  \isa{{\isasymsqdot}\ Fórmula\ de\ tipo\ G\ {\isasymlongrightarrow}\ H}: En este caso, sus componentes disyuntivas \isa{{\isasymbeta}\isactrlsub {\isadigit{1}}} y \isa{{\isasymbeta}\isactrlsub {\isadigit{2}}} son\\ 
    \isa{{\isasymnot}\ G} y \isa{H} respectivamente. Luego tenemos que o bien \isa{{\isacharbraceleft}{\isasymbeta}\isactrlsub {\isadigit{1}}{\isacharbraceright}\ {\isasymunion}\ S}  pertenece a \isa{C} o 
    bien\\ \isa{{\isacharbraceleft}{\isasymbeta}\isactrlsub {\isadigit{2}}{\isacharbraceright}\ {\isasymunion}\ S} pertenece a \isa{C} por la quinta condición de la definición de propiedad 
    de consistencia proposicional.

  \isa{{\isasymsqdot}\ Fórmula\ de\ tipo\ {\isasymnot}{\isacharparenleft}{\isasymnot}\ G{\isacharparenright}}: En este caso, sus componentes disyuntivas \isa{{\isasymbeta}\isactrlsub {\isadigit{1}}} y \isa{{\isasymbeta}\isactrlsub {\isadigit{2}}} son ambas 
    \isa{G}. Luego tenemos que, en particular, el conjunto \isa{{\isacharbraceleft}{\isasymbeta}\isactrlsub {\isadigit{1}}{\isacharbraceright}\ {\isasymunion}\ S} pertenece a \isa{C} por la 
    sexta condición de la definición de propiedad de consistencia proposicional. Por tanto, se 
    verifica que o bien \isa{{\isacharbraceleft}{\isasymbeta}\isactrlsub {\isadigit{1}}{\isacharbraceright}\ {\isasymunion}\ S} está en \isa{C} o bien \isa{{\isacharbraceleft}{\isasymbeta}\isactrlsub {\isadigit{2}}{\isacharbraceright}\ {\isasymunion}\ S} está en \isa{C}.

  \isa{{\isasymsqdot}\ Fórmula\ de\ tipo\ {\isasymnot}{\isacharparenleft}G\ {\isasymand}\ H{\isacharparenright}}: En este caso, sus componentes disyuntivas \isa{{\isasymbeta}\isactrlsub {\isadigit{1}}} y \isa{{\isasymbeta}\isactrlsub {\isadigit{2}}} son \\ 
    \isa{{\isasymnot}\ G} y \isa{{\isasymnot}\ H} respectivamente. Luego tenemos que o bien \isa{{\isacharbraceleft}{\isasymbeta}\isactrlsub {\isadigit{1}}{\isacharbraceright}\ {\isasymunion}\ S} pertenece a \isa{C} o 
    bien \isa{{\isacharbraceleft}{\isasymbeta}\isactrlsub {\isadigit{2}}{\isacharbraceright}\ {\isasymunion}\ S} pertenece a \isa{C} por la séptima condición de la definición de propiedad 
    de consistencia proposicional.

  De este modo, queda probada la primera implicación de la equivalencia. Veamos la prueba de la 
  implicación contraria.

\textbf{\isa{{\isadigit{2}}{\isacharparenright}\ {\isasymLongrightarrow}\ {\isadigit{1}}{\isacharparenright}}}

  Supongamos que, dada una colección de conjuntos de fórmulas proposicionales \isa{C}, para cualquier
  conjunto \isa{S} de la colección se verifica:
  \begin{itemize}
    \item \isa{{\isasymbottom}} no pertenece a \isa{S}.
    \item Dada \isa{p} una fórmula atómica cualquiera, no se tiene 
    simultáneamente que\\ \isa{p\ {\isasymin}\ S} y \isa{{\isasymnot}\ p\ {\isasymin}\ S}.
    \item Para toda fórmula de tipo \isa{{\isasymalpha}} con componentes \isa{{\isasymalpha}\isactrlsub {\isadigit{1}}} y \isa{{\isasymalpha}\isactrlsub {\isadigit{2}}} tal que \isa{{\isasymalpha}}
    pertenece a \isa{S}, se tiene que \isa{{\isacharbraceleft}{\isasymalpha}\isactrlsub {\isadigit{1}}{\isacharcomma}{\isasymalpha}\isactrlsub {\isadigit{2}}{\isacharbraceright}\ {\isasymunion}\ S} pertenece a \isa{C}.
    \item Para toda fórmula de tipo \isa{{\isasymbeta}} con componentes \isa{{\isasymbeta}\isactrlsub {\isadigit{1}}} y \isa{{\isasymbeta}\isactrlsub {\isadigit{2}}} tal que \isa{{\isasymbeta}}
    pertenece a \isa{S}, se tiene que o bien \isa{{\isacharbraceleft}{\isasymbeta}\isactrlsub {\isadigit{1}}{\isacharbraceright}\ {\isasymunion}\ S} pertenece a \isa{C} o 
    bien \isa{{\isacharbraceleft}{\isasymbeta}\isactrlsub {\isadigit{2}}{\isacharbraceright}\ {\isasymunion}\ S} pertenece a \isa{C}.
  \end{itemize}

  Probemos que \isa{C} verifica la propiedad de consistencia proposicional. Por la definición
  de la propiedad basta probar que, dado un conjunto cualquiera \isa{S} perteneciente a \isa{C}, se
  verifican las siguientes condiciones:
  \begin{enumerate}
    \item \isa{{\isasymbottom}\ {\isasymnotin}\ S}.
    \item Dada \isa{p} una fórmula atómica cualquiera, no se tiene 
      simultáneamente que\\ \isa{p\ {\isasymin}\ S} y \isa{{\isasymnot}\ p\ {\isasymin}\ S}.
    \item Si \isa{G\ {\isasymand}\ H\ {\isasymin}\ S}, entonces el conjunto \isa{{\isacharbraceleft}G{\isacharcomma}H{\isacharbraceright}\ {\isasymunion}\ S} pertenece a \isa{C}.
    \item Si \isa{G\ {\isasymor}\ H\ {\isasymin}\ S}, entonces o bien el conjunto \isa{{\isacharbraceleft}G{\isacharbraceright}\ {\isasymunion}\ S} pertenece a \isa{C}, o bien el conjunto 
      \isa{{\isacharbraceleft}H{\isacharbraceright}\ {\isasymunion}\ S} pertenece a \isa{C}.
    \item Si \isa{G\ {\isasymrightarrow}\ H\ {\isasymin}\ S}, entonces o bien el conjunto \isa{{\isacharbraceleft}{\isasymnot}\ G{\isacharbraceright}\ {\isasymunion}\ S} pertenece a \isa{C}, o bien el 
      conjunto \isa{{\isacharbraceleft}H{\isacharbraceright}\ {\isasymunion}\ S} pertenece a \isa{C}.
    \item Si \isa{{\isasymnot}{\isacharparenleft}{\isasymnot}\ G{\isacharparenright}\ {\isasymin}\ S}, entonces el conjunto \isa{{\isacharbraceleft}G{\isacharbraceright}\ {\isasymunion}\ S} pertenece a \isa{C}.
    \item Si \isa{{\isasymnot}{\isacharparenleft}G\ {\isasymand}\ H{\isacharparenright}\ {\isasymin}\ S}, entonces o bien el conjunto \isa{{\isacharbraceleft}{\isasymnot}\ G{\isacharbraceright}\ {\isasymunion}\ S} pertenece a \isa{C}, o bien el 
      conjunto \isa{{\isacharbraceleft}{\isasymnot}\ H{\isacharbraceright}\ {\isasymunion}\ S} pertenece a \isa{C}.
    \item Si \isa{{\isasymnot}{\isacharparenleft}G\ {\isasymor}\ H{\isacharparenright}\ {\isasymin}\ S}, entonces el conjunto \isa{{\isacharbraceleft}{\isasymnot}\ G{\isacharcomma}\ {\isasymnot}\ H{\isacharbraceright}\ {\isasymunion}\ S} pertenece a \isa{C}.
    \item Si \isa{{\isasymnot}{\isacharparenleft}G\ {\isasymrightarrow}\ H{\isacharparenright}\ {\isasymin}\ S}, entonces el conjunto \isa{{\isacharbraceleft}G{\isacharcomma}\ {\isasymnot}\ H{\isacharbraceright}\ {\isasymunion}\ S} pertenece a \isa{C}.
  \end{enumerate}

  En primer lugar, se observa que por hipótesis se cumplen las dos primeras condiciones de
  la definición.

  Por otra parte, vamos a deducir las condiciones tercera, sexta, octava y novena de la
  definición de la propiedad de consistencia proposicional a partir de la hipótesis sobre las 
  fórmulas de tipo \isa{{\isasymalpha}}.
  \begin{enumerate}
    \item[\isa{{\isadigit{3}}{\isacharparenright}}:] Supongamos que la fórmula \isa{G\ {\isasymand}\ H} pertenece a \isa{S} para fórmulas \isa{G} y \isa{H}
    cualesquiera. Observemos que se trata de una fórmula de tipo \isa{{\isasymalpha}} de componentes conjuntivas
    \isa{G} y \isa{H}. Luego, por hipótesis, tenemos que \isa{{\isacharbraceleft}G{\isacharcomma}\ H{\isacharbraceright}\ {\isasymunion}\ S} pertenece a \isa{C}.
    \item[\isa{{\isadigit{6}}{\isacharparenright}}:] Supongamos que la fórmula \isa{{\isasymnot}{\isacharparenleft}{\isasymnot}\ G{\isacharparenright}} pertenece a \isa{S} para la fórmula \isa{G} 
    cualquiera. Observemos que se trata de una fórmula de tipo \isa{{\isasymalpha}} cuyas componentes conjuntivas 
    son ambas la fórmula \isa{G}. Por hipótesis, tenemos que el conjunto \isa{{\isacharbraceleft}G{\isacharcomma}G{\isacharbraceright}\ {\isasymunion}\ S} pertence a \isa{C}
    y, puesto que dicho conjunto es equivalente a \isa{{\isacharbraceleft}G{\isacharbraceright}\ {\isasymunion}\ S}, tenemos el resultado.
    \item[\isa{{\isadigit{8}}{\isacharparenright}}:] Supongamos que la fórmula \isa{{\isasymnot}{\isacharparenleft}G\ {\isasymor}\ H{\isacharparenright}} pertenece a \isa{S} para fórmulas \isa{G} y \isa{H}
    cualesquiera. Observemos que se trata de una fórmula de tipo \isa{{\isasymalpha}} de componentes conjuntivas
    \isa{{\isasymnot}\ G} y \isa{{\isasymnot}\ H}. Luego, por hipótesis, tenemos que \isa{{\isacharbraceleft}{\isasymnot}\ G{\isacharcomma}\ {\isasymnot}\ H{\isacharbraceright}\ {\isasymunion}\ S} pertenece a \isa{C}.
    \item[\isa{{\isadigit{9}}{\isacharparenright}}:] Supongamos que la fórmula \isa{{\isasymnot}{\isacharparenleft}G\ {\isasymlongrightarrow}\ H{\isacharparenright}} pertenece a \isa{S} para fórmulas \isa{G} y \isa{H}
    cualesquiera. Observemos que se trata de una fórmula de tipo \isa{{\isasymalpha}} de componentes conjuntivas
    \isa{G} y \isa{{\isasymnot}\ H}. Luego, por hipótesis, tenemos que \isa{{\isacharbraceleft}G{\isacharcomma}\ {\isasymnot}\ H{\isacharbraceright}\ {\isasymunion}\ S} pertenece a \isa{C}.
  \end{enumerate} 

  Finalmente, deduzcamos el resto de condiciones de la definición de propiedad de consistencia
  proposicional a partir de la hipótesis referente a las fórmulas de tipo \isa{{\isasymbeta}}.
  \begin{enumerate}
    \item[\isa{{\isadigit{4}}{\isacharparenright}}:] Supongamos que la fórmula \isa{G\ {\isasymor}\ H} pertenece a \isa{S} para fórmulas \isa{G} y \isa{H}
    cualesquiera. Observemos que se trata de una fórmula de tipo \isa{{\isasymbeta}} de componentes disyuntivas
    \isa{G} y \isa{H}. Luego, por hipótesis, tenemos que o bien \isa{{\isacharbraceleft}G{\isacharbraceright}\ {\isasymunion}\ S} pertenece a \isa{C} o bien\\
    \isa{{\isacharbraceleft}H{\isacharbraceright}\ {\isasymunion}\ S} pertenece a \isa{C}.
    \item[\isa{{\isadigit{5}}{\isacharparenright}}:] Supongamos que la fórmula \isa{G\ {\isasymlongrightarrow}\ H} pertenece a \isa{S} para fórmulas \isa{G} y \isa{H}
    cualesquiera. Observemos que se trata de una fórmula de tipo \isa{{\isasymbeta}} de componentes disyuntivas
    \isa{{\isasymnot}\ G} y \isa{H}. Luego, por hipótesis, tenemos que o bien \isa{{\isacharbraceleft}{\isasymnot}\ G{\isacharbraceright}\ {\isasymunion}\ S} pertenece a \isa{C} o
    bien \isa{{\isacharbraceleft}H{\isacharbraceright}\ {\isasymunion}\ S} pertenece a \isa{C}.
    \item[\isa{{\isadigit{7}}{\isacharparenright}}:] Supongamos que la fórmula \isa{{\isasymnot}{\isacharparenleft}G\ {\isasymand}\ H{\isacharparenright}} pertenece a \isa{S} para fórmulas \isa{G} y \isa{H}
    cualesquiera. Observemos que se trata de una fórmula de tipo \isa{{\isasymbeta}} de componentes disyuntivas
    \isa{{\isasymnot}\ G} y \isa{{\isasymnot}\ H}. Luego, por hipótesis, tenemos que o bien \isa{{\isacharbraceleft}{\isasymnot}\ G{\isacharbraceright}\ {\isasymunion}\ S} pertenece a \isa{C} o
    bien \isa{{\isacharbraceleft}{\isasymnot}\ H{\isacharbraceright}\ {\isasymunion}\ S} pertenece \isa{C}.
  \end{enumerate} 

  De este modo, hemos probado a partir de la hipótesis todas las condiciones que garantizan que la
  colección \isa{C} cumple la propiedad de consistencia proposicional. Por lo tanto, queda demostrado el
  resultado.
\end{demostracion}

  Para probar este resultado de manera detallada en Isabelle vamos a demostrar cada una de las 
  implicaciones de la equivalencia por separado. La primera implicación del lema se basa en dos 
  lemas auxiliares. El primero de ellos deduce la condición de \isa{{\isadigit{2}}{\isacharparenright}} sobre fórmulas de tipo \isa{{\isasymalpha}} a 
  partir de las condiciones tercera, sexta, octava y novena de la definición de propiedad de 
  consistencia proposicional. Su demostración detallada en Isabelle se muestra a continuación.%
\end{isamarkuptext}\isamarkuptrue%
\isacommand{lemma}\isamarkupfalse%
\ pcp{\isacharunderscore}alt{\isadigit{1}}Con{\isacharcolon}\isanewline
\ \ \isakeyword{assumes}\ {\isachardoublequoteopen}{\isacharparenleft}{\isasymforall}G\ H{\isachardot}\ G\ \isactrlbold {\isasymand}\ H\ {\isasymin}\ S\ {\isasymlongrightarrow}\ {\isacharbraceleft}G{\isacharcomma}H{\isacharbraceright}\ {\isasymunion}\ S\ {\isasymin}\ C{\isacharparenright}\isanewline
\ \ {\isasymand}\ {\isacharparenleft}{\isasymforall}G{\isachardot}\ \isactrlbold {\isasymnot}\ {\isacharparenleft}\isactrlbold {\isasymnot}G{\isacharparenright}\ {\isasymin}\ S\ {\isasymlongrightarrow}\ {\isacharbraceleft}G{\isacharbraceright}\ {\isasymunion}\ S\ {\isasymin}\ C{\isacharparenright}\isanewline
\ \ {\isasymand}\ {\isacharparenleft}{\isasymforall}G\ H{\isachardot}\ \isactrlbold {\isasymnot}{\isacharparenleft}G\ \isactrlbold {\isasymor}\ H{\isacharparenright}\ {\isasymin}\ S\ {\isasymlongrightarrow}\ {\isacharbraceleft}\isactrlbold {\isasymnot}\ G{\isacharcomma}\ \isactrlbold {\isasymnot}\ H{\isacharbraceright}\ {\isasymunion}\ S\ {\isasymin}\ C{\isacharparenright}\isanewline
\ \ {\isasymand}\ {\isacharparenleft}{\isasymforall}G\ H{\isachardot}\ \isactrlbold {\isasymnot}{\isacharparenleft}G\ \isactrlbold {\isasymrightarrow}\ H{\isacharparenright}\ {\isasymin}\ S\ {\isasymlongrightarrow}\ {\isacharbraceleft}G{\isacharcomma}\isactrlbold {\isasymnot}\ H{\isacharbraceright}\ {\isasymunion}\ S\ {\isasymin}\ C{\isacharparenright}{\isachardoublequoteclose}\isanewline
\ \ \isakeyword{shows}\ {\isachardoublequoteopen}{\isasymforall}F\ G\ H{\isachardot}\ Con\ F\ G\ H\ {\isasymlongrightarrow}\ F\ {\isasymin}\ S\ {\isasymlongrightarrow}\ {\isacharbraceleft}G{\isacharcomma}H{\isacharbraceright}\ {\isasymunion}\ S\ {\isasymin}\ C{\isachardoublequoteclose}\isanewline
%
\isadelimproof
%
\endisadelimproof
%
\isatagproof
\isacommand{proof}\isamarkupfalse%
\ {\isacharminus}\isanewline
\ \ \isacommand{have}\isamarkupfalse%
\ C{\isadigit{1}}{\isacharcolon}{\isachardoublequoteopen}{\isasymforall}G\ H{\isachardot}\ G\ \isactrlbold {\isasymand}\ H\ {\isasymin}\ S\ {\isasymlongrightarrow}\ {\isacharbraceleft}G{\isacharcomma}H{\isacharbraceright}\ {\isasymunion}\ S\ {\isasymin}\ C{\isachardoublequoteclose}\isanewline
\ \ \ \ \isacommand{using}\isamarkupfalse%
\ assms\ \isacommand{by}\isamarkupfalse%
\ {\isacharparenleft}rule\ conjunct{\isadigit{1}}{\isacharparenright}\isanewline
\ \ \isacommand{have}\isamarkupfalse%
\ C{\isadigit{2}}{\isacharcolon}{\isachardoublequoteopen}{\isasymforall}G{\isachardot}\ \isactrlbold {\isasymnot}\ {\isacharparenleft}\isactrlbold {\isasymnot}G{\isacharparenright}\ {\isasymin}\ S\ {\isasymlongrightarrow}\ {\isacharbraceleft}G{\isacharbraceright}\ {\isasymunion}\ S\ {\isasymin}\ C{\isachardoublequoteclose}\isanewline
\ \ \ \ \isacommand{using}\isamarkupfalse%
\ assms\ \isacommand{by}\isamarkupfalse%
\ {\isacharparenleft}iprover\ elim{\isacharcolon}\ conjunct{\isadigit{2}}\ conjunct{\isadigit{1}}{\isacharparenright}\isanewline
\ \ \isacommand{have}\isamarkupfalse%
\ C{\isadigit{3}}{\isacharcolon}{\isachardoublequoteopen}{\isasymforall}G\ H{\isachardot}\ \isactrlbold {\isasymnot}{\isacharparenleft}G\ \isactrlbold {\isasymor}\ H{\isacharparenright}\ {\isasymin}\ S\ {\isasymlongrightarrow}\ {\isacharbraceleft}\isactrlbold {\isasymnot}\ G{\isacharcomma}\ \isactrlbold {\isasymnot}\ H{\isacharbraceright}\ {\isasymunion}\ S\ {\isasymin}\ C{\isachardoublequoteclose}\isanewline
\ \ \ \ \isacommand{using}\isamarkupfalse%
\ assms\ \isacommand{by}\isamarkupfalse%
\ {\isacharparenleft}iprover\ elim{\isacharcolon}\ conjunct{\isadigit{2}}\ conjunct{\isadigit{1}}{\isacharparenright}\isanewline
\ \ \isacommand{have}\isamarkupfalse%
\ C{\isadigit{4}}{\isacharcolon}{\isachardoublequoteopen}{\isasymforall}G\ H{\isachardot}\ \isactrlbold {\isasymnot}{\isacharparenleft}G\ \isactrlbold {\isasymrightarrow}\ H{\isacharparenright}\ {\isasymin}\ S\ {\isasymlongrightarrow}\ {\isacharbraceleft}G{\isacharcomma}\isactrlbold {\isasymnot}\ H{\isacharbraceright}\ {\isasymunion}\ S\ {\isasymin}\ C{\isachardoublequoteclose}\isanewline
\ \ \ \ \isacommand{using}\isamarkupfalse%
\ assms\ \isacommand{by}\isamarkupfalse%
\ {\isacharparenleft}iprover\ elim{\isacharcolon}\ conjunct{\isadigit{2}}{\isacharparenright}\ \isanewline
\ \ \isacommand{show}\isamarkupfalse%
\ {\isachardoublequoteopen}{\isasymforall}F\ G\ H{\isachardot}\ Con\ F\ G\ H\ {\isasymlongrightarrow}\ F\ {\isasymin}\ S\ {\isasymlongrightarrow}\ {\isacharbraceleft}G{\isacharcomma}H{\isacharbraceright}\ {\isasymunion}\ S\ {\isasymin}\ C{\isachardoublequoteclose}\isanewline
\ \ \isacommand{proof}\isamarkupfalse%
\ {\isacharparenleft}rule\ allI{\isacharparenright}{\isacharplus}\isanewline
\ \ \ \ \isacommand{fix}\isamarkupfalse%
\ F\ G\ H\isanewline
\ \ \ \ \isacommand{show}\isamarkupfalse%
\ {\isachardoublequoteopen}Con\ F\ G\ H\ {\isasymlongrightarrow}\ F\ {\isasymin}\ S\ {\isasymlongrightarrow}\ {\isacharbraceleft}G{\isacharcomma}H{\isacharbraceright}\ {\isasymunion}\ S\ {\isasymin}\ C{\isachardoublequoteclose}\isanewline
\ \ \ \ \isacommand{proof}\isamarkupfalse%
\ {\isacharparenleft}rule\ impI{\isacharparenright}\isanewline
\ \ \ \ \ \ \isacommand{assume}\isamarkupfalse%
\ {\isachardoublequoteopen}Con\ F\ G\ H{\isachardoublequoteclose}\isanewline
\ \ \ \ \ \ \isacommand{then}\isamarkupfalse%
\ \isacommand{have}\isamarkupfalse%
\ {\isachardoublequoteopen}F\ {\isacharequal}\ G\ \isactrlbold {\isasymand}\ H\ {\isasymor}\ \isanewline
\ \ \ \ \ \ \ \ \ \ \ \ \ \ \ \ {\isacharparenleft}{\isacharparenleft}{\isasymexists}G{\isadigit{1}}\ H{\isadigit{1}}{\isachardot}\ F\ {\isacharequal}\ \isactrlbold {\isasymnot}\ {\isacharparenleft}G{\isadigit{1}}\ \isactrlbold {\isasymor}\ H{\isadigit{1}}{\isacharparenright}\ {\isasymand}\ G\ {\isacharequal}\ \isactrlbold {\isasymnot}\ G{\isadigit{1}}\ {\isasymand}\ H\ {\isacharequal}\ \isactrlbold {\isasymnot}\ H{\isadigit{1}}{\isacharparenright}\ {\isasymor}\ \isanewline
\ \ \ \ \ \ \ \ \ \ \ \ \ \ \ \ {\isacharparenleft}{\isasymexists}H{\isadigit{2}}{\isachardot}\ F\ {\isacharequal}\ \isactrlbold {\isasymnot}\ {\isacharparenleft}G\ \isactrlbold {\isasymrightarrow}\ H{\isadigit{2}}{\isacharparenright}\ {\isasymand}\ H\ {\isacharequal}\ \isactrlbold {\isasymnot}\ H{\isadigit{2}}{\isacharparenright}\ {\isasymor}\ \isanewline
\ \ \ \ \ \ \ \ \ \ \ \ \ \ \ \ F\ {\isacharequal}\ \isactrlbold {\isasymnot}\ {\isacharparenleft}\isactrlbold {\isasymnot}\ G{\isacharparenright}\ {\isasymand}\ H\ {\isacharequal}\ G{\isacharparenright}{\isachardoublequoteclose}\isanewline
\ \ \ \ \ \ \ \ \isacommand{by}\isamarkupfalse%
\ {\isacharparenleft}simp\ only{\isacharcolon}\ con{\isacharunderscore}dis{\isacharunderscore}simps{\isacharparenleft}{\isadigit{1}}{\isacharparenright}{\isacharparenright}\isanewline
\ \ \ \ \ \ \isacommand{thus}\isamarkupfalse%
\ {\isachardoublequoteopen}F\ {\isasymin}\ S\ {\isasymlongrightarrow}\ {\isacharbraceleft}G{\isacharcomma}H{\isacharbraceright}\ {\isasymunion}\ S\ {\isasymin}\ C{\isachardoublequoteclose}\isanewline
\ \ \ \ \ \ \isacommand{proof}\isamarkupfalse%
\ {\isacharparenleft}rule\ disjE{\isacharparenright}\isanewline
\ \ \ \ \ \ \ \ \isacommand{assume}\isamarkupfalse%
\ {\isachardoublequoteopen}F\ {\isacharequal}\ G\ \isactrlbold {\isasymand}\ H{\isachardoublequoteclose}\isanewline
\ \ \ \ \ \ \ \ \isacommand{show}\isamarkupfalse%
\ {\isachardoublequoteopen}F\ {\isasymin}\ S\ {\isasymlongrightarrow}\ {\isacharbraceleft}G{\isacharcomma}H{\isacharbraceright}\ {\isasymunion}\ S\ {\isasymin}\ C{\isachardoublequoteclose}\isanewline
\ \ \ \ \ \ \ \ \ \ \isacommand{using}\isamarkupfalse%
\ C{\isadigit{1}}\ {\isacartoucheopen}F\ {\isacharequal}\ G\ \isactrlbold {\isasymand}\ H{\isacartoucheclose}\ \isacommand{by}\isamarkupfalse%
\ {\isacharparenleft}iprover\ elim{\isacharcolon}\ allE{\isacharparenright}\isanewline
\ \ \ \ \ \ \isacommand{next}\isamarkupfalse%
\isanewline
\ \ \ \ \ \ \ \ \isacommand{assume}\isamarkupfalse%
\ {\isachardoublequoteopen}{\isacharparenleft}{\isasymexists}G{\isadigit{1}}\ H{\isadigit{1}}{\isachardot}\ F\ {\isacharequal}\ \isactrlbold {\isasymnot}\ {\isacharparenleft}G{\isadigit{1}}\ \isactrlbold {\isasymor}\ H{\isadigit{1}}{\isacharparenright}\ {\isasymand}\ G\ {\isacharequal}\ \isactrlbold {\isasymnot}\ G{\isadigit{1}}\ {\isasymand}\ H\ {\isacharequal}\ \isactrlbold {\isasymnot}\ H{\isadigit{1}}{\isacharparenright}\ {\isasymor}\ \isanewline
\ \ \ \ \ \ \ \ \ \ \ \ \ \ \ \ {\isacharparenleft}{\isasymexists}H{\isadigit{2}}{\isachardot}\ F\ {\isacharequal}\ \isactrlbold {\isasymnot}\ {\isacharparenleft}G\ \isactrlbold {\isasymrightarrow}\ H{\isadigit{2}}{\isacharparenright}\ {\isasymand}\ H\ {\isacharequal}\ \isactrlbold {\isasymnot}\ H{\isadigit{2}}{\isacharparenright}\ {\isasymor}\ \isanewline
\ \ \ \ \ \ \ \ \ \ \ \ \ \ \ \ F\ {\isacharequal}\ \isactrlbold {\isasymnot}\ {\isacharparenleft}\isactrlbold {\isasymnot}\ G{\isacharparenright}\ {\isasymand}\ H\ {\isacharequal}\ G{\isachardoublequoteclose}\isanewline
\ \ \ \ \ \ \ \ \isacommand{thus}\isamarkupfalse%
\ {\isachardoublequoteopen}F\ {\isasymin}\ S\ {\isasymlongrightarrow}\ {\isacharbraceleft}G{\isacharcomma}H{\isacharbraceright}\ {\isasymunion}\ S\ {\isasymin}\ C{\isachardoublequoteclose}\isanewline
\ \ \ \ \ \ \ \ \isacommand{proof}\isamarkupfalse%
\ {\isacharparenleft}rule\ disjE{\isacharparenright}\isanewline
\ \ \ \ \ \ \ \ \ \ \isacommand{assume}\isamarkupfalse%
\ E{\isadigit{1}}{\isacharcolon}{\isachardoublequoteopen}{\isasymexists}G{\isadigit{1}}\ H{\isadigit{1}}{\isachardot}\ F\ {\isacharequal}\ \isactrlbold {\isasymnot}\ {\isacharparenleft}G{\isadigit{1}}\ \isactrlbold {\isasymor}\ H{\isadigit{1}}{\isacharparenright}\ {\isasymand}\ G\ {\isacharequal}\ \isactrlbold {\isasymnot}\ G{\isadigit{1}}\ {\isasymand}\ H\ {\isacharequal}\ \isactrlbold {\isasymnot}\ H{\isadigit{1}}{\isachardoublequoteclose}\isanewline
\ \ \ \ \ \ \ \ \ \ \isacommand{obtain}\isamarkupfalse%
\ G{\isadigit{1}}\ H{\isadigit{1}}\ \isakeyword{where}\ A{\isadigit{1}}{\isacharcolon}{\isachardoublequoteopen}F\ {\isacharequal}\ \isactrlbold {\isasymnot}\ {\isacharparenleft}G{\isadigit{1}}\ \isactrlbold {\isasymor}\ H{\isadigit{1}}{\isacharparenright}\ {\isasymand}\ G\ {\isacharequal}\ \isactrlbold {\isasymnot}\ G{\isadigit{1}}\ {\isasymand}\ H\ {\isacharequal}\ \isactrlbold {\isasymnot}\ H{\isadigit{1}}{\isachardoublequoteclose}\isanewline
\ \ \ \ \ \ \ \ \ \ \ \ \isacommand{using}\isamarkupfalse%
\ E{\isadigit{1}}\ \isacommand{by}\isamarkupfalse%
\ {\isacharparenleft}iprover\ elim{\isacharcolon}\ exE{\isacharparenright}\isanewline
\ \ \ \ \ \ \ \ \ \ \isacommand{have}\isamarkupfalse%
\ {\isachardoublequoteopen}F\ {\isacharequal}\ \isactrlbold {\isasymnot}\ {\isacharparenleft}G{\isadigit{1}}\ \isactrlbold {\isasymor}\ H{\isadigit{1}}{\isacharparenright}{\isachardoublequoteclose}\isanewline
\ \ \ \ \ \ \ \ \ \ \ \ \isacommand{using}\isamarkupfalse%
\ A{\isadigit{1}}\ \isacommand{by}\isamarkupfalse%
\ {\isacharparenleft}rule\ conjunct{\isadigit{1}}{\isacharparenright}\isanewline
\ \ \ \ \ \ \ \ \ \ \isacommand{have}\isamarkupfalse%
\ {\isachardoublequoteopen}G\ {\isacharequal}\ \isactrlbold {\isasymnot}\ G{\isadigit{1}}{\isachardoublequoteclose}\isanewline
\ \ \ \ \ \ \ \ \ \ \ \ \isacommand{using}\isamarkupfalse%
\ A{\isadigit{1}}\ \isacommand{by}\isamarkupfalse%
\ {\isacharparenleft}iprover\ elim{\isacharcolon}\ conjunct{\isadigit{2}}\ conjunct{\isadigit{1}}{\isacharparenright}\isanewline
\ \ \ \ \ \ \ \ \ \ \isacommand{have}\isamarkupfalse%
\ {\isachardoublequoteopen}H\ {\isacharequal}\ \isactrlbold {\isasymnot}\ H{\isadigit{1}}{\isachardoublequoteclose}\isanewline
\ \ \ \ \ \ \ \ \ \ \ \ \isacommand{using}\isamarkupfalse%
\ A{\isadigit{1}}\ \isacommand{by}\isamarkupfalse%
\ {\isacharparenleft}iprover\ elim{\isacharcolon}\ conjunct{\isadigit{2}}{\isacharparenright}\isanewline
\ \ \ \ \ \ \ \ \ \ \isacommand{show}\isamarkupfalse%
\ {\isachardoublequoteopen}F\ {\isasymin}\ S\ {\isasymlongrightarrow}\ {\isacharbraceleft}G{\isacharcomma}H{\isacharbraceright}\ {\isasymunion}\ S\ {\isasymin}\ C{\isachardoublequoteclose}\isanewline
\ \ \ \ \ \ \ \ \ \ \ \ \isacommand{using}\isamarkupfalse%
\ C{\isadigit{3}}\ {\isacartoucheopen}F\ {\isacharequal}\ \isactrlbold {\isasymnot}\ {\isacharparenleft}G{\isadigit{1}}\ \isactrlbold {\isasymor}\ H{\isadigit{1}}{\isacharparenright}{\isacartoucheclose}\ {\isacartoucheopen}G\ {\isacharequal}\ \isactrlbold {\isasymnot}\ G{\isadigit{1}}{\isacartoucheclose}\ {\isacartoucheopen}H\ {\isacharequal}\ \isactrlbold {\isasymnot}\ H{\isadigit{1}}{\isacartoucheclose}\ \isacommand{by}\isamarkupfalse%
\ {\isacharparenleft}iprover\ elim{\isacharcolon}\ allE{\isacharparenright}\isanewline
\ \ \ \ \ \ \ \ \isacommand{next}\isamarkupfalse%
\isanewline
\ \ \ \ \ \ \ \ \ \ \isacommand{assume}\isamarkupfalse%
\ {\isachardoublequoteopen}{\isacharparenleft}{\isasymexists}H{\isadigit{2}}{\isachardot}\ F\ {\isacharequal}\ \isactrlbold {\isasymnot}\ {\isacharparenleft}G\ \isactrlbold {\isasymrightarrow}\ H{\isadigit{2}}{\isacharparenright}\ {\isasymand}\ H\ {\isacharequal}\ \isactrlbold {\isasymnot}\ H{\isadigit{2}}{\isacharparenright}\ {\isasymor}\ \isanewline
\ \ \ \ \ \ \ \ \ \ \ \ \ \ \ \ \ \ \ F\ {\isacharequal}\ \isactrlbold {\isasymnot}\ {\isacharparenleft}\isactrlbold {\isasymnot}\ G{\isacharparenright}\ {\isasymand}\ H\ {\isacharequal}\ G{\isachardoublequoteclose}\ \isanewline
\ \ \ \ \ \ \ \ \ \ \isacommand{thus}\isamarkupfalse%
\ {\isachardoublequoteopen}F\ {\isasymin}\ S\ {\isasymlongrightarrow}\ {\isacharbraceleft}G{\isacharcomma}H{\isacharbraceright}\ {\isasymunion}\ S\ {\isasymin}\ C{\isachardoublequoteclose}\isanewline
\ \ \ \ \ \ \ \ \ \ \isacommand{proof}\isamarkupfalse%
\ {\isacharparenleft}rule\ disjE{\isacharparenright}\isanewline
\ \ \ \ \ \ \ \ \ \ \ \ \isacommand{assume}\isamarkupfalse%
\ E{\isadigit{2}}{\isacharcolon}{\isachardoublequoteopen}{\isasymexists}H{\isadigit{2}}{\isachardot}\ F\ {\isacharequal}\ \isactrlbold {\isasymnot}\ {\isacharparenleft}G\ \isactrlbold {\isasymrightarrow}\ H{\isadigit{2}}{\isacharparenright}\ {\isasymand}\ H\ {\isacharequal}\ \isactrlbold {\isasymnot}\ H{\isadigit{2}}{\isachardoublequoteclose}\isanewline
\ \ \ \ \ \ \ \ \ \ \ \ \isacommand{obtain}\isamarkupfalse%
\ H{\isadigit{2}}\ \isakeyword{where}\ A{\isadigit{2}}{\isacharcolon}{\isachardoublequoteopen}F\ {\isacharequal}\ \isactrlbold {\isasymnot}\ {\isacharparenleft}G\ \isactrlbold {\isasymrightarrow}\ H{\isadigit{2}}{\isacharparenright}\ {\isasymand}\ H\ {\isacharequal}\ \isactrlbold {\isasymnot}\ H{\isadigit{2}}{\isachardoublequoteclose}\isanewline
\ \ \ \ \ \ \ \ \ \ \ \ \ \ \isacommand{using}\isamarkupfalse%
\ E{\isadigit{2}}\ \isacommand{by}\isamarkupfalse%
\ {\isacharparenleft}rule\ exE{\isacharparenright}\isanewline
\ \ \ \ \ \ \ \ \ \ \ \ \isacommand{have}\isamarkupfalse%
\ {\isachardoublequoteopen}F\ {\isacharequal}\ \isactrlbold {\isasymnot}\ {\isacharparenleft}G\ \isactrlbold {\isasymrightarrow}\ H{\isadigit{2}}{\isacharparenright}{\isachardoublequoteclose}\isanewline
\ \ \ \ \ \ \ \ \ \ \ \ \ \ \isacommand{using}\isamarkupfalse%
\ A{\isadigit{2}}\ \isacommand{by}\isamarkupfalse%
\ {\isacharparenleft}rule\ conjunct{\isadigit{1}}{\isacharparenright}\isanewline
\ \ \ \ \ \ \ \ \ \ \ \ \isacommand{have}\isamarkupfalse%
\ {\isachardoublequoteopen}H\ {\isacharequal}\ \isactrlbold {\isasymnot}\ H{\isadigit{2}}{\isachardoublequoteclose}\isanewline
\ \ \ \ \ \ \ \ \ \ \ \ \ \ \isacommand{using}\isamarkupfalse%
\ A{\isadigit{2}}\ \isacommand{by}\isamarkupfalse%
\ {\isacharparenleft}rule\ conjunct{\isadigit{2}}{\isacharparenright}\isanewline
\ \ \ \ \ \ \ \ \ \ \ \ \isacommand{show}\isamarkupfalse%
\ {\isachardoublequoteopen}F\ {\isasymin}\ S\ {\isasymlongrightarrow}\ {\isacharbraceleft}G{\isacharcomma}H{\isacharbraceright}\ {\isasymunion}\ S\ {\isasymin}\ C{\isachardoublequoteclose}\isanewline
\ \ \ \ \ \ \ \ \ \ \ \ \ \ \isacommand{using}\isamarkupfalse%
\ C{\isadigit{4}}\ {\isacartoucheopen}F\ {\isacharequal}\ \isactrlbold {\isasymnot}\ {\isacharparenleft}G\ \isactrlbold {\isasymrightarrow}\ H{\isadigit{2}}{\isacharparenright}{\isacartoucheclose}\ {\isacartoucheopen}H\ {\isacharequal}\ \isactrlbold {\isasymnot}\ H{\isadigit{2}}{\isacartoucheclose}\ \isacommand{by}\isamarkupfalse%
\ {\isacharparenleft}iprover\ elim{\isacharcolon}\ allE{\isacharparenright}\isanewline
\ \ \ \ \ \ \ \ \ \ \isacommand{next}\isamarkupfalse%
\isanewline
\ \ \ \ \ \ \ \ \ \ \ \ \isacommand{assume}\isamarkupfalse%
\ A{\isadigit{3}}{\isacharcolon}{\isachardoublequoteopen}F\ {\isacharequal}\ \isactrlbold {\isasymnot}{\isacharparenleft}\isactrlbold {\isasymnot}\ G{\isacharparenright}\ {\isasymand}\ H\ {\isacharequal}\ G{\isachardoublequoteclose}\isanewline
\ \ \ \ \ \ \ \ \ \ \ \ \isacommand{then}\isamarkupfalse%
\ \isacommand{have}\isamarkupfalse%
\ {\isachardoublequoteopen}F\ {\isacharequal}\ \isactrlbold {\isasymnot}{\isacharparenleft}\isactrlbold {\isasymnot}\ G{\isacharparenright}{\isachardoublequoteclose}\isanewline
\ \ \ \ \ \ \ \ \ \ \ \ \ \ \isacommand{by}\isamarkupfalse%
\ {\isacharparenleft}rule\ conjunct{\isadigit{1}}{\isacharparenright}\isanewline
\ \ \ \ \ \ \ \ \ \ \ \ \isacommand{have}\isamarkupfalse%
\ {\isachardoublequoteopen}H\ {\isacharequal}\ G{\isachardoublequoteclose}\isanewline
\ \ \ \ \ \ \ \ \ \ \ \ \ \ \isacommand{using}\isamarkupfalse%
\ A{\isadigit{3}}\ \isacommand{by}\isamarkupfalse%
\ {\isacharparenleft}rule\ conjunct{\isadigit{2}}{\isacharparenright}\isanewline
\ \ \ \ \ \ \ \ \ \ \ \ \isacommand{have}\isamarkupfalse%
\ {\isachardoublequoteopen}F\ {\isasymin}\ S\ {\isasymlongrightarrow}\ {\isacharbraceleft}G{\isacharbraceright}\ {\isasymunion}\ S\ {\isasymin}\ C{\isachardoublequoteclose}\isanewline
\ \ \ \ \ \ \ \ \ \ \ \ \ \ \isacommand{using}\isamarkupfalse%
\ C{\isadigit{2}}\ {\isacartoucheopen}F\ {\isacharequal}\ \isactrlbold {\isasymnot}{\isacharparenleft}\isactrlbold {\isasymnot}\ G{\isacharparenright}{\isacartoucheclose}\ \isacommand{by}\isamarkupfalse%
\ {\isacharparenleft}iprover\ elim{\isacharcolon}\ allE{\isacharparenright}\isanewline
\ \ \ \ \ \ \ \ \ \ \ \ \isacommand{then}\isamarkupfalse%
\ \isacommand{have}\isamarkupfalse%
\ {\isachardoublequoteopen}F\ {\isasymin}\ S\ {\isasymlongrightarrow}\ {\isacharbraceleft}G{\isacharcomma}G{\isacharbraceright}\ {\isasymunion}\ S\ {\isasymin}\ C{\isachardoublequoteclose}\isanewline
\ \ \ \ \ \ \ \ \ \ \ \ \ \ \isacommand{by}\isamarkupfalse%
\ {\isacharparenleft}simp\ only{\isacharcolon}\ insert{\isacharunderscore}absorb{\isadigit{2}}{\isacharparenright}\isanewline
\ \ \ \ \ \ \ \ \ \ \ \ \isacommand{thus}\isamarkupfalse%
\ {\isachardoublequoteopen}F\ {\isasymin}\ S\ {\isasymlongrightarrow}\ {\isacharbraceleft}G{\isacharcomma}H{\isacharbraceright}\ {\isasymunion}\ S\ {\isasymin}\ C{\isachardoublequoteclose}\ \isanewline
\ \ \ \ \ \ \ \ \ \ \ \ \ \ \isacommand{by}\isamarkupfalse%
\ {\isacharparenleft}simp\ only{\isacharcolon}\ {\isacartoucheopen}H\ {\isacharequal}\ G{\isacartoucheclose}{\isacharparenright}\isanewline
\ \ \ \ \ \ \ \ \ \ \isacommand{qed}\isamarkupfalse%
\isanewline
\ \ \ \ \ \ \ \ \isacommand{qed}\isamarkupfalse%
\isanewline
\ \ \ \ \ \ \isacommand{qed}\isamarkupfalse%
\isanewline
\ \ \ \ \isacommand{qed}\isamarkupfalse%
\isanewline
\ \ \isacommand{qed}\isamarkupfalse%
\isanewline
\isacommand{qed}\isamarkupfalse%
%
\endisatagproof
{\isafoldproof}%
%
\isadelimproof
%
\endisadelimproof
%
\begin{isamarkuptext}%
Finalmente, el siguiente lema auxiliar deduce la condición de \isa{{\isadigit{2}}{\isacharparenright}} sobre fórmulas de tipo \isa{{\isasymbeta}} 
  a partir de las condiciones cuarta, quinta, sexta y séptima de la definición de propiedad de 
  consistencia proposicional.%
\end{isamarkuptext}\isamarkuptrue%
\isacommand{lemma}\isamarkupfalse%
\ pcp{\isacharunderscore}alt{\isadigit{1}}Dis{\isacharcolon}\isanewline
\ \ \isakeyword{assumes}\ {\isachardoublequoteopen}{\isacharparenleft}{\isasymforall}G\ H{\isachardot}\ G\ \isactrlbold {\isasymor}\ H\ {\isasymin}\ S\ {\isasymlongrightarrow}\ {\isacharbraceleft}G{\isacharbraceright}\ {\isasymunion}\ S\ {\isasymin}\ C\ {\isasymor}\ {\isacharbraceleft}H{\isacharbraceright}\ {\isasymunion}\ S\ {\isasymin}\ C{\isacharparenright}\isanewline
\ \ {\isasymand}\ {\isacharparenleft}{\isasymforall}G\ H{\isachardot}\ G\ \isactrlbold {\isasymrightarrow}\ H\ {\isasymin}\ S\ {\isasymlongrightarrow}\ {\isacharbraceleft}\isactrlbold {\isasymnot}\ G{\isacharbraceright}\ {\isasymunion}\ S\ {\isasymin}\ C\ {\isasymor}\ {\isacharbraceleft}H{\isacharbraceright}\ {\isasymunion}\ S\ {\isasymin}\ C{\isacharparenright}\isanewline
\ \ {\isasymand}\ {\isacharparenleft}{\isasymforall}G{\isachardot}\ \isactrlbold {\isasymnot}\ {\isacharparenleft}\isactrlbold {\isasymnot}G{\isacharparenright}\ {\isasymin}\ S\ {\isasymlongrightarrow}\ {\isacharbraceleft}G{\isacharbraceright}\ {\isasymunion}\ S\ {\isasymin}\ C{\isacharparenright}\isanewline
\ \ {\isasymand}\ {\isacharparenleft}{\isasymforall}G\ H{\isachardot}\ \isactrlbold {\isasymnot}{\isacharparenleft}G\ \isactrlbold {\isasymand}\ H{\isacharparenright}\ {\isasymin}\ S\ {\isasymlongrightarrow}\ {\isacharbraceleft}\isactrlbold {\isasymnot}\ G{\isacharbraceright}\ {\isasymunion}\ S\ {\isasymin}\ C\ {\isasymor}\ {\isacharbraceleft}\isactrlbold {\isasymnot}\ H{\isacharbraceright}\ {\isasymunion}\ S\ {\isasymin}\ C{\isacharparenright}{\isachardoublequoteclose}\isanewline
\ \ \isakeyword{shows}\ {\isachardoublequoteopen}{\isasymforall}F\ G\ H{\isachardot}\ Dis\ F\ G\ H\ {\isasymlongrightarrow}\ F\ {\isasymin}\ S\ {\isasymlongrightarrow}\ {\isacharbraceleft}G{\isacharbraceright}\ {\isasymunion}\ S\ {\isasymin}\ C\ {\isasymor}\ {\isacharbraceleft}H{\isacharbraceright}\ {\isasymunion}\ S\ {\isasymin}\ C{\isachardoublequoteclose}\isanewline
%
\isadelimproof
%
\endisadelimproof
%
\isatagproof
\isacommand{proof}\isamarkupfalse%
\ {\isacharminus}\isanewline
\ \ \isacommand{have}\isamarkupfalse%
\ C{\isadigit{1}}{\isacharcolon}{\isachardoublequoteopen}{\isasymforall}G\ H{\isachardot}\ G\ \isactrlbold {\isasymor}\ H\ {\isasymin}\ S\ {\isasymlongrightarrow}\ {\isacharbraceleft}G{\isacharbraceright}\ {\isasymunion}\ S\ {\isasymin}\ C\ {\isasymor}\ {\isacharbraceleft}H{\isacharbraceright}\ {\isasymunion}\ S\ {\isasymin}\ C{\isachardoublequoteclose}\isanewline
\ \ \ \ \isacommand{using}\isamarkupfalse%
\ assms\ \isacommand{by}\isamarkupfalse%
\ {\isacharparenleft}rule\ conjunct{\isadigit{1}}{\isacharparenright}\isanewline
\ \ \isacommand{have}\isamarkupfalse%
\ C{\isadigit{2}}{\isacharcolon}{\isachardoublequoteopen}{\isasymforall}G\ H{\isachardot}\ G\ \isactrlbold {\isasymrightarrow}\ H\ {\isasymin}\ S\ {\isasymlongrightarrow}\ {\isacharbraceleft}\isactrlbold {\isasymnot}\ G{\isacharbraceright}\ {\isasymunion}\ S\ {\isasymin}\ C\ {\isasymor}\ {\isacharbraceleft}H{\isacharbraceright}\ {\isasymunion}\ S\ {\isasymin}\ C{\isachardoublequoteclose}\isanewline
\ \ \ \ \isacommand{using}\isamarkupfalse%
\ assms\ \isacommand{by}\isamarkupfalse%
\ {\isacharparenleft}iprover\ elim{\isacharcolon}\ conjunct{\isadigit{2}}\ conjunct{\isadigit{1}}{\isacharparenright}\isanewline
\ \ \isacommand{have}\isamarkupfalse%
\ C{\isadigit{3}}{\isacharcolon}{\isachardoublequoteopen}{\isasymforall}G{\isachardot}\ \isactrlbold {\isasymnot}\ {\isacharparenleft}\isactrlbold {\isasymnot}G{\isacharparenright}\ {\isasymin}\ S\ {\isasymlongrightarrow}\ {\isacharbraceleft}G{\isacharbraceright}\ {\isasymunion}\ S\ {\isasymin}\ C{\isachardoublequoteclose}\isanewline
\ \ \ \ \isacommand{using}\isamarkupfalse%
\ assms\ \isacommand{by}\isamarkupfalse%
\ {\isacharparenleft}iprover\ elim{\isacharcolon}\ conjunct{\isadigit{2}}\ conjunct{\isadigit{1}}{\isacharparenright}\isanewline
\ \ \isacommand{have}\isamarkupfalse%
\ C{\isadigit{4}}{\isacharcolon}{\isachardoublequoteopen}{\isasymforall}G\ H{\isachardot}\ \isactrlbold {\isasymnot}{\isacharparenleft}G\ \isactrlbold {\isasymand}\ H{\isacharparenright}\ {\isasymin}\ S\ {\isasymlongrightarrow}\ {\isacharbraceleft}\isactrlbold {\isasymnot}\ G{\isacharbraceright}\ {\isasymunion}\ S\ {\isasymin}\ C\ {\isasymor}\ {\isacharbraceleft}\isactrlbold {\isasymnot}\ H{\isacharbraceright}\ {\isasymunion}\ S\ {\isasymin}\ C{\isachardoublequoteclose}\isanewline
\ \ \ \ \isacommand{using}\isamarkupfalse%
\ assms\ \isacommand{by}\isamarkupfalse%
\ {\isacharparenleft}iprover\ elim{\isacharcolon}\ conjunct{\isadigit{2}}{\isacharparenright}\ \isanewline
\ \ \isacommand{show}\isamarkupfalse%
\ {\isachardoublequoteopen}{\isasymforall}F\ G\ H{\isachardot}\ Dis\ F\ G\ H\ {\isasymlongrightarrow}\ F\ {\isasymin}\ S\ {\isasymlongrightarrow}\ {\isacharbraceleft}G{\isacharbraceright}\ {\isasymunion}\ S\ {\isasymin}\ C\ {\isasymor}\ {\isacharbraceleft}H{\isacharbraceright}\ {\isasymunion}\ S\ {\isasymin}\ C{\isachardoublequoteclose}\isanewline
\ \ \isacommand{proof}\isamarkupfalse%
\ {\isacharparenleft}rule\ allI{\isacharparenright}{\isacharplus}\isanewline
\ \ \ \ \isacommand{fix}\isamarkupfalse%
\ F\ G\ H\isanewline
\ \ \ \ \isacommand{show}\isamarkupfalse%
\ {\isachardoublequoteopen}Dis\ F\ G\ H\ {\isasymlongrightarrow}\ F\ {\isasymin}\ S\ {\isasymlongrightarrow}\ {\isacharbraceleft}G{\isacharbraceright}\ {\isasymunion}\ S\ {\isasymin}\ C\ {\isasymor}\ {\isacharbraceleft}H{\isacharbraceright}\ {\isasymunion}\ S\ {\isasymin}\ C{\isachardoublequoteclose}\isanewline
\ \ \ \ \isacommand{proof}\isamarkupfalse%
\ {\isacharparenleft}rule\ impI{\isacharparenright}\isanewline
\ \ \ \ \ \ \isacommand{assume}\isamarkupfalse%
\ {\isachardoublequoteopen}Dis\ F\ G\ H{\isachardoublequoteclose}\isanewline
\ \ \ \ \ \ \isacommand{then}\isamarkupfalse%
\ \isacommand{have}\isamarkupfalse%
\ {\isachardoublequoteopen}F\ {\isacharequal}\ G\ \isactrlbold {\isasymor}\ H\ {\isasymor}\ \isanewline
\ \ \ \ \ \ \ \ \ \ \ \ \ \ \ \ {\isacharparenleft}{\isasymexists}G{\isadigit{1}}\ H{\isadigit{1}}{\isachardot}\ F\ {\isacharequal}\ G{\isadigit{1}}\ \isactrlbold {\isasymrightarrow}\ H{\isadigit{1}}\ {\isasymand}\ G\ {\isacharequal}\ \isactrlbold {\isasymnot}\ G{\isadigit{1}}\ {\isasymand}\ H\ {\isacharequal}\ H{\isadigit{1}}{\isacharparenright}\ {\isasymor}\ \isanewline
\ \ \ \ \ \ \ \ \ \ \ \ \ \ \ \ {\isacharparenleft}{\isasymexists}G{\isadigit{2}}\ H{\isadigit{2}}{\isachardot}\ F\ {\isacharequal}\ \isactrlbold {\isasymnot}\ {\isacharparenleft}G{\isadigit{2}}\ \isactrlbold {\isasymand}\ H{\isadigit{2}}{\isacharparenright}\ {\isasymand}\ G\ {\isacharequal}\ \isactrlbold {\isasymnot}\ G{\isadigit{2}}\ {\isasymand}\ H\ {\isacharequal}\ \isactrlbold {\isasymnot}\ H{\isadigit{2}}{\isacharparenright}\ {\isasymor}\ \isanewline
\ \ \ \ \ \ \ \ \ \ \ \ \ \ \ \ F\ {\isacharequal}\ \isactrlbold {\isasymnot}\ {\isacharparenleft}\isactrlbold {\isasymnot}\ G{\isacharparenright}\ {\isasymand}\ H\ {\isacharequal}\ G{\isachardoublequoteclose}\ \isanewline
\ \ \ \ \ \ \ \ \isacommand{by}\isamarkupfalse%
\ {\isacharparenleft}simp\ only{\isacharcolon}\ con{\isacharunderscore}dis{\isacharunderscore}simps{\isacharparenleft}{\isadigit{2}}{\isacharparenright}{\isacharparenright}\isanewline
\ \ \ \ \ \ \isacommand{thus}\isamarkupfalse%
\ {\isachardoublequoteopen}F\ {\isasymin}\ S\ {\isasymlongrightarrow}\ {\isacharbraceleft}G{\isacharbraceright}\ {\isasymunion}\ S\ {\isasymin}\ C\ {\isasymor}\ {\isacharbraceleft}H{\isacharbraceright}\ {\isasymunion}\ S\ {\isasymin}\ C{\isachardoublequoteclose}\isanewline
\ \ \ \ \ \ \isacommand{proof}\isamarkupfalse%
\ {\isacharparenleft}rule\ disjE{\isacharparenright}\isanewline
\ \ \ \ \ \ \ \ \isacommand{assume}\isamarkupfalse%
\ {\isachardoublequoteopen}F\ {\isacharequal}\ G\ \isactrlbold {\isasymor}\ H{\isachardoublequoteclose}\isanewline
\ \ \ \ \ \ \ \ \isacommand{show}\isamarkupfalse%
\ {\isachardoublequoteopen}F\ {\isasymin}\ S\ {\isasymlongrightarrow}\ {\isacharbraceleft}G{\isacharbraceright}\ {\isasymunion}\ S\ {\isasymin}\ C\ {\isasymor}\ {\isacharbraceleft}H{\isacharbraceright}\ {\isasymunion}\ S\ {\isasymin}\ C{\isachardoublequoteclose}\isanewline
\ \ \ \ \ \ \ \ \ \ \isacommand{using}\isamarkupfalse%
\ C{\isadigit{1}}\ {\isacartoucheopen}F\ {\isacharequal}\ G\ \isactrlbold {\isasymor}\ H{\isacartoucheclose}\ \isacommand{by}\isamarkupfalse%
\ {\isacharparenleft}iprover\ elim{\isacharcolon}\ allE{\isacharparenright}\isanewline
\ \ \ \ \ \ \isacommand{next}\isamarkupfalse%
\isanewline
\ \ \ \ \ \ \ \ \isacommand{assume}\isamarkupfalse%
\ {\isachardoublequoteopen}{\isacharparenleft}{\isasymexists}G{\isadigit{1}}\ H{\isadigit{1}}{\isachardot}\ F\ {\isacharequal}\ G{\isadigit{1}}\ \isactrlbold {\isasymrightarrow}\ H{\isadigit{1}}\ {\isasymand}\ G\ {\isacharequal}\ \isactrlbold {\isasymnot}\ G{\isadigit{1}}\ {\isasymand}\ H\ {\isacharequal}\ H{\isadigit{1}}{\isacharparenright}\ {\isasymor}\ \isanewline
\ \ \ \ \ \ \ \ \ \ \ \ \ \ {\isacharparenleft}{\isasymexists}G{\isadigit{2}}\ H{\isadigit{2}}{\isachardot}\ F\ {\isacharequal}\ \isactrlbold {\isasymnot}\ {\isacharparenleft}G{\isadigit{2}}\ \isactrlbold {\isasymand}\ H{\isadigit{2}}{\isacharparenright}\ {\isasymand}\ G\ {\isacharequal}\ \isactrlbold {\isasymnot}\ G{\isadigit{2}}\ {\isasymand}\ H\ {\isacharequal}\ \isactrlbold {\isasymnot}\ H{\isadigit{2}}{\isacharparenright}\ {\isasymor}\ \isanewline
\ \ \ \ \ \ \ \ \ \ \ \ \ \ F\ {\isacharequal}\ \isactrlbold {\isasymnot}\ {\isacharparenleft}\isactrlbold {\isasymnot}\ G{\isacharparenright}\ {\isasymand}\ H\ {\isacharequal}\ G{\isachardoublequoteclose}\isanewline
\ \ \ \ \ \ \ \ \isacommand{thus}\isamarkupfalse%
\ {\isachardoublequoteopen}F\ {\isasymin}\ S\ {\isasymlongrightarrow}\ {\isacharbraceleft}G{\isacharbraceright}\ {\isasymunion}\ S\ {\isasymin}\ C\ {\isasymor}\ {\isacharbraceleft}H{\isacharbraceright}\ {\isasymunion}\ S\ {\isasymin}\ C{\isachardoublequoteclose}\isanewline
\ \ \ \ \ \ \ \ \isacommand{proof}\isamarkupfalse%
\ {\isacharparenleft}rule\ disjE{\isacharparenright}\isanewline
\ \ \ \ \ \ \ \ \ \ \isacommand{assume}\isamarkupfalse%
\ E{\isadigit{1}}{\isacharcolon}{\isachardoublequoteopen}{\isasymexists}G{\isadigit{1}}\ H{\isadigit{1}}{\isachardot}\ F\ {\isacharequal}\ {\isacharparenleft}G{\isadigit{1}}\ \isactrlbold {\isasymrightarrow}\ H{\isadigit{1}}{\isacharparenright}\ {\isasymand}\ G\ {\isacharequal}\ \isactrlbold {\isasymnot}\ G{\isadigit{1}}\ {\isasymand}\ H\ {\isacharequal}\ H{\isadigit{1}}{\isachardoublequoteclose}\isanewline
\ \ \ \ \ \ \ \ \ \ \isacommand{obtain}\isamarkupfalse%
\ G{\isadigit{1}}\ H{\isadigit{1}}\ \isakeyword{where}\ A{\isadigit{1}}{\isacharcolon}{\isachardoublequoteopen}\ F\ {\isacharequal}\ {\isacharparenleft}G{\isadigit{1}}\ \isactrlbold {\isasymrightarrow}\ H{\isadigit{1}}{\isacharparenright}\ {\isasymand}\ G\ {\isacharequal}\ \isactrlbold {\isasymnot}\ G{\isadigit{1}}\ {\isasymand}\ H\ {\isacharequal}\ H{\isadigit{1}}{\isachardoublequoteclose}\isanewline
\ \ \ \ \ \ \ \ \ \ \ \ \isacommand{using}\isamarkupfalse%
\ E{\isadigit{1}}\ \isacommand{by}\isamarkupfalse%
\ {\isacharparenleft}iprover\ elim{\isacharcolon}\ exE{\isacharparenright}\isanewline
\ \ \ \ \ \ \ \ \ \ \isacommand{have}\isamarkupfalse%
\ {\isachardoublequoteopen}F\ {\isacharequal}\ {\isacharparenleft}G{\isadigit{1}}\ \isactrlbold {\isasymrightarrow}\ H{\isadigit{1}}{\isacharparenright}{\isachardoublequoteclose}\isanewline
\ \ \ \ \ \ \ \ \ \ \ \ \isacommand{using}\isamarkupfalse%
\ A{\isadigit{1}}\ \isacommand{by}\isamarkupfalse%
\ {\isacharparenleft}rule\ conjunct{\isadigit{1}}{\isacharparenright}\isanewline
\ \ \ \ \ \ \ \ \ \ \isacommand{have}\isamarkupfalse%
\ {\isachardoublequoteopen}G\ {\isacharequal}\ \isactrlbold {\isasymnot}\ G{\isadigit{1}}{\isachardoublequoteclose}\isanewline
\ \ \ \ \ \ \ \ \ \ \ \ \isacommand{using}\isamarkupfalse%
\ A{\isadigit{1}}\ \isacommand{by}\isamarkupfalse%
\ {\isacharparenleft}iprover\ elim{\isacharcolon}\ conjunct{\isadigit{2}}\ conjunct{\isadigit{1}}{\isacharparenright}\isanewline
\ \ \ \ \ \ \ \ \ \ \isacommand{have}\isamarkupfalse%
\ {\isachardoublequoteopen}H\ {\isacharequal}\ H{\isadigit{1}}{\isachardoublequoteclose}\isanewline
\ \ \ \ \ \ \ \ \ \ \ \ \isacommand{using}\isamarkupfalse%
\ A{\isadigit{1}}\ \isacommand{by}\isamarkupfalse%
\ {\isacharparenleft}iprover\ elim{\isacharcolon}\ conjunct{\isadigit{2}}{\isacharparenright}\isanewline
\ \ \ \ \ \ \ \ \ \ \isacommand{show}\isamarkupfalse%
\ {\isachardoublequoteopen}F\ {\isasymin}\ S\ {\isasymlongrightarrow}\ {\isacharbraceleft}G{\isacharbraceright}\ {\isasymunion}\ S\ {\isasymin}\ C\ {\isasymor}\ {\isacharbraceleft}H{\isacharbraceright}\ {\isasymunion}\ S\ {\isasymin}\ C{\isachardoublequoteclose}\isanewline
\ \ \ \ \ \ \ \ \ \ \ \ \isacommand{using}\isamarkupfalse%
\ C{\isadigit{2}}\ {\isacartoucheopen}F\ {\isacharequal}\ {\isacharparenleft}G{\isadigit{1}}\ \isactrlbold {\isasymrightarrow}\ H{\isadigit{1}}{\isacharparenright}{\isacartoucheclose}\ {\isacartoucheopen}G\ {\isacharequal}\ \isactrlbold {\isasymnot}\ G{\isadigit{1}}{\isacartoucheclose}\ {\isacartoucheopen}H\ {\isacharequal}\ H{\isadigit{1}}{\isacartoucheclose}\ \isacommand{by}\isamarkupfalse%
\ {\isacharparenleft}iprover\ elim{\isacharcolon}\ allE{\isacharparenright}\isanewline
\ \ \ \ \ \ \ \ \isacommand{next}\isamarkupfalse%
\isanewline
\ \ \ \ \ \ \ \ \ \ \isacommand{assume}\isamarkupfalse%
\ {\isachardoublequoteopen}{\isacharparenleft}{\isasymexists}G{\isadigit{2}}\ H{\isadigit{2}}{\isachardot}\ F\ {\isacharequal}\ \isactrlbold {\isasymnot}\ {\isacharparenleft}G{\isadigit{2}}\ \isactrlbold {\isasymand}\ H{\isadigit{2}}{\isacharparenright}\ {\isasymand}\ G\ {\isacharequal}\ \isactrlbold {\isasymnot}\ G{\isadigit{2}}\ {\isasymand}\ H\ {\isacharequal}\ \isactrlbold {\isasymnot}\ H{\isadigit{2}}{\isacharparenright}\ {\isasymor}\ \isanewline
\ \ \ \ \ \ \ \ \ \ \ \ \ \ \ \ \ \ F\ {\isacharequal}\ \isactrlbold {\isasymnot}\ {\isacharparenleft}\isactrlbold {\isasymnot}\ G{\isacharparenright}\ {\isasymand}\ H\ {\isacharequal}\ G{\isachardoublequoteclose}\ \isanewline
\ \ \ \ \ \ \ \ \ \ \isacommand{thus}\isamarkupfalse%
\ {\isachardoublequoteopen}F\ {\isasymin}\ S\ {\isasymlongrightarrow}\ {\isacharbraceleft}G{\isacharbraceright}\ {\isasymunion}\ S\ {\isasymin}\ C\ {\isasymor}\ {\isacharbraceleft}H{\isacharbraceright}\ {\isasymunion}\ S\ {\isasymin}\ C{\isachardoublequoteclose}\isanewline
\ \ \ \ \ \ \ \ \ \ \isacommand{proof}\isamarkupfalse%
\ {\isacharparenleft}rule\ disjE{\isacharparenright}\isanewline
\ \ \ \ \ \ \ \ \ \ \ \ \isacommand{assume}\isamarkupfalse%
\ E{\isadigit{2}}{\isacharcolon}{\isachardoublequoteopen}{\isasymexists}G{\isadigit{2}}\ H{\isadigit{2}}{\isachardot}\ F\ {\isacharequal}\ \isactrlbold {\isasymnot}\ {\isacharparenleft}G{\isadigit{2}}\ \isactrlbold {\isasymand}\ H{\isadigit{2}}{\isacharparenright}\ {\isasymand}\ G\ {\isacharequal}\ \isactrlbold {\isasymnot}\ G{\isadigit{2}}\ {\isasymand}\ H\ {\isacharequal}\ \isactrlbold {\isasymnot}\ H{\isadigit{2}}{\isachardoublequoteclose}\isanewline
\ \ \ \ \ \ \ \ \ \ \ \ \isacommand{obtain}\isamarkupfalse%
\ G{\isadigit{2}}\ H{\isadigit{2}}\ \isakeyword{where}\ A{\isadigit{2}}{\isacharcolon}{\isachardoublequoteopen}F\ {\isacharequal}\ \isactrlbold {\isasymnot}\ {\isacharparenleft}G{\isadigit{2}}\ \isactrlbold {\isasymand}\ H{\isadigit{2}}{\isacharparenright}\ {\isasymand}\ G\ {\isacharequal}\ \isactrlbold {\isasymnot}\ G{\isadigit{2}}\ {\isasymand}\ H\ {\isacharequal}\ \isactrlbold {\isasymnot}\ H{\isadigit{2}}{\isachardoublequoteclose}\isanewline
\ \ \ \ \ \ \ \ \ \ \ \ \ \ \isacommand{using}\isamarkupfalse%
\ E{\isadigit{2}}\ \isacommand{by}\isamarkupfalse%
\ {\isacharparenleft}iprover\ elim{\isacharcolon}\ exE{\isacharparenright}\isanewline
\ \ \ \ \ \ \ \ \ \ \ \ \isacommand{have}\isamarkupfalse%
\ {\isachardoublequoteopen}F\ {\isacharequal}\ \isactrlbold {\isasymnot}\ {\isacharparenleft}G{\isadigit{2}}\ \isactrlbold {\isasymand}\ H{\isadigit{2}}{\isacharparenright}{\isachardoublequoteclose}\isanewline
\ \ \ \ \ \ \ \ \ \ \ \ \ \ \isacommand{using}\isamarkupfalse%
\ A{\isadigit{2}}\ \isacommand{by}\isamarkupfalse%
\ {\isacharparenleft}rule\ conjunct{\isadigit{1}}{\isacharparenright}\isanewline
\ \ \ \ \ \ \ \ \ \ \ \ \isacommand{have}\isamarkupfalse%
\ {\isachardoublequoteopen}G\ {\isacharequal}\ \isactrlbold {\isasymnot}\ G{\isadigit{2}}{\isachardoublequoteclose}\isanewline
\ \ \ \ \ \ \ \ \ \ \ \ \ \ \isacommand{using}\isamarkupfalse%
\ A{\isadigit{2}}\ \isacommand{by}\isamarkupfalse%
\ {\isacharparenleft}iprover\ elim{\isacharcolon}\ conjunct{\isadigit{2}}\ conjunct{\isadigit{1}}{\isacharparenright}\isanewline
\ \ \ \ \ \ \ \ \ \ \ \ \isacommand{have}\isamarkupfalse%
\ {\isachardoublequoteopen}H\ {\isacharequal}\ \isactrlbold {\isasymnot}\ H{\isadigit{2}}{\isachardoublequoteclose}\isanewline
\ \ \ \ \ \ \ \ \ \ \ \ \ \ \isacommand{using}\isamarkupfalse%
\ A{\isadigit{2}}\ \isacommand{by}\isamarkupfalse%
\ {\isacharparenleft}iprover\ elim{\isacharcolon}\ conjunct{\isadigit{2}}{\isacharparenright}\isanewline
\ \ \ \ \ \ \ \ \ \ \ \ \isacommand{show}\isamarkupfalse%
\ {\isachardoublequoteopen}F\ {\isasymin}\ S\ {\isasymlongrightarrow}\ {\isacharbraceleft}G{\isacharbraceright}\ {\isasymunion}\ S\ {\isasymin}\ C\ {\isasymor}\ {\isacharbraceleft}H{\isacharbraceright}\ {\isasymunion}\ S\ {\isasymin}\ C{\isachardoublequoteclose}\isanewline
\ \ \ \ \ \ \ \ \ \ \ \ \ \ \isacommand{using}\isamarkupfalse%
\ C{\isadigit{4}}\ {\isacartoucheopen}F\ {\isacharequal}\ \isactrlbold {\isasymnot}\ {\isacharparenleft}G{\isadigit{2}}\ \isactrlbold {\isasymand}\ H{\isadigit{2}}{\isacharparenright}{\isacartoucheclose}\ {\isacartoucheopen}G\ {\isacharequal}\ \isactrlbold {\isasymnot}\ G{\isadigit{2}}{\isacartoucheclose}\ {\isacartoucheopen}H\ {\isacharequal}\ \isactrlbold {\isasymnot}\ H{\isadigit{2}}{\isacartoucheclose}\ \isacommand{by}\isamarkupfalse%
\ {\isacharparenleft}iprover\ elim{\isacharcolon}\ allE{\isacharparenright}\isanewline
\ \ \ \ \ \ \ \ \ \ \isacommand{next}\isamarkupfalse%
\isanewline
\ \ \ \ \ \ \ \ \ \ \ \ \isacommand{assume}\isamarkupfalse%
\ A{\isadigit{3}}{\isacharcolon}{\isachardoublequoteopen}F\ {\isacharequal}\ \isactrlbold {\isasymnot}{\isacharparenleft}\isactrlbold {\isasymnot}\ G{\isacharparenright}\ {\isasymand}\ H\ {\isacharequal}\ G{\isachardoublequoteclose}\isanewline
\ \ \ \ \ \ \ \ \ \ \ \ \isacommand{then}\isamarkupfalse%
\ \isacommand{have}\isamarkupfalse%
\ {\isachardoublequoteopen}F\ {\isacharequal}\ \isactrlbold {\isasymnot}{\isacharparenleft}\isactrlbold {\isasymnot}\ G{\isacharparenright}{\isachardoublequoteclose}\isanewline
\ \ \ \ \ \ \ \ \ \ \ \ \ \ \isacommand{by}\isamarkupfalse%
\ {\isacharparenleft}rule\ conjunct{\isadigit{1}}{\isacharparenright}\isanewline
\ \ \ \ \ \ \ \ \ \ \ \ \isacommand{have}\isamarkupfalse%
\ {\isachardoublequoteopen}H\ {\isacharequal}\ G{\isachardoublequoteclose}\isanewline
\ \ \ \ \ \ \ \ \ \ \ \ \ \ \isacommand{using}\isamarkupfalse%
\ A{\isadigit{3}}\ \isacommand{by}\isamarkupfalse%
\ {\isacharparenleft}rule\ conjunct{\isadigit{2}}{\isacharparenright}\isanewline
\ \ \ \ \ \ \ \ \ \ \ \ \isacommand{have}\isamarkupfalse%
\ {\isachardoublequoteopen}F\ {\isasymin}\ S\ {\isasymlongrightarrow}\ {\isacharbraceleft}G{\isacharbraceright}\ {\isasymunion}\ S\ {\isasymin}\ C{\isachardoublequoteclose}\isanewline
\ \ \ \ \ \ \ \ \ \ \ \ \ \ \isacommand{using}\isamarkupfalse%
\ C{\isadigit{3}}\ {\isacartoucheopen}F\ {\isacharequal}\ \isactrlbold {\isasymnot}{\isacharparenleft}\isactrlbold {\isasymnot}\ G{\isacharparenright}{\isacartoucheclose}\ \isacommand{by}\isamarkupfalse%
\ {\isacharparenleft}iprover\ elim{\isacharcolon}\ allE{\isacharparenright}\isanewline
\ \ \ \ \ \ \ \ \ \ \ \ \isacommand{then}\isamarkupfalse%
\ \isacommand{have}\isamarkupfalse%
\ {\isachardoublequoteopen}F\ {\isasymin}\ S\ {\isasymlongrightarrow}\ {\isacharbraceleft}G{\isacharbraceright}\ {\isasymunion}\ S\ {\isasymin}\ C\ {\isasymor}\ {\isacharbraceleft}G{\isacharbraceright}\ {\isasymunion}\ S\ {\isasymin}\ C{\isachardoublequoteclose}\isanewline
\ \ \ \ \ \ \ \ \ \ \ \ \ \ \isacommand{by}\isamarkupfalse%
\ {\isacharparenleft}simp\ only{\isacharcolon}\ disj{\isacharunderscore}absorb{\isacharparenright}\isanewline
\ \ \ \ \ \ \ \ \ \ \ \ \isacommand{thus}\isamarkupfalse%
\ {\isachardoublequoteopen}F\ {\isasymin}\ S\ {\isasymlongrightarrow}\ {\isacharbraceleft}G{\isacharbraceright}\ {\isasymunion}\ S\ {\isasymin}\ C\ {\isasymor}\ {\isacharbraceleft}H{\isacharbraceright}\ {\isasymunion}\ S\ {\isasymin}\ C{\isachardoublequoteclose}\isanewline
\ \ \ \ \ \ \ \ \ \ \ \ \ \ \isacommand{by}\isamarkupfalse%
\ {\isacharparenleft}simp\ only{\isacharcolon}\ {\isacartoucheopen}H\ {\isacharequal}\ G{\isacartoucheclose}{\isacharparenright}\isanewline
\ \ \ \ \ \ \ \ \ \ \isacommand{qed}\isamarkupfalse%
\isanewline
\ \ \ \ \ \ \ \ \isacommand{qed}\isamarkupfalse%
\isanewline
\ \ \ \ \ \ \isacommand{qed}\isamarkupfalse%
\isanewline
\ \ \ \ \isacommand{qed}\isamarkupfalse%
\isanewline
\ \ \isacommand{qed}\isamarkupfalse%
\isanewline
\isacommand{qed}\isamarkupfalse%
%
\endisatagproof
{\isafoldproof}%
%
\isadelimproof
%
\endisadelimproof
%
\begin{isamarkuptext}%
De esta manera, mediante los anteriores lemas auxiliares podemos probar la primera
  implicación detalladamente en Isabelle.%
\end{isamarkuptext}\isamarkuptrue%
\isacommand{lemma}\isamarkupfalse%
\ pcp{\isacharunderscore}alt{\isadigit{1}}{\isacharcolon}\ \isanewline
\ \ \isakeyword{assumes}\ {\isachardoublequoteopen}pcp\ C{\isachardoublequoteclose}\isanewline
\ \ \isakeyword{shows}\ {\isachardoublequoteopen}{\isasymforall}S\ {\isasymin}\ C{\isachardot}\ {\isasymbottom}\ {\isasymnotin}\ S\isanewline
\ \ {\isasymand}\ {\isacharparenleft}{\isasymforall}k{\isachardot}\ Atom\ k\ {\isasymin}\ S\ {\isasymlongrightarrow}\ \isactrlbold {\isasymnot}\ {\isacharparenleft}Atom\ k{\isacharparenright}\ {\isasymin}\ S\ {\isasymlongrightarrow}\ False{\isacharparenright}\isanewline
\ \ {\isasymand}\ {\isacharparenleft}{\isasymforall}F\ G\ H{\isachardot}\ Con\ F\ G\ H\ {\isasymlongrightarrow}\ F\ {\isasymin}\ S\ {\isasymlongrightarrow}\ {\isacharbraceleft}G{\isacharcomma}H{\isacharbraceright}\ {\isasymunion}\ S\ {\isasymin}\ C{\isacharparenright}\isanewline
\ \ {\isasymand}\ {\isacharparenleft}{\isasymforall}F\ G\ H{\isachardot}\ Dis\ F\ G\ H\ {\isasymlongrightarrow}\ F\ {\isasymin}\ S\ {\isasymlongrightarrow}\ {\isacharbraceleft}G{\isacharbraceright}\ {\isasymunion}\ S\ {\isasymin}\ C\ {\isasymor}\ {\isacharbraceleft}H{\isacharbraceright}\ {\isasymunion}\ S\ {\isasymin}\ C{\isacharparenright}{\isachardoublequoteclose}\isanewline
%
\isadelimproof
%
\endisadelimproof
%
\isatagproof
\isacommand{proof}\isamarkupfalse%
\ {\isacharparenleft}rule\ ballI{\isacharparenright}\isanewline
\ \ \isacommand{fix}\isamarkupfalse%
\ S\isanewline
\ \ \isacommand{assume}\isamarkupfalse%
\ {\isachardoublequoteopen}S\ {\isasymin}\ C{\isachardoublequoteclose}\isanewline
\ \ \isacommand{have}\isamarkupfalse%
\ {\isachardoublequoteopen}{\isacharparenleft}{\isasymforall}S\ {\isasymin}\ C{\isachardot}\isanewline
\ \ {\isasymbottom}\ {\isasymnotin}\ S\isanewline
\ \ {\isasymand}\ {\isacharparenleft}{\isasymforall}k{\isachardot}\ Atom\ k\ {\isasymin}\ S\ {\isasymlongrightarrow}\ \isactrlbold {\isasymnot}\ {\isacharparenleft}Atom\ k{\isacharparenright}\ {\isasymin}\ S\ {\isasymlongrightarrow}\ False{\isacharparenright}\isanewline
\ \ {\isasymand}\ {\isacharparenleft}{\isasymforall}G\ H{\isachardot}\ G\ \isactrlbold {\isasymand}\ H\ {\isasymin}\ S\ {\isasymlongrightarrow}\ {\isacharbraceleft}G{\isacharcomma}H{\isacharbraceright}\ {\isasymunion}\ S\ {\isasymin}\ C{\isacharparenright}\isanewline
\ \ {\isasymand}\ {\isacharparenleft}{\isasymforall}G\ H{\isachardot}\ G\ \isactrlbold {\isasymor}\ H\ {\isasymin}\ S\ {\isasymlongrightarrow}\ {\isacharbraceleft}G{\isacharbraceright}\ {\isasymunion}\ S\ {\isasymin}\ C\ {\isasymor}\ {\isacharbraceleft}H{\isacharbraceright}\ {\isasymunion}\ S\ {\isasymin}\ C{\isacharparenright}\isanewline
\ \ {\isasymand}\ {\isacharparenleft}{\isasymforall}G\ H{\isachardot}\ G\ \isactrlbold {\isasymrightarrow}\ H\ {\isasymin}\ S\ {\isasymlongrightarrow}\ {\isacharbraceleft}\isactrlbold {\isasymnot}G{\isacharbraceright}\ {\isasymunion}\ S\ {\isasymin}\ C\ {\isasymor}\ {\isacharbraceleft}H{\isacharbraceright}\ {\isasymunion}\ S\ {\isasymin}\ C{\isacharparenright}\isanewline
\ \ {\isasymand}\ {\isacharparenleft}{\isasymforall}G{\isachardot}\ \isactrlbold {\isasymnot}\ {\isacharparenleft}\isactrlbold {\isasymnot}G{\isacharparenright}\ {\isasymin}\ S\ {\isasymlongrightarrow}\ {\isacharbraceleft}G{\isacharbraceright}\ {\isasymunion}\ S\ {\isasymin}\ C{\isacharparenright}\isanewline
\ \ {\isasymand}\ {\isacharparenleft}{\isasymforall}G\ H{\isachardot}\ \isactrlbold {\isasymnot}{\isacharparenleft}G\ \isactrlbold {\isasymand}\ H{\isacharparenright}\ {\isasymin}\ S\ {\isasymlongrightarrow}\ {\isacharbraceleft}\isactrlbold {\isasymnot}\ G{\isacharbraceright}\ {\isasymunion}\ S\ {\isasymin}\ C\ {\isasymor}\ {\isacharbraceleft}\isactrlbold {\isasymnot}\ H{\isacharbraceright}\ {\isasymunion}\ S\ {\isasymin}\ C{\isacharparenright}\isanewline
\ \ {\isasymand}\ {\isacharparenleft}{\isasymforall}G\ H{\isachardot}\ \isactrlbold {\isasymnot}{\isacharparenleft}G\ \isactrlbold {\isasymor}\ H{\isacharparenright}\ {\isasymin}\ S\ {\isasymlongrightarrow}\ {\isacharbraceleft}\isactrlbold {\isasymnot}\ G{\isacharcomma}\ \isactrlbold {\isasymnot}\ H{\isacharbraceright}\ {\isasymunion}\ S\ {\isasymin}\ C{\isacharparenright}\isanewline
\ \ {\isasymand}\ {\isacharparenleft}{\isasymforall}G\ H{\isachardot}\ \isactrlbold {\isasymnot}{\isacharparenleft}G\ \isactrlbold {\isasymrightarrow}\ H{\isacharparenright}\ {\isasymin}\ S\ {\isasymlongrightarrow}\ {\isacharbraceleft}G{\isacharcomma}\isactrlbold {\isasymnot}\ H{\isacharbraceright}\ {\isasymunion}\ S\ {\isasymin}\ C{\isacharparenright}{\isacharparenright}{\isachardoublequoteclose}\isanewline
\ \ \ \ \isacommand{using}\isamarkupfalse%
\ assms\ \isacommand{by}\isamarkupfalse%
\ {\isacharparenleft}simp\ only{\isacharcolon}\ pcp{\isacharunderscore}def{\isacharparenright}\isanewline
\ \ \isacommand{then}\isamarkupfalse%
\ \isacommand{have}\isamarkupfalse%
\ pcpS{\isacharcolon}{\isachardoublequoteopen}{\isasymbottom}\ {\isasymnotin}\ S\isanewline
\ \ {\isasymand}\ {\isacharparenleft}{\isasymforall}k{\isachardot}\ Atom\ k\ {\isasymin}\ S\ {\isasymlongrightarrow}\ \isactrlbold {\isasymnot}\ {\isacharparenleft}Atom\ k{\isacharparenright}\ {\isasymin}\ S\ {\isasymlongrightarrow}\ False{\isacharparenright}\isanewline
\ \ {\isasymand}\ {\isacharparenleft}{\isasymforall}G\ H{\isachardot}\ G\ \isactrlbold {\isasymand}\ H\ {\isasymin}\ S\ {\isasymlongrightarrow}\ {\isacharbraceleft}G{\isacharcomma}H{\isacharbraceright}\ {\isasymunion}\ S\ {\isasymin}\ C{\isacharparenright}\isanewline
\ \ {\isasymand}\ {\isacharparenleft}{\isasymforall}G\ H{\isachardot}\ G\ \isactrlbold {\isasymor}\ H\ {\isasymin}\ S\ {\isasymlongrightarrow}\ {\isacharbraceleft}G{\isacharbraceright}\ {\isasymunion}\ S\ {\isasymin}\ C\ {\isasymor}\ {\isacharbraceleft}H{\isacharbraceright}\ {\isasymunion}\ S\ {\isasymin}\ C{\isacharparenright}\isanewline
\ \ {\isasymand}\ {\isacharparenleft}{\isasymforall}G\ H{\isachardot}\ G\ \isactrlbold {\isasymrightarrow}\ H\ {\isasymin}\ S\ {\isasymlongrightarrow}\ {\isacharbraceleft}\isactrlbold {\isasymnot}G{\isacharbraceright}\ {\isasymunion}\ S\ {\isasymin}\ C\ {\isasymor}\ {\isacharbraceleft}H{\isacharbraceright}\ {\isasymunion}\ S\ {\isasymin}\ C{\isacharparenright}\isanewline
\ \ {\isasymand}\ {\isacharparenleft}{\isasymforall}G{\isachardot}\ \isactrlbold {\isasymnot}\ {\isacharparenleft}\isactrlbold {\isasymnot}G{\isacharparenright}\ {\isasymin}\ S\ {\isasymlongrightarrow}\ {\isacharbraceleft}G{\isacharbraceright}\ {\isasymunion}\ S\ {\isasymin}\ C{\isacharparenright}\isanewline
\ \ {\isasymand}\ {\isacharparenleft}{\isasymforall}G\ H{\isachardot}\ \isactrlbold {\isasymnot}{\isacharparenleft}G\ \isactrlbold {\isasymand}\ H{\isacharparenright}\ {\isasymin}\ S\ {\isasymlongrightarrow}\ {\isacharbraceleft}\isactrlbold {\isasymnot}\ G{\isacharbraceright}\ {\isasymunion}\ S\ {\isasymin}\ C\ {\isasymor}\ {\isacharbraceleft}\isactrlbold {\isasymnot}\ H{\isacharbraceright}\ {\isasymunion}\ S\ {\isasymin}\ C{\isacharparenright}\isanewline
\ \ {\isasymand}\ {\isacharparenleft}{\isasymforall}G\ H{\isachardot}\ \isactrlbold {\isasymnot}{\isacharparenleft}G\ \isactrlbold {\isasymor}\ H{\isacharparenright}\ {\isasymin}\ S\ {\isasymlongrightarrow}\ {\isacharbraceleft}\isactrlbold {\isasymnot}\ G{\isacharcomma}\ \isactrlbold {\isasymnot}\ H{\isacharbraceright}\ {\isasymunion}\ S\ {\isasymin}\ C{\isacharparenright}\isanewline
\ \ {\isasymand}\ {\isacharparenleft}{\isasymforall}G\ H{\isachardot}\ \isactrlbold {\isasymnot}{\isacharparenleft}G\ \isactrlbold {\isasymrightarrow}\ H{\isacharparenright}\ {\isasymin}\ S\ {\isasymlongrightarrow}\ {\isacharbraceleft}G{\isacharcomma}\isactrlbold {\isasymnot}\ H{\isacharbraceright}\ {\isasymunion}\ S\ {\isasymin}\ C{\isacharparenright}{\isachardoublequoteclose}\isanewline
\ \ \ \ \isacommand{using}\isamarkupfalse%
\ {\isacartoucheopen}S\ {\isasymin}\ C{\isacartoucheclose}\ \isacommand{by}\isamarkupfalse%
\ {\isacharparenleft}rule\ bspec{\isacharparenright}\isanewline
\ \ \isacommand{then}\isamarkupfalse%
\ \isacommand{have}\isamarkupfalse%
\ C{\isadigit{1}}{\isacharcolon}{\isachardoublequoteopen}{\isasymbottom}\ {\isasymnotin}\ S{\isachardoublequoteclose}\isanewline
\ \ \ \ \isacommand{by}\isamarkupfalse%
\ {\isacharparenleft}rule\ conjunct{\isadigit{1}}{\isacharparenright}\isanewline
\ \ \isacommand{have}\isamarkupfalse%
\ C{\isadigit{2}}{\isacharcolon}{\isachardoublequoteopen}{\isasymforall}k{\isachardot}\ Atom\ k\ {\isasymin}\ S\ {\isasymlongrightarrow}\ \isactrlbold {\isasymnot}\ {\isacharparenleft}Atom\ k{\isacharparenright}\ {\isasymin}\ S\ {\isasymlongrightarrow}\ False{\isachardoublequoteclose}\isanewline
\ \ \ \ \isacommand{using}\isamarkupfalse%
\ pcpS\ \isacommand{by}\isamarkupfalse%
\ {\isacharparenleft}iprover\ elim{\isacharcolon}\ conjunct{\isadigit{2}}\ conjunct{\isadigit{1}}{\isacharparenright}\isanewline
\ \ \isacommand{have}\isamarkupfalse%
\ C{\isadigit{3}}{\isacharcolon}{\isachardoublequoteopen}{\isasymforall}G\ H{\isachardot}\ G\ \isactrlbold {\isasymand}\ H\ {\isasymin}\ S\ {\isasymlongrightarrow}\ {\isacharbraceleft}G{\isacharcomma}H{\isacharbraceright}\ {\isasymunion}\ S\ {\isasymin}\ C{\isachardoublequoteclose}\isanewline
\ \ \ \ \isacommand{using}\isamarkupfalse%
\ pcpS\ \isacommand{by}\isamarkupfalse%
\ {\isacharparenleft}iprover\ elim{\isacharcolon}\ conjunct{\isadigit{2}}\ conjunct{\isadigit{1}}{\isacharparenright}\isanewline
\ \ \isacommand{have}\isamarkupfalse%
\ C{\isadigit{4}}{\isacharcolon}{\isachardoublequoteopen}{\isasymforall}G\ H{\isachardot}\ G\ \isactrlbold {\isasymor}\ H\ {\isasymin}\ S\ {\isasymlongrightarrow}\ {\isacharbraceleft}G{\isacharbraceright}\ {\isasymunion}\ S\ {\isasymin}\ C\ {\isasymor}\ {\isacharbraceleft}H{\isacharbraceright}\ {\isasymunion}\ S\ {\isasymin}\ C{\isachardoublequoteclose}\isanewline
\ \ \ \ \isacommand{using}\isamarkupfalse%
\ pcpS\ \isacommand{by}\isamarkupfalse%
\ {\isacharparenleft}iprover\ elim{\isacharcolon}\ conjunct{\isadigit{2}}\ conjunct{\isadigit{1}}{\isacharparenright}\isanewline
\ \ \isacommand{have}\isamarkupfalse%
\ C{\isadigit{5}}{\isacharcolon}{\isachardoublequoteopen}{\isasymforall}G\ H{\isachardot}\ G\ \isactrlbold {\isasymrightarrow}\ H\ {\isasymin}\ S\ {\isasymlongrightarrow}\ {\isacharbraceleft}\isactrlbold {\isasymnot}G{\isacharbraceright}\ {\isasymunion}\ S\ {\isasymin}\ C\ {\isasymor}\ {\isacharbraceleft}H{\isacharbraceright}\ {\isasymunion}\ S\ {\isasymin}\ C{\isachardoublequoteclose}\isanewline
\ \ \ \ \isacommand{using}\isamarkupfalse%
\ pcpS\ \isacommand{by}\isamarkupfalse%
\ {\isacharparenleft}iprover\ elim{\isacharcolon}\ conjunct{\isadigit{2}}\ conjunct{\isadigit{1}}{\isacharparenright}\isanewline
\ \ \isacommand{have}\isamarkupfalse%
\ C{\isadigit{6}}{\isacharcolon}{\isachardoublequoteopen}{\isasymforall}G{\isachardot}\ \isactrlbold {\isasymnot}\ {\isacharparenleft}\isactrlbold {\isasymnot}G{\isacharparenright}\ {\isasymin}\ S\ {\isasymlongrightarrow}\ {\isacharbraceleft}G{\isacharbraceright}\ {\isasymunion}\ S\ {\isasymin}\ C{\isachardoublequoteclose}\isanewline
\ \ \ \ \isacommand{using}\isamarkupfalse%
\ pcpS\ \isacommand{by}\isamarkupfalse%
\ {\isacharparenleft}iprover\ elim{\isacharcolon}\ conjunct{\isadigit{2}}\ conjunct{\isadigit{1}}{\isacharparenright}\isanewline
\ \ \isacommand{have}\isamarkupfalse%
\ C{\isadigit{7}}{\isacharcolon}{\isachardoublequoteopen}{\isasymforall}G\ H{\isachardot}\ \isactrlbold {\isasymnot}{\isacharparenleft}G\ \isactrlbold {\isasymand}\ H{\isacharparenright}\ {\isasymin}\ S\ {\isasymlongrightarrow}\ {\isacharbraceleft}\isactrlbold {\isasymnot}\ G{\isacharbraceright}\ {\isasymunion}\ S\ {\isasymin}\ C\ {\isasymor}\ {\isacharbraceleft}\isactrlbold {\isasymnot}\ H{\isacharbraceright}\ {\isasymunion}\ S\ {\isasymin}\ C{\isachardoublequoteclose}\isanewline
\ \ \ \ \isacommand{using}\isamarkupfalse%
\ pcpS\ \isacommand{by}\isamarkupfalse%
\ {\isacharparenleft}iprover\ elim{\isacharcolon}\ conjunct{\isadigit{2}}\ conjunct{\isadigit{1}}{\isacharparenright}\isanewline
\ \ \isacommand{have}\isamarkupfalse%
\ C{\isadigit{8}}{\isacharcolon}{\isachardoublequoteopen}{\isasymforall}G\ H{\isachardot}\ \isactrlbold {\isasymnot}{\isacharparenleft}G\ \isactrlbold {\isasymor}\ H{\isacharparenright}\ {\isasymin}\ S\ {\isasymlongrightarrow}\ {\isacharbraceleft}\isactrlbold {\isasymnot}\ G{\isacharcomma}\ \isactrlbold {\isasymnot}\ H{\isacharbraceright}\ {\isasymunion}\ S\ {\isasymin}\ C{\isachardoublequoteclose}\isanewline
\ \ \ \ \isacommand{using}\isamarkupfalse%
\ pcpS\ \isacommand{by}\isamarkupfalse%
\ {\isacharparenleft}iprover\ elim{\isacharcolon}\ conjunct{\isadigit{2}}\ conjunct{\isadigit{1}}{\isacharparenright}\isanewline
\ \ \isacommand{have}\isamarkupfalse%
\ C{\isadigit{9}}{\isacharcolon}{\isachardoublequoteopen}{\isasymforall}G\ H{\isachardot}\ \isactrlbold {\isasymnot}{\isacharparenleft}G\ \isactrlbold {\isasymrightarrow}\ H{\isacharparenright}\ {\isasymin}\ S\ {\isasymlongrightarrow}\ {\isacharbraceleft}G{\isacharcomma}\isactrlbold {\isasymnot}\ H{\isacharbraceright}\ {\isasymunion}\ S\ {\isasymin}\ C{\isachardoublequoteclose}\isanewline
\ \ \ \ \isacommand{using}\isamarkupfalse%
\ pcpS\ \isacommand{by}\isamarkupfalse%
\ {\isacharparenleft}iprover\ elim{\isacharcolon}\ conjunct{\isadigit{2}}{\isacharparenright}\isanewline
\ \ \isacommand{have}\isamarkupfalse%
\ {\isachardoublequoteopen}{\isacharparenleft}{\isasymforall}G\ H{\isachardot}\ G\ \isactrlbold {\isasymand}\ H\ {\isasymin}\ S\ {\isasymlongrightarrow}\ {\isacharbraceleft}G{\isacharcomma}H{\isacharbraceright}\ {\isasymunion}\ S\ {\isasymin}\ C{\isacharparenright}\isanewline
\ \ {\isasymand}\ {\isacharparenleft}{\isasymforall}G{\isachardot}\ \isactrlbold {\isasymnot}\ {\isacharparenleft}\isactrlbold {\isasymnot}G{\isacharparenright}\ {\isasymin}\ S\ {\isasymlongrightarrow}\ {\isacharbraceleft}G{\isacharbraceright}\ {\isasymunion}\ S\ {\isasymin}\ C{\isacharparenright}\isanewline
\ \ {\isasymand}\ {\isacharparenleft}{\isasymforall}G\ H{\isachardot}\ \isactrlbold {\isasymnot}{\isacharparenleft}G\ \isactrlbold {\isasymor}\ H{\isacharparenright}\ {\isasymin}\ S\ {\isasymlongrightarrow}\ {\isacharbraceleft}\isactrlbold {\isasymnot}\ G{\isacharcomma}\ \isactrlbold {\isasymnot}\ H{\isacharbraceright}\ {\isasymunion}\ S\ {\isasymin}\ C{\isacharparenright}\isanewline
\ \ {\isasymand}\ {\isacharparenleft}{\isasymforall}G\ H{\isachardot}\ \isactrlbold {\isasymnot}{\isacharparenleft}G\ \isactrlbold {\isasymrightarrow}\ H{\isacharparenright}\ {\isasymin}\ S\ {\isasymlongrightarrow}\ {\isacharbraceleft}G{\isacharcomma}\isactrlbold {\isasymnot}\ H{\isacharbraceright}\ {\isasymunion}\ S\ {\isasymin}\ C{\isacharparenright}{\isachardoublequoteclose}\isanewline
\ \ \ \ \isacommand{using}\isamarkupfalse%
\ C{\isadigit{3}}\ C{\isadigit{6}}\ C{\isadigit{8}}\ C{\isadigit{9}}\ \isacommand{by}\isamarkupfalse%
\ {\isacharparenleft}iprover\ intro{\isacharcolon}\ conjI{\isacharparenright}\isanewline
\ \ \isacommand{then}\isamarkupfalse%
\ \isacommand{have}\isamarkupfalse%
\ Con{\isacharcolon}{\isachardoublequoteopen}{\isasymforall}F\ G\ H{\isachardot}\ Con\ F\ G\ H\ {\isasymlongrightarrow}\ F\ {\isasymin}\ S\ {\isasymlongrightarrow}\ {\isacharbraceleft}G{\isacharcomma}H{\isacharbraceright}\ {\isasymunion}\ S\ {\isasymin}\ C{\isachardoublequoteclose}\isanewline
\ \ \ \ \isacommand{by}\isamarkupfalse%
\ {\isacharparenleft}rule\ pcp{\isacharunderscore}alt{\isadigit{1}}Con{\isacharparenright}\isanewline
\ \ \isacommand{have}\isamarkupfalse%
\ {\isachardoublequoteopen}{\isacharparenleft}{\isasymforall}G\ H{\isachardot}\ G\ \isactrlbold {\isasymor}\ H\ {\isasymin}\ S\ {\isasymlongrightarrow}\ {\isacharbraceleft}G{\isacharbraceright}\ {\isasymunion}\ S\ {\isasymin}\ C\ {\isasymor}\ {\isacharbraceleft}H{\isacharbraceright}\ {\isasymunion}\ S\ {\isasymin}\ C{\isacharparenright}\isanewline
\ \ {\isasymand}\ {\isacharparenleft}{\isasymforall}G\ H{\isachardot}\ G\ \isactrlbold {\isasymrightarrow}\ H\ {\isasymin}\ S\ {\isasymlongrightarrow}\ {\isacharbraceleft}\isactrlbold {\isasymnot}\ G{\isacharbraceright}\ {\isasymunion}\ S\ {\isasymin}\ C\ {\isasymor}\ {\isacharbraceleft}H{\isacharbraceright}\ {\isasymunion}\ S\ {\isasymin}\ C{\isacharparenright}\isanewline
\ \ {\isasymand}\ {\isacharparenleft}{\isasymforall}G{\isachardot}\ \isactrlbold {\isasymnot}\ {\isacharparenleft}\isactrlbold {\isasymnot}G{\isacharparenright}\ {\isasymin}\ S\ {\isasymlongrightarrow}\ {\isacharbraceleft}G{\isacharbraceright}\ {\isasymunion}\ S\ {\isasymin}\ C{\isacharparenright}\isanewline
\ \ {\isasymand}\ {\isacharparenleft}{\isasymforall}G\ H{\isachardot}\ \isactrlbold {\isasymnot}{\isacharparenleft}G\ \isactrlbold {\isasymand}\ H{\isacharparenright}\ {\isasymin}\ S\ {\isasymlongrightarrow}\ {\isacharbraceleft}\isactrlbold {\isasymnot}\ G{\isacharbraceright}\ {\isasymunion}\ S\ {\isasymin}\ C\ {\isasymor}\ {\isacharbraceleft}\isactrlbold {\isasymnot}\ H{\isacharbraceright}\ {\isasymunion}\ S\ {\isasymin}\ C{\isacharparenright}{\isachardoublequoteclose}\isanewline
\ \ \ \ \isacommand{using}\isamarkupfalse%
\ C{\isadigit{4}}\ C{\isadigit{5}}\ C{\isadigit{6}}\ C{\isadigit{7}}\ \isacommand{by}\isamarkupfalse%
\ {\isacharparenleft}iprover\ intro{\isacharcolon}\ conjI{\isacharparenright}\isanewline
\ \ \isacommand{then}\isamarkupfalse%
\ \isacommand{have}\isamarkupfalse%
\ Dis{\isacharcolon}{\isachardoublequoteopen}{\isasymforall}F\ G\ H{\isachardot}\ Dis\ F\ G\ H\ {\isasymlongrightarrow}\ F\ {\isasymin}\ S\ {\isasymlongrightarrow}\ {\isacharbraceleft}G{\isacharbraceright}\ {\isasymunion}\ S\ {\isasymin}\ C\ {\isasymor}\ {\isacharbraceleft}H{\isacharbraceright}\ {\isasymunion}\ S\ {\isasymin}\ C{\isachardoublequoteclose}\isanewline
\ \ \ \ \isacommand{by}\isamarkupfalse%
\ {\isacharparenleft}rule\ pcp{\isacharunderscore}alt{\isadigit{1}}Dis{\isacharparenright}\isanewline
\ \ \isacommand{thus}\isamarkupfalse%
\ {\isachardoublequoteopen}{\isasymbottom}\ {\isasymnotin}\ S\isanewline
\ \ {\isasymand}\ {\isacharparenleft}{\isasymforall}k{\isachardot}\ Atom\ k\ {\isasymin}\ S\ {\isasymlongrightarrow}\ \isactrlbold {\isasymnot}\ {\isacharparenleft}Atom\ k{\isacharparenright}\ {\isasymin}\ S\ {\isasymlongrightarrow}\ False{\isacharparenright}\isanewline
\ \ {\isasymand}\ {\isacharparenleft}{\isasymforall}F\ G\ H{\isachardot}\ Con\ F\ G\ H\ {\isasymlongrightarrow}\ F\ {\isasymin}\ S\ {\isasymlongrightarrow}\ {\isacharbraceleft}G{\isacharcomma}H{\isacharbraceright}\ {\isasymunion}\ S\ {\isasymin}\ C{\isacharparenright}\isanewline
\ \ {\isasymand}\ {\isacharparenleft}{\isasymforall}F\ G\ H{\isachardot}\ Dis\ F\ G\ H\ {\isasymlongrightarrow}\ F\ {\isasymin}\ S\ {\isasymlongrightarrow}\ {\isacharbraceleft}G{\isacharbraceright}\ {\isasymunion}\ S\ {\isasymin}\ C\ {\isasymor}\ {\isacharbraceleft}H{\isacharbraceright}\ {\isasymunion}\ S\ {\isasymin}\ C{\isacharparenright}{\isachardoublequoteclose}\isanewline
\ \ \ \ \isacommand{using}\isamarkupfalse%
\ C{\isadigit{1}}\ C{\isadigit{2}}\ Con\ Dis\ \isacommand{by}\isamarkupfalse%
\ {\isacharparenleft}iprover\ intro{\isacharcolon}\ conjI{\isacharparenright}\isanewline
\isacommand{qed}\isamarkupfalse%
%
\endisatagproof
{\isafoldproof}%
%
\isadelimproof
%
\endisadelimproof
%
\begin{isamarkuptext}%
Por otro lado, veamos la demostración detallada de la implicación recíproca de la
  equivalencia. Para ello, utilizaremos distintos lemas auxiliares para deducir cada una de las 
  condiciones de la definición de propiedad de consistencia proposicional a partir de las
  hipótesis sobre las fórmulas de tipo \isa{{\isasymalpha}} y \isa{{\isasymbeta}}. En primer lugar, veamos los lemas que se deducen
  condiciones a partir de la hipótesis referente a las fórmulas de tipo \isa{{\isasymalpha}}.%
\end{isamarkuptext}\isamarkuptrue%
\isacommand{lemma}\isamarkupfalse%
\ pcp{\isacharunderscore}alt{\isadigit{2}}Con{\isadigit{1}}{\isacharcolon}\isanewline
\ \ \isakeyword{assumes}\ {\isachardoublequoteopen}{\isasymforall}F\ G\ H{\isachardot}\ Con\ F\ G\ H\ {\isasymlongrightarrow}\ F\ {\isasymin}\ S\ {\isasymlongrightarrow}\ {\isacharbraceleft}G{\isacharcomma}H{\isacharbraceright}\ {\isasymunion}\ S\ {\isasymin}\ C{\isachardoublequoteclose}\isanewline
\ \ \isakeyword{shows}\ {\isachardoublequoteopen}{\isasymforall}G\ H{\isachardot}\ G\ \isactrlbold {\isasymand}\ H\ {\isasymin}\ S\ {\isasymlongrightarrow}\ {\isacharbraceleft}G{\isacharcomma}H{\isacharbraceright}\ {\isasymunion}\ S\ {\isasymin}\ C{\isachardoublequoteclose}\isanewline
%
\isadelimproof
%
\endisadelimproof
%
\isatagproof
\isacommand{proof}\isamarkupfalse%
\ {\isacharparenleft}rule\ allI{\isacharparenright}{\isacharplus}\isanewline
\ \ \isacommand{fix}\isamarkupfalse%
\ G\ H\isanewline
\ \ \isacommand{show}\isamarkupfalse%
\ {\isachardoublequoteopen}G\ \isactrlbold {\isasymand}\ H\ {\isasymin}\ S\ {\isasymlongrightarrow}\ {\isacharbraceleft}G{\isacharcomma}H{\isacharbraceright}\ {\isasymunion}\ S\ {\isasymin}\ C{\isachardoublequoteclose}\isanewline
\ \ \isacommand{proof}\isamarkupfalse%
\ {\isacharparenleft}rule\ impI{\isacharparenright}\isanewline
\ \ \ \ \isacommand{assume}\isamarkupfalse%
\ {\isachardoublequoteopen}G\ \isactrlbold {\isasymand}\ H\ {\isasymin}\ S{\isachardoublequoteclose}\isanewline
\ \ \ \ \isacommand{then}\isamarkupfalse%
\ \isacommand{have}\isamarkupfalse%
\ {\isachardoublequoteopen}Con\ {\isacharparenleft}G\ \isactrlbold {\isasymand}\ H{\isacharparenright}\ G\ H{\isachardoublequoteclose}\isanewline
\ \ \ \ \ \ \isacommand{by}\isamarkupfalse%
\ {\isacharparenleft}simp\ only{\isacharcolon}\ Con{\isachardot}intros{\isacharparenleft}{\isadigit{1}}{\isacharparenright}{\isacharparenright}\isanewline
\ \ \ \ \isacommand{let}\isamarkupfalse%
\ {\isacharquery}F{\isacharequal}{\isachardoublequoteopen}G\ \isactrlbold {\isasymand}\ H{\isachardoublequoteclose}\isanewline
\ \ \ \ \isacommand{have}\isamarkupfalse%
\ {\isachardoublequoteopen}Con\ {\isacharquery}F\ G\ H\ {\isasymlongrightarrow}\ {\isacharquery}F\ {\isasymin}\ S\ {\isasymlongrightarrow}\ {\isacharbraceleft}G{\isacharcomma}H{\isacharbraceright}\ {\isasymunion}\ S\ {\isasymin}\ C{\isachardoublequoteclose}\isanewline
\ \ \ \ \ \ \isacommand{using}\isamarkupfalse%
\ assms\ \isacommand{by}\isamarkupfalse%
\ {\isacharparenleft}iprover\ elim{\isacharcolon}\ allE{\isacharparenright}\isanewline
\ \ \ \ \isacommand{then}\isamarkupfalse%
\ \isacommand{have}\isamarkupfalse%
\ {\isachardoublequoteopen}{\isacharquery}F\ {\isasymin}\ S\ {\isasymlongrightarrow}\ {\isacharbraceleft}G{\isacharcomma}H{\isacharbraceright}\ {\isasymunion}\ S\ {\isasymin}\ C{\isachardoublequoteclose}\isanewline
\ \ \ \ \ \ \isacommand{using}\isamarkupfalse%
\ {\isacartoucheopen}Con\ {\isacharparenleft}G\ \isactrlbold {\isasymand}\ H{\isacharparenright}\ G\ H{\isacartoucheclose}\ \isacommand{by}\isamarkupfalse%
\ {\isacharparenleft}rule\ mp{\isacharparenright}\isanewline
\ \ \ \ \isacommand{thus}\isamarkupfalse%
\ {\isachardoublequoteopen}{\isacharbraceleft}G{\isacharcomma}H{\isacharbraceright}\ {\isasymunion}\ S\ {\isasymin}\ C{\isachardoublequoteclose}\isanewline
\ \ \ \ \ \ \isacommand{using}\isamarkupfalse%
\ {\isacartoucheopen}{\isacharparenleft}G\ \isactrlbold {\isasymand}\ H{\isacharparenright}\ {\isasymin}\ S{\isacartoucheclose}\ \isacommand{by}\isamarkupfalse%
\ {\isacharparenleft}rule\ mp{\isacharparenright}\isanewline
\ \ \isacommand{qed}\isamarkupfalse%
\isanewline
\isacommand{qed}\isamarkupfalse%
%
\endisatagproof
{\isafoldproof}%
%
\isadelimproof
\isanewline
%
\endisadelimproof
\isanewline
\isacommand{lemma}\isamarkupfalse%
\ pcp{\isacharunderscore}alt{\isadigit{2}}Con{\isadigit{2}}{\isacharcolon}\isanewline
\ \ \isakeyword{assumes}\ {\isachardoublequoteopen}{\isasymforall}F\ G\ H{\isachardot}\ Con\ F\ G\ H\ {\isasymlongrightarrow}\ F\ {\isasymin}\ S\ {\isasymlongrightarrow}\ {\isacharbraceleft}G{\isacharcomma}H{\isacharbraceright}\ {\isasymunion}\ S\ {\isasymin}\ C{\isachardoublequoteclose}\isanewline
\ \ \isakeyword{shows}\ {\isachardoublequoteopen}{\isasymforall}G{\isachardot}\ \isactrlbold {\isasymnot}\ {\isacharparenleft}\isactrlbold {\isasymnot}G{\isacharparenright}\ {\isasymin}\ S\ {\isasymlongrightarrow}\ {\isacharbraceleft}G{\isacharbraceright}\ {\isasymunion}\ S\ {\isasymin}\ C{\isachardoublequoteclose}\isanewline
%
\isadelimproof
%
\endisadelimproof
%
\isatagproof
\isacommand{proof}\isamarkupfalse%
\ {\isacharparenleft}rule\ allI{\isacharparenright}\isanewline
\ \ \isacommand{fix}\isamarkupfalse%
\ G\ \isanewline
\ \ \isacommand{show}\isamarkupfalse%
\ {\isachardoublequoteopen}\isactrlbold {\isasymnot}\ {\isacharparenleft}\isactrlbold {\isasymnot}G{\isacharparenright}\ {\isasymin}\ S\ {\isasymlongrightarrow}\ {\isacharbraceleft}G{\isacharbraceright}\ {\isasymunion}\ S\ {\isasymin}\ C{\isachardoublequoteclose}\isanewline
\ \ \isacommand{proof}\isamarkupfalse%
\ {\isacharparenleft}rule\ impI{\isacharparenright}\isanewline
\ \ \ \ \isacommand{assume}\isamarkupfalse%
\ {\isachardoublequoteopen}\isactrlbold {\isasymnot}{\isacharparenleft}\isactrlbold {\isasymnot}G{\isacharparenright}\ {\isasymin}\ S{\isachardoublequoteclose}\isanewline
\ \ \ \ \isacommand{then}\isamarkupfalse%
\ \isacommand{have}\isamarkupfalse%
\ {\isachardoublequoteopen}Con\ {\isacharparenleft}\isactrlbold {\isasymnot}{\isacharparenleft}\isactrlbold {\isasymnot}G{\isacharparenright}{\isacharparenright}\ G\ G{\isachardoublequoteclose}\isanewline
\ \ \ \ \ \ \isacommand{by}\isamarkupfalse%
\ {\isacharparenleft}simp\ only{\isacharcolon}\ Con{\isachardot}intros{\isacharparenleft}{\isadigit{4}}{\isacharparenright}{\isacharparenright}\isanewline
\ \ \ \ \isacommand{let}\isamarkupfalse%
\ {\isacharquery}F{\isacharequal}{\isachardoublequoteopen}\isactrlbold {\isasymnot}{\isacharparenleft}\isactrlbold {\isasymnot}\ G{\isacharparenright}{\isachardoublequoteclose}\isanewline
\ \ \ \ \isacommand{have}\isamarkupfalse%
\ {\isachardoublequoteopen}{\isasymforall}G\ H{\isachardot}\ Con\ {\isacharquery}F\ G\ H\ {\isasymlongrightarrow}\ {\isacharquery}F\ {\isasymin}\ S\ {\isasymlongrightarrow}\ {\isacharbraceleft}G{\isacharcomma}H{\isacharbraceright}\ {\isasymunion}\ S\ {\isasymin}\ C{\isachardoublequoteclose}\isanewline
\ \ \ \ \ \ \isacommand{using}\isamarkupfalse%
\ assms\ \isacommand{by}\isamarkupfalse%
\ {\isacharparenleft}rule\ allE{\isacharparenright}\isanewline
\ \ \ \ \isacommand{then}\isamarkupfalse%
\ \isacommand{have}\isamarkupfalse%
\ {\isachardoublequoteopen}{\isasymforall}H{\isachardot}\ Con\ {\isacharquery}F\ G\ H\ {\isasymlongrightarrow}\ {\isacharquery}F\ {\isasymin}\ S\ {\isasymlongrightarrow}\ {\isacharbraceleft}G{\isacharcomma}H{\isacharbraceright}\ {\isasymunion}\ S\ {\isasymin}\ C{\isachardoublequoteclose}\isanewline
\ \ \ \ \ \ \isacommand{by}\isamarkupfalse%
\ {\isacharparenleft}rule\ allE{\isacharparenright}\isanewline
\ \ \ \ \isacommand{then}\isamarkupfalse%
\ \isacommand{have}\isamarkupfalse%
\ {\isachardoublequoteopen}Con\ {\isacharquery}F\ G\ G\ {\isasymlongrightarrow}\ {\isacharquery}F\ {\isasymin}\ S\ {\isasymlongrightarrow}\ {\isacharbraceleft}G{\isacharcomma}G{\isacharbraceright}\ {\isasymunion}\ S\ {\isasymin}\ C{\isachardoublequoteclose}\isanewline
\ \ \ \ \ \ \isacommand{by}\isamarkupfalse%
\ {\isacharparenleft}rule\ allE{\isacharparenright}\isanewline
\ \ \ \ \isacommand{then}\isamarkupfalse%
\ \isacommand{have}\isamarkupfalse%
\ {\isachardoublequoteopen}{\isacharquery}F\ {\isasymin}\ S\ {\isasymlongrightarrow}\ {\isacharbraceleft}G{\isacharcomma}G{\isacharbraceright}\ {\isasymunion}\ S\ {\isasymin}\ C{\isachardoublequoteclose}\isanewline
\ \ \ \ \ \ \isacommand{using}\isamarkupfalse%
\ {\isacartoucheopen}Con\ {\isacharparenleft}\isactrlbold {\isasymnot}{\isacharparenleft}\isactrlbold {\isasymnot}G{\isacharparenright}{\isacharparenright}\ G\ G{\isacartoucheclose}\ \isacommand{by}\isamarkupfalse%
\ {\isacharparenleft}rule\ mp{\isacharparenright}\isanewline
\ \ \ \ \isacommand{then}\isamarkupfalse%
\ \isacommand{have}\isamarkupfalse%
\ {\isachardoublequoteopen}{\isacharbraceleft}G{\isacharcomma}G{\isacharbraceright}\ {\isasymunion}\ S\ {\isasymin}\ C{\isachardoublequoteclose}\isanewline
\ \ \ \ \ \ \isacommand{using}\isamarkupfalse%
\ {\isacartoucheopen}{\isacharparenleft}\isactrlbold {\isasymnot}{\isacharparenleft}\isactrlbold {\isasymnot}G{\isacharparenright}{\isacharparenright}\ {\isasymin}\ S{\isacartoucheclose}\ \isacommand{by}\isamarkupfalse%
\ {\isacharparenleft}rule\ mp{\isacharparenright}\isanewline
\ \ \ \ \isacommand{thus}\isamarkupfalse%
\ {\isachardoublequoteopen}{\isacharbraceleft}G{\isacharbraceright}\ {\isasymunion}\ S\ {\isasymin}\ C{\isachardoublequoteclose}\isanewline
\ \ \ \ \ \ \isacommand{by}\isamarkupfalse%
\ {\isacharparenleft}simp\ only{\isacharcolon}\ insert{\isacharunderscore}absorb{\isadigit{2}}{\isacharparenright}\isanewline
\ \ \isacommand{qed}\isamarkupfalse%
\isanewline
\isacommand{qed}\isamarkupfalse%
%
\endisatagproof
{\isafoldproof}%
%
\isadelimproof
\isanewline
%
\endisadelimproof
\isanewline
\isacommand{lemma}\isamarkupfalse%
\ pcp{\isacharunderscore}alt{\isadigit{2}}Con{\isadigit{3}}{\isacharcolon}\isanewline
\ \ \isakeyword{assumes}\ {\isachardoublequoteopen}{\isasymforall}F\ G\ H{\isachardot}\ Con\ F\ G\ H\ {\isasymlongrightarrow}\ F\ {\isasymin}\ S\ {\isasymlongrightarrow}\ {\isacharbraceleft}G{\isacharcomma}H{\isacharbraceright}\ {\isasymunion}\ S\ {\isasymin}\ C{\isachardoublequoteclose}\isanewline
\ \ \isakeyword{shows}\ {\isachardoublequoteopen}{\isasymforall}G\ H{\isachardot}\ \isactrlbold {\isasymnot}{\isacharparenleft}G\ \isactrlbold {\isasymor}\ H{\isacharparenright}\ {\isasymin}\ S\ {\isasymlongrightarrow}\ {\isacharbraceleft}\isactrlbold {\isasymnot}\ G{\isacharcomma}\ \isactrlbold {\isasymnot}\ H{\isacharbraceright}\ {\isasymunion}\ S\ {\isasymin}\ C{\isachardoublequoteclose}\isanewline
%
\isadelimproof
%
\endisadelimproof
%
\isatagproof
\isacommand{proof}\isamarkupfalse%
\ {\isacharparenleft}rule\ allI{\isacharparenright}{\isacharplus}\isanewline
\ \ \isacommand{fix}\isamarkupfalse%
\ G\ H\isanewline
\ \ \isacommand{show}\isamarkupfalse%
\ {\isachardoublequoteopen}\isactrlbold {\isasymnot}{\isacharparenleft}G\ \isactrlbold {\isasymor}\ H{\isacharparenright}\ {\isasymin}\ S\ {\isasymlongrightarrow}\ {\isacharbraceleft}\isactrlbold {\isasymnot}\ G{\isacharcomma}\ \isactrlbold {\isasymnot}\ H{\isacharbraceright}\ {\isasymunion}\ S\ {\isasymin}\ C{\isachardoublequoteclose}\isanewline
\ \ \isacommand{proof}\isamarkupfalse%
\ {\isacharparenleft}rule\ impI{\isacharparenright}\isanewline
\ \ \ \ \isacommand{assume}\isamarkupfalse%
\ {\isachardoublequoteopen}\isactrlbold {\isasymnot}{\isacharparenleft}G\ \isactrlbold {\isasymor}\ H{\isacharparenright}\ {\isasymin}\ S{\isachardoublequoteclose}\isanewline
\ \ \ \ \isacommand{then}\isamarkupfalse%
\ \isacommand{have}\isamarkupfalse%
\ {\isachardoublequoteopen}Con\ {\isacharparenleft}\isactrlbold {\isasymnot}{\isacharparenleft}G\ \isactrlbold {\isasymor}\ H{\isacharparenright}{\isacharparenright}\ {\isacharparenleft}\isactrlbold {\isasymnot}G{\isacharparenright}\ {\isacharparenleft}\isactrlbold {\isasymnot}H{\isacharparenright}{\isachardoublequoteclose}\isanewline
\ \ \ \ \ \ \isacommand{by}\isamarkupfalse%
\ {\isacharparenleft}simp\ only{\isacharcolon}\ Con{\isachardot}intros{\isacharparenleft}{\isadigit{2}}{\isacharparenright}{\isacharparenright}\isanewline
\ \ \ \ \isacommand{let}\isamarkupfalse%
\ {\isacharquery}F\ {\isacharequal}\ {\isachardoublequoteopen}\isactrlbold {\isasymnot}{\isacharparenleft}G\ \isactrlbold {\isasymor}\ H{\isacharparenright}{\isachardoublequoteclose}\isanewline
\ \ \ \ \isacommand{have}\isamarkupfalse%
\ {\isachardoublequoteopen}Con\ {\isacharquery}F\ {\isacharparenleft}\isactrlbold {\isasymnot}G{\isacharparenright}\ {\isacharparenleft}\isactrlbold {\isasymnot}H{\isacharparenright}\ {\isasymlongrightarrow}\ {\isacharquery}F\ {\isasymin}\ S\ {\isasymlongrightarrow}\ {\isacharbraceleft}\isactrlbold {\isasymnot}G{\isacharcomma}\isactrlbold {\isasymnot}H{\isacharbraceright}\ {\isasymunion}\ S\ {\isasymin}\ C{\isachardoublequoteclose}\isanewline
\ \ \ \ \ \ \isacommand{using}\isamarkupfalse%
\ assms\ \isacommand{by}\isamarkupfalse%
\ {\isacharparenleft}iprover\ elim{\isacharcolon}\ allE{\isacharparenright}\isanewline
\ \ \ \ \isacommand{then}\isamarkupfalse%
\ \isacommand{have}\isamarkupfalse%
\ {\isachardoublequoteopen}{\isacharquery}F\ {\isasymin}\ S\ {\isasymlongrightarrow}\ {\isacharbraceleft}\isactrlbold {\isasymnot}G{\isacharcomma}\isactrlbold {\isasymnot}H{\isacharbraceright}\ {\isasymunion}\ S\ {\isasymin}\ C{\isachardoublequoteclose}\isanewline
\ \ \ \ \ \ \isacommand{using}\isamarkupfalse%
\ {\isacartoucheopen}Con\ {\isacharparenleft}\isactrlbold {\isasymnot}{\isacharparenleft}G\ \isactrlbold {\isasymor}\ H{\isacharparenright}{\isacharparenright}\ {\isacharparenleft}\isactrlbold {\isasymnot}G{\isacharparenright}\ {\isacharparenleft}\isactrlbold {\isasymnot}H{\isacharparenright}{\isacartoucheclose}\ \isacommand{by}\isamarkupfalse%
\ {\isacharparenleft}rule\ mp{\isacharparenright}\isanewline
\ \ \ \ \isacommand{thus}\isamarkupfalse%
\ {\isachardoublequoteopen}{\isacharbraceleft}\isactrlbold {\isasymnot}G{\isacharcomma}\isactrlbold {\isasymnot}H{\isacharbraceright}\ {\isasymunion}\ S\ {\isasymin}\ C{\isachardoublequoteclose}\isanewline
\ \ \ \ \ \ \isacommand{using}\isamarkupfalse%
\ {\isacartoucheopen}\isactrlbold {\isasymnot}{\isacharparenleft}G\ \isactrlbold {\isasymor}\ H{\isacharparenright}\ {\isasymin}\ S{\isacartoucheclose}\ \isacommand{by}\isamarkupfalse%
\ {\isacharparenleft}rule\ mp{\isacharparenright}\isanewline
\ \ \isacommand{qed}\isamarkupfalse%
\isanewline
\isacommand{qed}\isamarkupfalse%
%
\endisatagproof
{\isafoldproof}%
%
\isadelimproof
\isanewline
%
\endisadelimproof
\isanewline
\isacommand{lemma}\isamarkupfalse%
\ pcp{\isacharunderscore}alt{\isadigit{2}}Con{\isadigit{4}}{\isacharcolon}\isanewline
\ \ \isakeyword{assumes}\ {\isachardoublequoteopen}{\isasymforall}F\ G\ H{\isachardot}\ Con\ F\ G\ H\ {\isasymlongrightarrow}\ F\ {\isasymin}\ S\ {\isasymlongrightarrow}\ {\isacharbraceleft}G{\isacharcomma}H{\isacharbraceright}\ {\isasymunion}\ S\ {\isasymin}\ C{\isachardoublequoteclose}\isanewline
\ \ \isakeyword{shows}\ {\isachardoublequoteopen}{\isasymforall}G\ H{\isachardot}\ \isactrlbold {\isasymnot}{\isacharparenleft}G\ \isactrlbold {\isasymrightarrow}\ H{\isacharparenright}\ {\isasymin}\ S\ {\isasymlongrightarrow}\ {\isacharbraceleft}G{\isacharcomma}\isactrlbold {\isasymnot}\ H{\isacharbraceright}\ {\isasymunion}\ S\ {\isasymin}\ C{\isachardoublequoteclose}\isanewline
%
\isadelimproof
%
\endisadelimproof
%
\isatagproof
\isacommand{proof}\isamarkupfalse%
\ {\isacharparenleft}rule\ allI{\isacharparenright}{\isacharplus}\isanewline
\ \ \isacommand{fix}\isamarkupfalse%
\ G\ H\isanewline
\ \ \isacommand{show}\isamarkupfalse%
\ {\isachardoublequoteopen}\isactrlbold {\isasymnot}{\isacharparenleft}G\ \isactrlbold {\isasymrightarrow}\ H{\isacharparenright}\ {\isasymin}\ S\ {\isasymlongrightarrow}\ {\isacharbraceleft}G{\isacharcomma}\isactrlbold {\isasymnot}\ H{\isacharbraceright}\ {\isasymunion}\ S\ {\isasymin}\ C{\isachardoublequoteclose}\isanewline
\ \ \isacommand{proof}\isamarkupfalse%
\ {\isacharparenleft}rule\ impI{\isacharparenright}\isanewline
\ \ \ \ \isacommand{assume}\isamarkupfalse%
\ {\isachardoublequoteopen}\isactrlbold {\isasymnot}{\isacharparenleft}G\ \isactrlbold {\isasymrightarrow}\ H{\isacharparenright}\ {\isasymin}\ S{\isachardoublequoteclose}\isanewline
\ \ \ \ \isacommand{then}\isamarkupfalse%
\ \isacommand{have}\isamarkupfalse%
\ {\isachardoublequoteopen}Con\ {\isacharparenleft}\isactrlbold {\isasymnot}{\isacharparenleft}G\ \isactrlbold {\isasymrightarrow}\ H{\isacharparenright}{\isacharparenright}\ G\ {\isacharparenleft}\isactrlbold {\isasymnot}H{\isacharparenright}{\isachardoublequoteclose}\isanewline
\ \ \ \ \ \ \isacommand{by}\isamarkupfalse%
\ {\isacharparenleft}simp\ only{\isacharcolon}\ Con{\isachardot}intros{\isacharparenleft}{\isadigit{3}}{\isacharparenright}{\isacharparenright}\isanewline
\ \ \ \ \isacommand{let}\isamarkupfalse%
\ {\isacharquery}F\ {\isacharequal}\ {\isachardoublequoteopen}\isactrlbold {\isasymnot}{\isacharparenleft}G\ \isactrlbold {\isasymrightarrow}\ H{\isacharparenright}{\isachardoublequoteclose}\isanewline
\ \ \ \ \isacommand{have}\isamarkupfalse%
\ {\isachardoublequoteopen}Con\ {\isacharquery}F\ G\ {\isacharparenleft}\isactrlbold {\isasymnot}H{\isacharparenright}\ {\isasymlongrightarrow}\ {\isacharquery}F\ {\isasymin}\ S\ {\isasymlongrightarrow}\ {\isacharbraceleft}G{\isacharcomma}\isactrlbold {\isasymnot}H{\isacharbraceright}\ {\isasymunion}\ S\ {\isasymin}\ C{\isachardoublequoteclose}\isanewline
\ \ \ \ \ \ \isacommand{using}\isamarkupfalse%
\ assms\ \isacommand{by}\isamarkupfalse%
\ {\isacharparenleft}iprover\ elim{\isacharcolon}\ allE{\isacharparenright}\isanewline
\ \ \ \ \isacommand{then}\isamarkupfalse%
\ \isacommand{have}\isamarkupfalse%
\ {\isachardoublequoteopen}{\isacharquery}F\ {\isasymin}\ S\ {\isasymlongrightarrow}\ {\isacharbraceleft}G{\isacharcomma}\isactrlbold {\isasymnot}H{\isacharbraceright}\ {\isasymunion}\ S\ {\isasymin}\ C{\isachardoublequoteclose}\ \ \isanewline
\ \ \ \ \ \ \isacommand{using}\isamarkupfalse%
\ {\isacartoucheopen}Con\ {\isacharparenleft}\isactrlbold {\isasymnot}{\isacharparenleft}G\ \isactrlbold {\isasymrightarrow}\ H{\isacharparenright}{\isacharparenright}\ G\ {\isacharparenleft}\isactrlbold {\isasymnot}H{\isacharparenright}{\isacartoucheclose}\ \isacommand{by}\isamarkupfalse%
\ {\isacharparenleft}rule\ mp{\isacharparenright}\isanewline
\ \ \ \ \isacommand{thus}\isamarkupfalse%
\ {\isachardoublequoteopen}{\isacharbraceleft}G{\isacharcomma}\isactrlbold {\isasymnot}H{\isacharbraceright}\ {\isasymunion}\ S\ {\isasymin}\ C{\isachardoublequoteclose}\isanewline
\ \ \ \ \ \ \isacommand{using}\isamarkupfalse%
\ {\isacartoucheopen}\isactrlbold {\isasymnot}{\isacharparenleft}G\ \isactrlbold {\isasymrightarrow}\ H{\isacharparenright}\ {\isasymin}\ S{\isacartoucheclose}\ \isacommand{by}\isamarkupfalse%
\ {\isacharparenleft}rule\ mp{\isacharparenright}\isanewline
\ \ \isacommand{qed}\isamarkupfalse%
\isanewline
\isacommand{qed}\isamarkupfalse%
%
\endisatagproof
{\isafoldproof}%
%
\isadelimproof
%
\endisadelimproof
%
\begin{isamarkuptext}%
Por otro lado, los siguientes lemas auxiliares prueban el resto de condiciones de la
  definición de propiedad de consistencia proposicional a partir de la hipótesis referente a 
  fórmulas de tipo \isa{{\isasymbeta}}.%
\end{isamarkuptext}\isamarkuptrue%
\isacommand{lemma}\isamarkupfalse%
\ pcp{\isacharunderscore}alt{\isadigit{2}}Dis{\isadigit{1}}{\isacharcolon}\isanewline
\ \ \isakeyword{assumes}\ {\isachardoublequoteopen}{\isasymforall}F\ G\ H{\isachardot}\ Dis\ F\ G\ H\ {\isasymlongrightarrow}\ F\ {\isasymin}\ S\ {\isasymlongrightarrow}\ {\isacharbraceleft}G{\isacharbraceright}\ {\isasymunion}\ S\ {\isasymin}\ C\ {\isasymor}\ {\isacharbraceleft}H{\isacharbraceright}\ {\isasymunion}\ S\ {\isasymin}\ C{\isachardoublequoteclose}\isanewline
\ \ \isakeyword{shows}\ {\isachardoublequoteopen}{\isasymforall}G\ H{\isachardot}\ G\ \isactrlbold {\isasymor}\ H\ {\isasymin}\ S\ {\isasymlongrightarrow}\ {\isacharbraceleft}G{\isacharbraceright}\ {\isasymunion}\ S\ {\isasymin}\ C\ {\isasymor}\ {\isacharbraceleft}H{\isacharbraceright}\ {\isasymunion}\ S\ {\isasymin}\ C{\isachardoublequoteclose}\isanewline
%
\isadelimproof
%
\endisadelimproof
%
\isatagproof
\isacommand{proof}\isamarkupfalse%
\ {\isacharparenleft}rule\ allI{\isacharparenright}{\isacharplus}\isanewline
\ \ \isacommand{fix}\isamarkupfalse%
\ G\ H\isanewline
\ \ \isacommand{show}\isamarkupfalse%
\ {\isachardoublequoteopen}G\ \isactrlbold {\isasymor}\ H\ {\isasymin}\ S\ {\isasymlongrightarrow}\ {\isacharbraceleft}G{\isacharbraceright}\ {\isasymunion}\ S\ {\isasymin}\ C\ {\isasymor}\ {\isacharbraceleft}H{\isacharbraceright}\ {\isasymunion}\ S\ {\isasymin}\ C{\isachardoublequoteclose}\isanewline
\ \ \isacommand{proof}\isamarkupfalse%
\ {\isacharparenleft}rule\ impI{\isacharparenright}\isanewline
\ \ \ \ \isacommand{assume}\isamarkupfalse%
\ {\isachardoublequoteopen}G\ \isactrlbold {\isasymor}\ H\ {\isasymin}\ S{\isachardoublequoteclose}\isanewline
\ \ \ \ \isacommand{then}\isamarkupfalse%
\ \isacommand{have}\isamarkupfalse%
\ {\isachardoublequoteopen}Dis\ {\isacharparenleft}G\ \isactrlbold {\isasymor}\ H{\isacharparenright}\ G\ H{\isachardoublequoteclose}\isanewline
\ \ \ \ \ \ \isacommand{by}\isamarkupfalse%
\ {\isacharparenleft}simp\ only{\isacharcolon}\ Dis{\isachardot}intros{\isacharparenleft}{\isadigit{1}}{\isacharparenright}{\isacharparenright}\isanewline
\ \ \ \ \isacommand{let}\isamarkupfalse%
\ {\isacharquery}F{\isacharequal}{\isachardoublequoteopen}G\ \isactrlbold {\isasymor}\ H{\isachardoublequoteclose}\isanewline
\ \ \ \ \isacommand{have}\isamarkupfalse%
\ {\isachardoublequoteopen}Dis\ {\isacharquery}F\ G\ H\ {\isasymlongrightarrow}\ {\isacharquery}F\ {\isasymin}\ S\ {\isasymlongrightarrow}\ {\isacharbraceleft}G{\isacharbraceright}\ {\isasymunion}\ S\ {\isasymin}\ C\ {\isasymor}\ {\isacharbraceleft}H{\isacharbraceright}\ {\isasymunion}\ S\ {\isasymin}\ C{\isachardoublequoteclose}\isanewline
\ \ \ \ \ \ \isacommand{using}\isamarkupfalse%
\ assms\ \isacommand{by}\isamarkupfalse%
\ {\isacharparenleft}iprover\ elim{\isacharcolon}\ allE{\isacharparenright}\isanewline
\ \ \ \ \isacommand{then}\isamarkupfalse%
\ \isacommand{have}\isamarkupfalse%
\ {\isachardoublequoteopen}{\isacharquery}F\ {\isasymin}\ S\ {\isasymlongrightarrow}\ {\isacharbraceleft}G{\isacharbraceright}\ {\isasymunion}\ S\ {\isasymin}\ C\ {\isasymor}\ {\isacharbraceleft}H{\isacharbraceright}\ {\isasymunion}\ S\ {\isasymin}\ C{\isachardoublequoteclose}\isanewline
\ \ \ \ \ \ \isacommand{using}\isamarkupfalse%
\ {\isacartoucheopen}Dis\ {\isacharparenleft}G\ \isactrlbold {\isasymor}\ H{\isacharparenright}\ G\ H{\isacartoucheclose}\ \isacommand{by}\isamarkupfalse%
\ {\isacharparenleft}rule\ mp{\isacharparenright}\isanewline
\ \ \ \ \isacommand{thus}\isamarkupfalse%
\ {\isachardoublequoteopen}{\isacharbraceleft}G{\isacharbraceright}\ {\isasymunion}\ S\ {\isasymin}\ C\ {\isasymor}\ {\isacharbraceleft}H{\isacharbraceright}\ {\isasymunion}\ S\ {\isasymin}\ C{\isachardoublequoteclose}\isanewline
\ \ \ \ \ \ \isacommand{using}\isamarkupfalse%
\ {\isacartoucheopen}{\isacharparenleft}G\ \isactrlbold {\isasymor}\ H{\isacharparenright}\ {\isasymin}\ S{\isacartoucheclose}\ \isacommand{by}\isamarkupfalse%
\ {\isacharparenleft}rule\ mp{\isacharparenright}\isanewline
\ \ \isacommand{qed}\isamarkupfalse%
\isanewline
\isacommand{qed}\isamarkupfalse%
%
\endisatagproof
{\isafoldproof}%
%
\isadelimproof
\isanewline
%
\endisadelimproof
\isanewline
\isacommand{lemma}\isamarkupfalse%
\ pcp{\isacharunderscore}alt{\isadigit{2}}Dis{\isadigit{2}}{\isacharcolon}\isanewline
\ \ \isakeyword{assumes}\ {\isachardoublequoteopen}{\isasymforall}F\ G\ H{\isachardot}\ Dis\ F\ G\ H\ {\isasymlongrightarrow}\ F\ {\isasymin}\ S\ {\isasymlongrightarrow}\ {\isacharbraceleft}G{\isacharbraceright}\ {\isasymunion}\ S\ {\isasymin}\ C\ {\isasymor}\ {\isacharbraceleft}H{\isacharbraceright}\ {\isasymunion}\ S\ {\isasymin}\ C{\isachardoublequoteclose}\isanewline
\ \ \isakeyword{shows}\ {\isachardoublequoteopen}{\isasymforall}G\ H{\isachardot}\ G\ \isactrlbold {\isasymrightarrow}\ H\ {\isasymin}\ S\ {\isasymlongrightarrow}\ {\isacharbraceleft}\isactrlbold {\isasymnot}\ G{\isacharbraceright}\ {\isasymunion}\ S\ {\isasymin}\ C\ {\isasymor}\ {\isacharbraceleft}H{\isacharbraceright}\ {\isasymunion}\ S\ {\isasymin}\ C{\isachardoublequoteclose}\isanewline
%
\isadelimproof
%
\endisadelimproof
%
\isatagproof
\isacommand{proof}\isamarkupfalse%
\ {\isacharparenleft}rule\ allI{\isacharparenright}{\isacharplus}\isanewline
\ \ \isacommand{fix}\isamarkupfalse%
\ G\ H\isanewline
\ \ \isacommand{show}\isamarkupfalse%
\ {\isachardoublequoteopen}G\ \isactrlbold {\isasymrightarrow}\ H\ {\isasymin}\ S\ {\isasymlongrightarrow}\ {\isacharbraceleft}\isactrlbold {\isasymnot}\ G{\isacharbraceright}\ {\isasymunion}\ S\ {\isasymin}\ C\ {\isasymor}\ {\isacharbraceleft}H{\isacharbraceright}\ {\isasymunion}\ S\ {\isasymin}\ C{\isachardoublequoteclose}\isanewline
\ \ \isacommand{proof}\isamarkupfalse%
\ {\isacharparenleft}rule\ impI{\isacharparenright}\isanewline
\ \ \ \ \isacommand{assume}\isamarkupfalse%
\ {\isachardoublequoteopen}G\ \isactrlbold {\isasymrightarrow}\ H\ {\isasymin}\ S{\isachardoublequoteclose}\isanewline
\ \ \ \ \isacommand{then}\isamarkupfalse%
\ \isacommand{have}\isamarkupfalse%
\ {\isachardoublequoteopen}Dis\ {\isacharparenleft}G\ \isactrlbold {\isasymrightarrow}\ H{\isacharparenright}\ {\isacharparenleft}\isactrlbold {\isasymnot}G{\isacharparenright}\ H{\isachardoublequoteclose}\isanewline
\ \ \ \ \ \ \isacommand{by}\isamarkupfalse%
\ {\isacharparenleft}simp\ only{\isacharcolon}\ Dis{\isachardot}intros{\isacharparenleft}{\isadigit{2}}{\isacharparenright}{\isacharparenright}\isanewline
\ \ \ \ \isacommand{let}\isamarkupfalse%
\ {\isacharquery}F{\isacharequal}{\isachardoublequoteopen}G\ \isactrlbold {\isasymrightarrow}\ H{\isachardoublequoteclose}\ \isanewline
\ \ \ \ \isacommand{have}\isamarkupfalse%
\ {\isachardoublequoteopen}Dis\ {\isacharquery}F\ {\isacharparenleft}\isactrlbold {\isasymnot}G{\isacharparenright}\ H\ {\isasymlongrightarrow}\ {\isacharquery}F\ {\isasymin}\ S\ {\isasymlongrightarrow}\ {\isacharbraceleft}\isactrlbold {\isasymnot}G{\isacharbraceright}\ {\isasymunion}\ S\ {\isasymin}\ C\ {\isasymor}\ {\isacharbraceleft}H{\isacharbraceright}\ {\isasymunion}\ S\ {\isasymin}\ C{\isachardoublequoteclose}\isanewline
\ \ \ \ \ \ \isacommand{using}\isamarkupfalse%
\ assms\ \isacommand{by}\isamarkupfalse%
\ {\isacharparenleft}iprover\ elim{\isacharcolon}\ allE{\isacharparenright}\isanewline
\ \ \ \ \isacommand{then}\isamarkupfalse%
\ \isacommand{have}\isamarkupfalse%
\ {\isachardoublequoteopen}{\isacharquery}F\ {\isasymin}\ S\ {\isasymlongrightarrow}\ {\isacharbraceleft}\isactrlbold {\isasymnot}G{\isacharbraceright}\ {\isasymunion}\ S\ {\isasymin}\ C\ {\isasymor}\ {\isacharbraceleft}H{\isacharbraceright}\ {\isasymunion}\ S\ {\isasymin}\ C{\isachardoublequoteclose}\isanewline
\ \ \ \ \ \ \isacommand{using}\isamarkupfalse%
\ {\isacartoucheopen}Dis\ {\isacharparenleft}G\ \isactrlbold {\isasymrightarrow}\ H{\isacharparenright}\ {\isacharparenleft}\isactrlbold {\isasymnot}G{\isacharparenright}\ H{\isacartoucheclose}\ \isacommand{by}\isamarkupfalse%
\ {\isacharparenleft}rule\ mp{\isacharparenright}\isanewline
\ \ \ \ \isacommand{thus}\isamarkupfalse%
\ {\isachardoublequoteopen}{\isacharbraceleft}\isactrlbold {\isasymnot}G{\isacharbraceright}\ {\isasymunion}\ S\ {\isasymin}\ C\ {\isasymor}\ {\isacharbraceleft}H{\isacharbraceright}\ {\isasymunion}\ S\ {\isasymin}\ C{\isachardoublequoteclose}\isanewline
\ \ \ \ \ \ \isacommand{using}\isamarkupfalse%
\ {\isacartoucheopen}{\isacharparenleft}G\ \isactrlbold {\isasymrightarrow}\ H{\isacharparenright}\ {\isasymin}\ S{\isacartoucheclose}\ \isacommand{by}\isamarkupfalse%
\ {\isacharparenleft}rule\ mp{\isacharparenright}\isanewline
\ \ \isacommand{qed}\isamarkupfalse%
\isanewline
\isacommand{qed}\isamarkupfalse%
%
\endisatagproof
{\isafoldproof}%
%
\isadelimproof
\isanewline
%
\endisadelimproof
\isanewline
\isacommand{lemma}\isamarkupfalse%
\ pcp{\isacharunderscore}alt{\isadigit{2}}Dis{\isadigit{3}}{\isacharcolon}\isanewline
\ \ \isakeyword{assumes}\ {\isachardoublequoteopen}{\isasymforall}F\ G\ H{\isachardot}\ Dis\ F\ G\ H\ {\isasymlongrightarrow}\ F\ {\isasymin}\ S\ {\isasymlongrightarrow}\ {\isacharbraceleft}G{\isacharbraceright}\ {\isasymunion}\ S\ {\isasymin}\ C\ {\isasymor}\ {\isacharbraceleft}H{\isacharbraceright}\ {\isasymunion}\ S\ {\isasymin}\ C{\isachardoublequoteclose}\isanewline
\ \ \isakeyword{shows}\ {\isachardoublequoteopen}{\isasymforall}G\ H{\isachardot}\ \isactrlbold {\isasymnot}{\isacharparenleft}G\ \isactrlbold {\isasymand}\ H{\isacharparenright}\ {\isasymin}\ S\ {\isasymlongrightarrow}\ {\isacharbraceleft}\isactrlbold {\isasymnot}\ G{\isacharbraceright}\ {\isasymunion}\ S\ {\isasymin}\ C\ {\isasymor}\ {\isacharbraceleft}\isactrlbold {\isasymnot}\ H{\isacharbraceright}\ {\isasymunion}\ S\ {\isasymin}\ C{\isachardoublequoteclose}\isanewline
%
\isadelimproof
%
\endisadelimproof
%
\isatagproof
\isacommand{proof}\isamarkupfalse%
\ {\isacharparenleft}rule\ allI{\isacharparenright}{\isacharplus}\isanewline
\ \ \isacommand{fix}\isamarkupfalse%
\ G\ H\isanewline
\ \ \isacommand{show}\isamarkupfalse%
\ {\isachardoublequoteopen}\isactrlbold {\isasymnot}{\isacharparenleft}G\ \isactrlbold {\isasymand}\ H{\isacharparenright}\ {\isasymin}\ S\ {\isasymlongrightarrow}\ {\isacharbraceleft}\isactrlbold {\isasymnot}\ G{\isacharbraceright}\ {\isasymunion}\ S\ {\isasymin}\ C\ {\isasymor}\ {\isacharbraceleft}\isactrlbold {\isasymnot}\ H{\isacharbraceright}\ {\isasymunion}\ S\ {\isasymin}\ C{\isachardoublequoteclose}\isanewline
\ \ \isacommand{proof}\isamarkupfalse%
\ {\isacharparenleft}rule\ impI{\isacharparenright}\isanewline
\ \ \ \ \isacommand{assume}\isamarkupfalse%
\ {\isachardoublequoteopen}\isactrlbold {\isasymnot}{\isacharparenleft}G\ \isactrlbold {\isasymand}\ H{\isacharparenright}\ {\isasymin}\ S{\isachardoublequoteclose}\isanewline
\ \ \ \ \isacommand{then}\isamarkupfalse%
\ \isacommand{have}\isamarkupfalse%
\ {\isachardoublequoteopen}Dis\ {\isacharparenleft}\isactrlbold {\isasymnot}{\isacharparenleft}G\ \isactrlbold {\isasymand}\ H{\isacharparenright}{\isacharparenright}\ {\isacharparenleft}\isactrlbold {\isasymnot}G{\isacharparenright}\ {\isacharparenleft}\isactrlbold {\isasymnot}H{\isacharparenright}{\isachardoublequoteclose}\isanewline
\ \ \ \ \ \ \isacommand{by}\isamarkupfalse%
\ {\isacharparenleft}simp\ only{\isacharcolon}\ Dis{\isachardot}intros{\isacharparenleft}{\isadigit{3}}{\isacharparenright}{\isacharparenright}\isanewline
\ \ \ \ \isacommand{let}\isamarkupfalse%
\ {\isacharquery}F{\isacharequal}{\isachardoublequoteopen}\isactrlbold {\isasymnot}{\isacharparenleft}G\ \isactrlbold {\isasymand}\ H{\isacharparenright}{\isachardoublequoteclose}\isanewline
\ \ \ \ \isacommand{have}\isamarkupfalse%
\ {\isachardoublequoteopen}Dis\ {\isacharquery}F\ {\isacharparenleft}\isactrlbold {\isasymnot}G{\isacharparenright}\ {\isacharparenleft}\isactrlbold {\isasymnot}H{\isacharparenright}\ {\isasymlongrightarrow}\ {\isacharquery}F\ {\isasymin}\ S\ {\isasymlongrightarrow}\ {\isacharbraceleft}\isactrlbold {\isasymnot}G{\isacharbraceright}\ {\isasymunion}\ S\ {\isasymin}\ C\ {\isasymor}\ {\isacharbraceleft}\isactrlbold {\isasymnot}H{\isacharbraceright}\ {\isasymunion}\ S\ {\isasymin}\ C{\isachardoublequoteclose}\isanewline
\ \ \ \ \ \ \isacommand{using}\isamarkupfalse%
\ assms\ \isacommand{by}\isamarkupfalse%
\ {\isacharparenleft}iprover\ elim{\isacharcolon}\ allE{\isacharparenright}\isanewline
\ \ \ \ \isacommand{then}\isamarkupfalse%
\ \isacommand{have}\isamarkupfalse%
\ {\isachardoublequoteopen}{\isacharquery}F\ {\isasymin}\ S\ {\isasymlongrightarrow}\ {\isacharbraceleft}\isactrlbold {\isasymnot}G{\isacharbraceright}\ {\isasymunion}\ S\ {\isasymin}\ C\ {\isasymor}\ {\isacharbraceleft}\isactrlbold {\isasymnot}H{\isacharbraceright}\ {\isasymunion}\ S\ {\isasymin}\ C{\isachardoublequoteclose}\isanewline
\ \ \ \ \ \ \isacommand{using}\isamarkupfalse%
\ {\isacartoucheopen}Dis\ {\isacharparenleft}\isactrlbold {\isasymnot}{\isacharparenleft}G\ \isactrlbold {\isasymand}\ H{\isacharparenright}{\isacharparenright}\ {\isacharparenleft}\isactrlbold {\isasymnot}G{\isacharparenright}\ {\isacharparenleft}\isactrlbold {\isasymnot}H{\isacharparenright}{\isacartoucheclose}\ \isacommand{by}\isamarkupfalse%
\ {\isacharparenleft}rule\ mp{\isacharparenright}\isanewline
\ \ \ \ \isacommand{thus}\isamarkupfalse%
\ {\isachardoublequoteopen}{\isacharbraceleft}\isactrlbold {\isasymnot}G{\isacharbraceright}\ {\isasymunion}\ S\ {\isasymin}\ C\ {\isasymor}\ {\isacharbraceleft}\isactrlbold {\isasymnot}H{\isacharbraceright}\ {\isasymunion}\ S\ {\isasymin}\ C{\isachardoublequoteclose}\isanewline
\ \ \ \ \ \ \isacommand{using}\isamarkupfalse%
\ {\isacartoucheopen}\isactrlbold {\isasymnot}{\isacharparenleft}G\ \isactrlbold {\isasymand}\ H{\isacharparenright}\ {\isasymin}\ S{\isacartoucheclose}\ \isacommand{by}\isamarkupfalse%
\ {\isacharparenleft}rule\ mp{\isacharparenright}\isanewline
\ \ \isacommand{qed}\isamarkupfalse%
\isanewline
\isacommand{qed}\isamarkupfalse%
%
\endisatagproof
{\isafoldproof}%
%
\isadelimproof
%
\endisadelimproof
%
\begin{isamarkuptext}%
De este modo, procedemos a la demostración detallada de esta implicación en Isabelle.%
\end{isamarkuptext}\isamarkuptrue%
\isacommand{lemma}\isamarkupfalse%
\ pcp{\isacharunderscore}alt{\isadigit{2}}{\isacharcolon}\ \isanewline
\ \ \isakeyword{assumes}\ {\isachardoublequoteopen}{\isasymforall}S\ {\isasymin}\ C{\isachardot}\ {\isasymbottom}\ {\isasymnotin}\ S\isanewline
{\isasymand}\ {\isacharparenleft}{\isasymforall}k{\isachardot}\ Atom\ k\ {\isasymin}\ S\ {\isasymlongrightarrow}\ \isactrlbold {\isasymnot}\ {\isacharparenleft}Atom\ k{\isacharparenright}\ {\isasymin}\ S\ {\isasymlongrightarrow}\ False{\isacharparenright}\isanewline
{\isasymand}\ {\isacharparenleft}{\isasymforall}F\ G\ H{\isachardot}\ Con\ F\ G\ H\ {\isasymlongrightarrow}\ F\ {\isasymin}\ S\ {\isasymlongrightarrow}\ {\isacharbraceleft}G{\isacharcomma}H{\isacharbraceright}\ {\isasymunion}\ S\ {\isasymin}\ C{\isacharparenright}\isanewline
{\isasymand}\ {\isacharparenleft}{\isasymforall}F\ G\ H{\isachardot}\ Dis\ F\ G\ H\ {\isasymlongrightarrow}\ F\ {\isasymin}\ S\ {\isasymlongrightarrow}\ {\isacharbraceleft}G{\isacharbraceright}\ {\isasymunion}\ S\ {\isasymin}\ C\ {\isasymor}\ {\isacharbraceleft}H{\isacharbraceright}\ {\isasymunion}\ S\ {\isasymin}\ C{\isacharparenright}{\isachardoublequoteclose}\isanewline
\ \ \isakeyword{shows}\ {\isachardoublequoteopen}pcp\ C{\isachardoublequoteclose}\isanewline
%
\isadelimproof
\ \ %
\endisadelimproof
%
\isatagproof
\isacommand{unfolding}\isamarkupfalse%
\ pcp{\isacharunderscore}def\isanewline
\isacommand{proof}\isamarkupfalse%
\ {\isacharparenleft}rule\ ballI{\isacharparenright}\isanewline
\ \ \isacommand{fix}\isamarkupfalse%
\ S\isanewline
\ \ \isacommand{assume}\isamarkupfalse%
\ {\isachardoublequoteopen}S\ {\isasymin}\ C{\isachardoublequoteclose}\isanewline
\ \ \isacommand{have}\isamarkupfalse%
\ H{\isacharcolon}{\isachardoublequoteopen}{\isasymbottom}\ {\isasymnotin}\ S\isanewline
\ \ \ \ {\isasymand}\ {\isacharparenleft}{\isasymforall}k{\isachardot}\ Atom\ k\ {\isasymin}\ S\ {\isasymlongrightarrow}\ \isactrlbold {\isasymnot}\ {\isacharparenleft}Atom\ k{\isacharparenright}\ {\isasymin}\ S\ {\isasymlongrightarrow}\ False{\isacharparenright}\isanewline
\ \ \ \ {\isasymand}\ {\isacharparenleft}{\isasymforall}F\ G\ H{\isachardot}\ Con\ F\ G\ H\ {\isasymlongrightarrow}\ F\ {\isasymin}\ S\ {\isasymlongrightarrow}\ {\isacharbraceleft}G{\isacharcomma}H{\isacharbraceright}\ {\isasymunion}\ S\ {\isasymin}\ C{\isacharparenright}\isanewline
\ \ \ \ {\isasymand}\ {\isacharparenleft}{\isasymforall}F\ G\ H{\isachardot}\ Dis\ F\ G\ H\ {\isasymlongrightarrow}\ F\ {\isasymin}\ S\ {\isasymlongrightarrow}\ {\isacharbraceleft}G{\isacharbraceright}\ {\isasymunion}\ S\ {\isasymin}\ C\ {\isasymor}\ {\isacharbraceleft}H{\isacharbraceright}\ {\isasymunion}\ S\ {\isasymin}\ C{\isacharparenright}{\isachardoublequoteclose}\isanewline
\ \ \ \ \isacommand{using}\isamarkupfalse%
\ assms\ {\isacartoucheopen}S\ {\isasymin}\ C{\isacartoucheclose}\ \isacommand{by}\isamarkupfalse%
\ {\isacharparenleft}rule\ bspec{\isacharparenright}\isanewline
\ \ \isacommand{then}\isamarkupfalse%
\ \isacommand{have}\isamarkupfalse%
\ Con{\isacharcolon}{\isachardoublequoteopen}{\isasymforall}F\ G\ H{\isachardot}\ Con\ F\ G\ H\ {\isasymlongrightarrow}\ F\ {\isasymin}\ S\ {\isasymlongrightarrow}\ {\isacharbraceleft}G{\isacharcomma}H{\isacharbraceright}\ {\isasymunion}\ S\ {\isasymin}\ C{\isachardoublequoteclose}\isanewline
\ \ \ \ \isacommand{by}\isamarkupfalse%
\ {\isacharparenleft}iprover\ elim{\isacharcolon}\ conjunct{\isadigit{1}}\ conjunct{\isadigit{2}}{\isacharparenright}\isanewline
\ \ \isacommand{have}\isamarkupfalse%
\ Dis{\isacharcolon}{\isachardoublequoteopen}{\isasymforall}F\ G\ H{\isachardot}\ Dis\ F\ G\ H\ {\isasymlongrightarrow}\ F\ {\isasymin}\ S\ {\isasymlongrightarrow}\ {\isacharbraceleft}G{\isacharbraceright}\ {\isasymunion}\ S\ {\isasymin}\ C\ {\isasymor}\ {\isacharbraceleft}H{\isacharbraceright}\ {\isasymunion}\ S\ {\isasymin}\ C{\isachardoublequoteclose}\isanewline
\ \ \ \ \isacommand{using}\isamarkupfalse%
\ H\ \isacommand{by}\isamarkupfalse%
\ {\isacharparenleft}iprover\ elim{\isacharcolon}\ conjunct{\isadigit{1}}\ conjunct{\isadigit{2}}{\isacharparenright}\isanewline
\ \ \isacommand{have}\isamarkupfalse%
\ {\isadigit{1}}{\isacharcolon}{\isachardoublequoteopen}{\isasymbottom}\ {\isasymnotin}\ S\isanewline
\ \ \ \ {\isasymand}\ {\isacharparenleft}{\isasymforall}k{\isachardot}\ Atom\ k\ {\isasymin}\ S\ {\isasymlongrightarrow}\ \isactrlbold {\isasymnot}\ {\isacharparenleft}Atom\ k{\isacharparenright}\ {\isasymin}\ S\ {\isasymlongrightarrow}\ False{\isacharparenright}{\isachardoublequoteclose}\isanewline
\ \ \ \ \isacommand{using}\isamarkupfalse%
\ H\ \isacommand{by}\isamarkupfalse%
\ {\isacharparenleft}iprover\ elim{\isacharcolon}\ conjunct{\isadigit{1}}{\isacharparenright}\isanewline
\ \ \isacommand{have}\isamarkupfalse%
\ {\isadigit{2}}{\isacharcolon}{\isachardoublequoteopen}{\isasymforall}G\ H{\isachardot}\ G\ \isactrlbold {\isasymand}\ H\ {\isasymin}\ S\ {\isasymlongrightarrow}\ {\isacharbraceleft}G{\isacharcomma}H{\isacharbraceright}\ {\isasymunion}\ S\ {\isasymin}\ C{\isachardoublequoteclose}\isanewline
\ \ \ \ \isacommand{using}\isamarkupfalse%
\ Con\ \isacommand{by}\isamarkupfalse%
\ {\isacharparenleft}rule\ pcp{\isacharunderscore}alt{\isadigit{2}}Con{\isadigit{1}}{\isacharparenright}\isanewline
\ \ \isacommand{have}\isamarkupfalse%
\ {\isadigit{3}}{\isacharcolon}{\isachardoublequoteopen}{\isasymforall}G\ H{\isachardot}\ G\ \isactrlbold {\isasymor}\ H\ {\isasymin}\ S\ {\isasymlongrightarrow}\ {\isacharbraceleft}G{\isacharbraceright}\ {\isasymunion}\ S\ {\isasymin}\ C\ {\isasymor}\ {\isacharbraceleft}H{\isacharbraceright}\ {\isasymunion}\ S\ {\isasymin}\ C{\isachardoublequoteclose}\isanewline
\ \ \ \ \isacommand{using}\isamarkupfalse%
\ Dis\ \isacommand{by}\isamarkupfalse%
\ {\isacharparenleft}rule\ pcp{\isacharunderscore}alt{\isadigit{2}}Dis{\isadigit{1}}{\isacharparenright}\isanewline
\ \ \isacommand{have}\isamarkupfalse%
\ {\isadigit{4}}{\isacharcolon}{\isachardoublequoteopen}{\isasymforall}G\ H{\isachardot}\ G\ \isactrlbold {\isasymrightarrow}\ H\ {\isasymin}\ S\ {\isasymlongrightarrow}\ {\isacharbraceleft}\isactrlbold {\isasymnot}G{\isacharbraceright}\ {\isasymunion}\ S\ {\isasymin}\ C\ {\isasymor}\ {\isacharbraceleft}H{\isacharbraceright}\ {\isasymunion}\ S\ {\isasymin}\ C{\isachardoublequoteclose}\isanewline
\ \ \ \ \isacommand{using}\isamarkupfalse%
\ Dis\ \isacommand{by}\isamarkupfalse%
\ {\isacharparenleft}rule\ pcp{\isacharunderscore}alt{\isadigit{2}}Dis{\isadigit{2}}{\isacharparenright}\isanewline
\ \ \isacommand{have}\isamarkupfalse%
\ {\isadigit{5}}{\isacharcolon}{\isachardoublequoteopen}{\isasymforall}G{\isachardot}\ \isactrlbold {\isasymnot}\ {\isacharparenleft}\isactrlbold {\isasymnot}G{\isacharparenright}\ {\isasymin}\ S\ {\isasymlongrightarrow}\ {\isacharbraceleft}G{\isacharbraceright}\ {\isasymunion}\ S\ {\isasymin}\ C{\isachardoublequoteclose}\isanewline
\ \ \ \ \isacommand{using}\isamarkupfalse%
\ Con\ \isacommand{by}\isamarkupfalse%
\ {\isacharparenleft}rule\ pcp{\isacharunderscore}alt{\isadigit{2}}Con{\isadigit{2}}{\isacharparenright}\isanewline
\ \ \isacommand{have}\isamarkupfalse%
\ {\isadigit{6}}{\isacharcolon}{\isachardoublequoteopen}{\isasymforall}G\ H{\isachardot}\ \isactrlbold {\isasymnot}{\isacharparenleft}G\ \isactrlbold {\isasymand}\ H{\isacharparenright}\ {\isasymin}\ S\ {\isasymlongrightarrow}\ {\isacharbraceleft}\isactrlbold {\isasymnot}\ G{\isacharbraceright}\ {\isasymunion}\ S\ {\isasymin}\ C\ {\isasymor}\ {\isacharbraceleft}\isactrlbold {\isasymnot}\ H{\isacharbraceright}\ {\isasymunion}\ S\ {\isasymin}\ C{\isachardoublequoteclose}\isanewline
\ \ \ \ \isacommand{using}\isamarkupfalse%
\ Dis\ \isacommand{by}\isamarkupfalse%
\ {\isacharparenleft}rule\ pcp{\isacharunderscore}alt{\isadigit{2}}Dis{\isadigit{3}}{\isacharparenright}\isanewline
\ \ \isacommand{have}\isamarkupfalse%
\ {\isadigit{7}}{\isacharcolon}{\isachardoublequoteopen}{\isasymforall}G\ H{\isachardot}\ \isactrlbold {\isasymnot}{\isacharparenleft}G\ \isactrlbold {\isasymor}\ H{\isacharparenright}\ {\isasymin}\ S\ {\isasymlongrightarrow}\ {\isacharbraceleft}\isactrlbold {\isasymnot}\ G{\isacharcomma}\ \isactrlbold {\isasymnot}\ H{\isacharbraceright}\ {\isasymunion}\ S\ {\isasymin}\ C{\isachardoublequoteclose}\isanewline
\ \ \ \ \isacommand{using}\isamarkupfalse%
\ Con\ \isacommand{by}\isamarkupfalse%
\ {\isacharparenleft}rule\ pcp{\isacharunderscore}alt{\isadigit{2}}Con{\isadigit{3}}{\isacharparenright}\isanewline
\ \ \isacommand{have}\isamarkupfalse%
\ {\isadigit{8}}{\isacharcolon}{\isachardoublequoteopen}{\isasymforall}G\ H{\isachardot}\ \isactrlbold {\isasymnot}{\isacharparenleft}G\ \isactrlbold {\isasymrightarrow}\ H{\isacharparenright}\ {\isasymin}\ S\ {\isasymlongrightarrow}\ {\isacharbraceleft}G{\isacharcomma}\isactrlbold {\isasymnot}\ H{\isacharbraceright}\ {\isasymunion}\ S\ {\isasymin}\ C{\isachardoublequoteclose}\isanewline
\ \ \ \ \isacommand{using}\isamarkupfalse%
\ Con\ \isacommand{by}\isamarkupfalse%
\ {\isacharparenleft}rule\ pcp{\isacharunderscore}alt{\isadigit{2}}Con{\isadigit{4}}{\isacharparenright}\isanewline
\ \ \isacommand{have}\isamarkupfalse%
\ A{\isacharcolon}{\isachardoublequoteopen}{\isasymbottom}\ {\isasymnotin}\ S\isanewline
\ \ \ \ {\isasymand}\ {\isacharparenleft}{\isasymforall}k{\isachardot}\ Atom\ k\ {\isasymin}\ S\ {\isasymlongrightarrow}\ \isactrlbold {\isasymnot}\ {\isacharparenleft}Atom\ k{\isacharparenright}\ {\isasymin}\ S\ {\isasymlongrightarrow}\ False{\isacharparenright}\isanewline
\ \ \ \ {\isasymand}\ {\isacharparenleft}{\isasymforall}G\ H{\isachardot}\ G\ \isactrlbold {\isasymand}\ H\ {\isasymin}\ S\ {\isasymlongrightarrow}\ {\isacharbraceleft}G{\isacharcomma}H{\isacharbraceright}\ {\isasymunion}\ S\ {\isasymin}\ C{\isacharparenright}\isanewline
\ \ \ \ {\isasymand}\ {\isacharparenleft}{\isasymforall}G\ H{\isachardot}\ G\ \isactrlbold {\isasymor}\ H\ {\isasymin}\ S\ {\isasymlongrightarrow}\ {\isacharbraceleft}G{\isacharbraceright}\ {\isasymunion}\ S\ {\isasymin}\ C\ {\isasymor}\ {\isacharbraceleft}H{\isacharbraceright}\ {\isasymunion}\ S\ {\isasymin}\ C{\isacharparenright}\isanewline
\ \ \ \ {\isasymand}\ {\isacharparenleft}{\isasymforall}G\ H{\isachardot}\ G\ \isactrlbold {\isasymrightarrow}\ H\ {\isasymin}\ S\ {\isasymlongrightarrow}\ {\isacharbraceleft}\isactrlbold {\isasymnot}G{\isacharbraceright}\ {\isasymunion}\ S\ {\isasymin}\ C\ {\isasymor}\ {\isacharbraceleft}H{\isacharbraceright}\ {\isasymunion}\ S\ {\isasymin}\ C{\isacharparenright}{\isachardoublequoteclose}\isanewline
\ \ \ \ \isacommand{using}\isamarkupfalse%
\ {\isadigit{1}}\ {\isadigit{2}}\ {\isadigit{3}}\ {\isadigit{4}}\ \isacommand{by}\isamarkupfalse%
\ {\isacharparenleft}iprover\ intro{\isacharcolon}\ conjI{\isacharparenright}\isanewline
\ \ \isacommand{have}\isamarkupfalse%
\ B{\isacharcolon}{\isachardoublequoteopen}{\isacharparenleft}{\isasymforall}G{\isachardot}\ \isactrlbold {\isasymnot}\ {\isacharparenleft}\isactrlbold {\isasymnot}G{\isacharparenright}\ {\isasymin}\ S\ {\isasymlongrightarrow}\ {\isacharbraceleft}G{\isacharbraceright}\ {\isasymunion}\ S\ {\isasymin}\ C{\isacharparenright}\isanewline
\ \ \ \ {\isasymand}\ {\isacharparenleft}{\isasymforall}G\ H{\isachardot}\ \isactrlbold {\isasymnot}{\isacharparenleft}G\ \isactrlbold {\isasymand}\ H{\isacharparenright}\ {\isasymin}\ S\ {\isasymlongrightarrow}\ {\isacharbraceleft}\isactrlbold {\isasymnot}\ G{\isacharbraceright}\ {\isasymunion}\ S\ {\isasymin}\ C\ {\isasymor}\ {\isacharbraceleft}\isactrlbold {\isasymnot}\ H{\isacharbraceright}\ {\isasymunion}\ S\ {\isasymin}\ C{\isacharparenright}\isanewline
\ \ \ \ {\isasymand}\ {\isacharparenleft}{\isasymforall}G\ H{\isachardot}\ \isactrlbold {\isasymnot}{\isacharparenleft}G\ \isactrlbold {\isasymor}\ H{\isacharparenright}\ {\isasymin}\ S\ {\isasymlongrightarrow}\ {\isacharbraceleft}\isactrlbold {\isasymnot}\ G{\isacharcomma}\ \isactrlbold {\isasymnot}\ H{\isacharbraceright}\ {\isasymunion}\ S\ {\isasymin}\ C{\isacharparenright}\isanewline
\ \ \ \ {\isasymand}\ {\isacharparenleft}{\isasymforall}G\ H{\isachardot}\ \isactrlbold {\isasymnot}{\isacharparenleft}G\ \isactrlbold {\isasymrightarrow}\ H{\isacharparenright}\ {\isasymin}\ S\ {\isasymlongrightarrow}\ {\isacharbraceleft}G{\isacharcomma}\isactrlbold {\isasymnot}\ H{\isacharbraceright}\ {\isasymunion}\ S\ {\isasymin}\ C{\isacharparenright}{\isachardoublequoteclose}\isanewline
\ \ \ \ \isacommand{using}\isamarkupfalse%
\ {\isadigit{5}}\ {\isadigit{6}}\ {\isadigit{7}}\ {\isadigit{8}}\ \isacommand{by}\isamarkupfalse%
\ {\isacharparenleft}iprover\ intro{\isacharcolon}\ conjI{\isacharparenright}\isanewline
\ \ \isacommand{have}\isamarkupfalse%
\ {\isachardoublequoteopen}{\isacharparenleft}{\isasymbottom}\ {\isasymnotin}\ S\isanewline
\ \ \ \ {\isasymand}\ {\isacharparenleft}{\isasymforall}k{\isachardot}\ Atom\ k\ {\isasymin}\ S\ {\isasymlongrightarrow}\ \isactrlbold {\isasymnot}\ {\isacharparenleft}Atom\ k{\isacharparenright}\ {\isasymin}\ S\ {\isasymlongrightarrow}\ False{\isacharparenright}\isanewline
\ \ \ \ {\isasymand}\ {\isacharparenleft}{\isasymforall}G\ H{\isachardot}\ G\ \isactrlbold {\isasymand}\ H\ {\isasymin}\ S\ {\isasymlongrightarrow}\ {\isacharbraceleft}G{\isacharcomma}H{\isacharbraceright}\ {\isasymunion}\ S\ {\isasymin}\ C{\isacharparenright}\isanewline
\ \ \ \ {\isasymand}\ {\isacharparenleft}{\isasymforall}G\ H{\isachardot}\ G\ \isactrlbold {\isasymor}\ H\ {\isasymin}\ S\ {\isasymlongrightarrow}\ {\isacharbraceleft}G{\isacharbraceright}\ {\isasymunion}\ S\ {\isasymin}\ C\ {\isasymor}\ {\isacharbraceleft}H{\isacharbraceright}\ {\isasymunion}\ S\ {\isasymin}\ C{\isacharparenright}\isanewline
\ \ \ \ {\isasymand}\ {\isacharparenleft}{\isasymforall}G\ H{\isachardot}\ G\ \isactrlbold {\isasymrightarrow}\ H\ {\isasymin}\ S\ {\isasymlongrightarrow}\ {\isacharbraceleft}\isactrlbold {\isasymnot}G{\isacharbraceright}\ {\isasymunion}\ S\ {\isasymin}\ C\ {\isasymor}\ {\isacharbraceleft}H{\isacharbraceright}\ {\isasymunion}\ S\ {\isasymin}\ C{\isacharparenright}{\isacharparenright}\isanewline
\ \ \ \ {\isasymand}\ {\isacharparenleft}{\isacharparenleft}{\isasymforall}G{\isachardot}\ \isactrlbold {\isasymnot}\ {\isacharparenleft}\isactrlbold {\isasymnot}G{\isacharparenright}\ {\isasymin}\ S\ {\isasymlongrightarrow}\ {\isacharbraceleft}G{\isacharbraceright}\ {\isasymunion}\ S\ {\isasymin}\ C{\isacharparenright}\isanewline
\ \ \ \ {\isasymand}\ {\isacharparenleft}{\isasymforall}G\ H{\isachardot}\ \isactrlbold {\isasymnot}{\isacharparenleft}G\ \isactrlbold {\isasymand}\ H{\isacharparenright}\ {\isasymin}\ S\ {\isasymlongrightarrow}\ {\isacharbraceleft}\isactrlbold {\isasymnot}\ G{\isacharbraceright}\ {\isasymunion}\ S\ {\isasymin}\ C\ {\isasymor}\ {\isacharbraceleft}\isactrlbold {\isasymnot}\ H{\isacharbraceright}\ {\isasymunion}\ S\ {\isasymin}\ C{\isacharparenright}\isanewline
\ \ \ \ {\isasymand}\ {\isacharparenleft}{\isasymforall}G\ H{\isachardot}\ \isactrlbold {\isasymnot}{\isacharparenleft}G\ \isactrlbold {\isasymor}\ H{\isacharparenright}\ {\isasymin}\ S\ {\isasymlongrightarrow}\ {\isacharbraceleft}\isactrlbold {\isasymnot}\ G{\isacharcomma}\ \isactrlbold {\isasymnot}\ H{\isacharbraceright}\ {\isasymunion}\ S\ {\isasymin}\ C{\isacharparenright}\isanewline
\ \ \ \ {\isasymand}\ {\isacharparenleft}{\isasymforall}G\ H{\isachardot}\ \isactrlbold {\isasymnot}{\isacharparenleft}G\ \isactrlbold {\isasymrightarrow}\ H{\isacharparenright}\ {\isasymin}\ S\ {\isasymlongrightarrow}\ {\isacharbraceleft}G{\isacharcomma}\isactrlbold {\isasymnot}\ H{\isacharbraceright}\ {\isasymunion}\ S\ {\isasymin}\ C{\isacharparenright}{\isacharparenright}{\isachardoublequoteclose}\isanewline
\ \ \ \ \isacommand{using}\isamarkupfalse%
\ A\ B\ \isacommand{by}\isamarkupfalse%
\ {\isacharparenleft}rule\ conjI{\isacharparenright}\isanewline
\ \ \isacommand{thus}\isamarkupfalse%
\ {\isachardoublequoteopen}{\isasymbottom}\ {\isasymnotin}\ S\isanewline
\ \ \ \ {\isasymand}\ {\isacharparenleft}{\isasymforall}k{\isachardot}\ Atom\ k\ {\isasymin}\ S\ {\isasymlongrightarrow}\ \isactrlbold {\isasymnot}\ {\isacharparenleft}Atom\ k{\isacharparenright}\ {\isasymin}\ S\ {\isasymlongrightarrow}\ False{\isacharparenright}\isanewline
\ \ \ \ {\isasymand}\ {\isacharparenleft}{\isasymforall}G\ H{\isachardot}\ G\ \isactrlbold {\isasymand}\ H\ {\isasymin}\ S\ {\isasymlongrightarrow}\ {\isacharbraceleft}G{\isacharcomma}H{\isacharbraceright}\ {\isasymunion}\ S\ {\isasymin}\ C{\isacharparenright}\isanewline
\ \ \ \ {\isasymand}\ {\isacharparenleft}{\isasymforall}G\ H{\isachardot}\ G\ \isactrlbold {\isasymor}\ H\ {\isasymin}\ S\ {\isasymlongrightarrow}\ {\isacharbraceleft}G{\isacharbraceright}\ {\isasymunion}\ S\ {\isasymin}\ C\ {\isasymor}\ {\isacharbraceleft}H{\isacharbraceright}\ {\isasymunion}\ S\ {\isasymin}\ C{\isacharparenright}\isanewline
\ \ \ \ {\isasymand}\ {\isacharparenleft}{\isasymforall}G\ H{\isachardot}\ G\ \isactrlbold {\isasymrightarrow}\ H\ {\isasymin}\ S\ {\isasymlongrightarrow}\ {\isacharbraceleft}\isactrlbold {\isasymnot}G{\isacharbraceright}\ {\isasymunion}\ S\ {\isasymin}\ C\ {\isasymor}\ {\isacharbraceleft}H{\isacharbraceright}\ {\isasymunion}\ S\ {\isasymin}\ C{\isacharparenright}\isanewline
\ \ \ \ {\isasymand}\ {\isacharparenleft}{\isasymforall}G{\isachardot}\ \isactrlbold {\isasymnot}\ {\isacharparenleft}\isactrlbold {\isasymnot}G{\isacharparenright}\ {\isasymin}\ S\ {\isasymlongrightarrow}\ {\isacharbraceleft}G{\isacharbraceright}\ {\isasymunion}\ S\ {\isasymin}\ C{\isacharparenright}\isanewline
\ \ \ \ {\isasymand}\ {\isacharparenleft}{\isasymforall}G\ H{\isachardot}\ \isactrlbold {\isasymnot}{\isacharparenleft}G\ \isactrlbold {\isasymand}\ H{\isacharparenright}\ {\isasymin}\ S\ {\isasymlongrightarrow}\ {\isacharbraceleft}\isactrlbold {\isasymnot}\ G{\isacharbraceright}\ {\isasymunion}\ S\ {\isasymin}\ C\ {\isasymor}\ {\isacharbraceleft}\isactrlbold {\isasymnot}\ H{\isacharbraceright}\ {\isasymunion}\ S\ {\isasymin}\ C{\isacharparenright}\isanewline
\ \ \ \ {\isasymand}\ {\isacharparenleft}{\isasymforall}G\ H{\isachardot}\ \isactrlbold {\isasymnot}{\isacharparenleft}G\ \isactrlbold {\isasymor}\ H{\isacharparenright}\ {\isasymin}\ S\ {\isasymlongrightarrow}\ {\isacharbraceleft}\isactrlbold {\isasymnot}\ G{\isacharcomma}\ \isactrlbold {\isasymnot}\ H{\isacharbraceright}\ {\isasymunion}\ S\ {\isasymin}\ C{\isacharparenright}\isanewline
\ \ \ \ {\isasymand}\ {\isacharparenleft}{\isasymforall}G\ H{\isachardot}\ \isactrlbold {\isasymnot}{\isacharparenleft}G\ \isactrlbold {\isasymrightarrow}\ H{\isacharparenright}\ {\isasymin}\ S\ {\isasymlongrightarrow}\ {\isacharbraceleft}G{\isacharcomma}\isactrlbold {\isasymnot}\ H{\isacharbraceright}\ {\isasymunion}\ S\ {\isasymin}\ C{\isacharparenright}{\isachardoublequoteclose}\isanewline
\ \ \ \ \isacommand{by}\isamarkupfalse%
\ {\isacharparenleft}iprover\ intro{\isacharcolon}\ conj{\isacharunderscore}assoc{\isacharparenright}\isanewline
\isacommand{qed}\isamarkupfalse%
%
\endisatagproof
{\isafoldproof}%
%
\isadelimproof
%
\endisadelimproof
%
\begin{isamarkuptext}%
Una vez probadas detalladamente en Isabelle cada una de las implicaciones de la
  equivalencia, podemos finalmente concluir con la demostración del lema completo.%
\end{isamarkuptext}\isamarkuptrue%
\isacommand{lemma}\isamarkupfalse%
\ {\isachardoublequoteopen}pcp\ C\ {\isacharequal}\ {\isacharparenleft}{\isasymforall}S\ {\isasymin}\ C{\isachardot}\ {\isasymbottom}\ {\isasymnotin}\ S\isanewline
{\isasymand}\ {\isacharparenleft}{\isasymforall}k{\isachardot}\ Atom\ k\ {\isasymin}\ S\ {\isasymlongrightarrow}\ \isactrlbold {\isasymnot}\ {\isacharparenleft}Atom\ k{\isacharparenright}\ {\isasymin}\ S\ {\isasymlongrightarrow}\ False{\isacharparenright}\isanewline
{\isasymand}\ {\isacharparenleft}{\isasymforall}F\ G\ H{\isachardot}\ Con\ F\ G\ H\ {\isasymlongrightarrow}\ F\ {\isasymin}\ S\ {\isasymlongrightarrow}\ {\isacharbraceleft}G{\isacharcomma}H{\isacharbraceright}\ {\isasymunion}\ S\ {\isasymin}\ C{\isacharparenright}\isanewline
{\isasymand}\ {\isacharparenleft}{\isasymforall}F\ G\ H{\isachardot}\ Dis\ F\ G\ H\ {\isasymlongrightarrow}\ F\ {\isasymin}\ S\ {\isasymlongrightarrow}\ {\isacharbraceleft}G{\isacharbraceright}\ {\isasymunion}\ S\ {\isasymin}\ C\ {\isasymor}\ {\isacharbraceleft}H{\isacharbraceright}\ {\isasymunion}\ S\ {\isasymin}\ C{\isacharparenright}{\isacharparenright}{\isachardoublequoteclose}\isanewline
%
\isadelimproof
%
\endisadelimproof
%
\isatagproof
\isacommand{proof}\isamarkupfalse%
\ {\isacharparenleft}rule\ iffI{\isacharparenright}\isanewline
\ \ \isacommand{assume}\isamarkupfalse%
\ {\isachardoublequoteopen}pcp\ C{\isachardoublequoteclose}\isanewline
\ \ \isacommand{thus}\isamarkupfalse%
\ {\isachardoublequoteopen}{\isasymforall}S\ {\isasymin}\ C{\isachardot}\ {\isasymbottom}\ {\isasymnotin}\ S\isanewline
{\isasymand}\ {\isacharparenleft}{\isasymforall}k{\isachardot}\ Atom\ k\ {\isasymin}\ S\ {\isasymlongrightarrow}\ \isactrlbold {\isasymnot}\ {\isacharparenleft}Atom\ k{\isacharparenright}\ {\isasymin}\ S\ {\isasymlongrightarrow}\ False{\isacharparenright}\isanewline
{\isasymand}\ {\isacharparenleft}{\isasymforall}F\ G\ H{\isachardot}\ Con\ F\ G\ H\ {\isasymlongrightarrow}\ F\ {\isasymin}\ S\ {\isasymlongrightarrow}\ {\isacharbraceleft}G{\isacharcomma}H{\isacharbraceright}\ {\isasymunion}\ S\ {\isasymin}\ C{\isacharparenright}\isanewline
{\isasymand}\ {\isacharparenleft}{\isasymforall}F\ G\ H{\isachardot}\ Dis\ F\ G\ H\ {\isasymlongrightarrow}\ F\ {\isasymin}\ S\ {\isasymlongrightarrow}\ {\isacharbraceleft}G{\isacharbraceright}\ {\isasymunion}\ S\ {\isasymin}\ C\ {\isasymor}\ {\isacharbraceleft}H{\isacharbraceright}\ {\isasymunion}\ S\ {\isasymin}\ C{\isacharparenright}{\isachardoublequoteclose}\isanewline
\ \ \ \ \isacommand{by}\isamarkupfalse%
\ {\isacharparenleft}rule\ pcp{\isacharunderscore}alt{\isadigit{1}}{\isacharparenright}\isanewline
\isacommand{next}\isamarkupfalse%
\isanewline
\ \ \isacommand{assume}\isamarkupfalse%
\ {\isachardoublequoteopen}{\isasymforall}S\ {\isasymin}\ C{\isachardot}\ {\isasymbottom}\ {\isasymnotin}\ S\isanewline
{\isasymand}\ {\isacharparenleft}{\isasymforall}k{\isachardot}\ Atom\ k\ {\isasymin}\ S\ {\isasymlongrightarrow}\ \isactrlbold {\isasymnot}\ {\isacharparenleft}Atom\ k{\isacharparenright}\ {\isasymin}\ S\ {\isasymlongrightarrow}\ False{\isacharparenright}\isanewline
{\isasymand}\ {\isacharparenleft}{\isasymforall}F\ G\ H{\isachardot}\ Con\ F\ G\ H\ {\isasymlongrightarrow}\ F\ {\isasymin}\ S\ {\isasymlongrightarrow}\ {\isacharbraceleft}G{\isacharcomma}H{\isacharbraceright}\ {\isasymunion}\ S\ {\isasymin}\ C{\isacharparenright}\isanewline
{\isasymand}\ {\isacharparenleft}{\isasymforall}F\ G\ H{\isachardot}\ Dis\ F\ G\ H\ {\isasymlongrightarrow}\ F\ {\isasymin}\ S\ {\isasymlongrightarrow}\ {\isacharbraceleft}G{\isacharbraceright}\ {\isasymunion}\ S\ {\isasymin}\ C\ {\isasymor}\ {\isacharbraceleft}H{\isacharbraceright}\ {\isasymunion}\ S\ {\isasymin}\ C{\isacharparenright}{\isachardoublequoteclose}\isanewline
\ \ \isacommand{thus}\isamarkupfalse%
\ {\isachardoublequoteopen}pcp\ C{\isachardoublequoteclose}\isanewline
\ \ \ \ \isacommand{by}\isamarkupfalse%
\ {\isacharparenleft}rule\ pcp{\isacharunderscore}alt{\isadigit{2}}{\isacharparenright}\isanewline
\isacommand{qed}\isamarkupfalse%
%
\endisatagproof
{\isafoldproof}%
%
\isadelimproof
%
\endisadelimproof
%
\begin{isamarkuptext}%
La demostración automática del resultado en Isabelle/HOL se muestra finalmente a 
  continuación.%
\end{isamarkuptext}\isamarkuptrue%
\isacommand{lemma}\isamarkupfalse%
\ pcp{\isacharunderscore}alt{\isacharcolon}\ {\isachardoublequoteopen}pcp\ C\ {\isacharequal}\ {\isacharparenleft}{\isasymforall}S\ {\isasymin}\ C{\isachardot}\isanewline
\ \ {\isasymbottom}\ {\isasymnotin}\ S\isanewline
{\isasymand}\ {\isacharparenleft}{\isasymforall}k{\isachardot}\ Atom\ k\ {\isasymin}\ S\ {\isasymlongrightarrow}\ \isactrlbold {\isasymnot}\ {\isacharparenleft}Atom\ k{\isacharparenright}\ {\isasymin}\ S\ {\isasymlongrightarrow}\ False{\isacharparenright}\isanewline
{\isasymand}\ {\isacharparenleft}{\isasymforall}F\ G\ H{\isachardot}\ Con\ F\ G\ H\ {\isasymlongrightarrow}\ F\ {\isasymin}\ S\ {\isasymlongrightarrow}\ {\isacharbraceleft}G{\isacharcomma}H{\isacharbraceright}\ {\isasymunion}\ S\ {\isasymin}\ C{\isacharparenright}\isanewline
{\isasymand}\ {\isacharparenleft}{\isasymforall}F\ G\ H{\isachardot}\ Dis\ F\ G\ H\ {\isasymlongrightarrow}\ F\ {\isasymin}\ S\ {\isasymlongrightarrow}\ {\isacharbraceleft}G{\isacharbraceright}\ {\isasymunion}\ S\ {\isasymin}\ C\ {\isasymor}\ {\isacharbraceleft}H{\isacharbraceright}\ {\isasymunion}\ S\ {\isasymin}\ C{\isacharparenright}{\isacharparenright}{\isachardoublequoteclose}\isanewline
%
\isadelimproof
\ \ %
\endisadelimproof
%
\isatagproof
\isacommand{apply}\isamarkupfalse%
{\isacharparenleft}simp\ add{\isacharcolon}\ pcp{\isacharunderscore}def\ con{\isacharunderscore}dis{\isacharunderscore}simps{\isacharparenright}\isanewline
\ \ \isacommand{apply}\isamarkupfalse%
{\isacharparenleft}rule\ iffI{\isacharsemicolon}\ unfold\ Ball{\isacharunderscore}def{\isacharsemicolon}\ elim\ all{\isacharunderscore}forward{\isacharparenright}\isanewline
\ \ \isacommand{by}\isamarkupfalse%
\ {\isacharparenleft}auto\ simp\ add{\isacharcolon}\ insert{\isacharunderscore}absorb\ split{\isacharcolon}\ formula{\isachardot}splits{\isacharparenright}\isanewline
%
\endisatagproof
{\isafoldproof}%
%
\isadelimproof
%
\endisadelimproof
%
\isadelimtheory
%
\endisadelimtheory
%
\isatagtheory
%
\endisatagtheory
{\isafoldtheory}%
%
\isadelimtheory
%
\endisadelimtheory
%
\end{isabellebody}%
\endinput
%:%file=~/TFM/TFM/Pcp.thy%:%
%:%19=12%:%
%:%20=13%:%
%:%20=14%:%
%:%21=15%:%
%:%22=16%:%
%:%23=17%:%
%:%24=18%:%
%:%25=19%:%
%:%26=20%:%
%:%27=21%:%
%:%28=22%:%
%:%29=23%:%
%:%30=24%:%
%:%31=25%:%
%:%32=26%:%
%:%33=27%:%
%:%34=28%:%
%:%35=29%:%
%:%36=30%:%
%:%37=31%:%
%:%38=32%:%
%:%39=33%:%
%:%40=34%:%
%:%41=35%:%
%:%42=36%:%
%:%43=37%:%
%:%44=38%:%
%:%45=39%:%
%:%47=41%:%
%:%48=41%:%
%:%59=52%:%
%:%60=53%:%
%:%61=54%:%
%:%63=56%:%
%:%64=56%:%
%:%67=57%:%
%:%71=57%:%
%:%72=57%:%
%:%73=57%:%
%:%82=59%:%
%:%83=60%:%
%:%84=61%:%
%:%86=63%:%
%:%87=63%:%
%:%90=64%:%
%:%94=64%:%
%:%95=64%:%
%:%96=64%:%
%:%101=64%:%
%:%104=65%:%
%:%105=66%:%
%:%106=66%:%
%:%108=68%:%
%:%111=69%:%
%:%115=69%:%
%:%116=69%:%
%:%117=69%:%
%:%126=71%:%
%:%127=72%:%
%:%128=73%:%
%:%129=74%:%
%:%131=76%:%
%:%132=76%:%
%:%133=77%:%
%:%136=78%:%
%:%140=78%:%
%:%141=78%:%
%:%142=78%:%
%:%151=80%:%
%:%152=81%:%
%:%153=82%:%
%:%154=83%:%
%:%155=84%:%
%:%156=85%:%
%:%157=86%:%
%:%158=87%:%
%:%159=88%:%
%:%160=89%:%
%:%161=90%:%
%:%162=91%:%
%:%163=92%:%
%:%164=93%:%
%:%165=94%:%
%:%166=95%:%
%:%167=96%:%
%:%168=97%:%
%:%169=98%:%
%:%170=99%:%
%:%171=100%:%
%:%172=101%:%
%:%173=102%:%
%:%175=104%:%
%:%176=104%:%
%:%179=107%:%
%:%182=108%:%
%:%186=108%:%
%:%196=110%:%
%:%197=111%:%
%:%198=112%:%
%:%199=113%:%
%:%200=114%:%
%:%201=115%:%
%:%202=116%:%
%:%203=117%:%
%:%204=118%:%
%:%205=119%:%
%:%206=120%:%
%:%207=121%:%
%:%208=122%:%
%:%209=123%:%
%:%210=124%:%
%:%211=125%:%
%:%212=126%:%
%:%213=127%:%
%:%214=128%:%
%:%215=129%:%
%:%216=130%:%
%:%217=131%:%
%:%218=132%:%
%:%219=133%:%
%:%220=134%:%
%:%221=135%:%
%:%222=136%:%
%:%223=137%:%
%:%224=138%:%
%:%225=139%:%
%:%226=140%:%
%:%227=141%:%
%:%228=142%:%
%:%229=143%:%
%:%230=144%:%
%:%231=145%:%
%:%232=146%:%
%:%233=147%:%
%:%234=148%:%
%:%235=149%:%
%:%236=150%:%
%:%237=151%:%
%:%238=152%:%
%:%239=153%:%
%:%240=154%:%
%:%241=155%:%
%:%242=156%:%
%:%243=157%:%
%:%244=158%:%
%:%245=159%:%
%:%246=160%:%
%:%247=161%:%
%:%248=162%:%
%:%249=163%:%
%:%250=164%:%
%:%251=165%:%
%:%252=166%:%
%:%253=167%:%
%:%254=168%:%
%:%255=169%:%
%:%256=170%:%
%:%257=171%:%
%:%258=172%:%
%:%259=173%:%
%:%260=174%:%
%:%261=175%:%
%:%262=176%:%
%:%263=177%:%
%:%264=178%:%
%:%265=179%:%
%:%266=180%:%
%:%267=181%:%
%:%268=182%:%
%:%269=183%:%
%:%270=184%:%
%:%271=185%:%
%:%272=186%:%
%:%273=187%:%
%:%274=188%:%
%:%275=189%:%
%:%276=190%:%
%:%277=191%:%
%:%278=192%:%
%:%279=193%:%
%:%280=194%:%
%:%281=195%:%
%:%282=196%:%
%:%283=197%:%
%:%284=198%:%
%:%285=199%:%
%:%286=200%:%
%:%287=201%:%
%:%288=202%:%
%:%289=203%:%
%:%290=204%:%
%:%291=205%:%
%:%292=206%:%
%:%293=207%:%
%:%294=208%:%
%:%295=209%:%
%:%296=210%:%
%:%297=211%:%
%:%298=212%:%
%:%299=213%:%
%:%300=214%:%
%:%301=215%:%
%:%302=216%:%
%:%303=217%:%
%:%304=218%:%
%:%305=219%:%
%:%306=220%:%
%:%307=221%:%
%:%308=222%:%
%:%309=223%:%
%:%310=224%:%
%:%311=225%:%
%:%312=226%:%
%:%313=227%:%
%:%314=228%:%
%:%315=229%:%
%:%316=230%:%
%:%317=231%:%
%:%318=232%:%
%:%319=233%:%
%:%320=234%:%
%:%321=235%:%
%:%322=236%:%
%:%323=237%:%
%:%324=238%:%
%:%325=239%:%
%:%326=240%:%
%:%327=241%:%
%:%328=242%:%
%:%329=243%:%
%:%330=244%:%
%:%331=245%:%
%:%332=246%:%
%:%333=247%:%
%:%334=248%:%
%:%335=249%:%
%:%336=250%:%
%:%337=251%:%
%:%338=252%:%
%:%339=253%:%
%:%340=254%:%
%:%341=255%:%
%:%342=256%:%
%:%343=257%:%
%:%344=258%:%
%:%345=259%:%
%:%346=260%:%
%:%347=261%:%
%:%348=262%:%
%:%349=263%:%
%:%350=264%:%
%:%351=265%:%
%:%352=266%:%
%:%353=267%:%
%:%354=268%:%
%:%355=269%:%
%:%356=270%:%
%:%357=271%:%
%:%358=272%:%
%:%359=273%:%
%:%360=274%:%
%:%361=275%:%
%:%362=276%:%
%:%363=277%:%
%:%364=278%:%
%:%365=279%:%
%:%366=280%:%
%:%367=281%:%
%:%368=282%:%
%:%369=283%:%
%:%370=284%:%
%:%371=285%:%
%:%372=286%:%
%:%373=287%:%
%:%374=288%:%
%:%375=289%:%
%:%377=291%:%
%:%378=291%:%
%:%379=292%:%
%:%382=295%:%
%:%383=296%:%
%:%390=297%:%
%:%391=297%:%
%:%392=298%:%
%:%393=298%:%
%:%394=299%:%
%:%395=299%:%
%:%396=299%:%
%:%397=300%:%
%:%398=300%:%
%:%399=301%:%
%:%400=301%:%
%:%401=301%:%
%:%402=302%:%
%:%403=302%:%
%:%404=303%:%
%:%405=303%:%
%:%406=303%:%
%:%407=304%:%
%:%408=304%:%
%:%409=305%:%
%:%410=305%:%
%:%411=305%:%
%:%412=306%:%
%:%413=306%:%
%:%414=307%:%
%:%415=307%:%
%:%416=308%:%
%:%417=308%:%
%:%418=309%:%
%:%419=309%:%
%:%420=310%:%
%:%421=310%:%
%:%422=311%:%
%:%423=311%:%
%:%424=312%:%
%:%425=312%:%
%:%426=312%:%
%:%429=315%:%
%:%430=316%:%
%:%431=316%:%
%:%432=317%:%
%:%433=317%:%
%:%434=318%:%
%:%435=318%:%
%:%436=319%:%
%:%437=319%:%
%:%438=320%:%
%:%439=320%:%
%:%440=321%:%
%:%441=321%:%
%:%442=321%:%
%:%443=322%:%
%:%444=322%:%
%:%445=323%:%
%:%446=323%:%
%:%448=325%:%
%:%449=326%:%
%:%450=326%:%
%:%451=327%:%
%:%452=327%:%
%:%453=328%:%
%:%454=328%:%
%:%455=329%:%
%:%456=329%:%
%:%457=330%:%
%:%458=330%:%
%:%459=330%:%
%:%460=331%:%
%:%461=331%:%
%:%462=332%:%
%:%463=332%:%
%:%464=332%:%
%:%465=333%:%
%:%466=333%:%
%:%467=334%:%
%:%468=334%:%
%:%469=334%:%
%:%470=335%:%
%:%471=335%:%
%:%472=336%:%
%:%473=336%:%
%:%474=336%:%
%:%475=337%:%
%:%476=337%:%
%:%477=338%:%
%:%478=338%:%
%:%479=338%:%
%:%480=339%:%
%:%481=339%:%
%:%482=340%:%
%:%483=340%:%
%:%484=341%:%
%:%485=342%:%
%:%486=342%:%
%:%487=343%:%
%:%488=343%:%
%:%489=344%:%
%:%490=344%:%
%:%491=345%:%
%:%492=345%:%
%:%493=346%:%
%:%494=346%:%
%:%495=346%:%
%:%496=347%:%
%:%497=347%:%
%:%498=348%:%
%:%499=348%:%
%:%500=348%:%
%:%501=349%:%
%:%502=349%:%
%:%503=350%:%
%:%504=350%:%
%:%505=350%:%
%:%506=351%:%
%:%507=351%:%
%:%508=352%:%
%:%509=352%:%
%:%510=352%:%
%:%511=353%:%
%:%512=353%:%
%:%513=354%:%
%:%514=354%:%
%:%515=355%:%
%:%516=355%:%
%:%517=355%:%
%:%518=356%:%
%:%519=356%:%
%:%520=357%:%
%:%521=357%:%
%:%522=358%:%
%:%523=358%:%
%:%524=358%:%
%:%525=359%:%
%:%526=359%:%
%:%527=360%:%
%:%528=360%:%
%:%529=360%:%
%:%530=361%:%
%:%531=361%:%
%:%532=361%:%
%:%533=362%:%
%:%534=362%:%
%:%535=363%:%
%:%536=363%:%
%:%537=364%:%
%:%538=364%:%
%:%539=365%:%
%:%540=365%:%
%:%541=366%:%
%:%542=366%:%
%:%543=367%:%
%:%544=367%:%
%:%545=368%:%
%:%546=368%:%
%:%547=369%:%
%:%548=369%:%
%:%549=370%:%
%:%559=372%:%
%:%560=373%:%
%:%561=374%:%
%:%563=376%:%
%:%564=376%:%
%:%565=377%:%
%:%568=380%:%
%:%569=381%:%
%:%576=382%:%
%:%577=382%:%
%:%578=383%:%
%:%579=383%:%
%:%580=384%:%
%:%581=384%:%
%:%582=384%:%
%:%583=385%:%
%:%584=385%:%
%:%585=386%:%
%:%586=386%:%
%:%587=386%:%
%:%588=387%:%
%:%589=387%:%
%:%590=388%:%
%:%591=388%:%
%:%592=388%:%
%:%593=389%:%
%:%594=389%:%
%:%595=390%:%
%:%596=390%:%
%:%597=390%:%
%:%598=391%:%
%:%599=391%:%
%:%600=392%:%
%:%601=392%:%
%:%602=393%:%
%:%603=393%:%
%:%604=394%:%
%:%605=394%:%
%:%606=395%:%
%:%607=395%:%
%:%608=396%:%
%:%609=396%:%
%:%610=397%:%
%:%611=397%:%
%:%612=397%:%
%:%615=400%:%
%:%616=401%:%
%:%617=401%:%
%:%618=402%:%
%:%619=402%:%
%:%620=403%:%
%:%621=403%:%
%:%622=404%:%
%:%623=404%:%
%:%624=405%:%
%:%625=405%:%
%:%626=406%:%
%:%627=406%:%
%:%628=406%:%
%:%629=407%:%
%:%630=407%:%
%:%631=408%:%
%:%632=408%:%
%:%634=410%:%
%:%635=411%:%
%:%636=411%:%
%:%637=412%:%
%:%638=412%:%
%:%639=413%:%
%:%640=413%:%
%:%641=414%:%
%:%642=414%:%
%:%643=415%:%
%:%644=415%:%
%:%645=415%:%
%:%646=416%:%
%:%647=416%:%
%:%648=417%:%
%:%649=417%:%
%:%650=417%:%
%:%651=418%:%
%:%652=418%:%
%:%653=419%:%
%:%654=419%:%
%:%655=419%:%
%:%656=420%:%
%:%657=420%:%
%:%658=421%:%
%:%659=421%:%
%:%660=421%:%
%:%661=422%:%
%:%662=422%:%
%:%663=423%:%
%:%664=423%:%
%:%665=423%:%
%:%666=424%:%
%:%667=424%:%
%:%668=425%:%
%:%669=425%:%
%:%670=426%:%
%:%671=427%:%
%:%672=427%:%
%:%673=428%:%
%:%674=428%:%
%:%675=429%:%
%:%676=429%:%
%:%677=430%:%
%:%678=430%:%
%:%679=431%:%
%:%680=431%:%
%:%681=431%:%
%:%682=432%:%
%:%683=432%:%
%:%684=433%:%
%:%685=433%:%
%:%686=433%:%
%:%687=434%:%
%:%688=434%:%
%:%689=435%:%
%:%690=435%:%
%:%691=435%:%
%:%692=436%:%
%:%693=436%:%
%:%694=437%:%
%:%695=437%:%
%:%696=437%:%
%:%697=438%:%
%:%698=438%:%
%:%699=439%:%
%:%700=439%:%
%:%701=439%:%
%:%702=440%:%
%:%703=440%:%
%:%704=441%:%
%:%705=441%:%
%:%706=442%:%
%:%707=442%:%
%:%708=442%:%
%:%709=443%:%
%:%710=443%:%
%:%711=444%:%
%:%712=444%:%
%:%713=445%:%
%:%714=445%:%
%:%715=445%:%
%:%716=446%:%
%:%717=446%:%
%:%718=447%:%
%:%719=447%:%
%:%720=447%:%
%:%721=448%:%
%:%722=448%:%
%:%723=448%:%
%:%724=449%:%
%:%725=449%:%
%:%726=450%:%
%:%727=450%:%
%:%728=451%:%
%:%729=451%:%
%:%730=452%:%
%:%731=452%:%
%:%732=453%:%
%:%733=453%:%
%:%734=454%:%
%:%735=454%:%
%:%736=455%:%
%:%737=455%:%
%:%738=456%:%
%:%739=456%:%
%:%740=457%:%
%:%750=459%:%
%:%751=460%:%
%:%753=462%:%
%:%754=462%:%
%:%755=463%:%
%:%756=464%:%
%:%759=467%:%
%:%766=468%:%
%:%767=468%:%
%:%768=469%:%
%:%769=469%:%
%:%770=470%:%
%:%771=470%:%
%:%772=471%:%
%:%773=471%:%
%:%782=480%:%
%:%783=481%:%
%:%784=481%:%
%:%785=481%:%
%:%786=482%:%
%:%787=482%:%
%:%788=482%:%
%:%796=490%:%
%:%797=491%:%
%:%798=491%:%
%:%799=491%:%
%:%800=492%:%
%:%801=492%:%
%:%802=492%:%
%:%803=493%:%
%:%804=493%:%
%:%805=494%:%
%:%806=494%:%
%:%807=495%:%
%:%808=495%:%
%:%809=495%:%
%:%810=496%:%
%:%811=496%:%
%:%812=497%:%
%:%813=497%:%
%:%814=497%:%
%:%815=498%:%
%:%816=498%:%
%:%817=499%:%
%:%818=499%:%
%:%819=499%:%
%:%820=500%:%
%:%821=500%:%
%:%822=501%:%
%:%823=501%:%
%:%824=501%:%
%:%825=502%:%
%:%826=502%:%
%:%827=503%:%
%:%828=503%:%
%:%829=503%:%
%:%830=504%:%
%:%831=504%:%
%:%832=505%:%
%:%833=505%:%
%:%834=505%:%
%:%835=506%:%
%:%836=506%:%
%:%837=507%:%
%:%838=507%:%
%:%839=507%:%
%:%840=508%:%
%:%841=508%:%
%:%842=509%:%
%:%843=509%:%
%:%844=509%:%
%:%845=510%:%
%:%846=510%:%
%:%849=513%:%
%:%850=514%:%
%:%851=514%:%
%:%852=514%:%
%:%853=515%:%
%:%854=515%:%
%:%855=515%:%
%:%856=516%:%
%:%857=516%:%
%:%858=517%:%
%:%859=517%:%
%:%862=520%:%
%:%863=521%:%
%:%864=521%:%
%:%865=521%:%
%:%866=522%:%
%:%867=522%:%
%:%868=522%:%
%:%869=523%:%
%:%870=523%:%
%:%871=524%:%
%:%872=524%:%
%:%875=527%:%
%:%876=528%:%
%:%877=528%:%
%:%878=528%:%
%:%879=529%:%
%:%889=531%:%
%:%890=532%:%
%:%891=533%:%
%:%892=534%:%
%:%893=535%:%
%:%895=537%:%
%:%896=537%:%
%:%897=538%:%
%:%898=539%:%
%:%905=540%:%
%:%906=540%:%
%:%907=541%:%
%:%908=541%:%
%:%909=542%:%
%:%910=542%:%
%:%911=543%:%
%:%912=543%:%
%:%913=544%:%
%:%914=544%:%
%:%915=545%:%
%:%916=545%:%
%:%917=545%:%
%:%918=546%:%
%:%919=546%:%
%:%920=547%:%
%:%921=547%:%
%:%922=548%:%
%:%923=548%:%
%:%924=549%:%
%:%925=549%:%
%:%926=549%:%
%:%927=550%:%
%:%928=550%:%
%:%929=550%:%
%:%930=551%:%
%:%931=551%:%
%:%932=551%:%
%:%933=552%:%
%:%934=552%:%
%:%935=553%:%
%:%936=553%:%
%:%937=553%:%
%:%938=554%:%
%:%939=554%:%
%:%940=555%:%
%:%946=555%:%
%:%949=556%:%
%:%950=557%:%
%:%951=557%:%
%:%952=558%:%
%:%953=559%:%
%:%960=560%:%
%:%961=560%:%
%:%962=561%:%
%:%963=561%:%
%:%964=562%:%
%:%965=562%:%
%:%966=563%:%
%:%967=563%:%
%:%968=564%:%
%:%969=564%:%
%:%970=565%:%
%:%971=565%:%
%:%972=565%:%
%:%973=566%:%
%:%974=566%:%
%:%975=567%:%
%:%976=567%:%
%:%977=568%:%
%:%978=568%:%
%:%979=569%:%
%:%980=569%:%
%:%981=569%:%
%:%982=570%:%
%:%983=570%:%
%:%984=570%:%
%:%985=571%:%
%:%986=571%:%
%:%987=572%:%
%:%988=572%:%
%:%989=572%:%
%:%990=573%:%
%:%991=573%:%
%:%992=574%:%
%:%993=574%:%
%:%994=574%:%
%:%995=575%:%
%:%996=575%:%
%:%997=575%:%
%:%998=576%:%
%:%999=576%:%
%:%1000=576%:%
%:%1001=577%:%
%:%1002=577%:%
%:%1003=577%:%
%:%1004=578%:%
%:%1005=578%:%
%:%1006=579%:%
%:%1007=579%:%
%:%1008=580%:%
%:%1009=580%:%
%:%1010=581%:%
%:%1016=581%:%
%:%1019=582%:%
%:%1020=583%:%
%:%1021=583%:%
%:%1022=584%:%
%:%1023=585%:%
%:%1030=586%:%
%:%1031=586%:%
%:%1032=587%:%
%:%1033=587%:%
%:%1034=588%:%
%:%1035=588%:%
%:%1036=589%:%
%:%1037=589%:%
%:%1038=590%:%
%:%1039=590%:%
%:%1040=591%:%
%:%1041=591%:%
%:%1042=591%:%
%:%1043=592%:%
%:%1044=592%:%
%:%1045=593%:%
%:%1046=593%:%
%:%1047=594%:%
%:%1048=594%:%
%:%1049=595%:%
%:%1050=595%:%
%:%1051=595%:%
%:%1052=596%:%
%:%1053=596%:%
%:%1054=596%:%
%:%1055=597%:%
%:%1056=597%:%
%:%1057=597%:%
%:%1058=598%:%
%:%1059=598%:%
%:%1060=599%:%
%:%1061=599%:%
%:%1062=599%:%
%:%1063=600%:%
%:%1064=600%:%
%:%1065=601%:%
%:%1071=601%:%
%:%1074=602%:%
%:%1075=603%:%
%:%1076=603%:%
%:%1077=604%:%
%:%1078=605%:%
%:%1085=606%:%
%:%1086=606%:%
%:%1087=607%:%
%:%1088=607%:%
%:%1089=608%:%
%:%1090=608%:%
%:%1091=609%:%
%:%1092=609%:%
%:%1093=610%:%
%:%1094=610%:%
%:%1095=611%:%
%:%1096=611%:%
%:%1097=611%:%
%:%1098=612%:%
%:%1099=612%:%
%:%1100=613%:%
%:%1101=613%:%
%:%1102=614%:%
%:%1103=614%:%
%:%1104=615%:%
%:%1105=615%:%
%:%1106=615%:%
%:%1107=616%:%
%:%1108=616%:%
%:%1109=616%:%
%:%1110=617%:%
%:%1111=617%:%
%:%1112=617%:%
%:%1113=618%:%
%:%1114=618%:%
%:%1115=619%:%
%:%1116=619%:%
%:%1117=619%:%
%:%1118=620%:%
%:%1119=620%:%
%:%1120=621%:%
%:%1130=623%:%
%:%1131=624%:%
%:%1132=625%:%
%:%1134=627%:%
%:%1135=627%:%
%:%1136=628%:%
%:%1137=629%:%
%:%1144=630%:%
%:%1145=630%:%
%:%1146=631%:%
%:%1147=631%:%
%:%1148=632%:%
%:%1149=632%:%
%:%1150=633%:%
%:%1151=633%:%
%:%1152=634%:%
%:%1153=634%:%
%:%1154=635%:%
%:%1155=635%:%
%:%1156=635%:%
%:%1157=636%:%
%:%1158=636%:%
%:%1159=637%:%
%:%1160=637%:%
%:%1161=638%:%
%:%1162=638%:%
%:%1163=639%:%
%:%1164=639%:%
%:%1165=639%:%
%:%1166=640%:%
%:%1167=640%:%
%:%1168=640%:%
%:%1169=641%:%
%:%1170=641%:%
%:%1171=641%:%
%:%1172=642%:%
%:%1173=642%:%
%:%1174=643%:%
%:%1175=643%:%
%:%1176=643%:%
%:%1177=644%:%
%:%1178=644%:%
%:%1179=645%:%
%:%1185=645%:%
%:%1188=646%:%
%:%1189=647%:%
%:%1190=647%:%
%:%1191=648%:%
%:%1192=649%:%
%:%1199=650%:%
%:%1200=650%:%
%:%1201=651%:%
%:%1202=651%:%
%:%1203=652%:%
%:%1204=652%:%
%:%1205=653%:%
%:%1206=653%:%
%:%1207=654%:%
%:%1208=654%:%
%:%1209=655%:%
%:%1210=655%:%
%:%1211=655%:%
%:%1212=656%:%
%:%1213=656%:%
%:%1214=657%:%
%:%1215=657%:%
%:%1216=658%:%
%:%1217=658%:%
%:%1218=659%:%
%:%1219=659%:%
%:%1220=659%:%
%:%1221=660%:%
%:%1222=660%:%
%:%1223=660%:%
%:%1224=661%:%
%:%1225=661%:%
%:%1226=661%:%
%:%1227=662%:%
%:%1228=662%:%
%:%1229=663%:%
%:%1230=663%:%
%:%1231=663%:%
%:%1232=664%:%
%:%1233=664%:%
%:%1234=665%:%
%:%1240=665%:%
%:%1243=666%:%
%:%1244=667%:%
%:%1245=667%:%
%:%1246=668%:%
%:%1247=669%:%
%:%1254=670%:%
%:%1255=670%:%
%:%1256=671%:%
%:%1257=671%:%
%:%1258=672%:%
%:%1259=672%:%
%:%1260=673%:%
%:%1261=673%:%
%:%1262=674%:%
%:%1263=674%:%
%:%1264=675%:%
%:%1265=675%:%
%:%1266=675%:%
%:%1267=676%:%
%:%1268=676%:%
%:%1269=677%:%
%:%1270=677%:%
%:%1271=678%:%
%:%1272=678%:%
%:%1273=679%:%
%:%1274=679%:%
%:%1275=679%:%
%:%1276=680%:%
%:%1277=680%:%
%:%1278=680%:%
%:%1279=681%:%
%:%1280=681%:%
%:%1281=681%:%
%:%1282=682%:%
%:%1283=682%:%
%:%1284=683%:%
%:%1285=683%:%
%:%1286=683%:%
%:%1287=684%:%
%:%1288=684%:%
%:%1289=685%:%
%:%1299=687%:%
%:%1301=689%:%
%:%1302=689%:%
%:%1303=690%:%
%:%1306=693%:%
%:%1307=694%:%
%:%1310=695%:%
%:%1314=695%:%
%:%1315=695%:%
%:%1316=696%:%
%:%1317=696%:%
%:%1318=697%:%
%:%1319=697%:%
%:%1320=698%:%
%:%1321=698%:%
%:%1322=699%:%
%:%1323=699%:%
%:%1326=702%:%
%:%1327=703%:%
%:%1328=703%:%
%:%1329=703%:%
%:%1330=704%:%
%:%1331=704%:%
%:%1332=704%:%
%:%1333=705%:%
%:%1334=705%:%
%:%1335=706%:%
%:%1336=706%:%
%:%1337=707%:%
%:%1338=707%:%
%:%1339=707%:%
%:%1340=708%:%
%:%1341=708%:%
%:%1342=709%:%
%:%1343=710%:%
%:%1344=710%:%
%:%1345=710%:%
%:%1346=711%:%
%:%1347=711%:%
%:%1348=712%:%
%:%1349=712%:%
%:%1350=712%:%
%:%1351=713%:%
%:%1352=713%:%
%:%1353=714%:%
%:%1354=714%:%
%:%1355=714%:%
%:%1356=715%:%
%:%1357=715%:%
%:%1358=716%:%
%:%1359=716%:%
%:%1360=716%:%
%:%1361=717%:%
%:%1362=717%:%
%:%1363=718%:%
%:%1364=718%:%
%:%1365=718%:%
%:%1366=719%:%
%:%1367=719%:%
%:%1368=720%:%
%:%1369=720%:%
%:%1370=720%:%
%:%1371=721%:%
%:%1372=721%:%
%:%1373=722%:%
%:%1374=722%:%
%:%1375=722%:%
%:%1376=723%:%
%:%1377=723%:%
%:%1378=724%:%
%:%1379=724%:%
%:%1380=724%:%
%:%1381=725%:%
%:%1382=725%:%
%:%1386=729%:%
%:%1387=730%:%
%:%1388=730%:%
%:%1389=730%:%
%:%1390=731%:%
%:%1391=731%:%
%:%1394=734%:%
%:%1395=735%:%
%:%1396=735%:%
%:%1397=735%:%
%:%1398=736%:%
%:%1399=736%:%
%:%1407=744%:%
%:%1408=745%:%
%:%1409=745%:%
%:%1410=745%:%
%:%1411=746%:%
%:%1412=746%:%
%:%1420=754%:%
%:%1421=755%:%
%:%1422=755%:%
%:%1423=756%:%
%:%1433=758%:%
%:%1434=759%:%
%:%1436=761%:%
%:%1437=761%:%
%:%1440=764%:%
%:%1447=765%:%
%:%1448=765%:%
%:%1449=766%:%
%:%1450=766%:%
%:%1451=767%:%
%:%1452=767%:%
%:%1455=770%:%
%:%1456=771%:%
%:%1457=771%:%
%:%1458=772%:%
%:%1459=772%:%
%:%1460=773%:%
%:%1461=773%:%
%:%1464=776%:%
%:%1465=777%:%
%:%1466=777%:%
%:%1467=778%:%
%:%1468=778%:%
%:%1469=779%:%
%:%1479=781%:%
%:%1480=782%:%
%:%1482=784%:%
%:%1483=784%:%
%:%1487=788%:%
%:%1490=789%:%
%:%1494=789%:%
%:%1495=789%:%
%:%1496=790%:%
%:%1497=790%:%
%:%1498=791%:%
%:%1499=791%:%