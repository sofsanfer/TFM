%
\begin{isabellebody}%
\setisabellecontext{Pcp}%
%
\isadelimtheory
%
\endisadelimtheory
%
\isatagtheory
%
\endisatagtheory
{\isafoldtheory}%
%
\isadelimtheory
%
\endisadelimtheory
%
\begin{isamarkuptext}%
En este capítulo se define la \isa{propiedad\ de\ consistencia\ proposicional} para una 
  colección de conjuntos de fórmulas proposicionales y se caracteriza la propiedad 
  de consistencia proposicional mediante la notación uniforme.

  \begin{definicion}
    Sea \isa{C} una colección de conjuntos de fórmulas proposicionales. Decimos que
    \isa{C} verifica la \isa{propiedad\ de\ consistencia\ proposicional} si, para todo
    conjunto \isa{S} perteneciente a la colección, se verifica:
    \begin{enumerate}
      \item \isa{{\isasymbottom}\ {\isasymnotin}\ S}.
      \item Dada \isa{p} una fórmula atómica cualquiera, no se tiene 
        simultáneamente que\\ \isa{p\ {\isasymin}\ S} y \isa{{\isasymnot}\ p\ {\isasymin}\ S}.
      \item Si \isa{F\ {\isasymand}\ G\ {\isasymin}\ S}, entonces el conjunto \isa{{\isacharbraceleft}F{\isacharcomma}G{\isacharbraceright}\ {\isasymunion}\ S} pertenece a \isa{C}.
      \item Si \isa{F\ {\isasymor}\ G\ {\isasymin}\ S}, entonces o bien el conjunto \isa{{\isacharbraceleft}F{\isacharbraceright}\ {\isasymunion}\ S} pertenece a \isa{C}, o bien el 
        conjunto \isa{{\isacharbraceleft}G{\isacharbraceright}\ {\isasymunion}\ S} pertenece a \isa{C}.
      \item Si \isa{F\ {\isasymrightarrow}\ G\ {\isasymin}\ S}, entonces o bien el conjunto \isa{{\isacharbraceleft}{\isasymnot}\ F{\isacharbraceright}\ {\isasymunion}\ S} pertenece a \isa{C}, o bien el 
        conjunto \isa{{\isacharbraceleft}G{\isacharbraceright}\ {\isasymunion}\ S} pertenece a \isa{C}.
      \item Si \isa{{\isasymnot}{\isacharparenleft}{\isasymnot}\ F{\isacharparenright}\ {\isasymin}\ S}, entonces el conjunto \isa{{\isacharbraceleft}F{\isacharbraceright}\ {\isasymunion}\ S} pertenece a \isa{C}.
      \item Si \isa{{\isasymnot}{\isacharparenleft}F\ {\isasymand}\ G{\isacharparenright}\ {\isasymin}\ S}, entonces o bien el conjunto \isa{{\isacharbraceleft}{\isasymnot}\ F{\isacharbraceright}\ {\isasymunion}\ S} pertenece a \isa{C}, o bien el 
        conjunto \isa{{\isacharbraceleft}{\isasymnot}\ G{\isacharbraceright}\ {\isasymunion}\ S} pertenece a \isa{C}.
      \item Si \isa{{\isasymnot}{\isacharparenleft}F\ {\isasymor}\ G{\isacharparenright}\ {\isasymin}\ S}, entonces el conjunto \isa{{\isacharbraceleft}{\isasymnot}\ F{\isacharcomma}\ {\isasymnot}\ G{\isacharbraceright}\ {\isasymunion}\ S} pertenece a \isa{C}.
      \item Si \isa{{\isasymnot}{\isacharparenleft}F\ {\isasymrightarrow}\ G{\isacharparenright}\ {\isasymin}\ S}, entonces el conjunto \isa{{\isacharbraceleft}F{\isacharcomma}\ {\isasymnot}\ G{\isacharbraceright}\ {\isasymunion}\ S} pertenece a \isa{C}.
    \end{enumerate}
  \end{definicion}

  Veamos, a continuación, su formalización en Isabelle mediante el tipo \isa{definition}.%
\end{isamarkuptext}\isamarkuptrue%
\isacommand{definition}\isamarkupfalse%
\ {\isachardoublequoteopen}pcp\ C\ {\isasymequiv}\ {\isacharparenleft}{\isasymforall}S\ {\isasymin}\ C{\isachardot}\isanewline
\ \ {\isasymbottom}\ {\isasymnotin}\ S\isanewline
{\isasymand}\ {\isacharparenleft}{\isasymforall}k{\isachardot}\ Atom\ k\ {\isasymin}\ S\ {\isasymlongrightarrow}\ \isactrlbold {\isasymnot}\ {\isacharparenleft}Atom\ k{\isacharparenright}\ {\isasymin}\ S\ {\isasymlongrightarrow}\ False{\isacharparenright}\isanewline
{\isasymand}\ {\isacharparenleft}{\isasymforall}F\ G{\isachardot}\ F\ \isactrlbold {\isasymand}\ G\ {\isasymin}\ S\ {\isasymlongrightarrow}\ {\isacharbraceleft}F{\isacharcomma}G{\isacharbraceright}\ {\isasymunion}\ S\ {\isasymin}\ C{\isacharparenright}\isanewline
{\isasymand}\ {\isacharparenleft}{\isasymforall}F\ G{\isachardot}\ F\ \isactrlbold {\isasymor}\ G\ {\isasymin}\ S\ {\isasymlongrightarrow}\ {\isacharbraceleft}F{\isacharbraceright}\ {\isasymunion}\ S\ {\isasymin}\ C\ {\isasymor}\ {\isacharbraceleft}G{\isacharbraceright}\ {\isasymunion}\ S\ {\isasymin}\ C{\isacharparenright}\isanewline
{\isasymand}\ {\isacharparenleft}{\isasymforall}F\ G{\isachardot}\ F\ \isactrlbold {\isasymrightarrow}\ G\ {\isasymin}\ S\ {\isasymlongrightarrow}\ {\isacharbraceleft}\isactrlbold {\isasymnot}F{\isacharbraceright}\ {\isasymunion}\ S\ {\isasymin}\ C\ {\isasymor}\ {\isacharbraceleft}G{\isacharbraceright}\ {\isasymunion}\ S\ {\isasymin}\ C{\isacharparenright}\isanewline
{\isasymand}\ {\isacharparenleft}{\isasymforall}F{\isachardot}\ \isactrlbold {\isasymnot}\ {\isacharparenleft}\isactrlbold {\isasymnot}F{\isacharparenright}\ {\isasymin}\ S\ {\isasymlongrightarrow}\ {\isacharbraceleft}F{\isacharbraceright}\ {\isasymunion}\ S\ {\isasymin}\ C{\isacharparenright}\isanewline
{\isasymand}\ {\isacharparenleft}{\isasymforall}F\ G{\isachardot}\ \isactrlbold {\isasymnot}{\isacharparenleft}F\ \isactrlbold {\isasymand}\ G{\isacharparenright}\ {\isasymin}\ S\ {\isasymlongrightarrow}\ {\isacharbraceleft}\isactrlbold {\isasymnot}\ F{\isacharbraceright}\ {\isasymunion}\ S\ {\isasymin}\ C\ {\isasymor}\ {\isacharbraceleft}\isactrlbold {\isasymnot}\ G{\isacharbraceright}\ {\isasymunion}\ S\ {\isasymin}\ C{\isacharparenright}\isanewline
{\isasymand}\ {\isacharparenleft}{\isasymforall}F\ G{\isachardot}\ \isactrlbold {\isasymnot}{\isacharparenleft}F\ \isactrlbold {\isasymor}\ G{\isacharparenright}\ {\isasymin}\ S\ {\isasymlongrightarrow}\ {\isacharbraceleft}\isactrlbold {\isasymnot}\ F{\isacharcomma}\ \isactrlbold {\isasymnot}\ G{\isacharbraceright}\ {\isasymunion}\ S\ {\isasymin}\ C{\isacharparenright}\isanewline
{\isasymand}\ {\isacharparenleft}{\isasymforall}F\ G{\isachardot}\ \isactrlbold {\isasymnot}{\isacharparenleft}F\ \isactrlbold {\isasymrightarrow}\ G{\isacharparenright}\ {\isasymin}\ S\ {\isasymlongrightarrow}\ {\isacharbraceleft}F{\isacharcomma}\isactrlbold {\isasymnot}\ G{\isacharbraceright}\ {\isasymunion}\ S\ {\isasymin}\ C{\isacharparenright}{\isacharparenright}{\isachardoublequoteclose}%
\begin{isamarkuptext}%
Observando la definición anterior, se prueba fácilmente que la colección trivial
  formada por el conjunto vacío de fórmulas verifica la propiedad de consistencia 
  proposicional.%
\end{isamarkuptext}\isamarkuptrue%
\isacommand{lemma}\isamarkupfalse%
\ {\isachardoublequoteopen}pcp\ {\isacharbraceleft}{\isacharbraceleft}{\isacharbraceright}{\isacharbraceright}{\isachardoublequoteclose}\isanewline
%
\isadelimproof
\ \ %
\endisadelimproof
%
\isatagproof
\isacommand{unfolding}\isamarkupfalse%
\ pcp{\isacharunderscore}def\ \isacommand{by}\isamarkupfalse%
\ simp%
\endisatagproof
{\isafoldproof}%
%
\isadelimproof
%
\endisadelimproof
%
\begin{isamarkuptext}%
Del mismo modo, aplicando la definición, se demuestra que los siguientes ejemplos
  de colecciones de conjuntos de fórmulas proposicionales verifican igualmente la 
  propiedad.%
\end{isamarkuptext}\isamarkuptrue%
\isacommand{lemma}\isamarkupfalse%
\ {\isachardoublequoteopen}pcp\ {\isacharbraceleft}{\isacharbraceleft}Atom\ {\isadigit{0}}{\isacharbraceright}{\isacharbraceright}{\isachardoublequoteclose}\isanewline
%
\isadelimproof
\ \ %
\endisadelimproof
%
\isatagproof
\isacommand{unfolding}\isamarkupfalse%
\ pcp{\isacharunderscore}def\ \isacommand{by}\isamarkupfalse%
\ simp%
\endisatagproof
{\isafoldproof}%
%
\isadelimproof
\isanewline
%
\endisadelimproof
\isanewline
\isacommand{lemma}\isamarkupfalse%
\ {\isachardoublequoteopen}pcp\ {\isacharbraceleft}{\isacharbraceleft}{\isacharparenleft}\isactrlbold {\isasymnot}\ {\isacharparenleft}Atom\ {\isadigit{1}}{\isacharparenright}{\isacharparenright}\ \isactrlbold {\isasymrightarrow}\ Atom\ {\isadigit{2}}{\isacharbraceright}{\isacharcomma}\isanewline
\ \ \ {\isacharbraceleft}{\isacharparenleft}{\isacharparenleft}\isactrlbold {\isasymnot}\ {\isacharparenleft}Atom\ {\isadigit{1}}{\isacharparenright}{\isacharparenright}\ \isactrlbold {\isasymrightarrow}\ Atom\ {\isadigit{2}}{\isacharparenright}{\isacharcomma}\ \isactrlbold {\isasymnot}{\isacharparenleft}\isactrlbold {\isasymnot}\ {\isacharparenleft}Atom\ {\isadigit{1}}{\isacharparenright}{\isacharparenright}{\isacharbraceright}{\isacharcomma}\isanewline
\ \ {\isacharbraceleft}{\isacharparenleft}{\isacharparenleft}\isactrlbold {\isasymnot}\ {\isacharparenleft}Atom\ {\isadigit{1}}{\isacharparenright}{\isacharparenright}\ \isactrlbold {\isasymrightarrow}\ Atom\ {\isadigit{2}}{\isacharparenright}{\isacharcomma}\ \isactrlbold {\isasymnot}{\isacharparenleft}\isactrlbold {\isasymnot}\ {\isacharparenleft}Atom\ {\isadigit{1}}{\isacharparenright}{\isacharparenright}{\isacharcomma}\ \ Atom\ {\isadigit{1}}{\isacharbraceright}{\isacharbraceright}{\isachardoublequoteclose}\ \isanewline
%
\isadelimproof
\ \ %
\endisadelimproof
%
\isatagproof
\isacommand{unfolding}\isamarkupfalse%
\ pcp{\isacharunderscore}def\ \isacommand{by}\isamarkupfalse%
\ auto%
\endisatagproof
{\isafoldproof}%
%
\isadelimproof
%
\endisadelimproof
%
\begin{isamarkuptext}%
Por último, en contraposición podemos ilustrar un caso de colección que no verifique la 
  propiedad con la siguiente colección obtenida al modificar el último ejemplo. De
  esta manera, aunque la colección verifique correctamente la quinta condición de la
  definición, no cumplirá la sexta.%
\end{isamarkuptext}\isamarkuptrue%
\isacommand{lemma}\isamarkupfalse%
\ {\isachardoublequoteopen}{\isasymnot}\ pcp\ {\isacharbraceleft}{\isacharbraceleft}{\isacharparenleft}\isactrlbold {\isasymnot}\ {\isacharparenleft}Atom\ {\isadigit{1}}{\isacharparenright}{\isacharparenright}\ \isactrlbold {\isasymrightarrow}\ Atom\ {\isadigit{2}}{\isacharbraceright}{\isacharcomma}\isanewline
\ \ \ {\isacharbraceleft}{\isacharparenleft}{\isacharparenleft}\isactrlbold {\isasymnot}\ {\isacharparenleft}Atom\ {\isadigit{1}}{\isacharparenright}{\isacharparenright}\ \isactrlbold {\isasymrightarrow}\ Atom\ {\isadigit{2}}{\isacharparenright}{\isacharcomma}\ \isactrlbold {\isasymnot}{\isacharparenleft}\isactrlbold {\isasymnot}\ {\isacharparenleft}Atom\ {\isadigit{1}}{\isacharparenright}{\isacharparenright}{\isacharbraceright}{\isacharbraceright}{\isachardoublequoteclose}\ \isanewline
%
\isadelimproof
\ \ %
\endisadelimproof
%
\isatagproof
\isacommand{unfolding}\isamarkupfalse%
\ pcp{\isacharunderscore}def\ \isacommand{by}\isamarkupfalse%
\ auto%
\endisatagproof
{\isafoldproof}%
%
\isadelimproof
%
\endisadelimproof
%
\begin{isamarkuptext}%
Por otra parte, veamos un resultado que permite la caracterización de la 
  propiedad de consistencia proposicional mediante la notación uniforme.

  \begin{lema}[Caracterización de \isa{P{\isachardot}C{\isachardot}P} mediante la notación uniforme]
    Dada una colección \isa{C} de conjuntos de fórmulas proposicionales, son equivalentes:
    \begin{enumerate}
      \item \isa{C} verifica la propiedad de consistencia proposicional.
      \item Para cualquier conjunto de fórmulas \isa{S} de la colección, se verifican las 
      condiciones:
      \begin{itemize}
        \item \isa{{\isasymbottom}} no pertenece a \isa{S}.
        \item Dada \isa{p} una fórmula atómica cualquiera, no se tiene 
        simultáneamente que\\ \isa{p\ {\isasymin}\ S} y \isa{{\isasymnot}\ p\ {\isasymin}\ S}.
        \item Para toda fórmula de tipo \isa{{\isasymalpha}} con componentes \isa{{\isasymalpha}\isactrlsub {\isadigit{1}}} y \isa{{\isasymalpha}\isactrlsub {\isadigit{2}}} tal que \isa{{\isasymalpha}}
        pertenece a \isa{S}, se tiene que \isa{{\isacharbraceleft}{\isasymalpha}\isactrlsub {\isadigit{1}}{\isacharcomma}{\isasymalpha}\isactrlsub {\isadigit{2}}{\isacharbraceright}\ {\isasymunion}\ S} pertenece a \isa{C}.
        \item Para toda fórmula de tipo \isa{{\isasymbeta}} con componentes \isa{{\isasymbeta}\isactrlsub {\isadigit{1}}} y \isa{{\isasymbeta}\isactrlsub {\isadigit{2}}} tal que \isa{{\isasymbeta}}
        pertenece a \isa{S}, se tiene que o bien \isa{{\isacharbraceleft}{\isasymbeta}\isactrlsub {\isadigit{1}}{\isacharbraceright}\ {\isasymunion}\ S} pertenece a \isa{C} o 
        bien \isa{{\isacharbraceleft}{\isasymbeta}\isactrlsub {\isadigit{2}}{\isacharbraceright}\ {\isasymunion}\ S} pertenece a \isa{C}.
      \end{itemize} 
    \end{enumerate}
  \end{lema}

  En Isabelle/HOL se formaliza el resultado como sigue.%
\end{isamarkuptext}\isamarkuptrue%
\isacommand{lemma}\isamarkupfalse%
\ {\isachardoublequoteopen}pcp\ C\ {\isacharequal}\ {\isacharparenleft}{\isasymforall}S\ {\isasymin}\ C{\isachardot}\ {\isasymbottom}\ {\isasymnotin}\ S\isanewline
{\isasymand}\ {\isacharparenleft}{\isasymforall}k{\isachardot}\ Atom\ k\ {\isasymin}\ S\ {\isasymlongrightarrow}\ \isactrlbold {\isasymnot}\ {\isacharparenleft}Atom\ k{\isacharparenright}\ {\isasymin}\ S\ {\isasymlongrightarrow}\ False{\isacharparenright}\isanewline
{\isasymand}\ {\isacharparenleft}{\isasymforall}F\ G\ H{\isachardot}\ Con\ F\ G\ H\ {\isasymlongrightarrow}\ F\ {\isasymin}\ S\ {\isasymlongrightarrow}\ {\isacharbraceleft}G{\isacharcomma}H{\isacharbraceright}\ {\isasymunion}\ S\ {\isasymin}\ C{\isacharparenright}\isanewline
{\isasymand}\ {\isacharparenleft}{\isasymforall}F\ G\ H{\isachardot}\ Dis\ F\ G\ H\ {\isasymlongrightarrow}\ F\ {\isasymin}\ S\ {\isasymlongrightarrow}\ {\isacharbraceleft}G{\isacharbraceright}\ {\isasymunion}\ S\ {\isasymin}\ C\ {\isasymor}\ {\isacharbraceleft}H{\isacharbraceright}\ {\isasymunion}\ S\ {\isasymin}\ C{\isacharparenright}{\isacharparenright}{\isachardoublequoteclose}\isanewline
%
\isadelimproof
\ \ %
\endisadelimproof
%
\isatagproof
\isacommand{oops}\isamarkupfalse%
%
\endisatagproof
{\isafoldproof}%
%
\isadelimproof
%
\endisadelimproof
%
\begin{isamarkuptext}%
En primer lugar, veamos la demostración del lema.

\begin{demostracion}
  Para probar la equivalencia, veamos cada una de las implicaciones por separado.

\textbf{\isa{{\isadigit{1}}{\isacharparenright}\ {\isasymLongrightarrow}\ {\isadigit{2}}{\isacharparenright}}}
  
  Supongamos que \isa{C} es una colección de conjuntos de fórmulas proposicionales que
  verifica la propiedad de consistencia proposicional. Vamos a probar que, en efecto,
  cumple las condiciones de \isa{{\isadigit{2}}{\isacharparenright}}. 

  Consideremos un conjunto de fórmulas \isa{S} perteneciente a la colección \isa{C}.
  Por hipótesis, de la definición de propiedad de consistencia proposicional obtenemos
  que \isa{S} verifica las siguientes condiciones:
 \begin{enumerate}
      \item \isa{{\isasymbottom}\ {\isasymnotin}\ S}.
      \item Dada \isa{p} una fórmula atómica cualquiera, no se tiene 
        simultáneamente que\\ \isa{p\ {\isasymin}\ S} y \isa{{\isasymnot}\ p\ {\isasymin}\ S}.
      \item Si \isa{G\ {\isasymand}\ H\ {\isasymin}\ S}, entonces el conjunto \isa{{\isacharbraceleft}G{\isacharcomma}H{\isacharbraceright}\ {\isasymunion}\ S} pertenece a \isa{C}.
      \item Si \isa{G\ {\isasymor}\ H\ {\isasymin}\ S}, entonces o bien el conjunto \isa{{\isacharbraceleft}G{\isacharbraceright}\ {\isasymunion}\ S} pertenece a \isa{C}, o bien el 
        conjunto \isa{{\isacharbraceleft}H{\isacharbraceright}\ {\isasymunion}\ S} pertenece a \isa{C}.
      \item Si \isa{G\ {\isasymrightarrow}\ H\ {\isasymin}\ S}, entonces o bien el conjunto \isa{{\isacharbraceleft}{\isasymnot}\ G{\isacharbraceright}\ {\isasymunion}\ S} pertenece a \isa{C}, o bien el 
        conjunto \isa{{\isacharbraceleft}H{\isacharbraceright}\ {\isasymunion}\ S} pertenece a \isa{C}.
      \item Si \isa{{\isasymnot}{\isacharparenleft}{\isasymnot}\ G{\isacharparenright}\ {\isasymin}\ S}, entonces el conjunto \isa{{\isacharbraceleft}G{\isacharbraceright}\ {\isasymunion}\ S} pertenece a \isa{C}.
      \item Si \isa{{\isasymnot}{\isacharparenleft}G\ {\isasymand}\ H{\isacharparenright}\ {\isasymin}\ S}, entonces o bien el conjunto \isa{{\isacharbraceleft}{\isasymnot}\ G{\isacharbraceright}\ {\isasymunion}\ S} pertenece a \isa{C}, o bien el 
        conjunto \isa{{\isacharbraceleft}{\isasymnot}\ H{\isacharbraceright}\ {\isasymunion}\ S} pertenece a \isa{C}.
      \item Si \isa{{\isasymnot}{\isacharparenleft}G\ {\isasymor}\ H{\isacharparenright}\ {\isasymin}\ S}, entonces el conjunto \isa{{\isacharbraceleft}{\isasymnot}\ G{\isacharcomma}\ {\isasymnot}\ H{\isacharbraceright}\ {\isasymunion}\ S} pertenece a \isa{C}.
      \item Si \isa{{\isasymnot}{\isacharparenleft}G\ {\isasymrightarrow}\ H{\isacharparenright}\ {\isasymin}\ S}, entonces el conjunto \isa{{\isacharbraceleft}G{\isacharcomma}\ {\isasymnot}\ H{\isacharbraceright}\ {\isasymunion}\ S} pertenece a \isa{C}.
 \end{enumerate}

  Las dos primeras condiciones se corresponden con los dos primeros resultados que queríamos
  demostrar. De este modo, falta probar:
  \begin{itemize}
     \item Para toda fórmula de tipo \isa{{\isasymalpha}} con componentes \isa{{\isasymalpha}\isactrlsub {\isadigit{1}}} y \isa{{\isasymalpha}\isactrlsub {\isadigit{2}}} tal que \isa{{\isasymalpha}}
     pertenece a \isa{S}, se tiene que \isa{{\isacharbraceleft}{\isasymalpha}\isactrlsub {\isadigit{1}}{\isacharcomma}{\isasymalpha}\isactrlsub {\isadigit{2}}{\isacharbraceright}\ {\isasymunion}\ S} pertenece a \isa{C}.
     \item Para toda fórmula de tipo \isa{{\isasymbeta}} con componentes \isa{{\isasymbeta}\isactrlsub {\isadigit{1}}} y \isa{{\isasymbeta}\isactrlsub {\isadigit{2}}} tal que \isa{{\isasymbeta}}
     pertenece a \isa{S}, se tiene que o bien \isa{{\isacharbraceleft}{\isasymbeta}\isactrlsub {\isadigit{1}}{\isacharbraceright}\ {\isasymunion}\ S} pertenece a \isa{C} o 
     bien \isa{{\isacharbraceleft}{\isasymbeta}\isactrlsub {\isadigit{2}}{\isacharbraceright}\ {\isasymunion}\ S} pertenece a \isa{C}.   
  \end{itemize} 

  En primer lugar, vamos a deducir el primer resultado correspondiente a las fórmulas
  de tipo \isa{{\isasymalpha}} de las condiciones tercera, sexta, octava y novena de la definición de 
  propiedad de consistencia proposicional. En efecto, consideremos una fórmula de tipo 
  \isa{{\isasymalpha}} cualquiera con componentes \isa{{\isasymalpha}\isactrlsub {\isadigit{1}}} y \isa{{\isasymalpha}\isactrlsub {\isadigit{2}}} tal que \isa{{\isasymalpha}} pertenece a \isa{S}. Sabemos que 
  la fórmula es de la forma \isa{G\ {\isasymand}\ H}, \isa{{\isasymnot}\ {\isacharparenleft}{\isasymnot}\ G{\isacharparenright}}, \isa{{\isasymnot}\ {\isacharparenleft}G\ {\isasymor}\ H{\isacharparenright}} o 
  \isa{{\isasymnot}{\isacharparenleft}G\ {\isasymlongrightarrow}\ H{\isacharparenright}} para ciertas fórmulas \isa{G} y \isa{H}. Vamos a probar que para cada caso se cumple que 
  \isa{{\isacharbraceleft}{\isasymalpha}\isactrlsub {\isadigit{1}}{\isacharcomma}\ {\isasymalpha}\isactrlsub {\isadigit{2}}{\isacharbraceright}\ {\isasymunion}\ S} pertenece a la colección:

  \isa{{\isasymsqdot}\ Fórmula\ de\ tipo\ G\ {\isasymand}\ H}: En este caso, sus componentes conjuntivas \isa{{\isasymalpha}\isactrlsub {\isadigit{1}}} y \isa{{\isasymalpha}\isactrlsub {\isadigit{2}}} son \isa{G} 
    y \isa{H} respectivamente. Luego tenemos que \isa{{\isacharbraceleft}{\isasymalpha}\isactrlsub {\isadigit{1}}{\isacharcomma}\ {\isasymalpha}\isactrlsub {\isadigit{2}}{\isacharbraceright}\ {\isasymunion}\ S}  pertenece a \isa{C} por
    la tercera condición de la definición de propiedad de consistencia
    proposicional.

  \isa{{\isasymsqdot}\ Fórmula\ de\ tipo\ {\isasymnot}\ {\isacharparenleft}{\isasymnot}\ G{\isacharparenright}}: En este caso, sus componentes conjuntivas \isa{{\isasymalpha}\isactrlsub {\isadigit{1}}} y \isa{{\isasymalpha}\isactrlsub {\isadigit{2}}} son 
    ambas \isa{G}. Como el conjunto \isa{{\isacharbraceleft}{\isasymalpha}\isactrlsub {\isadigit{1}}{\isacharbraceright}\ {\isasymunion}\ S} es equivalente a \isa{{\isacharbraceleft}{\isasymalpha}\isactrlsub {\isadigit{1}}{\isacharcomma}\ {\isasymalpha}\isactrlsub {\isadigit{2}}{\isacharbraceright}\ {\isasymunion}\ S} ya
    que \isa{{\isasymalpha}\isactrlsub {\isadigit{1}}} y \isa{{\isasymalpha}\isactrlsub {\isadigit{2}}} son iguales, tenemos que este último pertenece a \isa{C} por la sexta 
    condición de la definición de propiedad de consistencia proposicional.

  \isa{{\isasymsqdot}\ Fórmula\ de\ tipo\ {\isasymnot}{\isacharparenleft}G\ {\isasymor}\ H{\isacharparenright}}: En este caso, sus componentes conjuntivas \isa{{\isasymalpha}\isactrlsub {\isadigit{1}}} y \isa{{\isasymalpha}\isactrlsub {\isadigit{2}}} son\\ 
    \isa{{\isasymnot}\ G} y \isa{{\isasymnot}\ H} respectivamente. Luego tenemos que \isa{{\isacharbraceleft}{\isasymalpha}\isactrlsub {\isadigit{1}}{\isacharcomma}\ {\isasymalpha}\isactrlsub {\isadigit{2}}{\isacharbraceright}\ {\isasymunion}\ S}  pertenece a \isa{C} por
    la octava condición de la definición de propiedad de consistencia proposicional.

  \isa{{\isasymsqdot}\ Fórmula\ de\ tipo\ {\isasymnot}{\isacharparenleft}G\ {\isasymlongrightarrow}\ H{\isacharparenright}}: En este caso, sus componentes conjuntivas \isa{{\isasymalpha}\isactrlsub {\isadigit{1}}} y \isa{{\isasymalpha}\isactrlsub {\isadigit{2}}} son \isa{G} y 
    \isa{{\isasymnot}\ H} respectivamente. Luego tenemos que \isa{{\isacharbraceleft}{\isasymalpha}\isactrlsub {\isadigit{1}}{\isacharcomma}\ {\isasymalpha}\isactrlsub {\isadigit{2}}{\isacharbraceright}\ {\isasymunion}\ S}  pertenece a \isa{C} por la novena 
    condición de la definición de propiedad de consistencia proposicional.

  Finalmente, el resultado correspondiente a las fórmulas de tipo \isa{{\isasymbeta}} se obtiene de las 
  condiciones cuarta, quinta, sexta y séptima de la definición de propiedad de consistencia 
  proposicional. Para probarlo, consideremos una fórmula cualquiera de tipo \isa{{\isasymbeta}} perteneciente
  al conjunto \isa{S} y cuyas componentes disyuntivas son \isa{{\isasymbeta}\isactrlsub {\isadigit{1}}} y \isa{{\isasymbeta}\isactrlsub {\isadigit{2}}}. Por simplificación, sabemos 
  que dicha fórmula es de la forma \isa{G\ {\isasymor}\ H}, \isa{G\ {\isasymlongrightarrow}\ H}, \isa{{\isasymnot}\ {\isacharparenleft}{\isasymnot}\ G{\isacharparenright}} o \isa{{\isasymnot}{\isacharparenleft}G\ {\isasymand}\ H{\isacharparenright}} para ciertas 
  fórmulas \isa{G} y \isa{H}. Deduzcamos que, en efecto, tenemos que o bien \isa{{\isacharbraceleft}{\isasymbeta}\isactrlsub {\isadigit{1}}{\isacharbraceright}\ {\isasymunion}\ S} está en \isa{C} o bien 
  \isa{{\isacharbraceleft}{\isasymbeta}\isactrlsub {\isadigit{2}}{\isacharbraceright}\ {\isasymunion}\ S} está en \isa{C}.

  \isa{{\isasymsqdot}\ Fórmula\ de\ tipo\ G\ {\isasymor}\ H}: En este caso, sus componentes disyuntivas \isa{{\isasymbeta}\isactrlsub {\isadigit{1}}} y \isa{{\isasymbeta}\isactrlsub {\isadigit{2}}} son \isa{G} y 
    \isa{H} respectivamente. Luego tenemos que o bien \isa{{\isacharbraceleft}{\isasymbeta}\isactrlsub {\isadigit{1}}{\isacharbraceright}\ {\isasymunion}\ S}  pertenece a \isa{C} o bien\\
    \isa{{\isacharbraceleft}{\isasymbeta}\isactrlsub {\isadigit{2}}{\isacharbraceright}\ {\isasymunion}\ S} pertenece a \isa{C} por la cuarta condición de la definición de propiedad de 
    consistencia proposicional.

  \isa{{\isasymsqdot}\ Fórmula\ de\ tipo\ G\ {\isasymlongrightarrow}\ H}: En este caso, sus componentes disyuntivas \isa{{\isasymbeta}\isactrlsub {\isadigit{1}}} y \isa{{\isasymbeta}\isactrlsub {\isadigit{2}}} son\\ 
    \isa{{\isasymnot}\ G} y \isa{H} respectivamente. Luego tenemos que o bien \isa{{\isacharbraceleft}{\isasymbeta}\isactrlsub {\isadigit{1}}{\isacharbraceright}\ {\isasymunion}\ S}  pertenece a \isa{C} o 
    bien\\ \isa{{\isacharbraceleft}{\isasymbeta}\isactrlsub {\isadigit{2}}{\isacharbraceright}\ {\isasymunion}\ S} pertenece a \isa{C} por la quinta condición de la definición de propiedad 
    de consistencia proposicional.

  \isa{{\isasymsqdot}\ Fórmula\ de\ tipo\ {\isasymnot}{\isacharparenleft}{\isasymnot}\ G{\isacharparenright}}: En este caso, sus componentes disyuntivas \isa{{\isasymbeta}\isactrlsub {\isadigit{1}}} y \isa{{\isasymbeta}\isactrlsub {\isadigit{2}}} son ambas 
    \isa{G}. Luego tenemos que, en particular, el conjunto \isa{{\isacharbraceleft}{\isasymbeta}\isactrlsub {\isadigit{1}}{\isacharbraceright}\ {\isasymunion}\ S} pertenece a \isa{C} por la 
    sexta condición de la definición de propiedad de consistencia proposicional. Por tanto, se 
    verifica que o bien \isa{{\isacharbraceleft}{\isasymbeta}\isactrlsub {\isadigit{1}}{\isacharbraceright}\ {\isasymunion}\ S} está en \isa{C} o bien \isa{{\isacharbraceleft}{\isasymbeta}\isactrlsub {\isadigit{2}}{\isacharbraceright}\ {\isasymunion}\ S} está en \isa{C}.

  \isa{{\isasymsqdot}\ Fórmula\ de\ tipo\ {\isasymnot}{\isacharparenleft}G\ {\isasymand}\ H{\isacharparenright}}: En este caso, sus componentes disyuntivas \isa{{\isasymbeta}\isactrlsub {\isadigit{1}}} y \isa{{\isasymbeta}\isactrlsub {\isadigit{2}}} son \\ 
    \isa{{\isasymnot}\ G} y \isa{{\isasymnot}\ H} respectivamente. Luego tenemos que o bien \isa{{\isacharbraceleft}{\isasymbeta}\isactrlsub {\isadigit{1}}{\isacharbraceright}\ {\isasymunion}\ S} pertenece a \isa{C} o 
    bien \isa{{\isacharbraceleft}{\isasymbeta}\isactrlsub {\isadigit{2}}{\isacharbraceright}\ {\isasymunion}\ S} pertenece a \isa{C} por la séptima condición de la definición de propiedad 
    de consistencia proposicional.

  De este modo, queda probada la primera implicación de la equivalencia. Veamos la prueba de la 
  implicación contraria.

\textbf{\isa{{\isadigit{2}}{\isacharparenright}\ {\isasymLongrightarrow}\ {\isadigit{1}}{\isacharparenright}}}

  Supongamos que, dada una colección de conjuntos de fórmulas proposicionales \isa{C}, para cualquier
  conjunto \isa{S} de la colección se verifica:
  \begin{itemize}
    \item \isa{{\isasymbottom}} no pertenece a \isa{S}.
    \item Dada \isa{p} una fórmula atómica cualquiera, no se tiene 
    simultáneamente que\\ \isa{p\ {\isasymin}\ S} y \isa{{\isasymnot}\ p\ {\isasymin}\ S}.
    \item Para toda fórmula de tipo \isa{{\isasymalpha}} con componentes \isa{{\isasymalpha}\isactrlsub {\isadigit{1}}} y \isa{{\isasymalpha}\isactrlsub {\isadigit{2}}} tal que \isa{{\isasymalpha}}
    pertenece a \isa{S}, se tiene que \isa{{\isacharbraceleft}{\isasymalpha}\isactrlsub {\isadigit{1}}{\isacharcomma}{\isasymalpha}\isactrlsub {\isadigit{2}}{\isacharbraceright}\ {\isasymunion}\ S} pertenece a \isa{C}.
    \item Para toda fórmula de tipo \isa{{\isasymbeta}} con componentes \isa{{\isasymbeta}\isactrlsub {\isadigit{1}}} y \isa{{\isasymbeta}\isactrlsub {\isadigit{2}}} tal que \isa{{\isasymbeta}}
    pertenece a \isa{S}, se tiene que o bien \isa{{\isacharbraceleft}{\isasymbeta}\isactrlsub {\isadigit{1}}{\isacharbraceright}\ {\isasymunion}\ S} pertenece a \isa{C} o 
    bien \isa{{\isacharbraceleft}{\isasymbeta}\isactrlsub {\isadigit{2}}{\isacharbraceright}\ {\isasymunion}\ S} pertenece a \isa{C}.
  \end{itemize}

  Probemos que \isa{C} verifica la propiedad de consistencia proposicional. Por la definición
  de la propiedad basta probar que, dado un conjunto cualquiera \isa{S} perteneciente a \isa{C}, se
  verifican las siguientes condiciones:
  \begin{enumerate}
    \item \isa{{\isasymbottom}\ {\isasymnotin}\ S}.
    \item Dada \isa{p} una fórmula atómica cualquiera, no se tiene 
      simultáneamente que\\ \isa{p\ {\isasymin}\ S} y \isa{{\isasymnot}\ p\ {\isasymin}\ S}.
    \item Si \isa{G\ {\isasymand}\ H\ {\isasymin}\ S}, entonces el conjunto \isa{{\isacharbraceleft}G{\isacharcomma}H{\isacharbraceright}\ {\isasymunion}\ S} pertenece a \isa{C}.
    \item Si \isa{G\ {\isasymor}\ H\ {\isasymin}\ S}, entonces o bien el conjunto \isa{{\isacharbraceleft}G{\isacharbraceright}\ {\isasymunion}\ S} pertenece a \isa{C}, o bien el conjunto 
      \isa{{\isacharbraceleft}H{\isacharbraceright}\ {\isasymunion}\ S} pertenece a \isa{C}.
    \item Si \isa{G\ {\isasymrightarrow}\ H\ {\isasymin}\ S}, entonces o bien el conjunto \isa{{\isacharbraceleft}{\isasymnot}\ G{\isacharbraceright}\ {\isasymunion}\ S} pertenece a \isa{C}, o bien el 
      conjunto \isa{{\isacharbraceleft}H{\isacharbraceright}\ {\isasymunion}\ S} pertenece a \isa{C}.
    \item Si \isa{{\isasymnot}{\isacharparenleft}{\isasymnot}\ G{\isacharparenright}\ {\isasymin}\ S}, entonces el conjunto \isa{{\isacharbraceleft}G{\isacharbraceright}\ {\isasymunion}\ S} pertenece a \isa{C}.
    \item Si \isa{{\isasymnot}{\isacharparenleft}G\ {\isasymand}\ H{\isacharparenright}\ {\isasymin}\ S}, entonces o bien el conjunto \isa{{\isacharbraceleft}{\isasymnot}\ G{\isacharbraceright}\ {\isasymunion}\ S} pertenece a \isa{C}, o bien el 
      conjunto \isa{{\isacharbraceleft}{\isasymnot}\ H{\isacharbraceright}\ {\isasymunion}\ S} pertenece a \isa{C}.
    \item Si \isa{{\isasymnot}{\isacharparenleft}G\ {\isasymor}\ H{\isacharparenright}\ {\isasymin}\ S}, entonces el conjunto \isa{{\isacharbraceleft}{\isasymnot}\ G{\isacharcomma}\ {\isasymnot}\ H{\isacharbraceright}\ {\isasymunion}\ S} pertenece a \isa{C}.
    \item Si \isa{{\isasymnot}{\isacharparenleft}G\ {\isasymrightarrow}\ H{\isacharparenright}\ {\isasymin}\ S}, entonces el conjunto \isa{{\isacharbraceleft}G{\isacharcomma}\ {\isasymnot}\ H{\isacharbraceright}\ {\isasymunion}\ S} pertenece a \isa{C}.
  \end{enumerate}

  En primer lugar, se observa que por hipótesis se cumplen las dos primeras condiciones de
  la definición.

  Por otra parte, vamos a deducir las condiciones tercera, sexta, octava y novena de la
  definición de la propiedad de consistencia proposicional a partir de la hipótesis sobre las 
  fórmulas de tipo \isa{{\isasymalpha}}.
  \begin{enumerate}
    \item[\isa{{\isadigit{3}}{\isacharparenright}}:] Supongamos que la fórmula \isa{G\ {\isasymand}\ H} pertenece a \isa{S} para fórmulas \isa{G} y \isa{H}
    cualesquiera. Observemos que se trata de una fórmula de tipo \isa{{\isasymalpha}} de componentes conjuntivas
    \isa{G} y \isa{H}. Luego, por hipótesis, tenemos que \isa{{\isacharbraceleft}G{\isacharcomma}\ H{\isacharbraceright}\ {\isasymunion}\ S} pertenece a \isa{C}.
    \item[\isa{{\isadigit{6}}{\isacharparenright}}:] Supongamos que la fórmula \isa{{\isasymnot}{\isacharparenleft}{\isasymnot}\ G{\isacharparenright}} pertenece a \isa{S} para la fórmula \isa{G} 
    cualquiera. Observemos que se trata de una fórmula de tipo \isa{{\isasymalpha}} cuyas componentes conjuntivas 
    son ambas la fórmula \isa{G}. Por hipótesis, tenemos que el conjunto \isa{{\isacharbraceleft}G{\isacharcomma}G{\isacharbraceright}\ {\isasymunion}\ S} pertence a \isa{C}
    y, puesto que dicho conjunto es equivalente a \isa{{\isacharbraceleft}G{\isacharbraceright}\ {\isasymunion}\ S}, tenemos el resultado.
    \item[\isa{{\isadigit{8}}{\isacharparenright}}:] Supongamos que la fórmula \isa{{\isasymnot}{\isacharparenleft}G\ {\isasymor}\ H{\isacharparenright}} pertenece a \isa{S} para fórmulas \isa{G} y \isa{H}
    cualesquiera. Observemos que se trata de una fórmula de tipo \isa{{\isasymalpha}} de componentes conjuntivas
    \isa{{\isasymnot}\ G} y \isa{{\isasymnot}\ H}. Luego, por hipótesis, tenemos que \isa{{\isacharbraceleft}{\isasymnot}\ G{\isacharcomma}\ {\isasymnot}\ H{\isacharbraceright}\ {\isasymunion}\ S} pertenece a \isa{C}.
    \item[\isa{{\isadigit{9}}{\isacharparenright}}:] Supongamos que la fórmula \isa{{\isasymnot}{\isacharparenleft}G\ {\isasymlongrightarrow}\ H{\isacharparenright}} pertenece a \isa{S} para fórmulas \isa{G} y \isa{H}
    cualesquiera. Observemos que se trata de una fórmula de tipo \isa{{\isasymalpha}} de componentes conjuntivas
    \isa{G} y \isa{{\isasymnot}\ H}. Luego, por hipótesis, tenemos que \isa{{\isacharbraceleft}G{\isacharcomma}\ {\isasymnot}\ H{\isacharbraceright}\ {\isasymunion}\ S} pertenece a \isa{C}.
  \end{enumerate} 

  Finalmente, deduzcamos el resto de condiciones de la definición de propiedad de consistencia
  proposicional a partir de la hipótesis referente a las fórmulas de tipo \isa{{\isasymbeta}}.
  \begin{enumerate}
    \item[\isa{{\isadigit{4}}{\isacharparenright}}:] Supongamos que la fórmula \isa{G\ {\isasymor}\ H} pertenece a \isa{S} para fórmulas \isa{G} y \isa{H}
    cualesquiera. Observemos que se trata de una fórmula de tipo \isa{{\isasymbeta}} de componentes disyuntivas
    \isa{G} y \isa{H}. Luego, por hipótesis, tenemos que o bien \isa{{\isacharbraceleft}G{\isacharbraceright}\ {\isasymunion}\ S} pertenece a \isa{C} o bien\\
    \isa{{\isacharbraceleft}H{\isacharbraceright}\ {\isasymunion}\ S} pertenece a \isa{C}.
    \item[\isa{{\isadigit{5}}{\isacharparenright}}:] Supongamos que la fórmula \isa{G\ {\isasymlongrightarrow}\ H} pertenece a \isa{S} para fórmulas \isa{G} y \isa{H}
    cualesquiera. Observemos que se trata de una fórmula de tipo \isa{{\isasymbeta}} de componentes disyuntivas
    \isa{{\isasymnot}\ G} y \isa{H}. Luego, por hipótesis, tenemos que o bien \isa{{\isacharbraceleft}{\isasymnot}\ G{\isacharbraceright}\ {\isasymunion}\ S} pertenece a \isa{C} o
    bien \isa{{\isacharbraceleft}H{\isacharbraceright}\ {\isasymunion}\ S} pertenece a \isa{C}.
    \item[\isa{{\isadigit{7}}{\isacharparenright}}:] Supongamos que la fórmula \isa{{\isasymnot}{\isacharparenleft}G\ {\isasymand}\ H{\isacharparenright}} pertenece a \isa{S} para fórmulas \isa{G} y \isa{H}
    cualesquiera. Observemos que se trata de una fórmula de tipo \isa{{\isasymbeta}} de componentes disyuntivas
    \isa{{\isasymnot}\ G} y \isa{{\isasymnot}\ H}. Luego, por hipótesis, tenemos que o bien \isa{{\isacharbraceleft}{\isasymnot}\ G{\isacharbraceright}\ {\isasymunion}\ S} pertenece a \isa{C} o
    bien \isa{{\isacharbraceleft}{\isasymnot}\ H{\isacharbraceright}\ {\isasymunion}\ S} pertenece \isa{C}.
  \end{enumerate} 

  De este modo, hemos probado a partir de la hipótesis todas las condiciones que garantizan que la
  colección \isa{C} cumple la propiedad de consistencia proposicional. Por lo tanto, queda demostrado el
  resultado.
\end{demostracion}

  Para probar este resultado de manera detallada en Isabelle vamos a demostrar cada una de las 
  implicaciones de la equivalencia por separado. La primera implicación del lema se basa en dos 
  lemas auxiliares. El primero de ellos deduce la condición de \isa{{\isadigit{2}}{\isacharparenright}} sobre fórmulas de tipo \isa{{\isasymalpha}} a 
  partir de las condiciones tercera, sexta, octava y novena de la definición de propiedad de 
  consistencia proposicional. Su demostración detallada en Isabelle se muestra a continuación.%
\end{isamarkuptext}\isamarkuptrue%
\isacommand{lemma}\isamarkupfalse%
\ pcp{\isacharunderscore}alt{\isadigit{1}}Con{\isacharcolon}\isanewline
\ \ \isakeyword{assumes}\ {\isachardoublequoteopen}{\isacharparenleft}{\isasymforall}G\ H{\isachardot}\ G\ \isactrlbold {\isasymand}\ H\ {\isasymin}\ S\ {\isasymlongrightarrow}\ {\isacharbraceleft}G{\isacharcomma}H{\isacharbraceright}\ {\isasymunion}\ S\ {\isasymin}\ C{\isacharparenright}\isanewline
\ \ {\isasymand}\ {\isacharparenleft}{\isasymforall}G{\isachardot}\ \isactrlbold {\isasymnot}\ {\isacharparenleft}\isactrlbold {\isasymnot}G{\isacharparenright}\ {\isasymin}\ S\ {\isasymlongrightarrow}\ {\isacharbraceleft}G{\isacharbraceright}\ {\isasymunion}\ S\ {\isasymin}\ C{\isacharparenright}\isanewline
\ \ {\isasymand}\ {\isacharparenleft}{\isasymforall}G\ H{\isachardot}\ \isactrlbold {\isasymnot}{\isacharparenleft}G\ \isactrlbold {\isasymor}\ H{\isacharparenright}\ {\isasymin}\ S\ {\isasymlongrightarrow}\ {\isacharbraceleft}\isactrlbold {\isasymnot}\ G{\isacharcomma}\ \isactrlbold {\isasymnot}\ H{\isacharbraceright}\ {\isasymunion}\ S\ {\isasymin}\ C{\isacharparenright}\isanewline
\ \ {\isasymand}\ {\isacharparenleft}{\isasymforall}G\ H{\isachardot}\ \isactrlbold {\isasymnot}{\isacharparenleft}G\ \isactrlbold {\isasymrightarrow}\ H{\isacharparenright}\ {\isasymin}\ S\ {\isasymlongrightarrow}\ {\isacharbraceleft}G{\isacharcomma}\isactrlbold {\isasymnot}\ H{\isacharbraceright}\ {\isasymunion}\ S\ {\isasymin}\ C{\isacharparenright}{\isachardoublequoteclose}\isanewline
\ \ \isakeyword{shows}\ {\isachardoublequoteopen}{\isasymforall}F\ G\ H{\isachardot}\ Con\ F\ G\ H\ {\isasymlongrightarrow}\ F\ {\isasymin}\ S\ {\isasymlongrightarrow}\ {\isacharbraceleft}G{\isacharcomma}H{\isacharbraceright}\ {\isasymunion}\ S\ {\isasymin}\ C{\isachardoublequoteclose}\isanewline
%
\isadelimproof
%
\endisadelimproof
%
\isatagproof
\isacommand{proof}\isamarkupfalse%
\ {\isacharminus}\isanewline
\ \ \isacommand{have}\isamarkupfalse%
\ C{\isadigit{1}}{\isacharcolon}{\isachardoublequoteopen}{\isasymforall}G\ H{\isachardot}\ G\ \isactrlbold {\isasymand}\ H\ {\isasymin}\ S\ {\isasymlongrightarrow}\ {\isacharbraceleft}G{\isacharcomma}H{\isacharbraceright}\ {\isasymunion}\ S\ {\isasymin}\ C{\isachardoublequoteclose}\isanewline
\ \ \ \ \isacommand{using}\isamarkupfalse%
\ assms\ \isacommand{by}\isamarkupfalse%
\ {\isacharparenleft}rule\ conjunct{\isadigit{1}}{\isacharparenright}\isanewline
\ \ \isacommand{have}\isamarkupfalse%
\ C{\isadigit{2}}{\isacharcolon}{\isachardoublequoteopen}{\isasymforall}G{\isachardot}\ \isactrlbold {\isasymnot}\ {\isacharparenleft}\isactrlbold {\isasymnot}G{\isacharparenright}\ {\isasymin}\ S\ {\isasymlongrightarrow}\ {\isacharbraceleft}G{\isacharbraceright}\ {\isasymunion}\ S\ {\isasymin}\ C{\isachardoublequoteclose}\isanewline
\ \ \ \ \isacommand{using}\isamarkupfalse%
\ assms\ \isacommand{by}\isamarkupfalse%
\ {\isacharparenleft}iprover\ elim{\isacharcolon}\ conjunct{\isadigit{2}}\ conjunct{\isadigit{1}}{\isacharparenright}\isanewline
\ \ \isacommand{have}\isamarkupfalse%
\ C{\isadigit{3}}{\isacharcolon}{\isachardoublequoteopen}{\isasymforall}G\ H{\isachardot}\ \isactrlbold {\isasymnot}{\isacharparenleft}G\ \isactrlbold {\isasymor}\ H{\isacharparenright}\ {\isasymin}\ S\ {\isasymlongrightarrow}\ {\isacharbraceleft}\isactrlbold {\isasymnot}\ G{\isacharcomma}\ \isactrlbold {\isasymnot}\ H{\isacharbraceright}\ {\isasymunion}\ S\ {\isasymin}\ C{\isachardoublequoteclose}\isanewline
\ \ \ \ \isacommand{using}\isamarkupfalse%
\ assms\ \isacommand{by}\isamarkupfalse%
\ {\isacharparenleft}iprover\ elim{\isacharcolon}\ conjunct{\isadigit{2}}\ conjunct{\isadigit{1}}{\isacharparenright}\isanewline
\ \ \isacommand{have}\isamarkupfalse%
\ C{\isadigit{4}}{\isacharcolon}{\isachardoublequoteopen}{\isasymforall}G\ H{\isachardot}\ \isactrlbold {\isasymnot}{\isacharparenleft}G\ \isactrlbold {\isasymrightarrow}\ H{\isacharparenright}\ {\isasymin}\ S\ {\isasymlongrightarrow}\ {\isacharbraceleft}G{\isacharcomma}\isactrlbold {\isasymnot}\ H{\isacharbraceright}\ {\isasymunion}\ S\ {\isasymin}\ C{\isachardoublequoteclose}\isanewline
\ \ \ \ \isacommand{using}\isamarkupfalse%
\ assms\ \isacommand{by}\isamarkupfalse%
\ {\isacharparenleft}iprover\ elim{\isacharcolon}\ conjunct{\isadigit{2}}{\isacharparenright}\ \isanewline
\ \ \isacommand{show}\isamarkupfalse%
\ {\isachardoublequoteopen}{\isasymforall}F\ G\ H{\isachardot}\ Con\ F\ G\ H\ {\isasymlongrightarrow}\ F\ {\isasymin}\ S\ {\isasymlongrightarrow}\ {\isacharbraceleft}G{\isacharcomma}H{\isacharbraceright}\ {\isasymunion}\ S\ {\isasymin}\ C{\isachardoublequoteclose}\isanewline
\ \ \isacommand{proof}\isamarkupfalse%
\ {\isacharparenleft}rule\ allI{\isacharparenright}{\isacharplus}\isanewline
\ \ \ \ \isacommand{fix}\isamarkupfalse%
\ F\ G\ H\isanewline
\ \ \ \ \isacommand{show}\isamarkupfalse%
\ {\isachardoublequoteopen}Con\ F\ G\ H\ {\isasymlongrightarrow}\ F\ {\isasymin}\ S\ {\isasymlongrightarrow}\ {\isacharbraceleft}G{\isacharcomma}H{\isacharbraceright}\ {\isasymunion}\ S\ {\isasymin}\ C{\isachardoublequoteclose}\isanewline
\ \ \ \ \isacommand{proof}\isamarkupfalse%
\ {\isacharparenleft}rule\ impI{\isacharparenright}\isanewline
\ \ \ \ \ \ \isacommand{assume}\isamarkupfalse%
\ {\isachardoublequoteopen}Con\ F\ G\ H{\isachardoublequoteclose}\isanewline
\ \ \ \ \ \ \isacommand{then}\isamarkupfalse%
\ \isacommand{have}\isamarkupfalse%
\ {\isachardoublequoteopen}F\ {\isacharequal}\ G\ \isactrlbold {\isasymand}\ H\ {\isasymor}\ \isanewline
\ \ \ \ \ \ \ \ \ \ \ \ \ \ \ \ {\isacharparenleft}{\isacharparenleft}{\isasymexists}G{\isadigit{1}}\ H{\isadigit{1}}{\isachardot}\ F\ {\isacharequal}\ \isactrlbold {\isasymnot}\ {\isacharparenleft}G{\isadigit{1}}\ \isactrlbold {\isasymor}\ H{\isadigit{1}}{\isacharparenright}\ {\isasymand}\ G\ {\isacharequal}\ \isactrlbold {\isasymnot}\ G{\isadigit{1}}\ {\isasymand}\ H\ {\isacharequal}\ \isactrlbold {\isasymnot}\ H{\isadigit{1}}{\isacharparenright}\ {\isasymor}\ \isanewline
\ \ \ \ \ \ \ \ \ \ \ \ \ \ \ \ {\isacharparenleft}{\isasymexists}H{\isadigit{2}}{\isachardot}\ F\ {\isacharequal}\ \isactrlbold {\isasymnot}\ {\isacharparenleft}G\ \isactrlbold {\isasymrightarrow}\ H{\isadigit{2}}{\isacharparenright}\ {\isasymand}\ H\ {\isacharequal}\ \isactrlbold {\isasymnot}\ H{\isadigit{2}}{\isacharparenright}\ {\isasymor}\ \isanewline
\ \ \ \ \ \ \ \ \ \ \ \ \ \ \ \ F\ {\isacharequal}\ \isactrlbold {\isasymnot}\ {\isacharparenleft}\isactrlbold {\isasymnot}\ G{\isacharparenright}\ {\isasymand}\ H\ {\isacharequal}\ G{\isacharparenright}{\isachardoublequoteclose}\isanewline
\ \ \ \ \ \ \ \ \isacommand{by}\isamarkupfalse%
\ {\isacharparenleft}simp\ only{\isacharcolon}\ con{\isacharunderscore}dis{\isacharunderscore}simps{\isacharparenleft}{\isadigit{1}}{\isacharparenright}{\isacharparenright}\isanewline
\ \ \ \ \ \ \isacommand{thus}\isamarkupfalse%
\ {\isachardoublequoteopen}F\ {\isasymin}\ S\ {\isasymlongrightarrow}\ {\isacharbraceleft}G{\isacharcomma}H{\isacharbraceright}\ {\isasymunion}\ S\ {\isasymin}\ C{\isachardoublequoteclose}\isanewline
\ \ \ \ \ \ \isacommand{proof}\isamarkupfalse%
\ {\isacharparenleft}rule\ disjE{\isacharparenright}\isanewline
\ \ \ \ \ \ \ \ \isacommand{assume}\isamarkupfalse%
\ {\isachardoublequoteopen}F\ {\isacharequal}\ G\ \isactrlbold {\isasymand}\ H{\isachardoublequoteclose}\isanewline
\ \ \ \ \ \ \ \ \isacommand{show}\isamarkupfalse%
\ {\isachardoublequoteopen}F\ {\isasymin}\ S\ {\isasymlongrightarrow}\ {\isacharbraceleft}G{\isacharcomma}H{\isacharbraceright}\ {\isasymunion}\ S\ {\isasymin}\ C{\isachardoublequoteclose}\isanewline
\ \ \ \ \ \ \ \ \ \ \isacommand{using}\isamarkupfalse%
\ C{\isadigit{1}}\ {\isacartoucheopen}F\ {\isacharequal}\ G\ \isactrlbold {\isasymand}\ H{\isacartoucheclose}\ \isacommand{by}\isamarkupfalse%
\ {\isacharparenleft}iprover\ elim{\isacharcolon}\ allE{\isacharparenright}\isanewline
\ \ \ \ \ \ \isacommand{next}\isamarkupfalse%
\isanewline
\ \ \ \ \ \ \ \ \isacommand{assume}\isamarkupfalse%
\ {\isachardoublequoteopen}{\isacharparenleft}{\isasymexists}G{\isadigit{1}}\ H{\isadigit{1}}{\isachardot}\ F\ {\isacharequal}\ \isactrlbold {\isasymnot}\ {\isacharparenleft}G{\isadigit{1}}\ \isactrlbold {\isasymor}\ H{\isadigit{1}}{\isacharparenright}\ {\isasymand}\ G\ {\isacharequal}\ \isactrlbold {\isasymnot}\ G{\isadigit{1}}\ {\isasymand}\ H\ {\isacharequal}\ \isactrlbold {\isasymnot}\ H{\isadigit{1}}{\isacharparenright}\ {\isasymor}\ \isanewline
\ \ \ \ \ \ \ \ \ \ \ \ \ \ \ \ {\isacharparenleft}{\isasymexists}H{\isadigit{2}}{\isachardot}\ F\ {\isacharequal}\ \isactrlbold {\isasymnot}\ {\isacharparenleft}G\ \isactrlbold {\isasymrightarrow}\ H{\isadigit{2}}{\isacharparenright}\ {\isasymand}\ H\ {\isacharequal}\ \isactrlbold {\isasymnot}\ H{\isadigit{2}}{\isacharparenright}\ {\isasymor}\ \isanewline
\ \ \ \ \ \ \ \ \ \ \ \ \ \ \ \ F\ {\isacharequal}\ \isactrlbold {\isasymnot}\ {\isacharparenleft}\isactrlbold {\isasymnot}\ G{\isacharparenright}\ {\isasymand}\ H\ {\isacharequal}\ G{\isachardoublequoteclose}\isanewline
\ \ \ \ \ \ \ \ \isacommand{thus}\isamarkupfalse%
\ {\isachardoublequoteopen}F\ {\isasymin}\ S\ {\isasymlongrightarrow}\ {\isacharbraceleft}G{\isacharcomma}H{\isacharbraceright}\ {\isasymunion}\ S\ {\isasymin}\ C{\isachardoublequoteclose}\isanewline
\ \ \ \ \ \ \ \ \isacommand{proof}\isamarkupfalse%
\ {\isacharparenleft}rule\ disjE{\isacharparenright}\isanewline
\ \ \ \ \ \ \ \ \ \ \isacommand{assume}\isamarkupfalse%
\ E{\isadigit{1}}{\isacharcolon}{\isachardoublequoteopen}{\isasymexists}G{\isadigit{1}}\ H{\isadigit{1}}{\isachardot}\ F\ {\isacharequal}\ \isactrlbold {\isasymnot}\ {\isacharparenleft}G{\isadigit{1}}\ \isactrlbold {\isasymor}\ H{\isadigit{1}}{\isacharparenright}\ {\isasymand}\ G\ {\isacharequal}\ \isactrlbold {\isasymnot}\ G{\isadigit{1}}\ {\isasymand}\ H\ {\isacharequal}\ \isactrlbold {\isasymnot}\ H{\isadigit{1}}{\isachardoublequoteclose}\isanewline
\ \ \ \ \ \ \ \ \ \ \isacommand{obtain}\isamarkupfalse%
\ G{\isadigit{1}}\ H{\isadigit{1}}\ \isakeyword{where}\ A{\isadigit{1}}{\isacharcolon}{\isachardoublequoteopen}F\ {\isacharequal}\ \isactrlbold {\isasymnot}\ {\isacharparenleft}G{\isadigit{1}}\ \isactrlbold {\isasymor}\ H{\isadigit{1}}{\isacharparenright}\ {\isasymand}\ G\ {\isacharequal}\ \isactrlbold {\isasymnot}\ G{\isadigit{1}}\ {\isasymand}\ H\ {\isacharequal}\ \isactrlbold {\isasymnot}\ H{\isadigit{1}}{\isachardoublequoteclose}\isanewline
\ \ \ \ \ \ \ \ \ \ \ \ \isacommand{using}\isamarkupfalse%
\ E{\isadigit{1}}\ \isacommand{by}\isamarkupfalse%
\ {\isacharparenleft}iprover\ elim{\isacharcolon}\ exE{\isacharparenright}\isanewline
\ \ \ \ \ \ \ \ \ \ \isacommand{have}\isamarkupfalse%
\ {\isachardoublequoteopen}F\ {\isacharequal}\ \isactrlbold {\isasymnot}\ {\isacharparenleft}G{\isadigit{1}}\ \isactrlbold {\isasymor}\ H{\isadigit{1}}{\isacharparenright}{\isachardoublequoteclose}\isanewline
\ \ \ \ \ \ \ \ \ \ \ \ \isacommand{using}\isamarkupfalse%
\ A{\isadigit{1}}\ \isacommand{by}\isamarkupfalse%
\ {\isacharparenleft}rule\ conjunct{\isadigit{1}}{\isacharparenright}\isanewline
\ \ \ \ \ \ \ \ \ \ \isacommand{have}\isamarkupfalse%
\ {\isachardoublequoteopen}G\ {\isacharequal}\ \isactrlbold {\isasymnot}\ G{\isadigit{1}}{\isachardoublequoteclose}\isanewline
\ \ \ \ \ \ \ \ \ \ \ \ \isacommand{using}\isamarkupfalse%
\ A{\isadigit{1}}\ \isacommand{by}\isamarkupfalse%
\ {\isacharparenleft}iprover\ elim{\isacharcolon}\ conjunct{\isadigit{2}}\ conjunct{\isadigit{1}}{\isacharparenright}\isanewline
\ \ \ \ \ \ \ \ \ \ \isacommand{have}\isamarkupfalse%
\ {\isachardoublequoteopen}H\ {\isacharequal}\ \isactrlbold {\isasymnot}\ H{\isadigit{1}}{\isachardoublequoteclose}\isanewline
\ \ \ \ \ \ \ \ \ \ \ \ \isacommand{using}\isamarkupfalse%
\ A{\isadigit{1}}\ \isacommand{by}\isamarkupfalse%
\ {\isacharparenleft}iprover\ elim{\isacharcolon}\ conjunct{\isadigit{2}}{\isacharparenright}\isanewline
\ \ \ \ \ \ \ \ \ \ \isacommand{show}\isamarkupfalse%
\ {\isachardoublequoteopen}F\ {\isasymin}\ S\ {\isasymlongrightarrow}\ {\isacharbraceleft}G{\isacharcomma}H{\isacharbraceright}\ {\isasymunion}\ S\ {\isasymin}\ C{\isachardoublequoteclose}\isanewline
\ \ \ \ \ \ \ \ \ \ \ \ \isacommand{using}\isamarkupfalse%
\ C{\isadigit{3}}\ {\isacartoucheopen}F\ {\isacharequal}\ \isactrlbold {\isasymnot}\ {\isacharparenleft}G{\isadigit{1}}\ \isactrlbold {\isasymor}\ H{\isadigit{1}}{\isacharparenright}{\isacartoucheclose}\ {\isacartoucheopen}G\ {\isacharequal}\ \isactrlbold {\isasymnot}\ G{\isadigit{1}}{\isacartoucheclose}\ {\isacartoucheopen}H\ {\isacharequal}\ \isactrlbold {\isasymnot}\ H{\isadigit{1}}{\isacartoucheclose}\ \isacommand{by}\isamarkupfalse%
\ {\isacharparenleft}iprover\ elim{\isacharcolon}\ allE{\isacharparenright}\isanewline
\ \ \ \ \ \ \ \ \isacommand{next}\isamarkupfalse%
\isanewline
\ \ \ \ \ \ \ \ \ \ \isacommand{assume}\isamarkupfalse%
\ {\isachardoublequoteopen}{\isacharparenleft}{\isasymexists}H{\isadigit{2}}{\isachardot}\ F\ {\isacharequal}\ \isactrlbold {\isasymnot}\ {\isacharparenleft}G\ \isactrlbold {\isasymrightarrow}\ H{\isadigit{2}}{\isacharparenright}\ {\isasymand}\ H\ {\isacharequal}\ \isactrlbold {\isasymnot}\ H{\isadigit{2}}{\isacharparenright}\ {\isasymor}\ \isanewline
\ \ \ \ \ \ \ \ \ \ \ \ \ \ \ \ \ \ \ F\ {\isacharequal}\ \isactrlbold {\isasymnot}\ {\isacharparenleft}\isactrlbold {\isasymnot}\ G{\isacharparenright}\ {\isasymand}\ H\ {\isacharequal}\ G{\isachardoublequoteclose}\ \isanewline
\ \ \ \ \ \ \ \ \ \ \isacommand{thus}\isamarkupfalse%
\ {\isachardoublequoteopen}F\ {\isasymin}\ S\ {\isasymlongrightarrow}\ {\isacharbraceleft}G{\isacharcomma}H{\isacharbraceright}\ {\isasymunion}\ S\ {\isasymin}\ C{\isachardoublequoteclose}\isanewline
\ \ \ \ \ \ \ \ \ \ \isacommand{proof}\isamarkupfalse%
\ {\isacharparenleft}rule\ disjE{\isacharparenright}\isanewline
\ \ \ \ \ \ \ \ \ \ \ \ \isacommand{assume}\isamarkupfalse%
\ E{\isadigit{2}}{\isacharcolon}{\isachardoublequoteopen}{\isasymexists}H{\isadigit{2}}{\isachardot}\ F\ {\isacharequal}\ \isactrlbold {\isasymnot}\ {\isacharparenleft}G\ \isactrlbold {\isasymrightarrow}\ H{\isadigit{2}}{\isacharparenright}\ {\isasymand}\ H\ {\isacharequal}\ \isactrlbold {\isasymnot}\ H{\isadigit{2}}{\isachardoublequoteclose}\isanewline
\ \ \ \ \ \ \ \ \ \ \ \ \isacommand{obtain}\isamarkupfalse%
\ H{\isadigit{2}}\ \isakeyword{where}\ A{\isadigit{2}}{\isacharcolon}{\isachardoublequoteopen}F\ {\isacharequal}\ \isactrlbold {\isasymnot}\ {\isacharparenleft}G\ \isactrlbold {\isasymrightarrow}\ H{\isadigit{2}}{\isacharparenright}\ {\isasymand}\ H\ {\isacharequal}\ \isactrlbold {\isasymnot}\ H{\isadigit{2}}{\isachardoublequoteclose}\isanewline
\ \ \ \ \ \ \ \ \ \ \ \ \ \ \isacommand{using}\isamarkupfalse%
\ E{\isadigit{2}}\ \isacommand{by}\isamarkupfalse%
\ {\isacharparenleft}rule\ exE{\isacharparenright}\isanewline
\ \ \ \ \ \ \ \ \ \ \ \ \isacommand{have}\isamarkupfalse%
\ {\isachardoublequoteopen}F\ {\isacharequal}\ \isactrlbold {\isasymnot}\ {\isacharparenleft}G\ \isactrlbold {\isasymrightarrow}\ H{\isadigit{2}}{\isacharparenright}{\isachardoublequoteclose}\isanewline
\ \ \ \ \ \ \ \ \ \ \ \ \ \ \isacommand{using}\isamarkupfalse%
\ A{\isadigit{2}}\ \isacommand{by}\isamarkupfalse%
\ {\isacharparenleft}rule\ conjunct{\isadigit{1}}{\isacharparenright}\isanewline
\ \ \ \ \ \ \ \ \ \ \ \ \isacommand{have}\isamarkupfalse%
\ {\isachardoublequoteopen}H\ {\isacharequal}\ \isactrlbold {\isasymnot}\ H{\isadigit{2}}{\isachardoublequoteclose}\isanewline
\ \ \ \ \ \ \ \ \ \ \ \ \ \ \isacommand{using}\isamarkupfalse%
\ A{\isadigit{2}}\ \isacommand{by}\isamarkupfalse%
\ {\isacharparenleft}rule\ conjunct{\isadigit{2}}{\isacharparenright}\isanewline
\ \ \ \ \ \ \ \ \ \ \ \ \isacommand{show}\isamarkupfalse%
\ {\isachardoublequoteopen}F\ {\isasymin}\ S\ {\isasymlongrightarrow}\ {\isacharbraceleft}G{\isacharcomma}H{\isacharbraceright}\ {\isasymunion}\ S\ {\isasymin}\ C{\isachardoublequoteclose}\isanewline
\ \ \ \ \ \ \ \ \ \ \ \ \ \ \isacommand{using}\isamarkupfalse%
\ C{\isadigit{4}}\ {\isacartoucheopen}F\ {\isacharequal}\ \isactrlbold {\isasymnot}\ {\isacharparenleft}G\ \isactrlbold {\isasymrightarrow}\ H{\isadigit{2}}{\isacharparenright}{\isacartoucheclose}\ {\isacartoucheopen}H\ {\isacharequal}\ \isactrlbold {\isasymnot}\ H{\isadigit{2}}{\isacartoucheclose}\ \isacommand{by}\isamarkupfalse%
\ {\isacharparenleft}iprover\ elim{\isacharcolon}\ allE{\isacharparenright}\isanewline
\ \ \ \ \ \ \ \ \ \ \isacommand{next}\isamarkupfalse%
\isanewline
\ \ \ \ \ \ \ \ \ \ \ \ \isacommand{assume}\isamarkupfalse%
\ A{\isadigit{3}}{\isacharcolon}{\isachardoublequoteopen}F\ {\isacharequal}\ \isactrlbold {\isasymnot}{\isacharparenleft}\isactrlbold {\isasymnot}\ G{\isacharparenright}\ {\isasymand}\ H\ {\isacharequal}\ G{\isachardoublequoteclose}\isanewline
\ \ \ \ \ \ \ \ \ \ \ \ \isacommand{then}\isamarkupfalse%
\ \isacommand{have}\isamarkupfalse%
\ {\isachardoublequoteopen}F\ {\isacharequal}\ \isactrlbold {\isasymnot}{\isacharparenleft}\isactrlbold {\isasymnot}\ G{\isacharparenright}{\isachardoublequoteclose}\isanewline
\ \ \ \ \ \ \ \ \ \ \ \ \ \ \isacommand{by}\isamarkupfalse%
\ {\isacharparenleft}rule\ conjunct{\isadigit{1}}{\isacharparenright}\isanewline
\ \ \ \ \ \ \ \ \ \ \ \ \isacommand{have}\isamarkupfalse%
\ {\isachardoublequoteopen}H\ {\isacharequal}\ G{\isachardoublequoteclose}\isanewline
\ \ \ \ \ \ \ \ \ \ \ \ \ \ \isacommand{using}\isamarkupfalse%
\ A{\isadigit{3}}\ \isacommand{by}\isamarkupfalse%
\ {\isacharparenleft}rule\ conjunct{\isadigit{2}}{\isacharparenright}\isanewline
\ \ \ \ \ \ \ \ \ \ \ \ \isacommand{have}\isamarkupfalse%
\ {\isachardoublequoteopen}F\ {\isasymin}\ S\ {\isasymlongrightarrow}\ {\isacharbraceleft}G{\isacharbraceright}\ {\isasymunion}\ S\ {\isasymin}\ C{\isachardoublequoteclose}\isanewline
\ \ \ \ \ \ \ \ \ \ \ \ \ \ \isacommand{using}\isamarkupfalse%
\ C{\isadigit{2}}\ {\isacartoucheopen}F\ {\isacharequal}\ \isactrlbold {\isasymnot}{\isacharparenleft}\isactrlbold {\isasymnot}\ G{\isacharparenright}{\isacartoucheclose}\ \isacommand{by}\isamarkupfalse%
\ {\isacharparenleft}iprover\ elim{\isacharcolon}\ allE{\isacharparenright}\isanewline
\ \ \ \ \ \ \ \ \ \ \ \ \isacommand{then}\isamarkupfalse%
\ \isacommand{have}\isamarkupfalse%
\ {\isachardoublequoteopen}F\ {\isasymin}\ S\ {\isasymlongrightarrow}\ {\isacharbraceleft}G{\isacharcomma}G{\isacharbraceright}\ {\isasymunion}\ S\ {\isasymin}\ C{\isachardoublequoteclose}\isanewline
\ \ \ \ \ \ \ \ \ \ \ \ \ \ \isacommand{by}\isamarkupfalse%
\ {\isacharparenleft}simp\ only{\isacharcolon}\ insert{\isacharunderscore}absorb{\isadigit{2}}{\isacharparenright}\isanewline
\ \ \ \ \ \ \ \ \ \ \ \ \isacommand{thus}\isamarkupfalse%
\ {\isachardoublequoteopen}F\ {\isasymin}\ S\ {\isasymlongrightarrow}\ {\isacharbraceleft}G{\isacharcomma}H{\isacharbraceright}\ {\isasymunion}\ S\ {\isasymin}\ C{\isachardoublequoteclose}\ \isanewline
\ \ \ \ \ \ \ \ \ \ \ \ \ \ \isacommand{by}\isamarkupfalse%
\ {\isacharparenleft}simp\ only{\isacharcolon}\ {\isacartoucheopen}H\ {\isacharequal}\ G{\isacartoucheclose}{\isacharparenright}\isanewline
\ \ \ \ \ \ \ \ \ \ \isacommand{qed}\isamarkupfalse%
\isanewline
\ \ \ \ \ \ \ \ \isacommand{qed}\isamarkupfalse%
\isanewline
\ \ \ \ \ \ \isacommand{qed}\isamarkupfalse%
\isanewline
\ \ \ \ \isacommand{qed}\isamarkupfalse%
\isanewline
\ \ \isacommand{qed}\isamarkupfalse%
\isanewline
\isacommand{qed}\isamarkupfalse%
%
\endisatagproof
{\isafoldproof}%
%
\isadelimproof
%
\endisadelimproof
%
\begin{isamarkuptext}%
Finalmente, el siguiente lema auxiliar deduce la condición de \isa{{\isadigit{2}}{\isacharparenright}} sobre fórmulas de tipo \isa{{\isasymbeta}} 
  a partir de las condiciones cuarta, quinta, sexta y séptima de la definición de propiedad de 
  consistencia proposicional.%
\end{isamarkuptext}\isamarkuptrue%
\isacommand{lemma}\isamarkupfalse%
\ pcp{\isacharunderscore}alt{\isadigit{1}}Dis{\isacharcolon}\isanewline
\ \ \isakeyword{assumes}\ {\isachardoublequoteopen}{\isacharparenleft}{\isasymforall}G\ H{\isachardot}\ G\ \isactrlbold {\isasymor}\ H\ {\isasymin}\ S\ {\isasymlongrightarrow}\ {\isacharbraceleft}G{\isacharbraceright}\ {\isasymunion}\ S\ {\isasymin}\ C\ {\isasymor}\ {\isacharbraceleft}H{\isacharbraceright}\ {\isasymunion}\ S\ {\isasymin}\ C{\isacharparenright}\isanewline
\ \ {\isasymand}\ {\isacharparenleft}{\isasymforall}G\ H{\isachardot}\ G\ \isactrlbold {\isasymrightarrow}\ H\ {\isasymin}\ S\ {\isasymlongrightarrow}\ {\isacharbraceleft}\isactrlbold {\isasymnot}\ G{\isacharbraceright}\ {\isasymunion}\ S\ {\isasymin}\ C\ {\isasymor}\ {\isacharbraceleft}H{\isacharbraceright}\ {\isasymunion}\ S\ {\isasymin}\ C{\isacharparenright}\isanewline
\ \ {\isasymand}\ {\isacharparenleft}{\isasymforall}G{\isachardot}\ \isactrlbold {\isasymnot}\ {\isacharparenleft}\isactrlbold {\isasymnot}G{\isacharparenright}\ {\isasymin}\ S\ {\isasymlongrightarrow}\ {\isacharbraceleft}G{\isacharbraceright}\ {\isasymunion}\ S\ {\isasymin}\ C{\isacharparenright}\isanewline
\ \ {\isasymand}\ {\isacharparenleft}{\isasymforall}G\ H{\isachardot}\ \isactrlbold {\isasymnot}{\isacharparenleft}G\ \isactrlbold {\isasymand}\ H{\isacharparenright}\ {\isasymin}\ S\ {\isasymlongrightarrow}\ {\isacharbraceleft}\isactrlbold {\isasymnot}\ G{\isacharbraceright}\ {\isasymunion}\ S\ {\isasymin}\ C\ {\isasymor}\ {\isacharbraceleft}\isactrlbold {\isasymnot}\ H{\isacharbraceright}\ {\isasymunion}\ S\ {\isasymin}\ C{\isacharparenright}{\isachardoublequoteclose}\isanewline
\ \ \isakeyword{shows}\ {\isachardoublequoteopen}{\isasymforall}F\ G\ H{\isachardot}\ Dis\ F\ G\ H\ {\isasymlongrightarrow}\ F\ {\isasymin}\ S\ {\isasymlongrightarrow}\ {\isacharbraceleft}G{\isacharbraceright}\ {\isasymunion}\ S\ {\isasymin}\ C\ {\isasymor}\ {\isacharbraceleft}H{\isacharbraceright}\ {\isasymunion}\ S\ {\isasymin}\ C{\isachardoublequoteclose}\isanewline
%
\isadelimproof
%
\endisadelimproof
%
\isatagproof
\isacommand{proof}\isamarkupfalse%
\ {\isacharminus}\isanewline
\ \ \isacommand{have}\isamarkupfalse%
\ C{\isadigit{1}}{\isacharcolon}{\isachardoublequoteopen}{\isasymforall}G\ H{\isachardot}\ G\ \isactrlbold {\isasymor}\ H\ {\isasymin}\ S\ {\isasymlongrightarrow}\ {\isacharbraceleft}G{\isacharbraceright}\ {\isasymunion}\ S\ {\isasymin}\ C\ {\isasymor}\ {\isacharbraceleft}H{\isacharbraceright}\ {\isasymunion}\ S\ {\isasymin}\ C{\isachardoublequoteclose}\isanewline
\ \ \ \ \isacommand{using}\isamarkupfalse%
\ assms\ \isacommand{by}\isamarkupfalse%
\ {\isacharparenleft}rule\ conjunct{\isadigit{1}}{\isacharparenright}\isanewline
\ \ \isacommand{have}\isamarkupfalse%
\ C{\isadigit{2}}{\isacharcolon}{\isachardoublequoteopen}{\isasymforall}G\ H{\isachardot}\ G\ \isactrlbold {\isasymrightarrow}\ H\ {\isasymin}\ S\ {\isasymlongrightarrow}\ {\isacharbraceleft}\isactrlbold {\isasymnot}\ G{\isacharbraceright}\ {\isasymunion}\ S\ {\isasymin}\ C\ {\isasymor}\ {\isacharbraceleft}H{\isacharbraceright}\ {\isasymunion}\ S\ {\isasymin}\ C{\isachardoublequoteclose}\isanewline
\ \ \ \ \isacommand{using}\isamarkupfalse%
\ assms\ \isacommand{by}\isamarkupfalse%
\ {\isacharparenleft}iprover\ elim{\isacharcolon}\ conjunct{\isadigit{2}}\ conjunct{\isadigit{1}}{\isacharparenright}\isanewline
\ \ \isacommand{have}\isamarkupfalse%
\ C{\isadigit{3}}{\isacharcolon}{\isachardoublequoteopen}{\isasymforall}G{\isachardot}\ \isactrlbold {\isasymnot}\ {\isacharparenleft}\isactrlbold {\isasymnot}G{\isacharparenright}\ {\isasymin}\ S\ {\isasymlongrightarrow}\ {\isacharbraceleft}G{\isacharbraceright}\ {\isasymunion}\ S\ {\isasymin}\ C{\isachardoublequoteclose}\isanewline
\ \ \ \ \isacommand{using}\isamarkupfalse%
\ assms\ \isacommand{by}\isamarkupfalse%
\ {\isacharparenleft}iprover\ elim{\isacharcolon}\ conjunct{\isadigit{2}}\ conjunct{\isadigit{1}}{\isacharparenright}\isanewline
\ \ \isacommand{have}\isamarkupfalse%
\ C{\isadigit{4}}{\isacharcolon}{\isachardoublequoteopen}{\isasymforall}G\ H{\isachardot}\ \isactrlbold {\isasymnot}{\isacharparenleft}G\ \isactrlbold {\isasymand}\ H{\isacharparenright}\ {\isasymin}\ S\ {\isasymlongrightarrow}\ {\isacharbraceleft}\isactrlbold {\isasymnot}\ G{\isacharbraceright}\ {\isasymunion}\ S\ {\isasymin}\ C\ {\isasymor}\ {\isacharbraceleft}\isactrlbold {\isasymnot}\ H{\isacharbraceright}\ {\isasymunion}\ S\ {\isasymin}\ C{\isachardoublequoteclose}\isanewline
\ \ \ \ \isacommand{using}\isamarkupfalse%
\ assms\ \isacommand{by}\isamarkupfalse%
\ {\isacharparenleft}iprover\ elim{\isacharcolon}\ conjunct{\isadigit{2}}{\isacharparenright}\ \isanewline
\ \ \isacommand{show}\isamarkupfalse%
\ {\isachardoublequoteopen}{\isasymforall}F\ G\ H{\isachardot}\ Dis\ F\ G\ H\ {\isasymlongrightarrow}\ F\ {\isasymin}\ S\ {\isasymlongrightarrow}\ {\isacharbraceleft}G{\isacharbraceright}\ {\isasymunion}\ S\ {\isasymin}\ C\ {\isasymor}\ {\isacharbraceleft}H{\isacharbraceright}\ {\isasymunion}\ S\ {\isasymin}\ C{\isachardoublequoteclose}\isanewline
\ \ \isacommand{proof}\isamarkupfalse%
\ {\isacharparenleft}rule\ allI{\isacharparenright}{\isacharplus}\isanewline
\ \ \ \ \isacommand{fix}\isamarkupfalse%
\ F\ G\ H\isanewline
\ \ \ \ \isacommand{show}\isamarkupfalse%
\ {\isachardoublequoteopen}Dis\ F\ G\ H\ {\isasymlongrightarrow}\ F\ {\isasymin}\ S\ {\isasymlongrightarrow}\ {\isacharbraceleft}G{\isacharbraceright}\ {\isasymunion}\ S\ {\isasymin}\ C\ {\isasymor}\ {\isacharbraceleft}H{\isacharbraceright}\ {\isasymunion}\ S\ {\isasymin}\ C{\isachardoublequoteclose}\isanewline
\ \ \ \ \isacommand{proof}\isamarkupfalse%
\ {\isacharparenleft}rule\ impI{\isacharparenright}\isanewline
\ \ \ \ \ \ \isacommand{assume}\isamarkupfalse%
\ {\isachardoublequoteopen}Dis\ F\ G\ H{\isachardoublequoteclose}\isanewline
\ \ \ \ \ \ \isacommand{then}\isamarkupfalse%
\ \isacommand{have}\isamarkupfalse%
\ {\isachardoublequoteopen}F\ {\isacharequal}\ G\ \isactrlbold {\isasymor}\ H\ {\isasymor}\ \isanewline
\ \ \ \ \ \ \ \ \ \ \ \ \ \ \ \ {\isacharparenleft}{\isasymexists}G{\isadigit{1}}\ H{\isadigit{1}}{\isachardot}\ F\ {\isacharequal}\ G{\isadigit{1}}\ \isactrlbold {\isasymrightarrow}\ H{\isadigit{1}}\ {\isasymand}\ G\ {\isacharequal}\ \isactrlbold {\isasymnot}\ G{\isadigit{1}}\ {\isasymand}\ H\ {\isacharequal}\ H{\isadigit{1}}{\isacharparenright}\ {\isasymor}\ \isanewline
\ \ \ \ \ \ \ \ \ \ \ \ \ \ \ \ {\isacharparenleft}{\isasymexists}G{\isadigit{2}}\ H{\isadigit{2}}{\isachardot}\ F\ {\isacharequal}\ \isactrlbold {\isasymnot}\ {\isacharparenleft}G{\isadigit{2}}\ \isactrlbold {\isasymand}\ H{\isadigit{2}}{\isacharparenright}\ {\isasymand}\ G\ {\isacharequal}\ \isactrlbold {\isasymnot}\ G{\isadigit{2}}\ {\isasymand}\ H\ {\isacharequal}\ \isactrlbold {\isasymnot}\ H{\isadigit{2}}{\isacharparenright}\ {\isasymor}\ \isanewline
\ \ \ \ \ \ \ \ \ \ \ \ \ \ \ \ F\ {\isacharequal}\ \isactrlbold {\isasymnot}\ {\isacharparenleft}\isactrlbold {\isasymnot}\ G{\isacharparenright}\ {\isasymand}\ H\ {\isacharequal}\ G{\isachardoublequoteclose}\ \isanewline
\ \ \ \ \ \ \ \ \isacommand{by}\isamarkupfalse%
\ {\isacharparenleft}simp\ only{\isacharcolon}\ con{\isacharunderscore}dis{\isacharunderscore}simps{\isacharparenleft}{\isadigit{2}}{\isacharparenright}{\isacharparenright}\isanewline
\ \ \ \ \ \ \isacommand{thus}\isamarkupfalse%
\ {\isachardoublequoteopen}F\ {\isasymin}\ S\ {\isasymlongrightarrow}\ {\isacharbraceleft}G{\isacharbraceright}\ {\isasymunion}\ S\ {\isasymin}\ C\ {\isasymor}\ {\isacharbraceleft}H{\isacharbraceright}\ {\isasymunion}\ S\ {\isasymin}\ C{\isachardoublequoteclose}\isanewline
\ \ \ \ \ \ \isacommand{proof}\isamarkupfalse%
\ {\isacharparenleft}rule\ disjE{\isacharparenright}\isanewline
\ \ \ \ \ \ \ \ \isacommand{assume}\isamarkupfalse%
\ {\isachardoublequoteopen}F\ {\isacharequal}\ G\ \isactrlbold {\isasymor}\ H{\isachardoublequoteclose}\isanewline
\ \ \ \ \ \ \ \ \isacommand{show}\isamarkupfalse%
\ {\isachardoublequoteopen}F\ {\isasymin}\ S\ {\isasymlongrightarrow}\ {\isacharbraceleft}G{\isacharbraceright}\ {\isasymunion}\ S\ {\isasymin}\ C\ {\isasymor}\ {\isacharbraceleft}H{\isacharbraceright}\ {\isasymunion}\ S\ {\isasymin}\ C{\isachardoublequoteclose}\isanewline
\ \ \ \ \ \ \ \ \ \ \isacommand{using}\isamarkupfalse%
\ C{\isadigit{1}}\ {\isacartoucheopen}F\ {\isacharequal}\ G\ \isactrlbold {\isasymor}\ H{\isacartoucheclose}\ \isacommand{by}\isamarkupfalse%
\ {\isacharparenleft}iprover\ elim{\isacharcolon}\ allE{\isacharparenright}\isanewline
\ \ \ \ \ \ \isacommand{next}\isamarkupfalse%
\isanewline
\ \ \ \ \ \ \ \ \isacommand{assume}\isamarkupfalse%
\ {\isachardoublequoteopen}{\isacharparenleft}{\isasymexists}G{\isadigit{1}}\ H{\isadigit{1}}{\isachardot}\ F\ {\isacharequal}\ G{\isadigit{1}}\ \isactrlbold {\isasymrightarrow}\ H{\isadigit{1}}\ {\isasymand}\ G\ {\isacharequal}\ \isactrlbold {\isasymnot}\ G{\isadigit{1}}\ {\isasymand}\ H\ {\isacharequal}\ H{\isadigit{1}}{\isacharparenright}\ {\isasymor}\ \isanewline
\ \ \ \ \ \ \ \ \ \ \ \ \ \ {\isacharparenleft}{\isasymexists}G{\isadigit{2}}\ H{\isadigit{2}}{\isachardot}\ F\ {\isacharequal}\ \isactrlbold {\isasymnot}\ {\isacharparenleft}G{\isadigit{2}}\ \isactrlbold {\isasymand}\ H{\isadigit{2}}{\isacharparenright}\ {\isasymand}\ G\ {\isacharequal}\ \isactrlbold {\isasymnot}\ G{\isadigit{2}}\ {\isasymand}\ H\ {\isacharequal}\ \isactrlbold {\isasymnot}\ H{\isadigit{2}}{\isacharparenright}\ {\isasymor}\ \isanewline
\ \ \ \ \ \ \ \ \ \ \ \ \ \ F\ {\isacharequal}\ \isactrlbold {\isasymnot}\ {\isacharparenleft}\isactrlbold {\isasymnot}\ G{\isacharparenright}\ {\isasymand}\ H\ {\isacharequal}\ G{\isachardoublequoteclose}\isanewline
\ \ \ \ \ \ \ \ \isacommand{thus}\isamarkupfalse%
\ {\isachardoublequoteopen}F\ {\isasymin}\ S\ {\isasymlongrightarrow}\ {\isacharbraceleft}G{\isacharbraceright}\ {\isasymunion}\ S\ {\isasymin}\ C\ {\isasymor}\ {\isacharbraceleft}H{\isacharbraceright}\ {\isasymunion}\ S\ {\isasymin}\ C{\isachardoublequoteclose}\isanewline
\ \ \ \ \ \ \ \ \isacommand{proof}\isamarkupfalse%
\ {\isacharparenleft}rule\ disjE{\isacharparenright}\isanewline
\ \ \ \ \ \ \ \ \ \ \isacommand{assume}\isamarkupfalse%
\ E{\isadigit{1}}{\isacharcolon}{\isachardoublequoteopen}{\isasymexists}G{\isadigit{1}}\ H{\isadigit{1}}{\isachardot}\ F\ {\isacharequal}\ {\isacharparenleft}G{\isadigit{1}}\ \isactrlbold {\isasymrightarrow}\ H{\isadigit{1}}{\isacharparenright}\ {\isasymand}\ G\ {\isacharequal}\ \isactrlbold {\isasymnot}\ G{\isadigit{1}}\ {\isasymand}\ H\ {\isacharequal}\ H{\isadigit{1}}{\isachardoublequoteclose}\isanewline
\ \ \ \ \ \ \ \ \ \ \isacommand{obtain}\isamarkupfalse%
\ G{\isadigit{1}}\ H{\isadigit{1}}\ \isakeyword{where}\ A{\isadigit{1}}{\isacharcolon}{\isachardoublequoteopen}\ F\ {\isacharequal}\ {\isacharparenleft}G{\isadigit{1}}\ \isactrlbold {\isasymrightarrow}\ H{\isadigit{1}}{\isacharparenright}\ {\isasymand}\ G\ {\isacharequal}\ \isactrlbold {\isasymnot}\ G{\isadigit{1}}\ {\isasymand}\ H\ {\isacharequal}\ H{\isadigit{1}}{\isachardoublequoteclose}\isanewline
\ \ \ \ \ \ \ \ \ \ \ \ \isacommand{using}\isamarkupfalse%
\ E{\isadigit{1}}\ \isacommand{by}\isamarkupfalse%
\ {\isacharparenleft}iprover\ elim{\isacharcolon}\ exE{\isacharparenright}\isanewline
\ \ \ \ \ \ \ \ \ \ \isacommand{have}\isamarkupfalse%
\ {\isachardoublequoteopen}F\ {\isacharequal}\ {\isacharparenleft}G{\isadigit{1}}\ \isactrlbold {\isasymrightarrow}\ H{\isadigit{1}}{\isacharparenright}{\isachardoublequoteclose}\isanewline
\ \ \ \ \ \ \ \ \ \ \ \ \isacommand{using}\isamarkupfalse%
\ A{\isadigit{1}}\ \isacommand{by}\isamarkupfalse%
\ {\isacharparenleft}rule\ conjunct{\isadigit{1}}{\isacharparenright}\isanewline
\ \ \ \ \ \ \ \ \ \ \isacommand{have}\isamarkupfalse%
\ {\isachardoublequoteopen}G\ {\isacharequal}\ \isactrlbold {\isasymnot}\ G{\isadigit{1}}{\isachardoublequoteclose}\isanewline
\ \ \ \ \ \ \ \ \ \ \ \ \isacommand{using}\isamarkupfalse%
\ A{\isadigit{1}}\ \isacommand{by}\isamarkupfalse%
\ {\isacharparenleft}iprover\ elim{\isacharcolon}\ conjunct{\isadigit{2}}\ conjunct{\isadigit{1}}{\isacharparenright}\isanewline
\ \ \ \ \ \ \ \ \ \ \isacommand{have}\isamarkupfalse%
\ {\isachardoublequoteopen}H\ {\isacharequal}\ H{\isadigit{1}}{\isachardoublequoteclose}\isanewline
\ \ \ \ \ \ \ \ \ \ \ \ \isacommand{using}\isamarkupfalse%
\ A{\isadigit{1}}\ \isacommand{by}\isamarkupfalse%
\ {\isacharparenleft}iprover\ elim{\isacharcolon}\ conjunct{\isadigit{2}}{\isacharparenright}\isanewline
\ \ \ \ \ \ \ \ \ \ \isacommand{show}\isamarkupfalse%
\ {\isachardoublequoteopen}F\ {\isasymin}\ S\ {\isasymlongrightarrow}\ {\isacharbraceleft}G{\isacharbraceright}\ {\isasymunion}\ S\ {\isasymin}\ C\ {\isasymor}\ {\isacharbraceleft}H{\isacharbraceright}\ {\isasymunion}\ S\ {\isasymin}\ C{\isachardoublequoteclose}\isanewline
\ \ \ \ \ \ \ \ \ \ \ \ \isacommand{using}\isamarkupfalse%
\ C{\isadigit{2}}\ {\isacartoucheopen}F\ {\isacharequal}\ {\isacharparenleft}G{\isadigit{1}}\ \isactrlbold {\isasymrightarrow}\ H{\isadigit{1}}{\isacharparenright}{\isacartoucheclose}\ {\isacartoucheopen}G\ {\isacharequal}\ \isactrlbold {\isasymnot}\ G{\isadigit{1}}{\isacartoucheclose}\ {\isacartoucheopen}H\ {\isacharequal}\ H{\isadigit{1}}{\isacartoucheclose}\ \isacommand{by}\isamarkupfalse%
\ {\isacharparenleft}iprover\ elim{\isacharcolon}\ allE{\isacharparenright}\isanewline
\ \ \ \ \ \ \ \ \isacommand{next}\isamarkupfalse%
\isanewline
\ \ \ \ \ \ \ \ \ \ \isacommand{assume}\isamarkupfalse%
\ {\isachardoublequoteopen}{\isacharparenleft}{\isasymexists}G{\isadigit{2}}\ H{\isadigit{2}}{\isachardot}\ F\ {\isacharequal}\ \isactrlbold {\isasymnot}\ {\isacharparenleft}G{\isadigit{2}}\ \isactrlbold {\isasymand}\ H{\isadigit{2}}{\isacharparenright}\ {\isasymand}\ G\ {\isacharequal}\ \isactrlbold {\isasymnot}\ G{\isadigit{2}}\ {\isasymand}\ H\ {\isacharequal}\ \isactrlbold {\isasymnot}\ H{\isadigit{2}}{\isacharparenright}\ {\isasymor}\ \isanewline
\ \ \ \ \ \ \ \ \ \ \ \ \ \ \ \ \ \ F\ {\isacharequal}\ \isactrlbold {\isasymnot}\ {\isacharparenleft}\isactrlbold {\isasymnot}\ G{\isacharparenright}\ {\isasymand}\ H\ {\isacharequal}\ G{\isachardoublequoteclose}\ \isanewline
\ \ \ \ \ \ \ \ \ \ \isacommand{thus}\isamarkupfalse%
\ {\isachardoublequoteopen}F\ {\isasymin}\ S\ {\isasymlongrightarrow}\ {\isacharbraceleft}G{\isacharbraceright}\ {\isasymunion}\ S\ {\isasymin}\ C\ {\isasymor}\ {\isacharbraceleft}H{\isacharbraceright}\ {\isasymunion}\ S\ {\isasymin}\ C{\isachardoublequoteclose}\isanewline
\ \ \ \ \ \ \ \ \ \ \isacommand{proof}\isamarkupfalse%
\ {\isacharparenleft}rule\ disjE{\isacharparenright}\isanewline
\ \ \ \ \ \ \ \ \ \ \ \ \isacommand{assume}\isamarkupfalse%
\ E{\isadigit{2}}{\isacharcolon}{\isachardoublequoteopen}{\isasymexists}G{\isadigit{2}}\ H{\isadigit{2}}{\isachardot}\ F\ {\isacharequal}\ \isactrlbold {\isasymnot}\ {\isacharparenleft}G{\isadigit{2}}\ \isactrlbold {\isasymand}\ H{\isadigit{2}}{\isacharparenright}\ {\isasymand}\ G\ {\isacharequal}\ \isactrlbold {\isasymnot}\ G{\isadigit{2}}\ {\isasymand}\ H\ {\isacharequal}\ \isactrlbold {\isasymnot}\ H{\isadigit{2}}{\isachardoublequoteclose}\isanewline
\ \ \ \ \ \ \ \ \ \ \ \ \isacommand{obtain}\isamarkupfalse%
\ G{\isadigit{2}}\ H{\isadigit{2}}\ \isakeyword{where}\ A{\isadigit{2}}{\isacharcolon}{\isachardoublequoteopen}F\ {\isacharequal}\ \isactrlbold {\isasymnot}\ {\isacharparenleft}G{\isadigit{2}}\ \isactrlbold {\isasymand}\ H{\isadigit{2}}{\isacharparenright}\ {\isasymand}\ G\ {\isacharequal}\ \isactrlbold {\isasymnot}\ G{\isadigit{2}}\ {\isasymand}\ H\ {\isacharequal}\ \isactrlbold {\isasymnot}\ H{\isadigit{2}}{\isachardoublequoteclose}\isanewline
\ \ \ \ \ \ \ \ \ \ \ \ \ \ \isacommand{using}\isamarkupfalse%
\ E{\isadigit{2}}\ \isacommand{by}\isamarkupfalse%
\ {\isacharparenleft}iprover\ elim{\isacharcolon}\ exE{\isacharparenright}\isanewline
\ \ \ \ \ \ \ \ \ \ \ \ \isacommand{have}\isamarkupfalse%
\ {\isachardoublequoteopen}F\ {\isacharequal}\ \isactrlbold {\isasymnot}\ {\isacharparenleft}G{\isadigit{2}}\ \isactrlbold {\isasymand}\ H{\isadigit{2}}{\isacharparenright}{\isachardoublequoteclose}\isanewline
\ \ \ \ \ \ \ \ \ \ \ \ \ \ \isacommand{using}\isamarkupfalse%
\ A{\isadigit{2}}\ \isacommand{by}\isamarkupfalse%
\ {\isacharparenleft}rule\ conjunct{\isadigit{1}}{\isacharparenright}\isanewline
\ \ \ \ \ \ \ \ \ \ \ \ \isacommand{have}\isamarkupfalse%
\ {\isachardoublequoteopen}G\ {\isacharequal}\ \isactrlbold {\isasymnot}\ G{\isadigit{2}}{\isachardoublequoteclose}\isanewline
\ \ \ \ \ \ \ \ \ \ \ \ \ \ \isacommand{using}\isamarkupfalse%
\ A{\isadigit{2}}\ \isacommand{by}\isamarkupfalse%
\ {\isacharparenleft}iprover\ elim{\isacharcolon}\ conjunct{\isadigit{2}}\ conjunct{\isadigit{1}}{\isacharparenright}\isanewline
\ \ \ \ \ \ \ \ \ \ \ \ \isacommand{have}\isamarkupfalse%
\ {\isachardoublequoteopen}H\ {\isacharequal}\ \isactrlbold {\isasymnot}\ H{\isadigit{2}}{\isachardoublequoteclose}\isanewline
\ \ \ \ \ \ \ \ \ \ \ \ \ \ \isacommand{using}\isamarkupfalse%
\ A{\isadigit{2}}\ \isacommand{by}\isamarkupfalse%
\ {\isacharparenleft}iprover\ elim{\isacharcolon}\ conjunct{\isadigit{2}}{\isacharparenright}\isanewline
\ \ \ \ \ \ \ \ \ \ \ \ \isacommand{show}\isamarkupfalse%
\ {\isachardoublequoteopen}F\ {\isasymin}\ S\ {\isasymlongrightarrow}\ {\isacharbraceleft}G{\isacharbraceright}\ {\isasymunion}\ S\ {\isasymin}\ C\ {\isasymor}\ {\isacharbraceleft}H{\isacharbraceright}\ {\isasymunion}\ S\ {\isasymin}\ C{\isachardoublequoteclose}\isanewline
\ \ \ \ \ \ \ \ \ \ \ \ \ \ \isacommand{using}\isamarkupfalse%
\ C{\isadigit{4}}\ {\isacartoucheopen}F\ {\isacharequal}\ \isactrlbold {\isasymnot}\ {\isacharparenleft}G{\isadigit{2}}\ \isactrlbold {\isasymand}\ H{\isadigit{2}}{\isacharparenright}{\isacartoucheclose}\ {\isacartoucheopen}G\ {\isacharequal}\ \isactrlbold {\isasymnot}\ G{\isadigit{2}}{\isacartoucheclose}\ {\isacartoucheopen}H\ {\isacharequal}\ \isactrlbold {\isasymnot}\ H{\isadigit{2}}{\isacartoucheclose}\ \isacommand{by}\isamarkupfalse%
\ {\isacharparenleft}iprover\ elim{\isacharcolon}\ allE{\isacharparenright}\isanewline
\ \ \ \ \ \ \ \ \ \ \isacommand{next}\isamarkupfalse%
\isanewline
\ \ \ \ \ \ \ \ \ \ \ \ \isacommand{assume}\isamarkupfalse%
\ A{\isadigit{3}}{\isacharcolon}{\isachardoublequoteopen}F\ {\isacharequal}\ \isactrlbold {\isasymnot}{\isacharparenleft}\isactrlbold {\isasymnot}\ G{\isacharparenright}\ {\isasymand}\ H\ {\isacharequal}\ G{\isachardoublequoteclose}\isanewline
\ \ \ \ \ \ \ \ \ \ \ \ \isacommand{then}\isamarkupfalse%
\ \isacommand{have}\isamarkupfalse%
\ {\isachardoublequoteopen}F\ {\isacharequal}\ \isactrlbold {\isasymnot}{\isacharparenleft}\isactrlbold {\isasymnot}\ G{\isacharparenright}{\isachardoublequoteclose}\isanewline
\ \ \ \ \ \ \ \ \ \ \ \ \ \ \isacommand{by}\isamarkupfalse%
\ {\isacharparenleft}rule\ conjunct{\isadigit{1}}{\isacharparenright}\isanewline
\ \ \ \ \ \ \ \ \ \ \ \ \isacommand{have}\isamarkupfalse%
\ {\isachardoublequoteopen}H\ {\isacharequal}\ G{\isachardoublequoteclose}\isanewline
\ \ \ \ \ \ \ \ \ \ \ \ \ \ \isacommand{using}\isamarkupfalse%
\ A{\isadigit{3}}\ \isacommand{by}\isamarkupfalse%
\ {\isacharparenleft}rule\ conjunct{\isadigit{2}}{\isacharparenright}\isanewline
\ \ \ \ \ \ \ \ \ \ \ \ \isacommand{have}\isamarkupfalse%
\ {\isachardoublequoteopen}F\ {\isasymin}\ S\ {\isasymlongrightarrow}\ {\isacharbraceleft}G{\isacharbraceright}\ {\isasymunion}\ S\ {\isasymin}\ C{\isachardoublequoteclose}\isanewline
\ \ \ \ \ \ \ \ \ \ \ \ \ \ \isacommand{using}\isamarkupfalse%
\ C{\isadigit{3}}\ {\isacartoucheopen}F\ {\isacharequal}\ \isactrlbold {\isasymnot}{\isacharparenleft}\isactrlbold {\isasymnot}\ G{\isacharparenright}{\isacartoucheclose}\ \isacommand{by}\isamarkupfalse%
\ {\isacharparenleft}iprover\ elim{\isacharcolon}\ allE{\isacharparenright}\isanewline
\ \ \ \ \ \ \ \ \ \ \ \ \isacommand{then}\isamarkupfalse%
\ \isacommand{have}\isamarkupfalse%
\ {\isachardoublequoteopen}F\ {\isasymin}\ S\ {\isasymlongrightarrow}\ {\isacharbraceleft}G{\isacharbraceright}\ {\isasymunion}\ S\ {\isasymin}\ C\ {\isasymor}\ {\isacharbraceleft}G{\isacharbraceright}\ {\isasymunion}\ S\ {\isasymin}\ C{\isachardoublequoteclose}\isanewline
\ \ \ \ \ \ \ \ \ \ \ \ \ \ \isacommand{by}\isamarkupfalse%
\ {\isacharparenleft}simp\ only{\isacharcolon}\ disj{\isacharunderscore}absorb{\isacharparenright}\isanewline
\ \ \ \ \ \ \ \ \ \ \ \ \isacommand{thus}\isamarkupfalse%
\ {\isachardoublequoteopen}F\ {\isasymin}\ S\ {\isasymlongrightarrow}\ {\isacharbraceleft}G{\isacharbraceright}\ {\isasymunion}\ S\ {\isasymin}\ C\ {\isasymor}\ {\isacharbraceleft}H{\isacharbraceright}\ {\isasymunion}\ S\ {\isasymin}\ C{\isachardoublequoteclose}\isanewline
\ \ \ \ \ \ \ \ \ \ \ \ \ \ \isacommand{by}\isamarkupfalse%
\ {\isacharparenleft}simp\ only{\isacharcolon}\ {\isacartoucheopen}H\ {\isacharequal}\ G{\isacartoucheclose}{\isacharparenright}\isanewline
\ \ \ \ \ \ \ \ \ \ \isacommand{qed}\isamarkupfalse%
\isanewline
\ \ \ \ \ \ \ \ \isacommand{qed}\isamarkupfalse%
\isanewline
\ \ \ \ \ \ \isacommand{qed}\isamarkupfalse%
\isanewline
\ \ \ \ \isacommand{qed}\isamarkupfalse%
\isanewline
\ \ \isacommand{qed}\isamarkupfalse%
\isanewline
\isacommand{qed}\isamarkupfalse%
%
\endisatagproof
{\isafoldproof}%
%
\isadelimproof
%
\endisadelimproof
%
\begin{isamarkuptext}%
De esta manera, mediante los anteriores lemas auxiliares podemos probar la primera
  implicación detalladamente en Isabelle.%
\end{isamarkuptext}\isamarkuptrue%
\isacommand{lemma}\isamarkupfalse%
\ pcp{\isacharunderscore}alt{\isadigit{1}}{\isacharcolon}\ \isanewline
\ \ \isakeyword{assumes}\ {\isachardoublequoteopen}pcp\ C{\isachardoublequoteclose}\isanewline
\ \ \isakeyword{shows}\ {\isachardoublequoteopen}{\isasymforall}S\ {\isasymin}\ C{\isachardot}\ {\isasymbottom}\ {\isasymnotin}\ S\isanewline
\ \ {\isasymand}\ {\isacharparenleft}{\isasymforall}k{\isachardot}\ Atom\ k\ {\isasymin}\ S\ {\isasymlongrightarrow}\ \isactrlbold {\isasymnot}\ {\isacharparenleft}Atom\ k{\isacharparenright}\ {\isasymin}\ S\ {\isasymlongrightarrow}\ False{\isacharparenright}\isanewline
\ \ {\isasymand}\ {\isacharparenleft}{\isasymforall}F\ G\ H{\isachardot}\ Con\ F\ G\ H\ {\isasymlongrightarrow}\ F\ {\isasymin}\ S\ {\isasymlongrightarrow}\ {\isacharbraceleft}G{\isacharcomma}H{\isacharbraceright}\ {\isasymunion}\ S\ {\isasymin}\ C{\isacharparenright}\isanewline
\ \ {\isasymand}\ {\isacharparenleft}{\isasymforall}F\ G\ H{\isachardot}\ Dis\ F\ G\ H\ {\isasymlongrightarrow}\ F\ {\isasymin}\ S\ {\isasymlongrightarrow}\ {\isacharbraceleft}G{\isacharbraceright}\ {\isasymunion}\ S\ {\isasymin}\ C\ {\isasymor}\ {\isacharbraceleft}H{\isacharbraceright}\ {\isasymunion}\ S\ {\isasymin}\ C{\isacharparenright}{\isachardoublequoteclose}\isanewline
%
\isadelimproof
%
\endisadelimproof
%
\isatagproof
\isacommand{proof}\isamarkupfalse%
\ {\isacharparenleft}rule\ ballI{\isacharparenright}\isanewline
\ \ \isacommand{fix}\isamarkupfalse%
\ S\isanewline
\ \ \isacommand{assume}\isamarkupfalse%
\ {\isachardoublequoteopen}S\ {\isasymin}\ C{\isachardoublequoteclose}\isanewline
\ \ \isacommand{have}\isamarkupfalse%
\ {\isachardoublequoteopen}{\isacharparenleft}{\isasymforall}S\ {\isasymin}\ C{\isachardot}\isanewline
\ \ {\isasymbottom}\ {\isasymnotin}\ S\isanewline
\ \ {\isasymand}\ {\isacharparenleft}{\isasymforall}k{\isachardot}\ Atom\ k\ {\isasymin}\ S\ {\isasymlongrightarrow}\ \isactrlbold {\isasymnot}\ {\isacharparenleft}Atom\ k{\isacharparenright}\ {\isasymin}\ S\ {\isasymlongrightarrow}\ False{\isacharparenright}\isanewline
\ \ {\isasymand}\ {\isacharparenleft}{\isasymforall}G\ H{\isachardot}\ G\ \isactrlbold {\isasymand}\ H\ {\isasymin}\ S\ {\isasymlongrightarrow}\ {\isacharbraceleft}G{\isacharcomma}H{\isacharbraceright}\ {\isasymunion}\ S\ {\isasymin}\ C{\isacharparenright}\isanewline
\ \ {\isasymand}\ {\isacharparenleft}{\isasymforall}G\ H{\isachardot}\ G\ \isactrlbold {\isasymor}\ H\ {\isasymin}\ S\ {\isasymlongrightarrow}\ {\isacharbraceleft}G{\isacharbraceright}\ {\isasymunion}\ S\ {\isasymin}\ C\ {\isasymor}\ {\isacharbraceleft}H{\isacharbraceright}\ {\isasymunion}\ S\ {\isasymin}\ C{\isacharparenright}\isanewline
\ \ {\isasymand}\ {\isacharparenleft}{\isasymforall}G\ H{\isachardot}\ G\ \isactrlbold {\isasymrightarrow}\ H\ {\isasymin}\ S\ {\isasymlongrightarrow}\ {\isacharbraceleft}\isactrlbold {\isasymnot}G{\isacharbraceright}\ {\isasymunion}\ S\ {\isasymin}\ C\ {\isasymor}\ {\isacharbraceleft}H{\isacharbraceright}\ {\isasymunion}\ S\ {\isasymin}\ C{\isacharparenright}\isanewline
\ \ {\isasymand}\ {\isacharparenleft}{\isasymforall}G{\isachardot}\ \isactrlbold {\isasymnot}\ {\isacharparenleft}\isactrlbold {\isasymnot}G{\isacharparenright}\ {\isasymin}\ S\ {\isasymlongrightarrow}\ {\isacharbraceleft}G{\isacharbraceright}\ {\isasymunion}\ S\ {\isasymin}\ C{\isacharparenright}\isanewline
\ \ {\isasymand}\ {\isacharparenleft}{\isasymforall}G\ H{\isachardot}\ \isactrlbold {\isasymnot}{\isacharparenleft}G\ \isactrlbold {\isasymand}\ H{\isacharparenright}\ {\isasymin}\ S\ {\isasymlongrightarrow}\ {\isacharbraceleft}\isactrlbold {\isasymnot}\ G{\isacharbraceright}\ {\isasymunion}\ S\ {\isasymin}\ C\ {\isasymor}\ {\isacharbraceleft}\isactrlbold {\isasymnot}\ H{\isacharbraceright}\ {\isasymunion}\ S\ {\isasymin}\ C{\isacharparenright}\isanewline
\ \ {\isasymand}\ {\isacharparenleft}{\isasymforall}G\ H{\isachardot}\ \isactrlbold {\isasymnot}{\isacharparenleft}G\ \isactrlbold {\isasymor}\ H{\isacharparenright}\ {\isasymin}\ S\ {\isasymlongrightarrow}\ {\isacharbraceleft}\isactrlbold {\isasymnot}\ G{\isacharcomma}\ \isactrlbold {\isasymnot}\ H{\isacharbraceright}\ {\isasymunion}\ S\ {\isasymin}\ C{\isacharparenright}\isanewline
\ \ {\isasymand}\ {\isacharparenleft}{\isasymforall}G\ H{\isachardot}\ \isactrlbold {\isasymnot}{\isacharparenleft}G\ \isactrlbold {\isasymrightarrow}\ H{\isacharparenright}\ {\isasymin}\ S\ {\isasymlongrightarrow}\ {\isacharbraceleft}G{\isacharcomma}\isactrlbold {\isasymnot}\ H{\isacharbraceright}\ {\isasymunion}\ S\ {\isasymin}\ C{\isacharparenright}{\isacharparenright}{\isachardoublequoteclose}\isanewline
\ \ \ \ \isacommand{using}\isamarkupfalse%
\ assms\ \isacommand{by}\isamarkupfalse%
\ {\isacharparenleft}simp\ only{\isacharcolon}\ pcp{\isacharunderscore}def{\isacharparenright}\isanewline
\ \ \isacommand{then}\isamarkupfalse%
\ \isacommand{have}\isamarkupfalse%
\ pcpS{\isacharcolon}{\isachardoublequoteopen}{\isasymbottom}\ {\isasymnotin}\ S\isanewline
\ \ {\isasymand}\ {\isacharparenleft}{\isasymforall}k{\isachardot}\ Atom\ k\ {\isasymin}\ S\ {\isasymlongrightarrow}\ \isactrlbold {\isasymnot}\ {\isacharparenleft}Atom\ k{\isacharparenright}\ {\isasymin}\ S\ {\isasymlongrightarrow}\ False{\isacharparenright}\isanewline
\ \ {\isasymand}\ {\isacharparenleft}{\isasymforall}G\ H{\isachardot}\ G\ \isactrlbold {\isasymand}\ H\ {\isasymin}\ S\ {\isasymlongrightarrow}\ {\isacharbraceleft}G{\isacharcomma}H{\isacharbraceright}\ {\isasymunion}\ S\ {\isasymin}\ C{\isacharparenright}\isanewline
\ \ {\isasymand}\ {\isacharparenleft}{\isasymforall}G\ H{\isachardot}\ G\ \isactrlbold {\isasymor}\ H\ {\isasymin}\ S\ {\isasymlongrightarrow}\ {\isacharbraceleft}G{\isacharbraceright}\ {\isasymunion}\ S\ {\isasymin}\ C\ {\isasymor}\ {\isacharbraceleft}H{\isacharbraceright}\ {\isasymunion}\ S\ {\isasymin}\ C{\isacharparenright}\isanewline
\ \ {\isasymand}\ {\isacharparenleft}{\isasymforall}G\ H{\isachardot}\ G\ \isactrlbold {\isasymrightarrow}\ H\ {\isasymin}\ S\ {\isasymlongrightarrow}\ {\isacharbraceleft}\isactrlbold {\isasymnot}G{\isacharbraceright}\ {\isasymunion}\ S\ {\isasymin}\ C\ {\isasymor}\ {\isacharbraceleft}H{\isacharbraceright}\ {\isasymunion}\ S\ {\isasymin}\ C{\isacharparenright}\isanewline
\ \ {\isasymand}\ {\isacharparenleft}{\isasymforall}G{\isachardot}\ \isactrlbold {\isasymnot}\ {\isacharparenleft}\isactrlbold {\isasymnot}G{\isacharparenright}\ {\isasymin}\ S\ {\isasymlongrightarrow}\ {\isacharbraceleft}G{\isacharbraceright}\ {\isasymunion}\ S\ {\isasymin}\ C{\isacharparenright}\isanewline
\ \ {\isasymand}\ {\isacharparenleft}{\isasymforall}G\ H{\isachardot}\ \isactrlbold {\isasymnot}{\isacharparenleft}G\ \isactrlbold {\isasymand}\ H{\isacharparenright}\ {\isasymin}\ S\ {\isasymlongrightarrow}\ {\isacharbraceleft}\isactrlbold {\isasymnot}\ G{\isacharbraceright}\ {\isasymunion}\ S\ {\isasymin}\ C\ {\isasymor}\ {\isacharbraceleft}\isactrlbold {\isasymnot}\ H{\isacharbraceright}\ {\isasymunion}\ S\ {\isasymin}\ C{\isacharparenright}\isanewline
\ \ {\isasymand}\ {\isacharparenleft}{\isasymforall}G\ H{\isachardot}\ \isactrlbold {\isasymnot}{\isacharparenleft}G\ \isactrlbold {\isasymor}\ H{\isacharparenright}\ {\isasymin}\ S\ {\isasymlongrightarrow}\ {\isacharbraceleft}\isactrlbold {\isasymnot}\ G{\isacharcomma}\ \isactrlbold {\isasymnot}\ H{\isacharbraceright}\ {\isasymunion}\ S\ {\isasymin}\ C{\isacharparenright}\isanewline
\ \ {\isasymand}\ {\isacharparenleft}{\isasymforall}G\ H{\isachardot}\ \isactrlbold {\isasymnot}{\isacharparenleft}G\ \isactrlbold {\isasymrightarrow}\ H{\isacharparenright}\ {\isasymin}\ S\ {\isasymlongrightarrow}\ {\isacharbraceleft}G{\isacharcomma}\isactrlbold {\isasymnot}\ H{\isacharbraceright}\ {\isasymunion}\ S\ {\isasymin}\ C{\isacharparenright}{\isachardoublequoteclose}\isanewline
\ \ \ \ \isacommand{using}\isamarkupfalse%
\ {\isacartoucheopen}S\ {\isasymin}\ C{\isacartoucheclose}\ \isacommand{by}\isamarkupfalse%
\ {\isacharparenleft}rule\ bspec{\isacharparenright}\isanewline
\ \ \isacommand{then}\isamarkupfalse%
\ \isacommand{have}\isamarkupfalse%
\ C{\isadigit{1}}{\isacharcolon}{\isachardoublequoteopen}{\isasymbottom}\ {\isasymnotin}\ S{\isachardoublequoteclose}\isanewline
\ \ \ \ \isacommand{by}\isamarkupfalse%
\ {\isacharparenleft}rule\ conjunct{\isadigit{1}}{\isacharparenright}\isanewline
\ \ \isacommand{have}\isamarkupfalse%
\ C{\isadigit{2}}{\isacharcolon}{\isachardoublequoteopen}{\isasymforall}k{\isachardot}\ Atom\ k\ {\isasymin}\ S\ {\isasymlongrightarrow}\ \isactrlbold {\isasymnot}\ {\isacharparenleft}Atom\ k{\isacharparenright}\ {\isasymin}\ S\ {\isasymlongrightarrow}\ False{\isachardoublequoteclose}\isanewline
\ \ \ \ \isacommand{using}\isamarkupfalse%
\ pcpS\ \isacommand{by}\isamarkupfalse%
\ {\isacharparenleft}iprover\ elim{\isacharcolon}\ conjunct{\isadigit{2}}\ conjunct{\isadigit{1}}{\isacharparenright}\isanewline
\ \ \isacommand{have}\isamarkupfalse%
\ C{\isadigit{3}}{\isacharcolon}{\isachardoublequoteopen}{\isasymforall}G\ H{\isachardot}\ G\ \isactrlbold {\isasymand}\ H\ {\isasymin}\ S\ {\isasymlongrightarrow}\ {\isacharbraceleft}G{\isacharcomma}H{\isacharbraceright}\ {\isasymunion}\ S\ {\isasymin}\ C{\isachardoublequoteclose}\isanewline
\ \ \ \ \isacommand{using}\isamarkupfalse%
\ pcpS\ \isacommand{by}\isamarkupfalse%
\ {\isacharparenleft}iprover\ elim{\isacharcolon}\ conjunct{\isadigit{2}}\ conjunct{\isadigit{1}}{\isacharparenright}\isanewline
\ \ \isacommand{have}\isamarkupfalse%
\ C{\isadigit{4}}{\isacharcolon}{\isachardoublequoteopen}{\isasymforall}G\ H{\isachardot}\ G\ \isactrlbold {\isasymor}\ H\ {\isasymin}\ S\ {\isasymlongrightarrow}\ {\isacharbraceleft}G{\isacharbraceright}\ {\isasymunion}\ S\ {\isasymin}\ C\ {\isasymor}\ {\isacharbraceleft}H{\isacharbraceright}\ {\isasymunion}\ S\ {\isasymin}\ C{\isachardoublequoteclose}\isanewline
\ \ \ \ \isacommand{using}\isamarkupfalse%
\ pcpS\ \isacommand{by}\isamarkupfalse%
\ {\isacharparenleft}iprover\ elim{\isacharcolon}\ conjunct{\isadigit{2}}\ conjunct{\isadigit{1}}{\isacharparenright}\isanewline
\ \ \isacommand{have}\isamarkupfalse%
\ C{\isadigit{5}}{\isacharcolon}{\isachardoublequoteopen}{\isasymforall}G\ H{\isachardot}\ G\ \isactrlbold {\isasymrightarrow}\ H\ {\isasymin}\ S\ {\isasymlongrightarrow}\ {\isacharbraceleft}\isactrlbold {\isasymnot}G{\isacharbraceright}\ {\isasymunion}\ S\ {\isasymin}\ C\ {\isasymor}\ {\isacharbraceleft}H{\isacharbraceright}\ {\isasymunion}\ S\ {\isasymin}\ C{\isachardoublequoteclose}\isanewline
\ \ \ \ \isacommand{using}\isamarkupfalse%
\ pcpS\ \isacommand{by}\isamarkupfalse%
\ {\isacharparenleft}iprover\ elim{\isacharcolon}\ conjunct{\isadigit{2}}\ conjunct{\isadigit{1}}{\isacharparenright}\isanewline
\ \ \isacommand{have}\isamarkupfalse%
\ C{\isadigit{6}}{\isacharcolon}{\isachardoublequoteopen}{\isasymforall}G{\isachardot}\ \isactrlbold {\isasymnot}\ {\isacharparenleft}\isactrlbold {\isasymnot}G{\isacharparenright}\ {\isasymin}\ S\ {\isasymlongrightarrow}\ {\isacharbraceleft}G{\isacharbraceright}\ {\isasymunion}\ S\ {\isasymin}\ C{\isachardoublequoteclose}\isanewline
\ \ \ \ \isacommand{using}\isamarkupfalse%
\ pcpS\ \isacommand{by}\isamarkupfalse%
\ {\isacharparenleft}iprover\ elim{\isacharcolon}\ conjunct{\isadigit{2}}\ conjunct{\isadigit{1}}{\isacharparenright}\isanewline
\ \ \isacommand{have}\isamarkupfalse%
\ C{\isadigit{7}}{\isacharcolon}{\isachardoublequoteopen}{\isasymforall}G\ H{\isachardot}\ \isactrlbold {\isasymnot}{\isacharparenleft}G\ \isactrlbold {\isasymand}\ H{\isacharparenright}\ {\isasymin}\ S\ {\isasymlongrightarrow}\ {\isacharbraceleft}\isactrlbold {\isasymnot}\ G{\isacharbraceright}\ {\isasymunion}\ S\ {\isasymin}\ C\ {\isasymor}\ {\isacharbraceleft}\isactrlbold {\isasymnot}\ H{\isacharbraceright}\ {\isasymunion}\ S\ {\isasymin}\ C{\isachardoublequoteclose}\isanewline
\ \ \ \ \isacommand{using}\isamarkupfalse%
\ pcpS\ \isacommand{by}\isamarkupfalse%
\ {\isacharparenleft}iprover\ elim{\isacharcolon}\ conjunct{\isadigit{2}}\ conjunct{\isadigit{1}}{\isacharparenright}\isanewline
\ \ \isacommand{have}\isamarkupfalse%
\ C{\isadigit{8}}{\isacharcolon}{\isachardoublequoteopen}{\isasymforall}G\ H{\isachardot}\ \isactrlbold {\isasymnot}{\isacharparenleft}G\ \isactrlbold {\isasymor}\ H{\isacharparenright}\ {\isasymin}\ S\ {\isasymlongrightarrow}\ {\isacharbraceleft}\isactrlbold {\isasymnot}\ G{\isacharcomma}\ \isactrlbold {\isasymnot}\ H{\isacharbraceright}\ {\isasymunion}\ S\ {\isasymin}\ C{\isachardoublequoteclose}\isanewline
\ \ \ \ \isacommand{using}\isamarkupfalse%
\ pcpS\ \isacommand{by}\isamarkupfalse%
\ {\isacharparenleft}iprover\ elim{\isacharcolon}\ conjunct{\isadigit{2}}\ conjunct{\isadigit{1}}{\isacharparenright}\isanewline
\ \ \isacommand{have}\isamarkupfalse%
\ C{\isadigit{9}}{\isacharcolon}{\isachardoublequoteopen}{\isasymforall}G\ H{\isachardot}\ \isactrlbold {\isasymnot}{\isacharparenleft}G\ \isactrlbold {\isasymrightarrow}\ H{\isacharparenright}\ {\isasymin}\ S\ {\isasymlongrightarrow}\ {\isacharbraceleft}G{\isacharcomma}\isactrlbold {\isasymnot}\ H{\isacharbraceright}\ {\isasymunion}\ S\ {\isasymin}\ C{\isachardoublequoteclose}\isanewline
\ \ \ \ \isacommand{using}\isamarkupfalse%
\ pcpS\ \isacommand{by}\isamarkupfalse%
\ {\isacharparenleft}iprover\ elim{\isacharcolon}\ conjunct{\isadigit{2}}{\isacharparenright}\isanewline
\ \ \isacommand{have}\isamarkupfalse%
\ {\isachardoublequoteopen}{\isacharparenleft}{\isasymforall}G\ H{\isachardot}\ G\ \isactrlbold {\isasymand}\ H\ {\isasymin}\ S\ {\isasymlongrightarrow}\ {\isacharbraceleft}G{\isacharcomma}H{\isacharbraceright}\ {\isasymunion}\ S\ {\isasymin}\ C{\isacharparenright}\isanewline
\ \ {\isasymand}\ {\isacharparenleft}{\isasymforall}G{\isachardot}\ \isactrlbold {\isasymnot}\ {\isacharparenleft}\isactrlbold {\isasymnot}G{\isacharparenright}\ {\isasymin}\ S\ {\isasymlongrightarrow}\ {\isacharbraceleft}G{\isacharbraceright}\ {\isasymunion}\ S\ {\isasymin}\ C{\isacharparenright}\isanewline
\ \ {\isasymand}\ {\isacharparenleft}{\isasymforall}G\ H{\isachardot}\ \isactrlbold {\isasymnot}{\isacharparenleft}G\ \isactrlbold {\isasymor}\ H{\isacharparenright}\ {\isasymin}\ S\ {\isasymlongrightarrow}\ {\isacharbraceleft}\isactrlbold {\isasymnot}\ G{\isacharcomma}\ \isactrlbold {\isasymnot}\ H{\isacharbraceright}\ {\isasymunion}\ S\ {\isasymin}\ C{\isacharparenright}\isanewline
\ \ {\isasymand}\ {\isacharparenleft}{\isasymforall}G\ H{\isachardot}\ \isactrlbold {\isasymnot}{\isacharparenleft}G\ \isactrlbold {\isasymrightarrow}\ H{\isacharparenright}\ {\isasymin}\ S\ {\isasymlongrightarrow}\ {\isacharbraceleft}G{\isacharcomma}\isactrlbold {\isasymnot}\ H{\isacharbraceright}\ {\isasymunion}\ S\ {\isasymin}\ C{\isacharparenright}{\isachardoublequoteclose}\isanewline
\ \ \ \ \isacommand{using}\isamarkupfalse%
\ C{\isadigit{3}}\ C{\isadigit{6}}\ C{\isadigit{8}}\ C{\isadigit{9}}\ \isacommand{by}\isamarkupfalse%
\ {\isacharparenleft}iprover\ intro{\isacharcolon}\ conjI{\isacharparenright}\isanewline
\ \ \isacommand{then}\isamarkupfalse%
\ \isacommand{have}\isamarkupfalse%
\ Con{\isacharcolon}{\isachardoublequoteopen}{\isasymforall}F\ G\ H{\isachardot}\ Con\ F\ G\ H\ {\isasymlongrightarrow}\ F\ {\isasymin}\ S\ {\isasymlongrightarrow}\ {\isacharbraceleft}G{\isacharcomma}H{\isacharbraceright}\ {\isasymunion}\ S\ {\isasymin}\ C{\isachardoublequoteclose}\isanewline
\ \ \ \ \isacommand{by}\isamarkupfalse%
\ {\isacharparenleft}rule\ pcp{\isacharunderscore}alt{\isadigit{1}}Con{\isacharparenright}\isanewline
\ \ \isacommand{have}\isamarkupfalse%
\ {\isachardoublequoteopen}{\isacharparenleft}{\isasymforall}G\ H{\isachardot}\ G\ \isactrlbold {\isasymor}\ H\ {\isasymin}\ S\ {\isasymlongrightarrow}\ {\isacharbraceleft}G{\isacharbraceright}\ {\isasymunion}\ S\ {\isasymin}\ C\ {\isasymor}\ {\isacharbraceleft}H{\isacharbraceright}\ {\isasymunion}\ S\ {\isasymin}\ C{\isacharparenright}\isanewline
\ \ {\isasymand}\ {\isacharparenleft}{\isasymforall}G\ H{\isachardot}\ G\ \isactrlbold {\isasymrightarrow}\ H\ {\isasymin}\ S\ {\isasymlongrightarrow}\ {\isacharbraceleft}\isactrlbold {\isasymnot}\ G{\isacharbraceright}\ {\isasymunion}\ S\ {\isasymin}\ C\ {\isasymor}\ {\isacharbraceleft}H{\isacharbraceright}\ {\isasymunion}\ S\ {\isasymin}\ C{\isacharparenright}\isanewline
\ \ {\isasymand}\ {\isacharparenleft}{\isasymforall}G{\isachardot}\ \isactrlbold {\isasymnot}\ {\isacharparenleft}\isactrlbold {\isasymnot}G{\isacharparenright}\ {\isasymin}\ S\ {\isasymlongrightarrow}\ {\isacharbraceleft}G{\isacharbraceright}\ {\isasymunion}\ S\ {\isasymin}\ C{\isacharparenright}\isanewline
\ \ {\isasymand}\ {\isacharparenleft}{\isasymforall}G\ H{\isachardot}\ \isactrlbold {\isasymnot}{\isacharparenleft}G\ \isactrlbold {\isasymand}\ H{\isacharparenright}\ {\isasymin}\ S\ {\isasymlongrightarrow}\ {\isacharbraceleft}\isactrlbold {\isasymnot}\ G{\isacharbraceright}\ {\isasymunion}\ S\ {\isasymin}\ C\ {\isasymor}\ {\isacharbraceleft}\isactrlbold {\isasymnot}\ H{\isacharbraceright}\ {\isasymunion}\ S\ {\isasymin}\ C{\isacharparenright}{\isachardoublequoteclose}\isanewline
\ \ \ \ \isacommand{using}\isamarkupfalse%
\ C{\isadigit{4}}\ C{\isadigit{5}}\ C{\isadigit{6}}\ C{\isadigit{7}}\ \isacommand{by}\isamarkupfalse%
\ {\isacharparenleft}iprover\ intro{\isacharcolon}\ conjI{\isacharparenright}\isanewline
\ \ \isacommand{then}\isamarkupfalse%
\ \isacommand{have}\isamarkupfalse%
\ Dis{\isacharcolon}{\isachardoublequoteopen}{\isasymforall}F\ G\ H{\isachardot}\ Dis\ F\ G\ H\ {\isasymlongrightarrow}\ F\ {\isasymin}\ S\ {\isasymlongrightarrow}\ {\isacharbraceleft}G{\isacharbraceright}\ {\isasymunion}\ S\ {\isasymin}\ C\ {\isasymor}\ {\isacharbraceleft}H{\isacharbraceright}\ {\isasymunion}\ S\ {\isasymin}\ C{\isachardoublequoteclose}\isanewline
\ \ \ \ \isacommand{by}\isamarkupfalse%
\ {\isacharparenleft}rule\ pcp{\isacharunderscore}alt{\isadigit{1}}Dis{\isacharparenright}\isanewline
\ \ \isacommand{thus}\isamarkupfalse%
\ {\isachardoublequoteopen}{\isasymbottom}\ {\isasymnotin}\ S\isanewline
\ \ {\isasymand}\ {\isacharparenleft}{\isasymforall}k{\isachardot}\ Atom\ k\ {\isasymin}\ S\ {\isasymlongrightarrow}\ \isactrlbold {\isasymnot}\ {\isacharparenleft}Atom\ k{\isacharparenright}\ {\isasymin}\ S\ {\isasymlongrightarrow}\ False{\isacharparenright}\isanewline
\ \ {\isasymand}\ {\isacharparenleft}{\isasymforall}F\ G\ H{\isachardot}\ Con\ F\ G\ H\ {\isasymlongrightarrow}\ F\ {\isasymin}\ S\ {\isasymlongrightarrow}\ {\isacharbraceleft}G{\isacharcomma}H{\isacharbraceright}\ {\isasymunion}\ S\ {\isasymin}\ C{\isacharparenright}\isanewline
\ \ {\isasymand}\ {\isacharparenleft}{\isasymforall}F\ G\ H{\isachardot}\ Dis\ F\ G\ H\ {\isasymlongrightarrow}\ F\ {\isasymin}\ S\ {\isasymlongrightarrow}\ {\isacharbraceleft}G{\isacharbraceright}\ {\isasymunion}\ S\ {\isasymin}\ C\ {\isasymor}\ {\isacharbraceleft}H{\isacharbraceright}\ {\isasymunion}\ S\ {\isasymin}\ C{\isacharparenright}{\isachardoublequoteclose}\isanewline
\ \ \ \ \isacommand{using}\isamarkupfalse%
\ C{\isadigit{1}}\ C{\isadigit{2}}\ Con\ Dis\ \isacommand{by}\isamarkupfalse%
\ {\isacharparenleft}iprover\ intro{\isacharcolon}\ conjI{\isacharparenright}\isanewline
\isacommand{qed}\isamarkupfalse%
%
\endisatagproof
{\isafoldproof}%
%
\isadelimproof
%
\endisadelimproof
%
\begin{isamarkuptext}%
Por otro lado, veamos la demostración detallada de la implicación recíproca de la
  equivalencia. Para ello, utilizaremos distintos lemas auxiliares para deducir cada una de las 
  condiciones de la definición de propiedad de consistencia proposicional a partir de las
  hipótesis sobre las fórmulas de tipo \isa{{\isasymalpha}} y \isa{{\isasymbeta}}. En primer lugar, veamos los lemas que se deducen
  condiciones a partir de la hipótesis referente a las fórmulas de tipo \isa{{\isasymalpha}}.%
\end{isamarkuptext}\isamarkuptrue%
\isacommand{lemma}\isamarkupfalse%
\ pcp{\isacharunderscore}alt{\isadigit{2}}Con{\isadigit{1}}{\isacharcolon}\isanewline
\ \ \isakeyword{assumes}\ {\isachardoublequoteopen}{\isasymforall}F\ G\ H{\isachardot}\ Con\ F\ G\ H\ {\isasymlongrightarrow}\ F\ {\isasymin}\ S\ {\isasymlongrightarrow}\ {\isacharbraceleft}G{\isacharcomma}H{\isacharbraceright}\ {\isasymunion}\ S\ {\isasymin}\ C{\isachardoublequoteclose}\isanewline
\ \ \isakeyword{shows}\ {\isachardoublequoteopen}{\isasymforall}G\ H{\isachardot}\ G\ \isactrlbold {\isasymand}\ H\ {\isasymin}\ S\ {\isasymlongrightarrow}\ {\isacharbraceleft}G{\isacharcomma}H{\isacharbraceright}\ {\isasymunion}\ S\ {\isasymin}\ C{\isachardoublequoteclose}\isanewline
%
\isadelimproof
%
\endisadelimproof
%
\isatagproof
\isacommand{proof}\isamarkupfalse%
\ {\isacharparenleft}rule\ allI{\isacharparenright}{\isacharplus}\isanewline
\ \ \isacommand{fix}\isamarkupfalse%
\ G\ H\isanewline
\ \ \isacommand{show}\isamarkupfalse%
\ {\isachardoublequoteopen}G\ \isactrlbold {\isasymand}\ H\ {\isasymin}\ S\ {\isasymlongrightarrow}\ {\isacharbraceleft}G{\isacharcomma}H{\isacharbraceright}\ {\isasymunion}\ S\ {\isasymin}\ C{\isachardoublequoteclose}\isanewline
\ \ \isacommand{proof}\isamarkupfalse%
\ {\isacharparenleft}rule\ impI{\isacharparenright}\isanewline
\ \ \ \ \isacommand{assume}\isamarkupfalse%
\ {\isachardoublequoteopen}G\ \isactrlbold {\isasymand}\ H\ {\isasymin}\ S{\isachardoublequoteclose}\isanewline
\ \ \ \ \isacommand{then}\isamarkupfalse%
\ \isacommand{have}\isamarkupfalse%
\ {\isachardoublequoteopen}Con\ {\isacharparenleft}G\ \isactrlbold {\isasymand}\ H{\isacharparenright}\ G\ H{\isachardoublequoteclose}\isanewline
\ \ \ \ \ \ \isacommand{by}\isamarkupfalse%
\ {\isacharparenleft}simp\ only{\isacharcolon}\ Con{\isachardot}intros{\isacharparenleft}{\isadigit{1}}{\isacharparenright}{\isacharparenright}\isanewline
\ \ \ \ \isacommand{let}\isamarkupfalse%
\ {\isacharquery}F{\isacharequal}{\isachardoublequoteopen}G\ \isactrlbold {\isasymand}\ H{\isachardoublequoteclose}\isanewline
\ \ \ \ \isacommand{have}\isamarkupfalse%
\ {\isachardoublequoteopen}Con\ {\isacharquery}F\ G\ H\ {\isasymlongrightarrow}\ {\isacharquery}F\ {\isasymin}\ S\ {\isasymlongrightarrow}\ {\isacharbraceleft}G{\isacharcomma}H{\isacharbraceright}\ {\isasymunion}\ S\ {\isasymin}\ C{\isachardoublequoteclose}\isanewline
\ \ \ \ \ \ \isacommand{using}\isamarkupfalse%
\ assms\ \isacommand{by}\isamarkupfalse%
\ {\isacharparenleft}iprover\ elim{\isacharcolon}\ allE{\isacharparenright}\isanewline
\ \ \ \ \isacommand{then}\isamarkupfalse%
\ \isacommand{have}\isamarkupfalse%
\ {\isachardoublequoteopen}{\isacharquery}F\ {\isasymin}\ S\ {\isasymlongrightarrow}\ {\isacharbraceleft}G{\isacharcomma}H{\isacharbraceright}\ {\isasymunion}\ S\ {\isasymin}\ C{\isachardoublequoteclose}\isanewline
\ \ \ \ \ \ \isacommand{using}\isamarkupfalse%
\ {\isacartoucheopen}Con\ {\isacharparenleft}G\ \isactrlbold {\isasymand}\ H{\isacharparenright}\ G\ H{\isacartoucheclose}\ \isacommand{by}\isamarkupfalse%
\ {\isacharparenleft}rule\ mp{\isacharparenright}\isanewline
\ \ \ \ \isacommand{thus}\isamarkupfalse%
\ {\isachardoublequoteopen}{\isacharbraceleft}G{\isacharcomma}H{\isacharbraceright}\ {\isasymunion}\ S\ {\isasymin}\ C{\isachardoublequoteclose}\isanewline
\ \ \ \ \ \ \isacommand{using}\isamarkupfalse%
\ {\isacartoucheopen}{\isacharparenleft}G\ \isactrlbold {\isasymand}\ H{\isacharparenright}\ {\isasymin}\ S{\isacartoucheclose}\ \isacommand{by}\isamarkupfalse%
\ {\isacharparenleft}rule\ mp{\isacharparenright}\isanewline
\ \ \isacommand{qed}\isamarkupfalse%
\isanewline
\isacommand{qed}\isamarkupfalse%
%
\endisatagproof
{\isafoldproof}%
%
\isadelimproof
\isanewline
%
\endisadelimproof
\isanewline
\isacommand{lemma}\isamarkupfalse%
\ pcp{\isacharunderscore}alt{\isadigit{2}}Con{\isadigit{2}}{\isacharcolon}\isanewline
\ \ \isakeyword{assumes}\ {\isachardoublequoteopen}{\isasymforall}F\ G\ H{\isachardot}\ Con\ F\ G\ H\ {\isasymlongrightarrow}\ F\ {\isasymin}\ S\ {\isasymlongrightarrow}\ {\isacharbraceleft}G{\isacharcomma}H{\isacharbraceright}\ {\isasymunion}\ S\ {\isasymin}\ C{\isachardoublequoteclose}\isanewline
\ \ \isakeyword{shows}\ {\isachardoublequoteopen}{\isasymforall}G{\isachardot}\ \isactrlbold {\isasymnot}\ {\isacharparenleft}\isactrlbold {\isasymnot}G{\isacharparenright}\ {\isasymin}\ S\ {\isasymlongrightarrow}\ {\isacharbraceleft}G{\isacharbraceright}\ {\isasymunion}\ S\ {\isasymin}\ C{\isachardoublequoteclose}\isanewline
%
\isadelimproof
%
\endisadelimproof
%
\isatagproof
\isacommand{proof}\isamarkupfalse%
\ {\isacharparenleft}rule\ allI{\isacharparenright}\isanewline
\ \ \isacommand{fix}\isamarkupfalse%
\ G\ \isanewline
\ \ \isacommand{show}\isamarkupfalse%
\ {\isachardoublequoteopen}\isactrlbold {\isasymnot}\ {\isacharparenleft}\isactrlbold {\isasymnot}G{\isacharparenright}\ {\isasymin}\ S\ {\isasymlongrightarrow}\ {\isacharbraceleft}G{\isacharbraceright}\ {\isasymunion}\ S\ {\isasymin}\ C{\isachardoublequoteclose}\isanewline
\ \ \isacommand{proof}\isamarkupfalse%
\ {\isacharparenleft}rule\ impI{\isacharparenright}\isanewline
\ \ \ \ \isacommand{assume}\isamarkupfalse%
\ {\isachardoublequoteopen}\isactrlbold {\isasymnot}{\isacharparenleft}\isactrlbold {\isasymnot}G{\isacharparenright}\ {\isasymin}\ S{\isachardoublequoteclose}\isanewline
\ \ \ \ \isacommand{then}\isamarkupfalse%
\ \isacommand{have}\isamarkupfalse%
\ {\isachardoublequoteopen}Con\ {\isacharparenleft}\isactrlbold {\isasymnot}{\isacharparenleft}\isactrlbold {\isasymnot}G{\isacharparenright}{\isacharparenright}\ G\ G{\isachardoublequoteclose}\isanewline
\ \ \ \ \ \ \isacommand{by}\isamarkupfalse%
\ {\isacharparenleft}simp\ only{\isacharcolon}\ Con{\isachardot}intros{\isacharparenleft}{\isadigit{4}}{\isacharparenright}{\isacharparenright}\isanewline
\ \ \ \ \isacommand{let}\isamarkupfalse%
\ {\isacharquery}F{\isacharequal}{\isachardoublequoteopen}\isactrlbold {\isasymnot}{\isacharparenleft}\isactrlbold {\isasymnot}\ G{\isacharparenright}{\isachardoublequoteclose}\isanewline
\ \ \ \ \isacommand{have}\isamarkupfalse%
\ {\isachardoublequoteopen}{\isasymforall}G\ H{\isachardot}\ Con\ {\isacharquery}F\ G\ H\ {\isasymlongrightarrow}\ {\isacharquery}F\ {\isasymin}\ S\ {\isasymlongrightarrow}\ {\isacharbraceleft}G{\isacharcomma}H{\isacharbraceright}\ {\isasymunion}\ S\ {\isasymin}\ C{\isachardoublequoteclose}\isanewline
\ \ \ \ \ \ \isacommand{using}\isamarkupfalse%
\ assms\ \isacommand{by}\isamarkupfalse%
\ {\isacharparenleft}rule\ allE{\isacharparenright}\isanewline
\ \ \ \ \isacommand{then}\isamarkupfalse%
\ \isacommand{have}\isamarkupfalse%
\ {\isachardoublequoteopen}{\isasymforall}H{\isachardot}\ Con\ {\isacharquery}F\ G\ H\ {\isasymlongrightarrow}\ {\isacharquery}F\ {\isasymin}\ S\ {\isasymlongrightarrow}\ {\isacharbraceleft}G{\isacharcomma}H{\isacharbraceright}\ {\isasymunion}\ S\ {\isasymin}\ C{\isachardoublequoteclose}\isanewline
\ \ \ \ \ \ \isacommand{by}\isamarkupfalse%
\ {\isacharparenleft}rule\ allE{\isacharparenright}\isanewline
\ \ \ \ \isacommand{then}\isamarkupfalse%
\ \isacommand{have}\isamarkupfalse%
\ {\isachardoublequoteopen}Con\ {\isacharquery}F\ G\ G\ {\isasymlongrightarrow}\ {\isacharquery}F\ {\isasymin}\ S\ {\isasymlongrightarrow}\ {\isacharbraceleft}G{\isacharcomma}G{\isacharbraceright}\ {\isasymunion}\ S\ {\isasymin}\ C{\isachardoublequoteclose}\isanewline
\ \ \ \ \ \ \isacommand{by}\isamarkupfalse%
\ {\isacharparenleft}rule\ allE{\isacharparenright}\isanewline
\ \ \ \ \isacommand{then}\isamarkupfalse%
\ \isacommand{have}\isamarkupfalse%
\ {\isachardoublequoteopen}{\isacharquery}F\ {\isasymin}\ S\ {\isasymlongrightarrow}\ {\isacharbraceleft}G{\isacharcomma}G{\isacharbraceright}\ {\isasymunion}\ S\ {\isasymin}\ C{\isachardoublequoteclose}\isanewline
\ \ \ \ \ \ \isacommand{using}\isamarkupfalse%
\ {\isacartoucheopen}Con\ {\isacharparenleft}\isactrlbold {\isasymnot}{\isacharparenleft}\isactrlbold {\isasymnot}G{\isacharparenright}{\isacharparenright}\ G\ G{\isacartoucheclose}\ \isacommand{by}\isamarkupfalse%
\ {\isacharparenleft}rule\ mp{\isacharparenright}\isanewline
\ \ \ \ \isacommand{then}\isamarkupfalse%
\ \isacommand{have}\isamarkupfalse%
\ {\isachardoublequoteopen}{\isacharbraceleft}G{\isacharcomma}G{\isacharbraceright}\ {\isasymunion}\ S\ {\isasymin}\ C{\isachardoublequoteclose}\isanewline
\ \ \ \ \ \ \isacommand{using}\isamarkupfalse%
\ {\isacartoucheopen}{\isacharparenleft}\isactrlbold {\isasymnot}{\isacharparenleft}\isactrlbold {\isasymnot}G{\isacharparenright}{\isacharparenright}\ {\isasymin}\ S{\isacartoucheclose}\ \isacommand{by}\isamarkupfalse%
\ {\isacharparenleft}rule\ mp{\isacharparenright}\isanewline
\ \ \ \ \isacommand{thus}\isamarkupfalse%
\ {\isachardoublequoteopen}{\isacharbraceleft}G{\isacharbraceright}\ {\isasymunion}\ S\ {\isasymin}\ C{\isachardoublequoteclose}\isanewline
\ \ \ \ \ \ \isacommand{by}\isamarkupfalse%
\ {\isacharparenleft}simp\ only{\isacharcolon}\ insert{\isacharunderscore}absorb{\isadigit{2}}{\isacharparenright}\isanewline
\ \ \isacommand{qed}\isamarkupfalse%
\isanewline
\isacommand{qed}\isamarkupfalse%
%
\endisatagproof
{\isafoldproof}%
%
\isadelimproof
\isanewline
%
\endisadelimproof
\isanewline
\isacommand{lemma}\isamarkupfalse%
\ pcp{\isacharunderscore}alt{\isadigit{2}}Con{\isadigit{3}}{\isacharcolon}\isanewline
\ \ \isakeyword{assumes}\ {\isachardoublequoteopen}{\isasymforall}F\ G\ H{\isachardot}\ Con\ F\ G\ H\ {\isasymlongrightarrow}\ F\ {\isasymin}\ S\ {\isasymlongrightarrow}\ {\isacharbraceleft}G{\isacharcomma}H{\isacharbraceright}\ {\isasymunion}\ S\ {\isasymin}\ C{\isachardoublequoteclose}\isanewline
\ \ \isakeyword{shows}\ {\isachardoublequoteopen}{\isasymforall}G\ H{\isachardot}\ \isactrlbold {\isasymnot}{\isacharparenleft}G\ \isactrlbold {\isasymor}\ H{\isacharparenright}\ {\isasymin}\ S\ {\isasymlongrightarrow}\ {\isacharbraceleft}\isactrlbold {\isasymnot}\ G{\isacharcomma}\ \isactrlbold {\isasymnot}\ H{\isacharbraceright}\ {\isasymunion}\ S\ {\isasymin}\ C{\isachardoublequoteclose}\isanewline
%
\isadelimproof
%
\endisadelimproof
%
\isatagproof
\isacommand{proof}\isamarkupfalse%
\ {\isacharparenleft}rule\ allI{\isacharparenright}{\isacharplus}\isanewline
\ \ \isacommand{fix}\isamarkupfalse%
\ G\ H\isanewline
\ \ \isacommand{show}\isamarkupfalse%
\ {\isachardoublequoteopen}\isactrlbold {\isasymnot}{\isacharparenleft}G\ \isactrlbold {\isasymor}\ H{\isacharparenright}\ {\isasymin}\ S\ {\isasymlongrightarrow}\ {\isacharbraceleft}\isactrlbold {\isasymnot}\ G{\isacharcomma}\ \isactrlbold {\isasymnot}\ H{\isacharbraceright}\ {\isasymunion}\ S\ {\isasymin}\ C{\isachardoublequoteclose}\isanewline
\ \ \isacommand{proof}\isamarkupfalse%
\ {\isacharparenleft}rule\ impI{\isacharparenright}\isanewline
\ \ \ \ \isacommand{assume}\isamarkupfalse%
\ {\isachardoublequoteopen}\isactrlbold {\isasymnot}{\isacharparenleft}G\ \isactrlbold {\isasymor}\ H{\isacharparenright}\ {\isasymin}\ S{\isachardoublequoteclose}\isanewline
\ \ \ \ \isacommand{then}\isamarkupfalse%
\ \isacommand{have}\isamarkupfalse%
\ {\isachardoublequoteopen}Con\ {\isacharparenleft}\isactrlbold {\isasymnot}{\isacharparenleft}G\ \isactrlbold {\isasymor}\ H{\isacharparenright}{\isacharparenright}\ {\isacharparenleft}\isactrlbold {\isasymnot}G{\isacharparenright}\ {\isacharparenleft}\isactrlbold {\isasymnot}H{\isacharparenright}{\isachardoublequoteclose}\isanewline
\ \ \ \ \ \ \isacommand{by}\isamarkupfalse%
\ {\isacharparenleft}simp\ only{\isacharcolon}\ Con{\isachardot}intros{\isacharparenleft}{\isadigit{2}}{\isacharparenright}{\isacharparenright}\isanewline
\ \ \ \ \isacommand{let}\isamarkupfalse%
\ {\isacharquery}F\ {\isacharequal}\ {\isachardoublequoteopen}\isactrlbold {\isasymnot}{\isacharparenleft}G\ \isactrlbold {\isasymor}\ H{\isacharparenright}{\isachardoublequoteclose}\isanewline
\ \ \ \ \isacommand{have}\isamarkupfalse%
\ {\isachardoublequoteopen}Con\ {\isacharquery}F\ {\isacharparenleft}\isactrlbold {\isasymnot}G{\isacharparenright}\ {\isacharparenleft}\isactrlbold {\isasymnot}H{\isacharparenright}\ {\isasymlongrightarrow}\ {\isacharquery}F\ {\isasymin}\ S\ {\isasymlongrightarrow}\ {\isacharbraceleft}\isactrlbold {\isasymnot}G{\isacharcomma}\isactrlbold {\isasymnot}H{\isacharbraceright}\ {\isasymunion}\ S\ {\isasymin}\ C{\isachardoublequoteclose}\isanewline
\ \ \ \ \ \ \isacommand{using}\isamarkupfalse%
\ assms\ \isacommand{by}\isamarkupfalse%
\ {\isacharparenleft}iprover\ elim{\isacharcolon}\ allE{\isacharparenright}\isanewline
\ \ \ \ \isacommand{then}\isamarkupfalse%
\ \isacommand{have}\isamarkupfalse%
\ {\isachardoublequoteopen}{\isacharquery}F\ {\isasymin}\ S\ {\isasymlongrightarrow}\ {\isacharbraceleft}\isactrlbold {\isasymnot}G{\isacharcomma}\isactrlbold {\isasymnot}H{\isacharbraceright}\ {\isasymunion}\ S\ {\isasymin}\ C{\isachardoublequoteclose}\isanewline
\ \ \ \ \ \ \isacommand{using}\isamarkupfalse%
\ {\isacartoucheopen}Con\ {\isacharparenleft}\isactrlbold {\isasymnot}{\isacharparenleft}G\ \isactrlbold {\isasymor}\ H{\isacharparenright}{\isacharparenright}\ {\isacharparenleft}\isactrlbold {\isasymnot}G{\isacharparenright}\ {\isacharparenleft}\isactrlbold {\isasymnot}H{\isacharparenright}{\isacartoucheclose}\ \isacommand{by}\isamarkupfalse%
\ {\isacharparenleft}rule\ mp{\isacharparenright}\isanewline
\ \ \ \ \isacommand{thus}\isamarkupfalse%
\ {\isachardoublequoteopen}{\isacharbraceleft}\isactrlbold {\isasymnot}G{\isacharcomma}\isactrlbold {\isasymnot}H{\isacharbraceright}\ {\isasymunion}\ S\ {\isasymin}\ C{\isachardoublequoteclose}\isanewline
\ \ \ \ \ \ \isacommand{using}\isamarkupfalse%
\ {\isacartoucheopen}\isactrlbold {\isasymnot}{\isacharparenleft}G\ \isactrlbold {\isasymor}\ H{\isacharparenright}\ {\isasymin}\ S{\isacartoucheclose}\ \isacommand{by}\isamarkupfalse%
\ {\isacharparenleft}rule\ mp{\isacharparenright}\isanewline
\ \ \isacommand{qed}\isamarkupfalse%
\isanewline
\isacommand{qed}\isamarkupfalse%
%
\endisatagproof
{\isafoldproof}%
%
\isadelimproof
\isanewline
%
\endisadelimproof
\isanewline
\isacommand{lemma}\isamarkupfalse%
\ pcp{\isacharunderscore}alt{\isadigit{2}}Con{\isadigit{4}}{\isacharcolon}\isanewline
\ \ \isakeyword{assumes}\ {\isachardoublequoteopen}{\isasymforall}F\ G\ H{\isachardot}\ Con\ F\ G\ H\ {\isasymlongrightarrow}\ F\ {\isasymin}\ S\ {\isasymlongrightarrow}\ {\isacharbraceleft}G{\isacharcomma}H{\isacharbraceright}\ {\isasymunion}\ S\ {\isasymin}\ C{\isachardoublequoteclose}\isanewline
\ \ \isakeyword{shows}\ {\isachardoublequoteopen}{\isasymforall}G\ H{\isachardot}\ \isactrlbold {\isasymnot}{\isacharparenleft}G\ \isactrlbold {\isasymrightarrow}\ H{\isacharparenright}\ {\isasymin}\ S\ {\isasymlongrightarrow}\ {\isacharbraceleft}G{\isacharcomma}\isactrlbold {\isasymnot}\ H{\isacharbraceright}\ {\isasymunion}\ S\ {\isasymin}\ C{\isachardoublequoteclose}\isanewline
%
\isadelimproof
%
\endisadelimproof
%
\isatagproof
\isacommand{proof}\isamarkupfalse%
\ {\isacharparenleft}rule\ allI{\isacharparenright}{\isacharplus}\isanewline
\ \ \isacommand{fix}\isamarkupfalse%
\ G\ H\isanewline
\ \ \isacommand{show}\isamarkupfalse%
\ {\isachardoublequoteopen}\isactrlbold {\isasymnot}{\isacharparenleft}G\ \isactrlbold {\isasymrightarrow}\ H{\isacharparenright}\ {\isasymin}\ S\ {\isasymlongrightarrow}\ {\isacharbraceleft}G{\isacharcomma}\isactrlbold {\isasymnot}\ H{\isacharbraceright}\ {\isasymunion}\ S\ {\isasymin}\ C{\isachardoublequoteclose}\isanewline
\ \ \isacommand{proof}\isamarkupfalse%
\ {\isacharparenleft}rule\ impI{\isacharparenright}\isanewline
\ \ \ \ \isacommand{assume}\isamarkupfalse%
\ {\isachardoublequoteopen}\isactrlbold {\isasymnot}{\isacharparenleft}G\ \isactrlbold {\isasymrightarrow}\ H{\isacharparenright}\ {\isasymin}\ S{\isachardoublequoteclose}\isanewline
\ \ \ \ \isacommand{then}\isamarkupfalse%
\ \isacommand{have}\isamarkupfalse%
\ {\isachardoublequoteopen}Con\ {\isacharparenleft}\isactrlbold {\isasymnot}{\isacharparenleft}G\ \isactrlbold {\isasymrightarrow}\ H{\isacharparenright}{\isacharparenright}\ G\ {\isacharparenleft}\isactrlbold {\isasymnot}H{\isacharparenright}{\isachardoublequoteclose}\isanewline
\ \ \ \ \ \ \isacommand{by}\isamarkupfalse%
\ {\isacharparenleft}simp\ only{\isacharcolon}\ Con{\isachardot}intros{\isacharparenleft}{\isadigit{3}}{\isacharparenright}{\isacharparenright}\isanewline
\ \ \ \ \isacommand{let}\isamarkupfalse%
\ {\isacharquery}F\ {\isacharequal}\ {\isachardoublequoteopen}\isactrlbold {\isasymnot}{\isacharparenleft}G\ \isactrlbold {\isasymrightarrow}\ H{\isacharparenright}{\isachardoublequoteclose}\isanewline
\ \ \ \ \isacommand{have}\isamarkupfalse%
\ {\isachardoublequoteopen}Con\ {\isacharquery}F\ G\ {\isacharparenleft}\isactrlbold {\isasymnot}H{\isacharparenright}\ {\isasymlongrightarrow}\ {\isacharquery}F\ {\isasymin}\ S\ {\isasymlongrightarrow}\ {\isacharbraceleft}G{\isacharcomma}\isactrlbold {\isasymnot}H{\isacharbraceright}\ {\isasymunion}\ S\ {\isasymin}\ C{\isachardoublequoteclose}\isanewline
\ \ \ \ \ \ \isacommand{using}\isamarkupfalse%
\ assms\ \isacommand{by}\isamarkupfalse%
\ {\isacharparenleft}iprover\ elim{\isacharcolon}\ allE{\isacharparenright}\isanewline
\ \ \ \ \isacommand{then}\isamarkupfalse%
\ \isacommand{have}\isamarkupfalse%
\ {\isachardoublequoteopen}{\isacharquery}F\ {\isasymin}\ S\ {\isasymlongrightarrow}\ {\isacharbraceleft}G{\isacharcomma}\isactrlbold {\isasymnot}H{\isacharbraceright}\ {\isasymunion}\ S\ {\isasymin}\ C{\isachardoublequoteclose}\ \ \isanewline
\ \ \ \ \ \ \isacommand{using}\isamarkupfalse%
\ {\isacartoucheopen}Con\ {\isacharparenleft}\isactrlbold {\isasymnot}{\isacharparenleft}G\ \isactrlbold {\isasymrightarrow}\ H{\isacharparenright}{\isacharparenright}\ G\ {\isacharparenleft}\isactrlbold {\isasymnot}H{\isacharparenright}{\isacartoucheclose}\ \isacommand{by}\isamarkupfalse%
\ {\isacharparenleft}rule\ mp{\isacharparenright}\isanewline
\ \ \ \ \isacommand{thus}\isamarkupfalse%
\ {\isachardoublequoteopen}{\isacharbraceleft}G{\isacharcomma}\isactrlbold {\isasymnot}H{\isacharbraceright}\ {\isasymunion}\ S\ {\isasymin}\ C{\isachardoublequoteclose}\isanewline
\ \ \ \ \ \ \isacommand{using}\isamarkupfalse%
\ {\isacartoucheopen}\isactrlbold {\isasymnot}{\isacharparenleft}G\ \isactrlbold {\isasymrightarrow}\ H{\isacharparenright}\ {\isasymin}\ S{\isacartoucheclose}\ \isacommand{by}\isamarkupfalse%
\ {\isacharparenleft}rule\ mp{\isacharparenright}\isanewline
\ \ \isacommand{qed}\isamarkupfalse%
\isanewline
\isacommand{qed}\isamarkupfalse%
%
\endisatagproof
{\isafoldproof}%
%
\isadelimproof
%
\endisadelimproof
%
\begin{isamarkuptext}%
Por otro lado, los siguientes lemas auxiliares prueban el resto de condiciones de la
  definición de propiedad de consistencia proposicional a partir de la hipótesis referente a 
  fórmulas de tipo \isa{{\isasymbeta}}.%
\end{isamarkuptext}\isamarkuptrue%
\isacommand{lemma}\isamarkupfalse%
\ pcp{\isacharunderscore}alt{\isadigit{2}}Dis{\isadigit{1}}{\isacharcolon}\isanewline
\ \ \isakeyword{assumes}\ {\isachardoublequoteopen}{\isasymforall}F\ G\ H{\isachardot}\ Dis\ F\ G\ H\ {\isasymlongrightarrow}\ F\ {\isasymin}\ S\ {\isasymlongrightarrow}\ {\isacharbraceleft}G{\isacharbraceright}\ {\isasymunion}\ S\ {\isasymin}\ C\ {\isasymor}\ {\isacharbraceleft}H{\isacharbraceright}\ {\isasymunion}\ S\ {\isasymin}\ C{\isachardoublequoteclose}\isanewline
\ \ \isakeyword{shows}\ {\isachardoublequoteopen}{\isasymforall}G\ H{\isachardot}\ G\ \isactrlbold {\isasymor}\ H\ {\isasymin}\ S\ {\isasymlongrightarrow}\ {\isacharbraceleft}G{\isacharbraceright}\ {\isasymunion}\ S\ {\isasymin}\ C\ {\isasymor}\ {\isacharbraceleft}H{\isacharbraceright}\ {\isasymunion}\ S\ {\isasymin}\ C{\isachardoublequoteclose}\isanewline
%
\isadelimproof
%
\endisadelimproof
%
\isatagproof
\isacommand{proof}\isamarkupfalse%
\ {\isacharparenleft}rule\ allI{\isacharparenright}{\isacharplus}\isanewline
\ \ \isacommand{fix}\isamarkupfalse%
\ G\ H\isanewline
\ \ \isacommand{show}\isamarkupfalse%
\ {\isachardoublequoteopen}G\ \isactrlbold {\isasymor}\ H\ {\isasymin}\ S\ {\isasymlongrightarrow}\ {\isacharbraceleft}G{\isacharbraceright}\ {\isasymunion}\ S\ {\isasymin}\ C\ {\isasymor}\ {\isacharbraceleft}H{\isacharbraceright}\ {\isasymunion}\ S\ {\isasymin}\ C{\isachardoublequoteclose}\isanewline
\ \ \isacommand{proof}\isamarkupfalse%
\ {\isacharparenleft}rule\ impI{\isacharparenright}\isanewline
\ \ \ \ \isacommand{assume}\isamarkupfalse%
\ {\isachardoublequoteopen}G\ \isactrlbold {\isasymor}\ H\ {\isasymin}\ S{\isachardoublequoteclose}\isanewline
\ \ \ \ \isacommand{then}\isamarkupfalse%
\ \isacommand{have}\isamarkupfalse%
\ {\isachardoublequoteopen}Dis\ {\isacharparenleft}G\ \isactrlbold {\isasymor}\ H{\isacharparenright}\ G\ H{\isachardoublequoteclose}\isanewline
\ \ \ \ \ \ \isacommand{by}\isamarkupfalse%
\ {\isacharparenleft}simp\ only{\isacharcolon}\ Dis{\isachardot}intros{\isacharparenleft}{\isadigit{1}}{\isacharparenright}{\isacharparenright}\isanewline
\ \ \ \ \isacommand{let}\isamarkupfalse%
\ {\isacharquery}F{\isacharequal}{\isachardoublequoteopen}G\ \isactrlbold {\isasymor}\ H{\isachardoublequoteclose}\isanewline
\ \ \ \ \isacommand{have}\isamarkupfalse%
\ {\isachardoublequoteopen}Dis\ {\isacharquery}F\ G\ H\ {\isasymlongrightarrow}\ {\isacharquery}F\ {\isasymin}\ S\ {\isasymlongrightarrow}\ {\isacharbraceleft}G{\isacharbraceright}\ {\isasymunion}\ S\ {\isasymin}\ C\ {\isasymor}\ {\isacharbraceleft}H{\isacharbraceright}\ {\isasymunion}\ S\ {\isasymin}\ C{\isachardoublequoteclose}\isanewline
\ \ \ \ \ \ \isacommand{using}\isamarkupfalse%
\ assms\ \isacommand{by}\isamarkupfalse%
\ {\isacharparenleft}iprover\ elim{\isacharcolon}\ allE{\isacharparenright}\isanewline
\ \ \ \ \isacommand{then}\isamarkupfalse%
\ \isacommand{have}\isamarkupfalse%
\ {\isachardoublequoteopen}{\isacharquery}F\ {\isasymin}\ S\ {\isasymlongrightarrow}\ {\isacharbraceleft}G{\isacharbraceright}\ {\isasymunion}\ S\ {\isasymin}\ C\ {\isasymor}\ {\isacharbraceleft}H{\isacharbraceright}\ {\isasymunion}\ S\ {\isasymin}\ C{\isachardoublequoteclose}\isanewline
\ \ \ \ \ \ \isacommand{using}\isamarkupfalse%
\ {\isacartoucheopen}Dis\ {\isacharparenleft}G\ \isactrlbold {\isasymor}\ H{\isacharparenright}\ G\ H{\isacartoucheclose}\ \isacommand{by}\isamarkupfalse%
\ {\isacharparenleft}rule\ mp{\isacharparenright}\isanewline
\ \ \ \ \isacommand{thus}\isamarkupfalse%
\ {\isachardoublequoteopen}{\isacharbraceleft}G{\isacharbraceright}\ {\isasymunion}\ S\ {\isasymin}\ C\ {\isasymor}\ {\isacharbraceleft}H{\isacharbraceright}\ {\isasymunion}\ S\ {\isasymin}\ C{\isachardoublequoteclose}\isanewline
\ \ \ \ \ \ \isacommand{using}\isamarkupfalse%
\ {\isacartoucheopen}{\isacharparenleft}G\ \isactrlbold {\isasymor}\ H{\isacharparenright}\ {\isasymin}\ S{\isacartoucheclose}\ \isacommand{by}\isamarkupfalse%
\ {\isacharparenleft}rule\ mp{\isacharparenright}\isanewline
\ \ \isacommand{qed}\isamarkupfalse%
\isanewline
\isacommand{qed}\isamarkupfalse%
%
\endisatagproof
{\isafoldproof}%
%
\isadelimproof
\isanewline
%
\endisadelimproof
\isanewline
\isacommand{lemma}\isamarkupfalse%
\ pcp{\isacharunderscore}alt{\isadigit{2}}Dis{\isadigit{2}}{\isacharcolon}\isanewline
\ \ \isakeyword{assumes}\ {\isachardoublequoteopen}{\isasymforall}F\ G\ H{\isachardot}\ Dis\ F\ G\ H\ {\isasymlongrightarrow}\ F\ {\isasymin}\ S\ {\isasymlongrightarrow}\ {\isacharbraceleft}G{\isacharbraceright}\ {\isasymunion}\ S\ {\isasymin}\ C\ {\isasymor}\ {\isacharbraceleft}H{\isacharbraceright}\ {\isasymunion}\ S\ {\isasymin}\ C{\isachardoublequoteclose}\isanewline
\ \ \isakeyword{shows}\ {\isachardoublequoteopen}{\isasymforall}G\ H{\isachardot}\ G\ \isactrlbold {\isasymrightarrow}\ H\ {\isasymin}\ S\ {\isasymlongrightarrow}\ {\isacharbraceleft}\isactrlbold {\isasymnot}\ G{\isacharbraceright}\ {\isasymunion}\ S\ {\isasymin}\ C\ {\isasymor}\ {\isacharbraceleft}H{\isacharbraceright}\ {\isasymunion}\ S\ {\isasymin}\ C{\isachardoublequoteclose}\isanewline
%
\isadelimproof
%
\endisadelimproof
%
\isatagproof
\isacommand{proof}\isamarkupfalse%
\ {\isacharparenleft}rule\ allI{\isacharparenright}{\isacharplus}\isanewline
\ \ \isacommand{fix}\isamarkupfalse%
\ G\ H\isanewline
\ \ \isacommand{show}\isamarkupfalse%
\ {\isachardoublequoteopen}G\ \isactrlbold {\isasymrightarrow}\ H\ {\isasymin}\ S\ {\isasymlongrightarrow}\ {\isacharbraceleft}\isactrlbold {\isasymnot}\ G{\isacharbraceright}\ {\isasymunion}\ S\ {\isasymin}\ C\ {\isasymor}\ {\isacharbraceleft}H{\isacharbraceright}\ {\isasymunion}\ S\ {\isasymin}\ C{\isachardoublequoteclose}\isanewline
\ \ \isacommand{proof}\isamarkupfalse%
\ {\isacharparenleft}rule\ impI{\isacharparenright}\isanewline
\ \ \ \ \isacommand{assume}\isamarkupfalse%
\ {\isachardoublequoteopen}G\ \isactrlbold {\isasymrightarrow}\ H\ {\isasymin}\ S{\isachardoublequoteclose}\isanewline
\ \ \ \ \isacommand{then}\isamarkupfalse%
\ \isacommand{have}\isamarkupfalse%
\ {\isachardoublequoteopen}Dis\ {\isacharparenleft}G\ \isactrlbold {\isasymrightarrow}\ H{\isacharparenright}\ {\isacharparenleft}\isactrlbold {\isasymnot}G{\isacharparenright}\ H{\isachardoublequoteclose}\isanewline
\ \ \ \ \ \ \isacommand{by}\isamarkupfalse%
\ {\isacharparenleft}simp\ only{\isacharcolon}\ Dis{\isachardot}intros{\isacharparenleft}{\isadigit{2}}{\isacharparenright}{\isacharparenright}\isanewline
\ \ \ \ \isacommand{let}\isamarkupfalse%
\ {\isacharquery}F{\isacharequal}{\isachardoublequoteopen}G\ \isactrlbold {\isasymrightarrow}\ H{\isachardoublequoteclose}\ \isanewline
\ \ \ \ \isacommand{have}\isamarkupfalse%
\ {\isachardoublequoteopen}Dis\ {\isacharquery}F\ {\isacharparenleft}\isactrlbold {\isasymnot}G{\isacharparenright}\ H\ {\isasymlongrightarrow}\ {\isacharquery}F\ {\isasymin}\ S\ {\isasymlongrightarrow}\ {\isacharbraceleft}\isactrlbold {\isasymnot}G{\isacharbraceright}\ {\isasymunion}\ S\ {\isasymin}\ C\ {\isasymor}\ {\isacharbraceleft}H{\isacharbraceright}\ {\isasymunion}\ S\ {\isasymin}\ C{\isachardoublequoteclose}\isanewline
\ \ \ \ \ \ \isacommand{using}\isamarkupfalse%
\ assms\ \isacommand{by}\isamarkupfalse%
\ {\isacharparenleft}iprover\ elim{\isacharcolon}\ allE{\isacharparenright}\isanewline
\ \ \ \ \isacommand{then}\isamarkupfalse%
\ \isacommand{have}\isamarkupfalse%
\ {\isachardoublequoteopen}{\isacharquery}F\ {\isasymin}\ S\ {\isasymlongrightarrow}\ {\isacharbraceleft}\isactrlbold {\isasymnot}G{\isacharbraceright}\ {\isasymunion}\ S\ {\isasymin}\ C\ {\isasymor}\ {\isacharbraceleft}H{\isacharbraceright}\ {\isasymunion}\ S\ {\isasymin}\ C{\isachardoublequoteclose}\isanewline
\ \ \ \ \ \ \isacommand{using}\isamarkupfalse%
\ {\isacartoucheopen}Dis\ {\isacharparenleft}G\ \isactrlbold {\isasymrightarrow}\ H{\isacharparenright}\ {\isacharparenleft}\isactrlbold {\isasymnot}G{\isacharparenright}\ H{\isacartoucheclose}\ \isacommand{by}\isamarkupfalse%
\ {\isacharparenleft}rule\ mp{\isacharparenright}\isanewline
\ \ \ \ \isacommand{thus}\isamarkupfalse%
\ {\isachardoublequoteopen}{\isacharbraceleft}\isactrlbold {\isasymnot}G{\isacharbraceright}\ {\isasymunion}\ S\ {\isasymin}\ C\ {\isasymor}\ {\isacharbraceleft}H{\isacharbraceright}\ {\isasymunion}\ S\ {\isasymin}\ C{\isachardoublequoteclose}\isanewline
\ \ \ \ \ \ \isacommand{using}\isamarkupfalse%
\ {\isacartoucheopen}{\isacharparenleft}G\ \isactrlbold {\isasymrightarrow}\ H{\isacharparenright}\ {\isasymin}\ S{\isacartoucheclose}\ \isacommand{by}\isamarkupfalse%
\ {\isacharparenleft}rule\ mp{\isacharparenright}\isanewline
\ \ \isacommand{qed}\isamarkupfalse%
\isanewline
\isacommand{qed}\isamarkupfalse%
%
\endisatagproof
{\isafoldproof}%
%
\isadelimproof
\isanewline
%
\endisadelimproof
\isanewline
\isacommand{lemma}\isamarkupfalse%
\ pcp{\isacharunderscore}alt{\isadigit{2}}Dis{\isadigit{3}}{\isacharcolon}\isanewline
\ \ \isakeyword{assumes}\ {\isachardoublequoteopen}{\isasymforall}F\ G\ H{\isachardot}\ Dis\ F\ G\ H\ {\isasymlongrightarrow}\ F\ {\isasymin}\ S\ {\isasymlongrightarrow}\ {\isacharbraceleft}G{\isacharbraceright}\ {\isasymunion}\ S\ {\isasymin}\ C\ {\isasymor}\ {\isacharbraceleft}H{\isacharbraceright}\ {\isasymunion}\ S\ {\isasymin}\ C{\isachardoublequoteclose}\isanewline
\ \ \isakeyword{shows}\ {\isachardoublequoteopen}{\isasymforall}G\ H{\isachardot}\ \isactrlbold {\isasymnot}{\isacharparenleft}G\ \isactrlbold {\isasymand}\ H{\isacharparenright}\ {\isasymin}\ S\ {\isasymlongrightarrow}\ {\isacharbraceleft}\isactrlbold {\isasymnot}\ G{\isacharbraceright}\ {\isasymunion}\ S\ {\isasymin}\ C\ {\isasymor}\ {\isacharbraceleft}\isactrlbold {\isasymnot}\ H{\isacharbraceright}\ {\isasymunion}\ S\ {\isasymin}\ C{\isachardoublequoteclose}\isanewline
%
\isadelimproof
%
\endisadelimproof
%
\isatagproof
\isacommand{proof}\isamarkupfalse%
\ {\isacharparenleft}rule\ allI{\isacharparenright}{\isacharplus}\isanewline
\ \ \isacommand{fix}\isamarkupfalse%
\ G\ H\isanewline
\ \ \isacommand{show}\isamarkupfalse%
\ {\isachardoublequoteopen}\isactrlbold {\isasymnot}{\isacharparenleft}G\ \isactrlbold {\isasymand}\ H{\isacharparenright}\ {\isasymin}\ S\ {\isasymlongrightarrow}\ {\isacharbraceleft}\isactrlbold {\isasymnot}\ G{\isacharbraceright}\ {\isasymunion}\ S\ {\isasymin}\ C\ {\isasymor}\ {\isacharbraceleft}\isactrlbold {\isasymnot}\ H{\isacharbraceright}\ {\isasymunion}\ S\ {\isasymin}\ C{\isachardoublequoteclose}\isanewline
\ \ \isacommand{proof}\isamarkupfalse%
\ {\isacharparenleft}rule\ impI{\isacharparenright}\isanewline
\ \ \ \ \isacommand{assume}\isamarkupfalse%
\ {\isachardoublequoteopen}\isactrlbold {\isasymnot}{\isacharparenleft}G\ \isactrlbold {\isasymand}\ H{\isacharparenright}\ {\isasymin}\ S{\isachardoublequoteclose}\isanewline
\ \ \ \ \isacommand{then}\isamarkupfalse%
\ \isacommand{have}\isamarkupfalse%
\ {\isachardoublequoteopen}Dis\ {\isacharparenleft}\isactrlbold {\isasymnot}{\isacharparenleft}G\ \isactrlbold {\isasymand}\ H{\isacharparenright}{\isacharparenright}\ {\isacharparenleft}\isactrlbold {\isasymnot}G{\isacharparenright}\ {\isacharparenleft}\isactrlbold {\isasymnot}H{\isacharparenright}{\isachardoublequoteclose}\isanewline
\ \ \ \ \ \ \isacommand{by}\isamarkupfalse%
\ {\isacharparenleft}simp\ only{\isacharcolon}\ Dis{\isachardot}intros{\isacharparenleft}{\isadigit{3}}{\isacharparenright}{\isacharparenright}\isanewline
\ \ \ \ \isacommand{let}\isamarkupfalse%
\ {\isacharquery}F{\isacharequal}{\isachardoublequoteopen}\isactrlbold {\isasymnot}{\isacharparenleft}G\ \isactrlbold {\isasymand}\ H{\isacharparenright}{\isachardoublequoteclose}\isanewline
\ \ \ \ \isacommand{have}\isamarkupfalse%
\ {\isachardoublequoteopen}Dis\ {\isacharquery}F\ {\isacharparenleft}\isactrlbold {\isasymnot}G{\isacharparenright}\ {\isacharparenleft}\isactrlbold {\isasymnot}H{\isacharparenright}\ {\isasymlongrightarrow}\ {\isacharquery}F\ {\isasymin}\ S\ {\isasymlongrightarrow}\ {\isacharbraceleft}\isactrlbold {\isasymnot}G{\isacharbraceright}\ {\isasymunion}\ S\ {\isasymin}\ C\ {\isasymor}\ {\isacharbraceleft}\isactrlbold {\isasymnot}H{\isacharbraceright}\ {\isasymunion}\ S\ {\isasymin}\ C{\isachardoublequoteclose}\isanewline
\ \ \ \ \ \ \isacommand{using}\isamarkupfalse%
\ assms\ \isacommand{by}\isamarkupfalse%
\ {\isacharparenleft}iprover\ elim{\isacharcolon}\ allE{\isacharparenright}\isanewline
\ \ \ \ \isacommand{then}\isamarkupfalse%
\ \isacommand{have}\isamarkupfalse%
\ {\isachardoublequoteopen}{\isacharquery}F\ {\isasymin}\ S\ {\isasymlongrightarrow}\ {\isacharbraceleft}\isactrlbold {\isasymnot}G{\isacharbraceright}\ {\isasymunion}\ S\ {\isasymin}\ C\ {\isasymor}\ {\isacharbraceleft}\isactrlbold {\isasymnot}H{\isacharbraceright}\ {\isasymunion}\ S\ {\isasymin}\ C{\isachardoublequoteclose}\isanewline
\ \ \ \ \ \ \isacommand{using}\isamarkupfalse%
\ {\isacartoucheopen}Dis\ {\isacharparenleft}\isactrlbold {\isasymnot}{\isacharparenleft}G\ \isactrlbold {\isasymand}\ H{\isacharparenright}{\isacharparenright}\ {\isacharparenleft}\isactrlbold {\isasymnot}G{\isacharparenright}\ {\isacharparenleft}\isactrlbold {\isasymnot}H{\isacharparenright}{\isacartoucheclose}\ \isacommand{by}\isamarkupfalse%
\ {\isacharparenleft}rule\ mp{\isacharparenright}\isanewline
\ \ \ \ \isacommand{thus}\isamarkupfalse%
\ {\isachardoublequoteopen}{\isacharbraceleft}\isactrlbold {\isasymnot}G{\isacharbraceright}\ {\isasymunion}\ S\ {\isasymin}\ C\ {\isasymor}\ {\isacharbraceleft}\isactrlbold {\isasymnot}H{\isacharbraceright}\ {\isasymunion}\ S\ {\isasymin}\ C{\isachardoublequoteclose}\isanewline
\ \ \ \ \ \ \isacommand{using}\isamarkupfalse%
\ {\isacartoucheopen}\isactrlbold {\isasymnot}{\isacharparenleft}G\ \isactrlbold {\isasymand}\ H{\isacharparenright}\ {\isasymin}\ S{\isacartoucheclose}\ \isacommand{by}\isamarkupfalse%
\ {\isacharparenleft}rule\ mp{\isacharparenright}\isanewline
\ \ \isacommand{qed}\isamarkupfalse%
\isanewline
\isacommand{qed}\isamarkupfalse%
%
\endisatagproof
{\isafoldproof}%
%
\isadelimproof
%
\endisadelimproof
%
\begin{isamarkuptext}%
De este modo, procedemos a la demostración detallada de esta implicación en Isabelle.%
\end{isamarkuptext}\isamarkuptrue%
\isacommand{lemma}\isamarkupfalse%
\ pcp{\isacharunderscore}alt{\isadigit{2}}{\isacharcolon}\ \isanewline
\ \ \isakeyword{assumes}\ {\isachardoublequoteopen}{\isasymforall}S\ {\isasymin}\ C{\isachardot}\ {\isasymbottom}\ {\isasymnotin}\ S\isanewline
{\isasymand}\ {\isacharparenleft}{\isasymforall}k{\isachardot}\ Atom\ k\ {\isasymin}\ S\ {\isasymlongrightarrow}\ \isactrlbold {\isasymnot}\ {\isacharparenleft}Atom\ k{\isacharparenright}\ {\isasymin}\ S\ {\isasymlongrightarrow}\ False{\isacharparenright}\isanewline
{\isasymand}\ {\isacharparenleft}{\isasymforall}F\ G\ H{\isachardot}\ Con\ F\ G\ H\ {\isasymlongrightarrow}\ F\ {\isasymin}\ S\ {\isasymlongrightarrow}\ {\isacharbraceleft}G{\isacharcomma}H{\isacharbraceright}\ {\isasymunion}\ S\ {\isasymin}\ C{\isacharparenright}\isanewline
{\isasymand}\ {\isacharparenleft}{\isasymforall}F\ G\ H{\isachardot}\ Dis\ F\ G\ H\ {\isasymlongrightarrow}\ F\ {\isasymin}\ S\ {\isasymlongrightarrow}\ {\isacharbraceleft}G{\isacharbraceright}\ {\isasymunion}\ S\ {\isasymin}\ C\ {\isasymor}\ {\isacharbraceleft}H{\isacharbraceright}\ {\isasymunion}\ S\ {\isasymin}\ C{\isacharparenright}{\isachardoublequoteclose}\isanewline
\ \ \isakeyword{shows}\ {\isachardoublequoteopen}pcp\ C{\isachardoublequoteclose}\isanewline
%
\isadelimproof
\ \ %
\endisadelimproof
%
\isatagproof
\isacommand{unfolding}\isamarkupfalse%
\ pcp{\isacharunderscore}def\isanewline
\isacommand{proof}\isamarkupfalse%
\ {\isacharparenleft}rule\ ballI{\isacharparenright}\isanewline
\ \ \isacommand{fix}\isamarkupfalse%
\ S\isanewline
\ \ \isacommand{assume}\isamarkupfalse%
\ {\isachardoublequoteopen}S\ {\isasymin}\ C{\isachardoublequoteclose}\isanewline
\ \ \isacommand{have}\isamarkupfalse%
\ H{\isacharcolon}{\isachardoublequoteopen}{\isasymbottom}\ {\isasymnotin}\ S\isanewline
\ \ \ \ {\isasymand}\ {\isacharparenleft}{\isasymforall}k{\isachardot}\ Atom\ k\ {\isasymin}\ S\ {\isasymlongrightarrow}\ \isactrlbold {\isasymnot}\ {\isacharparenleft}Atom\ k{\isacharparenright}\ {\isasymin}\ S\ {\isasymlongrightarrow}\ False{\isacharparenright}\isanewline
\ \ \ \ {\isasymand}\ {\isacharparenleft}{\isasymforall}F\ G\ H{\isachardot}\ Con\ F\ G\ H\ {\isasymlongrightarrow}\ F\ {\isasymin}\ S\ {\isasymlongrightarrow}\ {\isacharbraceleft}G{\isacharcomma}H{\isacharbraceright}\ {\isasymunion}\ S\ {\isasymin}\ C{\isacharparenright}\isanewline
\ \ \ \ {\isasymand}\ {\isacharparenleft}{\isasymforall}F\ G\ H{\isachardot}\ Dis\ F\ G\ H\ {\isasymlongrightarrow}\ F\ {\isasymin}\ S\ {\isasymlongrightarrow}\ {\isacharbraceleft}G{\isacharbraceright}\ {\isasymunion}\ S\ {\isasymin}\ C\ {\isasymor}\ {\isacharbraceleft}H{\isacharbraceright}\ {\isasymunion}\ S\ {\isasymin}\ C{\isacharparenright}{\isachardoublequoteclose}\isanewline
\ \ \ \ \isacommand{using}\isamarkupfalse%
\ assms\ {\isacartoucheopen}S\ {\isasymin}\ C{\isacartoucheclose}\ \isacommand{by}\isamarkupfalse%
\ {\isacharparenleft}rule\ bspec{\isacharparenright}\isanewline
\ \ \isacommand{then}\isamarkupfalse%
\ \isacommand{have}\isamarkupfalse%
\ Con{\isacharcolon}{\isachardoublequoteopen}{\isasymforall}F\ G\ H{\isachardot}\ Con\ F\ G\ H\ {\isasymlongrightarrow}\ F\ {\isasymin}\ S\ {\isasymlongrightarrow}\ {\isacharbraceleft}G{\isacharcomma}H{\isacharbraceright}\ {\isasymunion}\ S\ {\isasymin}\ C{\isachardoublequoteclose}\isanewline
\ \ \ \ \isacommand{by}\isamarkupfalse%
\ {\isacharparenleft}iprover\ elim{\isacharcolon}\ conjunct{\isadigit{1}}\ conjunct{\isadigit{2}}{\isacharparenright}\isanewline
\ \ \isacommand{have}\isamarkupfalse%
\ Dis{\isacharcolon}{\isachardoublequoteopen}{\isasymforall}F\ G\ H{\isachardot}\ Dis\ F\ G\ H\ {\isasymlongrightarrow}\ F\ {\isasymin}\ S\ {\isasymlongrightarrow}\ {\isacharbraceleft}G{\isacharbraceright}\ {\isasymunion}\ S\ {\isasymin}\ C\ {\isasymor}\ {\isacharbraceleft}H{\isacharbraceright}\ {\isasymunion}\ S\ {\isasymin}\ C{\isachardoublequoteclose}\isanewline
\ \ \ \ \isacommand{using}\isamarkupfalse%
\ H\ \isacommand{by}\isamarkupfalse%
\ {\isacharparenleft}iprover\ elim{\isacharcolon}\ conjunct{\isadigit{1}}\ conjunct{\isadigit{2}}{\isacharparenright}\isanewline
\ \ \isacommand{have}\isamarkupfalse%
\ {\isadigit{1}}{\isacharcolon}{\isachardoublequoteopen}{\isasymbottom}\ {\isasymnotin}\ S\isanewline
\ \ \ \ {\isasymand}\ {\isacharparenleft}{\isasymforall}k{\isachardot}\ Atom\ k\ {\isasymin}\ S\ {\isasymlongrightarrow}\ \isactrlbold {\isasymnot}\ {\isacharparenleft}Atom\ k{\isacharparenright}\ {\isasymin}\ S\ {\isasymlongrightarrow}\ False{\isacharparenright}{\isachardoublequoteclose}\isanewline
\ \ \ \ \isacommand{using}\isamarkupfalse%
\ H\ \isacommand{by}\isamarkupfalse%
\ {\isacharparenleft}iprover\ elim{\isacharcolon}\ conjunct{\isadigit{1}}{\isacharparenright}\isanewline
\ \ \isacommand{have}\isamarkupfalse%
\ {\isadigit{2}}{\isacharcolon}{\isachardoublequoteopen}{\isasymforall}G\ H{\isachardot}\ G\ \isactrlbold {\isasymand}\ H\ {\isasymin}\ S\ {\isasymlongrightarrow}\ {\isacharbraceleft}G{\isacharcomma}H{\isacharbraceright}\ {\isasymunion}\ S\ {\isasymin}\ C{\isachardoublequoteclose}\isanewline
\ \ \ \ \isacommand{using}\isamarkupfalse%
\ Con\ \isacommand{by}\isamarkupfalse%
\ {\isacharparenleft}rule\ pcp{\isacharunderscore}alt{\isadigit{2}}Con{\isadigit{1}}{\isacharparenright}\isanewline
\ \ \isacommand{have}\isamarkupfalse%
\ {\isadigit{3}}{\isacharcolon}{\isachardoublequoteopen}{\isasymforall}G\ H{\isachardot}\ G\ \isactrlbold {\isasymor}\ H\ {\isasymin}\ S\ {\isasymlongrightarrow}\ {\isacharbraceleft}G{\isacharbraceright}\ {\isasymunion}\ S\ {\isasymin}\ C\ {\isasymor}\ {\isacharbraceleft}H{\isacharbraceright}\ {\isasymunion}\ S\ {\isasymin}\ C{\isachardoublequoteclose}\isanewline
\ \ \ \ \isacommand{using}\isamarkupfalse%
\ Dis\ \isacommand{by}\isamarkupfalse%
\ {\isacharparenleft}rule\ pcp{\isacharunderscore}alt{\isadigit{2}}Dis{\isadigit{1}}{\isacharparenright}\isanewline
\ \ \isacommand{have}\isamarkupfalse%
\ {\isadigit{4}}{\isacharcolon}{\isachardoublequoteopen}{\isasymforall}G\ H{\isachardot}\ G\ \isactrlbold {\isasymrightarrow}\ H\ {\isasymin}\ S\ {\isasymlongrightarrow}\ {\isacharbraceleft}\isactrlbold {\isasymnot}G{\isacharbraceright}\ {\isasymunion}\ S\ {\isasymin}\ C\ {\isasymor}\ {\isacharbraceleft}H{\isacharbraceright}\ {\isasymunion}\ S\ {\isasymin}\ C{\isachardoublequoteclose}\isanewline
\ \ \ \ \isacommand{using}\isamarkupfalse%
\ Dis\ \isacommand{by}\isamarkupfalse%
\ {\isacharparenleft}rule\ pcp{\isacharunderscore}alt{\isadigit{2}}Dis{\isadigit{2}}{\isacharparenright}\isanewline
\ \ \isacommand{have}\isamarkupfalse%
\ {\isadigit{5}}{\isacharcolon}{\isachardoublequoteopen}{\isasymforall}G{\isachardot}\ \isactrlbold {\isasymnot}\ {\isacharparenleft}\isactrlbold {\isasymnot}G{\isacharparenright}\ {\isasymin}\ S\ {\isasymlongrightarrow}\ {\isacharbraceleft}G{\isacharbraceright}\ {\isasymunion}\ S\ {\isasymin}\ C{\isachardoublequoteclose}\isanewline
\ \ \ \ \isacommand{using}\isamarkupfalse%
\ Con\ \isacommand{by}\isamarkupfalse%
\ {\isacharparenleft}rule\ pcp{\isacharunderscore}alt{\isadigit{2}}Con{\isadigit{2}}{\isacharparenright}\isanewline
\ \ \isacommand{have}\isamarkupfalse%
\ {\isadigit{6}}{\isacharcolon}{\isachardoublequoteopen}{\isasymforall}G\ H{\isachardot}\ \isactrlbold {\isasymnot}{\isacharparenleft}G\ \isactrlbold {\isasymand}\ H{\isacharparenright}\ {\isasymin}\ S\ {\isasymlongrightarrow}\ {\isacharbraceleft}\isactrlbold {\isasymnot}\ G{\isacharbraceright}\ {\isasymunion}\ S\ {\isasymin}\ C\ {\isasymor}\ {\isacharbraceleft}\isactrlbold {\isasymnot}\ H{\isacharbraceright}\ {\isasymunion}\ S\ {\isasymin}\ C{\isachardoublequoteclose}\isanewline
\ \ \ \ \isacommand{using}\isamarkupfalse%
\ Dis\ \isacommand{by}\isamarkupfalse%
\ {\isacharparenleft}rule\ pcp{\isacharunderscore}alt{\isadigit{2}}Dis{\isadigit{3}}{\isacharparenright}\isanewline
\ \ \isacommand{have}\isamarkupfalse%
\ {\isadigit{7}}{\isacharcolon}{\isachardoublequoteopen}{\isasymforall}G\ H{\isachardot}\ \isactrlbold {\isasymnot}{\isacharparenleft}G\ \isactrlbold {\isasymor}\ H{\isacharparenright}\ {\isasymin}\ S\ {\isasymlongrightarrow}\ {\isacharbraceleft}\isactrlbold {\isasymnot}\ G{\isacharcomma}\ \isactrlbold {\isasymnot}\ H{\isacharbraceright}\ {\isasymunion}\ S\ {\isasymin}\ C{\isachardoublequoteclose}\isanewline
\ \ \ \ \isacommand{using}\isamarkupfalse%
\ Con\ \isacommand{by}\isamarkupfalse%
\ {\isacharparenleft}rule\ pcp{\isacharunderscore}alt{\isadigit{2}}Con{\isadigit{3}}{\isacharparenright}\isanewline
\ \ \isacommand{have}\isamarkupfalse%
\ {\isadigit{8}}{\isacharcolon}{\isachardoublequoteopen}{\isasymforall}G\ H{\isachardot}\ \isactrlbold {\isasymnot}{\isacharparenleft}G\ \isactrlbold {\isasymrightarrow}\ H{\isacharparenright}\ {\isasymin}\ S\ {\isasymlongrightarrow}\ {\isacharbraceleft}G{\isacharcomma}\isactrlbold {\isasymnot}\ H{\isacharbraceright}\ {\isasymunion}\ S\ {\isasymin}\ C{\isachardoublequoteclose}\isanewline
\ \ \ \ \isacommand{using}\isamarkupfalse%
\ Con\ \isacommand{by}\isamarkupfalse%
\ {\isacharparenleft}rule\ pcp{\isacharunderscore}alt{\isadigit{2}}Con{\isadigit{4}}{\isacharparenright}\isanewline
\ \ \isacommand{have}\isamarkupfalse%
\ A{\isacharcolon}{\isachardoublequoteopen}{\isasymbottom}\ {\isasymnotin}\ S\isanewline
\ \ \ \ {\isasymand}\ {\isacharparenleft}{\isasymforall}k{\isachardot}\ Atom\ k\ {\isasymin}\ S\ {\isasymlongrightarrow}\ \isactrlbold {\isasymnot}\ {\isacharparenleft}Atom\ k{\isacharparenright}\ {\isasymin}\ S\ {\isasymlongrightarrow}\ False{\isacharparenright}\isanewline
\ \ \ \ {\isasymand}\ {\isacharparenleft}{\isasymforall}G\ H{\isachardot}\ G\ \isactrlbold {\isasymand}\ H\ {\isasymin}\ S\ {\isasymlongrightarrow}\ {\isacharbraceleft}G{\isacharcomma}H{\isacharbraceright}\ {\isasymunion}\ S\ {\isasymin}\ C{\isacharparenright}\isanewline
\ \ \ \ {\isasymand}\ {\isacharparenleft}{\isasymforall}G\ H{\isachardot}\ G\ \isactrlbold {\isasymor}\ H\ {\isasymin}\ S\ {\isasymlongrightarrow}\ {\isacharbraceleft}G{\isacharbraceright}\ {\isasymunion}\ S\ {\isasymin}\ C\ {\isasymor}\ {\isacharbraceleft}H{\isacharbraceright}\ {\isasymunion}\ S\ {\isasymin}\ C{\isacharparenright}\isanewline
\ \ \ \ {\isasymand}\ {\isacharparenleft}{\isasymforall}G\ H{\isachardot}\ G\ \isactrlbold {\isasymrightarrow}\ H\ {\isasymin}\ S\ {\isasymlongrightarrow}\ {\isacharbraceleft}\isactrlbold {\isasymnot}G{\isacharbraceright}\ {\isasymunion}\ S\ {\isasymin}\ C\ {\isasymor}\ {\isacharbraceleft}H{\isacharbraceright}\ {\isasymunion}\ S\ {\isasymin}\ C{\isacharparenright}{\isachardoublequoteclose}\isanewline
\ \ \ \ \isacommand{using}\isamarkupfalse%
\ {\isadigit{1}}\ {\isadigit{2}}\ {\isadigit{3}}\ {\isadigit{4}}\ \isacommand{by}\isamarkupfalse%
\ {\isacharparenleft}iprover\ intro{\isacharcolon}\ conjI{\isacharparenright}\isanewline
\ \ \isacommand{have}\isamarkupfalse%
\ B{\isacharcolon}{\isachardoublequoteopen}{\isacharparenleft}{\isasymforall}G{\isachardot}\ \isactrlbold {\isasymnot}\ {\isacharparenleft}\isactrlbold {\isasymnot}G{\isacharparenright}\ {\isasymin}\ S\ {\isasymlongrightarrow}\ {\isacharbraceleft}G{\isacharbraceright}\ {\isasymunion}\ S\ {\isasymin}\ C{\isacharparenright}\isanewline
\ \ \ \ {\isasymand}\ {\isacharparenleft}{\isasymforall}G\ H{\isachardot}\ \isactrlbold {\isasymnot}{\isacharparenleft}G\ \isactrlbold {\isasymand}\ H{\isacharparenright}\ {\isasymin}\ S\ {\isasymlongrightarrow}\ {\isacharbraceleft}\isactrlbold {\isasymnot}\ G{\isacharbraceright}\ {\isasymunion}\ S\ {\isasymin}\ C\ {\isasymor}\ {\isacharbraceleft}\isactrlbold {\isasymnot}\ H{\isacharbraceright}\ {\isasymunion}\ S\ {\isasymin}\ C{\isacharparenright}\isanewline
\ \ \ \ {\isasymand}\ {\isacharparenleft}{\isasymforall}G\ H{\isachardot}\ \isactrlbold {\isasymnot}{\isacharparenleft}G\ \isactrlbold {\isasymor}\ H{\isacharparenright}\ {\isasymin}\ S\ {\isasymlongrightarrow}\ {\isacharbraceleft}\isactrlbold {\isasymnot}\ G{\isacharcomma}\ \isactrlbold {\isasymnot}\ H{\isacharbraceright}\ {\isasymunion}\ S\ {\isasymin}\ C{\isacharparenright}\isanewline
\ \ \ \ {\isasymand}\ {\isacharparenleft}{\isasymforall}G\ H{\isachardot}\ \isactrlbold {\isasymnot}{\isacharparenleft}G\ \isactrlbold {\isasymrightarrow}\ H{\isacharparenright}\ {\isasymin}\ S\ {\isasymlongrightarrow}\ {\isacharbraceleft}G{\isacharcomma}\isactrlbold {\isasymnot}\ H{\isacharbraceright}\ {\isasymunion}\ S\ {\isasymin}\ C{\isacharparenright}{\isachardoublequoteclose}\isanewline
\ \ \ \ \isacommand{using}\isamarkupfalse%
\ {\isadigit{5}}\ {\isadigit{6}}\ {\isadigit{7}}\ {\isadigit{8}}\ \isacommand{by}\isamarkupfalse%
\ {\isacharparenleft}iprover\ intro{\isacharcolon}\ conjI{\isacharparenright}\isanewline
\ \ \isacommand{have}\isamarkupfalse%
\ {\isachardoublequoteopen}{\isacharparenleft}{\isasymbottom}\ {\isasymnotin}\ S\isanewline
\ \ \ \ {\isasymand}\ {\isacharparenleft}{\isasymforall}k{\isachardot}\ Atom\ k\ {\isasymin}\ S\ {\isasymlongrightarrow}\ \isactrlbold {\isasymnot}\ {\isacharparenleft}Atom\ k{\isacharparenright}\ {\isasymin}\ S\ {\isasymlongrightarrow}\ False{\isacharparenright}\isanewline
\ \ \ \ {\isasymand}\ {\isacharparenleft}{\isasymforall}G\ H{\isachardot}\ G\ \isactrlbold {\isasymand}\ H\ {\isasymin}\ S\ {\isasymlongrightarrow}\ {\isacharbraceleft}G{\isacharcomma}H{\isacharbraceright}\ {\isasymunion}\ S\ {\isasymin}\ C{\isacharparenright}\isanewline
\ \ \ \ {\isasymand}\ {\isacharparenleft}{\isasymforall}G\ H{\isachardot}\ G\ \isactrlbold {\isasymor}\ H\ {\isasymin}\ S\ {\isasymlongrightarrow}\ {\isacharbraceleft}G{\isacharbraceright}\ {\isasymunion}\ S\ {\isasymin}\ C\ {\isasymor}\ {\isacharbraceleft}H{\isacharbraceright}\ {\isasymunion}\ S\ {\isasymin}\ C{\isacharparenright}\isanewline
\ \ \ \ {\isasymand}\ {\isacharparenleft}{\isasymforall}G\ H{\isachardot}\ G\ \isactrlbold {\isasymrightarrow}\ H\ {\isasymin}\ S\ {\isasymlongrightarrow}\ {\isacharbraceleft}\isactrlbold {\isasymnot}G{\isacharbraceright}\ {\isasymunion}\ S\ {\isasymin}\ C\ {\isasymor}\ {\isacharbraceleft}H{\isacharbraceright}\ {\isasymunion}\ S\ {\isasymin}\ C{\isacharparenright}{\isacharparenright}\isanewline
\ \ \ \ {\isasymand}\ {\isacharparenleft}{\isacharparenleft}{\isasymforall}G{\isachardot}\ \isactrlbold {\isasymnot}\ {\isacharparenleft}\isactrlbold {\isasymnot}G{\isacharparenright}\ {\isasymin}\ S\ {\isasymlongrightarrow}\ {\isacharbraceleft}G{\isacharbraceright}\ {\isasymunion}\ S\ {\isasymin}\ C{\isacharparenright}\isanewline
\ \ \ \ {\isasymand}\ {\isacharparenleft}{\isasymforall}G\ H{\isachardot}\ \isactrlbold {\isasymnot}{\isacharparenleft}G\ \isactrlbold {\isasymand}\ H{\isacharparenright}\ {\isasymin}\ S\ {\isasymlongrightarrow}\ {\isacharbraceleft}\isactrlbold {\isasymnot}\ G{\isacharbraceright}\ {\isasymunion}\ S\ {\isasymin}\ C\ {\isasymor}\ {\isacharbraceleft}\isactrlbold {\isasymnot}\ H{\isacharbraceright}\ {\isasymunion}\ S\ {\isasymin}\ C{\isacharparenright}\isanewline
\ \ \ \ {\isasymand}\ {\isacharparenleft}{\isasymforall}G\ H{\isachardot}\ \isactrlbold {\isasymnot}{\isacharparenleft}G\ \isactrlbold {\isasymor}\ H{\isacharparenright}\ {\isasymin}\ S\ {\isasymlongrightarrow}\ {\isacharbraceleft}\isactrlbold {\isasymnot}\ G{\isacharcomma}\ \isactrlbold {\isasymnot}\ H{\isacharbraceright}\ {\isasymunion}\ S\ {\isasymin}\ C{\isacharparenright}\isanewline
\ \ \ \ {\isasymand}\ {\isacharparenleft}{\isasymforall}G\ H{\isachardot}\ \isactrlbold {\isasymnot}{\isacharparenleft}G\ \isactrlbold {\isasymrightarrow}\ H{\isacharparenright}\ {\isasymin}\ S\ {\isasymlongrightarrow}\ {\isacharbraceleft}G{\isacharcomma}\isactrlbold {\isasymnot}\ H{\isacharbraceright}\ {\isasymunion}\ S\ {\isasymin}\ C{\isacharparenright}{\isacharparenright}{\isachardoublequoteclose}\isanewline
\ \ \ \ \isacommand{using}\isamarkupfalse%
\ A\ B\ \isacommand{by}\isamarkupfalse%
\ {\isacharparenleft}rule\ conjI{\isacharparenright}\isanewline
\ \ \isacommand{thus}\isamarkupfalse%
\ {\isachardoublequoteopen}{\isasymbottom}\ {\isasymnotin}\ S\isanewline
\ \ \ \ {\isasymand}\ {\isacharparenleft}{\isasymforall}k{\isachardot}\ Atom\ k\ {\isasymin}\ S\ {\isasymlongrightarrow}\ \isactrlbold {\isasymnot}\ {\isacharparenleft}Atom\ k{\isacharparenright}\ {\isasymin}\ S\ {\isasymlongrightarrow}\ False{\isacharparenright}\isanewline
\ \ \ \ {\isasymand}\ {\isacharparenleft}{\isasymforall}G\ H{\isachardot}\ G\ \isactrlbold {\isasymand}\ H\ {\isasymin}\ S\ {\isasymlongrightarrow}\ {\isacharbraceleft}G{\isacharcomma}H{\isacharbraceright}\ {\isasymunion}\ S\ {\isasymin}\ C{\isacharparenright}\isanewline
\ \ \ \ {\isasymand}\ {\isacharparenleft}{\isasymforall}G\ H{\isachardot}\ G\ \isactrlbold {\isasymor}\ H\ {\isasymin}\ S\ {\isasymlongrightarrow}\ {\isacharbraceleft}G{\isacharbraceright}\ {\isasymunion}\ S\ {\isasymin}\ C\ {\isasymor}\ {\isacharbraceleft}H{\isacharbraceright}\ {\isasymunion}\ S\ {\isasymin}\ C{\isacharparenright}\isanewline
\ \ \ \ {\isasymand}\ {\isacharparenleft}{\isasymforall}G\ H{\isachardot}\ G\ \isactrlbold {\isasymrightarrow}\ H\ {\isasymin}\ S\ {\isasymlongrightarrow}\ {\isacharbraceleft}\isactrlbold {\isasymnot}G{\isacharbraceright}\ {\isasymunion}\ S\ {\isasymin}\ C\ {\isasymor}\ {\isacharbraceleft}H{\isacharbraceright}\ {\isasymunion}\ S\ {\isasymin}\ C{\isacharparenright}\isanewline
\ \ \ \ {\isasymand}\ {\isacharparenleft}{\isasymforall}G{\isachardot}\ \isactrlbold {\isasymnot}\ {\isacharparenleft}\isactrlbold {\isasymnot}G{\isacharparenright}\ {\isasymin}\ S\ {\isasymlongrightarrow}\ {\isacharbraceleft}G{\isacharbraceright}\ {\isasymunion}\ S\ {\isasymin}\ C{\isacharparenright}\isanewline
\ \ \ \ {\isasymand}\ {\isacharparenleft}{\isasymforall}G\ H{\isachardot}\ \isactrlbold {\isasymnot}{\isacharparenleft}G\ \isactrlbold {\isasymand}\ H{\isacharparenright}\ {\isasymin}\ S\ {\isasymlongrightarrow}\ {\isacharbraceleft}\isactrlbold {\isasymnot}\ G{\isacharbraceright}\ {\isasymunion}\ S\ {\isasymin}\ C\ {\isasymor}\ {\isacharbraceleft}\isactrlbold {\isasymnot}\ H{\isacharbraceright}\ {\isasymunion}\ S\ {\isasymin}\ C{\isacharparenright}\isanewline
\ \ \ \ {\isasymand}\ {\isacharparenleft}{\isasymforall}G\ H{\isachardot}\ \isactrlbold {\isasymnot}{\isacharparenleft}G\ \isactrlbold {\isasymor}\ H{\isacharparenright}\ {\isasymin}\ S\ {\isasymlongrightarrow}\ {\isacharbraceleft}\isactrlbold {\isasymnot}\ G{\isacharcomma}\ \isactrlbold {\isasymnot}\ H{\isacharbraceright}\ {\isasymunion}\ S\ {\isasymin}\ C{\isacharparenright}\isanewline
\ \ \ \ {\isasymand}\ {\isacharparenleft}{\isasymforall}G\ H{\isachardot}\ \isactrlbold {\isasymnot}{\isacharparenleft}G\ \isactrlbold {\isasymrightarrow}\ H{\isacharparenright}\ {\isasymin}\ S\ {\isasymlongrightarrow}\ {\isacharbraceleft}G{\isacharcomma}\isactrlbold {\isasymnot}\ H{\isacharbraceright}\ {\isasymunion}\ S\ {\isasymin}\ C{\isacharparenright}{\isachardoublequoteclose}\isanewline
\ \ \ \ \isacommand{by}\isamarkupfalse%
\ {\isacharparenleft}iprover\ intro{\isacharcolon}\ conj{\isacharunderscore}assoc{\isacharparenright}\isanewline
\isacommand{qed}\isamarkupfalse%
%
\endisatagproof
{\isafoldproof}%
%
\isadelimproof
%
\endisadelimproof
%
\begin{isamarkuptext}%
Una vez probadas detalladamente en Isabelle cada una de las implicaciones de la
  equivalencia, podemos finalmente concluir con la demostración del lema completo.%
\end{isamarkuptext}\isamarkuptrue%
\isacommand{lemma}\isamarkupfalse%
\ {\isachardoublequoteopen}pcp\ C\ {\isacharequal}\ {\isacharparenleft}{\isasymforall}S\ {\isasymin}\ C{\isachardot}\ {\isasymbottom}\ {\isasymnotin}\ S\isanewline
{\isasymand}\ {\isacharparenleft}{\isasymforall}k{\isachardot}\ Atom\ k\ {\isasymin}\ S\ {\isasymlongrightarrow}\ \isactrlbold {\isasymnot}\ {\isacharparenleft}Atom\ k{\isacharparenright}\ {\isasymin}\ S\ {\isasymlongrightarrow}\ False{\isacharparenright}\isanewline
{\isasymand}\ {\isacharparenleft}{\isasymforall}F\ G\ H{\isachardot}\ Con\ F\ G\ H\ {\isasymlongrightarrow}\ F\ {\isasymin}\ S\ {\isasymlongrightarrow}\ {\isacharbraceleft}G{\isacharcomma}H{\isacharbraceright}\ {\isasymunion}\ S\ {\isasymin}\ C{\isacharparenright}\isanewline
{\isasymand}\ {\isacharparenleft}{\isasymforall}F\ G\ H{\isachardot}\ Dis\ F\ G\ H\ {\isasymlongrightarrow}\ F\ {\isasymin}\ S\ {\isasymlongrightarrow}\ {\isacharbraceleft}G{\isacharbraceright}\ {\isasymunion}\ S\ {\isasymin}\ C\ {\isasymor}\ {\isacharbraceleft}H{\isacharbraceright}\ {\isasymunion}\ S\ {\isasymin}\ C{\isacharparenright}{\isacharparenright}{\isachardoublequoteclose}\isanewline
%
\isadelimproof
%
\endisadelimproof
%
\isatagproof
\isacommand{proof}\isamarkupfalse%
\ {\isacharparenleft}rule\ iffI{\isacharparenright}\isanewline
\ \ \isacommand{assume}\isamarkupfalse%
\ {\isachardoublequoteopen}pcp\ C{\isachardoublequoteclose}\isanewline
\ \ \isacommand{thus}\isamarkupfalse%
\ {\isachardoublequoteopen}{\isasymforall}S\ {\isasymin}\ C{\isachardot}\ {\isasymbottom}\ {\isasymnotin}\ S\isanewline
{\isasymand}\ {\isacharparenleft}{\isasymforall}k{\isachardot}\ Atom\ k\ {\isasymin}\ S\ {\isasymlongrightarrow}\ \isactrlbold {\isasymnot}\ {\isacharparenleft}Atom\ k{\isacharparenright}\ {\isasymin}\ S\ {\isasymlongrightarrow}\ False{\isacharparenright}\isanewline
{\isasymand}\ {\isacharparenleft}{\isasymforall}F\ G\ H{\isachardot}\ Con\ F\ G\ H\ {\isasymlongrightarrow}\ F\ {\isasymin}\ S\ {\isasymlongrightarrow}\ {\isacharbraceleft}G{\isacharcomma}H{\isacharbraceright}\ {\isasymunion}\ S\ {\isasymin}\ C{\isacharparenright}\isanewline
{\isasymand}\ {\isacharparenleft}{\isasymforall}F\ G\ H{\isachardot}\ Dis\ F\ G\ H\ {\isasymlongrightarrow}\ F\ {\isasymin}\ S\ {\isasymlongrightarrow}\ {\isacharbraceleft}G{\isacharbraceright}\ {\isasymunion}\ S\ {\isasymin}\ C\ {\isasymor}\ {\isacharbraceleft}H{\isacharbraceright}\ {\isasymunion}\ S\ {\isasymin}\ C{\isacharparenright}{\isachardoublequoteclose}\isanewline
\ \ \ \ \isacommand{by}\isamarkupfalse%
\ {\isacharparenleft}rule\ pcp{\isacharunderscore}alt{\isadigit{1}}{\isacharparenright}\isanewline
\isacommand{next}\isamarkupfalse%
\isanewline
\ \ \isacommand{assume}\isamarkupfalse%
\ {\isachardoublequoteopen}{\isasymforall}S\ {\isasymin}\ C{\isachardot}\ {\isasymbottom}\ {\isasymnotin}\ S\isanewline
{\isasymand}\ {\isacharparenleft}{\isasymforall}k{\isachardot}\ Atom\ k\ {\isasymin}\ S\ {\isasymlongrightarrow}\ \isactrlbold {\isasymnot}\ {\isacharparenleft}Atom\ k{\isacharparenright}\ {\isasymin}\ S\ {\isasymlongrightarrow}\ False{\isacharparenright}\isanewline
{\isasymand}\ {\isacharparenleft}{\isasymforall}F\ G\ H{\isachardot}\ Con\ F\ G\ H\ {\isasymlongrightarrow}\ F\ {\isasymin}\ S\ {\isasymlongrightarrow}\ {\isacharbraceleft}G{\isacharcomma}H{\isacharbraceright}\ {\isasymunion}\ S\ {\isasymin}\ C{\isacharparenright}\isanewline
{\isasymand}\ {\isacharparenleft}{\isasymforall}F\ G\ H{\isachardot}\ Dis\ F\ G\ H\ {\isasymlongrightarrow}\ F\ {\isasymin}\ S\ {\isasymlongrightarrow}\ {\isacharbraceleft}G{\isacharbraceright}\ {\isasymunion}\ S\ {\isasymin}\ C\ {\isasymor}\ {\isacharbraceleft}H{\isacharbraceright}\ {\isasymunion}\ S\ {\isasymin}\ C{\isacharparenright}{\isachardoublequoteclose}\isanewline
\ \ \isacommand{thus}\isamarkupfalse%
\ {\isachardoublequoteopen}pcp\ C{\isachardoublequoteclose}\isanewline
\ \ \ \ \isacommand{by}\isamarkupfalse%
\ {\isacharparenleft}rule\ pcp{\isacharunderscore}alt{\isadigit{2}}{\isacharparenright}\isanewline
\isacommand{qed}\isamarkupfalse%
%
\endisatagproof
{\isafoldproof}%
%
\isadelimproof
%
\endisadelimproof
%
\begin{isamarkuptext}%
La demostración automática del resultado en Isabelle/HOL se muestra finalmente a 
  continuación.%
\end{isamarkuptext}\isamarkuptrue%
\isacommand{lemma}\isamarkupfalse%
\ pcp{\isacharunderscore}alt{\isacharcolon}\ {\isachardoublequoteopen}pcp\ C\ {\isacharequal}\ {\isacharparenleft}{\isasymforall}S\ {\isasymin}\ C{\isachardot}\isanewline
\ \ {\isasymbottom}\ {\isasymnotin}\ S\isanewline
{\isasymand}\ {\isacharparenleft}{\isasymforall}k{\isachardot}\ Atom\ k\ {\isasymin}\ S\ {\isasymlongrightarrow}\ \isactrlbold {\isasymnot}\ {\isacharparenleft}Atom\ k{\isacharparenright}\ {\isasymin}\ S\ {\isasymlongrightarrow}\ False{\isacharparenright}\isanewline
{\isasymand}\ {\isacharparenleft}{\isasymforall}F\ G\ H{\isachardot}\ Con\ F\ G\ H\ {\isasymlongrightarrow}\ F\ {\isasymin}\ S\ {\isasymlongrightarrow}\ {\isacharbraceleft}G{\isacharcomma}H{\isacharbraceright}\ {\isasymunion}\ S\ {\isasymin}\ C{\isacharparenright}\isanewline
{\isasymand}\ {\isacharparenleft}{\isasymforall}F\ G\ H{\isachardot}\ Dis\ F\ G\ H\ {\isasymlongrightarrow}\ F\ {\isasymin}\ S\ {\isasymlongrightarrow}\ {\isacharbraceleft}G{\isacharbraceright}\ {\isasymunion}\ S\ {\isasymin}\ C\ {\isasymor}\ {\isacharbraceleft}H{\isacharbraceright}\ {\isasymunion}\ S\ {\isasymin}\ C{\isacharparenright}{\isacharparenright}{\isachardoublequoteclose}\isanewline
%
\isadelimproof
\ \ %
\endisadelimproof
%
\isatagproof
\isacommand{apply}\isamarkupfalse%
{\isacharparenleft}simp\ add{\isacharcolon}\ pcp{\isacharunderscore}def\ con{\isacharunderscore}dis{\isacharunderscore}simps{\isacharparenright}\isanewline
\ \ \isacommand{apply}\isamarkupfalse%
{\isacharparenleft}rule\ iffI{\isacharsemicolon}\ unfold\ Ball{\isacharunderscore}def{\isacharsemicolon}\ elim\ all{\isacharunderscore}forward{\isacharparenright}\isanewline
\ \ \isacommand{by}\isamarkupfalse%
\ {\isacharparenleft}auto\ simp\ add{\isacharcolon}\ insert{\isacharunderscore}absorb\ split{\isacharcolon}\ formula{\isachardot}splits{\isacharparenright}\isanewline
%
\endisatagproof
{\isafoldproof}%
%
\isadelimproof
%
\endisadelimproof
%
\isadelimtheory
%
\endisadelimtheory
%
\isatagtheory
%
\endisatagtheory
{\isafoldtheory}%
%
\isadelimtheory
%
\endisadelimtheory
%
\end{isabellebody}%
\endinput
%:%file=~/TFM/TFM/Pcp.thy%:%
%:%19=12%:%
%:%20=13%:%
%:%21=14%:%
%:%22=15%:%
%:%23=16%:%
%:%24=17%:%
%:%25=18%:%
%:%26=19%:%
%:%27=20%:%
%:%28=21%:%
%:%29=22%:%
%:%30=23%:%
%:%31=24%:%
%:%32=25%:%
%:%33=26%:%
%:%34=27%:%
%:%35=28%:%
%:%36=29%:%
%:%37=30%:%
%:%38=31%:%
%:%39=32%:%
%:%40=33%:%
%:%41=34%:%
%:%42=35%:%
%:%43=36%:%
%:%44=37%:%
%:%46=39%:%
%:%47=39%:%
%:%58=50%:%
%:%59=51%:%
%:%60=52%:%
%:%62=54%:%
%:%63=54%:%
%:%66=55%:%
%:%70=55%:%
%:%71=55%:%
%:%72=55%:%
%:%81=57%:%
%:%82=58%:%
%:%83=59%:%
%:%85=61%:%
%:%86=61%:%
%:%89=62%:%
%:%93=62%:%
%:%94=62%:%
%:%95=62%:%
%:%100=62%:%
%:%103=63%:%
%:%104=64%:%
%:%105=64%:%
%:%107=66%:%
%:%110=67%:%
%:%114=67%:%
%:%115=67%:%
%:%116=67%:%
%:%125=69%:%
%:%126=70%:%
%:%127=71%:%
%:%128=72%:%
%:%130=74%:%
%:%131=74%:%
%:%132=75%:%
%:%135=76%:%
%:%139=76%:%
%:%140=76%:%
%:%141=76%:%
%:%150=78%:%
%:%151=79%:%
%:%152=80%:%
%:%153=81%:%
%:%154=82%:%
%:%155=83%:%
%:%156=84%:%
%:%157=85%:%
%:%158=86%:%
%:%159=87%:%
%:%160=88%:%
%:%161=89%:%
%:%162=90%:%
%:%163=91%:%
%:%164=92%:%
%:%165=93%:%
%:%166=94%:%
%:%167=95%:%
%:%168=96%:%
%:%169=97%:%
%:%170=98%:%
%:%171=99%:%
%:%172=100%:%
%:%174=102%:%
%:%175=102%:%
%:%178=105%:%
%:%181=106%:%
%:%185=106%:%
%:%195=108%:%
%:%196=109%:%
%:%197=110%:%
%:%198=111%:%
%:%199=112%:%
%:%200=113%:%
%:%201=114%:%
%:%202=115%:%
%:%203=116%:%
%:%204=117%:%
%:%205=118%:%
%:%206=119%:%
%:%207=120%:%
%:%208=121%:%
%:%209=122%:%
%:%210=123%:%
%:%211=124%:%
%:%212=125%:%
%:%213=126%:%
%:%214=127%:%
%:%215=128%:%
%:%216=129%:%
%:%217=130%:%
%:%218=131%:%
%:%219=132%:%
%:%220=133%:%
%:%221=134%:%
%:%222=135%:%
%:%223=136%:%
%:%224=137%:%
%:%225=138%:%
%:%226=139%:%
%:%227=140%:%
%:%228=141%:%
%:%229=142%:%
%:%230=143%:%
%:%231=144%:%
%:%232=145%:%
%:%233=146%:%
%:%234=147%:%
%:%235=148%:%
%:%236=149%:%
%:%237=150%:%
%:%238=151%:%
%:%239=152%:%
%:%240=153%:%
%:%241=154%:%
%:%242=155%:%
%:%243=156%:%
%:%244=157%:%
%:%245=158%:%
%:%246=159%:%
%:%247=160%:%
%:%248=161%:%
%:%249=162%:%
%:%250=163%:%
%:%251=164%:%
%:%252=165%:%
%:%253=166%:%
%:%254=167%:%
%:%255=168%:%
%:%256=169%:%
%:%257=170%:%
%:%258=171%:%
%:%259=172%:%
%:%260=173%:%
%:%261=174%:%
%:%262=175%:%
%:%263=176%:%
%:%264=177%:%
%:%265=178%:%
%:%266=179%:%
%:%267=180%:%
%:%268=181%:%
%:%269=182%:%
%:%270=183%:%
%:%271=184%:%
%:%272=185%:%
%:%273=186%:%
%:%274=187%:%
%:%275=188%:%
%:%276=189%:%
%:%277=190%:%
%:%278=191%:%
%:%279=192%:%
%:%280=193%:%
%:%281=194%:%
%:%282=195%:%
%:%283=196%:%
%:%284=197%:%
%:%285=198%:%
%:%286=199%:%
%:%287=200%:%
%:%288=201%:%
%:%289=202%:%
%:%290=203%:%
%:%291=204%:%
%:%292=205%:%
%:%293=206%:%
%:%294=207%:%
%:%295=208%:%
%:%296=209%:%
%:%297=210%:%
%:%298=211%:%
%:%299=212%:%
%:%300=213%:%
%:%301=214%:%
%:%302=215%:%
%:%303=216%:%
%:%304=217%:%
%:%305=218%:%
%:%306=219%:%
%:%307=220%:%
%:%308=221%:%
%:%309=222%:%
%:%310=223%:%
%:%311=224%:%
%:%312=225%:%
%:%313=226%:%
%:%314=227%:%
%:%315=228%:%
%:%316=229%:%
%:%317=230%:%
%:%318=231%:%
%:%319=232%:%
%:%320=233%:%
%:%321=234%:%
%:%322=235%:%
%:%323=236%:%
%:%324=237%:%
%:%325=238%:%
%:%326=239%:%
%:%327=240%:%
%:%328=241%:%
%:%329=242%:%
%:%330=243%:%
%:%331=244%:%
%:%332=245%:%
%:%333=246%:%
%:%334=247%:%
%:%335=248%:%
%:%336=249%:%
%:%337=250%:%
%:%338=251%:%
%:%339=252%:%
%:%340=253%:%
%:%341=254%:%
%:%342=255%:%
%:%343=256%:%
%:%344=257%:%
%:%345=258%:%
%:%346=259%:%
%:%347=260%:%
%:%348=261%:%
%:%349=262%:%
%:%350=263%:%
%:%351=264%:%
%:%352=265%:%
%:%353=266%:%
%:%354=267%:%
%:%355=268%:%
%:%356=269%:%
%:%357=270%:%
%:%358=271%:%
%:%359=272%:%
%:%360=273%:%
%:%361=274%:%
%:%362=275%:%
%:%363=276%:%
%:%364=277%:%
%:%365=278%:%
%:%366=279%:%
%:%367=280%:%
%:%368=281%:%
%:%369=282%:%
%:%370=283%:%
%:%371=284%:%
%:%372=285%:%
%:%373=286%:%
%:%374=287%:%
%:%376=289%:%
%:%377=289%:%
%:%378=290%:%
%:%381=293%:%
%:%382=294%:%
%:%389=295%:%
%:%390=295%:%
%:%391=296%:%
%:%392=296%:%
%:%393=297%:%
%:%394=297%:%
%:%395=297%:%
%:%396=298%:%
%:%397=298%:%
%:%398=299%:%
%:%399=299%:%
%:%400=299%:%
%:%401=300%:%
%:%402=300%:%
%:%403=301%:%
%:%404=301%:%
%:%405=301%:%
%:%406=302%:%
%:%407=302%:%
%:%408=303%:%
%:%409=303%:%
%:%410=303%:%
%:%411=304%:%
%:%412=304%:%
%:%413=305%:%
%:%414=305%:%
%:%415=306%:%
%:%416=306%:%
%:%417=307%:%
%:%418=307%:%
%:%419=308%:%
%:%420=308%:%
%:%421=309%:%
%:%422=309%:%
%:%423=310%:%
%:%424=310%:%
%:%425=310%:%
%:%428=313%:%
%:%429=314%:%
%:%430=314%:%
%:%431=315%:%
%:%432=315%:%
%:%433=316%:%
%:%434=316%:%
%:%435=317%:%
%:%436=317%:%
%:%437=318%:%
%:%438=318%:%
%:%439=319%:%
%:%440=319%:%
%:%441=319%:%
%:%442=320%:%
%:%443=320%:%
%:%444=321%:%
%:%445=321%:%
%:%447=323%:%
%:%448=324%:%
%:%449=324%:%
%:%450=325%:%
%:%451=325%:%
%:%452=326%:%
%:%453=326%:%
%:%454=327%:%
%:%455=327%:%
%:%456=328%:%
%:%457=328%:%
%:%458=328%:%
%:%459=329%:%
%:%460=329%:%
%:%461=330%:%
%:%462=330%:%
%:%463=330%:%
%:%464=331%:%
%:%465=331%:%
%:%466=332%:%
%:%467=332%:%
%:%468=332%:%
%:%469=333%:%
%:%470=333%:%
%:%471=334%:%
%:%472=334%:%
%:%473=334%:%
%:%474=335%:%
%:%475=335%:%
%:%476=336%:%
%:%477=336%:%
%:%478=336%:%
%:%479=337%:%
%:%480=337%:%
%:%481=338%:%
%:%482=338%:%
%:%483=339%:%
%:%484=340%:%
%:%485=340%:%
%:%486=341%:%
%:%487=341%:%
%:%488=342%:%
%:%489=342%:%
%:%490=343%:%
%:%491=343%:%
%:%492=344%:%
%:%493=344%:%
%:%494=344%:%
%:%495=345%:%
%:%496=345%:%
%:%497=346%:%
%:%498=346%:%
%:%499=346%:%
%:%500=347%:%
%:%501=347%:%
%:%502=348%:%
%:%503=348%:%
%:%504=348%:%
%:%505=349%:%
%:%506=349%:%
%:%507=350%:%
%:%508=350%:%
%:%509=350%:%
%:%510=351%:%
%:%511=351%:%
%:%512=352%:%
%:%513=352%:%
%:%514=353%:%
%:%515=353%:%
%:%516=353%:%
%:%517=354%:%
%:%518=354%:%
%:%519=355%:%
%:%520=355%:%
%:%521=356%:%
%:%522=356%:%
%:%523=356%:%
%:%524=357%:%
%:%525=357%:%
%:%526=358%:%
%:%527=358%:%
%:%528=358%:%
%:%529=359%:%
%:%530=359%:%
%:%531=359%:%
%:%532=360%:%
%:%533=360%:%
%:%534=361%:%
%:%535=361%:%
%:%536=362%:%
%:%537=362%:%
%:%538=363%:%
%:%539=363%:%
%:%540=364%:%
%:%541=364%:%
%:%542=365%:%
%:%543=365%:%
%:%544=366%:%
%:%545=366%:%
%:%546=367%:%
%:%547=367%:%
%:%548=368%:%
%:%558=370%:%
%:%559=371%:%
%:%560=372%:%
%:%562=374%:%
%:%563=374%:%
%:%564=375%:%
%:%567=378%:%
%:%568=379%:%
%:%575=380%:%
%:%576=380%:%
%:%577=381%:%
%:%578=381%:%
%:%579=382%:%
%:%580=382%:%
%:%581=382%:%
%:%582=383%:%
%:%583=383%:%
%:%584=384%:%
%:%585=384%:%
%:%586=384%:%
%:%587=385%:%
%:%588=385%:%
%:%589=386%:%
%:%590=386%:%
%:%591=386%:%
%:%592=387%:%
%:%593=387%:%
%:%594=388%:%
%:%595=388%:%
%:%596=388%:%
%:%597=389%:%
%:%598=389%:%
%:%599=390%:%
%:%600=390%:%
%:%601=391%:%
%:%602=391%:%
%:%603=392%:%
%:%604=392%:%
%:%605=393%:%
%:%606=393%:%
%:%607=394%:%
%:%608=394%:%
%:%609=395%:%
%:%610=395%:%
%:%611=395%:%
%:%614=398%:%
%:%615=399%:%
%:%616=399%:%
%:%617=400%:%
%:%618=400%:%
%:%619=401%:%
%:%620=401%:%
%:%621=402%:%
%:%622=402%:%
%:%623=403%:%
%:%624=403%:%
%:%625=404%:%
%:%626=404%:%
%:%627=404%:%
%:%628=405%:%
%:%629=405%:%
%:%630=406%:%
%:%631=406%:%
%:%633=408%:%
%:%634=409%:%
%:%635=409%:%
%:%636=410%:%
%:%637=410%:%
%:%638=411%:%
%:%639=411%:%
%:%640=412%:%
%:%641=412%:%
%:%642=413%:%
%:%643=413%:%
%:%644=413%:%
%:%645=414%:%
%:%646=414%:%
%:%647=415%:%
%:%648=415%:%
%:%649=415%:%
%:%650=416%:%
%:%651=416%:%
%:%652=417%:%
%:%653=417%:%
%:%654=417%:%
%:%655=418%:%
%:%656=418%:%
%:%657=419%:%
%:%658=419%:%
%:%659=419%:%
%:%660=420%:%
%:%661=420%:%
%:%662=421%:%
%:%663=421%:%
%:%664=421%:%
%:%665=422%:%
%:%666=422%:%
%:%667=423%:%
%:%668=423%:%
%:%669=424%:%
%:%670=425%:%
%:%671=425%:%
%:%672=426%:%
%:%673=426%:%
%:%674=427%:%
%:%675=427%:%
%:%676=428%:%
%:%677=428%:%
%:%678=429%:%
%:%679=429%:%
%:%680=429%:%
%:%681=430%:%
%:%682=430%:%
%:%683=431%:%
%:%684=431%:%
%:%685=431%:%
%:%686=432%:%
%:%687=432%:%
%:%688=433%:%
%:%689=433%:%
%:%690=433%:%
%:%691=434%:%
%:%692=434%:%
%:%693=435%:%
%:%694=435%:%
%:%695=435%:%
%:%696=436%:%
%:%697=436%:%
%:%698=437%:%
%:%699=437%:%
%:%700=437%:%
%:%701=438%:%
%:%702=438%:%
%:%703=439%:%
%:%704=439%:%
%:%705=440%:%
%:%706=440%:%
%:%707=440%:%
%:%708=441%:%
%:%709=441%:%
%:%710=442%:%
%:%711=442%:%
%:%712=443%:%
%:%713=443%:%
%:%714=443%:%
%:%715=444%:%
%:%716=444%:%
%:%717=445%:%
%:%718=445%:%
%:%719=445%:%
%:%720=446%:%
%:%721=446%:%
%:%722=446%:%
%:%723=447%:%
%:%724=447%:%
%:%725=448%:%
%:%726=448%:%
%:%727=449%:%
%:%728=449%:%
%:%729=450%:%
%:%730=450%:%
%:%731=451%:%
%:%732=451%:%
%:%733=452%:%
%:%734=452%:%
%:%735=453%:%
%:%736=453%:%
%:%737=454%:%
%:%738=454%:%
%:%739=455%:%
%:%749=457%:%
%:%750=458%:%
%:%752=460%:%
%:%753=460%:%
%:%754=461%:%
%:%755=462%:%
%:%758=465%:%
%:%765=466%:%
%:%766=466%:%
%:%767=467%:%
%:%768=467%:%
%:%769=468%:%
%:%770=468%:%
%:%771=469%:%
%:%772=469%:%
%:%781=478%:%
%:%782=479%:%
%:%783=479%:%
%:%784=479%:%
%:%785=480%:%
%:%786=480%:%
%:%787=480%:%
%:%795=488%:%
%:%796=489%:%
%:%797=489%:%
%:%798=489%:%
%:%799=490%:%
%:%800=490%:%
%:%801=490%:%
%:%802=491%:%
%:%803=491%:%
%:%804=492%:%
%:%805=492%:%
%:%806=493%:%
%:%807=493%:%
%:%808=493%:%
%:%809=494%:%
%:%810=494%:%
%:%811=495%:%
%:%812=495%:%
%:%813=495%:%
%:%814=496%:%
%:%815=496%:%
%:%816=497%:%
%:%817=497%:%
%:%818=497%:%
%:%819=498%:%
%:%820=498%:%
%:%821=499%:%
%:%822=499%:%
%:%823=499%:%
%:%824=500%:%
%:%825=500%:%
%:%826=501%:%
%:%827=501%:%
%:%828=501%:%
%:%829=502%:%
%:%830=502%:%
%:%831=503%:%
%:%832=503%:%
%:%833=503%:%
%:%834=504%:%
%:%835=504%:%
%:%836=505%:%
%:%837=505%:%
%:%838=505%:%
%:%839=506%:%
%:%840=506%:%
%:%841=507%:%
%:%842=507%:%
%:%843=507%:%
%:%844=508%:%
%:%845=508%:%
%:%848=511%:%
%:%849=512%:%
%:%850=512%:%
%:%851=512%:%
%:%852=513%:%
%:%853=513%:%
%:%854=513%:%
%:%855=514%:%
%:%856=514%:%
%:%857=515%:%
%:%858=515%:%
%:%861=518%:%
%:%862=519%:%
%:%863=519%:%
%:%864=519%:%
%:%865=520%:%
%:%866=520%:%
%:%867=520%:%
%:%868=521%:%
%:%869=521%:%
%:%870=522%:%
%:%871=522%:%
%:%874=525%:%
%:%875=526%:%
%:%876=526%:%
%:%877=526%:%
%:%878=527%:%
%:%888=529%:%
%:%889=530%:%
%:%890=531%:%
%:%891=532%:%
%:%892=533%:%
%:%894=535%:%
%:%895=535%:%
%:%896=536%:%
%:%897=537%:%
%:%904=538%:%
%:%905=538%:%
%:%906=539%:%
%:%907=539%:%
%:%908=540%:%
%:%909=540%:%
%:%910=541%:%
%:%911=541%:%
%:%912=542%:%
%:%913=542%:%
%:%914=543%:%
%:%915=543%:%
%:%916=543%:%
%:%917=544%:%
%:%918=544%:%
%:%919=545%:%
%:%920=545%:%
%:%921=546%:%
%:%922=546%:%
%:%923=547%:%
%:%924=547%:%
%:%925=547%:%
%:%926=548%:%
%:%927=548%:%
%:%928=548%:%
%:%929=549%:%
%:%930=549%:%
%:%931=549%:%
%:%932=550%:%
%:%933=550%:%
%:%934=551%:%
%:%935=551%:%
%:%936=551%:%
%:%937=552%:%
%:%938=552%:%
%:%939=553%:%
%:%945=553%:%
%:%948=554%:%
%:%949=555%:%
%:%950=555%:%
%:%951=556%:%
%:%952=557%:%
%:%959=558%:%
%:%960=558%:%
%:%961=559%:%
%:%962=559%:%
%:%963=560%:%
%:%964=560%:%
%:%965=561%:%
%:%966=561%:%
%:%967=562%:%
%:%968=562%:%
%:%969=563%:%
%:%970=563%:%
%:%971=563%:%
%:%972=564%:%
%:%973=564%:%
%:%974=565%:%
%:%975=565%:%
%:%976=566%:%
%:%977=566%:%
%:%978=567%:%
%:%979=567%:%
%:%980=567%:%
%:%981=568%:%
%:%982=568%:%
%:%983=568%:%
%:%984=569%:%
%:%985=569%:%
%:%986=570%:%
%:%987=570%:%
%:%988=570%:%
%:%989=571%:%
%:%990=571%:%
%:%991=572%:%
%:%992=572%:%
%:%993=572%:%
%:%994=573%:%
%:%995=573%:%
%:%996=573%:%
%:%997=574%:%
%:%998=574%:%
%:%999=574%:%
%:%1000=575%:%
%:%1001=575%:%
%:%1002=575%:%
%:%1003=576%:%
%:%1004=576%:%
%:%1005=577%:%
%:%1006=577%:%
%:%1007=578%:%
%:%1008=578%:%
%:%1009=579%:%
%:%1015=579%:%
%:%1018=580%:%
%:%1019=581%:%
%:%1020=581%:%
%:%1021=582%:%
%:%1022=583%:%
%:%1029=584%:%
%:%1030=584%:%
%:%1031=585%:%
%:%1032=585%:%
%:%1033=586%:%
%:%1034=586%:%
%:%1035=587%:%
%:%1036=587%:%
%:%1037=588%:%
%:%1038=588%:%
%:%1039=589%:%
%:%1040=589%:%
%:%1041=589%:%
%:%1042=590%:%
%:%1043=590%:%
%:%1044=591%:%
%:%1045=591%:%
%:%1046=592%:%
%:%1047=592%:%
%:%1048=593%:%
%:%1049=593%:%
%:%1050=593%:%
%:%1051=594%:%
%:%1052=594%:%
%:%1053=594%:%
%:%1054=595%:%
%:%1055=595%:%
%:%1056=595%:%
%:%1057=596%:%
%:%1058=596%:%
%:%1059=597%:%
%:%1060=597%:%
%:%1061=597%:%
%:%1062=598%:%
%:%1063=598%:%
%:%1064=599%:%
%:%1070=599%:%
%:%1073=600%:%
%:%1074=601%:%
%:%1075=601%:%
%:%1076=602%:%
%:%1077=603%:%
%:%1084=604%:%
%:%1085=604%:%
%:%1086=605%:%
%:%1087=605%:%
%:%1088=606%:%
%:%1089=606%:%
%:%1090=607%:%
%:%1091=607%:%
%:%1092=608%:%
%:%1093=608%:%
%:%1094=609%:%
%:%1095=609%:%
%:%1096=609%:%
%:%1097=610%:%
%:%1098=610%:%
%:%1099=611%:%
%:%1100=611%:%
%:%1101=612%:%
%:%1102=612%:%
%:%1103=613%:%
%:%1104=613%:%
%:%1105=613%:%
%:%1106=614%:%
%:%1107=614%:%
%:%1108=614%:%
%:%1109=615%:%
%:%1110=615%:%
%:%1111=615%:%
%:%1112=616%:%
%:%1113=616%:%
%:%1114=617%:%
%:%1115=617%:%
%:%1116=617%:%
%:%1117=618%:%
%:%1118=618%:%
%:%1119=619%:%
%:%1129=621%:%
%:%1130=622%:%
%:%1131=623%:%
%:%1133=625%:%
%:%1134=625%:%
%:%1135=626%:%
%:%1136=627%:%
%:%1143=628%:%
%:%1144=628%:%
%:%1145=629%:%
%:%1146=629%:%
%:%1147=630%:%
%:%1148=630%:%
%:%1149=631%:%
%:%1150=631%:%
%:%1151=632%:%
%:%1152=632%:%
%:%1153=633%:%
%:%1154=633%:%
%:%1155=633%:%
%:%1156=634%:%
%:%1157=634%:%
%:%1158=635%:%
%:%1159=635%:%
%:%1160=636%:%
%:%1161=636%:%
%:%1162=637%:%
%:%1163=637%:%
%:%1164=637%:%
%:%1165=638%:%
%:%1166=638%:%
%:%1167=638%:%
%:%1168=639%:%
%:%1169=639%:%
%:%1170=639%:%
%:%1171=640%:%
%:%1172=640%:%
%:%1173=641%:%
%:%1174=641%:%
%:%1175=641%:%
%:%1176=642%:%
%:%1177=642%:%
%:%1178=643%:%
%:%1184=643%:%
%:%1187=644%:%
%:%1188=645%:%
%:%1189=645%:%
%:%1190=646%:%
%:%1191=647%:%
%:%1198=648%:%
%:%1199=648%:%
%:%1200=649%:%
%:%1201=649%:%
%:%1202=650%:%
%:%1203=650%:%
%:%1204=651%:%
%:%1205=651%:%
%:%1206=652%:%
%:%1207=652%:%
%:%1208=653%:%
%:%1209=653%:%
%:%1210=653%:%
%:%1211=654%:%
%:%1212=654%:%
%:%1213=655%:%
%:%1214=655%:%
%:%1215=656%:%
%:%1216=656%:%
%:%1217=657%:%
%:%1218=657%:%
%:%1219=657%:%
%:%1220=658%:%
%:%1221=658%:%
%:%1222=658%:%
%:%1223=659%:%
%:%1224=659%:%
%:%1225=659%:%
%:%1226=660%:%
%:%1227=660%:%
%:%1228=661%:%
%:%1229=661%:%
%:%1230=661%:%
%:%1231=662%:%
%:%1232=662%:%
%:%1233=663%:%
%:%1239=663%:%
%:%1242=664%:%
%:%1243=665%:%
%:%1244=665%:%
%:%1245=666%:%
%:%1246=667%:%
%:%1253=668%:%
%:%1254=668%:%
%:%1255=669%:%
%:%1256=669%:%
%:%1257=670%:%
%:%1258=670%:%
%:%1259=671%:%
%:%1260=671%:%
%:%1261=672%:%
%:%1262=672%:%
%:%1263=673%:%
%:%1264=673%:%
%:%1265=673%:%
%:%1266=674%:%
%:%1267=674%:%
%:%1268=675%:%
%:%1269=675%:%
%:%1270=676%:%
%:%1271=676%:%
%:%1272=677%:%
%:%1273=677%:%
%:%1274=677%:%
%:%1275=678%:%
%:%1276=678%:%
%:%1277=678%:%
%:%1278=679%:%
%:%1279=679%:%
%:%1280=679%:%
%:%1281=680%:%
%:%1282=680%:%
%:%1283=681%:%
%:%1284=681%:%
%:%1285=681%:%
%:%1286=682%:%
%:%1287=682%:%
%:%1288=683%:%
%:%1298=685%:%
%:%1300=687%:%
%:%1301=687%:%
%:%1302=688%:%
%:%1305=691%:%
%:%1306=692%:%
%:%1309=693%:%
%:%1313=693%:%
%:%1314=693%:%
%:%1315=694%:%
%:%1316=694%:%
%:%1317=695%:%
%:%1318=695%:%
%:%1319=696%:%
%:%1320=696%:%
%:%1321=697%:%
%:%1322=697%:%
%:%1325=700%:%
%:%1326=701%:%
%:%1327=701%:%
%:%1328=701%:%
%:%1329=702%:%
%:%1330=702%:%
%:%1331=702%:%
%:%1332=703%:%
%:%1333=703%:%
%:%1334=704%:%
%:%1335=704%:%
%:%1336=705%:%
%:%1337=705%:%
%:%1338=705%:%
%:%1339=706%:%
%:%1340=706%:%
%:%1341=707%:%
%:%1342=708%:%
%:%1343=708%:%
%:%1344=708%:%
%:%1345=709%:%
%:%1346=709%:%
%:%1347=710%:%
%:%1348=710%:%
%:%1349=710%:%
%:%1350=711%:%
%:%1351=711%:%
%:%1352=712%:%
%:%1353=712%:%
%:%1354=712%:%
%:%1355=713%:%
%:%1356=713%:%
%:%1357=714%:%
%:%1358=714%:%
%:%1359=714%:%
%:%1360=715%:%
%:%1361=715%:%
%:%1362=716%:%
%:%1363=716%:%
%:%1364=716%:%
%:%1365=717%:%
%:%1366=717%:%
%:%1367=718%:%
%:%1368=718%:%
%:%1369=718%:%
%:%1370=719%:%
%:%1371=719%:%
%:%1372=720%:%
%:%1373=720%:%
%:%1374=720%:%
%:%1375=721%:%
%:%1376=721%:%
%:%1377=722%:%
%:%1378=722%:%
%:%1379=722%:%
%:%1380=723%:%
%:%1381=723%:%
%:%1385=727%:%
%:%1386=728%:%
%:%1387=728%:%
%:%1388=728%:%
%:%1389=729%:%
%:%1390=729%:%
%:%1393=732%:%
%:%1394=733%:%
%:%1395=733%:%
%:%1396=733%:%
%:%1397=734%:%
%:%1398=734%:%
%:%1406=742%:%
%:%1407=743%:%
%:%1408=743%:%
%:%1409=743%:%
%:%1410=744%:%
%:%1411=744%:%
%:%1419=752%:%
%:%1420=753%:%
%:%1421=753%:%
%:%1422=754%:%
%:%1432=756%:%
%:%1433=757%:%
%:%1435=759%:%
%:%1436=759%:%
%:%1439=762%:%
%:%1446=763%:%
%:%1447=763%:%
%:%1448=764%:%
%:%1449=764%:%
%:%1450=765%:%
%:%1451=765%:%
%:%1454=768%:%
%:%1455=769%:%
%:%1456=769%:%
%:%1457=770%:%
%:%1458=770%:%
%:%1459=771%:%
%:%1460=771%:%
%:%1463=774%:%
%:%1464=775%:%
%:%1465=775%:%
%:%1466=776%:%
%:%1467=776%:%
%:%1468=777%:%
%:%1478=779%:%
%:%1479=780%:%
%:%1481=782%:%
%:%1482=782%:%
%:%1486=786%:%
%:%1489=787%:%
%:%1493=787%:%
%:%1494=787%:%
%:%1495=788%:%
%:%1496=788%:%
%:%1497=789%:%
%:%1498=789%:%