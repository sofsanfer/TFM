%
\begin{isabellebody}%
\setisabellecontext{ColecScfc}%
%
\isadelimtheory
%
\endisadelimtheory
%
\isatagtheory
%
\endisatagtheory
{\isafoldtheory}%
%
\isadelimtheory
%
\endisadelimtheory
%
\begin{isamarkuptext}%
En este capítulo se estudian las colecciones de conjuntos \isa{cerradas\ bajo\ subconjuntos} y de 
  \isa{carácter\ finito}, y se demuestran tres resultados sobre las mismas. El primero de ellos permite
  extender una colección con la propiedad de consistencia proposicional a otra que 
  también la verifique y sea cerrada bajo subconjuntos. Posteriormente probaremos que toda colección
  de carácter finito es cerrada bajo subconjuntos. Finalmente, se demuestra que una colección 
  cerrada bajo subconjuntos que verifique la propiedad de consistencia proposicional se puede 
  extender a otra que también verifique dicha propiedad y sea de carácter finito.

  \begin{definicion}
    Una colección de conjuntos es \isa{cerrada\ bajo\ subconjuntos} si todo subconjunto de cada conjunto 
    de la colección pertenece a la colección.
  \end{definicion}

  En Isabelle se formaliza de la siguiente manera.%
\end{isamarkuptext}\isamarkuptrue%
\isacommand{definition}\isamarkupfalse%
\ {\isachardoublequoteopen}subset{\isacharunderscore}closed\ C\ {\isasymequiv}\ {\isacharparenleft}{\isasymforall}S\ {\isasymin}\ C{\isachardot}\ {\isasymforall}S{\isacharprime}{\isasymsubseteq}S{\isachardot}\ S{\isacharprime}\ {\isasymin}\ C{\isacharparenright}{\isachardoublequoteclose}%
\begin{isamarkuptext}%
Mostremos algunos ejemplos para ilustrar la definición. Para ello, veamos si las colecciones
  de conjuntos de fórmulas proposicionales expuestas en los ejemplos anteriores son cerradas bajo 
  subconjuntos.%
\end{isamarkuptext}\isamarkuptrue%
\isacommand{lemma}\isamarkupfalse%
\ {\isachardoublequoteopen}subset{\isacharunderscore}closed\ {\isacharbraceleft}{\isacharbraceleft}{\isacharbraceright}{\isacharbraceright}{\isachardoublequoteclose}\isanewline
%
\isadelimproof
\ \ %
\endisadelimproof
%
\isatagproof
\isacommand{unfolding}\isamarkupfalse%
\ subset{\isacharunderscore}closed{\isacharunderscore}def\ \isacommand{by}\isamarkupfalse%
\ simp%
\endisatagproof
{\isafoldproof}%
%
\isadelimproof
\isanewline
%
\endisadelimproof
\isanewline
\isacommand{lemma}\isamarkupfalse%
\ {\isachardoublequoteopen}{\isasymnot}\ subset{\isacharunderscore}closed\ {\isacharbraceleft}{\isacharbraceleft}Atom\ {\isadigit{0}}{\isacharbraceright}{\isacharbraceright}{\isachardoublequoteclose}\isanewline
%
\isadelimproof
\ \ %
\endisadelimproof
%
\isatagproof
\isacommand{unfolding}\isamarkupfalse%
\ subset{\isacharunderscore}closed{\isacharunderscore}def\ \isacommand{by}\isamarkupfalse%
\ auto%
\endisatagproof
{\isafoldproof}%
%
\isadelimproof
%
\endisadelimproof
%
\begin{isamarkuptext}%
Observemos que, puesto que el conjunto vacío es subconjunto de todo conjunto, una
  condición necesaria para que una colección sea cerrada bajo subconjuntos es que contenga al
  conjunto vacío.%
\end{isamarkuptext}\isamarkuptrue%
\isacommand{lemma}\isamarkupfalse%
\ {\isachardoublequoteopen}subset{\isacharunderscore}closed\ {\isacharbraceleft}{\isacharbraceleft}Atom\ {\isadigit{0}}{\isacharbraceright}{\isacharcomma}{\isacharbraceleft}{\isacharbraceright}{\isacharbraceright}{\isachardoublequoteclose}\isanewline
%
\isadelimproof
\ \ %
\endisadelimproof
%
\isatagproof
\isacommand{unfolding}\isamarkupfalse%
\ subset{\isacharunderscore}closed{\isacharunderscore}def\ \isacommand{by}\isamarkupfalse%
\ auto%
\endisatagproof
{\isafoldproof}%
%
\isadelimproof
%
\endisadelimproof
%
\begin{isamarkuptext}%
De este modo, se deduce fácilmente que el resto de colecciones expuestas en los ejemplos
  anteriores no son cerradas bajo subconjuntos.%
\end{isamarkuptext}\isamarkuptrue%
\isacommand{lemma}\isamarkupfalse%
\ {\isachardoublequoteopen}{\isasymnot}\ subset{\isacharunderscore}closed\ {\isacharbraceleft}{\isacharbraceleft}{\isacharparenleft}\isactrlbold {\isasymnot}\ {\isacharparenleft}Atom\ {\isadigit{1}}{\isacharparenright}{\isacharparenright}\ \isactrlbold {\isasymrightarrow}\ Atom\ {\isadigit{2}}{\isacharbraceright}{\isacharcomma}\isanewline
\ \ \ {\isacharbraceleft}{\isacharparenleft}{\isacharparenleft}\isactrlbold {\isasymnot}\ {\isacharparenleft}Atom\ {\isadigit{1}}{\isacharparenright}{\isacharparenright}\ \isactrlbold {\isasymrightarrow}\ Atom\ {\isadigit{2}}{\isacharparenright}{\isacharcomma}\ \isactrlbold {\isasymnot}{\isacharparenleft}\isactrlbold {\isasymnot}\ {\isacharparenleft}Atom\ {\isadigit{1}}{\isacharparenright}{\isacharparenright}{\isacharbraceright}{\isacharcomma}\isanewline
\ \ {\isacharbraceleft}{\isacharparenleft}{\isacharparenleft}\isactrlbold {\isasymnot}\ {\isacharparenleft}Atom\ {\isadigit{1}}{\isacharparenright}{\isacharparenright}\ \isactrlbold {\isasymrightarrow}\ Atom\ {\isadigit{2}}{\isacharparenright}{\isacharcomma}\ \isactrlbold {\isasymnot}{\isacharparenleft}\isactrlbold {\isasymnot}\ {\isacharparenleft}Atom\ {\isadigit{1}}{\isacharparenright}{\isacharparenright}{\isacharcomma}\ \ Atom\ {\isadigit{1}}{\isacharbraceright}{\isacharbraceright}{\isachardoublequoteclose}\ \isanewline
%
\isadelimproof
\ \ %
\endisadelimproof
%
\isatagproof
\isacommand{unfolding}\isamarkupfalse%
\ subset{\isacharunderscore}closed{\isacharunderscore}def\ \isacommand{by}\isamarkupfalse%
\ auto%
\endisatagproof
{\isafoldproof}%
%
\isadelimproof
\isanewline
%
\endisadelimproof
\isanewline
\isacommand{lemma}\isamarkupfalse%
\ {\isachardoublequoteopen}{\isasymnot}\ subset{\isacharunderscore}closed\ {\isacharbraceleft}{\isacharbraceleft}{\isacharparenleft}\isactrlbold {\isasymnot}\ {\isacharparenleft}Atom\ {\isadigit{1}}{\isacharparenright}{\isacharparenright}\ \isactrlbold {\isasymrightarrow}\ Atom\ {\isadigit{2}}{\isacharbraceright}{\isacharcomma}\isanewline
\ \ \ {\isacharbraceleft}{\isacharparenleft}{\isacharparenleft}\isactrlbold {\isasymnot}\ {\isacharparenleft}Atom\ {\isadigit{1}}{\isacharparenright}{\isacharparenright}\ \isactrlbold {\isasymrightarrow}\ Atom\ {\isadigit{2}}{\isacharparenright}{\isacharcomma}\ \isactrlbold {\isasymnot}{\isacharparenleft}\isactrlbold {\isasymnot}\ {\isacharparenleft}Atom\ {\isadigit{1}}{\isacharparenright}{\isacharparenright}{\isacharbraceright}{\isacharbraceright}{\isachardoublequoteclose}\ \isanewline
%
\isadelimproof
\ \ %
\endisadelimproof
%
\isatagproof
\isacommand{unfolding}\isamarkupfalse%
\ subset{\isacharunderscore}closed{\isacharunderscore}def\ \isacommand{by}\isamarkupfalse%
\ auto%
\endisatagproof
{\isafoldproof}%
%
\isadelimproof
%
\endisadelimproof
%
\begin{isamarkuptext}%
Continuemos con la noción de propiedad de carácter finito.

\begin{definicion}
  Una colección de conjuntos tiene la \isa{propiedad\ de\ carácter\ finito} si para cualquier conjunto
  son equivalentes:
  \begin{enumerate}
    \item El conjunto pertenece a la colección.
    \item Todo subconjunto finito suyo pertenece a la colección.
  \end{enumerate}
\end{definicion}

  La formalización en Isabelle/HOL de dicha definición se muestra a continuación.%
\end{isamarkuptext}\isamarkuptrue%
\isacommand{definition}\isamarkupfalse%
\ {\isachardoublequoteopen}finite{\isacharunderscore}character\ C\ {\isasymequiv}\ \isanewline
\ \ \ \ \ \ \ \ \ \ \ \ {\isacharparenleft}{\isasymforall}S{\isachardot}\ S\ {\isasymin}\ C\ {\isasymlongleftrightarrow}\ {\isacharparenleft}{\isasymforall}S{\isacharprime}\ {\isasymsubseteq}\ S{\isachardot}\ finite\ S{\isacharprime}\ {\isasymlongrightarrow}\ S{\isacharprime}\ {\isasymin}\ C{\isacharparenright}{\isacharparenright}{\isachardoublequoteclose}%
\begin{isamarkuptext}%
Distingamos las colecciones de los ejemplos anteriores que tengan la propiedad de carácter 
  finito. Análogamente, puesto que el conjunto vacío es finito y subconjunto de cualquier conjunto, 
  se observa que una condición necesaria para que una colección tenga la propiedad de carácter 
  finito es que contenga al conjunto vacío.%
\end{isamarkuptext}\isamarkuptrue%
\isacommand{lemma}\isamarkupfalse%
\ {\isachardoublequoteopen}finite{\isacharunderscore}character\ {\isacharbraceleft}{\isacharbraceleft}{\isacharbraceright}{\isacharbraceright}{\isachardoublequoteclose}\isanewline
%
\isadelimproof
\ \ %
\endisadelimproof
%
\isatagproof
\isacommand{unfolding}\isamarkupfalse%
\ finite{\isacharunderscore}character{\isacharunderscore}def\ \isacommand{by}\isamarkupfalse%
\ auto%
\endisatagproof
{\isafoldproof}%
%
\isadelimproof
\isanewline
%
\endisadelimproof
\isanewline
\isacommand{lemma}\isamarkupfalse%
\ {\isachardoublequoteopen}{\isasymnot}\ finite{\isacharunderscore}character\ {\isacharbraceleft}{\isacharbraceleft}Atom\ {\isadigit{0}}{\isacharbraceright}{\isacharbraceright}{\isachardoublequoteclose}\isanewline
%
\isadelimproof
\ \ %
\endisadelimproof
%
\isatagproof
\isacommand{unfolding}\isamarkupfalse%
\ finite{\isacharunderscore}character{\isacharunderscore}def\ \isacommand{by}\isamarkupfalse%
\ auto%
\endisatagproof
{\isafoldproof}%
%
\isadelimproof
\isanewline
%
\endisadelimproof
\isanewline
\isacommand{lemma}\isamarkupfalse%
\ {\isachardoublequoteopen}finite{\isacharunderscore}character\ {\isacharbraceleft}{\isacharbraceleft}Atom\ {\isadigit{0}}{\isacharbraceright}{\isacharcomma}{\isacharbraceleft}{\isacharbraceright}{\isacharbraceright}{\isachardoublequoteclose}\isanewline
%
\isadelimproof
\ \ %
\endisadelimproof
%
\isatagproof
\isacommand{unfolding}\isamarkupfalse%
\ finite{\isacharunderscore}character{\isacharunderscore}def\ \isacommand{by}\isamarkupfalse%
\ auto%
\endisatagproof
{\isafoldproof}%
%
\isadelimproof
%
\endisadelimproof
%
\begin{isamarkuptext}%
\begin{lema}
    Toda colección de conjuntos con la propiedad de consistencia proposicional se puede extender a
    una colección que también verifique la propiedad de consistencia proposicional y sea cerrada 
    bajo subconjuntos.
  \end{lema}

  En Isabelle se formaliza el resultado de la siguiente manera.%
\end{isamarkuptext}\isamarkuptrue%
\isacommand{lemma}\isamarkupfalse%
\ {\isachardoublequoteopen}pcp\ C\ {\isasymLongrightarrow}\ {\isasymexists}C{\isacharprime}{\isachardot}\ C\ {\isasymsubseteq}\ C{\isacharprime}\ {\isasymand}\ pcp\ C{\isacharprime}\ {\isasymand}\ subset{\isacharunderscore}closed\ C{\isacharprime}{\isachardoublequoteclose}\isanewline
%
\isadelimproof
\ \ %
\endisadelimproof
%
\isatagproof
\isacommand{oops}\isamarkupfalse%
%
\endisatagproof
{\isafoldproof}%
%
\isadelimproof
%
\endisadelimproof
%
\begin{isamarkuptext}%
Procedamos con su demostración.

\begin{demostracion}
  Dada una colección de conjuntos cualquiera \isa{C}, consideremos la colección formada por los 
  conjuntos tales que son subconjuntos de algún conjunto de \isa{C}. Notemos esta colección por 
  \isa{C{\isacharprime}\ {\isacharequal}\ {\isacharbraceleft}S{\isacharprime}{\isachardot}\ {\isasymexists}S{\isasymin}C{\isachardot}\ S{\isacharprime}\ {\isasymsubseteq}\ S{\isacharbraceright}}. Vamos a probar que, en efecto, \isa{C{\isacharprime}} verifica  las condiciones del lema.

  En primer lugar, veamos que \isa{C} está contenida en \isa{C{\isacharprime}}. Para ello, consideremos un conjunto
  cualquiera perteneciente a \isa{C}. Puesto que la propiedad de contención es reflexiva, dicho conjunto 
  es subconjunto de sí mismo. De este modo, por definición de \isa{C{\isacharprime}}, se verifica que el conjunto 
  pertenece a \isa{C{\isacharprime}}.

  Por otro lado, comprobemos que \isa{C{\isacharprime}} tiene la propiedad de consistencia proposicional.
  Por el lema de caracterización de la propiedad de consistencia proposicional mediante la
  notación uniforme basta probar que, para cualquier conjunto de fórmulas \isa{S} de \isa{C{\isacharprime}}, se 
  verifican las condiciones:
  \begin{itemize}
    \item \isa{{\isasymbottom}} no pertenece a \isa{S}.
    \item Dada \isa{p} una fórmula atómica cualquiera, no se tiene 
    simultáneamente que\\ \isa{p\ {\isasymin}\ S} y \isa{{\isasymnot}\ p\ {\isasymin}\ S}.
    \item Para toda fórmula de tipo \isa{{\isasymalpha}} con componentes \isa{{\isasymalpha}\isactrlsub {\isadigit{1}}} y \isa{{\isasymalpha}\isactrlsub {\isadigit{2}}} tal que \isa{{\isasymalpha}}
    pertenece a \isa{S}, se tiene que \isa{{\isacharbraceleft}{\isasymalpha}\isactrlsub {\isadigit{1}}{\isacharcomma}{\isasymalpha}\isactrlsub {\isadigit{2}}{\isacharbraceright}\ {\isasymunion}\ S} pertenece a \isa{C{\isacharprime}}.
    \item Para toda fórmula de tipo \isa{{\isasymbeta}} con componentes \isa{{\isasymbeta}\isactrlsub {\isadigit{1}}} y \isa{{\isasymbeta}\isactrlsub {\isadigit{2}}} tal que \isa{{\isasymbeta}}
    pertenece a \isa{S}, se tiene que o bien \isa{{\isacharbraceleft}{\isasymbeta}\isactrlsub {\isadigit{1}}{\isacharbraceright}\ {\isasymunion}\ S} pertenece a \isa{C{\isacharprime}} o 
    bien \isa{{\isacharbraceleft}{\isasymbeta}\isactrlsub {\isadigit{2}}{\isacharbraceright}\ {\isasymunion}\ S} pertenece a \isa{C{\isacharprime}}.
  \end{itemize} 

  De este modo, sea \isa{S} un conjunto de fórmulas cualquiera de la colección \isa{C{\isacharprime}}. Por definición de
  dicha colección, existe un conjunto \isa{S{\isacharprime}} pertenciente a \isa{C} tal que \isa{S} está contenido en \isa{S{\isacharprime}}.
  Como \isa{C} tiene la propiedad de consistencia proposicional por hipótesis, verifica las condiciones
  del lema de caracterización de la propiedad de consistencia proposicional mediante la notación 
  uniforme. En particular, puesto que \isa{S{\isacharprime}} pertenece a \isa{C}, se verifica: 
  \begin{itemize}
    \item \isa{{\isasymbottom}} no pertenece a \isa{S{\isacharprime}}.
    \item Dada \isa{p} una fórmula atómica cualquiera, no se tiene 
    simultáneamente que\\ \isa{p\ {\isasymin}\ S{\isacharprime}} y \isa{{\isasymnot}\ p\ {\isasymin}\ S{\isacharprime}}.
    \item Para toda fórmula de tipo \isa{{\isasymalpha}} con componentes \isa{{\isasymalpha}\isactrlsub {\isadigit{1}}} y \isa{{\isasymalpha}\isactrlsub {\isadigit{2}}} tal que \isa{{\isasymalpha}}
    pertenece a \isa{S{\isacharprime}}, se tiene que \isa{{\isacharbraceleft}{\isasymalpha}\isactrlsub {\isadigit{1}}{\isacharcomma}{\isasymalpha}\isactrlsub {\isadigit{2}}{\isacharbraceright}\ {\isasymunion}\ S{\isacharprime}} pertenece a \isa{C}.
    \item Para toda fórmula de tipo \isa{{\isasymbeta}} con componentes \isa{{\isasymbeta}\isactrlsub {\isadigit{1}}} y \isa{{\isasymbeta}\isactrlsub {\isadigit{2}}} tal que \isa{{\isasymbeta}}
    pertenece a \isa{S{\isacharprime}}, se tiene que o bien \isa{{\isacharbraceleft}{\isasymbeta}\isactrlsub {\isadigit{1}}{\isacharbraceright}\ {\isasymunion}\ S{\isacharprime}} pertenece a \isa{C} o 
    bien \isa{{\isacharbraceleft}{\isasymbeta}\isactrlsub {\isadigit{2}}{\isacharbraceright}\ {\isasymunion}\ S{\isacharprime}} pertenece a \isa{C}.
  \end{itemize} 

  Por tanto, como \isa{S} está contenida en \isa{S{\isacharprime}}, se verifica análogamente que \isa{{\isasymbottom}} no pertence a \isa{S}
  y que dada una fórmula atómica cualquiera \isa{p}, no se tiene simultáneamente que\\ \isa{p\ {\isasymin}\ S} y 
  \isa{{\isasymnot}\ p\ {\isasymin}\ S{\isachardot}} Veamos que se verifican el resto de condiciones del lema de caracterización:

  \isa{{\isasymsqdot}\ Condición\ para\ fórmulas\ de\ tipo\ {\isasymalpha}}: Sea una fórmula de tipo \isa{{\isasymalpha}} con componentes \isa{{\isasymalpha}\isactrlsub {\isadigit{1}}} y 
    \isa{{\isasymalpha}\isactrlsub {\isadigit{2}}} tal que \isa{{\isasymalpha}} pertenece a \isa{S}. Como \isa{S} está contenida en \isa{S{\isacharprime}}, tenemos que la fórmula 
    pertence también a \isa{S{\isacharprime}}. De este modo, se verifica que \isa{{\isacharbraceleft}{\isasymalpha}\isactrlsub {\isadigit{1}}{\isacharcomma}{\isasymalpha}\isactrlsub {\isadigit{2}}{\isacharbraceright}\ {\isasymunion}\ S{\isacharprime}} pertenece a la colección 
    \isa{C}. Por otro lado, como el conjunto \isa{S} está contenido en \isa{S{\isacharprime}}, se observa fácilmente que\\
    \isa{{\isacharbraceleft}{\isasymalpha}\isactrlsub {\isadigit{1}}{\isacharcomma}{\isasymalpha}\isactrlsub {\isadigit{2}}{\isacharbraceright}\ {\isasymunion}\ S} está contenido en \isa{{\isacharbraceleft}{\isasymalpha}\isactrlsub {\isadigit{1}}{\isacharcomma}{\isasymalpha}\isactrlsub {\isadigit{2}}{\isacharbraceright}\ {\isasymunion}\ S{\isacharprime}}. Por lo tanto, el conjunto \isa{{\isacharbraceleft}{\isasymalpha}\isactrlsub {\isadigit{1}}{\isacharcomma}{\isasymalpha}\isactrlsub {\isadigit{2}}{\isacharbraceright}\ {\isasymunion}\ S} está en 
    \isa{C{\isacharprime}} por definición de esta, ya que es subconjunto de \isa{{\isacharbraceleft}{\isasymalpha}\isactrlsub {\isadigit{1}}{\isacharcomma}{\isasymalpha}\isactrlsub {\isadigit{2}}{\isacharbraceright}\ {\isasymunion}\ S{\isacharprime}} que pertence a \isa{C}.

  \isa{{\isasymsqdot}\ Condición\ para\ fórmulas\ de\ tipo\ {\isasymbeta}}: Sea una fórmula de tipo \isa{{\isasymbeta}} con componentes \isa{{\isasymbeta}\isactrlsub {\isadigit{1}}} y
    \isa{{\isasymbeta}\isactrlsub {\isadigit{2}}} tal que la fórmula pertenece a \isa{S}. Como el conjunto \isa{S} está contenido en \isa{S{\isacharprime}}, tenemos 
    que la fórmula pertence, a su vez, a \isa{S{\isacharprime}}. De este modo, se verifica que o bien \isa{{\isacharbraceleft}{\isasymbeta}\isactrlsub {\isadigit{1}}{\isacharbraceright}\ {\isasymunion}\ S{\isacharprime}}
    pertenece a \isa{C} o bien \isa{{\isacharbraceleft}{\isasymbeta}\isactrlsub {\isadigit{2}}{\isacharbraceright}\ {\isasymunion}\ S{\isacharprime}} pertence a \isa{C}. Por eliminación de la disyunción anterior, 
    vamos a probar que o bien \isa{{\isacharbraceleft}{\isasymbeta}\isactrlsub {\isadigit{1}}{\isacharbraceright}\ {\isasymunion}\ S} pertenece a \isa{C{\isacharprime}} o bien \isa{{\isacharbraceleft}{\isasymbeta}\isactrlsub {\isadigit{2}}{\isacharbraceright}\ {\isasymunion}\ S} pertenece a \isa{C{\isacharprime}}.
    \begin{itemize}
      \item Supongamos, en primer lugar, que \isa{{\isacharbraceleft}{\isasymbeta}\isactrlsub {\isadigit{1}}{\isacharbraceright}\ {\isasymunion}\ S{\isacharprime}} pertenece a \isa{C}. Puesto que el conjunto \isa{S}
      está contenido en \isa{S{\isacharprime}}, se observa fácilmente que \isa{{\isacharbraceleft}{\isasymbeta}\isactrlsub {\isadigit{1}}{\isacharbraceright}\ {\isasymunion}\ S} está contenido en\\ \isa{{\isacharbraceleft}{\isasymbeta}\isactrlsub {\isadigit{1}}{\isacharbraceright}\ {\isasymunion}\ S{\isacharprime}}.
      Por definición de la colección \isa{C{\isacharprime}}, tenemos que \isa{{\isacharbraceleft}{\isasymbeta}\isactrlsub {\isadigit{1}}{\isacharbraceright}\ {\isasymunion}\ S} pertenece a \isa{C{\isacharprime}}, ya que es
      subconjunto de \isa{{\isacharbraceleft}{\isasymbeta}\isactrlsub {\isadigit{1}}{\isacharbraceright}\ {\isasymunion}\ S{\isacharprime}} que pertenece a \isa{C}. Por tanto, hemos probado que o bien \isa{{\isacharbraceleft}{\isasymbeta}\isactrlsub {\isadigit{1}}{\isacharbraceright}\ {\isasymunion}\ S} 
      pertenece a \isa{C{\isacharprime}} o bien \isa{{\isacharbraceleft}{\isasymbeta}\isactrlsub {\isadigit{2}}{\isacharbraceright}\ {\isasymunion}\ S} pertenece a \isa{C{\isacharprime}}.
      \item Supongamos, finalmente, que \isa{{\isacharbraceleft}{\isasymbeta}\isactrlsub {\isadigit{2}}{\isacharbraceright}\ {\isasymunion}\ S{\isacharprime}} pertenece a \isa{C}. Análogamente obtenemos que
      \isa{{\isacharbraceleft}{\isasymbeta}\isactrlsub {\isadigit{2}}{\isacharbraceright}\ {\isasymunion}\ S} está contenido en \isa{{\isacharbraceleft}{\isasymbeta}\isactrlsub {\isadigit{2}}{\isacharbraceright}\ {\isasymunion}\ S{\isacharprime}}, luego \isa{{\isacharbraceleft}{\isasymbeta}\isactrlsub {\isadigit{2}}{\isacharbraceright}\ {\isasymunion}\ S} pertenece a \isa{C{\isacharprime}} por definición.
      Por tanto, o bien \isa{{\isacharbraceleft}{\isasymbeta}\isactrlsub {\isadigit{1}}{\isacharbraceright}\ {\isasymunion}\ S} pertenece a \isa{C{\isacharprime}} o bien \isa{{\isacharbraceleft}{\isasymbeta}\isactrlsub {\isadigit{2}}{\isacharbraceright}\ {\isasymunion}\ S} pertenece a \isa{C{\isacharprime}}.
    \end{itemize}
    De esta manera, queda probado que dada una fórmula de tipo \isa{{\isasymbeta}} y componentes \isa{{\isasymbeta}\isactrlsub {\isadigit{1}}} y \isa{{\isasymbeta}\isactrlsub {\isadigit{2}}} tal que
    pertenezca al conjunto \isa{S}, se verifica que o bien \isa{{\isacharbraceleft}{\isasymbeta}\isactrlsub {\isadigit{1}}{\isacharbraceright}\ {\isasymunion}\ S} pertenece a \isa{C{\isacharprime}} o bien \isa{{\isacharbraceleft}{\isasymbeta}\isactrlsub {\isadigit{2}}{\isacharbraceright}\ {\isasymunion}\ S}
    pertenece a \isa{C{\isacharprime}}.

  En conclusión, por el lema de caracterización de la propiedad de consistencia proposicional
  mediante la notación uniforme, queda probado que \isa{C{\isacharprime}} tiene la propiedad de consistencia
  proposicional. 

  Finalmente probemos que, además, \isa{C{\isacharprime}} es cerrada bajo subconjuntos. Por definición de ser cerrado
  bajo subconjuntos, basta probar que dado un conjunto perteneciente a \isa{C{\isacharprime}} verifica que todo 
  subconjunto suyo pertenece a \isa{C{\isacharprime}}. Consideremos \isa{S} un conjunto cualquiera de \isa{C{\isacharprime}}. Por
  definición de \isa{C{\isacharprime}}, existe un conjunto \isa{S{\isacharprime}} perteneciente a la colección \isa{C} tal que \isa{S} es
  subconjunto de \isa{S{\isacharprime}}. Sea \isa{S{\isacharprime}{\isacharprime}} un subconjunto cualquiera de \isa{S}. Como \isa{S} es subconjunto de \isa{S{\isacharprime}},
  se tiene que \isa{S{\isacharprime}{\isacharprime}} es, a su vez, subconjunto de \isa{S{\isacharprime}}. De este modo, existe un conjunto 
  perteneciente a la colección \isa{C} del cual \isa{S{\isacharprime}{\isacharprime}} es subconjunto. Por tanto, por definición de \isa{C{\isacharprime}}, 
  \isa{S{\isacharprime}{\isacharprime}} pertenece a la colección \isa{C{\isacharprime}}, como quería demostrar.
\end{demostracion}

  Procedamos con las demostraciones del lema en Isabelle/HOL.

  En primer lugar, vamos a introducir dos lemas auxiliares que emplearemos a lo largo de
  esta sección. El primero se trata de un lema similar al lema \isa{ballI} definido en Isabelle pero 
  considerando la relación de contención en lugar de la de pertenencia.%
\end{isamarkuptext}\isamarkuptrue%
\isacommand{lemma}\isamarkupfalse%
\ sallI{\isacharcolon}\ {\isachardoublequoteopen}{\isacharparenleft}{\isasymAnd}S{\isachardot}\ S\ {\isasymsubseteq}\ A\ {\isasymLongrightarrow}\ P\ S{\isacharparenright}\ {\isasymLongrightarrow}\ {\isasymforall}S\ {\isasymsubseteq}\ A{\isachardot}\ P\ S{\isachardoublequoteclose}\isanewline
%
\isadelimproof
\ \ %
\endisadelimproof
%
\isatagproof
\isacommand{by}\isamarkupfalse%
\ simp%
\endisatagproof
{\isafoldproof}%
%
\isadelimproof
%
\endisadelimproof
%
\begin{isamarkuptext}%
Por último definimos el siguiente lema auxiliar similar al lema \isa{bspec} de Isabelle/HOL
  considerando, análogamente, la relación de contención en lugar de la de pertenencia.%
\end{isamarkuptext}\isamarkuptrue%
\isacommand{lemma}\isamarkupfalse%
\ sspec{\isacharcolon}\ {\isachardoublequoteopen}{\isasymforall}S\ {\isasymsubseteq}\ A{\isachardot}\ P\ S\ {\isasymLongrightarrow}\ S\ {\isasymsubseteq}\ A\ {\isasymLongrightarrow}\ P\ S{\isachardoublequoteclose}\isanewline
%
\isadelimproof
\ \ %
\endisadelimproof
%
\isatagproof
\isacommand{by}\isamarkupfalse%
\ simp%
\endisatagproof
{\isafoldproof}%
%
\isadelimproof
%
\endisadelimproof
%
\begin{isamarkuptext}%
Veamos la prueba detallada del lema en Isabelle/HOL. Esta se fundamenta en tres lemas
  auxiliares: el primero prueba que la colección \isa{C} está contenida en \isa{C{\isacharprime}}, el segundo que
  \isa{C{\isacharprime}} tiene la propiedad de consistencia proposicional y, finalmente, el tercer lema demuestra que
  \isa{C{\isacharprime}} es cerrada bajo subconjuntos. En primer lugar, dada una colección cualquiera \isa{C}, definiremos 
  en Isabelle su extensión \isa{C{\isacharprime}} como sigue.%
\end{isamarkuptext}\isamarkuptrue%
\isacommand{definition}\isamarkupfalse%
\ extensionSC\ {\isacharcolon}{\isacharcolon}\ {\isachardoublequoteopen}{\isacharparenleft}{\isacharparenleft}{\isacharprime}a\ formula{\isacharparenright}\ set{\isacharparenright}\ set\ {\isasymRightarrow}\ {\isacharparenleft}{\isacharparenleft}{\isacharprime}a\ formula{\isacharparenright}\ set{\isacharparenright}\ set{\isachardoublequoteclose}\isanewline
\ \ \isakeyword{where}\ extensionSC{\isacharcolon}\ {\isachardoublequoteopen}extensionSC\ C\ {\isacharequal}\ {\isacharbraceleft}s{\isachardot}\ {\isasymexists}S{\isasymin}C{\isachardot}\ s\ {\isasymsubseteq}\ S{\isacharbraceright}{\isachardoublequoteclose}%
\begin{isamarkuptext}%
Una vez formalizada la extensión en Isabelle, comencemos probando de manera detallada que toda
  colección está contenida en su extensión así definida.%
\end{isamarkuptext}\isamarkuptrue%
\isacommand{lemma}\isamarkupfalse%
\ ex{\isadigit{1}}{\isacharunderscore}subset{\isacharcolon}\ {\isachardoublequoteopen}C\ {\isasymsubseteq}\ {\isacharparenleft}extensionSC\ C{\isacharparenright}{\isachardoublequoteclose}\isanewline
%
\isadelimproof
%
\endisadelimproof
%
\isatagproof
\isacommand{proof}\isamarkupfalse%
\ {\isacharparenleft}rule\ subsetI{\isacharparenright}\isanewline
\ \ \isacommand{fix}\isamarkupfalse%
\ s\isanewline
\ \ \isacommand{assume}\isamarkupfalse%
\ {\isachardoublequoteopen}s\ {\isasymin}\ C{\isachardoublequoteclose}\isanewline
\ \ \isacommand{have}\isamarkupfalse%
\ {\isachardoublequoteopen}s\ {\isasymsubseteq}\ s{\isachardoublequoteclose}\isanewline
\ \ \ \ \isacommand{by}\isamarkupfalse%
\ {\isacharparenleft}rule\ subset{\isacharunderscore}refl{\isacharparenright}\isanewline
\ \ \isacommand{then}\isamarkupfalse%
\ \isacommand{have}\isamarkupfalse%
\ {\isachardoublequoteopen}{\isasymexists}S{\isasymin}C{\isachardot}\ s\ {\isasymsubseteq}\ S{\isachardoublequoteclose}\isanewline
\ \ \ \ \isacommand{using}\isamarkupfalse%
\ {\isacartoucheopen}s\ {\isasymin}\ C{\isacartoucheclose}\ \isacommand{by}\isamarkupfalse%
\ {\isacharparenleft}rule\ bexI{\isacharparenright}\isanewline
\ \ \isacommand{thus}\isamarkupfalse%
\ {\isachardoublequoteopen}s\ {\isasymin}\ {\isacharparenleft}extensionSC\ C{\isacharparenright}{\isachardoublequoteclose}\isanewline
\ \ \ \ \isacommand{by}\isamarkupfalse%
\ {\isacharparenleft}simp\ only{\isacharcolon}\ mem{\isacharunderscore}Collect{\isacharunderscore}eq\ extensionSC{\isacharparenright}\isanewline
\isacommand{qed}\isamarkupfalse%
%
\endisatagproof
{\isafoldproof}%
%
\isadelimproof
%
\endisadelimproof
%
\begin{isamarkuptext}%
Prosigamos con la prueba del lema auxiliar que demuestra que \isa{C{\isacharprime}} tiene la propiedad
  de consistencia proposicional. Para ello, emplearemos un lema auxiliar que amplia el lema de 
  Isabelle \isa{insert{\isacharunderscore}is{\isacharunderscore}Un} para la unión de dos elementos y un conjunto, como se muestra a 
  continuación.%
\end{isamarkuptext}\isamarkuptrue%
\isacommand{lemma}\isamarkupfalse%
\ insertSetElem{\isacharcolon}\ {\isachardoublequoteopen}insert\ a\ {\isacharparenleft}insert\ b\ C{\isacharparenright}\ {\isacharequal}\ {\isacharbraceleft}a{\isacharcomma}b{\isacharbraceright}\ {\isasymunion}\ C{\isachardoublequoteclose}\isanewline
%
\isadelimproof
\ \ %
\endisadelimproof
%
\isatagproof
\isacommand{by}\isamarkupfalse%
\ simp%
\endisatagproof
{\isafoldproof}%
%
\isadelimproof
%
\endisadelimproof
%
\begin{isamarkuptext}%
Una vez introducido dicho lema auxiliar, podemos dar la prueba detallada del lema que 
  demuestra que \isa{C{\isacharprime}} tiene la propiedad de consistencia proposicional.%
\end{isamarkuptext}\isamarkuptrue%
\isacommand{lemma}\isamarkupfalse%
\ ex{\isadigit{1}}{\isacharunderscore}pcp{\isacharcolon}\ \isanewline
\ \ \isakeyword{assumes}\ {\isachardoublequoteopen}pcp\ C{\isachardoublequoteclose}\isanewline
\ \ \isakeyword{shows}\ {\isachardoublequoteopen}pcp\ {\isacharparenleft}extensionSC\ C{\isacharparenright}{\isachardoublequoteclose}\isanewline
%
\isadelimproof
%
\endisadelimproof
%
\isatagproof
\isacommand{proof}\isamarkupfalse%
\ {\isacharminus}\isanewline
\ \ \isacommand{have}\isamarkupfalse%
\ C{\isadigit{1}}{\isacharcolon}\ {\isachardoublequoteopen}C\ {\isasymsubseteq}\ {\isacharparenleft}extensionSC\ C{\isacharparenright}{\isachardoublequoteclose}\isanewline
\ \ \ \ \isacommand{by}\isamarkupfalse%
\ {\isacharparenleft}rule\ ex{\isadigit{1}}{\isacharunderscore}subset{\isacharparenright}\isanewline
\ \ \isacommand{show}\isamarkupfalse%
\ {\isachardoublequoteopen}pcp\ {\isacharparenleft}extensionSC\ C{\isacharparenright}{\isachardoublequoteclose}\isanewline
\ \ \isacommand{proof}\isamarkupfalse%
\ {\isacharparenleft}rule\ pcp{\isacharunderscore}alt{\isadigit{2}}{\isacharparenright}\isanewline
\ \ \ \ \isacommand{show}\isamarkupfalse%
\ {\isachardoublequoteopen}{\isasymforall}S\ {\isasymin}\ {\isacharparenleft}extensionSC\ C{\isacharparenright}{\isachardot}\ {\isacharparenleft}{\isasymbottom}\ {\isasymnotin}\ S\isanewline
\ \ \ \ {\isasymand}\ {\isacharparenleft}{\isasymforall}k{\isachardot}\ Atom\ k\ {\isasymin}\ S\ {\isasymlongrightarrow}\ \isactrlbold {\isasymnot}\ {\isacharparenleft}Atom\ k{\isacharparenright}\ {\isasymin}\ S\ {\isasymlongrightarrow}\ False{\isacharparenright}\isanewline
\ \ \ \ {\isasymand}\ {\isacharparenleft}{\isasymforall}F\ G\ H{\isachardot}\ Con\ F\ G\ H\ {\isasymlongrightarrow}\ F\ {\isasymin}\ S\ {\isasymlongrightarrow}\ {\isacharbraceleft}G{\isacharcomma}H{\isacharbraceright}\ {\isasymunion}\ S\ {\isasymin}\ {\isacharparenleft}extensionSC\ C{\isacharparenright}{\isacharparenright}\isanewline
\ \ \ \ {\isasymand}\ {\isacharparenleft}{\isasymforall}F\ G\ H{\isachardot}\ Dis\ F\ G\ H\ {\isasymlongrightarrow}\ F\ {\isasymin}\ S\ {\isasymlongrightarrow}\ {\isacharbraceleft}G{\isacharbraceright}\ {\isasymunion}\ S\ {\isasymin}\ {\isacharparenleft}extensionSC\ C{\isacharparenright}\ {\isasymor}\ {\isacharbraceleft}H{\isacharbraceright}\ {\isasymunion}\ S\ {\isasymin}\ {\isacharparenleft}extensionSC\ C{\isacharparenright}{\isacharparenright}{\isacharparenright}{\isachardoublequoteclose}\isanewline
\ \ \ \ \isacommand{proof}\isamarkupfalse%
\ {\isacharparenleft}rule\ ballI{\isacharparenright}\isanewline
\ \ \ \ \ \ \isacommand{fix}\isamarkupfalse%
\ S{\isacharprime}\isanewline
\ \ \ \ \ \ \isacommand{assume}\isamarkupfalse%
\ {\isachardoublequoteopen}S{\isacharprime}\ {\isasymin}\ {\isacharparenleft}extensionSC\ C{\isacharparenright}{\isachardoublequoteclose}\isanewline
\ \ \ \ \ \ \isacommand{then}\isamarkupfalse%
\ \isacommand{have}\isamarkupfalse%
\ {\isadigit{1}}{\isacharcolon}{\isachardoublequoteopen}{\isasymexists}S\ {\isasymin}\ C{\isachardot}\ S{\isacharprime}\ {\isasymsubseteq}\ S{\isachardoublequoteclose}\isanewline
\ \ \ \ \ \ \ \ \isacommand{unfolding}\isamarkupfalse%
\ extensionSC\ \isacommand{by}\isamarkupfalse%
\ {\isacharparenleft}rule\ CollectD{\isacharparenright}\ \ \isanewline
\ \ \ \ \ \ \isacommand{obtain}\isamarkupfalse%
\ S\ \isakeyword{where}\ {\isachardoublequoteopen}S\ {\isasymin}\ C{\isachardoublequoteclose}\ {\isachardoublequoteopen}S{\isacharprime}\ {\isasymsubseteq}\ S{\isachardoublequoteclose}\isanewline
\ \ \ \ \ \ \ \ \isacommand{using}\isamarkupfalse%
\ {\isadigit{1}}\ \isacommand{by}\isamarkupfalse%
\ {\isacharparenleft}rule\ bexE{\isacharparenright}\isanewline
\ \ \ \ \ \ \isacommand{have}\isamarkupfalse%
\ {\isachardoublequoteopen}{\isasymforall}S\ {\isasymin}\ C{\isachardot}\isanewline
\ \ \ \ \ \ {\isasymbottom}\ {\isasymnotin}\ S\isanewline
\ \ \ \ \ \ {\isasymand}\ {\isacharparenleft}{\isasymforall}k{\isachardot}\ Atom\ k\ {\isasymin}\ S\ {\isasymlongrightarrow}\ \isactrlbold {\isasymnot}\ {\isacharparenleft}Atom\ k{\isacharparenright}\ {\isasymin}\ S\ {\isasymlongrightarrow}\ False{\isacharparenright}\isanewline
\ \ \ \ \ \ {\isasymand}\ {\isacharparenleft}{\isasymforall}F\ G\ H{\isachardot}\ Con\ F\ G\ H\ {\isasymlongrightarrow}\ F\ {\isasymin}\ S\ {\isasymlongrightarrow}\ {\isacharbraceleft}G{\isacharcomma}H{\isacharbraceright}\ {\isasymunion}\ S\ {\isasymin}\ C{\isacharparenright}\isanewline
\ \ \ \ \ \ {\isasymand}\ {\isacharparenleft}{\isasymforall}F\ G\ H{\isachardot}\ Dis\ F\ G\ H\ {\isasymlongrightarrow}\ F\ {\isasymin}\ S\ {\isasymlongrightarrow}\ {\isacharbraceleft}G{\isacharbraceright}\ {\isasymunion}\ S\ {\isasymin}\ C\ {\isasymor}\ {\isacharbraceleft}H{\isacharbraceright}\ {\isasymunion}\ S\ {\isasymin}\ C{\isacharparenright}{\isachardoublequoteclose}\isanewline
\ \ \ \ \ \ \ \ \isacommand{using}\isamarkupfalse%
\ assms\ \isacommand{by}\isamarkupfalse%
\ {\isacharparenleft}rule\ pcp{\isacharunderscore}alt{\isadigit{1}}{\isacharparenright}\isanewline
\ \ \ \ \ \ \isacommand{then}\isamarkupfalse%
\ \isacommand{have}\isamarkupfalse%
\ H{\isacharcolon}{\isachardoublequoteopen}{\isasymbottom}\ {\isasymnotin}\ S\isanewline
\ \ \ \ \ \ {\isasymand}\ {\isacharparenleft}{\isasymforall}k{\isachardot}\ Atom\ k\ {\isasymin}\ S\ {\isasymlongrightarrow}\ \isactrlbold {\isasymnot}\ {\isacharparenleft}Atom\ k{\isacharparenright}\ {\isasymin}\ S\ {\isasymlongrightarrow}\ False{\isacharparenright}\isanewline
\ \ \ \ \ \ {\isasymand}\ {\isacharparenleft}{\isasymforall}F\ G\ H{\isachardot}\ Con\ F\ G\ H\ {\isasymlongrightarrow}\ F\ {\isasymin}\ S\ {\isasymlongrightarrow}\ {\isacharbraceleft}G{\isacharcomma}H{\isacharbraceright}\ {\isasymunion}\ S\ {\isasymin}\ C{\isacharparenright}\isanewline
\ \ \ \ \ \ {\isasymand}\ {\isacharparenleft}{\isasymforall}F\ G\ H{\isachardot}\ Dis\ F\ G\ H\ {\isasymlongrightarrow}\ F\ {\isasymin}\ S\ {\isasymlongrightarrow}\ {\isacharbraceleft}G{\isacharbraceright}\ {\isasymunion}\ S\ {\isasymin}\ C\ {\isasymor}\ {\isacharbraceleft}H{\isacharbraceright}\ {\isasymunion}\ S\ {\isasymin}\ C{\isacharparenright}{\isachardoublequoteclose}\isanewline
\ \ \ \ \ \ \ \ \isacommand{using}\isamarkupfalse%
\ {\isacartoucheopen}S\ {\isasymin}\ C{\isacartoucheclose}\ \isacommand{by}\isamarkupfalse%
\ {\isacharparenleft}rule\ bspec{\isacharparenright}\isanewline
\ \ \ \ \ \ \isacommand{then}\isamarkupfalse%
\ \isacommand{have}\isamarkupfalse%
\ {\isachardoublequoteopen}{\isasymbottom}\ {\isasymnotin}\ S{\isachardoublequoteclose}\isanewline
\ \ \ \ \ \ \ \ \isacommand{by}\isamarkupfalse%
\ {\isacharparenleft}rule\ conjunct{\isadigit{1}}{\isacharparenright}\isanewline
\ \ \ \ \ \ \isacommand{have}\isamarkupfalse%
\ S{\isadigit{1}}{\isacharcolon}{\isachardoublequoteopen}{\isasymbottom}\ {\isasymnotin}\ S{\isacharprime}{\isachardoublequoteclose}\isanewline
\ \ \ \ \ \ \ \ \isacommand{using}\isamarkupfalse%
\ {\isacartoucheopen}S{\isacharprime}\ {\isasymsubseteq}\ S{\isacartoucheclose}\ {\isacartoucheopen}{\isasymbottom}\ {\isasymnotin}\ S{\isacartoucheclose}\ \isacommand{by}\isamarkupfalse%
\ {\isacharparenleft}rule\ contra{\isacharunderscore}subsetD{\isacharparenright}\isanewline
\ \ \ \ \ \ \isacommand{have}\isamarkupfalse%
\ Atom{\isacharcolon}{\isachardoublequoteopen}{\isasymforall}k{\isachardot}\ Atom\ k\ {\isasymin}\ S\ {\isasymlongrightarrow}\ \isactrlbold {\isasymnot}\ {\isacharparenleft}Atom\ k{\isacharparenright}\ {\isasymin}\ S\ {\isasymlongrightarrow}\ False{\isachardoublequoteclose}\isanewline
\ \ \ \ \ \ \ \ \isacommand{using}\isamarkupfalse%
\ H\ \isacommand{by}\isamarkupfalse%
\ {\isacharparenleft}iprover\ elim{\isacharcolon}\ conjunct{\isadigit{1}}\ conjunct{\isadigit{2}}{\isacharparenright}\isanewline
\ \ \ \ \ \ \isacommand{have}\isamarkupfalse%
\ S{\isadigit{2}}{\isacharcolon}{\isachardoublequoteopen}{\isasymforall}k{\isachardot}\ Atom\ k\ {\isasymin}\ S{\isacharprime}\ {\isasymlongrightarrow}\ \isactrlbold {\isasymnot}\ {\isacharparenleft}Atom\ k{\isacharparenright}\ {\isasymin}\ S{\isacharprime}\ {\isasymlongrightarrow}\ False{\isachardoublequoteclose}\isanewline
\ \ \ \ \ \ \isacommand{proof}\isamarkupfalse%
\ {\isacharparenleft}rule\ allI{\isacharparenright}\isanewline
\ \ \ \ \ \ \ \ \isacommand{fix}\isamarkupfalse%
\ k\isanewline
\ \ \ \ \ \ \ \ \isacommand{show}\isamarkupfalse%
\ {\isachardoublequoteopen}Atom\ k\ {\isasymin}\ S{\isacharprime}\ {\isasymlongrightarrow}\ \isactrlbold {\isasymnot}\ {\isacharparenleft}Atom\ k{\isacharparenright}\ {\isasymin}\ S{\isacharprime}\ {\isasymlongrightarrow}\ False{\isachardoublequoteclose}\isanewline
\ \ \ \ \ \ \ \ \isacommand{proof}\isamarkupfalse%
\ {\isacharparenleft}rule\ impI{\isacharparenright}{\isacharplus}\isanewline
\ \ \ \ \ \ \ \ \ \ \isacommand{assume}\isamarkupfalse%
\ {\isachardoublequoteopen}Atom\ k\ {\isasymin}\ S{\isacharprime}{\isachardoublequoteclose}\isanewline
\ \ \ \ \ \ \ \ \ \ \isacommand{assume}\isamarkupfalse%
\ {\isachardoublequoteopen}\isactrlbold {\isasymnot}\ {\isacharparenleft}Atom\ k{\isacharparenright}\ {\isasymin}\ S{\isacharprime}{\isachardoublequoteclose}\isanewline
\ \ \ \ \ \ \ \ \ \ \isacommand{have}\isamarkupfalse%
\ {\isachardoublequoteopen}Atom\ k\ {\isasymin}\ S{\isachardoublequoteclose}\ \isanewline
\ \ \ \ \ \ \ \ \ \ \ \ \isacommand{using}\isamarkupfalse%
\ {\isacartoucheopen}S{\isacharprime}\ {\isasymsubseteq}\ S{\isacartoucheclose}\ {\isacartoucheopen}Atom\ k\ {\isasymin}\ S{\isacharprime}{\isacartoucheclose}\ \isacommand{by}\isamarkupfalse%
\ {\isacharparenleft}rule\ set{\isacharunderscore}mp{\isacharparenright}\isanewline
\ \ \ \ \ \ \ \ \ \ \isacommand{have}\isamarkupfalse%
\ {\isachardoublequoteopen}\isactrlbold {\isasymnot}\ {\isacharparenleft}Atom\ k{\isacharparenright}\ {\isasymin}\ S{\isachardoublequoteclose}\isanewline
\ \ \ \ \ \ \ \ \ \ \ \ \isacommand{using}\isamarkupfalse%
\ {\isacartoucheopen}S{\isacharprime}\ {\isasymsubseteq}\ S{\isacartoucheclose}\ {\isacartoucheopen}\isactrlbold {\isasymnot}\ {\isacharparenleft}Atom\ k{\isacharparenright}\ {\isasymin}\ S{\isacharprime}{\isacartoucheclose}\ \isacommand{by}\isamarkupfalse%
\ {\isacharparenleft}rule\ set{\isacharunderscore}mp{\isacharparenright}\isanewline
\ \ \ \ \ \ \ \ \ \ \isacommand{have}\isamarkupfalse%
\ {\isachardoublequoteopen}Atom\ k\ {\isasymin}\ S\ {\isasymlongrightarrow}\ \isactrlbold {\isasymnot}\ {\isacharparenleft}Atom\ k{\isacharparenright}\ {\isasymin}\ S\ {\isasymlongrightarrow}\ False{\isachardoublequoteclose}\isanewline
\ \ \ \ \ \ \ \ \ \ \ \ \isacommand{using}\isamarkupfalse%
\ Atom\ \isacommand{by}\isamarkupfalse%
\ {\isacharparenleft}rule\ allE{\isacharparenright}\isanewline
\ \ \ \ \ \ \ \ \ \ \isacommand{then}\isamarkupfalse%
\ \isacommand{have}\isamarkupfalse%
\ {\isachardoublequoteopen}\isactrlbold {\isasymnot}\ {\isacharparenleft}Atom\ k{\isacharparenright}\ {\isasymin}\ S\ {\isasymlongrightarrow}\ False{\isachardoublequoteclose}\isanewline
\ \ \ \ \ \ \ \ \ \ \ \ \isacommand{using}\isamarkupfalse%
\ {\isacartoucheopen}Atom\ k\ {\isasymin}\ S{\isacartoucheclose}\ \isacommand{by}\isamarkupfalse%
\ {\isacharparenleft}rule\ mp{\isacharparenright}\isanewline
\ \ \ \ \ \ \ \ \ \ \isacommand{thus}\isamarkupfalse%
\ {\isachardoublequoteopen}False{\isachardoublequoteclose}\isanewline
\ \ \ \ \ \ \ \ \ \ \ \ \isacommand{using}\isamarkupfalse%
\ {\isacartoucheopen}\isactrlbold {\isasymnot}\ {\isacharparenleft}Atom\ k{\isacharparenright}\ {\isasymin}\ S{\isacartoucheclose}\ \isacommand{by}\isamarkupfalse%
\ {\isacharparenleft}rule\ mp{\isacharparenright}\isanewline
\ \ \ \ \ \ \ \ \isacommand{qed}\isamarkupfalse%
\isanewline
\ \ \ \ \ \ \isacommand{qed}\isamarkupfalse%
\isanewline
\ \ \ \ \ \ \isacommand{have}\isamarkupfalse%
\ Con{\isacharcolon}{\isachardoublequoteopen}{\isasymforall}F\ G\ H{\isachardot}\ Con\ F\ G\ H\ {\isasymlongrightarrow}\ F\ {\isasymin}\ S\ {\isasymlongrightarrow}\ {\isacharbraceleft}G{\isacharcomma}H{\isacharbraceright}\ {\isasymunion}\ S\ {\isasymin}\ C{\isachardoublequoteclose}\isanewline
\ \ \ \ \ \ \ \ \isacommand{using}\isamarkupfalse%
\ H\ \isacommand{by}\isamarkupfalse%
\ {\isacharparenleft}iprover\ elim{\isacharcolon}\ conjunct{\isadigit{1}}\ conjunct{\isadigit{2}}{\isacharparenright}\isanewline
\ \ \ \ \ \ \isacommand{have}\isamarkupfalse%
\ S{\isadigit{3}}{\isacharcolon}{\isachardoublequoteopen}{\isasymforall}F\ G\ H{\isachardot}\ Con\ F\ G\ H\ {\isasymlongrightarrow}\ F\ {\isasymin}\ S{\isacharprime}\ {\isasymlongrightarrow}\ {\isacharbraceleft}G{\isacharcomma}H{\isacharbraceright}\ {\isasymunion}\ S{\isacharprime}\ {\isasymin}\ {\isacharparenleft}extensionSC\ C{\isacharparenright}{\isachardoublequoteclose}\isanewline
\ \ \ \ \ \ \isacommand{proof}\isamarkupfalse%
\ {\isacharparenleft}rule\ allI{\isacharparenright}{\isacharplus}\isanewline
\ \ \ \ \ \ \ \ \isacommand{fix}\isamarkupfalse%
\ F\ G\ H\isanewline
\ \ \ \ \ \ \ \ \isacommand{show}\isamarkupfalse%
\ {\isachardoublequoteopen}Con\ F\ G\ H\ {\isasymlongrightarrow}\ F\ {\isasymin}\ S{\isacharprime}\ {\isasymlongrightarrow}\ {\isacharbraceleft}G{\isacharcomma}H{\isacharbraceright}\ {\isasymunion}\ S{\isacharprime}\ {\isasymin}\ {\isacharparenleft}extensionSC\ C{\isacharparenright}{\isachardoublequoteclose}\isanewline
\ \ \ \ \ \ \ \ \isacommand{proof}\isamarkupfalse%
\ {\isacharparenleft}rule\ impI{\isacharparenright}{\isacharplus}\isanewline
\ \ \ \ \ \ \ \ \ \ \isacommand{assume}\isamarkupfalse%
\ {\isachardoublequoteopen}Con\ F\ G\ H{\isachardoublequoteclose}\isanewline
\ \ \ \ \ \ \ \ \ \ \isacommand{assume}\isamarkupfalse%
\ {\isachardoublequoteopen}F\ {\isasymin}\ S{\isacharprime}{\isachardoublequoteclose}\isanewline
\ \ \ \ \ \ \ \ \ \ \isacommand{have}\isamarkupfalse%
\ {\isachardoublequoteopen}F\ {\isasymin}\ S{\isachardoublequoteclose}\isanewline
\ \ \ \ \ \ \ \ \ \ \ \ \isacommand{using}\isamarkupfalse%
\ {\isacartoucheopen}S{\isacharprime}\ {\isasymsubseteq}\ S{\isacartoucheclose}\ {\isacartoucheopen}F\ {\isasymin}\ S{\isacharprime}{\isacartoucheclose}\ \isacommand{by}\isamarkupfalse%
\ {\isacharparenleft}rule\ set{\isacharunderscore}mp{\isacharparenright}\isanewline
\ \ \ \ \ \ \ \ \ \ \isacommand{have}\isamarkupfalse%
\ {\isachardoublequoteopen}Con\ F\ G\ H\ {\isasymlongrightarrow}\ F\ {\isasymin}\ S\ {\isasymlongrightarrow}\ {\isacharbraceleft}G{\isacharcomma}H{\isacharbraceright}\ {\isasymunion}\ S\ {\isasymin}\ C{\isachardoublequoteclose}\isanewline
\ \ \ \ \ \ \ \ \ \ \ \ \isacommand{using}\isamarkupfalse%
\ Con\ \isacommand{by}\isamarkupfalse%
\ {\isacharparenleft}iprover\ elim{\isacharcolon}\ allE{\isacharparenright}\isanewline
\ \ \ \ \ \ \ \ \ \ \isacommand{then}\isamarkupfalse%
\ \isacommand{have}\isamarkupfalse%
\ {\isachardoublequoteopen}F\ {\isasymin}\ S\ {\isasymlongrightarrow}\ {\isacharbraceleft}G{\isacharcomma}H{\isacharbraceright}\ {\isasymunion}\ S\ {\isasymin}\ C{\isachardoublequoteclose}\isanewline
\ \ \ \ \ \ \ \ \ \ \ \ \isacommand{using}\isamarkupfalse%
\ {\isacartoucheopen}Con\ F\ G\ H{\isacartoucheclose}\ \isacommand{by}\isamarkupfalse%
\ {\isacharparenleft}rule\ mp{\isacharparenright}\isanewline
\ \ \ \ \ \ \ \ \ \ \isacommand{then}\isamarkupfalse%
\ \isacommand{have}\isamarkupfalse%
\ {\isachardoublequoteopen}{\isacharbraceleft}G{\isacharcomma}H{\isacharbraceright}\ {\isasymunion}\ S\ {\isasymin}\ C{\isachardoublequoteclose}\isanewline
\ \ \ \ \ \ \ \ \ \ \ \ \isacommand{using}\isamarkupfalse%
\ {\isacartoucheopen}F\ {\isasymin}\ S{\isacartoucheclose}\ \isacommand{by}\isamarkupfalse%
\ {\isacharparenleft}rule\ mp{\isacharparenright}\isanewline
\ \ \ \ \ \ \ \ \ \ \isacommand{have}\isamarkupfalse%
\ {\isachardoublequoteopen}S{\isacharprime}\ {\isasymsubseteq}\ insert\ H\ S{\isachardoublequoteclose}\isanewline
\ \ \ \ \ \ \ \ \ \ \ \ \isacommand{using}\isamarkupfalse%
\ {\isacartoucheopen}S{\isacharprime}\ {\isasymsubseteq}\ S{\isacartoucheclose}\ \isacommand{by}\isamarkupfalse%
\ {\isacharparenleft}rule\ subset{\isacharunderscore}insertI{\isadigit{2}}{\isacharparenright}\ \isanewline
\ \ \ \ \ \ \ \ \ \ \isacommand{then}\isamarkupfalse%
\ \isacommand{have}\isamarkupfalse%
\ {\isachardoublequoteopen}insert\ H\ S{\isacharprime}\ {\isasymsubseteq}\ insert\ H\ {\isacharparenleft}insert\ H\ S{\isacharparenright}{\isachardoublequoteclose}\isanewline
\ \ \ \ \ \ \ \ \ \ \ \ \isacommand{by}\isamarkupfalse%
\ {\isacharparenleft}simp\ only{\isacharcolon}\ insert{\isacharunderscore}mono{\isacharparenright}\isanewline
\ \ \ \ \ \ \ \ \ \ \isacommand{then}\isamarkupfalse%
\ \isacommand{have}\isamarkupfalse%
\ {\isachardoublequoteopen}insert\ H\ S{\isacharprime}\ {\isasymsubseteq}\ insert\ H\ S{\isachardoublequoteclose}\isanewline
\ \ \ \ \ \ \ \ \ \ \ \ \isacommand{by}\isamarkupfalse%
\ {\isacharparenleft}simp\ only{\isacharcolon}\ insert{\isacharunderscore}absorb{\isadigit{2}}{\isacharparenright}\isanewline
\ \ \ \ \ \ \ \ \ \ \isacommand{then}\isamarkupfalse%
\ \isacommand{have}\isamarkupfalse%
\ {\isachardoublequoteopen}insert\ G\ {\isacharparenleft}insert\ H\ S{\isacharprime}{\isacharparenright}\ {\isasymsubseteq}\ insert\ G\ {\isacharparenleft}insert\ H\ S{\isacharparenright}{\isachardoublequoteclose}\isanewline
\ \ \ \ \ \ \ \ \ \ \ \ \isacommand{by}\isamarkupfalse%
\ {\isacharparenleft}simp\ only{\isacharcolon}\ insert{\isacharunderscore}mono{\isacharparenright}\isanewline
\ \ \ \ \ \ \ \ \ \ \isacommand{have}\isamarkupfalse%
\ A{\isacharcolon}{\isachardoublequoteopen}insert\ G\ {\isacharparenleft}insert\ H\ S{\isacharprime}{\isacharparenright}\ {\isacharequal}\ {\isacharbraceleft}G{\isacharcomma}H{\isacharbraceright}\ {\isasymunion}\ S{\isacharprime}{\isachardoublequoteclose}\isanewline
\ \ \ \ \ \ \ \ \ \ \ \ \isacommand{by}\isamarkupfalse%
\ {\isacharparenleft}rule\ insertSetElem{\isacharparenright}\ \isanewline
\ \ \ \ \ \ \ \ \ \ \isacommand{have}\isamarkupfalse%
\ B{\isacharcolon}{\isachardoublequoteopen}insert\ G\ {\isacharparenleft}insert\ H\ S{\isacharparenright}\ {\isacharequal}\ {\isacharbraceleft}G{\isacharcomma}H{\isacharbraceright}\ {\isasymunion}\ S{\isachardoublequoteclose}\isanewline
\ \ \ \ \ \ \ \ \ \ \ \ \isacommand{by}\isamarkupfalse%
\ {\isacharparenleft}rule\ insertSetElem{\isacharparenright}\isanewline
\ \ \ \ \ \ \ \ \ \ \isacommand{have}\isamarkupfalse%
\ {\isachardoublequoteopen}{\isacharbraceleft}G{\isacharcomma}H{\isacharbraceright}\ {\isasymunion}\ S{\isacharprime}\ {\isasymsubseteq}\ {\isacharbraceleft}G{\isacharcomma}H{\isacharbraceright}\ {\isasymunion}\ S{\isachardoublequoteclose}\ \isanewline
\ \ \ \ \ \ \ \ \ \ \ \ \isacommand{using}\isamarkupfalse%
\ {\isacartoucheopen}insert\ G\ {\isacharparenleft}insert\ H\ S{\isacharprime}{\isacharparenright}\ {\isasymsubseteq}\ insert\ G\ {\isacharparenleft}insert\ H\ S{\isacharparenright}{\isacartoucheclose}\ \isacommand{by}\isamarkupfalse%
\ {\isacharparenleft}simp\ only{\isacharcolon}\ A\ B{\isacharparenright}\isanewline
\ \ \ \ \ \ \ \ \ \ \isacommand{then}\isamarkupfalse%
\ \isacommand{have}\isamarkupfalse%
\ {\isachardoublequoteopen}{\isasymexists}S\ {\isasymin}\ C{\isachardot}\ {\isacharbraceleft}G{\isacharcomma}H{\isacharbraceright}\ {\isasymunion}\ S{\isacharprime}\ {\isasymsubseteq}\ S{\isachardoublequoteclose}\isanewline
\ \ \ \ \ \ \ \ \ \ \ \ \isacommand{using}\isamarkupfalse%
\ {\isacartoucheopen}{\isacharbraceleft}G{\isacharcomma}H{\isacharbraceright}\ {\isasymunion}\ S\ {\isasymin}\ C{\isacartoucheclose}\ \isacommand{by}\isamarkupfalse%
\ {\isacharparenleft}rule\ bexI{\isacharparenright}\isanewline
\ \ \ \ \ \ \ \ \ \ \isacommand{thus}\isamarkupfalse%
\ {\isachardoublequoteopen}{\isacharbraceleft}G{\isacharcomma}H{\isacharbraceright}\ {\isasymunion}\ S{\isacharprime}\ {\isasymin}\ {\isacharparenleft}extensionSC\ C{\isacharparenright}{\isachardoublequoteclose}\ \isanewline
\ \ \ \ \ \ \ \ \ \ \ \ \isacommand{unfolding}\isamarkupfalse%
\ extensionSC\ \isacommand{by}\isamarkupfalse%
\ {\isacharparenleft}rule\ CollectI{\isacharparenright}\isanewline
\ \ \ \ \ \ \ \ \isacommand{qed}\isamarkupfalse%
\isanewline
\ \ \ \ \ \ \isacommand{qed}\isamarkupfalse%
\isanewline
\ \ \ \ \ \ \isacommand{have}\isamarkupfalse%
\ Dis{\isacharcolon}{\isachardoublequoteopen}{\isasymforall}F\ G\ H{\isachardot}\ Dis\ F\ G\ H\ {\isasymlongrightarrow}\ F\ {\isasymin}\ S\ {\isasymlongrightarrow}\ {\isacharbraceleft}G{\isacharbraceright}\ {\isasymunion}\ S\ {\isasymin}\ C\ {\isasymor}\ {\isacharbraceleft}H{\isacharbraceright}\ {\isasymunion}\ S\ {\isasymin}\ C{\isachardoublequoteclose}\isanewline
\ \ \ \ \ \ \ \ \isacommand{using}\isamarkupfalse%
\ H\ \isacommand{by}\isamarkupfalse%
\ {\isacharparenleft}iprover\ elim{\isacharcolon}\ conjunct{\isadigit{2}}{\isacharparenright}\isanewline
\ \ \ \ \ \ \isacommand{have}\isamarkupfalse%
\ S{\isadigit{4}}{\isacharcolon}{\isachardoublequoteopen}{\isasymforall}F\ G\ H{\isachardot}\ Dis\ F\ G\ H\ {\isasymlongrightarrow}\ F\ {\isasymin}\ S{\isacharprime}\ {\isasymlongrightarrow}\ {\isacharbraceleft}G{\isacharbraceright}\ {\isasymunion}\ S{\isacharprime}\ {\isasymin}\ {\isacharparenleft}extensionSC\ C{\isacharparenright}\ {\isasymor}\ {\isacharbraceleft}H{\isacharbraceright}\ {\isasymunion}\ S{\isacharprime}\ {\isasymin}\ {\isacharparenleft}extensionSC\ C{\isacharparenright}{\isachardoublequoteclose}\isanewline
\ \ \ \ \ \ \isacommand{proof}\isamarkupfalse%
\ {\isacharparenleft}rule\ allI{\isacharparenright}{\isacharplus}\isanewline
\ \ \ \ \ \ \ \ \isacommand{fix}\isamarkupfalse%
\ F\ G\ H\isanewline
\ \ \ \ \ \ \ \ \isacommand{show}\isamarkupfalse%
\ {\isachardoublequoteopen}Dis\ F\ G\ H\ {\isasymlongrightarrow}\ F\ {\isasymin}\ S{\isacharprime}\ {\isasymlongrightarrow}\ {\isacharbraceleft}G{\isacharbraceright}\ {\isasymunion}\ S{\isacharprime}\ {\isasymin}\ {\isacharparenleft}extensionSC\ C{\isacharparenright}\ {\isasymor}\ {\isacharbraceleft}H{\isacharbraceright}\ {\isasymunion}\ S{\isacharprime}\ {\isasymin}\ {\isacharparenleft}extensionSC\ C{\isacharparenright}{\isachardoublequoteclose}\isanewline
\ \ \ \ \ \ \ \ \isacommand{proof}\isamarkupfalse%
\ {\isacharparenleft}rule\ impI{\isacharparenright}{\isacharplus}\isanewline
\ \ \ \ \ \ \ \ \ \ \isacommand{assume}\isamarkupfalse%
\ {\isachardoublequoteopen}Dis\ F\ G\ H{\isachardoublequoteclose}\isanewline
\ \ \ \ \ \ \ \ \ \ \isacommand{assume}\isamarkupfalse%
\ {\isachardoublequoteopen}F\ {\isasymin}\ S{\isacharprime}{\isachardoublequoteclose}\isanewline
\ \ \ \ \ \ \ \ \ \ \isacommand{have}\isamarkupfalse%
\ {\isachardoublequoteopen}F\ {\isasymin}\ S{\isachardoublequoteclose}\isanewline
\ \ \ \ \ \ \ \ \ \ \ \ \isacommand{using}\isamarkupfalse%
\ {\isacartoucheopen}S{\isacharprime}\ {\isasymsubseteq}\ S{\isacartoucheclose}\ {\isacartoucheopen}F\ {\isasymin}\ S{\isacharprime}{\isacartoucheclose}\ \isacommand{by}\isamarkupfalse%
\ {\isacharparenleft}rule\ set{\isacharunderscore}mp{\isacharparenright}\isanewline
\ \ \ \ \ \ \ \ \ \ \isacommand{have}\isamarkupfalse%
\ {\isachardoublequoteopen}Dis\ F\ G\ H\ {\isasymlongrightarrow}\ F\ {\isasymin}\ S\ {\isasymlongrightarrow}\ {\isacharbraceleft}G{\isacharbraceright}\ {\isasymunion}\ S\ {\isasymin}\ C\ {\isasymor}\ {\isacharbraceleft}H{\isacharbraceright}\ {\isasymunion}\ S\ {\isasymin}\ C{\isachardoublequoteclose}\isanewline
\ \ \ \ \ \ \ \ \ \ \ \ \isacommand{using}\isamarkupfalse%
\ Dis\ \isacommand{by}\isamarkupfalse%
\ {\isacharparenleft}iprover\ elim{\isacharcolon}\ allE{\isacharparenright}\isanewline
\ \ \ \ \ \ \ \ \ \ \isacommand{then}\isamarkupfalse%
\ \isacommand{have}\isamarkupfalse%
\ {\isachardoublequoteopen}F\ {\isasymin}\ S\ {\isasymlongrightarrow}\ {\isacharbraceleft}G{\isacharbraceright}\ {\isasymunion}\ S\ {\isasymin}\ C\ {\isasymor}\ {\isacharbraceleft}H{\isacharbraceright}\ {\isasymunion}\ S\ {\isasymin}\ C{\isachardoublequoteclose}\isanewline
\ \ \ \ \ \ \ \ \ \ \ \ \isacommand{using}\isamarkupfalse%
\ {\isacartoucheopen}Dis\ F\ G\ H{\isacartoucheclose}\ \isacommand{by}\isamarkupfalse%
\ {\isacharparenleft}rule\ mp{\isacharparenright}\isanewline
\ \ \ \ \ \ \ \ \ \ \isacommand{then}\isamarkupfalse%
\ \isacommand{have}\isamarkupfalse%
\ {\isadigit{9}}{\isacharcolon}{\isachardoublequoteopen}{\isacharbraceleft}G{\isacharbraceright}\ {\isasymunion}\ S\ {\isasymin}\ C\ {\isasymor}\ {\isacharbraceleft}H{\isacharbraceright}\ {\isasymunion}\ S\ {\isasymin}\ C{\isachardoublequoteclose}\isanewline
\ \ \ \ \ \ \ \ \ \ \ \ \isacommand{using}\isamarkupfalse%
\ {\isacartoucheopen}F\ {\isasymin}\ S{\isacartoucheclose}\ \isacommand{by}\isamarkupfalse%
\ {\isacharparenleft}rule\ mp{\isacharparenright}\isanewline
\ \ \ \ \ \ \ \ \ \ \isacommand{show}\isamarkupfalse%
\ {\isachardoublequoteopen}{\isacharbraceleft}G{\isacharbraceright}\ {\isasymunion}\ S{\isacharprime}\ {\isasymin}\ {\isacharparenleft}extensionSC\ C{\isacharparenright}\ {\isasymor}\ {\isacharbraceleft}H{\isacharbraceright}\ {\isasymunion}\ S{\isacharprime}\ {\isasymin}\ {\isacharparenleft}extensionSC\ C{\isacharparenright}{\isachardoublequoteclose}\isanewline
\ \ \ \ \ \ \ \ \ \ \ \ \isacommand{using}\isamarkupfalse%
\ {\isadigit{9}}\isanewline
\ \ \ \ \ \ \ \ \ \ \isacommand{proof}\isamarkupfalse%
\ {\isacharparenleft}rule\ disjE{\isacharparenright}\isanewline
\ \ \ \ \ \ \ \ \ \ \ \ \isacommand{assume}\isamarkupfalse%
\ {\isachardoublequoteopen}{\isacharbraceleft}G{\isacharbraceright}\ {\isasymunion}\ S\ {\isasymin}\ C{\isachardoublequoteclose}\isanewline
\ \ \ \ \ \ \ \ \ \ \ \ \isacommand{have}\isamarkupfalse%
\ {\isachardoublequoteopen}insert\ G\ S{\isacharprime}\ {\isasymsubseteq}\ insert\ G\ S{\isachardoublequoteclose}\isanewline
\ \ \ \ \ \ \ \ \ \ \ \ \ \ \isacommand{using}\isamarkupfalse%
\ {\isacartoucheopen}S{\isacharprime}\ {\isasymsubseteq}\ S{\isacartoucheclose}\ \isacommand{by}\isamarkupfalse%
\ {\isacharparenleft}simp\ only{\isacharcolon}\ insert{\isacharunderscore}mono{\isacharparenright}\isanewline
\ \ \ \ \ \ \ \ \ \ \ \ \isacommand{have}\isamarkupfalse%
\ C{\isacharcolon}{\isachardoublequoteopen}insert\ G\ S{\isacharprime}\ {\isacharequal}\ {\isacharbraceleft}G{\isacharbraceright}\ {\isasymunion}\ S{\isacharprime}{\isachardoublequoteclose}\isanewline
\ \ \ \ \ \ \ \ \ \ \ \ \ \ \isacommand{by}\isamarkupfalse%
\ {\isacharparenleft}rule\ insert{\isacharunderscore}is{\isacharunderscore}Un{\isacharparenright}\isanewline
\ \ \ \ \ \ \ \ \ \ \ \ \isacommand{have}\isamarkupfalse%
\ D{\isacharcolon}{\isachardoublequoteopen}insert\ G\ S\ {\isacharequal}\ {\isacharbraceleft}G{\isacharbraceright}\ {\isasymunion}\ S{\isachardoublequoteclose}\isanewline
\ \ \ \ \ \ \ \ \ \ \ \ \ \ \isacommand{by}\isamarkupfalse%
\ {\isacharparenleft}rule\ insert{\isacharunderscore}is{\isacharunderscore}Un{\isacharparenright}\isanewline
\ \ \ \ \ \ \ \ \ \ \ \ \isacommand{have}\isamarkupfalse%
\ {\isachardoublequoteopen}{\isacharbraceleft}G{\isacharbraceright}\ {\isasymunion}\ S{\isacharprime}\ {\isasymsubseteq}\ {\isacharbraceleft}G{\isacharbraceright}\ {\isasymunion}\ S{\isachardoublequoteclose}\isanewline
\ \ \ \ \ \ \ \ \ \ \ \ \ \ \isacommand{using}\isamarkupfalse%
\ {\isacartoucheopen}insert\ G\ S{\isacharprime}\ {\isasymsubseteq}\ insert\ G\ S{\isacartoucheclose}\ \isacommand{by}\isamarkupfalse%
\ {\isacharparenleft}simp\ only{\isacharcolon}\ C\ D{\isacharparenright}\isanewline
\ \ \ \ \ \ \ \ \ \ \ \ \isacommand{then}\isamarkupfalse%
\ \isacommand{have}\isamarkupfalse%
\ {\isachardoublequoteopen}{\isasymexists}S\ {\isasymin}\ C{\isachardot}\ {\isacharbraceleft}G{\isacharbraceright}\ {\isasymunion}\ S{\isacharprime}\ {\isasymsubseteq}\ S{\isachardoublequoteclose}\isanewline
\ \ \ \ \ \ \ \ \ \ \ \ \ \ \isacommand{using}\isamarkupfalse%
\ {\isacartoucheopen}{\isacharbraceleft}G{\isacharbraceright}\ {\isasymunion}\ S\ {\isasymin}\ C{\isacartoucheclose}\ \isacommand{by}\isamarkupfalse%
\ {\isacharparenleft}rule\ bexI{\isacharparenright}\isanewline
\ \ \ \ \ \ \ \ \ \ \ \ \isacommand{then}\isamarkupfalse%
\ \isacommand{have}\isamarkupfalse%
\ {\isachardoublequoteopen}{\isacharbraceleft}G{\isacharbraceright}\ {\isasymunion}\ S{\isacharprime}\ {\isasymin}\ {\isacharparenleft}extensionSC\ C{\isacharparenright}{\isachardoublequoteclose}\isanewline
\ \ \ \ \ \ \ \ \ \ \ \ \ \ \isacommand{unfolding}\isamarkupfalse%
\ extensionSC\ \isacommand{by}\isamarkupfalse%
\ {\isacharparenleft}rule\ CollectI{\isacharparenright}\isanewline
\ \ \ \ \ \ \ \ \ \ \ \ \isacommand{thus}\isamarkupfalse%
\ {\isachardoublequoteopen}{\isacharbraceleft}G{\isacharbraceright}\ {\isasymunion}\ S{\isacharprime}\ {\isasymin}\ {\isacharparenleft}extensionSC\ C{\isacharparenright}\ {\isasymor}\ {\isacharbraceleft}H{\isacharbraceright}\ {\isasymunion}\ S{\isacharprime}\ {\isasymin}\ {\isacharparenleft}extensionSC\ C{\isacharparenright}{\isachardoublequoteclose}\isanewline
\ \ \ \ \ \ \ \ \ \ \ \ \ \ \isacommand{by}\isamarkupfalse%
\ {\isacharparenleft}rule\ disjI{\isadigit{1}}{\isacharparenright}\isanewline
\ \ \ \ \ \ \ \ \ \ \isacommand{next}\isamarkupfalse%
\isanewline
\ \ \ \ \ \ \ \ \ \ \ \ \isacommand{assume}\isamarkupfalse%
\ {\isachardoublequoteopen}{\isacharbraceleft}H{\isacharbraceright}\ {\isasymunion}\ S\ {\isasymin}\ C{\isachardoublequoteclose}\isanewline
\ \ \ \ \ \ \ \ \ \ \ \ \isacommand{have}\isamarkupfalse%
\ {\isachardoublequoteopen}insert\ H\ S{\isacharprime}\ {\isasymsubseteq}\ insert\ H\ S{\isachardoublequoteclose}\isanewline
\ \ \ \ \ \ \ \ \ \ \ \ \ \ \isacommand{using}\isamarkupfalse%
\ {\isacartoucheopen}S{\isacharprime}\ {\isasymsubseteq}\ S{\isacartoucheclose}\ \isacommand{by}\isamarkupfalse%
\ {\isacharparenleft}simp\ only{\isacharcolon}\ insert{\isacharunderscore}mono{\isacharparenright}\isanewline
\ \ \ \ \ \ \ \ \ \ \ \ \isacommand{have}\isamarkupfalse%
\ E{\isacharcolon}{\isachardoublequoteopen}insert\ H\ S{\isacharprime}\ {\isacharequal}\ {\isacharbraceleft}H{\isacharbraceright}\ {\isasymunion}\ S{\isacharprime}{\isachardoublequoteclose}\isanewline
\ \ \ \ \ \ \ \ \ \ \ \ \ \ \isacommand{by}\isamarkupfalse%
\ {\isacharparenleft}rule\ insert{\isacharunderscore}is{\isacharunderscore}Un{\isacharparenright}\isanewline
\ \ \ \ \ \ \ \ \ \ \ \ \isacommand{have}\isamarkupfalse%
\ F{\isacharcolon}{\isachardoublequoteopen}insert\ H\ S\ {\isacharequal}\ {\isacharbraceleft}H{\isacharbraceright}\ {\isasymunion}\ S{\isachardoublequoteclose}\isanewline
\ \ \ \ \ \ \ \ \ \ \ \ \ \ \isacommand{by}\isamarkupfalse%
\ {\isacharparenleft}rule\ insert{\isacharunderscore}is{\isacharunderscore}Un{\isacharparenright}\isanewline
\ \ \ \ \ \ \ \ \ \ \ \ \isacommand{then}\isamarkupfalse%
\ \isacommand{have}\isamarkupfalse%
\ {\isachardoublequoteopen}{\isacharbraceleft}H{\isacharbraceright}\ {\isasymunion}\ S{\isacharprime}\ {\isasymsubseteq}\ {\isacharbraceleft}H{\isacharbraceright}\ {\isasymunion}\ S{\isachardoublequoteclose}\isanewline
\ \ \ \ \ \ \ \ \ \ \ \ \ \ \isacommand{using}\isamarkupfalse%
\ {\isacartoucheopen}insert\ H\ S{\isacharprime}\ {\isasymsubseteq}\ insert\ H\ S{\isacartoucheclose}\ \isacommand{by}\isamarkupfalse%
\ {\isacharparenleft}simp\ only{\isacharcolon}\ E\ F{\isacharparenright}\isanewline
\ \ \ \ \ \ \ \ \ \ \ \ \isacommand{then}\isamarkupfalse%
\ \isacommand{have}\isamarkupfalse%
\ {\isachardoublequoteopen}{\isasymexists}S\ {\isasymin}\ C{\isachardot}\ {\isacharbraceleft}H{\isacharbraceright}\ {\isasymunion}\ S{\isacharprime}\ {\isasymsubseteq}\ S{\isachardoublequoteclose}\isanewline
\ \ \ \ \ \ \ \ \ \ \ \ \ \ \isacommand{using}\isamarkupfalse%
\ {\isacartoucheopen}{\isacharbraceleft}H{\isacharbraceright}\ {\isasymunion}\ S\ {\isasymin}\ C{\isacartoucheclose}\ \isacommand{by}\isamarkupfalse%
\ {\isacharparenleft}rule\ bexI{\isacharparenright}\isanewline
\ \ \ \ \ \ \ \ \ \ \ \ \isacommand{then}\isamarkupfalse%
\ \isacommand{have}\isamarkupfalse%
\ {\isachardoublequoteopen}{\isacharbraceleft}H{\isacharbraceright}\ {\isasymunion}\ S{\isacharprime}\ {\isasymin}\ {\isacharparenleft}extensionSC\ C{\isacharparenright}{\isachardoublequoteclose}\isanewline
\ \ \ \ \ \ \ \ \ \ \ \ \ \ \isacommand{unfolding}\isamarkupfalse%
\ extensionSC\ \isacommand{by}\isamarkupfalse%
\ {\isacharparenleft}rule\ CollectI{\isacharparenright}\isanewline
\ \ \ \ \ \ \ \ \ \ \ \ \isacommand{thus}\isamarkupfalse%
\ {\isachardoublequoteopen}{\isacharbraceleft}G{\isacharbraceright}\ {\isasymunion}\ S{\isacharprime}\ {\isasymin}\ {\isacharparenleft}extensionSC\ C{\isacharparenright}\ {\isasymor}\ {\isacharbraceleft}H{\isacharbraceright}\ {\isasymunion}\ S{\isacharprime}\ {\isasymin}\ {\isacharparenleft}extensionSC\ C{\isacharparenright}{\isachardoublequoteclose}\isanewline
\ \ \ \ \ \ \ \ \ \ \ \ \ \ \isacommand{by}\isamarkupfalse%
\ {\isacharparenleft}rule\ disjI{\isadigit{2}}{\isacharparenright}\isanewline
\ \ \ \ \ \ \ \ \ \ \isacommand{qed}\isamarkupfalse%
\isanewline
\ \ \ \ \ \ \ \ \isacommand{qed}\isamarkupfalse%
\isanewline
\ \ \ \ \ \ \isacommand{qed}\isamarkupfalse%
\isanewline
\ \ \ \ \ \ \isacommand{show}\isamarkupfalse%
\ {\isachardoublequoteopen}{\isasymbottom}\ {\isasymnotin}\ S{\isacharprime}\isanewline
\ \ \ \ {\isasymand}\ {\isacharparenleft}{\isasymforall}k{\isachardot}\ Atom\ k\ {\isasymin}\ S{\isacharprime}\ {\isasymlongrightarrow}\ \isactrlbold {\isasymnot}\ {\isacharparenleft}Atom\ k{\isacharparenright}\ {\isasymin}\ S{\isacharprime}\ {\isasymlongrightarrow}\ False{\isacharparenright}\isanewline
\ \ \ \ {\isasymand}\ {\isacharparenleft}{\isasymforall}F\ G\ H{\isachardot}\ Con\ F\ G\ H\ {\isasymlongrightarrow}\ F\ {\isasymin}\ S{\isacharprime}\ {\isasymlongrightarrow}\ {\isacharbraceleft}G{\isacharcomma}H{\isacharbraceright}\ {\isasymunion}\ S{\isacharprime}\ {\isasymin}\ {\isacharparenleft}extensionSC\ C{\isacharparenright}{\isacharparenright}\isanewline
\ \ \ \ {\isasymand}\ {\isacharparenleft}{\isasymforall}F\ G\ H{\isachardot}\ Dis\ F\ G\ H\ {\isasymlongrightarrow}\ F\ {\isasymin}\ S{\isacharprime}\ {\isasymlongrightarrow}\ {\isacharbraceleft}G{\isacharbraceright}\ {\isasymunion}\ S{\isacharprime}\ {\isasymin}\ {\isacharparenleft}extensionSC\ C{\isacharparenright}\ {\isasymor}\ {\isacharbraceleft}H{\isacharbraceright}\ {\isasymunion}\ S{\isacharprime}\ {\isasymin}\ {\isacharparenleft}extensionSC\ C{\isacharparenright}{\isacharparenright}{\isachardoublequoteclose}\isanewline
\ \ \ \ \ \ \ \ \isacommand{using}\isamarkupfalse%
\ S{\isadigit{1}}\ S{\isadigit{2}}\ S{\isadigit{3}}\ S{\isadigit{4}}\ \isacommand{by}\isamarkupfalse%
\ {\isacharparenleft}iprover\ intro{\isacharcolon}\ conjI{\isacharparenright}\isanewline
\ \ \ \ \isacommand{qed}\isamarkupfalse%
\isanewline
\ \ \isacommand{qed}\isamarkupfalse%
\isanewline
\isacommand{qed}\isamarkupfalse%
%
\endisatagproof
{\isafoldproof}%
%
\isadelimproof
%
\endisadelimproof
%
\begin{isamarkuptext}%
Finalmente, el siguiente lema auxiliar prueba que \isa{C{\isacharprime}} es cerrada bajo subconjuntos.%
\end{isamarkuptext}\isamarkuptrue%
\isacommand{lemma}\isamarkupfalse%
\ ex{\isadigit{1}}{\isacharunderscore}subset{\isacharunderscore}closed{\isacharcolon}\isanewline
\ \ \isakeyword{assumes}\ {\isachardoublequoteopen}pcp\ C{\isachardoublequoteclose}\isanewline
\ \ \isakeyword{shows}\ {\isachardoublequoteopen}subset{\isacharunderscore}closed\ {\isacharparenleft}extensionSC\ C{\isacharparenright}{\isachardoublequoteclose}\isanewline
%
\isadelimproof
\ \ %
\endisadelimproof
%
\isatagproof
\isacommand{unfolding}\isamarkupfalse%
\ subset{\isacharunderscore}closed{\isacharunderscore}def\isanewline
\isacommand{proof}\isamarkupfalse%
\ {\isacharparenleft}rule\ ballI{\isacharparenright}\isanewline
\ \ \isacommand{fix}\isamarkupfalse%
\ S{\isacharprime}\isanewline
\ \ \isacommand{assume}\isamarkupfalse%
\ {\isachardoublequoteopen}S{\isacharprime}\ {\isasymin}\ {\isacharparenleft}extensionSC\ C{\isacharparenright}{\isachardoublequoteclose}\isanewline
\ \ \isacommand{then}\isamarkupfalse%
\ \isacommand{have}\isamarkupfalse%
\ H{\isacharcolon}{\isachardoublequoteopen}{\isasymexists}S\ {\isasymin}\ C{\isachardot}\ S{\isacharprime}\ {\isasymsubseteq}\ S{\isachardoublequoteclose}\isanewline
\ \ \ \ \isacommand{unfolding}\isamarkupfalse%
\ extensionSC\ \isacommand{by}\isamarkupfalse%
\ {\isacharparenleft}rule\ CollectD{\isacharparenright}\isanewline
\ \ \isacommand{obtain}\isamarkupfalse%
\ S\ \isakeyword{where}\ {\isacartoucheopen}S\ {\isasymin}\ C{\isacartoucheclose}\ \isakeyword{and}\ {\isacartoucheopen}S{\isacharprime}\ {\isasymsubseteq}\ S{\isacartoucheclose}\ \isanewline
\ \ \ \ \isacommand{using}\isamarkupfalse%
\ H\ \isacommand{by}\isamarkupfalse%
\ {\isacharparenleft}rule\ bexE{\isacharparenright}\ \isanewline
\ \ \isacommand{show}\isamarkupfalse%
\ {\isachardoublequoteopen}{\isasymforall}S{\isacharprime}{\isacharprime}\ {\isasymsubseteq}\ S{\isacharprime}{\isachardot}\ S{\isacharprime}{\isacharprime}\ {\isasymin}\ {\isacharparenleft}extensionSC\ C{\isacharparenright}{\isachardoublequoteclose}\isanewline
\ \ \isacommand{proof}\isamarkupfalse%
\ {\isacharparenleft}rule\ sallI{\isacharparenright}\isanewline
\ \ \ \ \isacommand{fix}\isamarkupfalse%
\ S{\isacharprime}{\isacharprime}\isanewline
\ \ \ \ \isacommand{assume}\isamarkupfalse%
\ {\isachardoublequoteopen}S{\isacharprime}{\isacharprime}\ {\isasymsubseteq}\ S{\isacharprime}{\isachardoublequoteclose}\ \isanewline
\ \ \ \ \isacommand{then}\isamarkupfalse%
\ \isacommand{have}\isamarkupfalse%
\ {\isachardoublequoteopen}S{\isacharprime}{\isacharprime}\ {\isasymsubseteq}\ S{\isachardoublequoteclose}\isanewline
\ \ \ \ \ \ \isacommand{using}\isamarkupfalse%
\ {\isacartoucheopen}S{\isacharprime}\ {\isasymsubseteq}\ S{\isacartoucheclose}\ \isacommand{by}\isamarkupfalse%
\ {\isacharparenleft}rule\ subset{\isacharunderscore}trans{\isacharparenright}\isanewline
\ \ \ \ \isacommand{then}\isamarkupfalse%
\ \isacommand{have}\isamarkupfalse%
\ {\isachardoublequoteopen}{\isasymexists}S\ {\isasymin}\ C{\isachardot}\ S{\isacharprime}{\isacharprime}\ {\isasymsubseteq}\ S{\isachardoublequoteclose}\isanewline
\ \ \ \ \ \ \isacommand{using}\isamarkupfalse%
\ {\isacartoucheopen}S\ {\isasymin}\ C{\isacartoucheclose}\ \isacommand{by}\isamarkupfalse%
\ {\isacharparenleft}rule\ bexI{\isacharparenright}\isanewline
\ \ \ \ \isacommand{thus}\isamarkupfalse%
\ {\isachardoublequoteopen}S{\isacharprime}{\isacharprime}\ {\isasymin}\ {\isacharparenleft}extensionSC\ C{\isacharparenright}{\isachardoublequoteclose}\isanewline
\ \ \ \ \ \ \isacommand{unfolding}\isamarkupfalse%
\ extensionSC\ \isacommand{by}\isamarkupfalse%
\ {\isacharparenleft}rule\ CollectI{\isacharparenright}\isanewline
\ \ \isacommand{qed}\isamarkupfalse%
\isanewline
\isacommand{qed}\isamarkupfalse%
%
\endisatagproof
{\isafoldproof}%
%
\isadelimproof
%
\endisadelimproof
%
\begin{isamarkuptext}%
En conclusión, la prueba detallada del lema completo se muestra a continuación.%
\end{isamarkuptext}\isamarkuptrue%
\isacommand{lemma}\isamarkupfalse%
\ ex{\isadigit{1}}{\isacharcolon}\ \isanewline
\ \ \isakeyword{assumes}\ {\isachardoublequoteopen}pcp\ C{\isachardoublequoteclose}\isanewline
\ \ \isakeyword{shows}\ {\isachardoublequoteopen}{\isasymexists}C{\isacharprime}{\isachardot}\ C\ {\isasymsubseteq}\ C{\isacharprime}\ {\isasymand}\ pcp\ C{\isacharprime}\ {\isasymand}\ subset{\isacharunderscore}closed\ C{\isacharprime}{\isachardoublequoteclose}\isanewline
%
\isadelimproof
%
\endisadelimproof
%
\isatagproof
\isacommand{proof}\isamarkupfalse%
\ {\isacharminus}\isanewline
\ \ \isacommand{have}\isamarkupfalse%
\ C{\isadigit{1}}{\isacharcolon}{\isachardoublequoteopen}C\ {\isasymsubseteq}\ {\isacharparenleft}extensionSC\ C{\isacharparenright}{\isachardoublequoteclose}\isanewline
\ \ \ \ \isacommand{by}\isamarkupfalse%
\ {\isacharparenleft}rule\ ex{\isadigit{1}}{\isacharunderscore}subset{\isacharparenright}\isanewline
\ \ \isacommand{have}\isamarkupfalse%
\ C{\isadigit{2}}{\isacharcolon}{\isachardoublequoteopen}pcp\ {\isacharparenleft}extensionSC\ C{\isacharparenright}{\isachardoublequoteclose}\isanewline
\ \ \ \ \isacommand{using}\isamarkupfalse%
\ assms\ \isacommand{by}\isamarkupfalse%
\ {\isacharparenleft}rule\ ex{\isadigit{1}}{\isacharunderscore}pcp{\isacharparenright}\isanewline
\ \ \isacommand{have}\isamarkupfalse%
\ C{\isadigit{3}}{\isacharcolon}{\isachardoublequoteopen}subset{\isacharunderscore}closed\ {\isacharparenleft}extensionSC\ C{\isacharparenright}{\isachardoublequoteclose}\isanewline
\ \ \ \ \isacommand{using}\isamarkupfalse%
\ assms\ \isacommand{by}\isamarkupfalse%
\ {\isacharparenleft}rule\ ex{\isadigit{1}}{\isacharunderscore}subset{\isacharunderscore}closed{\isacharparenright}\isanewline
\ \ \isacommand{have}\isamarkupfalse%
\ {\isachardoublequoteopen}C\ {\isasymsubseteq}\ {\isacharparenleft}extensionSC\ C{\isacharparenright}\ {\isasymand}\ pcp\ {\isacharparenleft}extensionSC\ C{\isacharparenright}\ {\isasymand}\ subset{\isacharunderscore}closed\ {\isacharparenleft}extensionSC\ C{\isacharparenright}{\isachardoublequoteclose}\ \isanewline
\ \ \ \ \isacommand{using}\isamarkupfalse%
\ C{\isadigit{1}}\ C{\isadigit{2}}\ C{\isadigit{3}}\ \isacommand{by}\isamarkupfalse%
\ {\isacharparenleft}iprover\ intro{\isacharcolon}\ conjI{\isacharparenright}\isanewline
\ \ \isacommand{thus}\isamarkupfalse%
\ {\isacharquery}thesis\isanewline
\ \ \ \ \isacommand{by}\isamarkupfalse%
\ {\isacharparenleft}rule\ exI{\isacharparenright}\isanewline
\isacommand{qed}\isamarkupfalse%
%
\endisatagproof
{\isafoldproof}%
%
\isadelimproof
%
\endisadelimproof
%
\begin{isamarkuptext}%
Continuemos con el segundo resultado de este apartado.

  \begin{lema}
  Toda colección de conjuntos con la propiedad de carácter finito es cerrada bajo subconjuntos.
  \end{lema}

  En Isabelle, se formaliza como sigue.%
\end{isamarkuptext}\isamarkuptrue%
\isacommand{lemma}\isamarkupfalse%
\ \isanewline
\ \ \isakeyword{assumes}\ {\isachardoublequoteopen}finite{\isacharunderscore}character\ C{\isachardoublequoteclose}\isanewline
\ \ \isakeyword{shows}\ {\isachardoublequoteopen}subset{\isacharunderscore}closed\ C{\isachardoublequoteclose}\isanewline
%
\isadelimproof
\ \ %
\endisadelimproof
%
\isatagproof
\isacommand{oops}\isamarkupfalse%
%
\endisatagproof
{\isafoldproof}%
%
\isadelimproof
%
\endisadelimproof
%
\begin{isamarkuptext}%
Procedamos con la demostración del resultado.

  \begin{demostracion}
    Consideremos una colección de conjuntos \isa{C} con la propiedad de carácter finito. Probemos que, 
    en efecto, es cerrada bajo subconjuntos. Por definición de esta última propiedad, basta 
    demostrar que todo subconjunto de cada conjunto de \isa{C} pertenece también a \isa{C}.

    Para ello, tomemos un conjunto \isa{S} cualquiera perteneciente a \isa{C} y un subconjunto cualquiera 
    \isa{S{\isacharprime}} de \isa{S}. Probemos que \isa{S{\isacharprime}} está en \isa{C}. Por hipótesis, como \isa{C} tiene la propiedad de carácter 
    finito, verifica que, para cualquier conjunto \isa{A}, son equivalentes:
    \begin{enumerate}
      \item \isa{A} pertenece a \isa{C}.
      \item Todo subconjunto finito de \isa{A} pertenece a \isa{C}.
    \end{enumerate}

    Para probar que el subconjunto \isa{S{\isacharprime}} pertenece a \isa{C}, vamos a demostrar que todo subconjunto 
    finito de \isa{S{\isacharprime}} pertenece a \isa{C}.

    De este modo, consideremos un subconjunto cualquiera \isa{S{\isacharprime}{\isacharprime}} de \isa{S{\isacharprime}}. Como \isa{S{\isacharprime}} es subconjunto de \isa{S}, 
    por la transitividad de la relación de contención de conjuntos, se tiene que \isa{S{\isacharprime}{\isacharprime}} es subconjunto 
    de \isa{S}. Aplicando la definición de propiedad de carácter finito de \isa{C} para el conjunto \isa{S}, 
    como este pertenece a \isa{C}, verifica que todo subconjunto finito de \isa{S} pertenece a \isa{C}. En
    particular, como \isa{S{\isacharprime}{\isacharprime}} es subconjunto de \isa{S}, verifica que, si \isa{S{\isacharprime}{\isacharprime}} es finito, entonces \isa{S{\isacharprime}{\isacharprime}} 
    pertenece a \isa{C}. Por tanto, hemos probado que cualquier conjunto finito de \isa{S{\isacharprime}} pertenece a la
    colección. Finalmente por la propiedad de carácter finito de \isa{C}, se verifica que \isa{S{\isacharprime}} pertenece 
    a \isa{C}, como queríamos demostrar.
  \end{demostracion}

  Veamos, a continuación, la demostración detallada del resultado en Isabelle.%
\end{isamarkuptext}\isamarkuptrue%
\isacommand{lemma}\isamarkupfalse%
\isanewline
\ \ \isakeyword{assumes}\ {\isachardoublequoteopen}finite{\isacharunderscore}character\ C{\isachardoublequoteclose}\isanewline
\ \ \isakeyword{shows}\ {\isachardoublequoteopen}subset{\isacharunderscore}closed\ C{\isachardoublequoteclose}\isanewline
%
\isadelimproof
\ \ %
\endisadelimproof
%
\isatagproof
\isacommand{unfolding}\isamarkupfalse%
\ subset{\isacharunderscore}closed{\isacharunderscore}def\isanewline
\isacommand{proof}\isamarkupfalse%
\ {\isacharparenleft}intro\ ballI\ sallI{\isacharparenright}\isanewline
\ \ \isacommand{fix}\isamarkupfalse%
\ S{\isacharprime}\ S\isanewline
\ \ \isacommand{assume}\isamarkupfalse%
\ \ {\isacartoucheopen}S\ {\isasymin}\ C{\isacartoucheclose}\ \isakeyword{and}\ {\isacartoucheopen}S{\isacharprime}\ {\isasymsubseteq}\ S{\isacartoucheclose}\isanewline
\ \ \isacommand{have}\isamarkupfalse%
\ H{\isacharcolon}{\isachardoublequoteopen}{\isasymforall}A{\isachardot}\ A\ {\isasymin}\ C\ {\isasymlongleftrightarrow}\ {\isacharparenleft}{\isasymforall}A{\isacharprime}\ {\isasymsubseteq}\ A{\isachardot}\ finite\ A{\isacharprime}\ {\isasymlongrightarrow}\ A{\isacharprime}\ {\isasymin}\ C{\isacharparenright}{\isachardoublequoteclose}\isanewline
\ \ \ \ \isacommand{using}\isamarkupfalse%
\ assms\ \isacommand{unfolding}\isamarkupfalse%
\ finite{\isacharunderscore}character{\isacharunderscore}def\ \isacommand{by}\isamarkupfalse%
\ this\isanewline
\ \ \isacommand{have}\isamarkupfalse%
\ QPQ{\isacharcolon}{\isachardoublequoteopen}{\isasymforall}S{\isacharprime}{\isacharprime}\ {\isasymsubseteq}\ S{\isacharprime}{\isachardot}\ finite\ S{\isacharprime}{\isacharprime}\ {\isasymlongrightarrow}\ S{\isacharprime}{\isacharprime}\ {\isasymin}\ C{\isachardoublequoteclose}\isanewline
\ \ \isacommand{proof}\isamarkupfalse%
\ {\isacharparenleft}rule\ sallI{\isacharparenright}\isanewline
\ \ \ \ \isacommand{fix}\isamarkupfalse%
\ S{\isacharprime}{\isacharprime}\isanewline
\ \ \ \ \isacommand{assume}\isamarkupfalse%
\ {\isachardoublequoteopen}S{\isacharprime}{\isacharprime}\ {\isasymsubseteq}\ S{\isacharprime}{\isachardoublequoteclose}\isanewline
\ \ \ \ \isacommand{then}\isamarkupfalse%
\ \isacommand{have}\isamarkupfalse%
\ {\isachardoublequoteopen}S{\isacharprime}{\isacharprime}\ {\isasymsubseteq}\ S{\isachardoublequoteclose}\isanewline
\ \ \ \ \ \ \isacommand{using}\isamarkupfalse%
\ {\isacartoucheopen}S{\isacharprime}\ {\isasymsubseteq}\ S{\isacartoucheclose}\ \isacommand{by}\isamarkupfalse%
\ {\isacharparenleft}simp\ only{\isacharcolon}\ subset{\isacharunderscore}trans{\isacharparenright}\isanewline
\ \ \ \ \isacommand{have}\isamarkupfalse%
\ {\isadigit{1}}{\isacharcolon}{\isachardoublequoteopen}S\ {\isasymin}\ C\ {\isasymlongleftrightarrow}\ {\isacharparenleft}{\isasymforall}S{\isacharprime}\ {\isasymsubseteq}\ S{\isachardot}\ finite\ S{\isacharprime}\ {\isasymlongrightarrow}\ S{\isacharprime}\ {\isasymin}\ C{\isacharparenright}{\isachardoublequoteclose}\isanewline
\ \ \ \ \ \ \isacommand{using}\isamarkupfalse%
\ H\ \isacommand{by}\isamarkupfalse%
\ {\isacharparenleft}rule\ allE{\isacharparenright}\isanewline
\ \ \ \ \isacommand{have}\isamarkupfalse%
\ {\isachardoublequoteopen}{\isasymforall}S{\isacharprime}\ {\isasymsubseteq}\ S{\isachardot}\ finite\ S{\isacharprime}\ {\isasymlongrightarrow}\ S{\isacharprime}\ {\isasymin}\ C{\isachardoublequoteclose}\isanewline
\ \ \ \ \ \ \isacommand{using}\isamarkupfalse%
\ {\isacartoucheopen}S\ {\isasymin}\ C{\isacartoucheclose}\ {\isadigit{1}}\ \isacommand{by}\isamarkupfalse%
\ {\isacharparenleft}rule\ back{\isacharunderscore}subst{\isacharparenright}\isanewline
\ \ \ \ \isacommand{thus}\isamarkupfalse%
\ {\isachardoublequoteopen}finite\ S{\isacharprime}{\isacharprime}\ {\isasymlongrightarrow}\ S{\isacharprime}{\isacharprime}\ {\isasymin}\ C{\isachardoublequoteclose}\isanewline
\ \ \ \ \ \ \isacommand{using}\isamarkupfalse%
\ {\isacartoucheopen}S{\isacharprime}{\isacharprime}\ {\isasymsubseteq}\ S{\isacartoucheclose}\ \isacommand{by}\isamarkupfalse%
\ {\isacharparenleft}rule\ sspec{\isacharparenright}\isanewline
\ \ \isacommand{qed}\isamarkupfalse%
\isanewline
\ \ \isacommand{have}\isamarkupfalse%
\ {\isachardoublequoteopen}S{\isacharprime}\ {\isasymin}\ C\ {\isasymlongleftrightarrow}\ {\isacharparenleft}{\isasymforall}S{\isacharprime}{\isacharprime}\ {\isasymsubseteq}\ S{\isacharprime}{\isachardot}\ finite\ S{\isacharprime}{\isacharprime}\ {\isasymlongrightarrow}\ S{\isacharprime}{\isacharprime}\ {\isasymin}\ C{\isacharparenright}{\isachardoublequoteclose}\isanewline
\ \ \ \ \isacommand{using}\isamarkupfalse%
\ H\ \isacommand{by}\isamarkupfalse%
\ {\isacharparenleft}rule\ allE{\isacharparenright}\isanewline
\ \ \isacommand{thus}\isamarkupfalse%
\ {\isachardoublequoteopen}S{\isacharprime}\ {\isasymin}\ C{\isachardoublequoteclose}\isanewline
\ \ \ \ \isacommand{using}\isamarkupfalse%
\ QPQ\ \isacommand{by}\isamarkupfalse%
\ {\isacharparenleft}rule\ forw{\isacharunderscore}subst{\isacharparenright}\isanewline
\isacommand{qed}\isamarkupfalse%
%
\endisatagproof
{\isafoldproof}%
%
\isadelimproof
%
\endisadelimproof
%
\begin{isamarkuptext}%
Finalmente, su prueba automática en Isabelle/HOL es la siguiente.%
\end{isamarkuptext}\isamarkuptrue%
\isacommand{lemma}\isamarkupfalse%
\ ex{\isadigit{2}}{\isacharcolon}\ \isanewline
\ \ \isakeyword{assumes}\ fc{\isacharcolon}\ {\isachardoublequoteopen}finite{\isacharunderscore}character\ C{\isachardoublequoteclose}\isanewline
\ \ \isakeyword{shows}\ {\isachardoublequoteopen}subset{\isacharunderscore}closed\ C{\isachardoublequoteclose}\isanewline
%
\isadelimproof
\ \ %
\endisadelimproof
%
\isatagproof
\isacommand{unfolding}\isamarkupfalse%
\ subset{\isacharunderscore}closed{\isacharunderscore}def\isanewline
\isacommand{proof}\isamarkupfalse%
\ {\isacharparenleft}intro\ ballI\ sallI{\isacharparenright}\isanewline
\ \ \isacommand{fix}\isamarkupfalse%
\ S{\isacharprime}\ S\isanewline
\ \ \isacommand{assume}\isamarkupfalse%
\ e{\isacharcolon}\ {\isacartoucheopen}S\ {\isasymin}\ C{\isacartoucheclose}\ \isakeyword{and}\ s{\isacharcolon}\ {\isacartoucheopen}S{\isacharprime}\ {\isasymsubseteq}\ S{\isacartoucheclose}\isanewline
\ \ \isacommand{hence}\isamarkupfalse%
\ {\isacharasterisk}{\isacharcolon}\ {\isachardoublequoteopen}S{\isacharprime}{\isacharprime}\ {\isasymsubseteq}\ S{\isacharprime}\ {\isasymLongrightarrow}\ S{\isacharprime}{\isacharprime}\ {\isasymsubseteq}\ S{\isachardoublequoteclose}\ \isakeyword{for}\ S{\isacharprime}{\isacharprime}\ \isacommand{by}\isamarkupfalse%
\ simp\isanewline
\ \ \isacommand{from}\isamarkupfalse%
\ fc\ \isacommand{have}\isamarkupfalse%
\ {\isachardoublequoteopen}S{\isacharprime}{\isacharprime}\ {\isasymsubseteq}\ S\ {\isasymLongrightarrow}\ finite\ S{\isacharprime}{\isacharprime}\ {\isasymLongrightarrow}\ S{\isacharprime}{\isacharprime}\ {\isasymin}\ C{\isachardoublequoteclose}\ \isakeyword{for}\ S{\isacharprime}{\isacharprime}\ \isanewline
\ \ \ \ \isacommand{unfolding}\isamarkupfalse%
\ finite{\isacharunderscore}character{\isacharunderscore}def\ \isacommand{using}\isamarkupfalse%
\ e\ \isacommand{by}\isamarkupfalse%
\ blast\isanewline
\ \ \isacommand{hence}\isamarkupfalse%
\ {\isachardoublequoteopen}S{\isacharprime}{\isacharprime}\ {\isasymsubseteq}\ S{\isacharprime}\ {\isasymLongrightarrow}\ finite\ S{\isacharprime}{\isacharprime}\ {\isasymLongrightarrow}\ S{\isacharprime}{\isacharprime}\ {\isasymin}\ C{\isachardoublequoteclose}\ \isakeyword{for}\ S{\isacharprime}{\isacharprime}\ \isacommand{using}\isamarkupfalse%
\ {\isacharasterisk}\ \isacommand{by}\isamarkupfalse%
\ simp\isanewline
\ \ \isacommand{with}\isamarkupfalse%
\ fc\ \isacommand{show}\isamarkupfalse%
\ {\isacartoucheopen}S{\isacharprime}\ {\isasymin}\ C{\isacartoucheclose}\ \isacommand{unfolding}\isamarkupfalse%
\ finite{\isacharunderscore}character{\isacharunderscore}def\ \isacommand{by}\isamarkupfalse%
\ blast\isanewline
\isacommand{qed}\isamarkupfalse%
%
\endisatagproof
{\isafoldproof}%
%
\isadelimproof
%
\endisadelimproof
%
\begin{isamarkuptext}%
Introduzcamos el último resultado de la sección.

 \begin{lema}
    Toda colección de conjuntos con la propiedad de consistencia proposicional y cerrada bajo 
    subconjuntos se puede extender a una colección que también verifique la propiedad de 
    consistencia proposicional y sea de carácter finito.
 \end{lema}

 \begin{demostracion}
   Dada una colección de conjuntos \isa{C} en las condiciones del enunciado, vamos a considerar su 
   extensión \isa{C{\isacharprime}} definida como la unión de \isa{C} y la colección formada por aquellos conjuntos
   cuyos subconjuntos finitos pertenecen a \isa{C}. Es decir,\\ \isa{C{\isacharprime}\ {\isacharequal}\ C\ {\isasymunion}\ E} donde 
   \isa{E\ {\isacharequal}\ {\isacharbraceleft}S{\isachardot}\ {\isasymforall}S{\isacharprime}\ {\isasymsubseteq}\ S{\isachardot}\ finite\ S{\isacharprime}\ {\isasymlongrightarrow}\ S{\isacharprime}\ {\isasymin}\ C{\isacharbraceright}}. Es evidente que es extensión pues contiene 
   a la colección \isa{C}. Vamos a probar que, además es de carácter finito y verifica la 
   propiedad de consistencia proposicional.

   En primer lugar, demostremos que \isa{C{\isacharprime}} es de carácter finito. Por definición de dicha propiedad, 
   basta probar que, para cualquier conjunto, son equivalentes:
   \begin{enumerate}
    \item El conjunto pertenece \isa{C{\isacharprime}}.
    \item Todo subconjunto finito suyo pertenece a \isa{C{\isacharprime}}.
   \end{enumerate}

   Comencemos probando \isa{{\isadigit{1}}{\isacharparenright}\ {\isasymLongrightarrow}\ {\isadigit{2}}{\isacharparenright}}. Para ello, sea un conjunto \isa{S} de \isa{C{\isacharprime}} de modo que \isa{S{\isacharprime}} es un
   subconjunto finito suyo. Como \isa{S} pertenece a la extensión, por definición de la misma tenemos
   que o bien \isa{S} está en \isa{C} o bien \isa{S} está en \isa{E}. Vamos a probar que \isa{S{\isacharprime}} está en \isa{C{\isacharprime}} por
   eliminación de la disyunción anterior. En primer lugar, si suponemos que \isa{S} está en \isa{C}, como
   se trata de una colección cerrada bajo subconjuntos, tenemos que todo subconjunto de \isa{S} está en 
   \isa{C}. En particular, \isa{S{\isacharprime}} está en \isa{C} y, por definición de la extensión, se prueba
   que \isa{S{\isacharprime}} está en \isa{C{\isacharprime}}. Por otro lado, suponiendo que \isa{S} esté en \isa{E}, por definición de dicha 
   colección tenemos que todo subconjunto finito de \isa{S} está en \isa{C}. De este modo, por las hipótesis 
   se prueba que \isa{S{\isacharprime}} está en \isa{C} y, por tanto, pertenece a la extensión. 

   Por último, probemos la implicación \isa{{\isadigit{2}}{\isacharparenright}\ {\isasymLongrightarrow}\ {\isadigit{1}}{\isacharparenright}}. Sea un conjunto cualquiera \isa{S} tal que todo
   subconjunto finito suyo pertenece a \isa{C{\isacharprime}}. Vamos a probar que \isa{S} también pertenece a \isa{C{\isacharprime}}. En
   particular, probaremos que pertenece a \isa{E}. Luego basta probar que todo subconjunto finito de 
   \isa{S} pertenece a \isa{C}. Para ello, consideremos \isa{S{\isacharprime}} un subconjunto finito cualquiera de \isa{S}. Por
   hipótesis, tenemos que \isa{S{\isacharprime}} pertenece a \isa{C{\isacharprime}}. Por definición de la extensión, tenemos entonces
   que o bien \isa{S{\isacharprime}} está en \isa{C} (lo que daría por concluida la prueba) o bien \isa{S{\isacharprime}} está en \isa{E}. 
   De este modo, si suponemos que \isa{S{\isacharprime}} está en \isa{E}, por definición de dicha colección tenemos que
   todo subconjunto finito suyo está en \isa{C}. En particular, como todo conjunto es subconjunto de si
   mismo y como hemos supuesto que \isa{S{\isacharprime}} es finito, tenemos que \isa{S{\isacharprime}} está en \isa{C}, lo que prueba la
   implicación.

   Probemos, finalmente, que \isa{C{\isacharprime}} verifica la propiedad de consistencia proposicional. Para ello,
   vamos a considerar un conjunto cualquiera \isa{S} perteneciente a \isa{C{\isacharprime}} y probaremos que se verifican 
   las cuatro condiciones del lema de caracterización de la propiedad de consistencia proposicional
   mediante la notación uniforme. Como el conjunto \isa{S} pertenece a \isa{C{\isacharprime}}, se observa fácilmente por
   definición de la extensión que, o bien \isa{S} está en \isa{C} o bien \isa{S} está en \isa{E}. Veamos que, para 
   ambos casos, se verifican dichas condiciones.

   En primer lugar, supongamos que \isa{S} está en \isa{C}. Como \isa{C} verifica la propiedad de consistencia 
   proposicional por hipótesis, verifica el lema de caracterización en particular para el conjunto 
   \isa{S}. De este modo, se cumple:
   \begin{itemize}
     \item \isa{{\isasymbottom}} no pertenece a \isa{S}.
     \item Dada \isa{p} una fórmula atómica cualquiera, no se tiene 
      simultáneamente que\\ \isa{p\ {\isasymin}\ S} y \isa{{\isasymnot}\ p\ {\isasymin}\ S}.
     \item Para toda fórmula de tipo \isa{{\isasymalpha}} con componentes \isa{{\isasymalpha}\isactrlsub {\isadigit{1}}} y \isa{{\isasymalpha}\isactrlsub {\isadigit{2}}} tal que \isa{{\isasymalpha}}
      pertenece a \isa{S}, se tiene que \isa{{\isacharbraceleft}{\isasymalpha}\isactrlsub {\isadigit{1}}{\isacharcomma}{\isasymalpha}\isactrlsub {\isadigit{2}}{\isacharbraceright}\ {\isasymunion}\ S} pertenece a \isa{C}.
     \item Para toda fórmula de tipo \isa{{\isasymbeta}} con componentes \isa{{\isasymbeta}\isactrlsub {\isadigit{1}}} y \isa{{\isasymbeta}\isactrlsub {\isadigit{2}}} tal que \isa{{\isasymbeta}}
      pertenece a \isa{S}, se tiene que o bien \isa{{\isacharbraceleft}{\isasymbeta}\isactrlsub {\isadigit{1}}{\isacharbraceright}\ {\isasymunion}\ S} pertenece a \isa{C} o 
      bien \isa{{\isacharbraceleft}{\isasymbeta}\isactrlsub {\isadigit{2}}{\isacharbraceright}\ {\isasymunion}\ S} pertenece a \isa{C}.
   \end{itemize} 
  
  Por lo tanto, puesto que \isa{C} está contenida en la extensión \isa{C{\isacharprime}}, se verifican las cuatro
  condiciones del lema para \isa{C{\isacharprime}}.

  Supongamos ahora que \isa{S} está en \isa{E}. Probemos que, en efecto, verifica las condiciones del lema 
  de caracterización.

  En primer lugar vamos a demostrar que \isa{{\isasymbottom}\ {\isasymnotin}\ S} por reducción al absurdo. Si suponemos que \isa{{\isasymbottom}\ {\isasymin}\ S},
  se deduce que el conjunto \isa{{\isacharbraceleft}{\isasymbottom}{\isacharbraceright}} es un subconjunto finito de \isa{S}. Como \isa{S} está en \isa{E}, por
  definición tenemos que \isa{{\isacharbraceleft}{\isasymbottom}{\isacharbraceright}\ {\isasymin}\ C}. De este modo, aplicando el lema de\\ caracterización de la
  propiedad de consistencia proposicional para la colección \isa{C} y el conjunto \isa{{\isacharbraceleft}{\isasymbottom}{\isacharbraceright}}, por la primera
  condición obtenemos que \isa{{\isasymbottom}\ {\isasymnotin}\ {\isacharbraceleft}{\isasymbottom}{\isacharbraceright}}, llegando a una contradicción.

  Demostremos que se verifica la segunda condición del lema para las fórmulas atómicas. De este
  modo, vamos a probar que dada \isa{p} una fórmula atómica cualquiera, no se tiene simultáneamente que
  \isa{p\ {\isasymin}\ S} y \isa{{\isasymnot}\ p\ {\isasymin}\ S}. La prueba se realizará por reducción al absurdo, luego supongamos que para
  cierta fórmula atómica se verifica \isa{p\ {\isasymin}\ S} y\\ \isa{{\isasymnot}\ p\ {\isasymin}\ S}. Análogamente, se observa que el conjunto
  \isa{{\isacharbraceleft}p{\isacharcomma}\ {\isasymnot}\ p{\isacharbraceright}} es un subconjunto finito de \isa{S}, luego pertenece a \isa{C}. Aplicando el lema de
  caracterización de la propiedad de consistencia proposicional para la colección \isa{C} y el conjunto
  \isa{{\isacharbraceleft}p{\isacharcomma}\ {\isasymnot}\ p{\isacharbraceright}}, por la segunda condición obtenemos que no se tiene simultáneamente \isa{q\ {\isasymin}\ {\isacharbraceleft}p{\isacharcomma}\ {\isasymnot}\ p{\isacharbraceright}} y
  \isa{{\isasymnot}\ q\ {\isasymin}\ {\isacharbraceleft}p{\isacharcomma}\ {\isasymnot}\ p{\isacharbraceright}} para ninguna fórmula atómica \isa{q}, llegando así a una contradicción para la
  fórmula atómica \isa{p}.

  Por otro lado, vamos a probar que se verifica la tercera condición del lema de\\ caracterización
  sobre las fórmulas de tipo \isa{{\isasymalpha}}. Consideremos una fórmula cualquiera \isa{F} de tipo \isa{{\isasymalpha}} y componentes 
  \isa{{\isasymalpha}\isactrlsub {\isadigit{1}}} y \isa{{\isasymalpha}\isactrlsub {\isadigit{2}}}, y supongamos que \isa{F\ {\isasymin}\ S}. Demostraremos que\\ \isa{{\isacharbraceleft}{\isasymalpha}\isactrlsub {\isadigit{1}}{\isacharcomma}{\isasymalpha}\isactrlsub {\isadigit{2}}{\isacharbraceright}\ {\isasymunion}\ S\ {\isasymin}\ C{\isacharprime}}. 

  Para ello, probaremos inicialmente que todo subconjunto finito \isa{S{\isacharprime}} de \isa{S} tal que\\ \isa{F\ {\isasymin}\ S{\isacharprime}} 
  verifica \isa{{\isacharbraceleft}{\isasymalpha}\isactrlsub {\isadigit{1}}{\isacharcomma}{\isasymalpha}\isactrlsub {\isadigit{2}}{\isacharbraceright}\ {\isasymunion}\ S{\isacharprime}\ {\isasymin}\ C}. Consideremos \isa{S{\isacharprime}} subconjunto finito cualquiera de \isa{S} en las
  condiciones anteriores. Como \isa{S\ {\isasymin}\ E}, por definición tenemos que \isa{S{\isacharprime}\ {\isasymin}\ C}. Aplicando el lema de 
  caracterización de la propiedad de consistencia proposicional para la colección \isa{C} y el conjunto
  \isa{S{\isacharprime}}, por la tercera condición obtenemos que \isa{{\isacharbraceleft}{\isasymalpha}\isactrlsub {\isadigit{1}}{\isacharcomma}{\isasymalpha}\isactrlsub {\isadigit{2}}{\isacharbraceright}\ {\isasymunion}\ S{\isacharprime}\ {\isasymin}\ C} ya que hemos supuesto que 
  \isa{F\ {\isasymin}\ S{\isacharprime}}.

  Una vez probado el resultado anterior, demostremos que \isa{{\isacharbraceleft}{\isasymalpha}\isactrlsub {\isadigit{1}}{\isacharcomma}{\isasymalpha}\isactrlsub {\isadigit{2}}{\isacharbraceright}\ {\isasymunion}\ S\ {\isasymin}\ E} y, por definición de 
  \isa{C{\isacharprime}}, obtendremos \isa{{\isacharbraceleft}{\isasymalpha}\isactrlsub {\isadigit{1}}{\isacharcomma}{\isasymalpha}\isactrlsub {\isadigit{2}}{\isacharbraceright}\ {\isasymunion}\ S\ {\isasymin}\ C{\isacharprime}}. Además, por definición de \isa{E}, basta probar que todo 
  subconjunto finito de \isa{{\isacharbraceleft}{\isasymalpha}\isactrlsub {\isadigit{1}}{\isacharcomma}{\isasymalpha}\isactrlsub {\isadigit{2}}{\isacharbraceright}\ {\isasymunion}\ S} pertenece a \isa{C}. Consideremos \isa{S{\isacharprime}} un subconjunto finito 
  cualquiera de \isa{{\isacharbraceleft}{\isasymalpha}\isactrlsub {\isadigit{1}}{\isacharcomma}{\isasymalpha}\isactrlsub {\isadigit{2}}{\isacharbraceright}\ {\isasymunion}\ S}. Como \isa{F\ {\isasymin}\ S}, es sencillo comprobar que el conjunto 
  \isa{{\isacharbraceleft}F{\isacharbraceright}\ {\isasymunion}\ {\isacharparenleft}S{\isacharprime}\ {\isacharminus}\ {\isacharbraceleft}{\isasymalpha}\isactrlsub {\isadigit{1}}{\isacharcomma}{\isasymalpha}\isactrlsub {\isadigit{2}}{\isacharbraceright}{\isacharparenright}} es un subconjunto finito de \isa{S}. Por el resultado probado anteriormente, 
  tenemos que el conjunto \isa{{\isacharbraceleft}{\isasymalpha}\isactrlsub {\isadigit{1}}{\isacharcomma}{\isasymalpha}\isactrlsub {\isadigit{2}}{\isacharbraceright}\ {\isasymunion}\ {\isacharparenleft}{\isacharbraceleft}F{\isacharbraceright}\ {\isasymunion}\ {\isacharparenleft}S{\isacharprime}\ {\isacharminus}\ {\isacharbraceleft}{\isasymalpha}\isactrlsub {\isadigit{1}}{\isacharcomma}{\isasymalpha}\isactrlsub {\isadigit{2}}{\isacharbraceright}{\isacharparenright}{\isacharparenright}\ {\isacharequal}} \\ \isa{{\isacharequal}\ {\isacharbraceleft}F{\isacharcomma}{\isasymalpha}\isactrlsub {\isadigit{1}}{\isacharcomma}{\isasymalpha}\isactrlsub {\isadigit{2}}{\isacharbraceright}\ {\isasymunion}\ S{\isacharprime}} pertenece a \isa{C}. 
  Además, como \isa{C} es cerrada bajo subconjuntos, todo conjunto de \isa{C} verifica que cualquier 
  subconjunto suyo pertenece a la colección. Luego, como \isa{S{\isacharprime}} es un subconjunto de 
  \isa{{\isacharbraceleft}F{\isacharcomma}{\isasymalpha}\isactrlsub {\isadigit{1}}{\isacharcomma}{\isasymalpha}\isactrlsub {\isadigit{2}}{\isacharbraceright}\ {\isasymunion}\ S{\isacharprime}}, queda probado que \isa{S{\isacharprime}\ {\isasymin}\ C}.

  Finalmente, veamos que se verifica la última condición del lema de caracterización de la propiedad
  de consistencia proposicional referente a las fórmulas de tipo \isa{{\isasymbeta}}. Consideremos una fórmula 
  cualquiera \isa{F} de tipo \isa{{\isasymbeta}} con componentes \isa{{\isasymbeta}\isactrlsub {\isadigit{1}}} y \isa{{\isasymbeta}\isactrlsub {\isadigit{2}}} tal que \isa{F\ {\isasymin}\ S}. Vamos a probar que se
  tiene que o bien \isa{{\isacharbraceleft}{\isasymbeta}\isactrlsub {\isadigit{1}}{\isacharbraceright}\ {\isasymunion}\ S\ {\isasymin}\ E} o bien \isa{{\isacharbraceleft}{\isasymbeta}\isactrlsub {\isadigit{1}}{\isacharbraceright}\ {\isasymunion}\ S\ {\isasymin}\ E}. En tal caso, por definición de \isa{C{\isacharprime}} se
  cumple que o bien \isa{{\isacharbraceleft}{\isasymbeta}\isactrlsub {\isadigit{1}}{\isacharbraceright}\ {\isasymunion}\ S\ {\isasymin}\ C{\isacharprime}} o bien \isa{{\isacharbraceleft}{\isasymbeta}\isactrlsub {\isadigit{1}}{\isacharbraceright}\ {\isasymunion}\ S\ {\isasymin}\ C{\isacharprime}}. La prueba se realizará por reducción al
  absurdo. Para ello, probemos inicialmente dos resultados previos.

  \begin{description}
    \item[\isa{{\isasymone}{\isacharparenright}}] En las condiciones anteriores, si consideramos \isa{S\isactrlsub {\isadigit{1}}} y \isa{S\isactrlsub {\isadigit{2}}} subconjuntos finitos 
    cualesquiera de \isa{S} tales que \isa{F\ {\isasymin}\ S\isactrlsub {\isadigit{1}}} y \isa{F\ {\isasymin}\ S\isactrlsub {\isadigit{2}}}, entonces existe una fórmula \isa{I\ {\isasymin}\ {\isacharbraceleft}{\isasymbeta}\isactrlsub {\isadigit{1}}{\isacharcomma}{\isasymbeta}\isactrlsub {\isadigit{2}}{\isacharbraceright}} tal 
    que se verifica que tanto \isa{{\isacharbraceleft}I{\isacharbraceright}\ {\isasymunion}\ S\isactrlsub {\isadigit{1}}} como \isa{{\isacharbraceleft}I{\isacharbraceright}\ {\isasymunion}\ S\isactrlsub {\isadigit{2}}} están en \isa{C}.
  \end{description}
  
  Para probar \isa{{\isasymone}{\isacharparenright}}, consideremos el conjunto finito \isa{S\isactrlsub {\isadigit{1}}\ {\isasymunion}\ S\isactrlsub {\isadigit{2}}} que es subconjunto de \isa{S} por las 
  hipótesis. De este modo, como \isa{S\ {\isasymin}\ E}, tenemos que \isa{S\isactrlsub {\isadigit{1}}\ {\isasymunion}\ S\isactrlsub {\isadigit{2}}\ {\isasymin}\ C}. Aplicando el lema de 
  caracterización de la propiedad de consistencia proposicional para la colección \isa{C} y el conjunto 
  \isa{S\isactrlsub {\isadigit{1}}\ {\isasymunion}\ S\isactrlsub {\isadigit{2}}}, por la última condición sobre las fórmulas de tipo \isa{{\isasymbeta}}, como\\ \isa{F\ {\isasymin}\ S\isactrlsub {\isadigit{1}}\ {\isasymunion}\ S\isactrlsub {\isadigit{2}}} por las 
  hipótesis, se tiene que o bien \isa{{\isacharbraceleft}{\isasymbeta}\isactrlsub {\isadigit{1}}{\isacharbraceright}\ {\isasymunion}\ S\isactrlsub {\isadigit{1}}\ {\isasymunion}\ S\isactrlsub {\isadigit{2}}\ {\isasymin}\ C} o bien\\ \isa{{\isacharbraceleft}{\isasymbeta}\isactrlsub {\isadigit{2}}{\isacharbraceright}\ {\isasymunion}\ S\isactrlsub {\isadigit{1}}\ {\isasymunion}\ S\isactrlsub {\isadigit{2}}\ {\isasymin}\ C}. Por tanto, 
  existe una fórmula \isa{I\ {\isasymin}\ {\isacharbraceleft}{\isasymbeta}\isactrlsub {\isadigit{1}}{\isacharcomma}{\isasymbeta}\isactrlsub {\isadigit{2}}{\isacharbraceright}} tal que\\ \isa{{\isacharbraceleft}I{\isacharbraceright}\ {\isasymunion}\ S\isactrlsub {\isadigit{1}}\ {\isasymunion}\ S\isactrlsub {\isadigit{2}}\ {\isasymin}\ C}. Sea \isa{I} la fórmula que cumple lo 
  anterior. Como \isa{C} es cerrada bajo subconjuntos, los subconjuntos \isa{{\isacharbraceleft}I{\isacharbraceright}\ {\isasymunion}\ S\isactrlsub {\isadigit{1}}} y \isa{{\isacharbraceleft}I{\isacharbraceright}\ {\isasymunion}\ S\isactrlsub {\isadigit{2}}} de 
  \isa{{\isacharbraceleft}I{\isacharbraceright}\ {\isasymunion}\ S\isactrlsub {\isadigit{1}}\ {\isasymunion}\ S\isactrlsub {\isadigit{2}}} pertenecen también a \isa{C}. Por tanto, hemos probado que existe una fórmula 
  \isa{I\ {\isasymin}\ {\isacharbraceleft}{\isasymbeta}\isactrlsub {\isadigit{1}}{\isacharcomma}{\isasymbeta}\isactrlsub {\isadigit{2}}{\isacharbraceright}} tal que \isa{{\isacharbraceleft}I{\isacharbraceright}\ {\isasymunion}\ S\isactrlsub {\isadigit{1}}\ {\isasymin}\ C} y \isa{{\isacharbraceleft}I{\isacharbraceright}\ {\isasymunion}\ S\isactrlsub {\isadigit{2}}\ {\isasymin}\ C}.

  Por otra parte, veamos el segundo resultado. 

  \begin{description}
    \item[\isa{{\isasymtwo}{\isacharparenright}}] En las condiciones de \isa{{\isasymone}{\isacharparenright}} para conjuntos cualesquiera \isa{S\isactrlsub {\isadigit{1}}} y \isa{S\isactrlsub {\isadigit{2}}}, si además 
    suponemos que \isa{{\isacharbraceleft}{\isasymbeta}\isactrlsub {\isadigit{1}}{\isacharbraceright}\ {\isasymunion}\ S\isactrlsub {\isadigit{1}}\ {\isasymnotin}\ C} y \isa{{\isacharbraceleft}{\isasymbeta}\isactrlsub {\isadigit{2}}{\isacharbraceright}\ {\isasymunion}\ S\isactrlsub {\isadigit{2}}\ {\isasymnotin}\ C}, llegamos a una contradicción. 
  \end{description}

  Para probarlo, utilizaremos \isa{{\isasymone}{\isacharparenright}} para los conjuntos \isa{{\isacharbraceleft}F{\isacharbraceright}\ {\isasymunion}\ S\isactrlsub {\isadigit{1}}} y \isa{{\isacharbraceleft}F{\isacharbraceright}\ {\isasymunion}\ S\isactrlsub {\isadigit{2}}}. Como es evidente, 
  puesto que \isa{F\ {\isasymin}\ S}, se verifica que ambos conjuntos son subconjuntos de \isa{S}. Además, como \isa{S\isactrlsub {\isadigit{1}}} y 
  \isa{S\isactrlsub {\isadigit{2}}} son finitos, se tiene que \isa{{\isacharbraceleft}F{\isacharbraceright}\ {\isasymunion}\ S\isactrlsub {\isadigit{1}}} y \isa{{\isacharbraceleft}F{\isacharbraceright}\ {\isasymunion}\ S\isactrlsub {\isadigit{2}}} también lo son. Por último, es claro que 
  \isa{F} pertenece a ambos conjuntos. Por lo tanto, por \isa{{\isasymone}{\isacharparenright}} tenemos que existe una fórmula 
  \isa{I\ {\isasymin}\ {\isacharbraceleft}{\isasymbeta}\isactrlsub {\isadigit{1}}{\isacharcomma}{\isasymbeta}\isactrlsub {\isadigit{2}}{\isacharbraceright}} tal que \isa{{\isacharbraceleft}I{\isacharbraceright}\ {\isasymunion}\ {\isacharbraceleft}F{\isacharbraceright}\ {\isasymunion}\ S\isactrlsub {\isadigit{1}}\ {\isasymin}\ C} y \isa{{\isacharbraceleft}I{\isacharbraceright}\ {\isasymunion}\ {\isacharbraceleft}F{\isacharbraceright}\ {\isasymunion}\ S\isactrlsub {\isadigit{2}}\ {\isasymin}\ C}. Por otro lado, podemos probar 
  que \isa{{\isacharbraceleft}{\isasymbeta}\isactrlsub {\isadigit{1}}{\isacharbraceright}\ {\isasymunion}\ {\isacharbraceleft}F{\isacharbraceright}\ {\isasymunion}\ S\isactrlsub {\isadigit{1}}\ {\isasymnotin}\ C}. Esto se debe a que, en caso contrario, como \isa{C} es cerrado bajo 
  subconjuntos, tendríamos que el subconjunto\\ \isa{{\isacharbraceleft}{\isasymbeta}\isactrlsub {\isadigit{1}}{\isacharbraceright}\ {\isasymunion}\ S\isactrlsub {\isadigit{1}}} pertenecería a \isa{C}, lo que contradice las 
  hipótesis. Análogamente, obtenemos que \isa{{\isacharbraceleft}{\isasymbeta}\isactrlsub {\isadigit{2}}{\isacharbraceright}\ {\isasymunion}\ {\isacharbraceleft}F{\isacharbraceright}\ {\isasymunion}\ S\isactrlsub {\isadigit{2}}\ {\isasymnotin}\ C}. De este modo, obtenemos que para 
  toda fórmula \isa{I\ {\isasymin}\ {\isacharbraceleft}{\isasymbeta}\isactrlsub {\isadigit{1}}{\isacharcomma}{\isasymbeta}\isactrlsub {\isadigit{2}}{\isacharbraceright}} se cumple que o bien \isa{{\isacharbraceleft}I{\isacharbraceright}\ {\isasymunion}\ {\isacharbraceleft}F{\isacharbraceright}\ {\isasymunion}\ S\isactrlsub {\isadigit{1}}\ {\isasymnotin}\ C} o bien \isa{{\isacharbraceleft}I{\isacharbraceright}\ {\isasymunion}\ {\isacharbraceleft}F{\isacharbraceright}\ {\isasymunion}\ S\isactrlsub {\isadigit{2}}\ {\isasymnotin}\ C}. 
  Esto es equivalente a que no existe ninguna fórmula \isa{I\ {\isasymin}\ {\isacharbraceleft}{\isasymbeta}\isactrlsub {\isadigit{1}}{\isacharcomma}{\isasymbeta}\isactrlsub {\isadigit{2}}{\isacharbraceright}} tal que \isa{{\isacharbraceleft}I{\isacharbraceright}\ {\isasymunion}\ {\isacharbraceleft}F{\isacharbraceright}\ {\isasymunion}\ S\isactrlsub {\isadigit{1}}\ {\isasymin}\ C} y\\ 
  \isa{{\isacharbraceleft}I{\isacharbraceright}\ {\isasymunion}\ {\isacharbraceleft}F{\isacharbraceright}\ {\isasymunion}\ S\isactrlsub {\isadigit{2}}\ {\isasymin}\ C}, lo que contradice lo obtenido para los conjuntos \isa{{\isacharbraceleft}F{\isacharbraceright}\ {\isasymunion}\ S\isactrlsub {\isadigit{1}}} y\\ \isa{{\isacharbraceleft}F{\isacharbraceright}\ {\isasymunion}\ S\isactrlsub {\isadigit{2}}} 
  por \isa{{\isasymone}{\isacharparenright}}.

  Finalmente, con los resultados anteriores, podemos probar que o bien\\ \isa{{\isacharbraceleft}{\isasymbeta}\isactrlsub {\isadigit{1}}{\isacharbraceright}\ {\isasymunion}\ S\ {\isasymin}\ E} o bien 
  \isa{{\isacharbraceleft}{\isasymbeta}\isactrlsub {\isadigit{2}}{\isacharbraceright}\ {\isasymunion}\ S\ {\isasymin}\ E} por reducción al absurdo. Supongamos que\\ \isa{{\isacharbraceleft}{\isasymbeta}\isactrlsub {\isadigit{1}}{\isacharbraceright}\ {\isasymunion}\ S\ {\isasymnotin}\ E} y \isa{{\isacharbraceleft}{\isasymbeta}\isactrlsub {\isadigit{2}}{\isacharbraceright}\ {\isasymunion}\ S\ {\isasymnotin}\ E}. Por
  definición de \isa{E}, se verifica que existe algún subconjunto finito de \isa{{\isacharbraceleft}{\isasymbeta}\isactrlsub {\isadigit{1}}{\isacharbraceright}\ {\isasymunion}\ S} y existe algún 
  subconjunto finito de \isa{{\isacharbraceleft}{\isasymbeta}\isactrlsub {\isadigit{2}}{\isacharbraceright}\ {\isasymunion}\ S} tales que no pertenecen a \isa{C}. Notemos por \isa{S\isactrlsub {\isadigit{1}}} y \isa{S\isactrlsub {\isadigit{2}}} 
  respectivamente a los subconjuntos anteriores. Vamos a aplicar \isa{{\isasymtwo}{\isacharparenright}} para los conjuntos \isa{S\isactrlsub {\isadigit{1}}\ {\isacharminus}\ {\isacharbraceleft}{\isasymbeta}\isactrlsub {\isadigit{1}}{\isacharbraceright}} 
  y \isa{S\isactrlsub {\isadigit{2}}\ {\isacharminus}\ {\isacharbraceleft}{\isasymbeta}\isactrlsub {\isadigit{2}}{\isacharbraceright}} para llegar a la contradicción.

  Para ello, debemos probar que se verifican las hipótesis del resultado para los conjuntos
  señalados. Es claro que tanto \isa{S\isactrlsub {\isadigit{1}}\ {\isacharminus}\ {\isacharbraceleft}{\isasymbeta}\isactrlsub {\isadigit{1}}{\isacharbraceright}} como \isa{S\isactrlsub {\isadigit{2}}\ {\isacharminus}\ {\isacharbraceleft}{\isasymbeta}\isactrlsub {\isadigit{2}}{\isacharbraceright}} son subconjuntos de \isa{S}, ya que \isa{S\isactrlsub {\isadigit{1}}} y
  \isa{S\isactrlsub {\isadigit{2}}} son subconjuntos de \isa{{\isacharbraceleft}{\isasymbeta}\isactrlsub {\isadigit{1}}{\isacharbraceright}\ {\isasymunion}\ S} y \isa{{\isacharbraceleft}{\isasymbeta}\isactrlsub {\isadigit{2}}{\isacharbraceright}\ {\isasymunion}\ S} respectivamente. Además, como \isa{S\isactrlsub {\isadigit{1}}} y \isa{S\isactrlsub {\isadigit{2}}} son
  finitos, es evidente que \isa{S\isactrlsub {\isadigit{1}}\ {\isacharminus}\ {\isacharbraceleft}{\isasymbeta}\isactrlsub {\isadigit{1}}{\isacharbraceright}} y \isa{S\isactrlsub {\isadigit{2}}\ {\isacharminus}\ {\isacharbraceleft}{\isasymbeta}\isactrlsub {\isadigit{2}}{\isacharbraceright}} también lo son. Queda probar que los conjuntos 
  \isa{{\isacharbraceleft}{\isasymbeta}\isactrlsub {\isadigit{1}}{\isacharbraceright}\ {\isasymunion}\ {\isacharparenleft}S\isactrlsub {\isadigit{1}}\ {\isacharminus}\ {\isacharbraceleft}{\isasymbeta}\isactrlsub {\isadigit{1}}{\isacharbraceright}{\isacharparenright}\ {\isacharequal}\ {\isacharbraceleft}{\isasymbeta}\isactrlsub {\isadigit{1}}{\isacharbraceright}\ {\isasymunion}\ S\isactrlsub {\isadigit{1}}} y \isa{{\isacharbraceleft}{\isasymbeta}\isactrlsub {\isadigit{2}}{\isacharbraceright}\ {\isasymunion}\ {\isacharparenleft}S\isactrlsub {\isadigit{2}}\ {\isacharminus}\ {\isacharbraceleft}{\isasymbeta}\isactrlsub {\isadigit{2}}{\isacharbraceright}{\isacharparenright}\ {\isacharequal}\ {\isacharbraceleft}{\isasymbeta}\isactrlsub {\isadigit{2}}{\isacharbraceright}\ {\isasymunion}\ S\isactrlsub {\isadigit{2}}} no pertenecen a \isa{C}. Como ni 
  \isa{S\isactrlsub {\isadigit{1}}} ni \isa{S\isactrlsub {\isadigit{2}}} están en la colección \isa{C} cerrada bajo subconjuntos, se cumple que ninguno de ellos 
  son subconjuntos de \isa{S}. Sin embargo, se verifica que \isa{S\isactrlsub {\isadigit{1}}} es subconjunto de \isa{{\isacharbraceleft}{\isasymbeta}\isactrlsub {\isadigit{1}}{\isacharbraceright}\ {\isasymunion}\ S} y \isa{S\isactrlsub {\isadigit{2}}} es 
  subconjunto de \isa{{\isacharbraceleft}{\isasymbeta}\isactrlsub {\isadigit{2}}{\isacharbraceright}\ {\isasymunion}\ S}. Por tanto, se cumple que\\ \isa{{\isasymbeta}\isactrlsub {\isadigit{1}}\ {\isasymin}\ S\isactrlsub {\isadigit{1}}} y \isa{{\isasymbeta}\isactrlsub {\isadigit{2}}\ {\isasymin}\ S\isactrlsub {\isadigit{2}}}. Por lo tanto,
  tenemos finalmente que los conjuntos \isa{{\isacharbraceleft}{\isasymbeta}\isactrlsub {\isadigit{1}}{\isacharbraceright}\ {\isasymunion}\ S\isactrlsub {\isadigit{1}}\ {\isacharequal}\ S\isactrlsub {\isadigit{1}}} y\\ \isa{{\isacharbraceleft}{\isasymbeta}\isactrlsub {\isadigit{2}}{\isacharbraceright}\ {\isasymunion}\ S\isactrlsub {\isadigit{2}}\ {\isacharequal}\ S\isactrlsub {\isadigit{2}}} no pertenecen a \isa{C}.
  Finalmente, como se cumplen las condiciones del resultado \isa{{\isadigit{2}}{\isacharparenright}}, llegamos a una contradicción para 
  los conjuntos \isa{S\isactrlsub {\isadigit{1}}\ {\isacharminus}\ {\isacharbraceleft}{\isasymbeta}\isactrlsub {\isadigit{1}}{\isacharbraceright}} y \isa{S\isactrlsub {\isadigit{2}}\ {\isacharminus}\ {\isacharbraceleft}{\isasymbeta}\isactrlsub {\isadigit{2}}{\isacharbraceright}}, probando que o bien \isa{{\isacharbraceleft}{\isasymbeta}\isactrlsub {\isadigit{1}}{\isacharbraceright}\ {\isasymunion}\ S\ {\isasymin}\ E} o bien \isa{{\isacharbraceleft}{\isasymbeta}\isactrlsub {\isadigit{1}}{\isacharbraceright}\ {\isasymunion}\ S\ {\isasymin}\ E}. 
  Por lo tanto, obtenemos por definición de \isa{C{\isacharprime}} que o bien \isa{{\isacharbraceleft}{\isasymbeta}\isactrlsub {\isadigit{1}}{\isacharbraceright}\ {\isasymunion}\ S\ {\isasymin}\ C{\isacharprime}} o bien \isa{{\isacharbraceleft}{\isasymbeta}\isactrlsub {\isadigit{1}}{\isacharbraceright}\ {\isasymunion}\ S\ {\isasymin}\ C{\isacharprime}}.
 \end{demostracion}

  Finalmente, veamos la demostración detallada del lema en Isabelle. Debido a la cantidad de lemas
  auxiliares empleados en la prueba detallada, para facilitar la comprensión mostraremos a
  continuación un grafo que estructura las relaciones de necesidad de los lemas introducidos.
  
 \begin{tikzpicture}
  [
    grow                    = down,
    level 1/.style          = {sibling distance=7cm},
    level 2/.style          = {sibling distance=4cm},
    level 3/.style          = {sibling distance=5.7cm},
    level distance          = 1.5cm,
    edge from parent/.style = {draw},
    every node/.style       = {font=\tiny},
    sloped
  ]
  \node [root] {\isa{ex{\isadigit{3}}}\\ \isa{{\isacharparenleft}Lema\ {\isadigit{3}}{\isachardot}{\isadigit{0}}{\isachardot}{\isadigit{5}}{\isacharparenright}}}
    child { node [env] {\isa{ex{\isadigit{3}}{\isacharunderscore}finite{\isacharunderscore}character}\\ \isa{{\isacharparenleft}C{\isacharprime}\ tiene\ la\ propiedad\ de\ carácter\ finito{\isacharparenright}}}}
    child { node [env] {\isa{ex{\isadigit{3}}{\isacharunderscore}pcp}\\ \isa{{\isacharparenleft}C{\isacharprime}\ tiene\ la\ propiedad\ de\ consistencia\ proposicional{\isacharparenright}}}
      		child { node [env] {\isa{ex{\isadigit{3}}{\isacharunderscore}pcp{\isacharunderscore}SinC}\\ \isa{{\isacharparenleft}Caso\ del\ conjunto\ en\ C{\isacharparenright}}}}
      		child { node [env] {\isa{ex{\isadigit{3}}{\isacharunderscore}pcp{\isacharunderscore}SinE}\\ \isa{{\isacharparenleft}Caso\ del\ conjunto\ en\ E{\isacharparenright}}}
        				child { node [env] {\isa{ex{\isadigit{3}}{\isacharunderscore}pcp{\isacharunderscore}SinE{\isacharunderscore}CON}\\ \isa{{\isacharparenleft}Condición\ fórmulas\ de\ tipo\ {\isasymalpha}{\isacharparenright}}}}
        				child { node [env] {\isa{ex{\isadigit{3}}{\isacharunderscore}pcp{\isacharunderscore}SinE{\isacharunderscore}DIS}\\ \isa{{\isacharparenleft}Condición\ fórmulas\ de\ tipo\ {\isasymbeta}{\isacharparenright}}}
                      child { node [env] {\isa{ex{\isadigit{3}}{\isacharunderscore}pcp{\isacharunderscore}SinE{\isacharunderscore}DIS{\isacharunderscore}auxFalse}\\ \isa{{\isacharparenleft}Resultado\ {\isasymone}{\isacharparenright}}}
                            child { node [env] {\isa{ex{\isadigit{3}}{\isacharunderscore}pcp{\isacharunderscore}SinE{\isacharunderscore}DIS{\isacharunderscore}auxEx}\\ \isa{{\isacharparenleft}Resultado\ {\isasymtwo}{\isacharparenright}}}}}}}};
\end{tikzpicture}

  De este modo, la prueba del \isa{lema\ {\isadigit{1}}{\isachardot}{\isadigit{3}}{\isachardot}{\isadigit{5}}} se estructura fundamentalmente en dos lemas auxiliares. 
  El primero, formalizado como \isa{ex{\isadigit{3}}{\isacharunderscore}finite{\isacharunderscore}character} en Isabelle, prueba que la extensión 
  \isa{C{\isacharprime}\ {\isacharequal}\ C\ {\isasymunion}\ E}, donde \isa{E} es la colección formada por aquellos conjuntos cuyos subconjuntos finitos 
  pertenecen a \isa{C}, tiene la propiedad de carácter finito. El segundo, formalizado como \isa{ex{\isadigit{3}}{\isacharunderscore}pcp}, 
  demuestra que \isa{C{\isacharprime}} verifica la propiedad de consistencia proposicional demostrando que cumple las 
  condiciones suficientes de dicha propiedad por el lema de caracterización \isa{{\isadigit{1}}{\isachardot}{\isadigit{2}}{\isachardot}{\isadigit{5}}}. De este modo, 
  considerando un conjunto \isa{S\ {\isasymin}\ C{\isacharprime}}, \isa{ex{\isadigit{3}}{\isacharunderscore}pcp} precisa, a su vez, de dos lemas auxiliares que 
  prueben las condiciones suficientes de \isa{{\isadigit{1}}{\isachardot}{\isadigit{2}}{\isachardot}{\isadigit{5}}}: uno para el caso en que \isa{S\ {\isasymin}\ C} (\isa{ex{\isadigit{3}}{\isacharunderscore}pcp{\isacharunderscore}SinC}) y 
  otro para el caso en que \isa{S\ {\isasymin}\ E} (\isa{ex{\isadigit{3}}{\isacharunderscore}pcp{\isacharunderscore}SinE}). Por otro lado, para el último caso en que 
  \isa{S\ {\isasymin}\ E}, utilizaremos dos lemas auxiliares. El primero, formalizado como \isa{ex{\isadigit{3}}{\isacharunderscore}pcp{\isacharunderscore}SinE{\isacharunderscore}CON}, 
  prueba que para \isa{C} una colección con la propiedad de consistencia proposicional y cerrada bajo 
  subconjuntos, \isa{S\ {\isasymin}\ E} y sea \isa{F} una fórmula de tipo \isa{{\isasymalpha}} y componentes \isa{{\isasymalpha}\isactrlsub {\isadigit{1}}} y \isa{{\isasymalpha}\isactrlsub {\isadigit{2}}}, se tiene que\\ 
  \isa{{\isacharbraceleft}{\isasymalpha}\isactrlsub {\isadigit{1}}{\isacharcomma}{\isasymalpha}\isactrlsub {\isadigit{2}}{\isacharbraceright}\ {\isasymunion}\ S\ {\isasymin}\ C{\isacharprime}}. El segundo lema, formalizado como \isa{ex{\isadigit{3}}{\isacharunderscore}pcp{\isacharunderscore}SinE{\isacharunderscore}DIS}, prueba que para \isa{C} una 
  colección con la propiedad de consistencia proposicional y cerrada bajo subconjuntos, \isa{S\ {\isasymin}\ E} y 
  sea \isa{F} una fórmula de tipo \isa{{\isasymbeta}} y componentes \isa{{\isasymbeta}\isactrlsub {\isadigit{1}}} y \isa{{\isasymbeta}\isactrlsub {\isadigit{2}}}, se tiene que o bien \isa{{\isacharbraceleft}{\isasymbeta}\isactrlsub {\isadigit{1}}{\isacharbraceright}\ {\isasymunion}\ S\ {\isasymin}\ C{\isacharprime}} o 
  bien \isa{{\isacharbraceleft}{\isasymbeta}\isactrlsub {\isadigit{2}}{\isacharbraceright}\ {\isasymunion}\ S\ {\isasymin}\ C{\isacharprime}}. Por último, este segundo lema auxiliar se probará por reducción al absurdo, 
  precisando para ello de los siguientes resultados auxiliares:
  
  \begin{description}
    \item[\isa{Resultado\ {\isasymone}}] Formalizado como \isa{ex{\isadigit{3}}{\isacharunderscore}pcp{\isacharunderscore}SinE{\isacharunderscore}DIS{\isacharunderscore}auxEx}. Prueba que dada \isa{C} una 
    colección con la propiedad de consistencia proposicional y cerrada bajo subconjuntos,\\ \isa{S\ {\isasymin}\ E} y 
    sea \isa{F} es una fórmula de tipo \isa{{\isasymbeta}} de componentes \isa{{\isasymbeta}\isactrlsub {\isadigit{1}}} y \isa{{\isasymbeta}\isactrlsub {\isadigit{2}}}, si consideramos \isa{S\isactrlsub {\isadigit{1}}} y \isa{S\isactrlsub {\isadigit{2}}} 
    subconjuntos finitos cualesquiera de \isa{S} tales que \isa{F\ {\isasymin}\ S\isactrlsub {\isadigit{1}}} y \isa{F\ {\isasymin}\ S\isactrlsub {\isadigit{2}}}, entonces existe una 
    fórmula \isa{I\ {\isasymin}\ {\isacharbraceleft}{\isasymbeta}\isactrlsub {\isadigit{1}}{\isacharcomma}{\isasymbeta}\isactrlsub {\isadigit{2}}{\isacharbraceright}} tal que se verifica que tanto \isa{{\isacharbraceleft}I{\isacharbraceright}\ {\isasymunion}\ S\isactrlsub {\isadigit{1}}} como \isa{{\isacharbraceleft}I{\isacharbraceright}\ {\isasymunion}\ S\isactrlsub {\isadigit{2}}} están en \isa{C}. 
    \item[\isa{Resultado\ {\isasymtwo}}] Formalizado como \isa{ex{\isadigit{3}}{\isacharunderscore}pcp{\isacharunderscore}SinE{\isacharunderscore}DIS{\isacharunderscore}auxFalse}. Utiliza 
    \isa{ex{\isadigit{3}}{\isacharunderscore}pcp{\isacharunderscore}SinE{\isacharunderscore}DIS{\isacharunderscore}auxEx} como lema auxiliar. Prueba que, en las condiciones del \isa{Resultado\ {\isasymone}}, 
    si además suponemos que \isa{{\isacharbraceleft}{\isasymbeta}\isactrlsub {\isadigit{1}}{\isacharbraceright}\ {\isasymunion}\ S\isactrlsub {\isadigit{1}}\ {\isasymnotin}\ C} y \isa{{\isacharbraceleft}{\isasymbeta}\isactrlsub {\isadigit{2}}{\isacharbraceright}\ {\isasymunion}\ S\isactrlsub {\isadigit{2}}\ {\isasymnotin}\ C}, llegamos a una contradicción.
  \end{description} 

  Por otro lado, para facilitar la notación, dada una colección cualquiera \isa{C}, formalizamos las 
  colecciones \isa{E} y \isa{C{\isacharprime}} como \isa{extF\ C} y \isa{extensionFin\ C} respectivamente como se muestra a 
  continuación.%
\end{isamarkuptext}\isamarkuptrue%
\isacommand{definition}\isamarkupfalse%
\ extF\ {\isacharcolon}{\isacharcolon}\ {\isachardoublequoteopen}{\isacharparenleft}{\isacharparenleft}{\isacharprime}a\ formula{\isacharparenright}\ set{\isacharparenright}\ set\ {\isasymRightarrow}\ {\isacharparenleft}{\isacharparenleft}{\isacharprime}a\ formula{\isacharparenright}\ set{\isacharparenright}\ set{\isachardoublequoteclose}\isanewline
\ \ \isakeyword{where}\ extF{\isacharcolon}\ {\isachardoublequoteopen}extF\ C\ {\isacharequal}\ {\isacharbraceleft}S{\isachardot}\ {\isasymforall}S{\isacharprime}\ {\isasymsubseteq}\ S{\isachardot}\ finite\ S{\isacharprime}\ {\isasymlongrightarrow}\ S{\isacharprime}\ {\isasymin}\ C{\isacharbraceright}{\isachardoublequoteclose}\isanewline
\isanewline
\isacommand{definition}\isamarkupfalse%
\ extensionFin\ {\isacharcolon}{\isacharcolon}\ {\isachardoublequoteopen}{\isacharparenleft}{\isacharparenleft}{\isacharprime}a\ formula{\isacharparenright}\ set{\isacharparenright}\ set\ {\isasymRightarrow}\ {\isacharparenleft}{\isacharparenleft}{\isacharprime}a\ formula{\isacharparenright}\ set{\isacharparenright}\ set{\isachardoublequoteclose}\isanewline
\ \ \isakeyword{where}\ extensionFin{\isacharcolon}\ {\isachardoublequoteopen}extensionFin\ C\ {\isacharequal}\ C\ {\isasymunion}\ {\isacharparenleft}extF\ C{\isacharparenright}{\isachardoublequoteclose}%
\begin{isamarkuptext}%
Una vez hechas las aclaraciones anteriores, procedamos ordenadamente con la demostración 
  detallada de cada lema auxiliar que conforma la prueba del lema \isa{{\isadigit{1}}{\isachardot}{\isadigit{3}}{\isachardot}{\isadigit{5}}}. En primer lugar, probemos 
  detalladamente que la extensión \isa{C{\isacharprime}} tiene la propiedad de carácter finito.%
\end{isamarkuptext}\isamarkuptrue%
\isacommand{lemma}\isamarkupfalse%
\ ex{\isadigit{3}}{\isacharunderscore}finite{\isacharunderscore}character{\isacharcolon}\isanewline
\ \ \isakeyword{assumes}\ {\isachardoublequoteopen}subset{\isacharunderscore}closed\ C{\isachardoublequoteclose}\isanewline
\ \ \ \ \ \ \ \ \isakeyword{shows}\ {\isachardoublequoteopen}finite{\isacharunderscore}character\ {\isacharparenleft}extensionFin\ C{\isacharparenright}{\isachardoublequoteclose}\isanewline
%
\isadelimproof
%
\endisadelimproof
%
\isatagproof
\isacommand{proof}\isamarkupfalse%
\ {\isacharminus}\isanewline
\ \ \isacommand{show}\isamarkupfalse%
\ {\isachardoublequoteopen}finite{\isacharunderscore}character\ {\isacharparenleft}extensionFin\ C{\isacharparenright}{\isachardoublequoteclose}\isanewline
\ \ \ \ \isacommand{unfolding}\isamarkupfalse%
\ finite{\isacharunderscore}character{\isacharunderscore}def\isanewline
\ \ \isacommand{proof}\isamarkupfalse%
\ {\isacharparenleft}rule\ allI{\isacharparenright}\isanewline
\ \ \ \isacommand{fix}\isamarkupfalse%
\ S\isanewline
\ \ \ \isacommand{show}\isamarkupfalse%
\ {\isachardoublequoteopen}S\ {\isasymin}\ {\isacharparenleft}extensionFin\ C{\isacharparenright}\ {\isasymlongleftrightarrow}\ {\isacharparenleft}{\isasymforall}S{\isacharprime}\ {\isasymsubseteq}\ S{\isachardot}\ finite\ S{\isacharprime}\ {\isasymlongrightarrow}\ S{\isacharprime}\ {\isasymin}\ {\isacharparenleft}extensionFin\ C{\isacharparenright}{\isacharparenright}{\isachardoublequoteclose}\isanewline
\ \ \ \isacommand{proof}\isamarkupfalse%
\ {\isacharparenleft}rule\ iffI{\isacharparenright}\isanewline
\ \ \ \ \ \isacommand{assume}\isamarkupfalse%
\ {\isachardoublequoteopen}S\ {\isasymin}\ {\isacharparenleft}extensionFin\ C{\isacharparenright}{\isachardoublequoteclose}\isanewline
\ \ \ \ \ \isacommand{show}\isamarkupfalse%
\ {\isachardoublequoteopen}{\isasymforall}S{\isacharprime}\ {\isasymsubseteq}\ S{\isachardot}\ finite\ S{\isacharprime}\ {\isasymlongrightarrow}\ S{\isacharprime}\ {\isasymin}\ {\isacharparenleft}extensionFin\ C{\isacharparenright}{\isachardoublequoteclose}\isanewline
\ \ \ \ \ \isacommand{proof}\isamarkupfalse%
\ {\isacharparenleft}intro\ sallI\ impI{\isacharparenright}\isanewline
\ \ \ \ \ \ \ \isacommand{fix}\isamarkupfalse%
\ S{\isacharprime}\isanewline
\ \ \ \ \ \ \ \isacommand{assume}\isamarkupfalse%
\ {\isachardoublequoteopen}S{\isacharprime}\ {\isasymsubseteq}\ S{\isachardoublequoteclose}\isanewline
\ \ \ \ \ \ \ \isacommand{assume}\isamarkupfalse%
\ {\isachardoublequoteopen}finite\ S{\isacharprime}{\isachardoublequoteclose}\isanewline
\ \ \ \ \ \ \ \isacommand{have}\isamarkupfalse%
\ {\isachardoublequoteopen}S\ {\isasymin}\ C\ {\isasymor}\ S\ {\isasymin}\ {\isacharparenleft}extF\ C{\isacharparenright}{\isachardoublequoteclose}\isanewline
\ \ \ \ \ \ \ \ \ \isacommand{using}\isamarkupfalse%
\ {\isacartoucheopen}S\ {\isasymin}\ {\isacharparenleft}extensionFin\ C{\isacharparenright}{\isacartoucheclose}\ \isacommand{by}\isamarkupfalse%
\ {\isacharparenleft}simp\ only{\isacharcolon}\ extensionFin\ Un{\isacharunderscore}iff{\isacharparenright}\isanewline
\ \ \ \ \ \ \ \isacommand{thus}\isamarkupfalse%
\ {\isachardoublequoteopen}S{\isacharprime}\ {\isasymin}\ {\isacharparenleft}extensionFin\ C{\isacharparenright}{\isachardoublequoteclose}\isanewline
\ \ \ \ \ \ \ \isacommand{proof}\isamarkupfalse%
\ {\isacharparenleft}rule\ disjE{\isacharparenright}\isanewline
\ \ \ \ \ \ \ \ \ \isacommand{assume}\isamarkupfalse%
\ {\isachardoublequoteopen}S\ {\isasymin}\ C{\isachardoublequoteclose}\isanewline
\ \ \ \ \ \ \ \ \ \isacommand{have}\isamarkupfalse%
\ {\isachardoublequoteopen}{\isasymforall}S\ {\isasymin}\ C{\isachardot}\ {\isasymforall}S{\isacharprime}\ {\isasymsubseteq}\ S{\isachardot}\ S{\isacharprime}\ {\isasymin}\ C{\isachardoublequoteclose}\isanewline
\ \ \ \ \ \ \ \ \ \ \ \isacommand{using}\isamarkupfalse%
\ assms\ \isacommand{by}\isamarkupfalse%
\ {\isacharparenleft}simp\ only{\isacharcolon}\ subset{\isacharunderscore}closed{\isacharunderscore}def{\isacharparenright}\isanewline
\ \ \ \ \ \ \ \ \ \isacommand{then}\isamarkupfalse%
\ \isacommand{have}\isamarkupfalse%
\ {\isachardoublequoteopen}{\isasymforall}S{\isacharprime}\ {\isasymsubseteq}\ S{\isachardot}\ S{\isacharprime}\ {\isasymin}\ C{\isachardoublequoteclose}\isanewline
\ \ \ \ \ \ \ \ \ \ \ \isacommand{using}\isamarkupfalse%
\ {\isacartoucheopen}S\ {\isasymin}\ C{\isacartoucheclose}\ \isacommand{by}\isamarkupfalse%
\ {\isacharparenleft}rule\ bspec{\isacharparenright}\isanewline
\ \ \ \ \ \ \ \ \ \isacommand{then}\isamarkupfalse%
\ \isacommand{have}\isamarkupfalse%
\ {\isachardoublequoteopen}S{\isacharprime}\ {\isasymin}\ C{\isachardoublequoteclose}\isanewline
\ \ \ \ \ \ \ \ \ \ \ \isacommand{using}\isamarkupfalse%
\ {\isacartoucheopen}S{\isacharprime}\ {\isasymsubseteq}\ S{\isacartoucheclose}\ \isacommand{by}\isamarkupfalse%
\ {\isacharparenleft}rule\ sspec{\isacharparenright}\isanewline
\ \ \ \ \ \ \ \ \ \isacommand{thus}\isamarkupfalse%
\ {\isachardoublequoteopen}S{\isacharprime}\ {\isasymin}\ {\isacharparenleft}extensionFin\ C{\isacharparenright}{\isachardoublequoteclose}\ \isanewline
\ \ \ \ \ \ \ \ \ \ \ \isacommand{by}\isamarkupfalse%
\ {\isacharparenleft}simp\ only{\isacharcolon}\ extensionFin\ UnI{\isadigit{1}}{\isacharparenright}\isanewline
\ \ \ \ \ \ \ \isacommand{next}\isamarkupfalse%
\isanewline
\ \ \ \ \ \ \ \ \ \isacommand{assume}\isamarkupfalse%
\ {\isachardoublequoteopen}S\ {\isasymin}\ {\isacharparenleft}extF\ C{\isacharparenright}{\isachardoublequoteclose}\isanewline
\ \ \ \ \ \ \ \ \ \isacommand{then}\isamarkupfalse%
\ \isacommand{have}\isamarkupfalse%
\ {\isachardoublequoteopen}{\isasymforall}S{\isacharprime}\ {\isasymsubseteq}\ S{\isachardot}\ finite\ S{\isacharprime}\ {\isasymlongrightarrow}\ S{\isacharprime}\ {\isasymin}\ C{\isachardoublequoteclose}\isanewline
\ \ \ \ \ \ \ \ \ \ \ \isacommand{unfolding}\isamarkupfalse%
\ extF\ \isacommand{by}\isamarkupfalse%
\ {\isacharparenleft}rule\ CollectD{\isacharparenright}\isanewline
\ \ \ \ \ \ \ \ \ \isacommand{then}\isamarkupfalse%
\ \isacommand{have}\isamarkupfalse%
\ {\isachardoublequoteopen}finite\ S{\isacharprime}\ {\isasymlongrightarrow}\ S{\isacharprime}\ {\isasymin}\ C{\isachardoublequoteclose}\isanewline
\ \ \ \ \ \ \ \ \ \ \ \isacommand{using}\isamarkupfalse%
\ {\isacartoucheopen}S{\isacharprime}\ {\isasymsubseteq}\ S{\isacartoucheclose}\ \isacommand{by}\isamarkupfalse%
\ {\isacharparenleft}rule\ sspec{\isacharparenright}\isanewline
\ \ \ \ \ \ \ \ \ \isacommand{then}\isamarkupfalse%
\ \isacommand{have}\isamarkupfalse%
\ {\isachardoublequoteopen}S{\isacharprime}\ {\isasymin}\ C{\isachardoublequoteclose}\isanewline
\ \ \ \ \ \ \ \ \ \ \ \isacommand{using}\isamarkupfalse%
\ {\isacartoucheopen}finite\ S{\isacharprime}{\isacartoucheclose}\ \isacommand{by}\isamarkupfalse%
\ {\isacharparenleft}rule\ mp{\isacharparenright}\isanewline
\ \ \ \ \ \ \ \ \ \isacommand{thus}\isamarkupfalse%
\ {\isachardoublequoteopen}S{\isacharprime}\ {\isasymin}\ {\isacharparenleft}extensionFin\ C{\isacharparenright}{\isachardoublequoteclose}\isanewline
\ \ \ \ \ \ \ \ \ \ \ \isacommand{by}\isamarkupfalse%
\ {\isacharparenleft}simp\ only{\isacharcolon}\ extensionFin\ UnI{\isadigit{1}}{\isacharparenright}\isanewline
\ \ \ \ \ \ \ \isacommand{qed}\isamarkupfalse%
\isanewline
\ \ \ \ \ \isacommand{qed}\isamarkupfalse%
\isanewline
\ \ \ \isacommand{next}\isamarkupfalse%
\isanewline
\ \ \ \ \ \isacommand{assume}\isamarkupfalse%
\ {\isachardoublequoteopen}{\isasymforall}S{\isacharprime}\ {\isasymsubseteq}\ S{\isachardot}\ finite\ S{\isacharprime}\ {\isasymlongrightarrow}\ S{\isacharprime}\ {\isasymin}\ {\isacharparenleft}extensionFin\ C{\isacharparenright}{\isachardoublequoteclose}\isanewline
\ \ \ \ \ \isacommand{then}\isamarkupfalse%
\ \isacommand{have}\isamarkupfalse%
\ F{\isacharcolon}{\isachardoublequoteopen}{\isasymforall}S{\isacharprime}\ {\isasymsubseteq}\ S{\isachardot}\ finite\ S{\isacharprime}\ {\isasymlongrightarrow}\ S{\isacharprime}\ {\isasymin}\ C\ {\isasymor}\ S{\isacharprime}\ {\isasymin}\ {\isacharparenleft}extF\ C{\isacharparenright}{\isachardoublequoteclose}\isanewline
\ \ \ \ \ \ \ \isacommand{by}\isamarkupfalse%
\ {\isacharparenleft}simp\ only{\isacharcolon}\ extensionFin\ Un{\isacharunderscore}iff{\isacharparenright}\isanewline
\ \ \ \ \ \isacommand{have}\isamarkupfalse%
\ {\isachardoublequoteopen}{\isasymforall}S{\isacharprime}\ {\isasymsubseteq}\ S{\isachardot}\ finite\ S{\isacharprime}\ {\isasymlongrightarrow}\ S{\isacharprime}\ {\isasymin}\ C{\isachardoublequoteclose}\isanewline
\ \ \ \ \ \isacommand{proof}\isamarkupfalse%
\ {\isacharparenleft}rule\ sallI{\isacharparenright}\isanewline
\ \ \ \ \ \ \ \isacommand{fix}\isamarkupfalse%
\ S{\isacharprime}\isanewline
\ \ \ \ \ \ \ \isacommand{assume}\isamarkupfalse%
\ {\isachardoublequoteopen}S{\isacharprime}\ {\isasymsubseteq}\ S{\isachardoublequoteclose}\isanewline
\ \ \ \ \ \ \ \isacommand{show}\isamarkupfalse%
\ {\isachardoublequoteopen}finite\ S{\isacharprime}\ {\isasymlongrightarrow}\ S{\isacharprime}\ {\isasymin}\ C{\isachardoublequoteclose}\isanewline
\ \ \ \ \ \ \ \isacommand{proof}\isamarkupfalse%
\ {\isacharparenleft}rule\ impI{\isacharparenright}\isanewline
\ \ \ \ \ \ \ \ \ \isacommand{assume}\isamarkupfalse%
\ {\isachardoublequoteopen}finite\ S{\isacharprime}{\isachardoublequoteclose}\isanewline
\ \ \ \ \ \ \ \ \ \isacommand{have}\isamarkupfalse%
\ {\isachardoublequoteopen}finite\ S{\isacharprime}\ {\isasymlongrightarrow}\ S{\isacharprime}\ {\isasymin}\ C\ {\isasymor}\ S{\isacharprime}\ {\isasymin}\ {\isacharparenleft}extF\ C{\isacharparenright}{\isachardoublequoteclose}\ \isanewline
\ \ \ \ \ \ \ \ \ \ \ \isacommand{using}\isamarkupfalse%
\ F\ {\isacartoucheopen}S{\isacharprime}\ {\isasymsubseteq}\ S{\isacartoucheclose}\ \isacommand{by}\isamarkupfalse%
\ {\isacharparenleft}rule\ sspec{\isacharparenright}\isanewline
\ \ \ \ \ \ \ \ \ \isacommand{then}\isamarkupfalse%
\ \isacommand{have}\isamarkupfalse%
\ {\isachardoublequoteopen}S{\isacharprime}\ {\isasymin}\ C\ {\isasymor}\ S{\isacharprime}\ {\isasymin}\ {\isacharparenleft}extF\ C{\isacharparenright}{\isachardoublequoteclose}\isanewline
\ \ \ \ \ \ \ \ \ \ \ \isacommand{using}\isamarkupfalse%
\ {\isacartoucheopen}finite\ S{\isacharprime}{\isacartoucheclose}\ \isacommand{by}\isamarkupfalse%
\ {\isacharparenleft}rule\ mp{\isacharparenright}\isanewline
\ \ \ \ \ \ \ \ \ \isacommand{thus}\isamarkupfalse%
\ {\isachardoublequoteopen}S{\isacharprime}\ {\isasymin}\ C{\isachardoublequoteclose}\isanewline
\ \ \ \ \ \ \ \ \ \isacommand{proof}\isamarkupfalse%
\ {\isacharparenleft}rule\ disjE{\isacharparenright}\isanewline
\ \ \ \ \ \ \ \ \ \ \ \isacommand{assume}\isamarkupfalse%
\ {\isachardoublequoteopen}S{\isacharprime}\ {\isasymin}\ C{\isachardoublequoteclose}\isanewline
\ \ \ \ \ \ \ \ \ \ \ \isacommand{thus}\isamarkupfalse%
\ {\isachardoublequoteopen}S{\isacharprime}\ {\isasymin}\ C{\isachardoublequoteclose}\isanewline
\ \ \ \ \ \ \ \ \ \ \ \ \ \isacommand{by}\isamarkupfalse%
\ this\isanewline
\ \ \ \ \ \ \ \ \ \isacommand{next}\isamarkupfalse%
\isanewline
\ \ \ \ \ \ \ \ \ \ \ \isacommand{assume}\isamarkupfalse%
\ {\isachardoublequoteopen}S{\isacharprime}\ {\isasymin}\ {\isacharparenleft}extF\ C{\isacharparenright}{\isachardoublequoteclose}\isanewline
\ \ \ \ \ \ \ \ \ \ \ \isacommand{then}\isamarkupfalse%
\ \isacommand{have}\isamarkupfalse%
\ S{\isacharprime}{\isacharcolon}{\isachardoublequoteopen}{\isasymforall}S{\isacharprime}{\isacharprime}\ {\isasymsubseteq}\ S{\isacharprime}{\isachardot}\ finite\ S{\isacharprime}{\isacharprime}\ {\isasymlongrightarrow}\ S{\isacharprime}{\isacharprime}\ {\isasymin}\ C{\isachardoublequoteclose}\isanewline
\ \ \ \ \ \ \ \ \ \ \ \ \ \isacommand{unfolding}\isamarkupfalse%
\ extF\ \isacommand{by}\isamarkupfalse%
\ {\isacharparenleft}rule\ CollectD{\isacharparenright}\isanewline
\ \ \ \ \ \ \ \ \ \ \ \isacommand{have}\isamarkupfalse%
\ {\isachardoublequoteopen}S{\isacharprime}\ {\isasymsubseteq}\ S{\isacharprime}{\isachardoublequoteclose}\isanewline
\ \ \ \ \ \ \ \ \ \ \ \ \ \isacommand{by}\isamarkupfalse%
\ {\isacharparenleft}simp\ only{\isacharcolon}\ subset{\isacharunderscore}refl{\isacharparenright}\isanewline
\ \ \ \ \ \ \ \ \ \ \ \isacommand{have}\isamarkupfalse%
\ {\isachardoublequoteopen}finite\ S{\isacharprime}\ {\isasymlongrightarrow}\ S{\isacharprime}\ {\isasymin}\ C{\isachardoublequoteclose}\isanewline
\ \ \ \ \ \ \ \ \ \ \ \ \ \isacommand{using}\isamarkupfalse%
\ S{\isacharprime}\ {\isacartoucheopen}S{\isacharprime}\ {\isasymsubseteq}\ S{\isacharprime}{\isacartoucheclose}\ \isacommand{by}\isamarkupfalse%
\ {\isacharparenleft}rule\ sspec{\isacharparenright}\isanewline
\ \ \ \ \ \ \ \ \ \ \ \isacommand{thus}\isamarkupfalse%
\ {\isachardoublequoteopen}S{\isacharprime}\ {\isasymin}\ C{\isachardoublequoteclose}\isanewline
\ \ \ \ \ \ \ \ \ \ \ \ \ \isacommand{using}\isamarkupfalse%
\ {\isacartoucheopen}finite\ S{\isacharprime}{\isacartoucheclose}\ \isacommand{by}\isamarkupfalse%
\ {\isacharparenleft}rule\ mp{\isacharparenright}\isanewline
\ \ \ \ \ \ \ \ \ \isacommand{qed}\isamarkupfalse%
\isanewline
\ \ \ \ \ \ \ \isacommand{qed}\isamarkupfalse%
\isanewline
\ \ \ \ \ \isacommand{qed}\isamarkupfalse%
\isanewline
\ \ \ \ \ \isacommand{then}\isamarkupfalse%
\ \isacommand{have}\isamarkupfalse%
\ {\isachardoublequoteopen}S\ {\isasymin}\ {\isacharbraceleft}S{\isachardot}\ {\isasymforall}S{\isacharprime}\ {\isasymsubseteq}\ S{\isachardot}\ finite\ S{\isacharprime}\ {\isasymlongrightarrow}\ S{\isacharprime}\ {\isasymin}\ C{\isacharbraceright}{\isachardoublequoteclose}\isanewline
\ \ \ \ \ \ \ \isacommand{by}\isamarkupfalse%
\ {\isacharparenleft}rule\ CollectI{\isacharparenright}\isanewline
\ \ \ \ \ \isacommand{thus}\isamarkupfalse%
\ {\isachardoublequoteopen}S\ {\isasymin}\ {\isacharparenleft}extensionFin\ C{\isacharparenright}{\isachardoublequoteclose}\isanewline
\ \ \ \ \ \ \ \isacommand{by}\isamarkupfalse%
\ {\isacharparenleft}simp\ only{\isacharcolon}\ extF\ extensionFin\ UnI{\isadigit{2}}{\isacharparenright}\isanewline
\ \ \ \isacommand{qed}\isamarkupfalse%
\isanewline
\ \isacommand{qed}\isamarkupfalse%
\isanewline
\isacommand{qed}\isamarkupfalse%
%
\endisatagproof
{\isafoldproof}%
%
\isadelimproof
%
\endisadelimproof
%
\begin{isamarkuptext}%
Por otro lado, para probar que  \isa{C{\isacharprime}\ {\isacharequal}\ C\ {\isasymunion}\ E}  verifica la propiedad de consistencia 
  proposicional, consideraremos un conjunto \isa{S\ {\isasymin}\ C{\isacharprime}} y utilizaremos fundamentalmente dos lemas 
  auxiliares: uno para el caso en que \isa{S\ {\isasymin}\ C} y otro para el caso en que \isa{S\ {\isasymin}\ E}. 

  En primer lugar, vamos a probar el primer lema auxiliar para el caso en que\\ \isa{S\ {\isasymin}\ C}, formalizado
  como \isa{ex{\isadigit{3}}{\isacharunderscore}pcp{\isacharunderscore}SinC}. Dicho lema prueba que, si \isa{C} es una colección con la propiedad de 
  consistencia proposicional y cerrada bajo subconjuntos, y sea \isa{S\ {\isasymin}\ C}, se verifican
  las condiciones del lema de caracterización de la propiedad de consistencia proposicional para
  la extensión \isa{C{\isacharprime}}:
  \begin{itemize}
    \item \isa{{\isasymbottom}\ {\isasymnotin}\ S}.
    \item Dada \isa{p} una fórmula atómica cualquiera, no se tiene 
    simultáneamente que\\ \isa{p\ {\isasymin}\ S} y \isa{{\isasymnot}\ p\ {\isasymin}\ S}.
    \item Para toda fórmula de tipo \isa{{\isasymalpha}} con componentes \isa{{\isasymalpha}\isactrlsub {\isadigit{1}}} y \isa{{\isasymalpha}\isactrlsub {\isadigit{2}}} tal que \isa{{\isasymalpha}}
    pertenece a \isa{S}, se tiene que \isa{{\isacharbraceleft}{\isasymalpha}\isactrlsub {\isadigit{1}}{\isacharcomma}{\isasymalpha}\isactrlsub {\isadigit{2}}{\isacharbraceright}\ {\isasymunion}\ S} pertenece a \isa{C{\isacharprime}}.
    \item Para toda fórmula de tipo \isa{{\isasymbeta}} con componentes \isa{{\isasymbeta}\isactrlsub {\isadigit{1}}} y \isa{{\isasymbeta}\isactrlsub {\isadigit{2}}} tal que \isa{{\isasymbeta}}
    pertenece a \isa{S}, se tiene que o bien \isa{{\isacharbraceleft}{\isasymbeta}\isactrlsub {\isadigit{1}}{\isacharbraceright}\ {\isasymunion}\ S} pertenece a \isa{C{\isacharprime}} o 
    bien \isa{{\isacharbraceleft}{\isasymbeta}\isactrlsub {\isadigit{2}}{\isacharbraceright}\ {\isasymunion}\ S} pertenece a \isa{C{\isacharprime}}.
  \end{itemize}%
\end{isamarkuptext}\isamarkuptrue%
\isacommand{lemma}\isamarkupfalse%
\ ex{\isadigit{3}}{\isacharunderscore}pcp{\isacharunderscore}SinC{\isacharcolon}\isanewline
\ \ \isakeyword{assumes}\ {\isachardoublequoteopen}pcp\ C{\isachardoublequoteclose}\isanewline
\ \ \ \ \ \ \ \ \ \ {\isachardoublequoteopen}subset{\isacharunderscore}closed\ C{\isachardoublequoteclose}\isanewline
\ \ \ \ \ \ \ \ \ \ {\isachardoublequoteopen}S\ {\isasymin}\ C{\isachardoublequoteclose}\ \isanewline
\ \ \isakeyword{shows}\ {\isachardoublequoteopen}{\isasymbottom}\ {\isasymnotin}\ S\ {\isasymand}\isanewline
\ \ \ \ \ \ \ \ \ {\isacharparenleft}{\isasymforall}k{\isachardot}\ Atom\ k\ {\isasymin}\ S\ {\isasymlongrightarrow}\ \isactrlbold {\isasymnot}\ {\isacharparenleft}Atom\ k{\isacharparenright}\ {\isasymin}\ S\ {\isasymlongrightarrow}\ False{\isacharparenright}\ {\isasymand}\isanewline
\ \ \ \ \ \ \ \ \ {\isacharparenleft}{\isasymforall}F\ G\ H{\isachardot}\ Con\ F\ G\ H\ {\isasymlongrightarrow}\ F\ {\isasymin}\ S\ {\isasymlongrightarrow}\ {\isacharbraceleft}G{\isacharcomma}\ H{\isacharbraceright}\ {\isasymunion}\ S\ {\isasymin}\ {\isacharparenleft}extensionFin\ C{\isacharparenright}{\isacharparenright}\ {\isasymand}\isanewline
\ \ \ \ \ \ \ \ \ {\isacharparenleft}{\isasymforall}F\ G\ H{\isachardot}\ Dis\ F\ G\ H\ {\isasymlongrightarrow}\ F\ {\isasymin}\ S\ {\isasymlongrightarrow}\ {\isacharbraceleft}G{\isacharbraceright}\ {\isasymunion}\ S\ {\isasymin}{\isacharparenleft}extensionFin\ C{\isacharparenright}\ {\isasymor}\ {\isacharbraceleft}H{\isacharbraceright}\ {\isasymunion}\ S\ {\isasymin}\ {\isacharparenleft}extensionFin\ C{\isacharparenright}{\isacharparenright}{\isachardoublequoteclose}\isanewline
%
\isadelimproof
%
\endisadelimproof
%
\isatagproof
\isacommand{proof}\isamarkupfalse%
\ {\isacharminus}\isanewline
\ \ \isacommand{have}\isamarkupfalse%
\ PCP{\isacharcolon}{\isachardoublequoteopen}{\isasymforall}S\ {\isasymin}\ C{\isachardot}\isanewline
\ \ \ \ {\isasymbottom}\ {\isasymnotin}\ S\isanewline
\ \ \ \ {\isasymand}\ {\isacharparenleft}{\isasymforall}k{\isachardot}\ Atom\ k\ {\isasymin}\ S\ {\isasymlongrightarrow}\ \isactrlbold {\isasymnot}\ {\isacharparenleft}Atom\ k{\isacharparenright}\ {\isasymin}\ S\ {\isasymlongrightarrow}\ False{\isacharparenright}\isanewline
\ \ \ \ {\isasymand}\ {\isacharparenleft}{\isasymforall}F\ G\ H{\isachardot}\ Con\ F\ G\ H\ {\isasymlongrightarrow}\ F\ {\isasymin}\ S\ {\isasymlongrightarrow}\ {\isacharbraceleft}G{\isacharcomma}H{\isacharbraceright}\ {\isasymunion}\ S\ {\isasymin}\ C{\isacharparenright}\isanewline
\ \ \ \ {\isasymand}\ {\isacharparenleft}{\isasymforall}F\ G\ H{\isachardot}\ Dis\ F\ G\ H\ {\isasymlongrightarrow}\ F\ {\isasymin}\ S\ {\isasymlongrightarrow}\ {\isacharbraceleft}G{\isacharbraceright}\ {\isasymunion}\ S\ {\isasymin}\ C\ {\isasymor}\ {\isacharbraceleft}H{\isacharbraceright}\ {\isasymunion}\ S\ {\isasymin}\ C{\isacharparenright}{\isachardoublequoteclose}\isanewline
\ \ \ \ \isacommand{using}\isamarkupfalse%
\ assms{\isacharparenleft}{\isadigit{1}}{\isacharparenright}\ \isacommand{by}\isamarkupfalse%
\ {\isacharparenleft}rule\ pcp{\isacharunderscore}alt{\isadigit{1}}{\isacharparenright}\isanewline
\ \ \isacommand{have}\isamarkupfalse%
\ H{\isacharcolon}{\isachardoublequoteopen}{\isasymbottom}\ {\isasymnotin}\ S\isanewline
\ \ \ \ {\isasymand}\ {\isacharparenleft}{\isasymforall}k{\isachardot}\ Atom\ k\ {\isasymin}\ S\ {\isasymlongrightarrow}\ \isactrlbold {\isasymnot}\ {\isacharparenleft}Atom\ k{\isacharparenright}\ {\isasymin}\ S\ {\isasymlongrightarrow}\ False{\isacharparenright}\isanewline
\ \ \ \ {\isasymand}\ {\isacharparenleft}{\isasymforall}F\ G\ H{\isachardot}\ Con\ F\ G\ H\ {\isasymlongrightarrow}\ F\ {\isasymin}\ S\ {\isasymlongrightarrow}\ {\isacharbraceleft}G{\isacharcomma}H{\isacharbraceright}\ {\isasymunion}\ S\ {\isasymin}\ C{\isacharparenright}\isanewline
\ \ \ \ {\isasymand}\ {\isacharparenleft}{\isasymforall}F\ G\ H{\isachardot}\ Dis\ F\ G\ H\ {\isasymlongrightarrow}\ F\ {\isasymin}\ S\ {\isasymlongrightarrow}\ {\isacharbraceleft}G{\isacharbraceright}\ {\isasymunion}\ S\ {\isasymin}\ C\ {\isasymor}\ {\isacharbraceleft}H{\isacharbraceright}\ {\isasymunion}\ S\ {\isasymin}\ C{\isacharparenright}{\isachardoublequoteclose}\isanewline
\ \ \ \ \ \isacommand{using}\isamarkupfalse%
\ PCP\ {\isacartoucheopen}S\ {\isasymin}\ C{\isacartoucheclose}\ \isacommand{by}\isamarkupfalse%
\ {\isacharparenleft}rule\ bspec{\isacharparenright}\isanewline
\ \ \isacommand{then}\isamarkupfalse%
\ \isacommand{have}\isamarkupfalse%
\ A{\isadigit{1}}{\isacharcolon}{\isachardoublequoteopen}{\isasymbottom}\ {\isasymnotin}\ S{\isachardoublequoteclose}\isanewline
\ \ \ \ \isacommand{by}\isamarkupfalse%
\ {\isacharparenleft}rule\ conjunct{\isadigit{1}}{\isacharparenright}\isanewline
\ \ \isacommand{have}\isamarkupfalse%
\ A{\isadigit{2}}{\isacharcolon}{\isachardoublequoteopen}{\isasymforall}k{\isachardot}\ Atom\ k\ {\isasymin}\ S\ {\isasymlongrightarrow}\ \isactrlbold {\isasymnot}\ {\isacharparenleft}Atom\ k{\isacharparenright}\ {\isasymin}\ S\ {\isasymlongrightarrow}\ False{\isachardoublequoteclose}\isanewline
\ \ \ \ \isacommand{using}\isamarkupfalse%
\ H\ \isacommand{by}\isamarkupfalse%
\ {\isacharparenleft}iprover\ elim{\isacharcolon}\ conjunct{\isadigit{2}}\ conjunct{\isadigit{1}}{\isacharparenright}\isanewline
\ \ \isacommand{have}\isamarkupfalse%
\ S{\isadigit{3}}{\isacharcolon}{\isachardoublequoteopen}{\isasymforall}F\ G\ H{\isachardot}\ Con\ F\ G\ H\ {\isasymlongrightarrow}\ F\ {\isasymin}\ S\ {\isasymlongrightarrow}\ {\isacharbraceleft}G{\isacharcomma}H{\isacharbraceright}\ {\isasymunion}\ S\ {\isasymin}\ C{\isachardoublequoteclose}\isanewline
\ \ \ \ \isacommand{using}\isamarkupfalse%
\ H\ \isacommand{by}\isamarkupfalse%
\ {\isacharparenleft}iprover\ elim{\isacharcolon}\ conjunct{\isadigit{2}}\ conjunct{\isadigit{1}}{\isacharparenright}\isanewline
\ \ \isacommand{have}\isamarkupfalse%
\ A{\isadigit{3}}{\isacharcolon}{\isachardoublequoteopen}{\isasymforall}F\ G\ H{\isachardot}\ Con\ F\ G\ H\ {\isasymlongrightarrow}\ F\ {\isasymin}\ S\ {\isasymlongrightarrow}\ {\isacharbraceleft}G{\isacharcomma}\ H{\isacharbraceright}\ {\isasymunion}\ S\ {\isasymin}\ {\isacharparenleft}extensionFin\ C{\isacharparenright}{\isachardoublequoteclose}\isanewline
\ \ \isacommand{proof}\isamarkupfalse%
\ {\isacharparenleft}rule\ allI{\isacharparenright}{\isacharplus}\isanewline
\ \ \ \ \isacommand{fix}\isamarkupfalse%
\ F\ G\ H\isanewline
\ \ \ \ \isacommand{show}\isamarkupfalse%
\ {\isachardoublequoteopen}Con\ F\ G\ H\ {\isasymlongrightarrow}\ F\ {\isasymin}\ S\ {\isasymlongrightarrow}\ {\isacharbraceleft}G{\isacharcomma}\ H{\isacharbraceright}\ {\isasymunion}\ S\ {\isasymin}\ {\isacharparenleft}extensionFin\ C{\isacharparenright}{\isachardoublequoteclose}\isanewline
\ \ \ \ \isacommand{proof}\isamarkupfalse%
\ {\isacharparenleft}rule\ impI{\isacharparenright}{\isacharplus}\isanewline
\ \ \ \ \ \ \isacommand{assume}\isamarkupfalse%
\ {\isachardoublequoteopen}Con\ F\ G\ H{\isachardoublequoteclose}\isanewline
\ \ \ \ \ \ \isacommand{assume}\isamarkupfalse%
\ {\isachardoublequoteopen}F\ {\isasymin}\ S{\isachardoublequoteclose}\ \isanewline
\ \ \ \ \ \ \isacommand{have}\isamarkupfalse%
\ {\isachardoublequoteopen}Con\ F\ G\ H\ {\isasymlongrightarrow}\ F\ {\isasymin}\ S\ {\isasymlongrightarrow}\ {\isacharbraceleft}G{\isacharcomma}H{\isacharbraceright}\ {\isasymunion}\ S\ {\isasymin}\ C{\isachardoublequoteclose}\isanewline
\ \ \ \ \ \ \ \ \isacommand{using}\isamarkupfalse%
\ S{\isadigit{3}}\ \isacommand{by}\isamarkupfalse%
\ {\isacharparenleft}iprover\ elim{\isacharcolon}\ allE{\isacharparenright}\isanewline
\ \ \ \ \ \ \isacommand{then}\isamarkupfalse%
\ \isacommand{have}\isamarkupfalse%
\ {\isachardoublequoteopen}F\ {\isasymin}\ S\ {\isasymlongrightarrow}\ {\isacharbraceleft}G{\isacharcomma}H{\isacharbraceright}\ {\isasymunion}\ S\ {\isasymin}\ C{\isachardoublequoteclose}\isanewline
\ \ \ \ \ \ \ \ \isacommand{using}\isamarkupfalse%
\ {\isacartoucheopen}Con\ F\ G\ H{\isacartoucheclose}\ \isacommand{by}\isamarkupfalse%
\ {\isacharparenleft}rule\ mp{\isacharparenright}\isanewline
\ \ \ \ \ \ \isacommand{then}\isamarkupfalse%
\ \isacommand{have}\isamarkupfalse%
\ {\isachardoublequoteopen}{\isacharbraceleft}G{\isacharcomma}H{\isacharbraceright}\ {\isasymunion}\ S\ {\isasymin}\ C{\isachardoublequoteclose}\isanewline
\ \ \ \ \ \ \ \ \isacommand{using}\isamarkupfalse%
\ {\isacartoucheopen}F\ {\isasymin}\ S{\isacartoucheclose}\ \isacommand{by}\isamarkupfalse%
\ {\isacharparenleft}rule\ mp{\isacharparenright}\isanewline
\ \ \ \ \ \ \isacommand{thus}\isamarkupfalse%
\ {\isachardoublequoteopen}{\isacharbraceleft}G{\isacharcomma}H{\isacharbraceright}\ {\isasymunion}\ S\ {\isasymin}\ {\isacharparenleft}extensionFin\ C{\isacharparenright}{\isachardoublequoteclose}\isanewline
\ \ \ \ \ \ \ \ \isacommand{unfolding}\isamarkupfalse%
\ extensionFin\ \isacommand{by}\isamarkupfalse%
\ {\isacharparenleft}rule\ UnI{\isadigit{1}}{\isacharparenright}\isanewline
\ \ \ \ \isacommand{qed}\isamarkupfalse%
\isanewline
\ \ \isacommand{qed}\isamarkupfalse%
\isanewline
\ \ \isacommand{have}\isamarkupfalse%
\ S{\isadigit{4}}{\isacharcolon}{\isachardoublequoteopen}{\isasymforall}F\ G\ H{\isachardot}\ Dis\ F\ G\ H\ {\isasymlongrightarrow}\ F\ {\isasymin}\ S\ {\isasymlongrightarrow}\ {\isacharbraceleft}G{\isacharbraceright}\ {\isasymunion}\ S\ {\isasymin}\ C\ {\isasymor}\ {\isacharbraceleft}H{\isacharbraceright}\ {\isasymunion}\ S\ {\isasymin}\ C{\isachardoublequoteclose}\isanewline
\ \ \ \ \isacommand{using}\isamarkupfalse%
\ H\ \isacommand{by}\isamarkupfalse%
\ {\isacharparenleft}iprover\ elim{\isacharcolon}\ conjunct{\isadigit{2}}{\isacharparenright}\isanewline
\ \ \isacommand{have}\isamarkupfalse%
\ A{\isadigit{4}}{\isacharcolon}{\isachardoublequoteopen}{\isasymforall}F\ G\ H{\isachardot}\ Dis\ F\ G\ H\ {\isasymlongrightarrow}\ F\ {\isasymin}\ S\ {\isasymlongrightarrow}\ {\isacharbraceleft}G{\isacharbraceright}\ {\isasymunion}\ S\ {\isasymin}\ {\isacharparenleft}extensionFin\ C{\isacharparenright}\ {\isasymor}\ {\isacharbraceleft}H{\isacharbraceright}\ {\isasymunion}\ S\ {\isasymin}\ {\isacharparenleft}extensionFin\ C{\isacharparenright}{\isachardoublequoteclose}\isanewline
\ \ \isacommand{proof}\isamarkupfalse%
\ {\isacharparenleft}rule\ allI{\isacharparenright}{\isacharplus}\isanewline
\ \ \ \ \isacommand{fix}\isamarkupfalse%
\ F\ G\ H\isanewline
\ \ \ \ \isacommand{show}\isamarkupfalse%
\ {\isachardoublequoteopen}Dis\ F\ G\ H\ {\isasymlongrightarrow}\ F\ {\isasymin}\ S\ {\isasymlongrightarrow}\ {\isacharbraceleft}G{\isacharbraceright}\ {\isasymunion}\ S\ {\isasymin}\ {\isacharparenleft}extensionFin\ C{\isacharparenright}\ {\isasymor}\ {\isacharbraceleft}H{\isacharbraceright}\ {\isasymunion}\ S\ {\isasymin}\ {\isacharparenleft}extensionFin\ C{\isacharparenright}{\isachardoublequoteclose}\isanewline
\ \ \ \ \isacommand{proof}\isamarkupfalse%
\ {\isacharparenleft}rule\ impI{\isacharparenright}{\isacharplus}\isanewline
\ \ \ \ \ \ \isacommand{assume}\isamarkupfalse%
\ {\isachardoublequoteopen}Dis\ F\ G\ H{\isachardoublequoteclose}\isanewline
\ \ \ \ \ \ \isacommand{assume}\isamarkupfalse%
\ {\isachardoublequoteopen}F\ {\isasymin}\ S{\isachardoublequoteclose}\ \isanewline
\ \ \ \ \ \ \isacommand{have}\isamarkupfalse%
\ {\isachardoublequoteopen}Dis\ F\ G\ H\ {\isasymlongrightarrow}\ F\ {\isasymin}\ S\ {\isasymlongrightarrow}\ {\isacharbraceleft}G{\isacharbraceright}\ {\isasymunion}\ S\ {\isasymin}\ C\ {\isasymor}\ {\isacharbraceleft}H{\isacharbraceright}\ {\isasymunion}\ S\ {\isasymin}\ C{\isachardoublequoteclose}\isanewline
\ \ \ \ \ \ \ \ \isacommand{using}\isamarkupfalse%
\ S{\isadigit{4}}\ \isacommand{by}\isamarkupfalse%
\ {\isacharparenleft}iprover\ elim{\isacharcolon}\ allE{\isacharparenright}\isanewline
\ \ \ \ \ \ \isacommand{then}\isamarkupfalse%
\ \isacommand{have}\isamarkupfalse%
\ {\isachardoublequoteopen}F\ {\isasymin}\ S\ {\isasymlongrightarrow}\ {\isacharbraceleft}G{\isacharbraceright}\ {\isasymunion}\ S\ {\isasymin}\ C\ {\isasymor}\ {\isacharbraceleft}H{\isacharbraceright}\ {\isasymunion}\ S\ {\isasymin}\ C{\isachardoublequoteclose}\isanewline
\ \ \ \ \ \ \ \ \isacommand{using}\isamarkupfalse%
\ {\isacartoucheopen}Dis\ F\ G\ H{\isacartoucheclose}\ \isacommand{by}\isamarkupfalse%
\ {\isacharparenleft}rule\ mp{\isacharparenright}\isanewline
\ \ \ \ \ \ \isacommand{then}\isamarkupfalse%
\ \isacommand{have}\isamarkupfalse%
\ {\isachardoublequoteopen}{\isacharbraceleft}G{\isacharbraceright}\ {\isasymunion}\ S\ {\isasymin}\ C\ {\isasymor}\ {\isacharbraceleft}H{\isacharbraceright}\ {\isasymunion}\ S\ {\isasymin}\ C{\isachardoublequoteclose}\isanewline
\ \ \ \ \ \ \ \ \isacommand{using}\isamarkupfalse%
\ {\isacartoucheopen}F\ {\isasymin}\ S{\isacartoucheclose}\ \isacommand{by}\isamarkupfalse%
\ {\isacharparenleft}rule\ mp{\isacharparenright}\isanewline
\ \ \ \ \ \ \isacommand{thus}\isamarkupfalse%
\ {\isachardoublequoteopen}{\isacharbraceleft}G{\isacharbraceright}\ {\isasymunion}\ S\ {\isasymin}\ {\isacharparenleft}extensionFin\ C{\isacharparenright}\ {\isasymor}\ {\isacharbraceleft}H{\isacharbraceright}\ {\isasymunion}\ S\ {\isasymin}\ {\isacharparenleft}extensionFin\ C{\isacharparenright}{\isachardoublequoteclose}\isanewline
\ \ \ \ \ \ \isacommand{proof}\isamarkupfalse%
\ {\isacharparenleft}rule\ disjE{\isacharparenright}\isanewline
\ \ \ \ \ \ \ \ \isacommand{assume}\isamarkupfalse%
\ {\isachardoublequoteopen}{\isacharbraceleft}G{\isacharbraceright}\ {\isasymunion}\ S\ {\isasymin}\ C{\isachardoublequoteclose}\isanewline
\ \ \ \ \ \ \ \ \isacommand{then}\isamarkupfalse%
\ \isacommand{have}\isamarkupfalse%
\ {\isachardoublequoteopen}{\isacharbraceleft}G{\isacharbraceright}\ {\isasymunion}\ S\ {\isasymin}\ {\isacharparenleft}extensionFin\ C{\isacharparenright}{\isachardoublequoteclose}\isanewline
\ \ \ \ \ \ \ \ \ \ \isacommand{unfolding}\isamarkupfalse%
\ extensionFin\ \isacommand{by}\isamarkupfalse%
\ {\isacharparenleft}rule\ UnI{\isadigit{1}}{\isacharparenright}\isanewline
\ \ \ \ \ \ \ \ \isacommand{thus}\isamarkupfalse%
\ {\isachardoublequoteopen}{\isacharbraceleft}G{\isacharbraceright}\ {\isasymunion}\ S\ {\isasymin}\ {\isacharparenleft}extensionFin\ C{\isacharparenright}\ {\isasymor}\ {\isacharbraceleft}H{\isacharbraceright}\ {\isasymunion}\ S\ {\isasymin}\ {\isacharparenleft}extensionFin\ C{\isacharparenright}{\isachardoublequoteclose}\isanewline
\ \ \ \ \ \ \ \ \ \ \isacommand{by}\isamarkupfalse%
\ {\isacharparenleft}rule\ disjI{\isadigit{1}}{\isacharparenright}\isanewline
\ \ \ \ \ \ \isacommand{next}\isamarkupfalse%
\isanewline
\ \ \ \ \ \ \ \ \isacommand{assume}\isamarkupfalse%
\ {\isachardoublequoteopen}{\isacharbraceleft}H{\isacharbraceright}\ {\isasymunion}\ S\ {\isasymin}\ C{\isachardoublequoteclose}\isanewline
\ \ \ \ \ \ \ \ \isacommand{then}\isamarkupfalse%
\ \isacommand{have}\isamarkupfalse%
\ {\isachardoublequoteopen}{\isacharbraceleft}H{\isacharbraceright}\ {\isasymunion}\ S\ {\isasymin}\ {\isacharparenleft}extensionFin\ C{\isacharparenright}{\isachardoublequoteclose}\isanewline
\ \ \ \ \ \ \ \ \ \ \isacommand{unfolding}\isamarkupfalse%
\ extensionFin\ \isacommand{by}\isamarkupfalse%
\ {\isacharparenleft}rule\ UnI{\isadigit{1}}{\isacharparenright}\isanewline
\ \ \ \ \ \ \ \ \isacommand{thus}\isamarkupfalse%
\ {\isachardoublequoteopen}{\isacharbraceleft}G{\isacharbraceright}\ {\isasymunion}\ S\ {\isasymin}\ {\isacharparenleft}extensionFin\ C{\isacharparenright}\ {\isasymor}\ {\isacharbraceleft}H{\isacharbraceright}\ {\isasymunion}\ S\ {\isasymin}\ {\isacharparenleft}extensionFin\ C{\isacharparenright}{\isachardoublequoteclose}\isanewline
\ \ \ \ \ \ \ \ \ \ \isacommand{by}\isamarkupfalse%
\ {\isacharparenleft}rule\ disjI{\isadigit{2}}{\isacharparenright}\isanewline
\ \ \ \ \ \ \isacommand{qed}\isamarkupfalse%
\isanewline
\ \ \ \ \isacommand{qed}\isamarkupfalse%
\isanewline
\ \ \isacommand{qed}\isamarkupfalse%
\isanewline
\ \ \isacommand{show}\isamarkupfalse%
\ {\isachardoublequoteopen}{\isasymbottom}\ {\isasymnotin}\ S\ {\isasymand}\isanewline
\ \ \ \ \ \ \ \ {\isacharparenleft}{\isasymforall}k{\isachardot}\ Atom\ k\ {\isasymin}\ S\ {\isasymlongrightarrow}\ \isactrlbold {\isasymnot}\ {\isacharparenleft}Atom\ k{\isacharparenright}\ {\isasymin}\ S\ {\isasymlongrightarrow}\ False{\isacharparenright}\ {\isasymand}\isanewline
\ \ \ \ \ \ \ \ {\isacharparenleft}{\isasymforall}F\ G\ H{\isachardot}\ Con\ F\ G\ H\ {\isasymlongrightarrow}\ F\ {\isasymin}\ S\ {\isasymlongrightarrow}\ {\isacharbraceleft}G{\isacharcomma}\ H{\isacharbraceright}\ {\isasymunion}\ S\ {\isasymin}\ {\isacharparenleft}extensionFin\ C{\isacharparenright}{\isacharparenright}\ {\isasymand}\isanewline
\ \ \ \ \ \ \ \ {\isacharparenleft}{\isasymforall}F\ G\ H{\isachardot}\ Dis\ F\ G\ H\ {\isasymlongrightarrow}\ F\ {\isasymin}\ S\ {\isasymlongrightarrow}\ {\isacharbraceleft}G{\isacharbraceright}\ {\isasymunion}\ S\ {\isasymin}\ {\isacharparenleft}extensionFin\ C{\isacharparenright}\ {\isasymor}\ {\isacharbraceleft}H{\isacharbraceright}\ {\isasymunion}\ S\ {\isasymin}\ {\isacharparenleft}extensionFin\ C{\isacharparenright}{\isacharparenright}{\isachardoublequoteclose}\isanewline
\ \ \ \ \isacommand{using}\isamarkupfalse%
\ A{\isadigit{1}}\ A{\isadigit{2}}\ A{\isadigit{3}}\ A{\isadigit{4}}\ \isacommand{by}\isamarkupfalse%
\ {\isacharparenleft}iprover\ intro{\isacharcolon}\ conjI{\isacharparenright}\isanewline
\isacommand{qed}\isamarkupfalse%
%
\endisatagproof
{\isafoldproof}%
%
\isadelimproof
%
\endisadelimproof
%
\begin{isamarkuptext}%
Como hemos señalado con anterioridad, para probar el caso en que \isa{S\ {\isasymin}\ E}, donde \isa{E} es la 
  colección formada por aquellos conjuntos cuyos subconjuntos finitos pertenecen a \isa{C}, precisaremos 
  de distintos lemas auxiliares. El primero de ellos demuestra detalladamente que si \isa{C} es una
  colección con la propiedad de consistencia proposicional y cerrada bajo subconjuntos, \isa{S\ {\isasymin}\ E}
  y sea \isa{F} una fórmula de tipo \isa{{\isasymalpha}} con componentes \isa{{\isasymalpha}\isactrlsub {\isadigit{1}}} y \isa{{\isasymalpha}\isactrlsub {\isadigit{2}}}, se verifica que \isa{{\isacharbraceleft}{\isasymalpha}\isactrlsub {\isadigit{1}}{\isacharcomma}{\isasymalpha}\isactrlsub {\isadigit{2}}{\isacharbraceright}\ {\isasymunion}\ S} 
  pertenece a la extensión \isa{C{\isacharprime}\ {\isacharequal}\ C\ {\isasymunion}\ E}.%
\end{isamarkuptext}\isamarkuptrue%
\isacommand{lemma}\isamarkupfalse%
\ ex{\isadigit{3}}{\isacharunderscore}pcp{\isacharunderscore}SinE{\isacharunderscore}CON{\isacharcolon}\isanewline
\ \ \isakeyword{assumes}\ {\isachardoublequoteopen}pcp\ C{\isachardoublequoteclose}\isanewline
\ \ \ \ \ \ \ \ \ \ {\isachardoublequoteopen}subset{\isacharunderscore}closed\ C{\isachardoublequoteclose}\isanewline
\ \ \ \ \ \ \ \ \ \ {\isachardoublequoteopen}S\ {\isasymin}\ {\isacharparenleft}extF\ C{\isacharparenright}{\isachardoublequoteclose}\isanewline
\ \ \ \ \ \ \ \ \ \ {\isachardoublequoteopen}Con\ F\ G\ H{\isachardoublequoteclose}\isanewline
\ \ \ \ \ \ \ \ \ \ {\isachardoublequoteopen}F\ {\isasymin}\ S{\isachardoublequoteclose}\isanewline
\ \ \isakeyword{shows}\ {\isachardoublequoteopen}{\isacharbraceleft}G{\isacharcomma}H{\isacharbraceright}\ {\isasymunion}\ S\ {\isasymin}\ {\isacharparenleft}extensionFin\ C{\isacharparenright}{\isachardoublequoteclose}\ \isanewline
%
\isadelimproof
%
\endisadelimproof
%
\isatagproof
\isacommand{proof}\isamarkupfalse%
\ {\isacharminus}\isanewline
\ \ \isacommand{have}\isamarkupfalse%
\ PCP{\isacharcolon}{\isachardoublequoteopen}{\isasymforall}S\ {\isasymin}\ C{\isachardot}\isanewline
\ \ {\isasymbottom}\ {\isasymnotin}\ S\isanewline
{\isasymand}\ {\isacharparenleft}{\isasymforall}k{\isachardot}\ Atom\ k\ {\isasymin}\ S\ {\isasymlongrightarrow}\ \isactrlbold {\isasymnot}\ {\isacharparenleft}Atom\ k{\isacharparenright}\ {\isasymin}\ S\ {\isasymlongrightarrow}\ False{\isacharparenright}\isanewline
{\isasymand}\ {\isacharparenleft}{\isasymforall}F\ G\ H{\isachardot}\ Con\ F\ G\ H\ {\isasymlongrightarrow}\ F\ {\isasymin}\ S\ {\isasymlongrightarrow}\ {\isacharbraceleft}G{\isacharcomma}H{\isacharbraceright}\ {\isasymunion}\ S\ {\isasymin}\ C{\isacharparenright}\isanewline
{\isasymand}\ {\isacharparenleft}{\isasymforall}F\ G\ H{\isachardot}\ Dis\ F\ G\ H\ {\isasymlongrightarrow}\ F\ {\isasymin}\ S\ {\isasymlongrightarrow}\ {\isacharbraceleft}G{\isacharbraceright}\ {\isasymunion}\ S\ {\isasymin}\ C\ {\isasymor}\ {\isacharbraceleft}H{\isacharbraceright}\ {\isasymunion}\ S\ {\isasymin}\ C{\isacharparenright}{\isachardoublequoteclose}\isanewline
\ \ \ \ \isacommand{using}\isamarkupfalse%
\ assms{\isacharparenleft}{\isadigit{1}}{\isacharparenright}\ \isacommand{by}\isamarkupfalse%
\ {\isacharparenleft}rule\ pcp{\isacharunderscore}alt{\isadigit{1}}{\isacharparenright}\isanewline
\ \ \isacommand{have}\isamarkupfalse%
\ {\isadigit{1}}{\isacharcolon}{\isachardoublequoteopen}{\isasymforall}S{\isacharprime}\ {\isasymsubseteq}\ S{\isachardot}\ finite\ S{\isacharprime}\ {\isasymlongrightarrow}\ F\ {\isasymin}\ S{\isacharprime}\ {\isasymlongrightarrow}\ {\isacharbraceleft}G{\isacharcomma}H{\isacharbraceright}\ {\isasymunion}\ S{\isacharprime}\ {\isasymin}\ C{\isachardoublequoteclose}\isanewline
\ \ \isacommand{proof}\isamarkupfalse%
\ {\isacharparenleft}rule\ sallI{\isacharparenright}\isanewline
\ \ \ \ \isacommand{fix}\isamarkupfalse%
\ S{\isacharprime}\isanewline
\ \ \ \ \isacommand{assume}\isamarkupfalse%
\ {\isachardoublequoteopen}S{\isacharprime}\ {\isasymsubseteq}\ S{\isachardoublequoteclose}\isanewline
\ \ \ \ \isacommand{show}\isamarkupfalse%
\ {\isachardoublequoteopen}finite\ S{\isacharprime}\ {\isasymlongrightarrow}\ F\ {\isasymin}\ S{\isacharprime}\ {\isasymlongrightarrow}\ {\isacharbraceleft}G{\isacharcomma}H{\isacharbraceright}\ {\isasymunion}\ S{\isacharprime}\ {\isasymin}\ C{\isachardoublequoteclose}\isanewline
\ \ \ \ \isacommand{proof}\isamarkupfalse%
\ {\isacharparenleft}rule\ impI{\isacharparenright}{\isacharplus}\isanewline
\ \ \ \ \ \ \isacommand{assume}\isamarkupfalse%
\ {\isachardoublequoteopen}finite\ S{\isacharprime}{\isachardoublequoteclose}\isanewline
\ \ \ \ \ \ \isacommand{assume}\isamarkupfalse%
\ {\isachardoublequoteopen}F\ {\isasymin}\ S{\isacharprime}{\isachardoublequoteclose}\isanewline
\ \ \ \ \ \ \isacommand{have}\isamarkupfalse%
\ E{\isacharcolon}{\isachardoublequoteopen}{\isasymforall}S{\isacharprime}\ {\isasymsubseteq}\ S{\isachardot}\ finite\ S{\isacharprime}\ {\isasymlongrightarrow}\ S{\isacharprime}\ {\isasymin}\ C{\isachardoublequoteclose}\isanewline
\ \ \ \ \ \ \ \ \isacommand{using}\isamarkupfalse%
\ assms{\isacharparenleft}{\isadigit{3}}{\isacharparenright}\ \isacommand{unfolding}\isamarkupfalse%
\ extF\ \isacommand{by}\isamarkupfalse%
\ {\isacharparenleft}rule\ CollectD{\isacharparenright}\isanewline
\ \ \ \ \ \ \isacommand{then}\isamarkupfalse%
\ \isacommand{have}\isamarkupfalse%
\ {\isachardoublequoteopen}finite\ S{\isacharprime}\ {\isasymlongrightarrow}\ S{\isacharprime}\ {\isasymin}\ C{\isachardoublequoteclose}\isanewline
\ \ \ \ \ \ \ \ \isacommand{using}\isamarkupfalse%
\ {\isacartoucheopen}S{\isacharprime}\ {\isasymsubseteq}\ S{\isacartoucheclose}\ \isacommand{by}\isamarkupfalse%
\ {\isacharparenleft}rule\ sspec{\isacharparenright}\isanewline
\ \ \ \ \ \ \isacommand{then}\isamarkupfalse%
\ \isacommand{have}\isamarkupfalse%
\ {\isachardoublequoteopen}S{\isacharprime}\ {\isasymin}\ C{\isachardoublequoteclose}\isanewline
\ \ \ \ \ \ \ \ \isacommand{using}\isamarkupfalse%
\ {\isacartoucheopen}finite\ S{\isacharprime}{\isacartoucheclose}\ \isacommand{by}\isamarkupfalse%
\ {\isacharparenleft}rule\ mp{\isacharparenright}\isanewline
\ \ \ \ \ \ \isacommand{have}\isamarkupfalse%
\ {\isachardoublequoteopen}{\isasymbottom}\ {\isasymnotin}\ S{\isacharprime}\isanewline
\ \ \ \ \ \ \ \ \ \ \ \ {\isasymand}\ {\isacharparenleft}{\isasymforall}k{\isachardot}\ Atom\ k\ {\isasymin}\ S{\isacharprime}\ {\isasymlongrightarrow}\ \isactrlbold {\isasymnot}\ {\isacharparenleft}Atom\ k{\isacharparenright}\ {\isasymin}\ S{\isacharprime}\ {\isasymlongrightarrow}\ False{\isacharparenright}\isanewline
\ \ \ \ \ \ \ \ \ \ \ \ {\isasymand}\ {\isacharparenleft}{\isasymforall}F\ G\ H{\isachardot}\ Con\ F\ G\ H\ {\isasymlongrightarrow}\ F\ {\isasymin}\ S{\isacharprime}\ {\isasymlongrightarrow}\ {\isacharbraceleft}G{\isacharcomma}H{\isacharbraceright}\ {\isasymunion}\ S{\isacharprime}\ {\isasymin}\ C{\isacharparenright}\isanewline
\ \ \ \ \ \ \ \ \ \ \ \ {\isasymand}\ {\isacharparenleft}{\isasymforall}F\ G\ H{\isachardot}\ Dis\ F\ G\ H\ {\isasymlongrightarrow}\ F\ {\isasymin}\ S{\isacharprime}\ {\isasymlongrightarrow}\ {\isacharbraceleft}G{\isacharbraceright}\ {\isasymunion}\ S{\isacharprime}\ {\isasymin}\ C\ {\isasymor}\ {\isacharbraceleft}H{\isacharbraceright}\ {\isasymunion}\ S{\isacharprime}\ {\isasymin}\ C{\isacharparenright}{\isachardoublequoteclose}\isanewline
\ \ \ \ \ \ \ \ \isacommand{using}\isamarkupfalse%
\ PCP\ {\isacartoucheopen}S{\isacharprime}\ {\isasymin}\ C{\isacartoucheclose}\ \isacommand{by}\isamarkupfalse%
\ {\isacharparenleft}rule\ bspec{\isacharparenright}\isanewline
\ \ \ \ \ \ \isacommand{then}\isamarkupfalse%
\ \isacommand{have}\isamarkupfalse%
\ {\isachardoublequoteopen}{\isasymforall}F\ G\ H{\isachardot}\ Con\ F\ G\ H\ {\isasymlongrightarrow}\ F\ {\isasymin}\ S{\isacharprime}\ {\isasymlongrightarrow}\ {\isacharbraceleft}G{\isacharcomma}\ H{\isacharbraceright}\ {\isasymunion}\ S{\isacharprime}\ {\isasymin}\ C{\isachardoublequoteclose}\isanewline
\ \ \ \ \ \ \ \ \isacommand{by}\isamarkupfalse%
\ {\isacharparenleft}iprover\ elim{\isacharcolon}\ conjunct{\isadigit{2}}\ conjunct{\isadigit{1}}{\isacharparenright}\isanewline
\ \ \ \ \ \ \isacommand{then}\isamarkupfalse%
\ \isacommand{have}\isamarkupfalse%
\ {\isachardoublequoteopen}Con\ F\ G\ H\ {\isasymlongrightarrow}\ F\ {\isasymin}\ S{\isacharprime}\ {\isasymlongrightarrow}\ {\isacharbraceleft}G{\isacharcomma}\ H{\isacharbraceright}\ {\isasymunion}\ S{\isacharprime}\ {\isasymin}\ C{\isachardoublequoteclose}\isanewline
\ \ \ \ \ \ \ \ \isacommand{by}\isamarkupfalse%
\ {\isacharparenleft}iprover\ elim{\isacharcolon}\ allE{\isacharparenright}\isanewline
\ \ \ \ \ \ \isacommand{then}\isamarkupfalse%
\ \isacommand{have}\isamarkupfalse%
\ {\isachardoublequoteopen}F\ {\isasymin}\ S{\isacharprime}\ {\isasymlongrightarrow}\ {\isacharbraceleft}G{\isacharcomma}H{\isacharbraceright}\ {\isasymunion}\ S{\isacharprime}\ {\isasymin}\ C{\isachardoublequoteclose}\isanewline
\ \ \ \ \ \ \ \ \isacommand{using}\isamarkupfalse%
\ assms{\isacharparenleft}{\isadigit{4}}{\isacharparenright}\ \isacommand{by}\isamarkupfalse%
\ {\isacharparenleft}rule\ mp{\isacharparenright}\isanewline
\ \ \ \ \ \ \isacommand{thus}\isamarkupfalse%
\ {\isachardoublequoteopen}{\isacharbraceleft}G{\isacharcomma}\ H{\isacharbraceright}\ {\isasymunion}\ S{\isacharprime}\ {\isasymin}\ C{\isachardoublequoteclose}\isanewline
\ \ \ \ \ \ \ \ \isacommand{using}\isamarkupfalse%
\ {\isacartoucheopen}F\ {\isasymin}\ S{\isacharprime}{\isacartoucheclose}\ \isacommand{by}\isamarkupfalse%
\ {\isacharparenleft}rule\ mp{\isacharparenright}\isanewline
\ \ \ \ \isacommand{qed}\isamarkupfalse%
\isanewline
\ \ \isacommand{qed}\isamarkupfalse%
\isanewline
\ \ \isacommand{have}\isamarkupfalse%
\ {\isachardoublequoteopen}{\isacharbraceleft}G{\isacharcomma}H{\isacharbraceright}\ {\isasymunion}\ S\ {\isasymin}\ {\isacharparenleft}extF\ C{\isacharparenright}{\isachardoublequoteclose}\isanewline
\ \ \ \ \isacommand{unfolding}\isamarkupfalse%
\ mem{\isacharunderscore}Collect{\isacharunderscore}eq\ Un{\isacharunderscore}iff\ extF\isanewline
\ \ \isacommand{proof}\isamarkupfalse%
\ {\isacharparenleft}rule\ sallI{\isacharparenright}\isanewline
\ \ \ \ \isacommand{fix}\isamarkupfalse%
\ S{\isacharprime}\isanewline
\ \ \ \ \isacommand{assume}\isamarkupfalse%
\ H{\isacharcolon}{\isachardoublequoteopen}S{\isacharprime}\ {\isasymsubseteq}\ {\isacharbraceleft}G{\isacharcomma}H{\isacharbraceright}\ {\isasymunion}\ S{\isachardoublequoteclose}\isanewline
\ \ \ \ \isacommand{show}\isamarkupfalse%
\ {\isachardoublequoteopen}finite\ S{\isacharprime}\ {\isasymlongrightarrow}\ S{\isacharprime}\ {\isasymin}\ C{\isachardoublequoteclose}\isanewline
\ \ \ \ \isacommand{proof}\isamarkupfalse%
\ {\isacharparenleft}rule\ impI{\isacharparenright}\isanewline
\ \ \ \ \ \ \isacommand{assume}\isamarkupfalse%
\ {\isachardoublequoteopen}finite\ S{\isacharprime}{\isachardoublequoteclose}\isanewline
\ \ \ \ \ \ \isacommand{have}\isamarkupfalse%
\ {\isachardoublequoteopen}S{\isacharprime}\ {\isacharminus}\ {\isacharbraceleft}G{\isacharcomma}H{\isacharbraceright}\ {\isasymsubseteq}\ S{\isachardoublequoteclose}\isanewline
\ \ \ \ \ \ \ \ \isacommand{using}\isamarkupfalse%
\ H\ \isacommand{by}\isamarkupfalse%
\ {\isacharparenleft}simp\ only{\isacharcolon}\ Diff{\isacharunderscore}subset{\isacharunderscore}conv{\isacharparenright}\isanewline
\ \ \ \ \ \ \isacommand{have}\isamarkupfalse%
\ {\isachardoublequoteopen}F\ {\isasymin}\ S\ {\isasymand}\ {\isacharparenleft}S{\isacharprime}\ {\isacharminus}\ {\isacharbraceleft}G{\isacharcomma}H{\isacharbraceright}\ {\isasymsubseteq}\ S{\isacharparenright}{\isachardoublequoteclose}\isanewline
\ \ \ \ \ \ \ \ \isacommand{using}\isamarkupfalse%
\ assms{\isacharparenleft}{\isadigit{5}}{\isacharparenright}\ {\isacartoucheopen}S{\isacharprime}\ {\isacharminus}\ {\isacharbraceleft}G{\isacharcomma}H{\isacharbraceright}\ {\isasymsubseteq}\ S{\isacartoucheclose}\ \isacommand{by}\isamarkupfalse%
\ {\isacharparenleft}rule\ conjI{\isacharparenright}\isanewline
\ \ \ \ \ \ \isacommand{then}\isamarkupfalse%
\ \isacommand{have}\isamarkupfalse%
\ {\isachardoublequoteopen}insert\ F\ \ {\isacharparenleft}S{\isacharprime}\ {\isacharminus}\ {\isacharbraceleft}G{\isacharcomma}H{\isacharbraceright}{\isacharparenright}\ {\isasymsubseteq}\ S{\isachardoublequoteclose}\ \isanewline
\ \ \ \ \ \ \ \ \isacommand{by}\isamarkupfalse%
\ {\isacharparenleft}simp\ only{\isacharcolon}\ insert{\isacharunderscore}subset{\isacharparenright}\isanewline
\ \ \ \ \ \ \isacommand{have}\isamarkupfalse%
\ F{\isadigit{1}}{\isacharcolon}{\isachardoublequoteopen}finite\ {\isacharparenleft}insert\ F\ \ {\isacharparenleft}S{\isacharprime}\ {\isacharminus}\ {\isacharbraceleft}G{\isacharcomma}H{\isacharbraceright}{\isacharparenright}{\isacharparenright}\ {\isasymlongrightarrow}\ F\ {\isasymin}\ {\isacharparenleft}insert\ F\ \ {\isacharparenleft}S{\isacharprime}\ {\isacharminus}\ {\isacharbraceleft}G{\isacharcomma}H{\isacharbraceright}{\isacharparenright}{\isacharparenright}\ {\isasymlongrightarrow}\ {\isacharbraceleft}G{\isacharcomma}H{\isacharbraceright}\ {\isasymunion}\ {\isacharparenleft}insert\ F\ \ {\isacharparenleft}S{\isacharprime}\ {\isacharminus}\ {\isacharbraceleft}G{\isacharcomma}H{\isacharbraceright}{\isacharparenright}{\isacharparenright}\ {\isasymin}\ C{\isachardoublequoteclose}\isanewline
\ \ \ \ \ \ \ \ \isacommand{using}\isamarkupfalse%
\ {\isadigit{1}}\ {\isacartoucheopen}insert\ F\ \ {\isacharparenleft}S{\isacharprime}\ {\isacharminus}\ {\isacharbraceleft}G{\isacharcomma}H{\isacharbraceright}{\isacharparenright}\ {\isasymsubseteq}\ S{\isacartoucheclose}\ \isacommand{by}\isamarkupfalse%
\ {\isacharparenleft}rule\ sspec{\isacharparenright}\isanewline
\ \ \ \ \ \ \isacommand{have}\isamarkupfalse%
\ {\isachardoublequoteopen}finite\ {\isacharparenleft}S{\isacharprime}\ {\isacharminus}\ {\isacharbraceleft}G{\isacharcomma}H{\isacharbraceright}{\isacharparenright}{\isachardoublequoteclose}\isanewline
\ \ \ \ \ \ \ \ \isacommand{using}\isamarkupfalse%
\ {\isacartoucheopen}finite\ S{\isacharprime}{\isacartoucheclose}\ \isacommand{by}\isamarkupfalse%
\ {\isacharparenleft}rule\ finite{\isacharunderscore}Diff{\isacharparenright}\isanewline
\ \ \ \ \ \ \isacommand{then}\isamarkupfalse%
\ \isacommand{have}\isamarkupfalse%
\ {\isachardoublequoteopen}finite\ {\isacharparenleft}insert\ F\ {\isacharparenleft}S{\isacharprime}\ {\isacharminus}\ {\isacharbraceleft}G{\isacharcomma}H{\isacharbraceright}{\isacharparenright}{\isacharparenright}{\isachardoublequoteclose}\ \isanewline
\ \ \ \ \ \ \ \ \isacommand{by}\isamarkupfalse%
\ {\isacharparenleft}rule\ finite{\isachardot}insertI{\isacharparenright}\isanewline
\ \ \ \ \ \ \isacommand{have}\isamarkupfalse%
\ F{\isadigit{2}}{\isacharcolon}{\isachardoublequoteopen}F\ {\isasymin}\ {\isacharparenleft}insert\ F\ \ {\isacharparenleft}S{\isacharprime}\ {\isacharminus}\ {\isacharbraceleft}G{\isacharcomma}H{\isacharbraceright}{\isacharparenright}{\isacharparenright}\ {\isasymlongrightarrow}\ {\isacharbraceleft}G{\isacharcomma}H{\isacharbraceright}\ {\isasymunion}\ {\isacharparenleft}insert\ F\ \ {\isacharparenleft}S{\isacharprime}\ {\isacharminus}\ {\isacharbraceleft}G{\isacharcomma}H{\isacharbraceright}{\isacharparenright}{\isacharparenright}\ {\isasymin}\ C{\isachardoublequoteclose}\isanewline
\ \ \ \ \ \ \ \ \isacommand{using}\isamarkupfalse%
\ F{\isadigit{1}}\ {\isacartoucheopen}finite\ {\isacharparenleft}insert\ F\ {\isacharparenleft}S{\isacharprime}\ {\isacharminus}\ {\isacharbraceleft}G{\isacharcomma}H{\isacharbraceright}{\isacharparenright}{\isacharparenright}{\isacartoucheclose}\ \isacommand{by}\isamarkupfalse%
\ {\isacharparenleft}rule\ mp{\isacharparenright}\isanewline
\ \ \ \ \ \ \isacommand{have}\isamarkupfalse%
\ {\isachardoublequoteopen}F\ {\isasymin}\ {\isacharparenleft}insert\ F\ \ {\isacharparenleft}S{\isacharprime}\ {\isacharminus}\ {\isacharbraceleft}G{\isacharcomma}H{\isacharbraceright}{\isacharparenright}{\isacharparenright}{\isachardoublequoteclose}\isanewline
\ \ \ \ \ \ \ \ \isacommand{by}\isamarkupfalse%
\ {\isacharparenleft}simp\ only{\isacharcolon}\ insertI{\isadigit{1}}{\isacharparenright}\isanewline
\ \ \ \ \ \ \isacommand{have}\isamarkupfalse%
\ F{\isadigit{3}}{\isacharcolon}{\isachardoublequoteopen}{\isacharbraceleft}G{\isacharcomma}H{\isacharbraceright}\ {\isasymunion}\ {\isacharparenleft}insert\ F\ {\isacharparenleft}S{\isacharprime}\ {\isacharminus}\ {\isacharbraceleft}G{\isacharcomma}H{\isacharbraceright}{\isacharparenright}{\isacharparenright}\ {\isasymin}\ C{\isachardoublequoteclose}\isanewline
\ \ \ \ \ \ \ \ \isacommand{using}\isamarkupfalse%
\ F{\isadigit{2}}\ {\isacartoucheopen}F\ {\isasymin}\ insert\ F\ {\isacharparenleft}S{\isacharprime}\ {\isacharminus}\ {\isacharbraceleft}G{\isacharcomma}H{\isacharbraceright}{\isacharparenright}{\isacartoucheclose}\ \isacommand{by}\isamarkupfalse%
\ {\isacharparenleft}rule\ mp{\isacharparenright}\isanewline
\ \ \ \ \ \ \isacommand{have}\isamarkupfalse%
\ IU{\isadigit{1}}{\isacharcolon}{\isachardoublequoteopen}insert\ F\ {\isacharparenleft}S{\isacharprime}\ {\isacharminus}\ {\isacharbraceleft}G{\isacharcomma}H{\isacharbraceright}{\isacharparenright}\ {\isacharequal}\ {\isacharbraceleft}F{\isacharbraceright}\ {\isasymunion}\ {\isacharparenleft}S{\isacharprime}\ {\isacharminus}\ {\isacharbraceleft}G{\isacharcomma}H{\isacharbraceright}{\isacharparenright}{\isachardoublequoteclose}\isanewline
\ \ \ \ \ \ \ \ \isacommand{by}\isamarkupfalse%
\ {\isacharparenleft}rule\ insert{\isacharunderscore}is{\isacharunderscore}Un{\isacharparenright}\isanewline
\ \ \ \ \ \ \isacommand{have}\isamarkupfalse%
\ IU{\isadigit{2}}{\isacharcolon}{\isachardoublequoteopen}insert\ F\ {\isacharparenleft}{\isacharbraceleft}G{\isacharcomma}H{\isacharbraceright}\ {\isasymunion}\ S{\isacharprime}{\isacharparenright}\ {\isacharequal}\ {\isacharbraceleft}F{\isacharbraceright}\ {\isasymunion}\ {\isacharparenleft}{\isacharbraceleft}G{\isacharcomma}H{\isacharbraceright}\ {\isasymunion}\ S{\isacharprime}{\isacharparenright}{\isachardoublequoteclose}\isanewline
\ \ \ \ \ \ \ \ \isacommand{by}\isamarkupfalse%
\ {\isacharparenleft}rule\ insert{\isacharunderscore}is{\isacharunderscore}Un{\isacharparenright}\isanewline
\ \ \ \ \ \ \isacommand{have}\isamarkupfalse%
\ GH{\isacharcolon}{\isachardoublequoteopen}insert\ G\ {\isacharparenleft}insert\ H\ S{\isacharprime}{\isacharparenright}\ {\isacharequal}\ {\isacharbraceleft}G{\isacharcomma}H{\isacharbraceright}\ {\isasymunion}\ S{\isacharprime}{\isachardoublequoteclose}\isanewline
\ \ \ \ \ \ \ \ \isacommand{by}\isamarkupfalse%
\ {\isacharparenleft}rule\ insertSetElem{\isacharparenright}\isanewline
\ \ \ \ \ \ \isacommand{have}\isamarkupfalse%
\ {\isachardoublequoteopen}{\isacharbraceleft}G{\isacharcomma}H{\isacharbraceright}\ {\isasymunion}\ {\isacharparenleft}insert\ F\ {\isacharparenleft}S{\isacharprime}\ {\isacharminus}\ {\isacharbraceleft}G{\isacharcomma}H{\isacharbraceright}{\isacharparenright}{\isacharparenright}\ {\isacharequal}\ {\isacharbraceleft}G{\isacharcomma}H{\isacharbraceright}\ {\isasymunion}\ {\isacharparenleft}{\isacharbraceleft}F{\isacharbraceright}\ {\isasymunion}\ {\isacharparenleft}S{\isacharprime}\ {\isacharminus}\ {\isacharbraceleft}G{\isacharcomma}H{\isacharbraceright}{\isacharparenright}{\isacharparenright}{\isachardoublequoteclose}\isanewline
\ \ \ \ \ \ \ \ \isacommand{by}\isamarkupfalse%
\ {\isacharparenleft}simp\ only{\isacharcolon}\ IU{\isadigit{1}}{\isacharparenright}\isanewline
\ \ \ \ \ \ \isacommand{also}\isamarkupfalse%
\ \isacommand{have}\isamarkupfalse%
\ {\isachardoublequoteopen}{\isasymdots}\ {\isacharequal}\ {\isacharbraceleft}F{\isacharbraceright}\ {\isasymunion}\ {\isacharparenleft}{\isacharbraceleft}G{\isacharcomma}H{\isacharbraceright}\ {\isasymunion}\ {\isacharparenleft}S{\isacharprime}\ {\isacharminus}\ {\isacharbraceleft}G{\isacharcomma}H{\isacharbraceright}{\isacharparenright}{\isacharparenright}{\isachardoublequoteclose}\isanewline
\ \ \ \ \ \ \ \ \isacommand{by}\isamarkupfalse%
\ {\isacharparenleft}simp\ only{\isacharcolon}\ Un{\isacharunderscore}left{\isacharunderscore}commute{\isacharparenright}\isanewline
\ \ \ \ \ \ \isacommand{also}\isamarkupfalse%
\ \isacommand{have}\isamarkupfalse%
\ {\isachardoublequoteopen}{\isasymdots}\ {\isacharequal}\ {\isacharbraceleft}F{\isacharbraceright}\ {\isasymunion}\ {\isacharparenleft}{\isacharbraceleft}G{\isacharcomma}H{\isacharbraceright}\ {\isasymunion}\ S{\isacharprime}{\isacharparenright}{\isachardoublequoteclose}\isanewline
\ \ \ \ \ \ \ \ \isacommand{by}\isamarkupfalse%
\ {\isacharparenleft}simp\ only{\isacharcolon}\ Un{\isacharunderscore}Diff{\isacharunderscore}cancel{\isacharparenright}\isanewline
\ \ \ \ \ \ \isacommand{also}\isamarkupfalse%
\ \isacommand{have}\isamarkupfalse%
\ {\isachardoublequoteopen}{\isasymdots}\ {\isacharequal}\ insert\ F\ {\isacharparenleft}{\isacharbraceleft}G{\isacharcomma}H{\isacharbraceright}\ {\isasymunion}\ S{\isacharprime}{\isacharparenright}{\isachardoublequoteclose}\isanewline
\ \ \ \ \ \ \ \ \isacommand{by}\isamarkupfalse%
\ {\isacharparenleft}simp\ only{\isacharcolon}\ IU{\isadigit{2}}{\isacharparenright}\isanewline
\ \ \ \ \ \ \isacommand{also}\isamarkupfalse%
\ \isacommand{have}\isamarkupfalse%
\ {\isachardoublequoteopen}{\isasymdots}\ {\isacharequal}\ insert\ F\ {\isacharparenleft}insert\ G\ {\isacharparenleft}insert\ H\ S{\isacharprime}{\isacharparenright}{\isacharparenright}{\isachardoublequoteclose}\isanewline
\ \ \ \ \ \ \ \ \isacommand{by}\isamarkupfalse%
\ {\isacharparenleft}simp\ only{\isacharcolon}\ GH{\isacharparenright}\isanewline
\ \ \ \ \ \ \isacommand{finally}\isamarkupfalse%
\ \isacommand{have}\isamarkupfalse%
\ F{\isadigit{4}}{\isacharcolon}{\isachardoublequoteopen}{\isacharbraceleft}G{\isacharcomma}H{\isacharbraceright}\ {\isasymunion}\ {\isacharparenleft}insert\ F\ {\isacharparenleft}S{\isacharprime}\ {\isacharminus}\ {\isacharbraceleft}G{\isacharcomma}H{\isacharbraceright}{\isacharparenright}{\isacharparenright}\ {\isacharequal}\ insert\ F\ {\isacharparenleft}insert\ G\ {\isacharparenleft}insert\ H\ S{\isacharprime}{\isacharparenright}{\isacharparenright}{\isachardoublequoteclose}\isanewline
\ \ \ \ \ \ \ \ \isacommand{by}\isamarkupfalse%
\ this\isanewline
\ \ \ \ \ \ \isacommand{have}\isamarkupfalse%
\ C{\isadigit{1}}{\isacharcolon}{\isachardoublequoteopen}insert\ F\ {\isacharparenleft}insert\ G\ {\isacharparenleft}insert\ H\ S{\isacharprime}{\isacharparenright}{\isacharparenright}\ {\isasymin}\ C{\isachardoublequoteclose}\isanewline
\ \ \ \ \ \ \ \ \isacommand{using}\isamarkupfalse%
\ F{\isadigit{3}}\ \isacommand{by}\isamarkupfalse%
\ {\isacharparenleft}simp\ only{\isacharcolon}\ F{\isadigit{4}}{\isacharparenright}\isanewline
\ \ \ \ \ \ \isacommand{have}\isamarkupfalse%
\ {\isachardoublequoteopen}S{\isacharprime}\ {\isasymsubseteq}\ insert\ F\ S{\isacharprime}{\isachardoublequoteclose}\isanewline
\ \ \ \ \ \ \ \ \isacommand{by}\isamarkupfalse%
\ {\isacharparenleft}rule\ subset{\isacharunderscore}insertI{\isacharparenright}\isanewline
\ \ \ \ \ \ \isacommand{then}\isamarkupfalse%
\ \isacommand{have}\isamarkupfalse%
\ C{\isadigit{2}}{\isacharcolon}{\isachardoublequoteopen}S{\isacharprime}\ {\isasymsubseteq}\ insert\ F\ {\isacharparenleft}insert\ G\ {\isacharparenleft}insert\ H\ S{\isacharprime}{\isacharparenright}{\isacharparenright}{\isachardoublequoteclose}\isanewline
\ \ \ \ \ \ \ \ \isacommand{by}\isamarkupfalse%
\ {\isacharparenleft}simp\ only{\isacharcolon}\ subset{\isacharunderscore}insertI{\isadigit{2}}{\isacharparenright}\isanewline
\ \ \ \ \ \ \isacommand{let}\isamarkupfalse%
\ {\isacharquery}S{\isacharequal}{\isachardoublequoteopen}insert\ F\ {\isacharparenleft}insert\ G\ {\isacharparenleft}insert\ H\ S{\isacharprime}{\isacharparenright}{\isacharparenright}{\isachardoublequoteclose}\isanewline
\ \ \ \ \ \ \isacommand{have}\isamarkupfalse%
\ {\isachardoublequoteopen}{\isasymforall}S\ {\isasymin}\ C{\isachardot}\ {\isasymforall}S{\isacharprime}\ {\isasymsubseteq}\ S{\isachardot}\ S{\isacharprime}\ {\isasymin}\ C{\isachardoublequoteclose}\isanewline
\ \ \ \ \ \ \ \ \isacommand{using}\isamarkupfalse%
\ assms{\isacharparenleft}{\isadigit{2}}{\isacharparenright}\ \isacommand{by}\isamarkupfalse%
\ {\isacharparenleft}simp\ only{\isacharcolon}\ subset{\isacharunderscore}closed{\isacharunderscore}def{\isacharparenright}\isanewline
\ \ \ \ \ \ \isacommand{then}\isamarkupfalse%
\ \isacommand{have}\isamarkupfalse%
\ {\isachardoublequoteopen}{\isasymforall}S{\isacharprime}\ {\isasymsubseteq}\ {\isacharquery}S{\isachardot}\ S{\isacharprime}\ {\isasymin}\ C{\isachardoublequoteclose}\isanewline
\ \ \ \ \ \ \ \ \isacommand{using}\isamarkupfalse%
\ C{\isadigit{1}}\ \isacommand{by}\isamarkupfalse%
\ {\isacharparenleft}rule\ bspec{\isacharparenright}\isanewline
\ \ \ \ \ \ \isacommand{thus}\isamarkupfalse%
\ {\isachardoublequoteopen}S{\isacharprime}\ {\isasymin}\ C{\isachardoublequoteclose}\isanewline
\ \ \ \ \ \ \ \ \isacommand{using}\isamarkupfalse%
\ C{\isadigit{2}}\ \isacommand{by}\isamarkupfalse%
\ {\isacharparenleft}rule\ sspec{\isacharparenright}\isanewline
\ \ \ \ \isacommand{qed}\isamarkupfalse%
\isanewline
\ \ \isacommand{qed}\isamarkupfalse%
\isanewline
\ \ \isacommand{thus}\isamarkupfalse%
\ {\isachardoublequoteopen}{\isacharbraceleft}G{\isacharcomma}H{\isacharbraceright}\ {\isasymunion}\ S\ {\isasymin}\ {\isacharparenleft}extensionFin\ C{\isacharparenright}{\isachardoublequoteclose}\isanewline
\ \ \ \ \isacommand{unfolding}\isamarkupfalse%
\ extensionFin\ \isacommand{by}\isamarkupfalse%
\ {\isacharparenleft}rule\ UnI{\isadigit{2}}{\isacharparenright}\isanewline
\isacommand{qed}\isamarkupfalse%
%
\endisatagproof
{\isafoldproof}%
%
\isadelimproof
%
\endisadelimproof
%
\begin{isamarkuptext}%
Seguidamente, vamos a probar el lema auxiliar \isa{ex{\isadigit{3}}{\isacharunderscore}pcp{\isacharunderscore}SinE{\isacharunderscore}DIS}. Este demuestra que si \isa{C} es 
  una colección con la propiedad de consistencia proposicional y cerrada bajo subconjuntos, \isa{S\ {\isasymin}\ E}
  y sea \isa{F} una fórmula de tipo \isa{{\isasymbeta}} con componentes \isa{{\isasymbeta}\isactrlsub {\isadigit{1}}} y \isa{{\isasymbeta}\isactrlsub {\isadigit{2}}}, se verifica que o bien 
  \isa{{\isacharbraceleft}{\isasymbeta}\isactrlsub {\isadigit{1}}{\isacharbraceright}\ {\isasymunion}\ S\ {\isasymin}\ C{\isacharprime}} o bien \isa{{\isacharbraceleft}{\isasymbeta}\isactrlsub {\isadigit{2}}{\isacharbraceright}\ {\isasymunion}\ S\ {\isasymin}\ C{\isacharprime}}. Dicha prueba se realizará por reducción al absurdo. Para
  ello precisaremos de dos lemas previos que nos permitan llegar a una contradicción: 
  \isa{ex{\isadigit{3}}{\isacharunderscore}pcp{\isacharunderscore}SinE{\isacharunderscore}DIS{\isacharunderscore}auxEx} y \isa{ex{\isadigit{3}}{\isacharunderscore}pcp{\isacharunderscore}SinE{\isacharunderscore}DIS{\isacharunderscore}auxFalse}.

  En primer lugar, veamos la demostración del lema \isa{ex{\isadigit{3}}{\isacharunderscore}pcp{\isacharunderscore}SinE{\isacharunderscore}DIS{\isacharunderscore}auxEx}. Este prueba que dada 
  \isa{C} una colección con la propiedad de consistencia proposicional y cerrada bajo subconjuntos, 
  \isa{S\ {\isasymin}\ E} y sea \isa{F} es una fórmula de tipo \isa{{\isasymbeta}} de componentes \isa{{\isasymbeta}\isactrlsub {\isadigit{1}}} y \isa{{\isasymbeta}\isactrlsub {\isadigit{2}}}, si consideramos \isa{S\isactrlsub {\isadigit{1}}} y 
  \isa{S\isactrlsub {\isadigit{2}}} subconjuntos finitos cualesquiera de \isa{S} tales que \isa{F\ {\isasymin}\ S\isactrlsub {\isadigit{1}}} y\\ \isa{F\ {\isasymin}\ S\isactrlsub {\isadigit{2}}}, entonces existe una 
  fórmula \isa{I\ {\isasymin}\ {\isacharbraceleft}{\isasymbeta}\isactrlsub {\isadigit{1}}{\isacharcomma}{\isasymbeta}\isactrlsub {\isadigit{2}}{\isacharbraceright}} tal que se verifica que tanto \isa{{\isacharbraceleft}I{\isacharbraceright}\ {\isasymunion}\ S\isactrlsub {\isadigit{1}}} como \isa{{\isacharbraceleft}I{\isacharbraceright}\ {\isasymunion}\ S\isactrlsub {\isadigit{2}}} están en \isa{C}.%
\end{isamarkuptext}\isamarkuptrue%
\isacommand{lemma}\isamarkupfalse%
\ ex{\isadigit{3}}{\isacharunderscore}pcp{\isacharunderscore}SinE{\isacharunderscore}DIS{\isacharunderscore}auxEx{\isacharcolon}\isanewline
\ \ \isakeyword{assumes}\ {\isachardoublequoteopen}pcp\ C{\isachardoublequoteclose}\isanewline
\ \ \ \ \ \ \ \ \ \ {\isachardoublequoteopen}subset{\isacharunderscore}closed\ C{\isachardoublequoteclose}\isanewline
\ \ \ \ \ \ \ \ \ \ {\isachardoublequoteopen}S\ {\isasymin}\ {\isacharparenleft}extF\ C{\isacharparenright}{\isachardoublequoteclose}\isanewline
\ \ \ \ \ \ \ \ \ \ {\isachardoublequoteopen}Dis\ F\ G\ H{\isachardoublequoteclose}\isanewline
\ \ \ \ \ \ \ \ \ \ {\isachardoublequoteopen}S{\isadigit{1}}\ {\isasymsubseteq}\ S{\isachardoublequoteclose}\isanewline
\ \ \ \ \ \ \ \ \ \ {\isachardoublequoteopen}finite\ S{\isadigit{1}}{\isachardoublequoteclose}\isanewline
\ \ \ \ \ \ \ \ \ \ {\isachardoublequoteopen}F\ {\isasymin}\ S{\isadigit{1}}{\isachardoublequoteclose}\isanewline
\ \ \ \ \ \ \ \ \ \ {\isachardoublequoteopen}S{\isadigit{2}}\ {\isasymsubseteq}\ S{\isachardoublequoteclose}\isanewline
\ \ \ \ \ \ \ \ \ \ {\isachardoublequoteopen}finite\ S{\isadigit{2}}{\isachardoublequoteclose}\isanewline
\ \ \ \ \ \ \ \ \ \ {\isachardoublequoteopen}F\ {\isasymin}\ S{\isadigit{2}}{\isachardoublequoteclose}\isanewline
\ \ \isakeyword{shows}\ {\isachardoublequoteopen}{\isasymexists}I{\isasymin}{\isacharbraceleft}G{\isacharcomma}H{\isacharbraceright}{\isachardot}\ insert\ I\ S{\isadigit{1}}\ {\isasymin}\ C\ {\isasymand}\ insert\ I\ S{\isadigit{2}}\ {\isasymin}\ C{\isachardoublequoteclose}\ \isanewline
%
\isadelimproof
%
\endisadelimproof
%
\isatagproof
\isacommand{proof}\isamarkupfalse%
\ {\isacharminus}\isanewline
\ \ \isacommand{let}\isamarkupfalse%
\ {\isacharquery}S\ {\isacharequal}\ {\isachardoublequoteopen}S{\isadigit{1}}\ {\isasymunion}\ S{\isadigit{2}}{\isachardoublequoteclose}\isanewline
\ \ \isacommand{have}\isamarkupfalse%
\ {\isachardoublequoteopen}S{\isadigit{1}}\ {\isasymsubseteq}\ {\isacharquery}S{\isachardoublequoteclose}\isanewline
\ \ \ \ \isacommand{by}\isamarkupfalse%
\ {\isacharparenleft}simp\ only{\isacharcolon}\ Un{\isacharunderscore}upper{\isadigit{1}}{\isacharparenright}\isanewline
\ \ \isacommand{have}\isamarkupfalse%
\ {\isachardoublequoteopen}S{\isadigit{2}}\ {\isasymsubseteq}\ {\isacharquery}S{\isachardoublequoteclose}\isanewline
\ \ \ \ \isacommand{by}\isamarkupfalse%
\ {\isacharparenleft}simp\ only{\isacharcolon}\ Un{\isacharunderscore}upper{\isadigit{2}}{\isacharparenright}\isanewline
\ \ \isacommand{have}\isamarkupfalse%
\ {\isachardoublequoteopen}finite\ {\isacharquery}S{\isachardoublequoteclose}\isanewline
\ \ \ \ \isacommand{using}\isamarkupfalse%
\ assms{\isacharparenleft}{\isadigit{6}}{\isacharparenright}\ assms{\isacharparenleft}{\isadigit{9}}{\isacharparenright}\ \isacommand{by}\isamarkupfalse%
\ {\isacharparenleft}rule\ finite{\isacharunderscore}UnI{\isacharparenright}\isanewline
\ \ \isacommand{have}\isamarkupfalse%
\ {\isachardoublequoteopen}{\isacharquery}S\ {\isasymsubseteq}\ S{\isachardoublequoteclose}\ \isanewline
\ \ \ \ \isacommand{using}\isamarkupfalse%
\ assms{\isacharparenleft}{\isadigit{5}}{\isacharparenright}\ assms{\isacharparenleft}{\isadigit{8}}{\isacharparenright}\ \isacommand{by}\isamarkupfalse%
\ {\isacharparenleft}simp\ only{\isacharcolon}\ Un{\isacharunderscore}subset{\isacharunderscore}iff{\isacharparenright}\isanewline
\ \ \isacommand{have}\isamarkupfalse%
\ {\isachardoublequoteopen}{\isasymforall}S{\isacharprime}\ {\isasymsubseteq}\ S{\isachardot}\ finite\ S{\isacharprime}\ {\isasymlongrightarrow}\ S{\isacharprime}\ {\isasymin}\ C{\isachardoublequoteclose}\isanewline
\ \ \ \ \isacommand{using}\isamarkupfalse%
\ assms{\isacharparenleft}{\isadigit{3}}{\isacharparenright}\ \isacommand{unfolding}\isamarkupfalse%
\ extF\ \isacommand{by}\isamarkupfalse%
\ {\isacharparenleft}rule\ CollectD{\isacharparenright}\isanewline
\ \ \isacommand{then}\isamarkupfalse%
\ \isacommand{have}\isamarkupfalse%
\ {\isachardoublequoteopen}finite\ {\isacharquery}S\ {\isasymlongrightarrow}\ {\isacharquery}S\ {\isasymin}\ C{\isachardoublequoteclose}\isanewline
\ \ \ \ \isacommand{using}\isamarkupfalse%
\ {\isacartoucheopen}{\isacharquery}S\ {\isasymsubseteq}\ S{\isacartoucheclose}\ \isacommand{by}\isamarkupfalse%
\ {\isacharparenleft}rule\ sspec{\isacharparenright}\isanewline
\ \ \isacommand{then}\isamarkupfalse%
\ \isacommand{have}\isamarkupfalse%
\ {\isachardoublequoteopen}{\isacharquery}S\ {\isasymin}\ C{\isachardoublequoteclose}\ \isanewline
\ \ \ \ \isacommand{using}\isamarkupfalse%
\ {\isacartoucheopen}finite\ {\isacharquery}S{\isacartoucheclose}\ \isacommand{by}\isamarkupfalse%
\ {\isacharparenleft}rule\ mp{\isacharparenright}\isanewline
\ \ \isacommand{have}\isamarkupfalse%
\ {\isachardoublequoteopen}F\ {\isasymin}\ {\isacharquery}S{\isachardoublequoteclose}\ \isanewline
\ \ \ \ \isacommand{using}\isamarkupfalse%
\ assms{\isacharparenleft}{\isadigit{7}}{\isacharparenright}\ \isacommand{by}\isamarkupfalse%
\ {\isacharparenleft}rule\ UnI{\isadigit{1}}{\isacharparenright}\isanewline
\ \ \isacommand{have}\isamarkupfalse%
\ {\isachardoublequoteopen}{\isasymforall}S\ {\isasymin}\ C{\isachardot}\ {\isasymbottom}\ {\isasymnotin}\ S\isanewline
\ \ {\isasymand}\ {\isacharparenleft}{\isasymforall}k{\isachardot}\ Atom\ k\ {\isasymin}\ S\ {\isasymlongrightarrow}\ \isactrlbold {\isasymnot}\ {\isacharparenleft}Atom\ k{\isacharparenright}\ {\isasymin}\ S\ {\isasymlongrightarrow}\ False{\isacharparenright}\isanewline
\ \ {\isasymand}\ {\isacharparenleft}{\isasymforall}F\ G\ H{\isachardot}\ Con\ F\ G\ H\ {\isasymlongrightarrow}\ F\ {\isasymin}\ S\ {\isasymlongrightarrow}\ {\isacharbraceleft}G{\isacharcomma}H{\isacharbraceright}\ {\isasymunion}\ S\ {\isasymin}\ C{\isacharparenright}\isanewline
\ \ {\isasymand}\ {\isacharparenleft}{\isasymforall}F\ G\ H{\isachardot}\ Dis\ F\ G\ H\ {\isasymlongrightarrow}\ F\ {\isasymin}\ S\ {\isasymlongrightarrow}\ {\isacharbraceleft}G{\isacharbraceright}\ {\isasymunion}\ S\ {\isasymin}\ C\ {\isasymor}\ {\isacharbraceleft}H{\isacharbraceright}\ {\isasymunion}\ S\ {\isasymin}\ C{\isacharparenright}{\isachardoublequoteclose}\isanewline
\ \ \ \ \isacommand{using}\isamarkupfalse%
\ assms{\isacharparenleft}{\isadigit{1}}{\isacharparenright}\ \isacommand{by}\isamarkupfalse%
\ {\isacharparenleft}rule\ pcp{\isacharunderscore}alt{\isadigit{1}}{\isacharparenright}\isanewline
\ \ \isacommand{then}\isamarkupfalse%
\ \isacommand{have}\isamarkupfalse%
\ {\isachardoublequoteopen}{\isasymbottom}\ {\isasymnotin}\ {\isacharquery}S\isanewline
\ \ \ \ \ \ \ \ {\isasymand}\ {\isacharparenleft}{\isasymforall}k{\isachardot}\ Atom\ k\ {\isasymin}\ {\isacharquery}S\ {\isasymlongrightarrow}\ \isactrlbold {\isasymnot}\ {\isacharparenleft}Atom\ k{\isacharparenright}\ {\isasymin}\ {\isacharquery}S\ {\isasymlongrightarrow}\ False{\isacharparenright}\isanewline
\ \ \ \ \ \ \ \ {\isasymand}\ {\isacharparenleft}{\isasymforall}F\ G\ H{\isachardot}\ Con\ F\ G\ H\ {\isasymlongrightarrow}\ F\ {\isasymin}\ {\isacharquery}S\ {\isasymlongrightarrow}\ {\isacharbraceleft}G{\isacharcomma}H{\isacharbraceright}\ {\isasymunion}\ {\isacharquery}S\ {\isasymin}\ C{\isacharparenright}\isanewline
\ \ \ \ \ \ \ \ {\isasymand}\ {\isacharparenleft}{\isasymforall}F\ G\ H{\isachardot}\ Dis\ F\ G\ H\ {\isasymlongrightarrow}\ F\ {\isasymin}\ {\isacharquery}S\ {\isasymlongrightarrow}\ {\isacharbraceleft}G{\isacharbraceright}\ {\isasymunion}\ {\isacharquery}S\ {\isasymin}\ C\ {\isasymor}\ {\isacharbraceleft}H{\isacharbraceright}\ {\isasymunion}\ {\isacharquery}S\ {\isasymin}\ C{\isacharparenright}{\isachardoublequoteclose}\isanewline
\ \ \ \ \isacommand{using}\isamarkupfalse%
\ {\isacartoucheopen}{\isacharquery}S\ {\isasymin}\ C{\isacartoucheclose}\ \isacommand{by}\isamarkupfalse%
\ {\isacharparenleft}rule\ bspec{\isacharparenright}\isanewline
\ \ \isacommand{then}\isamarkupfalse%
\ \isacommand{have}\isamarkupfalse%
\ {\isachardoublequoteopen}{\isasymforall}F\ G\ H{\isachardot}\ Dis\ F\ G\ H\ {\isasymlongrightarrow}\ F\ {\isasymin}\ {\isacharquery}S\ {\isasymlongrightarrow}\ {\isacharbraceleft}G{\isacharbraceright}\ {\isasymunion}\ {\isacharquery}S\ {\isasymin}\ C\ {\isasymor}\ {\isacharbraceleft}H{\isacharbraceright}\ {\isasymunion}\ {\isacharquery}S\ {\isasymin}\ C{\isachardoublequoteclose}\isanewline
\ \ \ \ \isacommand{by}\isamarkupfalse%
\ {\isacharparenleft}iprover\ elim{\isacharcolon}\ conjunct{\isadigit{2}}{\isacharparenright}\isanewline
\ \ \isacommand{then}\isamarkupfalse%
\ \isacommand{have}\isamarkupfalse%
\ {\isachardoublequoteopen}Dis\ F\ G\ H\ {\isasymlongrightarrow}\ F\ {\isasymin}\ {\isacharquery}S\ {\isasymlongrightarrow}\ {\isacharbraceleft}G{\isacharbraceright}\ {\isasymunion}\ {\isacharquery}S\ {\isasymin}\ C\ {\isasymor}\ {\isacharbraceleft}H{\isacharbraceright}\ {\isasymunion}\ {\isacharquery}S\ {\isasymin}\ C{\isachardoublequoteclose}\isanewline
\ \ \ \ \isacommand{by}\isamarkupfalse%
\ {\isacharparenleft}iprover\ elim{\isacharcolon}\ allE{\isacharparenright}\isanewline
\ \ \isacommand{then}\isamarkupfalse%
\ \isacommand{have}\isamarkupfalse%
\ {\isachardoublequoteopen}F\ {\isasymin}\ {\isacharquery}S\ {\isasymlongrightarrow}\ {\isacharbraceleft}G{\isacharbraceright}\ {\isasymunion}\ {\isacharquery}S\ {\isasymin}\ C\ {\isasymor}\ {\isacharbraceleft}H{\isacharbraceright}\ {\isasymunion}\ {\isacharquery}S\ {\isasymin}\ C{\isachardoublequoteclose}\isanewline
\ \ \ \ \isacommand{using}\isamarkupfalse%
\ assms{\isacharparenleft}{\isadigit{4}}{\isacharparenright}\ \isacommand{by}\isamarkupfalse%
\ {\isacharparenleft}rule\ mp{\isacharparenright}\isanewline
\ \ \isacommand{then}\isamarkupfalse%
\ \isacommand{have}\isamarkupfalse%
\ insIsUn{\isacharcolon}{\isachardoublequoteopen}{\isacharbraceleft}G{\isacharbraceright}\ {\isasymunion}\ {\isacharquery}S\ {\isasymin}\ C\ {\isasymor}\ {\isacharbraceleft}H{\isacharbraceright}\ {\isasymunion}\ {\isacharquery}S\ {\isasymin}\ C{\isachardoublequoteclose}\isanewline
\ \ \ \ \isacommand{using}\isamarkupfalse%
\ {\isacartoucheopen}F\ {\isasymin}\ {\isacharquery}S{\isacartoucheclose}\ \isacommand{by}\isamarkupfalse%
\ {\isacharparenleft}rule\ mp{\isacharparenright}\isanewline
\ \ \isacommand{have}\isamarkupfalse%
\ insG{\isacharcolon}{\isachardoublequoteopen}insert\ G\ {\isacharquery}S\ {\isacharequal}\ {\isacharbraceleft}G{\isacharbraceright}\ {\isasymunion}\ {\isacharquery}S{\isachardoublequoteclose}\ \isanewline
\ \ \ \ \isacommand{by}\isamarkupfalse%
\ {\isacharparenleft}rule\ insert{\isacharunderscore}is{\isacharunderscore}Un{\isacharparenright}\isanewline
\ \ \isacommand{have}\isamarkupfalse%
\ insH{\isacharcolon}{\isachardoublequoteopen}insert\ H\ {\isacharquery}S\ {\isacharequal}\ {\isacharbraceleft}H{\isacharbraceright}\ {\isasymunion}\ {\isacharquery}S{\isachardoublequoteclose}\isanewline
\ \ \ \ \isacommand{by}\isamarkupfalse%
\ {\isacharparenleft}rule\ insert{\isacharunderscore}is{\isacharunderscore}Un{\isacharparenright}\isanewline
\ \ \isacommand{have}\isamarkupfalse%
\ {\isachardoublequoteopen}insert\ G\ {\isacharquery}S\ {\isasymin}\ C\ {\isasymor}\ insert\ H\ {\isacharquery}S\ {\isasymin}\ C{\isachardoublequoteclose}\isanewline
\ \ \ \ \isacommand{using}\isamarkupfalse%
\ insG\ insH\ \isacommand{by}\isamarkupfalse%
\ {\isacharparenleft}simp\ only{\isacharcolon}\ insIsUn{\isacharparenright}\isanewline
\ \ \isacommand{then}\isamarkupfalse%
\ \isacommand{have}\isamarkupfalse%
\ {\isachardoublequoteopen}{\isacharparenleft}insert\ G\ {\isacharquery}S\ {\isasymin}\ C\ {\isasymor}\ insert\ H\ {\isacharquery}S\ {\isasymin}\ C{\isacharparenright}\ {\isasymor}\ {\isacharparenleft}{\isasymexists}I\ {\isasymin}\ {\isacharbraceleft}{\isacharbraceright}{\isachardot}\ insert\ I\ {\isacharquery}S\ {\isasymin}\ C{\isacharparenright}{\isachardoublequoteclose}\isanewline
\ \ \ \ \isacommand{by}\isamarkupfalse%
\ {\isacharparenleft}simp\ only{\isacharcolon}\ disjI{\isadigit{1}}{\isacharparenright}\isanewline
\ \ \isacommand{then}\isamarkupfalse%
\ \isacommand{have}\isamarkupfalse%
\ {\isachardoublequoteopen}insert\ G\ {\isacharquery}S\ {\isasymin}\ C\ {\isasymor}\ {\isacharparenleft}insert\ H\ {\isacharquery}S\ {\isasymin}\ C\ {\isasymor}\ {\isacharparenleft}{\isasymexists}I\ {\isasymin}\ {\isacharbraceleft}{\isacharbraceright}{\isachardot}\ insert\ I\ {\isacharquery}S\ {\isasymin}\ C{\isacharparenright}{\isacharparenright}{\isachardoublequoteclose}\isanewline
\ \ \ \ \isacommand{by}\isamarkupfalse%
\ {\isacharparenleft}simp\ only{\isacharcolon}\ disj{\isacharunderscore}assoc{\isacharparenright}\isanewline
\ \ \isacommand{then}\isamarkupfalse%
\ \isacommand{have}\isamarkupfalse%
\ {\isachardoublequoteopen}insert\ G\ {\isacharquery}S\ {\isasymin}\ C\ {\isasymor}\ {\isacharparenleft}{\isasymexists}I\ {\isasymin}\ {\isacharbraceleft}H{\isacharbraceright}{\isachardot}\ insert\ I\ {\isacharquery}S\ {\isasymin}\ C{\isacharparenright}{\isachardoublequoteclose}\isanewline
\ \ \ \ \isacommand{by}\isamarkupfalse%
\ {\isacharparenleft}simp\ only{\isacharcolon}\ bex{\isacharunderscore}simps{\isacharparenleft}{\isadigit{5}}{\isacharparenright}{\isacharparenright}\isanewline
\ \ \isacommand{then}\isamarkupfalse%
\ \isacommand{have}\isamarkupfalse%
\ {\isadigit{1}}{\isacharcolon}{\isachardoublequoteopen}{\isasymexists}I\ {\isasymin}\ {\isacharbraceleft}G{\isacharcomma}H{\isacharbraceright}{\isachardot}\ insert\ I\ {\isacharquery}S\ {\isasymin}\ C{\isachardoublequoteclose}\ \isanewline
\ \ \ \ \isacommand{by}\isamarkupfalse%
\ {\isacharparenleft}simp\ only{\isacharcolon}\ bex{\isacharunderscore}simps{\isacharparenleft}{\isadigit{5}}{\isacharparenright}{\isacharparenright}\isanewline
\ \ \isacommand{obtain}\isamarkupfalse%
\ I\ \isakeyword{where}\ {\isachardoublequoteopen}I\ {\isasymin}\ {\isacharbraceleft}G{\isacharcomma}H{\isacharbraceright}{\isachardoublequoteclose}\ \isakeyword{and}\ {\isachardoublequoteopen}insert\ I\ {\isacharquery}S\ {\isasymin}\ C{\isachardoublequoteclose}\isanewline
\ \ \ \ \isacommand{using}\isamarkupfalse%
\ {\isadigit{1}}\ \isacommand{by}\isamarkupfalse%
\ {\isacharparenleft}rule\ bexE{\isacharparenright}\isanewline
\ \ \isacommand{have}\isamarkupfalse%
\ SC{\isacharcolon}{\isachardoublequoteopen}{\isasymforall}S\ {\isasymin}\ C{\isachardot}\ {\isasymforall}S{\isacharprime}{\isasymsubseteq}S{\isachardot}\ S{\isacharprime}\ {\isasymin}\ C{\isachardoublequoteclose}\isanewline
\ \ \ \ \isacommand{using}\isamarkupfalse%
\ assms{\isacharparenleft}{\isadigit{2}}{\isacharparenright}\ \isacommand{by}\isamarkupfalse%
\ {\isacharparenleft}simp\ only{\isacharcolon}\ subset{\isacharunderscore}closed{\isacharunderscore}def{\isacharparenright}\isanewline
\ \ \isacommand{then}\isamarkupfalse%
\ \isacommand{have}\isamarkupfalse%
\ {\isadigit{2}}{\isacharcolon}{\isachardoublequoteopen}{\isasymforall}S{\isacharprime}\ {\isasymsubseteq}\ {\isacharparenleft}insert\ I\ {\isacharquery}S{\isacharparenright}{\isachardot}\ S{\isacharprime}\ {\isasymin}\ C{\isachardoublequoteclose}\isanewline
\ \ \ \ \isacommand{using}\isamarkupfalse%
\ {\isacartoucheopen}insert\ I\ {\isacharquery}S\ {\isasymin}\ C{\isacartoucheclose}\ \isacommand{by}\isamarkupfalse%
\ {\isacharparenleft}rule\ bspec{\isacharparenright}\isanewline
\ \ \isacommand{have}\isamarkupfalse%
\ {\isachardoublequoteopen}insert\ I\ S{\isadigit{1}}\ {\isasymsubseteq}\ insert\ I\ {\isacharquery}S{\isachardoublequoteclose}\ \isanewline
\ \ \ \ \isacommand{using}\isamarkupfalse%
\ {\isacartoucheopen}S{\isadigit{1}}\ {\isasymsubseteq}\ {\isacharquery}S{\isacartoucheclose}\ \isacommand{by}\isamarkupfalse%
\ {\isacharparenleft}rule\ insert{\isacharunderscore}mono{\isacharparenright}\isanewline
\ \ \isacommand{have}\isamarkupfalse%
\ {\isachardoublequoteopen}insert\ I\ S{\isadigit{1}}\ {\isasymin}\ C{\isachardoublequoteclose}\isanewline
\ \ \ \ \isacommand{using}\isamarkupfalse%
\ {\isadigit{2}}\ {\isacartoucheopen}insert\ I\ S{\isadigit{1}}\ {\isasymsubseteq}\ insert\ I\ {\isacharquery}S{\isacartoucheclose}\ \isacommand{by}\isamarkupfalse%
\ {\isacharparenleft}rule\ sspec{\isacharparenright}\isanewline
\ \ \isacommand{have}\isamarkupfalse%
\ {\isachardoublequoteopen}insert\ I\ S{\isadigit{2}}\ {\isasymsubseteq}\ insert\ I\ {\isacharquery}S{\isachardoublequoteclose}\isanewline
\ \ \ \ \isacommand{using}\isamarkupfalse%
\ {\isacartoucheopen}S{\isadigit{2}}\ {\isasymsubseteq}\ {\isacharquery}S{\isacartoucheclose}\ \isacommand{by}\isamarkupfalse%
\ {\isacharparenleft}rule\ insert{\isacharunderscore}mono{\isacharparenright}\isanewline
\ \ \isacommand{have}\isamarkupfalse%
\ {\isachardoublequoteopen}insert\ I\ S{\isadigit{2}}\ {\isasymin}\ C{\isachardoublequoteclose}\isanewline
\ \ \ \ \isacommand{using}\isamarkupfalse%
\ {\isadigit{2}}\ {\isacartoucheopen}insert\ I\ S{\isadigit{2}}\ {\isasymsubseteq}\ insert\ I\ {\isacharquery}S{\isacartoucheclose}\ \isacommand{by}\isamarkupfalse%
\ {\isacharparenleft}rule\ sspec{\isacharparenright}\isanewline
\ \ \isacommand{have}\isamarkupfalse%
\ {\isachardoublequoteopen}insert\ I\ S{\isadigit{1}}\ {\isasymin}\ C\ {\isasymand}\ insert\ I\ S{\isadigit{2}}\ {\isasymin}\ C{\isachardoublequoteclose}\isanewline
\ \ \ \ \isacommand{using}\isamarkupfalse%
\ {\isacartoucheopen}insert\ I\ S{\isadigit{1}}\ {\isasymin}\ C{\isacartoucheclose}\ {\isacartoucheopen}insert\ I\ S{\isadigit{2}}\ {\isasymin}\ C{\isacartoucheclose}\ \isacommand{by}\isamarkupfalse%
\ {\isacharparenleft}rule\ conjI{\isacharparenright}\isanewline
\ \ \isacommand{thus}\isamarkupfalse%
\ {\isachardoublequoteopen}{\isasymexists}I{\isasymin}{\isacharbraceleft}G{\isacharcomma}H{\isacharbraceright}{\isachardot}\ insert\ I\ S{\isadigit{1}}\ {\isasymin}\ C\ {\isasymand}\ insert\ I\ S{\isadigit{2}}\ {\isasymin}\ C{\isachardoublequoteclose}\isanewline
\ \ \ \ \isacommand{using}\isamarkupfalse%
\ {\isacartoucheopen}I\ {\isasymin}\ {\isacharbraceleft}G{\isacharcomma}H{\isacharbraceright}{\isacartoucheclose}\ \isacommand{by}\isamarkupfalse%
\ {\isacharparenleft}rule\ bexI{\isacharparenright}\isanewline
\isacommand{qed}\isamarkupfalse%
%
\endisatagproof
{\isafoldproof}%
%
\isadelimproof
%
\endisadelimproof
%
\begin{isamarkuptext}%
Finalmente, el lema \isa{ex{\isadigit{3}}{\isacharunderscore}pcp{\isacharunderscore}SinE{\isacharunderscore}DIS{\isacharunderscore}auxFalse} prueba que dada una colección \isa{C} con la 
  propiedad de consistencia proposicional y cerrada bajo subconjuntos, \isa{S\ {\isasymin}\ E} y sea \isa{F} es una 
  fórmula de tipo \isa{{\isasymbeta}} de componentes \isa{{\isasymbeta}\isactrlsub {\isadigit{1}}} y \isa{{\isasymbeta}\isactrlsub {\isadigit{2}}}, si consideramos \isa{S\isactrlsub {\isadigit{1}}} y \isa{S\isactrlsub {\isadigit{2}}} subconjuntos finitos 
  cualesquiera de \isa{S} tales que \isa{F\ {\isasymin}\ S\isactrlsub {\isadigit{1}}}, \isa{F\ {\isasymin}\ S\isactrlsub {\isadigit{2}}}, \isa{{\isacharbraceleft}{\isasymbeta}\isactrlsub {\isadigit{1}}{\isacharbraceright}\ {\isasymunion}\ S\isactrlsub {\isadigit{1}}\ {\isasymnotin}\ C} y \isa{{\isacharbraceleft}{\isasymbeta}\isactrlsub {\isadigit{2}}{\isacharbraceright}\ {\isasymunion}\ S\isactrlsub {\isadigit{2}}\ {\isasymnotin}\ C}, llegamos a 
  una contradicción.%
\end{isamarkuptext}\isamarkuptrue%
\isacommand{lemma}\isamarkupfalse%
\ ex{\isadigit{3}}{\isacharunderscore}pcp{\isacharunderscore}SinE{\isacharunderscore}DIS{\isacharunderscore}auxFalse{\isacharcolon}\isanewline
\ \ \isakeyword{assumes}\ {\isachardoublequoteopen}pcp\ C{\isachardoublequoteclose}\ \isanewline
\ \ \ \ \ \ \ \ \ \ {\isachardoublequoteopen}subset{\isacharunderscore}closed\ C{\isachardoublequoteclose}\isanewline
\ \ \ \ \ \ \ \ \ \ {\isachardoublequoteopen}S\ {\isasymin}\ {\isacharparenleft}extF\ C{\isacharparenright}{\isachardoublequoteclose}\isanewline
\ \ \ \ \ \ \ \ \ \ {\isachardoublequoteopen}Dis\ F\ G\ H{\isachardoublequoteclose}\isanewline
\ \ \ \ \ \ \ \ \ \ {\isachardoublequoteopen}F\ {\isasymin}\ S{\isachardoublequoteclose}\isanewline
\ \ \ \ \ \ \ \ \ \ {\isachardoublequoteopen}S{\isadigit{1}}\ {\isasymsubseteq}\ S{\isachardoublequoteclose}\ \isanewline
\ \ \ \ \ \ \ \ \ \ {\isachardoublequoteopen}finite\ S{\isadigit{1}}{\isachardoublequoteclose}\ \isanewline
\ \ \ \ \ \ \ \ \ \ {\isachardoublequoteopen}insert\ G\ S{\isadigit{1}}\ {\isasymnotin}\ C{\isachardoublequoteclose}\ \isanewline
\ \ \ \ \ \ \ \ \ \ {\isachardoublequoteopen}S{\isadigit{2}}\ {\isasymsubseteq}\ S{\isachardoublequoteclose}\ \isanewline
\ \ \ \ \ \ \ \ \ \ {\isachardoublequoteopen}finite\ S{\isadigit{2}}{\isachardoublequoteclose}\ \isanewline
\ \ \ \ \ \ \ \ \ \ {\isachardoublequoteopen}insert\ H\ S{\isadigit{2}}\ {\isasymnotin}\ C{\isachardoublequoteclose}\isanewline
\ \ \ \ \ \ \ \ \isakeyword{shows}\ {\isachardoublequoteopen}False{\isachardoublequoteclose}\isanewline
%
\isadelimproof
%
\endisadelimproof
%
\isatagproof
\isacommand{proof}\isamarkupfalse%
\ {\isacharminus}\isanewline
\ \ \isacommand{let}\isamarkupfalse%
\ {\isacharquery}S{\isadigit{1}}{\isacharequal}{\isachardoublequoteopen}insert\ F\ S{\isadigit{1}}{\isachardoublequoteclose}\isanewline
\ \ \isacommand{let}\isamarkupfalse%
\ {\isacharquery}S{\isadigit{2}}{\isacharequal}{\isachardoublequoteopen}insert\ F\ S{\isadigit{2}}{\isachardoublequoteclose}\isanewline
\ \ \isacommand{have}\isamarkupfalse%
\ SC{\isacharcolon}{\isachardoublequoteopen}{\isasymforall}S\ {\isasymin}\ C{\isachardot}\ {\isasymforall}S{\isacharprime}{\isasymsubseteq}S{\isachardot}\ S{\isacharprime}\ {\isasymin}\ C{\isachardoublequoteclose}\isanewline
\ \ \ \ \isacommand{using}\isamarkupfalse%
\ assms{\isacharparenleft}{\isadigit{2}}{\isacharparenright}\ \isacommand{by}\isamarkupfalse%
\ {\isacharparenleft}simp\ only{\isacharcolon}\ subset{\isacharunderscore}closed{\isacharunderscore}def{\isacharparenright}\isanewline
\ \ \isacommand{have}\isamarkupfalse%
\ {\isadigit{1}}{\isacharcolon}{\isachardoublequoteopen}{\isacharquery}S{\isadigit{1}}\ {\isasymsubseteq}\ S{\isachardoublequoteclose}\isanewline
\ \ \ \ \isacommand{using}\isamarkupfalse%
\ {\isacartoucheopen}F\ {\isasymin}\ S{\isacartoucheclose}\ {\isacartoucheopen}S{\isadigit{1}}\ {\isasymsubseteq}\ S{\isacartoucheclose}\ \isacommand{by}\isamarkupfalse%
\ {\isacharparenleft}simp\ only{\isacharcolon}\ insert{\isacharunderscore}subset{\isacharparenright}\ \isanewline
\ \ \isacommand{have}\isamarkupfalse%
\ {\isadigit{2}}{\isacharcolon}{\isachardoublequoteopen}finite\ {\isacharquery}S{\isadigit{1}}{\isachardoublequoteclose}\isanewline
\ \ \ \ \isacommand{using}\isamarkupfalse%
\ {\isacartoucheopen}finite\ S{\isadigit{1}}{\isacartoucheclose}\ \isacommand{by}\isamarkupfalse%
\ {\isacharparenleft}simp\ only{\isacharcolon}\ finite{\isacharunderscore}insert{\isacharparenright}\ \isanewline
\ \ \isacommand{have}\isamarkupfalse%
\ {\isadigit{3}}{\isacharcolon}{\isachardoublequoteopen}F\ {\isasymin}\ {\isacharquery}S{\isadigit{1}}{\isachardoublequoteclose}\isanewline
\ \ \ \ \isacommand{by}\isamarkupfalse%
\ {\isacharparenleft}simp\ only{\isacharcolon}\ insertI{\isadigit{1}}{\isacharparenright}\ \isanewline
\ \ \isacommand{have}\isamarkupfalse%
\ {\isadigit{4}}{\isacharcolon}{\isachardoublequoteopen}insert\ G\ {\isacharquery}S{\isadigit{1}}\ {\isasymnotin}\ C{\isachardoublequoteclose}\ \isanewline
\ \ \isacommand{proof}\isamarkupfalse%
\ {\isacharparenleft}rule\ ccontr{\isacharparenright}\isanewline
\ \ \ \ \isacommand{assume}\isamarkupfalse%
\ {\isachardoublequoteopen}{\isasymnot}{\isacharparenleft}insert\ G\ {\isacharquery}S{\isadigit{1}}\ {\isasymnotin}\ C{\isacharparenright}{\isachardoublequoteclose}\isanewline
\ \ \ \ \isacommand{then}\isamarkupfalse%
\ \isacommand{have}\isamarkupfalse%
\ {\isachardoublequoteopen}insert\ G\ {\isacharquery}S{\isadigit{1}}\ {\isasymin}\ C{\isachardoublequoteclose}\isanewline
\ \ \ \ \ \ \isacommand{by}\isamarkupfalse%
\ {\isacharparenleft}rule\ notnotD{\isacharparenright}\isanewline
\ \ \ \ \isacommand{have}\isamarkupfalse%
\ SC{\isadigit{1}}{\isacharcolon}{\isachardoublequoteopen}{\isasymforall}S{\isacharprime}\ {\isasymsubseteq}\ {\isacharparenleft}insert\ G\ {\isacharquery}S{\isadigit{1}}{\isacharparenright}{\isachardot}\ S{\isacharprime}\ {\isasymin}\ C{\isachardoublequoteclose}\isanewline
\ \ \ \ \ \ \isacommand{using}\isamarkupfalse%
\ SC\ {\isacartoucheopen}insert\ G\ {\isacharquery}S{\isadigit{1}}\ {\isasymin}\ C{\isacartoucheclose}\ \isacommand{by}\isamarkupfalse%
\ {\isacharparenleft}rule\ bspec{\isacharparenright}\isanewline
\ \ \ \ \isacommand{have}\isamarkupfalse%
\ {\isachardoublequoteopen}insert\ G\ S{\isadigit{1}}\ {\isasymsubseteq}\ insert\ F\ {\isacharparenleft}insert\ G\ S{\isadigit{1}}{\isacharparenright}{\isachardoublequoteclose}\isanewline
\ \ \ \ \ \ \isacommand{by}\isamarkupfalse%
\ {\isacharparenleft}rule\ subset{\isacharunderscore}insertI{\isacharparenright}\isanewline
\ \ \ \ \isacommand{then}\isamarkupfalse%
\ \isacommand{have}\isamarkupfalse%
\ {\isachardoublequoteopen}insert\ G\ S{\isadigit{1}}\ {\isasymsubseteq}\ insert\ G\ {\isacharquery}S{\isadigit{1}}{\isachardoublequoteclose}\isanewline
\ \ \ \ \ \ \isacommand{by}\isamarkupfalse%
\ {\isacharparenleft}simp\ only{\isacharcolon}\ insert{\isacharunderscore}commute{\isacharparenright}\isanewline
\ \ \ \ \isacommand{have}\isamarkupfalse%
\ {\isachardoublequoteopen}insert\ G\ S{\isadigit{1}}\ {\isasymin}\ C{\isachardoublequoteclose}\isanewline
\ \ \ \ \ \ \isacommand{using}\isamarkupfalse%
\ SC{\isadigit{1}}\ {\isacartoucheopen}insert\ G\ S{\isadigit{1}}\ {\isasymsubseteq}\ insert\ G\ {\isacharquery}S{\isadigit{1}}{\isacartoucheclose}\ \isacommand{by}\isamarkupfalse%
\ {\isacharparenleft}rule\ sspec{\isacharparenright}\isanewline
\ \ \ \ \isacommand{show}\isamarkupfalse%
\ {\isachardoublequoteopen}False{\isachardoublequoteclose}\isanewline
\ \ \ \ \ \ \isacommand{using}\isamarkupfalse%
\ assms{\isacharparenleft}{\isadigit{8}}{\isacharparenright}\ {\isacartoucheopen}insert\ G\ S{\isadigit{1}}\ {\isasymin}\ C{\isacartoucheclose}\ \isacommand{by}\isamarkupfalse%
\ {\isacharparenleft}rule\ notE{\isacharparenright}\isanewline
\ \ \isacommand{qed}\isamarkupfalse%
\isanewline
\ \ \isacommand{have}\isamarkupfalse%
\ {\isadigit{5}}{\isacharcolon}{\isachardoublequoteopen}{\isacharquery}S{\isadigit{2}}\ {\isasymsubseteq}\ S{\isachardoublequoteclose}\isanewline
\ \ \ \ \isacommand{using}\isamarkupfalse%
\ {\isacartoucheopen}F\ {\isasymin}\ S{\isacartoucheclose}\ {\isacartoucheopen}S{\isadigit{2}}\ {\isasymsubseteq}\ S{\isacartoucheclose}\ \isacommand{by}\isamarkupfalse%
\ {\isacharparenleft}simp\ only{\isacharcolon}\ insert{\isacharunderscore}subset{\isacharparenright}\isanewline
\ \ \isacommand{have}\isamarkupfalse%
\ {\isadigit{6}}{\isacharcolon}{\isachardoublequoteopen}finite\ {\isacharquery}S{\isadigit{2}}{\isachardoublequoteclose}\isanewline
\ \ \ \ \isacommand{using}\isamarkupfalse%
\ {\isacartoucheopen}finite\ S{\isadigit{2}}{\isacartoucheclose}\ \isacommand{by}\isamarkupfalse%
\ {\isacharparenleft}simp\ only{\isacharcolon}\ finite{\isacharunderscore}insert{\isacharparenright}\isanewline
\ \ \isacommand{have}\isamarkupfalse%
\ {\isadigit{7}}{\isacharcolon}{\isachardoublequoteopen}F\ {\isasymin}\ {\isacharquery}S{\isadigit{2}}{\isachardoublequoteclose}\isanewline
\ \ \ \ \isacommand{by}\isamarkupfalse%
\ {\isacharparenleft}simp\ only{\isacharcolon}\ insertI{\isadigit{1}}{\isacharparenright}\isanewline
\ \ \isacommand{have}\isamarkupfalse%
\ {\isadigit{8}}{\isacharcolon}{\isachardoublequoteopen}insert\ H\ {\isacharquery}S{\isadigit{2}}\ {\isasymnotin}\ C{\isachardoublequoteclose}\ \isanewline
\ \ \isacommand{proof}\isamarkupfalse%
\ {\isacharparenleft}rule\ ccontr{\isacharparenright}\isanewline
\ \ \ \ \isacommand{assume}\isamarkupfalse%
\ {\isachardoublequoteopen}{\isasymnot}{\isacharparenleft}insert\ H\ {\isacharquery}S{\isadigit{2}}\ {\isasymnotin}\ C{\isacharparenright}{\isachardoublequoteclose}\isanewline
\ \ \ \ \isacommand{then}\isamarkupfalse%
\ \isacommand{have}\isamarkupfalse%
\ {\isachardoublequoteopen}insert\ H\ {\isacharquery}S{\isadigit{2}}\ {\isasymin}\ C{\isachardoublequoteclose}\isanewline
\ \ \ \ \ \ \isacommand{by}\isamarkupfalse%
\ {\isacharparenleft}rule\ notnotD{\isacharparenright}\isanewline
\ \ \ \ \isacommand{have}\isamarkupfalse%
\ SC{\isadigit{2}}{\isacharcolon}{\isachardoublequoteopen}{\isasymforall}S{\isacharprime}\ {\isasymsubseteq}\ {\isacharparenleft}insert\ H\ {\isacharquery}S{\isadigit{2}}{\isacharparenright}{\isachardot}\ S{\isacharprime}\ {\isasymin}\ C{\isachardoublequoteclose}\isanewline
\ \ \ \ \ \ \isacommand{using}\isamarkupfalse%
\ SC\ {\isacartoucheopen}insert\ H\ {\isacharquery}S{\isadigit{2}}\ {\isasymin}\ C{\isacartoucheclose}\ \isacommand{by}\isamarkupfalse%
\ {\isacharparenleft}rule\ bspec{\isacharparenright}\isanewline
\ \ \ \ \isacommand{have}\isamarkupfalse%
\ {\isachardoublequoteopen}insert\ H\ S{\isadigit{2}}\ {\isasymsubseteq}\ insert\ F\ {\isacharparenleft}insert\ H\ S{\isadigit{2}}{\isacharparenright}{\isachardoublequoteclose}\isanewline
\ \ \ \ \ \ \isacommand{by}\isamarkupfalse%
\ {\isacharparenleft}rule\ subset{\isacharunderscore}insertI{\isacharparenright}\isanewline
\ \ \ \ \isacommand{then}\isamarkupfalse%
\ \isacommand{have}\isamarkupfalse%
\ {\isachardoublequoteopen}insert\ H\ S{\isadigit{2}}\ {\isasymsubseteq}\ insert\ H\ {\isacharquery}S{\isadigit{2}}{\isachardoublequoteclose}\isanewline
\ \ \ \ \ \ \isacommand{by}\isamarkupfalse%
\ {\isacharparenleft}simp\ only{\isacharcolon}\ insert{\isacharunderscore}commute{\isacharparenright}\isanewline
\ \ \ \ \isacommand{have}\isamarkupfalse%
\ {\isachardoublequoteopen}insert\ H\ S{\isadigit{2}}\ {\isasymin}\ C{\isachardoublequoteclose}\isanewline
\ \ \ \ \ \ \isacommand{using}\isamarkupfalse%
\ SC{\isadigit{2}}\ {\isacartoucheopen}insert\ H\ S{\isadigit{2}}\ {\isasymsubseteq}\ insert\ H\ {\isacharquery}S{\isadigit{2}}{\isacartoucheclose}\ \isacommand{by}\isamarkupfalse%
\ {\isacharparenleft}rule\ sspec{\isacharparenright}\isanewline
\ \ \ \ \isacommand{show}\isamarkupfalse%
\ {\isachardoublequoteopen}False{\isachardoublequoteclose}\isanewline
\ \ \ \ \ \ \isacommand{using}\isamarkupfalse%
\ assms{\isacharparenleft}{\isadigit{1}}{\isadigit{1}}{\isacharparenright}\ {\isacartoucheopen}insert\ H\ S{\isadigit{2}}\ {\isasymin}\ C{\isacartoucheclose}\ \isacommand{by}\isamarkupfalse%
\ {\isacharparenleft}rule\ notE{\isacharparenright}\isanewline
\ \ \isacommand{qed}\isamarkupfalse%
\isanewline
\ \ \isacommand{have}\isamarkupfalse%
\ Ex{\isacharcolon}{\isachardoublequoteopen}{\isasymexists}I\ {\isasymin}\ {\isacharbraceleft}G{\isacharcomma}H{\isacharbraceright}{\isachardot}\ insert\ I\ {\isacharquery}S{\isadigit{1}}\ {\isasymin}\ C\ {\isasymand}\ insert\ I\ {\isacharquery}S{\isadigit{2}}\ {\isasymin}\ C{\isachardoublequoteclose}\isanewline
\ \ \ \ \isacommand{using}\isamarkupfalse%
\ assms{\isacharparenleft}{\isadigit{1}}{\isacharparenright}\ assms{\isacharparenleft}{\isadigit{2}}{\isacharparenright}\ assms{\isacharparenleft}{\isadigit{3}}{\isacharparenright}\ assms{\isacharparenleft}{\isadigit{4}}{\isacharparenright}\ {\isadigit{1}}\ {\isadigit{2}}\ {\isadigit{3}}\ {\isadigit{5}}\ {\isadigit{6}}\ {\isadigit{7}}\ \isacommand{by}\isamarkupfalse%
\ {\isacharparenleft}rule\ ex{\isadigit{3}}{\isacharunderscore}pcp{\isacharunderscore}SinE{\isacharunderscore}DIS{\isacharunderscore}auxEx{\isacharparenright}\isanewline
\ \ \isacommand{have}\isamarkupfalse%
\ {\isachardoublequoteopen}{\isasymforall}I\ {\isasymin}\ {\isacharbraceleft}G{\isacharcomma}H{\isacharbraceright}{\isachardot}\ insert\ I\ {\isacharquery}S{\isadigit{1}}\ {\isasymnotin}\ C\ {\isasymor}\ insert\ I\ {\isacharquery}S{\isadigit{2}}\ {\isasymnotin}\ C{\isachardoublequoteclose}\isanewline
\ \ \ \ \isacommand{using}\isamarkupfalse%
\ {\isadigit{4}}\ {\isadigit{8}}\ \isacommand{by}\isamarkupfalse%
\ simp\isanewline
\ \ \isacommand{then}\isamarkupfalse%
\ \isacommand{have}\isamarkupfalse%
\ {\isachardoublequoteopen}{\isasymforall}I\ {\isasymin}\ {\isacharbraceleft}G{\isacharcomma}H{\isacharbraceright}{\isachardot}\ {\isasymnot}{\isacharparenleft}insert\ I\ {\isacharquery}S{\isadigit{1}}\ {\isasymin}\ C\ {\isasymand}\ insert\ I\ {\isacharquery}S{\isadigit{2}}\ {\isasymin}\ C{\isacharparenright}{\isachardoublequoteclose}\isanewline
\ \ \ \ \isacommand{by}\isamarkupfalse%
\ {\isacharparenleft}simp\ only{\isacharcolon}\ de{\isacharunderscore}Morgan{\isacharunderscore}conj{\isacharparenright}\isanewline
\ \ \isacommand{then}\isamarkupfalse%
\ \isacommand{have}\isamarkupfalse%
\ {\isachardoublequoteopen}{\isasymnot}{\isacharparenleft}{\isasymexists}I\ {\isasymin}\ {\isacharbraceleft}G{\isacharcomma}H{\isacharbraceright}{\isachardot}\ insert\ I\ {\isacharquery}S{\isadigit{1}}\ {\isasymin}\ C\ {\isasymand}\ insert\ I\ {\isacharquery}S{\isadigit{2}}\ {\isasymin}\ C{\isacharparenright}{\isachardoublequoteclose}\isanewline
\ \ \ \ \isacommand{by}\isamarkupfalse%
\ {\isacharparenleft}simp\ only{\isacharcolon}\ bex{\isacharunderscore}simps{\isacharparenleft}{\isadigit{8}}{\isacharparenright}{\isacharparenright}\ \isanewline
\ \ \isacommand{thus}\isamarkupfalse%
\ {\isachardoublequoteopen}False{\isachardoublequoteclose}\isanewline
\ \ \ \ \isacommand{using}\isamarkupfalse%
\ Ex\ \isacommand{by}\isamarkupfalse%
\ {\isacharparenleft}rule\ notE{\isacharparenright}\isanewline
\isacommand{qed}\isamarkupfalse%
%
\endisatagproof
{\isafoldproof}%
%
\isadelimproof
%
\endisadelimproof
%
\begin{isamarkuptext}%
Una vez introducidos los lemas anteriores, podemos probar el lema \isa{ex{\isadigit{3}}{\isacharunderscore}pcp{\isacharunderscore}SinE{\isacharunderscore}DIS} que
  demuestra que si \isa{C} es una colección con la propiedad de consistencia proposicional y cerrada 
  bajo subconjuntos, \isa{S\ {\isasymin}\ E} y sea \isa{F} una fórmula de tipo \isa{{\isasymbeta}} con componentes \isa{{\isasymbeta}\isactrlsub {\isadigit{1}}} y \isa{{\isasymbeta}\isactrlsub {\isadigit{2}}}, se 
  verifica que o bien \isa{{\isacharbraceleft}{\isasymbeta}\isactrlsub {\isadigit{1}}{\isacharbraceright}\ {\isasymunion}\ S\ {\isasymin}\ C{\isacharprime}} o bien \isa{{\isacharbraceleft}{\isasymbeta}\isactrlsub {\isadigit{2}}{\isacharbraceright}\ {\isasymunion}\ S\ {\isasymin}\ C{\isacharprime}}. Además, para dicha prueba 
  necesitaremos los siguientes lemas auxiliares en Isabelle.%
\end{isamarkuptext}\isamarkuptrue%
\isacommand{lemma}\isamarkupfalse%
\ sall{\isacharunderscore}simps{\isacharunderscore}not{\isacharunderscore}all{\isacharcolon}\isanewline
\ \ \isakeyword{assumes}\ {\isachardoublequoteopen}{\isasymnot}{\isacharparenleft}{\isasymforall}x\ {\isasymsubseteq}\ A{\isachardot}\ P\ x{\isacharparenright}{\isachardoublequoteclose}\isanewline
\ \ \isakeyword{shows}\ {\isachardoublequoteopen}{\isasymexists}x\ {\isasymsubseteq}\ A{\isachardot}\ {\isacharparenleft}{\isasymnot}\ P\ x{\isacharparenright}{\isachardoublequoteclose}\isanewline
%
\isadelimproof
\ \ %
\endisadelimproof
%
\isatagproof
\isacommand{using}\isamarkupfalse%
\ assms\ \isacommand{by}\isamarkupfalse%
\ blast%
\endisatagproof
{\isafoldproof}%
%
\isadelimproof
\isanewline
%
\endisadelimproof
\isanewline
\isacommand{lemma}\isamarkupfalse%
\ subexE{\isacharcolon}\ {\isachardoublequoteopen}{\isasymexists}x{\isasymsubseteq}A{\isachardot}\ P\ x\ {\isasymLongrightarrow}\ {\isacharparenleft}{\isasymAnd}x{\isachardot}\ x{\isasymsubseteq}A\ {\isasymLongrightarrow}\ P\ x\ {\isasymLongrightarrow}\ Q{\isacharparenright}\ {\isasymLongrightarrow}\ Q{\isachardoublequoteclose}\isanewline
%
\isadelimproof
\ \ %
\endisadelimproof
%
\isatagproof
\isacommand{by}\isamarkupfalse%
\ blast%
\endisatagproof
{\isafoldproof}%
%
\isadelimproof
%
\endisadelimproof
%
\begin{isamarkuptext}%
De este modo, procedamos con la demostración detallada de \isa{ex{\isadigit{3}}{\isacharunderscore}pcp{\isacharunderscore}SinE{\isacharunderscore}DIS}.%
\end{isamarkuptext}\isamarkuptrue%
\isacommand{lemma}\isamarkupfalse%
\ ex{\isadigit{3}}{\isacharunderscore}pcp{\isacharunderscore}SinE{\isacharunderscore}DIS{\isacharcolon}\isanewline
\ \ \isakeyword{assumes}\ {\isachardoublequoteopen}pcp\ C{\isachardoublequoteclose}\isanewline
\ \ \ \ \ \ \ \ \ \ {\isachardoublequoteopen}subset{\isacharunderscore}closed\ C{\isachardoublequoteclose}\isanewline
\ \ \ \ \ \ \ \ \ \ {\isachardoublequoteopen}S\ {\isasymin}\ {\isacharparenleft}extF\ C{\isacharparenright}{\isachardoublequoteclose}\isanewline
\ \ \ \ \ \ \ \ \ \ {\isachardoublequoteopen}Dis\ F\ G\ H{\isachardoublequoteclose}\isanewline
\ \ \ \ \ \ \ \ \ \ {\isachardoublequoteopen}F\ {\isasymin}\ S{\isachardoublequoteclose}\isanewline
\ \ \isakeyword{shows}\ {\isachardoublequoteopen}{\isacharbraceleft}G{\isacharbraceright}\ {\isasymunion}\ S\ {\isasymin}\ {\isacharparenleft}extensionFin\ C{\isacharparenright}\ {\isasymor}\ {\isacharbraceleft}H{\isacharbraceright}\ {\isasymunion}\ S\ {\isasymin}\ {\isacharparenleft}extensionFin\ C{\isacharparenright}{\isachardoublequoteclose}\isanewline
%
\isadelimproof
%
\endisadelimproof
%
\isatagproof
\isacommand{proof}\isamarkupfalse%
\ {\isacharminus}\isanewline
\ \ \isacommand{have}\isamarkupfalse%
\ {\isachardoublequoteopen}{\isacharparenleft}extF\ C{\isacharparenright}\ {\isasymsubseteq}\ {\isacharparenleft}extensionFin\ C{\isacharparenright}{\isachardoublequoteclose}\ \isanewline
\ \ \ \ \isacommand{unfolding}\isamarkupfalse%
\ extensionFin\ \isacommand{by}\isamarkupfalse%
\ {\isacharparenleft}rule\ Un{\isacharunderscore}upper{\isadigit{2}}{\isacharparenright}\ \isanewline
\ \ \isacommand{have}\isamarkupfalse%
\ PCP{\isacharcolon}{\isachardoublequoteopen}{\isasymforall}S\ {\isasymin}\ C{\isachardot}\isanewline
\ \ \ \ \ \ \ \ \ \ \ \ {\isasymbottom}\ {\isasymnotin}\ S\isanewline
\ \ \ \ \ \ \ \ \ \ \ \ {\isasymand}\ {\isacharparenleft}{\isasymforall}k{\isachardot}\ Atom\ k\ {\isasymin}\ S\ {\isasymlongrightarrow}\ \isactrlbold {\isasymnot}\ {\isacharparenleft}Atom\ k{\isacharparenright}\ {\isasymin}\ S\ {\isasymlongrightarrow}\ False{\isacharparenright}\isanewline
\ \ \ \ \ \ \ \ \ \ \ \ {\isasymand}\ {\isacharparenleft}{\isasymforall}F\ G\ H{\isachardot}\ Con\ F\ G\ H\ {\isasymlongrightarrow}\ F\ {\isasymin}\ S\ {\isasymlongrightarrow}\ {\isacharbraceleft}G{\isacharcomma}H{\isacharbraceright}\ {\isasymunion}\ S\ {\isasymin}\ C{\isacharparenright}\isanewline
\ \ \ \ \ \ \ \ \ \ \ \ {\isasymand}\ {\isacharparenleft}{\isasymforall}F\ G\ H{\isachardot}\ Dis\ F\ G\ H\ {\isasymlongrightarrow}\ F\ {\isasymin}\ S\ {\isasymlongrightarrow}\ {\isacharbraceleft}G{\isacharbraceright}\ {\isasymunion}\ S\ {\isasymin}\ C\ {\isasymor}\ {\isacharbraceleft}H{\isacharbraceright}\ {\isasymunion}\ S\ {\isasymin}\ C{\isacharparenright}{\isachardoublequoteclose}\isanewline
\ \ \ \ \isacommand{using}\isamarkupfalse%
\ assms{\isacharparenleft}{\isadigit{1}}{\isacharparenright}\ \isacommand{by}\isamarkupfalse%
\ {\isacharparenleft}rule\ pcp{\isacharunderscore}alt{\isadigit{1}}{\isacharparenright}\isanewline
\ \ \isacommand{have}\isamarkupfalse%
\ E{\isacharcolon}{\isachardoublequoteopen}{\isasymforall}S{\isacharprime}\ {\isasymsubseteq}\ S{\isachardot}\ finite\ S{\isacharprime}\ {\isasymlongrightarrow}\ S{\isacharprime}\ {\isasymin}\ C{\isachardoublequoteclose}\isanewline
\ \ \ \ \isacommand{using}\isamarkupfalse%
\ assms{\isacharparenleft}{\isadigit{3}}{\isacharparenright}\ \isacommand{unfolding}\isamarkupfalse%
\ extF\ \isacommand{by}\isamarkupfalse%
\ {\isacharparenleft}rule\ CollectD{\isacharparenright}\isanewline
\ \ \isacommand{then}\isamarkupfalse%
\ \isacommand{have}\isamarkupfalse%
\ E{\isacharprime}{\isacharcolon}{\isachardoublequoteopen}{\isasymforall}S{\isacharprime}{\isachardot}\ S{\isacharprime}\ {\isasymsubseteq}\ S\ {\isasymlongrightarrow}\ finite\ S{\isacharprime}\ {\isasymlongrightarrow}\ S{\isacharprime}\ {\isasymin}\ C{\isachardoublequoteclose}\isanewline
\ \ \ \ \isacommand{by}\isamarkupfalse%
\ blast\isanewline
\ \ \isacommand{have}\isamarkupfalse%
\ SC{\isacharcolon}{\isachardoublequoteopen}{\isasymforall}S\ {\isasymin}\ C{\isachardot}\ {\isasymforall}S{\isacharprime}{\isasymsubseteq}S{\isachardot}\ S{\isacharprime}\ {\isasymin}\ C{\isachardoublequoteclose}\isanewline
\ \ \ \ \isacommand{using}\isamarkupfalse%
\ assms{\isacharparenleft}{\isadigit{2}}{\isacharparenright}\ \isacommand{by}\isamarkupfalse%
\ {\isacharparenleft}simp\ only{\isacharcolon}\ subset{\isacharunderscore}closed{\isacharunderscore}def{\isacharparenright}\isanewline
\ \ \isacommand{have}\isamarkupfalse%
\ {\isachardoublequoteopen}insert\ G\ S\ {\isasymin}\ {\isacharparenleft}extF\ C{\isacharparenright}\ {\isasymor}\ insert\ H\ S\ {\isasymin}\ {\isacharparenleft}extF\ C{\isacharparenright}{\isachardoublequoteclose}\ \isanewline
\ \ \isacommand{proof}\isamarkupfalse%
\ {\isacharparenleft}rule\ ccontr{\isacharparenright}\isanewline
\ \ \ \ \isacommand{assume}\isamarkupfalse%
\ {\isachardoublequoteopen}{\isasymnot}{\isacharparenleft}insert\ G\ S\ {\isasymin}\ {\isacharparenleft}extF\ C{\isacharparenright}\ {\isasymor}\ insert\ H\ S\ {\isasymin}\ {\isacharparenleft}extF\ C{\isacharparenright}{\isacharparenright}{\isachardoublequoteclose}\ \ \isanewline
\ \ \ \ \isacommand{then}\isamarkupfalse%
\ \isacommand{have}\isamarkupfalse%
\ Conj{\isacharcolon}{\isachardoublequoteopen}{\isasymnot}{\isacharparenleft}insert\ G\ S\ {\isasymin}\ {\isacharparenleft}extF\ C{\isacharparenright}{\isacharparenright}\ {\isasymand}\ {\isasymnot}{\isacharparenleft}insert\ H\ S\ {\isasymin}\ {\isacharparenleft}extF\ C{\isacharparenright}{\isacharparenright}{\isachardoublequoteclose}\ \isanewline
\ \ \ \ \ \ \isacommand{by}\isamarkupfalse%
\ {\isacharparenleft}simp\ only{\isacharcolon}\ simp{\isacharunderscore}thms{\isacharparenleft}{\isadigit{8}}{\isacharcomma}{\isadigit{2}}{\isadigit{5}}{\isacharparenright}\ de{\isacharunderscore}Morgan{\isacharunderscore}disj{\isacharparenright}\isanewline
\ \ \ \ \isacommand{then}\isamarkupfalse%
\ \isacommand{have}\isamarkupfalse%
\ {\isachardoublequoteopen}{\isasymnot}{\isacharparenleft}insert\ G\ S\ {\isasymin}\ {\isacharparenleft}extF\ C{\isacharparenright}{\isacharparenright}{\isachardoublequoteclose}\isanewline
\ \ \ \ \ \ \isacommand{by}\isamarkupfalse%
\ {\isacharparenleft}rule\ conjunct{\isadigit{1}}{\isacharparenright}\isanewline
\ \ \ \ \isacommand{then}\isamarkupfalse%
\ \isacommand{have}\isamarkupfalse%
\ {\isachardoublequoteopen}{\isasymnot}{\isacharparenleft}{\isasymforall}S{\isacharprime}\ {\isasymsubseteq}\ {\isacharparenleft}insert\ G\ S{\isacharparenright}{\isachardot}\ finite\ S{\isacharprime}\ {\isasymlongrightarrow}\ S{\isacharprime}\ {\isasymin}\ C{\isacharparenright}{\isachardoublequoteclose}\isanewline
\ \ \ \ \ \ \isacommand{unfolding}\isamarkupfalse%
\ extF\ \isacommand{by}\isamarkupfalse%
\ {\isacharparenleft}simp\ add{\isacharcolon}\ mem{\isacharunderscore}Collect{\isacharunderscore}eq{\isacharparenright}\isanewline
\ \ \ \ \isacommand{then}\isamarkupfalse%
\ \isacommand{have}\isamarkupfalse%
\ Ex{\isadigit{1}}{\isacharcolon}{\isachardoublequoteopen}{\isasymexists}S{\isacharprime}{\isasymsubseteq}\ {\isacharparenleft}insert\ G\ S{\isacharparenright}{\isachardot}\ {\isasymnot}{\isacharparenleft}finite\ S{\isacharprime}\ {\isasymlongrightarrow}\ S{\isacharprime}\ {\isasymin}\ C{\isacharparenright}{\isachardoublequoteclose}\isanewline
\ \ \ \ \ \ \isacommand{by}\isamarkupfalse%
\ {\isacharparenleft}rule\ sall{\isacharunderscore}simps{\isacharunderscore}not{\isacharunderscore}all{\isacharparenright}\isanewline
\ \ \ \ \isacommand{obtain}\isamarkupfalse%
\ S{\isadigit{1}}\ \isakeyword{where}\ {\isachardoublequoteopen}S{\isadigit{1}}\ {\isasymsubseteq}\ insert\ G\ S{\isachardoublequoteclose}\ \isakeyword{and}\ {\isachardoublequoteopen}{\isasymnot}{\isacharparenleft}finite\ S{\isadigit{1}}\ {\isasymlongrightarrow}\ S{\isadigit{1}}\ {\isasymin}\ C{\isacharparenright}{\isachardoublequoteclose}\isanewline
\ \ \ \ \ \ \isacommand{using}\isamarkupfalse%
\ Ex{\isadigit{1}}\ \isacommand{by}\isamarkupfalse%
\ {\isacharparenleft}rule\ subexE{\isacharparenright}\isanewline
\ \ \ \ \isacommand{have}\isamarkupfalse%
\ {\isachardoublequoteopen}finite\ S{\isadigit{1}}\ {\isasymand}\ S{\isadigit{1}}\ {\isasymnotin}\ C{\isachardoublequoteclose}\ \isanewline
\ \ \ \ \ \ \isacommand{using}\isamarkupfalse%
\ {\isacartoucheopen}{\isasymnot}{\isacharparenleft}finite\ S{\isadigit{1}}\ {\isasymlongrightarrow}\ S{\isadigit{1}}\ {\isasymin}\ C{\isacharparenright}{\isacartoucheclose}\ \isacommand{by}\isamarkupfalse%
\ {\isacharparenleft}simp\ only{\isacharcolon}\ simp{\isacharunderscore}thms{\isacharparenleft}{\isadigit{8}}{\isacharparenright}\ not{\isacharunderscore}imp{\isacharparenright}\isanewline
\ \ \ \ \isacommand{then}\isamarkupfalse%
\ \isacommand{have}\isamarkupfalse%
\ {\isachardoublequoteopen}finite\ S{\isadigit{1}}{\isachardoublequoteclose}\isanewline
\ \ \ \ \ \ \isacommand{by}\isamarkupfalse%
\ {\isacharparenleft}rule\ conjunct{\isadigit{1}}{\isacharparenright}\isanewline
\ \ \ \ \isacommand{have}\isamarkupfalse%
\ {\isachardoublequoteopen}S{\isadigit{1}}\ {\isasymnotin}\ C{\isachardoublequoteclose}\isanewline
\ \ \ \ \ \ \isacommand{using}\isamarkupfalse%
\ {\isacartoucheopen}finite\ S{\isadigit{1}}\ {\isasymand}\ S{\isadigit{1}}\ {\isasymnotin}\ C{\isacartoucheclose}\ \isacommand{by}\isamarkupfalse%
\ {\isacharparenleft}rule\ conjunct{\isadigit{2}}{\isacharparenright}\isanewline
\ \ \ \ \isacommand{then}\isamarkupfalse%
\ \isacommand{have}\isamarkupfalse%
\ {\isachardoublequoteopen}insert\ G\ S{\isadigit{1}}\ {\isasymnotin}\ C{\isachardoublequoteclose}\isanewline
\ \ \ \ \isacommand{proof}\isamarkupfalse%
\ {\isacharminus}\ \isanewline
\ \ \ \ \ \ \isacommand{have}\isamarkupfalse%
\ {\isachardoublequoteopen}S{\isadigit{1}}\ {\isasymsubseteq}\ S\ {\isasymlongrightarrow}\ finite\ S{\isadigit{1}}\ {\isasymlongrightarrow}\ S{\isadigit{1}}\ {\isasymin}\ C{\isachardoublequoteclose}\isanewline
\ \ \ \ \ \ \ \ \isacommand{using}\isamarkupfalse%
\ E{\isacharprime}\ \isacommand{by}\isamarkupfalse%
\ {\isacharparenleft}rule\ allE{\isacharparenright}\isanewline
\ \ \ \ \ \ \isacommand{then}\isamarkupfalse%
\ \isacommand{have}\isamarkupfalse%
\ {\isachardoublequoteopen}{\isasymnot}\ S{\isadigit{1}}\ {\isasymsubseteq}\ S{\isachardoublequoteclose}\isanewline
\ \ \ \ \ \ \ \ \isacommand{using}\isamarkupfalse%
\ {\isacartoucheopen}{\isasymnot}\ {\isacharparenleft}finite\ S{\isadigit{1}}\ {\isasymlongrightarrow}\ S{\isadigit{1}}\ {\isasymin}\ C{\isacharparenright}{\isacartoucheclose}\ \isacommand{by}\isamarkupfalse%
\ {\isacharparenleft}rule\ mt{\isacharparenright}\isanewline
\ \ \ \ \ \ \isacommand{then}\isamarkupfalse%
\ \isacommand{have}\isamarkupfalse%
\ {\isachardoublequoteopen}{\isacharparenleft}S{\isadigit{1}}\ {\isasymsubseteq}\ insert\ G\ S{\isacharparenright}\ {\isasymnoteq}\ {\isacharparenleft}S{\isadigit{1}}\ {\isasymsubseteq}\ S{\isacharparenright}{\isachardoublequoteclose}\isanewline
\ \ \ \ \ \ \ \ \isacommand{using}\isamarkupfalse%
\ {\isacartoucheopen}S{\isadigit{1}}\ {\isasymsubseteq}\ insert\ G\ S{\isacartoucheclose}\ \isacommand{by}\isamarkupfalse%
\ simp\isanewline
\ \ \ \ \ \ \isacommand{then}\isamarkupfalse%
\ \isacommand{have}\isamarkupfalse%
\ notSI{\isacharcolon}{\isachardoublequoteopen}{\isasymnot}{\isacharparenleft}S{\isadigit{1}}\ {\isasymsubseteq}\ insert\ G\ S\ {\isasymlongleftrightarrow}\ S{\isadigit{1}}\ {\isasymsubseteq}\ S{\isacharparenright}{\isachardoublequoteclose}\isanewline
\ \ \ \ \ \ \ \ \isacommand{by}\isamarkupfalse%
\ blast\isanewline
\ \ \ \ \ \ \isacommand{have}\isamarkupfalse%
\ subsetInsert{\isacharcolon}{\isachardoublequoteopen}G\ {\isasymnotin}\ S{\isadigit{1}}\ {\isasymLongrightarrow}\ S{\isadigit{1}}\ {\isasymsubseteq}\ insert\ G\ S\ {\isasymlongleftrightarrow}\ S{\isadigit{1}}\ {\isasymsubseteq}\ S{\isachardoublequoteclose}\isanewline
\ \ \ \ \ \ \ \ \isacommand{by}\isamarkupfalse%
\ {\isacharparenleft}rule\ subset{\isacharunderscore}insert{\isacharparenright}\isanewline
\ \ \ \ \ \ \isacommand{have}\isamarkupfalse%
\ {\isachardoublequoteopen}{\isasymnot}{\isacharparenleft}G\ {\isasymnotin}\ S{\isadigit{1}}{\isacharparenright}{\isachardoublequoteclose}\isanewline
\ \ \ \ \ \ \ \ \isacommand{using}\isamarkupfalse%
\ notSI\ subsetInsert\ \isacommand{by}\isamarkupfalse%
\ {\isacharparenleft}rule\ contrapos{\isacharunderscore}nn{\isacharparenright}\isanewline
\ \ \ \ \ \ \isacommand{then}\isamarkupfalse%
\ \isacommand{have}\isamarkupfalse%
\ {\isachardoublequoteopen}G\ {\isasymin}\ S{\isadigit{1}}{\isachardoublequoteclose}\isanewline
\ \ \ \ \ \ \ \ \isacommand{by}\isamarkupfalse%
\ {\isacharparenleft}rule\ notnotD{\isacharparenright}\isanewline
\ \ \ \ \ \ \isacommand{then}\isamarkupfalse%
\ \isacommand{have}\isamarkupfalse%
\ {\isachardoublequoteopen}insert\ G\ S{\isadigit{1}}\ {\isacharequal}\ S{\isadigit{1}}{\isachardoublequoteclose}\isanewline
\ \ \ \ \ \ \ \ \isacommand{by}\isamarkupfalse%
\ {\isacharparenleft}rule\ insert{\isacharunderscore}absorb{\isacharparenright}\isanewline
\ \ \ \ \ \ \isacommand{show}\isamarkupfalse%
\ {\isacharquery}thesis\isanewline
\ \ \ \ \ \ \ \ \isacommand{using}\isamarkupfalse%
\ {\isacartoucheopen}S{\isadigit{1}}\ {\isasymnotin}\ C{\isacartoucheclose}\ \isacommand{by}\isamarkupfalse%
\ {\isacharparenleft}simp\ only{\isacharcolon}\ simp{\isacharunderscore}thms{\isacharparenleft}{\isadigit{8}}{\isacharparenright}\ {\isacartoucheopen}insert\ G\ S{\isadigit{1}}\ {\isacharequal}\ S{\isadigit{1}}{\isacartoucheclose}{\isacharparenright}\isanewline
\ \ \ \ \isacommand{qed}\isamarkupfalse%
\ \isanewline
\ \ \ \ \isacommand{let}\isamarkupfalse%
\ {\isacharquery}S{\isadigit{1}}{\isacharequal}{\isachardoublequoteopen}S{\isadigit{1}}\ {\isacharminus}\ {\isacharbraceleft}G{\isacharbraceright}{\isachardoublequoteclose}\isanewline
\ \ \ \ \isacommand{have}\isamarkupfalse%
\ {\isachardoublequoteopen}insert\ G\ S\ {\isacharequal}\ {\isacharbraceleft}G{\isacharbraceright}\ {\isasymunion}\ S{\isachardoublequoteclose}\isanewline
\ \ \ \ \ \ \isacommand{by}\isamarkupfalse%
\ {\isacharparenleft}rule\ insert{\isacharunderscore}is{\isacharunderscore}Un{\isacharparenright}\isanewline
\ \ \ \ \isacommand{have}\isamarkupfalse%
\ {\isachardoublequoteopen}S{\isadigit{1}}\ {\isasymsubseteq}\ {\isacharbraceleft}G{\isacharbraceright}\ {\isasymunion}\ S{\isachardoublequoteclose}\isanewline
\ \ \ \ \ \ \isacommand{using}\isamarkupfalse%
\ {\isacartoucheopen}S{\isadigit{1}}\ {\isasymsubseteq}\ insert\ G\ S{\isacartoucheclose}\ \isacommand{by}\isamarkupfalse%
\ {\isacharparenleft}simp\ only{\isacharcolon}\ {\isacartoucheopen}insert\ G\ S\ {\isacharequal}\ {\isacharbraceleft}G{\isacharbraceright}\ {\isasymunion}\ S{\isacartoucheclose}{\isacharparenright}\isanewline
\ \ \ \ \isacommand{have}\isamarkupfalse%
\ {\isadigit{1}}{\isacharcolon}{\isachardoublequoteopen}{\isacharquery}S{\isadigit{1}}\ {\isasymsubseteq}\ S{\isachardoublequoteclose}\ \isanewline
\ \ \ \ \ \ \isacommand{using}\isamarkupfalse%
\ {\isacartoucheopen}S{\isadigit{1}}\ {\isasymsubseteq}\ {\isacharbraceleft}G{\isacharbraceright}\ {\isasymunion}\ S{\isacartoucheclose}\ \isacommand{by}\isamarkupfalse%
\ {\isacharparenleft}simp\ only{\isacharcolon}\ Diff{\isacharunderscore}subset{\isacharunderscore}conv{\isacharparenright}\isanewline
\ \ \ \ \isacommand{have}\isamarkupfalse%
\ {\isadigit{2}}{\isacharcolon}{\isachardoublequoteopen}finite\ {\isacharquery}S{\isadigit{1}}{\isachardoublequoteclose}\isanewline
\ \ \ \ \ \ \isacommand{using}\isamarkupfalse%
\ {\isacartoucheopen}finite\ S{\isadigit{1}}{\isacartoucheclose}\ \isacommand{by}\isamarkupfalse%
\ {\isacharparenleft}simp\ only{\isacharcolon}\ finite{\isacharunderscore}Diff{\isacharparenright}\isanewline
\ \ \ \ \isacommand{have}\isamarkupfalse%
\ {\isachardoublequoteopen}insert\ G\ {\isacharquery}S{\isadigit{1}}\ {\isacharequal}\ insert\ G\ S{\isadigit{1}}{\isachardoublequoteclose}\isanewline
\ \ \ \ \ \ \isacommand{by}\isamarkupfalse%
\ {\isacharparenleft}simp\ only{\isacharcolon}\ insert{\isacharunderscore}Diff{\isacharunderscore}single{\isacharparenright}\isanewline
\ \ \ \ \isacommand{then}\isamarkupfalse%
\ \isacommand{have}\isamarkupfalse%
\ {\isadigit{3}}{\isacharcolon}{\isachardoublequoteopen}insert\ G\ {\isacharquery}S{\isadigit{1}}\ {\isasymnotin}\ C{\isachardoublequoteclose}\isanewline
\ \ \ \ \ \ \isacommand{using}\isamarkupfalse%
\ {\isacartoucheopen}insert\ G\ S{\isadigit{1}}\ {\isasymnotin}\ C{\isacartoucheclose}\ \isacommand{by}\isamarkupfalse%
\ {\isacharparenleft}simp\ only{\isacharcolon}\ simp{\isacharunderscore}thms{\isacharparenleft}{\isadigit{6}}{\isacharcomma}{\isadigit{8}}{\isacharparenright}\ {\isacartoucheopen}insert\ G\ {\isacharquery}S{\isadigit{1}}\ {\isacharequal}\ insert\ G\ S{\isadigit{1}}{\isacartoucheclose}{\isacharparenright}\isanewline
\ \ \ \ \isacommand{have}\isamarkupfalse%
\ {\isachardoublequoteopen}insert\ H\ S\ {\isasymnotin}\ {\isacharparenleft}extF\ C{\isacharparenright}{\isachardoublequoteclose}\isanewline
\ \ \ \ \ \ \isacommand{using}\isamarkupfalse%
\ Conj\ \isacommand{by}\isamarkupfalse%
\ {\isacharparenleft}rule\ conjunct{\isadigit{2}}{\isacharparenright}\isanewline
\ \ \ \ \isacommand{then}\isamarkupfalse%
\ \isacommand{have}\isamarkupfalse%
\ {\isachardoublequoteopen}{\isasymnot}{\isacharparenleft}{\isasymforall}S{\isacharprime}\ {\isasymsubseteq}\ {\isacharparenleft}insert\ H\ S{\isacharparenright}{\isachardot}\ finite\ S{\isacharprime}\ {\isasymlongrightarrow}\ S{\isacharprime}\ {\isasymin}\ C{\isacharparenright}{\isachardoublequoteclose}\isanewline
\ \ \ \ \ \ \isacommand{unfolding}\isamarkupfalse%
\ extF\ \isacommand{by}\isamarkupfalse%
\ {\isacharparenleft}simp\ add{\isacharcolon}\ mem{\isacharunderscore}Collect{\isacharunderscore}eq{\isacharparenright}\isanewline
\ \ \ \ \isacommand{then}\isamarkupfalse%
\ \isacommand{have}\isamarkupfalse%
\ Ex{\isadigit{2}}{\isacharcolon}{\isachardoublequoteopen}{\isasymexists}S{\isacharprime}{\isasymsubseteq}\ {\isacharparenleft}insert\ H\ S{\isacharparenright}{\isachardot}\ {\isasymnot}{\isacharparenleft}finite\ S{\isacharprime}\ {\isasymlongrightarrow}\ S{\isacharprime}\ {\isasymin}\ C{\isacharparenright}{\isachardoublequoteclose}\isanewline
\ \ \ \ \ \ \isacommand{by}\isamarkupfalse%
\ {\isacharparenleft}rule\ sall{\isacharunderscore}simps{\isacharunderscore}not{\isacharunderscore}all{\isacharparenright}\isanewline
\ \ \ \ \isacommand{obtain}\isamarkupfalse%
\ S{\isadigit{2}}\ \isakeyword{where}\ {\isachardoublequoteopen}S{\isadigit{2}}\ {\isasymsubseteq}\ insert\ H\ S{\isachardoublequoteclose}\ \isakeyword{and}\ {\isachardoublequoteopen}{\isasymnot}{\isacharparenleft}finite\ S{\isadigit{2}}\ {\isasymlongrightarrow}\ S{\isadigit{2}}\ {\isasymin}\ C{\isacharparenright}{\isachardoublequoteclose}\isanewline
\ \ \ \ \ \ \isacommand{using}\isamarkupfalse%
\ Ex{\isadigit{2}}\ \isacommand{by}\isamarkupfalse%
\ {\isacharparenleft}rule\ subexE{\isacharparenright}\isanewline
\ \ \ \ \isacommand{have}\isamarkupfalse%
\ {\isachardoublequoteopen}finite\ S{\isadigit{2}}\ {\isasymand}\ S{\isadigit{2}}\ {\isasymnotin}\ C{\isachardoublequoteclose}\isanewline
\ \ \ \ \ \ \isacommand{using}\isamarkupfalse%
\ {\isacartoucheopen}{\isasymnot}{\isacharparenleft}finite\ S{\isadigit{2}}\ {\isasymlongrightarrow}\ S{\isadigit{2}}\ {\isasymin}\ C{\isacharparenright}{\isacartoucheclose}\ \isacommand{by}\isamarkupfalse%
\ {\isacharparenleft}simp\ only{\isacharcolon}\ simp{\isacharunderscore}thms{\isacharparenleft}{\isadigit{8}}{\isacharcomma}{\isadigit{2}}{\isadigit{5}}{\isacharparenright}\ not{\isacharunderscore}imp{\isacharparenright}\isanewline
\ \ \ \ \isacommand{then}\isamarkupfalse%
\ \isacommand{have}\isamarkupfalse%
\ {\isachardoublequoteopen}finite\ S{\isadigit{2}}{\isachardoublequoteclose}\isanewline
\ \ \ \ \ \ \isacommand{by}\isamarkupfalse%
\ {\isacharparenleft}rule\ conjunct{\isadigit{1}}{\isacharparenright}\isanewline
\ \ \ \ \isacommand{have}\isamarkupfalse%
\ {\isachardoublequoteopen}S{\isadigit{2}}\ {\isasymnotin}\ C{\isachardoublequoteclose}\isanewline
\ \ \ \ \ \ \isacommand{using}\isamarkupfalse%
\ {\isacartoucheopen}finite\ S{\isadigit{2}}\ {\isasymand}\ S{\isadigit{2}}\ {\isasymnotin}\ C{\isacartoucheclose}\ \isacommand{by}\isamarkupfalse%
\ {\isacharparenleft}rule\ conjunct{\isadigit{2}}{\isacharparenright}\isanewline
\ \ \ \ \isacommand{then}\isamarkupfalse%
\ \isacommand{have}\isamarkupfalse%
\ {\isachardoublequoteopen}insert\ H\ S{\isadigit{2}}\ {\isasymnotin}\ C{\isachardoublequoteclose}\isanewline
\ \ \ \ \isacommand{proof}\isamarkupfalse%
\ {\isacharminus}\isanewline
\ \ \ \ \ \ \isacommand{have}\isamarkupfalse%
\ {\isachardoublequoteopen}S{\isadigit{2}}\ {\isasymsubseteq}\ S\ {\isasymlongrightarrow}\ finite\ S{\isadigit{2}}\ {\isasymlongrightarrow}\ S{\isadigit{2}}\ {\isasymin}\ C{\isachardoublequoteclose}\isanewline
\ \ \ \ \ \ \ \ \isacommand{using}\isamarkupfalse%
\ E{\isacharprime}\ \isacommand{by}\isamarkupfalse%
\ {\isacharparenleft}rule\ allE{\isacharparenright}\isanewline
\ \ \ \ \ \ \isacommand{then}\isamarkupfalse%
\ \isacommand{have}\isamarkupfalse%
\ {\isachardoublequoteopen}{\isasymnot}\ S{\isadigit{2}}\ {\isasymsubseteq}\ S{\isachardoublequoteclose}\isanewline
\ \ \ \ \ \ \ \ \isacommand{using}\isamarkupfalse%
\ {\isacartoucheopen}{\isasymnot}\ {\isacharparenleft}finite\ S{\isadigit{2}}\ {\isasymlongrightarrow}\ S{\isadigit{2}}\ {\isasymin}\ C{\isacharparenright}{\isacartoucheclose}\ \isacommand{by}\isamarkupfalse%
\ {\isacharparenleft}rule\ mt{\isacharparenright}\isanewline
\ \ \ \ \ \ \isacommand{then}\isamarkupfalse%
\ \isacommand{have}\isamarkupfalse%
\ {\isachardoublequoteopen}{\isacharparenleft}S{\isadigit{2}}\ {\isasymsubseteq}\ insert\ H\ S{\isacharparenright}\ {\isasymnoteq}\ {\isacharparenleft}S{\isadigit{2}}\ {\isasymsubseteq}\ S{\isacharparenright}{\isachardoublequoteclose}\isanewline
\ \ \ \ \ \ \ \ \isacommand{using}\isamarkupfalse%
\ {\isacartoucheopen}S{\isadigit{2}}\ {\isasymsubseteq}\ insert\ H\ S{\isacartoucheclose}\ \isacommand{by}\isamarkupfalse%
\ simp\ \isanewline
\ \ \ \ \ \ \isacommand{then}\isamarkupfalse%
\ \isacommand{have}\isamarkupfalse%
\ notSI{\isacharcolon}{\isachardoublequoteopen}{\isasymnot}{\isacharparenleft}S{\isadigit{2}}\ {\isasymsubseteq}\ insert\ H\ S\ {\isasymlongleftrightarrow}\ S{\isadigit{2}}\ {\isasymsubseteq}\ S{\isacharparenright}{\isachardoublequoteclose}\isanewline
\ \ \ \ \ \ \ \ \isacommand{by}\isamarkupfalse%
\ blast\ \isanewline
\ \ \ \ \ \ \isacommand{have}\isamarkupfalse%
\ subsetInsert{\isacharcolon}{\isachardoublequoteopen}H\ {\isasymnotin}\ S{\isadigit{2}}\ {\isasymLongrightarrow}\ S{\isadigit{2}}\ {\isasymsubseteq}\ insert\ H\ S\ {\isasymlongleftrightarrow}\ S{\isadigit{2}}\ {\isasymsubseteq}\ S{\isachardoublequoteclose}\isanewline
\ \ \ \ \ \ \ \ \isacommand{by}\isamarkupfalse%
\ {\isacharparenleft}rule\ subset{\isacharunderscore}insert{\isacharparenright}\isanewline
\ \ \ \ \ \ \isacommand{have}\isamarkupfalse%
\ {\isachardoublequoteopen}{\isasymnot}{\isacharparenleft}H\ {\isasymnotin}\ S{\isadigit{2}}{\isacharparenright}{\isachardoublequoteclose}\isanewline
\ \ \ \ \ \ \ \ \isacommand{using}\isamarkupfalse%
\ notSI\ subsetInsert\ \isacommand{by}\isamarkupfalse%
\ {\isacharparenleft}rule\ contrapos{\isacharunderscore}nn{\isacharparenright}\isanewline
\ \ \ \ \ \ \isacommand{then}\isamarkupfalse%
\ \isacommand{have}\isamarkupfalse%
\ {\isachardoublequoteopen}H\ {\isasymin}\ S{\isadigit{2}}{\isachardoublequoteclose}\isanewline
\ \ \ \ \ \ \ \ \isacommand{by}\isamarkupfalse%
\ {\isacharparenleft}rule\ notnotD{\isacharparenright}\isanewline
\ \ \ \ \ \ \isacommand{then}\isamarkupfalse%
\ \isacommand{have}\isamarkupfalse%
\ {\isachardoublequoteopen}insert\ H\ S{\isadigit{2}}\ {\isacharequal}\ S{\isadigit{2}}{\isachardoublequoteclose}\isanewline
\ \ \ \ \ \ \ \ \isacommand{by}\isamarkupfalse%
\ {\isacharparenleft}rule\ insert{\isacharunderscore}absorb{\isacharparenright}\isanewline
\ \ \ \ \ \ \isacommand{show}\isamarkupfalse%
\ {\isacharquery}thesis\isanewline
\ \ \ \ \ \ \ \ \isacommand{using}\isamarkupfalse%
\ {\isacartoucheopen}S{\isadigit{2}}\ {\isasymnotin}\ C{\isacartoucheclose}\ \isacommand{by}\isamarkupfalse%
\ {\isacharparenleft}simp\ only{\isacharcolon}\ simp{\isacharunderscore}thms{\isacharparenleft}{\isadigit{8}}{\isacharparenright}\ {\isacartoucheopen}insert\ H\ S{\isadigit{2}}\ {\isacharequal}\ S{\isadigit{2}}{\isacartoucheclose}{\isacharparenright}\isanewline
\ \ \ \ \isacommand{qed}\isamarkupfalse%
\ \isanewline
\ \ \ \ \isacommand{let}\isamarkupfalse%
\ {\isacharquery}S{\isadigit{2}}{\isacharequal}{\isachardoublequoteopen}S{\isadigit{2}}\ {\isacharminus}\ {\isacharbraceleft}H{\isacharbraceright}{\isachardoublequoteclose}\isanewline
\ \ \ \ \isacommand{have}\isamarkupfalse%
\ {\isachardoublequoteopen}insert\ H\ S\ {\isacharequal}\ {\isacharbraceleft}H{\isacharbraceright}\ {\isasymunion}\ S{\isachardoublequoteclose}\isanewline
\ \ \ \ \ \ \isacommand{by}\isamarkupfalse%
\ {\isacharparenleft}rule\ insert{\isacharunderscore}is{\isacharunderscore}Un{\isacharparenright}\isanewline
\ \ \ \ \isacommand{have}\isamarkupfalse%
\ {\isachardoublequoteopen}S{\isadigit{2}}\ {\isasymsubseteq}\ {\isacharbraceleft}H{\isacharbraceright}\ {\isasymunion}\ S{\isachardoublequoteclose}\isanewline
\ \ \ \ \ \ \isacommand{using}\isamarkupfalse%
\ {\isacartoucheopen}S{\isadigit{2}}\ {\isasymsubseteq}\ insert\ H\ S{\isacartoucheclose}\ \isacommand{by}\isamarkupfalse%
\ {\isacharparenleft}simp\ only{\isacharcolon}\ {\isacartoucheopen}insert\ H\ S\ {\isacharequal}\ {\isacharbraceleft}H{\isacharbraceright}\ {\isasymunion}\ S{\isacartoucheclose}{\isacharparenright}\isanewline
\ \ \ \ \isacommand{have}\isamarkupfalse%
\ {\isadigit{4}}{\isacharcolon}{\isachardoublequoteopen}{\isacharquery}S{\isadigit{2}}\ {\isasymsubseteq}\ S{\isachardoublequoteclose}\ \isanewline
\ \ \ \ \ \ \isacommand{using}\isamarkupfalse%
\ {\isacartoucheopen}S{\isadigit{2}}\ {\isasymsubseteq}\ {\isacharbraceleft}H{\isacharbraceright}\ {\isasymunion}\ S{\isacartoucheclose}\ \isacommand{by}\isamarkupfalse%
\ {\isacharparenleft}simp\ only{\isacharcolon}\ Diff{\isacharunderscore}subset{\isacharunderscore}conv{\isacharparenright}\isanewline
\ \ \ \ \isacommand{have}\isamarkupfalse%
\ {\isadigit{5}}{\isacharcolon}{\isachardoublequoteopen}finite\ {\isacharquery}S{\isadigit{2}}{\isachardoublequoteclose}\ \isanewline
\ \ \ \ \ \ \isacommand{using}\isamarkupfalse%
\ {\isacartoucheopen}finite\ S{\isadigit{2}}{\isacartoucheclose}\ \isacommand{by}\isamarkupfalse%
\ {\isacharparenleft}simp\ only{\isacharcolon}\ finite{\isacharunderscore}Diff{\isacharparenright}\isanewline
\ \ \ \ \isacommand{have}\isamarkupfalse%
\ {\isachardoublequoteopen}insert\ H\ {\isacharquery}S{\isadigit{2}}\ {\isacharequal}\ insert\ H\ S{\isadigit{2}}{\isachardoublequoteclose}\isanewline
\ \ \ \ \ \ \isacommand{by}\isamarkupfalse%
\ {\isacharparenleft}simp\ only{\isacharcolon}\ insert{\isacharunderscore}Diff{\isacharunderscore}single{\isacharparenright}\isanewline
\ \ \ \ \isacommand{then}\isamarkupfalse%
\ \isacommand{have}\isamarkupfalse%
\ {\isadigit{6}}{\isacharcolon}{\isachardoublequoteopen}insert\ H\ {\isacharquery}S{\isadigit{2}}\ {\isasymnotin}\ C{\isachardoublequoteclose}\isanewline
\ \ \ \ \ \ \isacommand{using}\isamarkupfalse%
\ {\isacartoucheopen}insert\ H\ S{\isadigit{2}}\ {\isasymnotin}\ C{\isacartoucheclose}\ \isacommand{by}\isamarkupfalse%
\ {\isacharparenleft}simp\ only{\isacharcolon}\ simp{\isacharunderscore}thms{\isacharparenleft}{\isadigit{6}}{\isacharcomma}{\isadigit{8}}{\isacharparenright}\ {\isacartoucheopen}insert\ H\ {\isacharquery}S{\isadigit{2}}\ {\isacharequal}\ insert\ H\ S{\isadigit{2}}{\isacartoucheclose}{\isacharparenright}\isanewline
\ \ \ \ \isacommand{show}\isamarkupfalse%
\ {\isachardoublequoteopen}False{\isachardoublequoteclose}\isanewline
\ \ \ \ \ \ \isacommand{using}\isamarkupfalse%
\ assms{\isacharparenleft}{\isadigit{1}}{\isacharparenright}\ assms{\isacharparenleft}{\isadigit{2}}{\isacharparenright}\ assms{\isacharparenleft}{\isadigit{3}}{\isacharparenright}\ assms{\isacharparenleft}{\isadigit{4}}{\isacharparenright}\ assms{\isacharparenleft}{\isadigit{5}}{\isacharparenright}\ {\isadigit{1}}\ {\isadigit{2}}\ {\isadigit{3}}\ {\isadigit{4}}\ {\isadigit{5}}\ {\isadigit{6}}\ \isacommand{by}\isamarkupfalse%
\ {\isacharparenleft}rule\ ex{\isadigit{3}}{\isacharunderscore}pcp{\isacharunderscore}SinE{\isacharunderscore}DIS{\isacharunderscore}auxFalse{\isacharparenright}\isanewline
\ \ \isacommand{qed}\isamarkupfalse%
\isanewline
\ \ \isacommand{thus}\isamarkupfalse%
\ {\isacharquery}thesis\isanewline
\ \ \isacommand{proof}\isamarkupfalse%
\ {\isacharparenleft}rule\ disjE{\isacharparenright}\isanewline
\ \ \ \ \isacommand{assume}\isamarkupfalse%
\ {\isachardoublequoteopen}insert\ G\ S\ {\isasymin}\ {\isacharparenleft}extF\ C{\isacharparenright}{\isachardoublequoteclose}\isanewline
\ \ \ \ \isacommand{have}\isamarkupfalse%
\ insG{\isacharcolon}{\isachardoublequoteopen}insert\ G\ S\ {\isasymin}\ {\isacharparenleft}extensionFin\ C{\isacharparenright}{\isachardoublequoteclose}\isanewline
\ \ \ \ \ \ \isacommand{using}\isamarkupfalse%
\ {\isacartoucheopen}{\isacharparenleft}extF\ C{\isacharparenright}\ {\isasymsubseteq}\ {\isacharparenleft}extensionFin\ C{\isacharparenright}{\isacartoucheclose}\ {\isacartoucheopen}insert\ G\ S\ {\isasymin}\ {\isacharparenleft}extF\ C{\isacharparenright}{\isacartoucheclose}\ \isacommand{by}\isamarkupfalse%
\ {\isacharparenleft}simp\ only{\isacharcolon}\ in{\isacharunderscore}mono{\isacharparenright}\isanewline
\ \ \ \ \isacommand{have}\isamarkupfalse%
\ {\isachardoublequoteopen}insert\ G\ S\ {\isacharequal}\ {\isacharbraceleft}G{\isacharbraceright}\ {\isasymunion}\ S{\isachardoublequoteclose}\isanewline
\ \ \ \ \ \ \isacommand{by}\isamarkupfalse%
\ {\isacharparenleft}rule\ insert{\isacharunderscore}is{\isacharunderscore}Un{\isacharparenright}\isanewline
\ \ \ \ \isacommand{then}\isamarkupfalse%
\ \isacommand{have}\isamarkupfalse%
\ {\isachardoublequoteopen}{\isacharbraceleft}G{\isacharbraceright}\ {\isasymunion}\ S\ {\isasymin}\ {\isacharparenleft}extensionFin\ C{\isacharparenright}{\isachardoublequoteclose}\isanewline
\ \ \ \ \ \ \isacommand{using}\isamarkupfalse%
\ insG\ {\isacartoucheopen}insert\ G\ S\ {\isacharequal}\ {\isacharbraceleft}G{\isacharbraceright}\ {\isasymunion}\ S{\isacartoucheclose}\ \isacommand{by}\isamarkupfalse%
\ {\isacharparenleft}simp\ only{\isacharcolon}\ insG{\isacharparenright}\isanewline
\ \ \ \ \isacommand{thus}\isamarkupfalse%
\ {\isacharquery}thesis\isanewline
\ \ \ \ \ \ \isacommand{by}\isamarkupfalse%
\ {\isacharparenleft}rule\ disjI{\isadigit{1}}{\isacharparenright}\isanewline
\ \ \isacommand{next}\isamarkupfalse%
\isanewline
\ \ \ \ \isacommand{assume}\isamarkupfalse%
\ {\isachardoublequoteopen}insert\ H\ S\ {\isasymin}\ {\isacharparenleft}extF\ C{\isacharparenright}{\isachardoublequoteclose}\isanewline
\ \ \ \ \isacommand{have}\isamarkupfalse%
\ insH{\isacharcolon}{\isachardoublequoteopen}insert\ H\ S\ {\isasymin}\ {\isacharparenleft}extensionFin\ C{\isacharparenright}{\isachardoublequoteclose}\isanewline
\ \ \ \ \ \ \isacommand{using}\isamarkupfalse%
\ {\isacartoucheopen}{\isacharparenleft}extF\ C{\isacharparenright}\ {\isasymsubseteq}\ {\isacharparenleft}extensionFin\ C{\isacharparenright}{\isacartoucheclose}\ {\isacartoucheopen}insert\ H\ S\ {\isasymin}\ {\isacharparenleft}extF\ C{\isacharparenright}{\isacartoucheclose}\ \isacommand{by}\isamarkupfalse%
\ {\isacharparenleft}simp\ only{\isacharcolon}\ in{\isacharunderscore}mono{\isacharparenright}\isanewline
\ \ \ \ \isacommand{have}\isamarkupfalse%
\ {\isachardoublequoteopen}insert\ H\ S\ {\isacharequal}\ {\isacharbraceleft}H{\isacharbraceright}\ {\isasymunion}\ S{\isachardoublequoteclose}\isanewline
\ \ \ \ \ \ \isacommand{by}\isamarkupfalse%
\ {\isacharparenleft}rule\ insert{\isacharunderscore}is{\isacharunderscore}Un{\isacharparenright}\isanewline
\ \ \ \ \isacommand{then}\isamarkupfalse%
\ \isacommand{have}\isamarkupfalse%
\ {\isachardoublequoteopen}{\isacharbraceleft}H{\isacharbraceright}\ {\isasymunion}\ S\ {\isasymin}\ {\isacharparenleft}extensionFin\ C{\isacharparenright}{\isachardoublequoteclose}\isanewline
\ \ \ \ \ \ \isacommand{using}\isamarkupfalse%
\ insH\ {\isacartoucheopen}insert\ H\ S\ {\isacharequal}\ {\isacharbraceleft}H{\isacharbraceright}\ {\isasymunion}\ S{\isacartoucheclose}\ \isacommand{by}\isamarkupfalse%
\ {\isacharparenleft}simp\ only{\isacharcolon}\ insH{\isacharparenright}\isanewline
\ \ \ \ \isacommand{thus}\isamarkupfalse%
\ {\isacharquery}thesis\isanewline
\ \ \ \ \ \ \isacommand{by}\isamarkupfalse%
\ {\isacharparenleft}rule\ disjI{\isadigit{2}}{\isacharparenright}\isanewline
\ \ \isacommand{qed}\isamarkupfalse%
\isanewline
\isacommand{qed}\isamarkupfalse%
%
\endisatagproof
{\isafoldproof}%
%
\isadelimproof
%
\endisadelimproof
%
\begin{isamarkuptext}%
Probados los lemas \isa{ex{\isadigit{3}}{\isacharunderscore}pcp{\isacharunderscore}SinE{\isacharunderscore}CON} y \isa{ex{\isadigit{3}}{\isacharunderscore}pcp{\isacharunderscore}SinE{\isacharunderscore}DIS}, podemos demostrar que \isa{C{\isacharprime}\ {\isacharequal}\ C\ {\isasymunion}\ E} 
  verifica las condiciones del lema de caracterización de la propiedad de consistencia proposicional 
  para el caso en que \isa{S\ {\isasymin}\ E}, formalizado como \isa{ex{\isadigit{3}}{\isacharunderscore}pcp{\isacharunderscore}SinE}. Dicho lema prueba que, si \isa{C} es 
  una colección con la propiedad de consistencia proposicional y cerrada bajo subconjuntos, y sea 
  \isa{S\ {\isasymin}\ E}, se verifican las condiciones:
  \begin{itemize}
    \item \isa{{\isasymbottom}\ {\isasymnotin}\ S}.
    \item Dada \isa{p} una fórmula atómica cualquiera, no se tiene 
    simultáneamente que\\ \isa{p\ {\isasymin}\ S} y \isa{{\isasymnot}\ p\ {\isasymin}\ S}.
    \item Para toda fórmula de tipo \isa{{\isasymalpha}} con componentes \isa{{\isasymalpha}\isactrlsub {\isadigit{1}}} y \isa{{\isasymalpha}\isactrlsub {\isadigit{2}}} tal que \isa{{\isasymalpha}}
    pertenece a \isa{S}, se tiene que \isa{{\isacharbraceleft}{\isasymalpha}\isactrlsub {\isadigit{1}}{\isacharcomma}{\isasymalpha}\isactrlsub {\isadigit{2}}{\isacharbraceright}\ {\isasymunion}\ S} pertenece a \isa{C{\isacharprime}}.
    \item Para toda fórmula de tipo \isa{{\isasymbeta}} con componentes \isa{{\isasymbeta}\isactrlsub {\isadigit{1}}} y \isa{{\isasymbeta}\isactrlsub {\isadigit{2}}} tal que \isa{{\isasymbeta}}
    pertenece a \isa{S}, se tiene que o bien \isa{{\isacharbraceleft}{\isasymbeta}\isactrlsub {\isadigit{1}}{\isacharbraceright}\ {\isasymunion}\ S} pertenece a \isa{C{\isacharprime}} o 
    bien \isa{{\isacharbraceleft}{\isasymbeta}\isactrlsub {\isadigit{2}}{\isacharbraceright}\ {\isasymunion}\ S} pertenece a \isa{C{\isacharprime}}.
  \end{itemize}%
\end{isamarkuptext}\isamarkuptrue%
\isacommand{lemma}\isamarkupfalse%
\ ex{\isadigit{3}}{\isacharunderscore}pcp{\isacharunderscore}SinE{\isacharcolon}\isanewline
\ \ \isakeyword{assumes}\ {\isachardoublequoteopen}pcp\ C{\isachardoublequoteclose}\isanewline
\ \ \ \ \ \ \ \ \ \ {\isachardoublequoteopen}subset{\isacharunderscore}closed\ C{\isachardoublequoteclose}\isanewline
\ \ \ \ \ \ \ \ \ \ {\isachardoublequoteopen}S\ {\isasymin}\ {\isacharparenleft}extF\ C{\isacharparenright}{\isachardoublequoteclose}\ \isanewline
\ \ \isakeyword{shows}\ {\isachardoublequoteopen}{\isasymbottom}\ {\isasymnotin}\ S\ {\isasymand}\isanewline
\ \ \ \ \ \ \ \ \ {\isacharparenleft}{\isasymforall}k{\isachardot}\ Atom\ k\ {\isasymin}\ S\ {\isasymlongrightarrow}\ \isactrlbold {\isasymnot}\ {\isacharparenleft}Atom\ k{\isacharparenright}\ {\isasymin}\ S\ {\isasymlongrightarrow}\ False{\isacharparenright}\ {\isasymand}\isanewline
\ \ \ \ \ \ \ \ \ {\isacharparenleft}{\isasymforall}F\ G\ H{\isachardot}\ Con\ F\ G\ H\ {\isasymlongrightarrow}\ F\ {\isasymin}\ S\ {\isasymlongrightarrow}\ {\isacharbraceleft}G{\isacharcomma}\ H{\isacharbraceright}\ {\isasymunion}\ S\ {\isasymin}\ {\isacharparenleft}extensionFin\ C{\isacharparenright}{\isacharparenright}\ {\isasymand}\isanewline
\ \ \ \ \ \ \ \ \ {\isacharparenleft}{\isasymforall}F\ G\ H{\isachardot}\ Dis\ F\ G\ H\ {\isasymlongrightarrow}\ F\ {\isasymin}\ S\ {\isasymlongrightarrow}\ {\isacharbraceleft}G{\isacharbraceright}\ {\isasymunion}\ S\ {\isasymin}\ {\isacharparenleft}extensionFin\ C{\isacharparenright}\ {\isasymor}\ {\isacharbraceleft}H{\isacharbraceright}\ {\isasymunion}\ S\ {\isasymin}\ {\isacharparenleft}extensionFin\ C{\isacharparenright}{\isacharparenright}{\isachardoublequoteclose}\isanewline
%
\isadelimproof
%
\endisadelimproof
%
\isatagproof
\isacommand{proof}\isamarkupfalse%
\ {\isacharminus}\isanewline
\ \ \isacommand{have}\isamarkupfalse%
\ PCP{\isacharcolon}{\isachardoublequoteopen}{\isasymforall}S\ {\isasymin}\ C{\isachardot}\isanewline
\ \ \ \ \ \ \ \ \ {\isasymbottom}\ {\isasymnotin}\ S\ {\isasymand}\isanewline
\ \ \ \ \ \ \ \ \ {\isacharparenleft}{\isasymforall}k{\isachardot}\ Atom\ k\ {\isasymin}\ S\ {\isasymlongrightarrow}\ \isactrlbold {\isasymnot}\ {\isacharparenleft}Atom\ k{\isacharparenright}\ {\isasymin}\ S\ {\isasymlongrightarrow}\ False{\isacharparenright}\ {\isasymand}\isanewline
\ \ \ \ \ \ \ \ \ {\isacharparenleft}{\isasymforall}F\ G\ H{\isachardot}\ Con\ F\ G\ H\ {\isasymlongrightarrow}\ F\ {\isasymin}\ S\ {\isasymlongrightarrow}\ {\isacharbraceleft}G{\isacharcomma}\ H{\isacharbraceright}\ {\isasymunion}\ S\ {\isasymin}\ C{\isacharparenright}\ {\isasymand}\isanewline
\ \ \ \ \ \ \ \ \ {\isacharparenleft}{\isasymforall}F\ G\ H{\isachardot}\ Dis\ F\ G\ H\ {\isasymlongrightarrow}\ F\ {\isasymin}\ S\ {\isasymlongrightarrow}\ {\isacharbraceleft}G{\isacharbraceright}\ {\isasymunion}\ S\ {\isasymin}\ C\ {\isasymor}\ {\isacharbraceleft}H{\isacharbraceright}\ {\isasymunion}\ S\ {\isasymin}\ C{\isacharparenright}{\isachardoublequoteclose}\isanewline
\ \ \ \ \isacommand{using}\isamarkupfalse%
\ assms{\isacharparenleft}{\isadigit{1}}{\isacharparenright}\ \isacommand{by}\isamarkupfalse%
\ {\isacharparenleft}rule\ pcp{\isacharunderscore}alt{\isadigit{1}}{\isacharparenright}\isanewline
\ \ \isacommand{have}\isamarkupfalse%
\ E{\isacharcolon}{\isachardoublequoteopen}{\isasymforall}S{\isacharprime}\ {\isasymsubseteq}\ S{\isachardot}\ finite\ S{\isacharprime}\ {\isasymlongrightarrow}\ S{\isacharprime}\ {\isasymin}\ C{\isachardoublequoteclose}\isanewline
\ \ \ \ \isacommand{using}\isamarkupfalse%
\ assms{\isacharparenleft}{\isadigit{3}}{\isacharparenright}\ \isacommand{unfolding}\isamarkupfalse%
\ extF\ \isacommand{by}\isamarkupfalse%
\ {\isacharparenleft}rule\ CollectD{\isacharparenright}\isanewline
\ \ \isacommand{have}\isamarkupfalse%
\ {\isachardoublequoteopen}{\isacharbraceleft}{\isacharbraceright}\ {\isasymsubseteq}\ S{\isachardoublequoteclose}\isanewline
\ \ \ \ \isacommand{by}\isamarkupfalse%
\ {\isacharparenleft}rule\ empty{\isacharunderscore}subsetI{\isacharparenright}\isanewline
\ \ \isacommand{have}\isamarkupfalse%
\ C{\isadigit{1}}{\isacharcolon}{\isachardoublequoteopen}{\isasymbottom}\ {\isasymnotin}\ S{\isachardoublequoteclose}\isanewline
\ \ \isacommand{proof}\isamarkupfalse%
\ {\isacharparenleft}rule\ ccontr{\isacharparenright}\isanewline
\ \ \ \ \isacommand{assume}\isamarkupfalse%
\ {\isachardoublequoteopen}{\isasymnot}{\isacharparenleft}{\isasymbottom}\ {\isasymnotin}\ S{\isacharparenright}{\isachardoublequoteclose}\isanewline
\ \ \ \ \isacommand{then}\isamarkupfalse%
\ \isacommand{have}\isamarkupfalse%
\ {\isachardoublequoteopen}{\isasymbottom}\ {\isasymin}\ S{\isachardoublequoteclose}\isanewline
\ \ \ \ \ \ \isacommand{by}\isamarkupfalse%
\ {\isacharparenleft}rule\ notnotD{\isacharparenright}\isanewline
\ \ \ \ \isacommand{then}\isamarkupfalse%
\ \isacommand{have}\isamarkupfalse%
\ {\isachardoublequoteopen}{\isasymbottom}\ {\isasymin}\ S\ {\isasymand}\ {\isacharbraceleft}{\isacharbraceright}\ {\isasymsubseteq}\ S{\isachardoublequoteclose}\isanewline
\ \ \ \ \ \ \isacommand{using}\isamarkupfalse%
\ {\isacartoucheopen}{\isacharbraceleft}{\isacharbraceright}\ {\isasymsubseteq}\ S{\isacartoucheclose}\ \isacommand{by}\isamarkupfalse%
\ {\isacharparenleft}rule\ conjI{\isacharparenright}\isanewline
\ \ \ \ \isacommand{then}\isamarkupfalse%
\ \isacommand{have}\isamarkupfalse%
\ {\isachardoublequoteopen}insert\ {\isasymbottom}\ {\isacharbraceleft}{\isacharbraceright}\ {\isasymsubseteq}\ S{\isachardoublequoteclose}\ \isanewline
\ \ \ \ \ \ \isacommand{by}\isamarkupfalse%
\ {\isacharparenleft}simp\ only{\isacharcolon}\ insert{\isacharunderscore}subset{\isacharparenright}\isanewline
\ \ \ \ \isacommand{have}\isamarkupfalse%
\ {\isachardoublequoteopen}finite\ {\isacharbraceleft}{\isacharbraceright}{\isachardoublequoteclose}\isanewline
\ \ \ \ \ \ \isacommand{by}\isamarkupfalse%
\ {\isacharparenleft}rule\ finite{\isachardot}emptyI{\isacharparenright}\isanewline
\ \ \ \ \isacommand{then}\isamarkupfalse%
\ \isacommand{have}\isamarkupfalse%
\ {\isachardoublequoteopen}finite\ {\isacharparenleft}insert\ {\isasymbottom}\ {\isacharbraceleft}{\isacharbraceright}{\isacharparenright}{\isachardoublequoteclose}\isanewline
\ \ \ \ \ \ \isacommand{by}\isamarkupfalse%
\ {\isacharparenleft}rule\ finite{\isachardot}insertI{\isacharparenright}\isanewline
\ \ \ \ \isacommand{have}\isamarkupfalse%
\ {\isachardoublequoteopen}finite\ {\isacharparenleft}insert\ {\isasymbottom}\ {\isacharbraceleft}{\isacharbraceright}{\isacharparenright}\ {\isasymlongrightarrow}\ {\isacharparenleft}insert\ {\isasymbottom}\ {\isacharbraceleft}{\isacharbraceright}{\isacharparenright}\ {\isasymin}\ C{\isachardoublequoteclose}\isanewline
\ \ \ \ \ \ \isacommand{using}\isamarkupfalse%
\ E\ {\isacartoucheopen}{\isacharparenleft}insert\ {\isasymbottom}\ {\isacharbraceleft}{\isacharbraceright}{\isacharparenright}\ {\isasymsubseteq}\ S{\isacartoucheclose}\ \isacommand{by}\isamarkupfalse%
\ simp\ \isanewline
\ \ \ \ \isacommand{then}\isamarkupfalse%
\ \isacommand{have}\isamarkupfalse%
\ {\isachardoublequoteopen}{\isacharparenleft}insert\ {\isasymbottom}\ {\isacharbraceleft}{\isacharbraceright}{\isacharparenright}\ {\isasymin}\ C{\isachardoublequoteclose}\isanewline
\ \ \ \ \ \ \isacommand{using}\isamarkupfalse%
\ {\isacartoucheopen}finite\ {\isacharparenleft}insert\ {\isasymbottom}\ {\isacharbraceleft}{\isacharbraceright}{\isacharparenright}{\isacartoucheclose}\ \isacommand{by}\isamarkupfalse%
\ {\isacharparenleft}rule\ mp{\isacharparenright}\isanewline
\ \ \ \ \isacommand{have}\isamarkupfalse%
\ {\isachardoublequoteopen}{\isasymbottom}\ {\isasymnotin}\ {\isacharparenleft}insert\ {\isasymbottom}\ {\isacharbraceleft}{\isacharbraceright}{\isacharparenright}\ {\isasymand}\isanewline
\ \ \ \ \ \ \ \ \ {\isacharparenleft}{\isasymforall}k{\isachardot}\ Atom\ k\ {\isasymin}\ {\isacharparenleft}insert\ {\isasymbottom}\ {\isacharbraceleft}{\isacharbraceright}{\isacharparenright}\ {\isasymlongrightarrow}\ \isactrlbold {\isasymnot}\ {\isacharparenleft}Atom\ k{\isacharparenright}\ {\isasymin}\ {\isacharparenleft}insert\ {\isasymbottom}\ {\isacharbraceleft}{\isacharbraceright}{\isacharparenright}\ {\isasymlongrightarrow}\ False{\isacharparenright}\ {\isasymand}\isanewline
\ \ \ \ \ \ \ \ \ {\isacharparenleft}{\isasymforall}F\ G\ H{\isachardot}\ Con\ F\ G\ H\ {\isasymlongrightarrow}\ F\ {\isasymin}\ {\isacharparenleft}insert\ {\isasymbottom}\ {\isacharbraceleft}{\isacharbraceright}{\isacharparenright}\ {\isasymlongrightarrow}\ {\isacharbraceleft}G{\isacharcomma}\ H{\isacharbraceright}\ {\isasymunion}\ {\isacharparenleft}insert\ {\isasymbottom}\ {\isacharbraceleft}{\isacharbraceright}{\isacharparenright}\ {\isasymin}\ C{\isacharparenright}\ {\isasymand}\isanewline
\ \ \ \ \ \ \ \ \ {\isacharparenleft}{\isasymforall}F\ G\ H{\isachardot}\ Dis\ F\ G\ H\ {\isasymlongrightarrow}\ F\ {\isasymin}\ {\isacharparenleft}insert\ {\isasymbottom}\ {\isacharbraceleft}{\isacharbraceright}{\isacharparenright}\ {\isasymlongrightarrow}\ {\isacharbraceleft}G{\isacharbraceright}\ {\isasymunion}\ {\isacharparenleft}insert\ {\isasymbottom}\ {\isacharbraceleft}{\isacharbraceright}{\isacharparenright}\ {\isasymin}\ C\ {\isasymor}\ {\isacharbraceleft}H{\isacharbraceright}\ {\isasymunion}\ {\isacharparenleft}insert\ {\isasymbottom}\ {\isacharbraceleft}{\isacharbraceright}{\isacharparenright}\ {\isasymin}\ C{\isacharparenright}{\isachardoublequoteclose}\isanewline
\ \ \ \ \ \ \isacommand{using}\isamarkupfalse%
\ PCP\ {\isacartoucheopen}{\isacharparenleft}insert\ {\isasymbottom}\ {\isacharbraceleft}{\isacharbraceright}{\isacharparenright}\ {\isasymin}\ C{\isacartoucheclose}\ \isacommand{by}\isamarkupfalse%
\ blast\ \isanewline
\ \ \ \ \isacommand{then}\isamarkupfalse%
\ \isacommand{have}\isamarkupfalse%
\ {\isachardoublequoteopen}{\isasymbottom}\ {\isasymnotin}\ {\isacharparenleft}insert\ {\isasymbottom}\ {\isacharbraceleft}{\isacharbraceright}{\isacharparenright}{\isachardoublequoteclose}\isanewline
\ \ \ \ \ \ \isacommand{by}\isamarkupfalse%
\ {\isacharparenleft}rule\ conjunct{\isadigit{1}}{\isacharparenright}\isanewline
\ \ \ \ \isacommand{have}\isamarkupfalse%
\ {\isachardoublequoteopen}{\isasymbottom}\ {\isasymin}\ {\isacharparenleft}insert\ {\isasymbottom}\ {\isacharbraceleft}{\isacharbraceright}{\isacharparenright}{\isachardoublequoteclose}\isanewline
\ \ \ \ \ \ \isacommand{by}\isamarkupfalse%
\ {\isacharparenleft}rule\ insertI{\isadigit{1}}{\isacharparenright}\isanewline
\ \ \ \ \isacommand{show}\isamarkupfalse%
\ {\isachardoublequoteopen}False{\isachardoublequoteclose}\isanewline
\ \ \ \ \ \ \isacommand{using}\isamarkupfalse%
\ {\isacartoucheopen}{\isasymbottom}\ {\isasymnotin}\ {\isacharparenleft}insert\ {\isasymbottom}\ {\isacharbraceleft}{\isacharbraceright}{\isacharparenright}{\isacartoucheclose}\ {\isacartoucheopen}{\isasymbottom}\ {\isasymin}\ {\isacharparenleft}insert\ {\isasymbottom}\ {\isacharbraceleft}{\isacharbraceright}{\isacharparenright}{\isacartoucheclose}\ \isacommand{by}\isamarkupfalse%
\ {\isacharparenleft}rule\ notE{\isacharparenright}\isanewline
\ \ \isacommand{qed}\isamarkupfalse%
\isanewline
\ \ \isacommand{have}\isamarkupfalse%
\ C{\isadigit{2}}{\isacharcolon}{\isachardoublequoteopen}{\isasymforall}k{\isachardot}\ Atom\ k\ {\isasymin}\ S\ {\isasymlongrightarrow}\ \isactrlbold {\isasymnot}\ {\isacharparenleft}Atom\ k{\isacharparenright}\ {\isasymin}\ S\ {\isasymlongrightarrow}\ False{\isachardoublequoteclose}\isanewline
\ \ \isacommand{proof}\isamarkupfalse%
\ {\isacharparenleft}rule\ allI{\isacharparenright}\isanewline
\ \ \ \ \isacommand{fix}\isamarkupfalse%
\ k\isanewline
\ \ \ \ \isacommand{show}\isamarkupfalse%
\ {\isachardoublequoteopen}Atom\ k\ {\isasymin}\ S\ {\isasymlongrightarrow}\ \isactrlbold {\isasymnot}{\isacharparenleft}Atom\ k{\isacharparenright}\ {\isasymin}\ S\ {\isasymlongrightarrow}\ False{\isachardoublequoteclose}\isanewline
\ \ \ \ \isacommand{proof}\isamarkupfalse%
\ {\isacharparenleft}rule\ impI{\isacharparenright}{\isacharplus}\isanewline
\ \ \ \ \ \ \isacommand{assume}\isamarkupfalse%
\ {\isachardoublequoteopen}Atom\ k\ {\isasymin}\ S{\isachardoublequoteclose}\isanewline
\ \ \ \ \ \ \isacommand{assume}\isamarkupfalse%
\ {\isachardoublequoteopen}\isactrlbold {\isasymnot}{\isacharparenleft}Atom\ k{\isacharparenright}\ {\isasymin}\ S{\isachardoublequoteclose}\isanewline
\ \ \ \ \ \ \isacommand{let}\isamarkupfalse%
\ {\isacharquery}A{\isacharequal}{\isachardoublequoteopen}insert\ {\isacharparenleft}Atom\ k{\isacharparenright}\ {\isacharparenleft}insert\ {\isacharparenleft}\isactrlbold {\isasymnot}{\isacharparenleft}Atom\ k{\isacharparenright}{\isacharparenright}\ {\isacharbraceleft}{\isacharbraceright}{\isacharparenright}{\isachardoublequoteclose}\isanewline
\ \ \ \ \ \ \isacommand{have}\isamarkupfalse%
\ {\isachardoublequoteopen}Atom\ k\ {\isasymin}\ {\isacharquery}A{\isachardoublequoteclose}\isanewline
\ \ \ \ \ \ \ \ \isacommand{by}\isamarkupfalse%
\ {\isacharparenleft}simp\ only{\isacharcolon}\ insert{\isacharunderscore}iff\ simp{\isacharunderscore}thms{\isacharparenright}\ \isanewline
\ \ \ \ \ \ \isacommand{have}\isamarkupfalse%
\ {\isachardoublequoteopen}\isactrlbold {\isasymnot}{\isacharparenleft}Atom\ k{\isacharparenright}\ {\isasymin}\ {\isacharquery}A{\isachardoublequoteclose}\isanewline
\ \ \ \ \ \ \ \ \isacommand{by}\isamarkupfalse%
\ {\isacharparenleft}simp\ only{\isacharcolon}\ insert{\isacharunderscore}iff\ simp{\isacharunderscore}thms{\isacharparenright}\ \isanewline
\ \ \ \ \ \ \isacommand{have}\isamarkupfalse%
\ inSubset{\isacharcolon}{\isachardoublequoteopen}insert\ {\isacharparenleft}\isactrlbold {\isasymnot}{\isacharparenleft}Atom\ k{\isacharparenright}{\isacharparenright}\ {\isacharbraceleft}{\isacharbraceright}\ {\isasymsubseteq}\ S{\isachardoublequoteclose}\isanewline
\ \ \ \ \ \ \ \ \isacommand{using}\isamarkupfalse%
\ {\isacartoucheopen}\isactrlbold {\isasymnot}{\isacharparenleft}Atom\ k{\isacharparenright}\ {\isasymin}\ S{\isacartoucheclose}\ {\isacartoucheopen}{\isacharbraceleft}{\isacharbraceright}\ {\isasymsubseteq}\ S{\isacartoucheclose}\ \isacommand{by}\isamarkupfalse%
\ {\isacharparenleft}simp\ only{\isacharcolon}\ insert{\isacharunderscore}subset{\isacharparenright}\isanewline
\ \ \ \ \ \ \isacommand{have}\isamarkupfalse%
\ {\isachardoublequoteopen}{\isacharquery}A\ {\isasymsubseteq}\ S{\isachardoublequoteclose}\isanewline
\ \ \ \ \ \ \ \ \isacommand{using}\isamarkupfalse%
\ inSubset\ {\isacartoucheopen}Atom\ k\ {\isasymin}\ S{\isacartoucheclose}\ \isacommand{by}\isamarkupfalse%
\ {\isacharparenleft}simp\ only{\isacharcolon}\ insert{\isacharunderscore}subset{\isacharparenright}\isanewline
\ \ \ \ \ \ \isacommand{have}\isamarkupfalse%
\ {\isachardoublequoteopen}finite\ {\isacharbraceleft}{\isacharbraceright}{\isachardoublequoteclose}\isanewline
\ \ \ \ \ \ \ \ \isacommand{by}\isamarkupfalse%
\ {\isacharparenleft}simp\ only{\isacharcolon}\ finite{\isachardot}emptyI{\isacharparenright}\isanewline
\ \ \ \ \ \ \isacommand{then}\isamarkupfalse%
\ \isacommand{have}\isamarkupfalse%
\ {\isachardoublequoteopen}finite\ {\isacharparenleft}insert\ {\isacharparenleft}\isactrlbold {\isasymnot}{\isacharparenleft}Atom\ k{\isacharparenright}{\isacharparenright}\ {\isacharbraceleft}{\isacharbraceright}{\isacharparenright}{\isachardoublequoteclose}\isanewline
\ \ \ \ \ \ \ \ \isacommand{by}\isamarkupfalse%
\ {\isacharparenleft}rule\ finite{\isachardot}insertI{\isacharparenright}\isanewline
\ \ \ \ \ \ \isacommand{then}\isamarkupfalse%
\ \isacommand{have}\isamarkupfalse%
\ {\isachardoublequoteopen}finite\ {\isacharquery}A{\isachardoublequoteclose}\isanewline
\ \ \ \ \ \ \ \ \isacommand{by}\isamarkupfalse%
\ {\isacharparenleft}rule\ finite{\isachardot}insertI{\isacharparenright}\isanewline
\ \ \ \ \ \ \isacommand{have}\isamarkupfalse%
\ {\isachardoublequoteopen}finite\ {\isacharquery}A\ {\isasymlongrightarrow}\ {\isacharquery}A\ {\isasymin}\ C{\isachardoublequoteclose}\isanewline
\ \ \ \ \ \ \ \ \isacommand{using}\isamarkupfalse%
\ E\ {\isacartoucheopen}{\isacharquery}A\ {\isasymsubseteq}\ S{\isacartoucheclose}\ \isacommand{by}\isamarkupfalse%
\ {\isacharparenleft}rule\ sspec{\isacharparenright}\isanewline
\ \ \ \ \ \ \isacommand{then}\isamarkupfalse%
\ \isacommand{have}\isamarkupfalse%
\ {\isachardoublequoteopen}{\isacharquery}A\ {\isasymin}\ C{\isachardoublequoteclose}\isanewline
\ \ \ \ \ \ \ \ \isacommand{using}\isamarkupfalse%
\ {\isacartoucheopen}finite\ {\isacharquery}A{\isacartoucheclose}\ \isacommand{by}\isamarkupfalse%
\ {\isacharparenleft}rule\ mp{\isacharparenright}\isanewline
\ \ \ \ \ \ \isacommand{have}\isamarkupfalse%
\ {\isachardoublequoteopen}{\isasymbottom}\ {\isasymnotin}\ {\isacharquery}A\isanewline
\ \ \ \ \ \ \ \ \ \ \ \ {\isasymand}\ {\isacharparenleft}{\isasymforall}k{\isachardot}\ Atom\ k\ {\isasymin}\ {\isacharquery}A\ {\isasymlongrightarrow}\ \isactrlbold {\isasymnot}\ {\isacharparenleft}Atom\ k{\isacharparenright}\ {\isasymin}\ {\isacharquery}A\ {\isasymlongrightarrow}\ False{\isacharparenright}\isanewline
\ \ \ \ \ \ \ \ \ \ \ \ {\isasymand}\ {\isacharparenleft}{\isasymforall}F\ G\ H{\isachardot}\ Con\ F\ G\ H\ {\isasymlongrightarrow}\ F\ {\isasymin}\ {\isacharquery}A\ {\isasymlongrightarrow}\ {\isacharbraceleft}G{\isacharcomma}H{\isacharbraceright}\ {\isasymunion}\ {\isacharquery}A\ {\isasymin}\ C{\isacharparenright}\isanewline
\ \ \ \ \ \ \ \ \ \ \ \ {\isasymand}\ {\isacharparenleft}{\isasymforall}F\ G\ H{\isachardot}\ Dis\ F\ G\ H\ {\isasymlongrightarrow}\ F\ {\isasymin}\ {\isacharquery}A\ {\isasymlongrightarrow}\ {\isacharbraceleft}G{\isacharbraceright}\ {\isasymunion}\ {\isacharquery}A\ {\isasymin}\ C\ {\isasymor}\ {\isacharbraceleft}H{\isacharbraceright}\ {\isasymunion}\ {\isacharquery}A\ {\isasymin}\ C{\isacharparenright}{\isachardoublequoteclose}\isanewline
\ \ \ \ \ \ \ \ \isacommand{using}\isamarkupfalse%
\ PCP\ {\isacartoucheopen}{\isacharquery}A\ {\isasymin}\ C{\isacartoucheclose}\ \isacommand{by}\isamarkupfalse%
\ {\isacharparenleft}rule\ bspec{\isacharparenright}\isanewline
\ \ \ \ \ \ \isacommand{then}\isamarkupfalse%
\ \isacommand{have}\isamarkupfalse%
\ {\isachardoublequoteopen}{\isasymforall}k{\isachardot}\ Atom\ k\ {\isasymin}\ {\isacharquery}A\ {\isasymlongrightarrow}\ \isactrlbold {\isasymnot}\ {\isacharparenleft}Atom\ k{\isacharparenright}\ {\isasymin}\ {\isacharquery}A\ {\isasymlongrightarrow}\ False{\isachardoublequoteclose}\isanewline
\ \ \ \ \ \ \ \ \isacommand{by}\isamarkupfalse%
\ {\isacharparenleft}iprover\ elim{\isacharcolon}\ conjunct{\isadigit{2}}\ conjunct{\isadigit{1}}{\isacharparenright}\isanewline
\ \ \ \ \ \ \isacommand{then}\isamarkupfalse%
\ \isacommand{have}\isamarkupfalse%
\ {\isachardoublequoteopen}Atom\ k\ {\isasymin}\ {\isacharquery}A\ {\isasymlongrightarrow}\ \isactrlbold {\isasymnot}\ {\isacharparenleft}Atom\ k{\isacharparenright}\ {\isasymin}\ {\isacharquery}A\ {\isasymlongrightarrow}\ False{\isachardoublequoteclose}\isanewline
\ \ \ \ \ \ \ \ \isacommand{by}\isamarkupfalse%
\ {\isacharparenleft}iprover\ elim{\isacharcolon}\ allE{\isacharparenright}\isanewline
\ \ \ \ \ \ \isacommand{then}\isamarkupfalse%
\ \isacommand{have}\isamarkupfalse%
\ {\isachardoublequoteopen}\isactrlbold {\isasymnot}{\isacharparenleft}Atom\ k{\isacharparenright}\ {\isasymin}\ {\isacharquery}A\ {\isasymlongrightarrow}\ False{\isachardoublequoteclose}\isanewline
\ \ \ \ \ \ \ \ \isacommand{using}\isamarkupfalse%
\ {\isacartoucheopen}Atom\ k\ {\isasymin}\ {\isacharquery}A{\isacartoucheclose}\ \isacommand{by}\isamarkupfalse%
\ {\isacharparenleft}rule\ mp{\isacharparenright}\isanewline
\ \ \ \ \ \ \isacommand{thus}\isamarkupfalse%
\ {\isachardoublequoteopen}False{\isachardoublequoteclose}\isanewline
\ \ \ \ \ \ \ \ \isacommand{using}\isamarkupfalse%
\ {\isacartoucheopen}\isactrlbold {\isasymnot}{\isacharparenleft}Atom\ k{\isacharparenright}\ {\isasymin}\ {\isacharquery}A{\isacartoucheclose}\ \isacommand{by}\isamarkupfalse%
\ {\isacharparenleft}rule\ mp{\isacharparenright}\isanewline
\ \ \ \ \isacommand{qed}\isamarkupfalse%
\isanewline
\ \ \isacommand{qed}\isamarkupfalse%
\isanewline
\ \ \isacommand{have}\isamarkupfalse%
\ C{\isadigit{3}}{\isacharcolon}{\isachardoublequoteopen}{\isasymforall}F\ G\ H{\isachardot}\ Con\ F\ G\ H\ {\isasymlongrightarrow}\ F\ {\isasymin}\ S\ {\isasymlongrightarrow}\ {\isacharbraceleft}G{\isacharcomma}H{\isacharbraceright}\ {\isasymunion}\ S\ {\isasymin}\ {\isacharparenleft}extensionFin\ C{\isacharparenright}{\isachardoublequoteclose}\isanewline
\ \ \isacommand{proof}\isamarkupfalse%
\ {\isacharparenleft}rule\ allI{\isacharparenright}{\isacharplus}\isanewline
\ \ \ \ \isacommand{fix}\isamarkupfalse%
\ F\ G\ H\isanewline
\ \ \ \ \isacommand{show}\isamarkupfalse%
\ {\isachardoublequoteopen}Con\ F\ G\ H\ {\isasymlongrightarrow}\ F\ {\isasymin}\ S\ {\isasymlongrightarrow}\ {\isacharbraceleft}G{\isacharcomma}H{\isacharbraceright}\ {\isasymunion}\ S\ {\isasymin}\ {\isacharparenleft}extensionFin\ C{\isacharparenright}{\isachardoublequoteclose}\isanewline
\ \ \ \ \isacommand{proof}\isamarkupfalse%
\ {\isacharparenleft}rule\ impI{\isacharparenright}{\isacharplus}\isanewline
\ \ \ \ \ \ \isacommand{assume}\isamarkupfalse%
\ {\isachardoublequoteopen}Con\ F\ G\ H{\isachardoublequoteclose}\isanewline
\ \ \ \ \ \ \isacommand{assume}\isamarkupfalse%
\ {\isachardoublequoteopen}F\ {\isasymin}\ S{\isachardoublequoteclose}\ \isanewline
\ \ \ \ \ \ \isacommand{show}\isamarkupfalse%
\ {\isachardoublequoteopen}{\isacharbraceleft}G{\isacharcomma}H{\isacharbraceright}\ {\isasymunion}\ S\ {\isasymin}\ {\isacharparenleft}extensionFin\ C{\isacharparenright}{\isachardoublequoteclose}\ \isanewline
\ \ \ \ \ \ \ \ \isacommand{using}\isamarkupfalse%
\ assms{\isacharparenleft}{\isadigit{1}}{\isacharparenright}\ assms{\isacharparenleft}{\isadigit{2}}{\isacharparenright}\ assms{\isacharparenleft}{\isadigit{3}}{\isacharparenright}\ {\isacartoucheopen}Con\ F\ G\ H{\isacartoucheclose}\ {\isacartoucheopen}F\ {\isasymin}\ S{\isacartoucheclose}\ \isacommand{by}\isamarkupfalse%
\ {\isacharparenleft}simp\ only{\isacharcolon}\ ex{\isadigit{3}}{\isacharunderscore}pcp{\isacharunderscore}SinE{\isacharunderscore}CON{\isacharparenright}\isanewline
\ \ \ \ \isacommand{qed}\isamarkupfalse%
\isanewline
\ \ \isacommand{qed}\isamarkupfalse%
\isanewline
\ \ \isacommand{have}\isamarkupfalse%
\ C{\isadigit{4}}{\isacharcolon}{\isachardoublequoteopen}{\isasymforall}F\ G\ H{\isachardot}\ Dis\ F\ G\ H\ {\isasymlongrightarrow}\ F\ {\isasymin}\ S\ {\isasymlongrightarrow}\ {\isacharbraceleft}G{\isacharbraceright}\ {\isasymunion}\ S\ {\isasymin}\ {\isacharparenleft}extensionFin\ C{\isacharparenright}\ {\isasymor}\ {\isacharbraceleft}H{\isacharbraceright}\ {\isasymunion}\ S\ {\isasymin}\ {\isacharparenleft}extensionFin\ C{\isacharparenright}{\isachardoublequoteclose}\isanewline
\ \ \isacommand{proof}\isamarkupfalse%
\ {\isacharparenleft}rule\ allI{\isacharparenright}{\isacharplus}\isanewline
\ \ \ \ \isacommand{fix}\isamarkupfalse%
\ F\ G\ H\isanewline
\ \ \ \ \isacommand{show}\isamarkupfalse%
\ {\isachardoublequoteopen}Dis\ F\ G\ H\ {\isasymlongrightarrow}\ F\ {\isasymin}\ S\ {\isasymlongrightarrow}\ {\isacharbraceleft}G{\isacharbraceright}\ {\isasymunion}\ S\ {\isasymin}\ {\isacharparenleft}extensionFin\ C{\isacharparenright}\ {\isasymor}\ {\isacharbraceleft}H{\isacharbraceright}\ {\isasymunion}\ S\ {\isasymin}\ {\isacharparenleft}extensionFin\ C{\isacharparenright}{\isachardoublequoteclose}\isanewline
\ \ \ \ \isacommand{proof}\isamarkupfalse%
\ {\isacharparenleft}rule\ impI{\isacharparenright}{\isacharplus}\isanewline
\ \ \ \ \ \ \isacommand{assume}\isamarkupfalse%
\ {\isachardoublequoteopen}Dis\ F\ G\ H{\isachardoublequoteclose}\isanewline
\ \ \ \ \ \ \isacommand{assume}\isamarkupfalse%
\ {\isachardoublequoteopen}F\ {\isasymin}\ S{\isachardoublequoteclose}\ \isanewline
\ \ \ \ \ \ \isacommand{show}\isamarkupfalse%
\ {\isachardoublequoteopen}{\isacharbraceleft}G{\isacharbraceright}\ {\isasymunion}\ S\ {\isasymin}\ {\isacharparenleft}extensionFin\ C{\isacharparenright}\ {\isasymor}\ {\isacharbraceleft}H{\isacharbraceright}\ {\isasymunion}\ S\ {\isasymin}\ {\isacharparenleft}extensionFin\ C{\isacharparenright}{\isachardoublequoteclose}\isanewline
\ \ \ \ \ \ \ \ \isacommand{using}\isamarkupfalse%
\ assms{\isacharparenleft}{\isadigit{1}}{\isacharparenright}\ assms{\isacharparenleft}{\isadigit{2}}{\isacharparenright}\ assms{\isacharparenleft}{\isadigit{3}}{\isacharparenright}\ {\isacartoucheopen}Dis\ F\ G\ H{\isacartoucheclose}\ {\isacartoucheopen}F\ {\isasymin}\ S{\isacartoucheclose}\ \isacommand{by}\isamarkupfalse%
\ {\isacharparenleft}rule\ ex{\isadigit{3}}{\isacharunderscore}pcp{\isacharunderscore}SinE{\isacharunderscore}DIS{\isacharparenright}\isanewline
\ \ \ \ \isacommand{qed}\isamarkupfalse%
\isanewline
\ \ \isacommand{qed}\isamarkupfalse%
\isanewline
\ \ \isacommand{show}\isamarkupfalse%
\ {\isacharquery}thesis\isanewline
\ \ \ \ \isacommand{using}\isamarkupfalse%
\ C{\isadigit{1}}\ C{\isadigit{2}}\ C{\isadigit{3}}\ C{\isadigit{4}}\ \isacommand{by}\isamarkupfalse%
\ {\isacharparenleft}iprover\ intro{\isacharcolon}\ conjI{\isacharparenright}\isanewline
\isacommand{qed}\isamarkupfalse%
%
\endisatagproof
{\isafoldproof}%
%
\isadelimproof
%
\endisadelimproof
%
\begin{isamarkuptext}%
En conclusión, la prueba detallada completa en Isabelle que demuestra que la extensión \isa{C{\isacharprime}} 
  verifica la propiedad de consistencia proposicional dada una colección \isa{C} que también la
  verifique y sea cerrada bajo subconjuntos es la siguiente.%
\end{isamarkuptext}\isamarkuptrue%
\isacommand{lemma}\isamarkupfalse%
\ ex{\isadigit{3}}{\isacharunderscore}pcp{\isacharcolon}\isanewline
\ \ \isakeyword{assumes}\ {\isachardoublequoteopen}pcp\ C{\isachardoublequoteclose}\isanewline
\ \ \ \ \ \ \ \ \ \ {\isachardoublequoteopen}subset{\isacharunderscore}closed\ C{\isachardoublequoteclose}\isanewline
\ \ \ \ \ \ \ \ \isakeyword{shows}\ {\isachardoublequoteopen}pcp\ {\isacharparenleft}extensionFin\ C{\isacharparenright}{\isachardoublequoteclose}\isanewline
%
\isadelimproof
\ \ %
\endisadelimproof
%
\isatagproof
\isacommand{unfolding}\isamarkupfalse%
\ pcp{\isacharunderscore}alt\isanewline
\isacommand{proof}\isamarkupfalse%
\ {\isacharparenleft}rule\ ballI{\isacharparenright}\isanewline
\ \ \isacommand{have}\isamarkupfalse%
\ PCP{\isacharcolon}{\isachardoublequoteopen}{\isasymforall}S\ {\isasymin}\ C{\isachardot}\isanewline
\ \ \ \ {\isasymbottom}\ {\isasymnotin}\ S\isanewline
\ \ \ \ {\isasymand}\ {\isacharparenleft}{\isasymforall}k{\isachardot}\ Atom\ k\ {\isasymin}\ S\ {\isasymlongrightarrow}\ \isactrlbold {\isasymnot}\ {\isacharparenleft}Atom\ k{\isacharparenright}\ {\isasymin}\ S\ {\isasymlongrightarrow}\ False{\isacharparenright}\isanewline
\ \ \ \ {\isasymand}\ {\isacharparenleft}{\isasymforall}F\ G\ H{\isachardot}\ Con\ F\ G\ H\ {\isasymlongrightarrow}\ F\ {\isasymin}\ S\ {\isasymlongrightarrow}\ {\isacharbraceleft}G{\isacharcomma}H{\isacharbraceright}\ {\isasymunion}\ S\ {\isasymin}\ C{\isacharparenright}\isanewline
\ \ \ \ {\isasymand}\ {\isacharparenleft}{\isasymforall}F\ G\ H{\isachardot}\ Dis\ F\ G\ H\ {\isasymlongrightarrow}\ F\ {\isasymin}\ S\ {\isasymlongrightarrow}\ {\isacharbraceleft}G{\isacharbraceright}\ {\isasymunion}\ S\ {\isasymin}\ C\ {\isasymor}\ {\isacharbraceleft}H{\isacharbraceright}\ {\isasymunion}\ S\ {\isasymin}\ C{\isacharparenright}{\isachardoublequoteclose}\isanewline
\ \ \ \ \isacommand{using}\isamarkupfalse%
\ assms{\isacharparenleft}{\isadigit{1}}{\isacharparenright}\ \isacommand{by}\isamarkupfalse%
\ {\isacharparenleft}rule\ pcp{\isacharunderscore}alt{\isadigit{1}}{\isacharparenright}\isanewline
\ \ \isacommand{fix}\isamarkupfalse%
\ S\isanewline
\ \ \isacommand{assume}\isamarkupfalse%
\ {\isachardoublequoteopen}S\ {\isasymin}\ {\isacharparenleft}extensionFin\ C{\isacharparenright}{\isachardoublequoteclose}\isanewline
\ \ \isacommand{then}\isamarkupfalse%
\ \isacommand{have}\isamarkupfalse%
\ {\isachardoublequoteopen}S\ {\isasymin}\ C\ {\isasymor}\ S\ {\isasymin}\ {\isacharparenleft}extF\ C{\isacharparenright}{\isachardoublequoteclose}\isanewline
\ \ \ \ \isacommand{unfolding}\isamarkupfalse%
\ extensionFin\ \isacommand{by}\isamarkupfalse%
\ {\isacharparenleft}simp\ only{\isacharcolon}\ Un{\isacharunderscore}iff{\isacharparenright}\isanewline
\ \ \isacommand{thus}\isamarkupfalse%
\ {\isachardoublequoteopen}{\isasymbottom}\ {\isasymnotin}\ S\ {\isasymand}\isanewline
\ \ \ \ \ \ \ \ \ {\isacharparenleft}{\isasymforall}k{\isachardot}\ Atom\ k\ {\isasymin}\ S\ {\isasymlongrightarrow}\ \isactrlbold {\isasymnot}\ {\isacharparenleft}Atom\ k{\isacharparenright}\ {\isasymin}\ S\ {\isasymlongrightarrow}\ False{\isacharparenright}\ {\isasymand}\isanewline
\ \ \ \ \ \ \ \ \ {\isacharparenleft}{\isasymforall}F\ G\ H{\isachardot}\ Con\ F\ G\ H\ {\isasymlongrightarrow}\ F\ {\isasymin}\ S\ {\isasymlongrightarrow}\ {\isacharbraceleft}G{\isacharcomma}\ H{\isacharbraceright}\ {\isasymunion}\ S\ {\isasymin}\ {\isacharparenleft}extensionFin\ C{\isacharparenright}{\isacharparenright}\ {\isasymand}\isanewline
\ \ \ \ \ \ \ \ \ {\isacharparenleft}{\isasymforall}F\ G\ H{\isachardot}\ Dis\ F\ G\ H\ {\isasymlongrightarrow}\ F\ {\isasymin}\ S\ {\isasymlongrightarrow}\ {\isacharbraceleft}G{\isacharbraceright}\ {\isasymunion}\ S\ {\isasymin}\ {\isacharparenleft}extensionFin\ C{\isacharparenright}\ {\isasymor}\ {\isacharbraceleft}H{\isacharbraceright}\ {\isasymunion}\ S\ {\isasymin}\ {\isacharparenleft}extensionFin\ C{\isacharparenright}{\isacharparenright}{\isachardoublequoteclose}\isanewline
\ \ \isacommand{proof}\isamarkupfalse%
\ {\isacharparenleft}rule\ disjE{\isacharparenright}\isanewline
\ \ \ \ \isacommand{assume}\isamarkupfalse%
\ {\isachardoublequoteopen}S\ {\isasymin}\ C{\isachardoublequoteclose}\isanewline
\ \ \ \ \isacommand{show}\isamarkupfalse%
\ {\isacharquery}thesis\isanewline
\ \ \ \ \ \ \isacommand{using}\isamarkupfalse%
\ assms\ {\isacartoucheopen}S\ {\isasymin}\ C{\isacartoucheclose}\ \isacommand{by}\isamarkupfalse%
\ {\isacharparenleft}rule\ ex{\isadigit{3}}{\isacharunderscore}pcp{\isacharunderscore}SinC{\isacharparenright}\isanewline
\ \ \isacommand{next}\isamarkupfalse%
\isanewline
\ \ \ \ \isacommand{assume}\isamarkupfalse%
\ {\isachardoublequoteopen}S\ {\isasymin}\ {\isacharparenleft}extF\ C{\isacharparenright}{\isachardoublequoteclose}\isanewline
\ \ \ \ \isacommand{show}\isamarkupfalse%
\ {\isacharquery}thesis\isanewline
\ \ \ \ \ \ \isacommand{using}\isamarkupfalse%
\ assms\ {\isacartoucheopen}S\ {\isasymin}\ {\isacharparenleft}extF\ C{\isacharparenright}{\isacartoucheclose}\ \isacommand{by}\isamarkupfalse%
\ {\isacharparenleft}rule\ ex{\isadigit{3}}{\isacharunderscore}pcp{\isacharunderscore}SinE{\isacharparenright}\isanewline
\ \ \isacommand{qed}\isamarkupfalse%
\isanewline
\isacommand{qed}\isamarkupfalse%
%
\endisatagproof
{\isafoldproof}%
%
\isadelimproof
%
\endisadelimproof
%
\begin{isamarkuptext}%
Por último, podemos dar la prueba completa del lema \isa{{\isadigit{3}}{\isachardot}{\isadigit{0}}{\isachardot}{\isadigit{5}}} en Isabelle.%
\end{isamarkuptext}\isamarkuptrue%
\isacommand{lemma}\isamarkupfalse%
\ ex{\isadigit{3}}{\isacharcolon}\isanewline
\ \ \isakeyword{assumes}\ {\isachardoublequoteopen}pcp\ C{\isachardoublequoteclose}\isanewline
\ \ \ \ \ \ \ \ \ \ {\isachardoublequoteopen}subset{\isacharunderscore}closed\ C{\isachardoublequoteclose}\isanewline
\ \ \isakeyword{shows}\ {\isachardoublequoteopen}{\isasymexists}C{\isacharprime}{\isachardot}\ C\ {\isasymsubseteq}\ C{\isacharprime}\ {\isasymand}\ pcp\ C{\isacharprime}\ {\isasymand}\ finite{\isacharunderscore}character\ C{\isacharprime}{\isachardoublequoteclose}\isanewline
%
\isadelimproof
%
\endisadelimproof
%
\isatagproof
\isacommand{proof}\isamarkupfalse%
\ {\isacharminus}\isanewline
\ \ \isacommand{let}\isamarkupfalse%
\ {\isacharquery}C{\isacharprime}{\isacharequal}{\isachardoublequoteopen}extensionFin\ C{\isachardoublequoteclose}\isanewline
\ \ \isacommand{have}\isamarkupfalse%
\ C{\isadigit{1}}{\isacharcolon}{\isachardoublequoteopen}C\ {\isasymsubseteq}\ {\isacharquery}C{\isacharprime}{\isachardoublequoteclose}\isanewline
\ \ \ \ \isacommand{unfolding}\isamarkupfalse%
\ extensionFin\ \isacommand{by}\isamarkupfalse%
\ {\isacharparenleft}simp\ only{\isacharcolon}\ Un{\isacharunderscore}upper{\isadigit{1}}{\isacharparenright}\isanewline
\ \ \isacommand{have}\isamarkupfalse%
\ C{\isadigit{2}}{\isacharcolon}{\isachardoublequoteopen}finite{\isacharunderscore}character\ {\isacharparenleft}{\isacharquery}C{\isacharprime}{\isacharparenright}{\isachardoublequoteclose}\isanewline
\ \ \ \ \isacommand{using}\isamarkupfalse%
\ assms{\isacharparenleft}{\isadigit{2}}{\isacharparenright}\ \isacommand{by}\isamarkupfalse%
\ {\isacharparenleft}rule\ ex{\isadigit{3}}{\isacharunderscore}finite{\isacharunderscore}character{\isacharparenright}\isanewline
\ \ \isacommand{have}\isamarkupfalse%
\ C{\isadigit{3}}{\isacharcolon}{\isachardoublequoteopen}pcp\ {\isacharparenleft}{\isacharquery}C{\isacharprime}{\isacharparenright}{\isachardoublequoteclose}\isanewline
\ \ \ \ \isacommand{using}\isamarkupfalse%
\ assms\ \isacommand{by}\isamarkupfalse%
\ {\isacharparenleft}rule\ ex{\isadigit{3}}{\isacharunderscore}pcp{\isacharparenright}\isanewline
\ \ \isacommand{have}\isamarkupfalse%
\ {\isachardoublequoteopen}C\ {\isasymsubseteq}\ {\isacharquery}C{\isacharprime}\ {\isasymand}\ pcp\ {\isacharquery}C{\isacharprime}\ {\isasymand}\ finite{\isacharunderscore}character\ {\isacharquery}C{\isacharprime}{\isachardoublequoteclose}\isanewline
\ \ \ \ \isacommand{using}\isamarkupfalse%
\ C{\isadigit{1}}\ C{\isadigit{2}}\ C{\isadigit{3}}\ \isacommand{by}\isamarkupfalse%
\ {\isacharparenleft}iprover\ intro{\isacharcolon}\ conjI{\isacharparenright}\isanewline
\ \ \isacommand{thus}\isamarkupfalse%
\ {\isacharquery}thesis\isanewline
\ \ \ \ \isacommand{by}\isamarkupfalse%
\ {\isacharparenleft}rule\ exI{\isacharparenright}\isanewline
\isacommand{qed}\isamarkupfalse%
\isanewline
%
\endisatagproof
{\isafoldproof}%
%
\isadelimproof
%
\endisadelimproof
%
\isadelimtheory
%
\endisadelimtheory
%
\isatagtheory
%
\endisatagtheory
{\isafoldtheory}%
%
\isadelimtheory
%
\endisadelimtheory
%
\end{isabellebody}%
\endinput
%:%file=~/TFM/TFM/ColecScfc.thy%:%
%:%19=12%:%
%:%20=13%:%
%:%21=14%:%
%:%22=15%:%
%:%23=16%:%
%:%24=17%:%
%:%25=18%:%
%:%26=19%:%
%:%27=20%:%
%:%28=21%:%
%:%29=22%:%
%:%30=23%:%
%:%31=24%:%
%:%32=25%:%
%:%34=27%:%
%:%35=27%:%
%:%37=29%:%
%:%38=30%:%
%:%39=31%:%
%:%41=33%:%
%:%42=33%:%
%:%45=34%:%
%:%49=34%:%
%:%50=34%:%
%:%51=34%:%
%:%56=34%:%
%:%59=35%:%
%:%60=36%:%
%:%61=36%:%
%:%64=37%:%
%:%68=37%:%
%:%69=37%:%
%:%70=37%:%
%:%79=39%:%
%:%80=40%:%
%:%81=41%:%
%:%83=43%:%
%:%84=43%:%
%:%87=44%:%
%:%91=44%:%
%:%92=44%:%
%:%93=44%:%
%:%102=46%:%
%:%103=47%:%
%:%105=49%:%
%:%106=49%:%
%:%108=51%:%
%:%111=52%:%
%:%115=52%:%
%:%116=52%:%
%:%117=52%:%
%:%122=52%:%
%:%125=53%:%
%:%126=54%:%
%:%127=54%:%
%:%128=55%:%
%:%131=56%:%
%:%135=56%:%
%:%136=56%:%
%:%137=56%:%
%:%146=58%:%
%:%147=59%:%
%:%148=60%:%
%:%149=61%:%
%:%150=62%:%
%:%151=63%:%
%:%152=64%:%
%:%153=65%:%
%:%154=66%:%
%:%155=67%:%
%:%156=68%:%
%:%157=69%:%
%:%159=71%:%
%:%160=71%:%
%:%163=74%:%
%:%164=75%:%
%:%165=76%:%
%:%166=77%:%
%:%168=79%:%
%:%169=79%:%
%:%172=80%:%
%:%176=80%:%
%:%177=80%:%
%:%178=80%:%
%:%183=80%:%
%:%186=81%:%
%:%187=82%:%
%:%188=82%:%
%:%191=83%:%
%:%195=83%:%
%:%196=83%:%
%:%197=83%:%
%:%202=83%:%
%:%205=84%:%
%:%206=85%:%
%:%207=85%:%
%:%210=86%:%
%:%214=86%:%
%:%215=86%:%
%:%216=86%:%
%:%225=89%:%
%:%226=90%:%
%:%227=91%:%
%:%228=92%:%
%:%229=93%:%
%:%230=94%:%
%:%231=95%:%
%:%233=97%:%
%:%234=97%:%
%:%237=98%:%
%:%241=98%:%
%:%251=100%:%
%:%252=101%:%
%:%253=102%:%
%:%254=103%:%
%:%255=104%:%
%:%256=105%:%
%:%257=106%:%
%:%258=107%:%
%:%259=108%:%
%:%260=109%:%
%:%261=110%:%
%:%262=111%:%
%:%263=112%:%
%:%264=113%:%
%:%265=114%:%
%:%266=115%:%
%:%267=116%:%
%:%268=117%:%
%:%269=118%:%
%:%270=119%:%
%:%271=120%:%
%:%272=121%:%
%:%273=122%:%
%:%274=123%:%
%:%275=124%:%
%:%276=125%:%
%:%277=126%:%
%:%278=127%:%
%:%279=128%:%
%:%280=129%:%
%:%281=130%:%
%:%282=131%:%
%:%283=132%:%
%:%284=133%:%
%:%285=134%:%
%:%286=135%:%
%:%287=136%:%
%:%288=137%:%
%:%289=138%:%
%:%290=139%:%
%:%291=140%:%
%:%292=141%:%
%:%293=142%:%
%:%294=143%:%
%:%295=144%:%
%:%296=145%:%
%:%297=146%:%
%:%298=147%:%
%:%299=148%:%
%:%300=149%:%
%:%301=150%:%
%:%302=151%:%
%:%303=152%:%
%:%304=153%:%
%:%305=154%:%
%:%306=155%:%
%:%307=156%:%
%:%308=157%:%
%:%309=158%:%
%:%310=159%:%
%:%311=160%:%
%:%312=161%:%
%:%313=162%:%
%:%314=163%:%
%:%315=164%:%
%:%316=165%:%
%:%317=166%:%
%:%318=167%:%
%:%319=168%:%
%:%320=169%:%
%:%321=170%:%
%:%322=171%:%
%:%323=172%:%
%:%324=173%:%
%:%325=174%:%
%:%326=175%:%
%:%327=176%:%
%:%328=177%:%
%:%329=178%:%
%:%330=179%:%
%:%331=180%:%
%:%332=181%:%
%:%333=182%:%
%:%334=183%:%
%:%335=184%:%
%:%336=185%:%
%:%337=186%:%
%:%338=187%:%
%:%339=188%:%
%:%340=189%:%
%:%341=190%:%
%:%342=191%:%
%:%344=193%:%
%:%345=193%:%
%:%348=194%:%
%:%352=194%:%
%:%353=194%:%
%:%362=196%:%
%:%363=197%:%
%:%365=199%:%
%:%366=199%:%
%:%369=200%:%
%:%373=200%:%
%:%374=200%:%
%:%383=202%:%
%:%384=203%:%
%:%385=204%:%
%:%386=205%:%
%:%387=206%:%
%:%389=208%:%
%:%390=208%:%
%:%391=209%:%
%:%393=211%:%
%:%394=212%:%
%:%396=214%:%
%:%397=214%:%
%:%404=215%:%
%:%405=215%:%
%:%406=216%:%
%:%407=216%:%
%:%408=217%:%
%:%409=217%:%
%:%410=218%:%
%:%411=218%:%
%:%412=219%:%
%:%413=219%:%
%:%414=220%:%
%:%415=220%:%
%:%416=220%:%
%:%417=221%:%
%:%418=221%:%
%:%419=221%:%
%:%420=222%:%
%:%421=222%:%
%:%422=223%:%
%:%423=223%:%
%:%424=224%:%
%:%434=226%:%
%:%435=227%:%
%:%436=228%:%
%:%437=229%:%
%:%439=231%:%
%:%440=231%:%
%:%443=232%:%
%:%447=232%:%
%:%448=232%:%
%:%457=234%:%
%:%458=235%:%
%:%460=237%:%
%:%461=237%:%
%:%462=238%:%
%:%463=239%:%
%:%470=240%:%
%:%471=240%:%
%:%472=241%:%
%:%473=241%:%
%:%474=242%:%
%:%475=242%:%
%:%476=243%:%
%:%477=243%:%
%:%478=244%:%
%:%479=244%:%
%:%480=245%:%
%:%481=245%:%
%:%484=248%:%
%:%485=249%:%
%:%486=249%:%
%:%487=250%:%
%:%488=250%:%
%:%489=251%:%
%:%490=251%:%
%:%491=252%:%
%:%492=252%:%
%:%493=252%:%
%:%494=253%:%
%:%495=253%:%
%:%496=253%:%
%:%497=254%:%
%:%498=254%:%
%:%499=255%:%
%:%500=255%:%
%:%501=255%:%
%:%502=256%:%
%:%503=256%:%
%:%507=260%:%
%:%508=261%:%
%:%509=261%:%
%:%510=261%:%
%:%511=262%:%
%:%512=262%:%
%:%513=262%:%
%:%516=265%:%
%:%517=266%:%
%:%518=266%:%
%:%519=266%:%
%:%520=267%:%
%:%521=267%:%
%:%522=267%:%
%:%523=268%:%
%:%524=268%:%
%:%525=269%:%
%:%526=269%:%
%:%527=270%:%
%:%528=270%:%
%:%529=270%:%
%:%530=271%:%
%:%531=271%:%
%:%532=272%:%
%:%533=272%:%
%:%534=272%:%
%:%535=273%:%
%:%536=273%:%
%:%537=274%:%
%:%538=274%:%
%:%539=275%:%
%:%540=275%:%
%:%541=276%:%
%:%542=276%:%
%:%543=277%:%
%:%544=277%:%
%:%545=278%:%
%:%546=278%:%
%:%547=279%:%
%:%548=279%:%
%:%549=280%:%
%:%550=280%:%
%:%551=281%:%
%:%552=281%:%
%:%553=281%:%
%:%554=282%:%
%:%555=282%:%
%:%556=283%:%
%:%557=283%:%
%:%558=283%:%
%:%559=284%:%
%:%560=284%:%
%:%561=285%:%
%:%562=285%:%
%:%563=285%:%
%:%564=286%:%
%:%565=286%:%
%:%566=286%:%
%:%567=287%:%
%:%568=287%:%
%:%569=287%:%
%:%570=288%:%
%:%571=288%:%
%:%572=289%:%
%:%573=289%:%
%:%574=289%:%
%:%575=290%:%
%:%576=290%:%
%:%577=291%:%
%:%578=291%:%
%:%579=292%:%
%:%580=292%:%
%:%581=293%:%
%:%582=293%:%
%:%583=293%:%
%:%584=294%:%
%:%585=294%:%
%:%586=295%:%
%:%587=295%:%
%:%588=296%:%
%:%589=296%:%
%:%590=297%:%
%:%591=297%:%
%:%592=298%:%
%:%593=298%:%
%:%594=299%:%
%:%595=299%:%
%:%596=300%:%
%:%597=300%:%
%:%598=301%:%
%:%599=301%:%
%:%600=302%:%
%:%601=302%:%
%:%602=302%:%
%:%603=303%:%
%:%604=303%:%
%:%605=304%:%
%:%606=304%:%
%:%607=304%:%
%:%608=305%:%
%:%609=305%:%
%:%610=305%:%
%:%611=306%:%
%:%612=306%:%
%:%613=306%:%
%:%614=307%:%
%:%615=307%:%
%:%616=307%:%
%:%617=308%:%
%:%618=308%:%
%:%619=308%:%
%:%620=309%:%
%:%621=309%:%
%:%622=310%:%
%:%623=310%:%
%:%624=310%:%
%:%625=311%:%
%:%626=311%:%
%:%627=311%:%
%:%628=312%:%
%:%629=312%:%
%:%630=313%:%
%:%631=313%:%
%:%632=313%:%
%:%633=314%:%
%:%634=314%:%
%:%635=315%:%
%:%636=315%:%
%:%637=315%:%
%:%638=316%:%
%:%639=316%:%
%:%640=317%:%
%:%641=317%:%
%:%642=318%:%
%:%643=318%:%
%:%644=319%:%
%:%645=319%:%
%:%646=320%:%
%:%647=320%:%
%:%648=321%:%
%:%649=321%:%
%:%650=322%:%
%:%651=322%:%
%:%652=322%:%
%:%653=323%:%
%:%654=323%:%
%:%655=323%:%
%:%656=324%:%
%:%657=324%:%
%:%658=324%:%
%:%659=325%:%
%:%660=325%:%
%:%661=326%:%
%:%662=326%:%
%:%663=326%:%
%:%664=327%:%
%:%665=327%:%
%:%666=328%:%
%:%667=328%:%
%:%668=329%:%
%:%669=329%:%
%:%670=330%:%
%:%671=330%:%
%:%672=330%:%
%:%673=331%:%
%:%674=331%:%
%:%675=332%:%
%:%676=332%:%
%:%677=333%:%
%:%678=333%:%
%:%679=334%:%
%:%680=334%:%
%:%681=335%:%
%:%682=335%:%
%:%683=336%:%
%:%684=336%:%
%:%685=337%:%
%:%686=337%:%
%:%687=338%:%
%:%688=338%:%
%:%689=339%:%
%:%690=339%:%
%:%691=339%:%
%:%692=340%:%
%:%693=340%:%
%:%694=341%:%
%:%695=341%:%
%:%696=341%:%
%:%697=342%:%
%:%698=342%:%
%:%699=342%:%
%:%700=343%:%
%:%701=343%:%
%:%702=343%:%
%:%703=344%:%
%:%704=344%:%
%:%705=344%:%
%:%706=345%:%
%:%707=345%:%
%:%708=345%:%
%:%709=346%:%
%:%710=346%:%
%:%711=347%:%
%:%712=347%:%
%:%713=348%:%
%:%714=348%:%
%:%715=349%:%
%:%716=349%:%
%:%717=350%:%
%:%718=350%:%
%:%719=351%:%
%:%720=351%:%
%:%721=351%:%
%:%722=352%:%
%:%723=352%:%
%:%724=353%:%
%:%725=353%:%
%:%726=354%:%
%:%727=354%:%
%:%728=355%:%
%:%729=355%:%
%:%730=356%:%
%:%731=356%:%
%:%732=357%:%
%:%733=357%:%
%:%734=357%:%
%:%735=358%:%
%:%736=358%:%
%:%737=358%:%
%:%738=359%:%
%:%739=359%:%
%:%740=359%:%
%:%741=360%:%
%:%742=360%:%
%:%743=360%:%
%:%744=361%:%
%:%745=361%:%
%:%746=361%:%
%:%747=362%:%
%:%748=362%:%
%:%749=363%:%
%:%750=363%:%
%:%751=364%:%
%:%752=364%:%
%:%753=365%:%
%:%754=365%:%
%:%755=366%:%
%:%756=366%:%
%:%757=367%:%
%:%758=367%:%
%:%759=367%:%
%:%760=368%:%
%:%761=368%:%
%:%762=369%:%
%:%763=369%:%
%:%764=370%:%
%:%765=370%:%
%:%766=371%:%
%:%767=371%:%
%:%768=372%:%
%:%769=372%:%
%:%770=372%:%
%:%771=373%:%
%:%772=373%:%
%:%773=373%:%
%:%774=374%:%
%:%775=374%:%
%:%776=374%:%
%:%777=375%:%
%:%778=375%:%
%:%779=375%:%
%:%780=376%:%
%:%781=376%:%
%:%782=376%:%
%:%783=377%:%
%:%784=377%:%
%:%785=377%:%
%:%786=378%:%
%:%787=378%:%
%:%788=379%:%
%:%789=379%:%
%:%790=380%:%
%:%791=380%:%
%:%792=381%:%
%:%793=381%:%
%:%794=382%:%
%:%795=382%:%
%:%796=383%:%
%:%797=383%:%
%:%800=386%:%
%:%801=387%:%
%:%802=387%:%
%:%803=387%:%
%:%804=388%:%
%:%805=388%:%
%:%806=389%:%
%:%807=389%:%
%:%808=390%:%
%:%818=392%:%
%:%820=394%:%
%:%821=394%:%
%:%822=395%:%
%:%823=396%:%
%:%826=397%:%
%:%830=397%:%
%:%831=397%:%
%:%832=398%:%
%:%833=398%:%
%:%834=399%:%
%:%835=399%:%
%:%836=400%:%
%:%837=400%:%
%:%838=401%:%
%:%839=401%:%
%:%840=401%:%
%:%841=402%:%
%:%842=402%:%
%:%843=402%:%
%:%844=403%:%
%:%845=403%:%
%:%846=404%:%
%:%847=404%:%
%:%848=404%:%
%:%849=405%:%
%:%850=405%:%
%:%851=406%:%
%:%852=406%:%
%:%853=407%:%
%:%854=407%:%
%:%855=408%:%
%:%856=408%:%
%:%857=409%:%
%:%858=409%:%
%:%859=409%:%
%:%860=410%:%
%:%861=410%:%
%:%862=410%:%
%:%863=411%:%
%:%864=411%:%
%:%865=411%:%
%:%866=412%:%
%:%867=412%:%
%:%868=412%:%
%:%869=413%:%
%:%870=413%:%
%:%871=414%:%
%:%872=414%:%
%:%873=414%:%
%:%874=415%:%
%:%875=415%:%
%:%876=416%:%
%:%886=418%:%
%:%888=420%:%
%:%889=420%:%
%:%890=421%:%
%:%891=422%:%
%:%898=423%:%
%:%899=423%:%
%:%900=424%:%
%:%901=424%:%
%:%902=425%:%
%:%903=425%:%
%:%904=426%:%
%:%905=426%:%
%:%906=427%:%
%:%907=427%:%
%:%908=427%:%
%:%909=428%:%
%:%910=428%:%
%:%911=429%:%
%:%912=429%:%
%:%913=429%:%
%:%914=430%:%
%:%915=430%:%
%:%916=431%:%
%:%917=431%:%
%:%918=431%:%
%:%919=432%:%
%:%920=432%:%
%:%921=433%:%
%:%922=433%:%
%:%923=434%:%
%:%933=436%:%
%:%934=437%:%
%:%935=438%:%
%:%936=439%:%
%:%937=440%:%
%:%938=441%:%
%:%939=442%:%
%:%941=444%:%
%:%942=444%:%
%:%943=445%:%
%:%944=446%:%
%:%947=447%:%
%:%951=447%:%
%:%961=449%:%
%:%962=450%:%
%:%963=451%:%
%:%964=452%:%
%:%965=453%:%
%:%966=454%:%
%:%967=455%:%
%:%968=456%:%
%:%969=457%:%
%:%970=458%:%
%:%971=459%:%
%:%972=460%:%
%:%973=461%:%
%:%974=462%:%
%:%975=463%:%
%:%976=464%:%
%:%977=465%:%
%:%978=466%:%
%:%979=467%:%
%:%980=468%:%
%:%981=469%:%
%:%982=470%:%
%:%983=471%:%
%:%984=472%:%
%:%985=473%:%
%:%986=474%:%
%:%987=475%:%
%:%988=476%:%
%:%989=477%:%
%:%991=479%:%
%:%992=479%:%
%:%993=480%:%
%:%994=481%:%
%:%997=482%:%
%:%1001=482%:%
%:%1002=482%:%
%:%1003=483%:%
%:%1004=483%:%
%:%1005=484%:%
%:%1006=484%:%
%:%1007=485%:%
%:%1008=485%:%
%:%1009=486%:%
%:%1010=486%:%
%:%1011=487%:%
%:%1012=487%:%
%:%1013=487%:%
%:%1014=487%:%
%:%1015=488%:%
%:%1016=488%:%
%:%1017=489%:%
%:%1018=489%:%
%:%1019=490%:%
%:%1020=490%:%
%:%1021=491%:%
%:%1022=491%:%
%:%1023=492%:%
%:%1024=492%:%
%:%1025=492%:%
%:%1026=493%:%
%:%1027=493%:%
%:%1028=493%:%
%:%1029=494%:%
%:%1030=494%:%
%:%1031=495%:%
%:%1032=495%:%
%:%1033=495%:%
%:%1034=496%:%
%:%1035=496%:%
%:%1036=497%:%
%:%1037=497%:%
%:%1038=497%:%
%:%1039=498%:%
%:%1040=498%:%
%:%1041=499%:%
%:%1042=499%:%
%:%1043=499%:%
%:%1044=500%:%
%:%1045=500%:%
%:%1046=501%:%
%:%1047=501%:%
%:%1048=502%:%
%:%1049=502%:%
%:%1050=502%:%
%:%1051=503%:%
%:%1052=503%:%
%:%1053=504%:%
%:%1054=504%:%
%:%1055=504%:%
%:%1056=505%:%
%:%1066=507%:%
%:%1068=509%:%
%:%1069=509%:%
%:%1070=510%:%
%:%1071=511%:%
%:%1074=512%:%
%:%1078=512%:%
%:%1079=512%:%
%:%1080=513%:%
%:%1081=513%:%
%:%1082=514%:%
%:%1083=514%:%
%:%1084=515%:%
%:%1085=515%:%
%:%1086=516%:%
%:%1087=516%:%
%:%1088=516%:%
%:%1089=517%:%
%:%1090=517%:%
%:%1091=517%:%
%:%1092=518%:%
%:%1093=518%:%
%:%1094=518%:%
%:%1095=518%:%
%:%1096=519%:%
%:%1097=519%:%
%:%1098=519%:%
%:%1099=519%:%
%:%1100=520%:%
%:%1101=520%:%
%:%1102=520%:%
%:%1103=520%:%
%:%1104=520%:%
%:%1105=521%:%
%:%1115=523%:%
%:%1116=524%:%
%:%1117=525%:%
%:%1118=526%:%
%:%1119=527%:%
%:%1120=528%:%
%:%1121=529%:%
%:%1122=530%:%
%:%1123=531%:%
%:%1124=532%:%
%:%1125=533%:%
%:%1126=534%:%
%:%1127=535%:%
%:%1128=536%:%
%:%1129=537%:%
%:%1130=538%:%
%:%1131=539%:%
%:%1132=540%:%
%:%1133=541%:%
%:%1134=542%:%
%:%1135=543%:%
%:%1136=544%:%
%:%1137=545%:%
%:%1138=546%:%
%:%1139=547%:%
%:%1140=548%:%
%:%1141=549%:%
%:%1142=550%:%
%:%1143=551%:%
%:%1144=552%:%
%:%1145=553%:%
%:%1146=554%:%
%:%1147=555%:%
%:%1148=556%:%
%:%1149=557%:%
%:%1150=558%:%
%:%1151=559%:%
%:%1152=560%:%
%:%1153=561%:%
%:%1154=562%:%
%:%1155=563%:%
%:%1156=564%:%
%:%1157=565%:%
%:%1158=566%:%
%:%1159=567%:%
%:%1160=568%:%
%:%1161=569%:%
%:%1162=570%:%
%:%1163=571%:%
%:%1164=572%:%
%:%1165=573%:%
%:%1166=574%:%
%:%1167=575%:%
%:%1168=576%:%
%:%1169=577%:%
%:%1170=578%:%
%:%1171=579%:%
%:%1172=580%:%
%:%1173=581%:%
%:%1174=582%:%
%:%1175=583%:%
%:%1176=584%:%
%:%1177=585%:%
%:%1178=586%:%
%:%1179=587%:%
%:%1180=588%:%
%:%1181=589%:%
%:%1182=590%:%
%:%1183=591%:%
%:%1184=592%:%
%:%1185=593%:%
%:%1186=594%:%
%:%1187=595%:%
%:%1188=596%:%
%:%1189=597%:%
%:%1190=598%:%
%:%1191=599%:%
%:%1192=600%:%
%:%1193=601%:%
%:%1194=602%:%
%:%1195=603%:%
%:%1196=604%:%
%:%1197=605%:%
%:%1198=606%:%
%:%1199=607%:%
%:%1200=608%:%
%:%1201=609%:%
%:%1202=610%:%
%:%1203=611%:%
%:%1204=612%:%
%:%1205=613%:%
%:%1206=614%:%
%:%1207=615%:%
%:%1208=616%:%
%:%1209=617%:%
%:%1210=618%:%
%:%1211=619%:%
%:%1212=620%:%
%:%1213=621%:%
%:%1214=622%:%
%:%1215=623%:%
%:%1216=624%:%
%:%1217=625%:%
%:%1218=626%:%
%:%1219=627%:%
%:%1220=628%:%
%:%1221=629%:%
%:%1222=630%:%
%:%1223=631%:%
%:%1224=632%:%
%:%1225=633%:%
%:%1226=634%:%
%:%1227=635%:%
%:%1228=636%:%
%:%1229=637%:%
%:%1230=638%:%
%:%1231=639%:%
%:%1232=640%:%
%:%1233=641%:%
%:%1234=642%:%
%:%1235=643%:%
%:%1236=644%:%
%:%1237=645%:%
%:%1238=646%:%
%:%1239=647%:%
%:%1240=648%:%
%:%1241=649%:%
%:%1242=650%:%
%:%1243=651%:%
%:%1244=652%:%
%:%1245=653%:%
%:%1246=654%:%
%:%1247=655%:%
%:%1248=656%:%
%:%1249=657%:%
%:%1250=658%:%
%:%1251=659%:%
%:%1252=660%:%
%:%1253=661%:%
%:%1254=662%:%
%:%1255=663%:%
%:%1256=664%:%
%:%1257=665%:%
%:%1258=666%:%
%:%1259=667%:%
%:%1260=668%:%
%:%1261=669%:%
%:%1262=670%:%
%:%1263=671%:%
%:%1264=672%:%
%:%1265=673%:%
%:%1266=674%:%
%:%1267=675%:%
%:%1268=676%:%
%:%1269=677%:%
%:%1270=678%:%
%:%1271=679%:%
%:%1272=680%:%
%:%1273=681%:%
%:%1274=682%:%
%:%1275=683%:%
%:%1276=684%:%
%:%1277=685%:%
%:%1278=686%:%
%:%1279=687%:%
%:%1280=688%:%
%:%1281=689%:%
%:%1282=690%:%
%:%1283=691%:%
%:%1284=692%:%
%:%1285=693%:%
%:%1286=694%:%
%:%1287=695%:%
%:%1288=696%:%
%:%1289=697%:%
%:%1290=698%:%
%:%1291=699%:%
%:%1292=700%:%
%:%1293=701%:%
%:%1294=702%:%
%:%1295=703%:%
%:%1296=704%:%
%:%1297=705%:%
%:%1298=706%:%
%:%1299=707%:%
%:%1300=708%:%
%:%1301=709%:%
%:%1302=710%:%
%:%1303=711%:%
%:%1304=712%:%
%:%1305=713%:%
%:%1306=714%:%
%:%1307=715%:%
%:%1308=716%:%
%:%1309=717%:%
%:%1310=718%:%
%:%1311=719%:%
%:%1312=720%:%
%:%1313=721%:%
%:%1314=722%:%
%:%1315=723%:%
%:%1316=724%:%
%:%1317=725%:%
%:%1318=726%:%
%:%1319=727%:%
%:%1320=728%:%
%:%1321=729%:%
%:%1322=730%:%
%:%1323=731%:%
%:%1324=732%:%
%:%1325=733%:%
%:%1326=734%:%
%:%1327=735%:%
%:%1328=736%:%
%:%1329=737%:%
%:%1330=738%:%
%:%1331=739%:%
%:%1332=740%:%
%:%1333=741%:%
%:%1334=742%:%
%:%1335=743%:%
%:%1336=744%:%
%:%1337=745%:%
%:%1338=746%:%
%:%1339=747%:%
%:%1340=748%:%
%:%1341=749%:%
%:%1342=750%:%
%:%1343=751%:%
%:%1344=752%:%
%:%1346=754%:%
%:%1347=754%:%
%:%1348=755%:%
%:%1349=756%:%
%:%1350=757%:%
%:%1351=757%:%
%:%1352=758%:%
%:%1354=760%:%
%:%1355=761%:%
%:%1356=762%:%
%:%1358=764%:%
%:%1359=764%:%
%:%1360=765%:%
%:%1361=766%:%
%:%1368=767%:%
%:%1369=767%:%
%:%1370=768%:%
%:%1371=768%:%
%:%1372=769%:%
%:%1373=769%:%
%:%1374=770%:%
%:%1375=770%:%
%:%1376=771%:%
%:%1377=771%:%
%:%1378=772%:%
%:%1379=772%:%
%:%1380=773%:%
%:%1381=773%:%
%:%1382=774%:%
%:%1383=774%:%
%:%1384=775%:%
%:%1385=775%:%
%:%1386=776%:%
%:%1387=776%:%
%:%1388=777%:%
%:%1389=777%:%
%:%1390=778%:%
%:%1391=778%:%
%:%1392=779%:%
%:%1393=779%:%
%:%1394=780%:%
%:%1395=780%:%
%:%1396=781%:%
%:%1397=781%:%
%:%1398=781%:%
%:%1399=782%:%
%:%1400=782%:%
%:%1401=783%:%
%:%1402=783%:%
%:%1403=784%:%
%:%1404=784%:%
%:%1405=785%:%
%:%1406=785%:%
%:%1407=786%:%
%:%1408=786%:%
%:%1409=786%:%
%:%1410=787%:%
%:%1411=787%:%
%:%1412=787%:%
%:%1413=788%:%
%:%1414=788%:%
%:%1415=788%:%
%:%1416=789%:%
%:%1417=789%:%
%:%1418=789%:%
%:%1419=790%:%
%:%1420=790%:%
%:%1421=790%:%
%:%1422=791%:%
%:%1423=791%:%
%:%1424=792%:%
%:%1425=792%:%
%:%1426=793%:%
%:%1427=793%:%
%:%1428=794%:%
%:%1429=794%:%
%:%1430=795%:%
%:%1431=795%:%
%:%1432=795%:%
%:%1433=796%:%
%:%1434=796%:%
%:%1435=796%:%
%:%1436=797%:%
%:%1437=797%:%
%:%1438=797%:%
%:%1439=798%:%
%:%1440=798%:%
%:%1441=798%:%
%:%1442=799%:%
%:%1443=799%:%
%:%1444=799%:%
%:%1445=800%:%
%:%1446=800%:%
%:%1447=800%:%
%:%1448=801%:%
%:%1449=801%:%
%:%1450=802%:%
%:%1451=802%:%
%:%1452=803%:%
%:%1453=803%:%
%:%1454=804%:%
%:%1455=804%:%
%:%1456=805%:%
%:%1457=805%:%
%:%1458=806%:%
%:%1459=806%:%
%:%1460=807%:%
%:%1461=807%:%
%:%1462=807%:%
%:%1463=808%:%
%:%1464=808%:%
%:%1465=809%:%
%:%1466=809%:%
%:%1467=810%:%
%:%1468=810%:%
%:%1469=811%:%
%:%1470=811%:%
%:%1471=812%:%
%:%1472=812%:%
%:%1473=813%:%
%:%1474=813%:%
%:%1475=814%:%
%:%1476=814%:%
%:%1477=815%:%
%:%1478=815%:%
%:%1479=816%:%
%:%1480=816%:%
%:%1481=817%:%
%:%1482=817%:%
%:%1483=817%:%
%:%1484=818%:%
%:%1485=818%:%
%:%1486=818%:%
%:%1487=819%:%
%:%1488=819%:%
%:%1489=819%:%
%:%1490=820%:%
%:%1491=820%:%
%:%1492=821%:%
%:%1493=821%:%
%:%1494=822%:%
%:%1495=822%:%
%:%1496=823%:%
%:%1497=823%:%
%:%1498=824%:%
%:%1499=824%:%
%:%1500=825%:%
%:%1501=825%:%
%:%1502=826%:%
%:%1503=826%:%
%:%1504=827%:%
%:%1505=827%:%
%:%1506=827%:%
%:%1507=828%:%
%:%1508=828%:%
%:%1509=828%:%
%:%1510=829%:%
%:%1511=829%:%
%:%1512=830%:%
%:%1513=830%:%
%:%1514=831%:%
%:%1515=831%:%
%:%1516=832%:%
%:%1517=832%:%
%:%1518=832%:%
%:%1519=833%:%
%:%1520=833%:%
%:%1521=834%:%
%:%1522=834%:%
%:%1523=834%:%
%:%1524=835%:%
%:%1525=835%:%
%:%1526=836%:%
%:%1527=836%:%
%:%1528=837%:%
%:%1529=837%:%
%:%1530=838%:%
%:%1531=838%:%
%:%1532=838%:%
%:%1533=839%:%
%:%1534=839%:%
%:%1535=840%:%
%:%1536=840%:%
%:%1537=841%:%
%:%1538=841%:%
%:%1539=842%:%
%:%1540=842%:%
%:%1541=843%:%
%:%1542=843%:%
%:%1543=844%:%
%:%1553=846%:%
%:%1554=847%:%
%:%1555=848%:%
%:%1556=849%:%
%:%1557=850%:%
%:%1558=851%:%
%:%1559=852%:%
%:%1560=853%:%
%:%1561=854%:%
%:%1562=855%:%
%:%1563=856%:%
%:%1564=857%:%
%:%1565=858%:%
%:%1566=859%:%
%:%1567=860%:%
%:%1568=861%:%
%:%1569=862%:%
%:%1570=863%:%
%:%1571=864%:%
%:%1573=866%:%
%:%1574=866%:%
%:%1575=867%:%
%:%1576=868%:%
%:%1577=869%:%
%:%1578=870%:%
%:%1581=873%:%
%:%1588=874%:%
%:%1589=874%:%
%:%1590=875%:%
%:%1591=875%:%
%:%1595=879%:%
%:%1596=880%:%
%:%1597=880%:%
%:%1598=880%:%
%:%1599=881%:%
%:%1600=881%:%
%:%1603=884%:%
%:%1604=885%:%
%:%1605=885%:%
%:%1606=885%:%
%:%1607=886%:%
%:%1608=886%:%
%:%1609=886%:%
%:%1610=887%:%
%:%1611=887%:%
%:%1612=888%:%
%:%1613=888%:%
%:%1614=889%:%
%:%1615=889%:%
%:%1616=889%:%
%:%1617=890%:%
%:%1618=890%:%
%:%1619=891%:%
%:%1620=891%:%
%:%1621=891%:%
%:%1622=892%:%
%:%1623=892%:%
%:%1624=893%:%
%:%1625=893%:%
%:%1626=894%:%
%:%1627=894%:%
%:%1628=895%:%
%:%1629=895%:%
%:%1630=896%:%
%:%1631=896%:%
%:%1632=897%:%
%:%1633=897%:%
%:%1634=898%:%
%:%1635=898%:%
%:%1636=899%:%
%:%1637=899%:%
%:%1638=900%:%
%:%1639=900%:%
%:%1640=900%:%
%:%1641=901%:%
%:%1642=901%:%
%:%1643=901%:%
%:%1644=902%:%
%:%1645=902%:%
%:%1646=902%:%
%:%1647=903%:%
%:%1648=903%:%
%:%1649=903%:%
%:%1650=904%:%
%:%1651=904%:%
%:%1652=904%:%
%:%1653=905%:%
%:%1654=905%:%
%:%1655=906%:%
%:%1656=906%:%
%:%1657=906%:%
%:%1658=907%:%
%:%1659=907%:%
%:%1660=908%:%
%:%1661=908%:%
%:%1662=909%:%
%:%1663=909%:%
%:%1664=910%:%
%:%1665=910%:%
%:%1666=910%:%
%:%1667=911%:%
%:%1668=911%:%
%:%1669=912%:%
%:%1670=912%:%
%:%1671=913%:%
%:%1672=913%:%
%:%1673=914%:%
%:%1674=914%:%
%:%1675=915%:%
%:%1676=915%:%
%:%1677=916%:%
%:%1678=916%:%
%:%1679=917%:%
%:%1680=917%:%
%:%1681=918%:%
%:%1682=918%:%
%:%1683=919%:%
%:%1684=919%:%
%:%1685=919%:%
%:%1686=920%:%
%:%1687=920%:%
%:%1688=920%:%
%:%1689=921%:%
%:%1690=921%:%
%:%1691=921%:%
%:%1692=922%:%
%:%1693=922%:%
%:%1694=922%:%
%:%1695=923%:%
%:%1696=923%:%
%:%1697=923%:%
%:%1698=924%:%
%:%1699=924%:%
%:%1700=925%:%
%:%1701=925%:%
%:%1702=926%:%
%:%1703=926%:%
%:%1704=927%:%
%:%1705=927%:%
%:%1706=927%:%
%:%1707=928%:%
%:%1708=928%:%
%:%1709=928%:%
%:%1710=929%:%
%:%1711=929%:%
%:%1712=930%:%
%:%1713=930%:%
%:%1714=931%:%
%:%1715=931%:%
%:%1716=932%:%
%:%1717=932%:%
%:%1718=933%:%
%:%1719=933%:%
%:%1720=933%:%
%:%1721=934%:%
%:%1722=934%:%
%:%1723=934%:%
%:%1724=935%:%
%:%1725=935%:%
%:%1726=936%:%
%:%1727=936%:%
%:%1728=937%:%
%:%1729=937%:%
%:%1730=938%:%
%:%1731=938%:%
%:%1732=939%:%
%:%1733=939%:%
%:%1734=940%:%
%:%1735=940%:%
%:%1738=943%:%
%:%1739=944%:%
%:%1740=944%:%
%:%1741=944%:%
%:%1742=945%:%
%:%1752=947%:%
%:%1753=948%:%
%:%1754=949%:%
%:%1755=950%:%
%:%1756=951%:%
%:%1757=952%:%
%:%1759=954%:%
%:%1760=954%:%
%:%1761=955%:%
%:%1762=956%:%
%:%1763=957%:%
%:%1764=958%:%
%:%1765=959%:%
%:%1766=960%:%
%:%1773=961%:%
%:%1774=961%:%
%:%1775=962%:%
%:%1776=962%:%
%:%1780=966%:%
%:%1781=967%:%
%:%1782=967%:%
%:%1783=967%:%
%:%1784=968%:%
%:%1785=968%:%
%:%1786=969%:%
%:%1787=969%:%
%:%1788=970%:%
%:%1789=970%:%
%:%1790=971%:%
%:%1791=971%:%
%:%1792=972%:%
%:%1793=972%:%
%:%1794=973%:%
%:%1795=973%:%
%:%1796=974%:%
%:%1797=974%:%
%:%1798=975%:%
%:%1799=975%:%
%:%1800=976%:%
%:%1801=976%:%
%:%1802=977%:%
%:%1803=977%:%
%:%1804=977%:%
%:%1805=977%:%
%:%1806=978%:%
%:%1807=978%:%
%:%1808=978%:%
%:%1809=979%:%
%:%1810=979%:%
%:%1811=979%:%
%:%1812=980%:%
%:%1813=980%:%
%:%1814=980%:%
%:%1815=981%:%
%:%1816=981%:%
%:%1817=981%:%
%:%1818=982%:%
%:%1819=982%:%
%:%1822=985%:%
%:%1823=986%:%
%:%1824=986%:%
%:%1825=986%:%
%:%1826=987%:%
%:%1827=987%:%
%:%1828=987%:%
%:%1829=988%:%
%:%1830=988%:%
%:%1831=989%:%
%:%1832=989%:%
%:%1833=989%:%
%:%1834=990%:%
%:%1835=990%:%
%:%1836=991%:%
%:%1837=991%:%
%:%1838=991%:%
%:%1839=992%:%
%:%1840=992%:%
%:%1841=992%:%
%:%1842=993%:%
%:%1843=993%:%
%:%1844=994%:%
%:%1845=994%:%
%:%1846=994%:%
%:%1847=995%:%
%:%1848=995%:%
%:%1849=996%:%
%:%1850=996%:%
%:%1851=997%:%
%:%1852=997%:%
%:%1853=998%:%
%:%1854=998%:%
%:%1855=999%:%
%:%1856=999%:%
%:%1857=1000%:%
%:%1858=1000%:%
%:%1859=1001%:%
%:%1860=1001%:%
%:%1861=1002%:%
%:%1862=1002%:%
%:%1863=1003%:%
%:%1864=1003%:%
%:%1865=1004%:%
%:%1866=1004%:%
%:%1867=1005%:%
%:%1868=1005%:%
%:%1869=1006%:%
%:%1870=1006%:%
%:%1871=1006%:%
%:%1872=1007%:%
%:%1873=1007%:%
%:%1874=1008%:%
%:%1875=1008%:%
%:%1876=1008%:%
%:%1877=1009%:%
%:%1878=1009%:%
%:%1879=1009%:%
%:%1880=1010%:%
%:%1881=1010%:%
%:%1882=1011%:%
%:%1883=1011%:%
%:%1884=1012%:%
%:%1885=1012%:%
%:%1886=1012%:%
%:%1887=1013%:%
%:%1888=1013%:%
%:%1889=1014%:%
%:%1890=1014%:%
%:%1891=1014%:%
%:%1892=1015%:%
%:%1893=1015%:%
%:%1894=1015%:%
%:%1895=1016%:%
%:%1896=1016%:%
%:%1897=1017%:%
%:%1898=1017%:%
%:%1899=1018%:%
%:%1900=1018%:%
%:%1901=1018%:%
%:%1902=1019%:%
%:%1903=1019%:%
%:%1904=1020%:%
%:%1905=1020%:%
%:%1906=1021%:%
%:%1907=1021%:%
%:%1908=1022%:%
%:%1909=1022%:%
%:%1910=1022%:%
%:%1911=1023%:%
%:%1912=1023%:%
%:%1913=1024%:%
%:%1914=1024%:%
%:%1915=1025%:%
%:%1916=1025%:%
%:%1917=1026%:%
%:%1918=1026%:%
%:%1919=1027%:%
%:%1920=1027%:%
%:%1921=1028%:%
%:%1922=1028%:%
%:%1923=1029%:%
%:%1924=1029%:%
%:%1925=1030%:%
%:%1926=1030%:%
%:%1927=1031%:%
%:%1928=1031%:%
%:%1929=1031%:%
%:%1930=1032%:%
%:%1931=1032%:%
%:%1932=1033%:%
%:%1933=1033%:%
%:%1934=1033%:%
%:%1935=1034%:%
%:%1936=1034%:%
%:%1937=1035%:%
%:%1938=1035%:%
%:%1939=1035%:%
%:%1940=1036%:%
%:%1941=1036%:%
%:%1942=1037%:%
%:%1943=1037%:%
%:%1944=1037%:%
%:%1945=1038%:%
%:%1946=1038%:%
%:%1947=1039%:%
%:%1948=1039%:%
%:%1949=1039%:%
%:%1950=1040%:%
%:%1951=1040%:%
%:%1952=1041%:%
%:%1953=1041%:%
%:%1954=1042%:%
%:%1955=1042%:%
%:%1956=1042%:%
%:%1957=1043%:%
%:%1958=1043%:%
%:%1959=1044%:%
%:%1960=1044%:%
%:%1961=1045%:%
%:%1962=1045%:%
%:%1963=1045%:%
%:%1964=1046%:%
%:%1965=1046%:%
%:%1966=1047%:%
%:%1967=1047%:%
%:%1968=1048%:%
%:%1969=1048%:%
%:%1970=1049%:%
%:%1971=1049%:%
%:%1972=1049%:%
%:%1973=1050%:%
%:%1974=1050%:%
%:%1975=1050%:%
%:%1976=1051%:%
%:%1977=1051%:%
%:%1978=1051%:%
%:%1979=1052%:%
%:%1980=1052%:%
%:%1981=1053%:%
%:%1982=1053%:%
%:%1983=1053%:%
%:%1984=1054%:%
%:%1985=1054%:%
%:%1986=1055%:%
%:%1987=1055%:%
%:%1988=1056%:%
%:%1989=1056%:%
%:%1990=1057%:%
%:%1991=1057%:%
%:%1992=1057%:%
%:%1993=1058%:%
%:%2003=1060%:%
%:%2004=1061%:%
%:%2005=1062%:%
%:%2006=1063%:%
%:%2007=1064%:%
%:%2008=1065%:%
%:%2009=1066%:%
%:%2010=1067%:%
%:%2011=1068%:%
%:%2012=1069%:%
%:%2013=1070%:%
%:%2014=1071%:%
%:%2016=1073%:%
%:%2017=1073%:%
%:%2018=1074%:%
%:%2019=1075%:%
%:%2020=1076%:%
%:%2021=1077%:%
%:%2022=1078%:%
%:%2023=1079%:%
%:%2024=1080%:%
%:%2025=1081%:%
%:%2026=1082%:%
%:%2027=1083%:%
%:%2028=1084%:%
%:%2035=1085%:%
%:%2036=1085%:%
%:%2037=1086%:%
%:%2038=1086%:%
%:%2039=1087%:%
%:%2040=1087%:%
%:%2041=1088%:%
%:%2042=1088%:%
%:%2043=1089%:%
%:%2044=1089%:%
%:%2045=1090%:%
%:%2046=1090%:%
%:%2047=1091%:%
%:%2048=1091%:%
%:%2049=1092%:%
%:%2050=1092%:%
%:%2051=1092%:%
%:%2052=1093%:%
%:%2053=1093%:%
%:%2054=1094%:%
%:%2055=1094%:%
%:%2056=1094%:%
%:%2057=1095%:%
%:%2058=1095%:%
%:%2059=1096%:%
%:%2060=1096%:%
%:%2061=1096%:%
%:%2062=1096%:%
%:%2063=1097%:%
%:%2064=1097%:%
%:%2065=1097%:%
%:%2066=1098%:%
%:%2067=1098%:%
%:%2068=1098%:%
%:%2069=1099%:%
%:%2070=1099%:%
%:%2071=1099%:%
%:%2072=1100%:%
%:%2073=1100%:%
%:%2074=1100%:%
%:%2075=1101%:%
%:%2076=1101%:%
%:%2077=1102%:%
%:%2078=1102%:%
%:%2079=1102%:%
%:%2080=1103%:%
%:%2081=1103%:%
%:%2084=1106%:%
%:%2085=1107%:%
%:%2086=1107%:%
%:%2087=1107%:%
%:%2088=1108%:%
%:%2089=1108%:%
%:%2090=1108%:%
%:%2093=1111%:%
%:%2094=1112%:%
%:%2095=1112%:%
%:%2096=1112%:%
%:%2097=1113%:%
%:%2098=1113%:%
%:%2099=1113%:%
%:%2100=1114%:%
%:%2101=1114%:%
%:%2102=1115%:%
%:%2103=1115%:%
%:%2104=1115%:%
%:%2105=1116%:%
%:%2106=1116%:%
%:%2107=1117%:%
%:%2108=1117%:%
%:%2109=1117%:%
%:%2110=1118%:%
%:%2111=1118%:%
%:%2112=1118%:%
%:%2113=1119%:%
%:%2114=1119%:%
%:%2115=1119%:%
%:%2116=1120%:%
%:%2117=1120%:%
%:%2118=1120%:%
%:%2119=1121%:%
%:%2120=1121%:%
%:%2121=1122%:%
%:%2122=1122%:%
%:%2123=1123%:%
%:%2124=1123%:%
%:%2125=1124%:%
%:%2126=1124%:%
%:%2127=1125%:%
%:%2128=1125%:%
%:%2129=1126%:%
%:%2130=1126%:%
%:%2131=1126%:%
%:%2132=1127%:%
%:%2133=1127%:%
%:%2134=1127%:%
%:%2135=1128%:%
%:%2136=1128%:%
%:%2137=1129%:%
%:%2138=1129%:%
%:%2139=1129%:%
%:%2140=1130%:%
%:%2141=1130%:%
%:%2142=1131%:%
%:%2143=1131%:%
%:%2144=1131%:%
%:%2145=1132%:%
%:%2146=1132%:%
%:%2147=1133%:%
%:%2148=1133%:%
%:%2149=1133%:%
%:%2150=1134%:%
%:%2151=1134%:%
%:%2152=1135%:%
%:%2153=1135%:%
%:%2154=1136%:%
%:%2155=1136%:%
%:%2156=1136%:%
%:%2157=1137%:%
%:%2158=1137%:%
%:%2159=1138%:%
%:%2160=1138%:%
%:%2161=1138%:%
%:%2162=1139%:%
%:%2163=1139%:%
%:%2164=1139%:%
%:%2165=1140%:%
%:%2166=1140%:%
%:%2167=1140%:%
%:%2168=1141%:%
%:%2169=1141%:%
%:%2170=1142%:%
%:%2171=1142%:%
%:%2172=1142%:%
%:%2173=1143%:%
%:%2174=1143%:%
%:%2175=1144%:%
%:%2176=1144%:%
%:%2177=1144%:%
%:%2178=1145%:%
%:%2179=1145%:%
%:%2180=1146%:%
%:%2181=1146%:%
%:%2182=1146%:%
%:%2183=1147%:%
%:%2184=1147%:%
%:%2185=1148%:%
%:%2186=1148%:%
%:%2187=1148%:%
%:%2188=1149%:%
%:%2189=1149%:%
%:%2190=1150%:%
%:%2191=1150%:%
%:%2192=1150%:%
%:%2193=1151%:%
%:%2194=1151%:%
%:%2195=1152%:%
%:%2196=1152%:%
%:%2197=1152%:%
%:%2198=1153%:%
%:%2208=1155%:%
%:%2209=1156%:%
%:%2210=1157%:%
%:%2211=1158%:%
%:%2212=1159%:%
%:%2214=1161%:%
%:%2215=1161%:%
%:%2216=1162%:%
%:%2217=1163%:%
%:%2218=1164%:%
%:%2219=1165%:%
%:%2220=1166%:%
%:%2221=1167%:%
%:%2222=1168%:%
%:%2223=1169%:%
%:%2224=1170%:%
%:%2225=1171%:%
%:%2226=1172%:%
%:%2227=1173%:%
%:%2234=1174%:%
%:%2235=1174%:%
%:%2236=1175%:%
%:%2237=1175%:%
%:%2238=1176%:%
%:%2239=1176%:%
%:%2240=1177%:%
%:%2241=1177%:%
%:%2242=1178%:%
%:%2243=1178%:%
%:%2244=1178%:%
%:%2245=1179%:%
%:%2246=1179%:%
%:%2247=1180%:%
%:%2248=1180%:%
%:%2249=1180%:%
%:%2250=1181%:%
%:%2251=1181%:%
%:%2252=1182%:%
%:%2253=1182%:%
%:%2254=1182%:%
%:%2255=1183%:%
%:%2256=1183%:%
%:%2257=1184%:%
%:%2258=1184%:%
%:%2259=1185%:%
%:%2260=1185%:%
%:%2261=1186%:%
%:%2262=1186%:%
%:%2263=1187%:%
%:%2264=1187%:%
%:%2265=1188%:%
%:%2266=1188%:%
%:%2267=1188%:%
%:%2268=1189%:%
%:%2269=1189%:%
%:%2270=1190%:%
%:%2271=1190%:%
%:%2272=1191%:%
%:%2273=1191%:%
%:%2274=1191%:%
%:%2275=1192%:%
%:%2276=1192%:%
%:%2277=1193%:%
%:%2278=1193%:%
%:%2279=1194%:%
%:%2280=1194%:%
%:%2281=1194%:%
%:%2282=1195%:%
%:%2283=1195%:%
%:%2284=1196%:%
%:%2285=1196%:%
%:%2286=1197%:%
%:%2287=1197%:%
%:%2288=1197%:%
%:%2289=1198%:%
%:%2290=1198%:%
%:%2291=1199%:%
%:%2292=1199%:%
%:%2293=1199%:%
%:%2294=1200%:%
%:%2295=1200%:%
%:%2296=1201%:%
%:%2297=1201%:%
%:%2298=1202%:%
%:%2299=1202%:%
%:%2300=1202%:%
%:%2301=1203%:%
%:%2302=1203%:%
%:%2303=1204%:%
%:%2304=1204%:%
%:%2305=1204%:%
%:%2306=1205%:%
%:%2307=1205%:%
%:%2308=1206%:%
%:%2309=1206%:%
%:%2310=1207%:%
%:%2311=1207%:%
%:%2312=1208%:%
%:%2313=1208%:%
%:%2314=1209%:%
%:%2315=1209%:%
%:%2316=1210%:%
%:%2317=1210%:%
%:%2318=1210%:%
%:%2319=1211%:%
%:%2320=1211%:%
%:%2321=1212%:%
%:%2322=1212%:%
%:%2323=1213%:%
%:%2324=1213%:%
%:%2325=1213%:%
%:%2326=1214%:%
%:%2327=1214%:%
%:%2328=1215%:%
%:%2329=1215%:%
%:%2330=1216%:%
%:%2331=1216%:%
%:%2332=1216%:%
%:%2333=1217%:%
%:%2334=1217%:%
%:%2335=1218%:%
%:%2336=1218%:%
%:%2337=1219%:%
%:%2338=1219%:%
%:%2339=1219%:%
%:%2340=1220%:%
%:%2341=1220%:%
%:%2342=1221%:%
%:%2343=1221%:%
%:%2344=1221%:%
%:%2345=1222%:%
%:%2346=1222%:%
%:%2347=1223%:%
%:%2348=1223%:%
%:%2349=1224%:%
%:%2350=1224%:%
%:%2351=1224%:%
%:%2352=1225%:%
%:%2353=1225%:%
%:%2354=1226%:%
%:%2355=1226%:%
%:%2356=1226%:%
%:%2357=1227%:%
%:%2358=1227%:%
%:%2359=1227%:%
%:%2360=1228%:%
%:%2361=1228%:%
%:%2362=1229%:%
%:%2363=1229%:%
%:%2364=1229%:%
%:%2365=1230%:%
%:%2366=1230%:%
%:%2367=1231%:%
%:%2368=1231%:%
%:%2369=1232%:%
%:%2370=1232%:%
%:%2371=1232%:%
%:%2372=1233%:%
%:%2382=1235%:%
%:%2383=1236%:%
%:%2384=1237%:%
%:%2385=1238%:%
%:%2386=1239%:%
%:%2388=1241%:%
%:%2389=1241%:%
%:%2390=1242%:%
%:%2391=1243%:%
%:%2394=1244%:%
%:%2398=1244%:%
%:%2399=1244%:%
%:%2400=1244%:%
%:%2405=1244%:%
%:%2408=1245%:%
%:%2409=1246%:%
%:%2410=1246%:%
%:%2413=1247%:%
%:%2417=1247%:%
%:%2418=1247%:%
%:%2427=1249%:%
%:%2429=1251%:%
%:%2430=1251%:%
%:%2431=1252%:%
%:%2432=1253%:%
%:%2433=1254%:%
%:%2434=1255%:%
%:%2435=1256%:%
%:%2436=1257%:%
%:%2443=1258%:%
%:%2444=1258%:%
%:%2445=1259%:%
%:%2446=1259%:%
%:%2447=1260%:%
%:%2448=1260%:%
%:%2449=1260%:%
%:%2450=1261%:%
%:%2451=1261%:%
%:%2455=1265%:%
%:%2456=1266%:%
%:%2457=1266%:%
%:%2458=1266%:%
%:%2459=1267%:%
%:%2460=1267%:%
%:%2461=1268%:%
%:%2462=1268%:%
%:%2463=1268%:%
%:%2464=1268%:%
%:%2465=1269%:%
%:%2466=1269%:%
%:%2467=1269%:%
%:%2468=1270%:%
%:%2469=1270%:%
%:%2470=1271%:%
%:%2471=1271%:%
%:%2472=1272%:%
%:%2473=1272%:%
%:%2474=1272%:%
%:%2475=1273%:%
%:%2476=1273%:%
%:%2477=1274%:%
%:%2478=1274%:%
%:%2479=1275%:%
%:%2480=1275%:%
%:%2481=1276%:%
%:%2482=1276%:%
%:%2483=1276%:%
%:%2484=1277%:%
%:%2485=1277%:%
%:%2486=1278%:%
%:%2487=1278%:%
%:%2488=1278%:%
%:%2489=1279%:%
%:%2490=1279%:%
%:%2491=1280%:%
%:%2492=1280%:%
%:%2493=1280%:%
%:%2494=1281%:%
%:%2495=1281%:%
%:%2496=1281%:%
%:%2497=1282%:%
%:%2498=1282%:%
%:%2499=1282%:%
%:%2500=1283%:%
%:%2501=1283%:%
%:%2502=1284%:%
%:%2503=1284%:%
%:%2504=1285%:%
%:%2505=1285%:%
%:%2506=1285%:%
%:%2507=1286%:%
%:%2508=1286%:%
%:%2509=1287%:%
%:%2510=1287%:%
%:%2511=1287%:%
%:%2512=1288%:%
%:%2513=1288%:%
%:%2514=1288%:%
%:%2515=1289%:%
%:%2516=1289%:%
%:%2517=1290%:%
%:%2518=1290%:%
%:%2519=1291%:%
%:%2520=1291%:%
%:%2521=1291%:%
%:%2522=1292%:%
%:%2523=1292%:%
%:%2524=1292%:%
%:%2525=1293%:%
%:%2526=1293%:%
%:%2527=1294%:%
%:%2528=1294%:%
%:%2529=1295%:%
%:%2530=1295%:%
%:%2531=1295%:%
%:%2532=1296%:%
%:%2533=1296%:%
%:%2534=1296%:%
%:%2535=1297%:%
%:%2536=1297%:%
%:%2537=1297%:%
%:%2538=1298%:%
%:%2539=1298%:%
%:%2540=1298%:%
%:%2541=1299%:%
%:%2542=1299%:%
%:%2543=1299%:%
%:%2544=1300%:%
%:%2545=1300%:%
%:%2546=1300%:%
%:%2547=1301%:%
%:%2548=1301%:%
%:%2549=1302%:%
%:%2550=1302%:%
%:%2551=1303%:%
%:%2552=1303%:%
%:%2553=1304%:%
%:%2554=1304%:%
%:%2555=1305%:%
%:%2556=1305%:%
%:%2557=1305%:%
%:%2558=1306%:%
%:%2559=1306%:%
%:%2560=1306%:%
%:%2561=1307%:%
%:%2562=1307%:%
%:%2563=1308%:%
%:%2564=1308%:%
%:%2565=1308%:%
%:%2566=1309%:%
%:%2567=1309%:%
%:%2568=1310%:%
%:%2569=1310%:%
%:%2570=1311%:%
%:%2571=1311%:%
%:%2572=1311%:%
%:%2573=1312%:%
%:%2574=1312%:%
%:%2575=1313%:%
%:%2576=1313%:%
%:%2577=1314%:%
%:%2578=1314%:%
%:%2579=1315%:%
%:%2580=1315%:%
%:%2581=1316%:%
%:%2582=1316%:%
%:%2583=1317%:%
%:%2584=1317%:%
%:%2585=1317%:%
%:%2586=1318%:%
%:%2587=1318%:%
%:%2588=1319%:%
%:%2589=1319%:%
%:%2590=1319%:%
%:%2591=1320%:%
%:%2592=1320%:%
%:%2593=1321%:%
%:%2594=1321%:%
%:%2595=1321%:%
%:%2596=1322%:%
%:%2597=1322%:%
%:%2598=1323%:%
%:%2599=1323%:%
%:%2600=1324%:%
%:%2601=1324%:%
%:%2602=1324%:%
%:%2603=1325%:%
%:%2604=1325%:%
%:%2605=1325%:%
%:%2606=1326%:%
%:%2607=1326%:%
%:%2608=1327%:%
%:%2609=1327%:%
%:%2610=1327%:%
%:%2611=1328%:%
%:%2612=1328%:%
%:%2613=1328%:%
%:%2614=1329%:%
%:%2615=1329%:%
%:%2616=1329%:%
%:%2617=1330%:%
%:%2618=1330%:%
%:%2619=1330%:%
%:%2620=1331%:%
%:%2621=1331%:%
%:%2622=1332%:%
%:%2623=1332%:%
%:%2624=1333%:%
%:%2625=1333%:%
%:%2626=1333%:%
%:%2627=1334%:%
%:%2628=1334%:%
%:%2629=1335%:%
%:%2630=1335%:%
%:%2631=1335%:%
%:%2632=1336%:%
%:%2633=1336%:%
%:%2634=1336%:%
%:%2635=1337%:%
%:%2636=1337%:%
%:%2637=1338%:%
%:%2638=1338%:%
%:%2639=1339%:%
%:%2640=1339%:%
%:%2641=1339%:%
%:%2642=1340%:%
%:%2643=1340%:%
%:%2644=1340%:%
%:%2645=1341%:%
%:%2646=1341%:%
%:%2647=1342%:%
%:%2648=1342%:%
%:%2649=1343%:%
%:%2650=1343%:%
%:%2651=1343%:%
%:%2652=1344%:%
%:%2653=1344%:%
%:%2654=1344%:%
%:%2655=1345%:%
%:%2656=1345%:%
%:%2657=1345%:%
%:%2658=1346%:%
%:%2659=1346%:%
%:%2660=1346%:%
%:%2661=1347%:%
%:%2662=1347%:%
%:%2663=1347%:%
%:%2664=1348%:%
%:%2665=1348%:%
%:%2666=1348%:%
%:%2667=1349%:%
%:%2668=1349%:%
%:%2669=1350%:%
%:%2670=1350%:%
%:%2671=1351%:%
%:%2672=1351%:%
%:%2673=1352%:%
%:%2674=1352%:%
%:%2675=1353%:%
%:%2676=1353%:%
%:%2677=1353%:%
%:%2678=1354%:%
%:%2679=1354%:%
%:%2680=1354%:%
%:%2681=1355%:%
%:%2682=1355%:%
%:%2683=1356%:%
%:%2684=1356%:%
%:%2685=1356%:%
%:%2686=1357%:%
%:%2687=1357%:%
%:%2688=1358%:%
%:%2689=1358%:%
%:%2690=1359%:%
%:%2691=1359%:%
%:%2692=1359%:%
%:%2693=1360%:%
%:%2694=1360%:%
%:%2695=1361%:%
%:%2696=1361%:%
%:%2697=1362%:%
%:%2698=1362%:%
%:%2699=1363%:%
%:%2700=1363%:%
%:%2701=1364%:%
%:%2702=1364%:%
%:%2703=1365%:%
%:%2704=1365%:%
%:%2705=1365%:%
%:%2706=1366%:%
%:%2707=1366%:%
%:%2708=1367%:%
%:%2709=1367%:%
%:%2710=1367%:%
%:%2711=1368%:%
%:%2712=1368%:%
%:%2713=1369%:%
%:%2714=1369%:%
%:%2715=1369%:%
%:%2716=1370%:%
%:%2717=1370%:%
%:%2718=1371%:%
%:%2719=1371%:%
%:%2720=1372%:%
%:%2721=1372%:%
%:%2722=1372%:%
%:%2723=1373%:%
%:%2724=1373%:%
%:%2725=1373%:%
%:%2726=1374%:%
%:%2727=1374%:%
%:%2728=1375%:%
%:%2729=1375%:%
%:%2730=1375%:%
%:%2731=1376%:%
%:%2732=1376%:%
%:%2733=1377%:%
%:%2734=1377%:%
%:%2735=1378%:%
%:%2736=1378%:%
%:%2737=1379%:%
%:%2738=1379%:%
%:%2739=1380%:%
%:%2740=1380%:%
%:%2741=1381%:%
%:%2742=1381%:%
%:%2743=1381%:%
%:%2744=1382%:%
%:%2745=1382%:%
%:%2746=1383%:%
%:%2747=1383%:%
%:%2748=1384%:%
%:%2749=1384%:%
%:%2750=1384%:%
%:%2751=1385%:%
%:%2752=1385%:%
%:%2753=1385%:%
%:%2754=1386%:%
%:%2755=1386%:%
%:%2756=1387%:%
%:%2757=1387%:%
%:%2758=1388%:%
%:%2759=1388%:%
%:%2760=1389%:%
%:%2761=1389%:%
%:%2762=1390%:%
%:%2763=1390%:%
%:%2764=1391%:%
%:%2765=1391%:%
%:%2766=1391%:%
%:%2767=1392%:%
%:%2768=1392%:%
%:%2769=1393%:%
%:%2770=1393%:%
%:%2771=1394%:%
%:%2772=1394%:%
%:%2773=1394%:%
%:%2774=1395%:%
%:%2775=1395%:%
%:%2776=1395%:%
%:%2777=1396%:%
%:%2778=1396%:%
%:%2779=1397%:%
%:%2780=1397%:%
%:%2781=1398%:%
%:%2782=1398%:%
%:%2783=1399%:%
%:%2793=1401%:%
%:%2794=1402%:%
%:%2795=1403%:%
%:%2796=1404%:%
%:%2797=1405%:%
%:%2798=1406%:%
%:%2799=1407%:%
%:%2800=1408%:%
%:%2801=1409%:%
%:%2802=1410%:%
%:%2803=1411%:%
%:%2804=1412%:%
%:%2805=1413%:%
%:%2806=1414%:%
%:%2807=1415%:%
%:%2809=1417%:%
%:%2810=1417%:%
%:%2811=1418%:%
%:%2812=1419%:%
%:%2813=1420%:%
%:%2814=1421%:%
%:%2817=1424%:%
%:%2824=1425%:%
%:%2825=1425%:%
%:%2826=1426%:%
%:%2827=1426%:%
%:%2831=1430%:%
%:%2832=1431%:%
%:%2833=1431%:%
%:%2834=1431%:%
%:%2835=1432%:%
%:%2836=1432%:%
%:%2837=1433%:%
%:%2838=1433%:%
%:%2839=1433%:%
%:%2840=1433%:%
%:%2841=1434%:%
%:%2842=1434%:%
%:%2843=1435%:%
%:%2844=1435%:%
%:%2845=1436%:%
%:%2846=1436%:%
%:%2847=1437%:%
%:%2848=1437%:%
%:%2849=1438%:%
%:%2850=1438%:%
%:%2851=1439%:%
%:%2852=1439%:%
%:%2853=1439%:%
%:%2854=1440%:%
%:%2855=1440%:%
%:%2856=1441%:%
%:%2857=1441%:%
%:%2858=1441%:%
%:%2859=1442%:%
%:%2860=1442%:%
%:%2861=1442%:%
%:%2862=1443%:%
%:%2863=1443%:%
%:%2864=1443%:%
%:%2865=1444%:%
%:%2866=1444%:%
%:%2867=1445%:%
%:%2868=1445%:%
%:%2869=1446%:%
%:%2870=1446%:%
%:%2871=1447%:%
%:%2872=1447%:%
%:%2873=1447%:%
%:%2874=1448%:%
%:%2875=1448%:%
%:%2876=1449%:%
%:%2877=1449%:%
%:%2878=1450%:%
%:%2879=1450%:%
%:%2880=1450%:%
%:%2881=1451%:%
%:%2882=1451%:%
%:%2883=1451%:%
%:%2884=1452%:%
%:%2885=1452%:%
%:%2886=1452%:%
%:%2887=1453%:%
%:%2888=1453%:%
%:%2891=1456%:%
%:%2892=1457%:%
%:%2893=1457%:%
%:%2894=1457%:%
%:%2895=1458%:%
%:%2896=1458%:%
%:%2897=1458%:%
%:%2898=1459%:%
%:%2899=1459%:%
%:%2900=1460%:%
%:%2901=1460%:%
%:%2902=1461%:%
%:%2903=1461%:%
%:%2904=1462%:%
%:%2905=1462%:%
%:%2906=1463%:%
%:%2907=1463%:%
%:%2908=1463%:%
%:%2909=1464%:%
%:%2910=1464%:%
%:%2911=1465%:%
%:%2912=1465%:%
%:%2913=1466%:%
%:%2914=1466%:%
%:%2915=1467%:%
%:%2916=1467%:%
%:%2917=1468%:%
%:%2918=1468%:%
%:%2919=1469%:%
%:%2920=1469%:%
%:%2921=1470%:%
%:%2922=1470%:%
%:%2923=1471%:%
%:%2924=1471%:%
%:%2925=1472%:%
%:%2926=1472%:%
%:%2927=1473%:%
%:%2928=1473%:%
%:%2929=1474%:%
%:%2930=1474%:%
%:%2931=1475%:%
%:%2932=1475%:%
%:%2933=1476%:%
%:%2934=1476%:%
%:%2935=1477%:%
%:%2936=1477%:%
%:%2937=1478%:%
%:%2938=1478%:%
%:%2939=1478%:%
%:%2940=1479%:%
%:%2941=1479%:%
%:%2942=1480%:%
%:%2943=1480%:%
%:%2944=1480%:%
%:%2945=1481%:%
%:%2946=1481%:%
%:%2947=1482%:%
%:%2948=1482%:%
%:%2949=1483%:%
%:%2950=1483%:%
%:%2951=1483%:%
%:%2952=1484%:%
%:%2953=1484%:%
%:%2954=1485%:%
%:%2955=1485%:%
%:%2956=1485%:%
%:%2957=1486%:%
%:%2958=1486%:%
%:%2959=1487%:%
%:%2960=1487%:%
%:%2961=1488%:%
%:%2962=1488%:%
%:%2963=1488%:%
%:%2964=1489%:%
%:%2965=1489%:%
%:%2966=1489%:%
%:%2967=1490%:%
%:%2968=1490%:%
%:%2969=1490%:%
%:%2970=1491%:%
%:%2971=1491%:%
%:%2974=1494%:%
%:%2975=1495%:%
%:%2976=1495%:%
%:%2977=1495%:%
%:%2978=1496%:%
%:%2979=1496%:%
%:%2980=1496%:%
%:%2981=1497%:%
%:%2982=1497%:%
%:%2983=1498%:%
%:%2984=1498%:%
%:%2985=1498%:%
%:%2986=1499%:%
%:%2987=1499%:%
%:%2988=1500%:%
%:%2989=1500%:%
%:%2990=1500%:%
%:%2991=1501%:%
%:%2992=1501%:%
%:%2993=1501%:%
%:%2994=1502%:%
%:%2995=1502%:%
%:%2996=1503%:%
%:%2997=1503%:%
%:%2998=1503%:%
%:%2999=1504%:%
%:%3000=1504%:%
%:%3001=1505%:%
%:%3002=1505%:%
%:%3003=1506%:%
%:%3004=1506%:%
%:%3005=1507%:%
%:%3006=1507%:%
%:%3007=1508%:%
%:%3008=1508%:%
%:%3009=1509%:%
%:%3010=1509%:%
%:%3011=1510%:%
%:%3012=1510%:%
%:%3013=1511%:%
%:%3014=1511%:%
%:%3015=1512%:%
%:%3016=1512%:%
%:%3017=1513%:%
%:%3018=1513%:%
%:%3019=1514%:%
%:%3020=1514%:%
%:%3021=1514%:%
%:%3022=1515%:%
%:%3023=1515%:%
%:%3024=1516%:%
%:%3025=1516%:%
%:%3026=1517%:%
%:%3027=1517%:%
%:%3028=1518%:%
%:%3029=1518%:%
%:%3030=1519%:%
%:%3031=1519%:%
%:%3032=1520%:%
%:%3033=1520%:%
%:%3034=1521%:%
%:%3035=1521%:%
%:%3036=1522%:%
%:%3037=1522%:%
%:%3038=1523%:%
%:%3039=1523%:%
%:%3040=1524%:%
%:%3041=1524%:%
%:%3042=1525%:%
%:%3043=1525%:%
%:%3044=1525%:%
%:%3045=1526%:%
%:%3046=1526%:%
%:%3047=1527%:%
%:%3048=1527%:%
%:%3049=1528%:%
%:%3050=1528%:%
%:%3051=1529%:%
%:%3052=1529%:%
%:%3053=1529%:%
%:%3054=1530%:%
%:%3064=1532%:%
%:%3065=1533%:%
%:%3066=1534%:%
%:%3068=1536%:%
%:%3069=1536%:%
%:%3070=1537%:%
%:%3071=1538%:%
%:%3072=1539%:%
%:%3075=1540%:%
%:%3079=1540%:%
%:%3080=1540%:%
%:%3081=1541%:%
%:%3082=1541%:%
%:%3083=1542%:%
%:%3084=1542%:%
%:%3088=1546%:%
%:%3089=1547%:%
%:%3090=1547%:%
%:%3091=1547%:%
%:%3092=1548%:%
%:%3093=1548%:%
%:%3094=1549%:%
%:%3095=1549%:%
%:%3096=1550%:%
%:%3097=1550%:%
%:%3098=1550%:%
%:%3099=1551%:%
%:%3100=1551%:%
%:%3101=1551%:%
%:%3102=1552%:%
%:%3103=1552%:%
%:%3106=1555%:%
%:%3107=1556%:%
%:%3108=1556%:%
%:%3109=1557%:%
%:%3110=1557%:%
%:%3111=1558%:%
%:%3112=1558%:%
%:%3113=1559%:%
%:%3114=1559%:%
%:%3115=1559%:%
%:%3116=1560%:%
%:%3117=1560%:%
%:%3118=1561%:%
%:%3119=1561%:%
%:%3120=1562%:%
%:%3121=1562%:%
%:%3122=1563%:%
%:%3123=1563%:%
%:%3124=1563%:%
%:%3125=1564%:%
%:%3126=1564%:%
%:%3127=1565%:%
%:%3137=1567%:%
%:%3139=1569%:%
%:%3140=1569%:%
%:%3141=1570%:%
%:%3142=1571%:%
%:%3143=1572%:%
%:%3150=1573%:%
%:%3151=1573%:%
%:%3152=1574%:%
%:%3153=1574%:%
%:%3154=1575%:%
%:%3155=1575%:%
%:%3156=1576%:%
%:%3157=1576%:%
%:%3158=1576%:%
%:%3159=1577%:%
%:%3160=1577%:%
%:%3161=1578%:%
%:%3162=1578%:%
%:%3163=1578%:%
%:%3164=1579%:%
%:%3165=1579%:%
%:%3166=1580%:%
%:%3167=1580%:%
%:%3168=1580%:%
%:%3169=1581%:%
%:%3170=1581%:%
%:%3171=1582%:%
%:%3172=1582%:%
%:%3173=1582%:%
%:%3174=1583%:%
%:%3175=1583%:%
%:%3176=1584%:%
%:%3177=1584%:%
%:%3178=1585%:%
%:%3179=1585%:%