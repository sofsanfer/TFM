%
\begin{isabellebody}%
\setisabellecontext{ColecScfc}%
%
\isadelimtheory
%
\endisadelimtheory
%
\isatagtheory
%
\endisatagtheory
{\isafoldtheory}%
%
\isadelimtheory
%
\endisadelimtheory
%
\begin{isamarkuptext}%
En este apartado definiremos colecciones de conjuntos \isa{cerradas\ bajo\ subconjuntos} y de 
  \isa{carácter\ finito}, junto con tres resultados sobre las mismas. El primero de ellos permite
  extender una colección que verifique la propiedad de consistencia proposicional a otra que 
  también la verifique y sea cerrada bajo subconjuntos. Posteriormente probaremos que toda colección
  de carácter finito es cerrada bajo subconjuntos. Finalmente, mostraremos un resultado que 
  permite extender una colección cerrada bajo subconjuntos que verifique la propiedad de
  consistencia proposicional a otra que también verifique dicha propiedad y sea de carácter 
  finito.

  \begin{definicion}
    Una colección de conjuntos es \isa{cerrada\ bajo\ subconjuntos} si todo subconjunto de cada conjunto 
    de la colección pertenece a la colección.
  \end{definicion}

  En Isabelle se formaliza de la siguiente manera.%
\end{isamarkuptext}\isamarkuptrue%
\isacommand{definition}\isamarkupfalse%
\ {\isachardoublequoteopen}subset{\isacharunderscore}closed\ C\ {\isasymequiv}\ {\isacharparenleft}{\isasymforall}S\ {\isasymin}\ C{\isachardot}\ {\isasymforall}S{\isacharprime}{\isasymsubseteq}S{\isachardot}\ S{\isacharprime}\ {\isasymin}\ C{\isacharparenright}{\isachardoublequoteclose}%
\begin{isamarkuptext}%
Mostremos algunos ejemplos para ilustrar la definición. Para ello, veamos si las colecciones
  de conjuntos de fórmulas proposicionales expuestas en los ejemplos anteriores son cerradas bajo 
  subconjuntos.%
\end{isamarkuptext}\isamarkuptrue%
\isacommand{lemma}\isamarkupfalse%
\ {\isachardoublequoteopen}subset{\isacharunderscore}closed\ {\isacharbraceleft}{\isacharbraceleft}{\isacharbraceright}{\isacharbraceright}{\isachardoublequoteclose}\isanewline
%
\isadelimproof
\ \ %
\endisadelimproof
%
\isatagproof
\isacommand{unfolding}\isamarkupfalse%
\ subset{\isacharunderscore}closed{\isacharunderscore}def\ \isacommand{by}\isamarkupfalse%
\ simp%
\endisatagproof
{\isafoldproof}%
%
\isadelimproof
\isanewline
%
\endisadelimproof
\isanewline
\isacommand{lemma}\isamarkupfalse%
\ {\isachardoublequoteopen}{\isasymnot}\ subset{\isacharunderscore}closed\ {\isacharbraceleft}{\isacharbraceleft}Atom\ {\isadigit{0}}{\isacharbraceright}{\isacharbraceright}{\isachardoublequoteclose}\isanewline
%
\isadelimproof
\ \ %
\endisadelimproof
%
\isatagproof
\isacommand{unfolding}\isamarkupfalse%
\ subset{\isacharunderscore}closed{\isacharunderscore}def\ \isacommand{by}\isamarkupfalse%
\ auto%
\endisatagproof
{\isafoldproof}%
%
\isadelimproof
%
\endisadelimproof
%
\begin{isamarkuptext}%
Observemos que, puesto que el conjunto vacío es subconjunto de todo conjunto, una
  condición necesaria para que una colección sea cerrada bajo subconjuntos es que contenga al
  conjunto vacío.%
\end{isamarkuptext}\isamarkuptrue%
\isacommand{lemma}\isamarkupfalse%
\ {\isachardoublequoteopen}subset{\isacharunderscore}closed\ {\isacharbraceleft}{\isacharbraceleft}Atom\ {\isadigit{0}}{\isacharbraceright}{\isacharcomma}{\isacharbraceleft}{\isacharbraceright}{\isacharbraceright}{\isachardoublequoteclose}\isanewline
%
\isadelimproof
\ \ %
\endisadelimproof
%
\isatagproof
\isacommand{unfolding}\isamarkupfalse%
\ subset{\isacharunderscore}closed{\isacharunderscore}def\ \isacommand{by}\isamarkupfalse%
\ auto%
\endisatagproof
{\isafoldproof}%
%
\isadelimproof
%
\endisadelimproof
%
\begin{isamarkuptext}%
De este modo, se deduce fácilmente que el resto de colecciones expuestas en los ejemplos
  anteriores no son cerradas bajo subconjuntos.%
\end{isamarkuptext}\isamarkuptrue%
\isacommand{lemma}\isamarkupfalse%
\ {\isachardoublequoteopen}{\isasymnot}\ subset{\isacharunderscore}closed\ {\isacharbraceleft}{\isacharbraceleft}{\isacharparenleft}\isactrlbold {\isasymnot}\ {\isacharparenleft}Atom\ {\isadigit{1}}{\isacharparenright}{\isacharparenright}\ \isactrlbold {\isasymrightarrow}\ Atom\ {\isadigit{2}}{\isacharbraceright}{\isacharcomma}\isanewline
\ \ \ {\isacharbraceleft}{\isacharparenleft}{\isacharparenleft}\isactrlbold {\isasymnot}\ {\isacharparenleft}Atom\ {\isadigit{1}}{\isacharparenright}{\isacharparenright}\ \isactrlbold {\isasymrightarrow}\ Atom\ {\isadigit{2}}{\isacharparenright}{\isacharcomma}\ \isactrlbold {\isasymnot}{\isacharparenleft}\isactrlbold {\isasymnot}\ {\isacharparenleft}Atom\ {\isadigit{1}}{\isacharparenright}{\isacharparenright}{\isacharbraceright}{\isacharcomma}\isanewline
\ \ {\isacharbraceleft}{\isacharparenleft}{\isacharparenleft}\isactrlbold {\isasymnot}\ {\isacharparenleft}Atom\ {\isadigit{1}}{\isacharparenright}{\isacharparenright}\ \isactrlbold {\isasymrightarrow}\ Atom\ {\isadigit{2}}{\isacharparenright}{\isacharcomma}\ \isactrlbold {\isasymnot}{\isacharparenleft}\isactrlbold {\isasymnot}\ {\isacharparenleft}Atom\ {\isadigit{1}}{\isacharparenright}{\isacharparenright}{\isacharcomma}\ \ Atom\ {\isadigit{1}}{\isacharbraceright}{\isacharbraceright}{\isachardoublequoteclose}\ \isanewline
%
\isadelimproof
\ \ %
\endisadelimproof
%
\isatagproof
\isacommand{unfolding}\isamarkupfalse%
\ subset{\isacharunderscore}closed{\isacharunderscore}def\ \isacommand{by}\isamarkupfalse%
\ auto%
\endisatagproof
{\isafoldproof}%
%
\isadelimproof
\isanewline
%
\endisadelimproof
\isanewline
\isacommand{lemma}\isamarkupfalse%
\ {\isachardoublequoteopen}{\isasymnot}\ subset{\isacharunderscore}closed\ {\isacharbraceleft}{\isacharbraceleft}{\isacharparenleft}\isactrlbold {\isasymnot}\ {\isacharparenleft}Atom\ {\isadigit{1}}{\isacharparenright}{\isacharparenright}\ \isactrlbold {\isasymrightarrow}\ Atom\ {\isadigit{2}}{\isacharbraceright}{\isacharcomma}\isanewline
\ \ \ {\isacharbraceleft}{\isacharparenleft}{\isacharparenleft}\isactrlbold {\isasymnot}\ {\isacharparenleft}Atom\ {\isadigit{1}}{\isacharparenright}{\isacharparenright}\ \isactrlbold {\isasymrightarrow}\ Atom\ {\isadigit{2}}{\isacharparenright}{\isacharcomma}\ \isactrlbold {\isasymnot}{\isacharparenleft}\isactrlbold {\isasymnot}\ {\isacharparenleft}Atom\ {\isadigit{1}}{\isacharparenright}{\isacharparenright}{\isacharbraceright}{\isacharbraceright}{\isachardoublequoteclose}\ \isanewline
%
\isadelimproof
\ \ %
\endisadelimproof
%
\isatagproof
\isacommand{unfolding}\isamarkupfalse%
\ subset{\isacharunderscore}closed{\isacharunderscore}def\ \isacommand{by}\isamarkupfalse%
\ auto%
\endisatagproof
{\isafoldproof}%
%
\isadelimproof
%
\endisadelimproof
%
\begin{isamarkuptext}%
Continuemos con la noción de propiedad de carácter finito.

\begin{definicion}
  Una colección de conjuntos tiene la \isa{propiedad\ de\ carácter\ finito} si para cualquier conjunto
  son equivalentes:
  \begin{enumerate}
    \item El conjunto pertenece a la colección.
    \item Todo subconjunto finito suyo pertenece a la colección.
  \end{enumerate}
\end{definicion}

  La formalización en Isabelle/HOL de dicha definición se muestra a continuación.%
\end{isamarkuptext}\isamarkuptrue%
\isacommand{definition}\isamarkupfalse%
\ {\isachardoublequoteopen}finite{\isacharunderscore}character\ C\ {\isasymequiv}\ \isanewline
\ \ \ \ \ \ \ \ \ \ \ \ {\isacharparenleft}{\isasymforall}S{\isachardot}\ S\ {\isasymin}\ C\ {\isasymlongleftrightarrow}\ {\isacharparenleft}{\isasymforall}S{\isacharprime}\ {\isasymsubseteq}\ S{\isachardot}\ finite\ S{\isacharprime}\ {\isasymlongrightarrow}\ S{\isacharprime}\ {\isasymin}\ C{\isacharparenright}{\isacharparenright}{\isachardoublequoteclose}%
\begin{isamarkuptext}%
Distingamos las colecciones de los ejemplos anteriores que tengan la propiedad de carácter 
  finito. Análogamente, puesto que el conjunto vacío es finito y subconjunto de cualquier conjunto, 
  se observa que una condición necesaria para que una colección tenga la propiedad de carácter 
  finito es que contenga al conjunto vacío.%
\end{isamarkuptext}\isamarkuptrue%
\isacommand{lemma}\isamarkupfalse%
\ {\isachardoublequoteopen}finite{\isacharunderscore}character\ {\isacharbraceleft}{\isacharbraceleft}{\isacharbraceright}{\isacharbraceright}{\isachardoublequoteclose}\isanewline
%
\isadelimproof
\ \ %
\endisadelimproof
%
\isatagproof
\isacommand{unfolding}\isamarkupfalse%
\ finite{\isacharunderscore}character{\isacharunderscore}def\ \isacommand{by}\isamarkupfalse%
\ auto%
\endisatagproof
{\isafoldproof}%
%
\isadelimproof
\isanewline
%
\endisadelimproof
\isanewline
\isacommand{lemma}\isamarkupfalse%
\ {\isachardoublequoteopen}{\isasymnot}\ finite{\isacharunderscore}character\ {\isacharbraceleft}{\isacharbraceleft}Atom\ {\isadigit{0}}{\isacharbraceright}{\isacharbraceright}{\isachardoublequoteclose}\isanewline
%
\isadelimproof
\ \ %
\endisadelimproof
%
\isatagproof
\isacommand{unfolding}\isamarkupfalse%
\ finite{\isacharunderscore}character{\isacharunderscore}def\ \isacommand{by}\isamarkupfalse%
\ auto%
\endisatagproof
{\isafoldproof}%
%
\isadelimproof
\isanewline
%
\endisadelimproof
\isanewline
\isacommand{lemma}\isamarkupfalse%
\ {\isachardoublequoteopen}finite{\isacharunderscore}character\ {\isacharbraceleft}{\isacharbraceleft}Atom\ {\isadigit{0}}{\isacharbraceright}{\isacharcomma}{\isacharbraceleft}{\isacharbraceright}{\isacharbraceright}{\isachardoublequoteclose}\isanewline
%
\isadelimproof
\ \ %
\endisadelimproof
%
\isatagproof
\isacommand{unfolding}\isamarkupfalse%
\ finite{\isacharunderscore}character{\isacharunderscore}def\ \isacommand{by}\isamarkupfalse%
\ auto%
\endisatagproof
{\isafoldproof}%
%
\isadelimproof
%
\endisadelimproof
%
\begin{isamarkuptext}%
Una vez introducidas las definiciones anteriores, veamos los resultados que las relacionan
  con la propiedad de consistencia proposicional. De este modo, combinándolos en la prueba del 
  \isa{teorema\ de\ existencia\ de\ modelo}, dada una colección \isa{C} cualquiera que verifique la propiedad 
  de consistencia proposicional, podemos extenderla a una colección \isa{C{\isacharprime}} que también la verifique y 
  además sea cerrada bajo subconjuntos y de carácter finito.

\comentario{Volver a revisar el párrafo anterior al final de la
redacción de la sección.}

  \begin{lema}
    Toda colección de conjuntos con la propiedad de consistencia proposicional se puede extender a
    una colección que también verifique la propiedad de consistencia proposicional y sea cerrada 
    bajo subconjuntos.
  \end{lema}

  En Isabelle se formaliza el resultado de la siguiente manera.%
\end{isamarkuptext}\isamarkuptrue%
\isacommand{lemma}\isamarkupfalse%
\ {\isachardoublequoteopen}pcp\ C\ {\isasymLongrightarrow}\ {\isasymexists}C{\isacharprime}{\isachardot}\ C\ {\isasymsubseteq}\ C{\isacharprime}\ {\isasymand}\ pcp\ C{\isacharprime}\ {\isasymand}\ subset{\isacharunderscore}closed\ C{\isacharprime}{\isachardoublequoteclose}\isanewline
%
\isadelimproof
\ \ %
\endisadelimproof
%
\isatagproof
\isacommand{oops}\isamarkupfalse%
%
\endisatagproof
{\isafoldproof}%
%
\isadelimproof
%
\endisadelimproof
%
\begin{isamarkuptext}%
Procedamos con su demostración.

\begin{demostracion}
  Dada una colección de conjuntos cualquiera \isa{C}, consideremos la colección formada por los 
  conjuntos tales que son subconjuntos de algún conjunto de \isa{C}. Notemos esta colección por 
  \isa{C{\isacharprime}\ {\isacharequal}\ {\isacharbraceleft}S{\isacharprime}{\isachardot}\ {\isasymexists}S{\isasymin}C{\isachardot}\ S{\isacharprime}\ {\isasymsubseteq}\ S{\isacharbraceright}}. Vamos a probar que, en efecto, \isa{C{\isacharprime}} verifica  las condiciones del lema.

  En primer lugar, veamos que \isa{C} está contenida en \isa{C{\isacharprime}}. Para ello, consideremos un conjunto
  cualquiera perteneciente a \isa{C}. Puesto que la propiedad de contención es reflexiva, dicho conjunto 
  es subconjunto de sí mismo. De este modo, por definición de \isa{C{\isacharprime}}, se verifica que el conjunto 
  pertenece a \isa{C{\isacharprime}}.

  Por otro lado, comprobemos que \isa{C{\isacharprime}} tiene la propiedad de consistencia proposicional.
  Por el lema de caracterización de la propiedad de consistencia proposicional mediante la
  notación uniforme basta probar que, para cualquier conjunto de fórmulas \isa{S} de \isa{C{\isacharprime}}, se 
  verifican las condiciones:
  \begin{itemize}
    \item \isa{{\isasymbottom}} no pertenece a \isa{S}.
    \item Dada \isa{p} una fórmula atómica cualquiera, no se tiene 
    simultáneamente que\\ \isa{p\ {\isasymin}\ S} y \isa{{\isasymnot}\ p\ {\isasymin}\ S}.
    \item Para toda fórmula de tipo \isa{{\isasymalpha}} con componentes \isa{{\isasymalpha}\isactrlsub {\isadigit{1}}} y \isa{{\isasymalpha}\isactrlsub {\isadigit{2}}} tal que \isa{{\isasymalpha}}
    pertenece a \isa{S}, se tiene que \isa{{\isacharbraceleft}{\isasymalpha}\isactrlsub {\isadigit{1}}{\isacharcomma}{\isasymalpha}\isactrlsub {\isadigit{2}}{\isacharbraceright}\ {\isasymunion}\ S} pertenece a \isa{C{\isacharprime}}.
    \item Para toda fórmula de tipo \isa{{\isasymbeta}} con componentes \isa{{\isasymbeta}\isactrlsub {\isadigit{1}}} y \isa{{\isasymbeta}\isactrlsub {\isadigit{2}}} tal que \isa{{\isasymbeta}}
    pertenece a \isa{S}, se tiene que o bien \isa{{\isacharbraceleft}{\isasymbeta}\isactrlsub {\isadigit{1}}{\isacharbraceright}\ {\isasymunion}\ S} pertenece a \isa{C{\isacharprime}} o 
    bien \isa{{\isacharbraceleft}{\isasymbeta}\isactrlsub {\isadigit{2}}{\isacharbraceright}\ {\isasymunion}\ S} pertenece a \isa{C{\isacharprime}}.
  \end{itemize} 

  De este modo, sea \isa{S} un conjunto de fórmulas cualquiera de la colección \isa{C{\isacharprime}}. Por definición de
  dicha colección, existe un conjunto \isa{S{\isacharprime}} pertenciente a \isa{C} tal que \isa{S} está contenido en \isa{S{\isacharprime}}.
  Como \isa{C} tiene la propiedad de consistencia proposicional por hipótesis, verifica las condiciones
  del lema de caracterización de la propiedad de consistencia proposicional mediante la notación 
  uniforme. En particular, puesto que \isa{S{\isacharprime}} pertenece a \isa{C}, se verifica: 
  \begin{itemize}
    \item \isa{{\isasymbottom}} no pertenece a \isa{S{\isacharprime}}.
    \item Dada \isa{p} una fórmula atómica cualquiera, no se tiene 
    simultáneamente que\\ \isa{p\ {\isasymin}\ S{\isacharprime}} y \isa{{\isasymnot}\ p\ {\isasymin}\ S{\isacharprime}}.
    \item Para toda fórmula de tipo \isa{{\isasymalpha}} con componentes \isa{{\isasymalpha}\isactrlsub {\isadigit{1}}} y \isa{{\isasymalpha}\isactrlsub {\isadigit{2}}} tal que \isa{{\isasymalpha}}
    pertenece a \isa{S{\isacharprime}}, se tiene que \isa{{\isacharbraceleft}{\isasymalpha}\isactrlsub {\isadigit{1}}{\isacharcomma}{\isasymalpha}\isactrlsub {\isadigit{2}}{\isacharbraceright}\ {\isasymunion}\ S{\isacharprime}} pertenece a \isa{C}.
    \item Para toda fórmula de tipo \isa{{\isasymbeta}} con componentes \isa{{\isasymbeta}\isactrlsub {\isadigit{1}}} y \isa{{\isasymbeta}\isactrlsub {\isadigit{2}}} tal que \isa{{\isasymbeta}}
    pertenece a \isa{S{\isacharprime}}, se tiene que o bien \isa{{\isacharbraceleft}{\isasymbeta}\isactrlsub {\isadigit{1}}{\isacharbraceright}\ {\isasymunion}\ S{\isacharprime}} pertenece a \isa{C} o 
    bien \isa{{\isacharbraceleft}{\isasymbeta}\isactrlsub {\isadigit{2}}{\isacharbraceright}\ {\isasymunion}\ S{\isacharprime}} pertenece a \isa{C}.
  \end{itemize} 

  Por tanto, como \isa{S} está contenida en \isa{S{\isacharprime}}, se verifica análogamente que \isa{{\isasymbottom}} no pertence a \isa{S}
  y que dada una fórmula atómica cualquiera \isa{p}, no se tiene simultáneamente que\\ \isa{p\ {\isasymin}\ S} y 
  \isa{{\isasymnot}\ p\ {\isasymin}\ S{\isachardot}} Veamos que se verifican el resto de condiciones del lema de caracterización:

  \isa{{\isasymsqdot}\ Condición\ para\ fórmulas\ de\ tipo\ {\isasymalpha}}: Sea una fórmula de tipo \isa{{\isasymalpha}} con componentes \isa{{\isasymalpha}\isactrlsub {\isadigit{1}}} y 
    \isa{{\isasymalpha}\isactrlsub {\isadigit{2}}} tal que \isa{{\isasymalpha}} pertenece a \isa{S}. Como \isa{S} está contenida en \isa{S{\isacharprime}}, tenemos que la fórmula 
    pertence también a \isa{S{\isacharprime}}. De este modo, se verifica que \isa{{\isacharbraceleft}{\isasymalpha}\isactrlsub {\isadigit{1}}{\isacharcomma}{\isasymalpha}\isactrlsub {\isadigit{2}}{\isacharbraceright}\ {\isasymunion}\ S{\isacharprime}} pertenece a la colección 
    \isa{C}. Por otro lado, como el conjunto \isa{S} está contenido en \isa{S{\isacharprime}}, se observa fácilmente que\\
    \isa{{\isacharbraceleft}{\isasymalpha}\isactrlsub {\isadigit{1}}{\isacharcomma}{\isasymalpha}\isactrlsub {\isadigit{2}}{\isacharbraceright}\ {\isasymunion}\ S} está contenido en \isa{{\isacharbraceleft}{\isasymalpha}\isactrlsub {\isadigit{1}}{\isacharcomma}{\isasymalpha}\isactrlsub {\isadigit{2}}{\isacharbraceright}\ {\isasymunion}\ S{\isacharprime}}. Por lo tanto, el conjunto \isa{{\isacharbraceleft}{\isasymalpha}\isactrlsub {\isadigit{1}}{\isacharcomma}{\isasymalpha}\isactrlsub {\isadigit{2}}{\isacharbraceright}\ {\isasymunion}\ S} está en 
    \isa{C{\isacharprime}} por definición de esta, ya que es subconjunto de \isa{{\isacharbraceleft}{\isasymalpha}\isactrlsub {\isadigit{1}}{\isacharcomma}{\isasymalpha}\isactrlsub {\isadigit{2}}{\isacharbraceright}\ {\isasymunion}\ S{\isacharprime}} que pertence a \isa{C}.

  \isa{{\isasymsqdot}\ Condición\ para\ fórmulas\ de\ tipo\ {\isasymbeta}}: Sea una fórmula de tipo \isa{{\isasymbeta}} con componentes \isa{{\isasymbeta}\isactrlsub {\isadigit{1}}} y
    \isa{{\isasymbeta}\isactrlsub {\isadigit{2}}} tal que la fórmula pertenece a \isa{S}. Como el conjunto \isa{S} está contenido en \isa{S{\isacharprime}}, tenemos 
    que la fórmula pertence, a su vez, a \isa{S{\isacharprime}}. De este modo, se verifica que o bien \isa{{\isacharbraceleft}{\isasymbeta}\isactrlsub {\isadigit{1}}{\isacharbraceright}\ {\isasymunion}\ S{\isacharprime}}
    pertenece a \isa{C} o bien \isa{{\isacharbraceleft}{\isasymbeta}\isactrlsub {\isadigit{2}}{\isacharbraceright}\ {\isasymunion}\ S{\isacharprime}} pertence a \isa{C}. Por eliminación de la disyunción anterior, 
    vamos a probar que o bien \isa{{\isacharbraceleft}{\isasymbeta}\isactrlsub {\isadigit{1}}{\isacharbraceright}\ {\isasymunion}\ S} pertenece a \isa{C{\isacharprime}} o bien \isa{{\isacharbraceleft}{\isasymbeta}\isactrlsub {\isadigit{2}}{\isacharbraceright}\ {\isasymunion}\ S} pertenece a \isa{C{\isacharprime}}.
    \begin{itemize}
      \item Supongamos, en primer lugar, que \isa{{\isacharbraceleft}{\isasymbeta}\isactrlsub {\isadigit{1}}{\isacharbraceright}\ {\isasymunion}\ S{\isacharprime}} pertenece a \isa{C}. Puesto que el conjunto \isa{S}
      está contenido en \isa{S{\isacharprime}}, se observa fácilmente que \isa{{\isacharbraceleft}{\isasymbeta}\isactrlsub {\isadigit{1}}{\isacharbraceright}\ {\isasymunion}\ S} está contenido en\\ \isa{{\isacharbraceleft}{\isasymbeta}\isactrlsub {\isadigit{1}}{\isacharbraceright}\ {\isasymunion}\ S{\isacharprime}}.
      Por definición de la colección \isa{C{\isacharprime}}, tenemos que \isa{{\isacharbraceleft}{\isasymbeta}\isactrlsub {\isadigit{1}}{\isacharbraceright}\ {\isasymunion}\ S} pertenece a \isa{C{\isacharprime}}, ya que es
      subconjunto de \isa{{\isacharbraceleft}{\isasymbeta}\isactrlsub {\isadigit{1}}{\isacharbraceright}\ {\isasymunion}\ S{\isacharprime}} que pertenece a \isa{C}. Por tanto, hemos probado que o bien \isa{{\isacharbraceleft}{\isasymbeta}\isactrlsub {\isadigit{1}}{\isacharbraceright}\ {\isasymunion}\ S} 
      pertenece a \isa{C{\isacharprime}} o bien \isa{{\isacharbraceleft}{\isasymbeta}\isactrlsub {\isadigit{2}}{\isacharbraceright}\ {\isasymunion}\ S} pertenece a \isa{C{\isacharprime}}.
      \item Supongamos, finalmente, que \isa{{\isacharbraceleft}{\isasymbeta}\isactrlsub {\isadigit{2}}{\isacharbraceright}\ {\isasymunion}\ S{\isacharprime}} pertenece a \isa{C}. Análogamente obtenemos que
      \isa{{\isacharbraceleft}{\isasymbeta}\isactrlsub {\isadigit{2}}{\isacharbraceright}\ {\isasymunion}\ S} está contenido en \isa{{\isacharbraceleft}{\isasymbeta}\isactrlsub {\isadigit{2}}{\isacharbraceright}\ {\isasymunion}\ S{\isacharprime}}, luego \isa{{\isacharbraceleft}{\isasymbeta}\isactrlsub {\isadigit{2}}{\isacharbraceright}\ {\isasymunion}\ S} pertenece a \isa{C{\isacharprime}} por definición.
      Por tanto, o bien \isa{{\isacharbraceleft}{\isasymbeta}\isactrlsub {\isadigit{1}}{\isacharbraceright}\ {\isasymunion}\ S} pertenece a \isa{C{\isacharprime}} o bien \isa{{\isacharbraceleft}{\isasymbeta}\isactrlsub {\isadigit{2}}{\isacharbraceright}\ {\isasymunion}\ S} pertenece a \isa{C{\isacharprime}}.
    \end{itemize}
    De esta manera, queda probado que dada una fórmula de tipo \isa{{\isasymbeta}} y componentes \isa{{\isasymbeta}\isactrlsub {\isadigit{1}}} y \isa{{\isasymbeta}\isactrlsub {\isadigit{2}}} tal que
    pertenezca al conjunto \isa{S}, se verifica que o bien \isa{{\isacharbraceleft}{\isasymbeta}\isactrlsub {\isadigit{1}}{\isacharbraceright}\ {\isasymunion}\ S} pertenece a \isa{C{\isacharprime}} o bien \isa{{\isacharbraceleft}{\isasymbeta}\isactrlsub {\isadigit{2}}{\isacharbraceright}\ {\isasymunion}\ S}
    pertenece a \isa{C{\isacharprime}}.

  En conclusión, por el lema de caracterización de la propiedad de consistencia proposicional
  mediante la notación uniforme, queda probado que \isa{C{\isacharprime}} tiene la propiedad de consistencia
  proposicional. 

  Finalmente probemos que, además, \isa{C{\isacharprime}} es cerrada bajo subconjuntos. Por definición de ser cerrado
  bajo subconjuntos, basta probar que dado un conjunto perteneciente a \isa{C{\isacharprime}} verifica que todo 
  subconjunto suyo pertenece a \isa{C{\isacharprime}}. Consideremos \isa{S} un conjunto cualquiera de \isa{C{\isacharprime}}. Por
  definición de \isa{C{\isacharprime}}, existe un conjunto \isa{S{\isacharprime}} perteneciente a la colección \isa{C} tal que \isa{S} es
  subconjunto de \isa{S{\isacharprime}}. Sea \isa{S{\isacharprime}{\isacharprime}} un subconjunto cualquiera de \isa{S}. Como \isa{S} es subconjunto de \isa{S{\isacharprime}},
  se tiene que \isa{S{\isacharprime}{\isacharprime}} es, a su vez, subconjunto de \isa{S{\isacharprime}}. De este modo, existe un conjunto 
  perteneciente a la colección \isa{C} del cual \isa{S{\isacharprime}{\isacharprime}} es subconjunto. Por tanto, por definición de \isa{C{\isacharprime}}, 
  \isa{S{\isacharprime}{\isacharprime}} pertenece a la colección \isa{C{\isacharprime}}, como quería demostrar.
\end{demostracion}

  Procedamos con las demostraciones del lema en Isabelle/HOL.

  En primer lugar, vamos a introducir dos lemas auxiliares que emplearemos a lo largo de
  esta sección. El primero se trata de un lema similar al lema \isa{ballI} definido en Isabelle pero 
  considerando la relación de contención en lugar de la de pertenencia.%
\end{isamarkuptext}\isamarkuptrue%
\isacommand{lemma}\isamarkupfalse%
\ sallI{\isacharcolon}\ {\isachardoublequoteopen}{\isacharparenleft}{\isasymAnd}S{\isachardot}\ S\ {\isasymsubseteq}\ A\ {\isasymLongrightarrow}\ P\ S{\isacharparenright}\ {\isasymLongrightarrow}\ {\isasymforall}S\ {\isasymsubseteq}\ A{\isachardot}\ P\ S{\isachardoublequoteclose}\isanewline
%
\isadelimproof
\ \ %
\endisadelimproof
%
\isatagproof
\isacommand{by}\isamarkupfalse%
\ simp%
\endisatagproof
{\isafoldproof}%
%
\isadelimproof
%
\endisadelimproof
%
\begin{isamarkuptext}%
Por último definimos el siguiente lema auxiliar similar al lema \isa{bspec} de Isabelle/HOL
  considerando, análogamente, la relación de contención en lugar de la de pertenencia.%
\end{isamarkuptext}\isamarkuptrue%
\isacommand{lemma}\isamarkupfalse%
\ sspec{\isacharcolon}\ {\isachardoublequoteopen}{\isasymforall}S\ {\isasymsubseteq}\ A{\isachardot}\ P\ S\ {\isasymLongrightarrow}\ S\ {\isasymsubseteq}\ A\ {\isasymLongrightarrow}\ P\ S{\isachardoublequoteclose}\isanewline
%
\isadelimproof
\ \ %
\endisadelimproof
%
\isatagproof
\isacommand{by}\isamarkupfalse%
\ simp%
\endisatagproof
{\isafoldproof}%
%
\isadelimproof
%
\endisadelimproof
%
\begin{isamarkuptext}%
Veamos la prueba detallada del lema en Isabelle/HOL. Esta se fundamenta en tres lemas
  auxiliares: el primero prueba que la colección \isa{C} está contenida en \isa{C{\isacharprime}}, el segundo que
  \isa{C{\isacharprime}} tiene la propiedad de consistencia proposicional y, finalmente, el tercer lema demuestra que
  \isa{C{\isacharprime}} es cerrada bajo subconjuntos. En primer lugar, dada una colección cualquiera \isa{C}, definiremos 
  en Isabelle su extensión \isa{C{\isacharprime}} como sigue.%
\end{isamarkuptext}\isamarkuptrue%
\isacommand{definition}\isamarkupfalse%
\ extensionSC\ {\isacharcolon}{\isacharcolon}\ {\isachardoublequoteopen}{\isacharparenleft}{\isacharparenleft}{\isacharprime}a\ formula{\isacharparenright}\ set{\isacharparenright}\ set\ {\isasymRightarrow}\ {\isacharparenleft}{\isacharparenleft}{\isacharprime}a\ formula{\isacharparenright}\ set{\isacharparenright}\ set{\isachardoublequoteclose}\isanewline
\ \ \isakeyword{where}\ extensionSC{\isacharcolon}\ {\isachardoublequoteopen}extensionSC\ C\ {\isacharequal}\ {\isacharbraceleft}s{\isachardot}\ {\isasymexists}S{\isasymin}C{\isachardot}\ s\ {\isasymsubseteq}\ S{\isacharbraceright}{\isachardoublequoteclose}%
\begin{isamarkuptext}%
Una vez formalizada la extensión en Isabelle, comencemos probando de manera detallada que toda
  colección está contenida en su extensión así definida.%
\end{isamarkuptext}\isamarkuptrue%
\isacommand{lemma}\isamarkupfalse%
\ ex{\isadigit{1}}{\isacharunderscore}subset{\isacharcolon}\ {\isachardoublequoteopen}C\ {\isasymsubseteq}\ {\isacharparenleft}extensionSC\ C{\isacharparenright}{\isachardoublequoteclose}\isanewline
%
\isadelimproof
%
\endisadelimproof
%
\isatagproof
\isacommand{proof}\isamarkupfalse%
\ {\isacharparenleft}rule\ subsetI{\isacharparenright}\isanewline
\ \ \isacommand{fix}\isamarkupfalse%
\ s\isanewline
\ \ \isacommand{assume}\isamarkupfalse%
\ {\isachardoublequoteopen}s\ {\isasymin}\ C{\isachardoublequoteclose}\isanewline
\ \ \isacommand{have}\isamarkupfalse%
\ {\isachardoublequoteopen}s\ {\isasymsubseteq}\ s{\isachardoublequoteclose}\isanewline
\ \ \ \ \isacommand{by}\isamarkupfalse%
\ {\isacharparenleft}rule\ subset{\isacharunderscore}refl{\isacharparenright}\isanewline
\ \ \isacommand{then}\isamarkupfalse%
\ \isacommand{have}\isamarkupfalse%
\ {\isachardoublequoteopen}{\isasymexists}S{\isasymin}C{\isachardot}\ s\ {\isasymsubseteq}\ S{\isachardoublequoteclose}\isanewline
\ \ \ \ \isacommand{using}\isamarkupfalse%
\ {\isacartoucheopen}s\ {\isasymin}\ C{\isacartoucheclose}\ \isacommand{by}\isamarkupfalse%
\ {\isacharparenleft}rule\ bexI{\isacharparenright}\isanewline
\ \ \isacommand{thus}\isamarkupfalse%
\ {\isachardoublequoteopen}s\ {\isasymin}\ {\isacharparenleft}extensionSC\ C{\isacharparenright}{\isachardoublequoteclose}\isanewline
\ \ \ \ \isacommand{by}\isamarkupfalse%
\ {\isacharparenleft}simp\ only{\isacharcolon}\ mem{\isacharunderscore}Collect{\isacharunderscore}eq\ extensionSC{\isacharparenright}\isanewline
\isacommand{qed}\isamarkupfalse%
%
\endisatagproof
{\isafoldproof}%
%
\isadelimproof
%
\endisadelimproof
%
\begin{isamarkuptext}%
Prosigamos con la prueba del lema auxiliar que demuestra que \isa{C{\isacharprime}} tiene la propiedad
  de consistencia proposicional. Para ello, emplearemos un lema auxiliar que amplia el lema de 
  Isabelle \isa{insert{\isacharunderscore}is{\isacharunderscore}Un} para la unión de dos elementos y un conjunto, como se muestra a 
  continuación.%
\end{isamarkuptext}\isamarkuptrue%
\isacommand{lemma}\isamarkupfalse%
\ insertSetElem{\isacharcolon}\ {\isachardoublequoteopen}insert\ a\ {\isacharparenleft}insert\ b\ C{\isacharparenright}\ {\isacharequal}\ {\isacharbraceleft}a{\isacharcomma}b{\isacharbraceright}\ {\isasymunion}\ C{\isachardoublequoteclose}\isanewline
%
\isadelimproof
\ \ %
\endisadelimproof
%
\isatagproof
\isacommand{by}\isamarkupfalse%
\ simp%
\endisatagproof
{\isafoldproof}%
%
\isadelimproof
%
\endisadelimproof
%
\begin{isamarkuptext}%
Una vez introducido dicho lema auxiliar, podemos dar la prueba detallada del lema que 
  demuestra que \isa{C{\isacharprime}} tiene la propiedad de consistencia proposicional.%
\end{isamarkuptext}\isamarkuptrue%
\isacommand{lemma}\isamarkupfalse%
\ ex{\isadigit{1}}{\isacharunderscore}pcp{\isacharcolon}\ \isanewline
\ \ \isakeyword{assumes}\ {\isachardoublequoteopen}pcp\ C{\isachardoublequoteclose}\isanewline
\ \ \isakeyword{shows}\ {\isachardoublequoteopen}pcp\ {\isacharparenleft}extensionSC\ C{\isacharparenright}{\isachardoublequoteclose}\isanewline
%
\isadelimproof
%
\endisadelimproof
%
\isatagproof
\isacommand{proof}\isamarkupfalse%
\ {\isacharminus}\isanewline
\ \ \isacommand{have}\isamarkupfalse%
\ C{\isadigit{1}}{\isacharcolon}\ {\isachardoublequoteopen}C\ {\isasymsubseteq}\ {\isacharparenleft}extensionSC\ C{\isacharparenright}{\isachardoublequoteclose}\isanewline
\ \ \ \ \isacommand{by}\isamarkupfalse%
\ {\isacharparenleft}rule\ ex{\isadigit{1}}{\isacharunderscore}subset{\isacharparenright}\isanewline
\ \ \isacommand{show}\isamarkupfalse%
\ {\isachardoublequoteopen}pcp\ {\isacharparenleft}extensionSC\ C{\isacharparenright}{\isachardoublequoteclose}\isanewline
\ \ \isacommand{proof}\isamarkupfalse%
\ {\isacharparenleft}rule\ pcp{\isacharunderscore}alt{\isadigit{2}}{\isacharparenright}\isanewline
\ \ \ \ \isacommand{show}\isamarkupfalse%
\ {\isachardoublequoteopen}{\isasymforall}S\ {\isasymin}\ {\isacharparenleft}extensionSC\ C{\isacharparenright}{\isachardot}\ {\isacharparenleft}{\isasymbottom}\ {\isasymnotin}\ S\isanewline
\ \ \ \ {\isasymand}\ {\isacharparenleft}{\isasymforall}k{\isachardot}\ Atom\ k\ {\isasymin}\ S\ {\isasymlongrightarrow}\ \isactrlbold {\isasymnot}\ {\isacharparenleft}Atom\ k{\isacharparenright}\ {\isasymin}\ S\ {\isasymlongrightarrow}\ False{\isacharparenright}\isanewline
\ \ \ \ {\isasymand}\ {\isacharparenleft}{\isasymforall}F\ G\ H{\isachardot}\ Con\ F\ G\ H\ {\isasymlongrightarrow}\ F\ {\isasymin}\ S\ {\isasymlongrightarrow}\ {\isacharbraceleft}G{\isacharcomma}H{\isacharbraceright}\ {\isasymunion}\ S\ {\isasymin}\ {\isacharparenleft}extensionSC\ C{\isacharparenright}{\isacharparenright}\isanewline
\ \ \ \ {\isasymand}\ {\isacharparenleft}{\isasymforall}F\ G\ H{\isachardot}\ Dis\ F\ G\ H\ {\isasymlongrightarrow}\ F\ {\isasymin}\ S\ {\isasymlongrightarrow}\ {\isacharbraceleft}G{\isacharbraceright}\ {\isasymunion}\ S\ {\isasymin}\ {\isacharparenleft}extensionSC\ C{\isacharparenright}\ {\isasymor}\ {\isacharbraceleft}H{\isacharbraceright}\ {\isasymunion}\ S\ {\isasymin}\ {\isacharparenleft}extensionSC\ C{\isacharparenright}{\isacharparenright}{\isacharparenright}{\isachardoublequoteclose}\isanewline
\ \ \ \ \isacommand{proof}\isamarkupfalse%
\ {\isacharparenleft}rule\ ballI{\isacharparenright}\isanewline
\ \ \ \ \ \ \isacommand{fix}\isamarkupfalse%
\ S{\isacharprime}\isanewline
\ \ \ \ \ \ \isacommand{assume}\isamarkupfalse%
\ {\isachardoublequoteopen}S{\isacharprime}\ {\isasymin}\ {\isacharparenleft}extensionSC\ C{\isacharparenright}{\isachardoublequoteclose}\isanewline
\ \ \ \ \ \ \isacommand{then}\isamarkupfalse%
\ \isacommand{have}\isamarkupfalse%
\ {\isadigit{1}}{\isacharcolon}{\isachardoublequoteopen}{\isasymexists}S\ {\isasymin}\ C{\isachardot}\ S{\isacharprime}\ {\isasymsubseteq}\ S{\isachardoublequoteclose}\isanewline
\ \ \ \ \ \ \ \ \isacommand{unfolding}\isamarkupfalse%
\ extensionSC\ \isacommand{by}\isamarkupfalse%
\ {\isacharparenleft}rule\ CollectD{\isacharparenright}\ \ \isanewline
\ \ \ \ \ \ \isacommand{obtain}\isamarkupfalse%
\ S\ \isakeyword{where}\ {\isachardoublequoteopen}S\ {\isasymin}\ C{\isachardoublequoteclose}\ {\isachardoublequoteopen}S{\isacharprime}\ {\isasymsubseteq}\ S{\isachardoublequoteclose}\isanewline
\ \ \ \ \ \ \ \ \isacommand{using}\isamarkupfalse%
\ {\isadigit{1}}\ \isacommand{by}\isamarkupfalse%
\ {\isacharparenleft}rule\ bexE{\isacharparenright}\isanewline
\ \ \ \ \ \ \isacommand{have}\isamarkupfalse%
\ {\isachardoublequoteopen}{\isasymforall}S\ {\isasymin}\ C{\isachardot}\isanewline
\ \ \ \ \ \ {\isasymbottom}\ {\isasymnotin}\ S\isanewline
\ \ \ \ \ \ {\isasymand}\ {\isacharparenleft}{\isasymforall}k{\isachardot}\ Atom\ k\ {\isasymin}\ S\ {\isasymlongrightarrow}\ \isactrlbold {\isasymnot}\ {\isacharparenleft}Atom\ k{\isacharparenright}\ {\isasymin}\ S\ {\isasymlongrightarrow}\ False{\isacharparenright}\isanewline
\ \ \ \ \ \ {\isasymand}\ {\isacharparenleft}{\isasymforall}F\ G\ H{\isachardot}\ Con\ F\ G\ H\ {\isasymlongrightarrow}\ F\ {\isasymin}\ S\ {\isasymlongrightarrow}\ {\isacharbraceleft}G{\isacharcomma}H{\isacharbraceright}\ {\isasymunion}\ S\ {\isasymin}\ C{\isacharparenright}\isanewline
\ \ \ \ \ \ {\isasymand}\ {\isacharparenleft}{\isasymforall}F\ G\ H{\isachardot}\ Dis\ F\ G\ H\ {\isasymlongrightarrow}\ F\ {\isasymin}\ S\ {\isasymlongrightarrow}\ {\isacharbraceleft}G{\isacharbraceright}\ {\isasymunion}\ S\ {\isasymin}\ C\ {\isasymor}\ {\isacharbraceleft}H{\isacharbraceright}\ {\isasymunion}\ S\ {\isasymin}\ C{\isacharparenright}{\isachardoublequoteclose}\isanewline
\ \ \ \ \ \ \ \ \isacommand{using}\isamarkupfalse%
\ assms\ \isacommand{by}\isamarkupfalse%
\ {\isacharparenleft}rule\ pcp{\isacharunderscore}alt{\isadigit{1}}{\isacharparenright}\isanewline
\ \ \ \ \ \ \isacommand{then}\isamarkupfalse%
\ \isacommand{have}\isamarkupfalse%
\ H{\isacharcolon}{\isachardoublequoteopen}{\isasymbottom}\ {\isasymnotin}\ S\isanewline
\ \ \ \ \ \ {\isasymand}\ {\isacharparenleft}{\isasymforall}k{\isachardot}\ Atom\ k\ {\isasymin}\ S\ {\isasymlongrightarrow}\ \isactrlbold {\isasymnot}\ {\isacharparenleft}Atom\ k{\isacharparenright}\ {\isasymin}\ S\ {\isasymlongrightarrow}\ False{\isacharparenright}\isanewline
\ \ \ \ \ \ {\isasymand}\ {\isacharparenleft}{\isasymforall}F\ G\ H{\isachardot}\ Con\ F\ G\ H\ {\isasymlongrightarrow}\ F\ {\isasymin}\ S\ {\isasymlongrightarrow}\ {\isacharbraceleft}G{\isacharcomma}H{\isacharbraceright}\ {\isasymunion}\ S\ {\isasymin}\ C{\isacharparenright}\isanewline
\ \ \ \ \ \ {\isasymand}\ {\isacharparenleft}{\isasymforall}F\ G\ H{\isachardot}\ Dis\ F\ G\ H\ {\isasymlongrightarrow}\ F\ {\isasymin}\ S\ {\isasymlongrightarrow}\ {\isacharbraceleft}G{\isacharbraceright}\ {\isasymunion}\ S\ {\isasymin}\ C\ {\isasymor}\ {\isacharbraceleft}H{\isacharbraceright}\ {\isasymunion}\ S\ {\isasymin}\ C{\isacharparenright}{\isachardoublequoteclose}\isanewline
\ \ \ \ \ \ \ \ \isacommand{using}\isamarkupfalse%
\ {\isacartoucheopen}S\ {\isasymin}\ C{\isacartoucheclose}\ \isacommand{by}\isamarkupfalse%
\ {\isacharparenleft}rule\ bspec{\isacharparenright}\isanewline
\ \ \ \ \ \ \isacommand{then}\isamarkupfalse%
\ \isacommand{have}\isamarkupfalse%
\ {\isachardoublequoteopen}{\isasymbottom}\ {\isasymnotin}\ S{\isachardoublequoteclose}\isanewline
\ \ \ \ \ \ \ \ \isacommand{by}\isamarkupfalse%
\ {\isacharparenleft}rule\ conjunct{\isadigit{1}}{\isacharparenright}\isanewline
\ \ \ \ \ \ \isacommand{have}\isamarkupfalse%
\ S{\isadigit{1}}{\isacharcolon}{\isachardoublequoteopen}{\isasymbottom}\ {\isasymnotin}\ S{\isacharprime}{\isachardoublequoteclose}\isanewline
\ \ \ \ \ \ \ \ \isacommand{using}\isamarkupfalse%
\ {\isacartoucheopen}S{\isacharprime}\ {\isasymsubseteq}\ S{\isacartoucheclose}\ {\isacartoucheopen}{\isasymbottom}\ {\isasymnotin}\ S{\isacartoucheclose}\ \isacommand{by}\isamarkupfalse%
\ {\isacharparenleft}rule\ contra{\isacharunderscore}subsetD{\isacharparenright}\isanewline
\ \ \ \ \ \ \isacommand{have}\isamarkupfalse%
\ Atom{\isacharcolon}{\isachardoublequoteopen}{\isasymforall}k{\isachardot}\ Atom\ k\ {\isasymin}\ S\ {\isasymlongrightarrow}\ \isactrlbold {\isasymnot}\ {\isacharparenleft}Atom\ k{\isacharparenright}\ {\isasymin}\ S\ {\isasymlongrightarrow}\ False{\isachardoublequoteclose}\isanewline
\ \ \ \ \ \ \ \ \isacommand{using}\isamarkupfalse%
\ H\ \isacommand{by}\isamarkupfalse%
\ {\isacharparenleft}iprover\ elim{\isacharcolon}\ conjunct{\isadigit{1}}\ conjunct{\isadigit{2}}{\isacharparenright}\isanewline
\ \ \ \ \ \ \isacommand{have}\isamarkupfalse%
\ S{\isadigit{2}}{\isacharcolon}{\isachardoublequoteopen}{\isasymforall}k{\isachardot}\ Atom\ k\ {\isasymin}\ S{\isacharprime}\ {\isasymlongrightarrow}\ \isactrlbold {\isasymnot}\ {\isacharparenleft}Atom\ k{\isacharparenright}\ {\isasymin}\ S{\isacharprime}\ {\isasymlongrightarrow}\ False{\isachardoublequoteclose}\isanewline
\ \ \ \ \ \ \isacommand{proof}\isamarkupfalse%
\ {\isacharparenleft}rule\ allI{\isacharparenright}\isanewline
\ \ \ \ \ \ \ \ \isacommand{fix}\isamarkupfalse%
\ k\isanewline
\ \ \ \ \ \ \ \ \isacommand{show}\isamarkupfalse%
\ {\isachardoublequoteopen}Atom\ k\ {\isasymin}\ S{\isacharprime}\ {\isasymlongrightarrow}\ \isactrlbold {\isasymnot}\ {\isacharparenleft}Atom\ k{\isacharparenright}\ {\isasymin}\ S{\isacharprime}\ {\isasymlongrightarrow}\ False{\isachardoublequoteclose}\isanewline
\ \ \ \ \ \ \ \ \isacommand{proof}\isamarkupfalse%
\ {\isacharparenleft}rule\ impI{\isacharparenright}{\isacharplus}\isanewline
\ \ \ \ \ \ \ \ \ \ \isacommand{assume}\isamarkupfalse%
\ {\isachardoublequoteopen}Atom\ k\ {\isasymin}\ S{\isacharprime}{\isachardoublequoteclose}\isanewline
\ \ \ \ \ \ \ \ \ \ \isacommand{assume}\isamarkupfalse%
\ {\isachardoublequoteopen}\isactrlbold {\isasymnot}\ {\isacharparenleft}Atom\ k{\isacharparenright}\ {\isasymin}\ S{\isacharprime}{\isachardoublequoteclose}\isanewline
\ \ \ \ \ \ \ \ \ \ \isacommand{have}\isamarkupfalse%
\ {\isachardoublequoteopen}Atom\ k\ {\isasymin}\ S{\isachardoublequoteclose}\ \isanewline
\ \ \ \ \ \ \ \ \ \ \ \ \isacommand{using}\isamarkupfalse%
\ {\isacartoucheopen}S{\isacharprime}\ {\isasymsubseteq}\ S{\isacartoucheclose}\ {\isacartoucheopen}Atom\ k\ {\isasymin}\ S{\isacharprime}{\isacartoucheclose}\ \isacommand{by}\isamarkupfalse%
\ {\isacharparenleft}rule\ set{\isacharunderscore}mp{\isacharparenright}\isanewline
\ \ \ \ \ \ \ \ \ \ \isacommand{have}\isamarkupfalse%
\ {\isachardoublequoteopen}\isactrlbold {\isasymnot}\ {\isacharparenleft}Atom\ k{\isacharparenright}\ {\isasymin}\ S{\isachardoublequoteclose}\isanewline
\ \ \ \ \ \ \ \ \ \ \ \ \isacommand{using}\isamarkupfalse%
\ {\isacartoucheopen}S{\isacharprime}\ {\isasymsubseteq}\ S{\isacartoucheclose}\ {\isacartoucheopen}\isactrlbold {\isasymnot}\ {\isacharparenleft}Atom\ k{\isacharparenright}\ {\isasymin}\ S{\isacharprime}{\isacartoucheclose}\ \isacommand{by}\isamarkupfalse%
\ {\isacharparenleft}rule\ set{\isacharunderscore}mp{\isacharparenright}\isanewline
\ \ \ \ \ \ \ \ \ \ \isacommand{have}\isamarkupfalse%
\ {\isachardoublequoteopen}Atom\ k\ {\isasymin}\ S\ {\isasymlongrightarrow}\ \isactrlbold {\isasymnot}\ {\isacharparenleft}Atom\ k{\isacharparenright}\ {\isasymin}\ S\ {\isasymlongrightarrow}\ False{\isachardoublequoteclose}\isanewline
\ \ \ \ \ \ \ \ \ \ \ \ \isacommand{using}\isamarkupfalse%
\ Atom\ \isacommand{by}\isamarkupfalse%
\ {\isacharparenleft}rule\ allE{\isacharparenright}\isanewline
\ \ \ \ \ \ \ \ \ \ \isacommand{then}\isamarkupfalse%
\ \isacommand{have}\isamarkupfalse%
\ {\isachardoublequoteopen}\isactrlbold {\isasymnot}\ {\isacharparenleft}Atom\ k{\isacharparenright}\ {\isasymin}\ S\ {\isasymlongrightarrow}\ False{\isachardoublequoteclose}\isanewline
\ \ \ \ \ \ \ \ \ \ \ \ \isacommand{using}\isamarkupfalse%
\ {\isacartoucheopen}Atom\ k\ {\isasymin}\ S{\isacartoucheclose}\ \isacommand{by}\isamarkupfalse%
\ {\isacharparenleft}rule\ mp{\isacharparenright}\isanewline
\ \ \ \ \ \ \ \ \ \ \isacommand{thus}\isamarkupfalse%
\ {\isachardoublequoteopen}False{\isachardoublequoteclose}\isanewline
\ \ \ \ \ \ \ \ \ \ \ \ \isacommand{using}\isamarkupfalse%
\ {\isacartoucheopen}\isactrlbold {\isasymnot}\ {\isacharparenleft}Atom\ k{\isacharparenright}\ {\isasymin}\ S{\isacartoucheclose}\ \isacommand{by}\isamarkupfalse%
\ {\isacharparenleft}rule\ mp{\isacharparenright}\isanewline
\ \ \ \ \ \ \ \ \isacommand{qed}\isamarkupfalse%
\isanewline
\ \ \ \ \ \ \isacommand{qed}\isamarkupfalse%
\isanewline
\ \ \ \ \ \ \isacommand{have}\isamarkupfalse%
\ Con{\isacharcolon}{\isachardoublequoteopen}{\isasymforall}F\ G\ H{\isachardot}\ Con\ F\ G\ H\ {\isasymlongrightarrow}\ F\ {\isasymin}\ S\ {\isasymlongrightarrow}\ {\isacharbraceleft}G{\isacharcomma}H{\isacharbraceright}\ {\isasymunion}\ S\ {\isasymin}\ C{\isachardoublequoteclose}\isanewline
\ \ \ \ \ \ \ \ \isacommand{using}\isamarkupfalse%
\ H\ \isacommand{by}\isamarkupfalse%
\ {\isacharparenleft}iprover\ elim{\isacharcolon}\ conjunct{\isadigit{1}}\ conjunct{\isadigit{2}}{\isacharparenright}\isanewline
\ \ \ \ \ \ \isacommand{have}\isamarkupfalse%
\ S{\isadigit{3}}{\isacharcolon}{\isachardoublequoteopen}{\isasymforall}F\ G\ H{\isachardot}\ Con\ F\ G\ H\ {\isasymlongrightarrow}\ F\ {\isasymin}\ S{\isacharprime}\ {\isasymlongrightarrow}\ {\isacharbraceleft}G{\isacharcomma}H{\isacharbraceright}\ {\isasymunion}\ S{\isacharprime}\ {\isasymin}\ {\isacharparenleft}extensionSC\ C{\isacharparenright}{\isachardoublequoteclose}\isanewline
\ \ \ \ \ \ \isacommand{proof}\isamarkupfalse%
\ {\isacharparenleft}rule\ allI{\isacharparenright}{\isacharplus}\isanewline
\ \ \ \ \ \ \ \ \isacommand{fix}\isamarkupfalse%
\ F\ G\ H\isanewline
\ \ \ \ \ \ \ \ \isacommand{show}\isamarkupfalse%
\ {\isachardoublequoteopen}Con\ F\ G\ H\ {\isasymlongrightarrow}\ F\ {\isasymin}\ S{\isacharprime}\ {\isasymlongrightarrow}\ {\isacharbraceleft}G{\isacharcomma}H{\isacharbraceright}\ {\isasymunion}\ S{\isacharprime}\ {\isasymin}\ {\isacharparenleft}extensionSC\ C{\isacharparenright}{\isachardoublequoteclose}\isanewline
\ \ \ \ \ \ \ \ \isacommand{proof}\isamarkupfalse%
\ {\isacharparenleft}rule\ impI{\isacharparenright}{\isacharplus}\isanewline
\ \ \ \ \ \ \ \ \ \ \isacommand{assume}\isamarkupfalse%
\ {\isachardoublequoteopen}Con\ F\ G\ H{\isachardoublequoteclose}\isanewline
\ \ \ \ \ \ \ \ \ \ \isacommand{assume}\isamarkupfalse%
\ {\isachardoublequoteopen}F\ {\isasymin}\ S{\isacharprime}{\isachardoublequoteclose}\isanewline
\ \ \ \ \ \ \ \ \ \ \isacommand{have}\isamarkupfalse%
\ {\isachardoublequoteopen}F\ {\isasymin}\ S{\isachardoublequoteclose}\isanewline
\ \ \ \ \ \ \ \ \ \ \ \ \isacommand{using}\isamarkupfalse%
\ {\isacartoucheopen}S{\isacharprime}\ {\isasymsubseteq}\ S{\isacartoucheclose}\ {\isacartoucheopen}F\ {\isasymin}\ S{\isacharprime}{\isacartoucheclose}\ \isacommand{by}\isamarkupfalse%
\ {\isacharparenleft}rule\ set{\isacharunderscore}mp{\isacharparenright}\isanewline
\ \ \ \ \ \ \ \ \ \ \isacommand{have}\isamarkupfalse%
\ {\isachardoublequoteopen}Con\ F\ G\ H\ {\isasymlongrightarrow}\ F\ {\isasymin}\ S\ {\isasymlongrightarrow}\ {\isacharbraceleft}G{\isacharcomma}H{\isacharbraceright}\ {\isasymunion}\ S\ {\isasymin}\ C{\isachardoublequoteclose}\isanewline
\ \ \ \ \ \ \ \ \ \ \ \ \isacommand{using}\isamarkupfalse%
\ Con\ \isacommand{by}\isamarkupfalse%
\ {\isacharparenleft}iprover\ elim{\isacharcolon}\ allE{\isacharparenright}\isanewline
\ \ \ \ \ \ \ \ \ \ \isacommand{then}\isamarkupfalse%
\ \isacommand{have}\isamarkupfalse%
\ {\isachardoublequoteopen}F\ {\isasymin}\ S\ {\isasymlongrightarrow}\ {\isacharbraceleft}G{\isacharcomma}H{\isacharbraceright}\ {\isasymunion}\ S\ {\isasymin}\ C{\isachardoublequoteclose}\isanewline
\ \ \ \ \ \ \ \ \ \ \ \ \isacommand{using}\isamarkupfalse%
\ {\isacartoucheopen}Con\ F\ G\ H{\isacartoucheclose}\ \isacommand{by}\isamarkupfalse%
\ {\isacharparenleft}rule\ mp{\isacharparenright}\isanewline
\ \ \ \ \ \ \ \ \ \ \isacommand{then}\isamarkupfalse%
\ \isacommand{have}\isamarkupfalse%
\ {\isachardoublequoteopen}{\isacharbraceleft}G{\isacharcomma}H{\isacharbraceright}\ {\isasymunion}\ S\ {\isasymin}\ C{\isachardoublequoteclose}\isanewline
\ \ \ \ \ \ \ \ \ \ \ \ \isacommand{using}\isamarkupfalse%
\ {\isacartoucheopen}F\ {\isasymin}\ S{\isacartoucheclose}\ \isacommand{by}\isamarkupfalse%
\ {\isacharparenleft}rule\ mp{\isacharparenright}\isanewline
\ \ \ \ \ \ \ \ \ \ \isacommand{have}\isamarkupfalse%
\ {\isachardoublequoteopen}S{\isacharprime}\ {\isasymsubseteq}\ insert\ H\ S{\isachardoublequoteclose}\isanewline
\ \ \ \ \ \ \ \ \ \ \ \ \isacommand{using}\isamarkupfalse%
\ {\isacartoucheopen}S{\isacharprime}\ {\isasymsubseteq}\ S{\isacartoucheclose}\ \isacommand{by}\isamarkupfalse%
\ {\isacharparenleft}rule\ subset{\isacharunderscore}insertI{\isadigit{2}}{\isacharparenright}\ \isanewline
\ \ \ \ \ \ \ \ \ \ \isacommand{then}\isamarkupfalse%
\ \isacommand{have}\isamarkupfalse%
\ {\isachardoublequoteopen}insert\ H\ S{\isacharprime}\ {\isasymsubseteq}\ insert\ H\ {\isacharparenleft}insert\ H\ S{\isacharparenright}{\isachardoublequoteclose}\isanewline
\ \ \ \ \ \ \ \ \ \ \ \ \isacommand{by}\isamarkupfalse%
\ {\isacharparenleft}simp\ only{\isacharcolon}\ insert{\isacharunderscore}mono{\isacharparenright}\isanewline
\ \ \ \ \ \ \ \ \ \ \isacommand{then}\isamarkupfalse%
\ \isacommand{have}\isamarkupfalse%
\ {\isachardoublequoteopen}insert\ H\ S{\isacharprime}\ {\isasymsubseteq}\ insert\ H\ S{\isachardoublequoteclose}\isanewline
\ \ \ \ \ \ \ \ \ \ \ \ \isacommand{by}\isamarkupfalse%
\ {\isacharparenleft}simp\ only{\isacharcolon}\ insert{\isacharunderscore}absorb{\isadigit{2}}{\isacharparenright}\isanewline
\ \ \ \ \ \ \ \ \ \ \isacommand{then}\isamarkupfalse%
\ \isacommand{have}\isamarkupfalse%
\ {\isachardoublequoteopen}insert\ G\ {\isacharparenleft}insert\ H\ S{\isacharprime}{\isacharparenright}\ {\isasymsubseteq}\ insert\ G\ {\isacharparenleft}insert\ H\ S{\isacharparenright}{\isachardoublequoteclose}\isanewline
\ \ \ \ \ \ \ \ \ \ \ \ \isacommand{by}\isamarkupfalse%
\ {\isacharparenleft}simp\ only{\isacharcolon}\ insert{\isacharunderscore}mono{\isacharparenright}\isanewline
\ \ \ \ \ \ \ \ \ \ \isacommand{have}\isamarkupfalse%
\ A{\isacharcolon}{\isachardoublequoteopen}insert\ G\ {\isacharparenleft}insert\ H\ S{\isacharprime}{\isacharparenright}\ {\isacharequal}\ {\isacharbraceleft}G{\isacharcomma}H{\isacharbraceright}\ {\isasymunion}\ S{\isacharprime}{\isachardoublequoteclose}\isanewline
\ \ \ \ \ \ \ \ \ \ \ \ \isacommand{by}\isamarkupfalse%
\ {\isacharparenleft}rule\ insertSetElem{\isacharparenright}\ \isanewline
\ \ \ \ \ \ \ \ \ \ \isacommand{have}\isamarkupfalse%
\ B{\isacharcolon}{\isachardoublequoteopen}insert\ G\ {\isacharparenleft}insert\ H\ S{\isacharparenright}\ {\isacharequal}\ {\isacharbraceleft}G{\isacharcomma}H{\isacharbraceright}\ {\isasymunion}\ S{\isachardoublequoteclose}\isanewline
\ \ \ \ \ \ \ \ \ \ \ \ \isacommand{by}\isamarkupfalse%
\ {\isacharparenleft}rule\ insertSetElem{\isacharparenright}\isanewline
\ \ \ \ \ \ \ \ \ \ \isacommand{have}\isamarkupfalse%
\ {\isachardoublequoteopen}{\isacharbraceleft}G{\isacharcomma}H{\isacharbraceright}\ {\isasymunion}\ S{\isacharprime}\ {\isasymsubseteq}\ {\isacharbraceleft}G{\isacharcomma}H{\isacharbraceright}\ {\isasymunion}\ S{\isachardoublequoteclose}\ \isanewline
\ \ \ \ \ \ \ \ \ \ \ \ \isacommand{using}\isamarkupfalse%
\ {\isacartoucheopen}insert\ G\ {\isacharparenleft}insert\ H\ S{\isacharprime}{\isacharparenright}\ {\isasymsubseteq}\ insert\ G\ {\isacharparenleft}insert\ H\ S{\isacharparenright}{\isacartoucheclose}\ \isacommand{by}\isamarkupfalse%
\ {\isacharparenleft}simp\ only{\isacharcolon}\ A\ B{\isacharparenright}\isanewline
\ \ \ \ \ \ \ \ \ \ \isacommand{then}\isamarkupfalse%
\ \isacommand{have}\isamarkupfalse%
\ {\isachardoublequoteopen}{\isasymexists}S\ {\isasymin}\ C{\isachardot}\ {\isacharbraceleft}G{\isacharcomma}H{\isacharbraceright}\ {\isasymunion}\ S{\isacharprime}\ {\isasymsubseteq}\ S{\isachardoublequoteclose}\isanewline
\ \ \ \ \ \ \ \ \ \ \ \ \isacommand{using}\isamarkupfalse%
\ {\isacartoucheopen}{\isacharbraceleft}G{\isacharcomma}H{\isacharbraceright}\ {\isasymunion}\ S\ {\isasymin}\ C{\isacartoucheclose}\ \isacommand{by}\isamarkupfalse%
\ {\isacharparenleft}rule\ bexI{\isacharparenright}\isanewline
\ \ \ \ \ \ \ \ \ \ \isacommand{thus}\isamarkupfalse%
\ {\isachardoublequoteopen}{\isacharbraceleft}G{\isacharcomma}H{\isacharbraceright}\ {\isasymunion}\ S{\isacharprime}\ {\isasymin}\ {\isacharparenleft}extensionSC\ C{\isacharparenright}{\isachardoublequoteclose}\ \isanewline
\ \ \ \ \ \ \ \ \ \ \ \ \isacommand{unfolding}\isamarkupfalse%
\ extensionSC\ \isacommand{by}\isamarkupfalse%
\ {\isacharparenleft}rule\ CollectI{\isacharparenright}\isanewline
\ \ \ \ \ \ \ \ \isacommand{qed}\isamarkupfalse%
\isanewline
\ \ \ \ \ \ \isacommand{qed}\isamarkupfalse%
\isanewline
\ \ \ \ \ \ \isacommand{have}\isamarkupfalse%
\ Dis{\isacharcolon}{\isachardoublequoteopen}{\isasymforall}F\ G\ H{\isachardot}\ Dis\ F\ G\ H\ {\isasymlongrightarrow}\ F\ {\isasymin}\ S\ {\isasymlongrightarrow}\ {\isacharbraceleft}G{\isacharbraceright}\ {\isasymunion}\ S\ {\isasymin}\ C\ {\isasymor}\ {\isacharbraceleft}H{\isacharbraceright}\ {\isasymunion}\ S\ {\isasymin}\ C{\isachardoublequoteclose}\isanewline
\ \ \ \ \ \ \ \ \isacommand{using}\isamarkupfalse%
\ H\ \isacommand{by}\isamarkupfalse%
\ {\isacharparenleft}iprover\ elim{\isacharcolon}\ conjunct{\isadigit{2}}{\isacharparenright}\isanewline
\ \ \ \ \ \ \isacommand{have}\isamarkupfalse%
\ S{\isadigit{4}}{\isacharcolon}{\isachardoublequoteopen}{\isasymforall}F\ G\ H{\isachardot}\ Dis\ F\ G\ H\ {\isasymlongrightarrow}\ F\ {\isasymin}\ S{\isacharprime}\ {\isasymlongrightarrow}\ {\isacharbraceleft}G{\isacharbraceright}\ {\isasymunion}\ S{\isacharprime}\ {\isasymin}\ {\isacharparenleft}extensionSC\ C{\isacharparenright}\ {\isasymor}\ {\isacharbraceleft}H{\isacharbraceright}\ {\isasymunion}\ S{\isacharprime}\ {\isasymin}\ {\isacharparenleft}extensionSC\ C{\isacharparenright}{\isachardoublequoteclose}\isanewline
\ \ \ \ \ \ \isacommand{proof}\isamarkupfalse%
\ {\isacharparenleft}rule\ allI{\isacharparenright}{\isacharplus}\isanewline
\ \ \ \ \ \ \ \ \isacommand{fix}\isamarkupfalse%
\ F\ G\ H\isanewline
\ \ \ \ \ \ \ \ \isacommand{show}\isamarkupfalse%
\ {\isachardoublequoteopen}Dis\ F\ G\ H\ {\isasymlongrightarrow}\ F\ {\isasymin}\ S{\isacharprime}\ {\isasymlongrightarrow}\ {\isacharbraceleft}G{\isacharbraceright}\ {\isasymunion}\ S{\isacharprime}\ {\isasymin}\ {\isacharparenleft}extensionSC\ C{\isacharparenright}\ {\isasymor}\ {\isacharbraceleft}H{\isacharbraceright}\ {\isasymunion}\ S{\isacharprime}\ {\isasymin}\ {\isacharparenleft}extensionSC\ C{\isacharparenright}{\isachardoublequoteclose}\isanewline
\ \ \ \ \ \ \ \ \isacommand{proof}\isamarkupfalse%
\ {\isacharparenleft}rule\ impI{\isacharparenright}{\isacharplus}\isanewline
\ \ \ \ \ \ \ \ \ \ \isacommand{assume}\isamarkupfalse%
\ {\isachardoublequoteopen}Dis\ F\ G\ H{\isachardoublequoteclose}\isanewline
\ \ \ \ \ \ \ \ \ \ \isacommand{assume}\isamarkupfalse%
\ {\isachardoublequoteopen}F\ {\isasymin}\ S{\isacharprime}{\isachardoublequoteclose}\isanewline
\ \ \ \ \ \ \ \ \ \ \isacommand{have}\isamarkupfalse%
\ {\isachardoublequoteopen}F\ {\isasymin}\ S{\isachardoublequoteclose}\isanewline
\ \ \ \ \ \ \ \ \ \ \ \ \isacommand{using}\isamarkupfalse%
\ {\isacartoucheopen}S{\isacharprime}\ {\isasymsubseteq}\ S{\isacartoucheclose}\ {\isacartoucheopen}F\ {\isasymin}\ S{\isacharprime}{\isacartoucheclose}\ \isacommand{by}\isamarkupfalse%
\ {\isacharparenleft}rule\ set{\isacharunderscore}mp{\isacharparenright}\isanewline
\ \ \ \ \ \ \ \ \ \ \isacommand{have}\isamarkupfalse%
\ {\isachardoublequoteopen}Dis\ F\ G\ H\ {\isasymlongrightarrow}\ F\ {\isasymin}\ S\ {\isasymlongrightarrow}\ {\isacharbraceleft}G{\isacharbraceright}\ {\isasymunion}\ S\ {\isasymin}\ C\ {\isasymor}\ {\isacharbraceleft}H{\isacharbraceright}\ {\isasymunion}\ S\ {\isasymin}\ C{\isachardoublequoteclose}\isanewline
\ \ \ \ \ \ \ \ \ \ \ \ \isacommand{using}\isamarkupfalse%
\ Dis\ \isacommand{by}\isamarkupfalse%
\ {\isacharparenleft}iprover\ elim{\isacharcolon}\ allE{\isacharparenright}\isanewline
\ \ \ \ \ \ \ \ \ \ \isacommand{then}\isamarkupfalse%
\ \isacommand{have}\isamarkupfalse%
\ {\isachardoublequoteopen}F\ {\isasymin}\ S\ {\isasymlongrightarrow}\ {\isacharbraceleft}G{\isacharbraceright}\ {\isasymunion}\ S\ {\isasymin}\ C\ {\isasymor}\ {\isacharbraceleft}H{\isacharbraceright}\ {\isasymunion}\ S\ {\isasymin}\ C{\isachardoublequoteclose}\isanewline
\ \ \ \ \ \ \ \ \ \ \ \ \isacommand{using}\isamarkupfalse%
\ {\isacartoucheopen}Dis\ F\ G\ H{\isacartoucheclose}\ \isacommand{by}\isamarkupfalse%
\ {\isacharparenleft}rule\ mp{\isacharparenright}\isanewline
\ \ \ \ \ \ \ \ \ \ \isacommand{then}\isamarkupfalse%
\ \isacommand{have}\isamarkupfalse%
\ {\isadigit{9}}{\isacharcolon}{\isachardoublequoteopen}{\isacharbraceleft}G{\isacharbraceright}\ {\isasymunion}\ S\ {\isasymin}\ C\ {\isasymor}\ {\isacharbraceleft}H{\isacharbraceright}\ {\isasymunion}\ S\ {\isasymin}\ C{\isachardoublequoteclose}\isanewline
\ \ \ \ \ \ \ \ \ \ \ \ \isacommand{using}\isamarkupfalse%
\ {\isacartoucheopen}F\ {\isasymin}\ S{\isacartoucheclose}\ \isacommand{by}\isamarkupfalse%
\ {\isacharparenleft}rule\ mp{\isacharparenright}\isanewline
\ \ \ \ \ \ \ \ \ \ \isacommand{show}\isamarkupfalse%
\ {\isachardoublequoteopen}{\isacharbraceleft}G{\isacharbraceright}\ {\isasymunion}\ S{\isacharprime}\ {\isasymin}\ {\isacharparenleft}extensionSC\ C{\isacharparenright}\ {\isasymor}\ {\isacharbraceleft}H{\isacharbraceright}\ {\isasymunion}\ S{\isacharprime}\ {\isasymin}\ {\isacharparenleft}extensionSC\ C{\isacharparenright}{\isachardoublequoteclose}\isanewline
\ \ \ \ \ \ \ \ \ \ \ \ \isacommand{using}\isamarkupfalse%
\ {\isadigit{9}}\isanewline
\ \ \ \ \ \ \ \ \ \ \isacommand{proof}\isamarkupfalse%
\ {\isacharparenleft}rule\ disjE{\isacharparenright}\isanewline
\ \ \ \ \ \ \ \ \ \ \ \ \isacommand{assume}\isamarkupfalse%
\ {\isachardoublequoteopen}{\isacharbraceleft}G{\isacharbraceright}\ {\isasymunion}\ S\ {\isasymin}\ C{\isachardoublequoteclose}\isanewline
\ \ \ \ \ \ \ \ \ \ \ \ \isacommand{have}\isamarkupfalse%
\ {\isachardoublequoteopen}insert\ G\ S{\isacharprime}\ {\isasymsubseteq}\ insert\ G\ S{\isachardoublequoteclose}\isanewline
\ \ \ \ \ \ \ \ \ \ \ \ \ \ \isacommand{using}\isamarkupfalse%
\ {\isacartoucheopen}S{\isacharprime}\ {\isasymsubseteq}\ S{\isacartoucheclose}\ \isacommand{by}\isamarkupfalse%
\ {\isacharparenleft}simp\ only{\isacharcolon}\ insert{\isacharunderscore}mono{\isacharparenright}\isanewline
\ \ \ \ \ \ \ \ \ \ \ \ \isacommand{have}\isamarkupfalse%
\ C{\isacharcolon}{\isachardoublequoteopen}insert\ G\ S{\isacharprime}\ {\isacharequal}\ {\isacharbraceleft}G{\isacharbraceright}\ {\isasymunion}\ S{\isacharprime}{\isachardoublequoteclose}\isanewline
\ \ \ \ \ \ \ \ \ \ \ \ \ \ \isacommand{by}\isamarkupfalse%
\ {\isacharparenleft}rule\ insert{\isacharunderscore}is{\isacharunderscore}Un{\isacharparenright}\isanewline
\ \ \ \ \ \ \ \ \ \ \ \ \isacommand{have}\isamarkupfalse%
\ D{\isacharcolon}{\isachardoublequoteopen}insert\ G\ S\ {\isacharequal}\ {\isacharbraceleft}G{\isacharbraceright}\ {\isasymunion}\ S{\isachardoublequoteclose}\isanewline
\ \ \ \ \ \ \ \ \ \ \ \ \ \ \isacommand{by}\isamarkupfalse%
\ {\isacharparenleft}rule\ insert{\isacharunderscore}is{\isacharunderscore}Un{\isacharparenright}\isanewline
\ \ \ \ \ \ \ \ \ \ \ \ \isacommand{have}\isamarkupfalse%
\ {\isachardoublequoteopen}{\isacharbraceleft}G{\isacharbraceright}\ {\isasymunion}\ S{\isacharprime}\ {\isasymsubseteq}\ {\isacharbraceleft}G{\isacharbraceright}\ {\isasymunion}\ S{\isachardoublequoteclose}\isanewline
\ \ \ \ \ \ \ \ \ \ \ \ \ \ \isacommand{using}\isamarkupfalse%
\ {\isacartoucheopen}insert\ G\ S{\isacharprime}\ {\isasymsubseteq}\ insert\ G\ S{\isacartoucheclose}\ \isacommand{by}\isamarkupfalse%
\ {\isacharparenleft}simp\ only{\isacharcolon}\ C\ D{\isacharparenright}\isanewline
\ \ \ \ \ \ \ \ \ \ \ \ \isacommand{then}\isamarkupfalse%
\ \isacommand{have}\isamarkupfalse%
\ {\isachardoublequoteopen}{\isasymexists}S\ {\isasymin}\ C{\isachardot}\ {\isacharbraceleft}G{\isacharbraceright}\ {\isasymunion}\ S{\isacharprime}\ {\isasymsubseteq}\ S{\isachardoublequoteclose}\isanewline
\ \ \ \ \ \ \ \ \ \ \ \ \ \ \isacommand{using}\isamarkupfalse%
\ {\isacartoucheopen}{\isacharbraceleft}G{\isacharbraceright}\ {\isasymunion}\ S\ {\isasymin}\ C{\isacartoucheclose}\ \isacommand{by}\isamarkupfalse%
\ {\isacharparenleft}rule\ bexI{\isacharparenright}\isanewline
\ \ \ \ \ \ \ \ \ \ \ \ \isacommand{then}\isamarkupfalse%
\ \isacommand{have}\isamarkupfalse%
\ {\isachardoublequoteopen}{\isacharbraceleft}G{\isacharbraceright}\ {\isasymunion}\ S{\isacharprime}\ {\isasymin}\ {\isacharparenleft}extensionSC\ C{\isacharparenright}{\isachardoublequoteclose}\isanewline
\ \ \ \ \ \ \ \ \ \ \ \ \ \ \isacommand{unfolding}\isamarkupfalse%
\ extensionSC\ \isacommand{by}\isamarkupfalse%
\ {\isacharparenleft}rule\ CollectI{\isacharparenright}\isanewline
\ \ \ \ \ \ \ \ \ \ \ \ \isacommand{thus}\isamarkupfalse%
\ {\isachardoublequoteopen}{\isacharbraceleft}G{\isacharbraceright}\ {\isasymunion}\ S{\isacharprime}\ {\isasymin}\ {\isacharparenleft}extensionSC\ C{\isacharparenright}\ {\isasymor}\ {\isacharbraceleft}H{\isacharbraceright}\ {\isasymunion}\ S{\isacharprime}\ {\isasymin}\ {\isacharparenleft}extensionSC\ C{\isacharparenright}{\isachardoublequoteclose}\isanewline
\ \ \ \ \ \ \ \ \ \ \ \ \ \ \isacommand{by}\isamarkupfalse%
\ {\isacharparenleft}rule\ disjI{\isadigit{1}}{\isacharparenright}\isanewline
\ \ \ \ \ \ \ \ \ \ \isacommand{next}\isamarkupfalse%
\isanewline
\ \ \ \ \ \ \ \ \ \ \ \ \isacommand{assume}\isamarkupfalse%
\ {\isachardoublequoteopen}{\isacharbraceleft}H{\isacharbraceright}\ {\isasymunion}\ S\ {\isasymin}\ C{\isachardoublequoteclose}\isanewline
\ \ \ \ \ \ \ \ \ \ \ \ \isacommand{have}\isamarkupfalse%
\ {\isachardoublequoteopen}insert\ H\ S{\isacharprime}\ {\isasymsubseteq}\ insert\ H\ S{\isachardoublequoteclose}\isanewline
\ \ \ \ \ \ \ \ \ \ \ \ \ \ \isacommand{using}\isamarkupfalse%
\ {\isacartoucheopen}S{\isacharprime}\ {\isasymsubseteq}\ S{\isacartoucheclose}\ \isacommand{by}\isamarkupfalse%
\ {\isacharparenleft}simp\ only{\isacharcolon}\ insert{\isacharunderscore}mono{\isacharparenright}\isanewline
\ \ \ \ \ \ \ \ \ \ \ \ \isacommand{have}\isamarkupfalse%
\ E{\isacharcolon}{\isachardoublequoteopen}insert\ H\ S{\isacharprime}\ {\isacharequal}\ {\isacharbraceleft}H{\isacharbraceright}\ {\isasymunion}\ S{\isacharprime}{\isachardoublequoteclose}\isanewline
\ \ \ \ \ \ \ \ \ \ \ \ \ \ \isacommand{by}\isamarkupfalse%
\ {\isacharparenleft}rule\ insert{\isacharunderscore}is{\isacharunderscore}Un{\isacharparenright}\isanewline
\ \ \ \ \ \ \ \ \ \ \ \ \isacommand{have}\isamarkupfalse%
\ F{\isacharcolon}{\isachardoublequoteopen}insert\ H\ S\ {\isacharequal}\ {\isacharbraceleft}H{\isacharbraceright}\ {\isasymunion}\ S{\isachardoublequoteclose}\isanewline
\ \ \ \ \ \ \ \ \ \ \ \ \ \ \isacommand{by}\isamarkupfalse%
\ {\isacharparenleft}rule\ insert{\isacharunderscore}is{\isacharunderscore}Un{\isacharparenright}\isanewline
\ \ \ \ \ \ \ \ \ \ \ \ \isacommand{then}\isamarkupfalse%
\ \isacommand{have}\isamarkupfalse%
\ {\isachardoublequoteopen}{\isacharbraceleft}H{\isacharbraceright}\ {\isasymunion}\ S{\isacharprime}\ {\isasymsubseteq}\ {\isacharbraceleft}H{\isacharbraceright}\ {\isasymunion}\ S{\isachardoublequoteclose}\isanewline
\ \ \ \ \ \ \ \ \ \ \ \ \ \ \isacommand{using}\isamarkupfalse%
\ {\isacartoucheopen}insert\ H\ S{\isacharprime}\ {\isasymsubseteq}\ insert\ H\ S{\isacartoucheclose}\ \isacommand{by}\isamarkupfalse%
\ {\isacharparenleft}simp\ only{\isacharcolon}\ E\ F{\isacharparenright}\isanewline
\ \ \ \ \ \ \ \ \ \ \ \ \isacommand{then}\isamarkupfalse%
\ \isacommand{have}\isamarkupfalse%
\ {\isachardoublequoteopen}{\isasymexists}S\ {\isasymin}\ C{\isachardot}\ {\isacharbraceleft}H{\isacharbraceright}\ {\isasymunion}\ S{\isacharprime}\ {\isasymsubseteq}\ S{\isachardoublequoteclose}\isanewline
\ \ \ \ \ \ \ \ \ \ \ \ \ \ \isacommand{using}\isamarkupfalse%
\ {\isacartoucheopen}{\isacharbraceleft}H{\isacharbraceright}\ {\isasymunion}\ S\ {\isasymin}\ C{\isacartoucheclose}\ \isacommand{by}\isamarkupfalse%
\ {\isacharparenleft}rule\ bexI{\isacharparenright}\isanewline
\ \ \ \ \ \ \ \ \ \ \ \ \isacommand{then}\isamarkupfalse%
\ \isacommand{have}\isamarkupfalse%
\ {\isachardoublequoteopen}{\isacharbraceleft}H{\isacharbraceright}\ {\isasymunion}\ S{\isacharprime}\ {\isasymin}\ {\isacharparenleft}extensionSC\ C{\isacharparenright}{\isachardoublequoteclose}\isanewline
\ \ \ \ \ \ \ \ \ \ \ \ \ \ \isacommand{unfolding}\isamarkupfalse%
\ extensionSC\ \isacommand{by}\isamarkupfalse%
\ {\isacharparenleft}rule\ CollectI{\isacharparenright}\isanewline
\ \ \ \ \ \ \ \ \ \ \ \ \isacommand{thus}\isamarkupfalse%
\ {\isachardoublequoteopen}{\isacharbraceleft}G{\isacharbraceright}\ {\isasymunion}\ S{\isacharprime}\ {\isasymin}\ {\isacharparenleft}extensionSC\ C{\isacharparenright}\ {\isasymor}\ {\isacharbraceleft}H{\isacharbraceright}\ {\isasymunion}\ S{\isacharprime}\ {\isasymin}\ {\isacharparenleft}extensionSC\ C{\isacharparenright}{\isachardoublequoteclose}\isanewline
\ \ \ \ \ \ \ \ \ \ \ \ \ \ \isacommand{by}\isamarkupfalse%
\ {\isacharparenleft}rule\ disjI{\isadigit{2}}{\isacharparenright}\isanewline
\ \ \ \ \ \ \ \ \ \ \isacommand{qed}\isamarkupfalse%
\isanewline
\ \ \ \ \ \ \ \ \isacommand{qed}\isamarkupfalse%
\isanewline
\ \ \ \ \ \ \isacommand{qed}\isamarkupfalse%
\isanewline
\ \ \ \ \ \ \isacommand{show}\isamarkupfalse%
\ {\isachardoublequoteopen}{\isasymbottom}\ {\isasymnotin}\ S{\isacharprime}\isanewline
\ \ \ \ {\isasymand}\ {\isacharparenleft}{\isasymforall}k{\isachardot}\ Atom\ k\ {\isasymin}\ S{\isacharprime}\ {\isasymlongrightarrow}\ \isactrlbold {\isasymnot}\ {\isacharparenleft}Atom\ k{\isacharparenright}\ {\isasymin}\ S{\isacharprime}\ {\isasymlongrightarrow}\ False{\isacharparenright}\isanewline
\ \ \ \ {\isasymand}\ {\isacharparenleft}{\isasymforall}F\ G\ H{\isachardot}\ Con\ F\ G\ H\ {\isasymlongrightarrow}\ F\ {\isasymin}\ S{\isacharprime}\ {\isasymlongrightarrow}\ {\isacharbraceleft}G{\isacharcomma}H{\isacharbraceright}\ {\isasymunion}\ S{\isacharprime}\ {\isasymin}\ {\isacharparenleft}extensionSC\ C{\isacharparenright}{\isacharparenright}\isanewline
\ \ \ \ {\isasymand}\ {\isacharparenleft}{\isasymforall}F\ G\ H{\isachardot}\ Dis\ F\ G\ H\ {\isasymlongrightarrow}\ F\ {\isasymin}\ S{\isacharprime}\ {\isasymlongrightarrow}\ {\isacharbraceleft}G{\isacharbraceright}\ {\isasymunion}\ S{\isacharprime}\ {\isasymin}\ {\isacharparenleft}extensionSC\ C{\isacharparenright}\ {\isasymor}\ {\isacharbraceleft}H{\isacharbraceright}\ {\isasymunion}\ S{\isacharprime}\ {\isasymin}\ {\isacharparenleft}extensionSC\ C{\isacharparenright}{\isacharparenright}{\isachardoublequoteclose}\isanewline
\ \ \ \ \ \ \ \ \isacommand{using}\isamarkupfalse%
\ S{\isadigit{1}}\ S{\isadigit{2}}\ S{\isadigit{3}}\ S{\isadigit{4}}\ \isacommand{by}\isamarkupfalse%
\ {\isacharparenleft}iprover\ intro{\isacharcolon}\ conjI{\isacharparenright}\isanewline
\ \ \ \ \isacommand{qed}\isamarkupfalse%
\isanewline
\ \ \isacommand{qed}\isamarkupfalse%
\isanewline
\isacommand{qed}\isamarkupfalse%
%
\endisatagproof
{\isafoldproof}%
%
\isadelimproof
%
\endisadelimproof
%
\begin{isamarkuptext}%
Finalmente, el siguiente lema auxiliar prueba que \isa{C{\isacharprime}} es cerrada bajo subconjuntos.%
\end{isamarkuptext}\isamarkuptrue%
\isacommand{lemma}\isamarkupfalse%
\ ex{\isadigit{1}}{\isacharunderscore}subset{\isacharunderscore}closed{\isacharcolon}\isanewline
\ \ \isakeyword{assumes}\ {\isachardoublequoteopen}pcp\ C{\isachardoublequoteclose}\isanewline
\ \ \isakeyword{shows}\ {\isachardoublequoteopen}subset{\isacharunderscore}closed\ {\isacharparenleft}extensionSC\ C{\isacharparenright}{\isachardoublequoteclose}\isanewline
%
\isadelimproof
\ \ %
\endisadelimproof
%
\isatagproof
\isacommand{unfolding}\isamarkupfalse%
\ subset{\isacharunderscore}closed{\isacharunderscore}def\isanewline
\isacommand{proof}\isamarkupfalse%
\ {\isacharparenleft}rule\ ballI{\isacharparenright}\isanewline
\ \ \isacommand{fix}\isamarkupfalse%
\ S{\isacharprime}\isanewline
\ \ \isacommand{assume}\isamarkupfalse%
\ {\isachardoublequoteopen}S{\isacharprime}\ {\isasymin}\ {\isacharparenleft}extensionSC\ C{\isacharparenright}{\isachardoublequoteclose}\isanewline
\ \ \isacommand{then}\isamarkupfalse%
\ \isacommand{have}\isamarkupfalse%
\ H{\isacharcolon}{\isachardoublequoteopen}{\isasymexists}S\ {\isasymin}\ C{\isachardot}\ S{\isacharprime}\ {\isasymsubseteq}\ S{\isachardoublequoteclose}\isanewline
\ \ \ \ \isacommand{unfolding}\isamarkupfalse%
\ extensionSC\ \isacommand{by}\isamarkupfalse%
\ {\isacharparenleft}rule\ CollectD{\isacharparenright}\isanewline
\ \ \isacommand{obtain}\isamarkupfalse%
\ S\ \isakeyword{where}\ {\isacartoucheopen}S\ {\isasymin}\ C{\isacartoucheclose}\ \isakeyword{and}\ {\isacartoucheopen}S{\isacharprime}\ {\isasymsubseteq}\ S{\isacartoucheclose}\ \isanewline
\ \ \ \ \isacommand{using}\isamarkupfalse%
\ H\ \isacommand{by}\isamarkupfalse%
\ {\isacharparenleft}rule\ bexE{\isacharparenright}\ \isanewline
\ \ \isacommand{show}\isamarkupfalse%
\ {\isachardoublequoteopen}{\isasymforall}S{\isacharprime}{\isacharprime}\ {\isasymsubseteq}\ S{\isacharprime}{\isachardot}\ S{\isacharprime}{\isacharprime}\ {\isasymin}\ {\isacharparenleft}extensionSC\ C{\isacharparenright}{\isachardoublequoteclose}\isanewline
\ \ \isacommand{proof}\isamarkupfalse%
\ {\isacharparenleft}rule\ sallI{\isacharparenright}\isanewline
\ \ \ \ \isacommand{fix}\isamarkupfalse%
\ S{\isacharprime}{\isacharprime}\isanewline
\ \ \ \ \isacommand{assume}\isamarkupfalse%
\ {\isachardoublequoteopen}S{\isacharprime}{\isacharprime}\ {\isasymsubseteq}\ S{\isacharprime}{\isachardoublequoteclose}\ \isanewline
\ \ \ \ \isacommand{then}\isamarkupfalse%
\ \isacommand{have}\isamarkupfalse%
\ {\isachardoublequoteopen}S{\isacharprime}{\isacharprime}\ {\isasymsubseteq}\ S{\isachardoublequoteclose}\isanewline
\ \ \ \ \ \ \isacommand{using}\isamarkupfalse%
\ {\isacartoucheopen}S{\isacharprime}\ {\isasymsubseteq}\ S{\isacartoucheclose}\ \isacommand{by}\isamarkupfalse%
\ {\isacharparenleft}rule\ subset{\isacharunderscore}trans{\isacharparenright}\isanewline
\ \ \ \ \isacommand{then}\isamarkupfalse%
\ \isacommand{have}\isamarkupfalse%
\ {\isachardoublequoteopen}{\isasymexists}S\ {\isasymin}\ C{\isachardot}\ S{\isacharprime}{\isacharprime}\ {\isasymsubseteq}\ S{\isachardoublequoteclose}\isanewline
\ \ \ \ \ \ \isacommand{using}\isamarkupfalse%
\ {\isacartoucheopen}S\ {\isasymin}\ C{\isacartoucheclose}\ \isacommand{by}\isamarkupfalse%
\ {\isacharparenleft}rule\ bexI{\isacharparenright}\isanewline
\ \ \ \ \isacommand{thus}\isamarkupfalse%
\ {\isachardoublequoteopen}S{\isacharprime}{\isacharprime}\ {\isasymin}\ {\isacharparenleft}extensionSC\ C{\isacharparenright}{\isachardoublequoteclose}\isanewline
\ \ \ \ \ \ \isacommand{unfolding}\isamarkupfalse%
\ extensionSC\ \isacommand{by}\isamarkupfalse%
\ {\isacharparenleft}rule\ CollectI{\isacharparenright}\isanewline
\ \ \isacommand{qed}\isamarkupfalse%
\isanewline
\isacommand{qed}\isamarkupfalse%
%
\endisatagproof
{\isafoldproof}%
%
\isadelimproof
%
\endisadelimproof
%
\begin{isamarkuptext}%
En conclusión, la prueba detallada del lema completo se muestra a continuación.%
\end{isamarkuptext}\isamarkuptrue%
\isacommand{lemma}\isamarkupfalse%
\ ex{\isadigit{1}}{\isacharcolon}\ \isanewline
\ \ \isakeyword{assumes}\ {\isachardoublequoteopen}pcp\ C{\isachardoublequoteclose}\isanewline
\ \ \isakeyword{shows}\ {\isachardoublequoteopen}{\isasymexists}C{\isacharprime}{\isachardot}\ C\ {\isasymsubseteq}\ C{\isacharprime}\ {\isasymand}\ pcp\ C{\isacharprime}\ {\isasymand}\ subset{\isacharunderscore}closed\ C{\isacharprime}{\isachardoublequoteclose}\isanewline
%
\isadelimproof
%
\endisadelimproof
%
\isatagproof
\isacommand{proof}\isamarkupfalse%
\ {\isacharminus}\isanewline
\ \ \isacommand{have}\isamarkupfalse%
\ C{\isadigit{1}}{\isacharcolon}{\isachardoublequoteopen}C\ {\isasymsubseteq}\ {\isacharparenleft}extensionSC\ C{\isacharparenright}{\isachardoublequoteclose}\isanewline
\ \ \ \ \isacommand{by}\isamarkupfalse%
\ {\isacharparenleft}rule\ ex{\isadigit{1}}{\isacharunderscore}subset{\isacharparenright}\isanewline
\ \ \isacommand{have}\isamarkupfalse%
\ C{\isadigit{2}}{\isacharcolon}{\isachardoublequoteopen}pcp\ {\isacharparenleft}extensionSC\ C{\isacharparenright}{\isachardoublequoteclose}\isanewline
\ \ \ \ \isacommand{using}\isamarkupfalse%
\ assms\ \isacommand{by}\isamarkupfalse%
\ {\isacharparenleft}rule\ ex{\isadigit{1}}{\isacharunderscore}pcp{\isacharparenright}\isanewline
\ \ \isacommand{have}\isamarkupfalse%
\ C{\isadigit{3}}{\isacharcolon}{\isachardoublequoteopen}subset{\isacharunderscore}closed\ {\isacharparenleft}extensionSC\ C{\isacharparenright}{\isachardoublequoteclose}\isanewline
\ \ \ \ \isacommand{using}\isamarkupfalse%
\ assms\ \isacommand{by}\isamarkupfalse%
\ {\isacharparenleft}rule\ ex{\isadigit{1}}{\isacharunderscore}subset{\isacharunderscore}closed{\isacharparenright}\isanewline
\ \ \isacommand{have}\isamarkupfalse%
\ {\isachardoublequoteopen}C\ {\isasymsubseteq}\ {\isacharparenleft}extensionSC\ C{\isacharparenright}\ {\isasymand}\ pcp\ {\isacharparenleft}extensionSC\ C{\isacharparenright}\ {\isasymand}\ subset{\isacharunderscore}closed\ {\isacharparenleft}extensionSC\ C{\isacharparenright}{\isachardoublequoteclose}\ \isanewline
\ \ \ \ \isacommand{using}\isamarkupfalse%
\ C{\isadigit{1}}\ C{\isadigit{2}}\ C{\isadigit{3}}\ \isacommand{by}\isamarkupfalse%
\ {\isacharparenleft}iprover\ intro{\isacharcolon}\ conjI{\isacharparenright}\isanewline
\ \ \isacommand{thus}\isamarkupfalse%
\ {\isacharquery}thesis\isanewline
\ \ \ \ \isacommand{by}\isamarkupfalse%
\ {\isacharparenleft}rule\ exI{\isacharparenright}\isanewline
\isacommand{qed}\isamarkupfalse%
%
\endisatagproof
{\isafoldproof}%
%
\isadelimproof
%
\endisadelimproof
%
\begin{isamarkuptext}%
Continuemos con el segundo resultado de este apartado.

  \begin{lema}
  Toda colección de conjuntos con la propiedad de carácter finito es cerrada bajo subconjuntos.
  \end{lema}

  En Isabelle, se formaliza como sigue.%
\end{isamarkuptext}\isamarkuptrue%
\isacommand{lemma}\isamarkupfalse%
\ \isanewline
\ \ \isakeyword{assumes}\ {\isachardoublequoteopen}finite{\isacharunderscore}character\ C{\isachardoublequoteclose}\isanewline
\ \ \isakeyword{shows}\ {\isachardoublequoteopen}subset{\isacharunderscore}closed\ C{\isachardoublequoteclose}\isanewline
%
\isadelimproof
\ \ %
\endisadelimproof
%
\isatagproof
\isacommand{oops}\isamarkupfalse%
%
\endisatagproof
{\isafoldproof}%
%
\isadelimproof
%
\endisadelimproof
%
\begin{isamarkuptext}%
Procedamos con la demostración del resultado.

  \begin{demostracion}
    Consideremos una colección de conjuntos \isa{C} con la propiedad de carácter finito. Probemos que, 
    en efecto, es cerrada bajo subconjuntos. Por definición de esta última propiedad, basta 
    demostrar que todo subconjunto de cada conjunto de \isa{C} pertenece también a \isa{C}.

    Para ello, tomemos un conjunto \isa{S} cualquiera perteneciente a \isa{C} y un subconjunto cualquiera 
    \isa{S{\isacharprime}} de \isa{S}. Probemos que \isa{S{\isacharprime}} está en \isa{C}. Por hipótesis, como \isa{C} tiene la propiedad de carácter 
    finito, verifica que, para cualquier conjunto \isa{A}, son equivalentes:
    \begin{enumerate}
      \item \isa{A} pertenece a \isa{C}.
      \item Todo subconjunto finito de \isa{A} pertenece a \isa{C}.
    \end{enumerate}

    Para probar que el subconjunto \isa{S{\isacharprime}} pertenece a \isa{C}, vamos a demostrar que todo subconjunto 
    finito de \isa{S{\isacharprime}} pertenece a \isa{C}.

    De este modo, consideremos un subconjunto cualquiera \isa{S{\isacharprime}{\isacharprime}} de \isa{S{\isacharprime}}. Como \isa{S{\isacharprime}} es subconjunto de \isa{S}, 
    por la transitividad de la relación de contención de conjuntos, se tiene que \isa{S{\isacharprime}{\isacharprime}} es subconjunto 
    de \isa{S}. Aplicando la definición de propiedad de carácter finito de \isa{C} para el conjunto \isa{S}, 
    como este pertenece a \isa{C}, verifica que todo subconjunto finito de \isa{S} pertenece a \isa{C}. En
    particular, como \isa{S{\isacharprime}{\isacharprime}} es subconjunto de \isa{S}, verifica que, si \isa{S{\isacharprime}{\isacharprime}} es finito, entonces \isa{S{\isacharprime}{\isacharprime}} 
    pertenece a \isa{C}. Por tanto, hemos probado que cualquier conjunto finito de \isa{S{\isacharprime}} pertenece a la
    colección. Finalmente por la propiedad de carácter finito de \isa{C}, se verifica que \isa{S{\isacharprime}} pertenece 
    a \isa{C}, como queríamos demostrar.
  \end{demostracion}

  Veamos, a continuación, la demostración detallada del resultado en Isabelle.%
\end{isamarkuptext}\isamarkuptrue%
\isacommand{lemma}\isamarkupfalse%
\isanewline
\ \ \isakeyword{assumes}\ {\isachardoublequoteopen}finite{\isacharunderscore}character\ C{\isachardoublequoteclose}\isanewline
\ \ \isakeyword{shows}\ {\isachardoublequoteopen}subset{\isacharunderscore}closed\ C{\isachardoublequoteclose}\isanewline
%
\isadelimproof
\ \ %
\endisadelimproof
%
\isatagproof
\isacommand{unfolding}\isamarkupfalse%
\ subset{\isacharunderscore}closed{\isacharunderscore}def\isanewline
\isacommand{proof}\isamarkupfalse%
\ {\isacharparenleft}intro\ ballI\ sallI{\isacharparenright}\isanewline
\ \ \isacommand{fix}\isamarkupfalse%
\ S{\isacharprime}\ S\isanewline
\ \ \isacommand{assume}\isamarkupfalse%
\ \ {\isacartoucheopen}S\ {\isasymin}\ C{\isacartoucheclose}\ \isakeyword{and}\ {\isacartoucheopen}S{\isacharprime}\ {\isasymsubseteq}\ S{\isacartoucheclose}\isanewline
\ \ \isacommand{have}\isamarkupfalse%
\ H{\isacharcolon}{\isachardoublequoteopen}{\isasymforall}A{\isachardot}\ A\ {\isasymin}\ C\ {\isasymlongleftrightarrow}\ {\isacharparenleft}{\isasymforall}A{\isacharprime}\ {\isasymsubseteq}\ A{\isachardot}\ finite\ A{\isacharprime}\ {\isasymlongrightarrow}\ A{\isacharprime}\ {\isasymin}\ C{\isacharparenright}{\isachardoublequoteclose}\isanewline
\ \ \ \ \isacommand{using}\isamarkupfalse%
\ assms\ \isacommand{unfolding}\isamarkupfalse%
\ finite{\isacharunderscore}character{\isacharunderscore}def\ \isacommand{by}\isamarkupfalse%
\ this\isanewline
\ \ \isacommand{have}\isamarkupfalse%
\ QPQ{\isacharcolon}{\isachardoublequoteopen}{\isasymforall}S{\isacharprime}{\isacharprime}\ {\isasymsubseteq}\ S{\isacharprime}{\isachardot}\ finite\ S{\isacharprime}{\isacharprime}\ {\isasymlongrightarrow}\ S{\isacharprime}{\isacharprime}\ {\isasymin}\ C{\isachardoublequoteclose}\isanewline
\ \ \isacommand{proof}\isamarkupfalse%
\ {\isacharparenleft}rule\ sallI{\isacharparenright}\isanewline
\ \ \ \ \isacommand{fix}\isamarkupfalse%
\ S{\isacharprime}{\isacharprime}\isanewline
\ \ \ \ \isacommand{assume}\isamarkupfalse%
\ {\isachardoublequoteopen}S{\isacharprime}{\isacharprime}\ {\isasymsubseteq}\ S{\isacharprime}{\isachardoublequoteclose}\isanewline
\ \ \ \ \isacommand{then}\isamarkupfalse%
\ \isacommand{have}\isamarkupfalse%
\ {\isachardoublequoteopen}S{\isacharprime}{\isacharprime}\ {\isasymsubseteq}\ S{\isachardoublequoteclose}\isanewline
\ \ \ \ \ \ \isacommand{using}\isamarkupfalse%
\ {\isacartoucheopen}S{\isacharprime}\ {\isasymsubseteq}\ S{\isacartoucheclose}\ \isacommand{by}\isamarkupfalse%
\ {\isacharparenleft}simp\ only{\isacharcolon}\ subset{\isacharunderscore}trans{\isacharparenright}\isanewline
\ \ \ \ \isacommand{have}\isamarkupfalse%
\ {\isadigit{1}}{\isacharcolon}{\isachardoublequoteopen}S\ {\isasymin}\ C\ {\isasymlongleftrightarrow}\ {\isacharparenleft}{\isasymforall}S{\isacharprime}\ {\isasymsubseteq}\ S{\isachardot}\ finite\ S{\isacharprime}\ {\isasymlongrightarrow}\ S{\isacharprime}\ {\isasymin}\ C{\isacharparenright}{\isachardoublequoteclose}\isanewline
\ \ \ \ \ \ \isacommand{using}\isamarkupfalse%
\ H\ \isacommand{by}\isamarkupfalse%
\ {\isacharparenleft}rule\ allE{\isacharparenright}\isanewline
\ \ \ \ \isacommand{have}\isamarkupfalse%
\ {\isachardoublequoteopen}{\isasymforall}S{\isacharprime}\ {\isasymsubseteq}\ S{\isachardot}\ finite\ S{\isacharprime}\ {\isasymlongrightarrow}\ S{\isacharprime}\ {\isasymin}\ C{\isachardoublequoteclose}\isanewline
\ \ \ \ \ \ \isacommand{using}\isamarkupfalse%
\ {\isacartoucheopen}S\ {\isasymin}\ C{\isacartoucheclose}\ {\isadigit{1}}\ \isacommand{by}\isamarkupfalse%
\ {\isacharparenleft}rule\ back{\isacharunderscore}subst{\isacharparenright}\isanewline
\ \ \ \ \isacommand{thus}\isamarkupfalse%
\ {\isachardoublequoteopen}finite\ S{\isacharprime}{\isacharprime}\ {\isasymlongrightarrow}\ S{\isacharprime}{\isacharprime}\ {\isasymin}\ C{\isachardoublequoteclose}\isanewline
\ \ \ \ \ \ \isacommand{using}\isamarkupfalse%
\ {\isacartoucheopen}S{\isacharprime}{\isacharprime}\ {\isasymsubseteq}\ S{\isacartoucheclose}\ \isacommand{by}\isamarkupfalse%
\ {\isacharparenleft}rule\ sspec{\isacharparenright}\isanewline
\ \ \isacommand{qed}\isamarkupfalse%
\isanewline
\ \ \isacommand{have}\isamarkupfalse%
\ {\isachardoublequoteopen}S{\isacharprime}\ {\isasymin}\ C\ {\isasymlongleftrightarrow}\ {\isacharparenleft}{\isasymforall}S{\isacharprime}{\isacharprime}\ {\isasymsubseteq}\ S{\isacharprime}{\isachardot}\ finite\ S{\isacharprime}{\isacharprime}\ {\isasymlongrightarrow}\ S{\isacharprime}{\isacharprime}\ {\isasymin}\ C{\isacharparenright}{\isachardoublequoteclose}\isanewline
\ \ \ \ \isacommand{using}\isamarkupfalse%
\ H\ \isacommand{by}\isamarkupfalse%
\ {\isacharparenleft}rule\ allE{\isacharparenright}\isanewline
\ \ \isacommand{thus}\isamarkupfalse%
\ {\isachardoublequoteopen}S{\isacharprime}\ {\isasymin}\ C{\isachardoublequoteclose}\isanewline
\ \ \ \ \isacommand{using}\isamarkupfalse%
\ QPQ\ \isacommand{by}\isamarkupfalse%
\ {\isacharparenleft}rule\ forw{\isacharunderscore}subst{\isacharparenright}\isanewline
\isacommand{qed}\isamarkupfalse%
%
\endisatagproof
{\isafoldproof}%
%
\isadelimproof
%
\endisadelimproof
%
\begin{isamarkuptext}%
Finalmente, su prueba automática en Isabelle/HOL es la siguiente.%
\end{isamarkuptext}\isamarkuptrue%
\isacommand{lemma}\isamarkupfalse%
\ ex{\isadigit{2}}{\isacharcolon}\ \isanewline
\ \ \isakeyword{assumes}\ fc{\isacharcolon}\ {\isachardoublequoteopen}finite{\isacharunderscore}character\ C{\isachardoublequoteclose}\isanewline
\ \ \isakeyword{shows}\ {\isachardoublequoteopen}subset{\isacharunderscore}closed\ C{\isachardoublequoteclose}\isanewline
%
\isadelimproof
\ \ %
\endisadelimproof
%
\isatagproof
\isacommand{unfolding}\isamarkupfalse%
\ subset{\isacharunderscore}closed{\isacharunderscore}def\isanewline
\isacommand{proof}\isamarkupfalse%
\ {\isacharparenleft}intro\ ballI\ sallI{\isacharparenright}\isanewline
\ \ \isacommand{fix}\isamarkupfalse%
\ S{\isacharprime}\ S\isanewline
\ \ \isacommand{assume}\isamarkupfalse%
\ e{\isacharcolon}\ {\isacartoucheopen}S\ {\isasymin}\ C{\isacartoucheclose}\ \isakeyword{and}\ s{\isacharcolon}\ {\isacartoucheopen}S{\isacharprime}\ {\isasymsubseteq}\ S{\isacartoucheclose}\isanewline
\ \ \isacommand{hence}\isamarkupfalse%
\ {\isacharasterisk}{\isacharcolon}\ {\isachardoublequoteopen}S{\isacharprime}{\isacharprime}\ {\isasymsubseteq}\ S{\isacharprime}\ {\isasymLongrightarrow}\ S{\isacharprime}{\isacharprime}\ {\isasymsubseteq}\ S{\isachardoublequoteclose}\ \isakeyword{for}\ S{\isacharprime}{\isacharprime}\ \isacommand{by}\isamarkupfalse%
\ simp\isanewline
\ \ \isacommand{from}\isamarkupfalse%
\ fc\ \isacommand{have}\isamarkupfalse%
\ {\isachardoublequoteopen}S{\isacharprime}{\isacharprime}\ {\isasymsubseteq}\ S\ {\isasymLongrightarrow}\ finite\ S{\isacharprime}{\isacharprime}\ {\isasymLongrightarrow}\ S{\isacharprime}{\isacharprime}\ {\isasymin}\ C{\isachardoublequoteclose}\ \isakeyword{for}\ S{\isacharprime}{\isacharprime}\ \isanewline
\ \ \ \ \isacommand{unfolding}\isamarkupfalse%
\ finite{\isacharunderscore}character{\isacharunderscore}def\ \isacommand{using}\isamarkupfalse%
\ e\ \isacommand{by}\isamarkupfalse%
\ blast\isanewline
\ \ \isacommand{hence}\isamarkupfalse%
\ {\isachardoublequoteopen}S{\isacharprime}{\isacharprime}\ {\isasymsubseteq}\ S{\isacharprime}\ {\isasymLongrightarrow}\ finite\ S{\isacharprime}{\isacharprime}\ {\isasymLongrightarrow}\ S{\isacharprime}{\isacharprime}\ {\isasymin}\ C{\isachardoublequoteclose}\ \isakeyword{for}\ S{\isacharprime}{\isacharprime}\ \isacommand{using}\isamarkupfalse%
\ {\isacharasterisk}\ \isacommand{by}\isamarkupfalse%
\ simp\isanewline
\ \ \isacommand{with}\isamarkupfalse%
\ fc\ \isacommand{show}\isamarkupfalse%
\ {\isacartoucheopen}S{\isacharprime}\ {\isasymin}\ C{\isacartoucheclose}\ \isacommand{unfolding}\isamarkupfalse%
\ finite{\isacharunderscore}character{\isacharunderscore}def\ \isacommand{by}\isamarkupfalse%
\ blast\isanewline
\isacommand{qed}\isamarkupfalse%
%
\endisatagproof
{\isafoldproof}%
%
\isadelimproof
%
\endisadelimproof
%
\begin{isamarkuptext}%
Introduzcamos el último resultado de la sección.

 \begin{lema}
    Toda colección de conjuntos con la propiedad de consistencia proposicional y cerrada bajo 
    subconjuntos se puede extender a una colección que también verifique la propiedad de 
    consistencia proposicional y sea de carácter finito.
 \end{lema}

 \begin{demostracion}
   Dada una colección de conjuntos \isa{C} en las condiciones del enunciado, vamos a considerar su 
   extensión \isa{C{\isacharprime}} definida como la unión de \isa{C} y la colección formada por aquellos conjuntos
   cuyos subconjuntos finitos pertenecen a \isa{C}. Es decir,\\ \isa{C{\isacharprime}\ {\isacharequal}\ C\ {\isasymunion}\ E} donde 
   \isa{E\ {\isacharequal}\ {\isacharbraceleft}S{\isachardot}\ {\isasymforall}S{\isacharprime}\ {\isasymsubseteq}\ S{\isachardot}\ finite\ S{\isacharprime}\ {\isasymlongrightarrow}\ S{\isacharprime}\ {\isasymin}\ C{\isacharbraceright}}. Es evidente que es extensión pues contiene 
   a la colección \isa{C}. Vamos a probar que, además es de carácter finito y verifica la 
   propiedad de consistencia proposicional.

   En primer lugar, demostremos que \isa{C{\isacharprime}} es de carácter finito. Por definición de dicha propiedad, 
   basta probar que, para cualquier conjunto, son equivalentes:
   \begin{enumerate}
    \item El conjunto pertenece \isa{C{\isacharprime}}.
    \item Todo subconjunto finito suyo pertenece a \isa{C{\isacharprime}}.
   \end{enumerate}

   Comencemos probando \isa{{\isadigit{1}}{\isacharparenright}\ {\isasymLongrightarrow}\ {\isadigit{2}}{\isacharparenright}}. Para ello, sea un conjunto \isa{S} de \isa{C{\isacharprime}} de modo que \isa{S{\isacharprime}} es un
   subconjunto finito suyo. Como \isa{S} pertenece a la extensión, por definición de la misma tenemos
   que o bien \isa{S} está en \isa{C} o bien \isa{S} está en \isa{E}. Vamos a probar que \isa{S{\isacharprime}} está en \isa{C{\isacharprime}} por
   eliminación de la disyunción anterior. En primer lugar, si suponemos que \isa{S} está en \isa{C}, como
   se trata de una colección cerrada bajo subconjuntos, tenemos que todo subconjunto de \isa{S} está en 
   \isa{C}. En particular, \isa{S{\isacharprime}} está en \isa{C} y, por definición de la extensión, se prueba
   que \isa{S{\isacharprime}} está en \isa{C{\isacharprime}}. Por otro lado, suponiendo que \isa{S} esté en \isa{E}, por definición de dicha 
   colección tenemos que todo subconjunto finito de \isa{S} está en \isa{C}. De este modo, por las hipótesis 
   se prueba que \isa{S{\isacharprime}} está en \isa{C} y, por tanto, pertenece a la extensión. 

   Por último, probemos la implicación \isa{{\isadigit{2}}{\isacharparenright}\ {\isasymLongrightarrow}\ {\isadigit{1}}{\isacharparenright}}. Sea un conjunto cualquiera \isa{S} tal que todo
   subconjunto finito suyo pertenece a \isa{C{\isacharprime}}. Vamos a probar que \isa{S} también pertenece a \isa{C{\isacharprime}}. En
   particular, probaremos que pertenece a \isa{E}. Luego basta probar que todo subconjunto finito de 
   \isa{S} pertenece a \isa{C}. Para ello, consideremos \isa{S{\isacharprime}} un subconjunto finito cualquiera de \isa{S}. Por
   hipótesis, tenemos que \isa{S{\isacharprime}} pertenece a \isa{C{\isacharprime}}. Por definición de la extensión, tenemos entonces
   que o bien \isa{S{\isacharprime}} está en \isa{C} (lo que daría por concluida la prueba) o bien \isa{S{\isacharprime}} está en \isa{E}. 
   De este modo, si suponemos que \isa{S{\isacharprime}} está en \isa{E}, por definición de dicha colección tenemos que
   todo subconjunto finito suyo está en \isa{C}. En particular, como todo conjunto es subconjunto de si
   mismo y como hemos supuesto que \isa{S{\isacharprime}} es finito, tenemos que \isa{S{\isacharprime}} está en \isa{C}, lo que prueba la
   implicación.

   Probemos, finalmente, que \isa{C{\isacharprime}} verifica la propiedad de consistencia proposicional. Para ello,
   vamos a considerar un conjunto cualquiera \isa{S} perteneciente a \isa{C{\isacharprime}} y probaremos que se verifican 
   las cuatro condiciones del lema de caracterización de la propiedad de consistencia proposicional
   mediante la notación uniforme. Como el conjunto \isa{S} pertenece a \isa{C{\isacharprime}}, se observa fácilmente por
   definición de la extensión que, o bien \isa{S} está en \isa{C} o bien \isa{S} está en \isa{E}. Veamos que, para 
   ambos casos, se verifican dichas condiciones.

   En primer lugar, supongamos que \isa{S} está en \isa{C}. Como \isa{C} verifica la propiedad de consistencia 
   proposicional por hipótesis, verifica el lema de caracterización en particular para el conjunto 
   \isa{S}. De este modo, se cumple:
   \begin{itemize}
     \item \isa{{\isasymbottom}} no pertenece a \isa{S}.
     \item Dada \isa{p} una fórmula atómica cualquiera, no se tiene 
      simultáneamente que\\ \isa{p\ {\isasymin}\ S} y \isa{{\isasymnot}\ p\ {\isasymin}\ S}.
     \item Para toda fórmula de tipo \isa{{\isasymalpha}} con componentes \isa{{\isasymalpha}\isactrlsub {\isadigit{1}}} y \isa{{\isasymalpha}\isactrlsub {\isadigit{2}}} tal que \isa{{\isasymalpha}}
      pertenece a \isa{S}, se tiene que \isa{{\isacharbraceleft}{\isasymalpha}\isactrlsub {\isadigit{1}}{\isacharcomma}{\isasymalpha}\isactrlsub {\isadigit{2}}{\isacharbraceright}\ {\isasymunion}\ S} pertenece a \isa{C}.
     \item Para toda fórmula de tipo \isa{{\isasymbeta}} con componentes \isa{{\isasymbeta}\isactrlsub {\isadigit{1}}} y \isa{{\isasymbeta}\isactrlsub {\isadigit{2}}} tal que \isa{{\isasymbeta}}
      pertenece a \isa{S}, se tiene que o bien \isa{{\isacharbraceleft}{\isasymbeta}\isactrlsub {\isadigit{1}}{\isacharbraceright}\ {\isasymunion}\ S} pertenece a \isa{C} o 
      bien \isa{{\isacharbraceleft}{\isasymbeta}\isactrlsub {\isadigit{2}}{\isacharbraceright}\ {\isasymunion}\ S} pertenece a \isa{C}.
   \end{itemize} 
  
  Por lo tanto, puesto que \isa{C} está contenida en la extensión \isa{C{\isacharprime}}, se verifican las cuatro
  condiciones del lema para \isa{C{\isacharprime}}.

  Supongamos ahora que \isa{S} está en \isa{E}. Probemos que, en efecto, verifica las condiciones del lema 
  de caracterización.

  En primer lugar vamos a demostrar que \isa{{\isasymbottom}\ {\isasymnotin}\ S} por reducción al absurdo. Si suponemos que \isa{{\isasymbottom}\ {\isasymin}\ S},
  se deduce que el conjunto \isa{{\isacharbraceleft}{\isasymbottom}{\isacharbraceright}} es un subconjunto finito de \isa{S}. Como \isa{S} está en \isa{E}, por
  definición tenemos que \isa{{\isacharbraceleft}{\isasymbottom}{\isacharbraceright}\ {\isasymin}\ C}. De este modo, aplicando el lema de\\ caracterización de la
  propiedad de consistencia proposicional para la colección \isa{C} y el conjunto \isa{{\isacharbraceleft}{\isasymbottom}{\isacharbraceright}}, por la primera
  condición obtenemos que \isa{{\isasymbottom}\ {\isasymnotin}\ {\isacharbraceleft}{\isasymbottom}{\isacharbraceright}}, llegando a una contradicción.

  Demostremos que se verifica la segunda condición del lema para las fórmulas atómicas. De este
  modo, vamos a probar que dada \isa{p} una fórmula atómica cualquiera, no se tiene simultáneamente que
  \isa{p\ {\isasymin}\ S} y \isa{{\isasymnot}\ p\ {\isasymin}\ S}. La prueba se realizará por reducción al absurdo, luego supongamos que para
  cierta fórmula atómica se verifica \isa{p\ {\isasymin}\ S} y\\ \isa{{\isasymnot}\ p\ {\isasymin}\ S}. Análogamente, se observa que el conjunto
  \isa{{\isacharbraceleft}p{\isacharcomma}\ {\isasymnot}\ p{\isacharbraceright}} es un subconjunto finito de \isa{S}, luego pertenece a \isa{C}. Aplicando el lema de
  caracterización de la propiedad de consistencia proposicional para la colección \isa{C} y el conjunto
  \isa{{\isacharbraceleft}p{\isacharcomma}\ {\isasymnot}\ p{\isacharbraceright}}, por la segunda condición obtenemos que no se tiene simultáneamente \isa{q\ {\isasymin}\ {\isacharbraceleft}p{\isacharcomma}\ {\isasymnot}\ p{\isacharbraceright}} y
  \isa{{\isasymnot}\ q\ {\isasymin}\ {\isacharbraceleft}p{\isacharcomma}\ {\isasymnot}\ p{\isacharbraceright}} para ninguna fórmula atómica \isa{q}, llegando así a una contradicción para la
  fórmula atómica \isa{p}.

  Por otro lado, vamos a probar que se verifica la tercera condición del lema de\\ caracterización
  sobre las fórmulas de tipo \isa{{\isasymalpha}}. Consideremos una fórmula cualquiera \isa{F} de tipo \isa{{\isasymalpha}} y componentes 
  \isa{{\isasymalpha}\isactrlsub {\isadigit{1}}} y \isa{{\isasymalpha}\isactrlsub {\isadigit{2}}}, y supongamos que \isa{F\ {\isasymin}\ S}. Demostraremos que\\ \isa{{\isacharbraceleft}{\isasymalpha}\isactrlsub {\isadigit{1}}{\isacharcomma}{\isasymalpha}\isactrlsub {\isadigit{2}}{\isacharbraceright}\ {\isasymunion}\ S\ {\isasymin}\ C{\isacharprime}}. 

  Para ello, probaremos inicialmente que todo subconjunto finito \isa{S{\isacharprime}} de \isa{S} tal que\\ \isa{F\ {\isasymin}\ S{\isacharprime}} 
  verifica \isa{{\isacharbraceleft}{\isasymalpha}\isactrlsub {\isadigit{1}}{\isacharcomma}{\isasymalpha}\isactrlsub {\isadigit{2}}{\isacharbraceright}\ {\isasymunion}\ S{\isacharprime}\ {\isasymin}\ C}. Consideremos \isa{S{\isacharprime}} subconjunto finito cualquiera de \isa{S} en las
  condiciones anteriores. Como \isa{S\ {\isasymin}\ E}, por definición tenemos que \isa{S{\isacharprime}\ {\isasymin}\ C}. Aplicando el lema de 
  caracterización de la propiedad de consistencia proposicional para la colección \isa{C} y el conjunto
  \isa{S{\isacharprime}}, por la tercera condición obtenemos que \isa{{\isacharbraceleft}{\isasymalpha}\isactrlsub {\isadigit{1}}{\isacharcomma}{\isasymalpha}\isactrlsub {\isadigit{2}}{\isacharbraceright}\ {\isasymunion}\ S{\isacharprime}\ {\isasymin}\ C} ya que hemos supuesto que 
  \isa{F\ {\isasymin}\ S{\isacharprime}}.

  Una vez probado el resultado anterior, demostremos que \isa{{\isacharbraceleft}{\isasymalpha}\isactrlsub {\isadigit{1}}{\isacharcomma}{\isasymalpha}\isactrlsub {\isadigit{2}}{\isacharbraceright}\ {\isasymunion}\ S\ {\isasymin}\ E} y, por definición de 
  \isa{C{\isacharprime}}, obtendremos \isa{{\isacharbraceleft}{\isasymalpha}\isactrlsub {\isadigit{1}}{\isacharcomma}{\isasymalpha}\isactrlsub {\isadigit{2}}{\isacharbraceright}\ {\isasymunion}\ S\ {\isasymin}\ C{\isacharprime}}. Además, por definición de \isa{E}, basta probar que todo 
  subconjunto finito de \isa{{\isacharbraceleft}{\isasymalpha}\isactrlsub {\isadigit{1}}{\isacharcomma}{\isasymalpha}\isactrlsub {\isadigit{2}}{\isacharbraceright}\ {\isasymunion}\ S} pertenece a \isa{C}. Consideremos \isa{S{\isacharprime}} un subconjunto finito 
  cualquiera de \isa{{\isacharbraceleft}{\isasymalpha}\isactrlsub {\isadigit{1}}{\isacharcomma}{\isasymalpha}\isactrlsub {\isadigit{2}}{\isacharbraceright}\ {\isasymunion}\ S}. Como \isa{F\ {\isasymin}\ S}, es sencillo comprobar que el conjunto 
  \isa{{\isacharbraceleft}F{\isacharbraceright}\ {\isasymunion}\ {\isacharparenleft}S{\isacharprime}\ {\isacharminus}\ {\isacharbraceleft}{\isasymalpha}\isactrlsub {\isadigit{1}}{\isacharcomma}{\isasymalpha}\isactrlsub {\isadigit{2}}{\isacharbraceright}{\isacharparenright}} es un subconjunto finito de \isa{S}. Por el resultado probado anteriormente, 
  tenemos que el conjunto \isa{{\isacharbraceleft}{\isasymalpha}\isactrlsub {\isadigit{1}}{\isacharcomma}{\isasymalpha}\isactrlsub {\isadigit{2}}{\isacharbraceright}\ {\isasymunion}\ {\isacharparenleft}{\isacharbraceleft}F{\isacharbraceright}\ {\isasymunion}\ {\isacharparenleft}S{\isacharprime}\ {\isacharminus}\ {\isacharbraceleft}{\isasymalpha}\isactrlsub {\isadigit{1}}{\isacharcomma}{\isasymalpha}\isactrlsub {\isadigit{2}}{\isacharbraceright}{\isacharparenright}{\isacharparenright}\ {\isacharequal}} \\ \isa{{\isacharequal}\ {\isacharbraceleft}F{\isacharcomma}{\isasymalpha}\isactrlsub {\isadigit{1}}{\isacharcomma}{\isasymalpha}\isactrlsub {\isadigit{2}}{\isacharbraceright}\ {\isasymunion}\ S{\isacharprime}} pertenece a \isa{C}. 
  Además, como \isa{C} es cerrada bajo subconjuntos, todo conjunto de \isa{C} verifica que cualquier 
  subconjunto suyo pertenece a la colección. Luego, como \isa{S{\isacharprime}} es un subconjunto de 
  \isa{{\isacharbraceleft}F{\isacharcomma}{\isasymalpha}\isactrlsub {\isadigit{1}}{\isacharcomma}{\isasymalpha}\isactrlsub {\isadigit{2}}{\isacharbraceright}\ {\isasymunion}\ S{\isacharprime}}, queda probado que \isa{S{\isacharprime}\ {\isasymin}\ C}.

  Finalmente, veamos que se verifica la última condición del lema de caracterización de la propiedad
  de consistencia proposicional referente a las fórmulas de tipo \isa{{\isasymbeta}}. Consideremos una fórmula 
  cualquiera \isa{F} de tipo \isa{{\isasymbeta}} con componentes \isa{{\isasymbeta}\isactrlsub {\isadigit{1}}} y \isa{{\isasymbeta}\isactrlsub {\isadigit{2}}} tal que \isa{F\ {\isasymin}\ S}. Vamos a probar que se
  tiene que o bien \isa{{\isacharbraceleft}{\isasymbeta}\isactrlsub {\isadigit{1}}{\isacharbraceright}\ {\isasymunion}\ S\ {\isasymin}\ E} o bien \isa{{\isacharbraceleft}{\isasymbeta}\isactrlsub {\isadigit{1}}{\isacharbraceright}\ {\isasymunion}\ S\ {\isasymin}\ E}. En tal caso, por definición de \isa{C{\isacharprime}} se
  cumple que o bien \isa{{\isacharbraceleft}{\isasymbeta}\isactrlsub {\isadigit{1}}{\isacharbraceright}\ {\isasymunion}\ S\ {\isasymin}\ C{\isacharprime}} o bien \isa{{\isacharbraceleft}{\isasymbeta}\isactrlsub {\isadigit{1}}{\isacharbraceright}\ {\isasymunion}\ S\ {\isasymin}\ C{\isacharprime}}. La prueba se realizará por reducción al
  absurdo. Para ello, probemos inicialmente dos resultados previos.

  \begin{description}
    \item[\isa{{\isasymone}{\isacharparenright}}] En las condiciones anteriores, si consideramos \isa{S\isactrlsub {\isadigit{1}}} y \isa{S\isactrlsub {\isadigit{2}}} subconjuntos finitos 
    cualesquiera de \isa{S} tales que \isa{F\ {\isasymin}\ S\isactrlsub {\isadigit{1}}} y \isa{F\ {\isasymin}\ S\isactrlsub {\isadigit{2}}}, entonces existe una fórmula \isa{I\ {\isasymin}\ {\isacharbraceleft}{\isasymbeta}\isactrlsub {\isadigit{1}}{\isacharcomma}{\isasymbeta}\isactrlsub {\isadigit{2}}{\isacharbraceright}} tal 
    que se verifica que tanto \isa{{\isacharbraceleft}I{\isacharbraceright}\ {\isasymunion}\ S\isactrlsub {\isadigit{1}}} como \isa{{\isacharbraceleft}I{\isacharbraceright}\ {\isasymunion}\ S\isactrlsub {\isadigit{2}}} están en \isa{C}.
  \end{description}
  
  Para probar \isa{{\isasymone}{\isacharparenright}}, consideremos el conjunto finito \isa{S\isactrlsub {\isadigit{1}}\ {\isasymunion}\ S\isactrlsub {\isadigit{2}}} que es subconjunto de \isa{S} por las 
  hipótesis. De este modo, como \isa{S\ {\isasymin}\ E}, tenemos que \isa{S\isactrlsub {\isadigit{1}}\ {\isasymunion}\ S\isactrlsub {\isadigit{2}}\ {\isasymin}\ C}. Aplicando el lema de 
  caracterización de la propiedad de consistencia proposicional para la colección \isa{C} y el conjunto 
  \isa{S\isactrlsub {\isadigit{1}}\ {\isasymunion}\ S\isactrlsub {\isadigit{2}}}, por la última condición sobre las fórmulas de tipo \isa{{\isasymbeta}}, como\\ \isa{F\ {\isasymin}\ S\isactrlsub {\isadigit{1}}\ {\isasymunion}\ S\isactrlsub {\isadigit{2}}} por las 
  hipótesis, se tiene que o bien \isa{{\isacharbraceleft}{\isasymbeta}\isactrlsub {\isadigit{1}}{\isacharbraceright}\ {\isasymunion}\ S\isactrlsub {\isadigit{1}}\ {\isasymunion}\ S\isactrlsub {\isadigit{2}}\ {\isasymin}\ C} o bien\\ \isa{{\isacharbraceleft}{\isasymbeta}\isactrlsub {\isadigit{2}}{\isacharbraceright}\ {\isasymunion}\ S\isactrlsub {\isadigit{1}}\ {\isasymunion}\ S\isactrlsub {\isadigit{2}}\ {\isasymin}\ C}. Por tanto, 
  existe una fórmula \isa{I\ {\isasymin}\ {\isacharbraceleft}{\isasymbeta}\isactrlsub {\isadigit{1}}{\isacharcomma}{\isasymbeta}\isactrlsub {\isadigit{2}}{\isacharbraceright}} tal que\\ \isa{{\isacharbraceleft}I{\isacharbraceright}\ {\isasymunion}\ S\isactrlsub {\isadigit{1}}\ {\isasymunion}\ S\isactrlsub {\isadigit{2}}\ {\isasymin}\ C}. Sea \isa{I} la fórmula que cumple lo 
  anterior. Como \isa{C} es cerrada bajo subconjuntos, los subconjuntos \isa{{\isacharbraceleft}I{\isacharbraceright}\ {\isasymunion}\ S\isactrlsub {\isadigit{1}}} y \isa{{\isacharbraceleft}I{\isacharbraceright}\ {\isasymunion}\ S\isactrlsub {\isadigit{2}}} de 
  \isa{{\isacharbraceleft}I{\isacharbraceright}\ {\isasymunion}\ S\isactrlsub {\isadigit{1}}\ {\isasymunion}\ S\isactrlsub {\isadigit{2}}} pertenecen también a \isa{C}. Por tanto, hemos probado que existe una fórmula 
  \isa{I\ {\isasymin}\ {\isacharbraceleft}{\isasymbeta}\isactrlsub {\isadigit{1}}{\isacharcomma}{\isasymbeta}\isactrlsub {\isadigit{2}}{\isacharbraceright}} tal que \isa{{\isacharbraceleft}I{\isacharbraceright}\ {\isasymunion}\ S\isactrlsub {\isadigit{1}}\ {\isasymin}\ C} y \isa{{\isacharbraceleft}I{\isacharbraceright}\ {\isasymunion}\ S\isactrlsub {\isadigit{2}}\ {\isasymin}\ C}.

  Por otra parte, veamos el segundo resultado. 

  \begin{description}
    \item[\isa{{\isasymtwo}{\isacharparenright}}] En las condiciones de \isa{{\isasymone}{\isacharparenright}} para conjuntos cualesquiera \isa{S\isactrlsub {\isadigit{1}}} y \isa{S\isactrlsub {\isadigit{2}}}, si además 
    suponemos que \isa{{\isacharbraceleft}{\isasymbeta}\isactrlsub {\isadigit{1}}{\isacharbraceright}\ {\isasymunion}\ S\isactrlsub {\isadigit{1}}\ {\isasymnotin}\ C} y \isa{{\isacharbraceleft}{\isasymbeta}\isactrlsub {\isadigit{2}}{\isacharbraceright}\ {\isasymunion}\ S\isactrlsub {\isadigit{2}}\ {\isasymnotin}\ C}, llegamos a una contradicción. 
  \end{description}

  Para probarlo, utilizaremos \isa{{\isasymone}{\isacharparenright}} para los conjuntos \isa{{\isacharbraceleft}F{\isacharbraceright}\ {\isasymunion}\ S\isactrlsub {\isadigit{1}}} y \isa{{\isacharbraceleft}F{\isacharbraceright}\ {\isasymunion}\ S\isactrlsub {\isadigit{2}}}. Como es evidente, 
  puesto que \isa{F\ {\isasymin}\ S}, se verifica que ambos conjuntos son subconjuntos de \isa{S}. Además, como \isa{S\isactrlsub {\isadigit{1}}} y 
  \isa{S\isactrlsub {\isadigit{2}}} son finitos, se tiene que \isa{{\isacharbraceleft}F{\isacharbraceright}\ {\isasymunion}\ S\isactrlsub {\isadigit{1}}} y \isa{{\isacharbraceleft}F{\isacharbraceright}\ {\isasymunion}\ S\isactrlsub {\isadigit{2}}} también lo son. Por último, es claro que 
  \isa{F} pertenece a ambos conjuntos. Por lo tanto, por \isa{{\isasymone}{\isacharparenright}} tenemos que existe una fórmula 
  \isa{I\ {\isasymin}\ {\isacharbraceleft}{\isasymbeta}\isactrlsub {\isadigit{1}}{\isacharcomma}{\isasymbeta}\isactrlsub {\isadigit{2}}{\isacharbraceright}} tal que \isa{{\isacharbraceleft}I{\isacharbraceright}\ {\isasymunion}\ {\isacharbraceleft}F{\isacharbraceright}\ {\isasymunion}\ S\isactrlsub {\isadigit{1}}\ {\isasymin}\ C} y \isa{{\isacharbraceleft}I{\isacharbraceright}\ {\isasymunion}\ {\isacharbraceleft}F{\isacharbraceright}\ {\isasymunion}\ S\isactrlsub {\isadigit{2}}\ {\isasymin}\ C}. Por otro lado, podemos probar 
  que \isa{{\isacharbraceleft}{\isasymbeta}\isactrlsub {\isadigit{1}}{\isacharbraceright}\ {\isasymunion}\ {\isacharbraceleft}F{\isacharbraceright}\ {\isasymunion}\ S\isactrlsub {\isadigit{1}}\ {\isasymnotin}\ C}. Esto se debe a que, en caso contrario, como \isa{C} es cerrado bajo 
  subconjuntos, tendríamos que el subconjunto\\ \isa{{\isacharbraceleft}{\isasymbeta}\isactrlsub {\isadigit{1}}{\isacharbraceright}\ {\isasymunion}\ S\isactrlsub {\isadigit{1}}} pertenecería a \isa{C}, lo que contradice las 
  hipótesis. Análogamente, obtenemos que \isa{{\isacharbraceleft}{\isasymbeta}\isactrlsub {\isadigit{2}}{\isacharbraceright}\ {\isasymunion}\ {\isacharbraceleft}F{\isacharbraceright}\ {\isasymunion}\ S\isactrlsub {\isadigit{2}}\ {\isasymnotin}\ C}. De este modo, obtenemos que para 
  toda fórmula \isa{I\ {\isasymin}\ {\isacharbraceleft}{\isasymbeta}\isactrlsub {\isadigit{1}}{\isacharcomma}{\isasymbeta}\isactrlsub {\isadigit{2}}{\isacharbraceright}} se cumple que o bien \isa{{\isacharbraceleft}I{\isacharbraceright}\ {\isasymunion}\ {\isacharbraceleft}F{\isacharbraceright}\ {\isasymunion}\ S\isactrlsub {\isadigit{1}}\ {\isasymnotin}\ C} o bien \isa{{\isacharbraceleft}I{\isacharbraceright}\ {\isasymunion}\ {\isacharbraceleft}F{\isacharbraceright}\ {\isasymunion}\ S\isactrlsub {\isadigit{2}}\ {\isasymnotin}\ C}. 
  Esto es equivalente a que no existe ninguna fórmula \isa{I\ {\isasymin}\ {\isacharbraceleft}{\isasymbeta}\isactrlsub {\isadigit{1}}{\isacharcomma}{\isasymbeta}\isactrlsub {\isadigit{2}}{\isacharbraceright}} tal que \isa{{\isacharbraceleft}I{\isacharbraceright}\ {\isasymunion}\ {\isacharbraceleft}F{\isacharbraceright}\ {\isasymunion}\ S\isactrlsub {\isadigit{1}}\ {\isasymin}\ C} y\\ 
  \isa{{\isacharbraceleft}I{\isacharbraceright}\ {\isasymunion}\ {\isacharbraceleft}F{\isacharbraceright}\ {\isasymunion}\ S\isactrlsub {\isadigit{2}}\ {\isasymin}\ C}, lo que contradice lo obtenido para los conjuntos \isa{{\isacharbraceleft}F{\isacharbraceright}\ {\isasymunion}\ S\isactrlsub {\isadigit{1}}} y\\ \isa{{\isacharbraceleft}F{\isacharbraceright}\ {\isasymunion}\ S\isactrlsub {\isadigit{2}}} 
  por \isa{{\isasymone}{\isacharparenright}}.

  Finalmente, con los resultados anteriores, podemos probar que o bien\\ \isa{{\isacharbraceleft}{\isasymbeta}\isactrlsub {\isadigit{1}}{\isacharbraceright}\ {\isasymunion}\ S\ {\isasymin}\ E} o bien 
  \isa{{\isacharbraceleft}{\isasymbeta}\isactrlsub {\isadigit{2}}{\isacharbraceright}\ {\isasymunion}\ S\ {\isasymin}\ E} por reducción al absurdo. Supongamos que\\ \isa{{\isacharbraceleft}{\isasymbeta}\isactrlsub {\isadigit{1}}{\isacharbraceright}\ {\isasymunion}\ S\ {\isasymnotin}\ E} y \isa{{\isacharbraceleft}{\isasymbeta}\isactrlsub {\isadigit{2}}{\isacharbraceright}\ {\isasymunion}\ S\ {\isasymnotin}\ E}. Por
  definición de \isa{E}, se verifica que existe algún subconjunto finito de \isa{{\isacharbraceleft}{\isasymbeta}\isactrlsub {\isadigit{1}}{\isacharbraceright}\ {\isasymunion}\ S} y existe algún 
  subconjunto finito de \isa{{\isacharbraceleft}{\isasymbeta}\isactrlsub {\isadigit{2}}{\isacharbraceright}\ {\isasymunion}\ S} tales que no pertenecen a \isa{C}. Notemos por \isa{S\isactrlsub {\isadigit{1}}} y \isa{S\isactrlsub {\isadigit{2}}} 
  respectivamente a los subconjuntos anteriores. Vamos a aplicar \isa{{\isasymtwo}{\isacharparenright}} para los conjuntos \isa{S\isactrlsub {\isadigit{1}}\ {\isacharminus}\ {\isacharbraceleft}{\isasymbeta}\isactrlsub {\isadigit{1}}{\isacharbraceright}} 
  y \isa{S\isactrlsub {\isadigit{2}}\ {\isacharminus}\ {\isacharbraceleft}{\isasymbeta}\isactrlsub {\isadigit{2}}{\isacharbraceright}} para llegar a la contradicción.

  Para ello, debemos probar que se verifican las hipótesis del resultado para los conjuntos
  señalados. Es claro que tanto \isa{S\isactrlsub {\isadigit{1}}\ {\isacharminus}\ {\isacharbraceleft}{\isasymbeta}\isactrlsub {\isadigit{1}}{\isacharbraceright}} como \isa{S\isactrlsub {\isadigit{2}}\ {\isacharminus}\ {\isacharbraceleft}{\isasymbeta}\isactrlsub {\isadigit{2}}{\isacharbraceright}} son subconjuntos de \isa{S}, ya que \isa{S\isactrlsub {\isadigit{1}}} y
  \isa{S\isactrlsub {\isadigit{2}}} son subconjuntos de \isa{{\isacharbraceleft}{\isasymbeta}\isactrlsub {\isadigit{1}}{\isacharbraceright}\ {\isasymunion}\ S} y \isa{{\isacharbraceleft}{\isasymbeta}\isactrlsub {\isadigit{2}}{\isacharbraceright}\ {\isasymunion}\ S} respectivamente. Además, como \isa{S\isactrlsub {\isadigit{1}}} y \isa{S\isactrlsub {\isadigit{2}}} son
  finitos, es evidente que \isa{S\isactrlsub {\isadigit{1}}\ {\isacharminus}\ {\isacharbraceleft}{\isasymbeta}\isactrlsub {\isadigit{1}}{\isacharbraceright}} y \isa{S\isactrlsub {\isadigit{2}}\ {\isacharminus}\ {\isacharbraceleft}{\isasymbeta}\isactrlsub {\isadigit{2}}{\isacharbraceright}} también lo son. Queda probar que los conjuntos 
  \isa{{\isacharbraceleft}{\isasymbeta}\isactrlsub {\isadigit{1}}{\isacharbraceright}\ {\isasymunion}\ {\isacharparenleft}S\isactrlsub {\isadigit{1}}\ {\isacharminus}\ {\isacharbraceleft}{\isasymbeta}\isactrlsub {\isadigit{1}}{\isacharbraceright}{\isacharparenright}\ {\isacharequal}\ {\isacharbraceleft}{\isasymbeta}\isactrlsub {\isadigit{1}}{\isacharbraceright}\ {\isasymunion}\ S\isactrlsub {\isadigit{1}}} y \isa{{\isacharbraceleft}{\isasymbeta}\isactrlsub {\isadigit{2}}{\isacharbraceright}\ {\isasymunion}\ {\isacharparenleft}S\isactrlsub {\isadigit{2}}\ {\isacharminus}\ {\isacharbraceleft}{\isasymbeta}\isactrlsub {\isadigit{2}}{\isacharbraceright}{\isacharparenright}\ {\isacharequal}\ {\isacharbraceleft}{\isasymbeta}\isactrlsub {\isadigit{2}}{\isacharbraceright}\ {\isasymunion}\ S\isactrlsub {\isadigit{2}}} no pertenecen a \isa{C}. Como ni 
  \isa{S\isactrlsub {\isadigit{1}}} ni \isa{S\isactrlsub {\isadigit{2}}} están en la colección \isa{C} cerrada bajo subconjuntos, se cumple que ninguno de ellos 
  son subconjuntos de \isa{S}. Sin embargo, se verifica que \isa{S\isactrlsub {\isadigit{1}}} es subconjunto de \isa{{\isacharbraceleft}{\isasymbeta}\isactrlsub {\isadigit{1}}{\isacharbraceright}\ {\isasymunion}\ S} y \isa{S\isactrlsub {\isadigit{2}}} es 
  subconjunto de \isa{{\isacharbraceleft}{\isasymbeta}\isactrlsub {\isadigit{2}}{\isacharbraceright}\ {\isasymunion}\ S}. Por tanto, se cumple que\\ \isa{{\isasymbeta}\isactrlsub {\isadigit{1}}\ {\isasymin}\ S\isactrlsub {\isadigit{1}}} y \isa{{\isasymbeta}\isactrlsub {\isadigit{2}}\ {\isasymin}\ S\isactrlsub {\isadigit{2}}}. Por lo tanto,
  tenemos finalmente que los conjuntos \isa{{\isacharbraceleft}{\isasymbeta}\isactrlsub {\isadigit{1}}{\isacharbraceright}\ {\isasymunion}\ S\isactrlsub {\isadigit{1}}\ {\isacharequal}\ S\isactrlsub {\isadigit{1}}} y\\ \isa{{\isacharbraceleft}{\isasymbeta}\isactrlsub {\isadigit{2}}{\isacharbraceright}\ {\isasymunion}\ S\isactrlsub {\isadigit{2}}\ {\isacharequal}\ S\isactrlsub {\isadigit{2}}} no pertenecen a \isa{C}.
  Finalmente, como se cumplen las condiciones del resultado \isa{{\isadigit{2}}{\isacharparenright}}, llegamos a una contradicción para 
  los conjuntos \isa{S\isactrlsub {\isadigit{1}}\ {\isacharminus}\ {\isacharbraceleft}{\isasymbeta}\isactrlsub {\isadigit{1}}{\isacharbraceright}} y \isa{S\isactrlsub {\isadigit{2}}\ {\isacharminus}\ {\isacharbraceleft}{\isasymbeta}\isactrlsub {\isadigit{2}}{\isacharbraceright}}, probando que o bien \isa{{\isacharbraceleft}{\isasymbeta}\isactrlsub {\isadigit{1}}{\isacharbraceright}\ {\isasymunion}\ S\ {\isasymin}\ E} o bien \isa{{\isacharbraceleft}{\isasymbeta}\isactrlsub {\isadigit{1}}{\isacharbraceright}\ {\isasymunion}\ S\ {\isasymin}\ E}. 
  Por lo tanto, obtenemos por definición de \isa{C{\isacharprime}} que o bien \isa{{\isacharbraceleft}{\isasymbeta}\isactrlsub {\isadigit{1}}{\isacharbraceright}\ {\isasymunion}\ S\ {\isasymin}\ C{\isacharprime}} o bien \isa{{\isacharbraceleft}{\isasymbeta}\isactrlsub {\isadigit{1}}{\isacharbraceright}\ {\isasymunion}\ S\ {\isasymin}\ C{\isacharprime}}.
 \end{demostracion}

  Finalmente, veamos la demostración detallada del lema en Isabelle. Debido a la cantidad de lemas
  auxiliares empleados en la prueba detallada, para facilitar la comprensión mostraremos a
  continuación un grafo que estructura las relaciones de necesidad de los lemas introducidos.
  
 \begin{tikzpicture}
  [
    grow                    = down,
    level 1/.style          = {sibling distance=7cm},
    level 2/.style          = {sibling distance=4cm},
    level 3/.style          = {sibling distance=5.7cm},
    level distance          = 1.5cm,
    edge from parent/.style = {draw},
    every node/.style       = {font=\tiny},
    sloped
  ]
  \node [root] {\isa{ex{\isadigit{3}}}\\ \isa{{\isacharparenleft}Lema\ {\isadigit{3}}{\isachardot}{\isadigit{0}}{\isachardot}{\isadigit{5}}{\isacharparenright}}}
    child { node [env] {\isa{ex{\isadigit{3}}{\isacharunderscore}finite{\isacharunderscore}character}\\ \isa{{\isacharparenleft}C{\isacharprime}\ tiene\ la\ propiedad\ de\ carácter\ finito{\isacharparenright}}}}
    child { node [env] {\isa{ex{\isadigit{3}}{\isacharunderscore}pcp}\\ \isa{{\isacharparenleft}C{\isacharprime}\ tiene\ la\ propiedad\ de\ consistencia\ proposicional{\isacharparenright}}}
      		child { node [env] {\isa{ex{\isadigit{3}}{\isacharunderscore}pcp{\isacharunderscore}SinC}\\ \isa{{\isacharparenleft}Caso\ del\ conjunto\ en\ C{\isacharparenright}}}}
      		child { node [env] {\isa{ex{\isadigit{3}}{\isacharunderscore}pcp{\isacharunderscore}SinE}\\ \isa{{\isacharparenleft}Caso\ del\ conjunto\ en\ E{\isacharparenright}}}
        				child { node [env] {\isa{ex{\isadigit{3}}{\isacharunderscore}pcp{\isacharunderscore}SinE{\isacharunderscore}CON}\\ \isa{{\isacharparenleft}Condición\ fórmulas\ de\ tipo\ {\isasymalpha}{\isacharparenright}}}}
        				child { node [env] {\isa{ex{\isadigit{3}}{\isacharunderscore}pcp{\isacharunderscore}SinE{\isacharunderscore}DIS}\\ \isa{{\isacharparenleft}Condición\ fórmulas\ de\ tipo\ {\isasymbeta}{\isacharparenright}}}
                      child { node [env] {\isa{ex{\isadigit{3}}{\isacharunderscore}pcp{\isacharunderscore}SinE{\isacharunderscore}DIS{\isacharunderscore}auxFalse}\\ \isa{{\isacharparenleft}Resultado\ {\isasymone}{\isacharparenright}}}
                            child { node [env] {\isa{ex{\isadigit{3}}{\isacharunderscore}pcp{\isacharunderscore}SinE{\isacharunderscore}DIS{\isacharunderscore}auxEx}\\ \isa{{\isacharparenleft}Resultado\ {\isasymtwo}{\isacharparenright}}}}}}}};
\end{tikzpicture}

  De este modo, la prueba del \isa{lema\ {\isadigit{1}}{\isachardot}{\isadigit{3}}{\isachardot}{\isadigit{5}}} se estructura fundamentalmente en dos lemas auxiliares. 
  El primero, formalizado como \isa{ex{\isadigit{3}}{\isacharunderscore}finite{\isacharunderscore}character} en Isabelle, prueba que la extensión 
  \isa{C{\isacharprime}\ {\isacharequal}\ C\ {\isasymunion}\ E}, donde \isa{E} es la colección formada por aquellos conjuntos cuyos subconjuntos finitos 
  pertenecen a \isa{C}, tiene la propiedad de carácter finito. El segundo, formalizado como \isa{ex{\isadigit{3}}{\isacharunderscore}pcp}, 
  demuestra que \isa{C{\isacharprime}} verifica la propiedad de consistencia proposicional demostrando que cumple las 
  condiciones suficientes de dicha propiedad por el lema de caracterización \isa{{\isadigit{1}}{\isachardot}{\isadigit{2}}{\isachardot}{\isadigit{5}}}. De este modo, 
  considerando un conjunto \isa{S\ {\isasymin}\ C{\isacharprime}}, \isa{ex{\isadigit{3}}{\isacharunderscore}pcp} precisa, a su vez, de dos lemas auxiliares que 
  prueben las condiciones suficientes de \isa{{\isadigit{1}}{\isachardot}{\isadigit{2}}{\isachardot}{\isadigit{5}}}: uno para el caso en que \isa{S\ {\isasymin}\ C} (\isa{ex{\isadigit{3}}{\isacharunderscore}pcp{\isacharunderscore}SinC}) y 
  otro para el caso en que \isa{S\ {\isasymin}\ E} (\isa{ex{\isadigit{3}}{\isacharunderscore}pcp{\isacharunderscore}SinE}). Por otro lado, para el último caso en que 
  \isa{S\ {\isasymin}\ E}, utilizaremos dos lemas auxiliares. El primero, formalizado como \isa{ex{\isadigit{3}}{\isacharunderscore}pcp{\isacharunderscore}SinE{\isacharunderscore}CON}, 
  prueba que para \isa{C} una colección con la propiedad de consistencia proposicional y cerrada bajo 
  subconjuntos, \isa{S\ {\isasymin}\ E} y sea \isa{F} una fórmula de tipo \isa{{\isasymalpha}} y componentes \isa{{\isasymalpha}\isactrlsub {\isadigit{1}}} y \isa{{\isasymalpha}\isactrlsub {\isadigit{2}}}, se tiene que\\ 
  \isa{{\isacharbraceleft}{\isasymalpha}\isactrlsub {\isadigit{1}}{\isacharcomma}{\isasymalpha}\isactrlsub {\isadigit{2}}{\isacharbraceright}\ {\isasymunion}\ S\ {\isasymin}\ C{\isacharprime}}. El segundo lema, formalizado como \isa{ex{\isadigit{3}}{\isacharunderscore}pcp{\isacharunderscore}SinE{\isacharunderscore}DIS}, prueba que para \isa{C} una 
  colección con la propiedad de consistencia proposicional y cerrada bajo subconjuntos, \isa{S\ {\isasymin}\ E} y 
  sea \isa{F} una fórmula de tipo \isa{{\isasymbeta}} y componentes \isa{{\isasymbeta}\isactrlsub {\isadigit{1}}} y \isa{{\isasymbeta}\isactrlsub {\isadigit{2}}}, se tiene que o bien \isa{{\isacharbraceleft}{\isasymbeta}\isactrlsub {\isadigit{1}}{\isacharbraceright}\ {\isasymunion}\ S\ {\isasymin}\ C{\isacharprime}} o 
  bien \isa{{\isacharbraceleft}{\isasymbeta}\isactrlsub {\isadigit{2}}{\isacharbraceright}\ {\isasymunion}\ S\ {\isasymin}\ C{\isacharprime}}. Por último, este segundo lema auxiliar se probará por reducción al absurdo, 
  precisando para ello de los siguientes resultados auxiliares:
  
  \begin{description}
    \item[\isa{Resultado\ {\isasymone}}] Formalizado como \isa{ex{\isadigit{3}}{\isacharunderscore}pcp{\isacharunderscore}SinE{\isacharunderscore}DIS{\isacharunderscore}auxEx}. Prueba que dada \isa{C} una 
    colección con la propiedad de consistencia proposicional y cerrada bajo subconjuntos,\\ \isa{S\ {\isasymin}\ E} y 
    sea \isa{F} es una fórmula de tipo \isa{{\isasymbeta}} de componentes \isa{{\isasymbeta}\isactrlsub {\isadigit{1}}} y \isa{{\isasymbeta}\isactrlsub {\isadigit{2}}}, si consideramos \isa{S\isactrlsub {\isadigit{1}}} y \isa{S\isactrlsub {\isadigit{2}}} 
    subconjuntos finitos cualesquiera de \isa{S} tales que \isa{F\ {\isasymin}\ S\isactrlsub {\isadigit{1}}} y \isa{F\ {\isasymin}\ S\isactrlsub {\isadigit{2}}}, entonces existe una 
    fórmula \isa{I\ {\isasymin}\ {\isacharbraceleft}{\isasymbeta}\isactrlsub {\isadigit{1}}{\isacharcomma}{\isasymbeta}\isactrlsub {\isadigit{2}}{\isacharbraceright}} tal que se verifica que tanto \isa{{\isacharbraceleft}I{\isacharbraceright}\ {\isasymunion}\ S\isactrlsub {\isadigit{1}}} como \isa{{\isacharbraceleft}I{\isacharbraceright}\ {\isasymunion}\ S\isactrlsub {\isadigit{2}}} están en \isa{C}. 
    \item[\isa{Resultado\ {\isasymtwo}}] Formalizado como \isa{ex{\isadigit{3}}{\isacharunderscore}pcp{\isacharunderscore}SinE{\isacharunderscore}DIS{\isacharunderscore}auxFalse}. Utiliza 
    \isa{ex{\isadigit{3}}{\isacharunderscore}pcp{\isacharunderscore}SinE{\isacharunderscore}DIS{\isacharunderscore}auxEx} como lema auxiliar. Prueba que, en las condiciones del \isa{Resultado\ {\isasymone}}, 
    si además suponemos que \isa{{\isacharbraceleft}{\isasymbeta}\isactrlsub {\isadigit{1}}{\isacharbraceright}\ {\isasymunion}\ S\isactrlsub {\isadigit{1}}\ {\isasymnotin}\ C} y \isa{{\isacharbraceleft}{\isasymbeta}\isactrlsub {\isadigit{2}}{\isacharbraceright}\ {\isasymunion}\ S\isactrlsub {\isadigit{2}}\ {\isasymnotin}\ C}, llegamos a una contradicción.
  \end{description} 

  Por otro lado, para facilitar la notación, dada una colección cualquiera \isa{C}, formalizamos las 
  colecciones \isa{E} y \isa{C{\isacharprime}} como \isa{extF\ C} y \isa{extensionFin\ C} respectivamente como se muestra a 
  continuación.%
\end{isamarkuptext}\isamarkuptrue%
\isacommand{definition}\isamarkupfalse%
\ extF\ {\isacharcolon}{\isacharcolon}\ {\isachardoublequoteopen}{\isacharparenleft}{\isacharparenleft}{\isacharprime}a\ formula{\isacharparenright}\ set{\isacharparenright}\ set\ {\isasymRightarrow}\ {\isacharparenleft}{\isacharparenleft}{\isacharprime}a\ formula{\isacharparenright}\ set{\isacharparenright}\ set{\isachardoublequoteclose}\isanewline
\ \ \isakeyword{where}\ extF{\isacharcolon}\ {\isachardoublequoteopen}extF\ C\ {\isacharequal}\ {\isacharbraceleft}S{\isachardot}\ {\isasymforall}S{\isacharprime}\ {\isasymsubseteq}\ S{\isachardot}\ finite\ S{\isacharprime}\ {\isasymlongrightarrow}\ S{\isacharprime}\ {\isasymin}\ C{\isacharbraceright}{\isachardoublequoteclose}\isanewline
\isanewline
\isacommand{definition}\isamarkupfalse%
\ extensionFin\ {\isacharcolon}{\isacharcolon}\ {\isachardoublequoteopen}{\isacharparenleft}{\isacharparenleft}{\isacharprime}a\ formula{\isacharparenright}\ set{\isacharparenright}\ set\ {\isasymRightarrow}\ {\isacharparenleft}{\isacharparenleft}{\isacharprime}a\ formula{\isacharparenright}\ set{\isacharparenright}\ set{\isachardoublequoteclose}\isanewline
\ \ \isakeyword{where}\ extensionFin{\isacharcolon}\ {\isachardoublequoteopen}extensionFin\ C\ {\isacharequal}\ C\ {\isasymunion}\ {\isacharparenleft}extF\ C{\isacharparenright}{\isachardoublequoteclose}%
\begin{isamarkuptext}%
Una vez hechas las aclaraciones anteriores, procedamos ordenadamente con la demostración 
  detallada de cada lema auxiliar que conforma la prueba del lema \isa{{\isadigit{1}}{\isachardot}{\isadigit{3}}{\isachardot}{\isadigit{5}}}. En primer lugar, probemos 
  detalladamente que la extensión \isa{C{\isacharprime}} tiene la propiedad de carácter finito.%
\end{isamarkuptext}\isamarkuptrue%
\isacommand{lemma}\isamarkupfalse%
\ ex{\isadigit{3}}{\isacharunderscore}finite{\isacharunderscore}character{\isacharcolon}\isanewline
\ \ \isakeyword{assumes}\ {\isachardoublequoteopen}subset{\isacharunderscore}closed\ C{\isachardoublequoteclose}\isanewline
\ \ \ \ \ \ \ \ \isakeyword{shows}\ {\isachardoublequoteopen}finite{\isacharunderscore}character\ {\isacharparenleft}extensionFin\ C{\isacharparenright}{\isachardoublequoteclose}\isanewline
%
\isadelimproof
%
\endisadelimproof
%
\isatagproof
\isacommand{proof}\isamarkupfalse%
\ {\isacharminus}\isanewline
\ \ \isacommand{show}\isamarkupfalse%
\ {\isachardoublequoteopen}finite{\isacharunderscore}character\ {\isacharparenleft}extensionFin\ C{\isacharparenright}{\isachardoublequoteclose}\isanewline
\ \ \ \ \isacommand{unfolding}\isamarkupfalse%
\ finite{\isacharunderscore}character{\isacharunderscore}def\isanewline
\ \ \isacommand{proof}\isamarkupfalse%
\ {\isacharparenleft}rule\ allI{\isacharparenright}\isanewline
\ \ \ \isacommand{fix}\isamarkupfalse%
\ S\isanewline
\ \ \ \isacommand{show}\isamarkupfalse%
\ {\isachardoublequoteopen}S\ {\isasymin}\ {\isacharparenleft}extensionFin\ C{\isacharparenright}\ {\isasymlongleftrightarrow}\ {\isacharparenleft}{\isasymforall}S{\isacharprime}\ {\isasymsubseteq}\ S{\isachardot}\ finite\ S{\isacharprime}\ {\isasymlongrightarrow}\ S{\isacharprime}\ {\isasymin}\ {\isacharparenleft}extensionFin\ C{\isacharparenright}{\isacharparenright}{\isachardoublequoteclose}\isanewline
\ \ \ \isacommand{proof}\isamarkupfalse%
\ {\isacharparenleft}rule\ iffI{\isacharparenright}\isanewline
\ \ \ \ \ \isacommand{assume}\isamarkupfalse%
\ {\isachardoublequoteopen}S\ {\isasymin}\ {\isacharparenleft}extensionFin\ C{\isacharparenright}{\isachardoublequoteclose}\isanewline
\ \ \ \ \ \isacommand{show}\isamarkupfalse%
\ {\isachardoublequoteopen}{\isasymforall}S{\isacharprime}\ {\isasymsubseteq}\ S{\isachardot}\ finite\ S{\isacharprime}\ {\isasymlongrightarrow}\ S{\isacharprime}\ {\isasymin}\ {\isacharparenleft}extensionFin\ C{\isacharparenright}{\isachardoublequoteclose}\isanewline
\ \ \ \ \ \isacommand{proof}\isamarkupfalse%
\ {\isacharparenleft}intro\ sallI\ impI{\isacharparenright}\isanewline
\ \ \ \ \ \ \ \isacommand{fix}\isamarkupfalse%
\ S{\isacharprime}\isanewline
\ \ \ \ \ \ \ \isacommand{assume}\isamarkupfalse%
\ {\isachardoublequoteopen}S{\isacharprime}\ {\isasymsubseteq}\ S{\isachardoublequoteclose}\isanewline
\ \ \ \ \ \ \ \isacommand{assume}\isamarkupfalse%
\ {\isachardoublequoteopen}finite\ S{\isacharprime}{\isachardoublequoteclose}\isanewline
\ \ \ \ \ \ \ \isacommand{have}\isamarkupfalse%
\ {\isachardoublequoteopen}S\ {\isasymin}\ C\ {\isasymor}\ S\ {\isasymin}\ {\isacharparenleft}extF\ C{\isacharparenright}{\isachardoublequoteclose}\isanewline
\ \ \ \ \ \ \ \ \ \isacommand{using}\isamarkupfalse%
\ {\isacartoucheopen}S\ {\isasymin}\ {\isacharparenleft}extensionFin\ C{\isacharparenright}{\isacartoucheclose}\ \isacommand{by}\isamarkupfalse%
\ {\isacharparenleft}simp\ only{\isacharcolon}\ extensionFin\ Un{\isacharunderscore}iff{\isacharparenright}\isanewline
\ \ \ \ \ \ \ \isacommand{thus}\isamarkupfalse%
\ {\isachardoublequoteopen}S{\isacharprime}\ {\isasymin}\ {\isacharparenleft}extensionFin\ C{\isacharparenright}{\isachardoublequoteclose}\isanewline
\ \ \ \ \ \ \ \isacommand{proof}\isamarkupfalse%
\ {\isacharparenleft}rule\ disjE{\isacharparenright}\isanewline
\ \ \ \ \ \ \ \ \ \isacommand{assume}\isamarkupfalse%
\ {\isachardoublequoteopen}S\ {\isasymin}\ C{\isachardoublequoteclose}\isanewline
\ \ \ \ \ \ \ \ \ \isacommand{have}\isamarkupfalse%
\ {\isachardoublequoteopen}{\isasymforall}S\ {\isasymin}\ C{\isachardot}\ {\isasymforall}S{\isacharprime}\ {\isasymsubseteq}\ S{\isachardot}\ S{\isacharprime}\ {\isasymin}\ C{\isachardoublequoteclose}\isanewline
\ \ \ \ \ \ \ \ \ \ \ \isacommand{using}\isamarkupfalse%
\ assms\ \isacommand{by}\isamarkupfalse%
\ {\isacharparenleft}simp\ only{\isacharcolon}\ subset{\isacharunderscore}closed{\isacharunderscore}def{\isacharparenright}\isanewline
\ \ \ \ \ \ \ \ \ \isacommand{then}\isamarkupfalse%
\ \isacommand{have}\isamarkupfalse%
\ {\isachardoublequoteopen}{\isasymforall}S{\isacharprime}\ {\isasymsubseteq}\ S{\isachardot}\ S{\isacharprime}\ {\isasymin}\ C{\isachardoublequoteclose}\isanewline
\ \ \ \ \ \ \ \ \ \ \ \isacommand{using}\isamarkupfalse%
\ {\isacartoucheopen}S\ {\isasymin}\ C{\isacartoucheclose}\ \isacommand{by}\isamarkupfalse%
\ {\isacharparenleft}rule\ bspec{\isacharparenright}\isanewline
\ \ \ \ \ \ \ \ \ \isacommand{then}\isamarkupfalse%
\ \isacommand{have}\isamarkupfalse%
\ {\isachardoublequoteopen}S{\isacharprime}\ {\isasymin}\ C{\isachardoublequoteclose}\isanewline
\ \ \ \ \ \ \ \ \ \ \ \isacommand{using}\isamarkupfalse%
\ {\isacartoucheopen}S{\isacharprime}\ {\isasymsubseteq}\ S{\isacartoucheclose}\ \isacommand{by}\isamarkupfalse%
\ {\isacharparenleft}rule\ sspec{\isacharparenright}\isanewline
\ \ \ \ \ \ \ \ \ \isacommand{thus}\isamarkupfalse%
\ {\isachardoublequoteopen}S{\isacharprime}\ {\isasymin}\ {\isacharparenleft}extensionFin\ C{\isacharparenright}{\isachardoublequoteclose}\ \isanewline
\ \ \ \ \ \ \ \ \ \ \ \isacommand{by}\isamarkupfalse%
\ {\isacharparenleft}simp\ only{\isacharcolon}\ extensionFin\ UnI{\isadigit{1}}{\isacharparenright}\isanewline
\ \ \ \ \ \ \ \isacommand{next}\isamarkupfalse%
\isanewline
\ \ \ \ \ \ \ \ \ \isacommand{assume}\isamarkupfalse%
\ {\isachardoublequoteopen}S\ {\isasymin}\ {\isacharparenleft}extF\ C{\isacharparenright}{\isachardoublequoteclose}\isanewline
\ \ \ \ \ \ \ \ \ \isacommand{then}\isamarkupfalse%
\ \isacommand{have}\isamarkupfalse%
\ {\isachardoublequoteopen}{\isasymforall}S{\isacharprime}\ {\isasymsubseteq}\ S{\isachardot}\ finite\ S{\isacharprime}\ {\isasymlongrightarrow}\ S{\isacharprime}\ {\isasymin}\ C{\isachardoublequoteclose}\isanewline
\ \ \ \ \ \ \ \ \ \ \ \isacommand{unfolding}\isamarkupfalse%
\ extF\ \isacommand{by}\isamarkupfalse%
\ {\isacharparenleft}rule\ CollectD{\isacharparenright}\isanewline
\ \ \ \ \ \ \ \ \ \isacommand{then}\isamarkupfalse%
\ \isacommand{have}\isamarkupfalse%
\ {\isachardoublequoteopen}finite\ S{\isacharprime}\ {\isasymlongrightarrow}\ S{\isacharprime}\ {\isasymin}\ C{\isachardoublequoteclose}\isanewline
\ \ \ \ \ \ \ \ \ \ \ \isacommand{using}\isamarkupfalse%
\ {\isacartoucheopen}S{\isacharprime}\ {\isasymsubseteq}\ S{\isacartoucheclose}\ \isacommand{by}\isamarkupfalse%
\ {\isacharparenleft}rule\ sspec{\isacharparenright}\isanewline
\ \ \ \ \ \ \ \ \ \isacommand{then}\isamarkupfalse%
\ \isacommand{have}\isamarkupfalse%
\ {\isachardoublequoteopen}S{\isacharprime}\ {\isasymin}\ C{\isachardoublequoteclose}\isanewline
\ \ \ \ \ \ \ \ \ \ \ \isacommand{using}\isamarkupfalse%
\ {\isacartoucheopen}finite\ S{\isacharprime}{\isacartoucheclose}\ \isacommand{by}\isamarkupfalse%
\ {\isacharparenleft}rule\ mp{\isacharparenright}\isanewline
\ \ \ \ \ \ \ \ \ \isacommand{thus}\isamarkupfalse%
\ {\isachardoublequoteopen}S{\isacharprime}\ {\isasymin}\ {\isacharparenleft}extensionFin\ C{\isacharparenright}{\isachardoublequoteclose}\isanewline
\ \ \ \ \ \ \ \ \ \ \ \isacommand{by}\isamarkupfalse%
\ {\isacharparenleft}simp\ only{\isacharcolon}\ extensionFin\ UnI{\isadigit{1}}{\isacharparenright}\isanewline
\ \ \ \ \ \ \ \isacommand{qed}\isamarkupfalse%
\isanewline
\ \ \ \ \ \isacommand{qed}\isamarkupfalse%
\isanewline
\ \ \ \isacommand{next}\isamarkupfalse%
\isanewline
\ \ \ \ \ \isacommand{assume}\isamarkupfalse%
\ {\isachardoublequoteopen}{\isasymforall}S{\isacharprime}\ {\isasymsubseteq}\ S{\isachardot}\ finite\ S{\isacharprime}\ {\isasymlongrightarrow}\ S{\isacharprime}\ {\isasymin}\ {\isacharparenleft}extensionFin\ C{\isacharparenright}{\isachardoublequoteclose}\isanewline
\ \ \ \ \ \isacommand{then}\isamarkupfalse%
\ \isacommand{have}\isamarkupfalse%
\ F{\isacharcolon}{\isachardoublequoteopen}{\isasymforall}S{\isacharprime}\ {\isasymsubseteq}\ S{\isachardot}\ finite\ S{\isacharprime}\ {\isasymlongrightarrow}\ S{\isacharprime}\ {\isasymin}\ C\ {\isasymor}\ S{\isacharprime}\ {\isasymin}\ {\isacharparenleft}extF\ C{\isacharparenright}{\isachardoublequoteclose}\isanewline
\ \ \ \ \ \ \ \isacommand{by}\isamarkupfalse%
\ {\isacharparenleft}simp\ only{\isacharcolon}\ extensionFin\ Un{\isacharunderscore}iff{\isacharparenright}\isanewline
\ \ \ \ \ \isacommand{have}\isamarkupfalse%
\ {\isachardoublequoteopen}{\isasymforall}S{\isacharprime}\ {\isasymsubseteq}\ S{\isachardot}\ finite\ S{\isacharprime}\ {\isasymlongrightarrow}\ S{\isacharprime}\ {\isasymin}\ C{\isachardoublequoteclose}\isanewline
\ \ \ \ \ \isacommand{proof}\isamarkupfalse%
\ {\isacharparenleft}rule\ sallI{\isacharparenright}\isanewline
\ \ \ \ \ \ \ \isacommand{fix}\isamarkupfalse%
\ S{\isacharprime}\isanewline
\ \ \ \ \ \ \ \isacommand{assume}\isamarkupfalse%
\ {\isachardoublequoteopen}S{\isacharprime}\ {\isasymsubseteq}\ S{\isachardoublequoteclose}\isanewline
\ \ \ \ \ \ \ \isacommand{show}\isamarkupfalse%
\ {\isachardoublequoteopen}finite\ S{\isacharprime}\ {\isasymlongrightarrow}\ S{\isacharprime}\ {\isasymin}\ C{\isachardoublequoteclose}\isanewline
\ \ \ \ \ \ \ \isacommand{proof}\isamarkupfalse%
\ {\isacharparenleft}rule\ impI{\isacharparenright}\isanewline
\ \ \ \ \ \ \ \ \ \isacommand{assume}\isamarkupfalse%
\ {\isachardoublequoteopen}finite\ S{\isacharprime}{\isachardoublequoteclose}\isanewline
\ \ \ \ \ \ \ \ \ \isacommand{have}\isamarkupfalse%
\ {\isachardoublequoteopen}finite\ S{\isacharprime}\ {\isasymlongrightarrow}\ S{\isacharprime}\ {\isasymin}\ C\ {\isasymor}\ S{\isacharprime}\ {\isasymin}\ {\isacharparenleft}extF\ C{\isacharparenright}{\isachardoublequoteclose}\ \isanewline
\ \ \ \ \ \ \ \ \ \ \ \isacommand{using}\isamarkupfalse%
\ F\ {\isacartoucheopen}S{\isacharprime}\ {\isasymsubseteq}\ S{\isacartoucheclose}\ \isacommand{by}\isamarkupfalse%
\ {\isacharparenleft}rule\ sspec{\isacharparenright}\isanewline
\ \ \ \ \ \ \ \ \ \isacommand{then}\isamarkupfalse%
\ \isacommand{have}\isamarkupfalse%
\ {\isachardoublequoteopen}S{\isacharprime}\ {\isasymin}\ C\ {\isasymor}\ S{\isacharprime}\ {\isasymin}\ {\isacharparenleft}extF\ C{\isacharparenright}{\isachardoublequoteclose}\isanewline
\ \ \ \ \ \ \ \ \ \ \ \isacommand{using}\isamarkupfalse%
\ {\isacartoucheopen}finite\ S{\isacharprime}{\isacartoucheclose}\ \isacommand{by}\isamarkupfalse%
\ {\isacharparenleft}rule\ mp{\isacharparenright}\isanewline
\ \ \ \ \ \ \ \ \ \isacommand{thus}\isamarkupfalse%
\ {\isachardoublequoteopen}S{\isacharprime}\ {\isasymin}\ C{\isachardoublequoteclose}\isanewline
\ \ \ \ \ \ \ \ \ \isacommand{proof}\isamarkupfalse%
\ {\isacharparenleft}rule\ disjE{\isacharparenright}\isanewline
\ \ \ \ \ \ \ \ \ \ \ \isacommand{assume}\isamarkupfalse%
\ {\isachardoublequoteopen}S{\isacharprime}\ {\isasymin}\ C{\isachardoublequoteclose}\isanewline
\ \ \ \ \ \ \ \ \ \ \ \isacommand{thus}\isamarkupfalse%
\ {\isachardoublequoteopen}S{\isacharprime}\ {\isasymin}\ C{\isachardoublequoteclose}\isanewline
\ \ \ \ \ \ \ \ \ \ \ \ \ \isacommand{by}\isamarkupfalse%
\ this\isanewline
\ \ \ \ \ \ \ \ \ \isacommand{next}\isamarkupfalse%
\isanewline
\ \ \ \ \ \ \ \ \ \ \ \isacommand{assume}\isamarkupfalse%
\ {\isachardoublequoteopen}S{\isacharprime}\ {\isasymin}\ {\isacharparenleft}extF\ C{\isacharparenright}{\isachardoublequoteclose}\isanewline
\ \ \ \ \ \ \ \ \ \ \ \isacommand{then}\isamarkupfalse%
\ \isacommand{have}\isamarkupfalse%
\ S{\isacharprime}{\isacharcolon}{\isachardoublequoteopen}{\isasymforall}S{\isacharprime}{\isacharprime}\ {\isasymsubseteq}\ S{\isacharprime}{\isachardot}\ finite\ S{\isacharprime}{\isacharprime}\ {\isasymlongrightarrow}\ S{\isacharprime}{\isacharprime}\ {\isasymin}\ C{\isachardoublequoteclose}\isanewline
\ \ \ \ \ \ \ \ \ \ \ \ \ \isacommand{unfolding}\isamarkupfalse%
\ extF\ \isacommand{by}\isamarkupfalse%
\ {\isacharparenleft}rule\ CollectD{\isacharparenright}\isanewline
\ \ \ \ \ \ \ \ \ \ \ \isacommand{have}\isamarkupfalse%
\ {\isachardoublequoteopen}S{\isacharprime}\ {\isasymsubseteq}\ S{\isacharprime}{\isachardoublequoteclose}\isanewline
\ \ \ \ \ \ \ \ \ \ \ \ \ \isacommand{by}\isamarkupfalse%
\ {\isacharparenleft}simp\ only{\isacharcolon}\ subset{\isacharunderscore}refl{\isacharparenright}\isanewline
\ \ \ \ \ \ \ \ \ \ \ \isacommand{have}\isamarkupfalse%
\ {\isachardoublequoteopen}finite\ S{\isacharprime}\ {\isasymlongrightarrow}\ S{\isacharprime}\ {\isasymin}\ C{\isachardoublequoteclose}\isanewline
\ \ \ \ \ \ \ \ \ \ \ \ \ \isacommand{using}\isamarkupfalse%
\ S{\isacharprime}\ {\isacartoucheopen}S{\isacharprime}\ {\isasymsubseteq}\ S{\isacharprime}{\isacartoucheclose}\ \isacommand{by}\isamarkupfalse%
\ {\isacharparenleft}rule\ sspec{\isacharparenright}\isanewline
\ \ \ \ \ \ \ \ \ \ \ \isacommand{thus}\isamarkupfalse%
\ {\isachardoublequoteopen}S{\isacharprime}\ {\isasymin}\ C{\isachardoublequoteclose}\isanewline
\ \ \ \ \ \ \ \ \ \ \ \ \ \isacommand{using}\isamarkupfalse%
\ {\isacartoucheopen}finite\ S{\isacharprime}{\isacartoucheclose}\ \isacommand{by}\isamarkupfalse%
\ {\isacharparenleft}rule\ mp{\isacharparenright}\isanewline
\ \ \ \ \ \ \ \ \ \isacommand{qed}\isamarkupfalse%
\isanewline
\ \ \ \ \ \ \ \isacommand{qed}\isamarkupfalse%
\isanewline
\ \ \ \ \ \isacommand{qed}\isamarkupfalse%
\isanewline
\ \ \ \ \ \isacommand{then}\isamarkupfalse%
\ \isacommand{have}\isamarkupfalse%
\ {\isachardoublequoteopen}S\ {\isasymin}\ {\isacharbraceleft}S{\isachardot}\ {\isasymforall}S{\isacharprime}\ {\isasymsubseteq}\ S{\isachardot}\ finite\ S{\isacharprime}\ {\isasymlongrightarrow}\ S{\isacharprime}\ {\isasymin}\ C{\isacharbraceright}{\isachardoublequoteclose}\isanewline
\ \ \ \ \ \ \ \isacommand{by}\isamarkupfalse%
\ {\isacharparenleft}rule\ CollectI{\isacharparenright}\isanewline
\ \ \ \ \ \isacommand{thus}\isamarkupfalse%
\ {\isachardoublequoteopen}S\ {\isasymin}\ {\isacharparenleft}extensionFin\ C{\isacharparenright}{\isachardoublequoteclose}\isanewline
\ \ \ \ \ \ \ \isacommand{by}\isamarkupfalse%
\ {\isacharparenleft}simp\ only{\isacharcolon}\ extF\ extensionFin\ UnI{\isadigit{2}}{\isacharparenright}\isanewline
\ \ \ \isacommand{qed}\isamarkupfalse%
\isanewline
\ \isacommand{qed}\isamarkupfalse%
\isanewline
\isacommand{qed}\isamarkupfalse%
%
\endisatagproof
{\isafoldproof}%
%
\isadelimproof
%
\endisadelimproof
%
\begin{isamarkuptext}%
Por otro lado, para probar que  \isa{C{\isacharprime}\ {\isacharequal}\ C\ {\isasymunion}\ E}  verifica la propiedad de consistencia 
  proposicional, consideraremos un conjunto \isa{S\ {\isasymin}\ C{\isacharprime}} y utilizaremos fundamentalmente dos lemas 
  auxiliares: uno para el caso en que \isa{S\ {\isasymin}\ C} y otro para el caso en que \isa{S\ {\isasymin}\ E}. 

  En primer lugar, vamos a probar el primer lema auxiliar para el caso en que\\ \isa{S\ {\isasymin}\ C}, formalizado
  como \isa{ex{\isadigit{3}}{\isacharunderscore}pcp{\isacharunderscore}SinC}. Dicho lema prueba que, si \isa{C} es una colección con la propiedad de 
  consistencia proposicional y cerrada bajo subconjuntos, y sea \isa{S\ {\isasymin}\ C}, se verifican
  las condiciones del lema de caracterización de la propiedad de consistencia proposicional para
  la extensión \isa{C{\isacharprime}}:
  \begin{itemize}
    \item \isa{{\isasymbottom}\ {\isasymnotin}\ S}.
    \item Dada \isa{p} una fórmula atómica cualquiera, no se tiene 
    simultáneamente que\\ \isa{p\ {\isasymin}\ S} y \isa{{\isasymnot}\ p\ {\isasymin}\ S}.
    \item Para toda fórmula de tipo \isa{{\isasymalpha}} con componentes \isa{{\isasymalpha}\isactrlsub {\isadigit{1}}} y \isa{{\isasymalpha}\isactrlsub {\isadigit{2}}} tal que \isa{{\isasymalpha}}
    pertenece a \isa{S}, se tiene que \isa{{\isacharbraceleft}{\isasymalpha}\isactrlsub {\isadigit{1}}{\isacharcomma}{\isasymalpha}\isactrlsub {\isadigit{2}}{\isacharbraceright}\ {\isasymunion}\ S} pertenece a \isa{C{\isacharprime}}.
    \item Para toda fórmula de tipo \isa{{\isasymbeta}} con componentes \isa{{\isasymbeta}\isactrlsub {\isadigit{1}}} y \isa{{\isasymbeta}\isactrlsub {\isadigit{2}}} tal que \isa{{\isasymbeta}}
    pertenece a \isa{S}, se tiene que o bien \isa{{\isacharbraceleft}{\isasymbeta}\isactrlsub {\isadigit{1}}{\isacharbraceright}\ {\isasymunion}\ S} pertenece a \isa{C{\isacharprime}} o 
    bien \isa{{\isacharbraceleft}{\isasymbeta}\isactrlsub {\isadigit{2}}{\isacharbraceright}\ {\isasymunion}\ S} pertenece a \isa{C{\isacharprime}}.
  \end{itemize}%
\end{isamarkuptext}\isamarkuptrue%
\isacommand{lemma}\isamarkupfalse%
\ ex{\isadigit{3}}{\isacharunderscore}pcp{\isacharunderscore}SinC{\isacharcolon}\isanewline
\ \ \isakeyword{assumes}\ {\isachardoublequoteopen}pcp\ C{\isachardoublequoteclose}\isanewline
\ \ \ \ \ \ \ \ \ \ {\isachardoublequoteopen}subset{\isacharunderscore}closed\ C{\isachardoublequoteclose}\isanewline
\ \ \ \ \ \ \ \ \ \ {\isachardoublequoteopen}S\ {\isasymin}\ C{\isachardoublequoteclose}\ \isanewline
\ \ \isakeyword{shows}\ {\isachardoublequoteopen}{\isasymbottom}\ {\isasymnotin}\ S\ {\isasymand}\isanewline
\ \ \ \ \ \ \ \ \ {\isacharparenleft}{\isasymforall}k{\isachardot}\ Atom\ k\ {\isasymin}\ S\ {\isasymlongrightarrow}\ \isactrlbold {\isasymnot}\ {\isacharparenleft}Atom\ k{\isacharparenright}\ {\isasymin}\ S\ {\isasymlongrightarrow}\ False{\isacharparenright}\ {\isasymand}\isanewline
\ \ \ \ \ \ \ \ \ {\isacharparenleft}{\isasymforall}F\ G\ H{\isachardot}\ Con\ F\ G\ H\ {\isasymlongrightarrow}\ F\ {\isasymin}\ S\ {\isasymlongrightarrow}\ {\isacharbraceleft}G{\isacharcomma}\ H{\isacharbraceright}\ {\isasymunion}\ S\ {\isasymin}\ {\isacharparenleft}extensionFin\ C{\isacharparenright}{\isacharparenright}\ {\isasymand}\isanewline
\ \ \ \ \ \ \ \ \ {\isacharparenleft}{\isasymforall}F\ G\ H{\isachardot}\ Dis\ F\ G\ H\ {\isasymlongrightarrow}\ F\ {\isasymin}\ S\ {\isasymlongrightarrow}\ {\isacharbraceleft}G{\isacharbraceright}\ {\isasymunion}\ S\ {\isasymin}{\isacharparenleft}extensionFin\ C{\isacharparenright}\ {\isasymor}\ {\isacharbraceleft}H{\isacharbraceright}\ {\isasymunion}\ S\ {\isasymin}\ {\isacharparenleft}extensionFin\ C{\isacharparenright}{\isacharparenright}{\isachardoublequoteclose}\isanewline
%
\isadelimproof
%
\endisadelimproof
%
\isatagproof
\isacommand{proof}\isamarkupfalse%
\ {\isacharminus}\isanewline
\ \ \isacommand{have}\isamarkupfalse%
\ PCP{\isacharcolon}{\isachardoublequoteopen}{\isasymforall}S\ {\isasymin}\ C{\isachardot}\isanewline
\ \ \ \ {\isasymbottom}\ {\isasymnotin}\ S\isanewline
\ \ \ \ {\isasymand}\ {\isacharparenleft}{\isasymforall}k{\isachardot}\ Atom\ k\ {\isasymin}\ S\ {\isasymlongrightarrow}\ \isactrlbold {\isasymnot}\ {\isacharparenleft}Atom\ k{\isacharparenright}\ {\isasymin}\ S\ {\isasymlongrightarrow}\ False{\isacharparenright}\isanewline
\ \ \ \ {\isasymand}\ {\isacharparenleft}{\isasymforall}F\ G\ H{\isachardot}\ Con\ F\ G\ H\ {\isasymlongrightarrow}\ F\ {\isasymin}\ S\ {\isasymlongrightarrow}\ {\isacharbraceleft}G{\isacharcomma}H{\isacharbraceright}\ {\isasymunion}\ S\ {\isasymin}\ C{\isacharparenright}\isanewline
\ \ \ \ {\isasymand}\ {\isacharparenleft}{\isasymforall}F\ G\ H{\isachardot}\ Dis\ F\ G\ H\ {\isasymlongrightarrow}\ F\ {\isasymin}\ S\ {\isasymlongrightarrow}\ {\isacharbraceleft}G{\isacharbraceright}\ {\isasymunion}\ S\ {\isasymin}\ C\ {\isasymor}\ {\isacharbraceleft}H{\isacharbraceright}\ {\isasymunion}\ S\ {\isasymin}\ C{\isacharparenright}{\isachardoublequoteclose}\isanewline
\ \ \ \ \isacommand{using}\isamarkupfalse%
\ assms{\isacharparenleft}{\isadigit{1}}{\isacharparenright}\ \isacommand{by}\isamarkupfalse%
\ {\isacharparenleft}rule\ pcp{\isacharunderscore}alt{\isadigit{1}}{\isacharparenright}\isanewline
\ \ \isacommand{have}\isamarkupfalse%
\ H{\isacharcolon}{\isachardoublequoteopen}{\isasymbottom}\ {\isasymnotin}\ S\isanewline
\ \ \ \ {\isasymand}\ {\isacharparenleft}{\isasymforall}k{\isachardot}\ Atom\ k\ {\isasymin}\ S\ {\isasymlongrightarrow}\ \isactrlbold {\isasymnot}\ {\isacharparenleft}Atom\ k{\isacharparenright}\ {\isasymin}\ S\ {\isasymlongrightarrow}\ False{\isacharparenright}\isanewline
\ \ \ \ {\isasymand}\ {\isacharparenleft}{\isasymforall}F\ G\ H{\isachardot}\ Con\ F\ G\ H\ {\isasymlongrightarrow}\ F\ {\isasymin}\ S\ {\isasymlongrightarrow}\ {\isacharbraceleft}G{\isacharcomma}H{\isacharbraceright}\ {\isasymunion}\ S\ {\isasymin}\ C{\isacharparenright}\isanewline
\ \ \ \ {\isasymand}\ {\isacharparenleft}{\isasymforall}F\ G\ H{\isachardot}\ Dis\ F\ G\ H\ {\isasymlongrightarrow}\ F\ {\isasymin}\ S\ {\isasymlongrightarrow}\ {\isacharbraceleft}G{\isacharbraceright}\ {\isasymunion}\ S\ {\isasymin}\ C\ {\isasymor}\ {\isacharbraceleft}H{\isacharbraceright}\ {\isasymunion}\ S\ {\isasymin}\ C{\isacharparenright}{\isachardoublequoteclose}\isanewline
\ \ \ \ \ \isacommand{using}\isamarkupfalse%
\ PCP\ {\isacartoucheopen}S\ {\isasymin}\ C{\isacartoucheclose}\ \isacommand{by}\isamarkupfalse%
\ {\isacharparenleft}rule\ bspec{\isacharparenright}\isanewline
\ \ \isacommand{then}\isamarkupfalse%
\ \isacommand{have}\isamarkupfalse%
\ A{\isadigit{1}}{\isacharcolon}{\isachardoublequoteopen}{\isasymbottom}\ {\isasymnotin}\ S{\isachardoublequoteclose}\isanewline
\ \ \ \ \isacommand{by}\isamarkupfalse%
\ {\isacharparenleft}rule\ conjunct{\isadigit{1}}{\isacharparenright}\isanewline
\ \ \isacommand{have}\isamarkupfalse%
\ A{\isadigit{2}}{\isacharcolon}{\isachardoublequoteopen}{\isasymforall}k{\isachardot}\ Atom\ k\ {\isasymin}\ S\ {\isasymlongrightarrow}\ \isactrlbold {\isasymnot}\ {\isacharparenleft}Atom\ k{\isacharparenright}\ {\isasymin}\ S\ {\isasymlongrightarrow}\ False{\isachardoublequoteclose}\isanewline
\ \ \ \ \isacommand{using}\isamarkupfalse%
\ H\ \isacommand{by}\isamarkupfalse%
\ {\isacharparenleft}iprover\ elim{\isacharcolon}\ conjunct{\isadigit{2}}\ conjunct{\isadigit{1}}{\isacharparenright}\isanewline
\ \ \isacommand{have}\isamarkupfalse%
\ S{\isadigit{3}}{\isacharcolon}{\isachardoublequoteopen}{\isasymforall}F\ G\ H{\isachardot}\ Con\ F\ G\ H\ {\isasymlongrightarrow}\ F\ {\isasymin}\ S\ {\isasymlongrightarrow}\ {\isacharbraceleft}G{\isacharcomma}H{\isacharbraceright}\ {\isasymunion}\ S\ {\isasymin}\ C{\isachardoublequoteclose}\isanewline
\ \ \ \ \isacommand{using}\isamarkupfalse%
\ H\ \isacommand{by}\isamarkupfalse%
\ {\isacharparenleft}iprover\ elim{\isacharcolon}\ conjunct{\isadigit{2}}\ conjunct{\isadigit{1}}{\isacharparenright}\isanewline
\ \ \isacommand{have}\isamarkupfalse%
\ A{\isadigit{3}}{\isacharcolon}{\isachardoublequoteopen}{\isasymforall}F\ G\ H{\isachardot}\ Con\ F\ G\ H\ {\isasymlongrightarrow}\ F\ {\isasymin}\ S\ {\isasymlongrightarrow}\ {\isacharbraceleft}G{\isacharcomma}\ H{\isacharbraceright}\ {\isasymunion}\ S\ {\isasymin}\ {\isacharparenleft}extensionFin\ C{\isacharparenright}{\isachardoublequoteclose}\isanewline
\ \ \isacommand{proof}\isamarkupfalse%
\ {\isacharparenleft}rule\ allI{\isacharparenright}{\isacharplus}\isanewline
\ \ \ \ \isacommand{fix}\isamarkupfalse%
\ F\ G\ H\isanewline
\ \ \ \ \isacommand{show}\isamarkupfalse%
\ {\isachardoublequoteopen}Con\ F\ G\ H\ {\isasymlongrightarrow}\ F\ {\isasymin}\ S\ {\isasymlongrightarrow}\ {\isacharbraceleft}G{\isacharcomma}\ H{\isacharbraceright}\ {\isasymunion}\ S\ {\isasymin}\ {\isacharparenleft}extensionFin\ C{\isacharparenright}{\isachardoublequoteclose}\isanewline
\ \ \ \ \isacommand{proof}\isamarkupfalse%
\ {\isacharparenleft}rule\ impI{\isacharparenright}{\isacharplus}\isanewline
\ \ \ \ \ \ \isacommand{assume}\isamarkupfalse%
\ {\isachardoublequoteopen}Con\ F\ G\ H{\isachardoublequoteclose}\isanewline
\ \ \ \ \ \ \isacommand{assume}\isamarkupfalse%
\ {\isachardoublequoteopen}F\ {\isasymin}\ S{\isachardoublequoteclose}\ \isanewline
\ \ \ \ \ \ \isacommand{have}\isamarkupfalse%
\ {\isachardoublequoteopen}Con\ F\ G\ H\ {\isasymlongrightarrow}\ F\ {\isasymin}\ S\ {\isasymlongrightarrow}\ {\isacharbraceleft}G{\isacharcomma}H{\isacharbraceright}\ {\isasymunion}\ S\ {\isasymin}\ C{\isachardoublequoteclose}\isanewline
\ \ \ \ \ \ \ \ \isacommand{using}\isamarkupfalse%
\ S{\isadigit{3}}\ \isacommand{by}\isamarkupfalse%
\ {\isacharparenleft}iprover\ elim{\isacharcolon}\ allE{\isacharparenright}\isanewline
\ \ \ \ \ \ \isacommand{then}\isamarkupfalse%
\ \isacommand{have}\isamarkupfalse%
\ {\isachardoublequoteopen}F\ {\isasymin}\ S\ {\isasymlongrightarrow}\ {\isacharbraceleft}G{\isacharcomma}H{\isacharbraceright}\ {\isasymunion}\ S\ {\isasymin}\ C{\isachardoublequoteclose}\isanewline
\ \ \ \ \ \ \ \ \isacommand{using}\isamarkupfalse%
\ {\isacartoucheopen}Con\ F\ G\ H{\isacartoucheclose}\ \isacommand{by}\isamarkupfalse%
\ {\isacharparenleft}rule\ mp{\isacharparenright}\isanewline
\ \ \ \ \ \ \isacommand{then}\isamarkupfalse%
\ \isacommand{have}\isamarkupfalse%
\ {\isachardoublequoteopen}{\isacharbraceleft}G{\isacharcomma}H{\isacharbraceright}\ {\isasymunion}\ S\ {\isasymin}\ C{\isachardoublequoteclose}\isanewline
\ \ \ \ \ \ \ \ \isacommand{using}\isamarkupfalse%
\ {\isacartoucheopen}F\ {\isasymin}\ S{\isacartoucheclose}\ \isacommand{by}\isamarkupfalse%
\ {\isacharparenleft}rule\ mp{\isacharparenright}\isanewline
\ \ \ \ \ \ \isacommand{thus}\isamarkupfalse%
\ {\isachardoublequoteopen}{\isacharbraceleft}G{\isacharcomma}H{\isacharbraceright}\ {\isasymunion}\ S\ {\isasymin}\ {\isacharparenleft}extensionFin\ C{\isacharparenright}{\isachardoublequoteclose}\isanewline
\ \ \ \ \ \ \ \ \isacommand{unfolding}\isamarkupfalse%
\ extensionFin\ \isacommand{by}\isamarkupfalse%
\ {\isacharparenleft}rule\ UnI{\isadigit{1}}{\isacharparenright}\isanewline
\ \ \ \ \isacommand{qed}\isamarkupfalse%
\isanewline
\ \ \isacommand{qed}\isamarkupfalse%
\isanewline
\ \ \isacommand{have}\isamarkupfalse%
\ S{\isadigit{4}}{\isacharcolon}{\isachardoublequoteopen}{\isasymforall}F\ G\ H{\isachardot}\ Dis\ F\ G\ H\ {\isasymlongrightarrow}\ F\ {\isasymin}\ S\ {\isasymlongrightarrow}\ {\isacharbraceleft}G{\isacharbraceright}\ {\isasymunion}\ S\ {\isasymin}\ C\ {\isasymor}\ {\isacharbraceleft}H{\isacharbraceright}\ {\isasymunion}\ S\ {\isasymin}\ C{\isachardoublequoteclose}\isanewline
\ \ \ \ \isacommand{using}\isamarkupfalse%
\ H\ \isacommand{by}\isamarkupfalse%
\ {\isacharparenleft}iprover\ elim{\isacharcolon}\ conjunct{\isadigit{2}}{\isacharparenright}\isanewline
\ \ \isacommand{have}\isamarkupfalse%
\ A{\isadigit{4}}{\isacharcolon}{\isachardoublequoteopen}{\isasymforall}F\ G\ H{\isachardot}\ Dis\ F\ G\ H\ {\isasymlongrightarrow}\ F\ {\isasymin}\ S\ {\isasymlongrightarrow}\ {\isacharbraceleft}G{\isacharbraceright}\ {\isasymunion}\ S\ {\isasymin}\ {\isacharparenleft}extensionFin\ C{\isacharparenright}\ {\isasymor}\ {\isacharbraceleft}H{\isacharbraceright}\ {\isasymunion}\ S\ {\isasymin}\ {\isacharparenleft}extensionFin\ C{\isacharparenright}{\isachardoublequoteclose}\isanewline
\ \ \isacommand{proof}\isamarkupfalse%
\ {\isacharparenleft}rule\ allI{\isacharparenright}{\isacharplus}\isanewline
\ \ \ \ \isacommand{fix}\isamarkupfalse%
\ F\ G\ H\isanewline
\ \ \ \ \isacommand{show}\isamarkupfalse%
\ {\isachardoublequoteopen}Dis\ F\ G\ H\ {\isasymlongrightarrow}\ F\ {\isasymin}\ S\ {\isasymlongrightarrow}\ {\isacharbraceleft}G{\isacharbraceright}\ {\isasymunion}\ S\ {\isasymin}\ {\isacharparenleft}extensionFin\ C{\isacharparenright}\ {\isasymor}\ {\isacharbraceleft}H{\isacharbraceright}\ {\isasymunion}\ S\ {\isasymin}\ {\isacharparenleft}extensionFin\ C{\isacharparenright}{\isachardoublequoteclose}\isanewline
\ \ \ \ \isacommand{proof}\isamarkupfalse%
\ {\isacharparenleft}rule\ impI{\isacharparenright}{\isacharplus}\isanewline
\ \ \ \ \ \ \isacommand{assume}\isamarkupfalse%
\ {\isachardoublequoteopen}Dis\ F\ G\ H{\isachardoublequoteclose}\isanewline
\ \ \ \ \ \ \isacommand{assume}\isamarkupfalse%
\ {\isachardoublequoteopen}F\ {\isasymin}\ S{\isachardoublequoteclose}\ \isanewline
\ \ \ \ \ \ \isacommand{have}\isamarkupfalse%
\ {\isachardoublequoteopen}Dis\ F\ G\ H\ {\isasymlongrightarrow}\ F\ {\isasymin}\ S\ {\isasymlongrightarrow}\ {\isacharbraceleft}G{\isacharbraceright}\ {\isasymunion}\ S\ {\isasymin}\ C\ {\isasymor}\ {\isacharbraceleft}H{\isacharbraceright}\ {\isasymunion}\ S\ {\isasymin}\ C{\isachardoublequoteclose}\isanewline
\ \ \ \ \ \ \ \ \isacommand{using}\isamarkupfalse%
\ S{\isadigit{4}}\ \isacommand{by}\isamarkupfalse%
\ {\isacharparenleft}iprover\ elim{\isacharcolon}\ allE{\isacharparenright}\isanewline
\ \ \ \ \ \ \isacommand{then}\isamarkupfalse%
\ \isacommand{have}\isamarkupfalse%
\ {\isachardoublequoteopen}F\ {\isasymin}\ S\ {\isasymlongrightarrow}\ {\isacharbraceleft}G{\isacharbraceright}\ {\isasymunion}\ S\ {\isasymin}\ C\ {\isasymor}\ {\isacharbraceleft}H{\isacharbraceright}\ {\isasymunion}\ S\ {\isasymin}\ C{\isachardoublequoteclose}\isanewline
\ \ \ \ \ \ \ \ \isacommand{using}\isamarkupfalse%
\ {\isacartoucheopen}Dis\ F\ G\ H{\isacartoucheclose}\ \isacommand{by}\isamarkupfalse%
\ {\isacharparenleft}rule\ mp{\isacharparenright}\isanewline
\ \ \ \ \ \ \isacommand{then}\isamarkupfalse%
\ \isacommand{have}\isamarkupfalse%
\ {\isachardoublequoteopen}{\isacharbraceleft}G{\isacharbraceright}\ {\isasymunion}\ S\ {\isasymin}\ C\ {\isasymor}\ {\isacharbraceleft}H{\isacharbraceright}\ {\isasymunion}\ S\ {\isasymin}\ C{\isachardoublequoteclose}\isanewline
\ \ \ \ \ \ \ \ \isacommand{using}\isamarkupfalse%
\ {\isacartoucheopen}F\ {\isasymin}\ S{\isacartoucheclose}\ \isacommand{by}\isamarkupfalse%
\ {\isacharparenleft}rule\ mp{\isacharparenright}\isanewline
\ \ \ \ \ \ \isacommand{thus}\isamarkupfalse%
\ {\isachardoublequoteopen}{\isacharbraceleft}G{\isacharbraceright}\ {\isasymunion}\ S\ {\isasymin}\ {\isacharparenleft}extensionFin\ C{\isacharparenright}\ {\isasymor}\ {\isacharbraceleft}H{\isacharbraceright}\ {\isasymunion}\ S\ {\isasymin}\ {\isacharparenleft}extensionFin\ C{\isacharparenright}{\isachardoublequoteclose}\isanewline
\ \ \ \ \ \ \isacommand{proof}\isamarkupfalse%
\ {\isacharparenleft}rule\ disjE{\isacharparenright}\isanewline
\ \ \ \ \ \ \ \ \isacommand{assume}\isamarkupfalse%
\ {\isachardoublequoteopen}{\isacharbraceleft}G{\isacharbraceright}\ {\isasymunion}\ S\ {\isasymin}\ C{\isachardoublequoteclose}\isanewline
\ \ \ \ \ \ \ \ \isacommand{then}\isamarkupfalse%
\ \isacommand{have}\isamarkupfalse%
\ {\isachardoublequoteopen}{\isacharbraceleft}G{\isacharbraceright}\ {\isasymunion}\ S\ {\isasymin}\ {\isacharparenleft}extensionFin\ C{\isacharparenright}{\isachardoublequoteclose}\isanewline
\ \ \ \ \ \ \ \ \ \ \isacommand{unfolding}\isamarkupfalse%
\ extensionFin\ \isacommand{by}\isamarkupfalse%
\ {\isacharparenleft}rule\ UnI{\isadigit{1}}{\isacharparenright}\isanewline
\ \ \ \ \ \ \ \ \isacommand{thus}\isamarkupfalse%
\ {\isachardoublequoteopen}{\isacharbraceleft}G{\isacharbraceright}\ {\isasymunion}\ S\ {\isasymin}\ {\isacharparenleft}extensionFin\ C{\isacharparenright}\ {\isasymor}\ {\isacharbraceleft}H{\isacharbraceright}\ {\isasymunion}\ S\ {\isasymin}\ {\isacharparenleft}extensionFin\ C{\isacharparenright}{\isachardoublequoteclose}\isanewline
\ \ \ \ \ \ \ \ \ \ \isacommand{by}\isamarkupfalse%
\ {\isacharparenleft}rule\ disjI{\isadigit{1}}{\isacharparenright}\isanewline
\ \ \ \ \ \ \isacommand{next}\isamarkupfalse%
\isanewline
\ \ \ \ \ \ \ \ \isacommand{assume}\isamarkupfalse%
\ {\isachardoublequoteopen}{\isacharbraceleft}H{\isacharbraceright}\ {\isasymunion}\ S\ {\isasymin}\ C{\isachardoublequoteclose}\isanewline
\ \ \ \ \ \ \ \ \isacommand{then}\isamarkupfalse%
\ \isacommand{have}\isamarkupfalse%
\ {\isachardoublequoteopen}{\isacharbraceleft}H{\isacharbraceright}\ {\isasymunion}\ S\ {\isasymin}\ {\isacharparenleft}extensionFin\ C{\isacharparenright}{\isachardoublequoteclose}\isanewline
\ \ \ \ \ \ \ \ \ \ \isacommand{unfolding}\isamarkupfalse%
\ extensionFin\ \isacommand{by}\isamarkupfalse%
\ {\isacharparenleft}rule\ UnI{\isadigit{1}}{\isacharparenright}\isanewline
\ \ \ \ \ \ \ \ \isacommand{thus}\isamarkupfalse%
\ {\isachardoublequoteopen}{\isacharbraceleft}G{\isacharbraceright}\ {\isasymunion}\ S\ {\isasymin}\ {\isacharparenleft}extensionFin\ C{\isacharparenright}\ {\isasymor}\ {\isacharbraceleft}H{\isacharbraceright}\ {\isasymunion}\ S\ {\isasymin}\ {\isacharparenleft}extensionFin\ C{\isacharparenright}{\isachardoublequoteclose}\isanewline
\ \ \ \ \ \ \ \ \ \ \isacommand{by}\isamarkupfalse%
\ {\isacharparenleft}rule\ disjI{\isadigit{2}}{\isacharparenright}\isanewline
\ \ \ \ \ \ \isacommand{qed}\isamarkupfalse%
\isanewline
\ \ \ \ \isacommand{qed}\isamarkupfalse%
\isanewline
\ \ \isacommand{qed}\isamarkupfalse%
\isanewline
\ \ \isacommand{show}\isamarkupfalse%
\ {\isachardoublequoteopen}{\isasymbottom}\ {\isasymnotin}\ S\ {\isasymand}\isanewline
\ \ \ \ \ \ \ \ {\isacharparenleft}{\isasymforall}k{\isachardot}\ Atom\ k\ {\isasymin}\ S\ {\isasymlongrightarrow}\ \isactrlbold {\isasymnot}\ {\isacharparenleft}Atom\ k{\isacharparenright}\ {\isasymin}\ S\ {\isasymlongrightarrow}\ False{\isacharparenright}\ {\isasymand}\isanewline
\ \ \ \ \ \ \ \ {\isacharparenleft}{\isasymforall}F\ G\ H{\isachardot}\ Con\ F\ G\ H\ {\isasymlongrightarrow}\ F\ {\isasymin}\ S\ {\isasymlongrightarrow}\ {\isacharbraceleft}G{\isacharcomma}\ H{\isacharbraceright}\ {\isasymunion}\ S\ {\isasymin}\ {\isacharparenleft}extensionFin\ C{\isacharparenright}{\isacharparenright}\ {\isasymand}\isanewline
\ \ \ \ \ \ \ \ {\isacharparenleft}{\isasymforall}F\ G\ H{\isachardot}\ Dis\ F\ G\ H\ {\isasymlongrightarrow}\ F\ {\isasymin}\ S\ {\isasymlongrightarrow}\ {\isacharbraceleft}G{\isacharbraceright}\ {\isasymunion}\ S\ {\isasymin}\ {\isacharparenleft}extensionFin\ C{\isacharparenright}\ {\isasymor}\ {\isacharbraceleft}H{\isacharbraceright}\ {\isasymunion}\ S\ {\isasymin}\ {\isacharparenleft}extensionFin\ C{\isacharparenright}{\isacharparenright}{\isachardoublequoteclose}\isanewline
\ \ \ \ \isacommand{using}\isamarkupfalse%
\ A{\isadigit{1}}\ A{\isadigit{2}}\ A{\isadigit{3}}\ A{\isadigit{4}}\ \isacommand{by}\isamarkupfalse%
\ {\isacharparenleft}iprover\ intro{\isacharcolon}\ conjI{\isacharparenright}\isanewline
\isacommand{qed}\isamarkupfalse%
%
\endisatagproof
{\isafoldproof}%
%
\isadelimproof
%
\endisadelimproof
%
\begin{isamarkuptext}%
Como hemos señalado con anterioridad, para probar el caso en que \isa{S\ {\isasymin}\ E}, donde \isa{E} es la 
  colección formada por aquellos conjuntos cuyos subconjuntos finitos pertenecen a \isa{C}, precisaremos 
  de distintos lemas auxiliares. El primero de ellos demuestra detalladamente que si \isa{C} es una
  colección con la propiedad de consistencia proposicional y cerrada bajo subconjuntos, \isa{S\ {\isasymin}\ E}
  y sea \isa{F} una fórmula de tipo \isa{{\isasymalpha}} con componentes \isa{{\isasymalpha}\isactrlsub {\isadigit{1}}} y \isa{{\isasymalpha}\isactrlsub {\isadigit{2}}}, se verifica que \isa{{\isacharbraceleft}{\isasymalpha}\isactrlsub {\isadigit{1}}{\isacharcomma}{\isasymalpha}\isactrlsub {\isadigit{2}}{\isacharbraceright}\ {\isasymunion}\ S} 
  pertenece a la extensión \isa{C{\isacharprime}\ {\isacharequal}\ C\ {\isasymunion}\ E}.%
\end{isamarkuptext}\isamarkuptrue%
\isacommand{lemma}\isamarkupfalse%
\ ex{\isadigit{3}}{\isacharunderscore}pcp{\isacharunderscore}SinE{\isacharunderscore}CON{\isacharcolon}\isanewline
\ \ \isakeyword{assumes}\ {\isachardoublequoteopen}pcp\ C{\isachardoublequoteclose}\isanewline
\ \ \ \ \ \ \ \ \ \ {\isachardoublequoteopen}subset{\isacharunderscore}closed\ C{\isachardoublequoteclose}\isanewline
\ \ \ \ \ \ \ \ \ \ {\isachardoublequoteopen}S\ {\isasymin}\ {\isacharparenleft}extF\ C{\isacharparenright}{\isachardoublequoteclose}\isanewline
\ \ \ \ \ \ \ \ \ \ {\isachardoublequoteopen}Con\ F\ G\ H{\isachardoublequoteclose}\isanewline
\ \ \ \ \ \ \ \ \ \ {\isachardoublequoteopen}F\ {\isasymin}\ S{\isachardoublequoteclose}\isanewline
\ \ \isakeyword{shows}\ {\isachardoublequoteopen}{\isacharbraceleft}G{\isacharcomma}H{\isacharbraceright}\ {\isasymunion}\ S\ {\isasymin}\ {\isacharparenleft}extensionFin\ C{\isacharparenright}{\isachardoublequoteclose}\ \isanewline
%
\isadelimproof
%
\endisadelimproof
%
\isatagproof
\isacommand{proof}\isamarkupfalse%
\ {\isacharminus}\isanewline
\ \ \isacommand{have}\isamarkupfalse%
\ PCP{\isacharcolon}{\isachardoublequoteopen}{\isasymforall}S\ {\isasymin}\ C{\isachardot}\isanewline
\ \ {\isasymbottom}\ {\isasymnotin}\ S\isanewline
{\isasymand}\ {\isacharparenleft}{\isasymforall}k{\isachardot}\ Atom\ k\ {\isasymin}\ S\ {\isasymlongrightarrow}\ \isactrlbold {\isasymnot}\ {\isacharparenleft}Atom\ k{\isacharparenright}\ {\isasymin}\ S\ {\isasymlongrightarrow}\ False{\isacharparenright}\isanewline
{\isasymand}\ {\isacharparenleft}{\isasymforall}F\ G\ H{\isachardot}\ Con\ F\ G\ H\ {\isasymlongrightarrow}\ F\ {\isasymin}\ S\ {\isasymlongrightarrow}\ {\isacharbraceleft}G{\isacharcomma}H{\isacharbraceright}\ {\isasymunion}\ S\ {\isasymin}\ C{\isacharparenright}\isanewline
{\isasymand}\ {\isacharparenleft}{\isasymforall}F\ G\ H{\isachardot}\ Dis\ F\ G\ H\ {\isasymlongrightarrow}\ F\ {\isasymin}\ S\ {\isasymlongrightarrow}\ {\isacharbraceleft}G{\isacharbraceright}\ {\isasymunion}\ S\ {\isasymin}\ C\ {\isasymor}\ {\isacharbraceleft}H{\isacharbraceright}\ {\isasymunion}\ S\ {\isasymin}\ C{\isacharparenright}{\isachardoublequoteclose}\isanewline
\ \ \ \ \isacommand{using}\isamarkupfalse%
\ assms{\isacharparenleft}{\isadigit{1}}{\isacharparenright}\ \isacommand{by}\isamarkupfalse%
\ {\isacharparenleft}rule\ pcp{\isacharunderscore}alt{\isadigit{1}}{\isacharparenright}\isanewline
\ \ \isacommand{have}\isamarkupfalse%
\ {\isadigit{1}}{\isacharcolon}{\isachardoublequoteopen}{\isasymforall}S{\isacharprime}\ {\isasymsubseteq}\ S{\isachardot}\ finite\ S{\isacharprime}\ {\isasymlongrightarrow}\ F\ {\isasymin}\ S{\isacharprime}\ {\isasymlongrightarrow}\ {\isacharbraceleft}G{\isacharcomma}H{\isacharbraceright}\ {\isasymunion}\ S{\isacharprime}\ {\isasymin}\ C{\isachardoublequoteclose}\isanewline
\ \ \isacommand{proof}\isamarkupfalse%
\ {\isacharparenleft}rule\ sallI{\isacharparenright}\isanewline
\ \ \ \ \isacommand{fix}\isamarkupfalse%
\ S{\isacharprime}\isanewline
\ \ \ \ \isacommand{assume}\isamarkupfalse%
\ {\isachardoublequoteopen}S{\isacharprime}\ {\isasymsubseteq}\ S{\isachardoublequoteclose}\isanewline
\ \ \ \ \isacommand{show}\isamarkupfalse%
\ {\isachardoublequoteopen}finite\ S{\isacharprime}\ {\isasymlongrightarrow}\ F\ {\isasymin}\ S{\isacharprime}\ {\isasymlongrightarrow}\ {\isacharbraceleft}G{\isacharcomma}H{\isacharbraceright}\ {\isasymunion}\ S{\isacharprime}\ {\isasymin}\ C{\isachardoublequoteclose}\isanewline
\ \ \ \ \isacommand{proof}\isamarkupfalse%
\ {\isacharparenleft}rule\ impI{\isacharparenright}{\isacharplus}\isanewline
\ \ \ \ \ \ \isacommand{assume}\isamarkupfalse%
\ {\isachardoublequoteopen}finite\ S{\isacharprime}{\isachardoublequoteclose}\isanewline
\ \ \ \ \ \ \isacommand{assume}\isamarkupfalse%
\ {\isachardoublequoteopen}F\ {\isasymin}\ S{\isacharprime}{\isachardoublequoteclose}\isanewline
\ \ \ \ \ \ \isacommand{have}\isamarkupfalse%
\ E{\isacharcolon}{\isachardoublequoteopen}{\isasymforall}S{\isacharprime}\ {\isasymsubseteq}\ S{\isachardot}\ finite\ S{\isacharprime}\ {\isasymlongrightarrow}\ S{\isacharprime}\ {\isasymin}\ C{\isachardoublequoteclose}\isanewline
\ \ \ \ \ \ \ \ \isacommand{using}\isamarkupfalse%
\ assms{\isacharparenleft}{\isadigit{3}}{\isacharparenright}\ \isacommand{unfolding}\isamarkupfalse%
\ extF\ \isacommand{by}\isamarkupfalse%
\ {\isacharparenleft}rule\ CollectD{\isacharparenright}\isanewline
\ \ \ \ \ \ \isacommand{then}\isamarkupfalse%
\ \isacommand{have}\isamarkupfalse%
\ {\isachardoublequoteopen}finite\ S{\isacharprime}\ {\isasymlongrightarrow}\ S{\isacharprime}\ {\isasymin}\ C{\isachardoublequoteclose}\isanewline
\ \ \ \ \ \ \ \ \isacommand{using}\isamarkupfalse%
\ {\isacartoucheopen}S{\isacharprime}\ {\isasymsubseteq}\ S{\isacartoucheclose}\ \isacommand{by}\isamarkupfalse%
\ {\isacharparenleft}rule\ sspec{\isacharparenright}\isanewline
\ \ \ \ \ \ \isacommand{then}\isamarkupfalse%
\ \isacommand{have}\isamarkupfalse%
\ {\isachardoublequoteopen}S{\isacharprime}\ {\isasymin}\ C{\isachardoublequoteclose}\isanewline
\ \ \ \ \ \ \ \ \isacommand{using}\isamarkupfalse%
\ {\isacartoucheopen}finite\ S{\isacharprime}{\isacartoucheclose}\ \isacommand{by}\isamarkupfalse%
\ {\isacharparenleft}rule\ mp{\isacharparenright}\isanewline
\ \ \ \ \ \ \isacommand{have}\isamarkupfalse%
\ {\isachardoublequoteopen}{\isasymbottom}\ {\isasymnotin}\ S{\isacharprime}\isanewline
\ \ \ \ \ \ \ \ \ \ \ \ {\isasymand}\ {\isacharparenleft}{\isasymforall}k{\isachardot}\ Atom\ k\ {\isasymin}\ S{\isacharprime}\ {\isasymlongrightarrow}\ \isactrlbold {\isasymnot}\ {\isacharparenleft}Atom\ k{\isacharparenright}\ {\isasymin}\ S{\isacharprime}\ {\isasymlongrightarrow}\ False{\isacharparenright}\isanewline
\ \ \ \ \ \ \ \ \ \ \ \ {\isasymand}\ {\isacharparenleft}{\isasymforall}F\ G\ H{\isachardot}\ Con\ F\ G\ H\ {\isasymlongrightarrow}\ F\ {\isasymin}\ S{\isacharprime}\ {\isasymlongrightarrow}\ {\isacharbraceleft}G{\isacharcomma}H{\isacharbraceright}\ {\isasymunion}\ S{\isacharprime}\ {\isasymin}\ C{\isacharparenright}\isanewline
\ \ \ \ \ \ \ \ \ \ \ \ {\isasymand}\ {\isacharparenleft}{\isasymforall}F\ G\ H{\isachardot}\ Dis\ F\ G\ H\ {\isasymlongrightarrow}\ F\ {\isasymin}\ S{\isacharprime}\ {\isasymlongrightarrow}\ {\isacharbraceleft}G{\isacharbraceright}\ {\isasymunion}\ S{\isacharprime}\ {\isasymin}\ C\ {\isasymor}\ {\isacharbraceleft}H{\isacharbraceright}\ {\isasymunion}\ S{\isacharprime}\ {\isasymin}\ C{\isacharparenright}{\isachardoublequoteclose}\isanewline
\ \ \ \ \ \ \ \ \isacommand{using}\isamarkupfalse%
\ PCP\ {\isacartoucheopen}S{\isacharprime}\ {\isasymin}\ C{\isacartoucheclose}\ \isacommand{by}\isamarkupfalse%
\ {\isacharparenleft}rule\ bspec{\isacharparenright}\isanewline
\ \ \ \ \ \ \isacommand{then}\isamarkupfalse%
\ \isacommand{have}\isamarkupfalse%
\ {\isachardoublequoteopen}{\isasymforall}F\ G\ H{\isachardot}\ Con\ F\ G\ H\ {\isasymlongrightarrow}\ F\ {\isasymin}\ S{\isacharprime}\ {\isasymlongrightarrow}\ {\isacharbraceleft}G{\isacharcomma}\ H{\isacharbraceright}\ {\isasymunion}\ S{\isacharprime}\ {\isasymin}\ C{\isachardoublequoteclose}\isanewline
\ \ \ \ \ \ \ \ \isacommand{by}\isamarkupfalse%
\ {\isacharparenleft}iprover\ elim{\isacharcolon}\ conjunct{\isadigit{2}}\ conjunct{\isadigit{1}}{\isacharparenright}\isanewline
\ \ \ \ \ \ \isacommand{then}\isamarkupfalse%
\ \isacommand{have}\isamarkupfalse%
\ {\isachardoublequoteopen}Con\ F\ G\ H\ {\isasymlongrightarrow}\ F\ {\isasymin}\ S{\isacharprime}\ {\isasymlongrightarrow}\ {\isacharbraceleft}G{\isacharcomma}\ H{\isacharbraceright}\ {\isasymunion}\ S{\isacharprime}\ {\isasymin}\ C{\isachardoublequoteclose}\isanewline
\ \ \ \ \ \ \ \ \isacommand{by}\isamarkupfalse%
\ {\isacharparenleft}iprover\ elim{\isacharcolon}\ allE{\isacharparenright}\isanewline
\ \ \ \ \ \ \isacommand{then}\isamarkupfalse%
\ \isacommand{have}\isamarkupfalse%
\ {\isachardoublequoteopen}F\ {\isasymin}\ S{\isacharprime}\ {\isasymlongrightarrow}\ {\isacharbraceleft}G{\isacharcomma}H{\isacharbraceright}\ {\isasymunion}\ S{\isacharprime}\ {\isasymin}\ C{\isachardoublequoteclose}\isanewline
\ \ \ \ \ \ \ \ \isacommand{using}\isamarkupfalse%
\ assms{\isacharparenleft}{\isadigit{4}}{\isacharparenright}\ \isacommand{by}\isamarkupfalse%
\ {\isacharparenleft}rule\ mp{\isacharparenright}\isanewline
\ \ \ \ \ \ \isacommand{thus}\isamarkupfalse%
\ {\isachardoublequoteopen}{\isacharbraceleft}G{\isacharcomma}\ H{\isacharbraceright}\ {\isasymunion}\ S{\isacharprime}\ {\isasymin}\ C{\isachardoublequoteclose}\isanewline
\ \ \ \ \ \ \ \ \isacommand{using}\isamarkupfalse%
\ {\isacartoucheopen}F\ {\isasymin}\ S{\isacharprime}{\isacartoucheclose}\ \isacommand{by}\isamarkupfalse%
\ {\isacharparenleft}rule\ mp{\isacharparenright}\isanewline
\ \ \ \ \isacommand{qed}\isamarkupfalse%
\isanewline
\ \ \isacommand{qed}\isamarkupfalse%
\isanewline
\ \ \isacommand{have}\isamarkupfalse%
\ {\isachardoublequoteopen}{\isacharbraceleft}G{\isacharcomma}H{\isacharbraceright}\ {\isasymunion}\ S\ {\isasymin}\ {\isacharparenleft}extF\ C{\isacharparenright}{\isachardoublequoteclose}\isanewline
\ \ \ \ \isacommand{unfolding}\isamarkupfalse%
\ mem{\isacharunderscore}Collect{\isacharunderscore}eq\ Un{\isacharunderscore}iff\ extF\isanewline
\ \ \isacommand{proof}\isamarkupfalse%
\ {\isacharparenleft}rule\ sallI{\isacharparenright}\isanewline
\ \ \ \ \isacommand{fix}\isamarkupfalse%
\ S{\isacharprime}\isanewline
\ \ \ \ \isacommand{assume}\isamarkupfalse%
\ H{\isacharcolon}{\isachardoublequoteopen}S{\isacharprime}\ {\isasymsubseteq}\ {\isacharbraceleft}G{\isacharcomma}H{\isacharbraceright}\ {\isasymunion}\ S{\isachardoublequoteclose}\isanewline
\ \ \ \ \isacommand{show}\isamarkupfalse%
\ {\isachardoublequoteopen}finite\ S{\isacharprime}\ {\isasymlongrightarrow}\ S{\isacharprime}\ {\isasymin}\ C{\isachardoublequoteclose}\isanewline
\ \ \ \ \isacommand{proof}\isamarkupfalse%
\ {\isacharparenleft}rule\ impI{\isacharparenright}\isanewline
\ \ \ \ \ \ \isacommand{assume}\isamarkupfalse%
\ {\isachardoublequoteopen}finite\ S{\isacharprime}{\isachardoublequoteclose}\isanewline
\ \ \ \ \ \ \isacommand{have}\isamarkupfalse%
\ {\isachardoublequoteopen}S{\isacharprime}\ {\isacharminus}\ {\isacharbraceleft}G{\isacharcomma}H{\isacharbraceright}\ {\isasymsubseteq}\ S{\isachardoublequoteclose}\isanewline
\ \ \ \ \ \ \ \ \isacommand{using}\isamarkupfalse%
\ H\ \isacommand{by}\isamarkupfalse%
\ {\isacharparenleft}simp\ only{\isacharcolon}\ Diff{\isacharunderscore}subset{\isacharunderscore}conv{\isacharparenright}\isanewline
\ \ \ \ \ \ \isacommand{have}\isamarkupfalse%
\ {\isachardoublequoteopen}F\ {\isasymin}\ S\ {\isasymand}\ {\isacharparenleft}S{\isacharprime}\ {\isacharminus}\ {\isacharbraceleft}G{\isacharcomma}H{\isacharbraceright}\ {\isasymsubseteq}\ S{\isacharparenright}{\isachardoublequoteclose}\isanewline
\ \ \ \ \ \ \ \ \isacommand{using}\isamarkupfalse%
\ assms{\isacharparenleft}{\isadigit{5}}{\isacharparenright}\ {\isacartoucheopen}S{\isacharprime}\ {\isacharminus}\ {\isacharbraceleft}G{\isacharcomma}H{\isacharbraceright}\ {\isasymsubseteq}\ S{\isacartoucheclose}\ \isacommand{by}\isamarkupfalse%
\ {\isacharparenleft}rule\ conjI{\isacharparenright}\isanewline
\ \ \ \ \ \ \isacommand{then}\isamarkupfalse%
\ \isacommand{have}\isamarkupfalse%
\ {\isachardoublequoteopen}insert\ F\ \ {\isacharparenleft}S{\isacharprime}\ {\isacharminus}\ {\isacharbraceleft}G{\isacharcomma}H{\isacharbraceright}{\isacharparenright}\ {\isasymsubseteq}\ S{\isachardoublequoteclose}\ \isanewline
\ \ \ \ \ \ \ \ \isacommand{by}\isamarkupfalse%
\ {\isacharparenleft}simp\ only{\isacharcolon}\ insert{\isacharunderscore}subset{\isacharparenright}\isanewline
\ \ \ \ \ \ \isacommand{have}\isamarkupfalse%
\ F{\isadigit{1}}{\isacharcolon}{\isachardoublequoteopen}finite\ {\isacharparenleft}insert\ F\ \ {\isacharparenleft}S{\isacharprime}\ {\isacharminus}\ {\isacharbraceleft}G{\isacharcomma}H{\isacharbraceright}{\isacharparenright}{\isacharparenright}\ {\isasymlongrightarrow}\ F\ {\isasymin}\ {\isacharparenleft}insert\ F\ \ {\isacharparenleft}S{\isacharprime}\ {\isacharminus}\ {\isacharbraceleft}G{\isacharcomma}H{\isacharbraceright}{\isacharparenright}{\isacharparenright}\ {\isasymlongrightarrow}\ {\isacharbraceleft}G{\isacharcomma}H{\isacharbraceright}\ {\isasymunion}\ {\isacharparenleft}insert\ F\ \ {\isacharparenleft}S{\isacharprime}\ {\isacharminus}\ {\isacharbraceleft}G{\isacharcomma}H{\isacharbraceright}{\isacharparenright}{\isacharparenright}\ {\isasymin}\ C{\isachardoublequoteclose}\isanewline
\ \ \ \ \ \ \ \ \isacommand{using}\isamarkupfalse%
\ {\isadigit{1}}\ {\isacartoucheopen}insert\ F\ \ {\isacharparenleft}S{\isacharprime}\ {\isacharminus}\ {\isacharbraceleft}G{\isacharcomma}H{\isacharbraceright}{\isacharparenright}\ {\isasymsubseteq}\ S{\isacartoucheclose}\ \isacommand{by}\isamarkupfalse%
\ {\isacharparenleft}rule\ sspec{\isacharparenright}\isanewline
\ \ \ \ \ \ \isacommand{have}\isamarkupfalse%
\ {\isachardoublequoteopen}finite\ {\isacharparenleft}S{\isacharprime}\ {\isacharminus}\ {\isacharbraceleft}G{\isacharcomma}H{\isacharbraceright}{\isacharparenright}{\isachardoublequoteclose}\isanewline
\ \ \ \ \ \ \ \ \isacommand{using}\isamarkupfalse%
\ {\isacartoucheopen}finite\ S{\isacharprime}{\isacartoucheclose}\ \isacommand{by}\isamarkupfalse%
\ {\isacharparenleft}rule\ finite{\isacharunderscore}Diff{\isacharparenright}\isanewline
\ \ \ \ \ \ \isacommand{then}\isamarkupfalse%
\ \isacommand{have}\isamarkupfalse%
\ {\isachardoublequoteopen}finite\ {\isacharparenleft}insert\ F\ {\isacharparenleft}S{\isacharprime}\ {\isacharminus}\ {\isacharbraceleft}G{\isacharcomma}H{\isacharbraceright}{\isacharparenright}{\isacharparenright}{\isachardoublequoteclose}\ \isanewline
\ \ \ \ \ \ \ \ \isacommand{by}\isamarkupfalse%
\ {\isacharparenleft}rule\ finite{\isachardot}insertI{\isacharparenright}\isanewline
\ \ \ \ \ \ \isacommand{have}\isamarkupfalse%
\ F{\isadigit{2}}{\isacharcolon}{\isachardoublequoteopen}F\ {\isasymin}\ {\isacharparenleft}insert\ F\ \ {\isacharparenleft}S{\isacharprime}\ {\isacharminus}\ {\isacharbraceleft}G{\isacharcomma}H{\isacharbraceright}{\isacharparenright}{\isacharparenright}\ {\isasymlongrightarrow}\ {\isacharbraceleft}G{\isacharcomma}H{\isacharbraceright}\ {\isasymunion}\ {\isacharparenleft}insert\ F\ \ {\isacharparenleft}S{\isacharprime}\ {\isacharminus}\ {\isacharbraceleft}G{\isacharcomma}H{\isacharbraceright}{\isacharparenright}{\isacharparenright}\ {\isasymin}\ C{\isachardoublequoteclose}\isanewline
\ \ \ \ \ \ \ \ \isacommand{using}\isamarkupfalse%
\ F{\isadigit{1}}\ {\isacartoucheopen}finite\ {\isacharparenleft}insert\ F\ {\isacharparenleft}S{\isacharprime}\ {\isacharminus}\ {\isacharbraceleft}G{\isacharcomma}H{\isacharbraceright}{\isacharparenright}{\isacharparenright}{\isacartoucheclose}\ \isacommand{by}\isamarkupfalse%
\ {\isacharparenleft}rule\ mp{\isacharparenright}\isanewline
\ \ \ \ \ \ \isacommand{have}\isamarkupfalse%
\ {\isachardoublequoteopen}F\ {\isasymin}\ {\isacharparenleft}insert\ F\ \ {\isacharparenleft}S{\isacharprime}\ {\isacharminus}\ {\isacharbraceleft}G{\isacharcomma}H{\isacharbraceright}{\isacharparenright}{\isacharparenright}{\isachardoublequoteclose}\isanewline
\ \ \ \ \ \ \ \ \isacommand{by}\isamarkupfalse%
\ {\isacharparenleft}simp\ only{\isacharcolon}\ insertI{\isadigit{1}}{\isacharparenright}\isanewline
\ \ \ \ \ \ \isacommand{have}\isamarkupfalse%
\ F{\isadigit{3}}{\isacharcolon}{\isachardoublequoteopen}{\isacharbraceleft}G{\isacharcomma}H{\isacharbraceright}\ {\isasymunion}\ {\isacharparenleft}insert\ F\ {\isacharparenleft}S{\isacharprime}\ {\isacharminus}\ {\isacharbraceleft}G{\isacharcomma}H{\isacharbraceright}{\isacharparenright}{\isacharparenright}\ {\isasymin}\ C{\isachardoublequoteclose}\isanewline
\ \ \ \ \ \ \ \ \isacommand{using}\isamarkupfalse%
\ F{\isadigit{2}}\ {\isacartoucheopen}F\ {\isasymin}\ insert\ F\ {\isacharparenleft}S{\isacharprime}\ {\isacharminus}\ {\isacharbraceleft}G{\isacharcomma}H{\isacharbraceright}{\isacharparenright}{\isacartoucheclose}\ \isacommand{by}\isamarkupfalse%
\ {\isacharparenleft}rule\ mp{\isacharparenright}\isanewline
\ \ \ \ \ \ \isacommand{have}\isamarkupfalse%
\ IU{\isadigit{1}}{\isacharcolon}{\isachardoublequoteopen}insert\ F\ {\isacharparenleft}S{\isacharprime}\ {\isacharminus}\ {\isacharbraceleft}G{\isacharcomma}H{\isacharbraceright}{\isacharparenright}\ {\isacharequal}\ {\isacharbraceleft}F{\isacharbraceright}\ {\isasymunion}\ {\isacharparenleft}S{\isacharprime}\ {\isacharminus}\ {\isacharbraceleft}G{\isacharcomma}H{\isacharbraceright}{\isacharparenright}{\isachardoublequoteclose}\isanewline
\ \ \ \ \ \ \ \ \isacommand{by}\isamarkupfalse%
\ {\isacharparenleft}rule\ insert{\isacharunderscore}is{\isacharunderscore}Un{\isacharparenright}\isanewline
\ \ \ \ \ \ \isacommand{have}\isamarkupfalse%
\ IU{\isadigit{2}}{\isacharcolon}{\isachardoublequoteopen}insert\ F\ {\isacharparenleft}{\isacharbraceleft}G{\isacharcomma}H{\isacharbraceright}\ {\isasymunion}\ S{\isacharprime}{\isacharparenright}\ {\isacharequal}\ {\isacharbraceleft}F{\isacharbraceright}\ {\isasymunion}\ {\isacharparenleft}{\isacharbraceleft}G{\isacharcomma}H{\isacharbraceright}\ {\isasymunion}\ S{\isacharprime}{\isacharparenright}{\isachardoublequoteclose}\isanewline
\ \ \ \ \ \ \ \ \isacommand{by}\isamarkupfalse%
\ {\isacharparenleft}rule\ insert{\isacharunderscore}is{\isacharunderscore}Un{\isacharparenright}\isanewline
\ \ \ \ \ \ \isacommand{have}\isamarkupfalse%
\ GH{\isacharcolon}{\isachardoublequoteopen}insert\ G\ {\isacharparenleft}insert\ H\ S{\isacharprime}{\isacharparenright}\ {\isacharequal}\ {\isacharbraceleft}G{\isacharcomma}H{\isacharbraceright}\ {\isasymunion}\ S{\isacharprime}{\isachardoublequoteclose}\isanewline
\ \ \ \ \ \ \ \ \isacommand{by}\isamarkupfalse%
\ {\isacharparenleft}rule\ insertSetElem{\isacharparenright}\isanewline
\ \ \ \ \ \ \isacommand{have}\isamarkupfalse%
\ {\isachardoublequoteopen}{\isacharbraceleft}G{\isacharcomma}H{\isacharbraceright}\ {\isasymunion}\ {\isacharparenleft}insert\ F\ {\isacharparenleft}S{\isacharprime}\ {\isacharminus}\ {\isacharbraceleft}G{\isacharcomma}H{\isacharbraceright}{\isacharparenright}{\isacharparenright}\ {\isacharequal}\ {\isacharbraceleft}G{\isacharcomma}H{\isacharbraceright}\ {\isasymunion}\ {\isacharparenleft}{\isacharbraceleft}F{\isacharbraceright}\ {\isasymunion}\ {\isacharparenleft}S{\isacharprime}\ {\isacharminus}\ {\isacharbraceleft}G{\isacharcomma}H{\isacharbraceright}{\isacharparenright}{\isacharparenright}{\isachardoublequoteclose}\isanewline
\ \ \ \ \ \ \ \ \isacommand{by}\isamarkupfalse%
\ {\isacharparenleft}simp\ only{\isacharcolon}\ IU{\isadigit{1}}{\isacharparenright}\isanewline
\ \ \ \ \ \ \isacommand{also}\isamarkupfalse%
\ \isacommand{have}\isamarkupfalse%
\ {\isachardoublequoteopen}{\isasymdots}\ {\isacharequal}\ {\isacharbraceleft}F{\isacharbraceright}\ {\isasymunion}\ {\isacharparenleft}{\isacharbraceleft}G{\isacharcomma}H{\isacharbraceright}\ {\isasymunion}\ {\isacharparenleft}S{\isacharprime}\ {\isacharminus}\ {\isacharbraceleft}G{\isacharcomma}H{\isacharbraceright}{\isacharparenright}{\isacharparenright}{\isachardoublequoteclose}\isanewline
\ \ \ \ \ \ \ \ \isacommand{by}\isamarkupfalse%
\ {\isacharparenleft}simp\ only{\isacharcolon}\ Un{\isacharunderscore}left{\isacharunderscore}commute{\isacharparenright}\isanewline
\ \ \ \ \ \ \isacommand{also}\isamarkupfalse%
\ \isacommand{have}\isamarkupfalse%
\ {\isachardoublequoteopen}{\isasymdots}\ {\isacharequal}\ {\isacharbraceleft}F{\isacharbraceright}\ {\isasymunion}\ {\isacharparenleft}{\isacharbraceleft}G{\isacharcomma}H{\isacharbraceright}\ {\isasymunion}\ S{\isacharprime}{\isacharparenright}{\isachardoublequoteclose}\isanewline
\ \ \ \ \ \ \ \ \isacommand{by}\isamarkupfalse%
\ {\isacharparenleft}simp\ only{\isacharcolon}\ Un{\isacharunderscore}Diff{\isacharunderscore}cancel{\isacharparenright}\isanewline
\ \ \ \ \ \ \isacommand{also}\isamarkupfalse%
\ \isacommand{have}\isamarkupfalse%
\ {\isachardoublequoteopen}{\isasymdots}\ {\isacharequal}\ insert\ F\ {\isacharparenleft}{\isacharbraceleft}G{\isacharcomma}H{\isacharbraceright}\ {\isasymunion}\ S{\isacharprime}{\isacharparenright}{\isachardoublequoteclose}\isanewline
\ \ \ \ \ \ \ \ \isacommand{by}\isamarkupfalse%
\ {\isacharparenleft}simp\ only{\isacharcolon}\ IU{\isadigit{2}}{\isacharparenright}\isanewline
\ \ \ \ \ \ \isacommand{also}\isamarkupfalse%
\ \isacommand{have}\isamarkupfalse%
\ {\isachardoublequoteopen}{\isasymdots}\ {\isacharequal}\ insert\ F\ {\isacharparenleft}insert\ G\ {\isacharparenleft}insert\ H\ S{\isacharprime}{\isacharparenright}{\isacharparenright}{\isachardoublequoteclose}\isanewline
\ \ \ \ \ \ \ \ \isacommand{by}\isamarkupfalse%
\ {\isacharparenleft}simp\ only{\isacharcolon}\ GH{\isacharparenright}\isanewline
\ \ \ \ \ \ \isacommand{finally}\isamarkupfalse%
\ \isacommand{have}\isamarkupfalse%
\ F{\isadigit{4}}{\isacharcolon}{\isachardoublequoteopen}{\isacharbraceleft}G{\isacharcomma}H{\isacharbraceright}\ {\isasymunion}\ {\isacharparenleft}insert\ F\ {\isacharparenleft}S{\isacharprime}\ {\isacharminus}\ {\isacharbraceleft}G{\isacharcomma}H{\isacharbraceright}{\isacharparenright}{\isacharparenright}\ {\isacharequal}\ insert\ F\ {\isacharparenleft}insert\ G\ {\isacharparenleft}insert\ H\ S{\isacharprime}{\isacharparenright}{\isacharparenright}{\isachardoublequoteclose}\isanewline
\ \ \ \ \ \ \ \ \isacommand{by}\isamarkupfalse%
\ this\isanewline
\ \ \ \ \ \ \isacommand{have}\isamarkupfalse%
\ C{\isadigit{1}}{\isacharcolon}{\isachardoublequoteopen}insert\ F\ {\isacharparenleft}insert\ G\ {\isacharparenleft}insert\ H\ S{\isacharprime}{\isacharparenright}{\isacharparenright}\ {\isasymin}\ C{\isachardoublequoteclose}\isanewline
\ \ \ \ \ \ \ \ \isacommand{using}\isamarkupfalse%
\ F{\isadigit{3}}\ \isacommand{by}\isamarkupfalse%
\ {\isacharparenleft}simp\ only{\isacharcolon}\ F{\isadigit{4}}{\isacharparenright}\isanewline
\ \ \ \ \ \ \isacommand{have}\isamarkupfalse%
\ {\isachardoublequoteopen}S{\isacharprime}\ {\isasymsubseteq}\ insert\ F\ S{\isacharprime}{\isachardoublequoteclose}\isanewline
\ \ \ \ \ \ \ \ \isacommand{by}\isamarkupfalse%
\ {\isacharparenleft}rule\ subset{\isacharunderscore}insertI{\isacharparenright}\isanewline
\ \ \ \ \ \ \isacommand{then}\isamarkupfalse%
\ \isacommand{have}\isamarkupfalse%
\ C{\isadigit{2}}{\isacharcolon}{\isachardoublequoteopen}S{\isacharprime}\ {\isasymsubseteq}\ insert\ F\ {\isacharparenleft}insert\ G\ {\isacharparenleft}insert\ H\ S{\isacharprime}{\isacharparenright}{\isacharparenright}{\isachardoublequoteclose}\isanewline
\ \ \ \ \ \ \ \ \isacommand{by}\isamarkupfalse%
\ {\isacharparenleft}simp\ only{\isacharcolon}\ subset{\isacharunderscore}insertI{\isadigit{2}}{\isacharparenright}\isanewline
\ \ \ \ \ \ \isacommand{let}\isamarkupfalse%
\ {\isacharquery}S{\isacharequal}{\isachardoublequoteopen}insert\ F\ {\isacharparenleft}insert\ G\ {\isacharparenleft}insert\ H\ S{\isacharprime}{\isacharparenright}{\isacharparenright}{\isachardoublequoteclose}\isanewline
\ \ \ \ \ \ \isacommand{have}\isamarkupfalse%
\ {\isachardoublequoteopen}{\isasymforall}S\ {\isasymin}\ C{\isachardot}\ {\isasymforall}S{\isacharprime}\ {\isasymsubseteq}\ S{\isachardot}\ S{\isacharprime}\ {\isasymin}\ C{\isachardoublequoteclose}\isanewline
\ \ \ \ \ \ \ \ \isacommand{using}\isamarkupfalse%
\ assms{\isacharparenleft}{\isadigit{2}}{\isacharparenright}\ \isacommand{by}\isamarkupfalse%
\ {\isacharparenleft}simp\ only{\isacharcolon}\ subset{\isacharunderscore}closed{\isacharunderscore}def{\isacharparenright}\isanewline
\ \ \ \ \ \ \isacommand{then}\isamarkupfalse%
\ \isacommand{have}\isamarkupfalse%
\ {\isachardoublequoteopen}{\isasymforall}S{\isacharprime}\ {\isasymsubseteq}\ {\isacharquery}S{\isachardot}\ S{\isacharprime}\ {\isasymin}\ C{\isachardoublequoteclose}\isanewline
\ \ \ \ \ \ \ \ \isacommand{using}\isamarkupfalse%
\ C{\isadigit{1}}\ \isacommand{by}\isamarkupfalse%
\ {\isacharparenleft}rule\ bspec{\isacharparenright}\isanewline
\ \ \ \ \ \ \isacommand{thus}\isamarkupfalse%
\ {\isachardoublequoteopen}S{\isacharprime}\ {\isasymin}\ C{\isachardoublequoteclose}\isanewline
\ \ \ \ \ \ \ \ \isacommand{using}\isamarkupfalse%
\ C{\isadigit{2}}\ \isacommand{by}\isamarkupfalse%
\ {\isacharparenleft}rule\ sspec{\isacharparenright}\isanewline
\ \ \ \ \isacommand{qed}\isamarkupfalse%
\isanewline
\ \ \isacommand{qed}\isamarkupfalse%
\isanewline
\ \ \isacommand{thus}\isamarkupfalse%
\ {\isachardoublequoteopen}{\isacharbraceleft}G{\isacharcomma}H{\isacharbraceright}\ {\isasymunion}\ S\ {\isasymin}\ {\isacharparenleft}extensionFin\ C{\isacharparenright}{\isachardoublequoteclose}\isanewline
\ \ \ \ \isacommand{unfolding}\isamarkupfalse%
\ extensionFin\ \isacommand{by}\isamarkupfalse%
\ {\isacharparenleft}rule\ UnI{\isadigit{2}}{\isacharparenright}\isanewline
\isacommand{qed}\isamarkupfalse%
%
\endisatagproof
{\isafoldproof}%
%
\isadelimproof
%
\endisadelimproof
%
\begin{isamarkuptext}%
Seguidamente, vamos a probar el lema auxiliar \isa{ex{\isadigit{3}}{\isacharunderscore}pcp{\isacharunderscore}SinE{\isacharunderscore}DIS}. Este demuestra que si \isa{C} es 
  una colección con la propiedad de consistencia proposicional y cerrada bajo subconjuntos, \isa{S\ {\isasymin}\ E}
  y sea \isa{F} una fórmula de tipo \isa{{\isasymbeta}} con componentes \isa{{\isasymbeta}\isactrlsub {\isadigit{1}}} y \isa{{\isasymbeta}\isactrlsub {\isadigit{2}}}, se verifica que o bien 
  \isa{{\isacharbraceleft}{\isasymbeta}\isactrlsub {\isadigit{1}}{\isacharbraceright}\ {\isasymunion}\ S\ {\isasymin}\ C{\isacharprime}} o bien \isa{{\isacharbraceleft}{\isasymbeta}\isactrlsub {\isadigit{2}}{\isacharbraceright}\ {\isasymunion}\ S\ {\isasymin}\ C{\isacharprime}}. Dicha prueba se realizará por reducción al absurdo. Para
  ello precisaremos de dos lemas previos que nos permitan llegar a una contradicción: 
  \isa{ex{\isadigit{3}}{\isacharunderscore}pcp{\isacharunderscore}SinE{\isacharunderscore}DIS{\isacharunderscore}auxEx} y \isa{ex{\isadigit{3}}{\isacharunderscore}pcp{\isacharunderscore}SinE{\isacharunderscore}DIS{\isacharunderscore}auxFalse}.

  En primer lugar, veamos la demostración del lema \isa{ex{\isadigit{3}}{\isacharunderscore}pcp{\isacharunderscore}SinE{\isacharunderscore}DIS{\isacharunderscore}auxEx}. Este prueba que dada 
  \isa{C} una colección con la propiedad de consistencia proposicional y cerrada bajo subconjuntos, 
  \isa{S\ {\isasymin}\ E} y sea \isa{F} es una fórmula de tipo \isa{{\isasymbeta}} de componentes \isa{{\isasymbeta}\isactrlsub {\isadigit{1}}} y \isa{{\isasymbeta}\isactrlsub {\isadigit{2}}}, si consideramos \isa{S\isactrlsub {\isadigit{1}}} y 
  \isa{S\isactrlsub {\isadigit{2}}} subconjuntos finitos cualesquiera de \isa{S} tales que \isa{F\ {\isasymin}\ S\isactrlsub {\isadigit{1}}} y\\ \isa{F\ {\isasymin}\ S\isactrlsub {\isadigit{2}}}, entonces existe una 
  fórmula \isa{I\ {\isasymin}\ {\isacharbraceleft}{\isasymbeta}\isactrlsub {\isadigit{1}}{\isacharcomma}{\isasymbeta}\isactrlsub {\isadigit{2}}{\isacharbraceright}} tal que se verifica que tanto \isa{{\isacharbraceleft}I{\isacharbraceright}\ {\isasymunion}\ S\isactrlsub {\isadigit{1}}} como \isa{{\isacharbraceleft}I{\isacharbraceright}\ {\isasymunion}\ S\isactrlsub {\isadigit{2}}} están en \isa{C}.%
\end{isamarkuptext}\isamarkuptrue%
\isacommand{lemma}\isamarkupfalse%
\ ex{\isadigit{3}}{\isacharunderscore}pcp{\isacharunderscore}SinE{\isacharunderscore}DIS{\isacharunderscore}auxEx{\isacharcolon}\isanewline
\ \ \isakeyword{assumes}\ {\isachardoublequoteopen}pcp\ C{\isachardoublequoteclose}\isanewline
\ \ \ \ \ \ \ \ \ \ {\isachardoublequoteopen}subset{\isacharunderscore}closed\ C{\isachardoublequoteclose}\isanewline
\ \ \ \ \ \ \ \ \ \ {\isachardoublequoteopen}S\ {\isasymin}\ {\isacharparenleft}extF\ C{\isacharparenright}{\isachardoublequoteclose}\isanewline
\ \ \ \ \ \ \ \ \ \ {\isachardoublequoteopen}Dis\ F\ G\ H{\isachardoublequoteclose}\isanewline
\ \ \ \ \ \ \ \ \ \ {\isachardoublequoteopen}S{\isadigit{1}}\ {\isasymsubseteq}\ S{\isachardoublequoteclose}\isanewline
\ \ \ \ \ \ \ \ \ \ {\isachardoublequoteopen}finite\ S{\isadigit{1}}{\isachardoublequoteclose}\isanewline
\ \ \ \ \ \ \ \ \ \ {\isachardoublequoteopen}F\ {\isasymin}\ S{\isadigit{1}}{\isachardoublequoteclose}\isanewline
\ \ \ \ \ \ \ \ \ \ {\isachardoublequoteopen}S{\isadigit{2}}\ {\isasymsubseteq}\ S{\isachardoublequoteclose}\isanewline
\ \ \ \ \ \ \ \ \ \ {\isachardoublequoteopen}finite\ S{\isadigit{2}}{\isachardoublequoteclose}\isanewline
\ \ \ \ \ \ \ \ \ \ {\isachardoublequoteopen}F\ {\isasymin}\ S{\isadigit{2}}{\isachardoublequoteclose}\isanewline
\ \ \isakeyword{shows}\ {\isachardoublequoteopen}{\isasymexists}I{\isasymin}{\isacharbraceleft}G{\isacharcomma}H{\isacharbraceright}{\isachardot}\ insert\ I\ S{\isadigit{1}}\ {\isasymin}\ C\ {\isasymand}\ insert\ I\ S{\isadigit{2}}\ {\isasymin}\ C{\isachardoublequoteclose}\ \isanewline
%
\isadelimproof
%
\endisadelimproof
%
\isatagproof
\isacommand{proof}\isamarkupfalse%
\ {\isacharminus}\isanewline
\ \ \isacommand{let}\isamarkupfalse%
\ {\isacharquery}S\ {\isacharequal}\ {\isachardoublequoteopen}S{\isadigit{1}}\ {\isasymunion}\ S{\isadigit{2}}{\isachardoublequoteclose}\isanewline
\ \ \isacommand{have}\isamarkupfalse%
\ {\isachardoublequoteopen}S{\isadigit{1}}\ {\isasymsubseteq}\ {\isacharquery}S{\isachardoublequoteclose}\isanewline
\ \ \ \ \isacommand{by}\isamarkupfalse%
\ {\isacharparenleft}simp\ only{\isacharcolon}\ Un{\isacharunderscore}upper{\isadigit{1}}{\isacharparenright}\isanewline
\ \ \isacommand{have}\isamarkupfalse%
\ {\isachardoublequoteopen}S{\isadigit{2}}\ {\isasymsubseteq}\ {\isacharquery}S{\isachardoublequoteclose}\isanewline
\ \ \ \ \isacommand{by}\isamarkupfalse%
\ {\isacharparenleft}simp\ only{\isacharcolon}\ Un{\isacharunderscore}upper{\isadigit{2}}{\isacharparenright}\isanewline
\ \ \isacommand{have}\isamarkupfalse%
\ {\isachardoublequoteopen}finite\ {\isacharquery}S{\isachardoublequoteclose}\isanewline
\ \ \ \ \isacommand{using}\isamarkupfalse%
\ assms{\isacharparenleft}{\isadigit{6}}{\isacharparenright}\ assms{\isacharparenleft}{\isadigit{9}}{\isacharparenright}\ \isacommand{by}\isamarkupfalse%
\ {\isacharparenleft}rule\ finite{\isacharunderscore}UnI{\isacharparenright}\isanewline
\ \ \isacommand{have}\isamarkupfalse%
\ {\isachardoublequoteopen}{\isacharquery}S\ {\isasymsubseteq}\ S{\isachardoublequoteclose}\ \isanewline
\ \ \ \ \isacommand{using}\isamarkupfalse%
\ assms{\isacharparenleft}{\isadigit{5}}{\isacharparenright}\ assms{\isacharparenleft}{\isadigit{8}}{\isacharparenright}\ \isacommand{by}\isamarkupfalse%
\ {\isacharparenleft}simp\ only{\isacharcolon}\ Un{\isacharunderscore}subset{\isacharunderscore}iff{\isacharparenright}\isanewline
\ \ \isacommand{have}\isamarkupfalse%
\ {\isachardoublequoteopen}{\isasymforall}S{\isacharprime}\ {\isasymsubseteq}\ S{\isachardot}\ finite\ S{\isacharprime}\ {\isasymlongrightarrow}\ S{\isacharprime}\ {\isasymin}\ C{\isachardoublequoteclose}\isanewline
\ \ \ \ \isacommand{using}\isamarkupfalse%
\ assms{\isacharparenleft}{\isadigit{3}}{\isacharparenright}\ \isacommand{unfolding}\isamarkupfalse%
\ extF\ \isacommand{by}\isamarkupfalse%
\ {\isacharparenleft}rule\ CollectD{\isacharparenright}\isanewline
\ \ \isacommand{then}\isamarkupfalse%
\ \isacommand{have}\isamarkupfalse%
\ {\isachardoublequoteopen}finite\ {\isacharquery}S\ {\isasymlongrightarrow}\ {\isacharquery}S\ {\isasymin}\ C{\isachardoublequoteclose}\isanewline
\ \ \ \ \isacommand{using}\isamarkupfalse%
\ {\isacartoucheopen}{\isacharquery}S\ {\isasymsubseteq}\ S{\isacartoucheclose}\ \isacommand{by}\isamarkupfalse%
\ {\isacharparenleft}rule\ sspec{\isacharparenright}\isanewline
\ \ \isacommand{then}\isamarkupfalse%
\ \isacommand{have}\isamarkupfalse%
\ {\isachardoublequoteopen}{\isacharquery}S\ {\isasymin}\ C{\isachardoublequoteclose}\ \isanewline
\ \ \ \ \isacommand{using}\isamarkupfalse%
\ {\isacartoucheopen}finite\ {\isacharquery}S{\isacartoucheclose}\ \isacommand{by}\isamarkupfalse%
\ {\isacharparenleft}rule\ mp{\isacharparenright}\isanewline
\ \ \isacommand{have}\isamarkupfalse%
\ {\isachardoublequoteopen}F\ {\isasymin}\ {\isacharquery}S{\isachardoublequoteclose}\ \isanewline
\ \ \ \ \isacommand{using}\isamarkupfalse%
\ assms{\isacharparenleft}{\isadigit{7}}{\isacharparenright}\ \isacommand{by}\isamarkupfalse%
\ {\isacharparenleft}rule\ UnI{\isadigit{1}}{\isacharparenright}\isanewline
\ \ \isacommand{have}\isamarkupfalse%
\ {\isachardoublequoteopen}{\isasymforall}S\ {\isasymin}\ C{\isachardot}\ {\isasymbottom}\ {\isasymnotin}\ S\isanewline
\ \ {\isasymand}\ {\isacharparenleft}{\isasymforall}k{\isachardot}\ Atom\ k\ {\isasymin}\ S\ {\isasymlongrightarrow}\ \isactrlbold {\isasymnot}\ {\isacharparenleft}Atom\ k{\isacharparenright}\ {\isasymin}\ S\ {\isasymlongrightarrow}\ False{\isacharparenright}\isanewline
\ \ {\isasymand}\ {\isacharparenleft}{\isasymforall}F\ G\ H{\isachardot}\ Con\ F\ G\ H\ {\isasymlongrightarrow}\ F\ {\isasymin}\ S\ {\isasymlongrightarrow}\ {\isacharbraceleft}G{\isacharcomma}H{\isacharbraceright}\ {\isasymunion}\ S\ {\isasymin}\ C{\isacharparenright}\isanewline
\ \ {\isasymand}\ {\isacharparenleft}{\isasymforall}F\ G\ H{\isachardot}\ Dis\ F\ G\ H\ {\isasymlongrightarrow}\ F\ {\isasymin}\ S\ {\isasymlongrightarrow}\ {\isacharbraceleft}G{\isacharbraceright}\ {\isasymunion}\ S\ {\isasymin}\ C\ {\isasymor}\ {\isacharbraceleft}H{\isacharbraceright}\ {\isasymunion}\ S\ {\isasymin}\ C{\isacharparenright}{\isachardoublequoteclose}\isanewline
\ \ \ \ \isacommand{using}\isamarkupfalse%
\ assms{\isacharparenleft}{\isadigit{1}}{\isacharparenright}\ \isacommand{by}\isamarkupfalse%
\ {\isacharparenleft}rule\ pcp{\isacharunderscore}alt{\isadigit{1}}{\isacharparenright}\isanewline
\ \ \isacommand{then}\isamarkupfalse%
\ \isacommand{have}\isamarkupfalse%
\ {\isachardoublequoteopen}{\isasymbottom}\ {\isasymnotin}\ {\isacharquery}S\isanewline
\ \ \ \ \ \ \ \ {\isasymand}\ {\isacharparenleft}{\isasymforall}k{\isachardot}\ Atom\ k\ {\isasymin}\ {\isacharquery}S\ {\isasymlongrightarrow}\ \isactrlbold {\isasymnot}\ {\isacharparenleft}Atom\ k{\isacharparenright}\ {\isasymin}\ {\isacharquery}S\ {\isasymlongrightarrow}\ False{\isacharparenright}\isanewline
\ \ \ \ \ \ \ \ {\isasymand}\ {\isacharparenleft}{\isasymforall}F\ G\ H{\isachardot}\ Con\ F\ G\ H\ {\isasymlongrightarrow}\ F\ {\isasymin}\ {\isacharquery}S\ {\isasymlongrightarrow}\ {\isacharbraceleft}G{\isacharcomma}H{\isacharbraceright}\ {\isasymunion}\ {\isacharquery}S\ {\isasymin}\ C{\isacharparenright}\isanewline
\ \ \ \ \ \ \ \ {\isasymand}\ {\isacharparenleft}{\isasymforall}F\ G\ H{\isachardot}\ Dis\ F\ G\ H\ {\isasymlongrightarrow}\ F\ {\isasymin}\ {\isacharquery}S\ {\isasymlongrightarrow}\ {\isacharbraceleft}G{\isacharbraceright}\ {\isasymunion}\ {\isacharquery}S\ {\isasymin}\ C\ {\isasymor}\ {\isacharbraceleft}H{\isacharbraceright}\ {\isasymunion}\ {\isacharquery}S\ {\isasymin}\ C{\isacharparenright}{\isachardoublequoteclose}\isanewline
\ \ \ \ \isacommand{using}\isamarkupfalse%
\ {\isacartoucheopen}{\isacharquery}S\ {\isasymin}\ C{\isacartoucheclose}\ \isacommand{by}\isamarkupfalse%
\ {\isacharparenleft}rule\ bspec{\isacharparenright}\isanewline
\ \ \isacommand{then}\isamarkupfalse%
\ \isacommand{have}\isamarkupfalse%
\ {\isachardoublequoteopen}{\isasymforall}F\ G\ H{\isachardot}\ Dis\ F\ G\ H\ {\isasymlongrightarrow}\ F\ {\isasymin}\ {\isacharquery}S\ {\isasymlongrightarrow}\ {\isacharbraceleft}G{\isacharbraceright}\ {\isasymunion}\ {\isacharquery}S\ {\isasymin}\ C\ {\isasymor}\ {\isacharbraceleft}H{\isacharbraceright}\ {\isasymunion}\ {\isacharquery}S\ {\isasymin}\ C{\isachardoublequoteclose}\isanewline
\ \ \ \ \isacommand{by}\isamarkupfalse%
\ {\isacharparenleft}iprover\ elim{\isacharcolon}\ conjunct{\isadigit{2}}{\isacharparenright}\isanewline
\ \ \isacommand{then}\isamarkupfalse%
\ \isacommand{have}\isamarkupfalse%
\ {\isachardoublequoteopen}Dis\ F\ G\ H\ {\isasymlongrightarrow}\ F\ {\isasymin}\ {\isacharquery}S\ {\isasymlongrightarrow}\ {\isacharbraceleft}G{\isacharbraceright}\ {\isasymunion}\ {\isacharquery}S\ {\isasymin}\ C\ {\isasymor}\ {\isacharbraceleft}H{\isacharbraceright}\ {\isasymunion}\ {\isacharquery}S\ {\isasymin}\ C{\isachardoublequoteclose}\isanewline
\ \ \ \ \isacommand{by}\isamarkupfalse%
\ {\isacharparenleft}iprover\ elim{\isacharcolon}\ allE{\isacharparenright}\isanewline
\ \ \isacommand{then}\isamarkupfalse%
\ \isacommand{have}\isamarkupfalse%
\ {\isachardoublequoteopen}F\ {\isasymin}\ {\isacharquery}S\ {\isasymlongrightarrow}\ {\isacharbraceleft}G{\isacharbraceright}\ {\isasymunion}\ {\isacharquery}S\ {\isasymin}\ C\ {\isasymor}\ {\isacharbraceleft}H{\isacharbraceright}\ {\isasymunion}\ {\isacharquery}S\ {\isasymin}\ C{\isachardoublequoteclose}\isanewline
\ \ \ \ \isacommand{using}\isamarkupfalse%
\ assms{\isacharparenleft}{\isadigit{4}}{\isacharparenright}\ \isacommand{by}\isamarkupfalse%
\ {\isacharparenleft}rule\ mp{\isacharparenright}\isanewline
\ \ \isacommand{then}\isamarkupfalse%
\ \isacommand{have}\isamarkupfalse%
\ insIsUn{\isacharcolon}{\isachardoublequoteopen}{\isacharbraceleft}G{\isacharbraceright}\ {\isasymunion}\ {\isacharquery}S\ {\isasymin}\ C\ {\isasymor}\ {\isacharbraceleft}H{\isacharbraceright}\ {\isasymunion}\ {\isacharquery}S\ {\isasymin}\ C{\isachardoublequoteclose}\isanewline
\ \ \ \ \isacommand{using}\isamarkupfalse%
\ {\isacartoucheopen}F\ {\isasymin}\ {\isacharquery}S{\isacartoucheclose}\ \isacommand{by}\isamarkupfalse%
\ {\isacharparenleft}rule\ mp{\isacharparenright}\isanewline
\ \ \isacommand{have}\isamarkupfalse%
\ insG{\isacharcolon}{\isachardoublequoteopen}insert\ G\ {\isacharquery}S\ {\isacharequal}\ {\isacharbraceleft}G{\isacharbraceright}\ {\isasymunion}\ {\isacharquery}S{\isachardoublequoteclose}\ \isanewline
\ \ \ \ \isacommand{by}\isamarkupfalse%
\ {\isacharparenleft}rule\ insert{\isacharunderscore}is{\isacharunderscore}Un{\isacharparenright}\isanewline
\ \ \isacommand{have}\isamarkupfalse%
\ insH{\isacharcolon}{\isachardoublequoteopen}insert\ H\ {\isacharquery}S\ {\isacharequal}\ {\isacharbraceleft}H{\isacharbraceright}\ {\isasymunion}\ {\isacharquery}S{\isachardoublequoteclose}\isanewline
\ \ \ \ \isacommand{by}\isamarkupfalse%
\ {\isacharparenleft}rule\ insert{\isacharunderscore}is{\isacharunderscore}Un{\isacharparenright}\isanewline
\ \ \isacommand{have}\isamarkupfalse%
\ {\isachardoublequoteopen}insert\ G\ {\isacharquery}S\ {\isasymin}\ C\ {\isasymor}\ insert\ H\ {\isacharquery}S\ {\isasymin}\ C{\isachardoublequoteclose}\isanewline
\ \ \ \ \isacommand{using}\isamarkupfalse%
\ insG\ insH\ \isacommand{by}\isamarkupfalse%
\ {\isacharparenleft}simp\ only{\isacharcolon}\ insIsUn{\isacharparenright}\isanewline
\ \ \isacommand{then}\isamarkupfalse%
\ \isacommand{have}\isamarkupfalse%
\ {\isachardoublequoteopen}{\isacharparenleft}insert\ G\ {\isacharquery}S\ {\isasymin}\ C\ {\isasymor}\ insert\ H\ {\isacharquery}S\ {\isasymin}\ C{\isacharparenright}\ {\isasymor}\ {\isacharparenleft}{\isasymexists}I\ {\isasymin}\ {\isacharbraceleft}{\isacharbraceright}{\isachardot}\ insert\ I\ {\isacharquery}S\ {\isasymin}\ C{\isacharparenright}{\isachardoublequoteclose}\isanewline
\ \ \ \ \isacommand{by}\isamarkupfalse%
\ {\isacharparenleft}simp\ only{\isacharcolon}\ disjI{\isadigit{1}}{\isacharparenright}\isanewline
\ \ \isacommand{then}\isamarkupfalse%
\ \isacommand{have}\isamarkupfalse%
\ {\isachardoublequoteopen}insert\ G\ {\isacharquery}S\ {\isasymin}\ C\ {\isasymor}\ {\isacharparenleft}insert\ H\ {\isacharquery}S\ {\isasymin}\ C\ {\isasymor}\ {\isacharparenleft}{\isasymexists}I\ {\isasymin}\ {\isacharbraceleft}{\isacharbraceright}{\isachardot}\ insert\ I\ {\isacharquery}S\ {\isasymin}\ C{\isacharparenright}{\isacharparenright}{\isachardoublequoteclose}\isanewline
\ \ \ \ \isacommand{by}\isamarkupfalse%
\ {\isacharparenleft}simp\ only{\isacharcolon}\ disj{\isacharunderscore}assoc{\isacharparenright}\isanewline
\ \ \isacommand{then}\isamarkupfalse%
\ \isacommand{have}\isamarkupfalse%
\ {\isachardoublequoteopen}insert\ G\ {\isacharquery}S\ {\isasymin}\ C\ {\isasymor}\ {\isacharparenleft}{\isasymexists}I\ {\isasymin}\ {\isacharbraceleft}H{\isacharbraceright}{\isachardot}\ insert\ I\ {\isacharquery}S\ {\isasymin}\ C{\isacharparenright}{\isachardoublequoteclose}\isanewline
\ \ \ \ \isacommand{by}\isamarkupfalse%
\ {\isacharparenleft}simp\ only{\isacharcolon}\ bex{\isacharunderscore}simps{\isacharparenleft}{\isadigit{5}}{\isacharparenright}{\isacharparenright}\isanewline
\ \ \isacommand{then}\isamarkupfalse%
\ \isacommand{have}\isamarkupfalse%
\ {\isadigit{1}}{\isacharcolon}{\isachardoublequoteopen}{\isasymexists}I\ {\isasymin}\ {\isacharbraceleft}G{\isacharcomma}H{\isacharbraceright}{\isachardot}\ insert\ I\ {\isacharquery}S\ {\isasymin}\ C{\isachardoublequoteclose}\ \isanewline
\ \ \ \ \isacommand{by}\isamarkupfalse%
\ {\isacharparenleft}simp\ only{\isacharcolon}\ bex{\isacharunderscore}simps{\isacharparenleft}{\isadigit{5}}{\isacharparenright}{\isacharparenright}\isanewline
\ \ \isacommand{obtain}\isamarkupfalse%
\ I\ \isakeyword{where}\ {\isachardoublequoteopen}I\ {\isasymin}\ {\isacharbraceleft}G{\isacharcomma}H{\isacharbraceright}{\isachardoublequoteclose}\ \isakeyword{and}\ {\isachardoublequoteopen}insert\ I\ {\isacharquery}S\ {\isasymin}\ C{\isachardoublequoteclose}\isanewline
\ \ \ \ \isacommand{using}\isamarkupfalse%
\ {\isadigit{1}}\ \isacommand{by}\isamarkupfalse%
\ {\isacharparenleft}rule\ bexE{\isacharparenright}\isanewline
\ \ \isacommand{have}\isamarkupfalse%
\ SC{\isacharcolon}{\isachardoublequoteopen}{\isasymforall}S\ {\isasymin}\ C{\isachardot}\ {\isasymforall}S{\isacharprime}{\isasymsubseteq}S{\isachardot}\ S{\isacharprime}\ {\isasymin}\ C{\isachardoublequoteclose}\isanewline
\ \ \ \ \isacommand{using}\isamarkupfalse%
\ assms{\isacharparenleft}{\isadigit{2}}{\isacharparenright}\ \isacommand{by}\isamarkupfalse%
\ {\isacharparenleft}simp\ only{\isacharcolon}\ subset{\isacharunderscore}closed{\isacharunderscore}def{\isacharparenright}\isanewline
\ \ \isacommand{then}\isamarkupfalse%
\ \isacommand{have}\isamarkupfalse%
\ {\isadigit{2}}{\isacharcolon}{\isachardoublequoteopen}{\isasymforall}S{\isacharprime}\ {\isasymsubseteq}\ {\isacharparenleft}insert\ I\ {\isacharquery}S{\isacharparenright}{\isachardot}\ S{\isacharprime}\ {\isasymin}\ C{\isachardoublequoteclose}\isanewline
\ \ \ \ \isacommand{using}\isamarkupfalse%
\ {\isacartoucheopen}insert\ I\ {\isacharquery}S\ {\isasymin}\ C{\isacartoucheclose}\ \isacommand{by}\isamarkupfalse%
\ {\isacharparenleft}rule\ bspec{\isacharparenright}\isanewline
\ \ \isacommand{have}\isamarkupfalse%
\ {\isachardoublequoteopen}insert\ I\ S{\isadigit{1}}\ {\isasymsubseteq}\ insert\ I\ {\isacharquery}S{\isachardoublequoteclose}\ \isanewline
\ \ \ \ \isacommand{using}\isamarkupfalse%
\ {\isacartoucheopen}S{\isadigit{1}}\ {\isasymsubseteq}\ {\isacharquery}S{\isacartoucheclose}\ \isacommand{by}\isamarkupfalse%
\ {\isacharparenleft}rule\ insert{\isacharunderscore}mono{\isacharparenright}\isanewline
\ \ \isacommand{have}\isamarkupfalse%
\ {\isachardoublequoteopen}insert\ I\ S{\isadigit{1}}\ {\isasymin}\ C{\isachardoublequoteclose}\isanewline
\ \ \ \ \isacommand{using}\isamarkupfalse%
\ {\isadigit{2}}\ {\isacartoucheopen}insert\ I\ S{\isadigit{1}}\ {\isasymsubseteq}\ insert\ I\ {\isacharquery}S{\isacartoucheclose}\ \isacommand{by}\isamarkupfalse%
\ {\isacharparenleft}rule\ sspec{\isacharparenright}\isanewline
\ \ \isacommand{have}\isamarkupfalse%
\ {\isachardoublequoteopen}insert\ I\ S{\isadigit{2}}\ {\isasymsubseteq}\ insert\ I\ {\isacharquery}S{\isachardoublequoteclose}\isanewline
\ \ \ \ \isacommand{using}\isamarkupfalse%
\ {\isacartoucheopen}S{\isadigit{2}}\ {\isasymsubseteq}\ {\isacharquery}S{\isacartoucheclose}\ \isacommand{by}\isamarkupfalse%
\ {\isacharparenleft}rule\ insert{\isacharunderscore}mono{\isacharparenright}\isanewline
\ \ \isacommand{have}\isamarkupfalse%
\ {\isachardoublequoteopen}insert\ I\ S{\isadigit{2}}\ {\isasymin}\ C{\isachardoublequoteclose}\isanewline
\ \ \ \ \isacommand{using}\isamarkupfalse%
\ {\isadigit{2}}\ {\isacartoucheopen}insert\ I\ S{\isadigit{2}}\ {\isasymsubseteq}\ insert\ I\ {\isacharquery}S{\isacartoucheclose}\ \isacommand{by}\isamarkupfalse%
\ {\isacharparenleft}rule\ sspec{\isacharparenright}\isanewline
\ \ \isacommand{have}\isamarkupfalse%
\ {\isachardoublequoteopen}insert\ I\ S{\isadigit{1}}\ {\isasymin}\ C\ {\isasymand}\ insert\ I\ S{\isadigit{2}}\ {\isasymin}\ C{\isachardoublequoteclose}\isanewline
\ \ \ \ \isacommand{using}\isamarkupfalse%
\ {\isacartoucheopen}insert\ I\ S{\isadigit{1}}\ {\isasymin}\ C{\isacartoucheclose}\ {\isacartoucheopen}insert\ I\ S{\isadigit{2}}\ {\isasymin}\ C{\isacartoucheclose}\ \isacommand{by}\isamarkupfalse%
\ {\isacharparenleft}rule\ conjI{\isacharparenright}\isanewline
\ \ \isacommand{thus}\isamarkupfalse%
\ {\isachardoublequoteopen}{\isasymexists}I{\isasymin}{\isacharbraceleft}G{\isacharcomma}H{\isacharbraceright}{\isachardot}\ insert\ I\ S{\isadigit{1}}\ {\isasymin}\ C\ {\isasymand}\ insert\ I\ S{\isadigit{2}}\ {\isasymin}\ C{\isachardoublequoteclose}\isanewline
\ \ \ \ \isacommand{using}\isamarkupfalse%
\ {\isacartoucheopen}I\ {\isasymin}\ {\isacharbraceleft}G{\isacharcomma}H{\isacharbraceright}{\isacartoucheclose}\ \isacommand{by}\isamarkupfalse%
\ {\isacharparenleft}rule\ bexI{\isacharparenright}\isanewline
\isacommand{qed}\isamarkupfalse%
%
\endisatagproof
{\isafoldproof}%
%
\isadelimproof
%
\endisadelimproof
%
\begin{isamarkuptext}%
Finalmente, el lema \isa{ex{\isadigit{3}}{\isacharunderscore}pcp{\isacharunderscore}SinE{\isacharunderscore}DIS{\isacharunderscore}auxFalse} prueba que dada una colección \isa{C} con la 
  propiedad de consistencia proposicional y cerrada bajo subconjuntos, \isa{S\ {\isasymin}\ E} y sea \isa{F} es una 
  fórmula de tipo \isa{{\isasymbeta}} de componentes \isa{{\isasymbeta}\isactrlsub {\isadigit{1}}} y \isa{{\isasymbeta}\isactrlsub {\isadigit{2}}}, si consideramos \isa{S\isactrlsub {\isadigit{1}}} y \isa{S\isactrlsub {\isadigit{2}}} subconjuntos finitos 
  cualesquiera de \isa{S} tales que \isa{F\ {\isasymin}\ S\isactrlsub {\isadigit{1}}}, \isa{F\ {\isasymin}\ S\isactrlsub {\isadigit{2}}}, \isa{{\isacharbraceleft}{\isasymbeta}\isactrlsub {\isadigit{1}}{\isacharbraceright}\ {\isasymunion}\ S\isactrlsub {\isadigit{1}}\ {\isasymnotin}\ C} y \isa{{\isacharbraceleft}{\isasymbeta}\isactrlsub {\isadigit{2}}{\isacharbraceright}\ {\isasymunion}\ S\isactrlsub {\isadigit{2}}\ {\isasymnotin}\ C}, llegamos a 
  una contradicción.%
\end{isamarkuptext}\isamarkuptrue%
\isacommand{lemma}\isamarkupfalse%
\ ex{\isadigit{3}}{\isacharunderscore}pcp{\isacharunderscore}SinE{\isacharunderscore}DIS{\isacharunderscore}auxFalse{\isacharcolon}\isanewline
\ \ \isakeyword{assumes}\ {\isachardoublequoteopen}pcp\ C{\isachardoublequoteclose}\ \isanewline
\ \ \ \ \ \ \ \ \ \ {\isachardoublequoteopen}subset{\isacharunderscore}closed\ C{\isachardoublequoteclose}\isanewline
\ \ \ \ \ \ \ \ \ \ {\isachardoublequoteopen}S\ {\isasymin}\ {\isacharparenleft}extF\ C{\isacharparenright}{\isachardoublequoteclose}\isanewline
\ \ \ \ \ \ \ \ \ \ {\isachardoublequoteopen}Dis\ F\ G\ H{\isachardoublequoteclose}\isanewline
\ \ \ \ \ \ \ \ \ \ {\isachardoublequoteopen}F\ {\isasymin}\ S{\isachardoublequoteclose}\isanewline
\ \ \ \ \ \ \ \ \ \ {\isachardoublequoteopen}S{\isadigit{1}}\ {\isasymsubseteq}\ S{\isachardoublequoteclose}\ \isanewline
\ \ \ \ \ \ \ \ \ \ {\isachardoublequoteopen}finite\ S{\isadigit{1}}{\isachardoublequoteclose}\ \isanewline
\ \ \ \ \ \ \ \ \ \ {\isachardoublequoteopen}insert\ G\ S{\isadigit{1}}\ {\isasymnotin}\ C{\isachardoublequoteclose}\ \isanewline
\ \ \ \ \ \ \ \ \ \ {\isachardoublequoteopen}S{\isadigit{2}}\ {\isasymsubseteq}\ S{\isachardoublequoteclose}\ \isanewline
\ \ \ \ \ \ \ \ \ \ {\isachardoublequoteopen}finite\ S{\isadigit{2}}{\isachardoublequoteclose}\ \isanewline
\ \ \ \ \ \ \ \ \ \ {\isachardoublequoteopen}insert\ H\ S{\isadigit{2}}\ {\isasymnotin}\ C{\isachardoublequoteclose}\isanewline
\ \ \ \ \ \ \ \ \isakeyword{shows}\ {\isachardoublequoteopen}False{\isachardoublequoteclose}\isanewline
%
\isadelimproof
%
\endisadelimproof
%
\isatagproof
\isacommand{proof}\isamarkupfalse%
\ {\isacharminus}\isanewline
\ \ \isacommand{let}\isamarkupfalse%
\ {\isacharquery}S{\isadigit{1}}{\isacharequal}{\isachardoublequoteopen}insert\ F\ S{\isadigit{1}}{\isachardoublequoteclose}\isanewline
\ \ \isacommand{let}\isamarkupfalse%
\ {\isacharquery}S{\isadigit{2}}{\isacharequal}{\isachardoublequoteopen}insert\ F\ S{\isadigit{2}}{\isachardoublequoteclose}\isanewline
\ \ \isacommand{have}\isamarkupfalse%
\ SC{\isacharcolon}{\isachardoublequoteopen}{\isasymforall}S\ {\isasymin}\ C{\isachardot}\ {\isasymforall}S{\isacharprime}{\isasymsubseteq}S{\isachardot}\ S{\isacharprime}\ {\isasymin}\ C{\isachardoublequoteclose}\isanewline
\ \ \ \ \isacommand{using}\isamarkupfalse%
\ assms{\isacharparenleft}{\isadigit{2}}{\isacharparenright}\ \isacommand{by}\isamarkupfalse%
\ {\isacharparenleft}simp\ only{\isacharcolon}\ subset{\isacharunderscore}closed{\isacharunderscore}def{\isacharparenright}\isanewline
\ \ \isacommand{have}\isamarkupfalse%
\ {\isadigit{1}}{\isacharcolon}{\isachardoublequoteopen}{\isacharquery}S{\isadigit{1}}\ {\isasymsubseteq}\ S{\isachardoublequoteclose}\isanewline
\ \ \ \ \isacommand{using}\isamarkupfalse%
\ {\isacartoucheopen}F\ {\isasymin}\ S{\isacartoucheclose}\ {\isacartoucheopen}S{\isadigit{1}}\ {\isasymsubseteq}\ S{\isacartoucheclose}\ \isacommand{by}\isamarkupfalse%
\ {\isacharparenleft}simp\ only{\isacharcolon}\ insert{\isacharunderscore}subset{\isacharparenright}\ \isanewline
\ \ \isacommand{have}\isamarkupfalse%
\ {\isadigit{2}}{\isacharcolon}{\isachardoublequoteopen}finite\ {\isacharquery}S{\isadigit{1}}{\isachardoublequoteclose}\isanewline
\ \ \ \ \isacommand{using}\isamarkupfalse%
\ {\isacartoucheopen}finite\ S{\isadigit{1}}{\isacartoucheclose}\ \isacommand{by}\isamarkupfalse%
\ {\isacharparenleft}simp\ only{\isacharcolon}\ finite{\isacharunderscore}insert{\isacharparenright}\ \isanewline
\ \ \isacommand{have}\isamarkupfalse%
\ {\isadigit{3}}{\isacharcolon}{\isachardoublequoteopen}F\ {\isasymin}\ {\isacharquery}S{\isadigit{1}}{\isachardoublequoteclose}\isanewline
\ \ \ \ \isacommand{by}\isamarkupfalse%
\ {\isacharparenleft}simp\ only{\isacharcolon}\ insertI{\isadigit{1}}{\isacharparenright}\ \isanewline
\ \ \isacommand{have}\isamarkupfalse%
\ {\isadigit{4}}{\isacharcolon}{\isachardoublequoteopen}insert\ G\ {\isacharquery}S{\isadigit{1}}\ {\isasymnotin}\ C{\isachardoublequoteclose}\ \isanewline
\ \ \isacommand{proof}\isamarkupfalse%
\ {\isacharparenleft}rule\ ccontr{\isacharparenright}\isanewline
\ \ \ \ \isacommand{assume}\isamarkupfalse%
\ {\isachardoublequoteopen}{\isasymnot}{\isacharparenleft}insert\ G\ {\isacharquery}S{\isadigit{1}}\ {\isasymnotin}\ C{\isacharparenright}{\isachardoublequoteclose}\isanewline
\ \ \ \ \isacommand{then}\isamarkupfalse%
\ \isacommand{have}\isamarkupfalse%
\ {\isachardoublequoteopen}insert\ G\ {\isacharquery}S{\isadigit{1}}\ {\isasymin}\ C{\isachardoublequoteclose}\isanewline
\ \ \ \ \ \ \isacommand{by}\isamarkupfalse%
\ {\isacharparenleft}rule\ notnotD{\isacharparenright}\isanewline
\ \ \ \ \isacommand{have}\isamarkupfalse%
\ SC{\isadigit{1}}{\isacharcolon}{\isachardoublequoteopen}{\isasymforall}S{\isacharprime}\ {\isasymsubseteq}\ {\isacharparenleft}insert\ G\ {\isacharquery}S{\isadigit{1}}{\isacharparenright}{\isachardot}\ S{\isacharprime}\ {\isasymin}\ C{\isachardoublequoteclose}\isanewline
\ \ \ \ \ \ \isacommand{using}\isamarkupfalse%
\ SC\ {\isacartoucheopen}insert\ G\ {\isacharquery}S{\isadigit{1}}\ {\isasymin}\ C{\isacartoucheclose}\ \isacommand{by}\isamarkupfalse%
\ {\isacharparenleft}rule\ bspec{\isacharparenright}\isanewline
\ \ \ \ \isacommand{have}\isamarkupfalse%
\ {\isachardoublequoteopen}insert\ G\ S{\isadigit{1}}\ {\isasymsubseteq}\ insert\ F\ {\isacharparenleft}insert\ G\ S{\isadigit{1}}{\isacharparenright}{\isachardoublequoteclose}\isanewline
\ \ \ \ \ \ \isacommand{by}\isamarkupfalse%
\ {\isacharparenleft}rule\ subset{\isacharunderscore}insertI{\isacharparenright}\isanewline
\ \ \ \ \isacommand{then}\isamarkupfalse%
\ \isacommand{have}\isamarkupfalse%
\ {\isachardoublequoteopen}insert\ G\ S{\isadigit{1}}\ {\isasymsubseteq}\ insert\ G\ {\isacharquery}S{\isadigit{1}}{\isachardoublequoteclose}\isanewline
\ \ \ \ \ \ \isacommand{by}\isamarkupfalse%
\ {\isacharparenleft}simp\ only{\isacharcolon}\ insert{\isacharunderscore}commute{\isacharparenright}\isanewline
\ \ \ \ \isacommand{have}\isamarkupfalse%
\ {\isachardoublequoteopen}insert\ G\ S{\isadigit{1}}\ {\isasymin}\ C{\isachardoublequoteclose}\isanewline
\ \ \ \ \ \ \isacommand{using}\isamarkupfalse%
\ SC{\isadigit{1}}\ {\isacartoucheopen}insert\ G\ S{\isadigit{1}}\ {\isasymsubseteq}\ insert\ G\ {\isacharquery}S{\isadigit{1}}{\isacartoucheclose}\ \isacommand{by}\isamarkupfalse%
\ {\isacharparenleft}rule\ sspec{\isacharparenright}\isanewline
\ \ \ \ \isacommand{show}\isamarkupfalse%
\ {\isachardoublequoteopen}False{\isachardoublequoteclose}\isanewline
\ \ \ \ \ \ \isacommand{using}\isamarkupfalse%
\ assms{\isacharparenleft}{\isadigit{8}}{\isacharparenright}\ {\isacartoucheopen}insert\ G\ S{\isadigit{1}}\ {\isasymin}\ C{\isacartoucheclose}\ \isacommand{by}\isamarkupfalse%
\ {\isacharparenleft}rule\ notE{\isacharparenright}\isanewline
\ \ \isacommand{qed}\isamarkupfalse%
\isanewline
\ \ \isacommand{have}\isamarkupfalse%
\ {\isadigit{5}}{\isacharcolon}{\isachardoublequoteopen}{\isacharquery}S{\isadigit{2}}\ {\isasymsubseteq}\ S{\isachardoublequoteclose}\isanewline
\ \ \ \ \isacommand{using}\isamarkupfalse%
\ {\isacartoucheopen}F\ {\isasymin}\ S{\isacartoucheclose}\ {\isacartoucheopen}S{\isadigit{2}}\ {\isasymsubseteq}\ S{\isacartoucheclose}\ \isacommand{by}\isamarkupfalse%
\ {\isacharparenleft}simp\ only{\isacharcolon}\ insert{\isacharunderscore}subset{\isacharparenright}\isanewline
\ \ \isacommand{have}\isamarkupfalse%
\ {\isadigit{6}}{\isacharcolon}{\isachardoublequoteopen}finite\ {\isacharquery}S{\isadigit{2}}{\isachardoublequoteclose}\isanewline
\ \ \ \ \isacommand{using}\isamarkupfalse%
\ {\isacartoucheopen}finite\ S{\isadigit{2}}{\isacartoucheclose}\ \isacommand{by}\isamarkupfalse%
\ {\isacharparenleft}simp\ only{\isacharcolon}\ finite{\isacharunderscore}insert{\isacharparenright}\isanewline
\ \ \isacommand{have}\isamarkupfalse%
\ {\isadigit{7}}{\isacharcolon}{\isachardoublequoteopen}F\ {\isasymin}\ {\isacharquery}S{\isadigit{2}}{\isachardoublequoteclose}\isanewline
\ \ \ \ \isacommand{by}\isamarkupfalse%
\ {\isacharparenleft}simp\ only{\isacharcolon}\ insertI{\isadigit{1}}{\isacharparenright}\isanewline
\ \ \isacommand{have}\isamarkupfalse%
\ {\isadigit{8}}{\isacharcolon}{\isachardoublequoteopen}insert\ H\ {\isacharquery}S{\isadigit{2}}\ {\isasymnotin}\ C{\isachardoublequoteclose}\ \isanewline
\ \ \isacommand{proof}\isamarkupfalse%
\ {\isacharparenleft}rule\ ccontr{\isacharparenright}\isanewline
\ \ \ \ \isacommand{assume}\isamarkupfalse%
\ {\isachardoublequoteopen}{\isasymnot}{\isacharparenleft}insert\ H\ {\isacharquery}S{\isadigit{2}}\ {\isasymnotin}\ C{\isacharparenright}{\isachardoublequoteclose}\isanewline
\ \ \ \ \isacommand{then}\isamarkupfalse%
\ \isacommand{have}\isamarkupfalse%
\ {\isachardoublequoteopen}insert\ H\ {\isacharquery}S{\isadigit{2}}\ {\isasymin}\ C{\isachardoublequoteclose}\isanewline
\ \ \ \ \ \ \isacommand{by}\isamarkupfalse%
\ {\isacharparenleft}rule\ notnotD{\isacharparenright}\isanewline
\ \ \ \ \isacommand{have}\isamarkupfalse%
\ SC{\isadigit{2}}{\isacharcolon}{\isachardoublequoteopen}{\isasymforall}S{\isacharprime}\ {\isasymsubseteq}\ {\isacharparenleft}insert\ H\ {\isacharquery}S{\isadigit{2}}{\isacharparenright}{\isachardot}\ S{\isacharprime}\ {\isasymin}\ C{\isachardoublequoteclose}\isanewline
\ \ \ \ \ \ \isacommand{using}\isamarkupfalse%
\ SC\ {\isacartoucheopen}insert\ H\ {\isacharquery}S{\isadigit{2}}\ {\isasymin}\ C{\isacartoucheclose}\ \isacommand{by}\isamarkupfalse%
\ {\isacharparenleft}rule\ bspec{\isacharparenright}\isanewline
\ \ \ \ \isacommand{have}\isamarkupfalse%
\ {\isachardoublequoteopen}insert\ H\ S{\isadigit{2}}\ {\isasymsubseteq}\ insert\ F\ {\isacharparenleft}insert\ H\ S{\isadigit{2}}{\isacharparenright}{\isachardoublequoteclose}\isanewline
\ \ \ \ \ \ \isacommand{by}\isamarkupfalse%
\ {\isacharparenleft}rule\ subset{\isacharunderscore}insertI{\isacharparenright}\isanewline
\ \ \ \ \isacommand{then}\isamarkupfalse%
\ \isacommand{have}\isamarkupfalse%
\ {\isachardoublequoteopen}insert\ H\ S{\isadigit{2}}\ {\isasymsubseteq}\ insert\ H\ {\isacharquery}S{\isadigit{2}}{\isachardoublequoteclose}\isanewline
\ \ \ \ \ \ \isacommand{by}\isamarkupfalse%
\ {\isacharparenleft}simp\ only{\isacharcolon}\ insert{\isacharunderscore}commute{\isacharparenright}\isanewline
\ \ \ \ \isacommand{have}\isamarkupfalse%
\ {\isachardoublequoteopen}insert\ H\ S{\isadigit{2}}\ {\isasymin}\ C{\isachardoublequoteclose}\isanewline
\ \ \ \ \ \ \isacommand{using}\isamarkupfalse%
\ SC{\isadigit{2}}\ {\isacartoucheopen}insert\ H\ S{\isadigit{2}}\ {\isasymsubseteq}\ insert\ H\ {\isacharquery}S{\isadigit{2}}{\isacartoucheclose}\ \isacommand{by}\isamarkupfalse%
\ {\isacharparenleft}rule\ sspec{\isacharparenright}\isanewline
\ \ \ \ \isacommand{show}\isamarkupfalse%
\ {\isachardoublequoteopen}False{\isachardoublequoteclose}\isanewline
\ \ \ \ \ \ \isacommand{using}\isamarkupfalse%
\ assms{\isacharparenleft}{\isadigit{1}}{\isadigit{1}}{\isacharparenright}\ {\isacartoucheopen}insert\ H\ S{\isadigit{2}}\ {\isasymin}\ C{\isacartoucheclose}\ \isacommand{by}\isamarkupfalse%
\ {\isacharparenleft}rule\ notE{\isacharparenright}\isanewline
\ \ \isacommand{qed}\isamarkupfalse%
\isanewline
\ \ \isacommand{have}\isamarkupfalse%
\ Ex{\isacharcolon}{\isachardoublequoteopen}{\isasymexists}I\ {\isasymin}\ {\isacharbraceleft}G{\isacharcomma}H{\isacharbraceright}{\isachardot}\ insert\ I\ {\isacharquery}S{\isadigit{1}}\ {\isasymin}\ C\ {\isasymand}\ insert\ I\ {\isacharquery}S{\isadigit{2}}\ {\isasymin}\ C{\isachardoublequoteclose}\isanewline
\ \ \ \ \isacommand{using}\isamarkupfalse%
\ assms{\isacharparenleft}{\isadigit{1}}{\isacharparenright}\ assms{\isacharparenleft}{\isadigit{2}}{\isacharparenright}\ assms{\isacharparenleft}{\isadigit{3}}{\isacharparenright}\ assms{\isacharparenleft}{\isadigit{4}}{\isacharparenright}\ {\isadigit{1}}\ {\isadigit{2}}\ {\isadigit{3}}\ {\isadigit{5}}\ {\isadigit{6}}\ {\isadigit{7}}\ \isacommand{by}\isamarkupfalse%
\ {\isacharparenleft}rule\ ex{\isadigit{3}}{\isacharunderscore}pcp{\isacharunderscore}SinE{\isacharunderscore}DIS{\isacharunderscore}auxEx{\isacharparenright}\isanewline
\ \ \isacommand{have}\isamarkupfalse%
\ {\isachardoublequoteopen}{\isasymforall}I\ {\isasymin}\ {\isacharbraceleft}G{\isacharcomma}H{\isacharbraceright}{\isachardot}\ insert\ I\ {\isacharquery}S{\isadigit{1}}\ {\isasymnotin}\ C\ {\isasymor}\ insert\ I\ {\isacharquery}S{\isadigit{2}}\ {\isasymnotin}\ C{\isachardoublequoteclose}\isanewline
\ \ \ \ \isacommand{using}\isamarkupfalse%
\ {\isadigit{4}}\ {\isadigit{8}}\ \isacommand{by}\isamarkupfalse%
\ simp\isanewline
\ \ \isacommand{then}\isamarkupfalse%
\ \isacommand{have}\isamarkupfalse%
\ {\isachardoublequoteopen}{\isasymforall}I\ {\isasymin}\ {\isacharbraceleft}G{\isacharcomma}H{\isacharbraceright}{\isachardot}\ {\isasymnot}{\isacharparenleft}insert\ I\ {\isacharquery}S{\isadigit{1}}\ {\isasymin}\ C\ {\isasymand}\ insert\ I\ {\isacharquery}S{\isadigit{2}}\ {\isasymin}\ C{\isacharparenright}{\isachardoublequoteclose}\isanewline
\ \ \ \ \isacommand{by}\isamarkupfalse%
\ {\isacharparenleft}simp\ only{\isacharcolon}\ de{\isacharunderscore}Morgan{\isacharunderscore}conj{\isacharparenright}\isanewline
\ \ \isacommand{then}\isamarkupfalse%
\ \isacommand{have}\isamarkupfalse%
\ {\isachardoublequoteopen}{\isasymnot}{\isacharparenleft}{\isasymexists}I\ {\isasymin}\ {\isacharbraceleft}G{\isacharcomma}H{\isacharbraceright}{\isachardot}\ insert\ I\ {\isacharquery}S{\isadigit{1}}\ {\isasymin}\ C\ {\isasymand}\ insert\ I\ {\isacharquery}S{\isadigit{2}}\ {\isasymin}\ C{\isacharparenright}{\isachardoublequoteclose}\isanewline
\ \ \ \ \isacommand{by}\isamarkupfalse%
\ {\isacharparenleft}simp\ only{\isacharcolon}\ bex{\isacharunderscore}simps{\isacharparenleft}{\isadigit{8}}{\isacharparenright}{\isacharparenright}\ \isanewline
\ \ \isacommand{thus}\isamarkupfalse%
\ {\isachardoublequoteopen}False{\isachardoublequoteclose}\isanewline
\ \ \ \ \isacommand{using}\isamarkupfalse%
\ Ex\ \isacommand{by}\isamarkupfalse%
\ {\isacharparenleft}rule\ notE{\isacharparenright}\isanewline
\isacommand{qed}\isamarkupfalse%
%
\endisatagproof
{\isafoldproof}%
%
\isadelimproof
%
\endisadelimproof
%
\begin{isamarkuptext}%
Una vez introducidos los lemas anteriores, podemos probar el lema \isa{ex{\isadigit{3}}{\isacharunderscore}pcp{\isacharunderscore}SinE{\isacharunderscore}DIS} que
  demuestra que si \isa{C} es una colección con la propiedad de consistencia proposicional y cerrada 
  bajo subconjuntos, \isa{S\ {\isasymin}\ E} y sea \isa{F} una fórmula de tipo \isa{{\isasymbeta}} con componentes \isa{{\isasymbeta}\isactrlsub {\isadigit{1}}} y \isa{{\isasymbeta}\isactrlsub {\isadigit{2}}}, se 
  verifica que o bien \isa{{\isacharbraceleft}{\isasymbeta}\isactrlsub {\isadigit{1}}{\isacharbraceright}\ {\isasymunion}\ S\ {\isasymin}\ C{\isacharprime}} o bien \isa{{\isacharbraceleft}{\isasymbeta}\isactrlsub {\isadigit{2}}{\isacharbraceright}\ {\isasymunion}\ S\ {\isasymin}\ C{\isacharprime}}. Además, para dicha prueba 
  necesitaremos los siguientes lemas auxiliares en Isabelle.%
\end{isamarkuptext}\isamarkuptrue%
\isacommand{lemma}\isamarkupfalse%
\ sall{\isacharunderscore}simps{\isacharunderscore}not{\isacharunderscore}all{\isacharcolon}\isanewline
\ \ \isakeyword{assumes}\ {\isachardoublequoteopen}{\isasymnot}{\isacharparenleft}{\isasymforall}x\ {\isasymsubseteq}\ A{\isachardot}\ P\ x{\isacharparenright}{\isachardoublequoteclose}\isanewline
\ \ \isakeyword{shows}\ {\isachardoublequoteopen}{\isasymexists}x\ {\isasymsubseteq}\ A{\isachardot}\ {\isacharparenleft}{\isasymnot}\ P\ x{\isacharparenright}{\isachardoublequoteclose}\isanewline
%
\isadelimproof
\ \ %
\endisadelimproof
%
\isatagproof
\isacommand{using}\isamarkupfalse%
\ assms\ \isacommand{by}\isamarkupfalse%
\ blast%
\endisatagproof
{\isafoldproof}%
%
\isadelimproof
\isanewline
%
\endisadelimproof
\isanewline
\isacommand{lemma}\isamarkupfalse%
\ subexE{\isacharcolon}\ {\isachardoublequoteopen}{\isasymexists}x{\isasymsubseteq}A{\isachardot}\ P\ x\ {\isasymLongrightarrow}\ {\isacharparenleft}{\isasymAnd}x{\isachardot}\ x{\isasymsubseteq}A\ {\isasymLongrightarrow}\ P\ x\ {\isasymLongrightarrow}\ Q{\isacharparenright}\ {\isasymLongrightarrow}\ Q{\isachardoublequoteclose}\isanewline
%
\isadelimproof
\ \ %
\endisadelimproof
%
\isatagproof
\isacommand{by}\isamarkupfalse%
\ blast%
\endisatagproof
{\isafoldproof}%
%
\isadelimproof
%
\endisadelimproof
%
\begin{isamarkuptext}%
De este modo, procedamos con la demostración detallada de \isa{ex{\isadigit{3}}{\isacharunderscore}pcp{\isacharunderscore}SinE{\isacharunderscore}DIS}.%
\end{isamarkuptext}\isamarkuptrue%
\isacommand{lemma}\isamarkupfalse%
\ ex{\isadigit{3}}{\isacharunderscore}pcp{\isacharunderscore}SinE{\isacharunderscore}DIS{\isacharcolon}\isanewline
\ \ \isakeyword{assumes}\ {\isachardoublequoteopen}pcp\ C{\isachardoublequoteclose}\isanewline
\ \ \ \ \ \ \ \ \ \ {\isachardoublequoteopen}subset{\isacharunderscore}closed\ C{\isachardoublequoteclose}\isanewline
\ \ \ \ \ \ \ \ \ \ {\isachardoublequoteopen}S\ {\isasymin}\ {\isacharparenleft}extF\ C{\isacharparenright}{\isachardoublequoteclose}\isanewline
\ \ \ \ \ \ \ \ \ \ {\isachardoublequoteopen}Dis\ F\ G\ H{\isachardoublequoteclose}\isanewline
\ \ \ \ \ \ \ \ \ \ {\isachardoublequoteopen}F\ {\isasymin}\ S{\isachardoublequoteclose}\isanewline
\ \ \isakeyword{shows}\ {\isachardoublequoteopen}{\isacharbraceleft}G{\isacharbraceright}\ {\isasymunion}\ S\ {\isasymin}\ {\isacharparenleft}extensionFin\ C{\isacharparenright}\ {\isasymor}\ {\isacharbraceleft}H{\isacharbraceright}\ {\isasymunion}\ S\ {\isasymin}\ {\isacharparenleft}extensionFin\ C{\isacharparenright}{\isachardoublequoteclose}\isanewline
%
\isadelimproof
%
\endisadelimproof
%
\isatagproof
\isacommand{proof}\isamarkupfalse%
\ {\isacharminus}\isanewline
\ \ \isacommand{have}\isamarkupfalse%
\ {\isachardoublequoteopen}{\isacharparenleft}extF\ C{\isacharparenright}\ {\isasymsubseteq}\ {\isacharparenleft}extensionFin\ C{\isacharparenright}{\isachardoublequoteclose}\ \isanewline
\ \ \ \ \isacommand{unfolding}\isamarkupfalse%
\ extensionFin\ \isacommand{by}\isamarkupfalse%
\ {\isacharparenleft}rule\ Un{\isacharunderscore}upper{\isadigit{2}}{\isacharparenright}\ \isanewline
\ \ \isacommand{have}\isamarkupfalse%
\ PCP{\isacharcolon}{\isachardoublequoteopen}{\isasymforall}S\ {\isasymin}\ C{\isachardot}\isanewline
\ \ \ \ \ \ \ \ \ \ \ \ {\isasymbottom}\ {\isasymnotin}\ S\isanewline
\ \ \ \ \ \ \ \ \ \ \ \ {\isasymand}\ {\isacharparenleft}{\isasymforall}k{\isachardot}\ Atom\ k\ {\isasymin}\ S\ {\isasymlongrightarrow}\ \isactrlbold {\isasymnot}\ {\isacharparenleft}Atom\ k{\isacharparenright}\ {\isasymin}\ S\ {\isasymlongrightarrow}\ False{\isacharparenright}\isanewline
\ \ \ \ \ \ \ \ \ \ \ \ {\isasymand}\ {\isacharparenleft}{\isasymforall}F\ G\ H{\isachardot}\ Con\ F\ G\ H\ {\isasymlongrightarrow}\ F\ {\isasymin}\ S\ {\isasymlongrightarrow}\ {\isacharbraceleft}G{\isacharcomma}H{\isacharbraceright}\ {\isasymunion}\ S\ {\isasymin}\ C{\isacharparenright}\isanewline
\ \ \ \ \ \ \ \ \ \ \ \ {\isasymand}\ {\isacharparenleft}{\isasymforall}F\ G\ H{\isachardot}\ Dis\ F\ G\ H\ {\isasymlongrightarrow}\ F\ {\isasymin}\ S\ {\isasymlongrightarrow}\ {\isacharbraceleft}G{\isacharbraceright}\ {\isasymunion}\ S\ {\isasymin}\ C\ {\isasymor}\ {\isacharbraceleft}H{\isacharbraceright}\ {\isasymunion}\ S\ {\isasymin}\ C{\isacharparenright}{\isachardoublequoteclose}\isanewline
\ \ \ \ \isacommand{using}\isamarkupfalse%
\ assms{\isacharparenleft}{\isadigit{1}}{\isacharparenright}\ \isacommand{by}\isamarkupfalse%
\ {\isacharparenleft}rule\ pcp{\isacharunderscore}alt{\isadigit{1}}{\isacharparenright}\isanewline
\ \ \isacommand{have}\isamarkupfalse%
\ E{\isacharcolon}{\isachardoublequoteopen}{\isasymforall}S{\isacharprime}\ {\isasymsubseteq}\ S{\isachardot}\ finite\ S{\isacharprime}\ {\isasymlongrightarrow}\ S{\isacharprime}\ {\isasymin}\ C{\isachardoublequoteclose}\isanewline
\ \ \ \ \isacommand{using}\isamarkupfalse%
\ assms{\isacharparenleft}{\isadigit{3}}{\isacharparenright}\ \isacommand{unfolding}\isamarkupfalse%
\ extF\ \isacommand{by}\isamarkupfalse%
\ {\isacharparenleft}rule\ CollectD{\isacharparenright}\isanewline
\ \ \isacommand{then}\isamarkupfalse%
\ \isacommand{have}\isamarkupfalse%
\ E{\isacharprime}{\isacharcolon}{\isachardoublequoteopen}{\isasymforall}S{\isacharprime}{\isachardot}\ S{\isacharprime}\ {\isasymsubseteq}\ S\ {\isasymlongrightarrow}\ finite\ S{\isacharprime}\ {\isasymlongrightarrow}\ S{\isacharprime}\ {\isasymin}\ C{\isachardoublequoteclose}\isanewline
\ \ \ \ \isacommand{by}\isamarkupfalse%
\ blast\isanewline
\ \ \isacommand{have}\isamarkupfalse%
\ SC{\isacharcolon}{\isachardoublequoteopen}{\isasymforall}S\ {\isasymin}\ C{\isachardot}\ {\isasymforall}S{\isacharprime}{\isasymsubseteq}S{\isachardot}\ S{\isacharprime}\ {\isasymin}\ C{\isachardoublequoteclose}\isanewline
\ \ \ \ \isacommand{using}\isamarkupfalse%
\ assms{\isacharparenleft}{\isadigit{2}}{\isacharparenright}\ \isacommand{by}\isamarkupfalse%
\ {\isacharparenleft}simp\ only{\isacharcolon}\ subset{\isacharunderscore}closed{\isacharunderscore}def{\isacharparenright}\isanewline
\ \ \isacommand{have}\isamarkupfalse%
\ {\isachardoublequoteopen}insert\ G\ S\ {\isasymin}\ {\isacharparenleft}extF\ C{\isacharparenright}\ {\isasymor}\ insert\ H\ S\ {\isasymin}\ {\isacharparenleft}extF\ C{\isacharparenright}{\isachardoublequoteclose}\ \isanewline
\ \ \isacommand{proof}\isamarkupfalse%
\ {\isacharparenleft}rule\ ccontr{\isacharparenright}\isanewline
\ \ \ \ \isacommand{assume}\isamarkupfalse%
\ {\isachardoublequoteopen}{\isasymnot}{\isacharparenleft}insert\ G\ S\ {\isasymin}\ {\isacharparenleft}extF\ C{\isacharparenright}\ {\isasymor}\ insert\ H\ S\ {\isasymin}\ {\isacharparenleft}extF\ C{\isacharparenright}{\isacharparenright}{\isachardoublequoteclose}\ \ \isanewline
\ \ \ \ \isacommand{then}\isamarkupfalse%
\ \isacommand{have}\isamarkupfalse%
\ Conj{\isacharcolon}{\isachardoublequoteopen}{\isasymnot}{\isacharparenleft}insert\ G\ S\ {\isasymin}\ {\isacharparenleft}extF\ C{\isacharparenright}{\isacharparenright}\ {\isasymand}\ {\isasymnot}{\isacharparenleft}insert\ H\ S\ {\isasymin}\ {\isacharparenleft}extF\ C{\isacharparenright}{\isacharparenright}{\isachardoublequoteclose}\ \isanewline
\ \ \ \ \ \ \isacommand{by}\isamarkupfalse%
\ {\isacharparenleft}simp\ only{\isacharcolon}\ simp{\isacharunderscore}thms{\isacharparenleft}{\isadigit{8}}{\isacharcomma}{\isadigit{2}}{\isadigit{5}}{\isacharparenright}\ de{\isacharunderscore}Morgan{\isacharunderscore}disj{\isacharparenright}\isanewline
\ \ \ \ \isacommand{then}\isamarkupfalse%
\ \isacommand{have}\isamarkupfalse%
\ {\isachardoublequoteopen}{\isasymnot}{\isacharparenleft}insert\ G\ S\ {\isasymin}\ {\isacharparenleft}extF\ C{\isacharparenright}{\isacharparenright}{\isachardoublequoteclose}\isanewline
\ \ \ \ \ \ \isacommand{by}\isamarkupfalse%
\ {\isacharparenleft}rule\ conjunct{\isadigit{1}}{\isacharparenright}\isanewline
\ \ \ \ \isacommand{then}\isamarkupfalse%
\ \isacommand{have}\isamarkupfalse%
\ {\isachardoublequoteopen}{\isasymnot}{\isacharparenleft}{\isasymforall}S{\isacharprime}\ {\isasymsubseteq}\ {\isacharparenleft}insert\ G\ S{\isacharparenright}{\isachardot}\ finite\ S{\isacharprime}\ {\isasymlongrightarrow}\ S{\isacharprime}\ {\isasymin}\ C{\isacharparenright}{\isachardoublequoteclose}\isanewline
\ \ \ \ \ \ \isacommand{unfolding}\isamarkupfalse%
\ extF\ \isacommand{by}\isamarkupfalse%
\ {\isacharparenleft}simp\ add{\isacharcolon}\ mem{\isacharunderscore}Collect{\isacharunderscore}eq{\isacharparenright}\isanewline
\ \ \ \ \isacommand{then}\isamarkupfalse%
\ \isacommand{have}\isamarkupfalse%
\ Ex{\isadigit{1}}{\isacharcolon}{\isachardoublequoteopen}{\isasymexists}S{\isacharprime}{\isasymsubseteq}\ {\isacharparenleft}insert\ G\ S{\isacharparenright}{\isachardot}\ {\isasymnot}{\isacharparenleft}finite\ S{\isacharprime}\ {\isasymlongrightarrow}\ S{\isacharprime}\ {\isasymin}\ C{\isacharparenright}{\isachardoublequoteclose}\isanewline
\ \ \ \ \ \ \isacommand{by}\isamarkupfalse%
\ {\isacharparenleft}rule\ sall{\isacharunderscore}simps{\isacharunderscore}not{\isacharunderscore}all{\isacharparenright}\isanewline
\ \ \ \ \isacommand{obtain}\isamarkupfalse%
\ S{\isadigit{1}}\ \isakeyword{where}\ {\isachardoublequoteopen}S{\isadigit{1}}\ {\isasymsubseteq}\ insert\ G\ S{\isachardoublequoteclose}\ \isakeyword{and}\ {\isachardoublequoteopen}{\isasymnot}{\isacharparenleft}finite\ S{\isadigit{1}}\ {\isasymlongrightarrow}\ S{\isadigit{1}}\ {\isasymin}\ C{\isacharparenright}{\isachardoublequoteclose}\isanewline
\ \ \ \ \ \ \isacommand{using}\isamarkupfalse%
\ Ex{\isadigit{1}}\ \isacommand{by}\isamarkupfalse%
\ {\isacharparenleft}rule\ subexE{\isacharparenright}\isanewline
\ \ \ \ \isacommand{have}\isamarkupfalse%
\ {\isachardoublequoteopen}finite\ S{\isadigit{1}}\ {\isasymand}\ S{\isadigit{1}}\ {\isasymnotin}\ C{\isachardoublequoteclose}\ \isanewline
\ \ \ \ \ \ \isacommand{using}\isamarkupfalse%
\ {\isacartoucheopen}{\isasymnot}{\isacharparenleft}finite\ S{\isadigit{1}}\ {\isasymlongrightarrow}\ S{\isadigit{1}}\ {\isasymin}\ C{\isacharparenright}{\isacartoucheclose}\ \isacommand{by}\isamarkupfalse%
\ {\isacharparenleft}simp\ only{\isacharcolon}\ simp{\isacharunderscore}thms{\isacharparenleft}{\isadigit{8}}{\isacharparenright}\ not{\isacharunderscore}imp{\isacharparenright}\isanewline
\ \ \ \ \isacommand{then}\isamarkupfalse%
\ \isacommand{have}\isamarkupfalse%
\ {\isachardoublequoteopen}finite\ S{\isadigit{1}}{\isachardoublequoteclose}\isanewline
\ \ \ \ \ \ \isacommand{by}\isamarkupfalse%
\ {\isacharparenleft}rule\ conjunct{\isadigit{1}}{\isacharparenright}\isanewline
\ \ \ \ \isacommand{have}\isamarkupfalse%
\ {\isachardoublequoteopen}S{\isadigit{1}}\ {\isasymnotin}\ C{\isachardoublequoteclose}\isanewline
\ \ \ \ \ \ \isacommand{using}\isamarkupfalse%
\ {\isacartoucheopen}finite\ S{\isadigit{1}}\ {\isasymand}\ S{\isadigit{1}}\ {\isasymnotin}\ C{\isacartoucheclose}\ \isacommand{by}\isamarkupfalse%
\ {\isacharparenleft}rule\ conjunct{\isadigit{2}}{\isacharparenright}\isanewline
\ \ \ \ \isacommand{then}\isamarkupfalse%
\ \isacommand{have}\isamarkupfalse%
\ {\isachardoublequoteopen}insert\ G\ S{\isadigit{1}}\ {\isasymnotin}\ C{\isachardoublequoteclose}\isanewline
\ \ \ \ \isacommand{proof}\isamarkupfalse%
\ {\isacharminus}\ \isanewline
\ \ \ \ \ \ \isacommand{have}\isamarkupfalse%
\ {\isachardoublequoteopen}S{\isadigit{1}}\ {\isasymsubseteq}\ S\ {\isasymlongrightarrow}\ finite\ S{\isadigit{1}}\ {\isasymlongrightarrow}\ S{\isadigit{1}}\ {\isasymin}\ C{\isachardoublequoteclose}\isanewline
\ \ \ \ \ \ \ \ \isacommand{using}\isamarkupfalse%
\ E{\isacharprime}\ \isacommand{by}\isamarkupfalse%
\ {\isacharparenleft}rule\ allE{\isacharparenright}\isanewline
\ \ \ \ \ \ \isacommand{then}\isamarkupfalse%
\ \isacommand{have}\isamarkupfalse%
\ {\isachardoublequoteopen}{\isasymnot}\ S{\isadigit{1}}\ {\isasymsubseteq}\ S{\isachardoublequoteclose}\isanewline
\ \ \ \ \ \ \ \ \isacommand{using}\isamarkupfalse%
\ {\isacartoucheopen}{\isasymnot}\ {\isacharparenleft}finite\ S{\isadigit{1}}\ {\isasymlongrightarrow}\ S{\isadigit{1}}\ {\isasymin}\ C{\isacharparenright}{\isacartoucheclose}\ \isacommand{by}\isamarkupfalse%
\ {\isacharparenleft}rule\ mt{\isacharparenright}\isanewline
\ \ \ \ \ \ \isacommand{then}\isamarkupfalse%
\ \isacommand{have}\isamarkupfalse%
\ {\isachardoublequoteopen}{\isacharparenleft}S{\isadigit{1}}\ {\isasymsubseteq}\ insert\ G\ S{\isacharparenright}\ {\isasymnoteq}\ {\isacharparenleft}S{\isadigit{1}}\ {\isasymsubseteq}\ S{\isacharparenright}{\isachardoublequoteclose}\isanewline
\ \ \ \ \ \ \ \ \isacommand{using}\isamarkupfalse%
\ {\isacartoucheopen}S{\isadigit{1}}\ {\isasymsubseteq}\ insert\ G\ S{\isacartoucheclose}\ \isacommand{by}\isamarkupfalse%
\ simp\isanewline
\ \ \ \ \ \ \isacommand{then}\isamarkupfalse%
\ \isacommand{have}\isamarkupfalse%
\ notSI{\isacharcolon}{\isachardoublequoteopen}{\isasymnot}{\isacharparenleft}S{\isadigit{1}}\ {\isasymsubseteq}\ insert\ G\ S\ {\isasymlongleftrightarrow}\ S{\isadigit{1}}\ {\isasymsubseteq}\ S{\isacharparenright}{\isachardoublequoteclose}\isanewline
\ \ \ \ \ \ \ \ \isacommand{by}\isamarkupfalse%
\ blast\isanewline
\ \ \ \ \ \ \isacommand{have}\isamarkupfalse%
\ subsetInsert{\isacharcolon}{\isachardoublequoteopen}G\ {\isasymnotin}\ S{\isadigit{1}}\ {\isasymLongrightarrow}\ S{\isadigit{1}}\ {\isasymsubseteq}\ insert\ G\ S\ {\isasymlongleftrightarrow}\ S{\isadigit{1}}\ {\isasymsubseteq}\ S{\isachardoublequoteclose}\isanewline
\ \ \ \ \ \ \ \ \isacommand{by}\isamarkupfalse%
\ {\isacharparenleft}rule\ subset{\isacharunderscore}insert{\isacharparenright}\isanewline
\ \ \ \ \ \ \isacommand{have}\isamarkupfalse%
\ {\isachardoublequoteopen}{\isasymnot}{\isacharparenleft}G\ {\isasymnotin}\ S{\isadigit{1}}{\isacharparenright}{\isachardoublequoteclose}\isanewline
\ \ \ \ \ \ \ \ \isacommand{using}\isamarkupfalse%
\ notSI\ subsetInsert\ \isacommand{by}\isamarkupfalse%
\ {\isacharparenleft}rule\ contrapos{\isacharunderscore}nn{\isacharparenright}\isanewline
\ \ \ \ \ \ \isacommand{then}\isamarkupfalse%
\ \isacommand{have}\isamarkupfalse%
\ {\isachardoublequoteopen}G\ {\isasymin}\ S{\isadigit{1}}{\isachardoublequoteclose}\isanewline
\ \ \ \ \ \ \ \ \isacommand{by}\isamarkupfalse%
\ {\isacharparenleft}rule\ notnotD{\isacharparenright}\isanewline
\ \ \ \ \ \ \isacommand{then}\isamarkupfalse%
\ \isacommand{have}\isamarkupfalse%
\ {\isachardoublequoteopen}insert\ G\ S{\isadigit{1}}\ {\isacharequal}\ S{\isadigit{1}}{\isachardoublequoteclose}\isanewline
\ \ \ \ \ \ \ \ \isacommand{by}\isamarkupfalse%
\ {\isacharparenleft}rule\ insert{\isacharunderscore}absorb{\isacharparenright}\isanewline
\ \ \ \ \ \ \isacommand{show}\isamarkupfalse%
\ {\isacharquery}thesis\isanewline
\ \ \ \ \ \ \ \ \isacommand{using}\isamarkupfalse%
\ {\isacartoucheopen}S{\isadigit{1}}\ {\isasymnotin}\ C{\isacartoucheclose}\ \isacommand{by}\isamarkupfalse%
\ {\isacharparenleft}simp\ only{\isacharcolon}\ simp{\isacharunderscore}thms{\isacharparenleft}{\isadigit{8}}{\isacharparenright}\ {\isacartoucheopen}insert\ G\ S{\isadigit{1}}\ {\isacharequal}\ S{\isadigit{1}}{\isacartoucheclose}{\isacharparenright}\isanewline
\ \ \ \ \isacommand{qed}\isamarkupfalse%
\ \isanewline
\ \ \ \ \isacommand{let}\isamarkupfalse%
\ {\isacharquery}S{\isadigit{1}}{\isacharequal}{\isachardoublequoteopen}S{\isadigit{1}}\ {\isacharminus}\ {\isacharbraceleft}G{\isacharbraceright}{\isachardoublequoteclose}\isanewline
\ \ \ \ \isacommand{have}\isamarkupfalse%
\ {\isachardoublequoteopen}insert\ G\ S\ {\isacharequal}\ {\isacharbraceleft}G{\isacharbraceright}\ {\isasymunion}\ S{\isachardoublequoteclose}\isanewline
\ \ \ \ \ \ \isacommand{by}\isamarkupfalse%
\ {\isacharparenleft}rule\ insert{\isacharunderscore}is{\isacharunderscore}Un{\isacharparenright}\isanewline
\ \ \ \ \isacommand{have}\isamarkupfalse%
\ {\isachardoublequoteopen}S{\isadigit{1}}\ {\isasymsubseteq}\ {\isacharbraceleft}G{\isacharbraceright}\ {\isasymunion}\ S{\isachardoublequoteclose}\isanewline
\ \ \ \ \ \ \isacommand{using}\isamarkupfalse%
\ {\isacartoucheopen}S{\isadigit{1}}\ {\isasymsubseteq}\ insert\ G\ S{\isacartoucheclose}\ \isacommand{by}\isamarkupfalse%
\ {\isacharparenleft}simp\ only{\isacharcolon}\ {\isacartoucheopen}insert\ G\ S\ {\isacharequal}\ {\isacharbraceleft}G{\isacharbraceright}\ {\isasymunion}\ S{\isacartoucheclose}{\isacharparenright}\isanewline
\ \ \ \ \isacommand{have}\isamarkupfalse%
\ {\isadigit{1}}{\isacharcolon}{\isachardoublequoteopen}{\isacharquery}S{\isadigit{1}}\ {\isasymsubseteq}\ S{\isachardoublequoteclose}\ \isanewline
\ \ \ \ \ \ \isacommand{using}\isamarkupfalse%
\ {\isacartoucheopen}S{\isadigit{1}}\ {\isasymsubseteq}\ {\isacharbraceleft}G{\isacharbraceright}\ {\isasymunion}\ S{\isacartoucheclose}\ \isacommand{by}\isamarkupfalse%
\ {\isacharparenleft}simp\ only{\isacharcolon}\ Diff{\isacharunderscore}subset{\isacharunderscore}conv{\isacharparenright}\isanewline
\ \ \ \ \isacommand{have}\isamarkupfalse%
\ {\isadigit{2}}{\isacharcolon}{\isachardoublequoteopen}finite\ {\isacharquery}S{\isadigit{1}}{\isachardoublequoteclose}\isanewline
\ \ \ \ \ \ \isacommand{using}\isamarkupfalse%
\ {\isacartoucheopen}finite\ S{\isadigit{1}}{\isacartoucheclose}\ \isacommand{by}\isamarkupfalse%
\ {\isacharparenleft}simp\ only{\isacharcolon}\ finite{\isacharunderscore}Diff{\isacharparenright}\isanewline
\ \ \ \ \isacommand{have}\isamarkupfalse%
\ {\isachardoublequoteopen}insert\ G\ {\isacharquery}S{\isadigit{1}}\ {\isacharequal}\ insert\ G\ S{\isadigit{1}}{\isachardoublequoteclose}\isanewline
\ \ \ \ \ \ \isacommand{by}\isamarkupfalse%
\ {\isacharparenleft}simp\ only{\isacharcolon}\ insert{\isacharunderscore}Diff{\isacharunderscore}single{\isacharparenright}\isanewline
\ \ \ \ \isacommand{then}\isamarkupfalse%
\ \isacommand{have}\isamarkupfalse%
\ {\isadigit{3}}{\isacharcolon}{\isachardoublequoteopen}insert\ G\ {\isacharquery}S{\isadigit{1}}\ {\isasymnotin}\ C{\isachardoublequoteclose}\isanewline
\ \ \ \ \ \ \isacommand{using}\isamarkupfalse%
\ {\isacartoucheopen}insert\ G\ S{\isadigit{1}}\ {\isasymnotin}\ C{\isacartoucheclose}\ \isacommand{by}\isamarkupfalse%
\ {\isacharparenleft}simp\ only{\isacharcolon}\ simp{\isacharunderscore}thms{\isacharparenleft}{\isadigit{6}}{\isacharcomma}{\isadigit{8}}{\isacharparenright}\ {\isacartoucheopen}insert\ G\ {\isacharquery}S{\isadigit{1}}\ {\isacharequal}\ insert\ G\ S{\isadigit{1}}{\isacartoucheclose}{\isacharparenright}\isanewline
\ \ \ \ \isacommand{have}\isamarkupfalse%
\ {\isachardoublequoteopen}insert\ H\ S\ {\isasymnotin}\ {\isacharparenleft}extF\ C{\isacharparenright}{\isachardoublequoteclose}\isanewline
\ \ \ \ \ \ \isacommand{using}\isamarkupfalse%
\ Conj\ \isacommand{by}\isamarkupfalse%
\ {\isacharparenleft}rule\ conjunct{\isadigit{2}}{\isacharparenright}\isanewline
\ \ \ \ \isacommand{then}\isamarkupfalse%
\ \isacommand{have}\isamarkupfalse%
\ {\isachardoublequoteopen}{\isasymnot}{\isacharparenleft}{\isasymforall}S{\isacharprime}\ {\isasymsubseteq}\ {\isacharparenleft}insert\ H\ S{\isacharparenright}{\isachardot}\ finite\ S{\isacharprime}\ {\isasymlongrightarrow}\ S{\isacharprime}\ {\isasymin}\ C{\isacharparenright}{\isachardoublequoteclose}\isanewline
\ \ \ \ \ \ \isacommand{unfolding}\isamarkupfalse%
\ extF\ \isacommand{by}\isamarkupfalse%
\ {\isacharparenleft}simp\ add{\isacharcolon}\ mem{\isacharunderscore}Collect{\isacharunderscore}eq{\isacharparenright}\isanewline
\ \ \ \ \isacommand{then}\isamarkupfalse%
\ \isacommand{have}\isamarkupfalse%
\ Ex{\isadigit{2}}{\isacharcolon}{\isachardoublequoteopen}{\isasymexists}S{\isacharprime}{\isasymsubseteq}\ {\isacharparenleft}insert\ H\ S{\isacharparenright}{\isachardot}\ {\isasymnot}{\isacharparenleft}finite\ S{\isacharprime}\ {\isasymlongrightarrow}\ S{\isacharprime}\ {\isasymin}\ C{\isacharparenright}{\isachardoublequoteclose}\isanewline
\ \ \ \ \ \ \isacommand{by}\isamarkupfalse%
\ {\isacharparenleft}rule\ sall{\isacharunderscore}simps{\isacharunderscore}not{\isacharunderscore}all{\isacharparenright}\isanewline
\ \ \ \ \isacommand{obtain}\isamarkupfalse%
\ S{\isadigit{2}}\ \isakeyword{where}\ {\isachardoublequoteopen}S{\isadigit{2}}\ {\isasymsubseteq}\ insert\ H\ S{\isachardoublequoteclose}\ \isakeyword{and}\ {\isachardoublequoteopen}{\isasymnot}{\isacharparenleft}finite\ S{\isadigit{2}}\ {\isasymlongrightarrow}\ S{\isadigit{2}}\ {\isasymin}\ C{\isacharparenright}{\isachardoublequoteclose}\isanewline
\ \ \ \ \ \ \isacommand{using}\isamarkupfalse%
\ Ex{\isadigit{2}}\ \isacommand{by}\isamarkupfalse%
\ {\isacharparenleft}rule\ subexE{\isacharparenright}\isanewline
\ \ \ \ \isacommand{have}\isamarkupfalse%
\ {\isachardoublequoteopen}finite\ S{\isadigit{2}}\ {\isasymand}\ S{\isadigit{2}}\ {\isasymnotin}\ C{\isachardoublequoteclose}\isanewline
\ \ \ \ \ \ \isacommand{using}\isamarkupfalse%
\ {\isacartoucheopen}{\isasymnot}{\isacharparenleft}finite\ S{\isadigit{2}}\ {\isasymlongrightarrow}\ S{\isadigit{2}}\ {\isasymin}\ C{\isacharparenright}{\isacartoucheclose}\ \isacommand{by}\isamarkupfalse%
\ {\isacharparenleft}simp\ only{\isacharcolon}\ simp{\isacharunderscore}thms{\isacharparenleft}{\isadigit{8}}{\isacharcomma}{\isadigit{2}}{\isadigit{5}}{\isacharparenright}\ not{\isacharunderscore}imp{\isacharparenright}\isanewline
\ \ \ \ \isacommand{then}\isamarkupfalse%
\ \isacommand{have}\isamarkupfalse%
\ {\isachardoublequoteopen}finite\ S{\isadigit{2}}{\isachardoublequoteclose}\isanewline
\ \ \ \ \ \ \isacommand{by}\isamarkupfalse%
\ {\isacharparenleft}rule\ conjunct{\isadigit{1}}{\isacharparenright}\isanewline
\ \ \ \ \isacommand{have}\isamarkupfalse%
\ {\isachardoublequoteopen}S{\isadigit{2}}\ {\isasymnotin}\ C{\isachardoublequoteclose}\isanewline
\ \ \ \ \ \ \isacommand{using}\isamarkupfalse%
\ {\isacartoucheopen}finite\ S{\isadigit{2}}\ {\isasymand}\ S{\isadigit{2}}\ {\isasymnotin}\ C{\isacartoucheclose}\ \isacommand{by}\isamarkupfalse%
\ {\isacharparenleft}rule\ conjunct{\isadigit{2}}{\isacharparenright}\isanewline
\ \ \ \ \isacommand{then}\isamarkupfalse%
\ \isacommand{have}\isamarkupfalse%
\ {\isachardoublequoteopen}insert\ H\ S{\isadigit{2}}\ {\isasymnotin}\ C{\isachardoublequoteclose}\isanewline
\ \ \ \ \isacommand{proof}\isamarkupfalse%
\ {\isacharminus}\isanewline
\ \ \ \ \ \ \isacommand{have}\isamarkupfalse%
\ {\isachardoublequoteopen}S{\isadigit{2}}\ {\isasymsubseteq}\ S\ {\isasymlongrightarrow}\ finite\ S{\isadigit{2}}\ {\isasymlongrightarrow}\ S{\isadigit{2}}\ {\isasymin}\ C{\isachardoublequoteclose}\isanewline
\ \ \ \ \ \ \ \ \isacommand{using}\isamarkupfalse%
\ E{\isacharprime}\ \isacommand{by}\isamarkupfalse%
\ {\isacharparenleft}rule\ allE{\isacharparenright}\isanewline
\ \ \ \ \ \ \isacommand{then}\isamarkupfalse%
\ \isacommand{have}\isamarkupfalse%
\ {\isachardoublequoteopen}{\isasymnot}\ S{\isadigit{2}}\ {\isasymsubseteq}\ S{\isachardoublequoteclose}\isanewline
\ \ \ \ \ \ \ \ \isacommand{using}\isamarkupfalse%
\ {\isacartoucheopen}{\isasymnot}\ {\isacharparenleft}finite\ S{\isadigit{2}}\ {\isasymlongrightarrow}\ S{\isadigit{2}}\ {\isasymin}\ C{\isacharparenright}{\isacartoucheclose}\ \isacommand{by}\isamarkupfalse%
\ {\isacharparenleft}rule\ mt{\isacharparenright}\isanewline
\ \ \ \ \ \ \isacommand{then}\isamarkupfalse%
\ \isacommand{have}\isamarkupfalse%
\ {\isachardoublequoteopen}{\isacharparenleft}S{\isadigit{2}}\ {\isasymsubseteq}\ insert\ H\ S{\isacharparenright}\ {\isasymnoteq}\ {\isacharparenleft}S{\isadigit{2}}\ {\isasymsubseteq}\ S{\isacharparenright}{\isachardoublequoteclose}\isanewline
\ \ \ \ \ \ \ \ \isacommand{using}\isamarkupfalse%
\ {\isacartoucheopen}S{\isadigit{2}}\ {\isasymsubseteq}\ insert\ H\ S{\isacartoucheclose}\ \isacommand{by}\isamarkupfalse%
\ simp\ \isanewline
\ \ \ \ \ \ \isacommand{then}\isamarkupfalse%
\ \isacommand{have}\isamarkupfalse%
\ notSI{\isacharcolon}{\isachardoublequoteopen}{\isasymnot}{\isacharparenleft}S{\isadigit{2}}\ {\isasymsubseteq}\ insert\ H\ S\ {\isasymlongleftrightarrow}\ S{\isadigit{2}}\ {\isasymsubseteq}\ S{\isacharparenright}{\isachardoublequoteclose}\isanewline
\ \ \ \ \ \ \ \ \isacommand{by}\isamarkupfalse%
\ blast\ \isanewline
\ \ \ \ \ \ \isacommand{have}\isamarkupfalse%
\ subsetInsert{\isacharcolon}{\isachardoublequoteopen}H\ {\isasymnotin}\ S{\isadigit{2}}\ {\isasymLongrightarrow}\ S{\isadigit{2}}\ {\isasymsubseteq}\ insert\ H\ S\ {\isasymlongleftrightarrow}\ S{\isadigit{2}}\ {\isasymsubseteq}\ S{\isachardoublequoteclose}\isanewline
\ \ \ \ \ \ \ \ \isacommand{by}\isamarkupfalse%
\ {\isacharparenleft}rule\ subset{\isacharunderscore}insert{\isacharparenright}\isanewline
\ \ \ \ \ \ \isacommand{have}\isamarkupfalse%
\ {\isachardoublequoteopen}{\isasymnot}{\isacharparenleft}H\ {\isasymnotin}\ S{\isadigit{2}}{\isacharparenright}{\isachardoublequoteclose}\isanewline
\ \ \ \ \ \ \ \ \isacommand{using}\isamarkupfalse%
\ notSI\ subsetInsert\ \isacommand{by}\isamarkupfalse%
\ {\isacharparenleft}rule\ contrapos{\isacharunderscore}nn{\isacharparenright}\isanewline
\ \ \ \ \ \ \isacommand{then}\isamarkupfalse%
\ \isacommand{have}\isamarkupfalse%
\ {\isachardoublequoteopen}H\ {\isasymin}\ S{\isadigit{2}}{\isachardoublequoteclose}\isanewline
\ \ \ \ \ \ \ \ \isacommand{by}\isamarkupfalse%
\ {\isacharparenleft}rule\ notnotD{\isacharparenright}\isanewline
\ \ \ \ \ \ \isacommand{then}\isamarkupfalse%
\ \isacommand{have}\isamarkupfalse%
\ {\isachardoublequoteopen}insert\ H\ S{\isadigit{2}}\ {\isacharequal}\ S{\isadigit{2}}{\isachardoublequoteclose}\isanewline
\ \ \ \ \ \ \ \ \isacommand{by}\isamarkupfalse%
\ {\isacharparenleft}rule\ insert{\isacharunderscore}absorb{\isacharparenright}\isanewline
\ \ \ \ \ \ \isacommand{show}\isamarkupfalse%
\ {\isacharquery}thesis\isanewline
\ \ \ \ \ \ \ \ \isacommand{using}\isamarkupfalse%
\ {\isacartoucheopen}S{\isadigit{2}}\ {\isasymnotin}\ C{\isacartoucheclose}\ \isacommand{by}\isamarkupfalse%
\ {\isacharparenleft}simp\ only{\isacharcolon}\ simp{\isacharunderscore}thms{\isacharparenleft}{\isadigit{8}}{\isacharparenright}\ {\isacartoucheopen}insert\ H\ S{\isadigit{2}}\ {\isacharequal}\ S{\isadigit{2}}{\isacartoucheclose}{\isacharparenright}\isanewline
\ \ \ \ \isacommand{qed}\isamarkupfalse%
\ \isanewline
\ \ \ \ \isacommand{let}\isamarkupfalse%
\ {\isacharquery}S{\isadigit{2}}{\isacharequal}{\isachardoublequoteopen}S{\isadigit{2}}\ {\isacharminus}\ {\isacharbraceleft}H{\isacharbraceright}{\isachardoublequoteclose}\isanewline
\ \ \ \ \isacommand{have}\isamarkupfalse%
\ {\isachardoublequoteopen}insert\ H\ S\ {\isacharequal}\ {\isacharbraceleft}H{\isacharbraceright}\ {\isasymunion}\ S{\isachardoublequoteclose}\isanewline
\ \ \ \ \ \ \isacommand{by}\isamarkupfalse%
\ {\isacharparenleft}rule\ insert{\isacharunderscore}is{\isacharunderscore}Un{\isacharparenright}\isanewline
\ \ \ \ \isacommand{have}\isamarkupfalse%
\ {\isachardoublequoteopen}S{\isadigit{2}}\ {\isasymsubseteq}\ {\isacharbraceleft}H{\isacharbraceright}\ {\isasymunion}\ S{\isachardoublequoteclose}\isanewline
\ \ \ \ \ \ \isacommand{using}\isamarkupfalse%
\ {\isacartoucheopen}S{\isadigit{2}}\ {\isasymsubseteq}\ insert\ H\ S{\isacartoucheclose}\ \isacommand{by}\isamarkupfalse%
\ {\isacharparenleft}simp\ only{\isacharcolon}\ {\isacartoucheopen}insert\ H\ S\ {\isacharequal}\ {\isacharbraceleft}H{\isacharbraceright}\ {\isasymunion}\ S{\isacartoucheclose}{\isacharparenright}\isanewline
\ \ \ \ \isacommand{have}\isamarkupfalse%
\ {\isadigit{4}}{\isacharcolon}{\isachardoublequoteopen}{\isacharquery}S{\isadigit{2}}\ {\isasymsubseteq}\ S{\isachardoublequoteclose}\ \isanewline
\ \ \ \ \ \ \isacommand{using}\isamarkupfalse%
\ {\isacartoucheopen}S{\isadigit{2}}\ {\isasymsubseteq}\ {\isacharbraceleft}H{\isacharbraceright}\ {\isasymunion}\ S{\isacartoucheclose}\ \isacommand{by}\isamarkupfalse%
\ {\isacharparenleft}simp\ only{\isacharcolon}\ Diff{\isacharunderscore}subset{\isacharunderscore}conv{\isacharparenright}\isanewline
\ \ \ \ \isacommand{have}\isamarkupfalse%
\ {\isadigit{5}}{\isacharcolon}{\isachardoublequoteopen}finite\ {\isacharquery}S{\isadigit{2}}{\isachardoublequoteclose}\ \isanewline
\ \ \ \ \ \ \isacommand{using}\isamarkupfalse%
\ {\isacartoucheopen}finite\ S{\isadigit{2}}{\isacartoucheclose}\ \isacommand{by}\isamarkupfalse%
\ {\isacharparenleft}simp\ only{\isacharcolon}\ finite{\isacharunderscore}Diff{\isacharparenright}\isanewline
\ \ \ \ \isacommand{have}\isamarkupfalse%
\ {\isachardoublequoteopen}insert\ H\ {\isacharquery}S{\isadigit{2}}\ {\isacharequal}\ insert\ H\ S{\isadigit{2}}{\isachardoublequoteclose}\isanewline
\ \ \ \ \ \ \isacommand{by}\isamarkupfalse%
\ {\isacharparenleft}simp\ only{\isacharcolon}\ insert{\isacharunderscore}Diff{\isacharunderscore}single{\isacharparenright}\isanewline
\ \ \ \ \isacommand{then}\isamarkupfalse%
\ \isacommand{have}\isamarkupfalse%
\ {\isadigit{6}}{\isacharcolon}{\isachardoublequoteopen}insert\ H\ {\isacharquery}S{\isadigit{2}}\ {\isasymnotin}\ C{\isachardoublequoteclose}\isanewline
\ \ \ \ \ \ \isacommand{using}\isamarkupfalse%
\ {\isacartoucheopen}insert\ H\ S{\isadigit{2}}\ {\isasymnotin}\ C{\isacartoucheclose}\ \isacommand{by}\isamarkupfalse%
\ {\isacharparenleft}simp\ only{\isacharcolon}\ simp{\isacharunderscore}thms{\isacharparenleft}{\isadigit{6}}{\isacharcomma}{\isadigit{8}}{\isacharparenright}\ {\isacartoucheopen}insert\ H\ {\isacharquery}S{\isadigit{2}}\ {\isacharequal}\ insert\ H\ S{\isadigit{2}}{\isacartoucheclose}{\isacharparenright}\isanewline
\ \ \ \ \isacommand{show}\isamarkupfalse%
\ {\isachardoublequoteopen}False{\isachardoublequoteclose}\isanewline
\ \ \ \ \ \ \isacommand{using}\isamarkupfalse%
\ assms{\isacharparenleft}{\isadigit{1}}{\isacharparenright}\ assms{\isacharparenleft}{\isadigit{2}}{\isacharparenright}\ assms{\isacharparenleft}{\isadigit{3}}{\isacharparenright}\ assms{\isacharparenleft}{\isadigit{4}}{\isacharparenright}\ assms{\isacharparenleft}{\isadigit{5}}{\isacharparenright}\ {\isadigit{1}}\ {\isadigit{2}}\ {\isadigit{3}}\ {\isadigit{4}}\ {\isadigit{5}}\ {\isadigit{6}}\ \isacommand{by}\isamarkupfalse%
\ {\isacharparenleft}rule\ ex{\isadigit{3}}{\isacharunderscore}pcp{\isacharunderscore}SinE{\isacharunderscore}DIS{\isacharunderscore}auxFalse{\isacharparenright}\isanewline
\ \ \isacommand{qed}\isamarkupfalse%
\isanewline
\ \ \isacommand{thus}\isamarkupfalse%
\ {\isacharquery}thesis\isanewline
\ \ \isacommand{proof}\isamarkupfalse%
\ {\isacharparenleft}rule\ disjE{\isacharparenright}\isanewline
\ \ \ \ \isacommand{assume}\isamarkupfalse%
\ {\isachardoublequoteopen}insert\ G\ S\ {\isasymin}\ {\isacharparenleft}extF\ C{\isacharparenright}{\isachardoublequoteclose}\isanewline
\ \ \ \ \isacommand{have}\isamarkupfalse%
\ insG{\isacharcolon}{\isachardoublequoteopen}insert\ G\ S\ {\isasymin}\ {\isacharparenleft}extensionFin\ C{\isacharparenright}{\isachardoublequoteclose}\isanewline
\ \ \ \ \ \ \isacommand{using}\isamarkupfalse%
\ {\isacartoucheopen}{\isacharparenleft}extF\ C{\isacharparenright}\ {\isasymsubseteq}\ {\isacharparenleft}extensionFin\ C{\isacharparenright}{\isacartoucheclose}\ {\isacartoucheopen}insert\ G\ S\ {\isasymin}\ {\isacharparenleft}extF\ C{\isacharparenright}{\isacartoucheclose}\ \isacommand{by}\isamarkupfalse%
\ {\isacharparenleft}simp\ only{\isacharcolon}\ in{\isacharunderscore}mono{\isacharparenright}\isanewline
\ \ \ \ \isacommand{have}\isamarkupfalse%
\ {\isachardoublequoteopen}insert\ G\ S\ {\isacharequal}\ {\isacharbraceleft}G{\isacharbraceright}\ {\isasymunion}\ S{\isachardoublequoteclose}\isanewline
\ \ \ \ \ \ \isacommand{by}\isamarkupfalse%
\ {\isacharparenleft}rule\ insert{\isacharunderscore}is{\isacharunderscore}Un{\isacharparenright}\isanewline
\ \ \ \ \isacommand{then}\isamarkupfalse%
\ \isacommand{have}\isamarkupfalse%
\ {\isachardoublequoteopen}{\isacharbraceleft}G{\isacharbraceright}\ {\isasymunion}\ S\ {\isasymin}\ {\isacharparenleft}extensionFin\ C{\isacharparenright}{\isachardoublequoteclose}\isanewline
\ \ \ \ \ \ \isacommand{using}\isamarkupfalse%
\ insG\ {\isacartoucheopen}insert\ G\ S\ {\isacharequal}\ {\isacharbraceleft}G{\isacharbraceright}\ {\isasymunion}\ S{\isacartoucheclose}\ \isacommand{by}\isamarkupfalse%
\ {\isacharparenleft}simp\ only{\isacharcolon}\ insG{\isacharparenright}\isanewline
\ \ \ \ \isacommand{thus}\isamarkupfalse%
\ {\isacharquery}thesis\isanewline
\ \ \ \ \ \ \isacommand{by}\isamarkupfalse%
\ {\isacharparenleft}rule\ disjI{\isadigit{1}}{\isacharparenright}\isanewline
\ \ \isacommand{next}\isamarkupfalse%
\isanewline
\ \ \ \ \isacommand{assume}\isamarkupfalse%
\ {\isachardoublequoteopen}insert\ H\ S\ {\isasymin}\ {\isacharparenleft}extF\ C{\isacharparenright}{\isachardoublequoteclose}\isanewline
\ \ \ \ \isacommand{have}\isamarkupfalse%
\ insH{\isacharcolon}{\isachardoublequoteopen}insert\ H\ S\ {\isasymin}\ {\isacharparenleft}extensionFin\ C{\isacharparenright}{\isachardoublequoteclose}\isanewline
\ \ \ \ \ \ \isacommand{using}\isamarkupfalse%
\ {\isacartoucheopen}{\isacharparenleft}extF\ C{\isacharparenright}\ {\isasymsubseteq}\ {\isacharparenleft}extensionFin\ C{\isacharparenright}{\isacartoucheclose}\ {\isacartoucheopen}insert\ H\ S\ {\isasymin}\ {\isacharparenleft}extF\ C{\isacharparenright}{\isacartoucheclose}\ \isacommand{by}\isamarkupfalse%
\ {\isacharparenleft}simp\ only{\isacharcolon}\ in{\isacharunderscore}mono{\isacharparenright}\isanewline
\ \ \ \ \isacommand{have}\isamarkupfalse%
\ {\isachardoublequoteopen}insert\ H\ S\ {\isacharequal}\ {\isacharbraceleft}H{\isacharbraceright}\ {\isasymunion}\ S{\isachardoublequoteclose}\isanewline
\ \ \ \ \ \ \isacommand{by}\isamarkupfalse%
\ {\isacharparenleft}rule\ insert{\isacharunderscore}is{\isacharunderscore}Un{\isacharparenright}\isanewline
\ \ \ \ \isacommand{then}\isamarkupfalse%
\ \isacommand{have}\isamarkupfalse%
\ {\isachardoublequoteopen}{\isacharbraceleft}H{\isacharbraceright}\ {\isasymunion}\ S\ {\isasymin}\ {\isacharparenleft}extensionFin\ C{\isacharparenright}{\isachardoublequoteclose}\isanewline
\ \ \ \ \ \ \isacommand{using}\isamarkupfalse%
\ insH\ {\isacartoucheopen}insert\ H\ S\ {\isacharequal}\ {\isacharbraceleft}H{\isacharbraceright}\ {\isasymunion}\ S{\isacartoucheclose}\ \isacommand{by}\isamarkupfalse%
\ {\isacharparenleft}simp\ only{\isacharcolon}\ insH{\isacharparenright}\isanewline
\ \ \ \ \isacommand{thus}\isamarkupfalse%
\ {\isacharquery}thesis\isanewline
\ \ \ \ \ \ \isacommand{by}\isamarkupfalse%
\ {\isacharparenleft}rule\ disjI{\isadigit{2}}{\isacharparenright}\isanewline
\ \ \isacommand{qed}\isamarkupfalse%
\isanewline
\isacommand{qed}\isamarkupfalse%
%
\endisatagproof
{\isafoldproof}%
%
\isadelimproof
%
\endisadelimproof
%
\begin{isamarkuptext}%
Probados los lemas \isa{ex{\isadigit{3}}{\isacharunderscore}pcp{\isacharunderscore}SinE{\isacharunderscore}CON} y \isa{ex{\isadigit{3}}{\isacharunderscore}pcp{\isacharunderscore}SinE{\isacharunderscore}DIS}, podemos demostrar que \isa{C{\isacharprime}\ {\isacharequal}\ C\ {\isasymunion}\ E} 
  verifica las condiciones del lema de caracterización de la propiedad de consistencia proposicional 
  para el caso en que \isa{S\ {\isasymin}\ E}, formalizado como \isa{ex{\isadigit{3}}{\isacharunderscore}pcp{\isacharunderscore}SinE}. Dicho lema prueba que, si \isa{C} es 
  una colección con la propiedad de consistencia proposicional y cerrada bajo subconjuntos, y sea 
  \isa{S\ {\isasymin}\ E}, se verifican las condiciones:
  \begin{itemize}
    \item \isa{{\isasymbottom}\ {\isasymnotin}\ S}.
    \item Dada \isa{p} una fórmula atómica cualquiera, no se tiene 
    simultáneamente que\\ \isa{p\ {\isasymin}\ S} y \isa{{\isasymnot}\ p\ {\isasymin}\ S}.
    \item Para toda fórmula de tipo \isa{{\isasymalpha}} con componentes \isa{{\isasymalpha}\isactrlsub {\isadigit{1}}} y \isa{{\isasymalpha}\isactrlsub {\isadigit{2}}} tal que \isa{{\isasymalpha}}
    pertenece a \isa{S}, se tiene que \isa{{\isacharbraceleft}{\isasymalpha}\isactrlsub {\isadigit{1}}{\isacharcomma}{\isasymalpha}\isactrlsub {\isadigit{2}}{\isacharbraceright}\ {\isasymunion}\ S} pertenece a \isa{C{\isacharprime}}.
    \item Para toda fórmula de tipo \isa{{\isasymbeta}} con componentes \isa{{\isasymbeta}\isactrlsub {\isadigit{1}}} y \isa{{\isasymbeta}\isactrlsub {\isadigit{2}}} tal que \isa{{\isasymbeta}}
    pertenece a \isa{S}, se tiene que o bien \isa{{\isacharbraceleft}{\isasymbeta}\isactrlsub {\isadigit{1}}{\isacharbraceright}\ {\isasymunion}\ S} pertenece a \isa{C{\isacharprime}} o 
    bien \isa{{\isacharbraceleft}{\isasymbeta}\isactrlsub {\isadigit{2}}{\isacharbraceright}\ {\isasymunion}\ S} pertenece a \isa{C{\isacharprime}}.
  \end{itemize}%
\end{isamarkuptext}\isamarkuptrue%
\isacommand{lemma}\isamarkupfalse%
\ ex{\isadigit{3}}{\isacharunderscore}pcp{\isacharunderscore}SinE{\isacharcolon}\isanewline
\ \ \isakeyword{assumes}\ {\isachardoublequoteopen}pcp\ C{\isachardoublequoteclose}\isanewline
\ \ \ \ \ \ \ \ \ \ {\isachardoublequoteopen}subset{\isacharunderscore}closed\ C{\isachardoublequoteclose}\isanewline
\ \ \ \ \ \ \ \ \ \ {\isachardoublequoteopen}S\ {\isasymin}\ {\isacharparenleft}extF\ C{\isacharparenright}{\isachardoublequoteclose}\ \isanewline
\ \ \isakeyword{shows}\ {\isachardoublequoteopen}{\isasymbottom}\ {\isasymnotin}\ S\ {\isasymand}\isanewline
\ \ \ \ \ \ \ \ \ {\isacharparenleft}{\isasymforall}k{\isachardot}\ Atom\ k\ {\isasymin}\ S\ {\isasymlongrightarrow}\ \isactrlbold {\isasymnot}\ {\isacharparenleft}Atom\ k{\isacharparenright}\ {\isasymin}\ S\ {\isasymlongrightarrow}\ False{\isacharparenright}\ {\isasymand}\isanewline
\ \ \ \ \ \ \ \ \ {\isacharparenleft}{\isasymforall}F\ G\ H{\isachardot}\ Con\ F\ G\ H\ {\isasymlongrightarrow}\ F\ {\isasymin}\ S\ {\isasymlongrightarrow}\ {\isacharbraceleft}G{\isacharcomma}\ H{\isacharbraceright}\ {\isasymunion}\ S\ {\isasymin}\ {\isacharparenleft}extensionFin\ C{\isacharparenright}{\isacharparenright}\ {\isasymand}\isanewline
\ \ \ \ \ \ \ \ \ {\isacharparenleft}{\isasymforall}F\ G\ H{\isachardot}\ Dis\ F\ G\ H\ {\isasymlongrightarrow}\ F\ {\isasymin}\ S\ {\isasymlongrightarrow}\ {\isacharbraceleft}G{\isacharbraceright}\ {\isasymunion}\ S\ {\isasymin}\ {\isacharparenleft}extensionFin\ C{\isacharparenright}\ {\isasymor}\ {\isacharbraceleft}H{\isacharbraceright}\ {\isasymunion}\ S\ {\isasymin}\ {\isacharparenleft}extensionFin\ C{\isacharparenright}{\isacharparenright}{\isachardoublequoteclose}\isanewline
%
\isadelimproof
%
\endisadelimproof
%
\isatagproof
\isacommand{proof}\isamarkupfalse%
\ {\isacharminus}\isanewline
\ \ \isacommand{have}\isamarkupfalse%
\ PCP{\isacharcolon}{\isachardoublequoteopen}{\isasymforall}S\ {\isasymin}\ C{\isachardot}\isanewline
\ \ \ \ \ \ \ \ \ {\isasymbottom}\ {\isasymnotin}\ S\ {\isasymand}\isanewline
\ \ \ \ \ \ \ \ \ {\isacharparenleft}{\isasymforall}k{\isachardot}\ Atom\ k\ {\isasymin}\ S\ {\isasymlongrightarrow}\ \isactrlbold {\isasymnot}\ {\isacharparenleft}Atom\ k{\isacharparenright}\ {\isasymin}\ S\ {\isasymlongrightarrow}\ False{\isacharparenright}\ {\isasymand}\isanewline
\ \ \ \ \ \ \ \ \ {\isacharparenleft}{\isasymforall}F\ G\ H{\isachardot}\ Con\ F\ G\ H\ {\isasymlongrightarrow}\ F\ {\isasymin}\ S\ {\isasymlongrightarrow}\ {\isacharbraceleft}G{\isacharcomma}\ H{\isacharbraceright}\ {\isasymunion}\ S\ {\isasymin}\ C{\isacharparenright}\ {\isasymand}\isanewline
\ \ \ \ \ \ \ \ \ {\isacharparenleft}{\isasymforall}F\ G\ H{\isachardot}\ Dis\ F\ G\ H\ {\isasymlongrightarrow}\ F\ {\isasymin}\ S\ {\isasymlongrightarrow}\ {\isacharbraceleft}G{\isacharbraceright}\ {\isasymunion}\ S\ {\isasymin}\ C\ {\isasymor}\ {\isacharbraceleft}H{\isacharbraceright}\ {\isasymunion}\ S\ {\isasymin}\ C{\isacharparenright}{\isachardoublequoteclose}\isanewline
\ \ \ \ \isacommand{using}\isamarkupfalse%
\ assms{\isacharparenleft}{\isadigit{1}}{\isacharparenright}\ \isacommand{by}\isamarkupfalse%
\ {\isacharparenleft}rule\ pcp{\isacharunderscore}alt{\isadigit{1}}{\isacharparenright}\isanewline
\ \ \isacommand{have}\isamarkupfalse%
\ E{\isacharcolon}{\isachardoublequoteopen}{\isasymforall}S{\isacharprime}\ {\isasymsubseteq}\ S{\isachardot}\ finite\ S{\isacharprime}\ {\isasymlongrightarrow}\ S{\isacharprime}\ {\isasymin}\ C{\isachardoublequoteclose}\isanewline
\ \ \ \ \isacommand{using}\isamarkupfalse%
\ assms{\isacharparenleft}{\isadigit{3}}{\isacharparenright}\ \isacommand{unfolding}\isamarkupfalse%
\ extF\ \isacommand{by}\isamarkupfalse%
\ {\isacharparenleft}rule\ CollectD{\isacharparenright}\isanewline
\ \ \isacommand{have}\isamarkupfalse%
\ {\isachardoublequoteopen}{\isacharbraceleft}{\isacharbraceright}\ {\isasymsubseteq}\ S{\isachardoublequoteclose}\isanewline
\ \ \ \ \isacommand{by}\isamarkupfalse%
\ {\isacharparenleft}rule\ empty{\isacharunderscore}subsetI{\isacharparenright}\isanewline
\ \ \isacommand{have}\isamarkupfalse%
\ C{\isadigit{1}}{\isacharcolon}{\isachardoublequoteopen}{\isasymbottom}\ {\isasymnotin}\ S{\isachardoublequoteclose}\isanewline
\ \ \isacommand{proof}\isamarkupfalse%
\ {\isacharparenleft}rule\ ccontr{\isacharparenright}\isanewline
\ \ \ \ \isacommand{assume}\isamarkupfalse%
\ {\isachardoublequoteopen}{\isasymnot}{\isacharparenleft}{\isasymbottom}\ {\isasymnotin}\ S{\isacharparenright}{\isachardoublequoteclose}\isanewline
\ \ \ \ \isacommand{then}\isamarkupfalse%
\ \isacommand{have}\isamarkupfalse%
\ {\isachardoublequoteopen}{\isasymbottom}\ {\isasymin}\ S{\isachardoublequoteclose}\isanewline
\ \ \ \ \ \ \isacommand{by}\isamarkupfalse%
\ {\isacharparenleft}rule\ notnotD{\isacharparenright}\isanewline
\ \ \ \ \isacommand{then}\isamarkupfalse%
\ \isacommand{have}\isamarkupfalse%
\ {\isachardoublequoteopen}{\isasymbottom}\ {\isasymin}\ S\ {\isasymand}\ {\isacharbraceleft}{\isacharbraceright}\ {\isasymsubseteq}\ S{\isachardoublequoteclose}\isanewline
\ \ \ \ \ \ \isacommand{using}\isamarkupfalse%
\ {\isacartoucheopen}{\isacharbraceleft}{\isacharbraceright}\ {\isasymsubseteq}\ S{\isacartoucheclose}\ \isacommand{by}\isamarkupfalse%
\ {\isacharparenleft}rule\ conjI{\isacharparenright}\isanewline
\ \ \ \ \isacommand{then}\isamarkupfalse%
\ \isacommand{have}\isamarkupfalse%
\ {\isachardoublequoteopen}insert\ {\isasymbottom}\ {\isacharbraceleft}{\isacharbraceright}\ {\isasymsubseteq}\ S{\isachardoublequoteclose}\ \isanewline
\ \ \ \ \ \ \isacommand{by}\isamarkupfalse%
\ {\isacharparenleft}simp\ only{\isacharcolon}\ insert{\isacharunderscore}subset{\isacharparenright}\isanewline
\ \ \ \ \isacommand{have}\isamarkupfalse%
\ {\isachardoublequoteopen}finite\ {\isacharbraceleft}{\isacharbraceright}{\isachardoublequoteclose}\isanewline
\ \ \ \ \ \ \isacommand{by}\isamarkupfalse%
\ {\isacharparenleft}rule\ finite{\isachardot}emptyI{\isacharparenright}\isanewline
\ \ \ \ \isacommand{then}\isamarkupfalse%
\ \isacommand{have}\isamarkupfalse%
\ {\isachardoublequoteopen}finite\ {\isacharparenleft}insert\ {\isasymbottom}\ {\isacharbraceleft}{\isacharbraceright}{\isacharparenright}{\isachardoublequoteclose}\isanewline
\ \ \ \ \ \ \isacommand{by}\isamarkupfalse%
\ {\isacharparenleft}rule\ finite{\isachardot}insertI{\isacharparenright}\isanewline
\ \ \ \ \isacommand{have}\isamarkupfalse%
\ {\isachardoublequoteopen}finite\ {\isacharparenleft}insert\ {\isasymbottom}\ {\isacharbraceleft}{\isacharbraceright}{\isacharparenright}\ {\isasymlongrightarrow}\ {\isacharparenleft}insert\ {\isasymbottom}\ {\isacharbraceleft}{\isacharbraceright}{\isacharparenright}\ {\isasymin}\ C{\isachardoublequoteclose}\isanewline
\ \ \ \ \ \ \isacommand{using}\isamarkupfalse%
\ E\ {\isacartoucheopen}{\isacharparenleft}insert\ {\isasymbottom}\ {\isacharbraceleft}{\isacharbraceright}{\isacharparenright}\ {\isasymsubseteq}\ S{\isacartoucheclose}\ \isacommand{by}\isamarkupfalse%
\ simp\ \isanewline
\ \ \ \ \isacommand{then}\isamarkupfalse%
\ \isacommand{have}\isamarkupfalse%
\ {\isachardoublequoteopen}{\isacharparenleft}insert\ {\isasymbottom}\ {\isacharbraceleft}{\isacharbraceright}{\isacharparenright}\ {\isasymin}\ C{\isachardoublequoteclose}\isanewline
\ \ \ \ \ \ \isacommand{using}\isamarkupfalse%
\ {\isacartoucheopen}finite\ {\isacharparenleft}insert\ {\isasymbottom}\ {\isacharbraceleft}{\isacharbraceright}{\isacharparenright}{\isacartoucheclose}\ \isacommand{by}\isamarkupfalse%
\ {\isacharparenleft}rule\ mp{\isacharparenright}\isanewline
\ \ \ \ \isacommand{have}\isamarkupfalse%
\ {\isachardoublequoteopen}{\isasymbottom}\ {\isasymnotin}\ {\isacharparenleft}insert\ {\isasymbottom}\ {\isacharbraceleft}{\isacharbraceright}{\isacharparenright}\ {\isasymand}\isanewline
\ \ \ \ \ \ \ \ \ {\isacharparenleft}{\isasymforall}k{\isachardot}\ Atom\ k\ {\isasymin}\ {\isacharparenleft}insert\ {\isasymbottom}\ {\isacharbraceleft}{\isacharbraceright}{\isacharparenright}\ {\isasymlongrightarrow}\ \isactrlbold {\isasymnot}\ {\isacharparenleft}Atom\ k{\isacharparenright}\ {\isasymin}\ {\isacharparenleft}insert\ {\isasymbottom}\ {\isacharbraceleft}{\isacharbraceright}{\isacharparenright}\ {\isasymlongrightarrow}\ False{\isacharparenright}\ {\isasymand}\isanewline
\ \ \ \ \ \ \ \ \ {\isacharparenleft}{\isasymforall}F\ G\ H{\isachardot}\ Con\ F\ G\ H\ {\isasymlongrightarrow}\ F\ {\isasymin}\ {\isacharparenleft}insert\ {\isasymbottom}\ {\isacharbraceleft}{\isacharbraceright}{\isacharparenright}\ {\isasymlongrightarrow}\ {\isacharbraceleft}G{\isacharcomma}\ H{\isacharbraceright}\ {\isasymunion}\ {\isacharparenleft}insert\ {\isasymbottom}\ {\isacharbraceleft}{\isacharbraceright}{\isacharparenright}\ {\isasymin}\ C{\isacharparenright}\ {\isasymand}\isanewline
\ \ \ \ \ \ \ \ \ {\isacharparenleft}{\isasymforall}F\ G\ H{\isachardot}\ Dis\ F\ G\ H\ {\isasymlongrightarrow}\ F\ {\isasymin}\ {\isacharparenleft}insert\ {\isasymbottom}\ {\isacharbraceleft}{\isacharbraceright}{\isacharparenright}\ {\isasymlongrightarrow}\ {\isacharbraceleft}G{\isacharbraceright}\ {\isasymunion}\ {\isacharparenleft}insert\ {\isasymbottom}\ {\isacharbraceleft}{\isacharbraceright}{\isacharparenright}\ {\isasymin}\ C\ {\isasymor}\ {\isacharbraceleft}H{\isacharbraceright}\ {\isasymunion}\ {\isacharparenleft}insert\ {\isasymbottom}\ {\isacharbraceleft}{\isacharbraceright}{\isacharparenright}\ {\isasymin}\ C{\isacharparenright}{\isachardoublequoteclose}\isanewline
\ \ \ \ \ \ \isacommand{using}\isamarkupfalse%
\ PCP\ {\isacartoucheopen}{\isacharparenleft}insert\ {\isasymbottom}\ {\isacharbraceleft}{\isacharbraceright}{\isacharparenright}\ {\isasymin}\ C{\isacartoucheclose}\ \isacommand{by}\isamarkupfalse%
\ blast\ \isanewline
\ \ \ \ \isacommand{then}\isamarkupfalse%
\ \isacommand{have}\isamarkupfalse%
\ {\isachardoublequoteopen}{\isasymbottom}\ {\isasymnotin}\ {\isacharparenleft}insert\ {\isasymbottom}\ {\isacharbraceleft}{\isacharbraceright}{\isacharparenright}{\isachardoublequoteclose}\isanewline
\ \ \ \ \ \ \isacommand{by}\isamarkupfalse%
\ {\isacharparenleft}rule\ conjunct{\isadigit{1}}{\isacharparenright}\isanewline
\ \ \ \ \isacommand{have}\isamarkupfalse%
\ {\isachardoublequoteopen}{\isasymbottom}\ {\isasymin}\ {\isacharparenleft}insert\ {\isasymbottom}\ {\isacharbraceleft}{\isacharbraceright}{\isacharparenright}{\isachardoublequoteclose}\isanewline
\ \ \ \ \ \ \isacommand{by}\isamarkupfalse%
\ {\isacharparenleft}rule\ insertI{\isadigit{1}}{\isacharparenright}\isanewline
\ \ \ \ \isacommand{show}\isamarkupfalse%
\ {\isachardoublequoteopen}False{\isachardoublequoteclose}\isanewline
\ \ \ \ \ \ \isacommand{using}\isamarkupfalse%
\ {\isacartoucheopen}{\isasymbottom}\ {\isasymnotin}\ {\isacharparenleft}insert\ {\isasymbottom}\ {\isacharbraceleft}{\isacharbraceright}{\isacharparenright}{\isacartoucheclose}\ {\isacartoucheopen}{\isasymbottom}\ {\isasymin}\ {\isacharparenleft}insert\ {\isasymbottom}\ {\isacharbraceleft}{\isacharbraceright}{\isacharparenright}{\isacartoucheclose}\ \isacommand{by}\isamarkupfalse%
\ {\isacharparenleft}rule\ notE{\isacharparenright}\isanewline
\ \ \isacommand{qed}\isamarkupfalse%
\isanewline
\ \ \isacommand{have}\isamarkupfalse%
\ C{\isadigit{2}}{\isacharcolon}{\isachardoublequoteopen}{\isasymforall}k{\isachardot}\ Atom\ k\ {\isasymin}\ S\ {\isasymlongrightarrow}\ \isactrlbold {\isasymnot}\ {\isacharparenleft}Atom\ k{\isacharparenright}\ {\isasymin}\ S\ {\isasymlongrightarrow}\ False{\isachardoublequoteclose}\isanewline
\ \ \isacommand{proof}\isamarkupfalse%
\ {\isacharparenleft}rule\ allI{\isacharparenright}\isanewline
\ \ \ \ \isacommand{fix}\isamarkupfalse%
\ k\isanewline
\ \ \ \ \isacommand{show}\isamarkupfalse%
\ {\isachardoublequoteopen}Atom\ k\ {\isasymin}\ S\ {\isasymlongrightarrow}\ \isactrlbold {\isasymnot}{\isacharparenleft}Atom\ k{\isacharparenright}\ {\isasymin}\ S\ {\isasymlongrightarrow}\ False{\isachardoublequoteclose}\isanewline
\ \ \ \ \isacommand{proof}\isamarkupfalse%
\ {\isacharparenleft}rule\ impI{\isacharparenright}{\isacharplus}\isanewline
\ \ \ \ \ \ \isacommand{assume}\isamarkupfalse%
\ {\isachardoublequoteopen}Atom\ k\ {\isasymin}\ S{\isachardoublequoteclose}\isanewline
\ \ \ \ \ \ \isacommand{assume}\isamarkupfalse%
\ {\isachardoublequoteopen}\isactrlbold {\isasymnot}{\isacharparenleft}Atom\ k{\isacharparenright}\ {\isasymin}\ S{\isachardoublequoteclose}\isanewline
\ \ \ \ \ \ \isacommand{let}\isamarkupfalse%
\ {\isacharquery}A{\isacharequal}{\isachardoublequoteopen}insert\ {\isacharparenleft}Atom\ k{\isacharparenright}\ {\isacharparenleft}insert\ {\isacharparenleft}\isactrlbold {\isasymnot}{\isacharparenleft}Atom\ k{\isacharparenright}{\isacharparenright}\ {\isacharbraceleft}{\isacharbraceright}{\isacharparenright}{\isachardoublequoteclose}\isanewline
\ \ \ \ \ \ \isacommand{have}\isamarkupfalse%
\ {\isachardoublequoteopen}Atom\ k\ {\isasymin}\ {\isacharquery}A{\isachardoublequoteclose}\isanewline
\ \ \ \ \ \ \ \ \isacommand{by}\isamarkupfalse%
\ {\isacharparenleft}simp\ only{\isacharcolon}\ insert{\isacharunderscore}iff\ simp{\isacharunderscore}thms{\isacharparenright}\ \isanewline
\ \ \ \ \ \ \isacommand{have}\isamarkupfalse%
\ {\isachardoublequoteopen}\isactrlbold {\isasymnot}{\isacharparenleft}Atom\ k{\isacharparenright}\ {\isasymin}\ {\isacharquery}A{\isachardoublequoteclose}\isanewline
\ \ \ \ \ \ \ \ \isacommand{by}\isamarkupfalse%
\ {\isacharparenleft}simp\ only{\isacharcolon}\ insert{\isacharunderscore}iff\ simp{\isacharunderscore}thms{\isacharparenright}\ \isanewline
\ \ \ \ \ \ \isacommand{have}\isamarkupfalse%
\ inSubset{\isacharcolon}{\isachardoublequoteopen}insert\ {\isacharparenleft}\isactrlbold {\isasymnot}{\isacharparenleft}Atom\ k{\isacharparenright}{\isacharparenright}\ {\isacharbraceleft}{\isacharbraceright}\ {\isasymsubseteq}\ S{\isachardoublequoteclose}\isanewline
\ \ \ \ \ \ \ \ \isacommand{using}\isamarkupfalse%
\ {\isacartoucheopen}\isactrlbold {\isasymnot}{\isacharparenleft}Atom\ k{\isacharparenright}\ {\isasymin}\ S{\isacartoucheclose}\ {\isacartoucheopen}{\isacharbraceleft}{\isacharbraceright}\ {\isasymsubseteq}\ S{\isacartoucheclose}\ \isacommand{by}\isamarkupfalse%
\ {\isacharparenleft}simp\ only{\isacharcolon}\ insert{\isacharunderscore}subset{\isacharparenright}\isanewline
\ \ \ \ \ \ \isacommand{have}\isamarkupfalse%
\ {\isachardoublequoteopen}{\isacharquery}A\ {\isasymsubseteq}\ S{\isachardoublequoteclose}\isanewline
\ \ \ \ \ \ \ \ \isacommand{using}\isamarkupfalse%
\ inSubset\ {\isacartoucheopen}Atom\ k\ {\isasymin}\ S{\isacartoucheclose}\ \isacommand{by}\isamarkupfalse%
\ {\isacharparenleft}simp\ only{\isacharcolon}\ insert{\isacharunderscore}subset{\isacharparenright}\isanewline
\ \ \ \ \ \ \isacommand{have}\isamarkupfalse%
\ {\isachardoublequoteopen}finite\ {\isacharbraceleft}{\isacharbraceright}{\isachardoublequoteclose}\isanewline
\ \ \ \ \ \ \ \ \isacommand{by}\isamarkupfalse%
\ {\isacharparenleft}simp\ only{\isacharcolon}\ finite{\isachardot}emptyI{\isacharparenright}\isanewline
\ \ \ \ \ \ \isacommand{then}\isamarkupfalse%
\ \isacommand{have}\isamarkupfalse%
\ {\isachardoublequoteopen}finite\ {\isacharparenleft}insert\ {\isacharparenleft}\isactrlbold {\isasymnot}{\isacharparenleft}Atom\ k{\isacharparenright}{\isacharparenright}\ {\isacharbraceleft}{\isacharbraceright}{\isacharparenright}{\isachardoublequoteclose}\isanewline
\ \ \ \ \ \ \ \ \isacommand{by}\isamarkupfalse%
\ {\isacharparenleft}rule\ finite{\isachardot}insertI{\isacharparenright}\isanewline
\ \ \ \ \ \ \isacommand{then}\isamarkupfalse%
\ \isacommand{have}\isamarkupfalse%
\ {\isachardoublequoteopen}finite\ {\isacharquery}A{\isachardoublequoteclose}\isanewline
\ \ \ \ \ \ \ \ \isacommand{by}\isamarkupfalse%
\ {\isacharparenleft}rule\ finite{\isachardot}insertI{\isacharparenright}\isanewline
\ \ \ \ \ \ \isacommand{have}\isamarkupfalse%
\ {\isachardoublequoteopen}finite\ {\isacharquery}A\ {\isasymlongrightarrow}\ {\isacharquery}A\ {\isasymin}\ C{\isachardoublequoteclose}\isanewline
\ \ \ \ \ \ \ \ \isacommand{using}\isamarkupfalse%
\ E\ {\isacartoucheopen}{\isacharquery}A\ {\isasymsubseteq}\ S{\isacartoucheclose}\ \isacommand{by}\isamarkupfalse%
\ {\isacharparenleft}rule\ sspec{\isacharparenright}\isanewline
\ \ \ \ \ \ \isacommand{then}\isamarkupfalse%
\ \isacommand{have}\isamarkupfalse%
\ {\isachardoublequoteopen}{\isacharquery}A\ {\isasymin}\ C{\isachardoublequoteclose}\isanewline
\ \ \ \ \ \ \ \ \isacommand{using}\isamarkupfalse%
\ {\isacartoucheopen}finite\ {\isacharquery}A{\isacartoucheclose}\ \isacommand{by}\isamarkupfalse%
\ {\isacharparenleft}rule\ mp{\isacharparenright}\isanewline
\ \ \ \ \ \ \isacommand{have}\isamarkupfalse%
\ {\isachardoublequoteopen}{\isasymbottom}\ {\isasymnotin}\ {\isacharquery}A\isanewline
\ \ \ \ \ \ \ \ \ \ \ \ {\isasymand}\ {\isacharparenleft}{\isasymforall}k{\isachardot}\ Atom\ k\ {\isasymin}\ {\isacharquery}A\ {\isasymlongrightarrow}\ \isactrlbold {\isasymnot}\ {\isacharparenleft}Atom\ k{\isacharparenright}\ {\isasymin}\ {\isacharquery}A\ {\isasymlongrightarrow}\ False{\isacharparenright}\isanewline
\ \ \ \ \ \ \ \ \ \ \ \ {\isasymand}\ {\isacharparenleft}{\isasymforall}F\ G\ H{\isachardot}\ Con\ F\ G\ H\ {\isasymlongrightarrow}\ F\ {\isasymin}\ {\isacharquery}A\ {\isasymlongrightarrow}\ {\isacharbraceleft}G{\isacharcomma}H{\isacharbraceright}\ {\isasymunion}\ {\isacharquery}A\ {\isasymin}\ C{\isacharparenright}\isanewline
\ \ \ \ \ \ \ \ \ \ \ \ {\isasymand}\ {\isacharparenleft}{\isasymforall}F\ G\ H{\isachardot}\ Dis\ F\ G\ H\ {\isasymlongrightarrow}\ F\ {\isasymin}\ {\isacharquery}A\ {\isasymlongrightarrow}\ {\isacharbraceleft}G{\isacharbraceright}\ {\isasymunion}\ {\isacharquery}A\ {\isasymin}\ C\ {\isasymor}\ {\isacharbraceleft}H{\isacharbraceright}\ {\isasymunion}\ {\isacharquery}A\ {\isasymin}\ C{\isacharparenright}{\isachardoublequoteclose}\isanewline
\ \ \ \ \ \ \ \ \isacommand{using}\isamarkupfalse%
\ PCP\ {\isacartoucheopen}{\isacharquery}A\ {\isasymin}\ C{\isacartoucheclose}\ \isacommand{by}\isamarkupfalse%
\ {\isacharparenleft}rule\ bspec{\isacharparenright}\isanewline
\ \ \ \ \ \ \isacommand{then}\isamarkupfalse%
\ \isacommand{have}\isamarkupfalse%
\ {\isachardoublequoteopen}{\isasymforall}k{\isachardot}\ Atom\ k\ {\isasymin}\ {\isacharquery}A\ {\isasymlongrightarrow}\ \isactrlbold {\isasymnot}\ {\isacharparenleft}Atom\ k{\isacharparenright}\ {\isasymin}\ {\isacharquery}A\ {\isasymlongrightarrow}\ False{\isachardoublequoteclose}\isanewline
\ \ \ \ \ \ \ \ \isacommand{by}\isamarkupfalse%
\ {\isacharparenleft}iprover\ elim{\isacharcolon}\ conjunct{\isadigit{2}}\ conjunct{\isadigit{1}}{\isacharparenright}\isanewline
\ \ \ \ \ \ \isacommand{then}\isamarkupfalse%
\ \isacommand{have}\isamarkupfalse%
\ {\isachardoublequoteopen}Atom\ k\ {\isasymin}\ {\isacharquery}A\ {\isasymlongrightarrow}\ \isactrlbold {\isasymnot}\ {\isacharparenleft}Atom\ k{\isacharparenright}\ {\isasymin}\ {\isacharquery}A\ {\isasymlongrightarrow}\ False{\isachardoublequoteclose}\isanewline
\ \ \ \ \ \ \ \ \isacommand{by}\isamarkupfalse%
\ {\isacharparenleft}iprover\ elim{\isacharcolon}\ allE{\isacharparenright}\isanewline
\ \ \ \ \ \ \isacommand{then}\isamarkupfalse%
\ \isacommand{have}\isamarkupfalse%
\ {\isachardoublequoteopen}\isactrlbold {\isasymnot}{\isacharparenleft}Atom\ k{\isacharparenright}\ {\isasymin}\ {\isacharquery}A\ {\isasymlongrightarrow}\ False{\isachardoublequoteclose}\isanewline
\ \ \ \ \ \ \ \ \isacommand{using}\isamarkupfalse%
\ {\isacartoucheopen}Atom\ k\ {\isasymin}\ {\isacharquery}A{\isacartoucheclose}\ \isacommand{by}\isamarkupfalse%
\ {\isacharparenleft}rule\ mp{\isacharparenright}\isanewline
\ \ \ \ \ \ \isacommand{thus}\isamarkupfalse%
\ {\isachardoublequoteopen}False{\isachardoublequoteclose}\isanewline
\ \ \ \ \ \ \ \ \isacommand{using}\isamarkupfalse%
\ {\isacartoucheopen}\isactrlbold {\isasymnot}{\isacharparenleft}Atom\ k{\isacharparenright}\ {\isasymin}\ {\isacharquery}A{\isacartoucheclose}\ \isacommand{by}\isamarkupfalse%
\ {\isacharparenleft}rule\ mp{\isacharparenright}\isanewline
\ \ \ \ \isacommand{qed}\isamarkupfalse%
\isanewline
\ \ \isacommand{qed}\isamarkupfalse%
\isanewline
\ \ \isacommand{have}\isamarkupfalse%
\ C{\isadigit{3}}{\isacharcolon}{\isachardoublequoteopen}{\isasymforall}F\ G\ H{\isachardot}\ Con\ F\ G\ H\ {\isasymlongrightarrow}\ F\ {\isasymin}\ S\ {\isasymlongrightarrow}\ {\isacharbraceleft}G{\isacharcomma}H{\isacharbraceright}\ {\isasymunion}\ S\ {\isasymin}\ {\isacharparenleft}extensionFin\ C{\isacharparenright}{\isachardoublequoteclose}\isanewline
\ \ \isacommand{proof}\isamarkupfalse%
\ {\isacharparenleft}rule\ allI{\isacharparenright}{\isacharplus}\isanewline
\ \ \ \ \isacommand{fix}\isamarkupfalse%
\ F\ G\ H\isanewline
\ \ \ \ \isacommand{show}\isamarkupfalse%
\ {\isachardoublequoteopen}Con\ F\ G\ H\ {\isasymlongrightarrow}\ F\ {\isasymin}\ S\ {\isasymlongrightarrow}\ {\isacharbraceleft}G{\isacharcomma}H{\isacharbraceright}\ {\isasymunion}\ S\ {\isasymin}\ {\isacharparenleft}extensionFin\ C{\isacharparenright}{\isachardoublequoteclose}\isanewline
\ \ \ \ \isacommand{proof}\isamarkupfalse%
\ {\isacharparenleft}rule\ impI{\isacharparenright}{\isacharplus}\isanewline
\ \ \ \ \ \ \isacommand{assume}\isamarkupfalse%
\ {\isachardoublequoteopen}Con\ F\ G\ H{\isachardoublequoteclose}\isanewline
\ \ \ \ \ \ \isacommand{assume}\isamarkupfalse%
\ {\isachardoublequoteopen}F\ {\isasymin}\ S{\isachardoublequoteclose}\ \isanewline
\ \ \ \ \ \ \isacommand{show}\isamarkupfalse%
\ {\isachardoublequoteopen}{\isacharbraceleft}G{\isacharcomma}H{\isacharbraceright}\ {\isasymunion}\ S\ {\isasymin}\ {\isacharparenleft}extensionFin\ C{\isacharparenright}{\isachardoublequoteclose}\ \isanewline
\ \ \ \ \ \ \ \ \isacommand{using}\isamarkupfalse%
\ assms{\isacharparenleft}{\isadigit{1}}{\isacharparenright}\ assms{\isacharparenleft}{\isadigit{2}}{\isacharparenright}\ assms{\isacharparenleft}{\isadigit{3}}{\isacharparenright}\ {\isacartoucheopen}Con\ F\ G\ H{\isacartoucheclose}\ {\isacartoucheopen}F\ {\isasymin}\ S{\isacartoucheclose}\ \isacommand{by}\isamarkupfalse%
\ {\isacharparenleft}simp\ only{\isacharcolon}\ ex{\isadigit{3}}{\isacharunderscore}pcp{\isacharunderscore}SinE{\isacharunderscore}CON{\isacharparenright}\isanewline
\ \ \ \ \isacommand{qed}\isamarkupfalse%
\isanewline
\ \ \isacommand{qed}\isamarkupfalse%
\isanewline
\ \ \isacommand{have}\isamarkupfalse%
\ C{\isadigit{4}}{\isacharcolon}{\isachardoublequoteopen}{\isasymforall}F\ G\ H{\isachardot}\ Dis\ F\ G\ H\ {\isasymlongrightarrow}\ F\ {\isasymin}\ S\ {\isasymlongrightarrow}\ {\isacharbraceleft}G{\isacharbraceright}\ {\isasymunion}\ S\ {\isasymin}\ {\isacharparenleft}extensionFin\ C{\isacharparenright}\ {\isasymor}\ {\isacharbraceleft}H{\isacharbraceright}\ {\isasymunion}\ S\ {\isasymin}\ {\isacharparenleft}extensionFin\ C{\isacharparenright}{\isachardoublequoteclose}\isanewline
\ \ \isacommand{proof}\isamarkupfalse%
\ {\isacharparenleft}rule\ allI{\isacharparenright}{\isacharplus}\isanewline
\ \ \ \ \isacommand{fix}\isamarkupfalse%
\ F\ G\ H\isanewline
\ \ \ \ \isacommand{show}\isamarkupfalse%
\ {\isachardoublequoteopen}Dis\ F\ G\ H\ {\isasymlongrightarrow}\ F\ {\isasymin}\ S\ {\isasymlongrightarrow}\ {\isacharbraceleft}G{\isacharbraceright}\ {\isasymunion}\ S\ {\isasymin}\ {\isacharparenleft}extensionFin\ C{\isacharparenright}\ {\isasymor}\ {\isacharbraceleft}H{\isacharbraceright}\ {\isasymunion}\ S\ {\isasymin}\ {\isacharparenleft}extensionFin\ C{\isacharparenright}{\isachardoublequoteclose}\isanewline
\ \ \ \ \isacommand{proof}\isamarkupfalse%
\ {\isacharparenleft}rule\ impI{\isacharparenright}{\isacharplus}\isanewline
\ \ \ \ \ \ \isacommand{assume}\isamarkupfalse%
\ {\isachardoublequoteopen}Dis\ F\ G\ H{\isachardoublequoteclose}\isanewline
\ \ \ \ \ \ \isacommand{assume}\isamarkupfalse%
\ {\isachardoublequoteopen}F\ {\isasymin}\ S{\isachardoublequoteclose}\ \isanewline
\ \ \ \ \ \ \isacommand{show}\isamarkupfalse%
\ {\isachardoublequoteopen}{\isacharbraceleft}G{\isacharbraceright}\ {\isasymunion}\ S\ {\isasymin}\ {\isacharparenleft}extensionFin\ C{\isacharparenright}\ {\isasymor}\ {\isacharbraceleft}H{\isacharbraceright}\ {\isasymunion}\ S\ {\isasymin}\ {\isacharparenleft}extensionFin\ C{\isacharparenright}{\isachardoublequoteclose}\isanewline
\ \ \ \ \ \ \ \ \isacommand{using}\isamarkupfalse%
\ assms{\isacharparenleft}{\isadigit{1}}{\isacharparenright}\ assms{\isacharparenleft}{\isadigit{2}}{\isacharparenright}\ assms{\isacharparenleft}{\isadigit{3}}{\isacharparenright}\ {\isacartoucheopen}Dis\ F\ G\ H{\isacartoucheclose}\ {\isacartoucheopen}F\ {\isasymin}\ S{\isacartoucheclose}\ \isacommand{by}\isamarkupfalse%
\ {\isacharparenleft}rule\ ex{\isadigit{3}}{\isacharunderscore}pcp{\isacharunderscore}SinE{\isacharunderscore}DIS{\isacharparenright}\isanewline
\ \ \ \ \isacommand{qed}\isamarkupfalse%
\isanewline
\ \ \isacommand{qed}\isamarkupfalse%
\isanewline
\ \ \isacommand{show}\isamarkupfalse%
\ {\isacharquery}thesis\isanewline
\ \ \ \ \isacommand{using}\isamarkupfalse%
\ C{\isadigit{1}}\ C{\isadigit{2}}\ C{\isadigit{3}}\ C{\isadigit{4}}\ \isacommand{by}\isamarkupfalse%
\ {\isacharparenleft}iprover\ intro{\isacharcolon}\ conjI{\isacharparenright}\isanewline
\isacommand{qed}\isamarkupfalse%
%
\endisatagproof
{\isafoldproof}%
%
\isadelimproof
%
\endisadelimproof
%
\begin{isamarkuptext}%
En conclusión, la prueba detallada completa en Isabelle que demuestra que la extensión \isa{C{\isacharprime}} 
  verifica la propiedad de consistencia proposicional dada una colección \isa{C} que también la
  verifique y sea cerrada bajo subconjuntos es la siguiente.%
\end{isamarkuptext}\isamarkuptrue%
\isacommand{lemma}\isamarkupfalse%
\ ex{\isadigit{3}}{\isacharunderscore}pcp{\isacharcolon}\isanewline
\ \ \isakeyword{assumes}\ {\isachardoublequoteopen}pcp\ C{\isachardoublequoteclose}\isanewline
\ \ \ \ \ \ \ \ \ \ {\isachardoublequoteopen}subset{\isacharunderscore}closed\ C{\isachardoublequoteclose}\isanewline
\ \ \ \ \ \ \ \ \isakeyword{shows}\ {\isachardoublequoteopen}pcp\ {\isacharparenleft}extensionFin\ C{\isacharparenright}{\isachardoublequoteclose}\isanewline
%
\isadelimproof
\ \ %
\endisadelimproof
%
\isatagproof
\isacommand{unfolding}\isamarkupfalse%
\ pcp{\isacharunderscore}alt\isanewline
\isacommand{proof}\isamarkupfalse%
\ {\isacharparenleft}rule\ ballI{\isacharparenright}\isanewline
\ \ \isacommand{have}\isamarkupfalse%
\ PCP{\isacharcolon}{\isachardoublequoteopen}{\isasymforall}S\ {\isasymin}\ C{\isachardot}\isanewline
\ \ \ \ {\isasymbottom}\ {\isasymnotin}\ S\isanewline
\ \ \ \ {\isasymand}\ {\isacharparenleft}{\isasymforall}k{\isachardot}\ Atom\ k\ {\isasymin}\ S\ {\isasymlongrightarrow}\ \isactrlbold {\isasymnot}\ {\isacharparenleft}Atom\ k{\isacharparenright}\ {\isasymin}\ S\ {\isasymlongrightarrow}\ False{\isacharparenright}\isanewline
\ \ \ \ {\isasymand}\ {\isacharparenleft}{\isasymforall}F\ G\ H{\isachardot}\ Con\ F\ G\ H\ {\isasymlongrightarrow}\ F\ {\isasymin}\ S\ {\isasymlongrightarrow}\ {\isacharbraceleft}G{\isacharcomma}H{\isacharbraceright}\ {\isasymunion}\ S\ {\isasymin}\ C{\isacharparenright}\isanewline
\ \ \ \ {\isasymand}\ {\isacharparenleft}{\isasymforall}F\ G\ H{\isachardot}\ Dis\ F\ G\ H\ {\isasymlongrightarrow}\ F\ {\isasymin}\ S\ {\isasymlongrightarrow}\ {\isacharbraceleft}G{\isacharbraceright}\ {\isasymunion}\ S\ {\isasymin}\ C\ {\isasymor}\ {\isacharbraceleft}H{\isacharbraceright}\ {\isasymunion}\ S\ {\isasymin}\ C{\isacharparenright}{\isachardoublequoteclose}\isanewline
\ \ \ \ \isacommand{using}\isamarkupfalse%
\ assms{\isacharparenleft}{\isadigit{1}}{\isacharparenright}\ \isacommand{by}\isamarkupfalse%
\ {\isacharparenleft}rule\ pcp{\isacharunderscore}alt{\isadigit{1}}{\isacharparenright}\isanewline
\ \ \isacommand{fix}\isamarkupfalse%
\ S\isanewline
\ \ \isacommand{assume}\isamarkupfalse%
\ {\isachardoublequoteopen}S\ {\isasymin}\ {\isacharparenleft}extensionFin\ C{\isacharparenright}{\isachardoublequoteclose}\isanewline
\ \ \isacommand{then}\isamarkupfalse%
\ \isacommand{have}\isamarkupfalse%
\ {\isachardoublequoteopen}S\ {\isasymin}\ C\ {\isasymor}\ S\ {\isasymin}\ {\isacharparenleft}extF\ C{\isacharparenright}{\isachardoublequoteclose}\isanewline
\ \ \ \ \isacommand{unfolding}\isamarkupfalse%
\ extensionFin\ \isacommand{by}\isamarkupfalse%
\ {\isacharparenleft}simp\ only{\isacharcolon}\ Un{\isacharunderscore}iff{\isacharparenright}\isanewline
\ \ \isacommand{thus}\isamarkupfalse%
\ {\isachardoublequoteopen}{\isasymbottom}\ {\isasymnotin}\ S\ {\isasymand}\isanewline
\ \ \ \ \ \ \ \ \ {\isacharparenleft}{\isasymforall}k{\isachardot}\ Atom\ k\ {\isasymin}\ S\ {\isasymlongrightarrow}\ \isactrlbold {\isasymnot}\ {\isacharparenleft}Atom\ k{\isacharparenright}\ {\isasymin}\ S\ {\isasymlongrightarrow}\ False{\isacharparenright}\ {\isasymand}\isanewline
\ \ \ \ \ \ \ \ \ {\isacharparenleft}{\isasymforall}F\ G\ H{\isachardot}\ Con\ F\ G\ H\ {\isasymlongrightarrow}\ F\ {\isasymin}\ S\ {\isasymlongrightarrow}\ {\isacharbraceleft}G{\isacharcomma}\ H{\isacharbraceright}\ {\isasymunion}\ S\ {\isasymin}\ {\isacharparenleft}extensionFin\ C{\isacharparenright}{\isacharparenright}\ {\isasymand}\isanewline
\ \ \ \ \ \ \ \ \ {\isacharparenleft}{\isasymforall}F\ G\ H{\isachardot}\ Dis\ F\ G\ H\ {\isasymlongrightarrow}\ F\ {\isasymin}\ S\ {\isasymlongrightarrow}\ {\isacharbraceleft}G{\isacharbraceright}\ {\isasymunion}\ S\ {\isasymin}\ {\isacharparenleft}extensionFin\ C{\isacharparenright}\ {\isasymor}\ {\isacharbraceleft}H{\isacharbraceright}\ {\isasymunion}\ S\ {\isasymin}\ {\isacharparenleft}extensionFin\ C{\isacharparenright}{\isacharparenright}{\isachardoublequoteclose}\isanewline
\ \ \isacommand{proof}\isamarkupfalse%
\ {\isacharparenleft}rule\ disjE{\isacharparenright}\isanewline
\ \ \ \ \isacommand{assume}\isamarkupfalse%
\ {\isachardoublequoteopen}S\ {\isasymin}\ C{\isachardoublequoteclose}\isanewline
\ \ \ \ \isacommand{show}\isamarkupfalse%
\ {\isacharquery}thesis\isanewline
\ \ \ \ \ \ \isacommand{using}\isamarkupfalse%
\ assms\ {\isacartoucheopen}S\ {\isasymin}\ C{\isacartoucheclose}\ \isacommand{by}\isamarkupfalse%
\ {\isacharparenleft}rule\ ex{\isadigit{3}}{\isacharunderscore}pcp{\isacharunderscore}SinC{\isacharparenright}\isanewline
\ \ \isacommand{next}\isamarkupfalse%
\isanewline
\ \ \ \ \isacommand{assume}\isamarkupfalse%
\ {\isachardoublequoteopen}S\ {\isasymin}\ {\isacharparenleft}extF\ C{\isacharparenright}{\isachardoublequoteclose}\isanewline
\ \ \ \ \isacommand{show}\isamarkupfalse%
\ {\isacharquery}thesis\isanewline
\ \ \ \ \ \ \isacommand{using}\isamarkupfalse%
\ assms\ {\isacartoucheopen}S\ {\isasymin}\ {\isacharparenleft}extF\ C{\isacharparenright}{\isacartoucheclose}\ \isacommand{by}\isamarkupfalse%
\ {\isacharparenleft}rule\ ex{\isadigit{3}}{\isacharunderscore}pcp{\isacharunderscore}SinE{\isacharparenright}\isanewline
\ \ \isacommand{qed}\isamarkupfalse%
\isanewline
\isacommand{qed}\isamarkupfalse%
%
\endisatagproof
{\isafoldproof}%
%
\isadelimproof
%
\endisadelimproof
%
\begin{isamarkuptext}%
Por último, podemos dar la prueba completa del lema \isa{{\isadigit{3}}{\isachardot}{\isadigit{0}}{\isachardot}{\isadigit{5}}} en Isabelle.%
\end{isamarkuptext}\isamarkuptrue%
\isacommand{lemma}\isamarkupfalse%
\ ex{\isadigit{3}}{\isacharcolon}\isanewline
\ \ \isakeyword{assumes}\ {\isachardoublequoteopen}pcp\ C{\isachardoublequoteclose}\isanewline
\ \ \ \ \ \ \ \ \ \ {\isachardoublequoteopen}subset{\isacharunderscore}closed\ C{\isachardoublequoteclose}\isanewline
\ \ \isakeyword{shows}\ {\isachardoublequoteopen}{\isasymexists}C{\isacharprime}{\isachardot}\ C\ {\isasymsubseteq}\ C{\isacharprime}\ {\isasymand}\ pcp\ C{\isacharprime}\ {\isasymand}\ finite{\isacharunderscore}character\ C{\isacharprime}{\isachardoublequoteclose}\isanewline
%
\isadelimproof
%
\endisadelimproof
%
\isatagproof
\isacommand{proof}\isamarkupfalse%
\ {\isacharminus}\isanewline
\ \ \isacommand{let}\isamarkupfalse%
\ {\isacharquery}C{\isacharprime}{\isacharequal}{\isachardoublequoteopen}extensionFin\ C{\isachardoublequoteclose}\isanewline
\ \ \isacommand{have}\isamarkupfalse%
\ C{\isadigit{1}}{\isacharcolon}{\isachardoublequoteopen}C\ {\isasymsubseteq}\ {\isacharquery}C{\isacharprime}{\isachardoublequoteclose}\isanewline
\ \ \ \ \isacommand{unfolding}\isamarkupfalse%
\ extensionFin\ \isacommand{by}\isamarkupfalse%
\ {\isacharparenleft}simp\ only{\isacharcolon}\ Un{\isacharunderscore}upper{\isadigit{1}}{\isacharparenright}\isanewline
\ \ \isacommand{have}\isamarkupfalse%
\ C{\isadigit{2}}{\isacharcolon}{\isachardoublequoteopen}finite{\isacharunderscore}character\ {\isacharparenleft}{\isacharquery}C{\isacharprime}{\isacharparenright}{\isachardoublequoteclose}\isanewline
\ \ \ \ \isacommand{using}\isamarkupfalse%
\ assms{\isacharparenleft}{\isadigit{2}}{\isacharparenright}\ \isacommand{by}\isamarkupfalse%
\ {\isacharparenleft}rule\ ex{\isadigit{3}}{\isacharunderscore}finite{\isacharunderscore}character{\isacharparenright}\isanewline
\ \ \isacommand{have}\isamarkupfalse%
\ C{\isadigit{3}}{\isacharcolon}{\isachardoublequoteopen}pcp\ {\isacharparenleft}{\isacharquery}C{\isacharprime}{\isacharparenright}{\isachardoublequoteclose}\isanewline
\ \ \ \ \isacommand{using}\isamarkupfalse%
\ assms\ \isacommand{by}\isamarkupfalse%
\ {\isacharparenleft}rule\ ex{\isadigit{3}}{\isacharunderscore}pcp{\isacharparenright}\isanewline
\ \ \isacommand{have}\isamarkupfalse%
\ {\isachardoublequoteopen}C\ {\isasymsubseteq}\ {\isacharquery}C{\isacharprime}\ {\isasymand}\ pcp\ {\isacharquery}C{\isacharprime}\ {\isasymand}\ finite{\isacharunderscore}character\ {\isacharquery}C{\isacharprime}{\isachardoublequoteclose}\isanewline
\ \ \ \ \isacommand{using}\isamarkupfalse%
\ C{\isadigit{1}}\ C{\isadigit{2}}\ C{\isadigit{3}}\ \isacommand{by}\isamarkupfalse%
\ {\isacharparenleft}iprover\ intro{\isacharcolon}\ conjI{\isacharparenright}\isanewline
\ \ \isacommand{thus}\isamarkupfalse%
\ {\isacharquery}thesis\isanewline
\ \ \ \ \isacommand{by}\isamarkupfalse%
\ {\isacharparenleft}rule\ exI{\isacharparenright}\isanewline
\isacommand{qed}\isamarkupfalse%
\isanewline
%
\endisatagproof
{\isafoldproof}%
%
\isadelimproof
%
\endisadelimproof
%
\isadelimtheory
%
\endisadelimtheory
%
\isatagtheory
%
\endisatagtheory
{\isafoldtheory}%
%
\isadelimtheory
%
\endisadelimtheory
%
\end{isabellebody}%
\endinput
%:%file=~/TFM/TFM/ColecScfc.thy%:%
%:%19=12%:%
%:%20=13%:%
%:%21=14%:%
%:%22=15%:%
%:%23=16%:%
%:%24=17%:%
%:%25=18%:%
%:%26=19%:%
%:%27=20%:%
%:%28=21%:%
%:%29=22%:%
%:%30=23%:%
%:%31=24%:%
%:%32=25%:%
%:%33=26%:%
%:%35=28%:%
%:%36=28%:%
%:%38=30%:%
%:%39=31%:%
%:%40=32%:%
%:%42=34%:%
%:%43=34%:%
%:%46=35%:%
%:%50=35%:%
%:%51=35%:%
%:%52=35%:%
%:%57=35%:%
%:%60=36%:%
%:%61=37%:%
%:%62=37%:%
%:%65=38%:%
%:%69=38%:%
%:%70=38%:%
%:%71=38%:%
%:%80=40%:%
%:%81=41%:%
%:%82=42%:%
%:%84=44%:%
%:%85=44%:%
%:%88=45%:%
%:%92=45%:%
%:%93=45%:%
%:%94=45%:%
%:%103=47%:%
%:%104=48%:%
%:%106=50%:%
%:%107=50%:%
%:%109=52%:%
%:%112=53%:%
%:%116=53%:%
%:%117=53%:%
%:%118=53%:%
%:%123=53%:%
%:%126=54%:%
%:%127=55%:%
%:%128=55%:%
%:%129=56%:%
%:%132=57%:%
%:%136=57%:%
%:%137=57%:%
%:%138=57%:%
%:%147=59%:%
%:%148=60%:%
%:%149=61%:%
%:%150=62%:%
%:%151=63%:%
%:%152=64%:%
%:%153=65%:%
%:%154=66%:%
%:%155=67%:%
%:%156=68%:%
%:%157=69%:%
%:%158=70%:%
%:%160=72%:%
%:%161=72%:%
%:%164=75%:%
%:%165=76%:%
%:%166=77%:%
%:%167=78%:%
%:%169=80%:%
%:%170=80%:%
%:%173=81%:%
%:%177=81%:%
%:%178=81%:%
%:%179=81%:%
%:%184=81%:%
%:%187=82%:%
%:%188=83%:%
%:%189=83%:%
%:%192=84%:%
%:%196=84%:%
%:%197=84%:%
%:%198=84%:%
%:%203=84%:%
%:%206=85%:%
%:%207=86%:%
%:%208=86%:%
%:%211=87%:%
%:%215=87%:%
%:%216=87%:%
%:%217=87%:%
%:%226=89%:%
%:%227=90%:%
%:%228=91%:%
%:%229=92%:%
%:%230=93%:%
%:%231=94%:%
%:%232=95%:%
%:%233=96%:%
%:%234=97%:%
%:%235=98%:%
%:%236=99%:%
%:%237=100%:%
%:%238=101%:%
%:%239=102%:%
%:%240=103%:%
%:%241=104%:%
%:%243=106%:%
%:%244=106%:%
%:%247=107%:%
%:%251=107%:%
%:%261=109%:%
%:%262=110%:%
%:%263=111%:%
%:%264=112%:%
%:%265=113%:%
%:%266=114%:%
%:%267=115%:%
%:%268=116%:%
%:%269=117%:%
%:%270=118%:%
%:%271=119%:%
%:%272=120%:%
%:%273=121%:%
%:%274=122%:%
%:%275=123%:%
%:%276=124%:%
%:%277=125%:%
%:%278=126%:%
%:%279=127%:%
%:%280=128%:%
%:%281=129%:%
%:%282=130%:%
%:%283=131%:%
%:%284=132%:%
%:%285=133%:%
%:%286=134%:%
%:%287=135%:%
%:%288=136%:%
%:%289=137%:%
%:%290=138%:%
%:%291=139%:%
%:%292=140%:%
%:%293=141%:%
%:%294=142%:%
%:%295=143%:%
%:%296=144%:%
%:%297=145%:%
%:%298=146%:%
%:%299=147%:%
%:%300=148%:%
%:%301=149%:%
%:%302=150%:%
%:%303=151%:%
%:%304=152%:%
%:%305=153%:%
%:%306=154%:%
%:%307=155%:%
%:%308=156%:%
%:%309=157%:%
%:%310=158%:%
%:%311=159%:%
%:%312=160%:%
%:%313=161%:%
%:%314=162%:%
%:%315=163%:%
%:%316=164%:%
%:%317=165%:%
%:%318=166%:%
%:%319=167%:%
%:%320=168%:%
%:%321=169%:%
%:%322=170%:%
%:%323=171%:%
%:%324=172%:%
%:%325=173%:%
%:%326=174%:%
%:%327=175%:%
%:%328=176%:%
%:%329=177%:%
%:%330=178%:%
%:%331=179%:%
%:%332=180%:%
%:%333=181%:%
%:%334=182%:%
%:%335=183%:%
%:%336=184%:%
%:%337=185%:%
%:%338=186%:%
%:%339=187%:%
%:%340=188%:%
%:%341=189%:%
%:%342=190%:%
%:%343=191%:%
%:%344=192%:%
%:%345=193%:%
%:%346=194%:%
%:%347=195%:%
%:%348=196%:%
%:%349=197%:%
%:%350=198%:%
%:%351=199%:%
%:%352=200%:%
%:%354=202%:%
%:%355=202%:%
%:%358=203%:%
%:%362=203%:%
%:%363=203%:%
%:%372=205%:%
%:%373=206%:%
%:%375=208%:%
%:%376=208%:%
%:%379=209%:%
%:%383=209%:%
%:%384=209%:%
%:%393=211%:%
%:%394=212%:%
%:%395=213%:%
%:%396=214%:%
%:%397=215%:%
%:%399=217%:%
%:%400=217%:%
%:%401=218%:%
%:%403=220%:%
%:%404=221%:%
%:%406=223%:%
%:%407=223%:%
%:%414=224%:%
%:%415=224%:%
%:%416=225%:%
%:%417=225%:%
%:%418=226%:%
%:%419=226%:%
%:%420=227%:%
%:%421=227%:%
%:%422=228%:%
%:%423=228%:%
%:%424=229%:%
%:%425=229%:%
%:%426=229%:%
%:%427=230%:%
%:%428=230%:%
%:%429=230%:%
%:%430=231%:%
%:%431=231%:%
%:%432=232%:%
%:%433=232%:%
%:%434=233%:%
%:%444=235%:%
%:%445=236%:%
%:%446=237%:%
%:%447=238%:%
%:%449=240%:%
%:%450=240%:%
%:%453=241%:%
%:%457=241%:%
%:%458=241%:%
%:%467=243%:%
%:%468=244%:%
%:%470=246%:%
%:%471=246%:%
%:%472=247%:%
%:%473=248%:%
%:%480=249%:%
%:%481=249%:%
%:%482=250%:%
%:%483=250%:%
%:%484=251%:%
%:%485=251%:%
%:%486=252%:%
%:%487=252%:%
%:%488=253%:%
%:%489=253%:%
%:%490=254%:%
%:%491=254%:%
%:%494=257%:%
%:%495=258%:%
%:%496=258%:%
%:%497=259%:%
%:%498=259%:%
%:%499=260%:%
%:%500=260%:%
%:%501=261%:%
%:%502=261%:%
%:%503=261%:%
%:%504=262%:%
%:%505=262%:%
%:%506=262%:%
%:%507=263%:%
%:%508=263%:%
%:%509=264%:%
%:%510=264%:%
%:%511=264%:%
%:%512=265%:%
%:%513=265%:%
%:%517=269%:%
%:%518=270%:%
%:%519=270%:%
%:%520=270%:%
%:%521=271%:%
%:%522=271%:%
%:%523=271%:%
%:%526=274%:%
%:%527=275%:%
%:%528=275%:%
%:%529=275%:%
%:%530=276%:%
%:%531=276%:%
%:%532=276%:%
%:%533=277%:%
%:%534=277%:%
%:%535=278%:%
%:%536=278%:%
%:%537=279%:%
%:%538=279%:%
%:%539=279%:%
%:%540=280%:%
%:%541=280%:%
%:%542=281%:%
%:%543=281%:%
%:%544=281%:%
%:%545=282%:%
%:%546=282%:%
%:%547=283%:%
%:%548=283%:%
%:%549=284%:%
%:%550=284%:%
%:%551=285%:%
%:%552=285%:%
%:%553=286%:%
%:%554=286%:%
%:%555=287%:%
%:%556=287%:%
%:%557=288%:%
%:%558=288%:%
%:%559=289%:%
%:%560=289%:%
%:%561=290%:%
%:%562=290%:%
%:%563=290%:%
%:%564=291%:%
%:%565=291%:%
%:%566=292%:%
%:%567=292%:%
%:%568=292%:%
%:%569=293%:%
%:%570=293%:%
%:%571=294%:%
%:%572=294%:%
%:%573=294%:%
%:%574=295%:%
%:%575=295%:%
%:%576=295%:%
%:%577=296%:%
%:%578=296%:%
%:%579=296%:%
%:%580=297%:%
%:%581=297%:%
%:%582=298%:%
%:%583=298%:%
%:%584=298%:%
%:%585=299%:%
%:%586=299%:%
%:%587=300%:%
%:%588=300%:%
%:%589=301%:%
%:%590=301%:%
%:%591=302%:%
%:%592=302%:%
%:%593=302%:%
%:%594=303%:%
%:%595=303%:%
%:%596=304%:%
%:%597=304%:%
%:%598=305%:%
%:%599=305%:%
%:%600=306%:%
%:%601=306%:%
%:%602=307%:%
%:%603=307%:%
%:%604=308%:%
%:%605=308%:%
%:%606=309%:%
%:%607=309%:%
%:%608=310%:%
%:%609=310%:%
%:%610=311%:%
%:%611=311%:%
%:%612=311%:%
%:%613=312%:%
%:%614=312%:%
%:%615=313%:%
%:%616=313%:%
%:%617=313%:%
%:%618=314%:%
%:%619=314%:%
%:%620=314%:%
%:%621=315%:%
%:%622=315%:%
%:%623=315%:%
%:%624=316%:%
%:%625=316%:%
%:%626=316%:%
%:%627=317%:%
%:%628=317%:%
%:%629=317%:%
%:%630=318%:%
%:%631=318%:%
%:%632=319%:%
%:%633=319%:%
%:%634=319%:%
%:%635=320%:%
%:%636=320%:%
%:%637=320%:%
%:%638=321%:%
%:%639=321%:%
%:%640=322%:%
%:%641=322%:%
%:%642=322%:%
%:%643=323%:%
%:%644=323%:%
%:%645=324%:%
%:%646=324%:%
%:%647=324%:%
%:%648=325%:%
%:%649=325%:%
%:%650=326%:%
%:%651=326%:%
%:%652=327%:%
%:%653=327%:%
%:%654=328%:%
%:%655=328%:%
%:%656=329%:%
%:%657=329%:%
%:%658=330%:%
%:%659=330%:%
%:%660=331%:%
%:%661=331%:%
%:%662=331%:%
%:%663=332%:%
%:%664=332%:%
%:%665=332%:%
%:%666=333%:%
%:%667=333%:%
%:%668=333%:%
%:%669=334%:%
%:%670=334%:%
%:%671=335%:%
%:%672=335%:%
%:%673=335%:%
%:%674=336%:%
%:%675=336%:%
%:%676=337%:%
%:%677=337%:%
%:%678=338%:%
%:%679=338%:%
%:%680=339%:%
%:%681=339%:%
%:%682=339%:%
%:%683=340%:%
%:%684=340%:%
%:%685=341%:%
%:%686=341%:%
%:%687=342%:%
%:%688=342%:%
%:%689=343%:%
%:%690=343%:%
%:%691=344%:%
%:%692=344%:%
%:%693=345%:%
%:%694=345%:%
%:%695=346%:%
%:%696=346%:%
%:%697=347%:%
%:%698=347%:%
%:%699=348%:%
%:%700=348%:%
%:%701=348%:%
%:%702=349%:%
%:%703=349%:%
%:%704=350%:%
%:%705=350%:%
%:%706=350%:%
%:%707=351%:%
%:%708=351%:%
%:%709=351%:%
%:%710=352%:%
%:%711=352%:%
%:%712=352%:%
%:%713=353%:%
%:%714=353%:%
%:%715=353%:%
%:%716=354%:%
%:%717=354%:%
%:%718=354%:%
%:%719=355%:%
%:%720=355%:%
%:%721=356%:%
%:%722=356%:%
%:%723=357%:%
%:%724=357%:%
%:%725=358%:%
%:%726=358%:%
%:%727=359%:%
%:%728=359%:%
%:%729=360%:%
%:%730=360%:%
%:%731=360%:%
%:%732=361%:%
%:%733=361%:%
%:%734=362%:%
%:%735=362%:%
%:%736=363%:%
%:%737=363%:%
%:%738=364%:%
%:%739=364%:%
%:%740=365%:%
%:%741=365%:%
%:%742=366%:%
%:%743=366%:%
%:%744=366%:%
%:%745=367%:%
%:%746=367%:%
%:%747=367%:%
%:%748=368%:%
%:%749=368%:%
%:%750=368%:%
%:%751=369%:%
%:%752=369%:%
%:%753=369%:%
%:%754=370%:%
%:%755=370%:%
%:%756=370%:%
%:%757=371%:%
%:%758=371%:%
%:%759=372%:%
%:%760=372%:%
%:%761=373%:%
%:%762=373%:%
%:%763=374%:%
%:%764=374%:%
%:%765=375%:%
%:%766=375%:%
%:%767=376%:%
%:%768=376%:%
%:%769=376%:%
%:%770=377%:%
%:%771=377%:%
%:%772=378%:%
%:%773=378%:%
%:%774=379%:%
%:%775=379%:%
%:%776=380%:%
%:%777=380%:%
%:%778=381%:%
%:%779=381%:%
%:%780=381%:%
%:%781=382%:%
%:%782=382%:%
%:%783=382%:%
%:%784=383%:%
%:%785=383%:%
%:%786=383%:%
%:%787=384%:%
%:%788=384%:%
%:%789=384%:%
%:%790=385%:%
%:%791=385%:%
%:%792=385%:%
%:%793=386%:%
%:%794=386%:%
%:%795=386%:%
%:%796=387%:%
%:%797=387%:%
%:%798=388%:%
%:%799=388%:%
%:%800=389%:%
%:%801=389%:%
%:%802=390%:%
%:%803=390%:%
%:%804=391%:%
%:%805=391%:%
%:%806=392%:%
%:%807=392%:%
%:%810=395%:%
%:%811=396%:%
%:%812=396%:%
%:%813=396%:%
%:%814=397%:%
%:%815=397%:%
%:%816=398%:%
%:%817=398%:%
%:%818=399%:%
%:%828=401%:%
%:%830=403%:%
%:%831=403%:%
%:%832=404%:%
%:%833=405%:%
%:%836=406%:%
%:%840=406%:%
%:%841=406%:%
%:%842=407%:%
%:%843=407%:%
%:%844=408%:%
%:%845=408%:%
%:%846=409%:%
%:%847=409%:%
%:%848=410%:%
%:%849=410%:%
%:%850=410%:%
%:%851=411%:%
%:%852=411%:%
%:%853=411%:%
%:%854=412%:%
%:%855=412%:%
%:%856=413%:%
%:%857=413%:%
%:%858=413%:%
%:%859=414%:%
%:%860=414%:%
%:%861=415%:%
%:%862=415%:%
%:%863=416%:%
%:%864=416%:%
%:%865=417%:%
%:%866=417%:%
%:%867=418%:%
%:%868=418%:%
%:%869=418%:%
%:%870=419%:%
%:%871=419%:%
%:%872=419%:%
%:%873=420%:%
%:%874=420%:%
%:%875=420%:%
%:%876=421%:%
%:%877=421%:%
%:%878=421%:%
%:%879=422%:%
%:%880=422%:%
%:%881=423%:%
%:%882=423%:%
%:%883=423%:%
%:%884=424%:%
%:%885=424%:%
%:%886=425%:%
%:%896=427%:%
%:%898=429%:%
%:%899=429%:%
%:%900=430%:%
%:%901=431%:%
%:%908=432%:%
%:%909=432%:%
%:%910=433%:%
%:%911=433%:%
%:%912=434%:%
%:%913=434%:%
%:%914=435%:%
%:%915=435%:%
%:%916=436%:%
%:%917=436%:%
%:%918=436%:%
%:%919=437%:%
%:%920=437%:%
%:%921=438%:%
%:%922=438%:%
%:%923=438%:%
%:%924=439%:%
%:%925=439%:%
%:%926=440%:%
%:%927=440%:%
%:%928=440%:%
%:%929=441%:%
%:%930=441%:%
%:%931=442%:%
%:%932=442%:%
%:%933=443%:%
%:%943=445%:%
%:%944=446%:%
%:%945=447%:%
%:%946=448%:%
%:%947=449%:%
%:%948=450%:%
%:%949=451%:%
%:%951=453%:%
%:%952=453%:%
%:%953=454%:%
%:%954=455%:%
%:%957=456%:%
%:%961=456%:%
%:%971=458%:%
%:%972=459%:%
%:%973=460%:%
%:%974=461%:%
%:%975=462%:%
%:%976=463%:%
%:%977=464%:%
%:%978=465%:%
%:%979=466%:%
%:%980=467%:%
%:%981=468%:%
%:%982=469%:%
%:%983=470%:%
%:%984=471%:%
%:%985=472%:%
%:%986=473%:%
%:%987=474%:%
%:%988=475%:%
%:%989=476%:%
%:%990=477%:%
%:%991=478%:%
%:%992=479%:%
%:%993=480%:%
%:%994=481%:%
%:%995=482%:%
%:%996=483%:%
%:%997=484%:%
%:%998=485%:%
%:%999=486%:%
%:%1001=488%:%
%:%1002=488%:%
%:%1003=489%:%
%:%1004=490%:%
%:%1007=491%:%
%:%1011=491%:%
%:%1012=491%:%
%:%1013=492%:%
%:%1014=492%:%
%:%1015=493%:%
%:%1016=493%:%
%:%1017=494%:%
%:%1018=494%:%
%:%1019=495%:%
%:%1020=495%:%
%:%1021=496%:%
%:%1022=496%:%
%:%1023=496%:%
%:%1024=496%:%
%:%1025=497%:%
%:%1026=497%:%
%:%1027=498%:%
%:%1028=498%:%
%:%1029=499%:%
%:%1030=499%:%
%:%1031=500%:%
%:%1032=500%:%
%:%1033=501%:%
%:%1034=501%:%
%:%1035=501%:%
%:%1036=502%:%
%:%1037=502%:%
%:%1038=502%:%
%:%1039=503%:%
%:%1040=503%:%
%:%1041=504%:%
%:%1042=504%:%
%:%1043=504%:%
%:%1044=505%:%
%:%1045=505%:%
%:%1046=506%:%
%:%1047=506%:%
%:%1048=506%:%
%:%1049=507%:%
%:%1050=507%:%
%:%1051=508%:%
%:%1052=508%:%
%:%1053=508%:%
%:%1054=509%:%
%:%1055=509%:%
%:%1056=510%:%
%:%1057=510%:%
%:%1058=511%:%
%:%1059=511%:%
%:%1060=511%:%
%:%1061=512%:%
%:%1062=512%:%
%:%1063=513%:%
%:%1064=513%:%
%:%1065=513%:%
%:%1066=514%:%
%:%1076=516%:%
%:%1078=518%:%
%:%1079=518%:%
%:%1080=519%:%
%:%1081=520%:%
%:%1084=521%:%
%:%1088=521%:%
%:%1089=521%:%
%:%1090=522%:%
%:%1091=522%:%
%:%1092=523%:%
%:%1093=523%:%
%:%1094=524%:%
%:%1095=524%:%
%:%1096=525%:%
%:%1097=525%:%
%:%1098=525%:%
%:%1099=526%:%
%:%1100=526%:%
%:%1101=526%:%
%:%1102=527%:%
%:%1103=527%:%
%:%1104=527%:%
%:%1105=527%:%
%:%1106=528%:%
%:%1107=528%:%
%:%1108=528%:%
%:%1109=528%:%
%:%1110=529%:%
%:%1111=529%:%
%:%1112=529%:%
%:%1113=529%:%
%:%1114=529%:%
%:%1115=530%:%
%:%1125=532%:%
%:%1126=533%:%
%:%1127=534%:%
%:%1128=535%:%
%:%1129=536%:%
%:%1130=537%:%
%:%1131=538%:%
%:%1132=539%:%
%:%1133=540%:%
%:%1134=541%:%
%:%1135=542%:%
%:%1136=543%:%
%:%1137=544%:%
%:%1138=545%:%
%:%1139=546%:%
%:%1140=547%:%
%:%1141=548%:%
%:%1142=549%:%
%:%1143=550%:%
%:%1144=551%:%
%:%1145=552%:%
%:%1146=553%:%
%:%1147=554%:%
%:%1148=555%:%
%:%1149=556%:%
%:%1150=557%:%
%:%1151=558%:%
%:%1152=559%:%
%:%1153=560%:%
%:%1154=561%:%
%:%1155=562%:%
%:%1156=563%:%
%:%1157=564%:%
%:%1158=565%:%
%:%1159=566%:%
%:%1160=567%:%
%:%1161=568%:%
%:%1162=569%:%
%:%1163=570%:%
%:%1164=571%:%
%:%1165=572%:%
%:%1166=573%:%
%:%1167=574%:%
%:%1168=575%:%
%:%1169=576%:%
%:%1170=577%:%
%:%1171=578%:%
%:%1172=579%:%
%:%1173=580%:%
%:%1174=581%:%
%:%1175=582%:%
%:%1176=583%:%
%:%1177=584%:%
%:%1178=585%:%
%:%1179=586%:%
%:%1180=587%:%
%:%1181=588%:%
%:%1182=589%:%
%:%1183=590%:%
%:%1184=591%:%
%:%1185=592%:%
%:%1186=593%:%
%:%1187=594%:%
%:%1188=595%:%
%:%1189=596%:%
%:%1190=597%:%
%:%1191=598%:%
%:%1192=599%:%
%:%1193=600%:%
%:%1194=601%:%
%:%1195=602%:%
%:%1196=603%:%
%:%1197=604%:%
%:%1198=605%:%
%:%1199=606%:%
%:%1200=607%:%
%:%1201=608%:%
%:%1202=609%:%
%:%1203=610%:%
%:%1204=611%:%
%:%1205=612%:%
%:%1206=613%:%
%:%1207=614%:%
%:%1208=615%:%
%:%1209=616%:%
%:%1210=617%:%
%:%1211=618%:%
%:%1212=619%:%
%:%1213=620%:%
%:%1214=621%:%
%:%1215=622%:%
%:%1216=623%:%
%:%1217=624%:%
%:%1218=625%:%
%:%1219=626%:%
%:%1220=627%:%
%:%1221=628%:%
%:%1222=629%:%
%:%1223=630%:%
%:%1224=631%:%
%:%1225=632%:%
%:%1226=633%:%
%:%1227=634%:%
%:%1228=635%:%
%:%1229=636%:%
%:%1230=637%:%
%:%1231=638%:%
%:%1232=639%:%
%:%1233=640%:%
%:%1234=641%:%
%:%1235=642%:%
%:%1236=643%:%
%:%1237=644%:%
%:%1238=645%:%
%:%1239=646%:%
%:%1240=647%:%
%:%1241=648%:%
%:%1242=649%:%
%:%1243=650%:%
%:%1244=651%:%
%:%1245=652%:%
%:%1246=653%:%
%:%1247=654%:%
%:%1248=655%:%
%:%1249=656%:%
%:%1250=657%:%
%:%1251=658%:%
%:%1252=659%:%
%:%1253=660%:%
%:%1254=661%:%
%:%1255=662%:%
%:%1256=663%:%
%:%1257=664%:%
%:%1258=665%:%
%:%1259=666%:%
%:%1260=667%:%
%:%1261=668%:%
%:%1262=669%:%
%:%1263=670%:%
%:%1264=671%:%
%:%1265=672%:%
%:%1266=673%:%
%:%1267=674%:%
%:%1268=675%:%
%:%1269=676%:%
%:%1270=677%:%
%:%1271=678%:%
%:%1272=679%:%
%:%1273=680%:%
%:%1274=681%:%
%:%1275=682%:%
%:%1276=683%:%
%:%1277=684%:%
%:%1278=685%:%
%:%1279=686%:%
%:%1280=687%:%
%:%1281=688%:%
%:%1282=689%:%
%:%1283=690%:%
%:%1284=691%:%
%:%1285=692%:%
%:%1286=693%:%
%:%1287=694%:%
%:%1288=695%:%
%:%1289=696%:%
%:%1290=697%:%
%:%1291=698%:%
%:%1292=699%:%
%:%1293=700%:%
%:%1294=701%:%
%:%1295=702%:%
%:%1296=703%:%
%:%1297=704%:%
%:%1298=705%:%
%:%1299=706%:%
%:%1300=707%:%
%:%1301=708%:%
%:%1302=709%:%
%:%1303=710%:%
%:%1304=711%:%
%:%1305=712%:%
%:%1306=713%:%
%:%1307=714%:%
%:%1308=715%:%
%:%1309=716%:%
%:%1310=717%:%
%:%1311=718%:%
%:%1312=719%:%
%:%1313=720%:%
%:%1314=721%:%
%:%1315=722%:%
%:%1316=723%:%
%:%1317=724%:%
%:%1318=725%:%
%:%1319=726%:%
%:%1320=727%:%
%:%1321=728%:%
%:%1322=729%:%
%:%1323=730%:%
%:%1324=731%:%
%:%1325=732%:%
%:%1326=733%:%
%:%1327=734%:%
%:%1328=735%:%
%:%1329=736%:%
%:%1330=737%:%
%:%1331=738%:%
%:%1332=739%:%
%:%1333=740%:%
%:%1334=741%:%
%:%1335=742%:%
%:%1336=743%:%
%:%1337=744%:%
%:%1338=745%:%
%:%1339=746%:%
%:%1340=747%:%
%:%1341=748%:%
%:%1342=749%:%
%:%1343=750%:%
%:%1344=751%:%
%:%1345=752%:%
%:%1346=753%:%
%:%1347=754%:%
%:%1348=755%:%
%:%1349=756%:%
%:%1350=757%:%
%:%1351=758%:%
%:%1352=759%:%
%:%1353=760%:%
%:%1354=761%:%
%:%1356=763%:%
%:%1357=763%:%
%:%1358=764%:%
%:%1359=765%:%
%:%1360=766%:%
%:%1361=766%:%
%:%1362=767%:%
%:%1364=769%:%
%:%1365=770%:%
%:%1366=771%:%
%:%1368=773%:%
%:%1369=773%:%
%:%1370=774%:%
%:%1371=775%:%
%:%1378=776%:%
%:%1379=776%:%
%:%1380=777%:%
%:%1381=777%:%
%:%1382=778%:%
%:%1383=778%:%
%:%1384=779%:%
%:%1385=779%:%
%:%1386=780%:%
%:%1387=780%:%
%:%1388=781%:%
%:%1389=781%:%
%:%1390=782%:%
%:%1391=782%:%
%:%1392=783%:%
%:%1393=783%:%
%:%1394=784%:%
%:%1395=784%:%
%:%1396=785%:%
%:%1397=785%:%
%:%1398=786%:%
%:%1399=786%:%
%:%1400=787%:%
%:%1401=787%:%
%:%1402=788%:%
%:%1403=788%:%
%:%1404=789%:%
%:%1405=789%:%
%:%1406=790%:%
%:%1407=790%:%
%:%1408=790%:%
%:%1409=791%:%
%:%1410=791%:%
%:%1411=792%:%
%:%1412=792%:%
%:%1413=793%:%
%:%1414=793%:%
%:%1415=794%:%
%:%1416=794%:%
%:%1417=795%:%
%:%1418=795%:%
%:%1419=795%:%
%:%1420=796%:%
%:%1421=796%:%
%:%1422=796%:%
%:%1423=797%:%
%:%1424=797%:%
%:%1425=797%:%
%:%1426=798%:%
%:%1427=798%:%
%:%1428=798%:%
%:%1429=799%:%
%:%1430=799%:%
%:%1431=799%:%
%:%1432=800%:%
%:%1433=800%:%
%:%1434=801%:%
%:%1435=801%:%
%:%1436=802%:%
%:%1437=802%:%
%:%1438=803%:%
%:%1439=803%:%
%:%1440=804%:%
%:%1441=804%:%
%:%1442=804%:%
%:%1443=805%:%
%:%1444=805%:%
%:%1445=805%:%
%:%1446=806%:%
%:%1447=806%:%
%:%1448=806%:%
%:%1449=807%:%
%:%1450=807%:%
%:%1451=807%:%
%:%1452=808%:%
%:%1453=808%:%
%:%1454=808%:%
%:%1455=809%:%
%:%1456=809%:%
%:%1457=809%:%
%:%1458=810%:%
%:%1459=810%:%
%:%1460=811%:%
%:%1461=811%:%
%:%1462=812%:%
%:%1463=812%:%
%:%1464=813%:%
%:%1465=813%:%
%:%1466=814%:%
%:%1467=814%:%
%:%1468=815%:%
%:%1469=815%:%
%:%1470=816%:%
%:%1471=816%:%
%:%1472=816%:%
%:%1473=817%:%
%:%1474=817%:%
%:%1475=818%:%
%:%1476=818%:%
%:%1477=819%:%
%:%1478=819%:%
%:%1479=820%:%
%:%1480=820%:%
%:%1481=821%:%
%:%1482=821%:%
%:%1483=822%:%
%:%1484=822%:%
%:%1485=823%:%
%:%1486=823%:%
%:%1487=824%:%
%:%1488=824%:%
%:%1489=825%:%
%:%1490=825%:%
%:%1491=826%:%
%:%1492=826%:%
%:%1493=826%:%
%:%1494=827%:%
%:%1495=827%:%
%:%1496=827%:%
%:%1497=828%:%
%:%1498=828%:%
%:%1499=828%:%
%:%1500=829%:%
%:%1501=829%:%
%:%1502=830%:%
%:%1503=830%:%
%:%1504=831%:%
%:%1505=831%:%
%:%1506=832%:%
%:%1507=832%:%
%:%1508=833%:%
%:%1509=833%:%
%:%1510=834%:%
%:%1511=834%:%
%:%1512=835%:%
%:%1513=835%:%
%:%1514=836%:%
%:%1515=836%:%
%:%1516=836%:%
%:%1517=837%:%
%:%1518=837%:%
%:%1519=837%:%
%:%1520=838%:%
%:%1521=838%:%
%:%1522=839%:%
%:%1523=839%:%
%:%1524=840%:%
%:%1525=840%:%
%:%1526=841%:%
%:%1527=841%:%
%:%1528=841%:%
%:%1529=842%:%
%:%1530=842%:%
%:%1531=843%:%
%:%1532=843%:%
%:%1533=843%:%
%:%1534=844%:%
%:%1535=844%:%
%:%1536=845%:%
%:%1537=845%:%
%:%1538=846%:%
%:%1539=846%:%
%:%1540=847%:%
%:%1541=847%:%
%:%1542=847%:%
%:%1543=848%:%
%:%1544=848%:%
%:%1545=849%:%
%:%1546=849%:%
%:%1547=850%:%
%:%1548=850%:%
%:%1549=851%:%
%:%1550=851%:%
%:%1551=852%:%
%:%1552=852%:%
%:%1553=853%:%
%:%1563=855%:%
%:%1564=856%:%
%:%1565=857%:%
%:%1566=858%:%
%:%1567=859%:%
%:%1568=860%:%
%:%1569=861%:%
%:%1570=862%:%
%:%1571=863%:%
%:%1572=864%:%
%:%1573=865%:%
%:%1574=866%:%
%:%1575=867%:%
%:%1576=868%:%
%:%1577=869%:%
%:%1578=870%:%
%:%1579=871%:%
%:%1580=872%:%
%:%1581=873%:%
%:%1583=875%:%
%:%1584=875%:%
%:%1585=876%:%
%:%1586=877%:%
%:%1587=878%:%
%:%1588=879%:%
%:%1591=882%:%
%:%1598=883%:%
%:%1599=883%:%
%:%1600=884%:%
%:%1601=884%:%
%:%1605=888%:%
%:%1606=889%:%
%:%1607=889%:%
%:%1608=889%:%
%:%1609=890%:%
%:%1610=890%:%
%:%1613=893%:%
%:%1614=894%:%
%:%1615=894%:%
%:%1616=894%:%
%:%1617=895%:%
%:%1618=895%:%
%:%1619=895%:%
%:%1620=896%:%
%:%1621=896%:%
%:%1622=897%:%
%:%1623=897%:%
%:%1624=898%:%
%:%1625=898%:%
%:%1626=898%:%
%:%1627=899%:%
%:%1628=899%:%
%:%1629=900%:%
%:%1630=900%:%
%:%1631=900%:%
%:%1632=901%:%
%:%1633=901%:%
%:%1634=902%:%
%:%1635=902%:%
%:%1636=903%:%
%:%1637=903%:%
%:%1638=904%:%
%:%1639=904%:%
%:%1640=905%:%
%:%1641=905%:%
%:%1642=906%:%
%:%1643=906%:%
%:%1644=907%:%
%:%1645=907%:%
%:%1646=908%:%
%:%1647=908%:%
%:%1648=909%:%
%:%1649=909%:%
%:%1650=909%:%
%:%1651=910%:%
%:%1652=910%:%
%:%1653=910%:%
%:%1654=911%:%
%:%1655=911%:%
%:%1656=911%:%
%:%1657=912%:%
%:%1658=912%:%
%:%1659=912%:%
%:%1660=913%:%
%:%1661=913%:%
%:%1662=913%:%
%:%1663=914%:%
%:%1664=914%:%
%:%1665=915%:%
%:%1666=915%:%
%:%1667=915%:%
%:%1668=916%:%
%:%1669=916%:%
%:%1670=917%:%
%:%1671=917%:%
%:%1672=918%:%
%:%1673=918%:%
%:%1674=919%:%
%:%1675=919%:%
%:%1676=919%:%
%:%1677=920%:%
%:%1678=920%:%
%:%1679=921%:%
%:%1680=921%:%
%:%1681=922%:%
%:%1682=922%:%
%:%1683=923%:%
%:%1684=923%:%
%:%1685=924%:%
%:%1686=924%:%
%:%1687=925%:%
%:%1688=925%:%
%:%1689=926%:%
%:%1690=926%:%
%:%1691=927%:%
%:%1692=927%:%
%:%1693=928%:%
%:%1694=928%:%
%:%1695=928%:%
%:%1696=929%:%
%:%1697=929%:%
%:%1698=929%:%
%:%1699=930%:%
%:%1700=930%:%
%:%1701=930%:%
%:%1702=931%:%
%:%1703=931%:%
%:%1704=931%:%
%:%1705=932%:%
%:%1706=932%:%
%:%1707=932%:%
%:%1708=933%:%
%:%1709=933%:%
%:%1710=934%:%
%:%1711=934%:%
%:%1712=935%:%
%:%1713=935%:%
%:%1714=936%:%
%:%1715=936%:%
%:%1716=936%:%
%:%1717=937%:%
%:%1718=937%:%
%:%1719=937%:%
%:%1720=938%:%
%:%1721=938%:%
%:%1722=939%:%
%:%1723=939%:%
%:%1724=940%:%
%:%1725=940%:%
%:%1726=941%:%
%:%1727=941%:%
%:%1728=942%:%
%:%1729=942%:%
%:%1730=942%:%
%:%1731=943%:%
%:%1732=943%:%
%:%1733=943%:%
%:%1734=944%:%
%:%1735=944%:%
%:%1736=945%:%
%:%1737=945%:%
%:%1738=946%:%
%:%1739=946%:%
%:%1740=947%:%
%:%1741=947%:%
%:%1742=948%:%
%:%1743=948%:%
%:%1744=949%:%
%:%1745=949%:%
%:%1748=952%:%
%:%1749=953%:%
%:%1750=953%:%
%:%1751=953%:%
%:%1752=954%:%
%:%1762=956%:%
%:%1763=957%:%
%:%1764=958%:%
%:%1765=959%:%
%:%1766=960%:%
%:%1767=961%:%
%:%1769=963%:%
%:%1770=963%:%
%:%1771=964%:%
%:%1772=965%:%
%:%1773=966%:%
%:%1774=967%:%
%:%1775=968%:%
%:%1776=969%:%
%:%1783=970%:%
%:%1784=970%:%
%:%1785=971%:%
%:%1786=971%:%
%:%1790=975%:%
%:%1791=976%:%
%:%1792=976%:%
%:%1793=976%:%
%:%1794=977%:%
%:%1795=977%:%
%:%1796=978%:%
%:%1797=978%:%
%:%1798=979%:%
%:%1799=979%:%
%:%1800=980%:%
%:%1801=980%:%
%:%1802=981%:%
%:%1803=981%:%
%:%1804=982%:%
%:%1805=982%:%
%:%1806=983%:%
%:%1807=983%:%
%:%1808=984%:%
%:%1809=984%:%
%:%1810=985%:%
%:%1811=985%:%
%:%1812=986%:%
%:%1813=986%:%
%:%1814=986%:%
%:%1815=986%:%
%:%1816=987%:%
%:%1817=987%:%
%:%1818=987%:%
%:%1819=988%:%
%:%1820=988%:%
%:%1821=988%:%
%:%1822=989%:%
%:%1823=989%:%
%:%1824=989%:%
%:%1825=990%:%
%:%1826=990%:%
%:%1827=990%:%
%:%1828=991%:%
%:%1829=991%:%
%:%1832=994%:%
%:%1833=995%:%
%:%1834=995%:%
%:%1835=995%:%
%:%1836=996%:%
%:%1837=996%:%
%:%1838=996%:%
%:%1839=997%:%
%:%1840=997%:%
%:%1841=998%:%
%:%1842=998%:%
%:%1843=998%:%
%:%1844=999%:%
%:%1845=999%:%
%:%1846=1000%:%
%:%1847=1000%:%
%:%1848=1000%:%
%:%1849=1001%:%
%:%1850=1001%:%
%:%1851=1001%:%
%:%1852=1002%:%
%:%1853=1002%:%
%:%1854=1003%:%
%:%1855=1003%:%
%:%1856=1003%:%
%:%1857=1004%:%
%:%1858=1004%:%
%:%1859=1005%:%
%:%1860=1005%:%
%:%1861=1006%:%
%:%1862=1006%:%
%:%1863=1007%:%
%:%1864=1007%:%
%:%1865=1008%:%
%:%1866=1008%:%
%:%1867=1009%:%
%:%1868=1009%:%
%:%1869=1010%:%
%:%1870=1010%:%
%:%1871=1011%:%
%:%1872=1011%:%
%:%1873=1012%:%
%:%1874=1012%:%
%:%1875=1013%:%
%:%1876=1013%:%
%:%1877=1014%:%
%:%1878=1014%:%
%:%1879=1015%:%
%:%1880=1015%:%
%:%1881=1015%:%
%:%1882=1016%:%
%:%1883=1016%:%
%:%1884=1017%:%
%:%1885=1017%:%
%:%1886=1017%:%
%:%1887=1018%:%
%:%1888=1018%:%
%:%1889=1018%:%
%:%1890=1019%:%
%:%1891=1019%:%
%:%1892=1020%:%
%:%1893=1020%:%
%:%1894=1021%:%
%:%1895=1021%:%
%:%1896=1021%:%
%:%1897=1022%:%
%:%1898=1022%:%
%:%1899=1023%:%
%:%1900=1023%:%
%:%1901=1023%:%
%:%1902=1024%:%
%:%1903=1024%:%
%:%1904=1024%:%
%:%1905=1025%:%
%:%1906=1025%:%
%:%1907=1026%:%
%:%1908=1026%:%
%:%1909=1027%:%
%:%1910=1027%:%
%:%1911=1027%:%
%:%1912=1028%:%
%:%1913=1028%:%
%:%1914=1029%:%
%:%1915=1029%:%
%:%1916=1030%:%
%:%1917=1030%:%
%:%1918=1031%:%
%:%1919=1031%:%
%:%1920=1031%:%
%:%1921=1032%:%
%:%1922=1032%:%
%:%1923=1033%:%
%:%1924=1033%:%
%:%1925=1034%:%
%:%1926=1034%:%
%:%1927=1035%:%
%:%1928=1035%:%
%:%1929=1036%:%
%:%1930=1036%:%
%:%1931=1037%:%
%:%1932=1037%:%
%:%1933=1038%:%
%:%1934=1038%:%
%:%1935=1039%:%
%:%1936=1039%:%
%:%1937=1040%:%
%:%1938=1040%:%
%:%1939=1040%:%
%:%1940=1041%:%
%:%1941=1041%:%
%:%1942=1042%:%
%:%1943=1042%:%
%:%1944=1042%:%
%:%1945=1043%:%
%:%1946=1043%:%
%:%1947=1044%:%
%:%1948=1044%:%
%:%1949=1044%:%
%:%1950=1045%:%
%:%1951=1045%:%
%:%1952=1046%:%
%:%1953=1046%:%
%:%1954=1046%:%
%:%1955=1047%:%
%:%1956=1047%:%
%:%1957=1048%:%
%:%1958=1048%:%
%:%1959=1048%:%
%:%1960=1049%:%
%:%1961=1049%:%
%:%1962=1050%:%
%:%1963=1050%:%
%:%1964=1051%:%
%:%1965=1051%:%
%:%1966=1051%:%
%:%1967=1052%:%
%:%1968=1052%:%
%:%1969=1053%:%
%:%1970=1053%:%
%:%1971=1054%:%
%:%1972=1054%:%
%:%1973=1054%:%
%:%1974=1055%:%
%:%1975=1055%:%
%:%1976=1056%:%
%:%1977=1056%:%
%:%1978=1057%:%
%:%1979=1057%:%
%:%1980=1058%:%
%:%1981=1058%:%
%:%1982=1058%:%
%:%1983=1059%:%
%:%1984=1059%:%
%:%1985=1059%:%
%:%1986=1060%:%
%:%1987=1060%:%
%:%1988=1060%:%
%:%1989=1061%:%
%:%1990=1061%:%
%:%1991=1062%:%
%:%1992=1062%:%
%:%1993=1062%:%
%:%1994=1063%:%
%:%1995=1063%:%
%:%1996=1064%:%
%:%1997=1064%:%
%:%1998=1065%:%
%:%1999=1065%:%
%:%2000=1066%:%
%:%2001=1066%:%
%:%2002=1066%:%
%:%2003=1067%:%
%:%2013=1069%:%
%:%2014=1070%:%
%:%2015=1071%:%
%:%2016=1072%:%
%:%2017=1073%:%
%:%2018=1074%:%
%:%2019=1075%:%
%:%2020=1076%:%
%:%2021=1077%:%
%:%2022=1078%:%
%:%2023=1079%:%
%:%2024=1080%:%
%:%2026=1082%:%
%:%2027=1082%:%
%:%2028=1083%:%
%:%2029=1084%:%
%:%2030=1085%:%
%:%2031=1086%:%
%:%2032=1087%:%
%:%2033=1088%:%
%:%2034=1089%:%
%:%2035=1090%:%
%:%2036=1091%:%
%:%2037=1092%:%
%:%2038=1093%:%
%:%2045=1094%:%
%:%2046=1094%:%
%:%2047=1095%:%
%:%2048=1095%:%
%:%2049=1096%:%
%:%2050=1096%:%
%:%2051=1097%:%
%:%2052=1097%:%
%:%2053=1098%:%
%:%2054=1098%:%
%:%2055=1099%:%
%:%2056=1099%:%
%:%2057=1100%:%
%:%2058=1100%:%
%:%2059=1101%:%
%:%2060=1101%:%
%:%2061=1101%:%
%:%2062=1102%:%
%:%2063=1102%:%
%:%2064=1103%:%
%:%2065=1103%:%
%:%2066=1103%:%
%:%2067=1104%:%
%:%2068=1104%:%
%:%2069=1105%:%
%:%2070=1105%:%
%:%2071=1105%:%
%:%2072=1105%:%
%:%2073=1106%:%
%:%2074=1106%:%
%:%2075=1106%:%
%:%2076=1107%:%
%:%2077=1107%:%
%:%2078=1107%:%
%:%2079=1108%:%
%:%2080=1108%:%
%:%2081=1108%:%
%:%2082=1109%:%
%:%2083=1109%:%
%:%2084=1109%:%
%:%2085=1110%:%
%:%2086=1110%:%
%:%2087=1111%:%
%:%2088=1111%:%
%:%2089=1111%:%
%:%2090=1112%:%
%:%2091=1112%:%
%:%2094=1115%:%
%:%2095=1116%:%
%:%2096=1116%:%
%:%2097=1116%:%
%:%2098=1117%:%
%:%2099=1117%:%
%:%2100=1117%:%
%:%2103=1120%:%
%:%2104=1121%:%
%:%2105=1121%:%
%:%2106=1121%:%
%:%2107=1122%:%
%:%2108=1122%:%
%:%2109=1122%:%
%:%2110=1123%:%
%:%2111=1123%:%
%:%2112=1124%:%
%:%2113=1124%:%
%:%2114=1124%:%
%:%2115=1125%:%
%:%2116=1125%:%
%:%2117=1126%:%
%:%2118=1126%:%
%:%2119=1126%:%
%:%2120=1127%:%
%:%2121=1127%:%
%:%2122=1127%:%
%:%2123=1128%:%
%:%2124=1128%:%
%:%2125=1128%:%
%:%2126=1129%:%
%:%2127=1129%:%
%:%2128=1129%:%
%:%2129=1130%:%
%:%2130=1130%:%
%:%2131=1131%:%
%:%2132=1131%:%
%:%2133=1132%:%
%:%2134=1132%:%
%:%2135=1133%:%
%:%2136=1133%:%
%:%2137=1134%:%
%:%2138=1134%:%
%:%2139=1135%:%
%:%2140=1135%:%
%:%2141=1135%:%
%:%2142=1136%:%
%:%2143=1136%:%
%:%2144=1136%:%
%:%2145=1137%:%
%:%2146=1137%:%
%:%2147=1138%:%
%:%2148=1138%:%
%:%2149=1138%:%
%:%2150=1139%:%
%:%2151=1139%:%
%:%2152=1140%:%
%:%2153=1140%:%
%:%2154=1140%:%
%:%2155=1141%:%
%:%2156=1141%:%
%:%2157=1142%:%
%:%2158=1142%:%
%:%2159=1142%:%
%:%2160=1143%:%
%:%2161=1143%:%
%:%2162=1144%:%
%:%2163=1144%:%
%:%2164=1145%:%
%:%2165=1145%:%
%:%2166=1145%:%
%:%2167=1146%:%
%:%2168=1146%:%
%:%2169=1147%:%
%:%2170=1147%:%
%:%2171=1147%:%
%:%2172=1148%:%
%:%2173=1148%:%
%:%2174=1148%:%
%:%2175=1149%:%
%:%2176=1149%:%
%:%2177=1149%:%
%:%2178=1150%:%
%:%2179=1150%:%
%:%2180=1151%:%
%:%2181=1151%:%
%:%2182=1151%:%
%:%2183=1152%:%
%:%2184=1152%:%
%:%2185=1153%:%
%:%2186=1153%:%
%:%2187=1153%:%
%:%2188=1154%:%
%:%2189=1154%:%
%:%2190=1155%:%
%:%2191=1155%:%
%:%2192=1155%:%
%:%2193=1156%:%
%:%2194=1156%:%
%:%2195=1157%:%
%:%2196=1157%:%
%:%2197=1157%:%
%:%2198=1158%:%
%:%2199=1158%:%
%:%2200=1159%:%
%:%2201=1159%:%
%:%2202=1159%:%
%:%2203=1160%:%
%:%2204=1160%:%
%:%2205=1161%:%
%:%2206=1161%:%
%:%2207=1161%:%
%:%2208=1162%:%
%:%2218=1164%:%
%:%2219=1165%:%
%:%2220=1166%:%
%:%2221=1167%:%
%:%2222=1168%:%
%:%2224=1170%:%
%:%2225=1170%:%
%:%2226=1171%:%
%:%2227=1172%:%
%:%2228=1173%:%
%:%2229=1174%:%
%:%2230=1175%:%
%:%2231=1176%:%
%:%2232=1177%:%
%:%2233=1178%:%
%:%2234=1179%:%
%:%2235=1180%:%
%:%2236=1181%:%
%:%2237=1182%:%
%:%2244=1183%:%
%:%2245=1183%:%
%:%2246=1184%:%
%:%2247=1184%:%
%:%2248=1185%:%
%:%2249=1185%:%
%:%2250=1186%:%
%:%2251=1186%:%
%:%2252=1187%:%
%:%2253=1187%:%
%:%2254=1187%:%
%:%2255=1188%:%
%:%2256=1188%:%
%:%2257=1189%:%
%:%2258=1189%:%
%:%2259=1189%:%
%:%2260=1190%:%
%:%2261=1190%:%
%:%2262=1191%:%
%:%2263=1191%:%
%:%2264=1191%:%
%:%2265=1192%:%
%:%2266=1192%:%
%:%2267=1193%:%
%:%2268=1193%:%
%:%2269=1194%:%
%:%2270=1194%:%
%:%2271=1195%:%
%:%2272=1195%:%
%:%2273=1196%:%
%:%2274=1196%:%
%:%2275=1197%:%
%:%2276=1197%:%
%:%2277=1197%:%
%:%2278=1198%:%
%:%2279=1198%:%
%:%2280=1199%:%
%:%2281=1199%:%
%:%2282=1200%:%
%:%2283=1200%:%
%:%2284=1200%:%
%:%2285=1201%:%
%:%2286=1201%:%
%:%2287=1202%:%
%:%2288=1202%:%
%:%2289=1203%:%
%:%2290=1203%:%
%:%2291=1203%:%
%:%2292=1204%:%
%:%2293=1204%:%
%:%2294=1205%:%
%:%2295=1205%:%
%:%2296=1206%:%
%:%2297=1206%:%
%:%2298=1206%:%
%:%2299=1207%:%
%:%2300=1207%:%
%:%2301=1208%:%
%:%2302=1208%:%
%:%2303=1208%:%
%:%2304=1209%:%
%:%2305=1209%:%
%:%2306=1210%:%
%:%2307=1210%:%
%:%2308=1211%:%
%:%2309=1211%:%
%:%2310=1211%:%
%:%2311=1212%:%
%:%2312=1212%:%
%:%2313=1213%:%
%:%2314=1213%:%
%:%2315=1213%:%
%:%2316=1214%:%
%:%2317=1214%:%
%:%2318=1215%:%
%:%2319=1215%:%
%:%2320=1216%:%
%:%2321=1216%:%
%:%2322=1217%:%
%:%2323=1217%:%
%:%2324=1218%:%
%:%2325=1218%:%
%:%2326=1219%:%
%:%2327=1219%:%
%:%2328=1219%:%
%:%2329=1220%:%
%:%2330=1220%:%
%:%2331=1221%:%
%:%2332=1221%:%
%:%2333=1222%:%
%:%2334=1222%:%
%:%2335=1222%:%
%:%2336=1223%:%
%:%2337=1223%:%
%:%2338=1224%:%
%:%2339=1224%:%
%:%2340=1225%:%
%:%2341=1225%:%
%:%2342=1225%:%
%:%2343=1226%:%
%:%2344=1226%:%
%:%2345=1227%:%
%:%2346=1227%:%
%:%2347=1228%:%
%:%2348=1228%:%
%:%2349=1228%:%
%:%2350=1229%:%
%:%2351=1229%:%
%:%2352=1230%:%
%:%2353=1230%:%
%:%2354=1230%:%
%:%2355=1231%:%
%:%2356=1231%:%
%:%2357=1232%:%
%:%2358=1232%:%
%:%2359=1233%:%
%:%2360=1233%:%
%:%2361=1233%:%
%:%2362=1234%:%
%:%2363=1234%:%
%:%2364=1235%:%
%:%2365=1235%:%
%:%2366=1235%:%
%:%2367=1236%:%
%:%2368=1236%:%
%:%2369=1236%:%
%:%2370=1237%:%
%:%2371=1237%:%
%:%2372=1238%:%
%:%2373=1238%:%
%:%2374=1238%:%
%:%2375=1239%:%
%:%2376=1239%:%
%:%2377=1240%:%
%:%2378=1240%:%
%:%2379=1241%:%
%:%2380=1241%:%
%:%2381=1241%:%
%:%2382=1242%:%
%:%2392=1244%:%
%:%2393=1245%:%
%:%2394=1246%:%
%:%2395=1247%:%
%:%2396=1248%:%
%:%2398=1250%:%
%:%2399=1250%:%
%:%2400=1251%:%
%:%2401=1252%:%
%:%2404=1253%:%
%:%2408=1253%:%
%:%2409=1253%:%
%:%2410=1253%:%
%:%2415=1253%:%
%:%2418=1254%:%
%:%2419=1255%:%
%:%2420=1255%:%
%:%2423=1256%:%
%:%2427=1256%:%
%:%2428=1256%:%
%:%2437=1258%:%
%:%2439=1260%:%
%:%2440=1260%:%
%:%2441=1261%:%
%:%2442=1262%:%
%:%2443=1263%:%
%:%2444=1264%:%
%:%2445=1265%:%
%:%2446=1266%:%
%:%2453=1267%:%
%:%2454=1267%:%
%:%2455=1268%:%
%:%2456=1268%:%
%:%2457=1269%:%
%:%2458=1269%:%
%:%2459=1269%:%
%:%2460=1270%:%
%:%2461=1270%:%
%:%2465=1274%:%
%:%2466=1275%:%
%:%2467=1275%:%
%:%2468=1275%:%
%:%2469=1276%:%
%:%2470=1276%:%
%:%2471=1277%:%
%:%2472=1277%:%
%:%2473=1277%:%
%:%2474=1277%:%
%:%2475=1278%:%
%:%2476=1278%:%
%:%2477=1278%:%
%:%2478=1279%:%
%:%2479=1279%:%
%:%2480=1280%:%
%:%2481=1280%:%
%:%2482=1281%:%
%:%2483=1281%:%
%:%2484=1281%:%
%:%2485=1282%:%
%:%2486=1282%:%
%:%2487=1283%:%
%:%2488=1283%:%
%:%2489=1284%:%
%:%2490=1284%:%
%:%2491=1285%:%
%:%2492=1285%:%
%:%2493=1285%:%
%:%2494=1286%:%
%:%2495=1286%:%
%:%2496=1287%:%
%:%2497=1287%:%
%:%2498=1287%:%
%:%2499=1288%:%
%:%2500=1288%:%
%:%2501=1289%:%
%:%2502=1289%:%
%:%2503=1289%:%
%:%2504=1290%:%
%:%2505=1290%:%
%:%2506=1290%:%
%:%2507=1291%:%
%:%2508=1291%:%
%:%2509=1291%:%
%:%2510=1292%:%
%:%2511=1292%:%
%:%2512=1293%:%
%:%2513=1293%:%
%:%2514=1294%:%
%:%2515=1294%:%
%:%2516=1294%:%
%:%2517=1295%:%
%:%2518=1295%:%
%:%2519=1296%:%
%:%2520=1296%:%
%:%2521=1296%:%
%:%2522=1297%:%
%:%2523=1297%:%
%:%2524=1297%:%
%:%2525=1298%:%
%:%2526=1298%:%
%:%2527=1299%:%
%:%2528=1299%:%
%:%2529=1300%:%
%:%2530=1300%:%
%:%2531=1300%:%
%:%2532=1301%:%
%:%2533=1301%:%
%:%2534=1301%:%
%:%2535=1302%:%
%:%2536=1302%:%
%:%2537=1303%:%
%:%2538=1303%:%
%:%2539=1304%:%
%:%2540=1304%:%
%:%2541=1304%:%
%:%2542=1305%:%
%:%2543=1305%:%
%:%2544=1305%:%
%:%2545=1306%:%
%:%2546=1306%:%
%:%2547=1306%:%
%:%2548=1307%:%
%:%2549=1307%:%
%:%2550=1307%:%
%:%2551=1308%:%
%:%2552=1308%:%
%:%2553=1308%:%
%:%2554=1309%:%
%:%2555=1309%:%
%:%2556=1309%:%
%:%2557=1310%:%
%:%2558=1310%:%
%:%2559=1311%:%
%:%2560=1311%:%
%:%2561=1312%:%
%:%2562=1312%:%
%:%2563=1313%:%
%:%2564=1313%:%
%:%2565=1314%:%
%:%2566=1314%:%
%:%2567=1314%:%
%:%2568=1315%:%
%:%2569=1315%:%
%:%2570=1315%:%
%:%2571=1316%:%
%:%2572=1316%:%
%:%2573=1317%:%
%:%2574=1317%:%
%:%2575=1317%:%
%:%2576=1318%:%
%:%2577=1318%:%
%:%2578=1319%:%
%:%2579=1319%:%
%:%2580=1320%:%
%:%2581=1320%:%
%:%2582=1320%:%
%:%2583=1321%:%
%:%2584=1321%:%
%:%2585=1322%:%
%:%2586=1322%:%
%:%2587=1323%:%
%:%2588=1323%:%
%:%2589=1324%:%
%:%2590=1324%:%
%:%2591=1325%:%
%:%2592=1325%:%
%:%2593=1326%:%
%:%2594=1326%:%
%:%2595=1326%:%
%:%2596=1327%:%
%:%2597=1327%:%
%:%2598=1328%:%
%:%2599=1328%:%
%:%2600=1328%:%
%:%2601=1329%:%
%:%2602=1329%:%
%:%2603=1330%:%
%:%2604=1330%:%
%:%2605=1330%:%
%:%2606=1331%:%
%:%2607=1331%:%
%:%2608=1332%:%
%:%2609=1332%:%
%:%2610=1333%:%
%:%2611=1333%:%
%:%2612=1333%:%
%:%2613=1334%:%
%:%2614=1334%:%
%:%2615=1334%:%
%:%2616=1335%:%
%:%2617=1335%:%
%:%2618=1336%:%
%:%2619=1336%:%
%:%2620=1336%:%
%:%2621=1337%:%
%:%2622=1337%:%
%:%2623=1337%:%
%:%2624=1338%:%
%:%2625=1338%:%
%:%2626=1338%:%
%:%2627=1339%:%
%:%2628=1339%:%
%:%2629=1339%:%
%:%2630=1340%:%
%:%2631=1340%:%
%:%2632=1341%:%
%:%2633=1341%:%
%:%2634=1342%:%
%:%2635=1342%:%
%:%2636=1342%:%
%:%2637=1343%:%
%:%2638=1343%:%
%:%2639=1344%:%
%:%2640=1344%:%
%:%2641=1344%:%
%:%2642=1345%:%
%:%2643=1345%:%
%:%2644=1345%:%
%:%2645=1346%:%
%:%2646=1346%:%
%:%2647=1347%:%
%:%2648=1347%:%
%:%2649=1348%:%
%:%2650=1348%:%
%:%2651=1348%:%
%:%2652=1349%:%
%:%2653=1349%:%
%:%2654=1349%:%
%:%2655=1350%:%
%:%2656=1350%:%
%:%2657=1351%:%
%:%2658=1351%:%
%:%2659=1352%:%
%:%2660=1352%:%
%:%2661=1352%:%
%:%2662=1353%:%
%:%2663=1353%:%
%:%2664=1353%:%
%:%2665=1354%:%
%:%2666=1354%:%
%:%2667=1354%:%
%:%2668=1355%:%
%:%2669=1355%:%
%:%2670=1355%:%
%:%2671=1356%:%
%:%2672=1356%:%
%:%2673=1356%:%
%:%2674=1357%:%
%:%2675=1357%:%
%:%2676=1357%:%
%:%2677=1358%:%
%:%2678=1358%:%
%:%2679=1359%:%
%:%2680=1359%:%
%:%2681=1360%:%
%:%2682=1360%:%
%:%2683=1361%:%
%:%2684=1361%:%
%:%2685=1362%:%
%:%2686=1362%:%
%:%2687=1362%:%
%:%2688=1363%:%
%:%2689=1363%:%
%:%2690=1363%:%
%:%2691=1364%:%
%:%2692=1364%:%
%:%2693=1365%:%
%:%2694=1365%:%
%:%2695=1365%:%
%:%2696=1366%:%
%:%2697=1366%:%
%:%2698=1367%:%
%:%2699=1367%:%
%:%2700=1368%:%
%:%2701=1368%:%
%:%2702=1368%:%
%:%2703=1369%:%
%:%2704=1369%:%
%:%2705=1370%:%
%:%2706=1370%:%
%:%2707=1371%:%
%:%2708=1371%:%
%:%2709=1372%:%
%:%2710=1372%:%
%:%2711=1373%:%
%:%2712=1373%:%
%:%2713=1374%:%
%:%2714=1374%:%
%:%2715=1374%:%
%:%2716=1375%:%
%:%2717=1375%:%
%:%2718=1376%:%
%:%2719=1376%:%
%:%2720=1376%:%
%:%2721=1377%:%
%:%2722=1377%:%
%:%2723=1378%:%
%:%2724=1378%:%
%:%2725=1378%:%
%:%2726=1379%:%
%:%2727=1379%:%
%:%2728=1380%:%
%:%2729=1380%:%
%:%2730=1381%:%
%:%2731=1381%:%
%:%2732=1381%:%
%:%2733=1382%:%
%:%2734=1382%:%
%:%2735=1382%:%
%:%2736=1383%:%
%:%2737=1383%:%
%:%2738=1384%:%
%:%2739=1384%:%
%:%2740=1384%:%
%:%2741=1385%:%
%:%2742=1385%:%
%:%2743=1386%:%
%:%2744=1386%:%
%:%2745=1387%:%
%:%2746=1387%:%
%:%2747=1388%:%
%:%2748=1388%:%
%:%2749=1389%:%
%:%2750=1389%:%
%:%2751=1390%:%
%:%2752=1390%:%
%:%2753=1390%:%
%:%2754=1391%:%
%:%2755=1391%:%
%:%2756=1392%:%
%:%2757=1392%:%
%:%2758=1393%:%
%:%2759=1393%:%
%:%2760=1393%:%
%:%2761=1394%:%
%:%2762=1394%:%
%:%2763=1394%:%
%:%2764=1395%:%
%:%2765=1395%:%
%:%2766=1396%:%
%:%2767=1396%:%
%:%2768=1397%:%
%:%2769=1397%:%
%:%2770=1398%:%
%:%2771=1398%:%
%:%2772=1399%:%
%:%2773=1399%:%
%:%2774=1400%:%
%:%2775=1400%:%
%:%2776=1400%:%
%:%2777=1401%:%
%:%2778=1401%:%
%:%2779=1402%:%
%:%2780=1402%:%
%:%2781=1403%:%
%:%2782=1403%:%
%:%2783=1403%:%
%:%2784=1404%:%
%:%2785=1404%:%
%:%2786=1404%:%
%:%2787=1405%:%
%:%2788=1405%:%
%:%2789=1406%:%
%:%2790=1406%:%
%:%2791=1407%:%
%:%2792=1407%:%
%:%2793=1408%:%
%:%2803=1410%:%
%:%2804=1411%:%
%:%2805=1412%:%
%:%2806=1413%:%
%:%2807=1414%:%
%:%2808=1415%:%
%:%2809=1416%:%
%:%2810=1417%:%
%:%2811=1418%:%
%:%2812=1419%:%
%:%2813=1420%:%
%:%2814=1421%:%
%:%2815=1422%:%
%:%2816=1423%:%
%:%2817=1424%:%
%:%2819=1426%:%
%:%2820=1426%:%
%:%2821=1427%:%
%:%2822=1428%:%
%:%2823=1429%:%
%:%2824=1430%:%
%:%2827=1433%:%
%:%2834=1434%:%
%:%2835=1434%:%
%:%2836=1435%:%
%:%2837=1435%:%
%:%2841=1439%:%
%:%2842=1440%:%
%:%2843=1440%:%
%:%2844=1440%:%
%:%2845=1441%:%
%:%2846=1441%:%
%:%2847=1442%:%
%:%2848=1442%:%
%:%2849=1442%:%
%:%2850=1442%:%
%:%2851=1443%:%
%:%2852=1443%:%
%:%2853=1444%:%
%:%2854=1444%:%
%:%2855=1445%:%
%:%2856=1445%:%
%:%2857=1446%:%
%:%2858=1446%:%
%:%2859=1447%:%
%:%2860=1447%:%
%:%2861=1448%:%
%:%2862=1448%:%
%:%2863=1448%:%
%:%2864=1449%:%
%:%2865=1449%:%
%:%2866=1450%:%
%:%2867=1450%:%
%:%2868=1450%:%
%:%2869=1451%:%
%:%2870=1451%:%
%:%2871=1451%:%
%:%2872=1452%:%
%:%2873=1452%:%
%:%2874=1452%:%
%:%2875=1453%:%
%:%2876=1453%:%
%:%2877=1454%:%
%:%2878=1454%:%
%:%2879=1455%:%
%:%2880=1455%:%
%:%2881=1456%:%
%:%2882=1456%:%
%:%2883=1456%:%
%:%2884=1457%:%
%:%2885=1457%:%
%:%2886=1458%:%
%:%2887=1458%:%
%:%2888=1459%:%
%:%2889=1459%:%
%:%2890=1459%:%
%:%2891=1460%:%
%:%2892=1460%:%
%:%2893=1460%:%
%:%2894=1461%:%
%:%2895=1461%:%
%:%2896=1461%:%
%:%2897=1462%:%
%:%2898=1462%:%
%:%2901=1465%:%
%:%2902=1466%:%
%:%2903=1466%:%
%:%2904=1466%:%
%:%2905=1467%:%
%:%2906=1467%:%
%:%2907=1467%:%
%:%2908=1468%:%
%:%2909=1468%:%
%:%2910=1469%:%
%:%2911=1469%:%
%:%2912=1470%:%
%:%2913=1470%:%
%:%2914=1471%:%
%:%2915=1471%:%
%:%2916=1472%:%
%:%2917=1472%:%
%:%2918=1472%:%
%:%2919=1473%:%
%:%2920=1473%:%
%:%2921=1474%:%
%:%2922=1474%:%
%:%2923=1475%:%
%:%2924=1475%:%
%:%2925=1476%:%
%:%2926=1476%:%
%:%2927=1477%:%
%:%2928=1477%:%
%:%2929=1478%:%
%:%2930=1478%:%
%:%2931=1479%:%
%:%2932=1479%:%
%:%2933=1480%:%
%:%2934=1480%:%
%:%2935=1481%:%
%:%2936=1481%:%
%:%2937=1482%:%
%:%2938=1482%:%
%:%2939=1483%:%
%:%2940=1483%:%
%:%2941=1484%:%
%:%2942=1484%:%
%:%2943=1485%:%
%:%2944=1485%:%
%:%2945=1486%:%
%:%2946=1486%:%
%:%2947=1487%:%
%:%2948=1487%:%
%:%2949=1487%:%
%:%2950=1488%:%
%:%2951=1488%:%
%:%2952=1489%:%
%:%2953=1489%:%
%:%2954=1489%:%
%:%2955=1490%:%
%:%2956=1490%:%
%:%2957=1491%:%
%:%2958=1491%:%
%:%2959=1492%:%
%:%2960=1492%:%
%:%2961=1492%:%
%:%2962=1493%:%
%:%2963=1493%:%
%:%2964=1494%:%
%:%2965=1494%:%
%:%2966=1494%:%
%:%2967=1495%:%
%:%2968=1495%:%
%:%2969=1496%:%
%:%2970=1496%:%
%:%2971=1497%:%
%:%2972=1497%:%
%:%2973=1497%:%
%:%2974=1498%:%
%:%2975=1498%:%
%:%2976=1498%:%
%:%2977=1499%:%
%:%2978=1499%:%
%:%2979=1499%:%
%:%2980=1500%:%
%:%2981=1500%:%
%:%2984=1503%:%
%:%2985=1504%:%
%:%2986=1504%:%
%:%2987=1504%:%
%:%2988=1505%:%
%:%2989=1505%:%
%:%2990=1505%:%
%:%2991=1506%:%
%:%2992=1506%:%
%:%2993=1507%:%
%:%2994=1507%:%
%:%2995=1507%:%
%:%2996=1508%:%
%:%2997=1508%:%
%:%2998=1509%:%
%:%2999=1509%:%
%:%3000=1509%:%
%:%3001=1510%:%
%:%3002=1510%:%
%:%3003=1510%:%
%:%3004=1511%:%
%:%3005=1511%:%
%:%3006=1512%:%
%:%3007=1512%:%
%:%3008=1512%:%
%:%3009=1513%:%
%:%3010=1513%:%
%:%3011=1514%:%
%:%3012=1514%:%
%:%3013=1515%:%
%:%3014=1515%:%
%:%3015=1516%:%
%:%3016=1516%:%
%:%3017=1517%:%
%:%3018=1517%:%
%:%3019=1518%:%
%:%3020=1518%:%
%:%3021=1519%:%
%:%3022=1519%:%
%:%3023=1520%:%
%:%3024=1520%:%
%:%3025=1521%:%
%:%3026=1521%:%
%:%3027=1522%:%
%:%3028=1522%:%
%:%3029=1523%:%
%:%3030=1523%:%
%:%3031=1523%:%
%:%3032=1524%:%
%:%3033=1524%:%
%:%3034=1525%:%
%:%3035=1525%:%
%:%3036=1526%:%
%:%3037=1526%:%
%:%3038=1527%:%
%:%3039=1527%:%
%:%3040=1528%:%
%:%3041=1528%:%
%:%3042=1529%:%
%:%3043=1529%:%
%:%3044=1530%:%
%:%3045=1530%:%
%:%3046=1531%:%
%:%3047=1531%:%
%:%3048=1532%:%
%:%3049=1532%:%
%:%3050=1533%:%
%:%3051=1533%:%
%:%3052=1534%:%
%:%3053=1534%:%
%:%3054=1534%:%
%:%3055=1535%:%
%:%3056=1535%:%
%:%3057=1536%:%
%:%3058=1536%:%
%:%3059=1537%:%
%:%3060=1537%:%
%:%3061=1538%:%
%:%3062=1538%:%
%:%3063=1538%:%
%:%3064=1539%:%
%:%3074=1541%:%
%:%3075=1542%:%
%:%3076=1543%:%
%:%3078=1545%:%
%:%3079=1545%:%
%:%3080=1546%:%
%:%3081=1547%:%
%:%3082=1548%:%
%:%3085=1549%:%
%:%3089=1549%:%
%:%3090=1549%:%
%:%3091=1550%:%
%:%3092=1550%:%
%:%3093=1551%:%
%:%3094=1551%:%
%:%3098=1555%:%
%:%3099=1556%:%
%:%3100=1556%:%
%:%3101=1556%:%
%:%3102=1557%:%
%:%3103=1557%:%
%:%3104=1558%:%
%:%3105=1558%:%
%:%3106=1559%:%
%:%3107=1559%:%
%:%3108=1559%:%
%:%3109=1560%:%
%:%3110=1560%:%
%:%3111=1560%:%
%:%3112=1561%:%
%:%3113=1561%:%
%:%3116=1564%:%
%:%3117=1565%:%
%:%3118=1565%:%
%:%3119=1566%:%
%:%3120=1566%:%
%:%3121=1567%:%
%:%3122=1567%:%
%:%3123=1568%:%
%:%3124=1568%:%
%:%3125=1568%:%
%:%3126=1569%:%
%:%3127=1569%:%
%:%3128=1570%:%
%:%3129=1570%:%
%:%3130=1571%:%
%:%3131=1571%:%
%:%3132=1572%:%
%:%3133=1572%:%
%:%3134=1572%:%
%:%3135=1573%:%
%:%3136=1573%:%
%:%3137=1574%:%
%:%3147=1576%:%
%:%3149=1578%:%
%:%3150=1578%:%
%:%3151=1579%:%
%:%3152=1580%:%
%:%3153=1581%:%
%:%3160=1582%:%
%:%3161=1582%:%
%:%3162=1583%:%
%:%3163=1583%:%
%:%3164=1584%:%
%:%3165=1584%:%
%:%3166=1585%:%
%:%3167=1585%:%
%:%3168=1585%:%
%:%3169=1586%:%
%:%3170=1586%:%
%:%3171=1587%:%
%:%3172=1587%:%
%:%3173=1587%:%
%:%3174=1588%:%
%:%3175=1588%:%
%:%3176=1589%:%
%:%3177=1589%:%
%:%3178=1589%:%
%:%3179=1590%:%
%:%3180=1590%:%
%:%3181=1591%:%
%:%3182=1591%:%
%:%3183=1591%:%
%:%3184=1592%:%
%:%3185=1592%:%
%:%3186=1593%:%
%:%3187=1593%:%
%:%3188=1594%:%
%:%3189=1594%:%