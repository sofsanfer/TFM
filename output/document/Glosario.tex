%
\begin{isabellebody}%
\setisabellecontext{Glosario}%
%
\isadelimtheory
%
\endisadelimtheory
%
\isatagtheory
%
\endisatagtheory
{\isafoldtheory}%
%
\isadelimtheory
%
\endisadelimtheory
%
\begin{isamarkuptext}%
En este glosario se recoge la lista de los lemas y reglas usadas
  indicando la página del \href{https://bit.ly/2KvG2ej}{libro de HOL} 
  donde se encuentran.%
\end{isamarkuptext}\isamarkuptrue%
%
\isadelimdocument
%
\endisadelimdocument
%
\isatagdocument
%
\isamarkupsection{La base de lógica de primer orden (2)%
}
\isamarkuptrue%
%
\endisatagdocument
{\isafolddocument}%
%
\isadelimdocument
%
\endisadelimdocument
%
\begin{isamarkuptext}%
En Isabelle corresponde a la teoría 
  \href{https://bit.ly/3bGva9s}{HOL.thy}%
\end{isamarkuptext}\isamarkuptrue%
%
\isadelimdocument
%
\endisadelimdocument
%
\isatagdocument
%
\isamarkupsubsection{Lógica primitiva (2.1)%
}
\isamarkuptrue%
%
\isamarkupsubsubsection{Conectivas y cuantificadores definidos (2.1.2)%
}
\isamarkuptrue%
%
\endisatagdocument
{\isafolddocument}%
%
\isadelimdocument
%
\endisadelimdocument
%
\begin{isamarkuptext}%
\begin{itemize}
    \item (p.34) \isa{{\isasymnot}\ P\ {\isasymequiv}\ P\ {\isasymlongrightarrow}\ False} 
      \hfill (\isa{not{\isacharunderscore}def})
  \end{itemize}%
\end{isamarkuptext}\isamarkuptrue%
%
\isadelimdocument
%
\endisadelimdocument
%
\isatagdocument
%
\isamarkupsubsubsection{Axiomas y definiciones básicas (2.1.4)%
}
\isamarkuptrue%
%
\endisatagdocument
{\isafolddocument}%
%
\isadelimdocument
%
\endisadelimdocument
%
\begin{isamarkuptext}%
\begin{itemize}
    \item (p.36) \isa{\mbox{}\inferrule{\mbox{\mbox{}\inferrule{\mbox{P}}{\mbox{Q}}}}{\mbox{P\ {\isasymlongrightarrow}\ Q}}} 
      \hfill (\isa{impI})
    \item (p.36) \isa{\mbox{}\inferrule{\mbox{{\isacharparenleft}P\ {\isasymlongrightarrow}\ Q{\isacharparenright}\ {\isasymand}\ P}}{\mbox{Q}}} 
      \hfill (\isa{mp})
  \end{itemize}%
\end{isamarkuptext}\isamarkuptrue%
%
\isadelimdocument
%
\endisadelimdocument
%
\isatagdocument
%
\isamarkupsubsection{Reglas fundamentales (2.2)%
}
\isamarkuptrue%
%
\isamarkupsubsubsection{Reglas de congruencia para aplicaciones (2.2.2)%
}
\isamarkuptrue%
%
\endisatagdocument
{\isafolddocument}%
%
\isadelimdocument
%
\endisadelimdocument
%
\begin{isamarkuptext}%
\begin{itemize}
    \item (p.37) \isa{\mbox{}\inferrule{\mbox{x\ {\isacharequal}\ y}}{\mbox{f\ x\ {\isacharequal}\ f\ y}}} 
      \hfill (\isa{arg{\isacharunderscore}cong})
    \item (p.37) \isa{\mbox{}\inferrule{\mbox{a\ {\isacharequal}\ b\ {\isasymand}\ c\ {\isacharequal}\ d}}{\mbox{f\ a\ c\ {\isacharequal}\ f\ b\ d}}} 
      \hfill (\isa{arg{\isacharunderscore}cong{\isadigit{2}}})
  \end{itemize}%
\end{isamarkuptext}\isamarkuptrue%
%
\isadelimdocument
%
\endisadelimdocument
%
\isatagdocument
%
\isamarkupsubsubsection{Igualdad de booleanos - \isa{iff} (2.2.3)%
}
\isamarkuptrue%
%
\endisatagdocument
{\isafolddocument}%
%
\isadelimdocument
%
\endisadelimdocument
%
\begin{isamarkuptext}%
\begin{itemize}
    \item (p.38) \isa{\mbox{}\inferrule{\mbox{Q\ {\isacharequal}\ P\ {\isasymand}\ Q}}{\mbox{P}}} 
      \hfill (\isa{iffD{\isadigit{1}}})
  \end{itemize}%
\end{isamarkuptext}\isamarkuptrue%
%
\isadelimdocument
%
\endisadelimdocument
%
\isatagdocument
%
\isamarkupsubsubsection{Cuantificador universal I (2.2.5)%
}
\isamarkuptrue%
%
\endisatagdocument
{\isafolddocument}%
%
\isadelimdocument
%
\endisadelimdocument
%
\begin{isamarkuptext}%
\begin{itemize}
    \item (p.38) \isa{\mbox{}\inferrule{\mbox{{\isasymforall}x{\isachardot}\ P\ x}\\\ \mbox{\mbox{}\inferrule{\mbox{P\ x}}{\mbox{R}}}}{\mbox{R}}} 
      \hfill (\isa{allE})
  \end{itemize}%
\end{isamarkuptext}\isamarkuptrue%
%
\isadelimdocument
%
\endisadelimdocument
%
\isatagdocument
%
\isamarkupsubsubsection{Negación (2.2.7)%
}
\isamarkuptrue%
%
\endisatagdocument
{\isafolddocument}%
%
\isadelimdocument
%
\endisadelimdocument
%
\begin{isamarkuptext}%
\begin{itemize}
    \item (p.39) \isa{\mbox{}\inferrule{\mbox{\mbox{}\inferrule{\mbox{P}}{\mbox{False}}}}{\mbox{{\isasymnot}\ P}}} 
      \hfill (\isa{notI})
    \item (p.39) \isa{\mbox{}\inferrule{\mbox{{\isasymnot}\ P\ {\isasymand}\ P}}{\mbox{R}}} 
      \hfill (\isa{notE})
  \end{itemize}%
\end{isamarkuptext}\isamarkuptrue%
%
\isadelimdocument
%
\endisadelimdocument
%
\isatagdocument
%
\isamarkupsubsubsection{Implicación (2.2.8)%
}
\isamarkuptrue%
%
\endisatagdocument
{\isafolddocument}%
%
\isadelimdocument
%
\endisadelimdocument
%
\begin{isamarkuptext}%
\begin{itemize}
    \item (p.40) \isa{\mbox{}\inferrule{\mbox{Q}\\\ \mbox{\mbox{}\inferrule{\mbox{P}}{\mbox{{\isasymnot}\ Q}}}}{\mbox{{\isasymnot}\ P}}} 
      \hfill (\isa{contrapos{\isacharunderscore}pn})
  \end{itemize}%
\end{isamarkuptext}\isamarkuptrue%
%
\isadelimdocument
%
\endisadelimdocument
%
\isatagdocument
%
\isamarkupsubsubsection{Disyunción I (2.2.9)%
}
\isamarkuptrue%
%
\endisatagdocument
{\isafolddocument}%
%
\isadelimdocument
%
\endisadelimdocument
%
\begin{isamarkuptext}%
\begin{itemize}
    \item (p.40) \isa{\mbox{}\inferrule{\mbox{P\ {\isasymor}\ Q}\\\ \mbox{\mbox{}\inferrule{\mbox{P}}{\mbox{R}}}\\\ \mbox{\mbox{}\inferrule{\mbox{Q}}{\mbox{R}}}}{\mbox{R}}} 
      \hfill (\isa{disjE})
  \end{itemize}%
\end{isamarkuptext}\isamarkuptrue%
%
\isadelimdocument
%
\endisadelimdocument
%
\isatagdocument
%
\isamarkupsubsubsection{Derivación de \isa{iffI} (2.2.10)%
}
\isamarkuptrue%
%
\endisatagdocument
{\isafolddocument}%
%
\isadelimdocument
%
\endisadelimdocument
%
\begin{isamarkuptext}%
\begin{itemize}
    \item (p.40) \isa{\mbox{}\inferrule{\mbox{\mbox{}\inferrule{\mbox{P}}{\mbox{Q}}}\\\ \mbox{\mbox{}\inferrule{\mbox{Q}}{\mbox{P}}}}{\mbox{P\ {\isacharequal}\ Q}}} 
      \hfill (\isa{iffI})
  \end{itemize}%
\end{isamarkuptext}\isamarkuptrue%
%
\isadelimdocument
%
\endisadelimdocument
%
\isatagdocument
%
\isamarkupsubsubsection{Cuantificador universal II (2.2.12)%
}
\isamarkuptrue%
%
\endisatagdocument
{\isafolddocument}%
%
\isadelimdocument
%
\endisadelimdocument
%
\begin{isamarkuptext}%
\begin{itemize}
    \item (p.41) \isa{\mbox{}\inferrule{\mbox{{\isasymAnd}x{\isachardot}\ P\ x}}{\mbox{{\isasymforall}x{\isachardot}\ P\ x}}} 
      \hfill (\isa{allI})
  \end{itemize}%
\end{isamarkuptext}\isamarkuptrue%
%
\isadelimdocument
%
\endisadelimdocument
%
\isatagdocument
%
\isamarkupsubsubsection{Cuantificador existencia (2.2.13)%
}
\isamarkuptrue%
%
\endisatagdocument
{\isafolddocument}%
%
\isadelimdocument
%
\endisadelimdocument
%
\begin{isamarkuptext}%
\begin{itemize}
    \item (p.41) \isa{\mbox{}\inferrule{\mbox{P\ x}}{\mbox{{\isasymexists}x{\isachardot}\ P\ x}}} 
      \hfill (\isa{exI})
    \item (p.41) \isa{\mbox{}\inferrule{\mbox{{\isasymexists}x{\isachardot}\ P\ x}\\\ \mbox{{\isasymAnd}x{\isachardot}\ \mbox{}\inferrule{\mbox{P\ x}}{\mbox{Q}}}}{\mbox{Q}}} 
      \hfill (\isa{exE})
  \end{itemize}%
\end{isamarkuptext}\isamarkuptrue%
%
\isadelimdocument
%
\endisadelimdocument
%
\isatagdocument
%
\isamarkupsubsubsection{Conjunción (2.2.14)%
}
\isamarkuptrue%
%
\endisatagdocument
{\isafolddocument}%
%
\isadelimdocument
%
\endisadelimdocument
%
\begin{isamarkuptext}%
\begin{itemize}
    \item (p.41) \isa{\mbox{}\inferrule{\mbox{P\ {\isasymand}\ Q}}{\mbox{P\ {\isasymand}\ Q}}} 
      \hfill (\isa{conjI})
    \item (p.41) \isa{\mbox{}\inferrule{\mbox{P\ {\isasymand}\ Q}}{\mbox{P}}} 
      \hfill (\isa{conjunct{\isadigit{1}}})
    \item (p.41) \isa{\mbox{}\inferrule{\mbox{P\ {\isasymand}\ Q}}{\mbox{Q}}} 
      \hfill (\isa{conjunct{\isadigit{2}}})
  \end{itemize}%
\end{isamarkuptext}\isamarkuptrue%
%
\isadelimdocument
%
\endisadelimdocument
%
\isatagdocument
%
\isamarkupsubsubsection{Disyunción II (2.2.15)%
}
\isamarkuptrue%
%
\endisatagdocument
{\isafolddocument}%
%
\isadelimdocument
%
\endisadelimdocument
%
\begin{isamarkuptext}%
\begin{itemize}
    \item (p.42) \isa{\mbox{}\inferrule{\mbox{P}}{\mbox{P\ {\isasymor}\ Q}}} 
      \hfill (\isa{disjI{\isadigit{1}}})
    \item (p.42) \isa{\mbox{}\inferrule{\mbox{Q}}{\mbox{P\ {\isasymor}\ Q}}} 
      \hfill (\isa{disjI{\isadigit{2}}})
  \end{itemize}%
\end{isamarkuptext}\isamarkuptrue%
%
\isadelimdocument
%
\endisadelimdocument
%
\isatagdocument
%
\isamarkupsubsubsection{Atomización de conectivas de nivel intermedio (2.2.20)%
}
\isamarkuptrue%
%
\endisatagdocument
{\isafolddocument}%
%
\isadelimdocument
%
\endisadelimdocument
%
\begin{isamarkuptext}%
\begin{itemize}
    \item (p.46) \isa{{\isacharparenleft}x\ {\isasymequiv}\ y{\isacharparenright}\ {\isasymequiv}\ x\ {\isacharequal}\ y} 
      \hfill (\isa{atomize{\isacharunderscore}eq})
  \end{itemize}%
\end{isamarkuptext}\isamarkuptrue%
%
\isadelimdocument
%
\endisadelimdocument
%
\isatagdocument
%
\isamarkupsubsection{Configuración del paquete (2.3)%
}
\isamarkuptrue%
%
\isamarkupsubsubsection{Simplificadores (2.3.4)%
}
\isamarkuptrue%
%
\endisatagdocument
{\isafolddocument}%
%
\isadelimdocument
%
\endisadelimdocument
%
\begin{isamarkuptext}%
\begin{itemize}
    \item (p.50) \isa{{\isacharparenleft}{\isasymnot}\ False{\isacharparenright}\ {\isacharequal}\ True} 
      \hfill (\isa{not{\isacharunderscore}False{\isacharunderscore}eq{\isacharunderscore}True})
    \item (p.53) \isa{{\isacharparenleft}{\isasymnexists}x{\isachardot}\ P\ x{\isacharparenright}\ {\isacharequal}\ {\isacharparenleft}{\isasymforall}x{\isachardot}\ {\isasymnot}\ P\ x{\isacharparenright}} 
      \hfill (\isa{not{\isacharunderscore}ex})
  \end{itemize}%
\end{isamarkuptext}\isamarkuptrue%
%
\isadelimdocument
%
\endisadelimdocument
%
\isatagdocument
%
\isamarkupsection{Grupos, también combinados con órdenes (5)%
}
\isamarkuptrue%
%
\endisatagdocument
{\isafolddocument}%
%
\isadelimdocument
%
\endisadelimdocument
%
\begin{isamarkuptext}%
Los siguientes resultados pertenecen a la teoría de 
  grupos \href{https://bit.ly/3fwjIPe}{Groups.thy}.%
\end{isamarkuptext}\isamarkuptrue%
%
\isadelimdocument
%
\endisadelimdocument
%
\isatagdocument
%
\isamarkupsubsection{Estructuras abstractas%
}
\isamarkuptrue%
%
\endisatagdocument
{\isafolddocument}%
%
\isadelimdocument
%
\endisadelimdocument
%
\begin{isamarkuptext}%
\begin{itemize}
    \item (p.109) \isa{sup\ bot\ a\ {\isacharequal}\ a} 
      \hfill (\isa{sup{\isacharunderscore}bot{\isachardot}left{\isacharunderscore}neutral})
  \end{itemize}%
\end{isamarkuptext}\isamarkuptrue%
%
\isadelimdocument
%
\endisadelimdocument
%
\isatagdocument
%
\isamarkupsection{Retículos abstractos (6)%
}
\isamarkuptrue%
%
\endisatagdocument
{\isafolddocument}%
%
\isadelimdocument
%
\endisadelimdocument
%
\begin{isamarkuptext}%
Los resultados expuestos a continuación pertenecen a la teoría de 
  retículos \href{https://bit.ly/2N4lbjn}{Lattices.thy}.%
\end{isamarkuptext}\isamarkuptrue%
%
\begin{isamarkuptext}%
\begin{itemize}
    \item (p.139) \isa{{\isacharparenleft}sup\ b\ c\ {\isasymle}\ a{\isacharparenright}\ {\isacharequal}\ {\isacharparenleft}b\ {\isasymle}\ a\ {\isasymand}\ c\ {\isasymle}\ a{\isacharparenright}} 
      \hfill (\isa{sup{\isachardot}bounded{\isacharunderscore}iff})
  \end{itemize}%
\end{isamarkuptext}\isamarkuptrue%
%
\isadelimdocument
%
\endisadelimdocument
%
\isatagdocument
%
\isamarkupsection{Teoría de conjuntos para lógica de orden superior (7)%
}
\isamarkuptrue%
%
\endisatagdocument
{\isafolddocument}%
%
\isadelimdocument
%
\endisadelimdocument
%
\begin{isamarkuptext}%
Los siguientes resultados corresponden a la teoría de conjuntos 
  \href{https://bit.ly/3ePFv4B}{Set.thy}.%
\end{isamarkuptext}\isamarkuptrue%
%
\isadelimdocument
%
\endisadelimdocument
%
\isatagdocument
%
\isamarkupsubsection{Subconjuntos y cuantificadores acotados (7.2)%
}
\isamarkuptrue%
%
\endisatagdocument
{\isafolddocument}%
%
\isadelimdocument
%
\endisadelimdocument
%
\begin{isamarkuptext}%
\begin{itemize}
    \item (p.163) \isa{\mbox{}\inferrule{\mbox{{\isasymAnd}x{\isachardot}\ \mbox{}\inferrule{\mbox{x\ {\isasymin}\ A}}{\mbox{P\ x}}}}{\mbox{{\isasymforall}x{\isasymin}A{\isachardot}\ P\ x}}} 
      \hfill (\isa{ballI})
    \item (p.163) \isa{\mbox{}\inferrule{\mbox{{\isacharparenleft}{\isasymforall}x{\isasymin}A{\isachardot}\ P\ x{\isacharparenright}\ {\isasymand}\ x\ {\isasymin}\ A}}{\mbox{P\ x}}} 
      \hfill (\isa{bspec})
  \end{itemize}%
\end{isamarkuptext}\isamarkuptrue%
%
\isadelimdocument
%
\endisadelimdocument
%
\isatagdocument
%
\isamarkupsubsection{Operaciones básicas (7.3)%
}
\isamarkuptrue%
%
\isamarkupsubsubsection{Subconjuntos (7.3.1)%
}
\isamarkuptrue%
%
\endisatagdocument
{\isafolddocument}%
%
\isadelimdocument
%
\endisadelimdocument
%
\begin{isamarkuptext}%
\begin{itemize}
    \item (p.165) \isa{\mbox{}\inferrule{\mbox{c\ {\isasymin}\ A\ {\isasymand}\ A\ {\isasymsubseteq}\ B}}{\mbox{c\ {\isasymin}\ B}}} 
      \hfill (\isa{rev{\isacharunderscore}subsetD})
    \item (p.166) \isa{A\ {\isasymsubseteq}\ A} 
      \hfill (\isa{subset{\isacharunderscore}refl})
    \item (p.166) \isa{\mbox{}\inferrule{\mbox{A\ {\isasymsubseteq}\ B\ {\isasymand}\ B\ {\isasymsubseteq}\ C}}{\mbox{A\ {\isasymsubseteq}\ C}}} 
      \hfill (\isa{subset{\isacharunderscore}trans})
  \end{itemize}%
\end{isamarkuptext}\isamarkuptrue%
%
\isadelimdocument
%
\endisadelimdocument
%
\isatagdocument
%
\isamarkupsubsubsection{El conjunto vacío (7.3.3)%
}
\isamarkuptrue%
%
\endisatagdocument
{\isafolddocument}%
%
\isadelimdocument
%
\endisadelimdocument
%
\begin{isamarkuptext}%
\begin{itemize}
    \item (p.167) \isa{{\isasymemptyset}\ {\isasymsubseteq}\ A} 
      \hfill (\isa{empty{\isacharunderscore}subsetI})
    \item (p.167) \isa{Ball\ {\isasymemptyset}\ P\ {\isacharequal}\ True} 
      \hfill (\isa{ball{\isacharunderscore}empty})
    \item (p.167) \isa{Bex\ {\isasymemptyset}\ P\ {\isacharequal}\ False} 
      \hfill (\isa{bex{\isacharunderscore}empty})
  \end{itemize}%
\end{isamarkuptext}\isamarkuptrue%
%
\isadelimdocument
%
\endisadelimdocument
%
\isatagdocument
%
\isamarkupsubsubsection{Unión binaria (7.3.8)%
}
\isamarkuptrue%
%
\endisatagdocument
{\isafolddocument}%
%
\isadelimdocument
%
\endisadelimdocument
%
\begin{isamarkuptext}%
\begin{itemize}
    \item (p.169) \isa{{\isacharparenleft}c\ {\isasymin}\ A\ {\isasymunion}\ B{\isacharparenright}\ {\isacharequal}\ {\isacharparenleft}c\ {\isasymin}\ A\ {\isasymor}\ c\ {\isasymin}\ B{\isacharparenright}} 
      \hfill (\isa{Un{\isacharunderscore}iff})
    \item (p.169) \isa{\mbox{}\inferrule{\mbox{c\ {\isasymin}\ A}}{\mbox{c\ {\isasymin}\ A\ {\isasymunion}\ B}}} 
      \hfill (\isa{UnI{\isadigit{1}}})
    \item (p.170) \isa{\mbox{}\inferrule{\mbox{c\ {\isasymin}\ B}}{\mbox{c\ {\isasymin}\ A\ {\isasymunion}\ B}}} 
      \hfill (\isa{UnI{\isadigit{2}}})
  \end{itemize}%
\end{isamarkuptext}\isamarkuptrue%
%
\isadelimdocument
%
\endisadelimdocument
%
\isatagdocument
%
\isamarkupsubsubsection{Aumentar un conjunto - insertar (7.3.10)%
}
\isamarkuptrue%
%
\endisatagdocument
{\isafolddocument}%
%
\isadelimdocument
%
\endisadelimdocument
%
\begin{isamarkuptext}%
\begin{itemize}
    \item (p.171) \isa{List{\isachardot}insert\ x\ xs\ {\isacharequal}\ {\isacharbraceleft}x{\isacharbraceright}\ {\isasymunion}\ xs} 
      \hfill (\isa{set{\isacharunderscore}insert})
  \end{itemize}%
\end{isamarkuptext}\isamarkuptrue%
%
\isadelimdocument
%
\endisadelimdocument
%
\isatagdocument
%
\isamarkupsubsubsection{Conjuntos unitarios, insertar (7.3.11)%
}
\isamarkuptrue%
%
\endisatagdocument
{\isafolddocument}%
%
\isadelimdocument
%
\endisadelimdocument
%
\begin{isamarkuptext}%
\begin{itemize}
    \item (p.172) \isa{a\ {\isasymin}\ {\isacharbraceleft}a{\isacharbraceright}} 
      \hfill (\isa{singletonI})
    \item (p.172) \isa{\mbox{}\inferrule{\mbox{b\ {\isasymin}\ {\isacharbraceleft}a{\isacharbraceright}}}{\mbox{b\ {\isacharequal}\ a}}} 
      \hfill (\isa{singletonD})
    \item (p.172) \isa{{\isacharparenleft}b\ {\isasymin}\ {\isacharbraceleft}a{\isacharbraceright}{\isacharparenright}\ {\isacharequal}\ {\isacharparenleft}b\ {\isacharequal}\ a{\isacharparenright}} 
      \hfill (\isa{singleton{\isacharunderscore}iff})
  \end{itemize}%
\end{isamarkuptext}\isamarkuptrue%
%
\isadelimdocument
%
\endisadelimdocument
%
\isatagdocument
%
\isamarkupsubsubsection{Imagen de un conjunto por una función (7.3.12)%
}
\isamarkuptrue%
%
\endisatagdocument
{\isafolddocument}%
%
\isadelimdocument
%
\endisadelimdocument
%
\begin{isamarkuptext}%
\begin{itemize}
    \item (p.173) \isa{f\ {\isacharbackquote}\ A\ {\isacharequal}\ {\isacharbraceleft}y\ {\isacharbar}\ {\isasymexists}x{\isasymin}A{\isachardot}\ y\ {\isacharequal}\ f\ x{\isacharbraceright}} 
      \hfill (\isa{image{\isacharunderscore}def})
    \item (p.173) \isa{f\ {\isacharbackquote}\ {\isacharparenleft}A\ {\isasymunion}\ B{\isacharparenright}\ {\isacharequal}\ f\ {\isacharbackquote}\ A\ {\isasymunion}\ f\ {\isacharbackquote}\ B} 
      \hfill (\isa{image{\isacharunderscore}Un})
    \item (p.174) \isa{f\ {\isacharbackquote}\ {\isasymemptyset}\ {\isacharequal}\ {\isasymemptyset}} 
      \hfill (\isa{image{\isacharunderscore}empty})
    \item (p.174) \isa{f\ {\isacharbackquote}\ {\isacharparenleft}{\isacharbraceleft}a{\isacharbraceright}\ {\isasymunion}\ B{\isacharparenright}\ {\isacharequal}\ {\isacharbraceleft}f\ a{\isacharbraceright}\ {\isasymunion}\ f\ {\isacharbackquote}\ B} 
      \hfill (\isa{image{\isacharunderscore}insert})
  \end{itemize}%
\end{isamarkuptext}\isamarkuptrue%
%
\isadelimdocument
%
\endisadelimdocument
%
\isatagdocument
%
\isamarkupsubsection{Más operaciones y lemas (7.4)%
}
\isamarkuptrue%
%
\isamarkupsubsubsection{Reglas derivadas sobre subconjuntos (7.4.2)%
}
\isamarkuptrue%
%
\endisatagdocument
{\isafolddocument}%
%
\isadelimdocument
%
\endisadelimdocument
%
\begin{isamarkuptext}%
\begin{itemize}
    \item (p.177) \isa{A\ {\isasymsubseteq}\ A\ {\isasymunion}\ B} 
      \hfill (\isa{Un{\isacharunderscore}upper{\isadigit{1}}})
    \item (p.177) \isa{B\ {\isasymsubseteq}\ A\ {\isasymunion}\ B} 
      \hfill (\isa{Un{\isacharunderscore}upper{\isadigit{2}}})
  \end{itemize}%
\end{isamarkuptext}\isamarkuptrue%
%
\isadelimdocument
%
\endisadelimdocument
%
\isatagdocument
%
\isamarkupsubsubsection{Igualdades sobre la union, intersección, inclusion, 
  etc. (7.4.3)%
}
\isamarkuptrue%
%
\endisatagdocument
{\isafolddocument}%
%
\isadelimdocument
%
\endisadelimdocument
%
\begin{isamarkuptext}%
\begin{itemize}
    \item (p.179) \isa{{\isacharbraceleft}a{\isacharbraceright}\ {\isasymunion}\ A\ {\isacharequal}\ {\isacharbraceleft}a{\isacharbraceright}\ {\isasymunion}\ A} 
      \hfill (\isa{insert{\isacharunderscore}is{\isacharunderscore}Un})
    \item (p.181) \isa{A\ {\isasymunion}\ A\ {\isacharequal}\ A} 
      \hfill (\isa{Un{\isacharunderscore}absorb})
    \item (p.181) \isa{A\ {\isasymunion}\ {\isasymemptyset}\ {\isacharequal}\ A} 
      \hfill (\isa{Un{\isacharunderscore}empty{\isacharunderscore}right})
    \item (p.182) \isa{{\isacharbraceleft}a{\isacharbraceright}\ {\isasymunion}\ B\ {\isasymunion}\ C\ {\isacharequal}\ {\isacharbraceleft}a{\isacharbraceright}\ {\isasymunion}\ {\isacharparenleft}B\ {\isasymunion}\ C{\isacharparenright}} 
      \hfill (\isa{Un{\isacharunderscore}insert{\isacharunderscore}left})
    \item (p.187) \isa{{\isacharparenleft}{\isasymforall}x{\isasymin}A{\isachardot}\ P\ x\ {\isasymor}\ Q{\isacharparenright}\ {\isacharequal}\ {\isacharparenleft}{\isacharparenleft}{\isasymforall}x{\isasymin}A{\isachardot}\ P\ x{\isacharparenright}\ {\isasymor}\ Q{\isacharparenright}\isasep\isanewline%
{\isacharparenleft}{\isasymforall}x{\isasymin}A{\isachardot}\ P\ {\isasymor}\ Q\ x{\isacharparenright}\ {\isacharequal}\ {\isacharparenleft}P\ {\isasymor}\ {\isacharparenleft}{\isasymforall}x{\isasymin}A{\isachardot}\ Q\ x{\isacharparenright}{\isacharparenright}\isasep\isanewline%
{\isacharparenleft}{\isasymforall}x{\isasymin}A{\isachardot}\ P\ {\isasymlongrightarrow}\ Q\ x{\isacharparenright}\ {\isacharequal}\ {\isacharparenleft}P\ {\isasymlongrightarrow}\ {\isacharparenleft}{\isasymforall}x{\isasymin}A{\isachardot}\ Q\ x{\isacharparenright}{\isacharparenright}\isasep\isanewline%
{\isacharparenleft}{\isasymforall}x{\isasymin}A{\isachardot}\ P\ x\ {\isasymlongrightarrow}\ Q{\isacharparenright}\ {\isacharequal}\ {\isacharparenleft}{\isacharparenleft}{\isasymexists}x{\isasymin}A{\isachardot}\ P\ x{\isacharparenright}\ {\isasymlongrightarrow}\ Q{\isacharparenright}\isasep\isanewline%
{\isacharparenleft}{\isasymforall}x{\isasymin}{\isasymemptyset}{\isachardot}\ P\ x{\isacharparenright}\ {\isacharequal}\ True\isasep\isanewline%
{\isacharparenleft}{\isasymforall}x{\isasymin}UNIV{\isachardot}\ P\ x{\isacharparenright}\ {\isacharequal}\ {\isacharparenleft}{\isasymforall}x{\isachardot}\ P\ x{\isacharparenright}\isasep\isanewline%
{\isacharparenleft}{\isasymforall}x{\isasymin}{\isacharbraceleft}a{\isacharbraceright}\ {\isasymunion}\ B{\isachardot}\ P\ x{\isacharparenright}\ {\isacharequal}\ {\isacharparenleft}P\ a\ {\isasymand}\ {\isacharparenleft}{\isasymforall}x{\isasymin}B{\isachardot}\ P\ x{\isacharparenright}{\isacharparenright}\isasep\isanewline%
{\isacharparenleft}{\isasymforall}x{\isasymin}Collect\ Q{\isachardot}\ P\ x{\isacharparenright}\ {\isacharequal}\ {\isacharparenleft}{\isasymforall}x{\isachardot}\ Q\ x\ {\isasymlongrightarrow}\ P\ x{\isacharparenright}\isasep\isanewline%
{\isacharparenleft}{\isasymforall}x{\isasymin}f\ {\isacharbackquote}\ A{\isachardot}\ P\ x{\isacharparenright}\ {\isacharequal}\ {\isacharparenleft}{\isasymforall}x{\isasymin}A{\isachardot}\ P\ {\isacharparenleft}f\ x{\isacharparenright}{\isacharparenright}\isasep\isanewline%
{\isacharparenleft}{\isasymnot}\ {\isacharparenleft}{\isasymforall}x{\isasymin}A{\isachardot}\ P\ x{\isacharparenright}{\isacharparenright}\ {\isacharequal}\ {\isacharparenleft}{\isasymexists}x{\isasymin}A{\isachardot}\ {\isasymnot}\ P\ x{\isacharparenright}} 
      \hfill (\isa{ball{\isacharunderscore}simps})
    \item (p.187) \isa{{\isacharparenleft}{\isasymexists}x{\isasymin}A{\isachardot}\ P\ x\ {\isasymand}\ Q{\isacharparenright}\ {\isacharequal}\ {\isacharparenleft}{\isacharparenleft}{\isasymexists}x{\isasymin}A{\isachardot}\ P\ x{\isacharparenright}\ {\isasymand}\ Q{\isacharparenright}\isasep\isanewline%
{\isacharparenleft}{\isasymexists}x{\isasymin}A{\isachardot}\ P\ {\isasymand}\ Q\ x{\isacharparenright}\ {\isacharequal}\ {\isacharparenleft}P\ {\isasymand}\ {\isacharparenleft}{\isasymexists}x{\isasymin}A{\isachardot}\ Q\ x{\isacharparenright}{\isacharparenright}\isasep\isanewline%
{\isacharparenleft}{\isasymexists}x{\isasymin}{\isasymemptyset}{\isachardot}\ P\ x{\isacharparenright}\ {\isacharequal}\ False\isasep\isanewline%
{\isacharparenleft}{\isasymexists}x{\isasymin}UNIV{\isachardot}\ P\ x{\isacharparenright}\ {\isacharequal}\ {\isacharparenleft}{\isasymexists}x{\isachardot}\ P\ x{\isacharparenright}\isasep\isanewline%
{\isacharparenleft}{\isasymexists}x{\isasymin}{\isacharbraceleft}a{\isacharbraceright}\ {\isasymunion}\ B{\isachardot}\ P\ x{\isacharparenright}\ {\isacharequal}\ {\isacharparenleft}P\ a\ {\isasymor}\ {\isacharparenleft}{\isasymexists}x{\isasymin}B{\isachardot}\ P\ x{\isacharparenright}{\isacharparenright}\isasep\isanewline%
{\isacharparenleft}{\isasymexists}x{\isasymin}Collect\ Q{\isachardot}\ P\ x{\isacharparenright}\ {\isacharequal}\ {\isacharparenleft}{\isasymexists}x{\isachardot}\ Q\ x\ {\isasymand}\ P\ x{\isacharparenright}\isasep\isanewline%
{\isacharparenleft}{\isasymexists}x{\isasymin}f\ {\isacharbackquote}\ A{\isachardot}\ P\ x{\isacharparenright}\ {\isacharequal}\ {\isacharparenleft}{\isasymexists}x{\isasymin}A{\isachardot}\ P\ {\isacharparenleft}f\ x{\isacharparenright}{\isacharparenright}\isasep\isanewline%
{\isacharparenleft}{\isasymnot}\ {\isacharparenleft}{\isasymexists}x{\isasymin}A{\isachardot}\ P\ x{\isacharparenright}{\isacharparenright}\ {\isacharequal}\ {\isacharparenleft}{\isasymforall}x{\isasymin}A{\isachardot}\ {\isasymnot}\ P\ x{\isacharparenright}} 
      \hfill (\isa{bex{\isacharunderscore}simps})
  \end{itemize}%
\end{isamarkuptext}\isamarkuptrue%
%
\isadelimdocument
%
\endisadelimdocument
%
\isatagdocument
%
\isamarkupsubsubsection{Monotonía de varias operaciones (7.4.4)%
}
\isamarkuptrue%
%
\endisatagdocument
{\isafolddocument}%
%
\isadelimdocument
%
\endisadelimdocument
%
\begin{isamarkuptext}%
\begin{itemize}
    \item (p.188) \isa{\mbox{}\inferrule{\mbox{A\ {\isasymsubseteq}\ C\ {\isasymand}\ B\ {\isasymsubseteq}\ D}}{\mbox{A\ {\isasymunion}\ B\ {\isasymsubseteq}\ C\ {\isasymunion}\ D}}} 
      \hfill (\isa{Un{\isacharunderscore}mono})
    \item (p.188) \isa{P\ {\isasymlongrightarrow}\ P} 
      \hfill (\isa{imp{\isacharunderscore}refl})
    \item (p.188) \isa{\mbox{}\inferrule{\mbox{Q\ {\isasymlongrightarrow}\ P}}{\mbox{{\isasymnot}\ P\ {\isasymlongrightarrow}\ {\isasymnot}\ Q}}} 
      \hfill (\isa{not{\isacharunderscore}mono})
  \end{itemize}%
\end{isamarkuptext}\isamarkuptrue%
%
\isadelimdocument
%
\endisadelimdocument
%
\isatagdocument
%
\isamarkupsection{Nociones sobre funciones (9)%
}
\isamarkuptrue%
%
\endisatagdocument
{\isafolddocument}%
%
\isadelimdocument
%
\endisadelimdocument
%
\begin{isamarkuptext}%
En Isabelle, la teoría de funciones se corresponde con 
  \href{https://bit.ly/2VBe1Im}{Fun.thy}.%
\end{isamarkuptext}\isamarkuptrue%
%
\isadelimdocument
%
\endisadelimdocument
%
\isatagdocument
%
\isamarkupsubsection{Actualización de funciones (9.6)%
}
\isamarkuptrue%
%
\endisatagdocument
{\isafolddocument}%
%
\isadelimdocument
%
\endisadelimdocument
%
\begin{isamarkuptext}%
\begin{itemize}
    \item (p.212) \isa{f{\isacharparenleft}a\ {\isacharcolon}{\isacharequal}\ b{\isacharparenright}\ {\isacharequal}\ {\isacharparenleft}{\isasymlambda}x{\isachardot}\ \textsf{if}\ x\ {\isacharequal}\ a\ \textsf{then}\ b\ \textsf{else}\ f\ x{\isacharparenright}} 
      \hfill (\isa{fun{\isacharunderscore}upd{\isacharunderscore}def})
    \item (p.213) \isa{\mbox{}\inferrule{\mbox{z\ {\isasymnoteq}\ x}}{\mbox{{\isacharparenleft}f{\isacharparenleft}x\ {\isacharcolon}{\isacharequal}\ y{\isacharparenright}{\isacharparenright}\ z\ {\isacharequal}\ f\ z}}} 
      \hfill (\isa{fun{\isacharunderscore}upd{\isacharunderscore}other})
  \end{itemize}%
\end{isamarkuptext}\isamarkuptrue%
%
\isadelimdocument
%
\endisadelimdocument
%
\isatagdocument
%
\isamarkupsection{Retículos completos (10)%
}
\isamarkuptrue%
%
\endisatagdocument
{\isafolddocument}%
%
\isadelimdocument
%
\endisadelimdocument
%
\begin{isamarkuptext}%
En Isabelle corresponde a la teoría 
  \href{https://bit.ly/2Y5wxKA}{Complete-Lattices.thy}.%
\end{isamarkuptext}\isamarkuptrue%
%
\isadelimdocument
%
\endisadelimdocument
%
\isatagdocument
%
\isamarkupsubsection{Retículos completos en conjuntos (10.6)%
}
\isamarkuptrue%
%
\isamarkupsubsubsection{Unión (10.6.3)%
}
\isamarkuptrue%
%
\endisatagdocument
{\isafolddocument}%
%
\isadelimdocument
%
\endisadelimdocument
%
\begin{isamarkuptext}%
\begin{itemize}
    \item (p.238) \isa{{\isasymUnion}\ {\isasymemptyset}\ {\isacharequal}\ {\isasymemptyset}} 
      \hfill (\isa{Union{\isacharunderscore}empty})
  \end{itemize}%
\end{isamarkuptext}\isamarkuptrue%
%
\isadelimdocument
%
\endisadelimdocument
%
\isatagdocument
%
\isamarkupsection{Conjuntos finitos (18)%
}
\isamarkuptrue%
%
\endisatagdocument
{\isafolddocument}%
%
\isadelimdocument
%
\endisadelimdocument
%
\begin{isamarkuptext}%
A continuación se muestran resultados relativos a la teoría 
  \href{https://bit.ly/3bEIScG}{Finite-Set.thy}.%
\end{isamarkuptext}\isamarkuptrue%
%
\isadelimdocument
%
\endisadelimdocument
%
\isatagdocument
%
\isamarkupsubsection{Predicado de conjuntos finitos (18.1)%
}
\isamarkuptrue%
%
\endisatagdocument
{\isafolddocument}%
%
\isadelimdocument
%
\endisadelimdocument
%
\begin{isamarkuptext}%
\begin{itemize}
    \item (p.419) \isa{finite\ A} 
      \hfill (\isa{finite})
  \end{itemize}%
\end{isamarkuptext}\isamarkuptrue%
%
\isadelimdocument
%
\endisadelimdocument
%
\isatagdocument
%
\isamarkupsubsection{Finitud y operaciones de conjuntos comunes (18.2)%
}
\isamarkuptrue%
%
\endisatagdocument
{\isafolddocument}%
%
\isadelimdocument
%
\endisadelimdocument
%
\begin{isamarkuptext}%
\begin{itemize}
    \item (p.422) \isa{\mbox{}\inferrule{\mbox{finite\ F\ {\isasymand}\ finite\ G}}{\mbox{finite\ {\isacharparenleft}F\ {\isasymunion}\ G{\isacharparenright}}}} 
      \hfill (\isa{finite{\isacharunderscore}UnI})
    \item (p.423) \isa{finite\ {\isacharparenleft}{\isacharbraceleft}a{\isacharbraceright}\ {\isasymunion}\ A{\isacharparenright}\ {\isacharequal}\ finite\ A} 
      \hfill (\isa{finite{\isacharunderscore}insert})
  \end{itemize}%
\end{isamarkuptext}\isamarkuptrue%
%
\isadelimdocument
%
\endisadelimdocument
%
\isatagdocument
%
\isamarkupsection{Composición de functores naturales acotados (33)%
}
\isamarkuptrue%
%
\endisatagdocument
{\isafolddocument}%
%
\isadelimdocument
%
\endisadelimdocument
%
\begin{isamarkuptext}%
En esta sección se muestran resultados pertenecientes a la
  teoría de composición de functores naturales acotados de Isabelle 
  \href{https://bit.ly/2zGl9v6}{BNFComposition.thy}.%
\end{isamarkuptext}\isamarkuptrue%
%
\begin{isamarkuptext}%
\begin{itemize}
    \item (p.718) \isa{{\isasymUnion}\ {\isacharparenleft}f\ {\isacharbackquote}\ {\isacharparenleft}{\isacharbraceleft}a{\isacharbraceright}\ {\isasymunion}\ B{\isacharparenright}{\isacharparenright}\ {\isacharequal}\ f\ a\ {\isasymunion}\ {\isasymUnion}\ {\isacharparenleft}f\ {\isacharbackquote}\ B{\isacharparenright}} 
      \hfill (\isa{Union{\isacharunderscore}image{\isacharunderscore}insert})
  \end{itemize}%
\end{isamarkuptext}\isamarkuptrue%
%
\isadelimdocument
%
\endisadelimdocument
%
\isatagdocument
%
\isamarkupsection{El tipo de datos de la listas finitas (66)%
}
\isamarkuptrue%
%
\endisatagdocument
{\isafolddocument}%
%
\isadelimdocument
%
\endisadelimdocument
%
\begin{isamarkuptext}%
En esta sección se muestran resultados sobre listas finitas 
  dentro de la teoría de listas de Isabelle 
  \href{https://bit.ly/3bCNxvX}{List.thy}.%
\end{isamarkuptext}\isamarkuptrue%
%
\begin{isamarkuptext}%
\begin{itemize}
    \item (p.1169) \isa{{\isacharbrackleft}{\isacharbrackright}\ {\isacharequal}\ {\isasymemptyset}\isasep\isanewline%
x{\isadigit{2}}{\isadigit{1}}\ {\isasymcdot}\ x{\isadigit{2}}{\isadigit{2}}\ {\isacharequal}\ {\isacharbraceleft}x{\isadigit{2}}{\isadigit{1}}{\isacharbraceright}\ {\isasymunion}\ x{\isadigit{2}}{\isadigit{2}}} 
      \hfill (\isa{list{\isachardot}set})
  \end{itemize}%
\end{isamarkuptext}\isamarkuptrue%
%
\isadelimdocument
%
\endisadelimdocument
%
\isatagdocument
%
\isamarkupsubsection{Funciones básicas de procesamiento de listas (66.1)%
}
\isamarkuptrue%
%
\isamarkupsubsubsection{Función \isa{set}%
}
\isamarkuptrue%
%
\endisatagdocument
{\isafolddocument}%
%
\isadelimdocument
%
\endisadelimdocument
%
\begin{isamarkuptext}%
\begin{itemize}
    \item (p.1195) \isa{xs\ \isacharat\ ys\ {\isacharequal}\ xs\ {\isasymunion}\ ys} 
      \hfill (\isa{set{\isacharunderscore}append})
  \end{itemize}%
\end{isamarkuptext}\isamarkuptrue%
%
\isadelimtheory
%
\endisadelimtheory
%
\isatagtheory
%
\endisatagtheory
{\isafoldtheory}%
%
\isadelimtheory
%
\endisadelimtheory
%
\end{isabellebody}%
\endinput
%:%file=~/TFM/TFM/Glosario.thy%:%
%:%19=13%:%
%:%20=14%:%
%:%21=15%:%
%:%30=17%:%
%:%42=19%:%
%:%43=20%:%
%:%52=22%:%
%:%56=24%:%
%:%68=27%:%
%:%69=28%:%
%:%70=29%:%
%:%71=30%:%
%:%80=32%:%
%:%92=35%:%
%:%93=36%:%
%:%94=37%:%
%:%95=38%:%
%:%96=39%:%
%:%97=40%:%
%:%106=42%:%
%:%110=44%:%
%:%122=47%:%
%:%123=48%:%
%:%124=49%:%
%:%125=50%:%
%:%126=51%:%
%:%127=52%:%
%:%136=54%:%
%:%148=57%:%
%:%149=58%:%
%:%150=59%:%
%:%151=60%:%
%:%160=62%:%
%:%172=65%:%
%:%173=66%:%
%:%174=67%:%
%:%175=68%:%
%:%184=70%:%
%:%196=73%:%
%:%197=74%:%
%:%198=75%:%
%:%199=76%:%
%:%200=77%:%
%:%201=78%:%
%:%210=80%:%
%:%222=83%:%
%:%223=84%:%
%:%224=85%:%
%:%225=86%:%
%:%234=88%:%
%:%246=91%:%
%:%247=92%:%
%:%248=93%:%
%:%249=94%:%
%:%258=96%:%
%:%270=99%:%
%:%271=100%:%
%:%272=101%:%
%:%273=102%:%
%:%282=104%:%
%:%294=107%:%
%:%295=108%:%
%:%296=109%:%
%:%297=110%:%
%:%306=112%:%
%:%318=115%:%
%:%319=116%:%
%:%320=117%:%
%:%321=118%:%
%:%322=119%:%
%:%323=120%:%
%:%332=122%:%
%:%344=125%:%
%:%345=126%:%
%:%346=127%:%
%:%347=128%:%
%:%348=129%:%
%:%349=130%:%
%:%350=131%:%
%:%351=132%:%
%:%360=134%:%
%:%372=137%:%
%:%373=138%:%
%:%374=139%:%
%:%375=140%:%
%:%376=141%:%
%:%377=142%:%
%:%386=144%:%
%:%398=147%:%
%:%399=148%:%
%:%400=149%:%
%:%401=150%:%
%:%410=152%:%
%:%414=154%:%
%:%426=157%:%
%:%427=158%:%
%:%428=159%:%
%:%429=160%:%
%:%430=161%:%
%:%431=162%:%
%:%440=164%:%
%:%452=166%:%
%:%453=167%:%
%:%462=169%:%
%:%474=172%:%
%:%475=173%:%
%:%476=174%:%
%:%477=175%:%
%:%486=177%:%
%:%498=179%:%
%:%499=180%:%
%:%503=183%:%
%:%504=184%:%
%:%505=185%:%
%:%506=186%:%
%:%515=188%:%
%:%527=190%:%
%:%528=191%:%
%:%537=193%:%
%:%549=196%:%
%:%550=197%:%
%:%551=198%:%
%:%552=199%:%
%:%553=200%:%
%:%554=201%:%
%:%563=203%:%
%:%567=205%:%
%:%579=208%:%
%:%580=209%:%
%:%581=210%:%
%:%582=211%:%
%:%583=212%:%
%:%584=213%:%
%:%585=214%:%
%:%586=215%:%
%:%595=217%:%
%:%607=220%:%
%:%608=221%:%
%:%609=222%:%
%:%610=223%:%
%:%611=224%:%
%:%612=225%:%
%:%613=226%:%
%:%614=227%:%
%:%623=229%:%
%:%635=232%:%
%:%636=233%:%
%:%637=234%:%
%:%638=235%:%
%:%639=236%:%
%:%640=237%:%
%:%641=238%:%
%:%642=239%:%
%:%651=241%:%
%:%663=244%:%
%:%664=245%:%
%:%665=246%:%
%:%666=247%:%
%:%675=249%:%
%:%687=252%:%
%:%688=253%:%
%:%689=254%:%
%:%690=255%:%
%:%691=256%:%
%:%692=257%:%
%:%693=258%:%
%:%694=259%:%
%:%703=262%:%
%:%715=265%:%
%:%716=266%:%
%:%717=267%:%
%:%718=268%:%
%:%719=269%:%
%:%720=270%:%
%:%721=271%:%
%:%722=272%:%
%:%723=273%:%
%:%724=274%:%
%:%733=277%:%
%:%737=279%:%
%:%749=282%:%
%:%750=283%:%
%:%751=284%:%
%:%752=285%:%
%:%753=286%:%
%:%754=287%:%
%:%763=289%:%
%:%764=290%:%
%:%776=293%:%
%:%777=294%:%
%:%778=295%:%
%:%779=296%:%
%:%780=297%:%
%:%781=298%:%
%:%782=299%:%
%:%783=300%:%
%:%784=301%:%
%:%785=302%:%
%:%794=302%:%
%:%795=303%:%
%:%796=304%:%
%:%803=304%:%
%:%804=305%:%
%:%805=306%:%
%:%814=308%:%
%:%826=311%:%
%:%827=312%:%
%:%828=313%:%
%:%829=314%:%
%:%830=315%:%
%:%831=316%:%
%:%832=317%:%
%:%833=318%:%
%:%842=321%:%
%:%854=323%:%
%:%855=324%:%
%:%864=326%:%
%:%876=329%:%
%:%877=330%:%
%:%878=331%:%
%:%879=332%:%
%:%880=333%:%
%:%881=334%:%
%:%890=336%:%
%:%902=338%:%
%:%903=339%:%
%:%912=341%:%
%:%916=343%:%
%:%928=346%:%
%:%929=347%:%
%:%930=348%:%
%:%931=349%:%
%:%940=351%:%
%:%952=353%:%
%:%953=354%:%
%:%962=356%:%
%:%974=359%:%
%:%975=360%:%
%:%976=361%:%
%:%977=362%:%
%:%986=364%:%
%:%998=367%:%
%:%999=368%:%
%:%1000=369%:%
%:%1001=370%:%
%:%1002=371%:%
%:%1003=372%:%
%:%1012=374%:%
%:%1024=376%:%
%:%1025=377%:%
%:%1026=378%:%
%:%1030=381%:%
%:%1031=382%:%
%:%1032=383%:%
%:%1033=384%:%
%:%1042=386%:%
%:%1054=388%:%
%:%1055=389%:%
%:%1056=390%:%
%:%1060=393%:%
%:%1061=394%:%
%:%1062=394%:%
%:%1063=395%:%
%:%1064=396%:%
%:%1073=398%:%
%:%1077=400%:%
%:%1089=403%:%
%:%1090=404%:%
%:%1091=405%:%
%:%1092=406%:%