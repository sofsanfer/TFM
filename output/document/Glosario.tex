%
\begin{isabellebody}%
\setisabellecontext{Glosario}%
%
\isadelimtheory
%
\endisadelimtheory
%
\isatagtheory
%
\endisatagtheory
{\isafoldtheory}%
%
\isadelimtheory
%
\endisadelimtheory
%
\begin{isamarkuptext}%
En este glosario se recoge la lista de los lemas y reglas usadas
  indicando la página del \href{https://acortar.link/BytjC}{libro de HOL} 
  donde se encuentran.%
\end{isamarkuptext}\isamarkuptrue%
%
\isadelimdocument
%
\endisadelimdocument
%
\isatagdocument
%
\isamarkupsection{La base de lógica de primer orden (1)%
}
\isamarkuptrue%
%
\endisatagdocument
{\isafolddocument}%
%
\isadelimdocument
%
\endisadelimdocument
%
\begin{isamarkuptext}%
En Isabelle corresponde a la teoría 
  \href{https://acortar.link/qTQCQ}{HOL.thy}%
\end{isamarkuptext}\isamarkuptrue%
%
\begin{isamarkuptext}%
\begin{itemize}
    \item (p.35) \isa{\mbox{}\inferrule{\mbox{\mbox{}\inferrule{\mbox{P}}{\mbox{Q}}}}{\mbox{P\ {\isasymlongrightarrow}\ Q}}} 
      \hfill (\isa{impI})
    \item (p.35) \isa{\mbox{}\inferrule{\mbox{{\isacharparenleft}P\ {\isasymlongrightarrow}\ Q{\isacharparenright}\ {\isasymand}\ P}}{\mbox{Q}}} 
      \hfill (\isa{mp})
    \item (p.35) \isa{Let\ s\ f\ {\isasymequiv}\ f\ s} 
      \hfill (\isa{Let{\isacharunderscore}def})
    \item (p.36) \isa{\mbox{}\inferrule{\mbox{a\ {\isacharequal}\ b\ {\isasymand}\ P\ b}}{\mbox{P\ a}}} 
      \hfill (\isa{forw{\isacharunderscore}subst})
    \item (p.36) \isa{\mbox{}\inferrule{\mbox{P\ a\ {\isasymand}\ a\ {\isacharequal}\ b}}{\mbox{P\ b}}} 
      \hfill (\isa{back{\isacharunderscore}subst})
    \item (p.36) \isa{\mbox{}\inferrule{\mbox{Q\ {\isacharequal}\ P\ {\isasymand}\ Q}}{\mbox{P}}} 
      \hfill (\isa{iffD{\isadigit{1}}})
    \item (p.37) \isa{\mbox{}\inferrule{\mbox{{\isasymforall}x{\isachardot}\ P\ x}\\\ \mbox{\mbox{}\inferrule{\mbox{P\ x}}{\mbox{R}}}}{\mbox{R}}} 
      \hfill (\isa{allE})
    \item (p.38) \isa{\mbox{}\inferrule{\mbox{{\isasymnot}\ P\ {\isasymand}\ P}}{\mbox{R}}} 
      \hfill (\isa{notE})
    \item (p.38) \isa{\mbox{}\inferrule{\mbox{{\isasymnot}\ Q}\\\ \mbox{\mbox{}\inferrule{\mbox{P}}{\mbox{Q}}}}{\mbox{{\isasymnot}\ P}}} 
      \hfill (\isa{contrapos{\isacharunderscore}nn})
    \item (p.39) \isa{\mbox{}\inferrule{\mbox{P\ {\isasymor}\ Q}\\\ \mbox{\mbox{}\inferrule{\mbox{P}}{\mbox{R}}}\\\ \mbox{\mbox{}\inferrule{\mbox{Q}}{\mbox{R}}}}{\mbox{R}}} 
      \hfill (\isa{disjE})
    \item (p.39) \isa{\mbox{}\inferrule{\mbox{\mbox{}\inferrule{\mbox{P}}{\mbox{Q}}}\\\ \mbox{\mbox{}\inferrule{\mbox{Q}}{\mbox{P}}}}{\mbox{P\ {\isacharequal}\ Q}}} 
      \hfill (\isa{iffI})
    \item (p.40) \isa{\mbox{}\inferrule{\mbox{{\isasymAnd}x{\isachardot}\ P\ x}}{\mbox{{\isasymforall}x{\isachardot}\ P\ x}}} 
      \hfill (\isa{allI})
    \item (p.40) \isa{\mbox{}\inferrule{\mbox{P\ x}}{\mbox{{\isasymexists}x{\isachardot}\ P\ x}}} 
      \hfill (\isa{exI})
    \item (p.40) \isa{\mbox{}\inferrule{\mbox{{\isasymexists}x{\isachardot}\ P\ x}\\\ \mbox{{\isasymAnd}x{\isachardot}\ \mbox{}\inferrule{\mbox{P\ x}}{\mbox{Q}}}}{\mbox{Q}}} 
      \hfill (\isa{exE})
    \item (p.40) \isa{\mbox{}\inferrule{\mbox{P\ {\isasymand}\ Q}}{\mbox{P\ {\isasymand}\ Q}}} 
      \hfill (\isa{conjI})
    \item (p.40) \isa{\mbox{}\inferrule{\mbox{P\ {\isasymand}\ Q}}{\mbox{P}}} 
      \hfill (\isa{conjunct{\isadigit{1}}})
    \item (p.40) \isa{\mbox{}\inferrule{\mbox{P\ {\isasymand}\ Q}}{\mbox{Q}}} 
      \hfill (\isa{conjunct{\isadigit{2}}})
    \item (p.41) \isa{\mbox{}\inferrule{\mbox{P}}{\mbox{P\ {\isasymor}\ Q}}} 
      \hfill (\isa{disjI{\isadigit{1}}})
    \item (p.41) \isa{\mbox{}\inferrule{\mbox{Q}}{\mbox{P\ {\isasymor}\ Q}}} 
      \hfill (\isa{disjI{\isadigit{2}}})
    \item (p.41) \isa{\mbox{}\inferrule{\mbox{\mbox{}\inferrule{\mbox{{\isasymnot}\ P}}{\mbox{False}}}}{\mbox{P}}} 
      \hfill (\isa{ccontr})
    \item (p.41) \isa{\mbox{}\inferrule{\mbox{{\isasymnot}\ {\isasymnot}\ P}}{\mbox{P}}} 
      \hfill (\isa{notnotD})
    \item (p.49) \isa{{\isacharparenleft}{\isasymnot}\ {\isasymnot}\ P{\isacharparenright}\ {\isacharequal}\ P\isasep\isanewline%
{\isacharparenleft}{\isacharparenleft}{\isasymnot}\ P{\isacharparenright}\ {\isacharequal}\ {\isacharparenleft}{\isasymnot}\ Q{\isacharparenright}{\isacharparenright}\ {\isacharequal}\ {\isacharparenleft}P\ {\isacharequal}\ Q{\isacharparenright}\isasep\isanewline%
{\isacharparenleft}P\ {\isasymnoteq}\ Q{\isacharparenright}\ {\isacharequal}\ {\isacharparenleft}P\ {\isacharequal}\ {\isacharparenleft}{\isasymnot}\ Q{\isacharparenright}{\isacharparenright}\isasep\isanewline%
{\isacharparenleft}P\ {\isasymor}\ {\isasymnot}\ P{\isacharparenright}\ {\isacharequal}\ True\isasep\isanewline%
{\isacharparenleft}{\isasymnot}\ P\ {\isasymor}\ P{\isacharparenright}\ {\isacharequal}\ True\isasep\isanewline%
{\isacharparenleft}x\ {\isacharequal}\ x{\isacharparenright}\ {\isacharequal}\ True\isasep\isanewline%
{\isacharparenleft}{\isasymnot}\ True{\isacharparenright}\ {\isacharequal}\ False\isasep\isanewline%
{\isacharparenleft}{\isasymnot}\ False{\isacharparenright}\ {\isacharequal}\ True\isasep\isanewline%
{\isacharparenleft}{\isasymnot}\ P{\isacharparenright}\ {\isasymnoteq}\ P\isasep\isanewline%
P\ {\isasymnoteq}\ {\isacharparenleft}{\isasymnot}\ P{\isacharparenright}\isasep\isanewline%
{\isacharparenleft}True\ {\isacharequal}\ P{\isacharparenright}\ {\isacharequal}\ P\isasep\isanewline%
{\isacharparenleft}P\ {\isacharequal}\ True{\isacharparenright}\ {\isacharequal}\ P\isasep\isanewline%
{\isacharparenleft}False\ {\isacharequal}\ P{\isacharparenright}\ {\isacharequal}\ {\isacharparenleft}{\isasymnot}\ P{\isacharparenright}\isasep\isanewline%
{\isacharparenleft}P\ {\isacharequal}\ False{\isacharparenright}\ {\isacharequal}\ {\isacharparenleft}{\isasymnot}\ P{\isacharparenright}\isasep\isanewline%
{\isacharparenleft}True\ {\isasymlongrightarrow}\ P{\isacharparenright}\ {\isacharequal}\ P\isasep\isanewline%
{\isacharparenleft}False\ {\isasymlongrightarrow}\ P{\isacharparenright}\ {\isacharequal}\ True\isasep\isanewline%
{\isacharparenleft}P\ {\isasymlongrightarrow}\ True{\isacharparenright}\ {\isacharequal}\ True\isasep\isanewline%
{\isacharparenleft}P\ {\isasymlongrightarrow}\ P{\isacharparenright}\ {\isacharequal}\ True\isasep\isanewline%
{\isacharparenleft}P\ {\isasymlongrightarrow}\ False{\isacharparenright}\ {\isacharequal}\ {\isacharparenleft}{\isasymnot}\ P{\isacharparenright}\isasep\isanewline%
{\isacharparenleft}P\ {\isasymlongrightarrow}\ {\isasymnot}\ P{\isacharparenright}\ {\isacharequal}\ {\isacharparenleft}{\isasymnot}\ P{\isacharparenright}\isasep\isanewline%
{\isacharparenleft}P\ {\isasymand}\ True{\isacharparenright}\ {\isacharequal}\ P\isasep\isanewline%
{\isacharparenleft}True\ {\isasymand}\ P{\isacharparenright}\ {\isacharequal}\ P\isasep\isanewline%
{\isacharparenleft}P\ {\isasymand}\ False{\isacharparenright}\ {\isacharequal}\ False\isasep\isanewline%
{\isacharparenleft}False\ {\isasymand}\ P{\isacharparenright}\ {\isacharequal}\ False\isasep\isanewline%
{\isacharparenleft}P\ {\isasymand}\ P{\isacharparenright}\ {\isacharequal}\ P\isasep\isanewline%
{\isacharparenleft}P\ {\isasymand}\ P\ {\isasymand}\ Q{\isacharparenright}\ {\isacharequal}\ {\isacharparenleft}P\ {\isasymand}\ Q{\isacharparenright}\isasep\isanewline%
{\isacharparenleft}P\ {\isasymand}\ {\isasymnot}\ P{\isacharparenright}\ {\isacharequal}\ False\isasep\isanewline%
{\isacharparenleft}{\isasymnot}\ P\ {\isasymand}\ P{\isacharparenright}\ {\isacharequal}\ False\isasep\isanewline%
{\isacharparenleft}P\ {\isasymor}\ True{\isacharparenright}\ {\isacharequal}\ True\isasep\isanewline%
{\isacharparenleft}True\ {\isasymor}\ P{\isacharparenright}\ {\isacharequal}\ True\isasep\isanewline%
{\isacharparenleft}P\ {\isasymor}\ False{\isacharparenright}\ {\isacharequal}\ P\isasep\isanewline%
{\isacharparenleft}False\ {\isasymor}\ P{\isacharparenright}\ {\isacharequal}\ P\isasep\isanewline%
{\isacharparenleft}P\ {\isasymor}\ P{\isacharparenright}\ {\isacharequal}\ P\isasep\isanewline%
{\isacharparenleft}P\ {\isasymor}\ P\ {\isasymor}\ Q{\isacharparenright}\ {\isacharequal}\ {\isacharparenleft}P\ {\isasymor}\ Q{\isacharparenright}\isasep\isanewline%
{\isacharparenleft}{\isasymforall}x{\isachardot}\ P{\isacharparenright}\ {\isacharequal}\ P\isasep\isanewline%
{\isacharparenleft}{\isasymexists}x{\isachardot}\ P{\isacharparenright}\ {\isacharequal}\ P\isasep\isanewline%
{\isasymexists}x{\isachardot}\ x\ {\isacharequal}\ t\isasep\isanewline%
{\isasymexists}x{\isachardot}\ t\ {\isacharequal}\ x\isasep\isanewline%
{\isacharparenleft}{\isasymexists}x{\isachardot}\ x\ {\isacharequal}\ t\ {\isasymand}\ P\ x{\isacharparenright}\ {\isacharequal}\ P\ t\isasep\isanewline%
{\isacharparenleft}{\isasymexists}x{\isachardot}\ t\ {\isacharequal}\ x\ {\isasymand}\ P\ x{\isacharparenright}\ {\isacharequal}\ P\ t\isasep\isanewline%
{\isacharparenleft}{\isasymforall}x{\isachardot}\ x\ {\isacharequal}\ t\ {\isasymlongrightarrow}\ P\ x{\isacharparenright}\ {\isacharequal}\ P\ t\isasep\isanewline%
{\isacharparenleft}{\isasymforall}x{\isachardot}\ t\ {\isacharequal}\ x\ {\isasymlongrightarrow}\ P\ x{\isacharparenright}\ {\isacharequal}\ P\ t\isasep\isanewline%
{\isacharparenleft}{\isasymforall}x{\isachardot}\ x\ {\isasymnoteq}\ t{\isacharparenright}\ {\isacharequal}\ False\isasep\isanewline%
{\isacharparenleft}{\isasymforall}x{\isachardot}\ t\ {\isasymnoteq}\ x{\isacharparenright}\ {\isacharequal}\ False} 
      \hfill (\isa{simp{\isacharunderscore}thms})
    \item (p.49) \isa{{\isacharparenleft}{\isasymnot}\ {\isasymnot}\ P{\isacharparenright}\ {\isacharequal}\ P} 
      \hfill (\isa{not{\isacharunderscore}not})
    \item (p.50) \isa{{\isacharparenleft}A\ {\isasymor}\ A{\isacharparenright}\ {\isacharequal}\ A} 
      \hfill (\isa{disj{\isacharunderscore}absorb})
    \item (p.50) \isa{{\isacharparenleft}A\ {\isasymand}\ A{\isacharparenright}\ {\isacharequal}\ A} 
      \hfill (\isa{conj{\isacharunderscore}absorb})
    \item (p.50) \isa{{\isacharparenleft}{\isacharparenleft}P\ {\isasymand}\ Q{\isacharparenright}\ {\isasymand}\ R{\isacharparenright}\ {\isacharequal}\ {\isacharparenleft}P\ {\isasymand}\ Q\ {\isasymand}\ R{\isacharparenright}} 
      \hfill (\isa{conj{\isacharunderscore}assoc})
    \item (p.50) \isa{{\isacharparenleft}{\isacharparenleft}P\ {\isasymor}\ Q{\isacharparenright}\ {\isasymor}\ R{\isacharparenright}\ {\isacharequal}\ {\isacharparenleft}P\ {\isasymor}\ Q\ {\isasymor}\ R{\isacharparenright}} 
      \hfill (\isa{disj{\isacharunderscore}assoc})
    \item (p.51) \isa{{\isacharparenleft}{\isasymnot}\ {\isacharparenleft}P\ {\isasymor}\ Q{\isacharparenright}{\isacharparenright}\ {\isacharequal}\ {\isacharparenleft}{\isasymnot}\ P\ {\isasymand}\ {\isasymnot}\ Q{\isacharparenright}} 
      \hfill (\isa{de{\isacharunderscore}Morgan{\isacharunderscore}disj})
    \item (p.51) \isa{{\isacharparenleft}{\isasymnot}\ {\isacharparenleft}P\ {\isasymand}\ Q{\isacharparenright}{\isacharparenright}\ {\isacharequal}\ {\isacharparenleft}{\isasymnot}\ P\ {\isasymor}\ {\isasymnot}\ Q{\isacharparenright}} 
      \hfill (\isa{de{\isacharunderscore}Morgan{\isacharunderscore}conj})
    \item (p.51) \isa{{\isacharparenleft}{\isasymnot}\ {\isacharparenleft}P\ {\isasymlongrightarrow}\ Q{\isacharparenright}{\isacharparenright}\ {\isacharequal}\ {\isacharparenleft}P\ {\isasymand}\ {\isasymnot}\ Q{\isacharparenright}} 
      \hfill (\isa{not{\isacharunderscore}imp})
    \item (p.51) \isa{{\isacharparenleft}P\ {\isasymor}\ Q{\isacharparenright}\ {\isacharequal}\ {\isacharparenleft}{\isasymnot}\ P\ {\isasymlongrightarrow}\ Q{\isacharparenright}} 
      \hfill (\isa{disj{\isacharunderscore}imp})
    \item (p.52) \isa{{\isacharparenleft}\textsf{if}\ True\ \textsf{then}\ x\ \textsf{else}\ y{\isacharparenright}\ {\isacharequal}\ x} 
      \hfill (\isa{if{\isacharunderscore}True})
    \item (p.52) \isa{{\isacharparenleft}\textsf{if}\ False\ \textsf{then}\ x\ \textsf{else}\ y{\isacharparenright}\ {\isacharequal}\ y} 
      \hfill (\isa{if{\isacharunderscore}False})
  \end{itemize}%
\end{isamarkuptext}\isamarkuptrue%
%
\isadelimdocument
%
\endisadelimdocument
%
\isatagdocument
%
\isamarkupsection{Teoría del orden (3)%
}
\isamarkuptrue%
%
\endisatagdocument
{\isafolddocument}%
%
\isadelimdocument
%
\endisadelimdocument
%
\begin{isamarkuptext}%
En Isabelle se corresponde con la teoría 
\href{https://acortar.link/LxuXO}{Orderings.thy}.%
\end{isamarkuptext}\isamarkuptrue%
%
\begin{isamarkuptext}%
\begin{itemize}
    \item (p.75) \isa{bot\ {\isasymle}\ a} 
      \hfill (\isa{extremum})
    \item (p.76) \isa{x\ {\isasymle}\ x} 
      \hfill (\isa{order{\isacharunderscore}refl})
    \item (p.77) \isa{\mbox{}\inferrule{\mbox{b\ {\isasymle}\ a\ {\isasymand}\ c\ {\isasymle}\ b}}{\mbox{c\ {\isasymle}\ a}}} 
      \hfill (\isa{trans})
  \end{itemize}%
\end{isamarkuptext}\isamarkuptrue%
%
\isadelimdocument
%
\endisadelimdocument
%
\isatagdocument
%
\isamarkupsection{Teoría de retículos (5)%
}
\isamarkuptrue%
%
\endisatagdocument
{\isafolddocument}%
%
\isadelimdocument
%
\endisadelimdocument
%
\begin{isamarkuptext}%
Los resultados expuestos a continuación pertenecen a la teoría de 
  retículos \href{https://acortar.link/ajRsg}{Lattices.thy}.%
\end{isamarkuptext}\isamarkuptrue%
%
\begin{isamarkuptext}%
\begin{itemize}
    \item (p.140) \isa{a\ {\isasymle}\ max\ a\ b} 
      \hfill (\isa{cobounded{\isadigit{1}}})
    \item (p.140) \isa{b\ {\isasymle}\ max\ a\ b} 
      \hfill (\isa{cobounded{\isadigit{2}}})
    \item (p.140) \isa{{\isacharparenleft}max\ b\ c\ {\isasymle}\ a{\isacharparenright}\ {\isacharequal}\ {\isacharparenleft}b\ {\isasymle}\ a\ {\isasymand}\ c\ {\isasymle}\ a{\isacharparenright}} 
      \hfill (\isa{bounded{\isacharunderscore}iff})
  \end{itemize}%
\end{isamarkuptext}\isamarkuptrue%
%
\isadelimdocument
%
\endisadelimdocument
%
\isatagdocument
%
\isamarkupsection{Teoría de conjuntos (6)%
}
\isamarkuptrue%
%
\endisatagdocument
{\isafolddocument}%
%
\isadelimdocument
%
\endisadelimdocument
%
\begin{isamarkuptext}%
Los siguientes resultados corresponden a la teoría de conjuntos 
  \href{https://acortar.link/adxMr}{Set.thy}.%
\end{isamarkuptext}\isamarkuptrue%
%
\begin{isamarkuptext}%
\begin{itemize}
    \item (p.158) \isa{{\isacharparenleft}a\ {\isasymin}\ Collect\ P{\isacharparenright}\ {\isacharequal}\ P\ a} 
      \hfill (\isa{mem{\isacharunderscore}Collect{\isacharunderscore}eq})
    \item (p.159) \isa{\mbox{}\inferrule{\mbox{P\ a}}{\mbox{a\ {\isasymin}\ {\isacharbraceleft}x\ {\isacharbar}\ P\ x{\isacharbraceright}}}} 
      \hfill (\isa{CollectI})
    \item (p.159) \isa{\mbox{}\inferrule{\mbox{a\ {\isasymin}\ {\isacharbraceleft}x\ {\isacharbar}\ P\ x{\isacharbraceright}}}{\mbox{P\ a}}} 
      \hfill (\isa{CollectD})
    \item (p.165) \isa{\mbox{}\inferrule{\mbox{{\isasymAnd}x{\isachardot}\ \mbox{}\inferrule{\mbox{x\ {\isasymin}\ A}}{\mbox{P\ x}}}}{\mbox{{\isasymforall}x{\isasymin}A{\isachardot}\ P\ x}}} 
      \hfill (\isa{ballI})
    \item (p.165) \isa{\mbox{}\inferrule{\mbox{{\isacharparenleft}{\isasymforall}x{\isasymin}A{\isachardot}\ P\ x{\isacharparenright}\ {\isasymand}\ x\ {\isasymin}\ A}}{\mbox{P\ x}}} 
      \hfill (\isa{bspec})
    \item (p.165) \isa{\mbox{}\inferrule{\mbox{P\ x\ {\isasymand}\ x\ {\isasymin}\ A}}{\mbox{{\isasymexists}x{\isasymin}A{\isachardot}\ P\ x}}} 
      \hfill (\isa{bexI})
    \item (p.166) \isa{\mbox{}\inferrule{\mbox{{\isasymexists}x{\isasymin}A{\isachardot}\ P\ x}\\\ \mbox{{\isasymAnd}x{\isachardot}\ \mbox{}\inferrule{\mbox{x\ {\isasymin}\ A\ {\isasymand}\ P\ x}}{\mbox{Q}}}}{\mbox{Q}}} 
      \hfill (\isa{bexE})
    \item (p.167) \isa{\mbox{}\inferrule{\mbox{{\isasymAnd}x{\isachardot}\ \mbox{}\inferrule{\mbox{x\ {\isasymin}\ A}}{\mbox{x\ {\isasymin}\ B}}}}{\mbox{A\ {\isasymsubseteq}\ B}}} 
      \hfill (\isa{subsetI})
    \item (p.167) \isa{\mbox{}\inferrule{\mbox{c\ {\isasymin}\ A\ {\isasymand}\ A\ {\isasymsubseteq}\ B}}{\mbox{c\ {\isasymin}\ B}}} 
      \hfill (\isa{rev{\isacharunderscore}subsetD})
    \item (p.167) \isa{\mbox{}\inferrule{\mbox{A\ {\isasymsubseteq}\ B}\\\ \mbox{\mbox{}\inferrule{\mbox{c\ {\isasymnotin}\ A}}{\mbox{P}}}\\\ \mbox{\mbox{}\inferrule{\mbox{c\ {\isasymin}\ B}}{\mbox{P}}}}{\mbox{P}}} 
      \hfill (\isa{subsetCE})
    \item (p.167) \isa{\mbox{}\inferrule{\mbox{A\ {\isasymsubseteq}\ B\ {\isasymand}\ c\ {\isasymnotin}\ B}}{\mbox{c\ {\isasymnotin}\ A}}} 
      \hfill (\isa{contra{\isacharunderscore}subsetD})
    \item (p.167) \isa{A\ {\isasymsubseteq}\ A} 
      \hfill (\isa{subset{\isacharunderscore}refl})
    \item (p.168) \isa{\mbox{}\inferrule{\mbox{A\ {\isasymsubseteq}\ B\ {\isasymand}\ B\ {\isasymsubseteq}\ C}}{\mbox{A\ {\isasymsubseteq}\ C}}} 
      \hfill (\isa{subset{\isacharunderscore}trans})
    \item (p.168) \isa{\mbox{}\inferrule{\mbox{A\ {\isacharequal}\ B}}{\mbox{B\ {\isasymsubseteq}\ A}}} 
      \hfill (\isa{equalityD{\isadigit{2}}})
    \item (p.169) \isa{{\isasymemptyset}\ {\isasymsubseteq}\ A} 
      \hfill (\isa{empty{\isacharunderscore}subsetI})
    \item (p.169) \isa{UNIV\ {\isacharequal}\ {\isacharbraceleft}x\ {\isacharbar}\ True{\isacharbraceright}} 
      \hfill (\isa{UNIV{\isacharunderscore}def})
    \item (p.169) \isa{x\ {\isasymin}\ UNIV} 
      \hfill (\isa{UNIV{\isacharunderscore}I})
    \item (p.169) \isa{Bex\ UNIV\ P\ {\isacharequal}\ Ex\ P} 
      \hfill (\isa{bex{\isacharunderscore}UNIV})
    \item (p.171) \isa{{\isacharparenleft}c\ {\isasymin}\ A\ {\isasymunion}\ B{\isacharparenright}\ {\isacharequal}\ {\isacharparenleft}c\ {\isasymin}\ A\ {\isasymor}\ c\ {\isasymin}\ B{\isacharparenright}} 
      \hfill (\isa{Un{\isacharunderscore}iff})
    \item (p.171) \isa{\mbox{}\inferrule{\mbox{c\ {\isasymin}\ A}}{\mbox{c\ {\isasymin}\ A\ {\isasymunion}\ B}}} 
      \hfill (\isa{UnI{\isadigit{1}}})
    \item (p.171) \isa{\mbox{}\inferrule{\mbox{c\ {\isasymin}\ B}}{\mbox{c\ {\isasymin}\ A\ {\isasymunion}\ B}}} 
      \hfill (\isa{UnI{\isadigit{2}}})
    \item (p.172) \isa{{\isacharparenleft}c\ {\isasymin}\ A\ {\isacharminus}\ B{\isacharparenright}\ {\isacharequal}\ {\isacharparenleft}c\ {\isasymin}\ A\ {\isasymand}\ c\ {\isasymnotin}\ B{\isacharparenright}} 
      \hfill (\isa{Diff{\isacharunderscore}iff})
    \item (p.172) \isa{{\isacharparenleft}a\ {\isasymin}\ {\isacharbraceleft}b{\isacharbraceright}\ {\isasymunion}\ A{\isacharparenright}\ {\isacharequal}\ {\isacharparenleft}a\ {\isacharequal}\ b\ {\isasymor}\ a\ {\isasymin}\ A{\isacharparenright}} 
      \hfill (\isa{insert{\isacharunderscore}iff})
    \item (p.172) \isa{a\ {\isasymin}\ {\isacharbraceleft}a{\isacharbraceright}\ {\isasymunion}\ B} 
      \hfill (\isa{insertI{\isadigit{1}}})
    \item (p.173) \isa{a\ {\isasymin}\ {\isacharbraceleft}a{\isacharbraceright}} 
      \hfill (\isa{singletonI})
    \item (p.174) \isa{{\isacharbraceleft}x\ {\isacharbar}\ x\ {\isacharequal}\ a{\isacharbraceright}\ {\isacharequal}\ {\isacharbraceleft}a{\isacharbraceright}} 
      \hfill (\isa{singleton{\isacharunderscore}conv})
    \item (p.175) \isa{\mbox{}\inferrule{\mbox{x\ {\isasymin}\ A}}{\mbox{f\ x\ {\isasymin}\ f\ {\isacharbackquote}\ A}}} 
      \hfill (\isa{imageI})
    \item (p.175) \isa{f\ {\isacharbackquote}\ {\isacharparenleft}A\ {\isasymunion}\ B{\isacharparenright}\ {\isacharequal}\ f\ {\isacharbackquote}\ A\ {\isasymunion}\ f\ {\isacharbackquote}\ B} 
      \hfill (\isa{image{\isacharunderscore}Un})
    \item (p.176) \isa{f\ {\isacharbackquote}\ {\isasymemptyset}\ {\isacharequal}\ {\isasymemptyset}} 
      \hfill (\isa{image{\isacharunderscore}empty})
    \item (p.176) \isa{f\ {\isacharbackquote}\ {\isacharparenleft}{\isacharbraceleft}a{\isacharbraceright}\ {\isasymunion}\ B{\isacharparenright}\ {\isacharequal}\ {\isacharbraceleft}f\ a{\isacharbraceright}\ {\isasymunion}\ f\ {\isacharbackquote}\ B} 
      \hfill (\isa{image{\isacharunderscore}insert})
    \item (p.176) \isa{f\ {\isacharbackquote}\ {\isacharbraceleft}x\ {\isacharbar}\ P\ x{\isacharbraceright}\ {\isacharequal}\ {\isacharbraceleft}f\ x\ {\isacharbar}\ P\ x{\isacharbraceright}} 
      \hfill (\isa{image{\isacharunderscore}Collect})
    \item (p.178) \isa{{\isacharparenleft}A\ {\isasymsubset}\ B{\isacharparenright}\ {\isacharequal}\ {\isacharparenleft}A\ {\isasymsubseteq}\ B\ {\isasymand}\ A\ {\isasymnoteq}\ B{\isacharparenright}} 
      \hfill (\isa{psubset{\isacharunderscore}eq})
    \item (p.179) \isa{\mbox{}\inferrule{\mbox{A\ {\isasymsubset}\ B}}{\mbox{{\isasymexists}b{\isachardot}\ b\ {\isasymin}\ B\ {\isacharminus}\ A}}} 
      \hfill (\isa{psubset{\isacharunderscore}imp{\isacharunderscore}ex{\isacharunderscore}mem})
    \item (p.179) \isa{B\ {\isasymsubseteq}\ {\isacharbraceleft}a{\isacharbraceright}\ {\isasymunion}\ B} 
      \hfill (\isa{subset{\isacharunderscore}insertI})
    \item (p.179) \isa{\mbox{}\inferrule{\mbox{A\ {\isasymsubseteq}\ B}}{\mbox{A\ {\isasymsubseteq}\ {\isacharbraceleft}b{\isacharbraceright}\ {\isasymunion}\ B}}} 
      \hfill (\isa{subset{\isacharunderscore}insertI{\isadigit{2}}})
    \item (p.179) \isa{A\ {\isasymsubseteq}\ A\ {\isasymunion}\ B} 
      \hfill (\isa{Un{\isacharunderscore}upper{\isadigit{1}}})
    \item (p.179) \isa{B\ {\isasymsubseteq}\ A\ {\isasymunion}\ B} 
      \hfill (\isa{Un{\isacharunderscore}upper{\isadigit{2}}})
    \item (p.179) \isa{\mbox{}\inferrule{\mbox{A\ {\isasymsubseteq}\ C\ {\isasymand}\ B\ {\isasymsubseteq}\ C}}{\mbox{A\ {\isasymunion}\ B\ {\isasymsubseteq}\ C}}} 
      \hfill (\isa{Un{\isacharunderscore}least})
    \item (p.180) \isa{{\isacharparenleft}A\ {\isacharminus}\ B\ {\isasymsubseteq}\ C{\isacharparenright}\ {\isacharequal}\ {\isacharparenleft}A\ {\isasymsubseteq}\ B\ {\isasymunion}\ C{\isacharparenright}} 
      \hfill (\isa{Diff{\isacharunderscore}subset{\isacharunderscore}conv})
    \item (p.180) \isa{{\isacharbraceleft}s\ {\isacharbar}\ P{\isacharbraceright}\ {\isacharequal}\ {\isacharparenleft}\textsf{if}\ P\ \textsf{then}\ UNIV\ \textsf{else}\ {\isasymemptyset}{\isacharparenright}} 
      \hfill (\isa{Collect{\isacharunderscore}const})
    \item (p.180) \isa{{\isacharbraceleft}x\ {\isacharbar}\ P\ x\ {\isasymor}\ Q\ x{\isacharbraceright}\ {\isacharequal}\ {\isacharbraceleft}x\ {\isacharbar}\ P\ x{\isacharbraceright}\ {\isasymunion}\ {\isacharbraceleft}x\ {\isacharbar}\ Q\ x{\isacharbraceright}} 
      \hfill (\isa{Collect{\isacharunderscore}disj{\isacharunderscore}eq})
    \item (p.181) \isa{{\isacharbraceleft}a{\isacharbraceright}\ {\isasymunion}\ A\ {\isacharequal}\ {\isacharbraceleft}a{\isacharbraceright}\ {\isasymunion}\ A} 
      \hfill (\isa{insert{\isacharunderscore}is{\isacharunderscore}Un})
    \item (p.181) \isa{\mbox{}\inferrule{\mbox{a\ {\isasymin}\ A}}{\mbox{{\isacharbraceleft}a{\isacharbraceright}\ {\isasymunion}\ A\ {\isacharequal}\ A}}} 
      \hfill (\isa{insert{\isacharunderscore}absorb})
    \item (p.181) \isa{{\isacharbraceleft}x{\isacharcomma}\ x{\isacharbraceright}\ {\isasymunion}\ A\ {\isacharequal}\ {\isacharbraceleft}x{\isacharbraceright}\ {\isasymunion}\ A} 
      \hfill (\isa{insert{\isacharunderscore}absorb{\isadigit{2}}})
    \item (p.181) \isa{{\isacharbraceleft}x{\isacharcomma}\ y{\isacharbraceright}\ {\isasymunion}\ A\ {\isacharequal}\ {\isacharbraceleft}y{\isacharcomma}\ x{\isacharbraceright}\ {\isasymunion}\ A} 
      \hfill (\isa{insert{\isacharunderscore}commute})
    \item (p.181) \isa{{\isacharparenleft}{\isacharbraceleft}x{\isacharbraceright}\ {\isasymunion}\ A\ {\isasymsubseteq}\ B{\isacharparenright}\ {\isacharequal}\ {\isacharparenleft}x\ {\isasymin}\ B\ {\isasymand}\ A\ {\isasymsubseteq}\ B{\isacharparenright}} 
      \hfill (\isa{insert{\isacharunderscore}subset})
    \item (p.183) \isa{A\ {\isasymunion}\ B\ {\isacharequal}\ B\ {\isasymunion}\ A} 
      \hfill (\isa{Un{\isacharunderscore}commute})
    \item (p.183) \isa{A\ {\isasymunion}\ {\isacharparenleft}B\ {\isasymunion}\ C{\isacharparenright}\ {\isacharequal}\ B\ {\isasymunion}\ {\isacharparenleft}A\ {\isasymunion}\ C{\isacharparenright}} 
      \hfill (\isa{Un{\isacharunderscore}left{\isacharunderscore}commute})
    \item (p.183) \isa{A\ {\isasymunion}\ B\ {\isasymunion}\ C\ {\isacharequal}\ A\ {\isasymunion}\ {\isacharparenleft}B\ {\isasymunion}\ C{\isacharparenright}} 
      \hfill (\isa{Un{\isacharunderscore}assoc})
    \item (p.184) \isa{{\isacharparenleft}A\ {\isasymunion}\ B\ {\isasymsubseteq}\ C{\isacharparenright}\ {\isacharequal}\ {\isacharparenleft}A\ {\isasymsubseteq}\ C\ {\isasymand}\ B\ {\isasymsubseteq}\ C{\isacharparenright}} 
      \hfill (\isa{Un{\isacharunderscore}subset{\isacharunderscore}iff})
    \item (p.187) \isa{{\isacharbraceleft}a{\isacharbraceright}\ {\isasymunion}\ {\isacharparenleft}A\ {\isacharminus}\ {\isacharbraceleft}a{\isacharbraceright}{\isacharparenright}\ {\isacharequal}\ {\isacharbraceleft}a{\isacharbraceright}\ {\isasymunion}\ A} 
      \hfill (\isa{insert{\isacharunderscore}Diff{\isacharunderscore}single})
    \item (p.187) \isa{\mbox{}\inferrule{\mbox{a\ {\isasymin}\ A}}{\mbox{{\isacharbraceleft}a{\isacharbraceright}\ {\isasymunion}\ {\isacharparenleft}A\ {\isacharminus}\ {\isacharbraceleft}a{\isacharbraceright}{\isacharparenright}\ {\isacharequal}\ A}}} 
      \hfill (\isa{insert{\isacharunderscore}Diff})
    \item (p.187) \isa{A\ {\isasymunion}\ {\isacharparenleft}B\ {\isacharminus}\ A{\isacharparenright}\ {\isacharequal}\ A\ {\isasymunion}\ B} 
      \hfill (\isa{Un{\isacharunderscore}Diff{\isacharunderscore}cancel})
    \item (p.189) \isa{{\isacharparenleft}{\isasymexists}x{\isasymin}A{\isachardot}\ P\ x\ {\isasymand}\ Q{\isacharparenright}\ {\isacharequal}\ {\isacharparenleft}{\isacharparenleft}{\isasymexists}x{\isasymin}A{\isachardot}\ P\ x{\isacharparenright}\ {\isasymand}\ Q{\isacharparenright}\isasep\isanewline%
{\isacharparenleft}{\isasymexists}x{\isasymin}A{\isachardot}\ P\ {\isasymand}\ Q\ x{\isacharparenright}\ {\isacharequal}\ {\isacharparenleft}P\ {\isasymand}\ {\isacharparenleft}{\isasymexists}x{\isasymin}A{\isachardot}\ Q\ x{\isacharparenright}{\isacharparenright}\isasep\isanewline%
{\isacharparenleft}{\isasymexists}x{\isasymin}{\isasymemptyset}{\isachardot}\ P\ x{\isacharparenright}\ {\isacharequal}\ False\isasep\isanewline%
{\isacharparenleft}{\isasymexists}x{\isasymin}UNIV{\isachardot}\ P\ x{\isacharparenright}\ {\isacharequal}\ {\isacharparenleft}{\isasymexists}x{\isachardot}\ P\ x{\isacharparenright}\isasep\isanewline%
{\isacharparenleft}{\isasymexists}x{\isasymin}{\isacharbraceleft}a{\isacharbraceright}\ {\isasymunion}\ B{\isachardot}\ P\ x{\isacharparenright}\ {\isacharequal}\ {\isacharparenleft}P\ a\ {\isasymor}\ {\isacharparenleft}{\isasymexists}x{\isasymin}B{\isachardot}\ P\ x{\isacharparenright}{\isacharparenright}\isasep\isanewline%
{\isacharparenleft}{\isasymexists}x{\isasymin}Collect\ Q{\isachardot}\ P\ x{\isacharparenright}\ {\isacharequal}\ {\isacharparenleft}{\isasymexists}x{\isachardot}\ Q\ x\ {\isasymand}\ P\ x{\isacharparenright}\isasep\isanewline%
{\isacharparenleft}{\isasymexists}x{\isasymin}f\ {\isacharbackquote}\ A{\isachardot}\ P\ x{\isacharparenright}\ {\isacharequal}\ {\isacharparenleft}{\isasymexists}x{\isasymin}A{\isachardot}\ P\ {\isacharparenleft}f\ x{\isacharparenright}{\isacharparenright}\isasep\isanewline%
{\isacharparenleft}{\isasymnot}\ {\isacharparenleft}{\isasymexists}x{\isasymin}A{\isachardot}\ P\ x{\isacharparenright}{\isacharparenright}\ {\isacharequal}\ {\isacharparenleft}{\isasymforall}x{\isasymin}A{\isachardot}\ {\isasymnot}\ P\ x{\isacharparenright}} 
      \hfill (\isa{bex{\isacharunderscore}simps})
    \item (p.190) \isa{\mbox{}\inferrule{\mbox{C\ {\isasymsubseteq}\ D}}{\mbox{{\isacharbraceleft}a{\isacharbraceright}\ {\isasymunion}\ C\ {\isasymsubseteq}\ {\isacharbraceleft}a{\isacharbraceright}\ {\isasymunion}\ D}}} 
      \hfill (\isa{insert{\isacharunderscore}mono})
    \item (p.190) \isa{\mbox{}\inferrule{\mbox{A\ {\isasymsubseteq}\ B}}{\mbox{x\ {\isasymin}\ A\ {\isasymlongrightarrow}\ x\ {\isasymin}\ B}}} 
      \hfill (\isa{in{\isacharunderscore}mono})
    \item (p.197) \isa{\mbox{}\inferrule{\mbox{A\ {\isasymsubseteq}\ B\ {\isasymand}\ c\ {\isasymin}\ A}}{\mbox{c\ {\isasymin}\ B}}} 
      \hfill (\isa{set{\isacharunderscore}mp})
  \end{itemize}%
\end{isamarkuptext}\isamarkuptrue%
%
\isadelimdocument
%
\endisadelimdocument
%
\isatagdocument
%
\isamarkupsection{Retículos completos (10)%
}
\isamarkuptrue%
%
\endisatagdocument
{\isafolddocument}%
%
\isadelimdocument
%
\endisadelimdocument
%
\begin{isamarkuptext}%
En Isabelle corresponde a la teoría 
  \href{https://acortar.link/iMt7h}{Complete-Lattices.thy}.%
\end{isamarkuptext}\isamarkuptrue%
%
\begin{isamarkuptext}%
\begin{itemize}
    \item (p.171) \isa{{\isacharparenleft}b\ {\isasymin}\ {\isasymUnion}\ {\isacharparenleft}B\ {\isacharbackquote}\ A{\isacharparenright}{\isacharparenright}\ {\isacharequal}\ {\isacharparenleft}{\isasymexists}x{\isasymin}A{\isachardot}\ b\ {\isasymin}\ B\ x{\isacharparenright}} 
      \hfill (\isa{UN{\isacharunderscore}iff})
    \item (p.241) \isa{\mbox{}\inferrule{\mbox{B\ {\isasymin}\ A}}{\mbox{B\ {\isasymsubseteq}\ {\isasymUnion}\ A}}} 
      \hfill (\isa{Union{\isacharunderscore}upper})
    \item (p.241) \isa{{\isasymUnion}\ {\isacharparenleft}A\ {\isasymunion}\ B{\isacharparenright}\ {\isacharequal}\ {\isasymUnion}\ A\ {\isasymunion}\ {\isasymUnion}\ B} 
      \hfill (\isa{Union{\isacharunderscore}Un{\isacharunderscore}distrib})
  \end{itemize}%
\end{isamarkuptext}\isamarkuptrue%
%
\isadelimdocument
%
\endisadelimdocument
%
\isatagdocument
%
\isamarkupsection{Números naturales (15)%
}
\isamarkuptrue%
%
\endisatagdocument
{\isafolddocument}%
%
\isadelimdocument
%
\endisadelimdocument
%
\begin{isamarkuptext}%
La teoría de los números naturales en Isabelle se corresponde a la teoría 
  \href{https://acortar.link/spxlz}{Nat.thy}%
\end{isamarkuptext}\isamarkuptrue%
%
\begin{isamarkuptext}%
\begin{itemize}
    \item (p.360) \isa{{\isacharparenleft}m\ {\isasymle}\ Suc\ n{\isacharparenright}\ {\isacharequal}\ {\isacharparenleft}m\ {\isasymle}\ n\ {\isasymor}\ m\ {\isacharequal}\ Suc\ n{\isacharparenright}} 
      \hfill (\isa{le{\isacharunderscore}Suc{\isacharunderscore}eq})
    \item (p.386) \isa{\mbox{}\inferrule{\mbox{{\isasymAnd}n{\isachardot}\ f\ n\ {\isasymle}\ f\ {\isacharparenleft}Suc\ n{\isacharparenright}}\\\ \mbox{n\ {\isasymle}\ n{\isacharprime}}}{\mbox{f\ n\ {\isasymle}\ f\ n{\isacharprime}}}} 
      \hfill (\isa{lift{\isacharunderscore}Suc{\isacharunderscore}mono{\isacharunderscore}le})
  \end{itemize}%
\end{isamarkuptext}\isamarkuptrue%
%
\isadelimdocument
%
\endisadelimdocument
%
\isatagdocument
%
\isamarkupsection{Conjuntos finitos (17)%
}
\isamarkuptrue%
%
\endisatagdocument
{\isafolddocument}%
%
\isadelimdocument
%
\endisadelimdocument
%
\begin{isamarkuptext}%
A continuación se muestran resultados relativos a la teoría 
  \href{https://acortar.link/F6WFh}{Finite-Set.thy}.%
\end{isamarkuptext}\isamarkuptrue%
%
\begin{isamarkuptext}%
\begin{itemize}
    \item (p.425) \isa{finite\ A} 
      \hfill (\isa{finite})
    \item (p.429) \isa{finite\ {\isacharparenleft}{\isacharbraceleft}a{\isacharbraceright}\ {\isasymunion}\ A{\isacharparenright}\ {\isacharequal}\ finite\ A} 
      \hfill (\isa{finite{\isacharunderscore}insert})
    \item (p.429) \isa{\mbox{}\inferrule{\mbox{finite\ A}}{\mbox{finite\ {\isacharparenleft}A\ {\isacharminus}\ B{\isacharparenright}}}} 
      \hfill (\isa{finite{\isacharunderscore}Diff})
  \end{itemize}%
\end{isamarkuptext}\isamarkuptrue%
%
\isadelimdocument
%
\endisadelimdocument
%
\isatagdocument
%
\isamarkupsection{Retículos condicionalmente completos (89)%
}
\isamarkuptrue%
%
\endisatagdocument
{\isafolddocument}%
%
\isadelimdocument
%
\endisadelimdocument
%
\begin{isamarkuptext}%
El Isabelle se corresponde con la teoría 
  \href{https://acortar.link/1suGL}{Conditionally-Complete-Lattices.thy}.%
\end{isamarkuptext}\isamarkuptrue%
%
\begin{isamarkuptext}%
\begin{itemize}
    \item (p.1632) \isa{Sup\ {\isacharbraceleft}x{\isacharbraceright}\ {\isacharequal}\ x} 
      \hfill (\isa{cSup{\isacharunderscore}singleton})
  \end{itemize}%
\end{isamarkuptext}\isamarkuptrue%
%
\isadelimtheory
%
\endisadelimtheory
%
\isatagtheory
%
\endisatagtheory
{\isafoldtheory}%
%
\isadelimtheory
%
\endisadelimtheory
%
\end{isabellebody}%
\endinput
%:%file=~/TFM/TFM/Glosario.thy%:%
%:%19=13%:%
%:%20=14%:%
%:%21=15%:%
%:%30=17%:%
%:%42=19%:%
%:%43=20%:%
%:%47=23%:%
%:%48=24%:%
%:%49=25%:%
%:%50=26%:%
%:%51=27%:%
%:%52=28%:%
%:%53=29%:%
%:%54=30%:%
%:%55=31%:%
%:%56=32%:%
%:%57=33%:%
%:%58=34%:%
%:%59=35%:%
%:%60=36%:%
%:%61=37%:%
%:%62=38%:%
%:%63=39%:%
%:%64=40%:%
%:%65=41%:%
%:%66=42%:%
%:%67=43%:%
%:%68=44%:%
%:%69=45%:%
%:%70=46%:%
%:%71=47%:%
%:%72=48%:%
%:%73=49%:%
%:%74=50%:%
%:%75=51%:%
%:%76=52%:%
%:%77=53%:%
%:%78=54%:%
%:%79=55%:%
%:%80=56%:%
%:%81=57%:%
%:%82=58%:%
%:%83=59%:%
%:%84=60%:%
%:%85=61%:%
%:%86=62%:%
%:%87=63%:%
%:%88=64%:%
%:%89=65%:%
%:%90=66%:%
%:%133=66%:%
%:%134=67%:%
%:%135=68%:%
%:%136=69%:%
%:%137=70%:%
%:%138=71%:%
%:%139=72%:%
%:%140=73%:%
%:%141=74%:%
%:%142=75%:%
%:%143=76%:%
%:%144=77%:%
%:%145=78%:%
%:%146=79%:%
%:%147=80%:%
%:%148=81%:%
%:%149=82%:%
%:%150=83%:%
%:%151=84%:%
%:%152=85%:%
%:%153=86%:%
%:%154=87%:%
%:%155=88%:%
%:%156=89%:%
%:%157=90%:%
%:%166=92%:%
%:%178=94%:%
%:%179=95%:%
%:%183=98%:%
%:%184=99%:%
%:%185=100%:%
%:%186=101%:%
%:%187=102%:%
%:%188=103%:%
%:%189=104%:%
%:%190=105%:%
%:%199=107%:%
%:%211=109%:%
%:%212=110%:%
%:%216=113%:%
%:%217=114%:%
%:%218=115%:%
%:%219=116%:%
%:%220=117%:%
%:%221=118%:%
%:%222=119%:%
%:%223=120%:%
%:%232=122%:%
%:%244=124%:%
%:%245=125%:%
%:%249=128%:%
%:%250=129%:%
%:%251=130%:%
%:%252=131%:%
%:%253=132%:%
%:%254=133%:%
%:%255=134%:%
%:%256=135%:%
%:%257=136%:%
%:%258=137%:%
%:%259=138%:%
%:%260=139%:%
%:%261=140%:%
%:%262=141%:%
%:%263=142%:%
%:%264=143%:%
%:%265=144%:%
%:%266=145%:%
%:%267=146%:%
%:%268=147%:%
%:%269=148%:%
%:%270=149%:%
%:%271=150%:%
%:%272=151%:%
%:%273=152%:%
%:%274=153%:%
%:%275=154%:%
%:%276=155%:%
%:%277=156%:%
%:%278=157%:%
%:%279=158%:%
%:%280=159%:%
%:%281=160%:%
%:%282=161%:%
%:%283=162%:%
%:%284=163%:%
%:%285=164%:%
%:%286=165%:%
%:%287=166%:%
%:%288=167%:%
%:%289=168%:%
%:%290=169%:%
%:%291=170%:%
%:%292=171%:%
%:%293=172%:%
%:%294=173%:%
%:%295=174%:%
%:%296=175%:%
%:%297=176%:%
%:%298=177%:%
%:%299=178%:%
%:%300=179%:%
%:%301=180%:%
%:%302=181%:%
%:%303=182%:%
%:%304=183%:%
%:%305=184%:%
%:%306=185%:%
%:%307=186%:%
%:%308=187%:%
%:%309=188%:%
%:%310=189%:%
%:%311=190%:%
%:%312=191%:%
%:%313=192%:%
%:%314=193%:%
%:%315=194%:%
%:%316=195%:%
%:%317=196%:%
%:%318=197%:%
%:%319=198%:%
%:%320=199%:%
%:%321=200%:%
%:%322=201%:%
%:%323=202%:%
%:%324=203%:%
%:%325=204%:%
%:%326=205%:%
%:%327=206%:%
%:%328=207%:%
%:%329=208%:%
%:%330=209%:%
%:%331=210%:%
%:%332=211%:%
%:%333=212%:%
%:%334=213%:%
%:%335=214%:%
%:%336=215%:%
%:%337=216%:%
%:%338=217%:%
%:%339=218%:%
%:%340=219%:%
%:%341=220%:%
%:%342=221%:%
%:%343=222%:%
%:%344=223%:%
%:%345=224%:%
%:%346=225%:%
%:%347=226%:%
%:%348=227%:%
%:%349=228%:%
%:%350=229%:%
%:%351=230%:%
%:%352=231%:%
%:%353=232%:%
%:%354=233%:%
%:%355=234%:%
%:%356=235%:%
%:%363=235%:%
%:%364=236%:%
%:%365=237%:%
%:%366=238%:%
%:%367=239%:%
%:%368=240%:%
%:%369=241%:%
%:%370=242%:%
%:%371=243%:%
%:%380=245%:%
%:%392=247%:%
%:%393=248%:%
%:%397=251%:%
%:%398=252%:%
%:%399=253%:%
%:%400=254%:%
%:%401=255%:%
%:%402=256%:%
%:%403=257%:%
%:%404=258%:%
%:%413=260%:%
%:%425=262%:%
%:%426=263%:%
%:%430=266%:%
%:%431=267%:%
%:%432=268%:%
%:%433=269%:%
%:%434=270%:%
%:%435=271%:%
%:%444=273%:%
%:%456=275%:%
%:%457=276%:%
%:%461=279%:%
%:%462=280%:%
%:%463=281%:%
%:%464=282%:%
%:%465=283%:%
%:%466=284%:%
%:%467=285%:%
%:%468=286%:%
%:%477=288%:%
%:%489=290%:%
%:%490=291%:%
%:%494=294%:%
%:%495=295%:%
%:%496=296%:%
%:%497=297%:%