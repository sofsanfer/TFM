\documentclass[12pt,a4paper,twoside]{book}
\usepackage{isabelle,isabellesym,latexsym}
\usepackage{ifthen,mathpartir}
\usepackage{amsmath}
\usepackage{forest}
\usepackage{textcomp}
\usepackage[only,bigsqcap]{stmaryrd}
\usepackage[nottoc,notlot,notlof]{tocbibind}

% further packages required for unusual symbols (see also
% isabellesym.sty), use only when needed

% Personalización
\usepackage{float}
\usepackage{color,graphicx}      % Usa figuras.
\usepackage[utf8x]{inputenc}       % Acentos de UTF8
% \usepackage[T1]{fontenc}         % Codificación T1 con European Computer
% \usepackage[spanish]{babel}      % Castellanización.
% \usepackage{ucs}
\usepackage{mathpazo}            % Tipo de fuente
\usepackage[scaled=.90]{helvet}  % Tipo de fuente
\usepackage{a4wide}              % Márgenes
\linespread{1.05}                % Distancia entre líneas
\setlength{\parindent}{2em}      % Indentación de comienzo de párrafo

\usepackage{tikz}
\tikzset{
  treenode/.style = {draw, align=center,
                     top color=white},
  root/.style     = {treenode, font=\small},
  env/.style      = {treenode, font=\ttfamily\footnotesize},
  dummy/.style    = {circle,draw}
}

\usetikzlibrary{trees}

\usepackage[colorinlistoftodos
           , backgroundcolor = yellow
           , textwidth = 4cm
           , shadow
           , spanish]{todonotes}

\setcounter{secnumdepth}{3}
           
\usepackage{amssymb}
  %for \<leadsto>, \<box>, \<diamond>, \<sqsupset>, \<mho>, \<Join>,
  %\<lhd>, \<lesssim>, \<greatersim>, \<lessapprox>, \<greaterapprox>,
  %\<triangleq>, \<yen>, \<lozenge>

%\usepackage{eurosym}
  %for \<euro>

%\usepackage[only,bigsqcap]{stmaryrd}
  %for \<Sqinter>

%\usepackage{eufrak}
  %for \<AA> ... \<ZZ>, \<aa> ... \<zz> (also included in amssymb)

% \usepackage{textcomp}
  %for \<onequarter>, \<onehalf>, \<threequarters>, \<degree>, \<cent>,
  %\<currency>

% this should be the last package used
\usepackage{pdfsetup}

% urls in roman style, theory text in math-similar italics
\urlstyle{rm}
\isabellestyle{it}

% for uniform font size
\renewcommand{\isastyle}{\isastyleminor}

% Nota: Definiciones
\input definiciones
\input castellano

% No ajusta los espacios verticales.
\raggedbottom

% Espacio entre párrafos
\parindent 2em\parskip 1ex
% Diagramas conmutativos 
%%%%%%%%%%%%%%%%%%%%%%%%%%%%%%%%%%%%%%%%%%%%%%%%%%%%%%%%%%%%%%%%%%%%%%%%%%%%%%
%% Cabeceras                                                              %%
%%%%%%%%%%%%%%%%%%%%%%%%%%%%%%%%%%%%%%%%%%%%%%%%%%%%%%%%%%%%%%%%%%%%%%%%%%%%%%

\usepackage{fancyhdr}

\addtolength{\headheight}{\baselineskip}

\pagestyle{fancy}

\cfoot{}

\fancyhead{}
\fancyhead[RE]{\slshape \nouppercase{\leftmark}}
\fancyhead[LO]{\slshape \nouppercase{\rightmark}}
\fancyhead[LE,RO]{\slshape \thepage}

%%%%%%%%%%%%%%%%%%%%%%%%%%%%%%%%%%%%%%%%%%%%%%%%%%%%%%%%%%%%%%%%%%%%%%%%%%%%%%
%% Documento
%%%%%%%%%%%%%%%%%%%%%%%%%%%%%%%%%%%%%%%%%%%%%%%%%%%%%%%%%%%%%%%%%%%%%%%%%%%%%%

\begin{document}

\begin{titlepage}
 \vspace*{2cm}
  \begin{center}
    {\huge \textbf{Elementos de lógica \\ \vspace*{1em}
                   formalizados en Isabelle/HOL}}
  \end{center}
  \vspace{3cm}
  \begin{center}
    \begin{figure}[h]
    \centering
    \includegraphics[height=6cm]{/Users/sofiasantiagofernandez/TFM/TFM/document/sello.png}
    % \includegraphics{sello.png}
    \end{figure}
  \vspace{3cm}
    {\normalsize Escuela Técnica Superior de Ingeniería Informática} \\
    {\normalsize Departamento de Ciencias de la Computación e Inteligencia Artificial}\\
    {\normalsize Trabajo Fin de Máster} \\
  \end{center}
  \begin{center}
    {\large \textbf{Sofía Santiago Fernández}}
  \end{center}
\end{titlepage}

\newpage


El presente Trabajo Fin de Grado se ha realizado en el Departamento de
Ciencias de la Computación e Inteligencia Artificial de la Universidad
de Sevilla y ha sido supervisado por José Antonio Alonso Jiménez y María
José Hidalgo Doblado.

\newpage

% \setcounter{tocdepth}{1}
\tableofcontents

% sane default for proof documents
% \parindent 0pt\parskip 0.5ex
\parindent 2em\parskip 1ex

% generated text of all theories
% %
\begin{isabellebody}%
\setisabellecontext{Resumen}%
%
\isadelimtheory
%
\endisadelimtheory
%
\isatagtheory
%
\endisatagtheory
{\isafoldtheory}%
%
\isadelimtheory
%
\endisadelimtheory
%
\begin{isamarkuptext}%
El objetivo de la Lógica es la formalización del conocimiento y su
razonamiento. Este trabajo constituye una continuación del Trabajo de Fin de
Grado \isa{Elementos\ de\ lógica\ formalizados\ en\ Isabelle{\isacharslash}HOL} \isa{{\isacharbrackleft}{\isadigit{4}}{\isacharbrackright}}, donde se estudiaron
la sintaxis, semántica y \isa{Lema\ de\ Hintikka} de la Lógica Proposicional
desde la perspectiva teórica de \isa{First−Order\ Logic\ and\ Automated\ Theorem\ Proving} 
\isa{{\isacharbrackleft}{\isadigit{5}}{\isacharbrackright}} de Melvin Fitting. Manteniendo dicha perspectiva, nos centraremos en la 
demostración del \isa{Teorema\ de\ Existencia\ de\ Modelo}, concluyendo con 
el \isa{Teorema\ de\ Compacidad} como consecuencia del mismo. Siguiendo la 
inspiración de \isa{Propositional\ Proof\ Systems} \isa{{\isacharbrackleft}{\isadigit{1}}{\isadigit{0}}{\isacharbrackright}} por Julius Michaelis y Tobias 
Nipkow, los resultados expuestos serán formalizados mediante Isabelle: un demostrador 
interactivo que incluye herramientas de razonamiento automático para guiar al usuario 
en el proceso de formalización, verificación y automatización de resultados. 
Concretamente, Isabelle/HOL es una especialización de Isabelle para la lógica de orden 
superior. Las demostraciones de los resultados en Isabelle/HOL se elaborarán siguiendo 
dos tácticas distintas a lo largo del trabajo. En primer lugar, cada lema será probado 
de manera detallada prescindiendo de toda herramienta de razonamiento automático, como 
resultado de una búsqueda inversa en cada paso de la prueba. En contraposición, 
elaboraremos una demostración automática alternativa de cada resultado que utilice todas 
las herramientas de razonamiento automático que proporciona el demostrador. De este modo, 
se evidenciará la capacidad de razonamiento automático de Isabelle.%
\end{isamarkuptext}\isamarkuptrue%
%
\begin{isamarkuptext}%
Logic’s purpose is about knowledge’s formalisation and its 
reasoning. This project is a continuation of \isa{Elementos\ de\ lógica\ formalizados\ en\ Isabelle{\isacharslash}HOL} \isa{{\isacharbrackleft}{\isadigit{4}}{\isacharbrackright}} in which we studied Syntax, Semantics and
a propositional version of Hintikka's lemma from the theoretical perspective 
of \isa{First−Order\ Logic\ and\ Automated\ Theorem\ Proving} \isa{{\isacharbrackleft}{\isadigit{5}}{\isacharbrackright}} by Melvin Fitting. 
Following the same perspective, we will focus on the demonstration of 
\isa{Propositional\ Model\ Existence} theorem, concluding with the \isa{Propositional\ Compactness} theorem as a consecuence. Inspired by \isa{Propositional\ Proof\ Systems} \isa{{\isacharbrackleft}{\isadigit{1}}{\isadigit{0}}{\isacharbrackright}} by 
Julius Michaelis and Tobias Nipkow, these results will be formalised using 
Isabelle: a proof assistant including automatic reasoning tools to guide the 
user on formalising, verifying and automating results. In particular, 
Isabelle/HOL is the specialization of Isabelle for High-Order Logic. The 
processing of the results formalised in Isabelle/HOL follows two directions. 
In the first place, each lemma will be proved on detail without any automation, 
as the result of an inverse research on every step of the demonstration until 
it is only completed with deductions based on elementary rules and definitions. 
Conversely, we will alternatively prove the results using all the 
automatic reasoning tools that are provide by the proof assistant. In 
this way, Isabelle’s power of automatic reasoning will be shown as the 
contrast between these two different proving tactics.%
\end{isamarkuptext}\isamarkuptrue%
%
\isadelimtheory
%
\endisadelimtheory
%
\isatagtheory
%
\endisatagtheory
{\isafoldtheory}%
%
\isadelimtheory
%
\endisadelimtheory
%
\end{isabellebody}%
\endinput
%:%file=~/TFM/TFM/Resumen.thy%:%
%:%19=8%:%
%:%20=9%:%
%:%21=10%:%
%:%22=11%:%
%:%23=12%:%
%:%24=13%:%
%:%25=14%:%
%:%26=15%:%
%:%27=16%:%
%:%28=17%:%
%:%29=18%:%
%:%30=19%:%
%:%31=20%:%
%:%32=21%:%
%:%33=22%:%
%:%34=23%:%
%:%35=24%:%
%:%36=25%:%
%:%37=26%:%
%:%38=27%:%
%:%42=31%:%
%:%43=32%:%
%:%43=33%:%
%:%44=34%:%
%:%45=35%:%
%:%46=36%:%
%:%47=37%:%
%:%47=38%:%
%:%48=39%:%
%:%49=40%:%
%:%50=41%:%
%:%51=42%:%
%:%52=43%:%
%:%53=44%:%
%:%54=45%:%
%:%55=46%:%
%:%56=47%:%
%:%57=48%:%
%:%58=49%:%
%:%59=50%:%

%
\begin{isabellebody}%
\setisabellecontext{Introduccion}%
%
\isadelimtheory
%
\endisadelimtheory
%
\isatagtheory
%
\endisatagtheory
{\isafoldtheory}%
%
\isadelimtheory
%
\endisadelimtheory
%
\begin{isamarkuptext}%
El objetivo de la Lógica es la formalización del conocimiento y el 
  razonamiento sobre el mismo. Tiene su origen en la Antigua Grecia con 
  Aristóteles y su investigación acerca de los principios del 
  razonamiento válido o correcto, recogidos fundamentalmente en su obra 
  \isa{Organon}. De este modo, dio lugar a la lógica silogística, que 
  consistía en la deducción de conclusiones a partir de dos premisas 
  iniciales. 

	Posteriormente, los estoicos (400-200 a.C) comenzaron a cuestionarse 
  temas relacionados con la semántica, como la naturaleza de la verdad. 
  Formularon la \isa{paradoja\ del\ mentiroso}, que plantea una incongruencia 
  acerca de la veracidad del siguiente predicado.

  \isa{Esta\ oración\ es\ falsa{\isachardot}}

  Sin embargo, no fue hasta el siglo XVII que el matemático y filósofo 
  Gottfried Wilhelm Leibniz (1646 – 1716) instaura un programa lógico 
  que propone la búsqueda de un sistema simbólico del lenguaje natural 
  junto con la matematización del concepto de validez. Estas ideas 
  fueron la principal motivación del desarrollo de la lógica moderna del 
  siglo XIX de la mano de matemáticos y filósofos como Bernard Bolzano 
  (1781 – 1848), George Boole (1815 – 1864), Charles Saunders Pierce 
  (1839 – 1914) y Gottlob Frege (1848 – 1925). Fue este último quien 
  introdujo el primer tratamiento sistemático de la lógica 
  proposicional. Frege basó su tesis en el desarrollo de una sintaxis 
  completa que combina el razonamiento de deducción de la silogística 
  aristotélica con la noción estoica de conectivas para relacionar 
  ideas. Paralelamente desarrolló una semántica asociada a dicha 
  sintaxis que permitiese verificar la validez de los procesos 
  deductivos. La lógica proposicional de Frege formó parte de la 
  escuela denominada logicismo. Su objetivo consistía en investigar 
  los fundamentos de las matemáticas con el fin de formalizarlos 
  lógicamente, para así realizar deducciones y razonamientos 
  válidos.

	En las últimas décadas, el desarrollo de la computación y la 
  inteligencia artificial ha permitido la formalización de las 
  matemáticas y la lógica mediante el lenguaje computacional. 
  Concretamente, el razonamiento automático es un área que investiga los
  distintos aspectos del razonamiento con el fin de crear programas y
  algoritmos para razonar de manera prácticamente automática.
  Se fundamenta en el programa lógico desarrollado por Leibniz,
  estructurado en base a dos principios: la formalización rigurosa
  de resultados y el desarrollo de algoritmos que permitan manipular
  y razonar a partir de dichas formalizaciones. Entre las principales 
  aplicaciones de este áres se encuentra la verificación y síntesis 
  automáticas de programas. De este modo, podemos validar distintos 
  razonamientos, así como crear herramientas de razonamiento automático 
  que permitan el desarrollo de nuevos resultados.

  En este contexto nace Isabelle en 1986, desarrollada por Larry Paulson 
  de la Universidad de Cambridge y Tobias Nipkow del Techniche 
  Universität München. Isabelle es un demostrador interactivo que,
  desde el razonamiento automático, facilita la formalización lógica de 
  resultados y proporciona herramientas para realizar deducciones. En 
  particular, Isabelle/HOL es la especialización de Isabelle para la 
  lógica de orden superior. Junto con Coq, ACL2 y PVS, entre
  otros, constituye uno de los demostradores interactivos más 
  influyentes.

  Como demostrador interactivo, Isabelle permite automatizar 
  razonamientos guiados por el usuario, verificando cada paso de una 
  deducción de manera precisa. Además, incorpora herramientas de 
  razonamiento automático para mejorar la productividad del proceso de 
  demostración. Para ello, cuenta con una extensa librería de resultados 
  lógicos y matemáticos que han sido formalizados y continúan en 
  desarrollo por parte de proyectos como \isa{The\ Alexandria\ Project{\isacharcolon}\ Large{\isacharminus}Scale\ Formal\ Proof\ for\ the\ Working\ Mathematician}. Este proyecto 
  comienza en 2017, dirigido por Lawrence Paulson desde la Universidad 
  de Cambridge. Tiene como finalidad la formalización de distintas 
  teorías para ampliar la librería de Isabelle, junto con la creación de 
  herramientas interactivas que asistan a los matemáticos en el proceso 
  de formalización, demostración y búsqueda de nuevos resultados. 

  Este trabajo constituye una continuación del Trabajo de Fin de Grado
  \isa{Elementos\ de\ lógica\ formalizados\ en\ Isabelle{\isacharslash}HOL} \isa{{\isacharbrackleft}{\isadigit{4}}{\isacharbrackright}} dedicado al estudio
  y formalización de la sintaxis, semántica y lema de Hintikka de la lógica
  proposicional. Inspirado en la primera sección de la publicación 
  \isa{Propositional\ Proof\ Systems} \isa{{\isacharbrackleft}{\isadigit{1}}{\isadigit{0}}{\isacharbrackright}} de Julius Michaelis y Tobias Nipkow, y
  basando su contenido teórico en el libro \isa{First{\isacharminus}Order\ Logic\ and\ Automated\ Theorem\ Proving} \isa{{\isacharbrackleft}{\isadigit{5}}{\isacharbrackright}} de Melvin Fitting, este trabajo tiene como objetivo la
  demostración y formalización del \isa{Teorema\ de\ Existencia\ de\ Modelo}, concluyendo
  con el \isa{Teorema\ de\ Compacidad} como consecuencia del mismo. Para ello, consta
  de cuatro capítulos. En el primero se introduce la notación uniforme para 
  fórmulas proposicionales, en el segundo capítulo se define la propiedad de 
  consistencia proposicional para colecciones de conjuntos de fórmulas, en el 
  tercero se definen las colecciones cerradas bajo subconjuntos y de carácter 
  finito y en el último capítulo se demuestra finalmente \isa{Teorema\ de\ Existencia\ de\ Modelo}, concluyendo con la prueba del \isa{Teorema\ de\ Compacidad}.

  El primer capítulo introduce la notación uniforme para las fórmulas
  proposicionales con el fin de reducir el número de casos a considerar sobre 
  la estructura de las fórmulas al clasificarlas en dos categorías. Para 
  justificar la clasificación, se introduce la definición de fórmulas 
  semánticamente equivalentes como aquellas que tienen el mismo 
  valor para toda interpretación. De este modo, las fórmulas proposicionales
  pueden ser de dos tipos: las de tipo conjuntivo (las fórmulas \isa{{\isasymalpha}}) y las 
  de tipo disyuntivo (las fórmulas \isa{{\isasymbeta}}). Cada fórmula de tipo \isa{{\isasymalpha}}, o \isa{{\isasymbeta}} 
  respectivamente, tiene asociada sus dos componentes \isa{{\isasymalpha}\isactrlsub {\isadigit{1}}} y \isa{{\isasymalpha}\isactrlsub {\isadigit{2}}}, o \isa{{\isasymbeta}\isactrlsub {\isadigit{1}}} y 
  \isa{{\isasymbeta}\isactrlsub {\isadigit{2}}} respectivamente. Intuitivamente, una fórmula es de tipo \isa{{\isasymalpha}} con 
  componentes \isa{{\isasymalpha}\isactrlsub {\isadigit{1}}} y \isa{{\isasymalpha}\isactrlsub {\isadigit{2}}} si es semánticamente equivalente a la fórmula 
  \isa{{\isasymalpha}\isactrlsub {\isadigit{1}}\ {\isasymand}\ {\isasymalpha}\isactrlsub {\isadigit{2}}}, y una fórmula será de tipo \isa{{\isasymbeta}} con componentes \isa{{\isasymbeta}\isactrlsub {\isadigit{1}}} y \isa{{\isasymbeta}\isactrlsub {\isadigit{2}}} si es 
  semánticamente equivalente a la fórmula\\ \isa{{\isasymbeta}\isactrlsub {\isadigit{1}}\ {\isasymor}\ {\isasymbeta}\isactrlsub {\isadigit{2}}}. En Isabelle, se formalizarán
  sintácticamente los conjuntos de fórmulas de tipo \isa{{\isasymalpha}} y de tipo \isa{{\isasymbeta}} como
  predicados inductivos, simplificando la intuición original al prescindir 
  de la noción de equivalencia semántica que permite clasificar la totalidad 
  de las fórmulas proposicionales. Finalmente, el capítulo concluye con un lema
  de caracterización de los conjuntos de Hintikka mediante la notación uniforme.

  En el siguiente capítulo se define la propiedad de consistencia proposicional
  para colecciones de conjuntos de fórmulas. Al final del capítulo se introduce 
  un lema que caracteriza dicha propiedad mediante notación uniforme, lo que 
  facilita las demostraciones de los resultados posteriores.

  En el tercer capítulo se definen las colecciones de conjuntos de fórmulas
  cerradas bajo subconjuntos y de carácter finito. Una colección de conjuntos 
  es cerrada bajo subconjuntos si todo subconjunto de cada conjunto de la 
  colección pertenece a la colección, mientras que una colección es de 
  carácter finito si para todo conjunto son equivalentes que el conjunto
  pertenezca a la colección y que todo subconjunto finito del mismo pertenezca
  a la colección. Posteriormente se muestran tres resultados sobre las mismas. 
  El primero de ellos permite extender una colección con la propiedad de 
  consistencia proposicional a otra que también la verifique y sea cerrada bajo 
  subconjuntos. Seguidamente se probará que toda colección de carácter finito
  es cerrada bajo subconjuntos. En último lugar, se demuestra que una colección
  cerrada bajo subconjuntos que verifique la propiedad de consistencia 
  proposicional se puede extender a otra que también verifique dicha propiedad 
  y sea de carácter finito. 

  Finalmente, en el cuarto capítulo se demuestra el \isa{Teorema\ de\ Existencia\ de\ Modelo} y el \isa{Teorema\ de\ Compacidad} como consecuencia de este. El capítulo 
  está dividido en tres apartados: el primero donde se definen ciertas 
  sucesiones de conjuntos de fórmulas a partir de una colección y un conjunto, 
  el segundo apartado dedicado a la demostración del \isa{Teorema\ de\ Existencia\ de\ Modelo} y el tercer apartado donde se demuestra el \isa{Teorema\ de\ Compacidad}. 

  En la primera parte del cuarto capítulo se definen ciertas sucesiones 
  monótonas \isa{{\isacharbraceleft}S\isactrlsub n{\isacharbraceright}} de conjuntos de fórmulas a partir de una colección \isa{C} y un 
  conjunto \isa{S\ {\isasymin}\ C}. De este modo, se prueban distintas propiedades sobre dichas 
  sucesiones de conjuntos que permiten la prueba del \isa{Teorema\ de\ Existencia\ de\ Modelo}. En primer lugar, se prueba que si \isa{C} verifica la propiedad de 
  consistencia proposicional, entonces todo elemento de la suceción \isa{{\isacharbraceleft}S\isactrlsub n{\isacharbraceright}} 
  pertenece a la colección. Igualmente, se dará un resultado que permite 
  caracterizar conjuntos de \isa{{\isacharbraceleft}S\isactrlsub n{\isacharbraceright}} en función de los anteriores. Por otro lado, 
  definiremos el límite de dichas sucesiones, probando que todo conjunto de 
  \isa{{\isacharbraceleft}S\isactrlsub n{\isacharbraceright}} está contenido en el límite. Además, se demostrará que si una fórmula 
  pertenece al límite, entonces pertenece a algún conjunto de la sucesión \isa{{\isacharbraceleft}S\isactrlsub n{\isacharbraceright}}. 
  Finalmente, se mostrará un resultado sobre conjuntos finitos contenidos en el 
  límite.

  En la segunda parte del capítulo se demuestra el \isa{Teorema\ de\ Existencia\ de\ Modelo}, que prueba que todo conjunto de fórmulas \isa{S} perteneciente a una 
  colección \isa{C} que verifique la propiedad de consistencia proposicional es 
  satisfacible. Para ello, extenderemos la colección \isa{C} a otra \isa{C{\isacharprime}} que 
  tenga la propiedad de consistencia proposicional, sea cerrada bajo 
  subconjuntos y sea de carácter finito, de modo que se introducirán
  distintos resultados sobre colecciones con las características anteriores. 
  En primer lugar, probaremos que el límite de la sucesión \isa{{\isacharbraceleft}S\isactrlsub n{\isacharbraceright}} definida a 
  partir de \isa{C{\isacharprime}} y \isa{S} pertenece a \isa{C{\isacharprime}}. En particular se probará que dicho 
  límite es un elemento maximal de la colección que lo define si esta es 
  cerrada bajo subconjuntos y verifica la propiedad de consistencia 
  proposicional. Por otra parte se demostrará que el límite de \isa{{\isacharbraceleft}S\isactrlsub n{\isacharbraceright}} es un 
  conjunto de Hintikka si está definido a partir de una colección \isa{C{\isacharprime}} con 
  las propiedades descritas luego, por el \isa{Teorema\ de\ Hintikka}, es
  satisfacible. Finalmente, como \isa{S\ {\isasymin}\ C} pertenece también a la extensión \isa{C{\isacharprime}}, 
  se verifica que está contenido en el límite de la sucesión \isa{{\isacharbraceleft}S\isactrlsub n{\isacharbraceright}} definida a 
  partir de \isa{C{\isacharprime}} y\\ \isa{S\ {\isasymin}\ C{\isacharprime}}. Por tanto, quedará demostrada la satisfacibilidad 
  del conjunto \isa{S} al heredarla por contención del límite.

  Por último, el apartado final del capítulo se dedica a la demostración del 
  \isa{Teorema\ de\ Compacidad} que prueba que todo conjunto de fórmulas finitamente 
  satisfacible es satisfacible. Para su demostración consideraremos la 
  colección formada por los conjuntos de fórmulas finitamente satisfacibles. 
  Probaremos que dicha colección verifica la propiedad de consistencia 
  proposicional y, por el \isa{Teorema\ de\ Existencia\ de\ Modelo}, todo conjunto 
  perteneciente a ella será satisfacible. 

  En lo referente a las demostraciones asistidas por Isabelle/HOL de
  los resultados formalizados a lo largo de las secciones, se elaborarán 
  dos tipos de pruebas correspondientes a dos tácticas distintas. En 
  primer lugar, se probará cada resultado siguiendo un esquema de 
  demostración detallado. En él utilizaremos únicamente y de manera 
  precisa las reglas de simplificación y definiciones incluidas en la 
  librería de Isabelle, prescindiendo de las herramientas de 
  razonamiento automático del demostrador. Para ello, se realiza una 
  búsqueda inversa en cada paso de la demostración automática hasta 
  llegar a un desarrollo de la prueba basado en deducciones a partir de
  resultados elementales que la completen de manera rigurosa. En 
  contraposición, se evidenciará la capacidad de razonamiento 
  automático de Isabelle/HOL mediante la realización de una prueba 
  alternativa siguiendo un esquema de demostración automático. Para 
  ello se utilizarán las herramientas de razonamiento que han sido 
  elaboradas en Isabelle/HOL con el objetivo de realizar deducciones de 
  la manera más eficiente.

  \isa{Este\ trabajo\ está\ disponible\ en\ la\ plataforma} GitHub \isa{mediante\ el\ siguiente\ enlace{\isacharcolon}} 

\href{https://github.com/sofsanfer/TFM}
                  {\url{https://github.com/sofsanfer/TFM}}%
\end{isamarkuptext}\isamarkuptrue%
%
\isadelimtheory
%
\endisadelimtheory
%
\isatagtheory
%
\endisatagtheory
{\isafoldtheory}%
%
\isadelimtheory
%
\endisadelimtheory
%
\end{isabellebody}%
\endinput
%:%file=~/TFM/TFM/Introduccion.thy%:%
%:%19=14%:%
%:%20=15%:%
%:%21=16%:%
%:%22=17%:%
%:%23=18%:%
%:%24=19%:%
%:%25=20%:%
%:%26=21%:%
%:%27=22%:%
%:%28=23%:%
%:%29=24%:%
%:%30=25%:%
%:%31=26%:%
%:%32=27%:%
%:%33=28%:%
%:%34=29%:%
%:%35=30%:%
%:%36=31%:%
%:%37=32%:%
%:%38=33%:%
%:%39=34%:%
%:%40=35%:%
%:%41=36%:%
%:%42=37%:%
%:%43=38%:%
%:%44=39%:%
%:%45=40%:%
%:%46=41%:%
%:%47=42%:%
%:%48=43%:%
%:%49=44%:%
%:%50=45%:%
%:%51=46%:%
%:%52=47%:%
%:%53=48%:%
%:%54=49%:%
%:%55=50%:%
%:%56=51%:%
%:%57=52%:%
%:%58=53%:%
%:%59=54%:%
%:%60=55%:%
%:%61=56%:%
%:%62=57%:%
%:%63=58%:%
%:%64=59%:%
%:%65=60%:%
%:%66=61%:%
%:%67=62%:%
%:%68=63%:%
%:%69=64%:%
%:%70=65%:%
%:%71=66%:%
%:%72=67%:%
%:%73=68%:%
%:%74=69%:%
%:%75=70%:%
%:%76=71%:%
%:%77=72%:%
%:%78=73%:%
%:%79=74%:%
%:%80=75%:%
%:%81=76%:%
%:%82=77%:%
%:%83=78%:%
%:%84=79%:%
%:%85=80%:%
%:%85=81%:%
%:%86=82%:%
%:%87=83%:%
%:%88=84%:%
%:%89=85%:%
%:%90=86%:%
%:%91=87%:%
%:%92=88%:%
%:%93=89%:%
%:%94=90%:%
%:%95=91%:%
%:%96=92%:%
%:%97=93%:%
%:%97=94%:%
%:%98=95%:%
%:%99=96%:%
%:%100=97%:%
%:%101=98%:%
%:%102=99%:%
%:%103=100%:%
%:%104=101%:%
%:%104=102%:%
%:%105=103%:%
%:%106=104%:%
%:%107=105%:%
%:%108=106%:%
%:%109=107%:%
%:%110=108%:%
%:%111=109%:%
%:%112=110%:%
%:%113=111%:%
%:%114=112%:%
%:%115=113%:%
%:%116=114%:%
%:%117=115%:%
%:%118=116%:%
%:%119=117%:%
%:%120=118%:%
%:%121=119%:%
%:%122=120%:%
%:%123=121%:%
%:%124=122%:%
%:%125=123%:%
%:%126=124%:%
%:%127=125%:%
%:%128=126%:%
%:%129=127%:%
%:%130=128%:%
%:%131=129%:%
%:%132=130%:%
%:%133=131%:%
%:%134=132%:%
%:%135=133%:%
%:%136=134%:%
%:%137=135%:%
%:%138=136%:%
%:%139=137%:%
%:%140=138%:%
%:%141=139%:%
%:%142=140%:%
%:%143=141%:%
%:%144=142%:%
%:%145=143%:%
%:%145=144%:%
%:%146=145%:%
%:%147=146%:%
%:%148=147%:%
%:%148=148%:%
%:%149=149%:%
%:%150=150%:%
%:%151=151%:%
%:%152=152%:%
%:%153=153%:%
%:%153=154%:%
%:%154=155%:%
%:%155=156%:%
%:%156=157%:%
%:%157=158%:%
%:%158=159%:%
%:%159=160%:%
%:%160=161%:%
%:%161=162%:%
%:%162=163%:%
%:%163=164%:%
%:%163=165%:%
%:%164=166%:%
%:%165=167%:%
%:%166=168%:%
%:%167=169%:%
%:%168=170%:%
%:%169=171%:%
%:%170=172%:%
%:%171=173%:%
%:%172=174%:%
%:%173=175%:%
%:%174=176%:%
%:%175=177%:%
%:%176=178%:%
%:%177=179%:%
%:%178=180%:%
%:%179=181%:%
%:%180=182%:%
%:%181=183%:%
%:%182=184%:%
%:%183=185%:%
%:%184=186%:%
%:%185=187%:%
%:%186=188%:%
%:%187=189%:%
%:%188=190%:%
%:%189=191%:%
%:%190=192%:%
%:%191=193%:%
%:%192=194%:%
%:%193=195%:%
%:%194=196%:%
%:%195=197%:%
%:%196=198%:%
%:%197=199%:%
%:%198=200%:%
%:%199=201%:%
%:%200=202%:%
%:%201=203%:%
%:%202=204%:%
%:%203=205%:%
%:%204=206%:%
%:%205=207%:%
%:%206=208%:%
%:%207=209%:%
%:%207=210%:%
%:%208=211%:%
%:%209=212%:%
%:%210=213%:%

%
\begin{isabellebody}%
\setisabellecontext{Sintaxis}%
%
\isadelimtheory
%
\endisadelimtheory
%
\isatagtheory
%
\endisatagtheory
{\isafoldtheory}%
%
\isadelimtheory
%
\endisadelimtheory
%
\isadelimdocument
%
\endisadelimdocument
%
\isatagdocument
%
\isamarkupsection{Fórmulas%
}
\isamarkuptrue%
%
\endisatagdocument
{\isafolddocument}%
%
\isadelimdocument
%
\endisadelimdocument
%
\begin{isamarkuptext}%
En esta sección presentaremos una formalización en Isabelle de la 
  sintaxis de la lógica proposicional, junto con resultados y pruebas 
  sobre la misma. En líneas generales, primero daremos las nociones de 
  forma clásica y, a continuación, su correspondiente formalización.

  En primer lugar, supondremos que disponemos de los siguientes 
  elementos:
  \begin{description}
    \item[Alfabeto:] Es una lista infinita de variables proposicionales. 
      También pueden ser llamadas átomos o símbolos proposicionales.
    \item[Conectivas:] Conjunto finito cuyos elementos interactúan con 
      las variables. Pueden ser monarias que afectan a un único elemento 
      o binarias que afectan a dos. En el primer grupo se encuentra la 
      negación (\isa{{\isasymnot}}) y en el segundo la conjunción (\isa{{\isasymand}}), la disyunción 
      (\isa{{\isasymor}}) y la implicación (\isa{{\isasymlongrightarrow}}).
  \end{description}

  A continuación definiremos la estructura de fórmula sobre los 
  elementos anteriores. Para ello daremos una definición recursiva 
  basada en dos elementos: un conjunto de fórmulas básicas y una serie 
  de procedimientos de definición de fórmulas a partir de otras. El 
  conjunto de las fórmulas será el menor conjunto de estructuras 
  sintácticas con dicho alfabeto y conectivas que contiene a las básicas 
  y es cerrado mediante los procedimientos de definición que mostraremos 
  a continuación.

  \begin{definicion}
    El conjunto de las fórmulas proposicionales está formado por las 
    siguientes:
    \begin{itemize}
      \item Las fórmulas atómicas, constituidas únicamente por una 
        variable del alfabeto. 
      \item La constante \isa{{\isasymbottom}}.
      \item Dada una fórmula \isa{F}, la negación \isa{{\isasymnot}\ F} es una fórmula.
      \item Dadas dos fórmulas \isa{F} y \isa{G}, la conjunción \isa{F\ {\isasymand}\ G} es una
        fórmula.
      \item Dadas dos fórmulas \isa{F} y \isa{G}, la disyunción \isa{F\ {\isasymor}\ G} es una
        fórmula.
      \item Dadas dos fórmulas \isa{F} y \isa{G}, la implicación \isa{F\ {\isasymlongrightarrow}\ G} es 
        una fórmula.
    \end{itemize}
  \end{definicion}

  Intuitivamente, las fórmulas proposicionales son entendidas como un 
  tipo de árbol sintáctico cuyos nodos son las conectivas y sus hojas
  las fórmulas atómicas. Veamos, por ejemplo, el árbol sintáctico de
  la fórmula \isa{p\ {\isasymrightarrow}\ {\isacharparenleft}{\isasymnot}\ q\ {\isasymor}\ p{\isacharparenright}}.

 \begin{forest} for tree = {parent anchor = south, child anchor = north}
        [\isa{p\ {\isasymrightarrow}\ {\isacharparenleft}{\isasymnot}\ q\ {\isasymor}\ p{\isacharparenright}}
            [\isa{p}]
            [\isa{{\isasymnot}\ q\ {\isasymor}\ p}
                [\isa{{\isasymnot}\ q}
                  [\isa{q}]]
                [\isa{p}]]
        ]
 \end{forest}

  A continuación, veamos la representación en Isabelle de la estructura
  de las fórmulas proposicionales.%
\end{isamarkuptext}\isamarkuptrue%
\isacommand{datatype}\isamarkupfalse%
\ {\isacharparenleft}atoms{\isacharcolon}\ {\isacharprime}a{\isacharparenright}\ formula\ {\isacharequal}\ \isanewline
\ \ Atom\ {\isacharprime}a\isanewline
{\isacharbar}\ Bot\ \ \ \ \ \ \ \ \ \ \ \ \ \ \ \ \ \ \ \ \ \ \ \ \ \ \ \ \ \ {\isacharparenleft}{\isachardoublequoteopen}{\isasymbottom}{\isachardoublequoteclose}{\isacharparenright}\ \ \isanewline
{\isacharbar}\ Not\ {\isachardoublequoteopen}{\isacharprime}a\ formula{\isachardoublequoteclose}\ \ \ \ \ \ \ \ \ \ \ \ \ \ \ \ \ {\isacharparenleft}{\isachardoublequoteopen}\isactrlbold {\isasymnot}{\isachardoublequoteclose}{\isacharparenright}\isanewline
{\isacharbar}\ And\ {\isachardoublequoteopen}{\isacharprime}a\ formula{\isachardoublequoteclose}\ {\isachardoublequoteopen}{\isacharprime}a\ formula{\isachardoublequoteclose}\ \ \ \ {\isacharparenleft}\isakeyword{infix}\ {\isachardoublequoteopen}\isactrlbold {\isasymand}{\isachardoublequoteclose}\ {\isadigit{6}}{\isadigit{8}}{\isacharparenright}\isanewline
{\isacharbar}\ Or\ {\isachardoublequoteopen}{\isacharprime}a\ formula{\isachardoublequoteclose}\ {\isachardoublequoteopen}{\isacharprime}a\ formula{\isachardoublequoteclose}\ \ \ \ \ {\isacharparenleft}\isakeyword{infix}\ {\isachardoublequoteopen}\isactrlbold {\isasymor}{\isachardoublequoteclose}\ {\isadigit{6}}{\isadigit{8}}{\isacharparenright}\isanewline
{\isacharbar}\ Imp\ {\isachardoublequoteopen}{\isacharprime}a\ formula{\isachardoublequoteclose}\ {\isachardoublequoteopen}{\isacharprime}a\ formula{\isachardoublequoteclose}\ \ \ \ {\isacharparenleft}\isakeyword{infixr}\ {\isachardoublequoteopen}\isactrlbold {\isasymrightarrow}{\isachardoublequoteclose}\ {\isadigit{6}}{\isadigit{8}}{\isacharparenright}%
\begin{isamarkuptext}%
Formulas are countable if their atoms are%
\end{isamarkuptext}\isamarkuptrue%
\isacommand{instance}\isamarkupfalse%
\ formula\ {\isacharcolon}{\isacharcolon}\ {\isacharparenleft}countable{\isacharparenright}\ countable%
\isadelimproof
\ %
\endisadelimproof
%
\isatagproof
\isacommand{by}\isamarkupfalse%
\ countable{\isacharunderscore}datatype%
\endisatagproof
{\isafoldproof}%
%
\isadelimproof
%
\endisadelimproof
%
\begin{isamarkuptext}%
Como podemos observar representamos las fórmulas proposicionales
  mediante un tipo de dato recursivo, \isa{formula}, con los 
  siguientes constructores sobre un tipo cualquiera:

  \begin{description}
    \item[Fórmulas básicas]:
      \begin{itemize}
        \item \isa{Atom\ {\isacharcolon}{\isacharcolon}\ {\isacharprime}a\ {\isasymRightarrow}\ {\isacharprime}a\ formula}
        \item \isa{{\isasymbottom}\ {\isacharcolon}{\isacharcolon}\ {\isacharprime}a\ formula}
      \end{itemize}
    \item [Fórmulas compuestas]:
      \begin{itemize}
        \item \isa{\isactrlbold {\isasymnot}\ {\isacharcolon}{\isacharcolon}\ {\isacharprime}a\ formula\ {\isasymRightarrow}\ {\isacharprime}a\ formula}
        \item \isa{{\isacharparenleft}\isactrlbold {\isasymand}{\isacharparenright}\ {\isacharcolon}{\isacharcolon}\ {\isacharprime}a\ formula\ {\isasymRightarrow}\ {\isacharprime}a\ formula\ {\isasymRightarrow}\ {\isacharprime}a\ formula}
        \item \isa{{\isacharparenleft}\isactrlbold {\isasymor}{\isacharparenright}\ {\isacharcolon}{\isacharcolon}\ {\isacharprime}a\ formula\ {\isasymRightarrow}\ {\isacharprime}a\ formula\ {\isasymRightarrow}\ {\isacharprime}a\ formula}
        \item \isa{{\isacharparenleft}\isactrlbold {\isasymrightarrow}{\isacharparenright}\ {\isacharcolon}{\isacharcolon}\ {\isacharprime}a\ formula\ {\isasymRightarrow}\ {\isacharprime}a\ formula\ {\isasymRightarrow}\ {\isacharprime}a\ formula}
      \end{itemize}
  \end{description}

  Cabe señalar que los términos \isa{infix} e \isa{infixr} nos señalan que 
  los constructores que representan a las conectivas se pueden usar de
  forma infija. En particular, \isa{infixr} se trata de un infijo asociado a 
  la derecha.

  Por otro lado, la definición de \isa{formula} 
  genera automáticamente los siguientes lemas sobre la función 
  \isa{atoms\ {\isacharcolon}{\isacharcolon}\ {\isacharprime}a\ formula\ {\isasymRightarrow}\ {\isacharprime}a\ set}, que obtiene el conjunto de átomos de una fórmula.

  \begin{itemize}
    \item[] \isa{atoms\ {\isacharparenleft}Atom\ x{\isadigit{1}}{\isachardot}{\isadigit{0}}{\isacharparenright}\ {\isacharequal}\ {\isacharbraceleft}x{\isadigit{1}}{\isachardot}{\isadigit{0}}{\isacharbraceright}\isasep\isanewline%
atoms\ {\isasymbottom}\ {\isacharequal}\ {\isasymemptyset}\isasep\isanewline%
atoms\ {\isacharparenleft}\isactrlbold {\isasymnot}\ x{\isadigit{3}}{\isachardot}{\isadigit{0}}{\isacharparenright}\ {\isacharequal}\ atoms\ x{\isadigit{3}}{\isachardot}{\isadigit{0}}\isasep\isanewline%
atoms\ {\isacharparenleft}x{\isadigit{4}}{\isadigit{1}}{\isachardot}{\isadigit{0}}\ \isactrlbold {\isasymand}\ x{\isadigit{4}}{\isadigit{2}}{\isachardot}{\isadigit{0}}{\isacharparenright}\ {\isacharequal}\ atoms\ x{\isadigit{4}}{\isadigit{1}}{\isachardot}{\isadigit{0}}\ {\isasymunion}\ atoms\ x{\isadigit{4}}{\isadigit{2}}{\isachardot}{\isadigit{0}}\isasep\isanewline%
atoms\ {\isacharparenleft}x{\isadigit{5}}{\isadigit{1}}{\isachardot}{\isadigit{0}}\ \isactrlbold {\isasymor}\ x{\isadigit{5}}{\isadigit{2}}{\isachardot}{\isadigit{0}}{\isacharparenright}\ {\isacharequal}\ atoms\ x{\isadigit{5}}{\isadigit{1}}{\isachardot}{\isadigit{0}}\ {\isasymunion}\ atoms\ x{\isadigit{5}}{\isadigit{2}}{\isachardot}{\isadigit{0}}\isasep\isanewline%
atoms\ {\isacharparenleft}x{\isadigit{6}}{\isadigit{1}}{\isachardot}{\isadigit{0}}\ \isactrlbold {\isasymrightarrow}\ x{\isadigit{6}}{\isadigit{2}}{\isachardot}{\isadigit{0}}{\isacharparenright}\ {\isacharequal}\ atoms\ x{\isadigit{6}}{\isadigit{1}}{\isachardot}{\isadigit{0}}\ {\isasymunion}\ atoms\ x{\isadigit{6}}{\isadigit{2}}{\isachardot}{\isadigit{0}}}
  \end{itemize} 

  A continuación veremos varios ejemplos de fórmulas y el conjunto de 
  sus variables proposicionales obtenido mediante \isa{atoms}. Se 
  observa que, por ser conjuntos, no contienen elementos repetidos.%
\end{isamarkuptext}\isamarkuptrue%
\isacommand{notepad}\isamarkupfalse%
\ \isanewline
\isakeyword{begin}\isanewline
%
\isadelimproof
\ \ %
\endisadelimproof
%
\isatagproof
\isacommand{fix}\isamarkupfalse%
\ p\ q\ r\ {\isacharcolon}{\isacharcolon}\ {\isacharprime}a\isanewline
\isanewline
\ \ \isacommand{have}\isamarkupfalse%
\ {\isachardoublequoteopen}atoms\ {\isacharparenleft}Atom\ p{\isacharparenright}\ {\isacharequal}\ {\isacharbraceleft}p{\isacharbraceright}{\isachardoublequoteclose}\isanewline
\ \ \ \ \isacommand{by}\isamarkupfalse%
\ {\isacharparenleft}simp\ only{\isacharcolon}\ formula{\isachardot}set{\isacharparenright}\isanewline
\isanewline
\ \ \isacommand{have}\isamarkupfalse%
\ {\isachardoublequoteopen}atoms\ {\isacharparenleft}\isactrlbold {\isasymnot}\ {\isacharparenleft}Atom\ p{\isacharparenright}{\isacharparenright}\ {\isacharequal}\ {\isacharbraceleft}p{\isacharbraceright}{\isachardoublequoteclose}\isanewline
\ \ \ \ \isacommand{by}\isamarkupfalse%
\ {\isacharparenleft}simp\ only{\isacharcolon}\ formula{\isachardot}set{\isacharparenright}\isanewline
\isanewline
\ \ \isacommand{have}\isamarkupfalse%
\ {\isachardoublequoteopen}atoms\ {\isacharparenleft}{\isacharparenleft}Atom\ p\ \isactrlbold {\isasymrightarrow}\ Atom\ q{\isacharparenright}\ \isactrlbold {\isasymor}\ Atom\ r{\isacharparenright}\ {\isacharequal}\ {\isacharbraceleft}p{\isacharcomma}q{\isacharcomma}r{\isacharbraceright}{\isachardoublequoteclose}\isanewline
\ \ \ \ \isacommand{by}\isamarkupfalse%
\ auto\isanewline
\isanewline
\ \ \isacommand{have}\isamarkupfalse%
\ {\isachardoublequoteopen}atoms\ {\isacharparenleft}{\isacharparenleft}Atom\ p\ \isactrlbold {\isasymrightarrow}\ Atom\ p{\isacharparenright}\ \isactrlbold {\isasymor}\ Atom\ r{\isacharparenright}\ {\isacharequal}\ {\isacharbraceleft}p{\isacharcomma}r{\isacharbraceright}{\isachardoublequoteclose}\isanewline
\ \ \ \ \isacommand{by}\isamarkupfalse%
\ auto%
\endisatagproof
{\isafoldproof}%
%
\isadelimproof
\ \ \isanewline
%
\endisadelimproof
\isacommand{end}\isamarkupfalse%
%
\begin{isamarkuptext}%
En particular, el conjunto de símbolos proposicionales de la 
  fórmula \isa{Bot} es vacío. Además, para calcular esta constante es 
  necesario especificar el tipo sobre el que se construye la fórmula.%
\end{isamarkuptext}\isamarkuptrue%
\isacommand{notepad}\isamarkupfalse%
\ \isanewline
\isakeyword{begin}\isanewline
%
\isadelimproof
\ \ %
\endisadelimproof
%
\isatagproof
\isacommand{fix}\isamarkupfalse%
\ p\ {\isacharcolon}{\isacharcolon}\ {\isacharprime}a\isanewline
\isanewline
\ \ \isacommand{have}\isamarkupfalse%
\ {\isachardoublequoteopen}atoms\ {\isasymbottom}\ {\isacharequal}\ {\isasymemptyset}{\isachardoublequoteclose}\isanewline
\ \ \ \ \isacommand{by}\isamarkupfalse%
\ {\isacharparenleft}simp\ only{\isacharcolon}\ formula{\isachardot}set{\isacharparenright}\isanewline
\isanewline
\ \ \isacommand{have}\isamarkupfalse%
\ {\isachardoublequoteopen}atoms\ {\isacharparenleft}Atom\ p\ \isactrlbold {\isasymor}\ {\isasymbottom}{\isacharparenright}\ {\isacharequal}\ {\isacharbraceleft}p{\isacharbraceright}{\isachardoublequoteclose}\isanewline
\ \ \isacommand{proof}\isamarkupfalse%
\ {\isacharminus}\isanewline
\ \ \ \ \isacommand{have}\isamarkupfalse%
\ {\isachardoublequoteopen}atoms\ {\isacharparenleft}Atom\ p\ \isactrlbold {\isasymor}\ {\isasymbottom}{\isacharparenright}\ {\isacharequal}\ atoms\ {\isacharparenleft}Atom\ p{\isacharparenright}\ {\isasymunion}\ atoms\ Bot{\isachardoublequoteclose}\isanewline
\ \ \ \ \ \ \isacommand{by}\isamarkupfalse%
\ {\isacharparenleft}simp\ only{\isacharcolon}\ formula{\isachardot}set{\isacharparenleft}{\isadigit{5}}{\isacharparenright}{\isacharparenright}\isanewline
\ \ \ \ \isacommand{also}\isamarkupfalse%
\ \isacommand{have}\isamarkupfalse%
\ {\isachardoublequoteopen}{\isasymdots}\ {\isacharequal}\ {\isacharbraceleft}p{\isacharbraceright}\ {\isasymunion}\ atoms\ Bot{\isachardoublequoteclose}\isanewline
\ \ \ \ \ \ \isacommand{by}\isamarkupfalse%
\ {\isacharparenleft}simp\ only{\isacharcolon}\ formula{\isachardot}set{\isacharparenleft}{\isadigit{1}}{\isacharparenright}{\isacharparenright}\isanewline
\ \ \ \ \isacommand{also}\isamarkupfalse%
\ \isacommand{have}\isamarkupfalse%
\ {\isachardoublequoteopen}{\isasymdots}\ {\isacharequal}\ {\isacharbraceleft}p{\isacharbraceright}\ {\isasymunion}\ {\isasymemptyset}{\isachardoublequoteclose}\isanewline
\ \ \ \ \ \ \isacommand{by}\isamarkupfalse%
\ {\isacharparenleft}simp\ only{\isacharcolon}\ formula{\isachardot}set{\isacharparenleft}{\isadigit{2}}{\isacharparenright}{\isacharparenright}\isanewline
\ \ \ \ \isacommand{also}\isamarkupfalse%
\ \isacommand{have}\isamarkupfalse%
\ {\isachardoublequoteopen}{\isasymdots}\ {\isacharequal}\ {\isacharbraceleft}p{\isacharbraceright}{\isachardoublequoteclose}\isanewline
\ \ \ \ \ \ \isacommand{by}\isamarkupfalse%
\ {\isacharparenleft}simp\ only{\isacharcolon}\ Un{\isacharunderscore}empty{\isacharunderscore}right{\isacharparenright}\isanewline
\ \ \ \ \isacommand{finally}\isamarkupfalse%
\ \isacommand{show}\isamarkupfalse%
\ {\isachardoublequoteopen}atoms\ {\isacharparenleft}Atom\ p\ \isactrlbold {\isasymor}\ {\isasymbottom}{\isacharparenright}\ {\isacharequal}\ {\isacharbraceleft}p{\isacharbraceright}{\isachardoublequoteclose}\isanewline
\ \ \ \ \ \ \isacommand{by}\isamarkupfalse%
\ this\isanewline
\ \ \isacommand{qed}\isamarkupfalse%
\isanewline
\isanewline
\ \ \isacommand{have}\isamarkupfalse%
\ {\isachardoublequoteopen}atoms\ {\isacharparenleft}Atom\ p\ \isactrlbold {\isasymor}\ {\isasymbottom}{\isacharparenright}\ {\isacharequal}\ {\isacharbraceleft}p{\isacharbraceright}{\isachardoublequoteclose}\isanewline
\ \ \ \ \isacommand{by}\isamarkupfalse%
\ {\isacharparenleft}simp\ only{\isacharcolon}\ formula{\isachardot}set\ Un{\isacharunderscore}empty{\isacharunderscore}right{\isacharparenright}%
\endisatagproof
{\isafoldproof}%
%
\isadelimproof
\isanewline
%
\endisadelimproof
\isacommand{end}\isamarkupfalse%
\isanewline
\isanewline
\isacommand{value}\isamarkupfalse%
\ {\isachardoublequoteopen}{\isacharparenleft}Bot{\isacharcolon}{\isacharcolon}nat\ formula{\isacharparenright}{\isachardoublequoteclose}%
\begin{isamarkuptext}%
Una vez definida la estructura de las fórmulas, vamos a introducir 
  el método de demostración que seguirán los resultados que aquí 
  presentaremos, tanto en la teoría clásica como en Isabelle. 

  Según la definición recursiva de las fórmulas, dispondremos de un 
  esquema de inducción sobre las mismas:

  \begin{teorema}[Principio de inducción sobre fórmulas
  proposicionales]
    Sea \isa{{\isasymP}} una propiedad sobre fórmulas que verifica las siguientes 
    condiciones:
    \begin{itemize}
      \item Las fórmulas atómicas la cumplen.
      \item La constante \isa{{\isasymbottom}} la cumple.
      \item Dada \isa{F} fórmula que la cumple, entonces \isa{{\isasymnot}\ F} la cumple.
      \item Dadas \isa{F} y \isa{G} fórmulas que la cumplen, entonces \isa{F\ {\isacharasterisk}\ G} la 
        cumple, donde \isa{{\isacharasterisk}} simboliza cualquier conectiva binaria.
    \end{itemize}
    Entonces, todas las fórmulas proposicionales tienen la propiedad 
    \isa{{\isasymP}}.
  \end{teorema}

  Análogamente, como las fórmulas proposicionales están definidas 
  mediante un tipo de datos recursivo, Isabelle genera de forma 
  automática el esquema de inducción correspondiente. De este modo, en 
  las pruebas formalizadas utilizaremos la táctica \isa{induction}, 
  que corresponde al siguiente esquema.

  \isa{{\isasymAnd}x{\isachardot}\ P{\isacharparenleft}Atom\ x{\isacharparenright}}

  \isa{P\ {\isasymbottom}}

  \isa{{\isasymAnd}x{\isachardot}\ P\ x\ {\isasymLongrightarrow}\ P{\isacharparenleft}\isactrlbold {\isasymnot}\ x{\isacharparenright}}

  \isa{{\isasymAnd}x{\isadigit{1}}\ x{\isadigit{2}}{\isachardot}\ P\ x{\isadigit{1}}\ {\isasymand}\ P\ x{\isadigit{2}}\ {\isasymLongrightarrow}\ P\ {\isacharparenleft}x{\isadigit{1}}\ \isactrlbold {\isasymand}\ x{\isadigit{2}}{\isacharparenright}}

  \isa{{\isasymAnd}x{\isadigit{1}}\ x{\isadigit{2}}{\isachardot}\ P\ x{\isadigit{1}}\ {\isasymand}\ P\ x{\isadigit{2}}\ {\isasymLongrightarrow}\ P\ {\isacharparenleft}x{\isadigit{1}}\ \isactrlbold {\isasymor}\ x{\isadigit{2}}{\isacharparenright}}

  \isa{{\isasymAnd}x{\isadigit{1}}\ x{\isadigit{2}}{\isachardot}\ P\ x{\isadigit{1}}\ {\isasymand}\ P\ x{\isadigit{2}}\ {\isasymLongrightarrow}\ P\ {\isacharparenleft}x{\isadigit{1}}\ \isactrlbold {\isasymrightarrow}\ x{\isadigit{2}}{\isacharparenright}}

  \rule{70mm}{0.1mm}

  \isa{P\ formula}

  Como hemos señalado, el esquema inductivo genera así seis casos 
  distintos como se muestra anteriormente. Además, todas las 
  demostraciones sobre casos de conectivas binarias son equivalentes en 
  esta sección, pues la construcción sintáctica de fórmulas es idéntica 
  entre ellas. Estas se diferencian esencialmente en la connotación 
  semántica que veremos más adelante.

  A continuación el primer resultado de este apartado:

  \begin{lema}
    El conjunto de los átomos de una fórmula proposicional es finito.
  \end{lema}

  Para proceder a la demostración, consideremos la siguiente
  definición inductiva de conjunto finito. Cabe añadir que la 
  demostración seguirá el esquema inductivo relativo a la estructura de 
  fórmula, y no el que induce esta última definición.

  \begin{definicion}
    Los conjuntos finitos son:
      \begin{itemize}
        \item El vacío.
        \item Dado un conjunto finito \isa{A} y un elemento cualquiera \isa{a}, 
          entonces \isa{{\isacharbraceleft}a{\isacharbraceright}\ {\isasymunion}\ A} es finito.
      \end{itemize}
  \end{definicion}

  La formalización en Isabelle de la definición anterior es precisamente 
  \isa{finite} perteneciente a la teoría 
  \href{https://n9.cl/x86r}{FiniteSet.thy}. Dicha definición inductiva
  genera dos reglas análogas a las anteriores que definen a los 
  conjuntos finitos y que emplearemos en la demostración del resultado.

  \begin{itemize}
    \item[] \isa{fold{\isacharunderscore}graph\ f\ z\ {\isasymemptyset}\ z} 
      \hfill (\isa{emptyI})
  \end{itemize}

  \begin{itemize}
    \item[] \isa{\mbox{}\inferrule{\mbox{x\ {\isasymnotin}\ A\ {\isasymand}\ fold{\isacharunderscore}graph\ f\ z\ A\ y}}{\mbox{fold{\isacharunderscore}graph\ f\ z\ {\isacharparenleft}{\isacharbraceleft}x{\isacharbraceright}\ {\isasymunion}\ A{\isacharparenright}\ {\isacharparenleft}f\ x\ y{\isacharparenright}}}} 
      \hfill (\isa{insertI})
  \end{itemize}

  De este modo, en Isabelle podemos especificar el lema como sigue.%
\end{isamarkuptext}\isamarkuptrue%
\isacommand{lemma}\isamarkupfalse%
\ {\isachardoublequoteopen}finite\ {\isacharparenleft}atoms\ F{\isacharparenright}{\isachardoublequoteclose}\isanewline
%
\isadelimproof
\ \ %
\endisadelimproof
%
\isatagproof
\isacommand{oops}\isamarkupfalse%
%
\endisatagproof
{\isafoldproof}%
%
\isadelimproof
%
\endisadelimproof
%
\begin{isamarkuptext}%
A continuación, veamos la demostración clásica del lema. 

  \begin{demostracion}
  La prueba es por inducción sobre el tipo recursivo de las fórmulas. 
  Veamos cada caso.
  
  Consideremos una fórmula atómica \isa{p} cualquiera. Entonces, 
  su conjunto de variables proposicionales es \isa{{\isacharbraceleft}p{\isacharbraceright}}, finito.

  Sea la fórmula \isa{{\isasymbottom}}. Entonces, su conjunto de átomos es vacío y, por 
  lo tanto, finito.
  
  Sea \isa{F} una fórmula cuyo conjunto de variables proposicionales sea 
  finito. Entonces, por definición, \isa{{\isasymnot}\ F} y \isa{F} tienen igual conjunto de
  átomos y, por hipótesis de inducción, es finito.

  Consideremos las fórmulas \isa{F} y \isa{G} cuyos conjuntos de átomos son 
  finitos. Por\\ construcción, el conjunto de variables de \isa{F{\isacharasterisk}G} es la 
  unión de sus respectivos conjuntos de átomos para cualquier \isa{{\isacharasterisk}} 
  conectiva binaria. Por lo tanto, usando la hipótesis de inducción, 
  dicho conjunto es finito. 
  \end{demostracion} 

  Veamos ahora la prueba detallada en Isabelle. Mostraremos con detalle 
  todos los casos de conectivas binarias, aunque se puede observar que 
  son completamente análogos. Para facilitar la lectura, primero 
  demostraremos por separado cada uno de los casos según el esquema 
  inductivo de fórmulas, y finalmente añadiremos la prueba para una 
  fórmula cualquiera a partir de los anteriores.%
\end{isamarkuptext}\isamarkuptrue%
\isacommand{lemma}\isamarkupfalse%
\ atoms{\isacharunderscore}finite{\isacharunderscore}atom{\isacharcolon}\isanewline
\ \ {\isachardoublequoteopen}finite\ {\isacharparenleft}atoms\ {\isacharparenleft}Atom\ x{\isacharparenright}{\isacharparenright}{\isachardoublequoteclose}\isanewline
%
\isadelimproof
%
\endisadelimproof
%
\isatagproof
\isacommand{proof}\isamarkupfalse%
\ {\isacharminus}\isanewline
\ \ \isacommand{have}\isamarkupfalse%
\ {\isachardoublequoteopen}finite\ {\isasymemptyset}{\isachardoublequoteclose}\isanewline
\ \ \ \ \isacommand{by}\isamarkupfalse%
\ {\isacharparenleft}simp\ only{\isacharcolon}\ finite{\isachardot}emptyI{\isacharparenright}\isanewline
\ \ \isacommand{then}\isamarkupfalse%
\ \isacommand{have}\isamarkupfalse%
\ {\isachardoublequoteopen}finite\ {\isacharbraceleft}x{\isacharbraceright}{\isachardoublequoteclose}\isanewline
\ \ \ \ \isacommand{by}\isamarkupfalse%
\ {\isacharparenleft}simp\ only{\isacharcolon}\ finite{\isacharunderscore}insert{\isacharparenright}\isanewline
\ \ \isacommand{then}\isamarkupfalse%
\ \isacommand{show}\isamarkupfalse%
\ {\isachardoublequoteopen}finite\ {\isacharparenleft}atoms\ {\isacharparenleft}Atom\ x{\isacharparenright}{\isacharparenright}{\isachardoublequoteclose}\isanewline
\ \ \ \ \isacommand{by}\isamarkupfalse%
\ {\isacharparenleft}simp\ only{\isacharcolon}\ formula{\isachardot}set{\isacharparenleft}{\isadigit{1}}{\isacharparenright}{\isacharparenright}\ \isanewline
\isacommand{qed}\isamarkupfalse%
%
\endisatagproof
{\isafoldproof}%
%
\isadelimproof
\isanewline
%
\endisadelimproof
\isanewline
\isacommand{lemma}\isamarkupfalse%
\ atoms{\isacharunderscore}finite{\isacharunderscore}bot{\isacharcolon}\isanewline
\ \ {\isachardoublequoteopen}finite\ {\isacharparenleft}atoms\ {\isasymbottom}{\isacharparenright}{\isachardoublequoteclose}\isanewline
%
\isadelimproof
%
\endisadelimproof
%
\isatagproof
\isacommand{proof}\isamarkupfalse%
\ {\isacharminus}\isanewline
\ \ \isacommand{have}\isamarkupfalse%
\ {\isachardoublequoteopen}finite\ {\isasymemptyset}{\isachardoublequoteclose}\isanewline
\ \ \ \ \isacommand{by}\isamarkupfalse%
\ {\isacharparenleft}simp\ only{\isacharcolon}\ finite{\isachardot}emptyI{\isacharparenright}\isanewline
\ \ \isacommand{then}\isamarkupfalse%
\ \isacommand{show}\isamarkupfalse%
\ {\isachardoublequoteopen}finite\ {\isacharparenleft}atoms\ {\isasymbottom}{\isacharparenright}{\isachardoublequoteclose}\isanewline
\ \ \ \ \isacommand{by}\isamarkupfalse%
\ {\isacharparenleft}simp\ only{\isacharcolon}\ formula{\isachardot}set{\isacharparenleft}{\isadigit{2}}{\isacharparenright}{\isacharparenright}\ \isanewline
\isacommand{qed}\isamarkupfalse%
%
\endisatagproof
{\isafoldproof}%
%
\isadelimproof
\isanewline
%
\endisadelimproof
\isanewline
\isacommand{lemma}\isamarkupfalse%
\ atoms{\isacharunderscore}finite{\isacharunderscore}not{\isacharcolon}\isanewline
\ \ \isakeyword{assumes}\ {\isachardoublequoteopen}finite\ {\isacharparenleft}atoms\ F{\isacharparenright}{\isachardoublequoteclose}\ \isanewline
\ \ \isakeyword{shows}\ \ \ {\isachardoublequoteopen}finite\ {\isacharparenleft}atoms\ {\isacharparenleft}\isactrlbold {\isasymnot}\ F{\isacharparenright}{\isacharparenright}{\isachardoublequoteclose}\isanewline
%
\isadelimproof
\ \ %
\endisadelimproof
%
\isatagproof
\isacommand{using}\isamarkupfalse%
\ assms\isanewline
\ \ \isacommand{by}\isamarkupfalse%
\ {\isacharparenleft}simp\ only{\isacharcolon}\ formula{\isachardot}set{\isacharparenleft}{\isadigit{3}}{\isacharparenright}{\isacharparenright}%
\endisatagproof
{\isafoldproof}%
%
\isadelimproof
\ \isanewline
%
\endisadelimproof
\isanewline
\isacommand{lemma}\isamarkupfalse%
\ atoms{\isacharunderscore}finite{\isacharunderscore}and{\isacharcolon}\isanewline
\ \ \isakeyword{assumes}\ {\isachardoublequoteopen}finite\ {\isacharparenleft}atoms\ F{\isadigit{1}}{\isacharparenright}{\isachardoublequoteclose}\isanewline
\ \ \ \ \ \ \ \ \ \ {\isachardoublequoteopen}finite\ {\isacharparenleft}atoms\ F{\isadigit{2}}{\isacharparenright}{\isachardoublequoteclose}\isanewline
\ \ \isakeyword{shows}\ \ \ {\isachardoublequoteopen}finite\ {\isacharparenleft}atoms\ {\isacharparenleft}F{\isadigit{1}}\ \isactrlbold {\isasymand}\ F{\isadigit{2}}{\isacharparenright}{\isacharparenright}{\isachardoublequoteclose}\isanewline
%
\isadelimproof
%
\endisadelimproof
%
\isatagproof
\isacommand{proof}\isamarkupfalse%
\ {\isacharminus}\isanewline
\ \ \isacommand{have}\isamarkupfalse%
\ {\isachardoublequoteopen}finite\ {\isacharparenleft}atoms\ F{\isadigit{1}}\ {\isasymunion}\ atoms\ F{\isadigit{2}}{\isacharparenright}{\isachardoublequoteclose}\isanewline
\ \ \ \ \isacommand{using}\isamarkupfalse%
\ assms\isanewline
\ \ \ \ \isacommand{by}\isamarkupfalse%
\ {\isacharparenleft}simp\ only{\isacharcolon}\ finite{\isacharunderscore}UnI{\isacharparenright}\isanewline
\ \ \isacommand{then}\isamarkupfalse%
\ \isacommand{show}\isamarkupfalse%
\ {\isachardoublequoteopen}finite\ {\isacharparenleft}atoms\ {\isacharparenleft}F{\isadigit{1}}\ \isactrlbold {\isasymand}\ F{\isadigit{2}}{\isacharparenright}{\isacharparenright}{\isachardoublequoteclose}\ \ \isanewline
\ \ \ \ \isacommand{by}\isamarkupfalse%
\ {\isacharparenleft}simp\ only{\isacharcolon}\ formula{\isachardot}set{\isacharparenleft}{\isadigit{4}}{\isacharparenright}{\isacharparenright}\isanewline
\isacommand{qed}\isamarkupfalse%
%
\endisatagproof
{\isafoldproof}%
%
\isadelimproof
\isanewline
%
\endisadelimproof
\isanewline
\isacommand{lemma}\isamarkupfalse%
\ atoms{\isacharunderscore}finite{\isacharunderscore}or{\isacharcolon}\isanewline
\ \ \isakeyword{assumes}\ {\isachardoublequoteopen}finite\ {\isacharparenleft}atoms\ F{\isadigit{1}}{\isacharparenright}{\isachardoublequoteclose}\isanewline
\ \ \ \ \ \ \ \ \ \ {\isachardoublequoteopen}finite\ {\isacharparenleft}atoms\ F{\isadigit{2}}{\isacharparenright}{\isachardoublequoteclose}\isanewline
\ \ \isakeyword{shows}\ \ \ {\isachardoublequoteopen}finite\ {\isacharparenleft}atoms\ {\isacharparenleft}F{\isadigit{1}}\ \isactrlbold {\isasymor}\ F{\isadigit{2}}{\isacharparenright}{\isacharparenright}{\isachardoublequoteclose}\isanewline
%
\isadelimproof
%
\endisadelimproof
%
\isatagproof
\isacommand{proof}\isamarkupfalse%
\ {\isacharminus}\isanewline
\ \ \isacommand{have}\isamarkupfalse%
\ {\isachardoublequoteopen}finite\ {\isacharparenleft}atoms\ F{\isadigit{1}}\ {\isasymunion}\ atoms\ F{\isadigit{2}}{\isacharparenright}{\isachardoublequoteclose}\isanewline
\ \ \ \ \isacommand{using}\isamarkupfalse%
\ assms\isanewline
\ \ \ \ \isacommand{by}\isamarkupfalse%
\ {\isacharparenleft}simp\ only{\isacharcolon}\ finite{\isacharunderscore}UnI{\isacharparenright}\isanewline
\ \ \isacommand{then}\isamarkupfalse%
\ \isacommand{show}\isamarkupfalse%
\ {\isachardoublequoteopen}finite\ {\isacharparenleft}atoms\ {\isacharparenleft}F{\isadigit{1}}\ \isactrlbold {\isasymor}\ F{\isadigit{2}}{\isacharparenright}{\isacharparenright}{\isachardoublequoteclose}\ \ \isanewline
\ \ \ \ \isacommand{by}\isamarkupfalse%
\ {\isacharparenleft}simp\ only{\isacharcolon}\ formula{\isachardot}set{\isacharparenleft}{\isadigit{5}}{\isacharparenright}{\isacharparenright}\isanewline
\isacommand{qed}\isamarkupfalse%
%
\endisatagproof
{\isafoldproof}%
%
\isadelimproof
\isanewline
%
\endisadelimproof
\isanewline
\isacommand{lemma}\isamarkupfalse%
\ atoms{\isacharunderscore}finite{\isacharunderscore}imp{\isacharcolon}\isanewline
\ \ \isakeyword{assumes}\ {\isachardoublequoteopen}finite\ {\isacharparenleft}atoms\ F{\isadigit{1}}{\isacharparenright}{\isachardoublequoteclose}\isanewline
\ \ \ \ \ \ \ \ \ \ {\isachardoublequoteopen}finite\ {\isacharparenleft}atoms\ F{\isadigit{2}}{\isacharparenright}{\isachardoublequoteclose}\isanewline
\ \ \isakeyword{shows}\ \ \ {\isachardoublequoteopen}finite\ {\isacharparenleft}atoms\ {\isacharparenleft}F{\isadigit{1}}\ \isactrlbold {\isasymrightarrow}\ F{\isadigit{2}}{\isacharparenright}{\isacharparenright}{\isachardoublequoteclose}\isanewline
%
\isadelimproof
%
\endisadelimproof
%
\isatagproof
\isacommand{proof}\isamarkupfalse%
\ {\isacharminus}\isanewline
\ \ \isacommand{have}\isamarkupfalse%
\ {\isachardoublequoteopen}finite\ {\isacharparenleft}atoms\ F{\isadigit{1}}\ {\isasymunion}\ atoms\ F{\isadigit{2}}{\isacharparenright}{\isachardoublequoteclose}\isanewline
\ \ \ \ \isacommand{using}\isamarkupfalse%
\ assms\isanewline
\ \ \ \ \isacommand{by}\isamarkupfalse%
\ {\isacharparenleft}simp\ only{\isacharcolon}\ finite{\isacharunderscore}UnI{\isacharparenright}\isanewline
\ \ \isacommand{then}\isamarkupfalse%
\ \isacommand{show}\isamarkupfalse%
\ {\isachardoublequoteopen}finite\ {\isacharparenleft}atoms\ {\isacharparenleft}F{\isadigit{1}}\ \isactrlbold {\isasymrightarrow}\ F{\isadigit{2}}{\isacharparenright}{\isacharparenright}{\isachardoublequoteclose}\ \ \isanewline
\ \ \ \ \isacommand{by}\isamarkupfalse%
\ {\isacharparenleft}simp\ only{\isacharcolon}\ formula{\isachardot}set{\isacharparenleft}{\isadigit{6}}{\isacharparenright}{\isacharparenright}\isanewline
\isacommand{qed}\isamarkupfalse%
%
\endisatagproof
{\isafoldproof}%
%
\isadelimproof
\isanewline
%
\endisadelimproof
\isanewline
\isacommand{lemma}\isamarkupfalse%
\ atoms{\isacharunderscore}finite{\isacharcolon}\ {\isachardoublequoteopen}finite\ {\isacharparenleft}atoms\ F{\isacharparenright}{\isachardoublequoteclose}\isanewline
%
\isadelimproof
%
\endisadelimproof
%
\isatagproof
\isacommand{proof}\isamarkupfalse%
\ {\isacharparenleft}induction\ F{\isacharparenright}\isanewline
\ \ \isacommand{case}\isamarkupfalse%
\ {\isacharparenleft}Atom\ x{\isacharparenright}\isanewline
\ \ \isacommand{then}\isamarkupfalse%
\ \isacommand{show}\isamarkupfalse%
\ {\isacharquery}case\ \isacommand{by}\isamarkupfalse%
\ {\isacharparenleft}simp\ only{\isacharcolon}\ atoms{\isacharunderscore}finite{\isacharunderscore}atom{\isacharparenright}\isanewline
\isacommand{next}\isamarkupfalse%
\isanewline
\ \ \isacommand{case}\isamarkupfalse%
\ Bot\isanewline
\ \ \isacommand{then}\isamarkupfalse%
\ \isacommand{show}\isamarkupfalse%
\ {\isacharquery}case\ \isacommand{by}\isamarkupfalse%
\ {\isacharparenleft}simp\ only{\isacharcolon}\ atoms{\isacharunderscore}finite{\isacharunderscore}bot{\isacharparenright}\isanewline
\isacommand{next}\isamarkupfalse%
\isanewline
\ \ \isacommand{case}\isamarkupfalse%
\ {\isacharparenleft}Not\ F{\isacharparenright}\isanewline
\ \ \isacommand{then}\isamarkupfalse%
\ \isacommand{show}\isamarkupfalse%
\ {\isacharquery}case\ \isacommand{by}\isamarkupfalse%
\ {\isacharparenleft}simp\ only{\isacharcolon}\ atoms{\isacharunderscore}finite{\isacharunderscore}not{\isacharparenright}\isanewline
\isacommand{next}\isamarkupfalse%
\isanewline
\ \ \isacommand{case}\isamarkupfalse%
\ {\isacharparenleft}And\ F{\isadigit{1}}\ F{\isadigit{2}}{\isacharparenright}\isanewline
\ \ \isacommand{then}\isamarkupfalse%
\ \isacommand{show}\isamarkupfalse%
\ {\isacharquery}case\ \isacommand{by}\isamarkupfalse%
\ {\isacharparenleft}simp\ only{\isacharcolon}\ atoms{\isacharunderscore}finite{\isacharunderscore}and{\isacharparenright}\isanewline
\isacommand{next}\isamarkupfalse%
\isanewline
\ \ \isacommand{case}\isamarkupfalse%
\ {\isacharparenleft}Or\ F{\isadigit{1}}\ F{\isadigit{2}}{\isacharparenright}\isanewline
\ \ \isacommand{then}\isamarkupfalse%
\ \isacommand{show}\isamarkupfalse%
\ {\isacharquery}case\ \isacommand{by}\isamarkupfalse%
\ {\isacharparenleft}simp\ only{\isacharcolon}\ atoms{\isacharunderscore}finite{\isacharunderscore}or{\isacharparenright}\isanewline
\isacommand{next}\isamarkupfalse%
\isanewline
\ \ \isacommand{case}\isamarkupfalse%
\ {\isacharparenleft}Imp\ F{\isadigit{1}}\ F{\isadigit{2}}{\isacharparenright}\isanewline
\ \ \isacommand{then}\isamarkupfalse%
\ \isacommand{show}\isamarkupfalse%
\ {\isacharquery}case\ \isacommand{by}\isamarkupfalse%
\ {\isacharparenleft}simp\ only{\isacharcolon}\ atoms{\isacharunderscore}finite{\isacharunderscore}imp{\isacharparenright}\isanewline
\isacommand{qed}\isamarkupfalse%
%
\endisatagproof
{\isafoldproof}%
%
\isadelimproof
%
\endisadelimproof
%
\begin{isamarkuptext}%
Su demostración automática es la siguiente.%
\end{isamarkuptext}\isamarkuptrue%
\isacommand{lemma}\isamarkupfalse%
\ {\isachardoublequoteopen}finite\ {\isacharparenleft}atoms\ F{\isacharparenright}{\isachardoublequoteclose}\ \isanewline
%
\isadelimproof
\ \ %
\endisadelimproof
%
\isatagproof
\isacommand{by}\isamarkupfalse%
\ {\isacharparenleft}induction\ F{\isacharparenright}\ simp{\isacharunderscore}all%
\endisatagproof
{\isafoldproof}%
%
\isadelimproof
%
\endisadelimproof
%
\isadelimdocument
%
\endisadelimdocument
%
\isatagdocument
%
\isamarkupsection{Subfórmulas%
}
\isamarkuptrue%
%
\endisatagdocument
{\isafolddocument}%
%
\isadelimdocument
%
\endisadelimdocument
%
\begin{isamarkuptext}%
Veamos la noción de subfórmulas.

  \begin{definicion}
  El conjunto de subfórmulas de una fórmula \isa{F}, notada \isa{Subf{\isacharparenleft}F{\isacharparenright}}, se 
  define recursivamente como:
    \begin{itemize}
      \item \isa{{\isacharbraceleft}F{\isacharbraceright}} si \isa{F} es una fórmula atómica.
      \item \isa{{\isacharbraceleft}{\isasymbottom}{\isacharbraceright}} si \isa{F} es \isa{{\isasymbottom}}.
      \item \isa{{\isacharbraceleft}F{\isacharbraceright}\ {\isasymunion}\ Subf{\isacharparenleft}G{\isacharparenright}} si \isa{F} es \isa{{\isasymnot}G}.
      \item \isa{{\isacharbraceleft}F{\isacharbraceright}\ {\isasymunion}\ Subf{\isacharparenleft}G{\isacharparenright}\ {\isasymunion}\ Subf{\isacharparenleft}H{\isacharparenright}} si \isa{F} es \isa{G{\isacharasterisk}H} donde \isa{{\isacharasterisk}} es 
        cualquier conectiva binaria.
    \end{itemize}
  \end{definicion}

  Para proceder a la formalización de Isabelle, seguiremos dos etapas. 
  En primer lugar, definimos la función primitiva recursiva 
  \isa{subformulae}. Esta nos devolverá la lista de todas las 
  subfórmulas de una fórmula original obtenidas recursivamente.%
\end{isamarkuptext}\isamarkuptrue%
\isacommand{primrec}\isamarkupfalse%
\ subformulae\ {\isacharcolon}{\isacharcolon}\ {\isachardoublequoteopen}{\isacharprime}a\ formula\ {\isasymRightarrow}\ {\isacharprime}a\ formula\ list{\isachardoublequoteclose}\ \isakeyword{where}\isanewline
\ \ {\isachardoublequoteopen}subformulae\ {\isacharparenleft}Atom\ k{\isacharparenright}\ {\isacharequal}\ {\isacharbrackleft}Atom\ k{\isacharbrackright}{\isachardoublequoteclose}\ \isanewline
{\isacharbar}\ {\isachardoublequoteopen}subformulae\ {\isasymbottom}\ \ \ \ \ \ \ \ {\isacharequal}\ {\isacharbrackleft}{\isasymbottom}{\isacharbrackright}{\isachardoublequoteclose}\ \isanewline
{\isacharbar}\ {\isachardoublequoteopen}subformulae\ {\isacharparenleft}\isactrlbold {\isasymnot}\ F{\isacharparenright}\ \ \ \ {\isacharequal}\ {\isacharparenleft}\isactrlbold {\isasymnot}\ F{\isacharparenright}\ {\isacharhash}\ subformulae\ F{\isachardoublequoteclose}\ \isanewline
{\isacharbar}\ {\isachardoublequoteopen}subformulae\ {\isacharparenleft}F\ \isactrlbold {\isasymand}\ G{\isacharparenright}\ \ {\isacharequal}\ {\isacharparenleft}F\ \isactrlbold {\isasymand}\ G{\isacharparenright}\ {\isacharhash}\ subformulae\ F\ {\isacharat}\ subformulae\ G{\isachardoublequoteclose}\ \isanewline
{\isacharbar}\ {\isachardoublequoteopen}subformulae\ {\isacharparenleft}F\ \isactrlbold {\isasymor}\ G{\isacharparenright}\ \ {\isacharequal}\ {\isacharparenleft}F\ \isactrlbold {\isasymor}\ G{\isacharparenright}\ {\isacharhash}\ subformulae\ F\ {\isacharat}\ subformulae\ G{\isachardoublequoteclose}\isanewline
{\isacharbar}\ {\isachardoublequoteopen}subformulae\ {\isacharparenleft}F\ \isactrlbold {\isasymrightarrow}\ G{\isacharparenright}\ {\isacharequal}\ {\isacharparenleft}F\ \isactrlbold {\isasymrightarrow}\ G{\isacharparenright}\ {\isacharhash}\ subformulae\ F\ {\isacharat}\ subformulae\ G{\isachardoublequoteclose}%
\begin{isamarkuptext}%
Observemos que, en la definición anterior, \isa{{\isacharhash}} es el operador que 
  añade un elemento al comienzo de una lista y \isa{{\isacharat}} concatena varias 
  listas. 

  Siguiendo con los ejemplos, apliquemos \isa{subformulae} en
  las distintas fórmulas. En particular, al tratarse de una lista pueden 
  aparecer elementos repetidos como se muestra a continuación.%
\end{isamarkuptext}\isamarkuptrue%
\isacommand{notepad}\isamarkupfalse%
\isanewline
\isakeyword{begin}\isanewline
%
\isadelimproof
\ \ %
\endisadelimproof
%
\isatagproof
\isacommand{fix}\isamarkupfalse%
\ p\ {\isacharcolon}{\isacharcolon}\ {\isacharprime}a\isanewline
\isanewline
\ \ \isacommand{have}\isamarkupfalse%
\ {\isachardoublequoteopen}subformulae\ {\isacharparenleft}Atom\ p{\isacharparenright}\ {\isacharequal}\ {\isacharbrackleft}Atom\ p{\isacharbrackright}{\isachardoublequoteclose}\isanewline
\ \ \ \ \isacommand{by}\isamarkupfalse%
\ simp\isanewline
\isanewline
\ \ \isacommand{have}\isamarkupfalse%
\ {\isachardoublequoteopen}subformulae\ {\isacharparenleft}\isactrlbold {\isasymnot}\ {\isacharparenleft}Atom\ p{\isacharparenright}{\isacharparenright}\ {\isacharequal}\ {\isacharbrackleft}\isactrlbold {\isasymnot}\ {\isacharparenleft}Atom\ p{\isacharparenright}{\isacharcomma}\ Atom\ p{\isacharbrackright}{\isachardoublequoteclose}\isanewline
\ \ \ \ \isacommand{by}\isamarkupfalse%
\ simp\isanewline
\isanewline
\ \ \isacommand{have}\isamarkupfalse%
\ {\isachardoublequoteopen}subformulae\ {\isacharparenleft}{\isacharparenleft}Atom\ p\ \isactrlbold {\isasymrightarrow}\ Atom\ q{\isacharparenright}\ \isactrlbold {\isasymor}\ Atom\ r{\isacharparenright}\ {\isacharequal}\ \isanewline
\ \ \ \ \ \ \ {\isacharbrackleft}{\isacharparenleft}Atom\ p\ \isactrlbold {\isasymrightarrow}\ Atom\ q{\isacharparenright}\ \isactrlbold {\isasymor}\ Atom\ r{\isacharcomma}\ Atom\ p\ \isactrlbold {\isasymrightarrow}\ Atom\ q{\isacharcomma}\ Atom\ p{\isacharcomma}\ \isanewline
\ \ \ \ \ \ \ \ Atom\ q{\isacharcomma}\ Atom\ r{\isacharbrackright}{\isachardoublequoteclose}\isanewline
\ \ \ \ \isacommand{by}\isamarkupfalse%
\ simp\isanewline
\isanewline
\ \ \isacommand{have}\isamarkupfalse%
\ {\isachardoublequoteopen}subformulae\ {\isacharparenleft}Atom\ p\ \isactrlbold {\isasymand}\ {\isasymbottom}{\isacharparenright}\ {\isacharequal}\ {\isacharbrackleft}Atom\ p\ \isactrlbold {\isasymand}\ {\isasymbottom}{\isacharcomma}\ Atom\ p{\isacharcomma}\ {\isasymbottom}{\isacharbrackright}{\isachardoublequoteclose}\isanewline
\ \ \ \ \isacommand{by}\isamarkupfalse%
\ simp\isanewline
\isanewline
\ \ \isacommand{have}\isamarkupfalse%
\ {\isachardoublequoteopen}subformulae\ {\isacharparenleft}Atom\ p\ \isactrlbold {\isasymor}\ Atom\ p{\isacharparenright}\ {\isacharequal}\ \isanewline
\ \ \ \ \ \ \ {\isacharbrackleft}Atom\ p\ \isactrlbold {\isasymor}\ Atom\ p{\isacharcomma}\ Atom\ p{\isacharcomma}\ Atom\ p{\isacharbrackright}{\isachardoublequoteclose}\isanewline
\ \ \ \ \isacommand{by}\isamarkupfalse%
\ simp%
\endisatagproof
{\isafoldproof}%
%
\isadelimproof
\isanewline
%
\endisadelimproof
\isacommand{end}\isamarkupfalse%
%
\begin{isamarkuptext}%
En la segunda etapa de formalización, definimos 
  \isa{setSubformulae}, que convierte al tipo conjunto la lista de 
  subfórmulas anterior.%
\end{isamarkuptext}\isamarkuptrue%
\isacommand{abbreviation}\isamarkupfalse%
\ setSubformulae\ {\isacharcolon}{\isacharcolon}\ {\isachardoublequoteopen}{\isacharprime}a\ formula\ {\isasymRightarrow}\ {\isacharprime}a\ formula\ set{\isachardoublequoteclose}\ \isakeyword{where}\isanewline
\ \ {\isachardoublequoteopen}setSubformulae\ F\ {\isasymequiv}\ set\ {\isacharparenleft}subformulae\ F{\isacharparenright}{\isachardoublequoteclose}%
\begin{isamarkuptext}%
De este modo, la función \isa{setSubformulae} es la formalización
  en Isabelle de \isa{Subf{\isacharparenleft}·{\isacharparenright}}. En Isabelle, primero hemos definido la lista 
  de subfórmulas pues, en algunos casos, es más sencilla la prueba de 
  resultados sobre este tipo. 
  Algunas de las ventajas del tipo conjuntos son la eliminación de 
  elementos repetidos o las operaciones propias de teoría de conjuntos. 
  Observemos los siguientes ejemplos con el tipo de conjuntos.%
\end{isamarkuptext}\isamarkuptrue%
\isacommand{notepad}\isamarkupfalse%
\isanewline
\isakeyword{begin}\isanewline
%
\isadelimproof
\ \ %
\endisadelimproof
%
\isatagproof
\isacommand{fix}\isamarkupfalse%
\ p\ q\ r\ {\isacharcolon}{\isacharcolon}\ {\isacharprime}a\isanewline
\isanewline
\ \ \isacommand{have}\isamarkupfalse%
\ {\isachardoublequoteopen}setSubformulae\ {\isacharparenleft}Atom\ p\ \isactrlbold {\isasymor}\ Atom\ p{\isacharparenright}\ {\isacharequal}\ {\isacharbraceleft}Atom\ p\ \isactrlbold {\isasymor}\ Atom\ p{\isacharcomma}\ Atom\ p{\isacharbraceright}{\isachardoublequoteclose}\isanewline
\ \ \ \ \isacommand{by}\isamarkupfalse%
\ simp\isanewline
\ \ \isanewline
\ \ \isacommand{have}\isamarkupfalse%
\ {\isachardoublequoteopen}setSubformulae\ {\isacharparenleft}{\isacharparenleft}Atom\ p\ \isactrlbold {\isasymrightarrow}\ Atom\ q{\isacharparenright}\ \isactrlbold {\isasymor}\ Atom\ r{\isacharparenright}\ {\isacharequal}\isanewline
\ \ \ \ \ \ \ \ {\isacharbraceleft}{\isacharparenleft}Atom\ p\ \isactrlbold {\isasymrightarrow}\ Atom\ q{\isacharparenright}\ \isactrlbold {\isasymor}\ Atom\ r{\isacharcomma}\ Atom\ p\ \isactrlbold {\isasymrightarrow}\ Atom\ q{\isacharcomma}\ Atom\ p{\isacharcomma}\ \isanewline
\ \ \ \ \ \ \ \ \ Atom\ q{\isacharcomma}\ Atom\ r{\isacharbraceright}{\isachardoublequoteclose}\isanewline
\ \ \isacommand{by}\isamarkupfalse%
\ auto%
\endisatagproof
{\isafoldproof}%
%
\isadelimproof
\ \ \ \isanewline
%
\endisadelimproof
\isacommand{end}\isamarkupfalse%
%
\begin{isamarkuptext}%
Por otro lado, debemos señalar que el uso de 
  \isa{abbreviation} para definir \isa{setSubformulae} no es 
  arbitrario. No es una definición propiamente dicha, sino 
  una forma de nombrar la composición de las funciones \isa{set} y 
  \isa{subformulae}.


  En primer lugar, veamos que \isa{setSubformulae} es una
  formalización de \isa{Subf} en Isabelle. Para ello 
  utilizaremos el siguiente resultado sobre listas, probado como sigue.%
\end{isamarkuptext}\isamarkuptrue%
\isacommand{lemma}\isamarkupfalse%
\ set{\isacharunderscore}insert{\isacharcolon}\ {\isachardoublequoteopen}set\ {\isacharparenleft}x\ {\isacharhash}\ ys{\isacharparenright}\ {\isacharequal}\ {\isacharbraceleft}x{\isacharbraceright}\ {\isasymunion}\ set\ ys{\isachardoublequoteclose}\isanewline
%
\isadelimproof
\ \ %
\endisadelimproof
%
\isatagproof
\isacommand{by}\isamarkupfalse%
\ {\isacharparenleft}simp\ only{\isacharcolon}\ list{\isachardot}set{\isacharparenleft}{\isadigit{2}}{\isacharparenright}\ Un{\isacharunderscore}insert{\isacharunderscore}left\ sup{\isacharunderscore}bot{\isachardot}left{\isacharunderscore}neutral{\isacharparenright}%
\endisatagproof
{\isafoldproof}%
%
\isadelimproof
%
\endisadelimproof
%
\begin{isamarkuptext}%
Por tanto, obtenemos la equivalencia como resultado de los 
  siguientes lemas, que aparecen demostrados de manera detallada.%
\end{isamarkuptext}\isamarkuptrue%
\isacommand{lemma}\isamarkupfalse%
\ setSubformulae{\isacharunderscore}atom{\isacharcolon}\isanewline
\ \ {\isachardoublequoteopen}setSubformulae\ {\isacharparenleft}Atom\ p{\isacharparenright}\ {\isacharequal}\ {\isacharbraceleft}Atom\ p{\isacharbraceright}{\isachardoublequoteclose}\isanewline
%
\isadelimproof
\ \ \ \ %
\endisadelimproof
%
\isatagproof
\isacommand{by}\isamarkupfalse%
\ {\isacharparenleft}simp\ only{\isacharcolon}\ subformulae{\isachardot}simps{\isacharparenleft}{\isadigit{1}}{\isacharparenright}\ list{\isachardot}set{\isacharparenright}%
\endisatagproof
{\isafoldproof}%
%
\isadelimproof
\isanewline
%
\endisadelimproof
\isanewline
\isacommand{lemma}\isamarkupfalse%
\ setSubformulae{\isacharunderscore}bot{\isacharcolon}\isanewline
\ \ {\isachardoublequoteopen}setSubformulae\ {\isacharparenleft}{\isasymbottom}{\isacharparenright}\ {\isacharequal}\ {\isacharbraceleft}{\isasymbottom}{\isacharbraceright}{\isachardoublequoteclose}\isanewline
%
\isadelimproof
\ \ \ \ %
\endisadelimproof
%
\isatagproof
\isacommand{by}\isamarkupfalse%
\ {\isacharparenleft}simp\ only{\isacharcolon}\ subformulae{\isachardot}simps{\isacharparenleft}{\isadigit{2}}{\isacharparenright}\ list{\isachardot}set{\isacharparenright}%
\endisatagproof
{\isafoldproof}%
%
\isadelimproof
\isanewline
%
\endisadelimproof
\isanewline
\isacommand{lemma}\isamarkupfalse%
\ setSubformulae{\isacharunderscore}not{\isacharcolon}\isanewline
\ \ \isakeyword{shows}\ {\isachardoublequoteopen}setSubformulae\ {\isacharparenleft}\isactrlbold {\isasymnot}\ F{\isacharparenright}\ {\isacharequal}\ {\isacharbraceleft}\isactrlbold {\isasymnot}\ F{\isacharbraceright}\ {\isasymunion}\ setSubformulae\ F{\isachardoublequoteclose}\isanewline
%
\isadelimproof
%
\endisadelimproof
%
\isatagproof
\isacommand{proof}\isamarkupfalse%
\ {\isacharminus}\isanewline
\ \ \isacommand{have}\isamarkupfalse%
\ {\isachardoublequoteopen}setSubformulae\ {\isacharparenleft}\isactrlbold {\isasymnot}\ F{\isacharparenright}\ {\isacharequal}\ set\ {\isacharparenleft}\isactrlbold {\isasymnot}\ F\ {\isacharhash}\ subformulae\ F{\isacharparenright}{\isachardoublequoteclose}\isanewline
\ \ \ \ \isacommand{by}\isamarkupfalse%
\ {\isacharparenleft}simp\ only{\isacharcolon}\ subformulae{\isachardot}simps{\isacharparenleft}{\isadigit{3}}{\isacharparenright}{\isacharparenright}\isanewline
\ \ \isacommand{also}\isamarkupfalse%
\ \isacommand{have}\isamarkupfalse%
\ {\isachardoublequoteopen}{\isasymdots}\ {\isacharequal}\ {\isacharbraceleft}\isactrlbold {\isasymnot}\ F{\isacharbraceright}\ {\isasymunion}\ set\ {\isacharparenleft}subformulae\ F{\isacharparenright}{\isachardoublequoteclose}\isanewline
\ \ \ \ \isacommand{by}\isamarkupfalse%
\ {\isacharparenleft}simp\ only{\isacharcolon}\ set{\isacharunderscore}insert{\isacharparenright}\isanewline
\ \ \isacommand{finally}\isamarkupfalse%
\ \isacommand{show}\isamarkupfalse%
\ {\isacharquery}thesis\isanewline
\ \ \ \ \isacommand{by}\isamarkupfalse%
\ this\isanewline
\isacommand{qed}\isamarkupfalse%
%
\endisatagproof
{\isafoldproof}%
%
\isadelimproof
\isanewline
%
\endisadelimproof
\isanewline
\isacommand{lemma}\isamarkupfalse%
\ setSubformulae{\isacharunderscore}and{\isacharcolon}\ \isanewline
\ \ {\isachardoublequoteopen}setSubformulae\ {\isacharparenleft}F{\isadigit{1}}\ \isactrlbold {\isasymand}\ F{\isadigit{2}}{\isacharparenright}\ \isanewline
\ \ \ {\isacharequal}\ {\isacharbraceleft}F{\isadigit{1}}\ \isactrlbold {\isasymand}\ F{\isadigit{2}}{\isacharbraceright}\ {\isasymunion}\ {\isacharparenleft}setSubformulae\ F{\isadigit{1}}\ {\isasymunion}\ setSubformulae\ F{\isadigit{2}}{\isacharparenright}{\isachardoublequoteclose}\isanewline
%
\isadelimproof
%
\endisadelimproof
%
\isatagproof
\isacommand{proof}\isamarkupfalse%
\ {\isacharminus}\isanewline
\ \ \isacommand{have}\isamarkupfalse%
\ {\isachardoublequoteopen}setSubformulae\ {\isacharparenleft}F{\isadigit{1}}\ \isactrlbold {\isasymand}\ F{\isadigit{2}}{\isacharparenright}\ \isanewline
\ \ \ \ \ \ \ \ {\isacharequal}\ set\ {\isacharparenleft}{\isacharparenleft}F{\isadigit{1}}\ \isactrlbold {\isasymand}\ F{\isadigit{2}}{\isacharparenright}\ {\isacharhash}\ {\isacharparenleft}subformulae\ F{\isadigit{1}}\ {\isacharat}\ subformulae\ F{\isadigit{2}}{\isacharparenright}{\isacharparenright}{\isachardoublequoteclose}\isanewline
\ \ \ \ \isacommand{by}\isamarkupfalse%
\ {\isacharparenleft}simp\ only{\isacharcolon}\ subformulae{\isachardot}simps{\isacharparenleft}{\isadigit{4}}{\isacharparenright}{\isacharparenright}\isanewline
\ \ \isacommand{also}\isamarkupfalse%
\ \isacommand{have}\isamarkupfalse%
\ {\isachardoublequoteopen}{\isasymdots}\ {\isacharequal}\ {\isacharbraceleft}F{\isadigit{1}}\ \isactrlbold {\isasymand}\ F{\isadigit{2}}{\isacharbraceright}\ {\isasymunion}\ {\isacharparenleft}set\ {\isacharparenleft}subformulae\ F{\isadigit{1}}\ {\isacharat}\ subformulae\ F{\isadigit{2}}{\isacharparenright}{\isacharparenright}{\isachardoublequoteclose}\isanewline
\ \ \ \ \isacommand{by}\isamarkupfalse%
\ {\isacharparenleft}simp\ only{\isacharcolon}\ set{\isacharunderscore}insert{\isacharparenright}\isanewline
\ \ \isacommand{also}\isamarkupfalse%
\ \isacommand{have}\isamarkupfalse%
\ {\isachardoublequoteopen}{\isasymdots}\ {\isacharequal}\ {\isacharbraceleft}F{\isadigit{1}}\ \isactrlbold {\isasymand}\ F{\isadigit{2}}{\isacharbraceright}\ {\isasymunion}\ {\isacharparenleft}setSubformulae\ F{\isadigit{1}}\ {\isasymunion}\ setSubformulae\ F{\isadigit{2}}{\isacharparenright}{\isachardoublequoteclose}\isanewline
\ \ \ \ \isacommand{by}\isamarkupfalse%
\ {\isacharparenleft}simp\ only{\isacharcolon}\ set{\isacharunderscore}append{\isacharparenright}\isanewline
\ \ \isacommand{finally}\isamarkupfalse%
\ \isacommand{show}\isamarkupfalse%
\ {\isacharquery}thesis\isanewline
\ \ \ \ \isacommand{by}\isamarkupfalse%
\ this\isanewline
\isacommand{qed}\isamarkupfalse%
%
\endisatagproof
{\isafoldproof}%
%
\isadelimproof
\isanewline
%
\endisadelimproof
\isanewline
\isacommand{lemma}\isamarkupfalse%
\ setSubformulae{\isacharunderscore}or{\isacharcolon}\ \isanewline
\ \ {\isachardoublequoteopen}setSubformulae\ {\isacharparenleft}F{\isadigit{1}}\ \isactrlbold {\isasymor}\ F{\isadigit{2}}{\isacharparenright}\ \isanewline
\ \ \ {\isacharequal}\ {\isacharbraceleft}F{\isadigit{1}}\ \isactrlbold {\isasymor}\ F{\isadigit{2}}{\isacharbraceright}\ {\isasymunion}\ {\isacharparenleft}setSubformulae\ F{\isadigit{1}}\ {\isasymunion}\ setSubformulae\ F{\isadigit{2}}{\isacharparenright}{\isachardoublequoteclose}\isanewline
%
\isadelimproof
%
\endisadelimproof
%
\isatagproof
\isacommand{proof}\isamarkupfalse%
\ {\isacharminus}\isanewline
\ \ \isacommand{have}\isamarkupfalse%
\ {\isachardoublequoteopen}setSubformulae\ {\isacharparenleft}F{\isadigit{1}}\ \isactrlbold {\isasymor}\ F{\isadigit{2}}{\isacharparenright}\ \isanewline
\ \ \ \ \ \ \ \ {\isacharequal}\ set\ {\isacharparenleft}{\isacharparenleft}F{\isadigit{1}}\ \isactrlbold {\isasymor}\ F{\isadigit{2}}{\isacharparenright}\ {\isacharhash}\ {\isacharparenleft}subformulae\ F{\isadigit{1}}\ {\isacharat}\ subformulae\ F{\isadigit{2}}{\isacharparenright}{\isacharparenright}{\isachardoublequoteclose}\isanewline
\ \ \ \ \isacommand{by}\isamarkupfalse%
\ {\isacharparenleft}simp\ only{\isacharcolon}\ subformulae{\isachardot}simps{\isacharparenleft}{\isadigit{5}}{\isacharparenright}{\isacharparenright}\isanewline
\ \ \isacommand{also}\isamarkupfalse%
\ \isacommand{have}\isamarkupfalse%
\ {\isachardoublequoteopen}{\isasymdots}\ {\isacharequal}\ {\isacharbraceleft}F{\isadigit{1}}\ \isactrlbold {\isasymor}\ F{\isadigit{2}}{\isacharbraceright}\ {\isasymunion}\ {\isacharparenleft}set\ {\isacharparenleft}subformulae\ F{\isadigit{1}}\ {\isacharat}\ subformulae\ F{\isadigit{2}}{\isacharparenright}{\isacharparenright}{\isachardoublequoteclose}\isanewline
\ \ \ \ \isacommand{by}\isamarkupfalse%
\ {\isacharparenleft}simp\ only{\isacharcolon}\ set{\isacharunderscore}insert{\isacharparenright}\isanewline
\ \ \isacommand{also}\isamarkupfalse%
\ \isacommand{have}\isamarkupfalse%
\ {\isachardoublequoteopen}{\isasymdots}\ {\isacharequal}\ {\isacharbraceleft}F{\isadigit{1}}\ \isactrlbold {\isasymor}\ F{\isadigit{2}}{\isacharbraceright}\ {\isasymunion}\ {\isacharparenleft}setSubformulae\ F{\isadigit{1}}\ {\isasymunion}\ setSubformulae\ F{\isadigit{2}}{\isacharparenright}{\isachardoublequoteclose}\isanewline
\ \ \ \ \isacommand{by}\isamarkupfalse%
\ {\isacharparenleft}simp\ only{\isacharcolon}\ set{\isacharunderscore}append{\isacharparenright}\isanewline
\ \ \isacommand{finally}\isamarkupfalse%
\ \isacommand{show}\isamarkupfalse%
\ {\isacharquery}thesis\isanewline
\ \ \ \ \isacommand{by}\isamarkupfalse%
\ this\isanewline
\isacommand{qed}\isamarkupfalse%
%
\endisatagproof
{\isafoldproof}%
%
\isadelimproof
\isanewline
%
\endisadelimproof
\isanewline
\isacommand{lemma}\isamarkupfalse%
\ setSubformulae{\isacharunderscore}imp{\isacharcolon}\ \isanewline
\ \ {\isachardoublequoteopen}setSubformulae\ {\isacharparenleft}F{\isadigit{1}}\ \isactrlbold {\isasymrightarrow}\ F{\isadigit{2}}{\isacharparenright}\ \isanewline
\ \ \ {\isacharequal}\ {\isacharbraceleft}F{\isadigit{1}}\ \isactrlbold {\isasymrightarrow}\ F{\isadigit{2}}{\isacharbraceright}\ {\isasymunion}\ {\isacharparenleft}setSubformulae\ F{\isadigit{1}}\ {\isasymunion}\ setSubformulae\ F{\isadigit{2}}{\isacharparenright}{\isachardoublequoteclose}\isanewline
%
\isadelimproof
%
\endisadelimproof
%
\isatagproof
\isacommand{proof}\isamarkupfalse%
\ {\isacharminus}\isanewline
\ \ \isacommand{have}\isamarkupfalse%
\ {\isachardoublequoteopen}setSubformulae\ {\isacharparenleft}F{\isadigit{1}}\ \isactrlbold {\isasymrightarrow}\ F{\isadigit{2}}{\isacharparenright}\ \isanewline
\ \ \ \ \ \ \ \ {\isacharequal}\ set\ {\isacharparenleft}{\isacharparenleft}F{\isadigit{1}}\ \isactrlbold {\isasymrightarrow}\ F{\isadigit{2}}{\isacharparenright}\ {\isacharhash}\ {\isacharparenleft}subformulae\ F{\isadigit{1}}\ {\isacharat}\ subformulae\ F{\isadigit{2}}{\isacharparenright}{\isacharparenright}{\isachardoublequoteclose}\isanewline
\ \ \ \ \isacommand{by}\isamarkupfalse%
\ {\isacharparenleft}simp\ only{\isacharcolon}\ subformulae{\isachardot}simps{\isacharparenleft}{\isadigit{6}}{\isacharparenright}{\isacharparenright}\isanewline
\ \ \isacommand{also}\isamarkupfalse%
\ \isacommand{have}\isamarkupfalse%
\ {\isachardoublequoteopen}{\isasymdots}\ {\isacharequal}\ {\isacharbraceleft}F{\isadigit{1}}\ \isactrlbold {\isasymrightarrow}\ F{\isadigit{2}}{\isacharbraceright}\ {\isasymunion}\ {\isacharparenleft}set\ {\isacharparenleft}subformulae\ F{\isadigit{1}}\ {\isacharat}\ subformulae\ F{\isadigit{2}}{\isacharparenright}{\isacharparenright}{\isachardoublequoteclose}\isanewline
\ \ \ \ \isacommand{by}\isamarkupfalse%
\ {\isacharparenleft}simp\ only{\isacharcolon}\ set{\isacharunderscore}insert{\isacharparenright}\isanewline
\ \ \isacommand{also}\isamarkupfalse%
\ \isacommand{have}\isamarkupfalse%
\ {\isachardoublequoteopen}{\isasymdots}\ {\isacharequal}\ {\isacharbraceleft}F{\isadigit{1}}\ \isactrlbold {\isasymrightarrow}\ F{\isadigit{2}}{\isacharbraceright}\ {\isasymunion}\ {\isacharparenleft}setSubformulae\ F{\isadigit{1}}\ {\isasymunion}\ setSubformulae\ F{\isadigit{2}}{\isacharparenright}{\isachardoublequoteclose}\isanewline
\ \ \ \ \isacommand{by}\isamarkupfalse%
\ {\isacharparenleft}simp\ only{\isacharcolon}\ set{\isacharunderscore}append{\isacharparenright}\isanewline
\ \ \isacommand{finally}\isamarkupfalse%
\ \isacommand{show}\isamarkupfalse%
\ {\isacharquery}thesis\isanewline
\ \ \ \ \isacommand{by}\isamarkupfalse%
\ this\isanewline
\isacommand{qed}\isamarkupfalse%
%
\endisatagproof
{\isafoldproof}%
%
\isadelimproof
%
\endisadelimproof
%
\begin{isamarkuptext}%
Una vez probada la equivalencia, comencemos con los resultados 
  correspondientes a las subfórmulas. En primer lugar, tenemos la 
  siguiente propiedad como consecuencia directa de la equivalencia de 
  funciones anterior.

  \begin{lema}
    Toda fórmula es subfórmula de ella misma.
  \end{lema}

  \begin{demostracion}
    La demostración se hace en cada caso de la estructura de las 
    fórmulas.
  
    Sea \isa{p} fórmula atómica cualquiera. Por definición, tenemos que su
    conjunto de subfórmulas es \isa{{\isacharbraceleft}p{\isacharbraceright}}, luego se tiene la propiedad.
  
    Sea la fórmula \isa{{\isasymbottom}}. Por definición, su conjunto de subfórmulas es
    \isa{{\isacharbraceleft}{\isasymbottom}{\isacharbraceright}}, luego se verifica el resultado.

    Sea la fórmula \isa{{\isasymnot}\ F}. Veamos que pertenece a su conjunto de
    subfórmulas.
    Por definición, tenemos que el conjunto de subfórmulas de \isa{{\isasymnot}\ F} es
    \isa{{\isacharbraceleft}{\isasymnot}\ F{\isacharbraceright}\ {\isasymunion}\ Subf{\isacharparenleft}F{\isacharparenright}}. Por tanto, \isa{{\isasymnot}\ F} pertence a su propio conjunto
    de subfórmulas como queríamos demostrar.

    Sea \isa{{\isacharasterisk}} una conectiva binaria cualquiera y las fórmulas \isa{F} y \isa{G}
    Veamos que \isa{F{\isacharasterisk}G} pertenece a su conjunto de subfórmulas.
    Por definición, tenemos que el conjunto de subfórmulas de \isa{F{\isacharasterisk}G} es
    \isa{{\isacharbraceleft}F{\isacharasterisk}G{\isacharbraceright}\ {\isasymunion}\ Subf{\isacharparenleft}F{\isacharparenright}\ {\isasymunion}\ Subf{\isacharparenleft}G{\isacharparenright}}. Por tanto, \isa{F{\isacharasterisk}G} pertence a su propio 
    conjunto de subfórmulas como queríamos demostrar.
  \end{demostracion}

  Formalicemos ahora el lema con su correspondiente demostración 
  detallada.%
\end{isamarkuptext}\isamarkuptrue%
\isacommand{lemma}\isamarkupfalse%
\ subformulae{\isacharunderscore}self{\isacharcolon}\ {\isachardoublequoteopen}F\ {\isasymin}\ setSubformulae\ F{\isachardoublequoteclose}\isanewline
%
\isadelimproof
%
\endisadelimproof
%
\isatagproof
\isacommand{proof}\isamarkupfalse%
\ {\isacharparenleft}cases\ F{\isacharparenright}\isanewline
\ \ \isacommand{case}\isamarkupfalse%
\ {\isacharparenleft}Atom\ x{\isadigit{1}}{\isacharparenright}\isanewline
\ \ \isacommand{then}\isamarkupfalse%
\ \isacommand{show}\isamarkupfalse%
\ {\isacharquery}thesis\ \isanewline
\ \ \ \ \isacommand{by}\isamarkupfalse%
\ {\isacharparenleft}simp\ only{\isacharcolon}\ singletonI\ setSubformulae{\isacharunderscore}atom{\isacharparenright}\isanewline
\isacommand{next}\isamarkupfalse%
\isanewline
\ \ \isacommand{case}\isamarkupfalse%
\ Bot\isanewline
\ \ \isacommand{then}\isamarkupfalse%
\ \isacommand{show}\isamarkupfalse%
\ {\isacharquery}thesis\isanewline
\ \ \ \ \isacommand{by}\isamarkupfalse%
\ {\isacharparenleft}simp\ only{\isacharcolon}\ singletonI\ setSubformulae{\isacharunderscore}bot{\isacharparenright}\isanewline
\isacommand{next}\isamarkupfalse%
\isanewline
\ \ \isacommand{case}\isamarkupfalse%
\ {\isacharparenleft}Not\ F{\isacharparenright}\isanewline
\ \ \isacommand{then}\isamarkupfalse%
\ \isacommand{show}\isamarkupfalse%
\ {\isacharquery}thesis\isanewline
\ \ \ \ \isacommand{by}\isamarkupfalse%
\ {\isacharparenleft}simp\ only{\isacharcolon}\ singletonI\ UnI{\isadigit{1}}\ setSubformulae{\isacharunderscore}not{\isacharparenright}\isanewline
\isacommand{next}\isamarkupfalse%
\isanewline
\ \ \isacommand{case}\isamarkupfalse%
\ {\isacharparenleft}And\ F{\isadigit{1}}\ F{\isadigit{2}}{\isacharparenright}\isanewline
\ \ \isacommand{then}\isamarkupfalse%
\ \isacommand{show}\isamarkupfalse%
\ {\isacharquery}thesis\isanewline
\ \ \ \isacommand{by}\isamarkupfalse%
\ {\isacharparenleft}simp\ only{\isacharcolon}\ singletonI\ UnI{\isadigit{1}}\ setSubformulae{\isacharunderscore}and{\isacharparenright}\isanewline
\isacommand{next}\isamarkupfalse%
\isanewline
\ \ \isacommand{case}\isamarkupfalse%
\ {\isacharparenleft}Or\ F{\isadigit{1}}\ F{\isadigit{2}}{\isacharparenright}\isanewline
\ \ \isacommand{then}\isamarkupfalse%
\ \isacommand{show}\isamarkupfalse%
\ {\isacharquery}thesis\isanewline
\ \ \ \isacommand{by}\isamarkupfalse%
\ {\isacharparenleft}simp\ only{\isacharcolon}\ singletonI\ UnI{\isadigit{1}}\ setSubformulae{\isacharunderscore}or{\isacharparenright}\isanewline
\isacommand{next}\isamarkupfalse%
\isanewline
\ \ \isacommand{case}\isamarkupfalse%
\ {\isacharparenleft}Imp\ F{\isadigit{1}}\ F{\isadigit{2}}{\isacharparenright}\isanewline
\ \ \isacommand{then}\isamarkupfalse%
\ \isacommand{show}\isamarkupfalse%
\ {\isacharquery}thesis\isanewline
\ \ \ \isacommand{by}\isamarkupfalse%
\ {\isacharparenleft}simp\ only{\isacharcolon}\ singletonI\ UnI{\isadigit{1}}\ setSubformulae{\isacharunderscore}imp{\isacharparenright}\isanewline
\isacommand{qed}\isamarkupfalse%
%
\endisatagproof
{\isafoldproof}%
%
\isadelimproof
%
\endisadelimproof
%
\begin{isamarkuptext}%
La demostración automática es la siguiente.%
\end{isamarkuptext}\isamarkuptrue%
\isacommand{lemma}\isamarkupfalse%
\ {\isachardoublequoteopen}F\ {\isasymin}\ setSubformulae\ F{\isachardoublequoteclose}\isanewline
%
\isadelimproof
\ \ %
\endisadelimproof
%
\isatagproof
\isacommand{by}\isamarkupfalse%
\ {\isacharparenleft}cases\ F{\isacharparenright}\ simp{\isacharunderscore}all%
\endisatagproof
{\isafoldproof}%
%
\isadelimproof
%
\endisadelimproof
%
\begin{isamarkuptext}%
Procedamos con los demás resultados de la sección. Como hemos 
  señalado con anterioridad, utilizaremos varias propiedades de 
  conjuntos pertenecientes a la teoría 
  \href{https://n9.cl/qatp}{Set.thy} de Isabelle, que apareceran en 
  el glosario final. 

  Además, definiremos dos reglas adicionales que utilizaremos con 
  frecuencia.%
\end{isamarkuptext}\isamarkuptrue%
\isacommand{lemma}\isamarkupfalse%
\ subContUnionRev{\isadigit{1}}{\isacharcolon}\ \isanewline
\ \ \isakeyword{assumes}\ {\isachardoublequoteopen}A\ {\isasymunion}\ B\ {\isasymsubseteq}\ C{\isachardoublequoteclose}\ \isanewline
\ \ \isakeyword{shows}\ \ \ {\isachardoublequoteopen}A\ {\isasymsubseteq}\ C{\isachardoublequoteclose}\isanewline
%
\isadelimproof
%
\endisadelimproof
%
\isatagproof
\isacommand{proof}\isamarkupfalse%
\ {\isacharminus}\isanewline
\ \ \isacommand{have}\isamarkupfalse%
\ {\isachardoublequoteopen}A\ {\isasymsubseteq}\ C\ {\isasymand}\ B\ {\isasymsubseteq}\ C{\isachardoublequoteclose}\isanewline
\ \ \ \ \isacommand{using}\isamarkupfalse%
\ assms\isanewline
\ \ \ \ \isacommand{by}\isamarkupfalse%
\ {\isacharparenleft}simp\ only{\isacharcolon}\ sup{\isachardot}bounded{\isacharunderscore}iff{\isacharparenright}\isanewline
\ \ \isacommand{then}\isamarkupfalse%
\ \isacommand{show}\isamarkupfalse%
\ {\isachardoublequoteopen}A\ {\isasymsubseteq}\ C{\isachardoublequoteclose}\isanewline
\ \ \ \ \isacommand{by}\isamarkupfalse%
\ {\isacharparenleft}rule\ conjunct{\isadigit{1}}{\isacharparenright}\isanewline
\isacommand{qed}\isamarkupfalse%
%
\endisatagproof
{\isafoldproof}%
%
\isadelimproof
\isanewline
%
\endisadelimproof
\isanewline
\isacommand{lemma}\isamarkupfalse%
\ subContUnionRev{\isadigit{2}}{\isacharcolon}\ \isanewline
\ \ \isakeyword{assumes}\ {\isachardoublequoteopen}A\ {\isasymunion}\ B\ {\isasymsubseteq}\ C{\isachardoublequoteclose}\ \isanewline
\ \ \isakeyword{shows}\ \ \ {\isachardoublequoteopen}B\ {\isasymsubseteq}\ C{\isachardoublequoteclose}\isanewline
%
\isadelimproof
%
\endisadelimproof
%
\isatagproof
\isacommand{proof}\isamarkupfalse%
\ {\isacharminus}\isanewline
\ \ \isacommand{have}\isamarkupfalse%
\ {\isachardoublequoteopen}A\ {\isasymsubseteq}\ C\ {\isasymand}\ B\ {\isasymsubseteq}\ C{\isachardoublequoteclose}\isanewline
\ \ \ \ \isacommand{using}\isamarkupfalse%
\ assms\isanewline
\ \ \ \ \isacommand{by}\isamarkupfalse%
\ {\isacharparenleft}simp\ only{\isacharcolon}\ sup{\isachardot}bounded{\isacharunderscore}iff{\isacharparenright}\isanewline
\ \ \isacommand{then}\isamarkupfalse%
\ \isacommand{show}\isamarkupfalse%
\ {\isachardoublequoteopen}B\ {\isasymsubseteq}\ C{\isachardoublequoteclose}\isanewline
\ \ \ \ \isacommand{by}\isamarkupfalse%
\ {\isacharparenleft}rule\ conjunct{\isadigit{2}}{\isacharparenright}\isanewline
\isacommand{qed}\isamarkupfalse%
%
\endisatagproof
{\isafoldproof}%
%
\isadelimproof
%
\endisadelimproof
%
\begin{isamarkuptext}%
Sus correspondientes demostraciones automáticas se muestran a 
  continuación.%
\end{isamarkuptext}\isamarkuptrue%
\isacommand{lemma}\isamarkupfalse%
\ {\isachardoublequoteopen}A\ {\isasymunion}\ B\ {\isasymsubseteq}\ C\ {\isasymLongrightarrow}\ A\ {\isasymsubseteq}\ C{\isachardoublequoteclose}\isanewline
%
\isadelimproof
\ \ %
\endisadelimproof
%
\isatagproof
\isacommand{by}\isamarkupfalse%
\ simp%
\endisatagproof
{\isafoldproof}%
%
\isadelimproof
\isanewline
%
\endisadelimproof
\isanewline
\isacommand{lemma}\isamarkupfalse%
\ {\isachardoublequoteopen}A\ {\isasymunion}\ B\ {\isasymsubseteq}\ C\ {\isasymLongrightarrow}\ B\ {\isasymsubseteq}\ C{\isachardoublequoteclose}\isanewline
%
\isadelimproof
\ \ %
\endisadelimproof
%
\isatagproof
\isacommand{by}\isamarkupfalse%
\ simp%
\endisatagproof
{\isafoldproof}%
%
\isadelimproof
%
\endisadelimproof
%
\begin{isamarkuptext}%
Veamos ahora los distintos resultados sobre subfórmulas.

  \begin{lema}
    Todas las fórmulas atómicas de una fórmula son subfórmulas.
  \end{lema}

  \begin{demostracion}
    Aclaremos que el conjunto de las fórmulas atómicas de una fórmula 
    cualquiera está formado a partir de cada elemento de su conjunto de 
    variables proposicionales. 
    Queremos demostrar que este conjunto está contenido en el conjunto 
    de\\ subfórmulas de dicha fórmula.
    De este modo, la prueba seguirá el esquema inductivo para la 
    estructura de fórmulas. Veamos cada caso:
  
    Consideremos la fórmula atómica \isa{p} cualquiera. Como su
    conjunto de átomos es \isa{{\isacharbraceleft}p{\isacharbraceright}}, el conjunto de sus fórmulas atómicas
    correspondiente será \isa{{\isacharbraceleft}p{\isacharbraceright}}. Por otro lado, su conjunto de
    subfórmulas es también \isa{{\isacharbraceleft}p{\isacharbraceright}}, luego el conjunto de sus fórmulas 
    atómicas está contenido en el conjunto de sus subfórmulas como 
    queríamos demostrar.

    Sea la fórmula \isa{{\isasymbottom}}. Como su conjunto de átomos es vacío, es claro 
    que el conjunto de sus fórmulas atómicas es también el vacío y, por
    tanto, está contenido en el conjunto de sus subfórmulas.

    Sea la fórmula \isa{F} tal que el conjunto de sus fórmulas atómicas está
    contenido en el conjunto de sus subfórmulas. Probemos el resultado 
    para \isa{{\isasymnot}\ F}. 
    En primer lugar, sabemos que los 
    conjuntos de variables proposicionales de \isa{F} y \isa{{\isasymnot}\ F} coinciden, 
    luego tendrán igual conjunto de fórmulas atómicas. Por lo tanto,
    por hipótesis de inducción tenemos que el conjunto de fórmulas
    atómicas de \isa{F} está contenido en el conjunto de subfórmulas de 
    \isa{F}. Por otro lado, como el conjunto de subfórmulas de \isa{{\isasymnot}\ F} está 
    definido como\\ \isa{Subf{\isacharparenleft}{\isasymnot}\ F{\isacharparenright}\ {\isacharequal}\ {\isacharbraceleft}{\isasymnot}\ F{\isacharbraceright}\ {\isasymunion}\ Subf{\isacharparenleft}F{\isacharparenright}}, tenemos que el 
    el conjunto de subfórmulas de \isa{F} está contenido en el de \isa{{\isasymnot}\ F}.
    Por tanto, por propiedades de contención, 
    tenemos que el conjunto de fórmulas atómicas de \isa{{\isasymnot}\ F} está 
    contenido en el conjunto de subfórmulas de \isa{{\isasymnot}\ F} como queríamos 
    demostrar.

    Sean las fórmulas \isa{F} y \isa{G} tales que sus conjuntos de fórmulas 
    atómicas están contenidos en sus conjuntos de subfórmulas 
    respectivamente. Probemos ahora el resultado para \isa{F{\isacharasterisk}G}, donde \isa{{\isacharasterisk}}
    simboliza una conectiva binaria cualquiera.
    En primer lugar, sabemos que el conjunto de átomos de \isa{F{\isacharasterisk}G}
    es la unión de sus correspondientes conjuntos de átomos. De este
    modo, el conjunto de fórmulas atómicas de \isa{F{\isacharasterisk}G} será la unión del 
    conjunto de fórmulas atómicas de \isa{F} y el correspondiente de \isa{G}. 
    Por tanto, por hipótesis de inducción tenemos que el conjunto de 
    fórmulas atómicas de \isa{F{\isacharasterisk}G} está contenido en la unión del conjunto
    de subfórmulas de \isa{F} y el conjunto de subfórmulas de \isa{G}. Como el
    conjunto de subfórmulas de \isa{F{\isacharasterisk}G} se define como\\
    \isa{Subf{\isacharparenleft}F{\isacharasterisk}G{\isacharparenright}\ {\isacharequal}\ {\isacharbraceleft}F{\isacharasterisk}G{\isacharbraceright}\ {\isasymunion}\ Subf{\isacharparenleft}F{\isacharparenright}\ {\isasymunion}\ Subf{\isacharparenleft}G{\isacharparenright}}, tenemos que la unión
    de los conjuntos de subfórmulas de \isa{F} y \isa{G} está contenida en el
    conjunto de subfórmulas de \isa{F{\isacharasterisk}G}. Por tanto, por propiedades
    de la contención, tenemos que le conjunto de fórmulas atómicas de
    \isa{F{\isacharasterisk}G} está contenido en el conjunto de subfórmulas de \isa{F{\isacharasterisk}G} como 
    queríamos demostrar.  
  \end{demostracion}

  En Isabelle, se especifica como sigue.%
\end{isamarkuptext}\isamarkuptrue%
\isacommand{lemma}\isamarkupfalse%
\ {\isachardoublequoteopen}Atom\ {\isacharbackquote}\ atoms\ F\ {\isasymsubseteq}\ setSubformulae\ F{\isachardoublequoteclose}\isanewline
%
\isadelimproof
\ \ %
\endisadelimproof
%
\isatagproof
\isacommand{oops}\isamarkupfalse%
%
\endisatagproof
{\isafoldproof}%
%
\isadelimproof
%
\endisadelimproof
%
\begin{isamarkuptext}%
Debemos observar que \isa{Atom\ {\isacharbackquote}\ atoms\ F} construye las fórmulas 
  atómicas a partir de cada uno de los elementos de \isa{atoms\ F}, creando 
  un conjunto de fórmulas atómicas. Para ello emplea el infijo \isa{{\isacharbackquote}} 
  definido como notación abreviada de \isa{{\isacharparenleft}{\isacharbackquote}{\isacharparenright}} que calcula la 
  imagen de un conjunto en la teoría \href{https://n9.cl/qatp}{Set.thy}.

  \begin{itemize}
    \item[] \isa{f\ {\isacharbackquote}\ A\ {\isacharequal}\ {\isacharbraceleft}y\ {\isacharbar}\ {\isasymexists}x{\isasymin}A{\isachardot}\ y\ {\isacharequal}\ f\ x{\isacharbraceright}} 
      \hfill (\isa{image{\isacharunderscore}def})
  \end{itemize}

  Para aclarar su funcionamiento, veamos ejemplos para distintos casos 
  de fórmulas.%
\end{isamarkuptext}\isamarkuptrue%
\isacommand{notepad}\isamarkupfalse%
\isanewline
\isakeyword{begin}\isanewline
%
\isadelimproof
\ \ %
\endisadelimproof
%
\isatagproof
\isacommand{fix}\isamarkupfalse%
\ p\ q\ r\ {\isacharcolon}{\isacharcolon}\ {\isacharprime}a\isanewline
\isanewline
\ \ \isacommand{have}\isamarkupfalse%
\ {\isachardoublequoteopen}Atom\ {\isacharbackquote}atoms\ {\isacharparenleft}Atom\ p\ \isactrlbold {\isasymor}\ {\isasymbottom}{\isacharparenright}\ {\isacharequal}\ {\isacharbraceleft}Atom\ p{\isacharbraceright}{\isachardoublequoteclose}\isanewline
\ \ \ \ \isacommand{by}\isamarkupfalse%
\ simp\isanewline
\isanewline
\ \ \isacommand{have}\isamarkupfalse%
\ {\isachardoublequoteopen}Atom\ {\isacharbackquote}atoms\ {\isacharparenleft}{\isacharparenleft}Atom\ p\ \isactrlbold {\isasymrightarrow}\ Atom\ q{\isacharparenright}\ \isactrlbold {\isasymor}\ Atom\ r{\isacharparenright}\ {\isacharequal}\ \isanewline
\ \ \ \ \ \ \ {\isacharbraceleft}Atom\ p{\isacharcomma}\ Atom\ q{\isacharcomma}\ Atom\ r{\isacharbraceright}{\isachardoublequoteclose}\isanewline
\ \ \ \ \isacommand{by}\isamarkupfalse%
\ auto\ \isanewline
\isanewline
\ \ \isacommand{have}\isamarkupfalse%
\ {\isachardoublequoteopen}Atom\ {\isacharbackquote}atoms\ {\isacharparenleft}{\isacharparenleft}Atom\ p\ \isactrlbold {\isasymrightarrow}\ Atom\ p{\isacharparenright}\ \isactrlbold {\isasymor}\ Atom\ r{\isacharparenright}\ {\isacharequal}\ \isanewline
\ \ \ \ \ \ \ {\isacharbraceleft}Atom\ p{\isacharcomma}\ Atom\ r{\isacharbraceright}{\isachardoublequoteclose}\isanewline
\ \ \ \ \isacommand{by}\isamarkupfalse%
\ auto%
\endisatagproof
{\isafoldproof}%
%
\isadelimproof
\isanewline
%
\endisadelimproof
\isacommand{end}\isamarkupfalse%
%
\begin{isamarkuptext}%
Además, esta función tiene las siguientes propiedades sobre 
  conjuntos que utilizaremos en la demostración.

  \begin{itemize}
    \item[] \isa{f\ {\isacharbackquote}\ {\isacharparenleft}A\ {\isasymunion}\ B{\isacharparenright}\ {\isacharequal}\ f\ {\isacharbackquote}\ A\ {\isasymunion}\ f\ {\isacharbackquote}\ B} 
      \hfill (\isa{image{\isacharunderscore}Un})
    \item[] \isa{f\ {\isacharbackquote}\ {\isacharparenleft}{\isacharbraceleft}a{\isacharbraceright}\ {\isasymunion}\ B{\isacharparenright}\ {\isacharequal}\ {\isacharbraceleft}f\ a{\isacharbraceright}\ {\isasymunion}\ f\ {\isacharbackquote}\ B} 
      \hfill (\isa{image{\isacharunderscore}insert})
    \item[] \isa{f\ {\isacharbackquote}\ {\isasymemptyset}\ {\isacharequal}\ {\isasymemptyset}} 
      \hfill (\isa{image{\isacharunderscore}empty})
  \end{itemize}

  Una vez hechas las aclaraciones necesarias, comencemos la demostración 
  estructurada. Esta seguirá el esquema inductivo señalado con 
  anterioridad.%
\end{isamarkuptext}\isamarkuptrue%
\isacommand{lemma}\isamarkupfalse%
\ atoms{\isacharunderscore}are{\isacharunderscore}subformulae{\isacharunderscore}atom{\isacharcolon}\ \isanewline
\ \ {\isachardoublequoteopen}Atom\ {\isacharbackquote}\ atoms\ {\isacharparenleft}Atom\ x{\isacharparenright}\ {\isasymsubseteq}\ setSubformulae\ {\isacharparenleft}Atom\ x{\isacharparenright}{\isachardoublequoteclose}\ \isanewline
%
\isadelimproof
%
\endisadelimproof
%
\isatagproof
\isacommand{proof}\isamarkupfalse%
\ {\isacharminus}\isanewline
\ \ \isacommand{have}\isamarkupfalse%
\ {\isachardoublequoteopen}Atom\ {\isacharbackquote}\ atoms\ {\isacharparenleft}Atom\ x{\isacharparenright}\ {\isacharequal}\ Atom\ {\isacharbackquote}\ {\isacharbraceleft}x{\isacharbraceright}{\isachardoublequoteclose}\isanewline
\ \ \ \ \isacommand{by}\isamarkupfalse%
\ {\isacharparenleft}simp\ only{\isacharcolon}\ formula{\isachardot}set{\isacharparenleft}{\isadigit{1}}{\isacharparenright}{\isacharparenright}\isanewline
\ \ \isacommand{also}\isamarkupfalse%
\ \isacommand{have}\isamarkupfalse%
\ {\isachardoublequoteopen}{\isasymdots}\ {\isacharequal}\ {\isacharbraceleft}Atom\ x{\isacharbraceright}{\isachardoublequoteclose}\isanewline
\ \ \ \ \isacommand{by}\isamarkupfalse%
\ {\isacharparenleft}simp\ only{\isacharcolon}\ image{\isacharunderscore}insert\ image{\isacharunderscore}empty{\isacharparenright}\isanewline
\ \ \isacommand{also}\isamarkupfalse%
\ \isacommand{have}\isamarkupfalse%
\ {\isachardoublequoteopen}{\isasymdots}\ {\isacharequal}\ set\ {\isacharbrackleft}Atom\ x{\isacharbrackright}{\isachardoublequoteclose}\isanewline
\ \ \ \ \isacommand{by}\isamarkupfalse%
\ {\isacharparenleft}simp\ only{\isacharcolon}\ list{\isachardot}set{\isacharparenleft}{\isadigit{1}}{\isacharparenright}\ list{\isachardot}set{\isacharparenleft}{\isadigit{2}}{\isacharparenright}{\isacharparenright}\isanewline
\ \ \isacommand{also}\isamarkupfalse%
\ \isacommand{have}\isamarkupfalse%
\ {\isachardoublequoteopen}{\isasymdots}\ {\isacharequal}\ set\ {\isacharparenleft}subformulae\ {\isacharparenleft}Atom\ x{\isacharparenright}{\isacharparenright}{\isachardoublequoteclose}\isanewline
\ \ \ \ \isacommand{by}\isamarkupfalse%
\ {\isacharparenleft}simp\ only{\isacharcolon}\ subformulae{\isachardot}simps{\isacharparenleft}{\isadigit{1}}{\isacharparenright}{\isacharparenright}\isanewline
\ \ \isacommand{finally}\isamarkupfalse%
\ \isacommand{have}\isamarkupfalse%
\ {\isachardoublequoteopen}Atom\ {\isacharbackquote}\ atoms\ {\isacharparenleft}Atom\ x{\isacharparenright}\ {\isacharequal}\ set\ {\isacharparenleft}subformulae\ {\isacharparenleft}Atom\ x{\isacharparenright}{\isacharparenright}{\isachardoublequoteclose}\isanewline
\ \ \ \ \isacommand{by}\isamarkupfalse%
\ this\isanewline
\ \ \isacommand{then}\isamarkupfalse%
\ \isacommand{show}\isamarkupfalse%
\ {\isacharquery}thesis\ \isanewline
\ \ \ \ \isacommand{by}\isamarkupfalse%
\ {\isacharparenleft}simp\ only{\isacharcolon}\ subset{\isacharunderscore}refl{\isacharparenright}\isanewline
\isacommand{qed}\isamarkupfalse%
%
\endisatagproof
{\isafoldproof}%
%
\isadelimproof
\isanewline
%
\endisadelimproof
\isanewline
\isacommand{lemma}\isamarkupfalse%
\ atoms{\isacharunderscore}are{\isacharunderscore}subformulae{\isacharunderscore}bot{\isacharcolon}\ \isanewline
\ \ {\isachardoublequoteopen}Atom\ {\isacharbackquote}\ atoms\ {\isasymbottom}\ {\isasymsubseteq}\ setSubformulae\ {\isasymbottom}{\isachardoublequoteclose}\ \ \isanewline
%
\isadelimproof
%
\endisadelimproof
%
\isatagproof
\isacommand{proof}\isamarkupfalse%
\ {\isacharminus}\isanewline
\ \ \isacommand{have}\isamarkupfalse%
\ {\isachardoublequoteopen}Atom\ {\isacharbackquote}\ atoms\ {\isasymbottom}\ {\isacharequal}\ Atom\ {\isacharbackquote}\ {\isasymemptyset}{\isachardoublequoteclose}\isanewline
\ \ \ \ \isacommand{by}\isamarkupfalse%
\ {\isacharparenleft}simp\ only{\isacharcolon}\ formula{\isachardot}set{\isacharparenleft}{\isadigit{2}}{\isacharparenright}{\isacharparenright}\isanewline
\ \ \isacommand{also}\isamarkupfalse%
\ \isacommand{have}\isamarkupfalse%
\ {\isachardoublequoteopen}{\isasymdots}\ {\isacharequal}\ {\isasymemptyset}{\isachardoublequoteclose}\isanewline
\ \ \ \ \isacommand{by}\isamarkupfalse%
\ {\isacharparenleft}simp\ only{\isacharcolon}\ image{\isacharunderscore}empty{\isacharparenright}\isanewline
\ \ \isacommand{also}\isamarkupfalse%
\ \isacommand{have}\isamarkupfalse%
\ {\isachardoublequoteopen}{\isasymdots}\ {\isasymsubseteq}\ setSubformulae\ {\isasymbottom}{\isachardoublequoteclose}\isanewline
\ \ \ \ \isacommand{by}\isamarkupfalse%
\ {\isacharparenleft}simp\ only{\isacharcolon}\ empty{\isacharunderscore}subsetI{\isacharparenright}\isanewline
\ \ \isacommand{finally}\isamarkupfalse%
\ \isacommand{show}\isamarkupfalse%
\ {\isacharquery}thesis\isanewline
\ \ \ \ \isacommand{by}\isamarkupfalse%
\ this\isanewline
\isacommand{qed}\isamarkupfalse%
%
\endisatagproof
{\isafoldproof}%
%
\isadelimproof
\isanewline
%
\endisadelimproof
\isanewline
\isacommand{lemma}\isamarkupfalse%
\ atoms{\isacharunderscore}are{\isacharunderscore}subformulae{\isacharunderscore}not{\isacharcolon}\ \isanewline
\ \ \isakeyword{assumes}\ {\isachardoublequoteopen}Atom\ {\isacharbackquote}\ atoms\ F\ {\isasymsubseteq}\ setSubformulae\ F{\isachardoublequoteclose}\ \isanewline
\ \ \isakeyword{shows}\ \ \ {\isachardoublequoteopen}Atom\ {\isacharbackquote}\ atoms\ {\isacharparenleft}\isactrlbold {\isasymnot}\ F{\isacharparenright}\ {\isasymsubseteq}\ setSubformulae\ {\isacharparenleft}\isactrlbold {\isasymnot}\ F{\isacharparenright}{\isachardoublequoteclose}\isanewline
%
\isadelimproof
%
\endisadelimproof
%
\isatagproof
\isacommand{proof}\isamarkupfalse%
\ {\isacharminus}\isanewline
\ \ \isacommand{have}\isamarkupfalse%
\ {\isachardoublequoteopen}Atom\ {\isacharbackquote}\ atoms\ {\isacharparenleft}\isactrlbold {\isasymnot}\ F{\isacharparenright}\ {\isacharequal}\ Atom\ {\isacharbackquote}\ atoms\ F{\isachardoublequoteclose}\isanewline
\ \ \ \ \isacommand{by}\isamarkupfalse%
\ {\isacharparenleft}simp\ only{\isacharcolon}\ formula{\isachardot}set{\isacharparenleft}{\isadigit{3}}{\isacharparenright}{\isacharparenright}\isanewline
\ \ \isacommand{also}\isamarkupfalse%
\ \isacommand{have}\isamarkupfalse%
\ {\isachardoublequoteopen}{\isasymdots}\ {\isasymsubseteq}\ setSubformulae\ F{\isachardoublequoteclose}\isanewline
\ \ \ \ \isacommand{by}\isamarkupfalse%
\ {\isacharparenleft}simp\ only{\isacharcolon}\ assms{\isacharparenright}\isanewline
\ \ \isacommand{also}\isamarkupfalse%
\ \isacommand{have}\isamarkupfalse%
\ {\isachardoublequoteopen}{\isasymdots}\ {\isasymsubseteq}\ {\isacharbraceleft}\isactrlbold {\isasymnot}\ F{\isacharbraceright}\ {\isasymunion}\ setSubformulae\ F{\isachardoublequoteclose}\isanewline
\ \ \ \ \isacommand{by}\isamarkupfalse%
\ {\isacharparenleft}simp\ only{\isacharcolon}\ Un{\isacharunderscore}upper{\isadigit{2}}{\isacharparenright}\isanewline
\ \ \isacommand{also}\isamarkupfalse%
\ \isacommand{have}\isamarkupfalse%
\ {\isachardoublequoteopen}{\isasymdots}\ {\isacharequal}\ setSubformulae\ {\isacharparenleft}\isactrlbold {\isasymnot}\ F{\isacharparenright}{\isachardoublequoteclose}\isanewline
\ \ \ \ \isacommand{by}\isamarkupfalse%
\ {\isacharparenleft}simp\ only{\isacharcolon}\ setSubformulae{\isacharunderscore}not{\isacharparenright}\isanewline
\ \ \isacommand{finally}\isamarkupfalse%
\ \isacommand{show}\isamarkupfalse%
\ {\isacharquery}thesis\isanewline
\ \ \ \ \isacommand{by}\isamarkupfalse%
\ this\isanewline
\isacommand{qed}\isamarkupfalse%
%
\endisatagproof
{\isafoldproof}%
%
\isadelimproof
\isanewline
%
\endisadelimproof
\isanewline
\isacommand{lemma}\isamarkupfalse%
\ atoms{\isacharunderscore}are{\isacharunderscore}subformulae{\isacharunderscore}and{\isacharcolon}\ \isanewline
\ \ \isakeyword{assumes}\ {\isachardoublequoteopen}Atom\ {\isacharbackquote}\ atoms\ F{\isadigit{1}}\ {\isasymsubseteq}\ setSubformulae\ F{\isadigit{1}}{\isachardoublequoteclose}\isanewline
\ \ \ \ \ \ \ \ \ \ {\isachardoublequoteopen}Atom\ {\isacharbackquote}\ atoms\ F{\isadigit{2}}\ {\isasymsubseteq}\ setSubformulae\ F{\isadigit{2}}{\isachardoublequoteclose}\isanewline
\ \ \isakeyword{shows}\ \ \ {\isachardoublequoteopen}Atom\ {\isacharbackquote}\ atoms\ {\isacharparenleft}F{\isadigit{1}}\ \isactrlbold {\isasymand}\ F{\isadigit{2}}{\isacharparenright}\ {\isasymsubseteq}\ setSubformulae\ {\isacharparenleft}F{\isadigit{1}}\ \isactrlbold {\isasymand}\ F{\isadigit{2}}{\isacharparenright}{\isachardoublequoteclose}\isanewline
%
\isadelimproof
%
\endisadelimproof
%
\isatagproof
\isacommand{proof}\isamarkupfalse%
\ {\isacharminus}\isanewline
\ \ \isacommand{have}\isamarkupfalse%
\ {\isachardoublequoteopen}Atom\ {\isacharbackquote}\ atoms\ {\isacharparenleft}F{\isadigit{1}}\ \isactrlbold {\isasymand}\ F{\isadigit{2}}{\isacharparenright}\ {\isacharequal}\ Atom\ {\isacharbackquote}\ {\isacharparenleft}atoms\ F{\isadigit{1}}\ {\isasymunion}\ atoms\ F{\isadigit{2}}{\isacharparenright}{\isachardoublequoteclose}\isanewline
\ \ \ \ \isacommand{by}\isamarkupfalse%
\ {\isacharparenleft}simp\ only{\isacharcolon}\ formula{\isachardot}set{\isacharparenleft}{\isadigit{4}}{\isacharparenright}{\isacharparenright}\isanewline
\ \ \isacommand{also}\isamarkupfalse%
\ \isacommand{have}\isamarkupfalse%
\ {\isachardoublequoteopen}{\isasymdots}\ {\isacharequal}\ Atom\ {\isacharbackquote}\ atoms\ F{\isadigit{1}}\ {\isasymunion}\ Atom\ {\isacharbackquote}\ atoms\ F{\isadigit{2}}{\isachardoublequoteclose}\ \isanewline
\ \ \ \ \isacommand{by}\isamarkupfalse%
\ {\isacharparenleft}rule\ image{\isacharunderscore}Un{\isacharparenright}\isanewline
\ \ \isacommand{also}\isamarkupfalse%
\ \isacommand{have}\isamarkupfalse%
\ {\isachardoublequoteopen}{\isasymdots}\ {\isasymsubseteq}\ setSubformulae\ F{\isadigit{1}}\ {\isasymunion}\ setSubformulae\ F{\isadigit{2}}{\isachardoublequoteclose}\isanewline
\ \ \ \ \isacommand{using}\isamarkupfalse%
\ assms\isanewline
\ \ \ \ \isacommand{by}\isamarkupfalse%
\ {\isacharparenleft}rule\ Un{\isacharunderscore}mono{\isacharparenright}\isanewline
\ \ \isacommand{also}\isamarkupfalse%
\ \isacommand{have}\isamarkupfalse%
\ {\isachardoublequoteopen}{\isasymdots}\ {\isasymsubseteq}\ {\isacharbraceleft}F{\isadigit{1}}\ \isactrlbold {\isasymand}\ F{\isadigit{2}}{\isacharbraceright}\ {\isasymunion}\ {\isacharparenleft}setSubformulae\ F{\isadigit{1}}\ {\isasymunion}\ setSubformulae\ F{\isadigit{2}}{\isacharparenright}{\isachardoublequoteclose}\isanewline
\ \ \ \ \isacommand{by}\isamarkupfalse%
\ {\isacharparenleft}simp\ only{\isacharcolon}\ Un{\isacharunderscore}upper{\isadigit{2}}{\isacharparenright}\isanewline
\ \ \isacommand{also}\isamarkupfalse%
\ \isacommand{have}\isamarkupfalse%
\ {\isachardoublequoteopen}{\isasymdots}\ {\isacharequal}\ setSubformulae\ {\isacharparenleft}F{\isadigit{1}}\ \isactrlbold {\isasymand}\ F{\isadigit{2}}{\isacharparenright}{\isachardoublequoteclose}\isanewline
\ \ \ \ \isacommand{by}\isamarkupfalse%
\ {\isacharparenleft}simp\ only{\isacharcolon}\ setSubformulae{\isacharunderscore}and{\isacharparenright}\isanewline
\ \ \isacommand{finally}\isamarkupfalse%
\ \isacommand{show}\isamarkupfalse%
\ {\isacharquery}thesis\isanewline
\ \ \ \ \isacommand{by}\isamarkupfalse%
\ this\isanewline
\isacommand{qed}\isamarkupfalse%
%
\endisatagproof
{\isafoldproof}%
%
\isadelimproof
\isanewline
%
\endisadelimproof
\isanewline
\isacommand{lemma}\isamarkupfalse%
\ atoms{\isacharunderscore}are{\isacharunderscore}subformulae{\isacharunderscore}or{\isacharcolon}\ \isanewline
\ \ \isakeyword{assumes}\ {\isachardoublequoteopen}Atom\ {\isacharbackquote}\ atoms\ F{\isadigit{1}}\ {\isasymsubseteq}\ setSubformulae\ F{\isadigit{1}}{\isachardoublequoteclose}\isanewline
\ \ \ \ \ \ \ \ \ \ {\isachardoublequoteopen}Atom\ {\isacharbackquote}\ atoms\ F{\isadigit{2}}\ {\isasymsubseteq}\ setSubformulae\ F{\isadigit{2}}{\isachardoublequoteclose}\isanewline
\ \ \isakeyword{shows}\ \ \ {\isachardoublequoteopen}Atom\ {\isacharbackquote}\ atoms\ {\isacharparenleft}F{\isadigit{1}}\ \isactrlbold {\isasymor}\ F{\isadigit{2}}{\isacharparenright}\ {\isasymsubseteq}\ setSubformulae\ {\isacharparenleft}F{\isadigit{1}}\ \isactrlbold {\isasymor}\ F{\isadigit{2}}{\isacharparenright}{\isachardoublequoteclose}\isanewline
%
\isadelimproof
%
\endisadelimproof
%
\isatagproof
\isacommand{proof}\isamarkupfalse%
\ {\isacharminus}\isanewline
\ \ \isacommand{have}\isamarkupfalse%
\ {\isachardoublequoteopen}Atom\ {\isacharbackquote}\ atoms\ {\isacharparenleft}F{\isadigit{1}}\ \isactrlbold {\isasymor}\ F{\isadigit{2}}{\isacharparenright}\ {\isacharequal}\ Atom\ {\isacharbackquote}\ {\isacharparenleft}atoms\ F{\isadigit{1}}\ {\isasymunion}\ atoms\ F{\isadigit{2}}{\isacharparenright}{\isachardoublequoteclose}\isanewline
\ \ \ \ \isacommand{by}\isamarkupfalse%
\ {\isacharparenleft}simp\ only{\isacharcolon}\ formula{\isachardot}set{\isacharparenleft}{\isadigit{5}}{\isacharparenright}{\isacharparenright}\isanewline
\ \ \isacommand{also}\isamarkupfalse%
\ \isacommand{have}\isamarkupfalse%
\ {\isachardoublequoteopen}{\isasymdots}\ {\isacharequal}\ Atom\ {\isacharbackquote}\ atoms\ F{\isadigit{1}}\ {\isasymunion}\ Atom\ {\isacharbackquote}\ atoms\ F{\isadigit{2}}{\isachardoublequoteclose}\ \isanewline
\ \ \ \ \isacommand{by}\isamarkupfalse%
\ {\isacharparenleft}rule\ image{\isacharunderscore}Un{\isacharparenright}\isanewline
\ \ \isacommand{also}\isamarkupfalse%
\ \isacommand{have}\isamarkupfalse%
\ {\isachardoublequoteopen}{\isasymdots}\ {\isasymsubseteq}\ setSubformulae\ F{\isadigit{1}}\ {\isasymunion}\ setSubformulae\ F{\isadigit{2}}{\isachardoublequoteclose}\isanewline
\ \ \ \ \isacommand{using}\isamarkupfalse%
\ assms\isanewline
\ \ \ \ \isacommand{by}\isamarkupfalse%
\ {\isacharparenleft}rule\ Un{\isacharunderscore}mono{\isacharparenright}\isanewline
\ \ \isacommand{also}\isamarkupfalse%
\ \isacommand{have}\isamarkupfalse%
\ {\isachardoublequoteopen}{\isasymdots}\ {\isasymsubseteq}\ {\isacharbraceleft}F{\isadigit{1}}\ \isactrlbold {\isasymor}\ F{\isadigit{2}}{\isacharbraceright}\ {\isasymunion}\ {\isacharparenleft}setSubformulae\ F{\isadigit{1}}\ {\isasymunion}\ setSubformulae\ F{\isadigit{2}}{\isacharparenright}{\isachardoublequoteclose}\isanewline
\ \ \ \ \isacommand{by}\isamarkupfalse%
\ {\isacharparenleft}simp\ only{\isacharcolon}\ Un{\isacharunderscore}upper{\isadigit{2}}{\isacharparenright}\isanewline
\ \ \isacommand{also}\isamarkupfalse%
\ \isacommand{have}\isamarkupfalse%
\ {\isachardoublequoteopen}{\isasymdots}\ {\isacharequal}\ setSubformulae\ {\isacharparenleft}F{\isadigit{1}}\ \isactrlbold {\isasymor}\ F{\isadigit{2}}{\isacharparenright}{\isachardoublequoteclose}\isanewline
\ \ \ \ \isacommand{by}\isamarkupfalse%
\ {\isacharparenleft}simp\ only{\isacharcolon}\ setSubformulae{\isacharunderscore}or{\isacharparenright}\isanewline
\ \ \isacommand{finally}\isamarkupfalse%
\ \isacommand{show}\isamarkupfalse%
\ {\isacharquery}thesis\isanewline
\ \ \ \ \isacommand{by}\isamarkupfalse%
\ this\isanewline
\isacommand{qed}\isamarkupfalse%
%
\endisatagproof
{\isafoldproof}%
%
\isadelimproof
\isanewline
%
\endisadelimproof
\isanewline
\isacommand{lemma}\isamarkupfalse%
\ atoms{\isacharunderscore}are{\isacharunderscore}subformulae{\isacharunderscore}imp{\isacharcolon}\ \isanewline
\ \ \isakeyword{assumes}\ {\isachardoublequoteopen}Atom\ {\isacharbackquote}\ atoms\ F{\isadigit{1}}\ {\isasymsubseteq}\ setSubformulae\ F{\isadigit{1}}{\isachardoublequoteclose}\isanewline
\ \ \ \ \ \ \ \ \ \ {\isachardoublequoteopen}Atom\ {\isacharbackquote}\ atoms\ F{\isadigit{2}}\ {\isasymsubseteq}\ setSubformulae\ F{\isadigit{2}}{\isachardoublequoteclose}\isanewline
\ \ \isakeyword{shows}\ \ \ {\isachardoublequoteopen}Atom\ {\isacharbackquote}\ atoms\ {\isacharparenleft}F{\isadigit{1}}\ \isactrlbold {\isasymrightarrow}\ F{\isadigit{2}}{\isacharparenright}\ {\isasymsubseteq}\ setSubformulae\ {\isacharparenleft}F{\isadigit{1}}\ \isactrlbold {\isasymrightarrow}\ F{\isadigit{2}}{\isacharparenright}{\isachardoublequoteclose}\isanewline
%
\isadelimproof
%
\endisadelimproof
%
\isatagproof
\isacommand{proof}\isamarkupfalse%
\ {\isacharminus}\isanewline
\ \ \isacommand{have}\isamarkupfalse%
\ {\isachardoublequoteopen}Atom\ {\isacharbackquote}\ atoms\ {\isacharparenleft}F{\isadigit{1}}\ \isactrlbold {\isasymrightarrow}\ F{\isadigit{2}}{\isacharparenright}\ {\isacharequal}\ Atom\ {\isacharbackquote}\ {\isacharparenleft}atoms\ F{\isadigit{1}}\ {\isasymunion}\ atoms\ F{\isadigit{2}}{\isacharparenright}{\isachardoublequoteclose}\isanewline
\ \ \ \ \isacommand{by}\isamarkupfalse%
\ {\isacharparenleft}simp\ only{\isacharcolon}\ formula{\isachardot}set{\isacharparenleft}{\isadigit{6}}{\isacharparenright}{\isacharparenright}\isanewline
\ \ \isacommand{also}\isamarkupfalse%
\ \isacommand{have}\isamarkupfalse%
\ {\isachardoublequoteopen}{\isasymdots}\ {\isacharequal}\ Atom\ {\isacharbackquote}\ atoms\ F{\isadigit{1}}\ {\isasymunion}\ Atom\ {\isacharbackquote}\ atoms\ F{\isadigit{2}}{\isachardoublequoteclose}\ \isanewline
\ \ \ \ \isacommand{by}\isamarkupfalse%
\ {\isacharparenleft}rule\ image{\isacharunderscore}Un{\isacharparenright}\isanewline
\ \ \isacommand{also}\isamarkupfalse%
\ \isacommand{have}\isamarkupfalse%
\ {\isachardoublequoteopen}{\isasymdots}\ {\isasymsubseteq}\ setSubformulae\ F{\isadigit{1}}\ {\isasymunion}\ setSubformulae\ F{\isadigit{2}}{\isachardoublequoteclose}\isanewline
\ \ \ \ \isacommand{using}\isamarkupfalse%
\ assms\isanewline
\ \ \ \ \isacommand{by}\isamarkupfalse%
\ {\isacharparenleft}rule\ Un{\isacharunderscore}mono{\isacharparenright}\isanewline
\ \ \isacommand{also}\isamarkupfalse%
\ \isacommand{have}\isamarkupfalse%
\ {\isachardoublequoteopen}{\isasymdots}\ {\isasymsubseteq}\ {\isacharbraceleft}F{\isadigit{1}}\ \isactrlbold {\isasymrightarrow}\ F{\isadigit{2}}{\isacharbraceright}\ {\isasymunion}\ {\isacharparenleft}setSubformulae\ F{\isadigit{1}}\ {\isasymunion}\ setSubformulae\ F{\isadigit{2}}{\isacharparenright}{\isachardoublequoteclose}\isanewline
\ \ \ \ \isacommand{by}\isamarkupfalse%
\ {\isacharparenleft}simp\ only{\isacharcolon}\ Un{\isacharunderscore}upper{\isadigit{2}}{\isacharparenright}\isanewline
\ \ \isacommand{also}\isamarkupfalse%
\ \isacommand{have}\isamarkupfalse%
\ {\isachardoublequoteopen}{\isasymdots}\ {\isacharequal}\ setSubformulae\ {\isacharparenleft}F{\isadigit{1}}\ \isactrlbold {\isasymrightarrow}\ F{\isadigit{2}}{\isacharparenright}{\isachardoublequoteclose}\isanewline
\ \ \ \ \isacommand{by}\isamarkupfalse%
\ {\isacharparenleft}simp\ only{\isacharcolon}\ setSubformulae{\isacharunderscore}imp{\isacharparenright}\isanewline
\ \ \isacommand{finally}\isamarkupfalse%
\ \isacommand{show}\isamarkupfalse%
\ {\isacharquery}thesis\isanewline
\ \ \ \ \isacommand{by}\isamarkupfalse%
\ this\isanewline
\isacommand{qed}\isamarkupfalse%
%
\endisatagproof
{\isafoldproof}%
%
\isadelimproof
\isanewline
%
\endisadelimproof
\isanewline
\isacommand{lemma}\isamarkupfalse%
\ atoms{\isacharunderscore}are{\isacharunderscore}subformulae{\isacharcolon}\ \isanewline
\ \ {\isachardoublequoteopen}Atom\ {\isacharbackquote}\ atoms\ F\ {\isasymsubseteq}\ setSubformulae\ F{\isachardoublequoteclose}\isanewline
%
\isadelimproof
%
\endisadelimproof
%
\isatagproof
\isacommand{proof}\isamarkupfalse%
\ {\isacharparenleft}induction\ F{\isacharparenright}\isanewline
\ \ \isacommand{case}\isamarkupfalse%
\ {\isacharparenleft}Atom\ x{\isacharparenright}\isanewline
\ \ \isacommand{then}\isamarkupfalse%
\ \isacommand{show}\isamarkupfalse%
\ {\isacharquery}case\ \isacommand{by}\isamarkupfalse%
\ {\isacharparenleft}simp\ only{\isacharcolon}\ atoms{\isacharunderscore}are{\isacharunderscore}subformulae{\isacharunderscore}atom{\isacharparenright}\ \isanewline
\isacommand{next}\isamarkupfalse%
\isanewline
\ \ \isacommand{case}\isamarkupfalse%
\ Bot\isanewline
\ \ \isacommand{then}\isamarkupfalse%
\ \isacommand{show}\isamarkupfalse%
\ {\isacharquery}case\ \isacommand{by}\isamarkupfalse%
\ {\isacharparenleft}simp\ only{\isacharcolon}\ atoms{\isacharunderscore}are{\isacharunderscore}subformulae{\isacharunderscore}bot{\isacharparenright}\ \isanewline
\isacommand{next}\isamarkupfalse%
\isanewline
\ \ \isacommand{case}\isamarkupfalse%
\ {\isacharparenleft}Not\ F{\isacharparenright}\isanewline
\ \ \isacommand{then}\isamarkupfalse%
\ \isacommand{show}\isamarkupfalse%
\ {\isacharquery}case\ \isacommand{by}\isamarkupfalse%
\ {\isacharparenleft}simp\ only{\isacharcolon}\ atoms{\isacharunderscore}are{\isacharunderscore}subformulae{\isacharunderscore}not{\isacharparenright}\ \isanewline
\isacommand{next}\isamarkupfalse%
\isanewline
\ \ \isacommand{case}\isamarkupfalse%
\ {\isacharparenleft}And\ F{\isadigit{1}}\ F{\isadigit{2}}{\isacharparenright}\isanewline
\ \ \isacommand{then}\isamarkupfalse%
\ \isacommand{show}\isamarkupfalse%
\ {\isacharquery}case\ \isacommand{by}\isamarkupfalse%
\ {\isacharparenleft}simp\ only{\isacharcolon}\ atoms{\isacharunderscore}are{\isacharunderscore}subformulae{\isacharunderscore}and{\isacharparenright}\ \isanewline
\isacommand{next}\isamarkupfalse%
\isanewline
\ \ \isacommand{case}\isamarkupfalse%
\ {\isacharparenleft}Or\ F{\isadigit{1}}\ F{\isadigit{2}}{\isacharparenright}\isanewline
\ \ \isacommand{then}\isamarkupfalse%
\ \isacommand{show}\isamarkupfalse%
\ {\isacharquery}case\ \isacommand{by}\isamarkupfalse%
\ {\isacharparenleft}simp\ only{\isacharcolon}\ atoms{\isacharunderscore}are{\isacharunderscore}subformulae{\isacharunderscore}or{\isacharparenright}\isanewline
\isacommand{next}\isamarkupfalse%
\isanewline
\ \ \isacommand{case}\isamarkupfalse%
\ {\isacharparenleft}Imp\ F{\isadigit{1}}\ F{\isadigit{2}}{\isacharparenright}\isanewline
\ \ \isacommand{then}\isamarkupfalse%
\ \isacommand{show}\isamarkupfalse%
\ {\isacharquery}case\ \isacommand{by}\isamarkupfalse%
\ {\isacharparenleft}simp\ only{\isacharcolon}\ atoms{\isacharunderscore}are{\isacharunderscore}subformulae{\isacharunderscore}imp{\isacharparenright}\isanewline
\isacommand{qed}\isamarkupfalse%
%
\endisatagproof
{\isafoldproof}%
%
\isadelimproof
%
\endisadelimproof
%
\begin{isamarkuptext}%
La demostración automática queda igualmente expuesta a 
  continuación.%
\end{isamarkuptext}\isamarkuptrue%
\isacommand{lemma}\isamarkupfalse%
\ {\isachardoublequoteopen}Atom\ {\isacharbackquote}\ atoms\ F\ {\isasymsubseteq}\ setSubformulae\ F{\isachardoublequoteclose}\isanewline
%
\isadelimproof
\ \ %
\endisadelimproof
%
\isatagproof
\isacommand{by}\isamarkupfalse%
\ {\isacharparenleft}induction\ F{\isacharparenright}\ \ auto%
\endisatagproof
{\isafoldproof}%
%
\isadelimproof
%
\endisadelimproof
%
\begin{isamarkuptext}%
La siguiente propiedad declara que el conjunto de átomos de una 
  subfórmula está contenido en el conjunto de átomos de la propia 
  fórmula.

  \begin{lema}
  Dada una fórmula, los átomos de sus subfórmulas son átomos de ella
  misma.
  \end{lema}

  \begin{demostracion}
  Procedemos mediante inducción en la estructura de las fórmulas según 
  los distintos casos:

  Sea \isa{p} una fórmula atómica cualquiera. Por definición de su conjunto
  de subfórmulas, su única subfórmula es ella misma, luego se verifica
  el resultado.

  Sea la fórmula \isa{{\isasymbottom}}. Por definición de su conjunto de subfórmulas, su 
  única subfórmula es ella misma, luego se verifica análogamente la
  propiedad en este caso.

  Sea la fórmula \isa{F} tal que para cualquier subfórmula suya se verifica 
  que el conjunto de sus átomos está contenido en el conjunto de átomos 
  de \isa{F}. Supongamos \isa{G} subfórmula cualquiera de \isa{{\isasymnot}\ F}. Vamos a
  probar que el conjunto de átomos de \isa{G} está contenido en el de 
  \isa{{\isasymnot}\ F}.
  Por definición, tenemos que el conjunto de subfórmulas de \isa{{\isasymnot}\ F} es de
  la forma \\ \isa{Subf{\isacharparenleft}{\isasymnot}\ F{\isacharparenright}\ {\isacharequal}\ {\isacharbraceleft}{\isasymnot}\ F{\isacharbraceright}\ {\isasymunion}\ Subf{\isacharparenleft}F{\isacharparenright}}. De este modo, tenemos dos 
  opciones posibles:\\ \isa{G\ {\isasymin}\ {\isacharbraceleft}{\isasymnot}\ F{\isacharbraceright}} o \isa{G\ {\isasymin}\ Subf{\isacharparenleft}F{\isacharparenright}}. 
  Del primer caso se deduce \isa{G\ {\isacharequal}\ {\isasymnot}\ F} 
  y, por tanto, tienen igual conjunto de átomos.
  Observando el segundo caso, por hipótesis de inducción, se tiene que 
  el conjunto de átomos de \isa{G} está contenido en el de \isa{F}. Además, como 
  el conjunto de átomos de \isa{F} y \isa{{\isasymnot}\ F} coinciden, se verifica el 
  resultado.

  Sea \isa{F{\isadigit{1}}} una fórmula proposicional tal que el conjunto de los átomos 
  de cualquier subfórmula suya está contenido en el conjunto de átomos 
  de \isa{F{\isadigit{1}}}. Sea también \isa{F{\isadigit{2}}}\\ cumpliendo dicha hipótesis de inducción 
  para sus correspondientes subfórmulas. Supongamos además que \isa{G} es
  subfórmula de \isa{F{\isadigit{1}}{\isacharasterisk}F{\isadigit{2}}}, donde \isa{{\isacharasterisk}} simboliza una conectiva binaria 
  cualquiera. Vamos a probar que el conjunto de átomos de \isa{G} está 
  contenido en el conjunto de átomos de \isa{F{\isadigit{1}}{\isacharasterisk}F{\isadigit{2}}}.
  En primer lugar, por definición tenemos que el conjunto de
  subfórmulas de \isa{F{\isadigit{1}}{\isacharasterisk}F{\isadigit{2}}} es de la forma\\
  \isa{Subf{\isacharparenleft}F{\isadigit{1}}{\isacharasterisk}F{\isadigit{2}}{\isacharparenright}\ {\isacharequal}\ {\isacharbraceleft}F{\isadigit{1}}{\isacharasterisk}F{\isadigit{2}}{\isacharbraceright}\ {\isasymunion}\ {\isacharparenleft}Subf{\isacharparenleft}F{\isadigit{1}}{\isacharparenright}\ {\isasymunion}\ Subf{\isacharparenleft}F{\isadigit{2}}{\isacharparenright}{\isacharparenright}}. De este modo, 
  tenemos dos posibles opciones:
  \isa{G\ {\isasymin}\ {\isacharbraceleft}F{\isadigit{1}}{\isacharasterisk}F{\isadigit{2}}{\isacharbraceright}} o \isa{G\ {\isasymin}\ Subf{\isacharparenleft}F{\isadigit{1}}{\isacharparenright}\ {\isasymunion}\ Subf{\isacharparenleft}F{\isadigit{2}}{\isacharparenright}}.
  Si \isa{G\ {\isasymin}\ {\isacharbraceleft}F{\isadigit{1}}{\isacharasterisk}F{\isadigit{2}}{\isacharbraceright}}, entonces \isa{G\ {\isacharequal}\ F{\isadigit{1}}{\isacharasterisk}F{\isadigit{2}}} y tienen igual conjunto de 
  átomos.
  Por otro lado, si \isa{G\ {\isasymin}\ Subf{\isacharparenleft}F{\isadigit{1}}{\isacharparenright}\ {\isasymunion}\ Subf{\isacharparenleft}F{\isadigit{2}}{\isacharparenright}} tenemos dos nuevas
  posibilidades: \isa{G} es subfórmula de \isa{F{\isadigit{1}}} o \isa{G} es subfórmula de \isa{F{\isadigit{2}}}.
  Suponiendo que fuese subfórmula de \isa{F{\isadigit{1}}}, aplicando hipótesis de
  inducción tendríamos que el conjunto de átomos de \isa{G} está contenido 
  en el de \isa{F{\isadigit{1}}}. De este modo, como el conjunto de átomos de \isa{F{\isadigit{1}}{\isacharasterisk}F{\isadigit{2}}} se
  define como la unión de los conjuntos de átomos de \isa{F{\isadigit{1}}} y \isa{F{\isadigit{2}}}, por
  propiedades de la contención se verifica que el conjunto de átomos de
  \isa{G} está contenido en el de \isa{F{\isadigit{1}}{\isacharasterisk}F{\isadigit{2}}}. Observemos que si \isa{G} es 
  subfórmula de \isa{F{\isadigit{2}}}, se demuestra análogamente cambiando los
  subíndices correspondientes. Por tanto, se tiene el resultado.      
  \end{demostracion}

  Formalizado en Isabelle:%
\end{isamarkuptext}\isamarkuptrue%
\isacommand{lemma}\isamarkupfalse%
\ {\isachardoublequoteopen}G\ {\isasymin}\ setSubformulae\ F\ {\isasymLongrightarrow}\ atoms\ G\ {\isasymsubseteq}\ atoms\ F{\isachardoublequoteclose}\isanewline
%
\isadelimproof
\ \ %
\endisadelimproof
%
\isatagproof
\isacommand{oops}\isamarkupfalse%
%
\endisatagproof
{\isafoldproof}%
%
\isadelimproof
%
\endisadelimproof
%
\begin{isamarkuptext}%
Veamos su demostración estructurada.%
\end{isamarkuptext}\isamarkuptrue%
\isacommand{lemma}\isamarkupfalse%
\ subformulas{\isacharunderscore}atoms{\isacharunderscore}atom{\isacharcolon}\isanewline
\ \ \isakeyword{assumes}\ {\isachardoublequoteopen}G\ {\isasymin}\ setSubformulae\ {\isacharparenleft}Atom\ x{\isacharparenright}{\isachardoublequoteclose}\ \isanewline
\ \ \isakeyword{shows}\ \ \ {\isachardoublequoteopen}atoms\ G\ {\isasymsubseteq}\ atoms\ {\isacharparenleft}Atom\ x{\isacharparenright}{\isachardoublequoteclose}\isanewline
%
\isadelimproof
%
\endisadelimproof
%
\isatagproof
\isacommand{proof}\isamarkupfalse%
\ {\isacharminus}\isanewline
\ \ \isacommand{have}\isamarkupfalse%
\ {\isachardoublequoteopen}G\ {\isasymin}\ {\isacharbraceleft}Atom\ x{\isacharbraceright}{\isachardoublequoteclose}\isanewline
\ \ \ \ \isacommand{using}\isamarkupfalse%
\ assms\isanewline
\ \ \ \ \isacommand{by}\isamarkupfalse%
\ {\isacharparenleft}simp\ only{\isacharcolon}\ setSubformulae{\isacharunderscore}atom{\isacharparenright}\isanewline
\ \ \isacommand{then}\isamarkupfalse%
\ \isacommand{have}\isamarkupfalse%
\ {\isachardoublequoteopen}G\ {\isacharequal}\ Atom\ x{\isachardoublequoteclose}\isanewline
\ \ \ \ \isacommand{by}\isamarkupfalse%
\ {\isacharparenleft}simp\ only{\isacharcolon}\ singletonD{\isacharparenright}\isanewline
\ \ \isacommand{then}\isamarkupfalse%
\ \isacommand{show}\isamarkupfalse%
\ {\isacharquery}thesis\isanewline
\ \ \ \ \isacommand{by}\isamarkupfalse%
\ {\isacharparenleft}simp\ only{\isacharcolon}\ subset{\isacharunderscore}refl{\isacharparenright}\isanewline
\isacommand{qed}\isamarkupfalse%
%
\endisatagproof
{\isafoldproof}%
%
\isadelimproof
\isanewline
%
\endisadelimproof
\isanewline
\isacommand{lemma}\isamarkupfalse%
\ subformulas{\isacharunderscore}atoms{\isacharunderscore}bot{\isacharcolon}\isanewline
\ \ \isakeyword{assumes}\ {\isachardoublequoteopen}G\ {\isasymin}\ setSubformulae\ {\isasymbottom}{\isachardoublequoteclose}\ \isanewline
\ \ \isakeyword{shows}\ \ \ {\isachardoublequoteopen}atoms\ G\ {\isasymsubseteq}\ atoms\ {\isasymbottom}{\isachardoublequoteclose}\isanewline
%
\isadelimproof
%
\endisadelimproof
%
\isatagproof
\isacommand{proof}\isamarkupfalse%
\ {\isacharminus}\isanewline
\ \ \isacommand{have}\isamarkupfalse%
\ {\isachardoublequoteopen}G\ {\isasymin}\ {\isacharbraceleft}{\isasymbottom}{\isacharbraceright}{\isachardoublequoteclose}\isanewline
\ \ \ \ \isacommand{using}\isamarkupfalse%
\ assms\isanewline
\ \ \ \ \isacommand{by}\isamarkupfalse%
\ {\isacharparenleft}simp\ only{\isacharcolon}\ setSubformulae{\isacharunderscore}bot{\isacharparenright}\isanewline
\ \ \isacommand{then}\isamarkupfalse%
\ \isacommand{have}\isamarkupfalse%
\ {\isachardoublequoteopen}G\ {\isacharequal}\ {\isasymbottom}{\isachardoublequoteclose}\isanewline
\ \ \ \ \isacommand{by}\isamarkupfalse%
\ {\isacharparenleft}simp\ only{\isacharcolon}\ singletonD{\isacharparenright}\isanewline
\ \ \isacommand{then}\isamarkupfalse%
\ \isacommand{show}\isamarkupfalse%
\ {\isacharquery}thesis\isanewline
\ \ \ \ \isacommand{by}\isamarkupfalse%
\ {\isacharparenleft}simp\ only{\isacharcolon}\ subset{\isacharunderscore}refl{\isacharparenright}\isanewline
\isacommand{qed}\isamarkupfalse%
%
\endisatagproof
{\isafoldproof}%
%
\isadelimproof
\isanewline
%
\endisadelimproof
\isanewline
\isacommand{lemma}\isamarkupfalse%
\ subformulas{\isacharunderscore}atoms{\isacharunderscore}not{\isacharcolon}\isanewline
\ \ \isakeyword{assumes}\ {\isachardoublequoteopen}G\ {\isasymin}\ setSubformulae\ F\ {\isasymLongrightarrow}\ atoms\ G\ {\isasymsubseteq}\ atoms\ F{\isachardoublequoteclose}\isanewline
\ \ \ \ \ \ \ \ \ \ {\isachardoublequoteopen}G\ {\isasymin}\ setSubformulae\ {\isacharparenleft}\isactrlbold {\isasymnot}\ F{\isacharparenright}{\isachardoublequoteclose}\isanewline
\ \ \isakeyword{shows}\ \ \ {\isachardoublequoteopen}atoms\ G\ {\isasymsubseteq}\ atoms\ {\isacharparenleft}\isactrlbold {\isasymnot}\ F{\isacharparenright}{\isachardoublequoteclose}\isanewline
%
\isadelimproof
%
\endisadelimproof
%
\isatagproof
\isacommand{proof}\isamarkupfalse%
\ {\isacharminus}\isanewline
\ \ \isacommand{have}\isamarkupfalse%
\ {\isachardoublequoteopen}G\ {\isasymin}\ {\isacharbraceleft}\isactrlbold {\isasymnot}\ F{\isacharbraceright}\ {\isasymunion}\ setSubformulae\ F{\isachardoublequoteclose}\isanewline
\ \ \ \ \isacommand{using}\isamarkupfalse%
\ assms{\isacharparenleft}{\isadigit{2}}{\isacharparenright}\isanewline
\ \ \ \ \isacommand{by}\isamarkupfalse%
\ {\isacharparenleft}simp\ only{\isacharcolon}\ setSubformulae{\isacharunderscore}not{\isacharparenright}\ \isanewline
\ \ \isacommand{then}\isamarkupfalse%
\ \isacommand{have}\isamarkupfalse%
\ {\isachardoublequoteopen}G\ {\isasymin}\ {\isacharbraceleft}\isactrlbold {\isasymnot}\ F{\isacharbraceright}\ {\isasymor}\ G\ {\isasymin}\ setSubformulae\ F{\isachardoublequoteclose}\isanewline
\ \ \ \ \isacommand{by}\isamarkupfalse%
\ {\isacharparenleft}simp\ only{\isacharcolon}\ Un{\isacharunderscore}iff{\isacharparenright}\isanewline
\ \ \isacommand{then}\isamarkupfalse%
\ \isacommand{show}\isamarkupfalse%
\ {\isachardoublequoteopen}atoms\ G\ {\isasymsubseteq}\ atoms\ {\isacharparenleft}\isactrlbold {\isasymnot}\ F{\isacharparenright}{\isachardoublequoteclose}\isanewline
\ \ \isacommand{proof}\isamarkupfalse%
\ {\isacharparenleft}rule\ disjE{\isacharparenright}\isanewline
\ \ \ \ \isacommand{assume}\isamarkupfalse%
\ {\isachardoublequoteopen}G\ {\isasymin}\ {\isacharbraceleft}\isactrlbold {\isasymnot}\ F{\isacharbraceright}{\isachardoublequoteclose}\isanewline
\ \ \ \ \isacommand{then}\isamarkupfalse%
\ \isacommand{have}\isamarkupfalse%
\ {\isachardoublequoteopen}G\ {\isacharequal}\ \isactrlbold {\isasymnot}\ F{\isachardoublequoteclose}\isanewline
\ \ \ \ \ \ \isacommand{by}\isamarkupfalse%
\ {\isacharparenleft}simp\ only{\isacharcolon}\ singletonD{\isacharparenright}\isanewline
\ \ \ \ \isacommand{then}\isamarkupfalse%
\ \isacommand{show}\isamarkupfalse%
\ {\isacharquery}thesis\isanewline
\ \ \ \ \ \ \isacommand{by}\isamarkupfalse%
\ {\isacharparenleft}simp\ only{\isacharcolon}\ subset{\isacharunderscore}refl{\isacharparenright}\isanewline
\ \ \isacommand{next}\isamarkupfalse%
\isanewline
\ \ \ \ \isacommand{assume}\isamarkupfalse%
\ {\isachardoublequoteopen}G\ {\isasymin}\ setSubformulae\ F{\isachardoublequoteclose}\isanewline
\ \ \ \ \isacommand{then}\isamarkupfalse%
\ \isacommand{have}\isamarkupfalse%
\ {\isachardoublequoteopen}atoms\ G\ {\isasymsubseteq}\ atoms\ F{\isachardoublequoteclose}\isanewline
\ \ \ \ \ \ \isacommand{by}\isamarkupfalse%
\ {\isacharparenleft}simp\ only{\isacharcolon}\ assms{\isacharparenleft}{\isadigit{1}}{\isacharparenright}{\isacharparenright}\isanewline
\ \ \ \ \isacommand{also}\isamarkupfalse%
\ \isacommand{have}\isamarkupfalse%
\ {\isachardoublequoteopen}{\isasymdots}\ {\isacharequal}\ atoms\ {\isacharparenleft}\isactrlbold {\isasymnot}\ F{\isacharparenright}{\isachardoublequoteclose}\isanewline
\ \ \ \ \ \ \isacommand{by}\isamarkupfalse%
\ {\isacharparenleft}simp\ only{\isacharcolon}\ formula{\isachardot}set{\isacharparenleft}{\isadigit{3}}{\isacharparenright}{\isacharparenright}\isanewline
\ \ \ \ \isacommand{finally}\isamarkupfalse%
\ \isacommand{show}\isamarkupfalse%
\ {\isacharquery}thesis\isanewline
\ \ \ \ \ \ \isacommand{by}\isamarkupfalse%
\ this\isanewline
\ \ \isacommand{qed}\isamarkupfalse%
\isanewline
\isacommand{qed}\isamarkupfalse%
%
\endisatagproof
{\isafoldproof}%
%
\isadelimproof
\isanewline
%
\endisadelimproof
\isanewline
\isacommand{lemma}\isamarkupfalse%
\ subformulas{\isacharunderscore}atoms{\isacharunderscore}and{\isacharcolon}\isanewline
\ \ \isakeyword{assumes}\ {\isachardoublequoteopen}G\ {\isasymin}\ setSubformulae\ F{\isadigit{1}}\ {\isasymLongrightarrow}\ atoms\ G\ {\isasymsubseteq}\ atoms\ F{\isadigit{1}}{\isachardoublequoteclose}\isanewline
\ \ \ \ \ \ \ \ \ \ {\isachardoublequoteopen}G\ {\isasymin}\ setSubformulae\ F{\isadigit{2}}\ {\isasymLongrightarrow}\ atoms\ G\ {\isasymsubseteq}\ atoms\ F{\isadigit{2}}{\isachardoublequoteclose}\isanewline
\ \ \ \ \ \ \ \ \ \ {\isachardoublequoteopen}G\ {\isasymin}\ setSubformulae\ {\isacharparenleft}F{\isadigit{1}}\ \isactrlbold {\isasymand}\ F{\isadigit{2}}{\isacharparenright}{\isachardoublequoteclose}\isanewline
\ \ \isakeyword{shows}\ \ \ {\isachardoublequoteopen}atoms\ G\ {\isasymsubseteq}\ atoms\ {\isacharparenleft}F{\isadigit{1}}\ \isactrlbold {\isasymand}\ F{\isadigit{2}}{\isacharparenright}{\isachardoublequoteclose}\isanewline
%
\isadelimproof
%
\endisadelimproof
%
\isatagproof
\isacommand{proof}\isamarkupfalse%
\ {\isacharminus}\isanewline
\ \ \isacommand{have}\isamarkupfalse%
\ {\isachardoublequoteopen}G\ {\isasymin}\ {\isacharbraceleft}F{\isadigit{1}}\ \isactrlbold {\isasymand}\ F{\isadigit{2}}{\isacharbraceright}\ {\isasymunion}\ {\isacharparenleft}setSubformulae\ F{\isadigit{1}}\ {\isasymunion}\ setSubformulae\ F{\isadigit{2}}{\isacharparenright}{\isachardoublequoteclose}\isanewline
\ \ \ \ \isacommand{using}\isamarkupfalse%
\ assms{\isacharparenleft}{\isadigit{3}}{\isacharparenright}\ \isanewline
\ \ \ \ \isacommand{by}\isamarkupfalse%
\ {\isacharparenleft}simp\ only{\isacharcolon}\ setSubformulae{\isacharunderscore}and{\isacharparenright}\isanewline
\ \ \isacommand{then}\isamarkupfalse%
\ \isacommand{have}\isamarkupfalse%
\ {\isachardoublequoteopen}G\ {\isasymin}\ {\isacharbraceleft}F{\isadigit{1}}\ \isactrlbold {\isasymand}\ F{\isadigit{2}}{\isacharbraceright}\ {\isasymor}\ G\ {\isasymin}\ setSubformulae\ F{\isadigit{1}}\ {\isasymunion}\ setSubformulae\ F{\isadigit{2}}{\isachardoublequoteclose}\isanewline
\ \ \ \ \isacommand{by}\isamarkupfalse%
\ {\isacharparenleft}simp\ only{\isacharcolon}\ Un{\isacharunderscore}iff{\isacharparenright}\isanewline
\ \ \isacommand{then}\isamarkupfalse%
\ \isacommand{show}\isamarkupfalse%
\ {\isacharquery}thesis\isanewline
\ \ \isacommand{proof}\isamarkupfalse%
\ {\isacharparenleft}rule\ disjE{\isacharparenright}\isanewline
\ \ \ \ \isacommand{assume}\isamarkupfalse%
\ {\isachardoublequoteopen}G\ {\isasymin}\ {\isacharbraceleft}F{\isadigit{1}}\ \isactrlbold {\isasymand}\ F{\isadigit{2}}{\isacharbraceright}{\isachardoublequoteclose}\isanewline
\ \ \ \ \isacommand{then}\isamarkupfalse%
\ \isacommand{have}\isamarkupfalse%
\ {\isachardoublequoteopen}G\ {\isacharequal}\ F{\isadigit{1}}\ \isactrlbold {\isasymand}\ F{\isadigit{2}}{\isachardoublequoteclose}\isanewline
\ \ \ \ \ \ \isacommand{by}\isamarkupfalse%
\ {\isacharparenleft}simp\ only{\isacharcolon}\ singletonD{\isacharparenright}\isanewline
\ \ \ \ \isacommand{then}\isamarkupfalse%
\ \isacommand{show}\isamarkupfalse%
\ {\isacharquery}thesis\isanewline
\ \ \ \ \ \ \isacommand{by}\isamarkupfalse%
\ {\isacharparenleft}simp\ only{\isacharcolon}\ subset{\isacharunderscore}refl{\isacharparenright}\isanewline
\ \ \isacommand{next}\isamarkupfalse%
\isanewline
\ \ \ \ \isacommand{assume}\isamarkupfalse%
\ {\isachardoublequoteopen}G\ {\isasymin}\ setSubformulae\ F{\isadigit{1}}\ {\isasymunion}\ setSubformulae\ F{\isadigit{2}}{\isachardoublequoteclose}\isanewline
\ \ \ \ \isacommand{then}\isamarkupfalse%
\ \isacommand{have}\isamarkupfalse%
\ {\isachardoublequoteopen}G\ {\isasymin}\ setSubformulae\ F{\isadigit{1}}\ {\isasymor}\ G\ {\isasymin}\ setSubformulae\ F{\isadigit{2}}{\isachardoublequoteclose}\ \ \isanewline
\ \ \ \ \ \ \isacommand{by}\isamarkupfalse%
\ {\isacharparenleft}simp\ only{\isacharcolon}\ Un{\isacharunderscore}iff{\isacharparenright}\isanewline
\ \ \ \ \isacommand{then}\isamarkupfalse%
\ \isacommand{show}\isamarkupfalse%
\ {\isacharquery}thesis\isanewline
\ \ \ \ \isacommand{proof}\isamarkupfalse%
\ {\isacharparenleft}rule\ disjE{\isacharparenright}\isanewline
\ \ \ \ \ \ \isacommand{assume}\isamarkupfalse%
\ {\isachardoublequoteopen}G\ {\isasymin}\ setSubformulae\ F{\isadigit{1}}{\isachardoublequoteclose}\isanewline
\ \ \ \ \ \ \isacommand{then}\isamarkupfalse%
\ \isacommand{have}\isamarkupfalse%
\ {\isachardoublequoteopen}atoms\ G\ {\isasymsubseteq}\ atoms\ F{\isadigit{1}}{\isachardoublequoteclose}\isanewline
\ \ \ \ \ \ \ \ \isacommand{by}\isamarkupfalse%
\ {\isacharparenleft}rule\ assms{\isacharparenleft}{\isadigit{1}}{\isacharparenright}{\isacharparenright}\isanewline
\ \ \ \ \ \ \isacommand{also}\isamarkupfalse%
\ \isacommand{have}\isamarkupfalse%
\ {\isachardoublequoteopen}{\isasymdots}\ {\isasymsubseteq}\ atoms\ F{\isadigit{1}}\ {\isasymunion}\ atoms\ F{\isadigit{2}}{\isachardoublequoteclose}\isanewline
\ \ \ \ \ \ \ \ \isacommand{by}\isamarkupfalse%
\ {\isacharparenleft}simp\ only{\isacharcolon}\ Un{\isacharunderscore}upper{\isadigit{1}}{\isacharparenright}\isanewline
\ \ \ \ \ \ \isacommand{also}\isamarkupfalse%
\ \isacommand{have}\isamarkupfalse%
\ {\isachardoublequoteopen}{\isasymdots}\ {\isacharequal}\ atoms\ {\isacharparenleft}F{\isadigit{1}}\ \isactrlbold {\isasymand}\ F{\isadigit{2}}{\isacharparenright}{\isachardoublequoteclose}\isanewline
\ \ \ \ \ \ \ \ \isacommand{by}\isamarkupfalse%
\ {\isacharparenleft}simp\ only{\isacharcolon}\ formula{\isachardot}set{\isacharparenleft}{\isadigit{4}}{\isacharparenright}{\isacharparenright}\isanewline
\ \ \ \ \ \ \isacommand{finally}\isamarkupfalse%
\ \isacommand{show}\isamarkupfalse%
\ {\isacharquery}thesis\isanewline
\ \ \ \ \ \ \ \ \isacommand{by}\isamarkupfalse%
\ this\isanewline
\ \ \ \ \isacommand{next}\isamarkupfalse%
\isanewline
\ \ \ \ \ \ \isacommand{assume}\isamarkupfalse%
\ {\isachardoublequoteopen}G\ {\isasymin}\ setSubformulae\ F{\isadigit{2}}{\isachardoublequoteclose}\isanewline
\ \ \ \ \ \ \isacommand{then}\isamarkupfalse%
\ \isacommand{have}\isamarkupfalse%
\ {\isachardoublequoteopen}atoms\ G\ {\isasymsubseteq}\ atoms\ F{\isadigit{2}}{\isachardoublequoteclose}\isanewline
\ \ \ \ \ \ \ \ \isacommand{by}\isamarkupfalse%
\ {\isacharparenleft}rule\ assms{\isacharparenleft}{\isadigit{2}}{\isacharparenright}{\isacharparenright}\isanewline
\ \ \ \ \ \ \isacommand{also}\isamarkupfalse%
\ \isacommand{have}\isamarkupfalse%
\ {\isachardoublequoteopen}{\isasymdots}\ {\isasymsubseteq}\ atoms\ F{\isadigit{1}}\ {\isasymunion}\ atoms\ F{\isadigit{2}}{\isachardoublequoteclose}\isanewline
\ \ \ \ \ \ \ \ \isacommand{by}\isamarkupfalse%
\ {\isacharparenleft}simp\ only{\isacharcolon}\ Un{\isacharunderscore}upper{\isadigit{2}}{\isacharparenright}\isanewline
\ \ \ \ \ \ \isacommand{also}\isamarkupfalse%
\ \isacommand{have}\isamarkupfalse%
\ {\isachardoublequoteopen}{\isasymdots}\ {\isacharequal}\ atoms\ {\isacharparenleft}F{\isadigit{1}}\ \isactrlbold {\isasymand}\ F{\isadigit{2}}{\isacharparenright}{\isachardoublequoteclose}\isanewline
\ \ \ \ \ \ \ \ \isacommand{by}\isamarkupfalse%
\ {\isacharparenleft}simp\ only{\isacharcolon}\ formula{\isachardot}set{\isacharparenleft}{\isadigit{4}}{\isacharparenright}{\isacharparenright}\isanewline
\ \ \ \ \ \ \isacommand{finally}\isamarkupfalse%
\ \isacommand{show}\isamarkupfalse%
\ {\isacharquery}thesis\isanewline
\ \ \ \ \ \ \ \ \isacommand{by}\isamarkupfalse%
\ this\isanewline
\ \ \ \ \isacommand{qed}\isamarkupfalse%
\isanewline
\ \ \isacommand{qed}\isamarkupfalse%
\isanewline
\isacommand{qed}\isamarkupfalse%
%
\endisatagproof
{\isafoldproof}%
%
\isadelimproof
\isanewline
%
\endisadelimproof
\isanewline
\isacommand{lemma}\isamarkupfalse%
\ subformulas{\isacharunderscore}atoms{\isacharunderscore}or{\isacharcolon}\isanewline
\ \ \isakeyword{assumes}\ {\isachardoublequoteopen}G\ {\isasymin}\ setSubformulae\ F{\isadigit{1}}\ {\isasymLongrightarrow}\ atoms\ G\ {\isasymsubseteq}\ atoms\ F{\isadigit{1}}{\isachardoublequoteclose}\isanewline
\ \ \ \ \ \ \ \ \ \ {\isachardoublequoteopen}G\ {\isasymin}\ setSubformulae\ F{\isadigit{2}}\ {\isasymLongrightarrow}\ atoms\ G\ {\isasymsubseteq}\ atoms\ F{\isadigit{2}}{\isachardoublequoteclose}\isanewline
\ \ \ \ \ \ \ \ \ \ {\isachardoublequoteopen}G\ {\isasymin}\ setSubformulae\ {\isacharparenleft}F{\isadigit{1}}\ \isactrlbold {\isasymor}\ F{\isadigit{2}}{\isacharparenright}{\isachardoublequoteclose}\isanewline
\ \ \isakeyword{shows}\ \ \ {\isachardoublequoteopen}atoms\ G\ {\isasymsubseteq}\ atoms\ {\isacharparenleft}F{\isadigit{1}}\ \isactrlbold {\isasymor}\ F{\isadigit{2}}{\isacharparenright}{\isachardoublequoteclose}\isanewline
%
\isadelimproof
%
\endisadelimproof
%
\isatagproof
\isacommand{proof}\isamarkupfalse%
\ {\isacharminus}\isanewline
\ \ \isacommand{have}\isamarkupfalse%
\ {\isachardoublequoteopen}G\ {\isasymin}\ {\isacharbraceleft}F{\isadigit{1}}\ \isactrlbold {\isasymor}\ F{\isadigit{2}}{\isacharbraceright}\ {\isasymunion}\ {\isacharparenleft}setSubformulae\ F{\isadigit{1}}\ {\isasymunion}\ setSubformulae\ F{\isadigit{2}}{\isacharparenright}{\isachardoublequoteclose}\isanewline
\ \ \ \ \isacommand{using}\isamarkupfalse%
\ assms{\isacharparenleft}{\isadigit{3}}{\isacharparenright}\ \isanewline
\ \ \ \ \isacommand{by}\isamarkupfalse%
\ {\isacharparenleft}simp\ only{\isacharcolon}\ setSubformulae{\isacharunderscore}or{\isacharparenright}\isanewline
\ \ \isacommand{then}\isamarkupfalse%
\ \isacommand{have}\isamarkupfalse%
\ {\isachardoublequoteopen}G\ {\isasymin}\ {\isacharbraceleft}F{\isadigit{1}}\ \isactrlbold {\isasymor}\ F{\isadigit{2}}{\isacharbraceright}\ {\isasymor}\ G\ {\isasymin}\ setSubformulae\ F{\isadigit{1}}\ {\isasymunion}\ setSubformulae\ F{\isadigit{2}}{\isachardoublequoteclose}\isanewline
\ \ \ \ \isacommand{by}\isamarkupfalse%
\ {\isacharparenleft}simp\ only{\isacharcolon}\ Un{\isacharunderscore}iff{\isacharparenright}\isanewline
\ \ \isacommand{then}\isamarkupfalse%
\ \isacommand{show}\isamarkupfalse%
\ {\isacharquery}thesis\isanewline
\ \ \isacommand{proof}\isamarkupfalse%
\ {\isacharparenleft}rule\ disjE{\isacharparenright}\isanewline
\ \ \ \ \isacommand{assume}\isamarkupfalse%
\ {\isachardoublequoteopen}G\ {\isasymin}\ {\isacharbraceleft}F{\isadigit{1}}\ \isactrlbold {\isasymor}\ F{\isadigit{2}}{\isacharbraceright}{\isachardoublequoteclose}\isanewline
\ \ \ \ \isacommand{then}\isamarkupfalse%
\ \isacommand{have}\isamarkupfalse%
\ {\isachardoublequoteopen}G\ {\isacharequal}\ F{\isadigit{1}}\ \isactrlbold {\isasymor}\ F{\isadigit{2}}{\isachardoublequoteclose}\isanewline
\ \ \ \ \ \ \isacommand{by}\isamarkupfalse%
\ {\isacharparenleft}simp\ only{\isacharcolon}\ singletonD{\isacharparenright}\isanewline
\ \ \ \ \isacommand{then}\isamarkupfalse%
\ \isacommand{show}\isamarkupfalse%
\ {\isacharquery}thesis\isanewline
\ \ \ \ \ \ \isacommand{by}\isamarkupfalse%
\ {\isacharparenleft}simp\ only{\isacharcolon}\ subset{\isacharunderscore}refl{\isacharparenright}\isanewline
\ \ \isacommand{next}\isamarkupfalse%
\isanewline
\ \ \ \ \isacommand{assume}\isamarkupfalse%
\ {\isachardoublequoteopen}G\ {\isasymin}\ setSubformulae\ F{\isadigit{1}}\ {\isasymunion}\ setSubformulae\ F{\isadigit{2}}{\isachardoublequoteclose}\isanewline
\ \ \ \ \isacommand{then}\isamarkupfalse%
\ \isacommand{have}\isamarkupfalse%
\ {\isachardoublequoteopen}G\ {\isasymin}\ setSubformulae\ F{\isadigit{1}}\ {\isasymor}\ G\ {\isasymin}\ setSubformulae\ F{\isadigit{2}}{\isachardoublequoteclose}\ \ \isanewline
\ \ \ \ \ \ \isacommand{by}\isamarkupfalse%
\ {\isacharparenleft}simp\ only{\isacharcolon}\ Un{\isacharunderscore}iff{\isacharparenright}\isanewline
\ \ \ \ \isacommand{then}\isamarkupfalse%
\ \isacommand{show}\isamarkupfalse%
\ {\isacharquery}thesis\isanewline
\ \ \ \ \isacommand{proof}\isamarkupfalse%
\ {\isacharparenleft}rule\ disjE{\isacharparenright}\isanewline
\ \ \ \ \ \ \isacommand{assume}\isamarkupfalse%
\ {\isachardoublequoteopen}G\ {\isasymin}\ setSubformulae\ F{\isadigit{1}}{\isachardoublequoteclose}\isanewline
\ \ \ \ \ \ \isacommand{then}\isamarkupfalse%
\ \isacommand{have}\isamarkupfalse%
\ {\isachardoublequoteopen}atoms\ G\ {\isasymsubseteq}\ atoms\ F{\isadigit{1}}{\isachardoublequoteclose}\isanewline
\ \ \ \ \ \ \ \ \isacommand{by}\isamarkupfalse%
\ {\isacharparenleft}rule\ assms{\isacharparenleft}{\isadigit{1}}{\isacharparenright}{\isacharparenright}\isanewline
\ \ \ \ \ \ \isacommand{also}\isamarkupfalse%
\ \isacommand{have}\isamarkupfalse%
\ {\isachardoublequoteopen}{\isasymdots}\ {\isasymsubseteq}\ atoms\ F{\isadigit{1}}\ {\isasymunion}\ atoms\ F{\isadigit{2}}{\isachardoublequoteclose}\isanewline
\ \ \ \ \ \ \ \ \isacommand{by}\isamarkupfalse%
\ {\isacharparenleft}simp\ only{\isacharcolon}\ Un{\isacharunderscore}upper{\isadigit{1}}{\isacharparenright}\isanewline
\ \ \ \ \ \ \isacommand{also}\isamarkupfalse%
\ \isacommand{have}\isamarkupfalse%
\ {\isachardoublequoteopen}{\isasymdots}\ {\isacharequal}\ atoms\ {\isacharparenleft}F{\isadigit{1}}\ \isactrlbold {\isasymor}\ F{\isadigit{2}}{\isacharparenright}{\isachardoublequoteclose}\isanewline
\ \ \ \ \ \ \ \ \isacommand{by}\isamarkupfalse%
\ {\isacharparenleft}simp\ only{\isacharcolon}\ formula{\isachardot}set{\isacharparenleft}{\isadigit{5}}{\isacharparenright}{\isacharparenright}\isanewline
\ \ \ \ \ \ \isacommand{finally}\isamarkupfalse%
\ \isacommand{show}\isamarkupfalse%
\ {\isacharquery}thesis\isanewline
\ \ \ \ \ \ \ \ \isacommand{by}\isamarkupfalse%
\ this\isanewline
\ \ \ \ \isacommand{next}\isamarkupfalse%
\isanewline
\ \ \ \ \ \ \isacommand{assume}\isamarkupfalse%
\ {\isachardoublequoteopen}G\ {\isasymin}\ setSubformulae\ F{\isadigit{2}}{\isachardoublequoteclose}\isanewline
\ \ \ \ \ \ \isacommand{then}\isamarkupfalse%
\ \isacommand{have}\isamarkupfalse%
\ {\isachardoublequoteopen}atoms\ G\ {\isasymsubseteq}\ atoms\ F{\isadigit{2}}{\isachardoublequoteclose}\isanewline
\ \ \ \ \ \ \ \ \isacommand{by}\isamarkupfalse%
\ {\isacharparenleft}rule\ assms{\isacharparenleft}{\isadigit{2}}{\isacharparenright}{\isacharparenright}\isanewline
\ \ \ \ \ \ \isacommand{also}\isamarkupfalse%
\ \isacommand{have}\isamarkupfalse%
\ {\isachardoublequoteopen}{\isasymdots}\ {\isasymsubseteq}\ atoms\ F{\isadigit{1}}\ {\isasymunion}\ atoms\ F{\isadigit{2}}{\isachardoublequoteclose}\isanewline
\ \ \ \ \ \ \ \ \isacommand{by}\isamarkupfalse%
\ {\isacharparenleft}simp\ only{\isacharcolon}\ Un{\isacharunderscore}upper{\isadigit{2}}{\isacharparenright}\isanewline
\ \ \ \ \ \ \isacommand{also}\isamarkupfalse%
\ \isacommand{have}\isamarkupfalse%
\ {\isachardoublequoteopen}{\isasymdots}\ {\isacharequal}\ atoms\ {\isacharparenleft}F{\isadigit{1}}\ \isactrlbold {\isasymor}\ F{\isadigit{2}}{\isacharparenright}{\isachardoublequoteclose}\isanewline
\ \ \ \ \ \ \ \ \isacommand{by}\isamarkupfalse%
\ {\isacharparenleft}simp\ only{\isacharcolon}\ formula{\isachardot}set{\isacharparenleft}{\isadigit{5}}{\isacharparenright}{\isacharparenright}\isanewline
\ \ \ \ \ \ \isacommand{finally}\isamarkupfalse%
\ \isacommand{show}\isamarkupfalse%
\ {\isacharquery}thesis\isanewline
\ \ \ \ \ \ \ \ \isacommand{by}\isamarkupfalse%
\ this\isanewline
\ \ \ \ \isacommand{qed}\isamarkupfalse%
\isanewline
\ \ \isacommand{qed}\isamarkupfalse%
\isanewline
\isacommand{qed}\isamarkupfalse%
%
\endisatagproof
{\isafoldproof}%
%
\isadelimproof
\isanewline
%
\endisadelimproof
\isanewline
\isacommand{lemma}\isamarkupfalse%
\ subformulas{\isacharunderscore}atoms{\isacharunderscore}imp{\isacharcolon}\isanewline
\ \ \isakeyword{assumes}\ {\isachardoublequoteopen}G\ {\isasymin}\ setSubformulae\ F{\isadigit{1}}\ {\isasymLongrightarrow}\ atoms\ G\ {\isasymsubseteq}\ atoms\ F{\isadigit{1}}{\isachardoublequoteclose}\isanewline
\ \ \ \ \ \ \ \ \ \ {\isachardoublequoteopen}G\ {\isasymin}\ setSubformulae\ F{\isadigit{2}}\ {\isasymLongrightarrow}\ atoms\ G\ {\isasymsubseteq}\ atoms\ F{\isadigit{2}}{\isachardoublequoteclose}\isanewline
\ \ \ \ \ \ \ \ \ \ {\isachardoublequoteopen}G\ {\isasymin}\ setSubformulae\ {\isacharparenleft}F{\isadigit{1}}\ \isactrlbold {\isasymrightarrow}\ F{\isadigit{2}}{\isacharparenright}{\isachardoublequoteclose}\isanewline
\ \ \isakeyword{shows}\ \ \ {\isachardoublequoteopen}atoms\ G\ {\isasymsubseteq}\ atoms\ {\isacharparenleft}F{\isadigit{1}}\ \isactrlbold {\isasymrightarrow}\ F{\isadigit{2}}{\isacharparenright}{\isachardoublequoteclose}\isanewline
%
\isadelimproof
%
\endisadelimproof
%
\isatagproof
\isacommand{proof}\isamarkupfalse%
\ {\isacharminus}\isanewline
\ \ \isacommand{have}\isamarkupfalse%
\ {\isachardoublequoteopen}G\ {\isasymin}\ {\isacharbraceleft}F{\isadigit{1}}\ \isactrlbold {\isasymrightarrow}\ F{\isadigit{2}}{\isacharbraceright}\ {\isasymunion}\ {\isacharparenleft}setSubformulae\ F{\isadigit{1}}\ {\isasymunion}\ setSubformulae\ F{\isadigit{2}}{\isacharparenright}{\isachardoublequoteclose}\isanewline
\ \ \ \ \isacommand{using}\isamarkupfalse%
\ assms{\isacharparenleft}{\isadigit{3}}{\isacharparenright}\ \isanewline
\ \ \ \ \isacommand{by}\isamarkupfalse%
\ {\isacharparenleft}simp\ only{\isacharcolon}\ setSubformulae{\isacharunderscore}imp{\isacharparenright}\isanewline
\ \ \isacommand{then}\isamarkupfalse%
\ \isacommand{have}\isamarkupfalse%
\ {\isachardoublequoteopen}G\ {\isasymin}\ {\isacharbraceleft}F{\isadigit{1}}\ \isactrlbold {\isasymrightarrow}\ F{\isadigit{2}}{\isacharbraceright}\ {\isasymor}\ G\ {\isasymin}\ setSubformulae\ F{\isadigit{1}}\ {\isasymunion}\ setSubformulae\ F{\isadigit{2}}{\isachardoublequoteclose}\isanewline
\ \ \ \ \isacommand{by}\isamarkupfalse%
\ {\isacharparenleft}simp\ only{\isacharcolon}\ Un{\isacharunderscore}iff{\isacharparenright}\isanewline
\ \ \isacommand{then}\isamarkupfalse%
\ \isacommand{show}\isamarkupfalse%
\ {\isacharquery}thesis\isanewline
\ \ \isacommand{proof}\isamarkupfalse%
\ {\isacharparenleft}rule\ disjE{\isacharparenright}\isanewline
\ \ \ \ \isacommand{assume}\isamarkupfalse%
\ {\isachardoublequoteopen}G\ {\isasymin}\ {\isacharbraceleft}F{\isadigit{1}}\ \isactrlbold {\isasymrightarrow}\ F{\isadigit{2}}{\isacharbraceright}{\isachardoublequoteclose}\isanewline
\ \ \ \ \isacommand{then}\isamarkupfalse%
\ \isacommand{have}\isamarkupfalse%
\ {\isachardoublequoteopen}G\ {\isacharequal}\ F{\isadigit{1}}\ \isactrlbold {\isasymrightarrow}\ F{\isadigit{2}}{\isachardoublequoteclose}\isanewline
\ \ \ \ \ \ \isacommand{by}\isamarkupfalse%
\ {\isacharparenleft}simp\ only{\isacharcolon}\ singletonD{\isacharparenright}\isanewline
\ \ \ \ \isacommand{then}\isamarkupfalse%
\ \isacommand{show}\isamarkupfalse%
\ {\isacharquery}thesis\isanewline
\ \ \ \ \ \ \isacommand{by}\isamarkupfalse%
\ {\isacharparenleft}simp\ only{\isacharcolon}\ subset{\isacharunderscore}refl{\isacharparenright}\isanewline
\ \ \isacommand{next}\isamarkupfalse%
\isanewline
\ \ \ \ \isacommand{assume}\isamarkupfalse%
\ {\isachardoublequoteopen}G\ {\isasymin}\ setSubformulae\ F{\isadigit{1}}\ {\isasymunion}\ setSubformulae\ F{\isadigit{2}}{\isachardoublequoteclose}\isanewline
\ \ \ \ \isacommand{then}\isamarkupfalse%
\ \isacommand{have}\isamarkupfalse%
\ {\isachardoublequoteopen}G\ {\isasymin}\ setSubformulae\ F{\isadigit{1}}\ {\isasymor}\ G\ {\isasymin}\ setSubformulae\ F{\isadigit{2}}{\isachardoublequoteclose}\ \ \isanewline
\ \ \ \ \ \ \isacommand{by}\isamarkupfalse%
\ {\isacharparenleft}simp\ only{\isacharcolon}\ Un{\isacharunderscore}iff{\isacharparenright}\isanewline
\ \ \ \ \isacommand{then}\isamarkupfalse%
\ \isacommand{show}\isamarkupfalse%
\ {\isacharquery}thesis\isanewline
\ \ \ \ \isacommand{proof}\isamarkupfalse%
\ {\isacharparenleft}rule\ disjE{\isacharparenright}\isanewline
\ \ \ \ \ \ \isacommand{assume}\isamarkupfalse%
\ {\isachardoublequoteopen}G\ {\isasymin}\ setSubformulae\ F{\isadigit{1}}{\isachardoublequoteclose}\isanewline
\ \ \ \ \ \ \isacommand{then}\isamarkupfalse%
\ \isacommand{have}\isamarkupfalse%
\ {\isachardoublequoteopen}atoms\ G\ {\isasymsubseteq}\ atoms\ F{\isadigit{1}}{\isachardoublequoteclose}\isanewline
\ \ \ \ \ \ \ \ \isacommand{by}\isamarkupfalse%
\ {\isacharparenleft}rule\ assms{\isacharparenleft}{\isadigit{1}}{\isacharparenright}{\isacharparenright}\isanewline
\ \ \ \ \ \ \isacommand{also}\isamarkupfalse%
\ \isacommand{have}\isamarkupfalse%
\ {\isachardoublequoteopen}{\isasymdots}\ {\isasymsubseteq}\ atoms\ F{\isadigit{1}}\ {\isasymunion}\ atoms\ F{\isadigit{2}}{\isachardoublequoteclose}\isanewline
\ \ \ \ \ \ \ \ \isacommand{by}\isamarkupfalse%
\ {\isacharparenleft}simp\ only{\isacharcolon}\ Un{\isacharunderscore}upper{\isadigit{1}}{\isacharparenright}\isanewline
\ \ \ \ \ \ \isacommand{also}\isamarkupfalse%
\ \isacommand{have}\isamarkupfalse%
\ {\isachardoublequoteopen}{\isasymdots}\ {\isacharequal}\ atoms\ {\isacharparenleft}F{\isadigit{1}}\ \isactrlbold {\isasymrightarrow}\ F{\isadigit{2}}{\isacharparenright}{\isachardoublequoteclose}\isanewline
\ \ \ \ \ \ \ \ \isacommand{by}\isamarkupfalse%
\ {\isacharparenleft}simp\ only{\isacharcolon}\ formula{\isachardot}set{\isacharparenleft}{\isadigit{6}}{\isacharparenright}{\isacharparenright}\isanewline
\ \ \ \ \ \ \isacommand{finally}\isamarkupfalse%
\ \isacommand{show}\isamarkupfalse%
\ {\isacharquery}thesis\isanewline
\ \ \ \ \ \ \ \ \isacommand{by}\isamarkupfalse%
\ this\isanewline
\ \ \ \ \isacommand{next}\isamarkupfalse%
\isanewline
\ \ \ \ \ \ \isacommand{assume}\isamarkupfalse%
\ {\isachardoublequoteopen}G\ {\isasymin}\ setSubformulae\ F{\isadigit{2}}{\isachardoublequoteclose}\isanewline
\ \ \ \ \ \ \isacommand{then}\isamarkupfalse%
\ \isacommand{have}\isamarkupfalse%
\ {\isachardoublequoteopen}atoms\ G\ {\isasymsubseteq}\ atoms\ F{\isadigit{2}}{\isachardoublequoteclose}\isanewline
\ \ \ \ \ \ \ \ \isacommand{by}\isamarkupfalse%
\ {\isacharparenleft}rule\ assms{\isacharparenleft}{\isadigit{2}}{\isacharparenright}{\isacharparenright}\isanewline
\ \ \ \ \ \ \isacommand{also}\isamarkupfalse%
\ \isacommand{have}\isamarkupfalse%
\ {\isachardoublequoteopen}{\isasymdots}\ {\isasymsubseteq}\ atoms\ F{\isadigit{1}}\ {\isasymunion}\ atoms\ F{\isadigit{2}}{\isachardoublequoteclose}\isanewline
\ \ \ \ \ \ \ \ \isacommand{by}\isamarkupfalse%
\ {\isacharparenleft}simp\ only{\isacharcolon}\ Un{\isacharunderscore}upper{\isadigit{2}}{\isacharparenright}\isanewline
\ \ \ \ \ \ \isacommand{also}\isamarkupfalse%
\ \isacommand{have}\isamarkupfalse%
\ {\isachardoublequoteopen}{\isasymdots}\ {\isacharequal}\ atoms\ {\isacharparenleft}F{\isadigit{1}}\ \isactrlbold {\isasymrightarrow}\ F{\isadigit{2}}{\isacharparenright}{\isachardoublequoteclose}\isanewline
\ \ \ \ \ \ \ \ \isacommand{by}\isamarkupfalse%
\ {\isacharparenleft}simp\ only{\isacharcolon}\ formula{\isachardot}set{\isacharparenleft}{\isadigit{6}}{\isacharparenright}{\isacharparenright}\isanewline
\ \ \ \ \ \ \isacommand{finally}\isamarkupfalse%
\ \isacommand{show}\isamarkupfalse%
\ {\isacharquery}thesis\isanewline
\ \ \ \ \ \ \ \ \isacommand{by}\isamarkupfalse%
\ this\isanewline
\ \ \ \ \isacommand{qed}\isamarkupfalse%
\isanewline
\ \ \isacommand{qed}\isamarkupfalse%
\isanewline
\isacommand{qed}\isamarkupfalse%
%
\endisatagproof
{\isafoldproof}%
%
\isadelimproof
\isanewline
%
\endisadelimproof
\isanewline
\isacommand{lemma}\isamarkupfalse%
\ subformulae{\isacharunderscore}atoms{\isacharcolon}\ \isanewline
\ \ {\isachardoublequoteopen}G\ {\isasymin}\ setSubformulae\ F\ {\isasymLongrightarrow}\ atoms\ G\ {\isasymsubseteq}\ atoms\ F{\isachardoublequoteclose}\isanewline
%
\isadelimproof
%
\endisadelimproof
%
\isatagproof
\isacommand{proof}\isamarkupfalse%
\ {\isacharparenleft}induction\ F{\isacharparenright}\isanewline
\ \ \isacommand{case}\isamarkupfalse%
\ {\isacharparenleft}Atom\ x{\isacharparenright}\isanewline
\ \ \isacommand{then}\isamarkupfalse%
\ \isacommand{show}\isamarkupfalse%
\ {\isacharquery}case\ \isacommand{by}\isamarkupfalse%
\ {\isacharparenleft}simp\ only{\isacharcolon}\ subformulas{\isacharunderscore}atoms{\isacharunderscore}atom{\isacharparenright}\ \isanewline
\isacommand{next}\isamarkupfalse%
\isanewline
\ \ \isacommand{case}\isamarkupfalse%
\ Bot\isanewline
\ \ \isacommand{then}\isamarkupfalse%
\ \isacommand{show}\isamarkupfalse%
\ {\isacharquery}case\ \isacommand{by}\isamarkupfalse%
\ {\isacharparenleft}simp\ only{\isacharcolon}\ subformulas{\isacharunderscore}atoms{\isacharunderscore}bot{\isacharparenright}\isanewline
\isacommand{next}\isamarkupfalse%
\isanewline
\ \ \isacommand{case}\isamarkupfalse%
\ {\isacharparenleft}Not\ F{\isacharparenright}\isanewline
\ \ \isacommand{then}\isamarkupfalse%
\ \isacommand{show}\isamarkupfalse%
\ {\isacharquery}case\ \isacommand{by}\isamarkupfalse%
\ {\isacharparenleft}simp\ only{\isacharcolon}\ subformulas{\isacharunderscore}atoms{\isacharunderscore}not{\isacharparenright}\isanewline
\isacommand{next}\isamarkupfalse%
\isanewline
\ \ \isacommand{case}\isamarkupfalse%
\ {\isacharparenleft}And\ F{\isadigit{1}}\ F{\isadigit{2}}{\isacharparenright}\isanewline
\ \ \isacommand{then}\isamarkupfalse%
\ \isacommand{show}\isamarkupfalse%
\ {\isacharquery}case\ \isacommand{by}\isamarkupfalse%
\ {\isacharparenleft}simp\ only{\isacharcolon}\ subformulas{\isacharunderscore}atoms{\isacharunderscore}and{\isacharparenright}\isanewline
\isacommand{next}\isamarkupfalse%
\isanewline
\ \ \isacommand{case}\isamarkupfalse%
\ {\isacharparenleft}Or\ F{\isadigit{1}}\ F{\isadigit{2}}{\isacharparenright}\isanewline
\ \ \isacommand{then}\isamarkupfalse%
\ \isacommand{show}\isamarkupfalse%
\ {\isacharquery}case\ \isacommand{by}\isamarkupfalse%
\ {\isacharparenleft}simp\ only{\isacharcolon}\ subformulas{\isacharunderscore}atoms{\isacharunderscore}or{\isacharparenright}\isanewline
\isacommand{next}\isamarkupfalse%
\isanewline
\ \ \isacommand{case}\isamarkupfalse%
\ {\isacharparenleft}Imp\ F{\isadigit{1}}\ F{\isadigit{2}}{\isacharparenright}\isanewline
\ \ \isacommand{then}\isamarkupfalse%
\ \isacommand{show}\isamarkupfalse%
\ {\isacharquery}case\ \isacommand{by}\isamarkupfalse%
\ {\isacharparenleft}simp\ only{\isacharcolon}\ subformulas{\isacharunderscore}atoms{\isacharunderscore}imp{\isacharparenright}\isanewline
\isacommand{qed}\isamarkupfalse%
%
\endisatagproof
{\isafoldproof}%
%
\isadelimproof
%
\endisadelimproof
%
\begin{isamarkuptext}%
Por último, su demostración automática.%
\end{isamarkuptext}\isamarkuptrue%
\isacommand{lemma}\isamarkupfalse%
\ {\isachardoublequoteopen}G\ {\isasymin}\ setSubformulae\ F\ {\isasymLongrightarrow}\ atoms\ G\ {\isasymsubseteq}\ atoms\ F{\isachardoublequoteclose}\isanewline
%
\isadelimproof
\ \ %
\endisadelimproof
%
\isatagproof
\isacommand{by}\isamarkupfalse%
\ {\isacharparenleft}induction\ F{\isacharparenright}\ auto%
\endisatagproof
{\isafoldproof}%
%
\isadelimproof
%
\endisadelimproof
%
\begin{isamarkuptext}%
A continuación vamos a introducir un lema para facilitar
   las siguientes demostraciones detalladas mediante contenciones en 
   cadena.

  \begin{lema}
    Sea \isa{G} una subfórmula de \isa{F}, entonces el conjunto de subfórmulas 
    de \isa{G} está contenido en el de \isa{F}.
  \end{lema} 

  \begin{demostracion}
  La prueba es por inducción en la estructura de fórmula.
  
  Sea \isa{p} una fórmula atómica cualquiera. Por definición, el conjunto de
  sus subfórmulas es \isa{{\isacharbraceleft}p{\isacharbraceright}}, luego su única subfórmula es ella misma y,
  por tanto, tienen igual conjunto de subfórmulas.

  Sea la fórmula \isa{{\isasymbottom}}. Por definición, el conjunto de
  sus subfórmulas es \isa{{\isacharbraceleft}{\isasymbottom}{\isacharbraceright}}, luego su única subfórmula es ella misma y,
  por tanto, tienen igual conjunto de subfórmulas.

  Sea una fórmula \isa{F} tal que para toda subfórmula suya se tiene que el
  conjunto de sus subfórmulas está contenido en el conjunto de 
  subfórmulas de \isa{F}.
  Supongamos \isa{G} subfórmula de \isa{{\isasymnot}\ F}. Vamos a probar que el conjunto de
  subfórmulas de \isa{G} está contenido en el de \isa{{\isasymnot}\ F}.
  En primer lugar, por definición se cumple que el conjunto de
  subfórmulas de \isa{{\isasymnot}\ F} es de la forma \isa{Subf{\isacharparenleft}{\isasymnot}\ F{\isacharparenright}\ {\isacharequal}\ {\isacharbraceleft}{\isasymnot}\ F{\isacharbraceright}\ {\isasymunion}\ Subf{\isacharparenleft}F{\isacharparenright}}.
  Como hemos supuesto \isa{G} subfórmula de \isa{{\isasymnot}\ F}, hay dos opciones 
  posibles: \isa{G\ {\isasymin}\ {\isacharbraceleft}{\isasymnot}\ F{\isacharbraceright}} o \isa{G\ {\isasymin}\ Subf{\isacharparenleft}F{\isacharparenright}}. 
  Del primer caso se obtiene que \isa{G\ {\isacharequal}\ {\isasymnot}\ F} y, por tanto, tienen igual 
  conjunto de subfórmulas. 
  Por otro lado si suponemos que \isa{G} es subfórmula de \isa{F}, por hipótesis
  de inducción tenemos que el conjunto de subfórmulas de \isa{G} está 
  contenido en el de \isa{F}. Como, a su vez, el conjunto de subfórmulas
  de \isa{F} está contenido en el de \isa{{\isasymnot}\ F} según la definición anterior, 
  por propiedades de la contención de verifica que el conjunto de 
  subfórmulas de \isa{G} está contenido en el de \isa{{\isasymnot}\ F}, como queríamos 
  demostrar.

  Sean las fórmulas \isa{F{\isadigit{1}}} y \isa{F{\isadigit{2}}} tales que para cualquier subfórmula
  de \isa{F{\isadigit{1}}} el conjunto de sus subfórmulas está contenido en el conjunto 
  de subfórmulas de \isa{F{\isadigit{1}}}, y para cualquier subfórmula de \isa{F{\isadigit{2}}} el 
  conjunto de sus subfórmulas está contenido en el conjunto de 
  subfórmulas de \isa{F{\isadigit{2}}}. Supongamos \isa{G} 
  subfórmula de \isa{F{\isadigit{1}}{\isacharasterisk}F{\isadigit{2}}} donde \isa{{\isacharasterisk}} simboliza una conectiva binaria 
  cualquiera. Vamos a probar que el conjunto de subfórmulas de \isa{G} está
  contenido en el de \isa{F{\isadigit{1}}{\isacharasterisk}F{\isadigit{2}}}. 
  En primer lugar, por definición se cumple que el conjunto de 
  subfórmulas de \isa{F{\isadigit{1}}{\isacharasterisk}F{\isadigit{2}}} es de la forma
  \isa{{\isacharbraceleft}F{\isadigit{1}}{\isacharasterisk}F{\isadigit{2}}{\isacharbraceright}\ {\isasymunion}\ {\isacharparenleft}Subf{\isacharparenleft}F{\isadigit{1}}{\isacharparenright}\ {\isasymunion}\ Subf{\isacharparenleft}F{\isadigit{2}}{\isacharparenright}{\isacharparenright}}. De este modo,
  tenemos dos opciones: \isa{G\ {\isasymin}\ {\isacharbraceleft}F{\isadigit{1}}{\isacharasterisk}F{\isadigit{2}}{\isacharbraceright}} o \isa{G\ {\isasymin}\ Subf{\isacharparenleft}F{\isadigit{1}}{\isacharparenright}\ {\isasymunion}\ Subf{\isacharparenleft}F{\isadigit{2}}{\isacharparenright}}.
  De la primera opción se deduce \isa{G\ {\isacharequal}\ F{\isadigit{1}}{\isacharasterisk}F{\isadigit{2}}} y, por
  tanto, tienen igual conjunto de subfórmulas. 
  Por otro lado, si \isa{G\ {\isasymin}\ Subf{\isacharparenleft}F{\isadigit{1}}{\isacharparenright}\ {\isasymunion}\ Subf{\isacharparenleft}F{\isadigit{2}}{\isacharparenright}}, tenemos a su vez dos 
  opciones: \isa{G} es subfórmula de \isa{F{\isadigit{1}}} o \isa{G} es subfórmula de \isa{F{\isadigit{2}}}.
  Supongamos que fuese subfórmula de \isa{F{\isadigit{1}}}. En este caso, por hipótesis 
  de inducción se tiene que el conjunto de subfórmulas de \isa{G} está 
  contenido en el de \isa{F{\isadigit{1}}}. Por la definición anterior del conjunto de 
  subfórmulas de \isa{F{\isadigit{1}}{\isacharasterisk}F{\isadigit{2}}}, se verifica que el conjunto de subfórmulas de 
  \isa{F{\isadigit{1}}} está contenido en el de \isa{F{\isadigit{1}}{\isacharasterisk}F{\isadigit{2}}}. Por tanto, por propiedades de
  contención se tiene que el conjunto de subfórmulas de \isa{G} está 
  contenido en el conjunto de subfórmulas de \isa{F{\isadigit{1}}{\isacharasterisk}F{\isadigit{2}}}. El caso de \isa{G} 
  subfórmula de \isa{F{\isadigit{2}}} se demuestra análogamente cambiando el 
  índice de la fórmula correspondiente. Por tanto, se verifica el 
  resultado en este caso.  
  \end{demostracion}

Veamos su formalización en Isabelle junto con su demostración 
  estructurada.%
\end{isamarkuptext}\isamarkuptrue%
\isacommand{lemma}\isamarkupfalse%
\ subContsubformulae{\isacharunderscore}atom{\isacharcolon}\ \isanewline
\ \ \isakeyword{assumes}\ {\isachardoublequoteopen}G\ {\isasymin}\ setSubformulae\ {\isacharparenleft}Atom\ x{\isacharparenright}{\isachardoublequoteclose}\ \isanewline
\ \ \isakeyword{shows}\ {\isachardoublequoteopen}setSubformulae\ G\ {\isasymsubseteq}\ setSubformulae\ {\isacharparenleft}Atom\ x{\isacharparenright}{\isachardoublequoteclose}\isanewline
%
\isadelimproof
%
\endisadelimproof
%
\isatagproof
\isacommand{proof}\isamarkupfalse%
\ {\isacharminus}\ \isanewline
\ \ \isacommand{have}\isamarkupfalse%
\ {\isachardoublequoteopen}G\ {\isasymin}\ {\isacharbraceleft}Atom\ x{\isacharbraceright}{\isachardoublequoteclose}\ \isacommand{using}\isamarkupfalse%
\ assms\ \isanewline
\ \ \ \ \isacommand{by}\isamarkupfalse%
\ {\isacharparenleft}simp\ only{\isacharcolon}\ setSubformulae{\isacharunderscore}atom{\isacharparenright}\isanewline
\ \ \isacommand{then}\isamarkupfalse%
\ \isacommand{have}\isamarkupfalse%
\ {\isachardoublequoteopen}G\ {\isacharequal}\ Atom\ x{\isachardoublequoteclose}\isanewline
\ \ \ \ \isacommand{by}\isamarkupfalse%
\ {\isacharparenleft}simp\ only{\isacharcolon}\ singletonD{\isacharparenright}\isanewline
\ \ \isacommand{then}\isamarkupfalse%
\ \isacommand{show}\isamarkupfalse%
\ {\isacharquery}thesis\isanewline
\ \ \ \ \isacommand{by}\isamarkupfalse%
\ {\isacharparenleft}simp\ only{\isacharcolon}\ subset{\isacharunderscore}refl{\isacharparenright}\isanewline
\isacommand{qed}\isamarkupfalse%
%
\endisatagproof
{\isafoldproof}%
%
\isadelimproof
\isanewline
%
\endisadelimproof
\isanewline
\isacommand{lemma}\isamarkupfalse%
\ subContsubformulae{\isacharunderscore}bot{\isacharcolon}\isanewline
\ \ \isakeyword{assumes}\ {\isachardoublequoteopen}G\ {\isasymin}\ setSubformulae\ {\isasymbottom}{\isachardoublequoteclose}\ \isanewline
\ \ \isakeyword{shows}\ \ \ {\isachardoublequoteopen}setSubformulae\ G\ {\isasymsubseteq}\ setSubformulae\ {\isasymbottom}{\isachardoublequoteclose}\isanewline
%
\isadelimproof
%
\endisadelimproof
%
\isatagproof
\isacommand{proof}\isamarkupfalse%
\ {\isacharminus}\isanewline
\ \ \isacommand{have}\isamarkupfalse%
\ {\isachardoublequoteopen}G\ {\isasymin}\ {\isacharbraceleft}{\isasymbottom}{\isacharbraceright}{\isachardoublequoteclose}\isanewline
\ \ \ \ \isacommand{using}\isamarkupfalse%
\ assms\isanewline
\ \ \ \ \isacommand{by}\isamarkupfalse%
\ {\isacharparenleft}simp\ only{\isacharcolon}\ setSubformulae{\isacharunderscore}bot{\isacharparenright}\isanewline
\ \ \isacommand{then}\isamarkupfalse%
\ \isacommand{have}\isamarkupfalse%
\ {\isachardoublequoteopen}G\ {\isacharequal}\ {\isasymbottom}{\isachardoublequoteclose}\isanewline
\ \ \ \ \isacommand{by}\isamarkupfalse%
\ {\isacharparenleft}simp\ only{\isacharcolon}\ singletonD{\isacharparenright}\isanewline
\ \ \isacommand{then}\isamarkupfalse%
\ \isacommand{show}\isamarkupfalse%
\ {\isacharquery}thesis\isanewline
\ \ \ \ \isacommand{by}\isamarkupfalse%
\ {\isacharparenleft}simp\ only{\isacharcolon}\ subset{\isacharunderscore}refl{\isacharparenright}\isanewline
\isacommand{qed}\isamarkupfalse%
%
\endisatagproof
{\isafoldproof}%
%
\isadelimproof
\isanewline
%
\endisadelimproof
\isanewline
\isacommand{lemma}\isamarkupfalse%
\ subContsubformulae{\isacharunderscore}not{\isacharcolon}\isanewline
\ \ \isakeyword{assumes}\ {\isachardoublequoteopen}G\ {\isasymin}\ setSubformulae\ F\ {\isasymLongrightarrow}\ setSubformulae\ G\ {\isasymsubseteq}\ setSubformulae\ F{\isachardoublequoteclose}\isanewline
\ \ \ \ \ \ \ \ \ \ {\isachardoublequoteopen}G\ {\isasymin}\ setSubformulae\ {\isacharparenleft}\isactrlbold {\isasymnot}\ F{\isacharparenright}{\isachardoublequoteclose}\isanewline
\ \ \isakeyword{shows}\ \ \ {\isachardoublequoteopen}setSubformulae\ G\ {\isasymsubseteq}\ setSubformulae\ {\isacharparenleft}\isactrlbold {\isasymnot}\ F{\isacharparenright}{\isachardoublequoteclose}\isanewline
%
\isadelimproof
%
\endisadelimproof
%
\isatagproof
\isacommand{proof}\isamarkupfalse%
\ {\isacharminus}\isanewline
\ \ \isacommand{have}\isamarkupfalse%
\ {\isachardoublequoteopen}G\ {\isasymin}\ {\isacharbraceleft}\isactrlbold {\isasymnot}\ F{\isacharbraceright}\ {\isasymunion}\ setSubformulae\ F{\isachardoublequoteclose}\isanewline
\ \ \ \ \isacommand{using}\isamarkupfalse%
\ assms{\isacharparenleft}{\isadigit{2}}{\isacharparenright}\isanewline
\ \ \ \ \isacommand{by}\isamarkupfalse%
\ {\isacharparenleft}simp\ only{\isacharcolon}\ setSubformulae{\isacharunderscore}not{\isacharparenright}\ \isanewline
\ \ \isacommand{then}\isamarkupfalse%
\ \isacommand{have}\isamarkupfalse%
\ {\isachardoublequoteopen}G\ {\isasymin}\ {\isacharbraceleft}\isactrlbold {\isasymnot}\ F{\isacharbraceright}\ {\isasymor}\ G\ {\isasymin}\ setSubformulae\ F{\isachardoublequoteclose}\isanewline
\ \ \ \ \isacommand{by}\isamarkupfalse%
\ {\isacharparenleft}simp\ only{\isacharcolon}\ Un{\isacharunderscore}iff{\isacharparenright}\isanewline
\ \ \isacommand{then}\isamarkupfalse%
\ \isacommand{show}\isamarkupfalse%
\ {\isachardoublequoteopen}setSubformulae\ G\ {\isasymsubseteq}\ setSubformulae\ {\isacharparenleft}\isactrlbold {\isasymnot}\ F{\isacharparenright}{\isachardoublequoteclose}\isanewline
\ \ \isacommand{proof}\isamarkupfalse%
\isanewline
\ \ \ \ \isacommand{assume}\isamarkupfalse%
\ {\isachardoublequoteopen}G\ {\isasymin}\ {\isacharbraceleft}\isactrlbold {\isasymnot}\ F{\isacharbraceright}{\isachardoublequoteclose}\isanewline
\ \ \ \ \isacommand{then}\isamarkupfalse%
\ \isacommand{have}\isamarkupfalse%
\ {\isachardoublequoteopen}G\ {\isacharequal}\ \isactrlbold {\isasymnot}\ F{\isachardoublequoteclose}\isanewline
\ \ \ \ \ \ \isacommand{by}\isamarkupfalse%
\ {\isacharparenleft}simp\ only{\isacharcolon}\ singletonD{\isacharparenright}\isanewline
\ \ \ \ \isacommand{then}\isamarkupfalse%
\ \isacommand{show}\isamarkupfalse%
\ {\isacharquery}thesis\isanewline
\ \ \ \ \ \ \isacommand{by}\isamarkupfalse%
\ {\isacharparenleft}simp\ only{\isacharcolon}\ subset{\isacharunderscore}refl{\isacharparenright}\isanewline
\ \ \isacommand{next}\isamarkupfalse%
\isanewline
\ \ \ \ \isacommand{assume}\isamarkupfalse%
\ {\isachardoublequoteopen}G\ {\isasymin}\ setSubformulae\ F{\isachardoublequoteclose}\isanewline
\ \ \ \ \isacommand{then}\isamarkupfalse%
\ \isacommand{have}\isamarkupfalse%
\ {\isachardoublequoteopen}setSubformulae\ G\ {\isasymsubseteq}\ setSubformulae\ F{\isachardoublequoteclose}\isanewline
\ \ \ \ \ \ \isacommand{by}\isamarkupfalse%
\ {\isacharparenleft}simp\ only{\isacharcolon}\ assms{\isacharparenleft}{\isadigit{1}}{\isacharparenright}{\isacharparenright}\isanewline
\ \ \ \ \isacommand{also}\isamarkupfalse%
\ \isacommand{have}\isamarkupfalse%
\ {\isachardoublequoteopen}setSubformulae\ F\ {\isasymsubseteq}\ setSubformulae\ {\isacharparenleft}\isactrlbold {\isasymnot}\ F{\isacharparenright}{\isachardoublequoteclose}\isanewline
\ \ \ \ \ \ \isacommand{by}\isamarkupfalse%
\ {\isacharparenleft}simp\ only{\isacharcolon}\ setSubformulae{\isacharunderscore}not\ Un{\isacharunderscore}upper{\isadigit{2}}{\isacharparenright}\isanewline
\ \ \ \ \isacommand{finally}\isamarkupfalse%
\ \isacommand{show}\isamarkupfalse%
\ {\isacharquery}thesis\ \isanewline
\ \ \ \ \ \ \isacommand{by}\isamarkupfalse%
\ this\isanewline
\ \ \isacommand{qed}\isamarkupfalse%
\isanewline
\isacommand{qed}\isamarkupfalse%
%
\endisatagproof
{\isafoldproof}%
%
\isadelimproof
\isanewline
%
\endisadelimproof
\isanewline
\isacommand{lemma}\isamarkupfalse%
\ subContsubformulae{\isacharunderscore}and{\isacharcolon}\isanewline
\ \ \isakeyword{assumes}\ {\isachardoublequoteopen}G\ {\isasymin}\ setSubformulae\ F{\isadigit{1}}\ \isanewline
\ \ \ \ \ \ \ \ \ \ \ \ {\isasymLongrightarrow}\ setSubformulae\ G\ {\isasymsubseteq}\ setSubformulae\ F{\isadigit{1}}{\isachardoublequoteclose}\isanewline
\ \ \ \ \ \ \ \ \ \ {\isachardoublequoteopen}G\ {\isasymin}\ setSubformulae\ F{\isadigit{2}}\ \isanewline
\ \ \ \ \ \ \ \ \ \ \ \ {\isasymLongrightarrow}\ setSubformulae\ G\ {\isasymsubseteq}\ setSubformulae\ F{\isadigit{2}}{\isachardoublequoteclose}\isanewline
\ \ \ \ \ \ \ \ \ \ {\isachardoublequoteopen}G\ {\isasymin}\ setSubformulae\ {\isacharparenleft}F{\isadigit{1}}\ \isactrlbold {\isasymand}\ F{\isadigit{2}}{\isacharparenright}{\isachardoublequoteclose}\isanewline
\ \ \isakeyword{shows}\ \ \ {\isachardoublequoteopen}setSubformulae\ G\ {\isasymsubseteq}\ setSubformulae\ {\isacharparenleft}F{\isadigit{1}}\ \isactrlbold {\isasymand}\ F{\isadigit{2}}{\isacharparenright}{\isachardoublequoteclose}\isanewline
%
\isadelimproof
%
\endisadelimproof
%
\isatagproof
\isacommand{proof}\isamarkupfalse%
\ {\isacharminus}\isanewline
\ \ \isacommand{have}\isamarkupfalse%
\ {\isachardoublequoteopen}G\ {\isasymin}\ {\isacharbraceleft}F{\isadigit{1}}\ \isactrlbold {\isasymand}\ F{\isadigit{2}}{\isacharbraceright}\ {\isasymunion}\ {\isacharparenleft}setSubformulae\ F{\isadigit{1}}\ {\isasymunion}\ setSubformulae\ F{\isadigit{2}}{\isacharparenright}{\isachardoublequoteclose}\isanewline
\ \ \ \ \isacommand{using}\isamarkupfalse%
\ assms{\isacharparenleft}{\isadigit{3}}{\isacharparenright}\ \isanewline
\ \ \ \ \isacommand{by}\isamarkupfalse%
\ {\isacharparenleft}simp\ only{\isacharcolon}\ setSubformulae{\isacharunderscore}and{\isacharparenright}\isanewline
\ \ \isacommand{then}\isamarkupfalse%
\ \isacommand{have}\isamarkupfalse%
\ {\isachardoublequoteopen}G\ {\isasymin}\ {\isacharbraceleft}F{\isadigit{1}}\ \isactrlbold {\isasymand}\ F{\isadigit{2}}{\isacharbraceright}\ {\isasymor}\ G\ {\isasymin}\ setSubformulae\ F{\isadigit{1}}\ {\isasymunion}\ setSubformulae\ F{\isadigit{2}}{\isachardoublequoteclose}\isanewline
\ \ \ \ \isacommand{by}\isamarkupfalse%
\ {\isacharparenleft}simp\ only{\isacharcolon}\ Un{\isacharunderscore}iff{\isacharparenright}\isanewline
\ \ \isacommand{then}\isamarkupfalse%
\ \isacommand{show}\isamarkupfalse%
\ {\isacharquery}thesis\isanewline
\ \ \isacommand{proof}\isamarkupfalse%
\ {\isacharparenleft}rule\ disjE{\isacharparenright}\isanewline
\ \ \ \ \isacommand{assume}\isamarkupfalse%
\ {\isachardoublequoteopen}G\ {\isasymin}\ {\isacharbraceleft}F{\isadigit{1}}\ \isactrlbold {\isasymand}\ F{\isadigit{2}}{\isacharbraceright}{\isachardoublequoteclose}\isanewline
\ \ \ \ \isacommand{then}\isamarkupfalse%
\ \isacommand{have}\isamarkupfalse%
\ {\isachardoublequoteopen}G\ {\isacharequal}\ F{\isadigit{1}}\ \isactrlbold {\isasymand}\ F{\isadigit{2}}{\isachardoublequoteclose}\isanewline
\ \ \ \ \ \ \isacommand{by}\isamarkupfalse%
\ {\isacharparenleft}simp\ only{\isacharcolon}\ singletonD{\isacharparenright}\isanewline
\ \ \ \ \isacommand{then}\isamarkupfalse%
\ \isacommand{show}\isamarkupfalse%
\ {\isacharquery}thesis\isanewline
\ \ \ \ \ \ \isacommand{by}\isamarkupfalse%
\ {\isacharparenleft}simp\ only{\isacharcolon}\ subset{\isacharunderscore}refl{\isacharparenright}\isanewline
\ \ \isacommand{next}\isamarkupfalse%
\isanewline
\ \ \ \ \isacommand{assume}\isamarkupfalse%
\ {\isachardoublequoteopen}G\ {\isasymin}\ setSubformulae\ F{\isadigit{1}}\ {\isasymunion}\ setSubformulae\ F{\isadigit{2}}{\isachardoublequoteclose}\isanewline
\ \ \ \ \isacommand{then}\isamarkupfalse%
\ \isacommand{have}\isamarkupfalse%
\ {\isachardoublequoteopen}G\ {\isasymin}\ setSubformulae\ F{\isadigit{1}}\ {\isasymor}\ G\ {\isasymin}\ setSubformulae\ F{\isadigit{2}}{\isachardoublequoteclose}\ \ \isanewline
\ \ \ \ \ \ \isacommand{by}\isamarkupfalse%
\ {\isacharparenleft}simp\ only{\isacharcolon}\ Un{\isacharunderscore}iff{\isacharparenright}\isanewline
\ \ \ \ \isacommand{then}\isamarkupfalse%
\ \isacommand{show}\isamarkupfalse%
\ {\isacharquery}thesis\isanewline
\ \ \ \ \isacommand{proof}\isamarkupfalse%
\ \isanewline
\ \ \ \ \ \ \isacommand{assume}\isamarkupfalse%
\ {\isachardoublequoteopen}G\ {\isasymin}\ setSubformulae\ F{\isadigit{1}}{\isachardoublequoteclose}\isanewline
\ \ \ \ \ \ \isacommand{then}\isamarkupfalse%
\ \isacommand{have}\isamarkupfalse%
\ {\isachardoublequoteopen}setSubformulae\ G\ {\isasymsubseteq}\ setSubformulae\ F{\isadigit{1}}{\isachardoublequoteclose}\isanewline
\ \ \ \ \ \ \ \ \isacommand{by}\isamarkupfalse%
\ {\isacharparenleft}simp\ only{\isacharcolon}\ assms{\isacharparenleft}{\isadigit{1}}{\isacharparenright}{\isacharparenright}\isanewline
\ \ \ \ \ \ \isacommand{also}\isamarkupfalse%
\ \isacommand{have}\isamarkupfalse%
\ {\isachardoublequoteopen}{\isasymdots}\ {\isasymsubseteq}\ setSubformulae\ F{\isadigit{1}}\ {\isasymunion}\ setSubformulae\ F{\isadigit{2}}{\isachardoublequoteclose}\isanewline
\ \ \ \ \ \ \ \ \isacommand{by}\isamarkupfalse%
\ {\isacharparenleft}simp\ only{\isacharcolon}\ Un{\isacharunderscore}upper{\isadigit{1}}{\isacharparenright}\isanewline
\ \ \ \ \ \ \isacommand{also}\isamarkupfalse%
\ \isacommand{have}\isamarkupfalse%
\ {\isachardoublequoteopen}{\isasymdots}\ {\isasymsubseteq}\ setSubformulae\ {\isacharparenleft}F{\isadigit{1}}\ \isactrlbold {\isasymand}\ F{\isadigit{2}}{\isacharparenright}{\isachardoublequoteclose}\isanewline
\ \ \ \ \ \ \ \ \isacommand{by}\isamarkupfalse%
\ {\isacharparenleft}simp\ only{\isacharcolon}\ setSubformulae{\isacharunderscore}and\ Un{\isacharunderscore}upper{\isadigit{2}}{\isacharparenright}\isanewline
\ \ \ \ \ \ \isacommand{finally}\isamarkupfalse%
\ \isacommand{show}\isamarkupfalse%
\ {\isacharquery}thesis\isanewline
\ \ \ \ \ \ \ \ \isacommand{by}\isamarkupfalse%
\ this\isanewline
\ \ \ \ \isacommand{next}\isamarkupfalse%
\isanewline
\ \ \ \ \ \ \isacommand{assume}\isamarkupfalse%
\ {\isachardoublequoteopen}G\ {\isasymin}\ setSubformulae\ F{\isadigit{2}}{\isachardoublequoteclose}\isanewline
\ \ \ \ \ \ \isacommand{then}\isamarkupfalse%
\ \isacommand{have}\isamarkupfalse%
\ {\isachardoublequoteopen}setSubformulae\ G\ {\isasymsubseteq}\ setSubformulae\ F{\isadigit{2}}{\isachardoublequoteclose}\isanewline
\ \ \ \ \ \ \ \ \isacommand{by}\isamarkupfalse%
\ {\isacharparenleft}rule\ assms{\isacharparenleft}{\isadigit{2}}{\isacharparenright}{\isacharparenright}\isanewline
\ \ \ \ \ \ \isacommand{also}\isamarkupfalse%
\ \isacommand{have}\isamarkupfalse%
\ {\isachardoublequoteopen}{\isasymdots}\ {\isasymsubseteq}\ setSubformulae\ F{\isadigit{1}}\ {\isasymunion}\ setSubformulae\ F{\isadigit{2}}{\isachardoublequoteclose}\isanewline
\ \ \ \ \ \ \ \ \isacommand{by}\isamarkupfalse%
\ {\isacharparenleft}simp\ only{\isacharcolon}\ Un{\isacharunderscore}upper{\isadigit{2}}{\isacharparenright}\isanewline
\ \ \ \ \ \ \isacommand{also}\isamarkupfalse%
\ \isacommand{have}\isamarkupfalse%
\ {\isachardoublequoteopen}{\isasymdots}\ {\isasymsubseteq}\ setSubformulae\ {\isacharparenleft}F{\isadigit{1}}\ \isactrlbold {\isasymand}\ F{\isadigit{2}}{\isacharparenright}{\isachardoublequoteclose}\isanewline
\ \ \ \ \ \ \ \ \isacommand{by}\isamarkupfalse%
\ {\isacharparenleft}simp\ only{\isacharcolon}\ setSubformulae{\isacharunderscore}and\ Un{\isacharunderscore}upper{\isadigit{2}}{\isacharparenright}\isanewline
\ \ \ \ \ \ \isacommand{finally}\isamarkupfalse%
\ \isacommand{show}\isamarkupfalse%
\ {\isacharquery}thesis\isanewline
\ \ \ \ \ \ \ \ \isacommand{by}\isamarkupfalse%
\ this\isanewline
\ \ \ \ \isacommand{qed}\isamarkupfalse%
\isanewline
\ \ \isacommand{qed}\isamarkupfalse%
\isanewline
\isacommand{qed}\isamarkupfalse%
%
\endisatagproof
{\isafoldproof}%
%
\isadelimproof
\isanewline
%
\endisadelimproof
\isanewline
\isacommand{lemma}\isamarkupfalse%
\ subContsubformulae{\isacharunderscore}or{\isacharcolon}\isanewline
\ \ \isakeyword{assumes}\ {\isachardoublequoteopen}G\ {\isasymin}\ setSubformulae\ F{\isadigit{1}}\ \isanewline
\ \ \ \ \ \ \ \ \ \ \ \ {\isasymLongrightarrow}\ setSubformulae\ G\ {\isasymsubseteq}\ setSubformulae\ F{\isadigit{1}}{\isachardoublequoteclose}\isanewline
\ \ \ \ \ \ \ \ \ \ {\isachardoublequoteopen}G\ {\isasymin}\ setSubformulae\ F{\isadigit{2}}\ \isanewline
\ \ \ \ \ \ \ \ \ \ \ \ {\isasymLongrightarrow}\ setSubformulae\ G\ {\isasymsubseteq}\ setSubformulae\ F{\isadigit{2}}{\isachardoublequoteclose}\isanewline
\ \ \ \ \ \ \ \ \ \ {\isachardoublequoteopen}G\ {\isasymin}\ setSubformulae\ {\isacharparenleft}F{\isadigit{1}}\ \isactrlbold {\isasymor}\ F{\isadigit{2}}{\isacharparenright}{\isachardoublequoteclose}\isanewline
\ \ \isakeyword{shows}\ \ \ {\isachardoublequoteopen}setSubformulae\ G\ {\isasymsubseteq}\ setSubformulae\ {\isacharparenleft}F{\isadigit{1}}\ \isactrlbold {\isasymor}\ F{\isadigit{2}}{\isacharparenright}{\isachardoublequoteclose}\isanewline
%
\isadelimproof
%
\endisadelimproof
%
\isatagproof
\isacommand{proof}\isamarkupfalse%
\ {\isacharminus}\isanewline
\ \ \isacommand{have}\isamarkupfalse%
\ {\isachardoublequoteopen}G\ {\isasymin}\ {\isacharbraceleft}F{\isadigit{1}}\ \isactrlbold {\isasymor}\ F{\isadigit{2}}{\isacharbraceright}\ {\isasymunion}\ {\isacharparenleft}setSubformulae\ F{\isadigit{1}}\ {\isasymunion}\ setSubformulae\ F{\isadigit{2}}{\isacharparenright}{\isachardoublequoteclose}\isanewline
\ \ \ \ \isacommand{using}\isamarkupfalse%
\ assms{\isacharparenleft}{\isadigit{3}}{\isacharparenright}\ \isanewline
\ \ \ \ \isacommand{by}\isamarkupfalse%
\ {\isacharparenleft}simp\ only{\isacharcolon}\ setSubformulae{\isacharunderscore}or{\isacharparenright}\isanewline
\ \ \isacommand{then}\isamarkupfalse%
\ \isacommand{have}\isamarkupfalse%
\ {\isachardoublequoteopen}G\ {\isasymin}\ {\isacharbraceleft}F{\isadigit{1}}\ \isactrlbold {\isasymor}\ F{\isadigit{2}}{\isacharbraceright}\ {\isasymor}\ G\ {\isasymin}\ setSubformulae\ F{\isadigit{1}}\ {\isasymunion}\ setSubformulae\ F{\isadigit{2}}{\isachardoublequoteclose}\isanewline
\ \ \ \ \isacommand{by}\isamarkupfalse%
\ {\isacharparenleft}simp\ only{\isacharcolon}\ Un{\isacharunderscore}iff{\isacharparenright}\isanewline
\ \ \isacommand{then}\isamarkupfalse%
\ \isacommand{show}\isamarkupfalse%
\ {\isacharquery}thesis\isanewline
\ \ \isacommand{proof}\isamarkupfalse%
\ {\isacharparenleft}rule\ disjE{\isacharparenright}\isanewline
\ \ \ \ \isacommand{assume}\isamarkupfalse%
\ {\isachardoublequoteopen}G\ {\isasymin}\ {\isacharbraceleft}F{\isadigit{1}}\ \isactrlbold {\isasymor}\ F{\isadigit{2}}{\isacharbraceright}{\isachardoublequoteclose}\isanewline
\ \ \ \ \isacommand{then}\isamarkupfalse%
\ \isacommand{have}\isamarkupfalse%
\ {\isachardoublequoteopen}G\ {\isacharequal}\ F{\isadigit{1}}\ \isactrlbold {\isasymor}\ F{\isadigit{2}}{\isachardoublequoteclose}\isanewline
\ \ \ \ \ \ \isacommand{by}\isamarkupfalse%
\ {\isacharparenleft}simp\ only{\isacharcolon}\ singletonD{\isacharparenright}\isanewline
\ \ \ \ \isacommand{then}\isamarkupfalse%
\ \isacommand{show}\isamarkupfalse%
\ {\isacharquery}thesis\isanewline
\ \ \ \ \ \ \isacommand{by}\isamarkupfalse%
\ {\isacharparenleft}simp\ only{\isacharcolon}\ subset{\isacharunderscore}refl{\isacharparenright}\isanewline
\ \ \isacommand{next}\isamarkupfalse%
\isanewline
\ \ \ \ \isacommand{assume}\isamarkupfalse%
\ {\isachardoublequoteopen}G\ {\isasymin}\ setSubformulae\ F{\isadigit{1}}\ {\isasymunion}\ setSubformulae\ F{\isadigit{2}}{\isachardoublequoteclose}\isanewline
\ \ \ \ \isacommand{then}\isamarkupfalse%
\ \isacommand{have}\isamarkupfalse%
\ {\isachardoublequoteopen}G\ {\isasymin}\ setSubformulae\ F{\isadigit{1}}\ {\isasymor}\ G\ {\isasymin}\ setSubformulae\ F{\isadigit{2}}{\isachardoublequoteclose}\ \ \isanewline
\ \ \ \ \ \ \isacommand{by}\isamarkupfalse%
\ {\isacharparenleft}simp\ only{\isacharcolon}\ Un{\isacharunderscore}iff{\isacharparenright}\isanewline
\ \ \ \ \isacommand{then}\isamarkupfalse%
\ \isacommand{show}\isamarkupfalse%
\ {\isacharquery}thesis\isanewline
\ \ \ \ \isacommand{proof}\isamarkupfalse%
\ {\isacharparenleft}rule\ disjE{\isacharparenright}\isanewline
\ \ \ \ \ \ \isacommand{assume}\isamarkupfalse%
\ {\isachardoublequoteopen}G\ {\isasymin}\ setSubformulae\ F{\isadigit{1}}{\isachardoublequoteclose}\isanewline
\ \ \ \ \ \ \isacommand{then}\isamarkupfalse%
\ \isacommand{have}\isamarkupfalse%
\ {\isachardoublequoteopen}setSubformulae\ G\ {\isasymsubseteq}\ setSubformulae\ F{\isadigit{1}}{\isachardoublequoteclose}\isanewline
\ \ \ \ \ \ \ \ \isacommand{by}\isamarkupfalse%
\ {\isacharparenleft}simp\ only{\isacharcolon}\ assms{\isacharparenleft}{\isadigit{1}}{\isacharparenright}{\isacharparenright}\isanewline
\ \ \ \ \ \ \isacommand{also}\isamarkupfalse%
\ \isacommand{have}\isamarkupfalse%
\ {\isachardoublequoteopen}{\isasymdots}\ {\isasymsubseteq}\ setSubformulae\ F{\isadigit{1}}\ {\isasymunion}\ setSubformulae\ F{\isadigit{2}}{\isachardoublequoteclose}\isanewline
\ \ \ \ \ \ \ \ \isacommand{by}\isamarkupfalse%
\ {\isacharparenleft}simp\ only{\isacharcolon}\ Un{\isacharunderscore}upper{\isadigit{1}}{\isacharparenright}\isanewline
\ \ \ \ \ \ \isacommand{also}\isamarkupfalse%
\ \isacommand{have}\isamarkupfalse%
\ {\isachardoublequoteopen}{\isasymdots}\ {\isasymsubseteq}\ setSubformulae\ {\isacharparenleft}F{\isadigit{1}}\ \isactrlbold {\isasymor}\ F{\isadigit{2}}{\isacharparenright}{\isachardoublequoteclose}\isanewline
\ \ \ \ \ \ \ \ \isacommand{by}\isamarkupfalse%
\ {\isacharparenleft}simp\ only{\isacharcolon}\ setSubformulae{\isacharunderscore}or\ Un{\isacharunderscore}upper{\isadigit{2}}{\isacharparenright}\isanewline
\ \ \ \ \ \ \isacommand{finally}\isamarkupfalse%
\ \isacommand{show}\isamarkupfalse%
\ {\isacharquery}thesis\isanewline
\ \ \ \ \ \ \ \ \isacommand{by}\isamarkupfalse%
\ this\isanewline
\ \ \ \ \isacommand{next}\isamarkupfalse%
\isanewline
\ \ \ \ \ \ \isacommand{assume}\isamarkupfalse%
\ {\isachardoublequoteopen}G\ {\isasymin}\ setSubformulae\ F{\isadigit{2}}{\isachardoublequoteclose}\isanewline
\ \ \ \ \ \ \isacommand{then}\isamarkupfalse%
\ \isacommand{have}\isamarkupfalse%
\ {\isachardoublequoteopen}setSubformulae\ G\ {\isasymsubseteq}\ setSubformulae\ F{\isadigit{2}}{\isachardoublequoteclose}\isanewline
\ \ \ \ \ \ \ \ \isacommand{by}\isamarkupfalse%
\ {\isacharparenleft}rule\ assms{\isacharparenleft}{\isadigit{2}}{\isacharparenright}{\isacharparenright}\isanewline
\ \ \ \ \ \ \isacommand{also}\isamarkupfalse%
\ \isacommand{have}\isamarkupfalse%
\ {\isachardoublequoteopen}{\isasymdots}\ {\isasymsubseteq}\ setSubformulae\ F{\isadigit{1}}\ {\isasymunion}\ setSubformulae\ F{\isadigit{2}}{\isachardoublequoteclose}\isanewline
\ \ \ \ \ \ \ \ \isacommand{by}\isamarkupfalse%
\ {\isacharparenleft}simp\ only{\isacharcolon}\ Un{\isacharunderscore}upper{\isadigit{2}}{\isacharparenright}\isanewline
\ \ \ \ \ \ \isacommand{also}\isamarkupfalse%
\ \isacommand{have}\isamarkupfalse%
\ {\isachardoublequoteopen}{\isasymdots}\ {\isasymsubseteq}\ setSubformulae\ {\isacharparenleft}F{\isadigit{1}}\ \isactrlbold {\isasymor}\ F{\isadigit{2}}{\isacharparenright}{\isachardoublequoteclose}\isanewline
\ \ \ \ \ \ \ \ \isacommand{by}\isamarkupfalse%
\ {\isacharparenleft}simp\ only{\isacharcolon}\ setSubformulae{\isacharunderscore}or\ Un{\isacharunderscore}upper{\isadigit{2}}{\isacharparenright}\isanewline
\ \ \ \ \ \ \isacommand{finally}\isamarkupfalse%
\ \isacommand{show}\isamarkupfalse%
\ {\isacharquery}thesis\isanewline
\ \ \ \ \ \ \ \ \isacommand{by}\isamarkupfalse%
\ this\isanewline
\ \ \ \ \isacommand{qed}\isamarkupfalse%
\isanewline
\ \ \isacommand{qed}\isamarkupfalse%
\isanewline
\isacommand{qed}\isamarkupfalse%
%
\endisatagproof
{\isafoldproof}%
%
\isadelimproof
\isanewline
%
\endisadelimproof
\isanewline
\isacommand{lemma}\isamarkupfalse%
\ subContsubformulae{\isacharunderscore}imp{\isacharcolon}\isanewline
\ \ \isakeyword{assumes}\ {\isachardoublequoteopen}G\ {\isasymin}\ setSubformulae\ F{\isadigit{1}}\ \isanewline
\ \ \ \ \ \ \ \ \ \ \ \ {\isasymLongrightarrow}\ setSubformulae\ G\ {\isasymsubseteq}\ setSubformulae\ F{\isadigit{1}}{\isachardoublequoteclose}\isanewline
\ \ \ \ \ \ \ \ \ \ {\isachardoublequoteopen}G\ {\isasymin}\ setSubformulae\ F{\isadigit{2}}\ \isanewline
\ \ \ \ \ \ \ \ \ \ \ \ {\isasymLongrightarrow}\ setSubformulae\ G\ {\isasymsubseteq}\ setSubformulae\ F{\isadigit{2}}{\isachardoublequoteclose}\isanewline
\ \ \ \ \ \ \ \ \ \ {\isachardoublequoteopen}G\ {\isasymin}\ setSubformulae\ {\isacharparenleft}F{\isadigit{1}}\ \isactrlbold {\isasymrightarrow}\ F{\isadigit{2}}{\isacharparenright}{\isachardoublequoteclose}\isanewline
\ \ \isakeyword{shows}\ \ \ {\isachardoublequoteopen}setSubformulae\ G\ {\isasymsubseteq}\ setSubformulae\ {\isacharparenleft}F{\isadigit{1}}\ \isactrlbold {\isasymrightarrow}\ F{\isadigit{2}}{\isacharparenright}{\isachardoublequoteclose}\isanewline
%
\isadelimproof
%
\endisadelimproof
%
\isatagproof
\isacommand{proof}\isamarkupfalse%
\ {\isacharminus}\isanewline
\ \ \isacommand{have}\isamarkupfalse%
\ {\isachardoublequoteopen}G\ {\isasymin}\ {\isacharbraceleft}F{\isadigit{1}}\ \isactrlbold {\isasymrightarrow}\ F{\isadigit{2}}{\isacharbraceright}\ {\isasymunion}\ {\isacharparenleft}setSubformulae\ F{\isadigit{1}}\ {\isasymunion}\ setSubformulae\ F{\isadigit{2}}{\isacharparenright}{\isachardoublequoteclose}\isanewline
\ \ \ \ \isacommand{using}\isamarkupfalse%
\ assms{\isacharparenleft}{\isadigit{3}}{\isacharparenright}\ \isanewline
\ \ \ \ \isacommand{by}\isamarkupfalse%
\ {\isacharparenleft}simp\ only{\isacharcolon}\ setSubformulae{\isacharunderscore}imp{\isacharparenright}\isanewline
\ \ \isacommand{then}\isamarkupfalse%
\ \isacommand{have}\isamarkupfalse%
\ {\isachardoublequoteopen}G\ {\isasymin}\ {\isacharbraceleft}F{\isadigit{1}}\ \isactrlbold {\isasymrightarrow}\ F{\isadigit{2}}{\isacharbraceright}\ {\isasymor}\ G\ {\isasymin}\ setSubformulae\ F{\isadigit{1}}\ {\isasymunion}\ setSubformulae\ F{\isadigit{2}}{\isachardoublequoteclose}\isanewline
\ \ \ \ \isacommand{by}\isamarkupfalse%
\ {\isacharparenleft}simp\ only{\isacharcolon}\ Un{\isacharunderscore}iff{\isacharparenright}\isanewline
\ \ \isacommand{then}\isamarkupfalse%
\ \isacommand{show}\isamarkupfalse%
\ {\isacharquery}thesis\isanewline
\ \ \isacommand{proof}\isamarkupfalse%
\ {\isacharparenleft}rule\ disjE{\isacharparenright}\isanewline
\ \ \ \ \isacommand{assume}\isamarkupfalse%
\ {\isachardoublequoteopen}G\ {\isasymin}\ {\isacharbraceleft}F{\isadigit{1}}\ \isactrlbold {\isasymrightarrow}\ F{\isadigit{2}}{\isacharbraceright}{\isachardoublequoteclose}\isanewline
\ \ \ \ \isacommand{then}\isamarkupfalse%
\ \isacommand{have}\isamarkupfalse%
\ {\isachardoublequoteopen}G\ {\isacharequal}\ F{\isadigit{1}}\ \isactrlbold {\isasymrightarrow}\ F{\isadigit{2}}{\isachardoublequoteclose}\isanewline
\ \ \ \ \ \ \isacommand{by}\isamarkupfalse%
\ {\isacharparenleft}simp\ only{\isacharcolon}\ singletonD{\isacharparenright}\isanewline
\ \ \ \ \isacommand{then}\isamarkupfalse%
\ \isacommand{show}\isamarkupfalse%
\ {\isacharquery}thesis\isanewline
\ \ \ \ \ \ \isacommand{by}\isamarkupfalse%
\ {\isacharparenleft}simp\ only{\isacharcolon}\ subset{\isacharunderscore}refl{\isacharparenright}\isanewline
\ \ \isacommand{next}\isamarkupfalse%
\isanewline
\ \ \ \ \isacommand{assume}\isamarkupfalse%
\ {\isachardoublequoteopen}G\ {\isasymin}\ setSubformulae\ F{\isadigit{1}}\ {\isasymunion}\ setSubformulae\ F{\isadigit{2}}{\isachardoublequoteclose}\isanewline
\ \ \ \ \isacommand{then}\isamarkupfalse%
\ \isacommand{have}\isamarkupfalse%
\ {\isachardoublequoteopen}G\ {\isasymin}\ setSubformulae\ F{\isadigit{1}}\ {\isasymor}\ G\ {\isasymin}\ setSubformulae\ F{\isadigit{2}}{\isachardoublequoteclose}\ \ \isanewline
\ \ \ \ \ \ \isacommand{by}\isamarkupfalse%
\ {\isacharparenleft}simp\ only{\isacharcolon}\ Un{\isacharunderscore}iff{\isacharparenright}\isanewline
\ \ \ \ \isacommand{then}\isamarkupfalse%
\ \isacommand{show}\isamarkupfalse%
\ {\isacharquery}thesis\isanewline
\ \ \ \ \isacommand{proof}\isamarkupfalse%
\ {\isacharparenleft}rule\ disjE{\isacharparenright}\isanewline
\ \ \ \ \ \ \isacommand{assume}\isamarkupfalse%
\ {\isachardoublequoteopen}G\ {\isasymin}\ setSubformulae\ F{\isadigit{1}}{\isachardoublequoteclose}\isanewline
\ \ \ \ \ \ \isacommand{then}\isamarkupfalse%
\ \isacommand{have}\isamarkupfalse%
\ {\isachardoublequoteopen}setSubformulae\ G\ {\isasymsubseteq}\ setSubformulae\ F{\isadigit{1}}{\isachardoublequoteclose}\isanewline
\ \ \ \ \ \ \ \ \isacommand{by}\isamarkupfalse%
\ {\isacharparenleft}simp\ only{\isacharcolon}\ assms{\isacharparenleft}{\isadigit{1}}{\isacharparenright}{\isacharparenright}\isanewline
\ \ \ \ \ \ \isacommand{also}\isamarkupfalse%
\ \isacommand{have}\isamarkupfalse%
\ {\isachardoublequoteopen}{\isasymdots}\ {\isasymsubseteq}\ setSubformulae\ F{\isadigit{1}}\ {\isasymunion}\ setSubformulae\ F{\isadigit{2}}{\isachardoublequoteclose}\isanewline
\ \ \ \ \ \ \ \ \isacommand{by}\isamarkupfalse%
\ {\isacharparenleft}simp\ only{\isacharcolon}\ Un{\isacharunderscore}upper{\isadigit{1}}{\isacharparenright}\isanewline
\ \ \ \ \ \ \isacommand{also}\isamarkupfalse%
\ \isacommand{have}\isamarkupfalse%
\ {\isachardoublequoteopen}{\isasymdots}\ {\isasymsubseteq}\ setSubformulae\ {\isacharparenleft}F{\isadigit{1}}\ \isactrlbold {\isasymrightarrow}\ F{\isadigit{2}}{\isacharparenright}{\isachardoublequoteclose}\isanewline
\ \ \ \ \ \ \ \ \isacommand{by}\isamarkupfalse%
\ {\isacharparenleft}simp\ only{\isacharcolon}\ setSubformulae{\isacharunderscore}imp\ Un{\isacharunderscore}upper{\isadigit{2}}{\isacharparenright}\isanewline
\ \ \ \ \ \ \isacommand{finally}\isamarkupfalse%
\ \isacommand{show}\isamarkupfalse%
\ {\isacharquery}thesis\isanewline
\ \ \ \ \ \ \ \ \isacommand{by}\isamarkupfalse%
\ this\isanewline
\ \ \ \ \isacommand{next}\isamarkupfalse%
\isanewline
\ \ \ \ \ \ \isacommand{assume}\isamarkupfalse%
\ {\isachardoublequoteopen}G\ {\isasymin}\ setSubformulae\ F{\isadigit{2}}{\isachardoublequoteclose}\isanewline
\ \ \ \ \ \ \isacommand{then}\isamarkupfalse%
\ \isacommand{have}\isamarkupfalse%
\ {\isachardoublequoteopen}setSubformulae\ G\ {\isasymsubseteq}\ setSubformulae\ F{\isadigit{2}}{\isachardoublequoteclose}\isanewline
\ \ \ \ \ \ \ \ \isacommand{by}\isamarkupfalse%
\ {\isacharparenleft}rule\ assms{\isacharparenleft}{\isadigit{2}}{\isacharparenright}{\isacharparenright}\isanewline
\ \ \ \ \ \ \isacommand{also}\isamarkupfalse%
\ \isacommand{have}\isamarkupfalse%
\ {\isachardoublequoteopen}{\isasymdots}\ {\isasymsubseteq}\ setSubformulae\ F{\isadigit{1}}\ {\isasymunion}\ setSubformulae\ F{\isadigit{2}}{\isachardoublequoteclose}\isanewline
\ \ \ \ \ \ \ \ \isacommand{by}\isamarkupfalse%
\ {\isacharparenleft}simp\ only{\isacharcolon}\ Un{\isacharunderscore}upper{\isadigit{2}}{\isacharparenright}\isanewline
\ \ \ \ \ \ \isacommand{also}\isamarkupfalse%
\ \isacommand{have}\isamarkupfalse%
\ {\isachardoublequoteopen}{\isasymdots}\ {\isasymsubseteq}\ setSubformulae\ {\isacharparenleft}F{\isadigit{1}}\ \isactrlbold {\isasymrightarrow}\ F{\isadigit{2}}{\isacharparenright}{\isachardoublequoteclose}\isanewline
\ \ \ \ \ \ \ \ \isacommand{by}\isamarkupfalse%
\ {\isacharparenleft}simp\ only{\isacharcolon}\ setSubformulae{\isacharunderscore}imp\ Un{\isacharunderscore}upper{\isadigit{2}}{\isacharparenright}\isanewline
\ \ \ \ \ \ \isacommand{finally}\isamarkupfalse%
\ \isacommand{show}\isamarkupfalse%
\ {\isacharquery}thesis\isanewline
\ \ \ \ \ \ \ \ \isacommand{by}\isamarkupfalse%
\ this\isanewline
\ \ \ \ \isacommand{qed}\isamarkupfalse%
\isanewline
\ \ \isacommand{qed}\isamarkupfalse%
\isanewline
\isacommand{qed}\isamarkupfalse%
%
\endisatagproof
{\isafoldproof}%
%
\isadelimproof
\isanewline
%
\endisadelimproof
\isanewline
\isacommand{lemma}\isamarkupfalse%
\isanewline
\ \ {\isachardoublequoteopen}G\ {\isasymin}\ setSubformulae\ F\ {\isasymLongrightarrow}\ setSubformulae\ G\ {\isasymsubseteq}\ setSubformulae\ F{\isachardoublequoteclose}\isanewline
%
\isadelimproof
%
\endisadelimproof
%
\isatagproof
\isacommand{proof}\isamarkupfalse%
\ {\isacharparenleft}induction\ F{\isacharparenright}\isanewline
\ \ \isacommand{case}\isamarkupfalse%
\ {\isacharparenleft}Atom\ x{\isacharparenright}\isanewline
\ \ \isacommand{then}\isamarkupfalse%
\ \isacommand{show}\isamarkupfalse%
\ {\isacharquery}case\ \isacommand{by}\isamarkupfalse%
\ {\isacharparenleft}rule\ subContsubformulae{\isacharunderscore}atom{\isacharparenright}\isanewline
\isacommand{next}\isamarkupfalse%
\isanewline
\ \ \isacommand{case}\isamarkupfalse%
\ Bot\isanewline
\ \ \isacommand{then}\isamarkupfalse%
\ \isacommand{show}\isamarkupfalse%
\ {\isacharquery}case\ \isacommand{by}\isamarkupfalse%
\ {\isacharparenleft}rule\ subContsubformulae{\isacharunderscore}bot{\isacharparenright}\isanewline
\isacommand{next}\isamarkupfalse%
\isanewline
\ \ \isacommand{case}\isamarkupfalse%
\ {\isacharparenleft}Not\ F{\isacharparenright}\isanewline
\ \ \isacommand{then}\isamarkupfalse%
\ \isacommand{show}\isamarkupfalse%
\ {\isacharquery}case\ \isacommand{by}\isamarkupfalse%
\ {\isacharparenleft}rule\ subContsubformulae{\isacharunderscore}not{\isacharparenright}\isanewline
\isacommand{next}\isamarkupfalse%
\isanewline
\ \ \isacommand{case}\isamarkupfalse%
\ {\isacharparenleft}And\ F{\isadigit{1}}\ F{\isadigit{2}}{\isacharparenright}\isanewline
\ \ \isacommand{then}\isamarkupfalse%
\ \isacommand{show}\isamarkupfalse%
\ {\isacharquery}case\ \isacommand{by}\isamarkupfalse%
\ {\isacharparenleft}rule\ subContsubformulae{\isacharunderscore}and{\isacharparenright}\isanewline
\isacommand{next}\isamarkupfalse%
\isanewline
\ \ \isacommand{case}\isamarkupfalse%
\ {\isacharparenleft}Or\ F{\isadigit{1}}\ F{\isadigit{2}}{\isacharparenright}\isanewline
\ \ \isacommand{then}\isamarkupfalse%
\ \isacommand{show}\isamarkupfalse%
\ {\isacharquery}case\ \isacommand{by}\isamarkupfalse%
\ {\isacharparenleft}rule\ subContsubformulae{\isacharunderscore}or{\isacharparenright}\isanewline
\isacommand{next}\isamarkupfalse%
\isanewline
\ \ \isacommand{case}\isamarkupfalse%
\ {\isacharparenleft}Imp\ F{\isadigit{1}}\ F{\isadigit{2}}{\isacharparenright}\isanewline
\ \ \isacommand{then}\isamarkupfalse%
\ \isacommand{show}\isamarkupfalse%
\ {\isacharquery}case\ \isacommand{by}\isamarkupfalse%
\ {\isacharparenleft}rule\ subContsubformulae{\isacharunderscore}imp{\isacharparenright}\isanewline
\isacommand{qed}\isamarkupfalse%
%
\endisatagproof
{\isafoldproof}%
%
\isadelimproof
%
\endisadelimproof
%
\begin{isamarkuptext}%
Finalmente, su demostración automática se muestra a continuación.%
\end{isamarkuptext}\isamarkuptrue%
\isacommand{lemma}\isamarkupfalse%
\ subContsubformulae{\isacharcolon}\isanewline
\ \ {\isachardoublequoteopen}G\ {\isasymin}\ setSubformulae\ F\ {\isasymLongrightarrow}\ setSubformulae\ G\ {\isasymsubseteq}\ setSubformulae\ F{\isachardoublequoteclose}\isanewline
%
\isadelimproof
\ \ %
\endisadelimproof
%
\isatagproof
\isacommand{by}\isamarkupfalse%
\ {\isacharparenleft}induction\ F{\isacharparenright}\ auto%
\endisatagproof
{\isafoldproof}%
%
\isadelimproof
%
\endisadelimproof
%
\begin{isamarkuptext}%
El siguiente lema nos da la noción de transitividad de contención 
  en cadena de las subfórmulas, de modo que la subfórmula de una 
  subfórmula es del mismo modo subfórmula de la mayor.

  \begin{lema}
    Sea \isa{H} una subfórmula de \isa{G} que es a su vez subfórmula de \isa{F}, 
    entonces \isa{H} es subfórmula de \isa{F}.
  \end{lema}

  \begin{demostracion}
  La prueba está basada en el lema anterior. Hemos demostrado que si 
  \isa{H} es subfórmula de \isa{G}, entonces el conjunto de subfórmulas de \isa{H} 
  está contenido en el conjunto de subfórmulas de \isa{G}. Del mismo modo, 
  como \isa{G} es subfórmula de \isa{F}, su conjunto de subfórmulas está 
  contenido en el conjunto de subfórmulas de \isa{F}. Por la
  transitividad de la contención, tenemos que el conjunto de subfórmulas
  de \isa{H} está contenido en el de \isa{F}. Por otro lema anterior, 
  como \isa{H} es subfórmula de ella misma, es decir, pertenece a su 
  conjunto de subfórmulas, por la contención anterior se verifica que
  pertenece al conjunto de subfórmulas de \isa{F} como queríamos demostrar. 
  \end{demostracion}

  Veamos su formalización y prueba estructurada en Isabelle.%
\end{isamarkuptext}\isamarkuptrue%
\isacommand{lemma}\isamarkupfalse%
\isanewline
\ \ \isakeyword{assumes}\ {\isachardoublequoteopen}G\ {\isasymin}\ setSubformulae\ F{\isachardoublequoteclose}\ \isanewline
\ \ \ \ \ \ \ \ \ \ {\isachardoublequoteopen}H\ {\isasymin}\ setSubformulae\ G{\isachardoublequoteclose}\isanewline
\ \ \isakeyword{shows}\ \ \ {\isachardoublequoteopen}H\ {\isasymin}\ setSubformulae\ F{\isachardoublequoteclose}\isanewline
%
\isadelimproof
%
\endisadelimproof
%
\isatagproof
\isacommand{proof}\isamarkupfalse%
\ {\isacharminus}\isanewline
\ \ \isacommand{have}\isamarkupfalse%
\ {\isadigit{1}}{\isacharcolon}\ {\isachardoublequoteopen}setSubformulae\ G\ {\isasymsubseteq}\ setSubformulae\ F{\isachardoublequoteclose}\ \isanewline
\ \ \ \ \isacommand{using}\isamarkupfalse%
\ assms{\isacharparenleft}{\isadigit{1}}{\isacharparenright}\ \isanewline
\ \ \ \ \isacommand{by}\isamarkupfalse%
\ {\isacharparenleft}rule\ subContsubformulae{\isacharparenright}\isanewline
\ \ \isacommand{have}\isamarkupfalse%
\ {\isachardoublequoteopen}setSubformulae\ H\ {\isasymsubseteq}\ setSubformulae\ G{\isachardoublequoteclose}\ \isanewline
\ \ \ \ \isacommand{using}\isamarkupfalse%
\ assms{\isacharparenleft}{\isadigit{2}}{\isacharparenright}\ \isanewline
\ \ \ \ \isacommand{by}\isamarkupfalse%
\ {\isacharparenleft}rule\ subContsubformulae{\isacharparenright}\isanewline
\ \ \isacommand{then}\isamarkupfalse%
\ \isacommand{have}\isamarkupfalse%
\ {\isadigit{2}}{\isacharcolon}\ {\isachardoublequoteopen}setSubformulae\ H\ {\isasymsubseteq}\ setSubformulae\ F{\isachardoublequoteclose}\ \isanewline
\ \ \ \ \isacommand{using}\isamarkupfalse%
\ {\isadigit{1}}\ \isanewline
\ \ \ \ \isacommand{by}\isamarkupfalse%
\ {\isacharparenleft}rule\ subset{\isacharunderscore}trans{\isacharparenright}\isanewline
\ \ \isacommand{have}\isamarkupfalse%
\ {\isachardoublequoteopen}H\ {\isasymin}\ setSubformulae\ H{\isachardoublequoteclose}\ \isanewline
\ \ \ \ \isacommand{by}\isamarkupfalse%
\ {\isacharparenleft}simp\ only{\isacharcolon}\ subformulae{\isacharunderscore}self{\isacharparenright}\isanewline
\ \ \isacommand{then}\isamarkupfalse%
\ \isacommand{show}\isamarkupfalse%
\ {\isachardoublequoteopen}H\ {\isasymin}\ setSubformulae\ F{\isachardoublequoteclose}\ \isanewline
\ \ \ \ \isacommand{using}\isamarkupfalse%
\ {\isadigit{2}}\ \isanewline
\ \ \ \ \isacommand{by}\isamarkupfalse%
\ {\isacharparenleft}rule\ rev{\isacharunderscore}subsetD{\isacharparenright}\isanewline
\isacommand{qed}\isamarkupfalse%
%
\endisatagproof
{\isafoldproof}%
%
\isadelimproof
%
\endisadelimproof
%
\begin{isamarkuptext}%
A continuación su demostración automática.%
\end{isamarkuptext}\isamarkuptrue%
\isacommand{lemma}\isamarkupfalse%
\ subsubformulae{\isacharcolon}\ \isanewline
\ \ {\isachardoublequoteopen}G\ {\isasymin}\ setSubformulae\ F\ \isanewline
\ \ \ {\isasymLongrightarrow}\ H\ {\isasymin}\ setSubformulae\ G\ \isanewline
\ \ \ {\isasymLongrightarrow}\ H\ {\isasymin}\ setSubformulae\ F{\isachardoublequoteclose}\isanewline
%
\isadelimproof
\ \ %
\endisadelimproof
%
\isatagproof
\isacommand{by}\isamarkupfalse%
\ {\isacharparenleft}drule\ subContsubformulae{\isacharcomma}\ erule\ subsetD{\isacharparenright}%
\endisatagproof
{\isafoldproof}%
%
\isadelimproof
%
\endisadelimproof
%
\begin{isamarkuptext}%
Presentemos ahora otro resultado que relaciona las conectivas
  con los conjuntos de subfórmulas.%
\end{isamarkuptext}\isamarkuptrue%
%
\begin{isamarkuptext}%
Para la demostración en Isabelle, probaremos cada caso de forma
 independiente.%
\end{isamarkuptext}\isamarkuptrue%
\isacommand{lemma}\isamarkupfalse%
\ subformulas{\isacharunderscore}in{\isacharunderscore}subformulas{\isacharunderscore}not{\isacharcolon}\isanewline
\ \ \isakeyword{assumes}\ {\isachardoublequoteopen}\isactrlbold {\isasymnot}\ G\ {\isasymin}\ setSubformulae\ F{\isachardoublequoteclose}\isanewline
\ \ \isakeyword{shows}\ {\isachardoublequoteopen}G\ {\isasymin}\ setSubformulae\ F{\isachardoublequoteclose}\isanewline
%
\isadelimproof
%
\endisadelimproof
%
\isatagproof
\isacommand{proof}\isamarkupfalse%
\ {\isacharminus}\isanewline
\ \ \isacommand{have}\isamarkupfalse%
\ {\isachardoublequoteopen}G\ {\isasymin}\ setSubformulae\ G{\isachardoublequoteclose}\isanewline
\ \ \ \ \isacommand{by}\isamarkupfalse%
\ {\isacharparenleft}simp\ only{\isacharcolon}\ subformulae{\isacharunderscore}self{\isacharparenright}\isanewline
\ \ \isacommand{then}\isamarkupfalse%
\ \isacommand{have}\isamarkupfalse%
\ {\isachardoublequoteopen}G\ {\isasymin}\ {\isacharbraceleft}\isactrlbold {\isasymnot}\ G{\isacharbraceright}\ {\isasymunion}\ setSubformulae\ G{\isachardoublequoteclose}\isanewline
\ \ \ \ \isacommand{by}\isamarkupfalse%
\ {\isacharparenleft}simp\ only{\isacharcolon}\ UnI{\isadigit{2}}{\isacharparenright}\isanewline
\ \ \isacommand{then}\isamarkupfalse%
\ \isacommand{have}\isamarkupfalse%
\ {\isadigit{1}}{\isacharcolon}{\isachardoublequoteopen}G\ {\isasymin}\ setSubformulae\ {\isacharparenleft}\isactrlbold {\isasymnot}\ G{\isacharparenright}{\isachardoublequoteclose}\ \isanewline
\ \ \ \ \isacommand{by}\isamarkupfalse%
\ {\isacharparenleft}simp\ only{\isacharcolon}\ setSubformulae{\isacharunderscore}not{\isacharparenright}\isanewline
\ \ \isacommand{show}\isamarkupfalse%
\ {\isachardoublequoteopen}G\ {\isasymin}\ setSubformulae\ F{\isachardoublequoteclose}\ \isacommand{using}\isamarkupfalse%
\ assms\ {\isadigit{1}}\ \isanewline
\ \ \ \ \isacommand{by}\isamarkupfalse%
\ {\isacharparenleft}rule\ subsubformulae{\isacharparenright}\isanewline
\isacommand{qed}\isamarkupfalse%
%
\endisatagproof
{\isafoldproof}%
%
\isadelimproof
\isanewline
%
\endisadelimproof
\isanewline
\isacommand{lemma}\isamarkupfalse%
\ subformulas{\isacharunderscore}in{\isacharunderscore}subformulas{\isacharunderscore}and{\isacharcolon}\isanewline
\ \ \isakeyword{assumes}\ {\isachardoublequoteopen}G\ \isactrlbold {\isasymand}\ H\ {\isasymin}\ setSubformulae\ F{\isachardoublequoteclose}\ \isanewline
\ \ \isakeyword{shows}\ {\isachardoublequoteopen}G\ {\isasymin}\ setSubformulae\ F\ {\isasymand}\ H\ {\isasymin}\ setSubformulae\ F{\isachardoublequoteclose}\isanewline
%
\isadelimproof
%
\endisadelimproof
%
\isatagproof
\isacommand{proof}\isamarkupfalse%
\ {\isacharparenleft}rule\ conjI{\isacharparenright}\isanewline
\ \ \isacommand{have}\isamarkupfalse%
\ {\isachardoublequoteopen}G\ {\isasymin}\ setSubformulae\ {\isacharparenleft}G\ \isactrlbold {\isasymand}\ H{\isacharparenright}{\isachardoublequoteclose}\ \isanewline
\ \ \ \ \isacommand{by}\isamarkupfalse%
\ {\isacharparenleft}simp\ only{\isacharcolon}\ subformulae{\isacharunderscore}self\ UnI{\isadigit{2}}\ UnI{\isadigit{1}}\ setSubformulae{\isacharunderscore}and{\isacharparenright}\isanewline
\ \ \isacommand{with}\isamarkupfalse%
\ assms\ \isacommand{show}\isamarkupfalse%
\ {\isachardoublequoteopen}G\ {\isasymin}\ setSubformulae\ F{\isachardoublequoteclose}\ \isanewline
\ \ \ \ \isacommand{by}\isamarkupfalse%
\ {\isacharparenleft}rule\ subsubformulae{\isacharparenright}\isanewline
\isacommand{next}\isamarkupfalse%
\isanewline
\ \ \isacommand{have}\isamarkupfalse%
\ {\isachardoublequoteopen}H\ {\isasymin}\ setSubformulae\ {\isacharparenleft}G\ \isactrlbold {\isasymand}\ H{\isacharparenright}{\isachardoublequoteclose}\ \ \isanewline
\ \ \ \ \isacommand{by}\isamarkupfalse%
\ {\isacharparenleft}simp\ only{\isacharcolon}\ subformulae{\isacharunderscore}self\ UnI{\isadigit{2}}\ UnI{\isadigit{1}}\ setSubformulae{\isacharunderscore}and{\isacharparenright}\isanewline
\ \ \isacommand{with}\isamarkupfalse%
\ assms\ \isacommand{show}\isamarkupfalse%
\ {\isachardoublequoteopen}H\ {\isasymin}\ setSubformulae\ F{\isachardoublequoteclose}\ \isanewline
\ \ \ \ \isacommand{by}\isamarkupfalse%
\ {\isacharparenleft}rule\ subsubformulae{\isacharparenright}\isanewline
\isacommand{qed}\isamarkupfalse%
%
\endisatagproof
{\isafoldproof}%
%
\isadelimproof
\isanewline
%
\endisadelimproof
\isanewline
\isacommand{lemma}\isamarkupfalse%
\ subformulas{\isacharunderscore}in{\isacharunderscore}subformulas{\isacharunderscore}or{\isacharcolon}\isanewline
\ \ \isakeyword{assumes}\ {\isachardoublequoteopen}G\ \isactrlbold {\isasymor}\ H\ {\isasymin}\ setSubformulae\ F{\isachardoublequoteclose}\ \isanewline
\ \ \isakeyword{shows}\ {\isachardoublequoteopen}G\ {\isasymin}\ setSubformulae\ F\ {\isasymand}\ H\ {\isasymin}\ setSubformulae\ F{\isachardoublequoteclose}\isanewline
%
\isadelimproof
%
\endisadelimproof
%
\isatagproof
\isacommand{proof}\isamarkupfalse%
\ {\isacharparenleft}rule\ conjI{\isacharparenright}\isanewline
\ \ \isacommand{have}\isamarkupfalse%
\ {\isachardoublequoteopen}G\ {\isasymin}\ setSubformulae\ {\isacharparenleft}G\ \isactrlbold {\isasymor}\ H{\isacharparenright}{\isachardoublequoteclose}\ \isanewline
\ \ \ \ \isacommand{by}\isamarkupfalse%
\ {\isacharparenleft}simp\ only{\isacharcolon}\ subformulae{\isacharunderscore}self\ UnI{\isadigit{2}}\ UnI{\isadigit{1}}\ setSubformulae{\isacharunderscore}or{\isacharparenright}\isanewline
\ \ \isacommand{with}\isamarkupfalse%
\ assms\ \isacommand{show}\isamarkupfalse%
\ {\isachardoublequoteopen}G\ {\isasymin}\ setSubformulae\ F{\isachardoublequoteclose}\ \isanewline
\ \ \ \ \isacommand{by}\isamarkupfalse%
\ {\isacharparenleft}rule\ subsubformulae{\isacharparenright}\isanewline
\isacommand{next}\isamarkupfalse%
\isanewline
\ \ \isacommand{have}\isamarkupfalse%
\ {\isachardoublequoteopen}H\ {\isasymin}\ setSubformulae\ {\isacharparenleft}G\ \isactrlbold {\isasymor}\ H{\isacharparenright}{\isachardoublequoteclose}\ \ \isanewline
\ \ \ \ \isacommand{by}\isamarkupfalse%
\ {\isacharparenleft}simp\ only{\isacharcolon}\ subformulae{\isacharunderscore}self\ UnI{\isadigit{2}}\ UnI{\isadigit{1}}\ setSubformulae{\isacharunderscore}or{\isacharparenright}\isanewline
\ \ \isacommand{with}\isamarkupfalse%
\ assms\ \isacommand{show}\isamarkupfalse%
\ {\isachardoublequoteopen}H\ {\isasymin}\ setSubformulae\ F{\isachardoublequoteclose}\ \isanewline
\ \ \ \ \isacommand{by}\isamarkupfalse%
\ {\isacharparenleft}rule\ subsubformulae{\isacharparenright}\isanewline
\isacommand{qed}\isamarkupfalse%
%
\endisatagproof
{\isafoldproof}%
%
\isadelimproof
\isanewline
%
\endisadelimproof
\isanewline
\isacommand{lemma}\isamarkupfalse%
\ subformulas{\isacharunderscore}in{\isacharunderscore}subformulas{\isacharunderscore}imp{\isacharcolon}\isanewline
\ \ \isakeyword{assumes}\ {\isachardoublequoteopen}G\ \isactrlbold {\isasymrightarrow}\ H\ {\isasymin}\ setSubformulae\ F{\isachardoublequoteclose}\ \isanewline
\ \ \isakeyword{shows}\ {\isachardoublequoteopen}G\ {\isasymin}\ setSubformulae\ F\ {\isasymand}\ H\ {\isasymin}\ setSubformulae\ F{\isachardoublequoteclose}\isanewline
%
\isadelimproof
%
\endisadelimproof
%
\isatagproof
\isacommand{proof}\isamarkupfalse%
\ {\isacharparenleft}rule\ conjI{\isacharparenright}\isanewline
\ \ \isacommand{have}\isamarkupfalse%
\ {\isachardoublequoteopen}G\ {\isasymin}\ setSubformulae\ {\isacharparenleft}G\ \isactrlbold {\isasymrightarrow}\ H{\isacharparenright}{\isachardoublequoteclose}\ \isanewline
\ \ \ \ \isacommand{by}\isamarkupfalse%
\ {\isacharparenleft}simp\ only{\isacharcolon}\ subformulae{\isacharunderscore}self\ UnI{\isadigit{2}}\ UnI{\isadigit{1}}\ setSubformulae{\isacharunderscore}imp{\isacharparenright}\isanewline
\ \ \isacommand{with}\isamarkupfalse%
\ assms\ \isacommand{show}\isamarkupfalse%
\ {\isachardoublequoteopen}G\ {\isasymin}\ setSubformulae\ F{\isachardoublequoteclose}\ \isanewline
\ \ \ \ \isacommand{by}\isamarkupfalse%
\ {\isacharparenleft}rule\ subsubformulae{\isacharparenright}\isanewline
\isacommand{next}\isamarkupfalse%
\isanewline
\ \ \isacommand{have}\isamarkupfalse%
\ {\isachardoublequoteopen}H\ {\isasymin}\ setSubformulae\ {\isacharparenleft}G\ \isactrlbold {\isasymrightarrow}\ H{\isacharparenright}{\isachardoublequoteclose}\ \ \isanewline
\ \ \ \ \isacommand{by}\isamarkupfalse%
\ {\isacharparenleft}simp\ only{\isacharcolon}\ subformulae{\isacharunderscore}self\ UnI{\isadigit{2}}\ UnI{\isadigit{1}}\ setSubformulae{\isacharunderscore}imp{\isacharparenright}\isanewline
\ \ \isacommand{with}\isamarkupfalse%
\ assms\ \isacommand{show}\isamarkupfalse%
\ {\isachardoublequoteopen}H\ {\isasymin}\ setSubformulae\ F{\isachardoublequoteclose}\ \isanewline
\ \ \ \ \isacommand{by}\isamarkupfalse%
\ {\isacharparenleft}rule\ subsubformulae{\isacharparenright}\isanewline
\isacommand{qed}\isamarkupfalse%
%
\endisatagproof
{\isafoldproof}%
%
\isadelimproof
\isanewline
%
\endisadelimproof
\isanewline
\isacommand{lemmas}\isamarkupfalse%
\ subformulas{\isacharunderscore}in{\isacharunderscore}subformulas\ {\isacharequal}\isanewline
\ \ subformulas{\isacharunderscore}in{\isacharunderscore}subformulas{\isacharunderscore}and\isanewline
\ \ subformulas{\isacharunderscore}in{\isacharunderscore}subformulas{\isacharunderscore}or\isanewline
\ \ subformulas{\isacharunderscore}in{\isacharunderscore}subformulas{\isacharunderscore}imp\isanewline
\ \ subformulas{\isacharunderscore}in{\isacharunderscore}subformulas{\isacharunderscore}not%
\isadelimdocument
%
\endisadelimdocument
%
\isatagdocument
%
\isamarkupsection{Conectivas generalizadas%
}
\isamarkuptrue%
%
\endisatagdocument
{\isafolddocument}%
%
\isadelimdocument
%
\endisadelimdocument
%
\begin{isamarkuptext}%
En esta sección definiremos nuevas conectivas y fórmulas a partir 
  de las ya definidas en el apartado anterior, junto con varios 
  resultados sobre las mismas. Veamos el primero.

  \begin{definicion}
    Se define la fórmula \isa{{\isasymtop}} como la implicación \isa{{\isasymbottom}\ {\isasymlongrightarrow}\ {\isasymbottom}}.
  \end{definicion}

  Se formaliza del siguiente modo.%
\end{isamarkuptext}\isamarkuptrue%
\isacommand{definition}\isamarkupfalse%
\ Top\ {\isacharparenleft}{\isachardoublequoteopen}{\isasymtop}{\isachardoublequoteclose}{\isacharparenright}\ \isakeyword{where}\isanewline
\ \ {\isachardoublequoteopen}{\isasymtop}\ {\isasymequiv}\ {\isasymbottom}\ \isactrlbold {\isasymrightarrow}\ {\isasymbottom}{\isachardoublequoteclose}%
\begin{isamarkuptext}%
Como podemos observar, se define mediante una relación de 
  equivalencia con otra fórmula ya conocida. El uso de dicha 
  equivalencia justifica el tipo \isa{definition} empleado en este 
  caso. 

  Por la propia definición, es claro que \isa{{\isasymtop}} no contiene ninguna
  variable proposicional, como se verifica a continuación en Isabelle.%
\end{isamarkuptext}\isamarkuptrue%
\isacommand{lemma}\isamarkupfalse%
\ {\isachardoublequoteopen}atoms\ {\isasymtop}\ {\isacharequal}\ {\isasymemptyset}{\isachardoublequoteclose}\isanewline
%
\isadelimproof
\ \ \ %
\endisadelimproof
%
\isatagproof
\isacommand{by}\isamarkupfalse%
\ {\isacharparenleft}simp\ only{\isacharcolon}\ Top{\isacharunderscore}def\ formula{\isachardot}set\ Un{\isacharunderscore}absorb{\isacharparenright}%
\endisatagproof
{\isafoldproof}%
%
\isadelimproof
%
\endisadelimproof
%
\begin{isamarkuptext}%
A continuación vamos a definir dos conectivas que generalizan la 
  conjunción y la disyunción para una lista finita de fórmulas. 

  En Isabelle está predefinido el tipo listas de la siguiente manera:

  \begin{definicion}
    Las listas de un tipo de elemento cualquiera se definen
    recursivamente como sigue.
    \begin{itemize}
      \item[] La lista vacía es una lista.
      \item[] Sea \isa{x} un elemento, y \isa{xs} una lista de elementos de su
      mismo tipo. Entonces, \isa{x{\isacharhash}xs} es una lista.
    \end{itemize}
  \end{definicion}

  La conjunción y disyunción generalizadas se definen sobre listas de
  fórmulas de manera recursiva:

  \begin{definicion}
  La conjunción generalizada de una lista de fórmulas se define 
  recursivamente como:
    \begin{itemize}
      \item La conjunción generalizada de la lista vacía es \isa{{\isasymnot}{\isasymbottom}}.
      \item Sea \isa{F} una fórmula y \isa{Fs} una lista de fórmulas. Entonces,
  la conjunción generalizada de \isa{F{\isacharhash}Fs} es la conjunción de \isa{F} con la 
  conjunción generalizada de \isa{Fs}.
    \end{itemize}
  \end{definicion}

  \begin{definicion}
  La disyunción generalizada de una lista de fórmulas se define 
  recursivamente como:
    \begin{itemize}
      \item La disyunción generalizada de la lista vacía es \isa{{\isasymbottom}}.
      \item Sea \isa{F} una fórmula y \isa{Fs} una lista de fórmulas. Entonces,
  la disyunción generalizada de \isa{F{\isacharhash}Fs} es la disyunción de \isa{F} con la 
  disyunción generalizada de \isa{Fs}.
    \end{itemize}
  \end{definicion}

  Notemos que al referirnos simplemente a disyunción o conjunción en las
  siguientes definiciones nos referiremos a la de dos elementos.

  Su formalización en Isabelle es la siguiente:%
\end{isamarkuptext}\isamarkuptrue%
\isacommand{primrec}\isamarkupfalse%
\ BigAnd\ {\isacharcolon}{\isacharcolon}\ {\isachardoublequoteopen}{\isacharprime}a\ formula\ list\ {\isasymRightarrow}\ {\isacharprime}a\ formula{\isachardoublequoteclose}\ {\isacharparenleft}{\isachardoublequoteopen}\isactrlbold {\isasymAnd}{\isacharunderscore}{\isachardoublequoteclose}{\isacharparenright}\ \isakeyword{where}\isanewline
\ \ {\isachardoublequoteopen}\isactrlbold {\isasymAnd}{\isacharbrackleft}{\isacharbrackright}\ {\isacharequal}\ {\isacharparenleft}\isactrlbold {\isasymnot}{\isasymbottom}{\isacharparenright}{\isachardoublequoteclose}\ \isanewline
{\isacharbar}\ {\isachardoublequoteopen}\isactrlbold {\isasymAnd}{\isacharparenleft}F{\isacharhash}Fs{\isacharparenright}\ {\isacharequal}\ F\ \isactrlbold {\isasymand}\ \isactrlbold {\isasymAnd}Fs{\isachardoublequoteclose}\isanewline
\isanewline
\isacommand{primrec}\isamarkupfalse%
\ BigOr\ {\isacharcolon}{\isacharcolon}\ {\isachardoublequoteopen}{\isacharprime}a\ formula\ list\ {\isasymRightarrow}\ {\isacharprime}a\ formula{\isachardoublequoteclose}\ {\isacharparenleft}{\isachardoublequoteopen}\isactrlbold {\isasymOr}{\isacharunderscore}{\isachardoublequoteclose}{\isacharparenright}\ \isakeyword{where}\isanewline
\ \ {\isachardoublequoteopen}\isactrlbold {\isasymOr}{\isacharbrackleft}{\isacharbrackright}\ {\isacharequal}\ {\isasymbottom}{\isachardoublequoteclose}\ \isanewline
{\isacharbar}\ {\isachardoublequoteopen}\isactrlbold {\isasymOr}{\isacharparenleft}F{\isacharhash}Fs{\isacharparenright}\ {\isacharequal}\ F\ \isactrlbold {\isasymor}\ \isactrlbold {\isasymOr}Fs{\isachardoublequoteclose}%
\begin{isamarkuptext}%
Ambas nuevas conectivas se definen con el tipo funciones 
  primitivas recursivas. Estas se basan en los dos casos descritos
  anteriormente según la definición recursiva de listas que se genera en
  Isabelle: la lista vacía representada como \isa{{\isacharbrackleft}{\isacharbrackright}} y la lista
  construida añadiendo una fórmula a una lista de fórmulas. 
  Además, se observa en cada definición el nuevo símbolo de 
  notación que aparece entre paréntesis.

  Por otro lado, como es habitual, de acuerdo a la definición recursiva
  de listas, Isabelle genera automáticamente un esquema inductivo que 
  emplearemos más adelante.

  Vamos a mostrar una propiedad sobre la conjunción plural.

  \begin{lema}
  El conjunto de átomos de la conjunción generalizada de una lista 
  de fórmulas es la unión de los conjuntos de átomos de cada fórmula de 
  la lista.
  \end{lema}

  \begin{demostracion}
  La prueba se hace por inducción sobre listas, en particular,
  listas de fórmulas. Para ello, demostremos el resultado en los casos
  siguientes. 

  En primer lugar lo probaremos para la lista vacía de fórmulas. Es
  claro por definición que la conjunción generalizada de la lista vacía
  es \isa{{\isasymnot}\ {\isasymbottom}}. De este modo, su conjunto de átomos coincide con los de 
  \isa{{\isasymbottom}}, luego es el vacío. Por tanto, queda demostrado el resultado, 
  pues el vacío es igual a la unión del conjunto de átomos de cada 
  elemento de la lista vacía de fórmulas. 

  Supongamos ahora una lista de fórmulas \isa{Fs} verificando el enunciado.
  Sea la fórmula \isa{F}, vamos a probar que \isa{F{\isacharhash}Fs} cumple la propiedad.
  Por definición de la nueva conectiva, el conjunto de átomos de 
  la conjunción generalizada de \isa{F{\isacharhash}Fs} es igual al conjunto de átomos de 
  la conjunción de \isa{F} con la conjunción generalizada de \isa{Fs}. De este
  modo, por propiedades del conjunto de átomos de la conjunción, tenemos
  que dicho conjunto es la unión del conjunto de átomos de \isa{F} y el
  conjunto de átomos de la conjunción generalizada de \isa{Fs}. Aplicando 
  ahora la hipótesis de inducción sobre \isa{Fs}, tenemos que lo anterior es 
  igual a la unión del conjunto de átomos de \isa{F} con la unión
 (generalizada) de los conjuntos de átomos de cada fórmula de \isa{Fs}. 
  Luego, por propiedades de la unión, es equivalente a la unión de los 
  conjuntos de átomos de cada elemento de \isa{F{\isacharhash}Fs} como queríamos 
  demostrar.
  \end{demostracion}

  En Isabelle se formaliza como sigue.%
\end{isamarkuptext}\isamarkuptrue%
\isacommand{lemma}\isamarkupfalse%
\ atoms{\isacharunderscore}BigAnd{\isacharcolon}\ \isanewline
\ \ {\isachardoublequoteopen}atoms\ {\isacharparenleft}\isactrlbold {\isasymAnd}Fs{\isacharparenright}\ {\isacharequal}\ {\isasymUnion}{\isacharparenleft}atoms\ {\isacharbackquote}\ set\ Fs{\isacharparenright}{\isachardoublequoteclose}\isanewline
%
\isadelimproof
\ \ %
\endisadelimproof
%
\isatagproof
\isacommand{oops}\isamarkupfalse%
%
\endisatagproof
{\isafoldproof}%
%
\isadelimproof
%
\endisadelimproof
%
\begin{isamarkuptext}%
Observemos el lado izquierdo de la igualdad. \isa{Fs} es una 
  lista de fórmulas, luego la conjunción generalizada de dicha lista se 
  trata de una fórmula. Al aplicarle \isa{atoms} a dicha fórmula, obtenemos 
  finalmente el conjunto de sus átomos. Por otro lado, en el lado 
  derecho de la igualdad tenemos el conjunto \isa{set\ Fs} cuyos elementos 
  son las fórmulas de la lista \isa{Fs}. De este modo, al aplicar \isa{atoms\ {\isacharbackquote}} 
  a dicho conjunto obtenemos la imagen por \isa{atoms} de cada uno de sus
  elementos, es decir, un conjunto cuyos elementos son los 
  conjuntos de átomos de cada fórmula de \isa{Fs}. Por último, mediante la 
  unión se obtiene el conjunto de los átomos de cada fórmula de la 
  lista inicial.

  Veamos ahora la demostración detallada. Esta seguirá el esquema de 
  inducción sobre listas. Previamente vamos a probar cada caso por
  separado.%
\end{isamarkuptext}\isamarkuptrue%
\isacommand{lemma}\isamarkupfalse%
\ atoms{\isacharunderscore}BigAnd{\isacharunderscore}nil{\isacharcolon}\ \isanewline
\ \ {\isachardoublequoteopen}atoms\ {\isacharparenleft}\isactrlbold {\isasymAnd}{\isacharbrackleft}{\isacharbrackright}{\isacharparenright}\ {\isacharequal}\ {\isasymUnion}\ {\isacharparenleft}atoms\ {\isacharbackquote}\ set\ Nil{\isacharparenright}{\isachardoublequoteclose}\isanewline
%
\isadelimproof
%
\endisadelimproof
%
\isatagproof
\isacommand{proof}\isamarkupfalse%
\ {\isacharminus}\isanewline
\ \ \isacommand{have}\isamarkupfalse%
\ {\isachardoublequoteopen}atoms\ {\isacharparenleft}\isactrlbold {\isasymAnd}{\isacharbrackleft}{\isacharbrackright}{\isacharparenright}\ {\isacharequal}\ atoms\ {\isacharparenleft}\isactrlbold {\isasymnot}\ {\isasymbottom}{\isacharparenright}{\isachardoublequoteclose}\ \isanewline
\ \ \ \ \isacommand{by}\isamarkupfalse%
\ {\isacharparenleft}simp\ only{\isacharcolon}\ BigAnd{\isachardot}simps{\isacharparenleft}{\isadigit{1}}{\isacharparenright}{\isacharparenright}\isanewline
\ \ \isacommand{also}\isamarkupfalse%
\ \isacommand{have}\isamarkupfalse%
\ {\isachardoublequoteopen}{\isasymdots}\ {\isacharequal}\ atoms\ {\isasymbottom}{\isachardoublequoteclose}\ \isanewline
\ \ \ \ \isacommand{by}\isamarkupfalse%
\ {\isacharparenleft}simp\ only{\isacharcolon}\ formula{\isachardot}set{\isacharparenleft}{\isadigit{3}}{\isacharparenright}{\isacharparenright}\isanewline
\ \ \isacommand{also}\isamarkupfalse%
\ \isacommand{have}\isamarkupfalse%
\ {\isachardoublequoteopen}{\isasymdots}\ {\isacharequal}\ {\isasymemptyset}{\isachardoublequoteclose}\ \isanewline
\ \ \ \ \isacommand{by}\isamarkupfalse%
\ {\isacharparenleft}simp\ only{\isacharcolon}\ formula{\isachardot}set{\isacharparenleft}{\isadigit{2}}{\isacharparenright}{\isacharparenright}\isanewline
\ \ \isacommand{also}\isamarkupfalse%
\ \isacommand{have}\isamarkupfalse%
\ {\isachardoublequoteopen}{\isasymdots}\ {\isacharequal}\ {\isasymUnion}\ {\isasymemptyset}{\isachardoublequoteclose}\isanewline
\ \ \ \ \isacommand{by}\isamarkupfalse%
\ {\isacharparenleft}simp\ only{\isacharcolon}\ Union{\isacharunderscore}empty{\isacharparenright}\isanewline
\ \ \isacommand{also}\isamarkupfalse%
\ \isacommand{have}\isamarkupfalse%
\ {\isachardoublequoteopen}{\isasymdots}\ {\isacharequal}\ \ {\isasymUnion}\ {\isacharparenleft}atoms\ {\isacharbackquote}\ {\isasymemptyset}{\isacharparenright}{\isachardoublequoteclose}\isanewline
\ \ \ \ \isacommand{by}\isamarkupfalse%
\ {\isacharparenleft}simp\ only{\isacharcolon}\ image{\isacharunderscore}empty{\isacharparenright}\isanewline
\ \ \isacommand{also}\isamarkupfalse%
\ \isacommand{have}\isamarkupfalse%
\ {\isachardoublequoteopen}{\isasymdots}\ {\isacharequal}\ {\isasymUnion}\ {\isacharparenleft}atoms\ {\isacharbackquote}\ set\ {\isacharbrackleft}{\isacharbrackright}{\isacharparenright}{\isachardoublequoteclose}\isanewline
\ \ \ \ \isacommand{by}\isamarkupfalse%
\ {\isacharparenleft}simp\ only{\isacharcolon}\ list{\isachardot}set{\isacharparenright}\isanewline
\ \ \isacommand{finally}\isamarkupfalse%
\ \isacommand{show}\isamarkupfalse%
\ {\isacharquery}thesis\isanewline
\ \ \ \ \isacommand{by}\isamarkupfalse%
\ this\isanewline
\isacommand{qed}\isamarkupfalse%
%
\endisatagproof
{\isafoldproof}%
%
\isadelimproof
%
\endisadelimproof
%
\begin{isamarkuptext}%
Mostramos el siguiente lema auxiliar que utilizaremos en la
  demostración del último caso de inducción.%
\end{isamarkuptext}\isamarkuptrue%
\isacommand{lemma}\isamarkupfalse%
\ union{\isacharunderscore}imagen{\isacharcolon}\ {\isachardoublequoteopen}f\ a\ {\isasymunion}\ {\isasymUnion}\ {\isacharparenleft}f\ {\isacharbackquote}\ B{\isacharparenright}\ {\isacharequal}\ {\isasymUnion}\ {\isacharparenleft}f\ {\isacharbackquote}\ {\isacharparenleft}{\isacharbraceleft}a{\isacharbraceright}\ {\isasymunion}\ B{\isacharparenright}{\isacharparenright}{\isachardoublequoteclose}\isanewline
%
\isadelimproof
\ \ %
\endisadelimproof
%
\isatagproof
\isacommand{by}\isamarkupfalse%
\ {\isacharparenleft}simp\ only{\isacharcolon}\ Union{\isacharunderscore}image{\isacharunderscore}insert\isanewline
\ \ \ \ \ \ \ \ \ \ \ \ \ \ \ \ \ insert{\isacharunderscore}is{\isacharunderscore}Un{\isacharbrackleft}THEN\ sym{\isacharbrackright}{\isacharparenright}%
\endisatagproof
{\isafoldproof}%
%
\isadelimproof
%
\endisadelimproof
%
\begin{isamarkuptext}%
Se trata de una modificación del lema \isa{Union{\isacharunderscore}image{\isacharunderscore}insert} en 
  Isabelle para adaptarlo al caso particular. 

  \begin{itemize}
    \item[] \isa{{\isasymUnion}\ {\isacharparenleft}f\ {\isacharbackquote}\ {\isacharparenleft}{\isacharbraceleft}a{\isacharbraceright}\ {\isasymunion}\ B{\isacharparenright}{\isacharparenright}\ {\isacharequal}\ f\ a\ {\isasymunion}\ {\isasymUnion}\ {\isacharparenleft}f\ {\isacharbackquote}\ B{\isacharparenright}} 
      \hfill (\isa{Union{\isacharunderscore}image{\isacharunderscore}insert})
  \end{itemize}

  Para ello empleamos el lema \isa{insert{\isacharunderscore}is{\isacharunderscore}Un}.

  \begin{itemize}
    \item[] \isa{insert\ a\ A\ {\isacharequal}\ {\isacharbraceleft}a{\isacharbraceright}\ {\isasymunion}\ A} \hspace{5cm} \isa{{\isacharparenleft}insert{\isacharunderscore}is{\isacharunderscore}Un{\isacharparenright}}
  \end{itemize}

  De esta manera, la unión de un conjunto de un solo elemento y otro 
  conjunto cualquiera es equivalente a insertar dicho elemento en el 
  conjunto. Además, aplicamos el lema seguido de \isa{{\isacharbrackleft}THEN\ sym{\isacharbrackright}} para 
  mostrar la equivalencia en el sentido en el que acaba de ser
  enunciada por simetría, pues en Isabelle aparece en sentido opuesto.
  Por tanto, el lema auxiliar \isa{union{\isacharunderscore}imagen} es fundamentalmente el
  lema de Isabelle\\ \isa{Union{\isacharunderscore}image{\isacharunderscore}insert} teniendo en cuenta las
  equivalencias anteriores.

  Procedamos a la demostración del último caso de inducción.%
\end{isamarkuptext}\isamarkuptrue%
\isacommand{lemma}\isamarkupfalse%
\ atoms{\isacharunderscore}BigAnd{\isacharunderscore}cons{\isacharcolon}\isanewline
\ \ \isakeyword{assumes}\ {\isachardoublequoteopen}atoms\ {\isacharparenleft}\isactrlbold {\isasymAnd}Fs{\isacharparenright}\ {\isacharequal}\ {\isasymUnion}\ {\isacharparenleft}atoms\ {\isacharbackquote}\ set\ Fs{\isacharparenright}{\isachardoublequoteclose}\isanewline
\ \ \isakeyword{shows}\ {\isachardoublequoteopen}atoms\ {\isacharparenleft}\isactrlbold {\isasymAnd}{\isacharparenleft}F{\isacharhash}Fs{\isacharparenright}{\isacharparenright}\ {\isacharequal}\ {\isasymUnion}\ {\isacharparenleft}atoms\ {\isacharbackquote}\ set\ {\isacharparenleft}F{\isacharhash}Fs{\isacharparenright}{\isacharparenright}{\isachardoublequoteclose}\isanewline
%
\isadelimproof
%
\endisadelimproof
%
\isatagproof
\isacommand{proof}\isamarkupfalse%
\ {\isacharminus}\isanewline
\ \ \isacommand{have}\isamarkupfalse%
\ {\isachardoublequoteopen}atoms\ {\isacharparenleft}\isactrlbold {\isasymAnd}{\isacharparenleft}F{\isacharhash}Fs{\isacharparenright}{\isacharparenright}\ {\isacharequal}\ atoms\ {\isacharparenleft}F\ \isactrlbold {\isasymand}\ \isactrlbold {\isasymAnd}Fs{\isacharparenright}{\isachardoublequoteclose}\isanewline
\ \ \ \ \isacommand{by}\isamarkupfalse%
\ {\isacharparenleft}simp\ only{\isacharcolon}\ BigAnd{\isachardot}simps{\isacharparenleft}{\isadigit{2}}{\isacharparenright}{\isacharparenright}\isanewline
\ \ \isacommand{also}\isamarkupfalse%
\ \isacommand{have}\isamarkupfalse%
\ {\isachardoublequoteopen}{\isasymdots}\ {\isacharequal}\ atoms\ F\ {\isasymunion}\ atoms\ {\isacharparenleft}\isactrlbold {\isasymAnd}Fs{\isacharparenright}{\isachardoublequoteclose}\isanewline
\ \ \ \ \isacommand{by}\isamarkupfalse%
\ {\isacharparenleft}simp\ only{\isacharcolon}\ formula{\isachardot}set{\isacharparenleft}{\isadigit{4}}{\isacharparenright}{\isacharparenright}\isanewline
\ \ \isacommand{also}\isamarkupfalse%
\ \isacommand{have}\isamarkupfalse%
\ {\isachardoublequoteopen}{\isasymdots}\ {\isacharequal}\ atoms\ F\ {\isasymunion}\ {\isasymUnion}{\isacharparenleft}atoms\ {\isacharbackquote}\ set\ Fs{\isacharparenright}{\isachardoublequoteclose}\isanewline
\ \ \ \ \isacommand{by}\isamarkupfalse%
\ {\isacharparenleft}simp\ only{\isacharcolon}\ assms{\isacharparenright}\isanewline
\ \ \isacommand{also}\isamarkupfalse%
\ \isacommand{have}\isamarkupfalse%
\ {\isachardoublequoteopen}{\isasymdots}\ {\isacharequal}\ {\isasymUnion}{\isacharparenleft}atoms\ {\isacharbackquote}\ {\isacharparenleft}{\isacharbraceleft}F{\isacharbraceright}\ {\isasymunion}\ set\ Fs{\isacharparenright}{\isacharparenright}{\isachardoublequoteclose}\ \isanewline
\ \ \ \ \isacommand{by}\isamarkupfalse%
\ {\isacharparenleft}simp\ only{\isacharcolon}\ union{\isacharunderscore}imagen{\isacharparenright}\isanewline
\ \ \isacommand{also}\isamarkupfalse%
\ \isacommand{have}\isamarkupfalse%
\ {\isachardoublequoteopen}{\isasymdots}\ {\isacharequal}\ \ {\isasymUnion}{\isacharparenleft}atoms\ {\isacharbackquote}\ set\ {\isacharparenleft}F{\isacharhash}Fs{\isacharparenright}{\isacharparenright}{\isachardoublequoteclose}\isanewline
\ \ \ \ \isacommand{by}\isamarkupfalse%
\ {\isacharparenleft}simp\ only{\isacharcolon}\ set{\isacharunderscore}insert{\isacharparenright}\isanewline
\ \ \isacommand{finally}\isamarkupfalse%
\ \isacommand{show}\isamarkupfalse%
\ \ {\isachardoublequoteopen}atoms\ {\isacharparenleft}\isactrlbold {\isasymAnd}{\isacharparenleft}F{\isacharhash}Fs{\isacharparenright}{\isacharparenright}\ {\isacharequal}\ {\isasymUnion}{\isacharparenleft}atoms\ {\isacharbackquote}\ set\ {\isacharparenleft}F{\isacharhash}Fs{\isacharparenright}{\isacharparenright}{\isachardoublequoteclose}\ \isanewline
\ \ \ \ \isacommand{by}\isamarkupfalse%
\ this\isanewline
\isacommand{qed}\isamarkupfalse%
%
\endisatagproof
{\isafoldproof}%
%
\isadelimproof
%
\endisadelimproof
%
\begin{isamarkuptext}%
Por tanto, la demostración detallada completa es la siguiente.%
\end{isamarkuptext}\isamarkuptrue%
\isacommand{lemma}\isamarkupfalse%
\ {\isachardoublequoteopen}atoms\ {\isacharparenleft}\isactrlbold {\isasymAnd}Fs{\isacharparenright}\ {\isacharequal}\ {\isasymUnion}\ {\isacharparenleft}atoms\ {\isacharbackquote}\ set\ Fs{\isacharparenright}{\isachardoublequoteclose}\isanewline
%
\isadelimproof
%
\endisadelimproof
%
\isatagproof
\isacommand{proof}\isamarkupfalse%
\ {\isacharparenleft}induction\ Fs{\isacharparenright}\isanewline
\ \ \isacommand{case}\isamarkupfalse%
\ Nil\isanewline
\ \ \isacommand{then}\isamarkupfalse%
\ \isacommand{show}\isamarkupfalse%
\ {\isacharquery}case\ \isacommand{by}\isamarkupfalse%
\ {\isacharparenleft}rule\ atoms{\isacharunderscore}BigAnd{\isacharunderscore}nil{\isacharparenright}\isanewline
\isacommand{next}\isamarkupfalse%
\isanewline
\ \ \isacommand{case}\isamarkupfalse%
\ {\isacharparenleft}Cons\ a\ Fs{\isacharparenright}\isanewline
\ \ \isacommand{assume}\isamarkupfalse%
\ {\isachardoublequoteopen}atoms\ {\isacharparenleft}\isactrlbold {\isasymAnd}Fs{\isacharparenright}\ {\isacharequal}\ {\isasymUnion}{\isacharparenleft}atoms\ {\isacharbackquote}\ set\ Fs{\isacharparenright}{\isachardoublequoteclose}\ \isanewline
\ \ \isacommand{then}\isamarkupfalse%
\ \isacommand{show}\isamarkupfalse%
\ {\isacharquery}case\ \isanewline
\ \ \ \ \isacommand{by}\isamarkupfalse%
\ {\isacharparenleft}rule\ atoms{\isacharunderscore}BigAnd{\isacharunderscore}cons{\isacharparenright}\isanewline
\isacommand{qed}\isamarkupfalse%
%
\endisatagproof
{\isafoldproof}%
%
\isadelimproof
%
\endisadelimproof
%
\begin{isamarkuptext}%
Por último, su demostración automática.%
\end{isamarkuptext}\isamarkuptrue%
\isacommand{lemma}\isamarkupfalse%
\ atoms{\isacharunderscore}BigAnd{\isacharcolon}\ \isanewline
\ \ {\isachardoublequoteopen}atoms\ {\isacharparenleft}\isactrlbold {\isasymAnd}Fs{\isacharparenright}\ {\isacharequal}\ {\isasymUnion}{\isacharparenleft}atoms\ {\isacharbackquote}\ set\ Fs{\isacharparenright}{\isachardoublequoteclose}\isanewline
%
\isadelimproof
\ \ %
\endisadelimproof
%
\isatagproof
\isacommand{by}\isamarkupfalse%
\ {\isacharparenleft}induction\ Fs{\isacharparenright}\ simp{\isacharunderscore}all%
\endisatagproof
{\isafoldproof}%
%
\isadelimproof
%
\endisadelimproof
%
\isadelimtheory
%
\endisadelimtheory
%
\isatagtheory
%
\endisatagtheory
{\isafoldtheory}%
%
\isadelimtheory
%
\endisadelimtheory
%
\end{isabellebody}%
\endinput
%:%file=~/TFM/TFM/Sintaxis.thy%:%
%:%24=13%:%
%:%36=15%:%
%:%37=16%:%
%:%38=17%:%
%:%39=18%:%
%:%40=19%:%
%:%41=20%:%
%:%42=21%:%
%:%43=22%:%
%:%44=23%:%
%:%45=24%:%
%:%46=25%:%
%:%47=26%:%
%:%48=27%:%
%:%49=28%:%
%:%50=29%:%
%:%51=30%:%
%:%52=31%:%
%:%53=32%:%
%:%54=33%:%
%:%55=34%:%
%:%56=35%:%
%:%57=36%:%
%:%58=37%:%
%:%59=38%:%
%:%60=39%:%
%:%61=40%:%
%:%62=41%:%
%:%63=42%:%
%:%64=43%:%
%:%65=44%:%
%:%66=45%:%
%:%67=46%:%
%:%68=47%:%
%:%69=48%:%
%:%70=49%:%
%:%71=50%:%
%:%72=51%:%
%:%73=52%:%
%:%74=53%:%
%:%75=54%:%
%:%76=55%:%
%:%77=56%:%
%:%78=57%:%
%:%79=58%:%
%:%80=59%:%
%:%81=60%:%
%:%82=61%:%
%:%83=62%:%
%:%84=63%:%
%:%85=64%:%
%:%86=65%:%
%:%87=66%:%
%:%88=67%:%
%:%89=68%:%
%:%90=69%:%
%:%91=70%:%
%:%92=71%:%
%:%93=72%:%
%:%94=73%:%
%:%95=74%:%
%:%97=76%:%
%:%98=76%:%
%:%99=77%:%
%:%100=78%:%
%:%101=79%:%
%:%102=80%:%
%:%103=81%:%
%:%104=82%:%
%:%106=84%:%
%:%108=85%:%
%:%109=85%:%
%:%111=85%:%
%:%115=85%:%
%:%116=85%:%
%:%125=87%:%
%:%126=88%:%
%:%127=89%:%
%:%128=90%:%
%:%129=91%:%
%:%130=92%:%
%:%131=93%:%
%:%132=94%:%
%:%133=95%:%
%:%134=96%:%
%:%135=97%:%
%:%136=98%:%
%:%137=99%:%
%:%138=100%:%
%:%139=101%:%
%:%140=102%:%
%:%141=103%:%
%:%142=104%:%
%:%143=105%:%
%:%144=106%:%
%:%145=107%:%
%:%146=108%:%
%:%147=109%:%
%:%148=110%:%
%:%149=111%:%
%:%150=112%:%
%:%151=113%:%
%:%152=114%:%
%:%153=115%:%
%:%154=116%:%
%:%159=116%:%
%:%160=117%:%
%:%161=118%:%
%:%162=119%:%
%:%163=120%:%
%:%164=121%:%
%:%166=123%:%
%:%167=123%:%
%:%168=124%:%
%:%171=125%:%
%:%175=125%:%
%:%176=125%:%
%:%177=126%:%
%:%178=127%:%
%:%179=127%:%
%:%180=128%:%
%:%181=128%:%
%:%182=129%:%
%:%183=130%:%
%:%184=130%:%
%:%185=131%:%
%:%186=131%:%
%:%187=132%:%
%:%188=133%:%
%:%189=133%:%
%:%190=134%:%
%:%191=134%:%
%:%192=135%:%
%:%193=136%:%
%:%194=136%:%
%:%195=137%:%
%:%196=137%:%
%:%201=137%:%
%:%204=138%:%
%:%207=140%:%
%:%208=141%:%
%:%209=142%:%
%:%211=144%:%
%:%212=144%:%
%:%213=145%:%
%:%216=146%:%
%:%220=146%:%
%:%221=146%:%
%:%222=147%:%
%:%223=148%:%
%:%224=148%:%
%:%225=149%:%
%:%226=149%:%
%:%227=150%:%
%:%228=151%:%
%:%229=151%:%
%:%230=152%:%
%:%231=152%:%
%:%232=153%:%
%:%233=153%:%
%:%234=154%:%
%:%235=154%:%
%:%236=155%:%
%:%237=155%:%
%:%238=155%:%
%:%239=156%:%
%:%240=156%:%
%:%241=157%:%
%:%242=157%:%
%:%243=157%:%
%:%244=158%:%
%:%245=158%:%
%:%246=159%:%
%:%247=159%:%
%:%248=159%:%
%:%249=160%:%
%:%250=160%:%
%:%251=161%:%
%:%252=161%:%
%:%253=161%:%
%:%254=162%:%
%:%255=162%:%
%:%256=163%:%
%:%257=163%:%
%:%258=164%:%
%:%259=165%:%
%:%260=165%:%
%:%261=166%:%
%:%262=166%:%
%:%267=166%:%
%:%270=167%:%
%:%271=167%:%
%:%272=168%:%
%:%273=169%:%
%:%274=169%:%
%:%276=171%:%
%:%277=172%:%
%:%278=173%:%
%:%279=174%:%
%:%280=175%:%
%:%281=176%:%
%:%282=177%:%
%:%283=178%:%
%:%284=179%:%
%:%285=180%:%
%:%286=181%:%
%:%287=182%:%
%:%288=183%:%
%:%289=184%:%
%:%290=185%:%
%:%291=186%:%
%:%292=187%:%
%:%293=188%:%
%:%294=189%:%
%:%295=190%:%
%:%296=191%:%
%:%297=192%:%
%:%298=193%:%
%:%299=194%:%
%:%300=195%:%
%:%301=196%:%
%:%302=197%:%
%:%303=198%:%
%:%304=199%:%
%:%305=200%:%
%:%306=201%:%
%:%307=202%:%
%:%308=203%:%
%:%309=204%:%
%:%310=205%:%
%:%311=206%:%
%:%312=207%:%
%:%313=208%:%
%:%314=209%:%
%:%315=210%:%
%:%316=211%:%
%:%317=212%:%
%:%318=213%:%
%:%319=214%:%
%:%320=215%:%
%:%321=216%:%
%:%322=217%:%
%:%323=218%:%
%:%324=219%:%
%:%325=220%:%
%:%326=221%:%
%:%327=222%:%
%:%328=223%:%
%:%329=224%:%
%:%330=225%:%
%:%331=226%:%
%:%332=227%:%
%:%333=228%:%
%:%334=229%:%
%:%335=230%:%
%:%336=231%:%
%:%337=232%:%
%:%338=233%:%
%:%339=234%:%
%:%340=235%:%
%:%341=236%:%
%:%342=237%:%
%:%343=238%:%
%:%344=239%:%
%:%345=240%:%
%:%346=241%:%
%:%347=242%:%
%:%348=243%:%
%:%349=244%:%
%:%350=245%:%
%:%351=246%:%
%:%352=247%:%
%:%353=248%:%
%:%354=249%:%
%:%355=250%:%
%:%356=251%:%
%:%357=252%:%
%:%358=253%:%
%:%359=254%:%
%:%360=255%:%
%:%361=256%:%
%:%362=257%:%
%:%363=258%:%
%:%365=260%:%
%:%366=260%:%
%:%369=261%:%
%:%373=261%:%
%:%383=263%:%
%:%384=264%:%
%:%385=265%:%
%:%386=266%:%
%:%387=267%:%
%:%388=268%:%
%:%389=269%:%
%:%390=270%:%
%:%391=271%:%
%:%392=272%:%
%:%393=273%:%
%:%394=274%:%
%:%395=275%:%
%:%396=276%:%
%:%397=277%:%
%:%398=278%:%
%:%399=279%:%
%:%400=280%:%
%:%401=281%:%
%:%402=282%:%
%:%403=283%:%
%:%404=284%:%
%:%405=285%:%
%:%406=286%:%
%:%407=287%:%
%:%408=288%:%
%:%409=289%:%
%:%410=290%:%
%:%411=291%:%
%:%413=293%:%
%:%414=293%:%
%:%415=294%:%
%:%422=295%:%
%:%423=295%:%
%:%424=296%:%
%:%425=296%:%
%:%426=297%:%
%:%427=297%:%
%:%428=298%:%
%:%429=298%:%
%:%430=298%:%
%:%431=299%:%
%:%432=299%:%
%:%433=300%:%
%:%434=300%:%
%:%435=300%:%
%:%436=301%:%
%:%437=301%:%
%:%438=302%:%
%:%444=302%:%
%:%447=303%:%
%:%448=304%:%
%:%449=304%:%
%:%450=305%:%
%:%457=306%:%
%:%458=306%:%
%:%459=307%:%
%:%460=307%:%
%:%461=308%:%
%:%462=308%:%
%:%463=309%:%
%:%464=309%:%
%:%465=309%:%
%:%466=310%:%
%:%467=310%:%
%:%468=311%:%
%:%474=311%:%
%:%477=312%:%
%:%478=313%:%
%:%479=313%:%
%:%480=314%:%
%:%481=315%:%
%:%484=316%:%
%:%488=316%:%
%:%489=316%:%
%:%490=317%:%
%:%491=317%:%
%:%496=317%:%
%:%499=318%:%
%:%500=319%:%
%:%501=319%:%
%:%502=320%:%
%:%503=321%:%
%:%504=322%:%
%:%511=323%:%
%:%512=323%:%
%:%513=324%:%
%:%514=324%:%
%:%515=325%:%
%:%516=325%:%
%:%517=326%:%
%:%518=326%:%
%:%519=327%:%
%:%520=327%:%
%:%521=327%:%
%:%522=328%:%
%:%523=328%:%
%:%524=329%:%
%:%530=329%:%
%:%533=330%:%
%:%534=331%:%
%:%535=331%:%
%:%536=332%:%
%:%537=333%:%
%:%538=334%:%
%:%545=335%:%
%:%546=335%:%
%:%547=336%:%
%:%548=336%:%
%:%549=337%:%
%:%550=337%:%
%:%551=338%:%
%:%552=338%:%
%:%553=339%:%
%:%554=339%:%
%:%555=339%:%
%:%556=340%:%
%:%557=340%:%
%:%558=341%:%
%:%564=341%:%
%:%567=342%:%
%:%568=343%:%
%:%569=343%:%
%:%570=344%:%
%:%571=345%:%
%:%572=346%:%
%:%579=347%:%
%:%580=347%:%
%:%581=348%:%
%:%582=348%:%
%:%583=349%:%
%:%584=349%:%
%:%585=350%:%
%:%586=350%:%
%:%587=351%:%
%:%588=351%:%
%:%589=351%:%
%:%590=352%:%
%:%591=352%:%
%:%592=353%:%
%:%598=353%:%
%:%601=354%:%
%:%602=355%:%
%:%603=355%:%
%:%610=356%:%
%:%611=356%:%
%:%612=357%:%
%:%613=357%:%
%:%614=358%:%
%:%615=358%:%
%:%616=358%:%
%:%617=358%:%
%:%618=359%:%
%:%619=359%:%
%:%620=360%:%
%:%621=360%:%
%:%622=361%:%
%:%623=361%:%
%:%624=361%:%
%:%625=361%:%
%:%626=362%:%
%:%627=362%:%
%:%628=363%:%
%:%629=363%:%
%:%630=364%:%
%:%631=364%:%
%:%632=364%:%
%:%633=364%:%
%:%634=365%:%
%:%635=365%:%
%:%636=366%:%
%:%637=366%:%
%:%638=367%:%
%:%639=367%:%
%:%640=367%:%
%:%641=367%:%
%:%642=368%:%
%:%643=368%:%
%:%644=369%:%
%:%645=369%:%
%:%646=370%:%
%:%647=370%:%
%:%648=370%:%
%:%649=370%:%
%:%650=371%:%
%:%651=371%:%
%:%652=372%:%
%:%653=372%:%
%:%654=373%:%
%:%655=373%:%
%:%656=373%:%
%:%657=373%:%
%:%658=374%:%
%:%668=376%:%
%:%670=378%:%
%:%671=378%:%
%:%674=379%:%
%:%678=379%:%
%:%679=379%:%
%:%693=381%:%
%:%705=383%:%
%:%706=384%:%
%:%707=385%:%
%:%708=386%:%
%:%709=387%:%
%:%710=388%:%
%:%711=389%:%
%:%712=390%:%
%:%713=391%:%
%:%714=392%:%
%:%715=393%:%
%:%716=394%:%
%:%717=395%:%
%:%718=396%:%
%:%719=397%:%
%:%720=398%:%
%:%721=399%:%
%:%722=400%:%
%:%724=402%:%
%:%725=402%:%
%:%726=403%:%
%:%727=404%:%
%:%728=405%:%
%:%729=406%:%
%:%730=407%:%
%:%731=408%:%
%:%733=410%:%
%:%734=411%:%
%:%735=412%:%
%:%736=413%:%
%:%737=414%:%
%:%738=415%:%
%:%739=416%:%
%:%741=419%:%
%:%742=419%:%
%:%743=420%:%
%:%746=421%:%
%:%750=421%:%
%:%751=421%:%
%:%752=422%:%
%:%753=423%:%
%:%754=423%:%
%:%755=424%:%
%:%756=424%:%
%:%757=425%:%
%:%758=426%:%
%:%759=426%:%
%:%760=427%:%
%:%761=427%:%
%:%762=428%:%
%:%763=429%:%
%:%764=429%:%
%:%766=431%:%
%:%767=432%:%
%:%768=432%:%
%:%769=433%:%
%:%770=434%:%
%:%771=434%:%
%:%772=435%:%
%:%773=435%:%
%:%774=436%:%
%:%775=437%:%
%:%776=437%:%
%:%777=438%:%
%:%778=439%:%
%:%779=439%:%
%:%784=439%:%
%:%787=440%:%
%:%790=442%:%
%:%791=443%:%
%:%792=444%:%
%:%794=446%:%
%:%795=446%:%
%:%796=447%:%
%:%798=449%:%
%:%799=450%:%
%:%800=451%:%
%:%801=452%:%
%:%802=453%:%
%:%803=454%:%
%:%804=455%:%
%:%806=458%:%
%:%807=458%:%
%:%808=459%:%
%:%811=460%:%
%:%815=460%:%
%:%816=460%:%
%:%817=461%:%
%:%818=462%:%
%:%819=462%:%
%:%820=463%:%
%:%821=463%:%
%:%822=464%:%
%:%823=465%:%
%:%824=465%:%
%:%826=467%:%
%:%827=468%:%
%:%828=468%:%
%:%833=468%:%
%:%836=469%:%
%:%839=471%:%
%:%840=472%:%
%:%841=473%:%
%:%842=474%:%
%:%843=475%:%
%:%844=476%:%
%:%845=477%:%
%:%846=478%:%
%:%847=479%:%
%:%848=480%:%
%:%850=482%:%
%:%851=482%:%
%:%854=483%:%
%:%858=483%:%
%:%859=483%:%
%:%868=485%:%
%:%869=486%:%
%:%871=488%:%
%:%872=488%:%
%:%873=489%:%
%:%876=490%:%
%:%880=490%:%
%:%881=490%:%
%:%886=490%:%
%:%889=491%:%
%:%890=492%:%
%:%891=492%:%
%:%892=493%:%
%:%895=494%:%
%:%899=494%:%
%:%900=494%:%
%:%905=494%:%
%:%908=495%:%
%:%909=496%:%
%:%910=496%:%
%:%911=497%:%
%:%918=498%:%
%:%919=498%:%
%:%920=499%:%
%:%921=499%:%
%:%922=500%:%
%:%923=500%:%
%:%924=501%:%
%:%925=501%:%
%:%926=501%:%
%:%927=502%:%
%:%928=502%:%
%:%929=503%:%
%:%930=503%:%
%:%931=503%:%
%:%932=504%:%
%:%933=504%:%
%:%934=505%:%
%:%940=505%:%
%:%943=506%:%
%:%944=507%:%
%:%945=507%:%
%:%946=508%:%
%:%947=509%:%
%:%954=510%:%
%:%955=510%:%
%:%956=511%:%
%:%957=511%:%
%:%958=512%:%
%:%959=513%:%
%:%960=513%:%
%:%961=514%:%
%:%962=514%:%
%:%963=514%:%
%:%964=515%:%
%:%965=515%:%
%:%966=516%:%
%:%967=516%:%
%:%968=516%:%
%:%969=517%:%
%:%970=517%:%
%:%971=518%:%
%:%972=518%:%
%:%973=518%:%
%:%974=519%:%
%:%975=519%:%
%:%976=520%:%
%:%982=520%:%
%:%985=521%:%
%:%986=522%:%
%:%987=522%:%
%:%988=523%:%
%:%989=524%:%
%:%996=525%:%
%:%997=525%:%
%:%998=526%:%
%:%999=526%:%
%:%1000=527%:%
%:%1001=528%:%
%:%1002=528%:%
%:%1003=529%:%
%:%1004=529%:%
%:%1005=529%:%
%:%1006=530%:%
%:%1007=530%:%
%:%1008=531%:%
%:%1009=531%:%
%:%1010=531%:%
%:%1011=532%:%
%:%1012=532%:%
%:%1013=533%:%
%:%1014=533%:%
%:%1015=533%:%
%:%1016=534%:%
%:%1017=534%:%
%:%1018=535%:%
%:%1024=535%:%
%:%1027=536%:%
%:%1028=537%:%
%:%1029=537%:%
%:%1030=538%:%
%:%1031=539%:%
%:%1038=540%:%
%:%1039=540%:%
%:%1040=541%:%
%:%1041=541%:%
%:%1042=542%:%
%:%1043=543%:%
%:%1044=543%:%
%:%1045=544%:%
%:%1046=544%:%
%:%1047=544%:%
%:%1048=545%:%
%:%1049=545%:%
%:%1050=546%:%
%:%1051=546%:%
%:%1052=546%:%
%:%1053=547%:%
%:%1054=547%:%
%:%1055=548%:%
%:%1056=548%:%
%:%1057=548%:%
%:%1058=549%:%
%:%1059=549%:%
%:%1060=550%:%
%:%1070=552%:%
%:%1071=553%:%
%:%1072=554%:%
%:%1073=555%:%
%:%1074=556%:%
%:%1075=557%:%
%:%1076=558%:%
%:%1077=559%:%
%:%1078=560%:%
%:%1079=561%:%
%:%1080=562%:%
%:%1081=563%:%
%:%1082=564%:%
%:%1083=565%:%
%:%1084=566%:%
%:%1085=567%:%
%:%1086=568%:%
%:%1087=569%:%
%:%1088=570%:%
%:%1089=571%:%
%:%1090=572%:%
%:%1091=573%:%
%:%1092=574%:%
%:%1093=575%:%
%:%1094=576%:%
%:%1095=577%:%
%:%1096=578%:%
%:%1097=579%:%
%:%1098=580%:%
%:%1099=581%:%
%:%1100=582%:%
%:%1101=583%:%
%:%1102=584%:%
%:%1103=585%:%
%:%1105=587%:%
%:%1106=587%:%
%:%1113=588%:%
%:%1114=588%:%
%:%1115=589%:%
%:%1116=589%:%
%:%1117=590%:%
%:%1118=590%:%
%:%1119=590%:%
%:%1120=591%:%
%:%1121=591%:%
%:%1122=592%:%
%:%1123=592%:%
%:%1124=593%:%
%:%1125=593%:%
%:%1126=594%:%
%:%1127=594%:%
%:%1128=594%:%
%:%1129=595%:%
%:%1130=595%:%
%:%1131=596%:%
%:%1132=596%:%
%:%1133=597%:%
%:%1134=597%:%
%:%1135=598%:%
%:%1136=598%:%
%:%1137=598%:%
%:%1138=599%:%
%:%1139=599%:%
%:%1140=600%:%
%:%1141=600%:%
%:%1142=601%:%
%:%1143=601%:%
%:%1144=602%:%
%:%1145=602%:%
%:%1146=602%:%
%:%1147=603%:%
%:%1148=603%:%
%:%1149=604%:%
%:%1150=604%:%
%:%1151=605%:%
%:%1152=605%:%
%:%1153=606%:%
%:%1154=606%:%
%:%1155=606%:%
%:%1156=607%:%
%:%1157=607%:%
%:%1158=608%:%
%:%1159=608%:%
%:%1160=609%:%
%:%1161=609%:%
%:%1162=610%:%
%:%1163=610%:%
%:%1164=610%:%
%:%1165=611%:%
%:%1166=611%:%
%:%1167=612%:%
%:%1177=614%:%
%:%1179=616%:%
%:%1180=616%:%
%:%1183=617%:%
%:%1187=617%:%
%:%1188=617%:%
%:%1197=619%:%
%:%1198=620%:%
%:%1199=621%:%
%:%1200=622%:%
%:%1201=623%:%
%:%1202=624%:%
%:%1203=625%:%
%:%1204=626%:%
%:%1206=628%:%
%:%1207=628%:%
%:%1208=629%:%
%:%1209=630%:%
%:%1216=631%:%
%:%1217=631%:%
%:%1218=632%:%
%:%1219=632%:%
%:%1220=633%:%
%:%1221=633%:%
%:%1222=634%:%
%:%1223=634%:%
%:%1224=635%:%
%:%1225=635%:%
%:%1226=635%:%
%:%1227=636%:%
%:%1228=636%:%
%:%1229=637%:%
%:%1235=637%:%
%:%1238=638%:%
%:%1239=639%:%
%:%1240=639%:%
%:%1241=640%:%
%:%1242=641%:%
%:%1249=642%:%
%:%1250=642%:%
%:%1251=643%:%
%:%1252=643%:%
%:%1253=644%:%
%:%1254=644%:%
%:%1255=645%:%
%:%1256=645%:%
%:%1257=646%:%
%:%1258=646%:%
%:%1259=646%:%
%:%1260=647%:%
%:%1261=647%:%
%:%1262=648%:%
%:%1272=650%:%
%:%1273=651%:%
%:%1275=653%:%
%:%1276=653%:%
%:%1279=654%:%
%:%1283=654%:%
%:%1284=654%:%
%:%1289=654%:%
%:%1292=655%:%
%:%1293=656%:%
%:%1294=656%:%
%:%1297=657%:%
%:%1301=657%:%
%:%1302=657%:%
%:%1311=659%:%
%:%1312=660%:%
%:%1313=661%:%
%:%1314=662%:%
%:%1315=663%:%
%:%1316=664%:%
%:%1317=665%:%
%:%1318=666%:%
%:%1319=667%:%
%:%1320=668%:%
%:%1321=669%:%
%:%1322=670%:%
%:%1323=671%:%
%:%1324=672%:%
%:%1325=673%:%
%:%1326=674%:%
%:%1327=675%:%
%:%1328=676%:%
%:%1329=677%:%
%:%1330=678%:%
%:%1331=679%:%
%:%1332=680%:%
%:%1333=681%:%
%:%1334=682%:%
%:%1335=683%:%
%:%1336=684%:%
%:%1337=685%:%
%:%1338=686%:%
%:%1339=687%:%
%:%1340=688%:%
%:%1341=689%:%
%:%1342=690%:%
%:%1343=691%:%
%:%1344=692%:%
%:%1345=693%:%
%:%1346=694%:%
%:%1347=695%:%
%:%1348=696%:%
%:%1349=697%:%
%:%1350=698%:%
%:%1351=699%:%
%:%1352=700%:%
%:%1353=701%:%
%:%1354=702%:%
%:%1355=703%:%
%:%1356=704%:%
%:%1357=705%:%
%:%1358=706%:%
%:%1359=707%:%
%:%1360=708%:%
%:%1361=709%:%
%:%1362=710%:%
%:%1363=711%:%
%:%1364=712%:%
%:%1365=713%:%
%:%1366=714%:%
%:%1367=715%:%
%:%1368=716%:%
%:%1369=717%:%
%:%1370=718%:%
%:%1371=719%:%
%:%1372=720%:%
%:%1373=721%:%
%:%1375=723%:%
%:%1376=723%:%
%:%1379=724%:%
%:%1383=724%:%
%:%1393=726%:%
%:%1394=727%:%
%:%1395=728%:%
%:%1396=729%:%
%:%1397=730%:%
%:%1398=731%:%
%:%1399=732%:%
%:%1400=733%:%
%:%1401=734%:%
%:%1402=735%:%
%:%1403=736%:%
%:%1404=737%:%
%:%1405=738%:%
%:%1407=740%:%
%:%1408=740%:%
%:%1409=741%:%
%:%1412=742%:%
%:%1416=742%:%
%:%1417=742%:%
%:%1418=743%:%
%:%1419=744%:%
%:%1420=744%:%
%:%1421=745%:%
%:%1422=745%:%
%:%1423=746%:%
%:%1424=747%:%
%:%1425=747%:%
%:%1426=748%:%
%:%1427=749%:%
%:%1428=749%:%
%:%1429=750%:%
%:%1430=751%:%
%:%1431=751%:%
%:%1432=752%:%
%:%1433=753%:%
%:%1434=753%:%
%:%1439=753%:%
%:%1442=754%:%
%:%1445=756%:%
%:%1446=757%:%
%:%1447=758%:%
%:%1448=759%:%
%:%1449=760%:%
%:%1450=761%:%
%:%1451=762%:%
%:%1452=763%:%
%:%1453=764%:%
%:%1454=765%:%
%:%1455=766%:%
%:%1456=767%:%
%:%1457=768%:%
%:%1458=769%:%
%:%1459=770%:%
%:%1461=772%:%
%:%1462=772%:%
%:%1463=773%:%
%:%1470=774%:%
%:%1471=774%:%
%:%1472=775%:%
%:%1473=775%:%
%:%1474=776%:%
%:%1475=776%:%
%:%1476=777%:%
%:%1477=777%:%
%:%1478=777%:%
%:%1479=778%:%
%:%1480=778%:%
%:%1481=779%:%
%:%1482=779%:%
%:%1483=779%:%
%:%1484=780%:%
%:%1485=780%:%
%:%1486=781%:%
%:%1487=781%:%
%:%1488=781%:%
%:%1489=782%:%
%:%1490=782%:%
%:%1491=783%:%
%:%1492=783%:%
%:%1493=783%:%
%:%1494=784%:%
%:%1495=784%:%
%:%1496=785%:%
%:%1497=785%:%
%:%1498=785%:%
%:%1499=786%:%
%:%1500=786%:%
%:%1501=787%:%
%:%1507=787%:%
%:%1510=788%:%
%:%1511=789%:%
%:%1512=789%:%
%:%1513=790%:%
%:%1520=791%:%
%:%1521=791%:%
%:%1522=792%:%
%:%1523=792%:%
%:%1524=793%:%
%:%1525=793%:%
%:%1526=794%:%
%:%1527=794%:%
%:%1528=794%:%
%:%1529=795%:%
%:%1530=795%:%
%:%1531=796%:%
%:%1532=796%:%
%:%1533=796%:%
%:%1534=797%:%
%:%1535=797%:%
%:%1536=798%:%
%:%1537=798%:%
%:%1538=798%:%
%:%1539=799%:%
%:%1540=799%:%
%:%1541=800%:%
%:%1547=800%:%
%:%1550=801%:%
%:%1551=802%:%
%:%1552=802%:%
%:%1553=803%:%
%:%1554=804%:%
%:%1561=805%:%
%:%1562=805%:%
%:%1563=806%:%
%:%1564=806%:%
%:%1565=807%:%
%:%1566=807%:%
%:%1567=808%:%
%:%1568=808%:%
%:%1569=808%:%
%:%1570=809%:%
%:%1571=809%:%
%:%1572=810%:%
%:%1573=810%:%
%:%1574=810%:%
%:%1575=811%:%
%:%1576=811%:%
%:%1577=812%:%
%:%1578=812%:%
%:%1579=812%:%
%:%1580=813%:%
%:%1581=813%:%
%:%1582=814%:%
%:%1583=814%:%
%:%1584=814%:%
%:%1585=815%:%
%:%1586=815%:%
%:%1587=816%:%
%:%1593=816%:%
%:%1596=817%:%
%:%1597=818%:%
%:%1598=818%:%
%:%1599=819%:%
%:%1600=820%:%
%:%1601=821%:%
%:%1608=822%:%
%:%1609=822%:%
%:%1610=823%:%
%:%1611=823%:%
%:%1612=824%:%
%:%1613=824%:%
%:%1614=825%:%
%:%1615=825%:%
%:%1616=825%:%
%:%1617=826%:%
%:%1618=826%:%
%:%1619=827%:%
%:%1620=827%:%
%:%1621=827%:%
%:%1622=828%:%
%:%1623=828%:%
%:%1624=829%:%
%:%1625=829%:%
%:%1626=830%:%
%:%1627=830%:%
%:%1628=830%:%
%:%1629=831%:%
%:%1630=831%:%
%:%1631=832%:%
%:%1632=832%:%
%:%1633=832%:%
%:%1634=833%:%
%:%1635=833%:%
%:%1636=834%:%
%:%1637=834%:%
%:%1638=834%:%
%:%1639=835%:%
%:%1640=835%:%
%:%1641=836%:%
%:%1647=836%:%
%:%1650=837%:%
%:%1651=838%:%
%:%1652=838%:%
%:%1653=839%:%
%:%1654=840%:%
%:%1655=841%:%
%:%1662=842%:%
%:%1663=842%:%
%:%1664=843%:%
%:%1665=843%:%
%:%1666=844%:%
%:%1667=844%:%
%:%1668=845%:%
%:%1669=845%:%
%:%1670=845%:%
%:%1671=846%:%
%:%1672=846%:%
%:%1673=847%:%
%:%1674=847%:%
%:%1675=847%:%
%:%1676=848%:%
%:%1677=848%:%
%:%1678=849%:%
%:%1679=849%:%
%:%1680=850%:%
%:%1681=850%:%
%:%1682=850%:%
%:%1683=851%:%
%:%1684=851%:%
%:%1685=852%:%
%:%1686=852%:%
%:%1687=852%:%
%:%1688=853%:%
%:%1689=853%:%
%:%1690=854%:%
%:%1691=854%:%
%:%1692=854%:%
%:%1693=855%:%
%:%1694=855%:%
%:%1695=856%:%
%:%1701=856%:%
%:%1704=857%:%
%:%1705=858%:%
%:%1706=858%:%
%:%1707=859%:%
%:%1708=860%:%
%:%1709=861%:%
%:%1716=862%:%
%:%1717=862%:%
%:%1718=863%:%
%:%1719=863%:%
%:%1720=864%:%
%:%1721=864%:%
%:%1722=865%:%
%:%1723=865%:%
%:%1724=865%:%
%:%1725=866%:%
%:%1726=866%:%
%:%1727=867%:%
%:%1728=867%:%
%:%1729=867%:%
%:%1730=868%:%
%:%1731=868%:%
%:%1732=869%:%
%:%1733=869%:%
%:%1734=870%:%
%:%1735=870%:%
%:%1736=870%:%
%:%1737=871%:%
%:%1738=871%:%
%:%1739=872%:%
%:%1740=872%:%
%:%1741=872%:%
%:%1742=873%:%
%:%1743=873%:%
%:%1744=874%:%
%:%1745=874%:%
%:%1746=874%:%
%:%1747=875%:%
%:%1748=875%:%
%:%1749=876%:%
%:%1755=876%:%
%:%1758=877%:%
%:%1759=878%:%
%:%1760=878%:%
%:%1761=879%:%
%:%1768=880%:%
%:%1769=880%:%
%:%1770=881%:%
%:%1771=881%:%
%:%1772=882%:%
%:%1773=882%:%
%:%1774=882%:%
%:%1775=882%:%
%:%1776=883%:%
%:%1777=883%:%
%:%1778=884%:%
%:%1779=884%:%
%:%1780=885%:%
%:%1781=885%:%
%:%1782=885%:%
%:%1783=885%:%
%:%1784=886%:%
%:%1785=886%:%
%:%1786=887%:%
%:%1787=887%:%
%:%1788=888%:%
%:%1789=888%:%
%:%1790=888%:%
%:%1791=888%:%
%:%1792=889%:%
%:%1793=889%:%
%:%1794=890%:%
%:%1795=890%:%
%:%1796=891%:%
%:%1797=891%:%
%:%1798=891%:%
%:%1799=891%:%
%:%1800=892%:%
%:%1801=892%:%
%:%1802=893%:%
%:%1803=893%:%
%:%1804=894%:%
%:%1805=894%:%
%:%1806=894%:%
%:%1807=894%:%
%:%1808=895%:%
%:%1809=895%:%
%:%1810=896%:%
%:%1811=896%:%
%:%1812=897%:%
%:%1813=897%:%
%:%1814=897%:%
%:%1815=897%:%
%:%1816=898%:%
%:%1826=900%:%
%:%1827=901%:%
%:%1829=903%:%
%:%1830=903%:%
%:%1833=904%:%
%:%1837=904%:%
%:%1838=904%:%
%:%1847=906%:%
%:%1848=907%:%
%:%1849=908%:%
%:%1850=909%:%
%:%1851=910%:%
%:%1852=911%:%
%:%1853=912%:%
%:%1854=913%:%
%:%1855=914%:%
%:%1856=915%:%
%:%1857=916%:%
%:%1858=917%:%
%:%1859=918%:%
%:%1860=919%:%
%:%1861=920%:%
%:%1862=921%:%
%:%1863=922%:%
%:%1864=923%:%
%:%1865=924%:%
%:%1866=925%:%
%:%1867=926%:%
%:%1868=927%:%
%:%1869=928%:%
%:%1870=929%:%
%:%1871=930%:%
%:%1872=931%:%
%:%1873=932%:%
%:%1874=933%:%
%:%1875=934%:%
%:%1876=935%:%
%:%1877=936%:%
%:%1878=937%:%
%:%1879=938%:%
%:%1880=939%:%
%:%1881=940%:%
%:%1882=941%:%
%:%1883=942%:%
%:%1884=943%:%
%:%1885=944%:%
%:%1886=945%:%
%:%1887=946%:%
%:%1888=947%:%
%:%1889=948%:%
%:%1890=949%:%
%:%1891=950%:%
%:%1892=951%:%
%:%1893=952%:%
%:%1894=953%:%
%:%1895=954%:%
%:%1896=955%:%
%:%1897=956%:%
%:%1898=957%:%
%:%1899=958%:%
%:%1900=959%:%
%:%1901=960%:%
%:%1902=961%:%
%:%1903=962%:%
%:%1904=963%:%
%:%1905=964%:%
%:%1906=965%:%
%:%1907=966%:%
%:%1908=967%:%
%:%1909=968%:%
%:%1911=970%:%
%:%1912=970%:%
%:%1915=971%:%
%:%1919=971%:%
%:%1929=973%:%
%:%1931=975%:%
%:%1932=975%:%
%:%1933=976%:%
%:%1934=977%:%
%:%1941=978%:%
%:%1942=978%:%
%:%1943=979%:%
%:%1944=979%:%
%:%1945=980%:%
%:%1946=980%:%
%:%1947=981%:%
%:%1948=981%:%
%:%1949=982%:%
%:%1950=982%:%
%:%1951=982%:%
%:%1952=983%:%
%:%1953=983%:%
%:%1954=984%:%
%:%1955=984%:%
%:%1956=984%:%
%:%1957=985%:%
%:%1958=985%:%
%:%1959=986%:%
%:%1965=986%:%
%:%1968=987%:%
%:%1969=988%:%
%:%1970=988%:%
%:%1971=989%:%
%:%1972=990%:%
%:%1979=991%:%
%:%1980=991%:%
%:%1981=992%:%
%:%1982=992%:%
%:%1983=993%:%
%:%1984=993%:%
%:%1985=994%:%
%:%1986=994%:%
%:%1987=995%:%
%:%1988=995%:%
%:%1989=995%:%
%:%1990=996%:%
%:%1991=996%:%
%:%1992=997%:%
%:%1993=997%:%
%:%1994=997%:%
%:%1995=998%:%
%:%1996=998%:%
%:%1997=999%:%
%:%2003=999%:%
%:%2006=1000%:%
%:%2007=1001%:%
%:%2008=1001%:%
%:%2009=1002%:%
%:%2010=1003%:%
%:%2011=1004%:%
%:%2018=1005%:%
%:%2019=1005%:%
%:%2020=1006%:%
%:%2021=1006%:%
%:%2022=1007%:%
%:%2023=1007%:%
%:%2024=1008%:%
%:%2025=1008%:%
%:%2026=1009%:%
%:%2027=1009%:%
%:%2028=1009%:%
%:%2029=1010%:%
%:%2030=1010%:%
%:%2031=1011%:%
%:%2032=1011%:%
%:%2033=1011%:%
%:%2034=1012%:%
%:%2035=1012%:%
%:%2036=1013%:%
%:%2037=1013%:%
%:%2038=1014%:%
%:%2039=1014%:%
%:%2040=1014%:%
%:%2041=1015%:%
%:%2042=1015%:%
%:%2043=1016%:%
%:%2044=1016%:%
%:%2045=1016%:%
%:%2046=1017%:%
%:%2047=1017%:%
%:%2048=1018%:%
%:%2049=1018%:%
%:%2050=1019%:%
%:%2051=1019%:%
%:%2052=1020%:%
%:%2053=1020%:%
%:%2054=1020%:%
%:%2055=1021%:%
%:%2056=1021%:%
%:%2057=1022%:%
%:%2058=1022%:%
%:%2059=1022%:%
%:%2060=1023%:%
%:%2061=1023%:%
%:%2062=1024%:%
%:%2063=1024%:%
%:%2064=1024%:%
%:%2065=1025%:%
%:%2066=1025%:%
%:%2067=1026%:%
%:%2068=1026%:%
%:%2069=1027%:%
%:%2075=1027%:%
%:%2078=1028%:%
%:%2079=1029%:%
%:%2080=1029%:%
%:%2081=1030%:%
%:%2082=1031%:%
%:%2083=1032%:%
%:%2084=1033%:%
%:%2091=1034%:%
%:%2092=1034%:%
%:%2093=1035%:%
%:%2094=1035%:%
%:%2095=1036%:%
%:%2096=1036%:%
%:%2097=1037%:%
%:%2098=1037%:%
%:%2099=1038%:%
%:%2100=1038%:%
%:%2101=1038%:%
%:%2102=1039%:%
%:%2103=1039%:%
%:%2104=1040%:%
%:%2105=1040%:%
%:%2106=1040%:%
%:%2107=1041%:%
%:%2108=1041%:%
%:%2109=1042%:%
%:%2110=1042%:%
%:%2111=1043%:%
%:%2112=1043%:%
%:%2113=1043%:%
%:%2114=1044%:%
%:%2115=1044%:%
%:%2116=1045%:%
%:%2117=1045%:%
%:%2118=1045%:%
%:%2119=1046%:%
%:%2120=1046%:%
%:%2121=1047%:%
%:%2122=1047%:%
%:%2123=1048%:%
%:%2124=1048%:%
%:%2125=1049%:%
%:%2126=1049%:%
%:%2127=1049%:%
%:%2128=1050%:%
%:%2129=1050%:%
%:%2130=1051%:%
%:%2131=1051%:%
%:%2132=1051%:%
%:%2133=1052%:%
%:%2134=1052%:%
%:%2135=1053%:%
%:%2136=1053%:%
%:%2137=1054%:%
%:%2138=1054%:%
%:%2139=1054%:%
%:%2140=1055%:%
%:%2141=1055%:%
%:%2142=1056%:%
%:%2143=1056%:%
%:%2144=1056%:%
%:%2145=1057%:%
%:%2146=1057%:%
%:%2147=1058%:%
%:%2148=1058%:%
%:%2149=1058%:%
%:%2150=1059%:%
%:%2151=1059%:%
%:%2152=1060%:%
%:%2153=1060%:%
%:%2154=1060%:%
%:%2155=1061%:%
%:%2156=1061%:%
%:%2157=1062%:%
%:%2158=1062%:%
%:%2159=1063%:%
%:%2160=1063%:%
%:%2161=1064%:%
%:%2162=1064%:%
%:%2163=1064%:%
%:%2164=1065%:%
%:%2165=1065%:%
%:%2166=1066%:%
%:%2167=1066%:%
%:%2168=1066%:%
%:%2169=1067%:%
%:%2170=1067%:%
%:%2171=1068%:%
%:%2172=1068%:%
%:%2173=1068%:%
%:%2174=1069%:%
%:%2175=1069%:%
%:%2176=1070%:%
%:%2177=1070%:%
%:%2178=1070%:%
%:%2179=1071%:%
%:%2180=1071%:%
%:%2181=1072%:%
%:%2182=1072%:%
%:%2183=1073%:%
%:%2184=1073%:%
%:%2185=1074%:%
%:%2191=1074%:%
%:%2194=1075%:%
%:%2195=1076%:%
%:%2196=1076%:%
%:%2197=1077%:%
%:%2198=1078%:%
%:%2199=1079%:%
%:%2200=1080%:%
%:%2207=1081%:%
%:%2208=1081%:%
%:%2209=1082%:%
%:%2210=1082%:%
%:%2211=1083%:%
%:%2212=1083%:%
%:%2213=1084%:%
%:%2214=1084%:%
%:%2215=1085%:%
%:%2216=1085%:%
%:%2217=1085%:%
%:%2218=1086%:%
%:%2219=1086%:%
%:%2220=1087%:%
%:%2221=1087%:%
%:%2222=1087%:%
%:%2223=1088%:%
%:%2224=1088%:%
%:%2225=1089%:%
%:%2226=1089%:%
%:%2227=1090%:%
%:%2228=1090%:%
%:%2229=1090%:%
%:%2230=1091%:%
%:%2231=1091%:%
%:%2232=1092%:%
%:%2233=1092%:%
%:%2234=1092%:%
%:%2235=1093%:%
%:%2236=1093%:%
%:%2237=1094%:%
%:%2238=1094%:%
%:%2239=1095%:%
%:%2240=1095%:%
%:%2241=1096%:%
%:%2242=1096%:%
%:%2243=1096%:%
%:%2244=1097%:%
%:%2245=1097%:%
%:%2246=1098%:%
%:%2247=1098%:%
%:%2248=1098%:%
%:%2249=1099%:%
%:%2250=1099%:%
%:%2251=1100%:%
%:%2252=1100%:%
%:%2253=1101%:%
%:%2254=1101%:%
%:%2255=1101%:%
%:%2256=1102%:%
%:%2257=1102%:%
%:%2258=1103%:%
%:%2259=1103%:%
%:%2260=1103%:%
%:%2261=1104%:%
%:%2262=1104%:%
%:%2263=1105%:%
%:%2264=1105%:%
%:%2265=1105%:%
%:%2266=1106%:%
%:%2267=1106%:%
%:%2268=1107%:%
%:%2269=1107%:%
%:%2270=1107%:%
%:%2271=1108%:%
%:%2272=1108%:%
%:%2273=1109%:%
%:%2274=1109%:%
%:%2275=1110%:%
%:%2276=1110%:%
%:%2277=1111%:%
%:%2278=1111%:%
%:%2279=1111%:%
%:%2280=1112%:%
%:%2281=1112%:%
%:%2282=1113%:%
%:%2283=1113%:%
%:%2284=1113%:%
%:%2285=1114%:%
%:%2286=1114%:%
%:%2287=1115%:%
%:%2288=1115%:%
%:%2289=1115%:%
%:%2290=1116%:%
%:%2291=1116%:%
%:%2292=1117%:%
%:%2293=1117%:%
%:%2294=1117%:%
%:%2295=1118%:%
%:%2296=1118%:%
%:%2297=1119%:%
%:%2298=1119%:%
%:%2299=1120%:%
%:%2300=1120%:%
%:%2301=1121%:%
%:%2307=1121%:%
%:%2310=1122%:%
%:%2311=1123%:%
%:%2312=1123%:%
%:%2313=1124%:%
%:%2314=1125%:%
%:%2315=1126%:%
%:%2316=1127%:%
%:%2323=1128%:%
%:%2324=1128%:%
%:%2325=1129%:%
%:%2326=1129%:%
%:%2327=1130%:%
%:%2328=1130%:%
%:%2329=1131%:%
%:%2330=1131%:%
%:%2331=1132%:%
%:%2332=1132%:%
%:%2333=1132%:%
%:%2334=1133%:%
%:%2335=1133%:%
%:%2336=1134%:%
%:%2337=1134%:%
%:%2338=1134%:%
%:%2339=1135%:%
%:%2340=1135%:%
%:%2341=1136%:%
%:%2342=1136%:%
%:%2343=1137%:%
%:%2344=1137%:%
%:%2345=1137%:%
%:%2346=1138%:%
%:%2347=1138%:%
%:%2348=1139%:%
%:%2349=1139%:%
%:%2350=1139%:%
%:%2351=1140%:%
%:%2352=1140%:%
%:%2353=1141%:%
%:%2354=1141%:%
%:%2355=1142%:%
%:%2356=1142%:%
%:%2357=1143%:%
%:%2358=1143%:%
%:%2359=1143%:%
%:%2360=1144%:%
%:%2361=1144%:%
%:%2362=1145%:%
%:%2363=1145%:%
%:%2364=1145%:%
%:%2365=1146%:%
%:%2366=1146%:%
%:%2367=1147%:%
%:%2368=1147%:%
%:%2369=1148%:%
%:%2370=1148%:%
%:%2371=1148%:%
%:%2372=1149%:%
%:%2373=1149%:%
%:%2374=1150%:%
%:%2375=1150%:%
%:%2376=1150%:%
%:%2377=1151%:%
%:%2378=1151%:%
%:%2379=1152%:%
%:%2380=1152%:%
%:%2381=1152%:%
%:%2382=1153%:%
%:%2383=1153%:%
%:%2384=1154%:%
%:%2385=1154%:%
%:%2386=1154%:%
%:%2387=1155%:%
%:%2388=1155%:%
%:%2389=1156%:%
%:%2390=1156%:%
%:%2391=1157%:%
%:%2392=1157%:%
%:%2393=1158%:%
%:%2394=1158%:%
%:%2395=1158%:%
%:%2396=1159%:%
%:%2397=1159%:%
%:%2398=1160%:%
%:%2399=1160%:%
%:%2400=1160%:%
%:%2401=1161%:%
%:%2402=1161%:%
%:%2403=1162%:%
%:%2404=1162%:%
%:%2405=1162%:%
%:%2406=1163%:%
%:%2407=1163%:%
%:%2408=1164%:%
%:%2409=1164%:%
%:%2410=1164%:%
%:%2411=1165%:%
%:%2412=1165%:%
%:%2413=1166%:%
%:%2414=1166%:%
%:%2415=1167%:%
%:%2416=1167%:%
%:%2417=1168%:%
%:%2423=1168%:%
%:%2426=1169%:%
%:%2427=1170%:%
%:%2428=1170%:%
%:%2429=1171%:%
%:%2436=1172%:%
%:%2437=1172%:%
%:%2438=1173%:%
%:%2439=1173%:%
%:%2440=1174%:%
%:%2441=1174%:%
%:%2442=1174%:%
%:%2443=1174%:%
%:%2444=1175%:%
%:%2445=1175%:%
%:%2446=1176%:%
%:%2447=1176%:%
%:%2448=1177%:%
%:%2449=1177%:%
%:%2450=1177%:%
%:%2451=1177%:%
%:%2452=1178%:%
%:%2453=1178%:%
%:%2454=1179%:%
%:%2455=1179%:%
%:%2456=1180%:%
%:%2457=1180%:%
%:%2458=1180%:%
%:%2459=1180%:%
%:%2460=1181%:%
%:%2461=1181%:%
%:%2462=1182%:%
%:%2463=1182%:%
%:%2464=1183%:%
%:%2465=1183%:%
%:%2466=1183%:%
%:%2467=1183%:%
%:%2468=1184%:%
%:%2469=1184%:%
%:%2470=1185%:%
%:%2471=1185%:%
%:%2472=1186%:%
%:%2473=1186%:%
%:%2474=1186%:%
%:%2475=1186%:%
%:%2476=1187%:%
%:%2477=1187%:%
%:%2478=1188%:%
%:%2479=1188%:%
%:%2480=1189%:%
%:%2481=1189%:%
%:%2482=1189%:%
%:%2483=1189%:%
%:%2484=1190%:%
%:%2494=1192%:%
%:%2496=1194%:%
%:%2497=1194%:%
%:%2500=1195%:%
%:%2504=1195%:%
%:%2505=1195%:%
%:%2514=1197%:%
%:%2515=1198%:%
%:%2516=1199%:%
%:%2517=1200%:%
%:%2518=1201%:%
%:%2519=1202%:%
%:%2520=1203%:%
%:%2521=1204%:%
%:%2522=1205%:%
%:%2523=1206%:%
%:%2524=1207%:%
%:%2525=1208%:%
%:%2526=1209%:%
%:%2527=1210%:%
%:%2528=1211%:%
%:%2529=1212%:%
%:%2530=1213%:%
%:%2531=1214%:%
%:%2532=1215%:%
%:%2533=1216%:%
%:%2534=1217%:%
%:%2535=1218%:%
%:%2536=1219%:%
%:%2537=1220%:%
%:%2538=1221%:%
%:%2539=1222%:%
%:%2540=1223%:%
%:%2541=1224%:%
%:%2542=1225%:%
%:%2543=1226%:%
%:%2544=1227%:%
%:%2545=1228%:%
%:%2546=1229%:%
%:%2547=1230%:%
%:%2548=1231%:%
%:%2549=1232%:%
%:%2550=1233%:%
%:%2551=1234%:%
%:%2552=1235%:%
%:%2553=1236%:%
%:%2554=1237%:%
%:%2555=1238%:%
%:%2556=1239%:%
%:%2557=1240%:%
%:%2558=1241%:%
%:%2559=1242%:%
%:%2560=1243%:%
%:%2561=1244%:%
%:%2562=1245%:%
%:%2563=1246%:%
%:%2564=1247%:%
%:%2565=1248%:%
%:%2566=1249%:%
%:%2567=1250%:%
%:%2568=1251%:%
%:%2569=1252%:%
%:%2570=1253%:%
%:%2571=1254%:%
%:%2572=1255%:%
%:%2573=1256%:%
%:%2574=1257%:%
%:%2575=1258%:%
%:%2576=1259%:%
%:%2577=1260%:%
%:%2578=1261%:%
%:%2579=1262%:%
%:%2580=1263%:%
%:%2581=1264%:%
%:%2582=1265%:%
%:%2584=1267%:%
%:%2585=1267%:%
%:%2586=1268%:%
%:%2587=1269%:%
%:%2594=1270%:%
%:%2595=1270%:%
%:%2596=1271%:%
%:%2597=1271%:%
%:%2598=1271%:%
%:%2599=1272%:%
%:%2600=1272%:%
%:%2601=1273%:%
%:%2602=1273%:%
%:%2603=1273%:%
%:%2604=1274%:%
%:%2605=1274%:%
%:%2606=1275%:%
%:%2607=1275%:%
%:%2608=1275%:%
%:%2609=1276%:%
%:%2610=1276%:%
%:%2611=1277%:%
%:%2617=1277%:%
%:%2620=1278%:%
%:%2621=1279%:%
%:%2622=1279%:%
%:%2623=1280%:%
%:%2624=1281%:%
%:%2631=1282%:%
%:%2632=1282%:%
%:%2633=1283%:%
%:%2634=1283%:%
%:%2635=1284%:%
%:%2636=1284%:%
%:%2637=1285%:%
%:%2638=1285%:%
%:%2639=1286%:%
%:%2640=1286%:%
%:%2641=1286%:%
%:%2642=1287%:%
%:%2643=1287%:%
%:%2644=1288%:%
%:%2645=1288%:%
%:%2646=1288%:%
%:%2647=1289%:%
%:%2648=1289%:%
%:%2649=1290%:%
%:%2655=1290%:%
%:%2658=1291%:%
%:%2659=1292%:%
%:%2660=1292%:%
%:%2661=1293%:%
%:%2662=1294%:%
%:%2663=1295%:%
%:%2670=1296%:%
%:%2671=1296%:%
%:%2672=1297%:%
%:%2673=1297%:%
%:%2674=1298%:%
%:%2675=1298%:%
%:%2676=1299%:%
%:%2677=1299%:%
%:%2678=1300%:%
%:%2679=1300%:%
%:%2680=1300%:%
%:%2681=1301%:%
%:%2682=1301%:%
%:%2683=1302%:%
%:%2684=1302%:%
%:%2685=1302%:%
%:%2686=1303%:%
%:%2687=1303%:%
%:%2688=1304%:%
%:%2689=1304%:%
%:%2690=1305%:%
%:%2691=1305%:%
%:%2692=1305%:%
%:%2693=1306%:%
%:%2694=1306%:%
%:%2695=1307%:%
%:%2696=1307%:%
%:%2697=1307%:%
%:%2698=1308%:%
%:%2699=1308%:%
%:%2700=1309%:%
%:%2701=1309%:%
%:%2702=1310%:%
%:%2703=1310%:%
%:%2704=1311%:%
%:%2705=1311%:%
%:%2706=1311%:%
%:%2707=1312%:%
%:%2708=1312%:%
%:%2709=1313%:%
%:%2710=1313%:%
%:%2711=1313%:%
%:%2712=1314%:%
%:%2713=1314%:%
%:%2714=1315%:%
%:%2715=1315%:%
%:%2716=1315%:%
%:%2717=1316%:%
%:%2718=1316%:%
%:%2719=1317%:%
%:%2720=1317%:%
%:%2721=1318%:%
%:%2727=1318%:%
%:%2730=1319%:%
%:%2731=1320%:%
%:%2732=1320%:%
%:%2733=1321%:%
%:%2734=1322%:%
%:%2735=1323%:%
%:%2736=1324%:%
%:%2737=1325%:%
%:%2738=1326%:%
%:%2745=1327%:%
%:%2746=1327%:%
%:%2747=1328%:%
%:%2748=1328%:%
%:%2749=1329%:%
%:%2750=1329%:%
%:%2751=1330%:%
%:%2752=1330%:%
%:%2753=1331%:%
%:%2754=1331%:%
%:%2755=1331%:%
%:%2756=1332%:%
%:%2757=1332%:%
%:%2758=1333%:%
%:%2759=1333%:%
%:%2760=1333%:%
%:%2761=1334%:%
%:%2762=1334%:%
%:%2763=1335%:%
%:%2764=1335%:%
%:%2765=1336%:%
%:%2766=1336%:%
%:%2767=1336%:%
%:%2768=1337%:%
%:%2769=1337%:%
%:%2770=1338%:%
%:%2771=1338%:%
%:%2772=1338%:%
%:%2773=1339%:%
%:%2774=1339%:%
%:%2775=1340%:%
%:%2776=1340%:%
%:%2777=1341%:%
%:%2778=1341%:%
%:%2779=1342%:%
%:%2780=1342%:%
%:%2781=1342%:%
%:%2782=1343%:%
%:%2783=1343%:%
%:%2784=1344%:%
%:%2785=1344%:%
%:%2786=1344%:%
%:%2787=1345%:%
%:%2788=1345%:%
%:%2789=1346%:%
%:%2790=1346%:%
%:%2791=1347%:%
%:%2792=1347%:%
%:%2793=1347%:%
%:%2794=1348%:%
%:%2795=1348%:%
%:%2796=1349%:%
%:%2797=1349%:%
%:%2798=1349%:%
%:%2799=1350%:%
%:%2800=1350%:%
%:%2801=1351%:%
%:%2802=1351%:%
%:%2803=1351%:%
%:%2804=1352%:%
%:%2805=1352%:%
%:%2806=1353%:%
%:%2807=1353%:%
%:%2808=1353%:%
%:%2809=1354%:%
%:%2810=1354%:%
%:%2811=1355%:%
%:%2812=1355%:%
%:%2813=1356%:%
%:%2814=1356%:%
%:%2815=1357%:%
%:%2816=1357%:%
%:%2817=1357%:%
%:%2818=1358%:%
%:%2819=1358%:%
%:%2820=1359%:%
%:%2821=1359%:%
%:%2822=1359%:%
%:%2823=1360%:%
%:%2824=1360%:%
%:%2825=1361%:%
%:%2826=1361%:%
%:%2827=1361%:%
%:%2828=1362%:%
%:%2829=1362%:%
%:%2830=1363%:%
%:%2831=1363%:%
%:%2832=1363%:%
%:%2833=1364%:%
%:%2834=1364%:%
%:%2835=1365%:%
%:%2836=1365%:%
%:%2837=1366%:%
%:%2838=1366%:%
%:%2839=1367%:%
%:%2845=1367%:%
%:%2848=1368%:%
%:%2849=1369%:%
%:%2850=1369%:%
%:%2851=1370%:%
%:%2852=1371%:%
%:%2853=1372%:%
%:%2854=1373%:%
%:%2855=1374%:%
%:%2856=1375%:%
%:%2863=1376%:%
%:%2864=1376%:%
%:%2865=1377%:%
%:%2866=1377%:%
%:%2867=1378%:%
%:%2868=1378%:%
%:%2869=1379%:%
%:%2870=1379%:%
%:%2871=1380%:%
%:%2872=1380%:%
%:%2873=1380%:%
%:%2874=1381%:%
%:%2875=1381%:%
%:%2876=1382%:%
%:%2877=1382%:%
%:%2878=1382%:%
%:%2879=1383%:%
%:%2880=1383%:%
%:%2881=1384%:%
%:%2882=1384%:%
%:%2883=1385%:%
%:%2884=1385%:%
%:%2885=1385%:%
%:%2886=1386%:%
%:%2887=1386%:%
%:%2888=1387%:%
%:%2889=1387%:%
%:%2890=1387%:%
%:%2891=1388%:%
%:%2892=1388%:%
%:%2893=1389%:%
%:%2894=1389%:%
%:%2895=1390%:%
%:%2896=1390%:%
%:%2897=1391%:%
%:%2898=1391%:%
%:%2899=1391%:%
%:%2900=1392%:%
%:%2901=1392%:%
%:%2902=1393%:%
%:%2903=1393%:%
%:%2904=1393%:%
%:%2905=1394%:%
%:%2906=1394%:%
%:%2907=1395%:%
%:%2908=1395%:%
%:%2909=1396%:%
%:%2910=1396%:%
%:%2911=1396%:%
%:%2912=1397%:%
%:%2913=1397%:%
%:%2914=1398%:%
%:%2915=1398%:%
%:%2916=1398%:%
%:%2917=1399%:%
%:%2918=1399%:%
%:%2919=1400%:%
%:%2920=1400%:%
%:%2921=1400%:%
%:%2922=1401%:%
%:%2923=1401%:%
%:%2924=1402%:%
%:%2925=1402%:%
%:%2926=1402%:%
%:%2927=1403%:%
%:%2928=1403%:%
%:%2929=1404%:%
%:%2930=1404%:%
%:%2931=1405%:%
%:%2932=1405%:%
%:%2933=1406%:%
%:%2934=1406%:%
%:%2935=1406%:%
%:%2936=1407%:%
%:%2937=1407%:%
%:%2938=1408%:%
%:%2939=1408%:%
%:%2940=1408%:%
%:%2941=1409%:%
%:%2942=1409%:%
%:%2943=1410%:%
%:%2944=1410%:%
%:%2945=1410%:%
%:%2946=1411%:%
%:%2947=1411%:%
%:%2948=1412%:%
%:%2949=1412%:%
%:%2950=1412%:%
%:%2951=1413%:%
%:%2952=1413%:%
%:%2953=1414%:%
%:%2954=1414%:%
%:%2955=1415%:%
%:%2956=1415%:%
%:%2957=1416%:%
%:%2963=1416%:%
%:%2966=1417%:%
%:%2967=1418%:%
%:%2968=1418%:%
%:%2969=1419%:%
%:%2970=1420%:%
%:%2971=1421%:%
%:%2972=1422%:%
%:%2973=1423%:%
%:%2974=1424%:%
%:%2981=1425%:%
%:%2982=1425%:%
%:%2983=1426%:%
%:%2984=1426%:%
%:%2985=1427%:%
%:%2986=1427%:%
%:%2987=1428%:%
%:%2988=1428%:%
%:%2989=1429%:%
%:%2990=1429%:%
%:%2991=1429%:%
%:%2992=1430%:%
%:%2993=1430%:%
%:%2994=1431%:%
%:%2995=1431%:%
%:%2996=1431%:%
%:%2997=1432%:%
%:%2998=1432%:%
%:%2999=1433%:%
%:%3000=1433%:%
%:%3001=1434%:%
%:%3002=1434%:%
%:%3003=1434%:%
%:%3004=1435%:%
%:%3005=1435%:%
%:%3006=1436%:%
%:%3007=1436%:%
%:%3008=1436%:%
%:%3009=1437%:%
%:%3010=1437%:%
%:%3011=1438%:%
%:%3012=1438%:%
%:%3013=1439%:%
%:%3014=1439%:%
%:%3015=1440%:%
%:%3016=1440%:%
%:%3017=1440%:%
%:%3018=1441%:%
%:%3019=1441%:%
%:%3020=1442%:%
%:%3021=1442%:%
%:%3022=1442%:%
%:%3023=1443%:%
%:%3024=1443%:%
%:%3025=1444%:%
%:%3026=1444%:%
%:%3027=1445%:%
%:%3028=1445%:%
%:%3029=1445%:%
%:%3030=1446%:%
%:%3031=1446%:%
%:%3032=1447%:%
%:%3033=1447%:%
%:%3034=1447%:%
%:%3035=1448%:%
%:%3036=1448%:%
%:%3037=1449%:%
%:%3038=1449%:%
%:%3039=1449%:%
%:%3040=1450%:%
%:%3041=1450%:%
%:%3042=1451%:%
%:%3043=1451%:%
%:%3044=1451%:%
%:%3045=1452%:%
%:%3046=1452%:%
%:%3047=1453%:%
%:%3048=1453%:%
%:%3049=1454%:%
%:%3050=1454%:%
%:%3051=1455%:%
%:%3052=1455%:%
%:%3053=1455%:%
%:%3054=1456%:%
%:%3055=1456%:%
%:%3056=1457%:%
%:%3057=1457%:%
%:%3058=1457%:%
%:%3059=1458%:%
%:%3060=1458%:%
%:%3061=1459%:%
%:%3062=1459%:%
%:%3063=1459%:%
%:%3064=1460%:%
%:%3065=1460%:%
%:%3066=1461%:%
%:%3067=1461%:%
%:%3068=1461%:%
%:%3069=1462%:%
%:%3070=1462%:%
%:%3071=1463%:%
%:%3072=1463%:%
%:%3073=1464%:%
%:%3074=1464%:%
%:%3075=1465%:%
%:%3081=1465%:%
%:%3084=1466%:%
%:%3085=1467%:%
%:%3086=1467%:%
%:%3087=1468%:%
%:%3094=1469%:%
%:%3095=1469%:%
%:%3096=1470%:%
%:%3097=1470%:%
%:%3098=1471%:%
%:%3099=1471%:%
%:%3100=1471%:%
%:%3101=1471%:%
%:%3102=1472%:%
%:%3103=1472%:%
%:%3104=1473%:%
%:%3105=1473%:%
%:%3106=1474%:%
%:%3107=1474%:%
%:%3108=1474%:%
%:%3109=1474%:%
%:%3110=1475%:%
%:%3111=1475%:%
%:%3112=1476%:%
%:%3113=1476%:%
%:%3114=1477%:%
%:%3115=1477%:%
%:%3116=1477%:%
%:%3117=1477%:%
%:%3118=1478%:%
%:%3119=1478%:%
%:%3120=1479%:%
%:%3121=1479%:%
%:%3122=1480%:%
%:%3123=1480%:%
%:%3124=1480%:%
%:%3125=1480%:%
%:%3126=1481%:%
%:%3127=1481%:%
%:%3128=1482%:%
%:%3129=1482%:%
%:%3130=1483%:%
%:%3131=1483%:%
%:%3132=1483%:%
%:%3133=1483%:%
%:%3134=1484%:%
%:%3135=1484%:%
%:%3136=1485%:%
%:%3137=1485%:%
%:%3138=1486%:%
%:%3139=1486%:%
%:%3140=1486%:%
%:%3141=1486%:%
%:%3142=1487%:%
%:%3152=1489%:%
%:%3154=1491%:%
%:%3155=1491%:%
%:%3156=1492%:%
%:%3159=1493%:%
%:%3163=1493%:%
%:%3164=1493%:%
%:%3173=1495%:%
%:%3174=1496%:%
%:%3175=1497%:%
%:%3176=1498%:%
%:%3177=1499%:%
%:%3178=1500%:%
%:%3179=1501%:%
%:%3180=1502%:%
%:%3181=1503%:%
%:%3182=1504%:%
%:%3183=1505%:%
%:%3184=1506%:%
%:%3185=1507%:%
%:%3186=1508%:%
%:%3187=1509%:%
%:%3188=1510%:%
%:%3189=1511%:%
%:%3190=1512%:%
%:%3191=1513%:%
%:%3192=1514%:%
%:%3193=1515%:%
%:%3194=1516%:%
%:%3195=1517%:%
%:%3197=1519%:%
%:%3198=1519%:%
%:%3199=1520%:%
%:%3200=1521%:%
%:%3201=1522%:%
%:%3208=1523%:%
%:%3209=1523%:%
%:%3210=1524%:%
%:%3211=1524%:%
%:%3212=1525%:%
%:%3213=1525%:%
%:%3214=1526%:%
%:%3215=1526%:%
%:%3216=1527%:%
%:%3217=1527%:%
%:%3218=1528%:%
%:%3219=1528%:%
%:%3220=1529%:%
%:%3221=1529%:%
%:%3222=1530%:%
%:%3223=1530%:%
%:%3224=1530%:%
%:%3225=1531%:%
%:%3226=1531%:%
%:%3227=1532%:%
%:%3228=1532%:%
%:%3229=1533%:%
%:%3230=1533%:%
%:%3231=1534%:%
%:%3232=1534%:%
%:%3233=1535%:%
%:%3234=1535%:%
%:%3235=1535%:%
%:%3236=1536%:%
%:%3237=1536%:%
%:%3238=1537%:%
%:%3239=1537%:%
%:%3240=1538%:%
%:%3250=1540%:%
%:%3252=1542%:%
%:%3253=1542%:%
%:%3254=1543%:%
%:%3256=1545%:%
%:%3259=1546%:%
%:%3263=1546%:%
%:%3264=1546%:%
%:%3273=1548%:%
%:%3274=1549%:%
%:%3278=1551%:%
%:%3279=1552%:%
%:%3281=1554%:%
%:%3282=1554%:%
%:%3283=1555%:%
%:%3284=1556%:%
%:%3291=1557%:%
%:%3292=1557%:%
%:%3293=1558%:%
%:%3294=1558%:%
%:%3295=1559%:%
%:%3296=1559%:%
%:%3297=1560%:%
%:%3298=1560%:%
%:%3299=1560%:%
%:%3300=1561%:%
%:%3301=1561%:%
%:%3302=1562%:%
%:%3303=1562%:%
%:%3304=1562%:%
%:%3305=1563%:%
%:%3306=1563%:%
%:%3307=1564%:%
%:%3308=1564%:%
%:%3309=1564%:%
%:%3310=1565%:%
%:%3311=1565%:%
%:%3312=1566%:%
%:%3318=1566%:%
%:%3321=1567%:%
%:%3322=1568%:%
%:%3323=1568%:%
%:%3324=1569%:%
%:%3325=1570%:%
%:%3332=1571%:%
%:%3333=1571%:%
%:%3334=1572%:%
%:%3335=1572%:%
%:%3336=1573%:%
%:%3337=1573%:%
%:%3338=1574%:%
%:%3339=1574%:%
%:%3340=1574%:%
%:%3341=1575%:%
%:%3342=1575%:%
%:%3343=1576%:%
%:%3344=1576%:%
%:%3345=1577%:%
%:%3346=1577%:%
%:%3347=1578%:%
%:%3348=1578%:%
%:%3349=1579%:%
%:%3350=1579%:%
%:%3351=1579%:%
%:%3352=1580%:%
%:%3353=1580%:%
%:%3354=1581%:%
%:%3360=1581%:%
%:%3363=1582%:%
%:%3364=1583%:%
%:%3365=1583%:%
%:%3366=1584%:%
%:%3367=1585%:%
%:%3374=1586%:%
%:%3375=1586%:%
%:%3376=1587%:%
%:%3377=1587%:%
%:%3378=1588%:%
%:%3379=1588%:%
%:%3380=1589%:%
%:%3381=1589%:%
%:%3382=1589%:%
%:%3383=1590%:%
%:%3384=1590%:%
%:%3385=1591%:%
%:%3386=1591%:%
%:%3387=1592%:%
%:%3388=1592%:%
%:%3389=1593%:%
%:%3390=1593%:%
%:%3391=1594%:%
%:%3392=1594%:%
%:%3393=1594%:%
%:%3394=1595%:%
%:%3395=1595%:%
%:%3396=1596%:%
%:%3402=1596%:%
%:%3405=1597%:%
%:%3406=1598%:%
%:%3407=1598%:%
%:%3408=1599%:%
%:%3409=1600%:%
%:%3416=1601%:%
%:%3417=1601%:%
%:%3418=1602%:%
%:%3419=1602%:%
%:%3420=1603%:%
%:%3421=1603%:%
%:%3422=1604%:%
%:%3423=1604%:%
%:%3424=1604%:%
%:%3425=1605%:%
%:%3426=1605%:%
%:%3427=1606%:%
%:%3428=1606%:%
%:%3429=1607%:%
%:%3430=1607%:%
%:%3431=1608%:%
%:%3432=1608%:%
%:%3433=1609%:%
%:%3434=1609%:%
%:%3435=1609%:%
%:%3436=1610%:%
%:%3437=1610%:%
%:%3438=1611%:%
%:%3444=1611%:%
%:%3447=1612%:%
%:%3448=1613%:%
%:%3449=1613%:%
%:%3450=1614%:%
%:%3451=1615%:%
%:%3452=1616%:%
%:%3453=1617%:%
%:%3460=1619%:%
%:%3472=1621%:%
%:%3473=1622%:%
%:%3474=1623%:%
%:%3475=1624%:%
%:%3476=1625%:%
%:%3477=1626%:%
%:%3478=1627%:%
%:%3479=1628%:%
%:%3480=1629%:%
%:%3482=1631%:%
%:%3483=1631%:%
%:%3484=1632%:%
%:%3486=1634%:%
%:%3487=1635%:%
%:%3488=1636%:%
%:%3489=1637%:%
%:%3490=1638%:%
%:%3491=1639%:%
%:%3492=1640%:%
%:%3494=1642%:%
%:%3495=1642%:%
%:%3498=1643%:%
%:%3502=1643%:%
%:%3503=1643%:%
%:%3512=1645%:%
%:%3513=1646%:%
%:%3514=1647%:%
%:%3515=1648%:%
%:%3516=1649%:%
%:%3517=1650%:%
%:%3518=1651%:%
%:%3519=1652%:%
%:%3520=1653%:%
%:%3521=1654%:%
%:%3522=1655%:%
%:%3523=1656%:%
%:%3524=1657%:%
%:%3525=1658%:%
%:%3526=1659%:%
%:%3527=1660%:%
%:%3528=1661%:%
%:%3529=1662%:%
%:%3530=1663%:%
%:%3531=1664%:%
%:%3532=1665%:%
%:%3533=1666%:%
%:%3534=1667%:%
%:%3535=1668%:%
%:%3536=1669%:%
%:%3537=1670%:%
%:%3538=1671%:%
%:%3539=1672%:%
%:%3540=1673%:%
%:%3541=1674%:%
%:%3542=1675%:%
%:%3543=1676%:%
%:%3544=1677%:%
%:%3545=1678%:%
%:%3546=1679%:%
%:%3547=1680%:%
%:%3548=1681%:%
%:%3549=1682%:%
%:%3550=1683%:%
%:%3551=1684%:%
%:%3552=1685%:%
%:%3553=1686%:%
%:%3554=1687%:%
%:%3555=1688%:%
%:%3557=1690%:%
%:%3558=1690%:%
%:%3559=1691%:%
%:%3560=1692%:%
%:%3561=1693%:%
%:%3562=1694%:%
%:%3563=1694%:%
%:%3564=1695%:%
%:%3565=1696%:%
%:%3567=1698%:%
%:%3568=1699%:%
%:%3569=1700%:%
%:%3570=1701%:%
%:%3571=1702%:%
%:%3572=1703%:%
%:%3573=1704%:%
%:%3574=1705%:%
%:%3575=1706%:%
%:%3576=1707%:%
%:%3577=1708%:%
%:%3578=1709%:%
%:%3579=1710%:%
%:%3580=1711%:%
%:%3581=1712%:%
%:%3582=1713%:%
%:%3583=1714%:%
%:%3584=1715%:%
%:%3585=1716%:%
%:%3586=1717%:%
%:%3587=1718%:%
%:%3588=1719%:%
%:%3589=1720%:%
%:%3590=1721%:%
%:%3591=1722%:%
%:%3592=1723%:%
%:%3593=1724%:%
%:%3594=1725%:%
%:%3595=1726%:%
%:%3596=1727%:%
%:%3597=1728%:%
%:%3598=1729%:%
%:%3599=1730%:%
%:%3600=1731%:%
%:%3601=1732%:%
%:%3602=1733%:%
%:%3603=1734%:%
%:%3604=1735%:%
%:%3605=1736%:%
%:%3606=1737%:%
%:%3607=1738%:%
%:%3608=1739%:%
%:%3609=1740%:%
%:%3610=1741%:%
%:%3611=1742%:%
%:%3612=1743%:%
%:%3613=1744%:%
%:%3614=1745%:%
%:%3615=1746%:%
%:%3617=1748%:%
%:%3618=1748%:%
%:%3619=1749%:%
%:%3622=1750%:%
%:%3626=1750%:%
%:%3636=1752%:%
%:%3637=1753%:%
%:%3638=1754%:%
%:%3639=1755%:%
%:%3640=1756%:%
%:%3641=1757%:%
%:%3642=1758%:%
%:%3643=1759%:%
%:%3644=1760%:%
%:%3645=1761%:%
%:%3646=1762%:%
%:%3647=1763%:%
%:%3648=1764%:%
%:%3649=1765%:%
%:%3650=1766%:%
%:%3652=1768%:%
%:%3653=1768%:%
%:%3654=1769%:%
%:%3661=1770%:%
%:%3662=1770%:%
%:%3663=1771%:%
%:%3664=1771%:%
%:%3665=1772%:%
%:%3666=1772%:%
%:%3667=1773%:%
%:%3668=1773%:%
%:%3669=1773%:%
%:%3670=1774%:%
%:%3671=1774%:%
%:%3672=1775%:%
%:%3673=1775%:%
%:%3674=1775%:%
%:%3675=1776%:%
%:%3676=1776%:%
%:%3677=1777%:%
%:%3678=1777%:%
%:%3679=1777%:%
%:%3680=1778%:%
%:%3681=1778%:%
%:%3682=1779%:%
%:%3683=1779%:%
%:%3684=1779%:%
%:%3685=1780%:%
%:%3686=1780%:%
%:%3687=1781%:%
%:%3688=1781%:%
%:%3689=1781%:%
%:%3690=1782%:%
%:%3691=1782%:%
%:%3692=1783%:%
%:%3693=1783%:%
%:%3694=1783%:%
%:%3695=1784%:%
%:%3696=1784%:%
%:%3697=1785%:%
%:%3707=1787%:%
%:%3708=1788%:%
%:%3710=1790%:%
%:%3711=1790%:%
%:%3714=1791%:%
%:%3718=1791%:%
%:%3719=1791%:%
%:%3720=1792%:%
%:%3729=1794%:%
%:%3730=1795%:%
%:%3731=1796%:%
%:%3732=1797%:%
%:%3733=1798%:%
%:%3734=1799%:%
%:%3735=1800%:%
%:%3736=1801%:%
%:%3737=1802%:%
%:%3738=1803%:%
%:%3739=1804%:%
%:%3740=1805%:%
%:%3741=1806%:%
%:%3742=1807%:%
%:%3743=1808%:%
%:%3744=1809%:%
%:%3745=1810%:%
%:%3746=1811%:%
%:%3747=1812%:%
%:%3748=1813%:%
%:%3749=1814%:%
%:%3750=1815%:%
%:%3751=1816%:%
%:%3752=1817%:%
%:%3754=1819%:%
%:%3755=1819%:%
%:%3756=1820%:%
%:%3757=1821%:%
%:%3764=1822%:%
%:%3765=1822%:%
%:%3766=1823%:%
%:%3767=1823%:%
%:%3768=1824%:%
%:%3769=1824%:%
%:%3770=1825%:%
%:%3771=1825%:%
%:%3772=1825%:%
%:%3773=1826%:%
%:%3774=1826%:%
%:%3775=1827%:%
%:%3776=1827%:%
%:%3777=1827%:%
%:%3778=1828%:%
%:%3779=1828%:%
%:%3780=1829%:%
%:%3781=1829%:%
%:%3782=1829%:%
%:%3783=1830%:%
%:%3784=1830%:%
%:%3785=1831%:%
%:%3786=1831%:%
%:%3787=1831%:%
%:%3788=1832%:%
%:%3789=1832%:%
%:%3790=1833%:%
%:%3791=1833%:%
%:%3792=1833%:%
%:%3793=1834%:%
%:%3794=1834%:%
%:%3795=1835%:%
%:%3805=1837%:%
%:%3807=1839%:%
%:%3808=1839%:%
%:%3815=1840%:%
%:%3816=1840%:%
%:%3817=1841%:%
%:%3818=1841%:%
%:%3819=1842%:%
%:%3820=1842%:%
%:%3821=1842%:%
%:%3822=1842%:%
%:%3823=1843%:%
%:%3824=1843%:%
%:%3825=1844%:%
%:%3826=1844%:%
%:%3827=1845%:%
%:%3828=1845%:%
%:%3829=1846%:%
%:%3830=1846%:%
%:%3831=1846%:%
%:%3832=1847%:%
%:%3833=1847%:%
%:%3834=1848%:%
%:%3844=1850%:%
%:%3846=1852%:%
%:%3847=1852%:%
%:%3848=1853%:%
%:%3851=1854%:%
%:%3855=1854%:%
%:%3856=1854%:%

%
\begin{isabellebody}%
\setisabellecontext{Semantica}%
%
\isadelimtheory
%
\endisadelimtheory
%
\isatagtheory
%
\endisatagtheory
{\isafoldtheory}%
%
\isadelimtheory
%
\endisadelimtheory
%
\isadelimdocument
%
\endisadelimdocument
%
\isatagdocument
%
\isamarkupsection{Semántica%
}
\isamarkuptrue%
%
\endisatagdocument
{\isafolddocument}%
%
\isadelimdocument
%
\endisadelimdocument
%
\begin{isamarkuptext}%
En esta sección presentaremos la semántica de las fórmulas
  proposicionales y su formalización en Isabelle/HOL. 

  \begin{definicion}
  Una interpretación es una aplicación del conjunto de variables
  proposicionales en el conjunto \isa{{\isasymBB}} de los booleanos.
  \end{definicion}

  De este modo, las interpretaciones asignan valores de verdad a las 
  variables proposicionales.

  En Isabelle, se formaliza como sigue.%
\end{isamarkuptext}\isamarkuptrue%
\isacommand{type{\isacharunderscore}synonym}\isamarkupfalse%
\ {\isacharprime}a\ valuation\ {\isacharequal}\ {\isachardoublequoteopen}{\isacharprime}a\ {\isasymRightarrow}\ bool{\isachardoublequoteclose}%
\begin{isamarkuptext}%
Como podemos observar, \isa{{\isacharprime}a\ valuation} representa
  una función entre elementos de tipo \isa{{\isacharprime}a} cualquiera que conforman los
  átomos de una fórmula proposicional a los que les asigna un booleano. 
  Se define mediante el tipo \isa{type{\isacharunderscore}synonym}, pues consiste en renombrar 
  una construcción ya existente en Isabelle.

  Dada una interpretación, vamos a definir el valor de verdad de una 
  fórmula proposicional en dicha interpretación.

  \begin{definicion}
  Para cada interpretación \isa{{\isasymA}} existe una única aplicación \isa{{\isasymI}\isactrlsub {\isasymA}} desde 
  el conjunto de fórmulas al conjunto \isa{{\isasymBB}} de los booleanos definida 
  recursivamente sobre la estructura de las fórmulas como sigue:\\
  Sea \isa{F} una fórmula cualquiera,
    \begin{itemize}
      \item Si \isa{F} es una fórmula atómica de la forma \isa{p}, entonces \\ 
        \isa{{\isasymI}\isactrlsub {\isasymA}{\isacharparenleft}F{\isacharparenright}\ {\isacharequal}\ {\isasymA}{\isacharparenleft}p{\isacharparenright}}.
      \item Si \isa{F} es la fórmula \isa{{\isasymbottom}}, entonces \isa{{\isasymI}\isactrlsub {\isasymA}{\isacharparenleft}F{\isacharparenright}\ {\isacharequal}\ False}.
      \item Si \isa{F} es de la forma \isa{{\isasymnot}\ G} para cierta fórmula \isa{G}, 
        entonces\\ \isa{{\isasymI}\isactrlsub {\isasymA}{\isacharparenleft}F{\isacharparenright}\ {\isacharequal}\ {\isasymnot}\ {\isasymI}\isactrlsub {\isasymA}{\isacharparenleft}G{\isacharparenright}}.
      \item Si \isa{F} es de la forma \isa{G\ {\isasymand}\ H} para ciertas fórmulas \isa{G} y 
        \isa{H}, entonces\\ \isa{{\isasymI}\isactrlsub {\isasymA}{\isacharparenleft}F{\isacharparenright}{\isacharequal}\ {\isasymI}\isactrlsub {\isasymA}{\isacharparenleft}G{\isacharparenright}\ {\isasymand}\ {\isasymI}\isactrlsub {\isasymA}{\isacharparenleft}H{\isacharparenright}}.
      \item Si \isa{F} es de la forma \isa{G\ {\isasymor}\ H} para ciertas fórmulas \isa{G} y 
        \isa{H}, entonces\\ \isa{{\isasymI}\isactrlsub {\isasymA}{\isacharparenleft}F{\isacharparenright}{\isacharequal}\ {\isasymI}\isactrlsub {\isasymA}{\isacharparenleft}G{\isacharparenright}\ {\isasymor}\ {\isasymI}\isactrlsub {\isasymA}{\isacharparenleft}H{\isacharparenright}}.
      \item Si \isa{F} es de la forma \isa{G\ {\isasymlongrightarrow}\ H} para ciertas fórmulas \isa{G} y 
        \isa{H}, entonces\\ \isa{{\isasymI}\isactrlsub {\isasymA}{\isacharparenleft}F{\isacharparenright}{\isacharequal}\ {\isasymI}\isactrlsub {\isasymA}{\isacharparenleft}G{\isacharparenright}\ {\isasymlongrightarrow}\ {\isasymI}\isactrlsub {\isasymA}{\isacharparenleft}H{\isacharparenright}}.
    \end{itemize}
  En estas condiciones se dice que \isa{{\isasymI}\isactrlsub {\isasymA}{\isacharparenleft}F{\isacharparenright}} es el valor de la fórmula 
  \isa{F} en la interpretación \isa{{\isasymA}}.
  \end{definicion}

  En Isabelle, dada una interpretación \isa{{\isasymA}} y una fórmula \isa{F}, vamos a 
  definir \isa{{\isasymI}\isactrlsub {\isasymA}{\isacharparenleft}F{\isacharparenright}} mediante la función \isa{formula{\isacharunderscore}semantics\ {\isasymA}\ F}, 
  notado como \isa{{\isasymA}\ {\isasymTurnstile}\ F}.%
\end{isamarkuptext}\isamarkuptrue%
\isacommand{primrec}\isamarkupfalse%
\ formula{\isacharunderscore}semantics\ {\isacharcolon}{\isacharcolon}\ \isanewline
\ \ {\isachardoublequoteopen}{\isacharprime}a\ valuation\ {\isasymRightarrow}\ {\isacharprime}a\ formula\ {\isasymRightarrow}\ bool{\isachardoublequoteclose}\ {\isacharparenleft}\isakeyword{infix}\ {\isachardoublequoteopen}{\isasymTurnstile}{\isachardoublequoteclose}\ {\isadigit{5}}{\isadigit{1}}{\isacharparenright}\ \isakeyword{where}\isanewline
\ \ {\isachardoublequoteopen}{\isasymA}\ {\isasymTurnstile}\ Atom\ k\ {\isacharequal}\ {\isasymA}\ k{\isachardoublequoteclose}\ \isanewline
{\isacharbar}\ {\isachardoublequoteopen}{\isasymA}\ {\isasymTurnstile}\ {\isasymbottom}\ {\isacharequal}\ False{\isachardoublequoteclose}\ \isanewline
{\isacharbar}\ {\isachardoublequoteopen}{\isasymA}\ {\isasymTurnstile}\ Not\ F\ {\isacharequal}\ {\isacharparenleft}{\isasymnot}\ {\isasymA}\ {\isasymTurnstile}\ F{\isacharparenright}{\isachardoublequoteclose}\ \isanewline
{\isacharbar}\ {\isachardoublequoteopen}{\isasymA}\ {\isasymTurnstile}\ And\ F\ G\ {\isacharequal}\ {\isacharparenleft}{\isasymA}\ {\isasymTurnstile}\ F\ {\isasymand}\ {\isasymA}\ {\isasymTurnstile}\ G{\isacharparenright}{\isachardoublequoteclose}\ \isanewline
{\isacharbar}\ {\isachardoublequoteopen}{\isasymA}\ {\isasymTurnstile}\ Or\ F\ G\ {\isacharequal}\ {\isacharparenleft}{\isasymA}\ {\isasymTurnstile}\ F\ {\isasymor}\ {\isasymA}\ {\isasymTurnstile}\ G{\isacharparenright}{\isachardoublequoteclose}\ \isanewline
{\isacharbar}\ {\isachardoublequoteopen}{\isasymA}\ {\isasymTurnstile}\ Imp\ F\ G\ {\isacharequal}\ {\isacharparenleft}{\isasymA}\ {\isasymTurnstile}\ F\ {\isasymlongrightarrow}\ {\isasymA}\ {\isasymTurnstile}\ G{\isacharparenright}{\isachardoublequoteclose}%
\begin{isamarkuptext}%
Como podemos observar, \isa{formula{\isacharunderscore}semantics} es una función
  primitiva recursiva, como indica el tipo \isa{primrec}, notada con el 
  símbolo infijo \isa{{\isasymTurnstile}}. De este modo, dada una interpretación \isa{{\isasymA}} sobre 
  variables proposicionales de un tipo \isa{{\isacharprime}a} cualquiera y una fórmula,
  se define el valor de la fórmula en la interpretación \isa{{\isasymA}} como se 
  muestra. Veamos algunos ejemplos.%
\end{isamarkuptext}\isamarkuptrue%
\isacommand{notepad}\isamarkupfalse%
\isanewline
\isakeyword{begin}\isanewline
%
\isadelimproof
\ \ %
\endisadelimproof
%
\isatagproof
\isacommand{fix}\isamarkupfalse%
\ {\isasymA}\ {\isacharcolon}{\isacharcolon}\ {\isachardoublequoteopen}nat\ valuation{\isachardoublequoteclose}\isanewline
\isanewline
\isacommand{have}\isamarkupfalse%
\ {\isachardoublequoteopen}{\isacharparenleft}{\isasymA}\ {\isacharparenleft}{\isadigit{1}}\ {\isacharcolon}{\isacharequal}\ True{\isacharcomma}\ {\isadigit{2}}\ {\isacharcolon}{\isacharequal}\ False{\isacharcomma}\ {\isadigit{3}}\ {\isacharcolon}{\isacharequal}\ True{\isacharparenright}\ \isanewline
\ \ \ \ \ \ \ \ \ \ \ \ {\isasymTurnstile}\ {\isacharparenleft}\isactrlbold {\isasymnot}\ {\isacharparenleft}{\isacharparenleft}Atom\ {\isadigit{1}}\ \isactrlbold {\isasymor}\ Atom\ {\isadigit{2}}{\isacharparenright}{\isacharparenright}\ \isactrlbold {\isasymrightarrow}\ Atom\ {\isadigit{3}}{\isacharparenright}{\isacharparenright}\ {\isacharequal}\ True{\isachardoublequoteclose}\ \isanewline
\ \ \isacommand{by}\isamarkupfalse%
\ simp\isanewline
\ \ \isanewline
\ \ \isacommand{have}\isamarkupfalse%
\ {\isachardoublequoteopen}{\isacharparenleft}{\isasymA}\ {\isacharparenleft}{\isadigit{1}}\ {\isacharcolon}{\isacharequal}\ True{\isacharparenright}\ {\isasymTurnstile}\ Atom\ {\isadigit{1}}{\isacharparenright}\ {\isacharequal}\ True{\isachardoublequoteclose}\isanewline
\ \ \ \ \isacommand{by}\isamarkupfalse%
\ simp\isanewline
\isanewline
\ \ \isacommand{have}\isamarkupfalse%
\ {\isachardoublequoteopen}{\isacharparenleft}{\isasymA}\ {\isacharparenleft}{\isadigit{1}}\ {\isacharcolon}{\isacharequal}\ True{\isacharparenright}\ {\isasymTurnstile}\ \isactrlbold {\isasymnot}\ {\isacharparenleft}Atom\ {\isadigit{1}}{\isacharparenright}{\isacharparenright}\ {\isacharequal}\ False{\isachardoublequoteclose}\isanewline
\ \ \ \ \isacommand{by}\isamarkupfalse%
\ simp\isanewline
\isanewline
\ \ \isacommand{have}\isamarkupfalse%
\ {\isachardoublequoteopen}{\isacharparenleft}{\isasymA}\ {\isacharparenleft}{\isadigit{1}}\ {\isacharcolon}{\isacharequal}\ True{\isacharcomma}\ {\isadigit{2}}\ {\isacharcolon}{\isacharequal}\ False{\isacharparenright}\ {\isasymTurnstile}\ \isactrlbold {\isasymnot}\ {\isacharparenleft}Atom\ {\isadigit{1}}{\isacharparenright}\ \isactrlbold {\isasymand}\ {\isacharparenleft}Atom\ {\isadigit{2}}{\isacharparenright}{\isacharparenright}\ {\isacharequal}\ False{\isachardoublequoteclose}\isanewline
\ \ \ \ \isacommand{by}\isamarkupfalse%
\ simp\isanewline
\isanewline
\ \ \isacommand{have}\isamarkupfalse%
\ {\isachardoublequoteopen}{\isacharparenleft}{\isasymA}\ {\isacharparenleft}{\isadigit{1}}\ {\isacharcolon}{\isacharequal}\ True{\isacharcomma}\ {\isadigit{2}}\ {\isacharcolon}{\isacharequal}\ False{\isacharcomma}\ {\isadigit{3}}\ {\isacharcolon}{\isacharequal}\ False{\isacharparenright}\ \isanewline
\ \ \ \ \ \ \ \ \ \ \ \ {\isasymTurnstile}\ {\isacharparenleft}\isactrlbold {\isasymnot}\ {\isacharparenleft}{\isacharparenleft}Atom\ {\isadigit{1}}\ \isactrlbold {\isasymand}\ Atom\ {\isadigit{2}}{\isacharparenright}{\isacharparenright}\ \isactrlbold {\isasymrightarrow}\ Atom\ {\isadigit{3}}{\isacharparenright}{\isacharparenright}\ {\isacharequal}\ False{\isachardoublequoteclose}\ \isanewline
\ \ \ \ \isacommand{by}\isamarkupfalse%
\ simp%
\endisatagproof
{\isafoldproof}%
%
\isadelimproof
\isanewline
%
\endisadelimproof
\isanewline
\isacommand{end}\isamarkupfalse%
%
\begin{isamarkuptext}%
En los ejemplos anteriores se ha usado la notación para
  funciones\\ \isa{f\ {\isacharparenleft}a\ {\isacharcolon}{\isacharequal}\ b{\isacharparenright}}. Dicha notación abreviada se corresponde con 
  la definción de \isa{fun{\isacharunderscore}upd\ f\ a\ b}.

  \begin{itemize}
    \item[] \isa{f{\isacharparenleft}a\ {\isacharcolon}{\isacharequal}\ b{\isacharparenright}\ {\isacharequal}\ {\isacharparenleft}{\isasymlambda}x{\isachardot}\ \textsf{if}\ x\ {\isacharequal}\ a\ \textsf{then}\ b\ \textsf{else}\ f\ x{\isacharparenright}} 
      \hfill (\isa{fun{\isacharunderscore}upd{\isacharunderscore}def})
  \end{itemize}

  Es decir, \isa{f\ {\isacharparenleft}a{\isacharcolon}{\isacharequal}b{\isacharparenright}} es la función que para cualquier valor \isa{x} del 
  dominio, si \isa{x\ {\isacharequal}\ a}, entonces devuelve \isa{b}. En caso contrario, 
  devuelve el valor \isa{f\ x}.

  A continuación veamos una serie de definiciones sobre fórmulas e 
  interpretaciones. En primer lugar, la noción de modelo de una 
  fórmula.

  \begin{definicion}
  Una interpretación es modelo de una fórmula si el valor de la
  fórmula dada dicha interpretación es \isa{Verdadero}. 
  \end{definicion}

  En Isabelle se formaliza de la siguiente manera.%
\end{isamarkuptext}\isamarkuptrue%
\isacommand{definition}\isamarkupfalse%
\ {\isachardoublequoteopen}isModel\ {\isasymA}\ F\ {\isasymequiv}\ {\isasymA}\ {\isasymTurnstile}\ F{\isachardoublequoteclose}%
\begin{isamarkuptext}%
Veamos cuáles de las interpretaciones de los ejemplos anteriores
  son modelos de las fórmulas dadas.%
\end{isamarkuptext}\isamarkuptrue%
\isacommand{notepad}\isamarkupfalse%
\isanewline
\isakeyword{begin}\isanewline
%
\isadelimproof
\ \ %
\endisadelimproof
%
\isatagproof
\isacommand{fix}\isamarkupfalse%
\ {\isasymA}\ {\isacharcolon}{\isacharcolon}\ {\isachardoublequoteopen}nat\ valuation{\isachardoublequoteclose}\isanewline
\isanewline
\ \ \isacommand{have}\isamarkupfalse%
\ {\isachardoublequoteopen}isModel\ {\isacharparenleft}{\isasymA}\ {\isacharparenleft}{\isadigit{1}}\ {\isacharcolon}{\isacharequal}\ True{\isacharparenright}{\isacharparenright}\ {\isacharparenleft}Atom\ {\isadigit{1}}{\isacharparenright}{\isachardoublequoteclose}\isanewline
\ \ \ \ \isacommand{by}\isamarkupfalse%
\ {\isacharparenleft}simp\ add{\isacharcolon}\ isModel{\isacharunderscore}def{\isacharparenright}\isanewline
\isanewline
\ \ \isacommand{have}\isamarkupfalse%
\ {\isachardoublequoteopen}{\isasymnot}\ isModel\ {\isacharparenleft}{\isasymA}\ {\isacharparenleft}{\isadigit{1}}\ {\isacharcolon}{\isacharequal}\ True{\isacharparenright}{\isacharparenright}\ {\isacharparenleft}\isactrlbold {\isasymnot}\ {\isacharparenleft}Atom\ {\isadigit{1}}{\isacharparenright}{\isacharparenright}{\isachardoublequoteclose}\isanewline
\ \ \ \ \isacommand{by}\isamarkupfalse%
\ {\isacharparenleft}simp\ add{\isacharcolon}\ isModel{\isacharunderscore}def{\isacharparenright}\isanewline
\isanewline
\ \ \isacommand{have}\isamarkupfalse%
\ {\isachardoublequoteopen}{\isasymnot}\ isModel\ {\isacharparenleft}{\isasymA}\ {\isacharparenleft}{\isadigit{1}}\ {\isacharcolon}{\isacharequal}\ True{\isacharcomma}\ {\isadigit{2}}\ {\isacharcolon}{\isacharequal}\ False{\isacharparenright}{\isacharparenright}\ {\isacharparenleft}\isactrlbold {\isasymnot}\ {\isacharparenleft}Atom\ {\isadigit{1}}{\isacharparenright}\ \isactrlbold {\isasymand}\ {\isacharparenleft}Atom\ {\isadigit{2}}{\isacharparenright}{\isacharparenright}{\isachardoublequoteclose}\isanewline
\ \ \ \ \isacommand{by}\isamarkupfalse%
\ {\isacharparenleft}simp\ add{\isacharcolon}\ isModel{\isacharunderscore}def{\isacharparenright}\isanewline
\isanewline
\ \ \isacommand{have}\isamarkupfalse%
\ {\isachardoublequoteopen}{\isasymnot}\ isModel\ {\isacharparenleft}{\isasymA}\ {\isacharparenleft}{\isadigit{1}}\ {\isacharcolon}{\isacharequal}\ True{\isacharcomma}\ {\isadigit{2}}\ {\isacharcolon}{\isacharequal}\ False{\isacharcomma}\ {\isadigit{3}}\ {\isacharcolon}{\isacharequal}\ False{\isacharparenright}{\isacharparenright}\ \isanewline
\ \ \ \ \ \ \ \ \ \ {\isacharparenleft}\isactrlbold {\isasymnot}\ {\isacharparenleft}{\isacharparenleft}Atom\ {\isadigit{1}}\ \isactrlbold {\isasymand}\ Atom\ {\isadigit{2}}{\isacharparenright}{\isacharparenright}\ \isactrlbold {\isasymrightarrow}\ Atom\ {\isadigit{3}}{\isacharparenright}{\isachardoublequoteclose}\isanewline
\ \ \ \ \isacommand{by}\isamarkupfalse%
\ {\isacharparenleft}simp\ add{\isacharcolon}\ isModel{\isacharunderscore}def{\isacharparenright}\isanewline
\isanewline
\ \ \isacommand{have}\isamarkupfalse%
\ {\isachardoublequoteopen}isModel\ {\isacharparenleft}{\isasymA}\ {\isacharparenleft}{\isadigit{1}}\ {\isacharcolon}{\isacharequal}\ True{\isacharcomma}\ {\isadigit{2}}\ {\isacharcolon}{\isacharequal}\ False{\isacharcomma}\ {\isadigit{3}}\ {\isacharcolon}{\isacharequal}\ True{\isacharparenright}{\isacharparenright}\ \isanewline
\ \ \ \ \ \ \ \ \ \ {\isacharparenleft}\isactrlbold {\isasymnot}\ {\isacharparenleft}{\isacharparenleft}Atom\ {\isadigit{1}}\ \isactrlbold {\isasymor}\ Atom\ {\isadigit{2}}{\isacharparenright}{\isacharparenright}\ \isactrlbold {\isasymrightarrow}\ Atom\ {\isadigit{3}}{\isacharparenright}{\isachardoublequoteclose}\isanewline
\ \ \ \ \isacommand{by}\isamarkupfalse%
\ {\isacharparenleft}simp\ add{\isacharcolon}\ isModel{\isacharunderscore}def{\isacharparenright}%
\endisatagproof
{\isafoldproof}%
%
\isadelimproof
\isanewline
%
\endisadelimproof
\isanewline
\isacommand{end}\isamarkupfalse%
%
\begin{isamarkuptext}%
Demos ahora la noción de fórmula satisfacible.

  \begin{definicion}
    Una fórmula es satisfacible si tiene algún modelo.
  \end{definicion}

  Se concreta en Isabelle como sigue.%
\end{isamarkuptext}\isamarkuptrue%
\isacommand{definition}\isamarkupfalse%
\ {\isachardoublequoteopen}satF{\isacharparenleft}F{\isacharparenright}\ {\isasymequiv}\ {\isasymexists}{\isasymA}{\isachardot}\ {\isasymA}\ {\isasymTurnstile}\ F{\isachardoublequoteclose}%
\begin{isamarkuptext}%
Mostremos ejemplos de fórmulas satisfacibles y no satisfacibles.
  Estas últimas son también llamadas contradicciones, pues son
  falsas para cualquier interpretación.%
\end{isamarkuptext}\isamarkuptrue%
\isacommand{notepad}\isamarkupfalse%
\isanewline
\isakeyword{begin}\isanewline
%
\isadelimproof
\ \ %
\endisadelimproof
%
\isatagproof
\isacommand{fix}\isamarkupfalse%
\ {\isasymA}\ {\isacharcolon}{\isacharcolon}\ {\isachardoublequoteopen}nat\ valuation{\isachardoublequoteclose}\isanewline
\isanewline
\ \ \isacommand{have}\isamarkupfalse%
\ {\isachardoublequoteopen}satF\ {\isacharparenleft}Atom\ {\isadigit{1}}{\isacharparenright}{\isachardoublequoteclose}\isanewline
\ \ \ \ \isacommand{by}\isamarkupfalse%
\ {\isacharparenleft}meson\ formula{\isacharunderscore}semantics{\isachardot}simps{\isacharparenleft}{\isadigit{1}}{\isacharparenright}\ satF{\isacharunderscore}def{\isacharparenright}\isanewline
\isanewline
\ \ \isacommand{have}\isamarkupfalse%
\ {\isachardoublequoteopen}satF\ {\isacharparenleft}Atom\ {\isadigit{1}}\ \isactrlbold {\isasymand}\ Atom\ {\isadigit{3}}{\isacharparenright}{\isachardoublequoteclose}\ \isanewline
\ \ \ \ \isacommand{using}\isamarkupfalse%
\ satF{\isacharunderscore}def\ \isacommand{by}\isamarkupfalse%
\ force\isanewline
\isanewline
\ \ \isacommand{have}\isamarkupfalse%
\ {\isachardoublequoteopen}{\isasymnot}\ satF\ {\isacharparenleft}Atom\ {\isadigit{2}}\ \isactrlbold {\isasymand}\ {\isacharparenleft}\isactrlbold {\isasymnot}\ {\isacharparenleft}Atom\ {\isadigit{2}}{\isacharparenright}{\isacharparenright}{\isacharparenright}{\isachardoublequoteclose}\isanewline
\ \ \ \ \isacommand{using}\isamarkupfalse%
\ satF{\isacharunderscore}def\ \isacommand{by}\isamarkupfalse%
\ force%
\endisatagproof
{\isafoldproof}%
%
\isadelimproof
\isanewline
%
\endisadelimproof
\isanewline
\isacommand{end}\isamarkupfalse%
%
\begin{isamarkuptext}%
Como podemos observar, \isa{isModel} y \isa{satF} se han 
  formalizado usando el tipo \isa{definition} pues, en ambos casos, hemos
  renombrado una construcción no recursiva ya existente en Isabelle/HOL.

  Continuemos con la noción de fórmula válida o tautología.

  \begin{definicion} 
  \isa{F} es una fórmula válida o tautología (\isa{{\isasymTurnstile}\ F}) si toda interpretación 
  es modelo de \isa{F}. 
  \end{definicion}

  Es decir, una tautología es una fórmula que es verdadera para 
  cualquier interpretación. En otras palabras, toda interpretación es
  modelo de dicha fórmula. En Isabelle se formaliza de la siguiente
  manera.%
\end{isamarkuptext}\isamarkuptrue%
\isacommand{abbreviation}\isamarkupfalse%
\ valid\ {\isacharparenleft}{\isachardoublequoteopen}{\isasymTurnstile}\ {\isacharunderscore}{\isachardoublequoteclose}\ {\isadigit{5}}{\isadigit{1}}{\isacharparenright}\ \isakeyword{where}\isanewline
\ \ {\isachardoublequoteopen}{\isasymTurnstile}\ F\ {\isasymequiv}\ {\isasymforall}{\isasymA}{\isachardot}\ {\isasymA}\ {\isasymTurnstile}\ F{\isachardoublequoteclose}%
\begin{isamarkuptext}%
Por otro lado, podemos observar que se ha definido mediante el 
  tipo \isa{abbreviation}, introduciendo una nueva notación para la 
  construcción formada por un cuantificador universal aplicado 
  al primer argumento de \isa{formula{\isacharunderscore}semantics}.

  Veamos un ejemplo clásico de tautología: el principio del tercio
  excluso.%
\end{isamarkuptext}\isamarkuptrue%
\isacommand{notepad}\isamarkupfalse%
\isanewline
\isakeyword{begin}\isanewline
%
\isadelimproof
\ \ %
\endisadelimproof
%
\isatagproof
\isacommand{fix}\isamarkupfalse%
\ {\isasymA}\ {\isacharcolon}{\isacharcolon}\ {\isachardoublequoteopen}nat\ valuation{\isachardoublequoteclose}\isanewline
\isanewline
\ \ \isacommand{have}\isamarkupfalse%
\ {\isachardoublequoteopen}{\isasymTurnstile}\ {\isacharparenleft}Atom\ {\isadigit{5}}\ \isactrlbold {\isasymor}\ {\isacharparenleft}\isactrlbold {\isasymnot}\ {\isacharparenleft}Atom\ {\isadigit{5}}{\isacharparenright}{\isacharparenright}{\isacharparenright}{\isachardoublequoteclose}\isanewline
\ \ \ \ \isacommand{by}\isamarkupfalse%
\ simp%
\endisatagproof
{\isafoldproof}%
%
\isadelimproof
\isanewline
%
\endisadelimproof
\isanewline
\isacommand{end}\isamarkupfalse%
%
\begin{isamarkuptext}%
Otro ejemplo de tautología se muestra con el siguiente lema.

  \begin{lema}
    La fórmula \isa{{\isasymtop}} es una tautología.
  \end{lema}
 
  Veamos su prueba.

  \begin{demostracion}
    Sea una interpretación cualquiera \isa{{\isasymA}}. Es obvio que, aplicando la\\
    propiedad reflexiva de la implicación, tenemos que \isa{Verdadero} es
    equivalente a suponer que valor de \isa{{\isasymbottom}} en \isa{{\isasymA}} se implica a sí 
    mismo. Por definición, se tiene que la implicación anterior es, a 
    su vez, equivalente al valor de la fórmula \isa{{\isasymbottom}\ {\isasymrightarrow}\ {\isasymbottom}} en la 
    interpretación \isa{{\isasymA}}. Según la definición de \isa{{\isasymtop}}, tenemos que esto es
    igual al valor de la fórmula \isa{{\isasymtop}} en la interpretación \isa{{\isasymA}}.
    Finalmente, mediante esta cadena de equivalencias se observa que
    el valor de \isa{{\isasymtop}} en una interpretación \isa{{\isasymA}} cualquiera es 
    \isa{Verdadero} como queríamos probar.    
  \end{demostracion}

  En Isabelle se enuncia y demuestra de manera detallada como sigue.%
\end{isamarkuptext}\isamarkuptrue%
\isacommand{lemma}\isamarkupfalse%
\ \ {\isachardoublequoteopen}{\isasymA}\ {\isasymTurnstile}\ {\isasymtop}{\isachardoublequoteclose}\ \isanewline
%
\isadelimproof
%
\endisadelimproof
%
\isatagproof
\isacommand{proof}\isamarkupfalse%
\ {\isacharminus}\isanewline
\ \isacommand{have}\isamarkupfalse%
\ {\isachardoublequoteopen}{\isasymA}\ {\isasymTurnstile}\ {\isasymbottom}\ {\isasymlongrightarrow}\ {\isasymA}\ {\isasymTurnstile}\ {\isasymbottom}{\isachardoublequoteclose}\ \isanewline
\ \ \ \isacommand{by}\isamarkupfalse%
\ {\isacharparenleft}rule\ imp{\isacharunderscore}refl{\isacharparenright}\isanewline
\ \isacommand{then}\isamarkupfalse%
\ \isacommand{have}\isamarkupfalse%
\ {\isachardoublequoteopen}{\isasymA}\ {\isasymTurnstile}\ {\isacharparenleft}{\isasymbottom}\ \isactrlbold {\isasymrightarrow}\ {\isasymbottom}{\isacharparenright}{\isachardoublequoteclose}\isanewline
\ \ \ \isacommand{by}\isamarkupfalse%
\ {\isacharparenleft}simp\ only{\isacharcolon}\ formula{\isacharunderscore}semantics{\isachardot}simps{\isacharparenleft}{\isadigit{6}}{\isacharparenright}{\isacharparenright}\isanewline
\ \isacommand{thus}\isamarkupfalse%
\ {\isachardoublequoteopen}{\isasymA}\ {\isasymTurnstile}\ {\isasymtop}{\isachardoublequoteclose}\ \isacommand{unfolding}\isamarkupfalse%
\ Top{\isacharunderscore}def\ \isacommand{by}\isamarkupfalse%
\ this\isanewline
\isacommand{qed}\isamarkupfalse%
%
\endisatagproof
{\isafoldproof}%
%
\isadelimproof
%
\endisadelimproof
%
\begin{isamarkuptext}%
Se demuestra automáticamente como sigue.%
\end{isamarkuptext}\isamarkuptrue%
\isacommand{lemma}\isamarkupfalse%
\ top{\isacharunderscore}semantics{\isacharcolon}\ {\isachardoublequoteopen}{\isasymA}\ {\isasymTurnstile}\ {\isasymtop}{\isachardoublequoteclose}\isanewline
%
\isadelimproof
\ \ %
\endisadelimproof
%
\isatagproof
\isacommand{unfolding}\isamarkupfalse%
\ Top{\isacharunderscore}def\ \isacommand{by}\isamarkupfalse%
\ simp%
\endisatagproof
{\isafoldproof}%
%
\isadelimproof
%
\endisadelimproof
%
\begin{isamarkuptext}%
Una vez presentados los conceptos y ejemplos anteriores, 
  continuemos con el siguiente resultado de la sección.

  \begin{lema}
  Sea una fórmula \isa{F} de modo que \isa{p} es una variable proposicional que 
  no pertenece a su conjunto de átomos. Entonces, el valor de \isa{F} en 
  una interpretación no depende del valor de la variable \isa{p} en dicha 
  interpretación.
  \end{lema}

  En Isabelle se formaliza de la siguiente manera empleando la notación
  de \isa{fun{\isacharunderscore}upd}.%
\end{isamarkuptext}\isamarkuptrue%
\isacommand{lemma}\isamarkupfalse%
\ {\isachardoublequoteopen}p\ {\isasymnotin}\ atoms\ F\ {\isasymLongrightarrow}\ {\isacharparenleft}{\isasymA}{\isacharparenleft}p\ {\isacharcolon}{\isacharequal}\ V{\isacharparenright}{\isacharparenright}\ {\isasymTurnstile}\ F\ {\isasymlongleftrightarrow}\ {\isasymA}\ {\isasymTurnstile}\ F{\isachardoublequoteclose}\isanewline
%
\isadelimproof
\ \ %
\endisadelimproof
%
\isatagproof
\isacommand{oops}\isamarkupfalse%
%
\endisatagproof
{\isafoldproof}%
%
\isadelimproof
%
\endisadelimproof
%
\begin{isamarkuptext}%
Veamos ahora la prueba del lema.

  \begin{demostracion}
  Vamos a probar el resultado por inducción en la estructura recursiva
  de las fórmulas. Para ello, dada una interpretación cualquier \isa{{\isasymA}} y 
  una variable \isa{p} que no pertenece al conjunto de átomos de una 
  fórmula, definimos la interpretación \isa{{\isasymA}{\isacharprime}} como aquella que devuelve
  \isa{{\isasymA}{\isacharparenleft}q{\isacharparenright}} para cualquier variable \isa{q} distinta de \isa{p}, y un valor 
  cualquiera \isa{V} en caso contrario. Para demostrar que el valor de una 
  fórmula en una interpretación no depende del valor de \isa{p} en dicha 
  interpretación, basta probar que el valor de la fórmula en \isa{{\isasymA}} 
  coincide con su valor en \isa{{\isasymA}{\isacharprime}} según la definición dada anteriormente. 
  Demostremos los siguientes casos.

  Sea \isa{q} una fórmula atómica cualquiera tal que \isa{p} no pertenece
  al conjunto de sus átomos \isa{{\isacharbraceleft}q{\isacharbraceright}}. De este modo, se tiene \isa{q\ {\isasymnoteq}\ p}. 
  Por definición, el valor de la fórmula atómica \isa{q} en la 
  interpretación \isa{{\isasymA}{\isacharprime}}, es \isa{{\isasymA}{\isacharprime}{\isacharparenleft}q{\isacharparenright}}. Como hemos visto que \isa{q\ {\isasymnoteq}\ p}, 
  tenemos a su vez\\ \isa{{\isasymA}{\isacharprime}{\isacharparenleft}q{\isacharparenright}\ {\isacharequal}\ {\isasymA}{\isacharparenleft}q{\isacharparenright}} según la definición de \isa{{\isasymA}{\isacharprime}}. A su
  vez, \isa{{\isasymA}{\isacharparenleft}q{\isacharparenright}} es el valor de la fórmula atómica \isa{q} en la 
  interpretación \isa{{\isasymA}}, luego se tiene finalmente que ambos valores
  coinciden. 

  Sea la fórmula \isa{{\isasymbottom}}. Por definición, el valor de dicha fórmula es 
  \isa{Falso} en cualquier interpretación, luego se verifica el
  resultado en particular.

  Sea \isa{F} una fórmula tal que para cualquier variable que no pertenezca
  al conjunto de sus átomos, entonces el valor de \isa{F} en la 
  interpretación \isa{{\isasymA}} coincide con su valor en la interpretación \isa{{\isasymA}{\isacharprime}} 
  construida como se indica en el enunciado. Vamos a demostrar el 
  resultado para la fórmula \isa{{\isasymnot}\ F} considerando una variable \isa{p} 
  cualquiera que no pertenezca al conjunto de átomos de \isa{{\isasymnot}\ F}. Como 
  los conjuntos de átomos de \isa{F} y \isa{{\isasymnot}\ F} son el mismo, entonces \isa{p} 
  tampoco pertenece al conjunto de átomos de \isa{F}. De este modo, por 
  hipótesis de inducción, el valor de la fórmula \isa{F} en la 
  interpretación \isa{{\isasymA}} coincide con su valor en la interpretación 
  \isa{{\isasymA}{\isacharprime}}. Por otro lado, por definición tenemos que el valor de la 
  fórmula \isa{{\isasymnot}\ F} en \isa{{\isasymA}} es la negación del valor de \isa{F} en \isa{{\isasymA}}. 
  Por lo visto anteriormente según la hipótesis de inducción, esto es 
  igual a la negación del valor de \isa{F} en \isa{{\isasymA}{\isacharprime}}. Por último, 
  por definición, esto es igual al valor de \isa{{\isasymnot}\ F} en \isa{{\isasymA}{\isacharprime}}, como 
  quería demostrar.

  Sean ahora las fórmulas \isa{G} y \isa{H} tales que, para cada una, su valor
  en la interpretación \isa{{\isasymA}} coincide con su valor en la
  interpretación \isa{{\isasymA}{\isacharprime}} construida como se indica en el enunciado a 
  partir de una variable que no pertenezca al conjunto de sus átomos. 
  Veamos que se verifica el resultado para la fórmula \isa{G\ {\isasymand}\ H}.
  Sea \isa{p} una variable proposicional que no pertenece al conjunto de
  átomos de \isa{G\ {\isasymand}\ H}. Por definición, dicho conjunto es igual a la unión
  del conjunto de átomos de \isa{G} y el conjunto de átomos de \isa{H}.
  Por tanto, en particular \isa{p} no pertenece al conjunto de átomos de
  \isa{G} y, por hipótesis de inducción, el valor de \isa{G} en la
  interpretación \isa{{\isasymA}} coincide con su valor en la
  interpretación \isa{{\isasymA}{\isacharprime}}. Por el mismo motivo, \isa{p} no pertenece al
  conjunto de átomos de \isa{H} y, por hipótesis de inducción,
  el valor de \isa{H} es el mismo en las interpretaciones \isa{{\isasymA}} y \isa{{\isasymA}{\isacharprime}}. 
  Aclaradas estas observaciones, se tiene por definición que el valor 
  de la fórmula \isa{G\ {\isasymand}\ H} en la interpretación \isa{{\isasymA}{\isacharprime}} es 
  la conjunción del valor de \isa{G} en \isa{{\isasymA}{\isacharprime}} y el valor de \isa{H} en \isa{{\isasymA}{\isacharprime}}. 
  Por lo demostrado anteriormente según las hipótesis de inducción, 
  esto es igual a la conjunción del valor de \isa{G} en \isa{{\isasymA}} y el valor de 
  \isa{H} en \isa{{\isasymA}}. Aplicando la definición, esto es el valor de \isa{G\ {\isasymand}\ H} 
  en la interpretación \isa{{\isasymA}}. Por tanto, queda probada la equivalencia 
  en este caso.

  Sean las fórmulas \isa{G} y \isa{H} cumpliendo las hipótesis supuestas
  para el caso anterior. Veamos que el resultado se verifica para la
  fórmula \isa{G\ {\isasymor}\ H}. Sea \isa{p} una variable proposicional que no pertenece
  al conjunto de átomos de \isa{G\ {\isasymor}\ H}. Por definición, dicho conjunto es
  la unión de los conjuntos de átomos de \isa{G} y \isa{H}. Por tanto, como se
  ha probado en el caso anterior, tenemos que \isa{p} no pertenece al
  conjunto de átomos de \isa{G}. Por tanto, aplicando la hipótesis de 
  inducción se tiene que el valor de \isa{G} en la interpretación \isa{{\isasymA}} 
  coincide con su valor en la interpretación \isa{{\isasymA}{\isacharprime}}. Análogamente 
  ocurre para la fórmula \isa{H} como vimos en el caso anterior, de modo
  que \isa{p} no pertenece al conjunto de átomos de \isa{H}. Por tanto, por 
  hipótesis de inducción, el valor de \isa{H} es el mismo en \isa{{\isasymA}} y \isa{{\isasymA}{\isacharprime}}.
  Veamos la equivalencia. 
  Por definición tenemos que el valor de la fórmula \isa{G\ {\isasymor}\ H} en la 
  interpretación \isa{{\isasymA}{\isacharprime}} es la disyunción entre el valor de \isa{G} en \isa{{\isasymA}{\isacharprime}}
  y el valor de \isa{H} en \isa{{\isasymA}{\isacharprime}}. Por lo probado anteriormente según las 
  hipótesis de inducicón, esto es igual a la disyunción entre el valor 
  de \isa{G} en \isa{{\isasymA}} y el valor de \isa{H} en \isa{{\isasymA}}. Por definición, se 
  verifica que es igual al valor de \isa{G\ {\isasymor}\ H} en la interpretación \isa{{\isasymA}}, 
  como queríamos demostrar.

  Probemos finalmente el último caso considerando las fórmulas \isa{G} y
  \isa{H} bajo las condiciones de los dos casos anteriores. Sea \isa{p} una 
  variable proposicional que no pertenece al conjunto de átomos de 
  \isa{G\ {\isasymrightarrow}\ H}. Como dicho conjunto es la unión del conjunto de átomos de
  \isa{G} y el de \isa{H}, \isa{p} no pertenece ni al conjunto de átomos de \isa{G} ni
  al de \isa{H}. Por lo tanto, por hipótesis de inducción, el valor de \isa{G} 
  es el mismo en las interpretaciones \isa{{\isasymA}} y \isa{{\isasymA}{\isacharprime}}, y lo mismo ocurre 
  para \isa{H}. Veamos ahora la cadena de equivalencias. Por definición 
  tenemos que el valor de la fórmula \isa{G\ {\isasymrightarrow}\ H} en la interpretación 
  \isa{{\isasymA}{\isacharprime}} es la implicación del valor de \isa{G} en \isa{{\isasymA}{\isacharprime}} y el valor de \isa{H}
  en \isa{{\isasymA}{\isacharprime}}. Análogamente a los casos anteriores por las hipótesis de 
  inducción, esto es igual a la implicación del valor de \isa{G} en \isa{{\isasymA}} y 
  el valor de \isa{H} en \isa{{\isasymA}}. Por definición, esto es igual al valor de 
  \isa{G\ {\isasymrightarrow}\ H} en la interpretación \isa{{\isasymA}}, probando así el resultado.
  \end{demostracion}

  Veamos a continuación la demostración detallada del lema en
  Isabelle/HOL. Para facilitar la lectura, inicialmente se ha
  probado el resultado para cada caso de la estructura de las fórmulas
  como es habitual. Además, se han empleado los lemas auxiliares 
  \isa{irrelevant{\isacharunderscore}atom{\isacharunderscore}atomic{\isacharunderscore}l{\isadigit{1}}}, \isa{irrelevant{\isacharunderscore}atom{\isacharunderscore}not{\isacharunderscore}l{\isadigit{1}}},
  \isa{irrelevant{\isacharunderscore}atom{\isacharunderscore}and{\isacharunderscore}l{\isadigit{1}}},\\ \isa{irrelevant{\isacharunderscore}atom{\isacharunderscore}or{\isacharunderscore}l{\isadigit{1}}} e 
  \isa{irrelevant{\isacharunderscore}atom{\isacharunderscore}imp{\isacharunderscore}l{\isadigit{1}}} para mostrar resultados sobre la no
  pertenencia a los conjuntos de átomos en cada caso. Es fácil observar 
  que no ha sido necesario el uso de lemas auxiliares en el caso de la 
  fórmula \isa{{\isasymbottom}}, pues su conjunto de átomos es el vacío.%
\end{isamarkuptext}\isamarkuptrue%
\isacommand{lemma}\isamarkupfalse%
\ irrelevant{\isacharunderscore}atom{\isacharunderscore}atomic{\isacharunderscore}l{\isadigit{1}}{\isacharcolon}\isanewline
\ \ \isakeyword{assumes}\ {\isachardoublequoteopen}p\ {\isasymnotin}\ atoms\ {\isacharparenleft}Atom\ x{\isacharparenright}{\isachardoublequoteclose}\ \isanewline
\ \ \isakeyword{shows}\ {\isachardoublequoteopen}x\ {\isasymnoteq}\ p{\isachardoublequoteclose}\isanewline
%
\isadelimproof
%
\endisadelimproof
%
\isatagproof
\isacommand{proof}\isamarkupfalse%
\ {\isacharparenleft}rule\ notI{\isacharparenright}\isanewline
\ \ \isacommand{assume}\isamarkupfalse%
\ {\isachardoublequoteopen}x\ {\isacharequal}\ p{\isachardoublequoteclose}\isanewline
\ \ \isacommand{then}\isamarkupfalse%
\ \isacommand{have}\isamarkupfalse%
\ {\isachardoublequoteopen}p\ {\isasymin}\ {\isacharbraceleft}x{\isacharbraceright}{\isachardoublequoteclose}\ \isanewline
\ \ \ \ \isacommand{by}\isamarkupfalse%
\ {\isacharparenleft}simp\ only{\isacharcolon}\ singleton{\isacharunderscore}iff{\isacharparenright}\isanewline
\ \ \isacommand{also}\isamarkupfalse%
\ \isacommand{have}\isamarkupfalse%
\ {\isachardoublequoteopen}{\isasymdots}\ {\isacharequal}\ atoms\ {\isacharparenleft}Atom\ x{\isacharparenright}{\isachardoublequoteclose}\isanewline
\ \ \ \ \isacommand{by}\isamarkupfalse%
\ {\isacharparenleft}simp\ only{\isacharcolon}\ formula{\isachardot}set{\isacharparenleft}{\isadigit{1}}{\isacharparenright}{\isacharparenright}\isanewline
\ \ \isacommand{finally}\isamarkupfalse%
\ \isacommand{have}\isamarkupfalse%
\ {\isachardoublequoteopen}p\ {\isasymin}\ atoms\ {\isacharparenleft}Atom\ x{\isacharparenright}{\isachardoublequoteclose}\isanewline
\ \ \ \ \isacommand{by}\isamarkupfalse%
\ this\ \isanewline
\ \ \isacommand{with}\isamarkupfalse%
\ assms\ \isacommand{show}\isamarkupfalse%
\ {\isachardoublequoteopen}False{\isachardoublequoteclose}\ \ \isanewline
\ \ \ \ \isacommand{by}\isamarkupfalse%
\ {\isacharparenleft}rule\ notE{\isacharparenright}\isanewline
\isacommand{qed}\isamarkupfalse%
%
\endisatagproof
{\isafoldproof}%
%
\isadelimproof
\isanewline
%
\endisadelimproof
\isanewline
\isacommand{lemma}\isamarkupfalse%
\ irrelevant{\isacharunderscore}atom{\isacharunderscore}atomic{\isacharcolon}\isanewline
\ \ \isakeyword{assumes}\ {\isachardoublequoteopen}p\ {\isasymnotin}\ atoms\ {\isacharparenleft}Atom\ x{\isacharparenright}{\isachardoublequoteclose}\ \isanewline
\ \ \isakeyword{shows}\ {\isachardoublequoteopen}{\isacharparenleft}{\isasymA}{\isacharparenleft}p\ {\isacharcolon}{\isacharequal}\ V{\isacharparenright}{\isacharparenright}\ {\isasymTurnstile}\ {\isacharparenleft}Atom\ x{\isacharparenright}\ {\isasymlongleftrightarrow}\ {\isasymA}\ {\isasymTurnstile}\ {\isacharparenleft}Atom\ x{\isacharparenright}{\isachardoublequoteclose}\isanewline
%
\isadelimproof
%
\endisadelimproof
%
\isatagproof
\isacommand{proof}\isamarkupfalse%
\ {\isacharminus}\isanewline
\ \ \isacommand{have}\isamarkupfalse%
\ {\isachardoublequoteopen}x\ {\isasymnoteq}\ p{\isachardoublequoteclose}\isanewline
\ \ \ \ \isacommand{using}\isamarkupfalse%
\ assms\isanewline
\ \ \ \ \isacommand{by}\isamarkupfalse%
\ {\isacharparenleft}rule\ irrelevant{\isacharunderscore}atom{\isacharunderscore}atomic{\isacharunderscore}l{\isadigit{1}}{\isacharparenright}\isanewline
\ \ \isacommand{have}\isamarkupfalse%
\ {\isachardoublequoteopen}{\isacharparenleft}{\isasymA}{\isacharparenleft}p\ {\isacharcolon}{\isacharequal}\ V{\isacharparenright}{\isacharparenright}\ {\isasymTurnstile}\ {\isacharparenleft}Atom\ x{\isacharparenright}\ {\isacharequal}\ {\isacharparenleft}{\isasymA}{\isacharparenleft}p\ {\isacharcolon}{\isacharequal}\ V{\isacharparenright}{\isacharparenright}\ x{\isachardoublequoteclose}\isanewline
\ \ \ \ \isacommand{by}\isamarkupfalse%
\ {\isacharparenleft}simp\ only{\isacharcolon}\ formula{\isacharunderscore}semantics{\isachardot}simps{\isacharparenleft}{\isadigit{1}}{\isacharparenright}{\isacharparenright}\isanewline
\ \ \isacommand{also}\isamarkupfalse%
\ \isacommand{have}\isamarkupfalse%
\ {\isachardoublequoteopen}{\isasymdots}\ {\isacharequal}\ {\isasymA}\ x{\isachardoublequoteclose}\isanewline
\ \ \ \ \isacommand{using}\isamarkupfalse%
\ {\isacartoucheopen}x\ {\isasymnoteq}\ p{\isacartoucheclose}\isanewline
\ \ \ \ \isacommand{by}\isamarkupfalse%
\ {\isacharparenleft}rule\ fun{\isacharunderscore}upd{\isacharunderscore}other{\isacharparenright}\isanewline
\ \ \isacommand{also}\isamarkupfalse%
\ \isacommand{have}\isamarkupfalse%
\ {\isachardoublequoteopen}{\isasymdots}\ {\isacharequal}\ {\isasymA}\ {\isasymTurnstile}\ {\isacharparenleft}Atom\ x{\isacharparenright}{\isachardoublequoteclose}\isanewline
\ \ \ \ \isacommand{by}\isamarkupfalse%
\ {\isacharparenleft}simp\ only{\isacharcolon}\ formula{\isacharunderscore}semantics{\isachardot}simps{\isacharparenleft}{\isadigit{1}}{\isacharparenright}{\isacharparenright}\isanewline
\ \ \isacommand{finally}\isamarkupfalse%
\ \isacommand{show}\isamarkupfalse%
\ {\isacharquery}thesis\isanewline
\ \ \ \ \isacommand{by}\isamarkupfalse%
\ this\isanewline
\isacommand{qed}\isamarkupfalse%
%
\endisatagproof
{\isafoldproof}%
%
\isadelimproof
\isanewline
%
\endisadelimproof
\isanewline
\isacommand{lemma}\isamarkupfalse%
\ irrelevant{\isacharunderscore}atom{\isacharunderscore}bot{\isacharcolon}\isanewline
\ \ \isakeyword{assumes}\ {\isachardoublequoteopen}p\ {\isasymnotin}\ atoms\ {\isasymbottom}{\isachardoublequoteclose}\ \isanewline
\ \ \isakeyword{shows}\ {\isachardoublequoteopen}{\isacharparenleft}{\isasymA}{\isacharparenleft}p\ {\isacharcolon}{\isacharequal}\ V{\isacharparenright}{\isacharparenright}\ {\isasymTurnstile}\ {\isasymbottom}\ {\isasymlongleftrightarrow}\ {\isasymA}\ {\isasymTurnstile}\ {\isasymbottom}{\isachardoublequoteclose}\isanewline
%
\isadelimproof
\ \ %
\endisadelimproof
%
\isatagproof
\isacommand{by}\isamarkupfalse%
\ {\isacharparenleft}simp\ only{\isacharcolon}\ formula{\isacharunderscore}semantics{\isachardot}simps{\isacharparenleft}{\isadigit{2}}{\isacharparenright}{\isacharparenright}%
\endisatagproof
{\isafoldproof}%
%
\isadelimproof
\isanewline
%
\endisadelimproof
\isanewline
\isacommand{lemma}\isamarkupfalse%
\ irrelevant{\isacharunderscore}atom{\isacharunderscore}not{\isacharunderscore}l{\isadigit{1}}{\isacharcolon}\isanewline
\ \ \isakeyword{assumes}\ {\isachardoublequoteopen}p\ {\isasymnotin}\ atoms\ {\isacharparenleft}\isactrlbold {\isasymnot}\ F{\isacharparenright}{\isachardoublequoteclose}\isanewline
\ \ \isakeyword{shows}\ \ \ {\isachardoublequoteopen}p\ {\isasymnotin}\ atoms\ F{\isachardoublequoteclose}\isanewline
%
\isadelimproof
%
\endisadelimproof
%
\isatagproof
\isacommand{proof}\isamarkupfalse%
\isanewline
\ \ \isacommand{assume}\isamarkupfalse%
\ {\isachardoublequoteopen}p\ {\isasymin}\ atoms\ F{\isachardoublequoteclose}\isanewline
\ \ \isacommand{then}\isamarkupfalse%
\ \isacommand{have}\isamarkupfalse%
\ {\isachardoublequoteopen}p\ {\isasymin}\ atoms\ {\isacharparenleft}\isactrlbold {\isasymnot}\ F{\isacharparenright}{\isachardoublequoteclose}\isanewline
\ \ \ \ \isacommand{by}\isamarkupfalse%
\ {\isacharparenleft}simp\ only{\isacharcolon}\ formula{\isachardot}set{\isacharparenleft}{\isadigit{3}}{\isacharparenright}{\isacharparenright}\ \isanewline
\ \ \isacommand{with}\isamarkupfalse%
\ assms\ \isacommand{show}\isamarkupfalse%
\ False\isanewline
\ \ \ \ \isacommand{by}\isamarkupfalse%
\ {\isacharparenleft}rule\ notE{\isacharparenright}\isanewline
\isacommand{qed}\isamarkupfalse%
%
\endisatagproof
{\isafoldproof}%
%
\isadelimproof
\isanewline
%
\endisadelimproof
\isanewline
\isacommand{lemma}\isamarkupfalse%
\ irrelevant{\isacharunderscore}atom{\isacharunderscore}not{\isacharcolon}\isanewline
\ \ \isakeyword{assumes}\ {\isachardoublequoteopen}p\ {\isasymnotin}\ atoms\ F\ {\isasymLongrightarrow}\ {\isasymA}{\isacharparenleft}p\ {\isacharcolon}{\isacharequal}\ V{\isacharparenright}\ {\isasymTurnstile}\ F\ {\isasymlongleftrightarrow}\ {\isasymA}\ {\isasymTurnstile}\ F{\isachardoublequoteclose}\isanewline
\ \ \ \ \ \ \ \ \ \ {\isachardoublequoteopen}p\ {\isasymnotin}\ atoms\ {\isacharparenleft}\isactrlbold {\isasymnot}\ F{\isacharparenright}{\isachardoublequoteclose}\isanewline
\ \isakeyword{shows}\ {\isachardoublequoteopen}{\isasymA}{\isacharparenleft}p\ {\isacharcolon}{\isacharequal}\ V{\isacharparenright}\ {\isasymTurnstile}\ \isactrlbold {\isasymnot}\ F\ {\isasymlongleftrightarrow}\ {\isasymA}\ {\isasymTurnstile}\ \isactrlbold {\isasymnot}\ F{\isachardoublequoteclose}\isanewline
%
\isadelimproof
%
\endisadelimproof
%
\isatagproof
\isacommand{proof}\isamarkupfalse%
\ {\isacharminus}\isanewline
\ \ \isacommand{have}\isamarkupfalse%
\ {\isachardoublequoteopen}p\ {\isasymnotin}\ atoms\ F{\isachardoublequoteclose}\isanewline
\ \ \ \ \isacommand{using}\isamarkupfalse%
\ assms{\isacharparenleft}{\isadigit{2}}{\isacharparenright}\ \isacommand{by}\isamarkupfalse%
\ {\isacharparenleft}rule\ irrelevant{\isacharunderscore}atom{\isacharunderscore}not{\isacharunderscore}l{\isadigit{1}}{\isacharparenright}\isanewline
\ \ \isacommand{then}\isamarkupfalse%
\ \isacommand{have}\isamarkupfalse%
\ {\isachardoublequoteopen}{\isasymA}{\isacharparenleft}p\ {\isacharcolon}{\isacharequal}\ V{\isacharparenright}\ {\isasymTurnstile}\ F\ {\isasymlongleftrightarrow}\ {\isasymA}\ {\isasymTurnstile}\ F{\isachardoublequoteclose}\isanewline
\ \ \ \ \isacommand{by}\isamarkupfalse%
\ {\isacharparenleft}rule\ assms{\isacharparenleft}{\isadigit{1}}{\isacharparenright}{\isacharparenright}\isanewline
\ \ \isacommand{have}\isamarkupfalse%
\ {\isachardoublequoteopen}{\isasymA}{\isacharparenleft}p\ {\isacharcolon}{\isacharequal}\ V{\isacharparenright}\ {\isasymTurnstile}\ \isactrlbold {\isasymnot}\ F\ {\isacharequal}\ {\isacharparenleft}{\isasymnot}\ {\isasymA}{\isacharparenleft}p\ {\isacharcolon}{\isacharequal}\ V{\isacharparenright}\ {\isasymTurnstile}\ F{\isacharparenright}{\isachardoublequoteclose}\isanewline
\ \ \ \ \isacommand{by}\isamarkupfalse%
\ {\isacharparenleft}simp\ only{\isacharcolon}\ formula{\isacharunderscore}semantics{\isachardot}simps{\isacharparenleft}{\isadigit{3}}{\isacharparenright}{\isacharparenright}\isanewline
\ \ \isacommand{also}\isamarkupfalse%
\ \isacommand{have}\isamarkupfalse%
\ {\isachardoublequoteopen}{\isasymdots}\ {\isacharequal}\ {\isacharparenleft}{\isasymnot}\ {\isasymA}\ {\isasymTurnstile}\ F{\isacharparenright}{\isachardoublequoteclose}\isanewline
\ \ \ \ \isacommand{by}\isamarkupfalse%
\ {\isacharparenleft}simp\ only{\isacharcolon}\ {\isacartoucheopen}{\isasymA}{\isacharparenleft}p\ {\isacharcolon}{\isacharequal}\ V{\isacharparenright}\ {\isasymTurnstile}\ F\ {\isasymlongleftrightarrow}\ {\isasymA}\ {\isasymTurnstile}\ F{\isacartoucheclose}{\isacharparenright}\isanewline
\ \ \isacommand{also}\isamarkupfalse%
\ \isacommand{have}\isamarkupfalse%
\ {\isachardoublequoteopen}{\isasymdots}\ {\isacharequal}\ {\isasymA}\ {\isasymTurnstile}\ \isactrlbold {\isasymnot}\ F{\isachardoublequoteclose}\isanewline
\ \ \ \ \isacommand{by}\isamarkupfalse%
\ {\isacharparenleft}simp\ only{\isacharcolon}\ formula{\isacharunderscore}semantics{\isachardot}simps{\isacharparenleft}{\isadigit{3}}{\isacharparenright}{\isacharparenright}\isanewline
\ \ \isacommand{finally}\isamarkupfalse%
\ \isacommand{show}\isamarkupfalse%
\ {\isachardoublequoteopen}{\isasymA}{\isacharparenleft}p\ {\isacharcolon}{\isacharequal}\ V{\isacharparenright}\ {\isasymTurnstile}\ \isactrlbold {\isasymnot}\ F\ {\isasymlongleftrightarrow}\ {\isasymA}\ {\isasymTurnstile}\ \isactrlbold {\isasymnot}\ F{\isachardoublequoteclose}\isanewline
\ \ \ \ \isacommand{by}\isamarkupfalse%
\ this\isanewline
\isacommand{qed}\isamarkupfalse%
%
\endisatagproof
{\isafoldproof}%
%
\isadelimproof
\isanewline
%
\endisadelimproof
\isanewline
\isacommand{lemma}\isamarkupfalse%
\ irrelevant{\isacharunderscore}atom{\isacharunderscore}and{\isacharunderscore}l{\isadigit{1}}{\isacharcolon}\isanewline
\ \ \isakeyword{assumes}\ {\isachardoublequoteopen}p\ {\isasymnotin}\ atoms\ {\isacharparenleft}F\ \isactrlbold {\isasymand}\ G{\isacharparenright}{\isachardoublequoteclose}\isanewline
\ \ \isakeyword{shows}\ \ \ {\isachardoublequoteopen}p\ {\isasymnotin}\ atoms\ F{\isachardoublequoteclose}\isanewline
%
\isadelimproof
%
\endisadelimproof
%
\isatagproof
\isacommand{proof}\isamarkupfalse%
\ \isanewline
\ \ \isacommand{assume}\isamarkupfalse%
\ {\isachardoublequoteopen}p\ {\isasymin}\ atoms\ F{\isachardoublequoteclose}\isanewline
\ \ \isacommand{then}\isamarkupfalse%
\ \isacommand{have}\isamarkupfalse%
\ {\isachardoublequoteopen}p\ {\isasymin}\ atoms\ F\ {\isasymunion}\ atoms\ G{\isachardoublequoteclose}\isanewline
\ \ \ \ \isacommand{by}\isamarkupfalse%
\ {\isacharparenleft}rule\ UnI{\isadigit{1}}{\isacharparenright}\isanewline
\ \ \isacommand{then}\isamarkupfalse%
\ \isacommand{have}\isamarkupfalse%
\ {\isachardoublequoteopen}p\ {\isasymin}\ atoms\ {\isacharparenleft}F\ \isactrlbold {\isasymand}\ G{\isacharparenright}{\isachardoublequoteclose}\isanewline
\ \ \ \ \isacommand{by}\isamarkupfalse%
\ {\isacharparenleft}simp\ only{\isacharcolon}\ formula{\isachardot}set{\isacharparenleft}{\isadigit{4}}{\isacharparenright}{\isacharparenright}\isanewline
\ \ \isacommand{with}\isamarkupfalse%
\ assms\ \isacommand{show}\isamarkupfalse%
\ False\ \isanewline
\ \ \ \ \isacommand{by}\isamarkupfalse%
\ {\isacharparenleft}rule\ notE{\isacharparenright}\isanewline
\isacommand{qed}\isamarkupfalse%
%
\endisatagproof
{\isafoldproof}%
%
\isadelimproof
\isanewline
%
\endisadelimproof
\isanewline
\isacommand{lemma}\isamarkupfalse%
\ irrelevant{\isacharunderscore}atom{\isacharunderscore}and{\isacharunderscore}l{\isadigit{2}}{\isacharcolon}\isanewline
\ \ \isakeyword{assumes}\ {\isachardoublequoteopen}p\ {\isasymnotin}\ atoms\ {\isacharparenleft}F\ \isactrlbold {\isasymand}\ G{\isacharparenright}{\isachardoublequoteclose}\isanewline
\ \ \isakeyword{shows}\ \ \ {\isachardoublequoteopen}p\ {\isasymnotin}\ atoms\ G{\isachardoublequoteclose}\isanewline
%
\isadelimproof
%
\endisadelimproof
%
\isatagproof
\isacommand{proof}\isamarkupfalse%
\ \isanewline
\ \ \isacommand{assume}\isamarkupfalse%
\ {\isachardoublequoteopen}p\ {\isasymin}\ atoms\ G{\isachardoublequoteclose}\isanewline
\ \ \isacommand{then}\isamarkupfalse%
\ \isacommand{have}\isamarkupfalse%
\ {\isachardoublequoteopen}p\ {\isasymin}\ atoms\ F\ {\isasymunion}\ atoms\ G{\isachardoublequoteclose}\isanewline
\ \ \ \ \isacommand{by}\isamarkupfalse%
\ {\isacharparenleft}rule\ UnI{\isadigit{2}}{\isacharparenright}\isanewline
\ \ \isacommand{then}\isamarkupfalse%
\ \isacommand{have}\isamarkupfalse%
\ {\isachardoublequoteopen}p\ {\isasymin}\ atoms\ {\isacharparenleft}F\ \isactrlbold {\isasymand}\ G{\isacharparenright}{\isachardoublequoteclose}\isanewline
\ \ \ \ \isacommand{by}\isamarkupfalse%
\ {\isacharparenleft}simp\ only{\isacharcolon}\ formula{\isachardot}set{\isacharparenleft}{\isadigit{4}}{\isacharparenright}{\isacharparenright}\isanewline
\ \ \isacommand{with}\isamarkupfalse%
\ assms\ \isacommand{show}\isamarkupfalse%
\ False\ \isanewline
\ \ \ \ \isacommand{by}\isamarkupfalse%
\ {\isacharparenleft}rule\ notE{\isacharparenright}\isanewline
\isacommand{qed}\isamarkupfalse%
%
\endisatagproof
{\isafoldproof}%
%
\isadelimproof
\isanewline
%
\endisadelimproof
\isanewline
\isacommand{lemma}\isamarkupfalse%
\ irrelevant{\isacharunderscore}atom{\isacharunderscore}and{\isacharcolon}\isanewline
\ \ \isakeyword{assumes}\ {\isachardoublequoteopen}p\ {\isasymnotin}\ atoms\ F\ {\isasymLongrightarrow}\ {\isasymA}{\isacharparenleft}p\ {\isacharcolon}{\isacharequal}\ V{\isacharparenright}\ {\isasymTurnstile}\ F\ {\isasymlongleftrightarrow}\ {\isasymA}\ {\isasymTurnstile}\ F{\isachardoublequoteclose}\isanewline
\ \ \ \ \ \ \ \ \ \ {\isachardoublequoteopen}p\ {\isasymnotin}\ atoms\ G\ {\isasymLongrightarrow}\ {\isasymA}{\isacharparenleft}p\ {\isacharcolon}{\isacharequal}\ V{\isacharparenright}\ {\isasymTurnstile}\ G\ {\isasymlongleftrightarrow}\ {\isasymA}\ {\isasymTurnstile}\ G{\isachardoublequoteclose}\isanewline
\ \ \ \ \ \ \ \ \ \ {\isachardoublequoteopen}p\ {\isasymnotin}\ atoms\ {\isacharparenleft}F\ \isactrlbold {\isasymand}\ G{\isacharparenright}{\isachardoublequoteclose}\isanewline
\ \ \isakeyword{shows}\ {\isachardoublequoteopen}{\isasymA}{\isacharparenleft}p\ {\isacharcolon}{\isacharequal}\ V{\isacharparenright}\ {\isasymTurnstile}\ {\isacharparenleft}F\ \isactrlbold {\isasymand}\ G{\isacharparenright}\ {\isasymlongleftrightarrow}\ {\isasymA}\ {\isasymTurnstile}\ {\isacharparenleft}F\ \isactrlbold {\isasymand}\ G{\isacharparenright}{\isachardoublequoteclose}\isanewline
%
\isadelimproof
%
\endisadelimproof
%
\isatagproof
\isacommand{proof}\isamarkupfalse%
\ {\isacharminus}\isanewline
\ \ \isacommand{have}\isamarkupfalse%
\ {\isachardoublequoteopen}p\ {\isasymnotin}\ atoms\ F{\isachardoublequoteclose}\isanewline
\ \ \ \ \isacommand{using}\isamarkupfalse%
\ assms{\isacharparenleft}{\isadigit{3}}{\isacharparenright}\isanewline
\ \ \ \ \isacommand{by}\isamarkupfalse%
\ {\isacharparenleft}rule\ irrelevant{\isacharunderscore}atom{\isacharunderscore}and{\isacharunderscore}l{\isadigit{1}}{\isacharparenright}\isanewline
\ \ \isacommand{then}\isamarkupfalse%
\ \isacommand{have}\isamarkupfalse%
\ HF{\isacharcolon}\ {\isachardoublequoteopen}{\isasymA}{\isacharparenleft}p\ {\isacharcolon}{\isacharequal}\ V{\isacharparenright}\ {\isasymTurnstile}\ F\ {\isasymlongleftrightarrow}\ {\isasymA}\ {\isasymTurnstile}\ F{\isachardoublequoteclose}\isanewline
\ \ \ \ \isacommand{by}\isamarkupfalse%
\ {\isacharparenleft}rule\ assms{\isacharparenleft}{\isadigit{1}}{\isacharparenright}{\isacharparenright}\isanewline
\ \ \isacommand{have}\isamarkupfalse%
\ {\isachardoublequoteopen}p\ {\isasymnotin}\ atoms\ G{\isachardoublequoteclose}\isanewline
\ \ \ \ \isacommand{using}\isamarkupfalse%
\ assms{\isacharparenleft}{\isadigit{3}}{\isacharparenright}\isanewline
\ \ \ \ \isacommand{by}\isamarkupfalse%
\ {\isacharparenleft}rule\ irrelevant{\isacharunderscore}atom{\isacharunderscore}and{\isacharunderscore}l{\isadigit{2}}{\isacharparenright}\isanewline
\ \ \isacommand{then}\isamarkupfalse%
\ \isacommand{have}\isamarkupfalse%
\ HG{\isacharcolon}\ {\isachardoublequoteopen}{\isasymA}{\isacharparenleft}p\ {\isacharcolon}{\isacharequal}\ V{\isacharparenright}\ {\isasymTurnstile}\ G\ {\isasymlongleftrightarrow}\ {\isasymA}\ {\isasymTurnstile}\ G{\isachardoublequoteclose}\isanewline
\ \ \ \ \isacommand{by}\isamarkupfalse%
\ {\isacharparenleft}rule\ assms{\isacharparenleft}{\isadigit{2}}{\isacharparenright}{\isacharparenright}\isanewline
\ \ \isacommand{have}\isamarkupfalse%
\ {\isachardoublequoteopen}{\isasymA}{\isacharparenleft}p\ {\isacharcolon}{\isacharequal}\ V{\isacharparenright}\ {\isasymTurnstile}\ {\isacharparenleft}F\ \isactrlbold {\isasymand}\ G{\isacharparenright}\ {\isacharequal}\ {\isacharparenleft}{\isasymA}{\isacharparenleft}p\ {\isacharcolon}{\isacharequal}\ V{\isacharparenright}\ {\isasymTurnstile}\ F\ {\isasymand}\ {\isasymA}{\isacharparenleft}p\ {\isacharcolon}{\isacharequal}\ V{\isacharparenright}\ {\isasymTurnstile}\ G{\isacharparenright}{\isachardoublequoteclose}\isanewline
\ \ \ \ \isacommand{by}\isamarkupfalse%
\ {\isacharparenleft}simp\ only{\isacharcolon}\ formula{\isacharunderscore}semantics{\isachardot}simps{\isacharparenleft}{\isadigit{4}}{\isacharparenright}{\isacharparenright}\isanewline
\ \ \isacommand{also}\isamarkupfalse%
\ \isacommand{have}\isamarkupfalse%
\ {\isachardoublequoteopen}{\isasymdots}\ {\isacharequal}\ {\isacharparenleft}{\isasymA}\ {\isasymTurnstile}\ F\ {\isasymand}\ {\isasymA}\ {\isasymTurnstile}\ G{\isacharparenright}{\isachardoublequoteclose}\isanewline
\ \ \ \ \isacommand{by}\isamarkupfalse%
\ {\isacharparenleft}simp\ only{\isacharcolon}\ HF\ HG{\isacharparenright}\isanewline
\ \ \isacommand{also}\isamarkupfalse%
\ \isacommand{have}\isamarkupfalse%
\ {\isachardoublequoteopen}{\isasymdots}\ {\isacharequal}\ {\isasymA}\ {\isasymTurnstile}\ {\isacharparenleft}F\ \isactrlbold {\isasymand}\ G{\isacharparenright}{\isachardoublequoteclose}\isanewline
\ \ \ \ \isacommand{by}\isamarkupfalse%
\ {\isacharparenleft}simp\ only{\isacharcolon}\ formula{\isacharunderscore}semantics{\isachardot}simps{\isacharparenleft}{\isadigit{4}}{\isacharparenright}{\isacharparenright}\isanewline
\ \ \isacommand{finally}\isamarkupfalse%
\ \isacommand{show}\isamarkupfalse%
\ {\isachardoublequoteopen}{\isasymA}{\isacharparenleft}p\ {\isacharcolon}{\isacharequal}\ V{\isacharparenright}\ {\isasymTurnstile}\ {\isacharparenleft}F\ \isactrlbold {\isasymand}\ G{\isacharparenright}\ {\isasymlongleftrightarrow}\ {\isasymA}\ {\isasymTurnstile}\ {\isacharparenleft}F\ \isactrlbold {\isasymand}\ G{\isacharparenright}{\isachardoublequoteclose}\isanewline
\ \ \ \ \isacommand{by}\isamarkupfalse%
\ this\isanewline
\isacommand{qed}\isamarkupfalse%
%
\endisatagproof
{\isafoldproof}%
%
\isadelimproof
\isanewline
%
\endisadelimproof
\isanewline
\isacommand{lemma}\isamarkupfalse%
\ irrelevant{\isacharunderscore}atom{\isacharunderscore}or{\isacharunderscore}l{\isadigit{1}}{\isacharcolon}\isanewline
\ \ \isakeyword{assumes}\ {\isachardoublequoteopen}p\ {\isasymnotin}\ atoms\ {\isacharparenleft}F\ \isactrlbold {\isasymor}\ G{\isacharparenright}{\isachardoublequoteclose}\isanewline
\ \ \isakeyword{shows}\ \ \ {\isachardoublequoteopen}p\ {\isasymnotin}\ atoms\ F{\isachardoublequoteclose}\isanewline
%
\isadelimproof
%
\endisadelimproof
%
\isatagproof
\isacommand{proof}\isamarkupfalse%
\ \isanewline
\ \ \isacommand{assume}\isamarkupfalse%
\ {\isachardoublequoteopen}p\ {\isasymin}\ atoms\ F{\isachardoublequoteclose}\isanewline
\ \ \isacommand{then}\isamarkupfalse%
\ \isacommand{have}\isamarkupfalse%
\ {\isachardoublequoteopen}p\ {\isasymin}\ atoms\ F\ {\isasymunion}\ atoms\ G{\isachardoublequoteclose}\isanewline
\ \ \ \ \isacommand{by}\isamarkupfalse%
\ {\isacharparenleft}rule\ UnI{\isadigit{1}}{\isacharparenright}\isanewline
\ \ \isacommand{then}\isamarkupfalse%
\ \isacommand{have}\isamarkupfalse%
\ {\isachardoublequoteopen}p\ {\isasymin}\ atoms\ {\isacharparenleft}F\ \isactrlbold {\isasymor}\ G{\isacharparenright}{\isachardoublequoteclose}\isanewline
\ \ \ \ \isacommand{by}\isamarkupfalse%
\ {\isacharparenleft}simp\ only{\isacharcolon}\ formula{\isachardot}set{\isacharparenleft}{\isadigit{5}}{\isacharparenright}{\isacharparenright}\isanewline
\ \ \isacommand{with}\isamarkupfalse%
\ assms\ \isacommand{show}\isamarkupfalse%
\ False\ \isanewline
\ \ \ \ \isacommand{by}\isamarkupfalse%
\ {\isacharparenleft}rule\ notE{\isacharparenright}\isanewline
\isacommand{qed}\isamarkupfalse%
%
\endisatagproof
{\isafoldproof}%
%
\isadelimproof
\isanewline
%
\endisadelimproof
\isanewline
\isacommand{lemma}\isamarkupfalse%
\ irrelevant{\isacharunderscore}atom{\isacharunderscore}or{\isacharunderscore}l{\isadigit{2}}{\isacharcolon}\isanewline
\ \ \isakeyword{assumes}\ {\isachardoublequoteopen}p\ {\isasymnotin}\ atoms\ {\isacharparenleft}F\ \isactrlbold {\isasymor}\ G{\isacharparenright}{\isachardoublequoteclose}\isanewline
\ \ \isakeyword{shows}\ \ \ {\isachardoublequoteopen}p\ {\isasymnotin}\ atoms\ G{\isachardoublequoteclose}\isanewline
%
\isadelimproof
%
\endisadelimproof
%
\isatagproof
\isacommand{proof}\isamarkupfalse%
\ \isanewline
\ \ \isacommand{assume}\isamarkupfalse%
\ {\isachardoublequoteopen}p\ {\isasymin}\ atoms\ G{\isachardoublequoteclose}\isanewline
\ \ \isacommand{then}\isamarkupfalse%
\ \isacommand{have}\isamarkupfalse%
\ {\isachardoublequoteopen}p\ {\isasymin}\ atoms\ F\ {\isasymunion}\ atoms\ G{\isachardoublequoteclose}\isanewline
\ \ \ \ \isacommand{by}\isamarkupfalse%
\ {\isacharparenleft}rule\ UnI{\isadigit{2}}{\isacharparenright}\isanewline
\ \ \isacommand{then}\isamarkupfalse%
\ \isacommand{have}\isamarkupfalse%
\ {\isachardoublequoteopen}p\ {\isasymin}\ atoms\ {\isacharparenleft}F\ \isactrlbold {\isasymor}\ G{\isacharparenright}{\isachardoublequoteclose}\isanewline
\ \ \ \ \isacommand{by}\isamarkupfalse%
\ {\isacharparenleft}simp\ only{\isacharcolon}\ formula{\isachardot}set{\isacharparenleft}{\isadigit{5}}{\isacharparenright}{\isacharparenright}\isanewline
\ \ \isacommand{with}\isamarkupfalse%
\ assms\ \isacommand{show}\isamarkupfalse%
\ False\ \isanewline
\ \ \ \ \isacommand{by}\isamarkupfalse%
\ {\isacharparenleft}rule\ notE{\isacharparenright}\isanewline
\isacommand{qed}\isamarkupfalse%
%
\endisatagproof
{\isafoldproof}%
%
\isadelimproof
\isanewline
%
\endisadelimproof
\isanewline
\isacommand{lemma}\isamarkupfalse%
\ irrelevant{\isacharunderscore}atom{\isacharunderscore}or{\isacharcolon}\isanewline
\ \ \isakeyword{assumes}\ {\isachardoublequoteopen}p\ {\isasymnotin}\ atoms\ F\ {\isasymLongrightarrow}\ {\isasymA}{\isacharparenleft}p\ {\isacharcolon}{\isacharequal}\ V{\isacharparenright}\ {\isasymTurnstile}\ F\ {\isasymlongleftrightarrow}\ {\isasymA}\ {\isasymTurnstile}\ F{\isachardoublequoteclose}\isanewline
\ \ \ \ \ \ \ \ \ \ {\isachardoublequoteopen}p\ {\isasymnotin}\ atoms\ G\ {\isasymLongrightarrow}\ {\isasymA}{\isacharparenleft}p\ {\isacharcolon}{\isacharequal}\ V{\isacharparenright}\ {\isasymTurnstile}\ G\ {\isasymlongleftrightarrow}\ {\isasymA}\ {\isasymTurnstile}\ G{\isachardoublequoteclose}\isanewline
\ \ \ \ \ \ \ \ \ \ {\isachardoublequoteopen}p\ {\isasymnotin}\ atoms\ {\isacharparenleft}F\ \isactrlbold {\isasymor}\ G{\isacharparenright}{\isachardoublequoteclose}\isanewline
\ \ \isakeyword{shows}\ \ \ {\isachardoublequoteopen}{\isasymA}{\isacharparenleft}p\ {\isacharcolon}{\isacharequal}\ V{\isacharparenright}\ {\isasymTurnstile}\ {\isacharparenleft}F\ \isactrlbold {\isasymor}\ G{\isacharparenright}\ {\isasymlongleftrightarrow}\ {\isasymA}\ {\isasymTurnstile}\ {\isacharparenleft}F\ \isactrlbold {\isasymor}\ G{\isacharparenright}{\isachardoublequoteclose}\isanewline
%
\isadelimproof
%
\endisadelimproof
%
\isatagproof
\isacommand{proof}\isamarkupfalse%
\ {\isacharminus}\isanewline
\ \ \isacommand{have}\isamarkupfalse%
\ {\isachardoublequoteopen}p\ {\isasymnotin}\ atoms\ F{\isachardoublequoteclose}\isanewline
\ \ \ \ \isacommand{using}\isamarkupfalse%
\ assms{\isacharparenleft}{\isadigit{3}}{\isacharparenright}\isanewline
\ \ \ \ \isacommand{by}\isamarkupfalse%
\ {\isacharparenleft}rule\ irrelevant{\isacharunderscore}atom{\isacharunderscore}or{\isacharunderscore}l{\isadigit{1}}{\isacharparenright}\isanewline
\ \ \isacommand{then}\isamarkupfalse%
\ \isacommand{have}\isamarkupfalse%
\ HF{\isacharcolon}\ {\isachardoublequoteopen}{\isasymA}{\isacharparenleft}p\ {\isacharcolon}{\isacharequal}\ V{\isacharparenright}\ {\isasymTurnstile}\ F\ {\isasymlongleftrightarrow}\ {\isasymA}\ {\isasymTurnstile}\ F{\isachardoublequoteclose}\isanewline
\ \ \ \ \isacommand{by}\isamarkupfalse%
\ {\isacharparenleft}rule\ assms{\isacharparenleft}{\isadigit{1}}{\isacharparenright}{\isacharparenright}\isanewline
\ \ \isacommand{have}\isamarkupfalse%
\ {\isachardoublequoteopen}p\ {\isasymnotin}\ atoms\ G{\isachardoublequoteclose}\isanewline
\ \ \ \ \isacommand{using}\isamarkupfalse%
\ assms{\isacharparenleft}{\isadigit{3}}{\isacharparenright}\isanewline
\ \ \ \ \isacommand{by}\isamarkupfalse%
\ {\isacharparenleft}rule\ irrelevant{\isacharunderscore}atom{\isacharunderscore}or{\isacharunderscore}l{\isadigit{2}}{\isacharparenright}\isanewline
\ \ \isacommand{then}\isamarkupfalse%
\ \isacommand{have}\isamarkupfalse%
\ HG{\isacharcolon}\ {\isachardoublequoteopen}{\isasymA}{\isacharparenleft}p\ {\isacharcolon}{\isacharequal}\ V{\isacharparenright}\ {\isasymTurnstile}\ G\ {\isasymlongleftrightarrow}\ {\isasymA}\ {\isasymTurnstile}\ G{\isachardoublequoteclose}\isanewline
\ \ \ \ \isacommand{by}\isamarkupfalse%
\ {\isacharparenleft}rule\ assms{\isacharparenleft}{\isadigit{2}}{\isacharparenright}{\isacharparenright}\isanewline
\ \ \isacommand{have}\isamarkupfalse%
\ {\isachardoublequoteopen}{\isasymA}{\isacharparenleft}p\ {\isacharcolon}{\isacharequal}\ V{\isacharparenright}\ {\isasymTurnstile}\ {\isacharparenleft}F\ \isactrlbold {\isasymor}\ G{\isacharparenright}\ {\isacharequal}\ {\isacharparenleft}{\isasymA}{\isacharparenleft}p\ {\isacharcolon}{\isacharequal}\ V{\isacharparenright}\ {\isasymTurnstile}\ F\ {\isasymor}\ {\isasymA}{\isacharparenleft}p\ {\isacharcolon}{\isacharequal}\ V{\isacharparenright}\ {\isasymTurnstile}\ G{\isacharparenright}{\isachardoublequoteclose}\isanewline
\ \ \ \ \isacommand{by}\isamarkupfalse%
\ {\isacharparenleft}simp\ only{\isacharcolon}\ formula{\isacharunderscore}semantics{\isachardot}simps{\isacharparenleft}{\isadigit{5}}{\isacharparenright}{\isacharparenright}\isanewline
\ \ \isacommand{also}\isamarkupfalse%
\ \isacommand{have}\isamarkupfalse%
\ {\isachardoublequoteopen}{\isasymdots}\ {\isacharequal}\ {\isacharparenleft}{\isasymA}\ {\isasymTurnstile}\ F\ {\isasymor}\ {\isasymA}\ {\isasymTurnstile}\ G{\isacharparenright}{\isachardoublequoteclose}\isanewline
\ \ \ \ \isacommand{by}\isamarkupfalse%
\ {\isacharparenleft}simp\ only{\isacharcolon}\ HF\ HG{\isacharparenright}\isanewline
\ \ \isacommand{also}\isamarkupfalse%
\ \isacommand{have}\isamarkupfalse%
\ {\isachardoublequoteopen}{\isasymdots}\ {\isacharequal}\ {\isasymA}\ {\isasymTurnstile}\ {\isacharparenleft}F\ \isactrlbold {\isasymor}\ G{\isacharparenright}{\isachardoublequoteclose}\isanewline
\ \ \ \ \isacommand{by}\isamarkupfalse%
\ {\isacharparenleft}simp\ only{\isacharcolon}\ formula{\isacharunderscore}semantics{\isachardot}simps{\isacharparenleft}{\isadigit{5}}{\isacharparenright}{\isacharparenright}\isanewline
\ \ \isacommand{finally}\isamarkupfalse%
\ \isacommand{show}\isamarkupfalse%
\ {\isachardoublequoteopen}{\isasymA}{\isacharparenleft}p\ {\isacharcolon}{\isacharequal}\ V{\isacharparenright}\ {\isasymTurnstile}\ {\isacharparenleft}F\ \isactrlbold {\isasymor}\ G{\isacharparenright}\ {\isasymlongleftrightarrow}\ {\isasymA}\ {\isasymTurnstile}\ {\isacharparenleft}F\ \isactrlbold {\isasymor}\ G{\isacharparenright}{\isachardoublequoteclose}\isanewline
\ \ \ \ \isacommand{by}\isamarkupfalse%
\ this\isanewline
\isacommand{qed}\isamarkupfalse%
%
\endisatagproof
{\isafoldproof}%
%
\isadelimproof
\isanewline
%
\endisadelimproof
\isanewline
\isacommand{lemma}\isamarkupfalse%
\ irrelevant{\isacharunderscore}atom{\isacharunderscore}imp{\isacharunderscore}l{\isadigit{1}}{\isacharcolon}\isanewline
\ \ \isakeyword{assumes}\ {\isachardoublequoteopen}p\ {\isasymnotin}\ atoms\ {\isacharparenleft}F\ \isactrlbold {\isasymrightarrow}\ G{\isacharparenright}{\isachardoublequoteclose}\isanewline
\ \ \isakeyword{shows}\ \ \ {\isachardoublequoteopen}p\ {\isasymnotin}\ atoms\ F{\isachardoublequoteclose}\isanewline
%
\isadelimproof
%
\endisadelimproof
%
\isatagproof
\isacommand{proof}\isamarkupfalse%
\ \isanewline
\ \ \isacommand{assume}\isamarkupfalse%
\ {\isachardoublequoteopen}p\ {\isasymin}\ atoms\ F{\isachardoublequoteclose}\isanewline
\ \ \isacommand{then}\isamarkupfalse%
\ \isacommand{have}\isamarkupfalse%
\ {\isachardoublequoteopen}p\ {\isasymin}\ atoms\ F\ {\isasymunion}\ atoms\ G{\isachardoublequoteclose}\isanewline
\ \ \ \ \isacommand{by}\isamarkupfalse%
\ {\isacharparenleft}rule\ UnI{\isadigit{1}}{\isacharparenright}\isanewline
\ \ \isacommand{then}\isamarkupfalse%
\ \isacommand{have}\isamarkupfalse%
\ {\isachardoublequoteopen}p\ {\isasymin}\ atoms\ {\isacharparenleft}F\ \isactrlbold {\isasymrightarrow}\ G{\isacharparenright}{\isachardoublequoteclose}\isanewline
\ \ \ \ \isacommand{by}\isamarkupfalse%
\ {\isacharparenleft}simp\ only{\isacharcolon}\ formula{\isachardot}set{\isacharparenleft}{\isadigit{6}}{\isacharparenright}{\isacharparenright}\isanewline
\ \ \isacommand{with}\isamarkupfalse%
\ assms\ \isacommand{show}\isamarkupfalse%
\ False\ \isanewline
\ \ \ \ \isacommand{by}\isamarkupfalse%
\ {\isacharparenleft}rule\ notE{\isacharparenright}\isanewline
\isacommand{qed}\isamarkupfalse%
%
\endisatagproof
{\isafoldproof}%
%
\isadelimproof
\isanewline
%
\endisadelimproof
\isanewline
\isacommand{lemma}\isamarkupfalse%
\ irrelevant{\isacharunderscore}atom{\isacharunderscore}imp{\isacharunderscore}l{\isadigit{2}}{\isacharcolon}\isanewline
\ \ \isakeyword{assumes}\ {\isachardoublequoteopen}p\ {\isasymnotin}\ atoms\ {\isacharparenleft}F\ \isactrlbold {\isasymrightarrow}\ G{\isacharparenright}{\isachardoublequoteclose}\isanewline
\ \ \isakeyword{shows}\ \ \ {\isachardoublequoteopen}p\ {\isasymnotin}\ atoms\ G{\isachardoublequoteclose}\isanewline
%
\isadelimproof
%
\endisadelimproof
%
\isatagproof
\isacommand{proof}\isamarkupfalse%
\ \isanewline
\ \ \isacommand{assume}\isamarkupfalse%
\ {\isachardoublequoteopen}p\ {\isasymin}\ atoms\ G{\isachardoublequoteclose}\isanewline
\ \ \isacommand{then}\isamarkupfalse%
\ \isacommand{have}\isamarkupfalse%
\ {\isachardoublequoteopen}p\ {\isasymin}\ atoms\ F\ {\isasymunion}\ atoms\ G{\isachardoublequoteclose}\isanewline
\ \ \ \ \isacommand{by}\isamarkupfalse%
\ {\isacharparenleft}rule\ UnI{\isadigit{2}}{\isacharparenright}\isanewline
\ \ \isacommand{then}\isamarkupfalse%
\ \isacommand{have}\isamarkupfalse%
\ {\isachardoublequoteopen}p\ {\isasymin}\ atoms\ {\isacharparenleft}F\ \isactrlbold {\isasymrightarrow}\ G{\isacharparenright}{\isachardoublequoteclose}\isanewline
\ \ \ \ \isacommand{by}\isamarkupfalse%
\ {\isacharparenleft}simp\ only{\isacharcolon}\ formula{\isachardot}set{\isacharparenleft}{\isadigit{6}}{\isacharparenright}{\isacharparenright}\isanewline
\ \ \isacommand{with}\isamarkupfalse%
\ assms\ \isacommand{show}\isamarkupfalse%
\ False\ \isanewline
\ \ \ \ \isacommand{by}\isamarkupfalse%
\ {\isacharparenleft}rule\ notE{\isacharparenright}\isanewline
\isacommand{qed}\isamarkupfalse%
%
\endisatagproof
{\isafoldproof}%
%
\isadelimproof
\isanewline
%
\endisadelimproof
\isanewline
\isacommand{lemma}\isamarkupfalse%
\ irrelevant{\isacharunderscore}atom{\isacharunderscore}imp{\isacharcolon}\isanewline
\ \ \isakeyword{assumes}\ {\isachardoublequoteopen}p\ {\isasymnotin}\ atoms\ F\ {\isasymLongrightarrow}\ {\isasymA}{\isacharparenleft}p\ {\isacharcolon}{\isacharequal}\ V{\isacharparenright}\ {\isasymTurnstile}\ F\ {\isasymlongleftrightarrow}\ {\isasymA}\ {\isasymTurnstile}\ F{\isachardoublequoteclose}\isanewline
\ \ \ \ \ \ \ \ \ \ {\isachardoublequoteopen}p\ {\isasymnotin}\ atoms\ G\ {\isasymLongrightarrow}\ {\isasymA}{\isacharparenleft}p\ {\isacharcolon}{\isacharequal}\ V{\isacharparenright}\ {\isasymTurnstile}\ G\ {\isasymlongleftrightarrow}\ {\isasymA}\ {\isasymTurnstile}\ G{\isachardoublequoteclose}\isanewline
\ \ \ \ \ \ \ \ \ \ {\isachardoublequoteopen}p\ {\isasymnotin}\ atoms\ {\isacharparenleft}F\ \isactrlbold {\isasymrightarrow}\ G{\isacharparenright}{\isachardoublequoteclose}\isanewline
\ \ \isakeyword{shows}\ {\isachardoublequoteopen}{\isasymA}{\isacharparenleft}p\ {\isacharcolon}{\isacharequal}\ V{\isacharparenright}\ {\isasymTurnstile}\ {\isacharparenleft}F\ \isactrlbold {\isasymrightarrow}\ G{\isacharparenright}\ {\isasymlongleftrightarrow}\ {\isasymA}\ {\isasymTurnstile}\ {\isacharparenleft}F\ \isactrlbold {\isasymrightarrow}\ G{\isacharparenright}{\isachardoublequoteclose}\isanewline
%
\isadelimproof
%
\endisadelimproof
%
\isatagproof
\isacommand{proof}\isamarkupfalse%
\ {\isacharminus}\isanewline
\ \ \isacommand{have}\isamarkupfalse%
\ {\isachardoublequoteopen}p\ {\isasymnotin}\ atoms\ F{\isachardoublequoteclose}\isanewline
\ \ \ \ \isacommand{using}\isamarkupfalse%
\ assms{\isacharparenleft}{\isadigit{3}}{\isacharparenright}\isanewline
\ \ \ \ \isacommand{by}\isamarkupfalse%
\ {\isacharparenleft}rule\ irrelevant{\isacharunderscore}atom{\isacharunderscore}imp{\isacharunderscore}l{\isadigit{1}}{\isacharparenright}\isanewline
\ \ \isacommand{then}\isamarkupfalse%
\ \isacommand{have}\isamarkupfalse%
\ HF{\isacharcolon}\ {\isachardoublequoteopen}{\isasymA}{\isacharparenleft}p\ {\isacharcolon}{\isacharequal}\ V{\isacharparenright}\ {\isasymTurnstile}\ F\ {\isasymlongleftrightarrow}\ {\isasymA}\ {\isasymTurnstile}\ F{\isachardoublequoteclose}\isanewline
\ \ \ \ \isacommand{by}\isamarkupfalse%
\ {\isacharparenleft}rule\ assms{\isacharparenleft}{\isadigit{1}}{\isacharparenright}{\isacharparenright}\isanewline
\ \ \isacommand{have}\isamarkupfalse%
\ {\isachardoublequoteopen}p\ {\isasymnotin}\ atoms\ G{\isachardoublequoteclose}\isanewline
\ \ \ \ \isacommand{using}\isamarkupfalse%
\ assms{\isacharparenleft}{\isadigit{3}}{\isacharparenright}\isanewline
\ \ \ \ \isacommand{by}\isamarkupfalse%
\ {\isacharparenleft}rule\ irrelevant{\isacharunderscore}atom{\isacharunderscore}imp{\isacharunderscore}l{\isadigit{2}}{\isacharparenright}\isanewline
\ \ \isacommand{then}\isamarkupfalse%
\ \isacommand{have}\isamarkupfalse%
\ HG{\isacharcolon}\ {\isachardoublequoteopen}{\isasymA}{\isacharparenleft}p\ {\isacharcolon}{\isacharequal}\ V{\isacharparenright}\ {\isasymTurnstile}\ G\ {\isasymlongleftrightarrow}\ {\isasymA}\ {\isasymTurnstile}\ G{\isachardoublequoteclose}\isanewline
\ \ \ \ \isacommand{by}\isamarkupfalse%
\ {\isacharparenleft}rule\ assms{\isacharparenleft}{\isadigit{2}}{\isacharparenright}{\isacharparenright}\isanewline
\ \ \isacommand{have}\isamarkupfalse%
\ {\isachardoublequoteopen}{\isasymA}{\isacharparenleft}p\ {\isacharcolon}{\isacharequal}\ V{\isacharparenright}\ {\isasymTurnstile}\ {\isacharparenleft}F\ \isactrlbold {\isasymrightarrow}\ G{\isacharparenright}\ {\isacharequal}\ {\isacharparenleft}{\isasymA}{\isacharparenleft}p\ {\isacharcolon}{\isacharequal}\ V{\isacharparenright}\ {\isasymTurnstile}\ F\ {\isasymlongrightarrow}\ {\isasymA}{\isacharparenleft}p\ {\isacharcolon}{\isacharequal}\ V{\isacharparenright}\ {\isasymTurnstile}\ G{\isacharparenright}{\isachardoublequoteclose}\isanewline
\ \ \ \ \isacommand{by}\isamarkupfalse%
\ {\isacharparenleft}simp\ only{\isacharcolon}\ formula{\isacharunderscore}semantics{\isachardot}simps{\isacharparenleft}{\isadigit{6}}{\isacharparenright}{\isacharparenright}\isanewline
\ \ \isacommand{also}\isamarkupfalse%
\ \isacommand{have}\isamarkupfalse%
\ {\isachardoublequoteopen}{\isasymdots}\ {\isacharequal}\ {\isacharparenleft}{\isasymA}\ {\isasymTurnstile}\ F\ {\isasymlongrightarrow}\ {\isasymA}\ {\isasymTurnstile}\ G{\isacharparenright}{\isachardoublequoteclose}\isanewline
\ \ \ \ \isacommand{by}\isamarkupfalse%
\ {\isacharparenleft}simp\ only{\isacharcolon}\ HF\ HG{\isacharparenright}\isanewline
\ \ \isacommand{also}\isamarkupfalse%
\ \isacommand{have}\isamarkupfalse%
\ {\isachardoublequoteopen}{\isasymdots}\ {\isacharequal}\ {\isasymA}\ {\isasymTurnstile}\ {\isacharparenleft}F\ \isactrlbold {\isasymrightarrow}\ G{\isacharparenright}{\isachardoublequoteclose}\isanewline
\ \ \ \ \isacommand{by}\isamarkupfalse%
\ {\isacharparenleft}simp\ only{\isacharcolon}\ formula{\isacharunderscore}semantics{\isachardot}simps{\isacharparenleft}{\isadigit{6}}{\isacharparenright}{\isacharparenright}\isanewline
\ \ \isacommand{finally}\isamarkupfalse%
\ \isacommand{show}\isamarkupfalse%
\ {\isachardoublequoteopen}{\isasymA}{\isacharparenleft}p\ {\isacharcolon}{\isacharequal}\ V{\isacharparenright}\ {\isasymTurnstile}\ {\isacharparenleft}F\ \isactrlbold {\isasymrightarrow}\ G{\isacharparenright}\ {\isasymlongleftrightarrow}\ {\isasymA}\ {\isasymTurnstile}\ {\isacharparenleft}F\ \isactrlbold {\isasymrightarrow}\ G{\isacharparenright}{\isachardoublequoteclose}\isanewline
\ \ \ \ \isacommand{by}\isamarkupfalse%
\ this\isanewline
\isacommand{qed}\isamarkupfalse%
%
\endisatagproof
{\isafoldproof}%
%
\isadelimproof
\isanewline
%
\endisadelimproof
\isanewline
\isacommand{lemma}\isamarkupfalse%
\ {\isachardoublequoteopen}p\ {\isasymnotin}\ atoms\ F\ {\isasymLongrightarrow}\ {\isacharparenleft}{\isasymA}{\isacharparenleft}p\ {\isacharcolon}{\isacharequal}\ V{\isacharparenright}{\isacharparenright}\ {\isasymTurnstile}\ F\ {\isasymlongleftrightarrow}\ {\isasymA}\ {\isasymTurnstile}\ F{\isachardoublequoteclose}\isanewline
%
\isadelimproof
%
\endisadelimproof
%
\isatagproof
\isacommand{proof}\isamarkupfalse%
\ {\isacharparenleft}induction\ F{\isacharparenright}\isanewline
\ \ \isacommand{case}\isamarkupfalse%
\ {\isacharparenleft}Atom\ x{\isacharparenright}\isanewline
\ \ \isacommand{then}\isamarkupfalse%
\ \isacommand{show}\isamarkupfalse%
\ {\isacharquery}case\ \isacommand{by}\isamarkupfalse%
\ {\isacharparenleft}rule\ irrelevant{\isacharunderscore}atom{\isacharunderscore}atomic{\isacharparenright}\isanewline
\isacommand{next}\isamarkupfalse%
\isanewline
\ \ \isacommand{case}\isamarkupfalse%
\ Bot\isanewline
\ \ \isacommand{then}\isamarkupfalse%
\ \isacommand{show}\isamarkupfalse%
\ {\isacharquery}case\ \isacommand{by}\isamarkupfalse%
\ {\isacharparenleft}rule\ irrelevant{\isacharunderscore}atom{\isacharunderscore}bot{\isacharparenright}\isanewline
\isacommand{next}\isamarkupfalse%
\isanewline
\ \ \isacommand{case}\isamarkupfalse%
\ {\isacharparenleft}Not\ F{\isacharparenright}\isanewline
\ \ \isacommand{then}\isamarkupfalse%
\ \isacommand{show}\isamarkupfalse%
\ {\isacharquery}case\ \isacommand{by}\isamarkupfalse%
\ {\isacharparenleft}rule\ irrelevant{\isacharunderscore}atom{\isacharunderscore}not{\isacharparenright}\isanewline
\isacommand{next}\isamarkupfalse%
\isanewline
\ \ \isacommand{case}\isamarkupfalse%
\ {\isacharparenleft}And\ F{\isadigit{1}}\ F{\isadigit{2}}{\isacharparenright}\isanewline
\ \ \isacommand{then}\isamarkupfalse%
\ \isacommand{show}\isamarkupfalse%
\ {\isacharquery}case\ \isacommand{by}\isamarkupfalse%
\ {\isacharparenleft}rule\ irrelevant{\isacharunderscore}atom{\isacharunderscore}and{\isacharparenright}\isanewline
\isacommand{next}\isamarkupfalse%
\isanewline
\ \ \isacommand{case}\isamarkupfalse%
\ {\isacharparenleft}Or\ F{\isadigit{1}}\ F{\isadigit{2}}{\isacharparenright}\isanewline
\ \ \isacommand{then}\isamarkupfalse%
\ \isacommand{show}\isamarkupfalse%
\ {\isacharquery}case\ \isacommand{by}\isamarkupfalse%
\ {\isacharparenleft}rule\ irrelevant{\isacharunderscore}atom{\isacharunderscore}or{\isacharparenright}\isanewline
\isacommand{next}\isamarkupfalse%
\isanewline
\ \ \isacommand{case}\isamarkupfalse%
\ {\isacharparenleft}Imp\ F{\isadigit{1}}\ F{\isadigit{2}}{\isacharparenright}\isanewline
\ \ \isacommand{then}\isamarkupfalse%
\ \isacommand{show}\isamarkupfalse%
\ {\isacharquery}case\ \isacommand{by}\isamarkupfalse%
\ {\isacharparenleft}rule\ irrelevant{\isacharunderscore}atom{\isacharunderscore}imp{\isacharparenright}\isanewline
\isacommand{qed}\isamarkupfalse%
%
\endisatagproof
{\isafoldproof}%
%
\isadelimproof
%
\endisadelimproof
%
\begin{isamarkuptext}%
Por último, su demostración automática.%
\end{isamarkuptext}\isamarkuptrue%
\isacommand{lemma}\isamarkupfalse%
\ irrelevant{\isacharunderscore}atom{\isacharcolon}\ \isanewline
\ \ {\isachardoublequoteopen}p\ {\isasymnotin}\ atoms\ F\ {\isasymLongrightarrow}\ {\isacharparenleft}{\isasymA}{\isacharparenleft}p\ {\isacharcolon}{\isacharequal}\ V{\isacharparenright}{\isacharparenright}\ {\isasymTurnstile}\ F\ {\isasymlongleftrightarrow}\ {\isasymA}\ {\isasymTurnstile}\ F{\isachardoublequoteclose}\isanewline
%
\isadelimproof
\ \ %
\endisadelimproof
%
\isatagproof
\isacommand{by}\isamarkupfalse%
\ {\isacharparenleft}induction\ F{\isacharparenright}\ simp{\isacharunderscore}all%
\endisatagproof
{\isafoldproof}%
%
\isadelimproof
%
\endisadelimproof
%
\begin{isamarkuptext}%
Procedamos con el siguiente lema de la sección.

  \begin{lema}
    Si dos interpretaciones coinciden sobre el conjunto de átomos de una 
    fórmula, entonces dicha fórmula tiene el mismo valor en ambas
    interpretaciones. 
  \end{lema}

  Su formalización en Isabelle es la siguiente.%
\end{isamarkuptext}\isamarkuptrue%
\isacommand{lemma}\isamarkupfalse%
\ {\isachardoublequoteopen}{\isasymforall}k\ {\isasymin}\ atoms\ F{\isachardot}\ {\isasymA}\isactrlsub {\isadigit{1}}\ k\ {\isacharequal}\ {\isasymA}\isactrlsub {\isadigit{2}}\ k\ {\isasymLongrightarrow}\ {\isasymA}\isactrlsub {\isadigit{1}}\ {\isasymTurnstile}\ F\ {\isasymlongleftrightarrow}\ {\isasymA}\isactrlsub {\isadigit{2}}\ {\isasymTurnstile}\ F{\isachardoublequoteclose}\isanewline
%
\isadelimproof
\ \ %
\endisadelimproof
%
\isatagproof
\isacommand{oops}\isamarkupfalse%
%
\endisatagproof
{\isafoldproof}%
%
\isadelimproof
%
\endisadelimproof
%
\begin{isamarkuptext}%
Vamos a probar el resultado.

  \begin{demostracion}
    La prueba sigue el esquema de inducción sobre la estructura de las
    fórmulas. De este modo, procedamos con la demostración de cada
    caso.

    En primer lugar sea una fórmula atómica \isa{p}, donde \isa{p} es una 
    variable proposicional cualquiera. Sean las interpretaciones
    \isa{{\isasymA}\isactrlsub {\isadigit{1}}} y \isa{{\isasymA}\isactrlsub {\isadigit{2}}} tales que toman los mismos valores sobre el conjunto de
    átomos de \isa{p}. Veamos que el valor de \isa{p} en \isa{{\isasymA}\isactrlsub {\isadigit{1}}} coincide con
    su valor en \isa{{\isasymA}\isactrlsub {\isadigit{2}}}. Por definición, el valor de \isa{p} en la
    interpretación \isa{{\isasymA}\isactrlsub {\isadigit{1}}} es \isa{{\isasymA}\isactrlsub {\isadigit{1}}{\isacharparenleft}p{\isacharparenright}}. Como el conjunto de átomos de
    \isa{p} es \isa{{\isacharbraceleft}p{\isacharbraceright}}, se tiene por hipótesis que \isa{{\isasymA}\isactrlsub {\isadigit{1}}{\isacharparenleft}p{\isacharparenright}\ {\isacharequal}\ {\isasymA}\isactrlsub {\isadigit{2}}{\isacharparenleft}p{\isacharparenright}}.
    Finalmente, aplicando la definición, esto es igual al valor de la 
    fórmula \isa{p} en la interpretación \isa{{\isasymA}\isactrlsub {\isadigit{2}}}, como queríamos probar.

    Consideremos ahora la fórmula \isa{{\isasymbottom}} y dos interpretaciones en las 
    condiciones del enunciado. Es fácil observar que, como el valor de 
    \isa{{\isasymbottom}} es \isa{Falso} en cualquier interpretación, se tiene en 
    particular el resultado.

    Sea una fórmula \isa{F} tal que, si dos interpretaciones coinciden sobre
    el conjunto de átomos de \isa{F}, entonces el valor de \isa{F} es el mismo
    en ambas interpretaciones. Sean dos interpretaciones cualesquiera
    \isa{{\isasymA}\isactrlsub {\isadigit{1}}} y \isa{{\isasymA}\isactrlsub {\isadigit{2}}} que toman los mismos valores sobre el conjunto de
    átomos de \isa{{\isasymnot}\ F}. Vamos a probar que el valor de \isa{{\isasymnot}\ F} en \isa{{\isasymA}\isactrlsub {\isadigit{1}}}
    coincide con su valor en \isa{{\isasymA}\isactrlsub {\isadigit{2}}}.
    Observemos que, como el conjunto de átomos de \isa{F} y \isa{{\isasymnot}\ F}
    coinciden, se tiene por hipótesis de inducción que el valor de \isa{F}
    en \isa{{\isasymA}\isactrlsub {\isadigit{1}}} coincide con su valor en \isa{{\isasymA}\isactrlsub {\isadigit{2}}}. Por otro lado, por
    definición, el valor de \isa{{\isasymnot}\ F} en \isa{{\isasymA}\isactrlsub {\isadigit{1}}} es la negación del valor
    de \isa{F} en \isa{{\isasymA}\isactrlsub {\isadigit{1}}}. Por la observación anterior, esto es igual a la
    negación del valor de \isa{F} en \isa{{\isasymA}\isactrlsub {\isadigit{2}}} que, por definición, es el
    valor de \isa{{\isasymnot}\ F} en \isa{{\isasymA}\isactrlsub {\isadigit{2}}}, probando así el resultado.

    Consideremos las fórmulas \isa{F} y \isa{G} con las mismas hipótesis que
    la fórmula del caso anterior. Sean las interpretaciones \isa{{\isasymA}\isactrlsub {\isadigit{1}}} y \isa{{\isasymA}\isactrlsub {\isadigit{2}}} 
    tales que coinciden sobre el conjunto de átomos de \isa{F\ {\isasymand}\ G}. Vamos a
    probar que el valor de \isa{F\ {\isasymand}\ G} en \isa{{\isasymA}\isactrlsub {\isadigit{1}}} es el mismo que en \isa{{\isasymA}\isactrlsub {\isadigit{2}}}.
    Como el conjunto de átomos de \isa{F\ {\isasymand}\ G} es la unión del conjunto de
    átomos de \isa{F} y el conjunto de átomos de \isa{G}, tenemos que \isa{{\isasymA}\isactrlsub {\isadigit{1}}} y 
    \isa{{\isasymA}\isactrlsub {\isadigit{2}}} coinciden sobre los elementos de dicha unión. En particular,
    coinciden sobre los elementos del conjunto de átomos de \isa{F} y, por
    hipótesis de inducción, tenemos que el valor de \isa{F} en \isa{{\isasymA}\isactrlsub {\isadigit{1}}} 
    coincide con su valor en \isa{{\isasymA}\isactrlsub {\isadigit{2}}}. Del mismo modo, las
    interpretaciones anteriores coinciden también sobre los elementos
    del conjunto de átomos de \isa{G} luego, aplicando análogamente la 
    hipótesis de inducción, tenemos que el valor de \isa{G} es el mismo 
    en las interpretaciones \isa{{\isasymA}\isactrlsub {\isadigit{1}}} y \isa{{\isasymA}\isactrlsub {\isadigit{2}}}. Veamos ahora que el valor
    de \isa{F\ {\isasymand}\ G} también coincide en dichas interpretaciones.
    Por definición, el valor de \isa{F\ {\isasymand}\ G} en \isa{{\isasymA}\isactrlsub {\isadigit{1}}} es la conjunción
    del valor de \isa{F} en \isa{{\isasymA}\isactrlsub {\isadigit{1}}} y el valor de \isa{G} en \isa{{\isasymA}\isactrlsub {\isadigit{1}}}. Por lo 
    obtenido anteriormente por las hipótesis de inducción, tenemos que
    esto es igual a la conjunción del valor de \isa{F} en \isa{{\isasymA}\isactrlsub {\isadigit{2}}} y el
    valor de \isa{G} en \isa{{\isasymA}\isactrlsub {\isadigit{2}}}. Por último se tiene que esto es igual al
    valor de \isa{F\ {\isasymand}\ G} en \isa{{\isasymA}\isactrlsub {\isadigit{2}}} tras aplicar la definición.

    Volvamos a considerar \isa{F} y \isa{G} en las condiciones anteriores y
    dos interpretaciones \isa{{\isasymA}\isactrlsub {\isadigit{1}}} y \isa{{\isasymA}\isactrlsub {\isadigit{2}}} que coinciden sobre el
    conjunto de átomos de \isa{F\ {\isasymor}\ G}. Vamos a probar que el valor de 
    dicha fórmula es el mismo en ambas interpretaciones.
    De manera análoga al caso anterior, como el conjunto de átomos de
    \isa{F\ {\isasymor}\ G} es la unión del conjunto de átomos de \isa{F} y el conjunto de
    átomos de \isa{G}, tenemos que las interpretaciones coinciden sobre los
    elementos de esta unión. En particular, coinciden sobre el conjunto
    de átomos de \isa{F}. Por tanto, por hipótesis de inducción, el valor
    de \isa{F} en \isa{{\isasymA}\isactrlsub {\isadigit{1}}} coincide con su valor en \isa{{\isasymA}\isactrlsub {\isadigit{2}}}. Igualmente 
    obtenemos que las interpretaciones coinciden sobre el conjunto de
    átomos de \isa{G} y, aplicando de nuevo hipótesis de inducción, el 
    valor de \isa{G} es el mismo en ambas interpretaciones. 
    Por otra parte, por definición tenemos que le valor de \isa{F\ {\isasymor}\ G} en
    la interpretación \isa{{\isasymA}\isactrlsub {\isadigit{1}}} es la disyunción entre el valor de \isa{F} en
    \isa{{\isasymA}\isactrlsub {\isadigit{1}}} y el valor de \isa{G} en \isa{{\isasymA}\isactrlsub {\isadigit{1}}}. Por las observaciones
    anteriores derivadas de las hipótesis de inducción, tenemos que
    esto es igual a la disyunción entre el valor de \isa{F} en \isa{{\isasymA}\isactrlsub {\isadigit{2}}} y
    el valor de \isa{G} en \isa{{\isasymA}\isactrlsub {\isadigit{2}}}. Por definición, esto es el valor de 
    \isa{F\ {\isasymor}\ G} en \isa{{\isasymA}\isactrlsub {\isadigit{2}}}, como queríamos demostrar.

    Veamos el último caso de las fórmulas. Sean \isa{F} y \isa{G} fórmulas en 
    las condiciones de los casos anteriores. Consideremos las
    interpretaciones \isa{{\isasymA}\isactrlsub {\isadigit{1}}} y \isa{{\isasymA}\isactrlsub {\isadigit{2}}} que coinciden sobre los elementos
    del conjunto de átomos de \isa{F\ {\isasymrightarrow}\ G}. Probemos que el valor de 
    \isa{F\ {\isasymrightarrow}\ G} es el mismo en ambas interpretaciones. Por definición, 
    el conjunto de átomos de\\ \isa{F\ {\isasymrightarrow}\ G} es la unión de los
    conjuntos de átomos de \isa{F} y \isa{G}. Por tanto, dichas
    interpretaciones coinciden sobre los elementos de dicha unión. 
    Como hemos visto en casos anteriores, en particular coinciden sobre
    los átomos de \isa{F} luego, por hipótesis de inducción, el valor de 
    \isa{F} en \isa{{\isasymA}\isactrlsub {\isadigit{1}}} coincide con su valor en \isa{{\isasymA}\isactrlsub {\isadigit{2}}}. Análogamente, las
    interpretaciones coinciden sobre los átomos de \isa{G} y, por hipótesis
    de inducción, el valor de \isa{G} es el mismo en ambas
    interpretaciones. Probemos que también coincide el valor de \isa{F\ {\isasymrightarrow}\ G}
    en \isa{{\isasymA}\isactrlsub {\isadigit{1}}} y \isa{{\isasymA}\isactrlsub {\isadigit{2}}}.
    Por definición, el valor de \isa{F\ {\isasymrightarrow}\ G} en \isa{{\isasymA}\isactrlsub {\isadigit{1}}} es la implicación
    entre el valor de \isa{F} en \isa{{\isasymA}\isactrlsub {\isadigit{1}}} y el valor de \isa{G} en \isa{{\isasymA}\isactrlsub {\isadigit{1}}}. De
    esta manera, por las observaciones anteriores tenemos que esto es
    igual a la implicación entre el valor de \isa{F} en \isa{{\isasymA}\isactrlsub {\isadigit{2}}} y el valor
    de \isa{G} en \isa{{\isasymA}\isactrlsub {\isadigit{2}}}. Finalmente, por definición, esto es el valor de
    \isa{F\ {\isasymrightarrow}\ G} en la interpretación \isa{{\isasymA}\isactrlsub {\isadigit{2}}}, probando así el resultado.    
  \end{demostracion}

  Probemos ahora el lema de forma detallada en Isabelle, haciendo cada
  caso de la estructura de las fórmulas por separado y empleando lemas
  auxiliares sobre la pertenencia a los conjuntos de átomos cuando sea 
  necesario.%
\end{isamarkuptext}\isamarkuptrue%
\isacommand{lemma}\isamarkupfalse%
\ relevant{\isacharunderscore}atoms{\isacharunderscore}same{\isacharunderscore}semantics{\isacharunderscore}atomic{\isacharunderscore}l{\isadigit{1}}{\isacharcolon}\isanewline
\ \ {\isachardoublequoteopen}x\ {\isasymin}\ atoms\ {\isacharparenleft}Atom\ x{\isacharparenright}{\isachardoublequoteclose}\isanewline
%
\isadelimproof
%
\endisadelimproof
%
\isatagproof
\isacommand{proof}\isamarkupfalse%
\ \isanewline
\ \ \isacommand{show}\isamarkupfalse%
\ {\isachardoublequoteopen}x\ {\isasymin}\ {\isacharbraceleft}x{\isacharbraceright}{\isachardoublequoteclose}\isanewline
\ \ \ \ \isacommand{by}\isamarkupfalse%
\ {\isacharparenleft}simp\ only{\isacharcolon}\ singleton{\isacharunderscore}iff{\isacharparenright}\isanewline
\isacommand{next}\isamarkupfalse%
\isanewline
\ \ \isacommand{show}\isamarkupfalse%
\ {\isachardoublequoteopen}{\isacharbraceleft}x{\isacharbraceright}\ {\isasymsubseteq}\ atoms\ {\isacharparenleft}Atom\ x{\isacharparenright}{\isachardoublequoteclose}\isanewline
\ \ \ \ \isacommand{by}\isamarkupfalse%
\ {\isacharparenleft}simp\ only{\isacharcolon}\ formula{\isachardot}set{\isacharparenleft}{\isadigit{1}}{\isacharparenright}{\isacharparenright}\isanewline
\isacommand{qed}\isamarkupfalse%
%
\endisatagproof
{\isafoldproof}%
%
\isadelimproof
\isanewline
%
\endisadelimproof
\isanewline
\isacommand{lemma}\isamarkupfalse%
\ relevant{\isacharunderscore}atoms{\isacharunderscore}same{\isacharunderscore}semantics{\isacharunderscore}atomic{\isacharcolon}\ \isanewline
\ \ \isakeyword{assumes}\ {\isachardoublequoteopen}{\isasymforall}k\ {\isasymin}\ atoms\ {\isacharparenleft}Atom\ x{\isacharparenright}{\isachardot}\ {\isasymA}\isactrlsub {\isadigit{1}}\ k\ {\isacharequal}\ {\isasymA}\isactrlsub {\isadigit{2}}\ k{\isachardoublequoteclose}\isanewline
\ \ \isakeyword{shows}\ \ \ {\isachardoublequoteopen}{\isasymA}\isactrlsub {\isadigit{1}}\ {\isasymTurnstile}\ Atom\ x\ {\isasymlongleftrightarrow}\ {\isasymA}\isactrlsub {\isadigit{2}}\ {\isasymTurnstile}\ Atom\ x{\isachardoublequoteclose}\isanewline
%
\isadelimproof
%
\endisadelimproof
%
\isatagproof
\isacommand{proof}\isamarkupfalse%
\ {\isacharminus}\isanewline
\ \ \isacommand{have}\isamarkupfalse%
\ {\isachardoublequoteopen}{\isasymA}\isactrlsub {\isadigit{1}}\ {\isasymTurnstile}\ Atom\ x\ {\isacharequal}\ {\isasymA}\isactrlsub {\isadigit{1}}\ x{\isachardoublequoteclose}\isanewline
\ \ \ \ \isacommand{by}\isamarkupfalse%
\ {\isacharparenleft}simp\ only{\isacharcolon}\ formula{\isacharunderscore}semantics{\isachardot}simps{\isacharparenleft}{\isadigit{1}}{\isacharparenright}{\isacharparenright}\isanewline
\ \ \isacommand{also}\isamarkupfalse%
\ \isacommand{have}\isamarkupfalse%
\ {\isachardoublequoteopen}{\isasymdots}\ {\isacharequal}\ {\isasymA}\isactrlsub {\isadigit{2}}\ x{\isachardoublequoteclose}\isanewline
\ \ \ \ \isacommand{using}\isamarkupfalse%
\ \ assms{\isacharparenleft}{\isadigit{1}}{\isacharparenright}\isanewline
\ \ \ \ \isacommand{by}\isamarkupfalse%
\ {\isacharparenleft}simp\ only{\isacharcolon}\ relevant{\isacharunderscore}atoms{\isacharunderscore}same{\isacharunderscore}semantics{\isacharunderscore}atomic{\isacharunderscore}l{\isadigit{1}}{\isacharparenright}\isanewline
\ \ \isacommand{also}\isamarkupfalse%
\ \isacommand{have}\isamarkupfalse%
\ {\isachardoublequoteopen}{\isasymdots}\ {\isacharequal}\ {\isasymA}\isactrlsub {\isadigit{2}}\ {\isasymTurnstile}\ Atom\ x{\isachardoublequoteclose}\isanewline
\ \ \ \ \isacommand{by}\isamarkupfalse%
\ {\isacharparenleft}simp\ only{\isacharcolon}\ formula{\isacharunderscore}semantics{\isachardot}simps{\isacharparenleft}{\isadigit{1}}{\isacharparenright}{\isacharparenright}\isanewline
\ \ \isacommand{finally}\isamarkupfalse%
\ \isacommand{show}\isamarkupfalse%
\ {\isacharquery}thesis\isanewline
\ \ \ \ \isacommand{by}\isamarkupfalse%
\ this\isanewline
\isacommand{qed}\isamarkupfalse%
%
\endisatagproof
{\isafoldproof}%
%
\isadelimproof
%
\endisadelimproof
%
\begin{isamarkuptext}%
Para las fórmulas atómicas, se observa el uso del lema 
  auxiliar\\ \isa{relevant{\isacharunderscore}atoms{\isacharunderscore}same{\isacharunderscore}semantics{\isacharunderscore}atomic{\isacharunderscore}l}. Sigamos con los
  siguientes casos.%
\end{isamarkuptext}\isamarkuptrue%
\isacommand{lemma}\isamarkupfalse%
\ relevant{\isacharunderscore}atoms{\isacharunderscore}same{\isacharunderscore}semantics{\isacharunderscore}bot{\isacharcolon}\ \isanewline
\ \ \isakeyword{assumes}\ {\isachardoublequoteopen}{\isasymforall}k\ {\isasymin}\ atoms\ {\isasymbottom}{\isachardot}\ {\isasymA}\isactrlsub {\isadigit{1}}\ k\ {\isacharequal}\ {\isasymA}\isactrlsub {\isadigit{2}}\ k{\isachardoublequoteclose}\isanewline
\ \ \isakeyword{shows}\ {\isachardoublequoteopen}{\isasymA}\isactrlsub {\isadigit{1}}\ {\isasymTurnstile}\ {\isasymbottom}\ {\isasymlongleftrightarrow}\ {\isasymA}\isactrlsub {\isadigit{2}}\ {\isasymTurnstile}\ {\isasymbottom}{\isachardoublequoteclose}\isanewline
%
\isadelimproof
\ \ %
\endisadelimproof
%
\isatagproof
\isacommand{by}\isamarkupfalse%
\ {\isacharparenleft}simp\ only{\isacharcolon}\ formula{\isacharunderscore}semantics{\isachardot}simps{\isacharparenleft}{\isadigit{2}}{\isacharparenright}{\isacharparenright}%
\endisatagproof
{\isafoldproof}%
%
\isadelimproof
\isanewline
%
\endisadelimproof
\isanewline
\isacommand{lemma}\isamarkupfalse%
\ relevant{\isacharunderscore}atoms{\isacharunderscore}same{\isacharunderscore}semantics{\isacharunderscore}not{\isacharcolon}\ \isanewline
\ \ \isakeyword{assumes}\ {\isachardoublequoteopen}{\isasymforall}k\ {\isasymin}\ atoms\ F{\isachardot}\ {\isasymA}\isactrlsub {\isadigit{1}}\ k\ {\isacharequal}\ {\isasymA}\isactrlsub {\isadigit{2}}\ k\ {\isasymLongrightarrow}\ {\isasymA}\isactrlsub {\isadigit{1}}\ {\isasymTurnstile}\ F\ {\isasymlongleftrightarrow}\ {\isasymA}\isactrlsub {\isadigit{2}}\ {\isasymTurnstile}\ F{\isachardoublequoteclose}\isanewline
\ \ \ \ \ \ \ \ \ \ {\isachardoublequoteopen}{\isasymforall}k\ {\isasymin}\ atoms\ {\isacharparenleft}\isactrlbold {\isasymnot}\ F{\isacharparenright}{\isachardot}\ {\isasymA}\isactrlsub {\isadigit{1}}\ k\ {\isacharequal}\ {\isasymA}\isactrlsub {\isadigit{2}}\ k{\isachardoublequoteclose}\isanewline
\ \ \ \ \ \ \ \ \isakeyword{shows}\ {\isachardoublequoteopen}{\isasymA}\isactrlsub {\isadigit{1}}\ {\isasymTurnstile}\ {\isacharparenleft}\isactrlbold {\isasymnot}\ F{\isacharparenright}\ {\isasymlongleftrightarrow}\ {\isasymA}\isactrlsub {\isadigit{2}}\ {\isasymTurnstile}\ {\isacharparenleft}\isactrlbold {\isasymnot}\ F{\isacharparenright}{\isachardoublequoteclose}\isanewline
%
\isadelimproof
%
\endisadelimproof
%
\isatagproof
\isacommand{proof}\isamarkupfalse%
\ {\isacharminus}\isanewline
\ \ \isacommand{have}\isamarkupfalse%
\ {\isachardoublequoteopen}{\isasymforall}k\ {\isasymin}\ atoms\ F{\isachardot}\ {\isasymA}\isactrlsub {\isadigit{1}}\ k\ {\isacharequal}\ {\isasymA}\isactrlsub {\isadigit{2}}\ k{\isachardoublequoteclose}\ \isacommand{using}\isamarkupfalse%
\ assms{\isacharparenleft}{\isadigit{2}}{\isacharparenright}\isanewline
\ \ \ \ \isacommand{by}\isamarkupfalse%
\ {\isacharparenleft}simp\ only{\isacharcolon}\ formula{\isachardot}set{\isacharparenleft}{\isadigit{3}}{\isacharparenright}{\isacharparenright}\isanewline
\ \ \isacommand{then}\isamarkupfalse%
\ \isacommand{have}\isamarkupfalse%
\ H{\isacharcolon}{\isachardoublequoteopen}{\isasymA}\isactrlsub {\isadigit{1}}\ {\isasymTurnstile}\ F\ {\isasymlongleftrightarrow}\ {\isasymA}\isactrlsub {\isadigit{2}}\ {\isasymTurnstile}\ F{\isachardoublequoteclose}\isanewline
\ \ \ \ \isacommand{by}\isamarkupfalse%
\ {\isacharparenleft}simp\ only{\isacharcolon}\ assms{\isacharparenleft}{\isadigit{1}}{\isacharparenright}{\isacharparenright}\isanewline
\ \ \isacommand{have}\isamarkupfalse%
\ {\isachardoublequoteopen}{\isasymA}\isactrlsub {\isadigit{1}}\ {\isasymTurnstile}\ {\isacharparenleft}\isactrlbold {\isasymnot}\ F{\isacharparenright}\ {\isacharequal}\ {\isacharparenleft}{\isasymnot}\ {\isasymA}\isactrlsub {\isadigit{1}}\ {\isasymTurnstile}\ F{\isacharparenright}{\isachardoublequoteclose}\isanewline
\ \ \ \ \isacommand{by}\isamarkupfalse%
\ {\isacharparenleft}simp\ only{\isacharcolon}\ formula{\isacharunderscore}semantics{\isachardot}simps{\isacharparenleft}{\isadigit{3}}{\isacharparenright}{\isacharparenright}\isanewline
\ \ \isacommand{also}\isamarkupfalse%
\ \isacommand{have}\isamarkupfalse%
\ {\isachardoublequoteopen}{\isasymdots}\ {\isacharequal}\ {\isacharparenleft}{\isasymnot}\ {\isasymA}\isactrlsub {\isadigit{2}}\ {\isasymTurnstile}\ F{\isacharparenright}{\isachardoublequoteclose}\isanewline
\ \ \ \ \isacommand{using}\isamarkupfalse%
\ H\ \isacommand{by}\isamarkupfalse%
\ {\isacharparenleft}rule\ arg{\isacharunderscore}cong{\isacharparenright}\isanewline
\ \ \isacommand{also}\isamarkupfalse%
\ \isacommand{have}\isamarkupfalse%
\ {\isachardoublequoteopen}{\isasymdots}\ {\isacharequal}\ {\isasymA}\isactrlsub {\isadigit{2}}\ {\isasymTurnstile}\ {\isacharparenleft}\isactrlbold {\isasymnot}\ F{\isacharparenright}{\isachardoublequoteclose}\isanewline
\ \ \ \ \isacommand{by}\isamarkupfalse%
\ {\isacharparenleft}simp\ only{\isacharcolon}\ formula{\isacharunderscore}semantics{\isachardot}simps{\isacharparenleft}{\isadigit{3}}{\isacharparenright}{\isacharparenright}\isanewline
\ \ \isacommand{finally}\isamarkupfalse%
\ \isacommand{show}\isamarkupfalse%
\ {\isacharquery}thesis\isanewline
\ \ \ \ \isacommand{by}\isamarkupfalse%
\ this\isanewline
\isacommand{qed}\isamarkupfalse%
%
\endisatagproof
{\isafoldproof}%
%
\isadelimproof
%
\endisadelimproof
%
\begin{isamarkuptext}%
Para los casos de la fórmula \isa{{\isasymbottom}} y la negación \isa{{\isasymnot}\ F} no han sido
  necesarios los lemas auxiliares pues, en el primer caso, el conjunto
  de átomos es el vacío y, en el segundo caso, el conjunto de átomos de
  \isa{{\isasymnot}\ F} coincide con el de \isa{F}. Finalmente, introducimos los siguientes 
  lemas auxiliares para facilitar las demostraciones detalladas en 
  Isabelle de los casos correspondientes a las conectivas binarias.%
\end{isamarkuptext}\isamarkuptrue%
\isacommand{lemma}\isamarkupfalse%
\ forall{\isacharunderscore}union{\isadigit{1}}{\isacharcolon}\ \isanewline
\ \ \isakeyword{assumes}\ {\isachardoublequoteopen}{\isasymforall}x\ {\isasymin}\ A\ {\isasymunion}\ B{\isachardot}\ P\ x{\isachardoublequoteclose}\isanewline
\ \ \isakeyword{shows}\ {\isachardoublequoteopen}{\isasymforall}x\ {\isasymin}\ A{\isachardot}\ P\ x{\isachardoublequoteclose}\isanewline
%
\isadelimproof
%
\endisadelimproof
%
\isatagproof
\isacommand{proof}\isamarkupfalse%
\ {\isacharparenleft}rule\ ballI{\isacharparenright}\isanewline
\ \ \isacommand{fix}\isamarkupfalse%
\ x\isanewline
\ \ \isacommand{assume}\isamarkupfalse%
\ {\isachardoublequoteopen}x\ {\isasymin}\ A{\isachardoublequoteclose}\isanewline
\ \ \isacommand{then}\isamarkupfalse%
\ \isacommand{have}\isamarkupfalse%
\ {\isachardoublequoteopen}x\ {\isasymin}\ A\ {\isasymunion}\ B{\isachardoublequoteclose}\ \isanewline
\ \ \ \ \isacommand{by}\isamarkupfalse%
\ {\isacharparenleft}simp\ only{\isacharcolon}\ UnI{\isadigit{1}}{\isacharparenright}\isanewline
\ \ \isacommand{then}\isamarkupfalse%
\ \isacommand{show}\isamarkupfalse%
\ {\isachardoublequoteopen}P\ x{\isachardoublequoteclose}\ \isanewline
\ \ \ \ \isacommand{by}\isamarkupfalse%
\ {\isacharparenleft}simp\ only{\isacharcolon}\ assms{\isacharparenright}\isanewline
\isacommand{qed}\isamarkupfalse%
%
\endisatagproof
{\isafoldproof}%
%
\isadelimproof
\isanewline
%
\endisadelimproof
\isanewline
\isacommand{lemma}\isamarkupfalse%
\ forall{\isacharunderscore}union{\isadigit{2}}{\isacharcolon}\isanewline
\ \ \isakeyword{assumes}\ {\isachardoublequoteopen}{\isasymforall}x\ {\isasymin}\ A\ {\isasymunion}\ B{\isachardot}\ P\ x{\isachardoublequoteclose}\isanewline
\ \ \isakeyword{shows}\ {\isachardoublequoteopen}{\isasymforall}x\ {\isasymin}\ B{\isachardot}\ P\ x{\isachardoublequoteclose}\isanewline
%
\isadelimproof
%
\endisadelimproof
%
\isatagproof
\isacommand{proof}\isamarkupfalse%
\ {\isacharparenleft}rule\ ballI{\isacharparenright}\isanewline
\ \ \isacommand{fix}\isamarkupfalse%
\ x\isanewline
\ \ \isacommand{assume}\isamarkupfalse%
\ {\isachardoublequoteopen}x\ {\isasymin}\ B{\isachardoublequoteclose}\isanewline
\ \ \isacommand{then}\isamarkupfalse%
\ \isacommand{have}\isamarkupfalse%
\ {\isachardoublequoteopen}x\ {\isasymin}\ A\ {\isasymunion}\ B{\isachardoublequoteclose}\ \isanewline
\ \ \ \ \isacommand{by}\isamarkupfalse%
\ {\isacharparenleft}simp\ only{\isacharcolon}\ UnI{\isadigit{2}}{\isacharparenright}\isanewline
\ \ \isacommand{then}\isamarkupfalse%
\ \isacommand{show}\isamarkupfalse%
\ {\isachardoublequoteopen}P\ x{\isachardoublequoteclose}\ \isanewline
\ \ \ \ \isacommand{by}\isamarkupfalse%
\ {\isacharparenleft}simp\ only{\isacharcolon}\ assms{\isacharparenright}\isanewline
\isacommand{qed}\isamarkupfalse%
%
\endisatagproof
{\isafoldproof}%
%
\isadelimproof
%
\endisadelimproof
%
\begin{isamarkuptext}%
Empleando dichos resultados, veamos las demostraciones detalladas
  de los tres últimos casos. Después se mostrará la demostración 
  detallada del lema completo en Isabelle.%
\end{isamarkuptext}\isamarkuptrue%
\isacommand{lemma}\isamarkupfalse%
\ relevant{\isacharunderscore}atoms{\isacharunderscore}same{\isacharunderscore}semantics{\isacharunderscore}and{\isacharcolon}\ \isanewline
\ \ \isakeyword{assumes}\ {\isachardoublequoteopen}{\isasymforall}k\ {\isasymin}\ atoms\ F{\isachardot}\ {\isasymA}\isactrlsub {\isadigit{1}}\ k\ {\isacharequal}\ {\isasymA}\isactrlsub {\isadigit{2}}\ k\ {\isasymLongrightarrow}\ {\isasymA}\isactrlsub {\isadigit{1}}\ {\isasymTurnstile}\ F\ {\isasymlongleftrightarrow}\ {\isasymA}\isactrlsub {\isadigit{2}}\ {\isasymTurnstile}\ F{\isachardoublequoteclose}\isanewline
\ \ \ \ \ \ \ \ \ \ {\isachardoublequoteopen}{\isasymforall}k\ {\isasymin}\ atoms\ G{\isachardot}\ {\isasymA}\isactrlsub {\isadigit{1}}\ k\ {\isacharequal}\ {\isasymA}\isactrlsub {\isadigit{2}}\ k\ {\isasymLongrightarrow}\ {\isasymA}\isactrlsub {\isadigit{1}}\ {\isasymTurnstile}\ G\ {\isasymlongleftrightarrow}\ {\isasymA}\isactrlsub {\isadigit{2}}\ {\isasymTurnstile}\ G{\isachardoublequoteclose}\isanewline
\ \ \ \ \ \ \ \ \ \ {\isachardoublequoteopen}{\isasymforall}k\ {\isasymin}\ atoms\ {\isacharparenleft}F\ \isactrlbold {\isasymand}\ G{\isacharparenright}{\isachardot}\ {\isasymA}\isactrlsub {\isadigit{1}}\ k\ {\isacharequal}\ {\isasymA}\isactrlsub {\isadigit{2}}\ k{\isachardoublequoteclose}\isanewline
\ \ \ \ \ \ \ \ \isakeyword{shows}\ {\isachardoublequoteopen}{\isasymA}\isactrlsub {\isadigit{1}}\ {\isasymTurnstile}\ {\isacharparenleft}F\ \isactrlbold {\isasymand}\ G{\isacharparenright}\ {\isasymlongleftrightarrow}\ {\isasymA}\isactrlsub {\isadigit{2}}\ {\isasymTurnstile}\ {\isacharparenleft}F\ \isactrlbold {\isasymand}\ G{\isacharparenright}{\isachardoublequoteclose}\isanewline
%
\isadelimproof
%
\endisadelimproof
%
\isatagproof
\isacommand{proof}\isamarkupfalse%
\ {\isacharminus}\isanewline
\ \ \isacommand{have}\isamarkupfalse%
\ H{\isacharcolon}{\isachardoublequoteopen}{\isasymforall}k\ {\isasymin}\ atoms\ F\ {\isasymunion}\ atoms\ G{\isachardot}\ {\isasymA}\isactrlsub {\isadigit{1}}\ k\ {\isacharequal}\ {\isasymA}\isactrlsub {\isadigit{2}}\ k{\isachardoublequoteclose}\ \isacommand{using}\isamarkupfalse%
\ assms{\isacharparenleft}{\isadigit{3}}{\isacharparenright}\isanewline
\ \ \ \ \isacommand{by}\isamarkupfalse%
\ {\isacharparenleft}simp\ only{\isacharcolon}\ formula{\isachardot}set{\isacharparenleft}{\isadigit{4}}{\isacharparenright}{\isacharparenright}\isanewline
\ \ \isacommand{then}\isamarkupfalse%
\ \isacommand{have}\isamarkupfalse%
\ {\isachardoublequoteopen}{\isasymforall}k\ {\isasymin}\ atoms\ F{\isachardot}\ {\isasymA}\isactrlsub {\isadigit{1}}\ k\ {\isacharequal}\ {\isasymA}\isactrlsub {\isadigit{2}}\ k{\isachardoublequoteclose}\isanewline
\ \ \ \ \isacommand{by}\isamarkupfalse%
\ {\isacharparenleft}rule\ forall{\isacharunderscore}union{\isadigit{1}}{\isacharparenright}\isanewline
\ \ \isacommand{then}\isamarkupfalse%
\ \isacommand{have}\isamarkupfalse%
\ H{\isadigit{1}}{\isacharcolon}{\isachardoublequoteopen}{\isasymA}\isactrlsub {\isadigit{1}}\ {\isasymTurnstile}\ F\ {\isasymlongleftrightarrow}\ {\isasymA}\isactrlsub {\isadigit{2}}\ {\isasymTurnstile}\ F{\isachardoublequoteclose}\isanewline
\ \ \ \ \isacommand{by}\isamarkupfalse%
\ {\isacharparenleft}simp\ only{\isacharcolon}\ assms{\isacharparenleft}{\isadigit{1}}{\isacharparenright}{\isacharparenright}\isanewline
\ \ \isacommand{have}\isamarkupfalse%
\ {\isachardoublequoteopen}{\isasymforall}k\ {\isasymin}\ atoms\ G{\isachardot}\ {\isasymA}\isactrlsub {\isadigit{1}}\ k\ {\isacharequal}\ {\isasymA}\isactrlsub {\isadigit{2}}\ k{\isachardoublequoteclose}\isanewline
\ \ \ \ \isacommand{using}\isamarkupfalse%
\ H\ \isacommand{by}\isamarkupfalse%
\ {\isacharparenleft}rule\ forall{\isacharunderscore}union{\isadigit{2}}{\isacharparenright}\isanewline
\ \ \isacommand{then}\isamarkupfalse%
\ \isacommand{have}\isamarkupfalse%
\ H{\isadigit{2}}{\isacharcolon}{\isachardoublequoteopen}{\isasymA}\isactrlsub {\isadigit{1}}\ {\isasymTurnstile}\ G\ {\isasymlongleftrightarrow}\ {\isasymA}\isactrlsub {\isadigit{2}}\ {\isasymTurnstile}\ G{\isachardoublequoteclose}\isanewline
\ \ \ \ \isacommand{by}\isamarkupfalse%
\ {\isacharparenleft}simp\ only{\isacharcolon}\ assms{\isacharparenleft}{\isadigit{2}}{\isacharparenright}{\isacharparenright}\isanewline
\ \ \isacommand{have}\isamarkupfalse%
\ {\isachardoublequoteopen}{\isasymA}\isactrlsub {\isadigit{1}}\ {\isasymTurnstile}\ F\ \isactrlbold {\isasymand}\ G\ {\isacharequal}\ {\isacharparenleft}{\isasymA}\isactrlsub {\isadigit{1}}\ {\isasymTurnstile}\ F\ {\isasymand}\ {\isasymA}\isactrlsub {\isadigit{1}}\ {\isasymTurnstile}\ G{\isacharparenright}{\isachardoublequoteclose}\isanewline
\ \ \ \ \isacommand{by}\isamarkupfalse%
\ {\isacharparenleft}simp\ only{\isacharcolon}\ formula{\isacharunderscore}semantics{\isachardot}simps{\isacharparenleft}{\isadigit{4}}{\isacharparenright}{\isacharparenright}\isanewline
\ \ \isacommand{also}\isamarkupfalse%
\ \isacommand{have}\isamarkupfalse%
\ {\isachardoublequoteopen}{\isasymdots}\ {\isacharequal}\ {\isacharparenleft}{\isasymA}\isactrlsub {\isadigit{2}}\ {\isasymTurnstile}\ F\ {\isasymand}\ {\isasymA}\isactrlsub {\isadigit{2}}\ {\isasymTurnstile}\ G{\isacharparenright}{\isachardoublequoteclose}\isanewline
\ \ \ \ \isacommand{using}\isamarkupfalse%
\ H{\isadigit{1}}\ H{\isadigit{2}}\ \isacommand{by}\isamarkupfalse%
\ {\isacharparenleft}rule\ arg{\isacharunderscore}cong{\isadigit{2}}{\isacharparenright}\isanewline
\ \ \isacommand{also}\isamarkupfalse%
\ \isacommand{have}\isamarkupfalse%
\ {\isachardoublequoteopen}{\isasymdots}\ {\isacharequal}\ {\isasymA}\isactrlsub {\isadigit{2}}\ {\isasymTurnstile}\ F\ \isactrlbold {\isasymand}\ G{\isachardoublequoteclose}\isanewline
\ \ \ \ \isacommand{by}\isamarkupfalse%
\ {\isacharparenleft}simp\ only{\isacharcolon}\ formula{\isacharunderscore}semantics{\isachardot}simps{\isacharparenleft}{\isadigit{4}}{\isacharparenright}{\isacharparenright}\isanewline
\ \ \isacommand{finally}\isamarkupfalse%
\ \isacommand{show}\isamarkupfalse%
\ {\isacharquery}thesis\ \isanewline
\ \ \ \ \isacommand{by}\isamarkupfalse%
\ this\isanewline
\isacommand{qed}\isamarkupfalse%
%
\endisatagproof
{\isafoldproof}%
%
\isadelimproof
\isanewline
%
\endisadelimproof
\isanewline
\isacommand{lemma}\isamarkupfalse%
\ relevant{\isacharunderscore}atoms{\isacharunderscore}same{\isacharunderscore}semantics{\isacharunderscore}or{\isacharcolon}\ \isanewline
\ \ \isakeyword{assumes}\ {\isachardoublequoteopen}{\isasymforall}k\ {\isasymin}\ atoms\ F{\isachardot}\ {\isasymA}\isactrlsub {\isadigit{1}}\ k\ {\isacharequal}\ {\isasymA}\isactrlsub {\isadigit{2}}\ k\ {\isasymLongrightarrow}\ {\isasymA}\isactrlsub {\isadigit{1}}\ {\isasymTurnstile}\ F\ {\isasymlongleftrightarrow}\ {\isasymA}\isactrlsub {\isadigit{2}}\ {\isasymTurnstile}\ F{\isachardoublequoteclose}\isanewline
\ \ \ \ \ \ \ \ \ \ {\isachardoublequoteopen}{\isasymforall}k\ {\isasymin}\ atoms\ G{\isachardot}\ {\isasymA}\isactrlsub {\isadigit{1}}\ k\ {\isacharequal}\ {\isasymA}\isactrlsub {\isadigit{2}}\ k\ {\isasymLongrightarrow}\ {\isasymA}\isactrlsub {\isadigit{1}}\ {\isasymTurnstile}\ G\ {\isasymlongleftrightarrow}\ {\isasymA}\isactrlsub {\isadigit{2}}\ {\isasymTurnstile}\ G{\isachardoublequoteclose}\isanewline
\ \ \ \ \ \ \ \ \ \ {\isachardoublequoteopen}{\isasymforall}k\ {\isasymin}\ atoms\ {\isacharparenleft}F\ \isactrlbold {\isasymor}\ G{\isacharparenright}{\isachardot}\ {\isasymA}\isactrlsub {\isadigit{1}}\ k\ {\isacharequal}\ {\isasymA}\isactrlsub {\isadigit{2}}\ k{\isachardoublequoteclose}\isanewline
\ \ \ \ \ \isakeyword{shows}\ {\isachardoublequoteopen}{\isasymA}\isactrlsub {\isadigit{1}}\ {\isasymTurnstile}\ {\isacharparenleft}F\ \isactrlbold {\isasymor}\ G{\isacharparenright}\ {\isasymlongleftrightarrow}\ {\isasymA}\isactrlsub {\isadigit{2}}\ {\isasymTurnstile}\ {\isacharparenleft}F\ \isactrlbold {\isasymor}\ G{\isacharparenright}{\isachardoublequoteclose}\isanewline
%
\isadelimproof
%
\endisadelimproof
%
\isatagproof
\isacommand{proof}\isamarkupfalse%
\ {\isacharminus}\isanewline
\ \ \isacommand{have}\isamarkupfalse%
\ H{\isacharcolon}{\isachardoublequoteopen}{\isasymforall}k\ {\isasymin}\ atoms\ F\ {\isasymunion}\ atoms\ G{\isachardot}\ {\isasymA}\isactrlsub {\isadigit{1}}\ k\ {\isacharequal}\ {\isasymA}\isactrlsub {\isadigit{2}}\ k{\isachardoublequoteclose}\ \isacommand{using}\isamarkupfalse%
\ assms{\isacharparenleft}{\isadigit{3}}{\isacharparenright}\isanewline
\ \ \ \ \isacommand{by}\isamarkupfalse%
\ {\isacharparenleft}simp\ only{\isacharcolon}\ formula{\isachardot}set{\isacharparenleft}{\isadigit{5}}{\isacharparenright}{\isacharparenright}\isanewline
\ \ \isacommand{then}\isamarkupfalse%
\ \isacommand{have}\isamarkupfalse%
\ {\isachardoublequoteopen}{\isasymforall}k\ {\isasymin}\ atoms\ F{\isachardot}\ {\isasymA}\isactrlsub {\isadigit{1}}\ k\ {\isacharequal}\ {\isasymA}\isactrlsub {\isadigit{2}}\ k{\isachardoublequoteclose}\isanewline
\ \ \ \ \isacommand{by}\isamarkupfalse%
\ {\isacharparenleft}rule\ forall{\isacharunderscore}union{\isadigit{1}}{\isacharparenright}\isanewline
\ \ \isacommand{then}\isamarkupfalse%
\ \isacommand{have}\isamarkupfalse%
\ H{\isadigit{1}}{\isacharcolon}{\isachardoublequoteopen}{\isasymA}\isactrlsub {\isadigit{1}}\ {\isasymTurnstile}\ F\ {\isasymlongleftrightarrow}\ {\isasymA}\isactrlsub {\isadigit{2}}\ {\isasymTurnstile}\ F{\isachardoublequoteclose}\isanewline
\ \ \ \ \isacommand{by}\isamarkupfalse%
\ {\isacharparenleft}simp\ only{\isacharcolon}\ assms{\isacharparenleft}{\isadigit{1}}{\isacharparenright}{\isacharparenright}\isanewline
\ \ \isacommand{have}\isamarkupfalse%
\ {\isachardoublequoteopen}{\isasymforall}k\ {\isasymin}\ atoms\ G{\isachardot}\ {\isasymA}\isactrlsub {\isadigit{1}}\ k\ {\isacharequal}\ {\isasymA}\isactrlsub {\isadigit{2}}\ k{\isachardoublequoteclose}\isanewline
\ \ \ \ \isacommand{using}\isamarkupfalse%
\ H\ \isacommand{by}\isamarkupfalse%
\ {\isacharparenleft}rule\ forall{\isacharunderscore}union{\isadigit{2}}{\isacharparenright}\isanewline
\ \ \isacommand{then}\isamarkupfalse%
\ \isacommand{have}\isamarkupfalse%
\ H{\isadigit{2}}{\isacharcolon}{\isachardoublequoteopen}{\isasymA}\isactrlsub {\isadigit{1}}\ {\isasymTurnstile}\ G\ {\isasymlongleftrightarrow}\ {\isasymA}\isactrlsub {\isadigit{2}}\ {\isasymTurnstile}\ G{\isachardoublequoteclose}\isanewline
\ \ \ \ \isacommand{by}\isamarkupfalse%
\ {\isacharparenleft}simp\ only{\isacharcolon}\ assms{\isacharparenleft}{\isadigit{2}}{\isacharparenright}{\isacharparenright}\isanewline
\ \ \isacommand{have}\isamarkupfalse%
\ {\isachardoublequoteopen}{\isasymA}\isactrlsub {\isadigit{1}}\ {\isasymTurnstile}\ F\ \isactrlbold {\isasymor}\ G\ {\isacharequal}\ {\isacharparenleft}{\isasymA}\isactrlsub {\isadigit{1}}\ {\isasymTurnstile}\ F\ {\isasymor}\ {\isasymA}\isactrlsub {\isadigit{1}}\ {\isasymTurnstile}\ G{\isacharparenright}{\isachardoublequoteclose}\isanewline
\ \ \ \ \isacommand{by}\isamarkupfalse%
\ {\isacharparenleft}simp\ only{\isacharcolon}\ formula{\isacharunderscore}semantics{\isachardot}simps{\isacharparenleft}{\isadigit{5}}{\isacharparenright}{\isacharparenright}\isanewline
\ \ \isacommand{also}\isamarkupfalse%
\ \isacommand{have}\isamarkupfalse%
\ {\isachardoublequoteopen}{\isasymdots}\ {\isacharequal}\ {\isacharparenleft}{\isasymA}\isactrlsub {\isadigit{2}}\ {\isasymTurnstile}\ F\ {\isasymor}\ {\isasymA}\isactrlsub {\isadigit{2}}\ {\isasymTurnstile}\ G{\isacharparenright}{\isachardoublequoteclose}\isanewline
\ \ \ \ \isacommand{using}\isamarkupfalse%
\ H{\isadigit{1}}\ H{\isadigit{2}}\ \isacommand{by}\isamarkupfalse%
\ {\isacharparenleft}rule\ arg{\isacharunderscore}cong{\isadigit{2}}{\isacharparenright}\isanewline
\ \ \isacommand{also}\isamarkupfalse%
\ \isacommand{have}\isamarkupfalse%
\ {\isachardoublequoteopen}{\isasymdots}\ {\isacharequal}\ {\isasymA}\isactrlsub {\isadigit{2}}\ {\isasymTurnstile}\ F\ \isactrlbold {\isasymor}\ G{\isachardoublequoteclose}\isanewline
\ \ \ \ \isacommand{by}\isamarkupfalse%
\ {\isacharparenleft}simp\ only{\isacharcolon}\ formula{\isacharunderscore}semantics{\isachardot}simps{\isacharparenleft}{\isadigit{5}}{\isacharparenright}{\isacharparenright}\isanewline
\ \ \isacommand{finally}\isamarkupfalse%
\ \isacommand{show}\isamarkupfalse%
\ {\isacharquery}thesis\ \isanewline
\ \ \ \ \isacommand{by}\isamarkupfalse%
\ this\isanewline
\isacommand{qed}\isamarkupfalse%
%
\endisatagproof
{\isafoldproof}%
%
\isadelimproof
\isanewline
%
\endisadelimproof
\isanewline
\isacommand{lemma}\isamarkupfalse%
\ relevant{\isacharunderscore}atoms{\isacharunderscore}same{\isacharunderscore}semantics{\isacharunderscore}imp{\isacharcolon}\ \isanewline
\ \ \isakeyword{assumes}\ {\isachardoublequoteopen}{\isasymforall}k\ {\isasymin}\ atoms\ F{\isachardot}\ {\isasymA}\isactrlsub {\isadigit{1}}\ k\ {\isacharequal}\ {\isasymA}\isactrlsub {\isadigit{2}}\ k\ {\isasymLongrightarrow}\ {\isasymA}\isactrlsub {\isadigit{1}}\ {\isasymTurnstile}\ F\ {\isasymlongleftrightarrow}\ {\isasymA}\isactrlsub {\isadigit{2}}\ {\isasymTurnstile}\ F{\isachardoublequoteclose}\isanewline
\ \ \ \ \ \ \ \ \ \ {\isachardoublequoteopen}{\isasymforall}k\ {\isasymin}\ atoms\ G{\isachardot}\ {\isasymA}\isactrlsub {\isadigit{1}}\ k\ {\isacharequal}\ {\isasymA}\isactrlsub {\isadigit{2}}\ k\ {\isasymLongrightarrow}\ {\isasymA}\isactrlsub {\isadigit{1}}\ {\isasymTurnstile}\ G\ {\isasymlongleftrightarrow}\ {\isasymA}\isactrlsub {\isadigit{2}}\ {\isasymTurnstile}\ G{\isachardoublequoteclose}\isanewline
\ \ \ \ \ \ \ \ \ \ {\isachardoublequoteopen}{\isasymforall}k\ {\isasymin}\ atoms\ {\isacharparenleft}F\ \isactrlbold {\isasymrightarrow}\ G{\isacharparenright}{\isachardot}\ {\isasymA}\isactrlsub {\isadigit{1}}\ k\ {\isacharequal}\ {\isasymA}\isactrlsub {\isadigit{2}}\ k{\isachardoublequoteclose}\isanewline
\ \ \ \ \ \isakeyword{shows}\ {\isachardoublequoteopen}{\isasymA}\isactrlsub {\isadigit{1}}\ {\isasymTurnstile}\ {\isacharparenleft}F\ \isactrlbold {\isasymrightarrow}\ G{\isacharparenright}\ {\isasymlongleftrightarrow}\ {\isasymA}\isactrlsub {\isadigit{2}}\ {\isasymTurnstile}\ {\isacharparenleft}F\ \isactrlbold {\isasymrightarrow}\ G{\isacharparenright}{\isachardoublequoteclose}\isanewline
%
\isadelimproof
%
\endisadelimproof
%
\isatagproof
\isacommand{proof}\isamarkupfalse%
\ {\isacharminus}\isanewline
\ \ \isacommand{have}\isamarkupfalse%
\ H{\isacharcolon}{\isachardoublequoteopen}{\isasymforall}k\ {\isasymin}\ atoms\ F\ {\isasymunion}\ atoms\ G{\isachardot}\ {\isasymA}\isactrlsub {\isadigit{1}}\ k\ {\isacharequal}\ {\isasymA}\isactrlsub {\isadigit{2}}\ k{\isachardoublequoteclose}\ \isacommand{using}\isamarkupfalse%
\ assms{\isacharparenleft}{\isadigit{3}}{\isacharparenright}\isanewline
\ \ \ \ \isacommand{by}\isamarkupfalse%
\ {\isacharparenleft}simp\ only{\isacharcolon}\ formula{\isachardot}set{\isacharparenleft}{\isadigit{6}}{\isacharparenright}{\isacharparenright}\isanewline
\ \ \isacommand{then}\isamarkupfalse%
\ \isacommand{have}\isamarkupfalse%
\ {\isachardoublequoteopen}{\isasymforall}k\ {\isasymin}\ atoms\ F{\isachardot}\ {\isasymA}\isactrlsub {\isadigit{1}}\ k\ {\isacharequal}\ {\isasymA}\isactrlsub {\isadigit{2}}\ k{\isachardoublequoteclose}\isanewline
\ \ \ \ \isacommand{by}\isamarkupfalse%
\ {\isacharparenleft}rule\ forall{\isacharunderscore}union{\isadigit{1}}{\isacharparenright}\isanewline
\ \ \isacommand{then}\isamarkupfalse%
\ \isacommand{have}\isamarkupfalse%
\ H{\isadigit{1}}{\isacharcolon}{\isachardoublequoteopen}{\isasymA}\isactrlsub {\isadigit{1}}\ {\isasymTurnstile}\ F\ {\isasymlongleftrightarrow}\ {\isasymA}\isactrlsub {\isadigit{2}}\ {\isasymTurnstile}\ F{\isachardoublequoteclose}\isanewline
\ \ \ \ \isacommand{by}\isamarkupfalse%
\ {\isacharparenleft}simp\ only{\isacharcolon}\ assms{\isacharparenleft}{\isadigit{1}}{\isacharparenright}{\isacharparenright}\isanewline
\ \ \isacommand{have}\isamarkupfalse%
\ {\isachardoublequoteopen}{\isasymforall}k\ {\isasymin}\ atoms\ G{\isachardot}\ {\isasymA}\isactrlsub {\isadigit{1}}\ k\ {\isacharequal}\ {\isasymA}\isactrlsub {\isadigit{2}}\ k{\isachardoublequoteclose}\isanewline
\ \ \ \ \isacommand{using}\isamarkupfalse%
\ H\ \isacommand{by}\isamarkupfalse%
\ {\isacharparenleft}rule\ forall{\isacharunderscore}union{\isadigit{2}}{\isacharparenright}\isanewline
\ \ \isacommand{then}\isamarkupfalse%
\ \isacommand{have}\isamarkupfalse%
\ H{\isadigit{2}}{\isacharcolon}{\isachardoublequoteopen}{\isasymA}\isactrlsub {\isadigit{1}}\ {\isasymTurnstile}\ G\ {\isasymlongleftrightarrow}\ {\isasymA}\isactrlsub {\isadigit{2}}\ {\isasymTurnstile}\ G{\isachardoublequoteclose}\isanewline
\ \ \ \ \isacommand{by}\isamarkupfalse%
\ {\isacharparenleft}simp\ only{\isacharcolon}\ assms{\isacharparenleft}{\isadigit{2}}{\isacharparenright}{\isacharparenright}\isanewline
\ \ \isacommand{have}\isamarkupfalse%
\ {\isachardoublequoteopen}{\isasymA}\isactrlsub {\isadigit{1}}\ {\isasymTurnstile}\ F\ \isactrlbold {\isasymrightarrow}\ G\ {\isacharequal}\ {\isacharparenleft}{\isasymA}\isactrlsub {\isadigit{1}}\ {\isasymTurnstile}\ F\ {\isasymlongrightarrow}\ {\isasymA}\isactrlsub {\isadigit{1}}\ {\isasymTurnstile}\ G{\isacharparenright}{\isachardoublequoteclose}\isanewline
\ \ \ \ \isacommand{by}\isamarkupfalse%
\ {\isacharparenleft}simp\ only{\isacharcolon}\ formula{\isacharunderscore}semantics{\isachardot}simps{\isacharparenleft}{\isadigit{6}}{\isacharparenright}{\isacharparenright}\isanewline
\ \ \isacommand{also}\isamarkupfalse%
\ \isacommand{have}\isamarkupfalse%
\ {\isachardoublequoteopen}{\isasymdots}\ {\isacharequal}\ {\isacharparenleft}{\isasymA}\isactrlsub {\isadigit{2}}\ {\isasymTurnstile}\ F\ {\isasymlongrightarrow}\ {\isasymA}\isactrlsub {\isadigit{2}}\ {\isasymTurnstile}\ G{\isacharparenright}{\isachardoublequoteclose}\isanewline
\ \ \ \ \isacommand{using}\isamarkupfalse%
\ H{\isadigit{1}}\ H{\isadigit{2}}\ \isacommand{by}\isamarkupfalse%
\ {\isacharparenleft}rule\ arg{\isacharunderscore}cong{\isadigit{2}}{\isacharparenright}\isanewline
\ \ \isacommand{also}\isamarkupfalse%
\ \isacommand{have}\isamarkupfalse%
\ {\isachardoublequoteopen}{\isasymdots}\ {\isacharequal}\ {\isasymA}\isactrlsub {\isadigit{2}}\ {\isasymTurnstile}\ F\ \isactrlbold {\isasymrightarrow}\ G{\isachardoublequoteclose}\isanewline
\ \ \ \ \isacommand{by}\isamarkupfalse%
\ {\isacharparenleft}simp\ only{\isacharcolon}\ formula{\isacharunderscore}semantics{\isachardot}simps{\isacharparenleft}{\isadigit{6}}{\isacharparenright}{\isacharparenright}\isanewline
\ \ \isacommand{finally}\isamarkupfalse%
\ \isacommand{show}\isamarkupfalse%
\ {\isacharquery}thesis\ \isanewline
\ \ \ \ \isacommand{by}\isamarkupfalse%
\ this\isanewline
\isacommand{qed}\isamarkupfalse%
%
\endisatagproof
{\isafoldproof}%
%
\isadelimproof
\isanewline
%
\endisadelimproof
\isanewline
\isacommand{lemma}\isamarkupfalse%
\ {\isachardoublequoteopen}{\isasymforall}k\ {\isasymin}\ atoms\ F{\isachardot}\ {\isasymA}\isactrlsub {\isadigit{1}}\ k\ {\isacharequal}\ {\isasymA}\isactrlsub {\isadigit{2}}\ k\ {\isasymLongrightarrow}\ {\isasymA}\isactrlsub {\isadigit{1}}\ {\isasymTurnstile}\ F\ {\isasymlongleftrightarrow}\ {\isasymA}\isactrlsub {\isadigit{2}}\ {\isasymTurnstile}\ F{\isachardoublequoteclose}\isanewline
%
\isadelimproof
%
\endisadelimproof
%
\isatagproof
\isacommand{proof}\isamarkupfalse%
\ {\isacharparenleft}induction\ F{\isacharparenright}\isanewline
\isacommand{case}\isamarkupfalse%
\ {\isacharparenleft}Atom\ x{\isacharparenright}\isanewline
\ \ \isacommand{then}\isamarkupfalse%
\ \isacommand{show}\isamarkupfalse%
\ {\isacharquery}case\ \isacommand{by}\isamarkupfalse%
\ {\isacharparenleft}rule\ relevant{\isacharunderscore}atoms{\isacharunderscore}same{\isacharunderscore}semantics{\isacharunderscore}atomic{\isacharparenright}\isanewline
\isacommand{next}\isamarkupfalse%
\isanewline
\ \ \isacommand{case}\isamarkupfalse%
\ Bot\isanewline
\isacommand{then}\isamarkupfalse%
\ \isacommand{show}\isamarkupfalse%
\ {\isacharquery}case\ \isacommand{by}\isamarkupfalse%
\ {\isacharparenleft}rule\ relevant{\isacharunderscore}atoms{\isacharunderscore}same{\isacharunderscore}semantics{\isacharunderscore}bot{\isacharparenright}\isanewline
\isacommand{next}\isamarkupfalse%
\isanewline
\ \ \isacommand{case}\isamarkupfalse%
\ {\isacharparenleft}Not\ F{\isacharparenright}\isanewline
\isacommand{then}\isamarkupfalse%
\ \isacommand{show}\isamarkupfalse%
\ {\isacharquery}case\ \isacommand{by}\isamarkupfalse%
\ {\isacharparenleft}rule\ relevant{\isacharunderscore}atoms{\isacharunderscore}same{\isacharunderscore}semantics{\isacharunderscore}not{\isacharparenright}\isanewline
\isacommand{next}\isamarkupfalse%
\isanewline
\ \ \isacommand{case}\isamarkupfalse%
\ {\isacharparenleft}And\ F{\isadigit{1}}\ F{\isadigit{2}}{\isacharparenright}\isanewline
\ \ \isacommand{then}\isamarkupfalse%
\ \isacommand{show}\isamarkupfalse%
\ {\isacharquery}case\ \isacommand{by}\isamarkupfalse%
\ {\isacharparenleft}rule\ relevant{\isacharunderscore}atoms{\isacharunderscore}same{\isacharunderscore}semantics{\isacharunderscore}and{\isacharparenright}\isanewline
\isacommand{next}\isamarkupfalse%
\isanewline
\isacommand{case}\isamarkupfalse%
\ {\isacharparenleft}Or\ F{\isadigit{1}}\ F{\isadigit{2}}{\isacharparenright}\isanewline
\ \ \isacommand{then}\isamarkupfalse%
\ \isacommand{show}\isamarkupfalse%
\ {\isacharquery}case\ \isacommand{by}\isamarkupfalse%
\ {\isacharparenleft}rule\ relevant{\isacharunderscore}atoms{\isacharunderscore}same{\isacharunderscore}semantics{\isacharunderscore}or{\isacharparenright}\isanewline
\isacommand{next}\isamarkupfalse%
\isanewline
\ \ \isacommand{case}\isamarkupfalse%
\ {\isacharparenleft}Imp\ F{\isadigit{1}}\ F{\isadigit{2}}{\isacharparenright}\isanewline
\ \ \isacommand{then}\isamarkupfalse%
\ \isacommand{show}\isamarkupfalse%
\ {\isacharquery}case\ \isacommand{by}\isamarkupfalse%
\ {\isacharparenleft}rule\ relevant{\isacharunderscore}atoms{\isacharunderscore}same{\isacharunderscore}semantics{\isacharunderscore}imp{\isacharparenright}\isanewline
\isacommand{qed}\isamarkupfalse%
%
\endisatagproof
{\isafoldproof}%
%
\isadelimproof
%
\endisadelimproof
%
\begin{isamarkuptext}%
Su demostración automática es la siguiente.%
\end{isamarkuptext}\isamarkuptrue%
\isacommand{lemma}\isamarkupfalse%
\ relevant{\isacharunderscore}atoms{\isacharunderscore}same{\isacharunderscore}semantics{\isacharcolon}\ \isanewline
\ \ \ {\isachardoublequoteopen}{\isasymforall}k\ {\isasymin}\ atoms\ F{\isachardot}\ {\isasymA}\isactrlsub {\isadigit{1}}\ k\ {\isacharequal}\ {\isasymA}\isactrlsub {\isadigit{2}}\ k\ {\isasymLongrightarrow}\ {\isasymA}\isactrlsub {\isadigit{1}}\ {\isasymTurnstile}\ F\ {\isasymlongleftrightarrow}\ {\isasymA}\isactrlsub {\isadigit{2}}\ {\isasymTurnstile}\ F{\isachardoublequoteclose}\isanewline
%
\isadelimproof
\ \ %
\endisadelimproof
%
\isatagproof
\isacommand{by}\isamarkupfalse%
\ {\isacharparenleft}induction\ F{\isacharparenright}\ simp{\isacharunderscore}all%
\endisatagproof
{\isafoldproof}%
%
\isadelimproof
%
\endisadelimproof
%
\isadelimdocument
%
\endisadelimdocument
%
\isatagdocument
%
\isamarkupsection{Semántica de fórmulas con conectivas generalizadas%
}
\isamarkuptrue%
%
\endisatagdocument
{\isafolddocument}%
%
\isadelimdocument
%
\endisadelimdocument
%
\begin{isamarkuptext}%
En este apartado mostraremos varios resultados relativos a la 
  semántica de las fórmulas construidas con conectivas generalizadas.

  \begin{lema}
    Una interpretación es modelo de la conjunción generalizada de una 
    lista de fórmulas si y solo si es modelo de cada fórmula de la
    lista.
  \end{lema}

  Su formalización en Isabelle es la siguiente.%
\end{isamarkuptext}\isamarkuptrue%
\isacommand{lemma}\isamarkupfalse%
\ {\isachardoublequoteopen}{\isacharparenleft}{\isasymA}\ {\isasymTurnstile}\ \isactrlbold {\isasymAnd}Fs{\isacharparenright}\ {\isasymlongleftrightarrow}\ {\isacharparenleft}{\isasymforall}f\ {\isasymin}\ set\ Fs{\isachardot}\ {\isasymA}\ {\isasymTurnstile}\ f{\isacharparenright}{\isachardoublequoteclose}\isanewline
%
\isadelimproof
\ \ %
\endisadelimproof
%
\isatagproof
\isacommand{oops}\isamarkupfalse%
%
\endisatagproof
{\isafoldproof}%
%
\isadelimproof
%
\endisadelimproof
%
\begin{isamarkuptext}%
Como podemos observar, en el enunciado de la derecha hemos
  empleado \isa{set} para cambiar al tipo de los conjuntos la lista de 
  fórmulas, pues esto permite emplear el cuantificador universal.

  Procedamos con la prueba del lema.

  \begin{demostracion}
   La prueba se basa en el esquema de inducción sobre listas. Para ello, 
   demostraremos el resultado mediante cadenas de equivalencias en los 
   siguientes casos.

   En primer lugar, lo probamos para la lista vacía de fórmulas. Sea la
   interpretación \isa{{\isasymA}} tal que es modelo de la conjunción generalizada
   de la lista vacía. Por definición de la conjunción generalizada,
   \isa{{\isasymA}} es modelo de \isa{{\isasymnot}\ {\isasymbottom}}. Aplicando la definición del valor de una
   fórmula en una interpretación para el caso de la negación,
   tenemos que esto es equivalente a que \isa{{\isasymA}} no es modelo de \isa{{\isasymbottom}}.
   Análogamente, como sabemos que el valor de \isa{{\isasymbottom}} es \isa{Falso} en 
   cualquier interpretación, se tiene que lo anterior es equivalente a
   \isa{{\isasymnot}\ Falso}, es decir, \isa{Verdadero}. Por otro lado, por propiedades
   del conjunto vacío, se tiene que toda propiedad sobre sus elementos
   es verdadera. Por tanto, lo anterior es equivalente a decir que \isa{{\isasymA}} 
   es modelo de todos los elementos del conjunto vacío. Es decir, \isa{{\isasymA}}
   es modelo de todos los elementos de la lista vacía, como queríamos
   demostrar. 

   Consideramos ahora la interpretación \isa{{\isasymA}} y la lista de fórmulas
   \isa{Fs} de modo que \isa{{\isasymA}} es modelo de la conjunción generalizada de \isa{Fs} 
   si y solo si es modelo de cada fórmula de \isa{Fs}. Veamos ahora que se 
   verifica la propiedad para la lista \isa{F{\isacharhash}Fs} formada al añadir la 
   fórmula \isa{F}.
   En primer lugar, si \isa{{\isasymA}} es modelo de la conjunción generalizada de
   \isa{F{\isacharhash}Fs}, por definición de dicha conjunción, esto es equivalente a
   que \isa{{\isasymA}} es modelo de la conjunción de \isa{F} y la conjunción
   generalizada de \isa{Fs}. Según el valor de una fórmula en una
   interpretación, esto es a su vez equivalente a la conjunción de
   "\isa{{\isasymA}} es modelo de \isa{F}" y "\isa{{\isasymA}} es modelo de la conjunción 
   generalizada de \isa{Fs}". Aplicando la hipótesis de inducción sobre el 
   segundo término de la conjunción, es equivalente a la conjunción de 
   "\isa{{\isasymA}} es  modelo de \isa{F}" y "\isa{{\isasymA}} es modelo de toda fórmula del 
   conjunto formado por los elementos de \isa{Fs}". Equivalentemente, \isa{{\isasymA}} 
   es modelo de toda fórmula del conjunto formado por la unión de \isa{{\isacharbraceleft}F{\isacharbraceright}} 
   y el conjunto de los elementos de \isa{Fs}. Es decir, \isa{{\isasymA}} es modelo de 
   toda fórmula del conjunto formado por los elementos de \isa{F{\isacharhash}Fs}, 
   probando así el resultado.
  \end{demostracion}

  Veamos ahora su demostración detallada en Isabelle. Primero se
  probarán los dos casos correspondientes a la estructura de listas por
  separado.%
\end{isamarkuptext}\isamarkuptrue%
\isacommand{lemma}\isamarkupfalse%
\ BigAnd{\isacharunderscore}semantics{\isacharunderscore}nil{\isacharcolon}\ {\isachardoublequoteopen}{\isacharparenleft}{\isasymA}\ {\isasymTurnstile}\ \isactrlbold {\isasymAnd}{\isacharbrackleft}{\isacharbrackright}{\isacharparenright}\ {\isasymlongleftrightarrow}\ {\isacharparenleft}{\isasymforall}f\ {\isasymin}\ set\ {\isacharbrackleft}{\isacharbrackright}{\isachardot}\ {\isasymA}\ {\isasymTurnstile}\ f{\isacharparenright}{\isachardoublequoteclose}\isanewline
%
\isadelimproof
%
\endisadelimproof
%
\isatagproof
\isacommand{proof}\isamarkupfalse%
{\isacharminus}\isanewline
\ \ \isacommand{have}\isamarkupfalse%
\ {\isachardoublequoteopen}{\isacharparenleft}{\isasymA}\ {\isasymTurnstile}\ \isactrlbold {\isasymAnd}{\isacharbrackleft}{\isacharbrackright}{\isacharparenright}\ {\isacharequal}\ {\isasymA}\ {\isasymTurnstile}\ {\isacharparenleft}\isactrlbold {\isasymnot}{\isasymbottom}{\isacharparenright}{\isachardoublequoteclose}\isanewline
\ \ \ \ \isacommand{by}\isamarkupfalse%
\ {\isacharparenleft}simp\ only{\isacharcolon}\ BigAnd{\isachardot}simps{\isacharparenleft}{\isadigit{1}}{\isacharparenright}{\isacharparenright}\isanewline
\ \ \isacommand{also}\isamarkupfalse%
\ \isacommand{have}\isamarkupfalse%
\ {\isachardoublequoteopen}{\isasymdots}\ {\isacharequal}\ {\isacharparenleft}{\isasymnot}\ {\isasymA}\ {\isasymTurnstile}\ {\isasymbottom}{\isacharparenright}{\isachardoublequoteclose}\isanewline
\ \ \ \ \isacommand{by}\isamarkupfalse%
\ {\isacharparenleft}simp\ only{\isacharcolon}\ formula{\isacharunderscore}semantics{\isachardot}simps{\isacharparenleft}{\isadigit{3}}{\isacharparenright}{\isacharparenright}\isanewline
\ \ \isacommand{also}\isamarkupfalse%
\ \isacommand{have}\isamarkupfalse%
\ {\isachardoublequoteopen}{\isasymdots}\ {\isacharequal}\ {\isacharparenleft}{\isasymnot}\ False{\isacharparenright}{\isachardoublequoteclose}\isanewline
\ \ \ \ \isacommand{by}\isamarkupfalse%
\ {\isacharparenleft}simp\ only{\isacharcolon}\ formula{\isacharunderscore}semantics{\isachardot}simps{\isacharparenleft}{\isadigit{2}}{\isacharparenright}{\isacharparenright}\isanewline
\ \ \isacommand{also}\isamarkupfalse%
\ \isacommand{have}\isamarkupfalse%
\ {\isachardoublequoteopen}{\isasymdots}\ {\isacharequal}\ True{\isachardoublequoteclose}\isanewline
\ \ \ \ \isacommand{by}\isamarkupfalse%
\ {\isacharparenleft}simp\ only{\isacharcolon}\ not{\isacharunderscore}False{\isacharunderscore}eq{\isacharunderscore}True{\isacharparenright}\isanewline
\ \ \isacommand{also}\isamarkupfalse%
\ \isacommand{have}\isamarkupfalse%
\ {\isachardoublequoteopen}{\isasymdots}\ {\isacharequal}\ {\isacharparenleft}{\isasymforall}f\ {\isasymin}\ {\isasymemptyset}{\isachardot}\ {\isasymA}\ {\isasymTurnstile}\ f{\isacharparenright}{\isachardoublequoteclose}\isanewline
\ \ \ \ \isacommand{by}\isamarkupfalse%
\ {\isacharparenleft}simp\ only{\isacharcolon}\ ball{\isacharunderscore}empty{\isacharparenright}\ \isanewline
\ \ \isacommand{also}\isamarkupfalse%
\ \isacommand{have}\isamarkupfalse%
\ {\isachardoublequoteopen}{\isasymdots}\ {\isacharequal}\ {\isacharparenleft}{\isasymforall}f\ {\isasymin}\ set\ {\isacharbrackleft}{\isacharbrackright}{\isachardot}\ {\isasymA}\ {\isasymTurnstile}\ f{\isacharparenright}{\isachardoublequoteclose}\isanewline
\ \ \ \ \isacommand{by}\isamarkupfalse%
\ {\isacharparenleft}simp\ only{\isacharcolon}\ list{\isachardot}set{\isacharparenright}\isanewline
\ \ \isacommand{finally}\isamarkupfalse%
\ \isacommand{show}\isamarkupfalse%
\ {\isacharquery}thesis\isanewline
\ \ \ \ \isacommand{by}\isamarkupfalse%
\ this\isanewline
\isacommand{qed}\isamarkupfalse%
%
\endisatagproof
{\isafoldproof}%
%
\isadelimproof
%
\endisadelimproof
%
\begin{isamarkuptext}%
Para el caso de la lista formada añadiendo un elemento, vamos a
  emplear el siguiente lema auxiliar.%
\end{isamarkuptext}\isamarkuptrue%
\isacommand{lemma}\isamarkupfalse%
\ Bigprop{\isadigit{1}}{\isacharcolon}\ {\isachardoublequoteopen}{\isacharparenleft}{\isasymforall}x{\isasymin}{\isacharparenleft}{\isacharbraceleft}a{\isacharbraceright}\ {\isasymunion}\ B{\isacharparenright}{\isachardot}\ P\ x{\isacharparenright}\ {\isacharequal}\ {\isacharparenleft}P\ a\ {\isasymand}\ {\isacharparenleft}{\isasymforall}x{\isasymin}B{\isachardot}\ P\ x{\isacharparenright}{\isacharparenright}{\isachardoublequoteclose}\isanewline
%
\isadelimproof
\ \ %
\endisadelimproof
%
\isatagproof
\isacommand{by}\isamarkupfalse%
\ {\isacharparenleft}simp\ only{\isacharcolon}\ ball{\isacharunderscore}simps{\isacharparenleft}{\isadigit{7}}{\isacharparenright}\isanewline
\ \ \ \ \ \ \ \ \ \ \ \ \ \ \ \ \ insert{\isacharunderscore}is{\isacharunderscore}Un{\isacharbrackleft}THEN\ sym{\isacharbrackright}{\isacharparenright}%
\endisatagproof
{\isafoldproof}%
%
\isadelimproof
%
\endisadelimproof
%
\begin{isamarkuptext}%
Se trata de una adaptación del séptimo apartado del lema 
  \isa{ball{\isacharunderscore}simps} en Isabelle, para adecuarlo a nuestro caso particular. 

  \begin{itemize}
    \item[] \isa{{\isacharparenleft}{\isasymforall}x{\isasymin}insert\ a\ B{\isachardot}\ P\ x{\isacharparenright}\ {\isasymlongleftrightarrow}\ {\isacharparenleft}P\ a\ {\isasymand}\ {\isacharparenleft}{\isasymforall}x{\isasymin}B{\isachardot}\ P\ x{\isacharparenright}{\isacharparenright}} 
        \hspace{3cm} \isa{{\isacharparenleft}ball{\isacharunderscore}simps{\isacharparenleft}{\isadigit{7}}{\isacharparenright}{\isacharparenright}}
  \end{itemize}

  Para ello, empleamos el lema \isa{insert{\isacharunderscore}is{\isacharunderscore}Un} seguido de \isa{{\isacharbrackleft}THEN\ sym{\isacharbrackright}}
  como hemos hecho anteriormente.

  Procedamos, así, con la demostración del caso del lema 
  correspondiente. Por último, mostraremos también su demostración
  detallada completa.%
\end{isamarkuptext}\isamarkuptrue%
\isacommand{lemma}\isamarkupfalse%
\ BigAnd{\isacharunderscore}semantics{\isacharunderscore}cons{\isacharcolon}\ \isanewline
\ \ \isakeyword{assumes}\ {\isachardoublequoteopen}{\isacharparenleft}{\isasymA}\ {\isasymTurnstile}\ \isactrlbold {\isasymAnd}Fs{\isacharparenright}\ {\isasymlongleftrightarrow}\ {\isacharparenleft}{\isasymforall}f\ {\isasymin}\ set\ Fs{\isachardot}\ {\isasymA}\ {\isasymTurnstile}\ f{\isacharparenright}{\isachardoublequoteclose}\isanewline
\ \ \isakeyword{shows}\ {\isachardoublequoteopen}{\isacharparenleft}{\isasymA}\ {\isasymTurnstile}\ \isactrlbold {\isasymAnd}{\isacharparenleft}F{\isacharhash}Fs{\isacharparenright}{\isacharparenright}\ {\isasymlongleftrightarrow}\ {\isacharparenleft}{\isasymforall}f\ {\isasymin}\ set\ {\isacharparenleft}F{\isacharhash}Fs{\isacharparenright}{\isachardot}\ {\isasymA}\ {\isasymTurnstile}\ f{\isacharparenright}{\isachardoublequoteclose}\isanewline
%
\isadelimproof
%
\endisadelimproof
%
\isatagproof
\isacommand{proof}\isamarkupfalse%
\ {\isacharminus}\isanewline
\ \ \isacommand{have}\isamarkupfalse%
\ {\isachardoublequoteopen}{\isacharparenleft}{\isasymA}\ {\isasymTurnstile}\ \isactrlbold {\isasymAnd}{\isacharparenleft}F{\isacharhash}Fs{\isacharparenright}{\isacharparenright}\ {\isacharequal}\ {\isasymA}\ {\isasymTurnstile}\ F\ \isactrlbold {\isasymand}\ \isactrlbold {\isasymAnd}Fs{\isachardoublequoteclose}\isanewline
\ \ \ \ \isacommand{by}\isamarkupfalse%
\ {\isacharparenleft}simp\ only{\isacharcolon}\ BigAnd{\isachardot}simps{\isacharparenleft}{\isadigit{2}}{\isacharparenright}{\isacharparenright}\isanewline
\ \ \isacommand{also}\isamarkupfalse%
\ \isacommand{have}\isamarkupfalse%
\ {\isachardoublequoteopen}{\isasymdots}\ {\isacharequal}\ {\isacharparenleft}{\isasymA}\ {\isasymTurnstile}\ F\ {\isasymand}\ {\isasymA}\ {\isasymTurnstile}\ \isactrlbold {\isasymAnd}Fs{\isacharparenright}{\isachardoublequoteclose}\isanewline
\ \ \ \ \isacommand{by}\isamarkupfalse%
\ {\isacharparenleft}simp\ only{\isacharcolon}\ formula{\isacharunderscore}semantics{\isachardot}simps{\isacharparenleft}{\isadigit{4}}{\isacharparenright}{\isacharparenright}\isanewline
\ \ \isacommand{also}\isamarkupfalse%
\ \isacommand{have}\isamarkupfalse%
\ {\isachardoublequoteopen}{\isasymdots}\ {\isacharequal}\ {\isacharparenleft}{\isasymA}\ {\isasymTurnstile}\ F\ {\isasymand}\ {\isacharparenleft}{\isasymforall}f\ {\isasymin}\ set\ Fs{\isachardot}\ {\isasymA}\ {\isasymTurnstile}\ f{\isacharparenright}{\isacharparenright}{\isachardoublequoteclose}\isanewline
\ \ \ \ \isacommand{by}\isamarkupfalse%
\ {\isacharparenleft}simp\ only{\isacharcolon}\ assms{\isacharparenright}\isanewline
\ \ \isacommand{also}\isamarkupfalse%
\ \isacommand{have}\isamarkupfalse%
\ {\isachardoublequoteopen}{\isasymdots}\ {\isacharequal}\ {\isacharparenleft}{\isasymforall}f\ {\isasymin}\ {\isacharparenleft}{\isacharbraceleft}F{\isacharbraceright}\ {\isasymunion}\ set\ Fs{\isacharparenright}{\isachardot}\ {\isasymA}\ {\isasymTurnstile}\ f{\isacharparenright}{\isachardoublequoteclose}\isanewline
\ \ \ \ \isacommand{by}\isamarkupfalse%
\ {\isacharparenleft}simp\ only{\isacharcolon}\ Bigprop{\isadigit{1}}{\isacharparenright}\isanewline
\ \ \isacommand{also}\isamarkupfalse%
\ \isacommand{have}\isamarkupfalse%
\ {\isachardoublequoteopen}{\isasymdots}\ {\isacharequal}\ {\isacharparenleft}{\isasymforall}f\ {\isasymin}\ set\ {\isacharparenleft}F{\isacharhash}Fs{\isacharparenright}{\isachardot}\ {\isasymA}\ {\isasymTurnstile}\ f{\isacharparenright}{\isachardoublequoteclose}\isanewline
\ \ \ \ \isacommand{by}\isamarkupfalse%
\ {\isacharparenleft}simp\ only{\isacharcolon}\ set{\isacharunderscore}insert{\isacharparenright}\isanewline
\ \ \isacommand{finally}\isamarkupfalse%
\ \isacommand{show}\isamarkupfalse%
\ {\isacharquery}thesis\isanewline
\ \ \ \ \isacommand{by}\isamarkupfalse%
\ this\isanewline
\isacommand{qed}\isamarkupfalse%
%
\endisatagproof
{\isafoldproof}%
%
\isadelimproof
\isanewline
%
\endisadelimproof
\isanewline
\isacommand{lemma}\isamarkupfalse%
\ {\isachardoublequoteopen}{\isacharparenleft}{\isasymA}\ {\isasymTurnstile}\ \isactrlbold {\isasymAnd}Fs{\isacharparenright}\ {\isasymlongleftrightarrow}\ {\isacharparenleft}{\isasymforall}f\ {\isasymin}\ set\ Fs{\isachardot}\ {\isasymA}\ {\isasymTurnstile}\ f{\isacharparenright}{\isachardoublequoteclose}\isanewline
%
\isadelimproof
%
\endisadelimproof
%
\isatagproof
\isacommand{proof}\isamarkupfalse%
\ {\isacharparenleft}induction\ Fs{\isacharparenright}\isanewline
\ \ \isacommand{case}\isamarkupfalse%
\ Nil\isanewline
\ \ \isacommand{then}\isamarkupfalse%
\ \isacommand{show}\isamarkupfalse%
\ {\isacharquery}case\ \isacommand{by}\isamarkupfalse%
\ {\isacharparenleft}rule\ BigAnd{\isacharunderscore}semantics{\isacharunderscore}nil{\isacharparenright}\isanewline
\isacommand{next}\isamarkupfalse%
\isanewline
\ \ \isacommand{case}\isamarkupfalse%
\ {\isacharparenleft}Cons\ a\ Fs{\isacharparenright}\isanewline
\ \ \isacommand{then}\isamarkupfalse%
\ \isacommand{show}\isamarkupfalse%
\ {\isacharquery}case\ \isacommand{by}\isamarkupfalse%
\ {\isacharparenleft}rule\ BigAnd{\isacharunderscore}semantics{\isacharunderscore}cons{\isacharparenright}\isanewline
\isacommand{qed}\isamarkupfalse%
%
\endisatagproof
{\isafoldproof}%
%
\isadelimproof
%
\endisadelimproof
%
\begin{isamarkuptext}%
Su demostración automática es la siguiente.%
\end{isamarkuptext}\isamarkuptrue%
\isacommand{lemma}\isamarkupfalse%
\ BigAnd{\isacharunderscore}semantics{\isacharcolon}\ \isanewline
\ \ {\isachardoublequoteopen}{\isacharparenleft}{\isasymA}\ {\isasymTurnstile}\ \isactrlbold {\isasymAnd}Fs{\isacharparenright}\ {\isasymlongleftrightarrow}\ {\isacharparenleft}{\isasymforall}f\ {\isasymin}\ set\ Fs{\isachardot}\ {\isasymA}\ {\isasymTurnstile}\ f{\isacharparenright}{\isachardoublequoteclose}\isanewline
%
\isadelimproof
\ \ %
\endisadelimproof
%
\isatagproof
\isacommand{by}\isamarkupfalse%
\ {\isacharparenleft}induction\ Fs{\isacharparenright}\ simp{\isacharunderscore}all%
\endisatagproof
{\isafoldproof}%
%
\isadelimproof
%
\endisadelimproof
%
\begin{isamarkuptext}%
Finalmente, un resultado sobre la disyunción generalizada.

  \begin{lema}
    Una interpretación es modelo de la disyunción generalizada de una 
    lista de fórmulas si y solo si existe una fórmula en la lista de la
    cual sea modelo.
  \end{lema}

  Su formalización en Isabelle es la siguiente.%
\end{isamarkuptext}\isamarkuptrue%
\isacommand{lemma}\isamarkupfalse%
\ {\isachardoublequoteopen}{\isacharparenleft}{\isasymA}\ {\isasymTurnstile}\ \isactrlbold {\isasymOr}Fs{\isacharparenright}\ {\isasymlongleftrightarrow}\ {\isacharparenleft}{\isasymexists}f\ {\isasymin}\ set\ Fs{\isachardot}\ {\isasymA}\ {\isasymTurnstile}\ f{\isacharparenright}{\isachardoublequoteclose}\ \isanewline
%
\isadelimproof
\ \ %
\endisadelimproof
%
\isatagproof
\isacommand{oops}\isamarkupfalse%
%
\endisatagproof
{\isafoldproof}%
%
\isadelimproof
%
\endisadelimproof
%
\begin{isamarkuptext}%
Procedamos con la demostración del resultado.

  \begin{demostracion}
    La prueba sigue el esquema de inducción sobre la estructura de
    listas. Vamos a probar los dos casos mediante cadenas de 
    equivalencias.

    Sea la lista vacía de fórmulas y una interpretación cualquiera
    \isa{{\isasymA}}. Por definición de la disyunción generalizada, si \isa{{\isasymA}} es
    modelo de la disyunción generalizada de la lista vacía,
    equivalentemente tenemos que es modelo de \isa{{\isasymbottom}}, es decir, \isa{Falso}.
    En particular, esto es equivalente a suponer que existe una fórmula 
    en el conjunto vacío tal que \isa{{\isasymA}} es modelo suyo.

    Consideremos ahora la lista de fórmulas \isa{Fs} y una interpretación
    \isa{{\isasymA}} tal que es modelo de \isa{Fs} si y solo si es modelo de cada
    fórmula del conjunto formado por los elementos de \isa{Fs}. Vamos a
    probar el resultado para la lista \isa{F{\isacharhash}Fs} dada cualquier fórmula
    \isa{F}. Si \isa{{\isasymA}} es modelo de la disyunción generalizada de \isa{F{\isacharhash}Fs}, por
    definición, es equivalente a la disyunción entre "\isa{{\isasymA}} es modelo de
    \isa{F}" y "\isa{{\isasymA}} es modelo de la disyunción generalizada de \isa{Fs}". 
    Aplicando la hipótesis de inducción, tenemos equivalentemente la
    disyunción entre "\isa{{\isasymA}} es modelo de \isa{F}" y "existe una fórmula
    perteneciente al conjunto de elementos de \isa{Fs} tal que \isa{{\isasymA}} es
    modelo suyo". Por tanto, por propiedades de la disyunción, esto es 
    equivalente a que exista una fórmula perteneciente a la unión de
    \isa{{\isacharbraceleft}F{\isacharbraceright}} y el conjunto de los elementos de \isa{Fs} que tiene a \isa{{\isasymA}} como 
    modelo. Finalmente, tenemos que esto ocurre si y solo si
    existe una fórmula en el conjunto de los elementos de \isa{F{\isacharhash}Fs} de la 
    cual \isa{{\isasymA}} sea modelo, como queríamos demostrar.
  \end{demostracion}

  A continuación lo probamos de manera detallada con Isabelle/HOL, 
  haciendo previamente cada paso por separado.%
\end{isamarkuptext}\isamarkuptrue%
\isacommand{lemma}\isamarkupfalse%
\ BigOr{\isacharunderscore}semantics{\isacharunderscore}nil{\isacharcolon}\ {\isachardoublequoteopen}{\isacharparenleft}{\isasymA}\ {\isasymTurnstile}\ \isactrlbold {\isasymOr}{\isacharbrackleft}{\isacharbrackright}{\isacharparenright}\ {\isasymlongleftrightarrow}\ {\isacharparenleft}{\isasymexists}f\ {\isasymin}\ set\ {\isacharbrackleft}{\isacharbrackright}{\isachardot}\ {\isasymA}\ {\isasymTurnstile}\ f{\isacharparenright}{\isachardoublequoteclose}\ \isanewline
%
\isadelimproof
%
\endisadelimproof
%
\isatagproof
\isacommand{proof}\isamarkupfalse%
\ {\isacharminus}\isanewline
\ \ \isacommand{have}\isamarkupfalse%
\ {\isachardoublequoteopen}{\isacharparenleft}{\isasymA}\ {\isasymTurnstile}\ \isactrlbold {\isasymOr}{\isacharbrackleft}{\isacharbrackright}{\isacharparenright}\ {\isacharequal}\ {\isasymA}\ {\isasymTurnstile}\ {\isasymbottom}{\isachardoublequoteclose}\isanewline
\ \ \ \ \isacommand{by}\isamarkupfalse%
\ {\isacharparenleft}simp\ only{\isacharcolon}\ BigOr{\isachardot}simps{\isacharparenleft}{\isadigit{1}}{\isacharparenright}{\isacharparenright}\isanewline
\ \ \isacommand{also}\isamarkupfalse%
\ \isacommand{have}\isamarkupfalse%
\ {\isachardoublequoteopen}{\isasymdots}\ {\isacharequal}\ False{\isachardoublequoteclose}\isanewline
\ \ \ \ \isacommand{by}\isamarkupfalse%
\ {\isacharparenleft}simp\ only{\isacharcolon}\ formula{\isacharunderscore}semantics{\isachardot}simps{\isacharparenleft}{\isadigit{2}}{\isacharparenright}{\isacharparenright}\isanewline
\ \ \isacommand{also}\isamarkupfalse%
\ \isacommand{have}\isamarkupfalse%
\ {\isachardoublequoteopen}{\isasymdots}\ {\isacharequal}\ {\isacharparenleft}{\isasymexists}f\ {\isasymin}\ {\isasymemptyset}{\isachardot}\ {\isasymA}\ {\isasymTurnstile}\ f{\isacharparenright}{\isachardoublequoteclose}\isanewline
\ \ \ \ \isacommand{by}\isamarkupfalse%
\ {\isacharparenleft}simp\ only{\isacharcolon}\ bex{\isacharunderscore}empty{\isacharparenright}\isanewline
\ \ \isacommand{also}\isamarkupfalse%
\ \isacommand{have}\isamarkupfalse%
\ {\isachardoublequoteopen}{\isasymdots}\ {\isacharequal}\ {\isacharparenleft}{\isasymexists}f\ {\isasymin}\ set\ {\isacharbrackleft}{\isacharbrackright}{\isachardot}\ {\isasymA}\ {\isasymTurnstile}\ f{\isacharparenright}{\isachardoublequoteclose}\isanewline
\ \ \ \ \isacommand{by}\isamarkupfalse%
\ {\isacharparenleft}simp\ only{\isacharcolon}\ list{\isachardot}set{\isacharparenright}\isanewline
\ \ \isacommand{finally}\isamarkupfalse%
\ \isacommand{show}\isamarkupfalse%
\ {\isacharquery}thesis\isanewline
\ \ \ \ \isacommand{by}\isamarkupfalse%
\ this\isanewline
\isacommand{qed}\isamarkupfalse%
%
\endisatagproof
{\isafoldproof}%
%
\isadelimproof
%
\endisadelimproof
%
\begin{isamarkuptext}%
Para el segundo caso de las listas emplearemos el siguiente lema
  auxiliar.%
\end{isamarkuptext}\isamarkuptrue%
\isacommand{lemma}\isamarkupfalse%
\ Bigprop{\isadigit{2}}{\isacharcolon}\ {\isachardoublequoteopen}{\isacharparenleft}{\isasymexists}x{\isasymin}{\isacharbraceleft}a{\isacharbraceright}\ {\isasymunion}\ B{\isachardot}\ P\ x{\isacharparenright}\ {\isacharequal}\ {\isacharparenleft}P\ a\ {\isasymor}\ {\isacharparenleft}{\isasymexists}x{\isasymin}B{\isachardot}\ P\ x{\isacharparenright}{\isacharparenright}{\isachardoublequoteclose}\isanewline
%
\isadelimproof
\ \ %
\endisadelimproof
%
\isatagproof
\isacommand{by}\isamarkupfalse%
\ {\isacharparenleft}simp\ only{\isacharcolon}\ bex{\isacharunderscore}simps{\isacharparenleft}{\isadigit{5}}{\isacharparenright}\isanewline
\ \ \ \ \ \ \ \ \ \ \ \ \ \ \ \ \ insert{\isacharunderscore}is{\isacharunderscore}Un{\isacharbrackleft}THEN\ sym{\isacharbrackright}{\isacharparenright}%
\endisatagproof
{\isafoldproof}%
%
\isadelimproof
%
\endisadelimproof
%
\begin{isamarkuptext}%
De forma similar al resultado sobre conjunción generalizada, se 
  trata de una modificación del quinto apartado del lema \isa{bex{\isacharunderscore}simps} en 
  Isabelle para adaptarlo al caso particular. 

  \begin{itemize}
    \item[] \isa{{\isacharparenleft}{\isasymexists}x{\isasymin}insert\ a\ B{\isachardot}\ P\ x{\isacharparenright}\ {\isasymlongleftrightarrow}\ {\isacharparenleft}P\ a\ {\isasymor}\ {\isacharparenleft}{\isasymexists}x{\isasymin}B{\isachardot}\ P\ x{\isacharparenright}{\isacharparenright}} 
        \hspace{3cm} \isa{{\isacharparenleft}bex{\isacharunderscore}simps{\isacharparenleft}{\isadigit{5}}{\isacharparenright}{\isacharparenright}}
  \end{itemize}
  
  Para modificarlo, empleamos análogamente el lema \isa{insert{\isacharunderscore}is{\isacharunderscore}Un}
  seguido de \isa{{\isacharbrackleft}THEN\ sym{\isacharbrackright}}, procediendo de manera similar a los casos 
  en los que se ha usado con anterioridad.
  
  Seguimos, así, con la demostración detallada del lema.%
\end{isamarkuptext}\isamarkuptrue%
\isacommand{lemma}\isamarkupfalse%
\ BigOr{\isacharunderscore}semantics{\isacharunderscore}cons{\isacharcolon}\ \isanewline
\ \ \isakeyword{assumes}\ {\isachardoublequoteopen}{\isacharparenleft}{\isasymA}\ {\isasymTurnstile}\ \isactrlbold {\isasymOr}Fs{\isacharparenright}\ {\isasymlongleftrightarrow}\ {\isacharparenleft}{\isasymexists}f\ {\isasymin}\ set\ Fs{\isachardot}\ {\isasymA}\ {\isasymTurnstile}\ f{\isacharparenright}{\isachardoublequoteclose}\ \isanewline
\ \ \isakeyword{shows}\ {\isachardoublequoteopen}{\isacharparenleft}{\isasymA}\ {\isasymTurnstile}\ \isactrlbold {\isasymOr}{\isacharparenleft}F{\isacharhash}Fs{\isacharparenright}{\isacharparenright}\ {\isasymlongleftrightarrow}\ {\isacharparenleft}{\isasymexists}f\ {\isasymin}\ set\ {\isacharparenleft}F{\isacharhash}Fs{\isacharparenright}{\isachardot}\ {\isasymA}\ {\isasymTurnstile}\ f{\isacharparenright}{\isachardoublequoteclose}\ \isanewline
%
\isadelimproof
%
\endisadelimproof
%
\isatagproof
\isacommand{proof}\isamarkupfalse%
\ {\isacharminus}\isanewline
\ \ \isacommand{have}\isamarkupfalse%
\ {\isachardoublequoteopen}{\isacharparenleft}{\isasymA}\ {\isasymTurnstile}\ \isactrlbold {\isasymOr}{\isacharparenleft}F{\isacharhash}Fs{\isacharparenright}{\isacharparenright}\ {\isacharequal}\ {\isasymA}\ {\isasymTurnstile}\ F\ \isactrlbold {\isasymor}\ \isactrlbold {\isasymOr}Fs{\isachardoublequoteclose}\isanewline
\ \ \ \ \isacommand{by}\isamarkupfalse%
\ {\isacharparenleft}simp\ only{\isacharcolon}\ BigOr{\isachardot}simps{\isacharparenleft}{\isadigit{2}}{\isacharparenright}{\isacharparenright}\isanewline
\ \ \isacommand{also}\isamarkupfalse%
\ \isacommand{have}\isamarkupfalse%
\ {\isachardoublequoteopen}{\isasymdots}\ {\isacharequal}\ {\isacharparenleft}{\isasymA}\ {\isasymTurnstile}\ F\ {\isasymor}\ {\isasymA}\ {\isasymTurnstile}\ \isactrlbold {\isasymOr}Fs{\isacharparenright}{\isachardoublequoteclose}\isanewline
\ \ \ \ \isacommand{by}\isamarkupfalse%
\ {\isacharparenleft}simp\ only{\isacharcolon}\ formula{\isacharunderscore}semantics{\isachardot}simps{\isacharparenleft}{\isadigit{5}}{\isacharparenright}{\isacharparenright}\isanewline
\ \ \isacommand{also}\isamarkupfalse%
\ \isacommand{have}\isamarkupfalse%
\ {\isachardoublequoteopen}{\isasymdots}\ {\isacharequal}\ {\isacharparenleft}{\isasymA}\ {\isasymTurnstile}\ F\ {\isasymor}\ {\isacharparenleft}{\isasymexists}f\ {\isasymin}\ set\ Fs{\isachardot}\ {\isasymA}\ {\isasymTurnstile}\ f{\isacharparenright}{\isacharparenright}{\isachardoublequoteclose}\isanewline
\ \ \ \ \isacommand{by}\isamarkupfalse%
\ {\isacharparenleft}simp\ only{\isacharcolon}\ assms{\isacharparenright}\isanewline
\ \ \isacommand{also}\isamarkupfalse%
\ \isacommand{have}\isamarkupfalse%
\ {\isachardoublequoteopen}{\isasymdots}\ {\isacharequal}\ {\isacharparenleft}{\isasymexists}f\ {\isasymin}\ {\isacharparenleft}{\isacharbraceleft}F{\isacharbraceright}\ {\isasymunion}\ set\ Fs{\isacharparenright}{\isachardot}\ {\isasymA}\ {\isasymTurnstile}\ f{\isacharparenright}{\isachardoublequoteclose}\isanewline
\ \ \ \ \isacommand{by}\isamarkupfalse%
\ {\isacharparenleft}simp\ only{\isacharcolon}\ Bigprop{\isadigit{2}}{\isacharparenright}\isanewline
\ \ \isacommand{also}\isamarkupfalse%
\ \isacommand{have}\isamarkupfalse%
\ {\isachardoublequoteopen}{\isasymdots}\ {\isacharequal}\ {\isacharparenleft}{\isasymexists}f\ {\isasymin}\ set\ {\isacharparenleft}F{\isacharhash}Fs{\isacharparenright}{\isachardot}\ {\isasymA}\ {\isasymTurnstile}\ f{\isacharparenright}{\isachardoublequoteclose}\isanewline
\ \ \ \ \isacommand{by}\isamarkupfalse%
\ {\isacharparenleft}simp\ only{\isacharcolon}\ set{\isacharunderscore}insert{\isacharparenright}\isanewline
\ \ \isacommand{finally}\isamarkupfalse%
\ \isacommand{show}\isamarkupfalse%
\ {\isacharquery}thesis\isanewline
\ \ \ \ \isacommand{by}\isamarkupfalse%
\ this\isanewline
\isacommand{qed}\isamarkupfalse%
%
\endisatagproof
{\isafoldproof}%
%
\isadelimproof
\isanewline
%
\endisadelimproof
\isanewline
\isacommand{lemma}\isamarkupfalse%
\ {\isachardoublequoteopen}{\isacharparenleft}{\isasymA}\ {\isasymTurnstile}\ \isactrlbold {\isasymOr}Fs{\isacharparenright}\ {\isasymlongleftrightarrow}\ {\isacharparenleft}{\isasymexists}f\ {\isasymin}\ set\ Fs{\isachardot}\ {\isasymA}\ {\isasymTurnstile}\ f{\isacharparenright}{\isachardoublequoteclose}\ \isanewline
%
\isadelimproof
%
\endisadelimproof
%
\isatagproof
\isacommand{proof}\isamarkupfalse%
\ {\isacharparenleft}induction\ Fs{\isacharparenright}\isanewline
\isacommand{case}\isamarkupfalse%
\ Nil\isanewline
\ \ \isacommand{then}\isamarkupfalse%
\ \isacommand{show}\isamarkupfalse%
\ {\isacharquery}case\ \isacommand{by}\isamarkupfalse%
\ {\isacharparenleft}rule\ BigOr{\isacharunderscore}semantics{\isacharunderscore}nil{\isacharparenright}\isanewline
\isacommand{next}\isamarkupfalse%
\isanewline
\ \ \isacommand{case}\isamarkupfalse%
\ {\isacharparenleft}Cons\ a\ Fs{\isacharparenright}\isanewline
\isacommand{then}\isamarkupfalse%
\ \isacommand{show}\isamarkupfalse%
\ {\isacharquery}case\ \isacommand{by}\isamarkupfalse%
\ {\isacharparenleft}rule\ BigOr{\isacharunderscore}semantics{\isacharunderscore}cons{\isacharparenright}\isanewline
\isacommand{qed}\isamarkupfalse%
%
\endisatagproof
{\isafoldproof}%
%
\isadelimproof
\isanewline
%
\endisadelimproof
\isanewline
\isacommand{lemma}\isamarkupfalse%
\ BigOr{\isacharunderscore}semantics{\isacharcolon}\ \isanewline
\ \ {\isachardoublequoteopen}{\isacharparenleft}{\isasymA}\ {\isasymTurnstile}\ \isactrlbold {\isasymOr}Fs{\isacharparenright}\ {\isasymlongleftrightarrow}\ {\isacharparenleft}{\isasymexists}f\ {\isasymin}\ set\ Fs{\isachardot}\ {\isasymA}\ {\isasymTurnstile}\ f{\isacharparenright}{\isachardoublequoteclose}\ \isanewline
%
\isadelimproof
\ \ %
\endisadelimproof
%
\isatagproof
\isacommand{by}\isamarkupfalse%
\ {\isacharparenleft}induction\ Fs{\isacharparenright}\ simp{\isacharunderscore}all%
\endisatagproof
{\isafoldproof}%
%
\isadelimproof
%
\endisadelimproof
%
\isadelimdocument
%
\endisadelimdocument
%
\isatagdocument
%
\isamarkupsection{Semántica de conjuntos de fórmulas%
}
\isamarkuptrue%
%
\endisatagdocument
{\isafolddocument}%
%
\isadelimdocument
%
\endisadelimdocument
%
\begin{isamarkuptext}%
Veamos definiciones y resultados relativos a la semántica de un
  conjunto de fórmulas.
  
  En primer lugar, extendamos la noción de modelo a un conjunto de 
  fórmulas.

  \begin{definicion}
    Una interpretación es modelo de un conjunto de fórmulas si es 
    modelo de todas las fórmulas del conjunto.
  \end{definicion}

  Su formalización en Isabelle es la siguiente.%
\end{isamarkuptext}\isamarkuptrue%
\isacommand{definition}\isamarkupfalse%
\ {\isachardoublequoteopen}isModelSet\ {\isasymA}\ S\ {\isasymequiv}\ {\isasymforall}F{\isachardot}\ {\isacharparenleft}F\ {\isasymin}\ S\ {\isasymlongrightarrow}\ {\isasymA}\ {\isasymTurnstile}\ F{\isacharparenright}{\isachardoublequoteclose}%
\begin{isamarkuptext}%
Continuando con los ejemplos anteriores, veamos una interpretación
  que es modelo de un conjunto de fórmulas.%
\end{isamarkuptext}\isamarkuptrue%
\isacommand{notepad}\isamarkupfalse%
\isanewline
\isakeyword{begin}\isanewline
%
\isadelimproof
\ \ %
\endisadelimproof
%
\isatagproof
\isacommand{fix}\isamarkupfalse%
\ {\isasymA}\ {\isacharcolon}{\isacharcolon}\ {\isachardoublequoteopen}nat\ valuation{\isachardoublequoteclose}\isanewline
\isanewline
\ \ \isacommand{have}\isamarkupfalse%
\ {\isachardoublequoteopen}isModelSet\ {\isacharparenleft}{\isasymA}\ {\isacharparenleft}{\isadigit{4}}\ {\isacharcolon}{\isacharequal}\ True{\isacharparenright}{\isacharparenright}\isanewline
\ \ \ \ \ {\isacharbraceleft}Atom\ {\isadigit{4}}{\isacharcomma}\ {\isacharparenleft}Atom\ {\isadigit{4}}\ \isactrlbold {\isasymand}\ Atom\ {\isadigit{4}}{\isacharparenright}\ \isactrlbold {\isasymrightarrow}\ Atom\ {\isadigit{4}}{\isacharbraceright}{\isachardoublequoteclose}\isanewline
\ \ \ \ \isacommand{by}\isamarkupfalse%
\ {\isacharparenleft}simp\ add{\isacharcolon}\ isModelSet{\isacharunderscore}def{\isacharparenright}%
\endisatagproof
{\isafoldproof}%
%
\isadelimproof
\isanewline
%
\endisadelimproof
\isanewline
\isacommand{end}\isamarkupfalse%
%
\begin{isamarkuptext}%
El siguiente resultado relaciona los conceptos de modelo de 
  una fórmula y modelo de un conjunto de fórmulas en Isabelle.
  La equivalencia se demostrará fácilmente mediante las definiciones
  de \isa{isModel} e \isa{isModelSet}.%
\end{isamarkuptext}\isamarkuptrue%
\isacommand{lemma}\isamarkupfalse%
\ modelSet{\isacharcolon}\isanewline
\ \ {\isachardoublequoteopen}isModelSet\ {\isasymA}\ S\ {\isasymequiv}\ {\isasymforall}F{\isachardot}\ {\isacharparenleft}F\ {\isasymin}\ S\ {\isasymlongrightarrow}\ isModel\ {\isasymA}\ F{\isacharparenright}{\isachardoublequoteclose}\ \isanewline
%
\isadelimproof
\ \ %
\endisadelimproof
%
\isatagproof
\isacommand{by}\isamarkupfalse%
\ {\isacharparenleft}simp\ only{\isacharcolon}\ isModelSet{\isacharunderscore}def\ isModel{\isacharunderscore}def{\isacharparenright}%
\endisatagproof
{\isafoldproof}%
%
\isadelimproof
%
\endisadelimproof
%
\begin{isamarkuptext}%
Veamos la noción de satisfacibilidad para un conjunto de fórmulas.

  \begin{definicion}
    Un conjunto de fórmulas es satisfacible si tiene algún modelo.
  \end{definicion}

  En Isabelle se formaliza de la siguiente manera.%
\end{isamarkuptext}\isamarkuptrue%
\isacommand{definition}\isamarkupfalse%
\ {\isachardoublequoteopen}sat\ S\ {\isasymequiv}\ {\isasymexists}{\isasymA}{\isachardot}\ {\isasymforall}F\ {\isasymin}\ S{\isachardot}\ {\isasymA}\ {\isasymTurnstile}\ F{\isachardoublequoteclose}%
\begin{isamarkuptext}%
En otras palabras, la satisfacibilidad de un conjunto de fórmulas 
  depende de la existencia de una interpretación que sea modelo de dicho 
  conjunto, es decir, que sea modelo de todas las fórmulas del conjunto.
  
  El siguiente lema muestra una forma alternativa de definir
  un conjunto de fórmulas satisfacible en Isabelle empleando 
  \isa{isModelSet} y \isa{sat}, según la observación anterior.%
\end{isamarkuptext}\isamarkuptrue%
\isacommand{lemma}\isamarkupfalse%
\ {\isachardoublequoteopen}sat\ S\ {\isasymequiv}\ {\isasymexists}{\isasymA}{\isachardot}\ isModelSet\ {\isasymA}\ S{\isachardoublequoteclose}\isanewline
%
\isadelimproof
\ \ %
\endisadelimproof
%
\isatagproof
\isacommand{oops}\isamarkupfalse%
%
\endisatagproof
{\isafoldproof}%
%
\isadelimproof
%
\endisadelimproof
%
\begin{isamarkuptext}%
Veamos sus demostraciones detallada y automática. Para ello, 
  introducimos inicialmente el lema auxiliar \isa{forall{\isacharunderscore}set}.%
\end{isamarkuptext}\isamarkuptrue%
\isacommand{lemma}\isamarkupfalse%
\ forall{\isacharunderscore}set{\isacharcolon}\isanewline
\ \ {\isachardoublequoteopen}{\isacharparenleft}{\isasymforall}x{\isachardot}\ {\isacharparenleft}x\ {\isasymin}\ A\ {\isasymlongrightarrow}\ P\ x{\isacharparenright}{\isacharparenright}\ {\isacharequal}\ {\isacharparenleft}{\isasymforall}x\ {\isasymin}\ A{\isachardot}\ P\ x{\isacharparenright}{\isachardoublequoteclose}\isanewline
%
\isadelimproof
%
\endisadelimproof
%
\isatagproof
\isacommand{proof}\isamarkupfalse%
\ {\isacharparenleft}rule\ iffI{\isacharparenright}\isanewline
\ \ \isacommand{assume}\isamarkupfalse%
\ H{\isadigit{1}}{\isacharcolon}{\isachardoublequoteopen}{\isasymforall}x{\isachardot}\ {\isacharparenleft}x\ {\isasymin}\ A\ {\isasymlongrightarrow}\ P\ x{\isacharparenright}{\isachardoublequoteclose}\isanewline
\ \ \isacommand{show}\isamarkupfalse%
\ {\isachardoublequoteopen}{\isasymforall}x\ {\isasymin}\ A{\isachardot}\ P\ x{\isachardoublequoteclose}\isanewline
\ \ \isacommand{proof}\isamarkupfalse%
\ {\isacharparenleft}rule\ ballI{\isacharparenright}\isanewline
\ \ \ \ \isacommand{fix}\isamarkupfalse%
\ x\isanewline
\ \ \ \ \isacommand{have}\isamarkupfalse%
\ {\isachardoublequoteopen}x\ {\isasymin}\ A\ {\isasymlongrightarrow}\ P\ x{\isachardoublequoteclose}\isanewline
\ \ \ \ \ \ \isacommand{using}\isamarkupfalse%
\ H{\isadigit{1}}\ \isacommand{by}\isamarkupfalse%
\ {\isacharparenleft}rule\ allE{\isacharparenright}\isanewline
\ \ \ \ \isacommand{thus}\isamarkupfalse%
\ {\isachardoublequoteopen}x\ {\isasymin}\ A\ {\isasymLongrightarrow}\ P\ x{\isachardoublequoteclose}\isanewline
\ \ \ \ \ \ \isacommand{by}\isamarkupfalse%
\ {\isacharparenleft}rule\ mp{\isacharparenright}\isanewline
\ \ \isacommand{qed}\isamarkupfalse%
\isanewline
\isacommand{next}\isamarkupfalse%
\isanewline
\ \ \isacommand{assume}\isamarkupfalse%
\ H{\isadigit{2}}{\isacharcolon}\ {\isachardoublequoteopen}{\isasymforall}x\ {\isasymin}\ A{\isachardot}\ P\ x{\isachardoublequoteclose}\isanewline
\ \ \isacommand{show}\isamarkupfalse%
\ {\isachardoublequoteopen}{\isasymforall}x{\isachardot}\ {\isacharparenleft}x\ {\isasymin}\ A\ {\isasymlongrightarrow}\ P\ x{\isacharparenright}{\isachardoublequoteclose}\isanewline
\ \ \isacommand{proof}\isamarkupfalse%
\ {\isacharparenleft}rule\ allI{\isacharparenright}\isanewline
\ \ \ \ \isacommand{fix}\isamarkupfalse%
\ x\isanewline
\ \ \ \ \isacommand{show}\isamarkupfalse%
\ {\isachardoublequoteopen}x\ {\isasymin}\ A\ {\isasymlongrightarrow}\ P\ x{\isachardoublequoteclose}\isanewline
\ \ \ \ \isacommand{proof}\isamarkupfalse%
\ {\isacharparenleft}rule\ impI{\isacharparenright}\isanewline
\ \ \ \ \ \ \isacommand{assume}\isamarkupfalse%
\ {\isachardoublequoteopen}x\ {\isasymin}\ A{\isachardoublequoteclose}\isanewline
\ \ \ \ \ \ \isacommand{show}\isamarkupfalse%
\ {\isachardoublequoteopen}P\ x{\isachardoublequoteclose}\ \isanewline
\ \ \ \ \ \ \ \ \isacommand{using}\isamarkupfalse%
\ H{\isadigit{2}}\ {\isacartoucheopen}x\ {\isasymin}\ A{\isacartoucheclose}\ \isacommand{by}\isamarkupfalse%
\ {\isacharparenleft}rule\ bspec{\isacharparenright}\isanewline
\ \ \ \ \isacommand{qed}\isamarkupfalse%
\isanewline
\ \ \isacommand{qed}\isamarkupfalse%
\isanewline
\isacommand{qed}\isamarkupfalse%
%
\endisatagproof
{\isafoldproof}%
%
\isadelimproof
\isanewline
%
\endisadelimproof
\isanewline
\isacommand{lemma}\isamarkupfalse%
\ {\isachardoublequoteopen}sat\ S\ {\isasymequiv}\ {\isasymexists}{\isasymA}{\isachardot}\ isModelSet\ {\isasymA}\ S{\isachardoublequoteclose}\isanewline
%
\isadelimproof
%
\endisadelimproof
%
\isatagproof
\isacommand{proof}\isamarkupfalse%
\ {\isacharminus}\isanewline
\ \ \isacommand{have}\isamarkupfalse%
\ {\isachardoublequoteopen}sat\ S\ {\isacharequal}\ {\isacharparenleft}{\isasymexists}{\isasymA}{\isachardot}\ isModelSet\ {\isasymA}\ S{\isacharparenright}{\isachardoublequoteclose}\isanewline
\ \ \isacommand{proof}\isamarkupfalse%
\ {\isacharparenleft}rule\ iffI{\isacharparenright}\isanewline
\ \ \ \ \isacommand{assume}\isamarkupfalse%
\ H{\isadigit{1}}{\isacharcolon}{\isachardoublequoteopen}{\isasymexists}{\isasymA}{\isachardot}\ isModelSet\ {\isasymA}\ S{\isachardoublequoteclose}\isanewline
\ \ \ \ \isacommand{obtain}\isamarkupfalse%
\ {\isachardoublequoteopen}{\isasymA}{\isachardoublequoteclose}\ \isakeyword{where}\ {\isachardoublequoteopen}isModelSet\ {\isasymA}\ S{\isachardoublequoteclose}\isanewline
\ \ \ \ \ \ \isacommand{using}\isamarkupfalse%
\ H{\isadigit{1}}\ \isacommand{by}\isamarkupfalse%
\ {\isacharparenleft}rule\ exE{\isacharparenright}\isanewline
\ \ \ \ \isacommand{then}\isamarkupfalse%
\ \isacommand{have}\isamarkupfalse%
\ {\isachardoublequoteopen}{\isasymforall}F{\isachardot}\ {\isacharparenleft}F\ {\isasymin}\ S\ {\isasymlongrightarrow}\ {\isasymA}\ {\isasymTurnstile}\ F{\isacharparenright}{\isachardoublequoteclose}\isanewline
\ \ \ \ \ \ \isacommand{by}\isamarkupfalse%
\ {\isacharparenleft}simp\ only{\isacharcolon}\ isModelSet{\isacharunderscore}def{\isacharparenright}\isanewline
\ \ \ \ \isacommand{then}\isamarkupfalse%
\ \isacommand{have}\isamarkupfalse%
\ {\isachardoublequoteopen}{\isasymforall}F\ {\isasymin}\ S{\isachardot}\ {\isasymA}\ {\isasymTurnstile}\ F{\isachardoublequoteclose}\isanewline
\ \ \ \ \ \ \isacommand{by}\isamarkupfalse%
\ {\isacharparenleft}simp\ only{\isacharcolon}\ forall{\isacharunderscore}set{\isacharparenright}\isanewline
\ \ \ \ \isacommand{then}\isamarkupfalse%
\ \isacommand{have}\isamarkupfalse%
\ {\isachardoublequoteopen}{\isasymexists}{\isasymA}{\isachardot}\ {\isasymforall}F\ {\isasymin}\ S{\isachardot}\ {\isasymA}\ {\isasymTurnstile}\ F{\isachardoublequoteclose}\ \isanewline
\ \ \ \ \ \ \isacommand{by}\isamarkupfalse%
\ {\isacharparenleft}simp\ only{\isacharcolon}\ exI{\isacharparenright}\isanewline
\ \ \ \ \isacommand{thus}\isamarkupfalse%
\ {\isachardoublequoteopen}sat\ S{\isachardoublequoteclose}\isanewline
\ \ \ \ \ \ \isacommand{by}\isamarkupfalse%
\ {\isacharparenleft}simp\ only{\isacharcolon}\ sat{\isacharunderscore}def{\isacharparenright}\isanewline
\ \ \isacommand{next}\isamarkupfalse%
\isanewline
\ \ \ \ \isacommand{assume}\isamarkupfalse%
\ {\isachardoublequoteopen}sat\ S{\isachardoublequoteclose}\isanewline
\ \ \ \ \isacommand{then}\isamarkupfalse%
\ \isacommand{have}\isamarkupfalse%
\ H{\isadigit{2}}{\isacharcolon}{\isachardoublequoteopen}{\isasymexists}{\isasymA}{\isachardot}\ {\isasymforall}F\ {\isasymin}\ S{\isachardot}\ {\isasymA}\ {\isasymTurnstile}\ F{\isachardoublequoteclose}\isanewline
\ \ \ \ \ \ \isacommand{by}\isamarkupfalse%
\ {\isacharparenleft}simp\ only{\isacharcolon}\ sat{\isacharunderscore}def{\isacharparenright}\isanewline
\ \ \ \ \isacommand{obtain}\isamarkupfalse%
\ {\isachardoublequoteopen}{\isasymA}{\isachardoublequoteclose}\ \isakeyword{where}\ {\isachardoublequoteopen}{\isasymforall}F\ {\isasymin}\ S{\isachardot}\ {\isasymA}\ {\isasymTurnstile}\ F{\isachardoublequoteclose}\isanewline
\ \ \ \ \ \ \isacommand{using}\isamarkupfalse%
\ H{\isadigit{2}}\ \isacommand{by}\isamarkupfalse%
\ {\isacharparenleft}rule\ exE{\isacharparenright}\isanewline
\ \ \ \ \isacommand{then}\isamarkupfalse%
\ \isacommand{have}\isamarkupfalse%
\ {\isachardoublequoteopen}{\isasymforall}F{\isachardot}\ {\isacharparenleft}F\ {\isasymin}\ S\ {\isasymlongrightarrow}\ {\isasymA}\ {\isasymTurnstile}\ F{\isacharparenright}{\isachardoublequoteclose}\isanewline
\ \ \ \ \ \ \isacommand{by}\isamarkupfalse%
\ {\isacharparenleft}simp\ only{\isacharcolon}\ forall{\isacharunderscore}set{\isacharparenright}\isanewline
\ \ \ \ \isacommand{then}\isamarkupfalse%
\ \isacommand{have}\isamarkupfalse%
\ {\isachardoublequoteopen}isModelSet\ {\isasymA}\ S{\isachardoublequoteclose}\isanewline
\ \ \ \ \ \ \isacommand{by}\isamarkupfalse%
\ {\isacharparenleft}simp\ only{\isacharcolon}\ isModelSet{\isacharunderscore}def{\isacharparenright}\isanewline
\ \ \ \ \isacommand{thus}\isamarkupfalse%
\ {\isachardoublequoteopen}{\isasymexists}{\isasymA}{\isachardot}\ isModelSet\ {\isasymA}\ S{\isachardoublequoteclose}\isanewline
\ \ \ \ \ \ \isacommand{by}\isamarkupfalse%
\ {\isacharparenleft}simp\ only{\isacharcolon}\ exI{\isacharparenright}\isanewline
\ \ \isacommand{qed}\isamarkupfalse%
\isanewline
\ \ \isacommand{thus}\isamarkupfalse%
\ {\isachardoublequoteopen}sat\ S\ \ {\isasymequiv}\ {\isasymexists}{\isasymA}{\isachardot}\ isModelSet\ {\isasymA}\ S{\isachardoublequoteclose}\isanewline
\ \ \ \ \isacommand{by}\isamarkupfalse%
\ {\isacharparenleft}simp\ only{\isacharcolon}\ atomize{\isacharunderscore}eq{\isacharparenright}\isanewline
\isacommand{qed}\isamarkupfalse%
%
\endisatagproof
{\isafoldproof}%
%
\isadelimproof
\isanewline
%
\endisadelimproof
\isanewline
\isacommand{lemma}\isamarkupfalse%
\ satAlt{\isacharcolon}\isanewline
\ {\isachardoublequoteopen}sat\ S\ {\isasymequiv}\ {\isasymexists}{\isasymA}{\isachardot}\ isModelSet\ {\isasymA}\ S{\isachardoublequoteclose}\isanewline
%
\isadelimproof
\ \ %
\endisadelimproof
%
\isatagproof
\isacommand{by}\isamarkupfalse%
\ {\isacharparenleft}smt\ isModelSet{\isacharunderscore}def\ sat{\isacharunderscore}def{\isacharparenright}%
\endisatagproof
{\isafoldproof}%
%
\isadelimproof
%
\endisadelimproof
%
\begin{isamarkuptext}%
Mostremos algunos ejemplos de conjuncto satisfacible. Por 
  definición, se observa que el conjunto de fórmulas utilizado 
  en el ejemplo de \isa{modelSet} es satisfacible. Por otro lado, un 
  caso de conjunto de fórmulas no satisfacible es cualquiera que 
  incluya una contradicción entre sus elementos.
  
  Por otra parte, en particular, se puede definir un conjunto de 
  fórmulas finitamente satisfacible.

  \begin{definicion}
    Un conjunto de fórmulas es finitamente satisfacible si todo
    subconjunto finito suyo es satisfacible.
  \end{definicion}

  Su formalización en Isabelle se muestra a continuación.%
\end{isamarkuptext}\isamarkuptrue%
\isacommand{definition}\isamarkupfalse%
\ {\isachardoublequoteopen}fin{\isacharunderscore}sat\ S\ {\isasymequiv}\ {\isacharparenleft}{\isasymforall}s\ {\isasymsubseteq}\ S{\isachardot}\ finite\ s\ {\isasymlongrightarrow}\ sat\ s{\isacharparenright}{\isachardoublequoteclose}%
\begin{isamarkuptext}%
Continuemos con la noción de consecuencia lógica.

  \begin{definicion}
    Una fórmula es consecuencia lógica de un conjunto de fórmulas si
    todos los modelos del conjunto son modelos de la fórmula.
  \end{definicion}

  Teniendo en cuenta la definición de modelo de una fórmula y modelo de
  un conjunto de fórmulas, su formalización en Isabelle es la 
  siguiente.%
\end{isamarkuptext}\isamarkuptrue%
\isacommand{definition}\isamarkupfalse%
\ entailment\ {\isacharcolon}{\isacharcolon}\ \isanewline
\ \ {\isachardoublequoteopen}{\isacharprime}a\ formula\ set\ {\isasymRightarrow}\ {\isacharprime}a\ formula\ {\isasymRightarrow}\ bool{\isachardoublequoteclose}\ {\isacharparenleft}{\isachardoublequoteopen}{\isacharparenleft}{\isacharunderscore}\ {\isasymTTurnstile}{\isacharslash}\ {\isacharunderscore}{\isacharparenright}{\isachardoublequoteclose}\ \isanewline
\ \ \ \ \ {\isacharbrackleft}{\isadigit{5}}{\isadigit{3}}{\isacharcomma}{\isadigit{5}}{\isadigit{3}}{\isacharbrackright}\ {\isadigit{5}}{\isadigit{3}}{\isacharparenright}\ \isakeyword{where}\isanewline
\ \ {\isachardoublequoteopen}{\isasymGamma}\ {\isasymTTurnstile}\ F\ {\isasymequiv}\ {\isacharparenleft}{\isasymforall}{\isasymA}{\isachardot}\ {\isacharparenleft}{\isacharparenleft}{\isasymforall}G\ {\isasymin}\ {\isasymGamma}{\isachardot}\ {\isasymA}\ {\isasymTurnstile}\ G{\isacharparenright}\ {\isasymlongrightarrow}\ {\isacharparenleft}{\isasymA}\ {\isasymTurnstile}\ F{\isacharparenright}{\isacharparenright}{\isacharparenright}{\isachardoublequoteclose}%
\begin{isamarkuptext}%
Hagamos varias observaciones sobre esta definición. En primer
  lugar, hemos usado el tipo \isa{definition}. Por otro lado, no hemos 
  empleado \isa{isModel} ni \isa{isModelSet} para su\\ definición ya que, de 
  este modo, no tenemos que desplegar dichas definiciones en las 
  demostraciones detalladas y automáticas de los lemas posteriores. 
  Finalmente se puede observar la notación \isa{{\isasymTTurnstile}}. En la teoría clásica no 
  se suele emplear una nueva notación, ya que se diferencia por el 
  contexto. En Isabelle/HOL es imprescindible aclarar la diferencia.

  Mostremos algún ejemplo de fórmula que sea consecuencia lógica de
  un conjunto.%
\end{isamarkuptext}\isamarkuptrue%
\isacommand{notepad}\isamarkupfalse%
\isanewline
\isakeyword{begin}\isanewline
%
\isadelimproof
\ \ %
\endisadelimproof
%
\isatagproof
\isacommand{fix}\isamarkupfalse%
\ {\isasymA}\ {\isacharcolon}{\isacharcolon}\ {\isachardoublequoteopen}nat\ valuation{\isachardoublequoteclose}\isanewline
\isanewline
\ \ \isacommand{have}\isamarkupfalse%
\ {\isachardoublequoteopen}{\isacharbraceleft}Atom\ {\isadigit{1}}\ \isactrlbold {\isasymrightarrow}\ Atom\ {\isadigit{2}}{\isacharcomma}\ Atom\ {\isadigit{2}}\ \isactrlbold {\isasymrightarrow}\ Atom\ {\isadigit{3}}{\isacharbraceright}\ {\isasymTTurnstile}\ {\isacharparenleft}Atom\ {\isadigit{1}}\ \isactrlbold {\isasymrightarrow}\ Atom\ {\isadigit{3}}{\isacharparenright}{\isachardoublequoteclose}\isanewline
\ \ \ \ \isacommand{by}\isamarkupfalse%
\ {\isacharparenleft}simp\ add{\isacharcolon}\ entailment{\isacharunderscore}def{\isacharparenright}%
\endisatagproof
{\isafoldproof}%
%
\isadelimproof
\isanewline
%
\endisadelimproof
\isanewline
\isacommand{end}\isamarkupfalse%
%
\begin{isamarkuptext}%
Llegamos así al último lema de la sección.

  \begin{lema}
    \isa{{\isasymbottom}} es consecuencia lógica de un conjunto si y solo si el conjunto
    es insatisfacible.
  \end{lema}

  Su formalización en Isabelle es la siguiente.%
\end{isamarkuptext}\isamarkuptrue%
\isacommand{lemma}\isamarkupfalse%
\ {\isachardoublequoteopen}{\isasymGamma}\ {\isasymTTurnstile}\ {\isasymbottom}\ {\isasymlongleftrightarrow}\ {\isasymnot}\ sat\ {\isasymGamma}{\isachardoublequoteclose}\ \isanewline
%
\isadelimproof
\ \ %
\endisadelimproof
%
\isatagproof
\isacommand{oops}\isamarkupfalse%
%
\endisatagproof
{\isafoldproof}%
%
\isadelimproof
%
\endisadelimproof
%
\begin{isamarkuptext}%
Comencemos las demostraciones del resultado.

  \begin{demostracion}
    Vamos a probar la doble implicación mediante una cadena de
    equivalencias.

    Sea un conjunto de fórmulas \isa{{\isasymGamma}} tal que la fórmula \isa{{\isasymbottom}} es
    consecuencia lógica de dicho conjunto. Por definición, esto es
    equivalente a que toda interpretación que sea modelo de
    \isa{{\isasymGamma}} es, a su vez, modelo de \isa{{\isasymbottom}}. Como el valor de \isa{{\isasymbottom}} es \isa{Falso} 
    en cualquier interpretación, ninguna interpretación es modelo suyo. 
    Por tanto, por la hipótesis inicial, se verifica que ninguna 
    interpretación es modelo del conjunto \isa{{\isasymGamma}}. Es decir, no existe 
    ninguna interpretación que sea modelo de dicho conjunto. Según la 
    definición, esto es equivalente a que el conjunto \isa{{\isasymGamma}} es 
    insatisfacible, como queríamos demostrar.
  \end{demostracion}

  Procedamos con las pruebas en Isabelle/HOL.%
\end{isamarkuptext}\isamarkuptrue%
\isacommand{lemma}\isamarkupfalse%
\ {\isachardoublequoteopen}{\isasymGamma}\ {\isasymTTurnstile}\ {\isasymbottom}\ {\isasymlongleftrightarrow}\ {\isasymnot}\ sat\ {\isasymGamma}{\isachardoublequoteclose}\ \isanewline
%
\isadelimproof
%
\endisadelimproof
%
\isatagproof
\isacommand{proof}\isamarkupfalse%
\ {\isacharminus}\isanewline
\ \ \isacommand{have}\isamarkupfalse%
\ {\isachardoublequoteopen}{\isasymGamma}\ {\isasymTTurnstile}\ {\isasymbottom}\ {\isacharequal}\ {\isacharparenleft}{\isasymforall}{\isasymA}{\isachardot}\ {\isacharparenleft}{\isacharparenleft}{\isasymforall}G\ {\isasymin}\ {\isasymGamma}{\isachardot}\ {\isasymA}\ {\isasymTurnstile}\ G{\isacharparenright}\ {\isasymlongrightarrow}\ {\isasymA}\ {\isasymTurnstile}\ {\isasymbottom}{\isacharparenright}{\isacharparenright}{\isachardoublequoteclose}\isanewline
\ \ \ \ \isacommand{by}\isamarkupfalse%
\ {\isacharparenleft}simp\ only{\isacharcolon}\ entailment{\isacharunderscore}def{\isacharparenright}\isanewline
\ \ \isacommand{also}\isamarkupfalse%
\ \isacommand{have}\isamarkupfalse%
\ {\isachardoublequoteopen}{\isasymdots}\ {\isacharequal}\ {\isacharparenleft}{\isasymforall}{\isasymA}{\isachardot}\ {\isacharparenleft}{\isacharparenleft}{\isasymforall}G\ {\isasymin}\ {\isasymGamma}{\isachardot}\ {\isasymA}\ {\isasymTurnstile}\ G{\isacharparenright}\ {\isasymlongrightarrow}\ False{\isacharparenright}{\isacharparenright}{\isachardoublequoteclose}\isanewline
\ \ \ \ \isacommand{by}\isamarkupfalse%
\ {\isacharparenleft}simp\ only{\isacharcolon}\ formula{\isacharunderscore}semantics{\isachardot}simps{\isacharparenleft}{\isadigit{2}}{\isacharparenright}{\isacharparenright}\isanewline
\ \ \isacommand{also}\isamarkupfalse%
\ \isacommand{have}\isamarkupfalse%
\ {\isachardoublequoteopen}{\isasymdots}\ {\isacharequal}\ {\isacharparenleft}{\isasymforall}{\isasymA}{\isachardot}\ {\isasymnot}{\isacharparenleft}{\isasymforall}G\ {\isasymin}\ {\isasymGamma}{\isachardot}\ {\isasymA}\ {\isasymTurnstile}\ G{\isacharparenright}{\isacharparenright}{\isachardoublequoteclose}\isanewline
\ \ \ \ \isacommand{by}\isamarkupfalse%
\ {\isacharparenleft}simp\ only{\isacharcolon}\ not{\isacharunderscore}def{\isacharparenright}\isanewline
\ \ \isacommand{also}\isamarkupfalse%
\ \isacommand{have}\isamarkupfalse%
\ {\isachardoublequoteopen}{\isasymdots}\ {\isacharequal}\ \ {\isacharparenleft}{\isasymnot}{\isacharparenleft}{\isasymexists}{\isasymA}{\isachardot}\ {\isasymforall}G\ {\isasymin}\ {\isasymGamma}{\isachardot}\ {\isasymA}\ {\isasymTurnstile}\ G{\isacharparenright}{\isacharparenright}{\isachardoublequoteclose}\isanewline
\ \ \ \ \isacommand{by}\isamarkupfalse%
\ {\isacharparenleft}simp\ only{\isacharcolon}\ not{\isacharunderscore}ex{\isacharparenright}\ \isanewline
\ \ \isacommand{also}\isamarkupfalse%
\ \isacommand{have}\isamarkupfalse%
\ {\isachardoublequoteopen}{\isasymdots}\ {\isacharequal}\ {\isacharparenleft}{\isasymnot}\ sat\ {\isasymGamma}{\isacharparenright}{\isachardoublequoteclose}\isanewline
\ \ \ \ \isacommand{by}\isamarkupfalse%
\ {\isacharparenleft}simp\ only{\isacharcolon}\ sat{\isacharunderscore}def{\isacharparenright}\isanewline
\ \ \isacommand{finally}\isamarkupfalse%
\ \isacommand{show}\isamarkupfalse%
\ {\isacharquery}thesis\isanewline
\ \ \ \ \isacommand{by}\isamarkupfalse%
\ this\isanewline
\isacommand{qed}\isamarkupfalse%
%
\endisatagproof
{\isafoldproof}%
%
\isadelimproof
%
\endisadelimproof
%
\begin{isamarkuptext}%
Finalmente su demostración automática es la siguiente.%
\end{isamarkuptext}\isamarkuptrue%
\isacommand{lemma}\isamarkupfalse%
\ entail{\isacharunderscore}sat{\isacharcolon}\ \isanewline
\ \ {\isachardoublequoteopen}{\isasymGamma}\ {\isasymTTurnstile}\ {\isasymbottom}\ {\isasymlongleftrightarrow}\ {\isasymnot}\ sat\ {\isasymGamma}{\isachardoublequoteclose}\ \isanewline
%
\isadelimproof
\ \ %
\endisadelimproof
%
\isatagproof
\isacommand{unfolding}\isamarkupfalse%
\ sat{\isacharunderscore}def\ entailment{\isacharunderscore}def\ \isanewline
\ \ \isacommand{by}\isamarkupfalse%
\ simp\isanewline
%
\endisatagproof
{\isafoldproof}%
%
\isadelimproof
%
\endisadelimproof
%
\isadelimtheory
%
\endisadelimtheory
%
\isatagtheory
%
\endisatagtheory
{\isafoldtheory}%
%
\isadelimtheory
%
\endisadelimtheory
%
\end{isabellebody}%
\endinput
%:%file=~/TFM/TFM/Semantica.thy%:%
%:%24=8%:%
%:%36=10%:%
%:%37=11%:%
%:%38=12%:%
%:%39=13%:%
%:%40=14%:%
%:%41=15%:%
%:%42=16%:%
%:%43=17%:%
%:%44=18%:%
%:%45=19%:%
%:%46=20%:%
%:%47=21%:%
%:%49=23%:%
%:%50=23%:%
%:%52=25%:%
%:%53=26%:%
%:%54=27%:%
%:%55=28%:%
%:%56=29%:%
%:%57=30%:%
%:%58=31%:%
%:%59=32%:%
%:%60=33%:%
%:%61=34%:%
%:%62=35%:%
%:%63=36%:%
%:%64=37%:%
%:%65=38%:%
%:%66=39%:%
%:%67=40%:%
%:%68=41%:%
%:%69=42%:%
%:%70=43%:%
%:%71=44%:%
%:%72=45%:%
%:%73=46%:%
%:%74=47%:%
%:%75=48%:%
%:%76=49%:%
%:%77=50%:%
%:%78=51%:%
%:%79=52%:%
%:%80=53%:%
%:%81=54%:%
%:%82=55%:%
%:%83=56%:%
%:%84=57%:%
%:%85=58%:%
%:%87=60%:%
%:%88=60%:%
%:%89=61%:%
%:%90=62%:%
%:%91=63%:%
%:%92=64%:%
%:%93=65%:%
%:%94=66%:%
%:%95=67%:%
%:%97=69%:%
%:%98=70%:%
%:%99=71%:%
%:%100=72%:%
%:%101=73%:%
%:%102=74%:%
%:%104=76%:%
%:%105=76%:%
%:%106=77%:%
%:%109=78%:%
%:%113=78%:%
%:%114=78%:%
%:%115=79%:%
%:%116=80%:%
%:%117=80%:%
%:%118=81%:%
%:%119=82%:%
%:%120=82%:%
%:%121=83%:%
%:%122=84%:%
%:%123=84%:%
%:%124=85%:%
%:%125=85%:%
%:%126=86%:%
%:%127=87%:%
%:%128=87%:%
%:%129=88%:%
%:%130=88%:%
%:%131=89%:%
%:%132=90%:%
%:%133=90%:%
%:%134=91%:%
%:%135=91%:%
%:%136=92%:%
%:%137=93%:%
%:%138=93%:%
%:%139=94%:%
%:%140=95%:%
%:%141=95%:%
%:%146=95%:%
%:%149=96%:%
%:%150=97%:%
%:%153=99%:%
%:%154=100%:%
%:%155=101%:%
%:%156=102%:%
%:%157=103%:%
%:%158=104%:%
%:%159=105%:%
%:%160=106%:%
%:%161=107%:%
%:%162=108%:%
%:%163=109%:%
%:%164=110%:%
%:%165=111%:%
%:%166=112%:%
%:%167=113%:%
%:%168=114%:%
%:%169=115%:%
%:%170=116%:%
%:%171=117%:%
%:%172=118%:%
%:%173=119%:%
%:%174=120%:%
%:%175=121%:%
%:%177=123%:%
%:%178=123%:%
%:%180=125%:%
%:%181=126%:%
%:%183=128%:%
%:%184=128%:%
%:%185=129%:%
%:%188=130%:%
%:%192=130%:%
%:%193=130%:%
%:%194=131%:%
%:%195=132%:%
%:%196=132%:%
%:%197=133%:%
%:%198=133%:%
%:%199=134%:%
%:%200=135%:%
%:%201=135%:%
%:%202=136%:%
%:%203=136%:%
%:%204=137%:%
%:%205=138%:%
%:%206=138%:%
%:%207=139%:%
%:%208=139%:%
%:%209=140%:%
%:%210=141%:%
%:%211=141%:%
%:%212=142%:%
%:%213=143%:%
%:%214=143%:%
%:%215=144%:%
%:%216=145%:%
%:%217=145%:%
%:%218=146%:%
%:%219=147%:%
%:%220=147%:%
%:%225=147%:%
%:%228=148%:%
%:%229=149%:%
%:%232=151%:%
%:%233=152%:%
%:%234=153%:%
%:%235=154%:%
%:%236=155%:%
%:%237=156%:%
%:%238=157%:%
%:%240=159%:%
%:%241=159%:%
%:%243=161%:%
%:%244=162%:%
%:%245=163%:%
%:%247=165%:%
%:%248=165%:%
%:%249=166%:%
%:%252=167%:%
%:%256=167%:%
%:%257=167%:%
%:%258=168%:%
%:%259=169%:%
%:%260=169%:%
%:%261=170%:%
%:%262=170%:%
%:%263=171%:%
%:%264=172%:%
%:%265=172%:%
%:%266=173%:%
%:%267=173%:%
%:%268=173%:%
%:%269=174%:%
%:%270=175%:%
%:%271=175%:%
%:%272=176%:%
%:%273=176%:%
%:%274=176%:%
%:%279=176%:%
%:%282=177%:%
%:%283=178%:%
%:%286=180%:%
%:%287=181%:%
%:%288=182%:%
%:%289=183%:%
%:%290=184%:%
%:%291=185%:%
%:%292=186%:%
%:%293=187%:%
%:%294=188%:%
%:%295=189%:%
%:%296=190%:%
%:%297=191%:%
%:%298=192%:%
%:%299=193%:%
%:%300=194%:%
%:%302=196%:%
%:%303=196%:%
%:%304=197%:%
%:%306=199%:%
%:%307=200%:%
%:%308=201%:%
%:%309=202%:%
%:%310=203%:%
%:%311=204%:%
%:%312=205%:%
%:%314=207%:%
%:%315=207%:%
%:%316=208%:%
%:%319=209%:%
%:%323=209%:%
%:%324=209%:%
%:%325=210%:%
%:%326=211%:%
%:%327=211%:%
%:%328=212%:%
%:%329=212%:%
%:%334=212%:%
%:%337=213%:%
%:%338=214%:%
%:%341=216%:%
%:%342=217%:%
%:%343=218%:%
%:%344=219%:%
%:%345=220%:%
%:%346=221%:%
%:%347=222%:%
%:%348=223%:%
%:%349=224%:%
%:%350=225%:%
%:%351=226%:%
%:%352=227%:%
%:%353=228%:%
%:%354=229%:%
%:%355=230%:%
%:%356=231%:%
%:%357=232%:%
%:%358=233%:%
%:%359=234%:%
%:%360=235%:%
%:%361=236%:%
%:%362=237%:%
%:%364=239%:%
%:%365=239%:%
%:%372=240%:%
%:%373=240%:%
%:%374=241%:%
%:%375=241%:%
%:%376=242%:%
%:%377=242%:%
%:%378=243%:%
%:%379=243%:%
%:%380=243%:%
%:%381=244%:%
%:%382=244%:%
%:%383=245%:%
%:%384=245%:%
%:%385=245%:%
%:%386=245%:%
%:%387=246%:%
%:%397=248%:%
%:%399=250%:%
%:%400=250%:%
%:%403=251%:%
%:%407=251%:%
%:%408=251%:%
%:%409=251%:%
%:%418=253%:%
%:%419=254%:%
%:%420=255%:%
%:%421=256%:%
%:%422=257%:%
%:%423=258%:%
%:%424=259%:%
%:%425=260%:%
%:%426=261%:%
%:%427=262%:%
%:%428=263%:%
%:%429=264%:%
%:%431=266%:%
%:%432=266%:%
%:%435=267%:%
%:%439=267%:%
%:%449=269%:%
%:%450=270%:%
%:%451=271%:%
%:%452=272%:%
%:%453=273%:%
%:%454=274%:%
%:%455=275%:%
%:%456=276%:%
%:%457=277%:%
%:%458=278%:%
%:%459=279%:%
%:%460=280%:%
%:%461=281%:%
%:%462=282%:%
%:%463=283%:%
%:%464=284%:%
%:%465=285%:%
%:%466=286%:%
%:%467=287%:%
%:%468=288%:%
%:%469=289%:%
%:%470=290%:%
%:%471=291%:%
%:%472=292%:%
%:%473=293%:%
%:%474=294%:%
%:%475=295%:%
%:%476=296%:%
%:%477=297%:%
%:%478=298%:%
%:%479=299%:%
%:%480=300%:%
%:%481=301%:%
%:%482=302%:%
%:%483=303%:%
%:%484=304%:%
%:%485=305%:%
%:%486=306%:%
%:%487=307%:%
%:%488=308%:%
%:%489=309%:%
%:%490=310%:%
%:%491=311%:%
%:%492=312%:%
%:%493=313%:%
%:%494=314%:%
%:%495=315%:%
%:%496=316%:%
%:%497=317%:%
%:%498=318%:%
%:%499=319%:%
%:%500=320%:%
%:%501=321%:%
%:%502=322%:%
%:%503=323%:%
%:%504=324%:%
%:%505=325%:%
%:%506=326%:%
%:%507=327%:%
%:%508=328%:%
%:%509=329%:%
%:%510=330%:%
%:%511=331%:%
%:%512=332%:%
%:%513=333%:%
%:%514=334%:%
%:%515=335%:%
%:%516=336%:%
%:%517=337%:%
%:%518=338%:%
%:%519=339%:%
%:%520=340%:%
%:%521=341%:%
%:%522=342%:%
%:%523=343%:%
%:%524=344%:%
%:%525=345%:%
%:%526=346%:%
%:%527=347%:%
%:%528=348%:%
%:%529=349%:%
%:%530=350%:%
%:%531=351%:%
%:%532=352%:%
%:%533=353%:%
%:%534=354%:%
%:%535=355%:%
%:%536=356%:%
%:%537=357%:%
%:%538=358%:%
%:%539=359%:%
%:%540=360%:%
%:%541=361%:%
%:%542=362%:%
%:%543=363%:%
%:%544=364%:%
%:%545=365%:%
%:%546=366%:%
%:%547=367%:%
%:%548=368%:%
%:%549=369%:%
%:%550=370%:%
%:%551=371%:%
%:%552=372%:%
%:%553=373%:%
%:%554=374%:%
%:%555=375%:%
%:%556=376%:%
%:%557=377%:%
%:%558=378%:%
%:%559=379%:%
%:%560=380%:%
%:%561=381%:%
%:%562=382%:%
%:%564=384%:%
%:%565=384%:%
%:%566=385%:%
%:%567=386%:%
%:%574=387%:%
%:%575=387%:%
%:%576=388%:%
%:%577=388%:%
%:%578=389%:%
%:%579=389%:%
%:%580=389%:%
%:%581=390%:%
%:%582=390%:%
%:%583=391%:%
%:%584=391%:%
%:%585=391%:%
%:%586=392%:%
%:%587=392%:%
%:%588=393%:%
%:%589=393%:%
%:%590=393%:%
%:%591=394%:%
%:%592=394%:%
%:%593=395%:%
%:%594=395%:%
%:%595=395%:%
%:%596=396%:%
%:%597=396%:%
%:%598=397%:%
%:%604=397%:%
%:%607=398%:%
%:%608=399%:%
%:%609=399%:%
%:%610=400%:%
%:%611=401%:%
%:%618=402%:%
%:%619=402%:%
%:%620=403%:%
%:%621=403%:%
%:%622=404%:%
%:%623=404%:%
%:%624=405%:%
%:%625=405%:%
%:%626=406%:%
%:%627=406%:%
%:%628=407%:%
%:%629=407%:%
%:%630=408%:%
%:%631=408%:%
%:%632=408%:%
%:%633=409%:%
%:%634=409%:%
%:%635=410%:%
%:%636=410%:%
%:%637=411%:%
%:%638=411%:%
%:%639=411%:%
%:%640=412%:%
%:%641=412%:%
%:%642=413%:%
%:%643=413%:%
%:%644=413%:%
%:%645=414%:%
%:%646=414%:%
%:%647=415%:%
%:%653=415%:%
%:%656=416%:%
%:%657=417%:%
%:%658=417%:%
%:%659=418%:%
%:%660=419%:%
%:%663=420%:%
%:%667=420%:%
%:%668=420%:%
%:%673=420%:%
%:%676=421%:%
%:%677=422%:%
%:%678=422%:%
%:%679=423%:%
%:%680=424%:%
%:%687=425%:%
%:%688=425%:%
%:%689=426%:%
%:%690=426%:%
%:%691=427%:%
%:%692=427%:%
%:%693=427%:%
%:%694=428%:%
%:%695=428%:%
%:%696=429%:%
%:%697=429%:%
%:%698=429%:%
%:%699=430%:%
%:%700=430%:%
%:%701=431%:%
%:%707=431%:%
%:%710=432%:%
%:%711=433%:%
%:%712=433%:%
%:%713=434%:%
%:%714=435%:%
%:%715=436%:%
%:%722=437%:%
%:%723=437%:%
%:%724=438%:%
%:%725=438%:%
%:%726=439%:%
%:%727=439%:%
%:%728=439%:%
%:%729=440%:%
%:%730=440%:%
%:%731=440%:%
%:%732=441%:%
%:%733=441%:%
%:%734=442%:%
%:%735=442%:%
%:%736=443%:%
%:%737=443%:%
%:%738=444%:%
%:%739=444%:%
%:%740=444%:%
%:%741=445%:%
%:%742=445%:%
%:%743=446%:%
%:%744=446%:%
%:%745=446%:%
%:%746=447%:%
%:%747=447%:%
%:%748=448%:%
%:%749=448%:%
%:%750=448%:%
%:%751=449%:%
%:%752=449%:%
%:%753=450%:%
%:%759=450%:%
%:%762=451%:%
%:%763=452%:%
%:%764=452%:%
%:%765=453%:%
%:%766=454%:%
%:%773=455%:%
%:%774=455%:%
%:%775=456%:%
%:%776=456%:%
%:%777=457%:%
%:%778=457%:%
%:%779=457%:%
%:%780=458%:%
%:%781=458%:%
%:%782=459%:%
%:%783=459%:%
%:%784=459%:%
%:%785=460%:%
%:%786=460%:%
%:%787=461%:%
%:%788=461%:%
%:%789=461%:%
%:%790=462%:%
%:%791=462%:%
%:%792=463%:%
%:%798=463%:%
%:%801=464%:%
%:%802=465%:%
%:%803=465%:%
%:%804=466%:%
%:%805=467%:%
%:%812=468%:%
%:%813=468%:%
%:%814=469%:%
%:%815=469%:%
%:%816=470%:%
%:%817=470%:%
%:%818=470%:%
%:%819=471%:%
%:%820=471%:%
%:%821=472%:%
%:%822=472%:%
%:%823=472%:%
%:%824=473%:%
%:%825=473%:%
%:%826=474%:%
%:%827=474%:%
%:%828=474%:%
%:%829=475%:%
%:%830=475%:%
%:%831=476%:%
%:%837=476%:%
%:%840=477%:%
%:%841=478%:%
%:%842=478%:%
%:%843=479%:%
%:%844=480%:%
%:%845=481%:%
%:%846=482%:%
%:%853=483%:%
%:%854=483%:%
%:%855=484%:%
%:%856=484%:%
%:%857=485%:%
%:%858=485%:%
%:%859=486%:%
%:%860=486%:%
%:%861=487%:%
%:%862=487%:%
%:%863=487%:%
%:%864=488%:%
%:%865=488%:%
%:%866=489%:%
%:%867=489%:%
%:%868=490%:%
%:%869=490%:%
%:%870=491%:%
%:%871=491%:%
%:%872=492%:%
%:%873=492%:%
%:%874=492%:%
%:%875=493%:%
%:%876=493%:%
%:%877=494%:%
%:%878=494%:%
%:%879=495%:%
%:%880=495%:%
%:%881=496%:%
%:%882=496%:%
%:%883=496%:%
%:%884=497%:%
%:%885=497%:%
%:%886=498%:%
%:%887=498%:%
%:%888=498%:%
%:%889=499%:%
%:%890=499%:%
%:%891=500%:%
%:%892=500%:%
%:%893=500%:%
%:%894=501%:%
%:%895=501%:%
%:%896=502%:%
%:%902=502%:%
%:%905=503%:%
%:%906=504%:%
%:%907=504%:%
%:%908=505%:%
%:%909=506%:%
%:%916=507%:%
%:%917=507%:%
%:%918=508%:%
%:%919=508%:%
%:%920=509%:%
%:%921=509%:%
%:%922=509%:%
%:%923=510%:%
%:%924=510%:%
%:%925=511%:%
%:%926=511%:%
%:%927=511%:%
%:%928=512%:%
%:%929=512%:%
%:%930=513%:%
%:%931=513%:%
%:%932=513%:%
%:%933=514%:%
%:%934=514%:%
%:%935=515%:%
%:%941=515%:%
%:%944=516%:%
%:%945=517%:%
%:%946=517%:%
%:%947=518%:%
%:%948=519%:%
%:%955=520%:%
%:%956=520%:%
%:%957=521%:%
%:%958=521%:%
%:%959=522%:%
%:%960=522%:%
%:%961=522%:%
%:%962=523%:%
%:%963=523%:%
%:%964=524%:%
%:%965=524%:%
%:%966=524%:%
%:%967=525%:%
%:%968=525%:%
%:%969=526%:%
%:%970=526%:%
%:%971=526%:%
%:%972=527%:%
%:%973=527%:%
%:%974=528%:%
%:%980=528%:%
%:%983=529%:%
%:%984=530%:%
%:%985=530%:%
%:%986=531%:%
%:%987=532%:%
%:%988=533%:%
%:%989=534%:%
%:%996=535%:%
%:%997=535%:%
%:%998=536%:%
%:%999=536%:%
%:%1000=537%:%
%:%1001=537%:%
%:%1002=538%:%
%:%1003=538%:%
%:%1004=539%:%
%:%1005=539%:%
%:%1006=539%:%
%:%1007=540%:%
%:%1008=540%:%
%:%1009=541%:%
%:%1010=541%:%
%:%1011=542%:%
%:%1012=542%:%
%:%1013=543%:%
%:%1014=543%:%
%:%1015=544%:%
%:%1016=544%:%
%:%1017=544%:%
%:%1018=545%:%
%:%1019=545%:%
%:%1020=546%:%
%:%1021=546%:%
%:%1022=547%:%
%:%1023=547%:%
%:%1024=548%:%
%:%1025=548%:%
%:%1026=548%:%
%:%1027=549%:%
%:%1028=549%:%
%:%1029=550%:%
%:%1030=550%:%
%:%1031=550%:%
%:%1032=551%:%
%:%1033=551%:%
%:%1034=552%:%
%:%1035=552%:%
%:%1036=552%:%
%:%1037=553%:%
%:%1038=553%:%
%:%1039=554%:%
%:%1045=554%:%
%:%1048=555%:%
%:%1049=556%:%
%:%1050=556%:%
%:%1051=557%:%
%:%1052=558%:%
%:%1059=559%:%
%:%1060=559%:%
%:%1061=560%:%
%:%1062=560%:%
%:%1063=561%:%
%:%1064=561%:%
%:%1065=561%:%
%:%1066=562%:%
%:%1067=562%:%
%:%1068=563%:%
%:%1069=563%:%
%:%1070=563%:%
%:%1071=564%:%
%:%1072=564%:%
%:%1073=565%:%
%:%1074=565%:%
%:%1075=565%:%
%:%1076=566%:%
%:%1077=566%:%
%:%1078=567%:%
%:%1084=567%:%
%:%1087=568%:%
%:%1088=569%:%
%:%1089=569%:%
%:%1090=570%:%
%:%1091=571%:%
%:%1098=572%:%
%:%1099=572%:%
%:%1100=573%:%
%:%1101=573%:%
%:%1102=574%:%
%:%1103=574%:%
%:%1104=574%:%
%:%1105=575%:%
%:%1106=575%:%
%:%1107=576%:%
%:%1108=576%:%
%:%1109=576%:%
%:%1110=577%:%
%:%1111=577%:%
%:%1112=578%:%
%:%1113=578%:%
%:%1114=578%:%
%:%1115=579%:%
%:%1116=579%:%
%:%1117=580%:%
%:%1123=580%:%
%:%1126=581%:%
%:%1127=582%:%
%:%1128=582%:%
%:%1129=583%:%
%:%1130=584%:%
%:%1131=585%:%
%:%1132=586%:%
%:%1139=587%:%
%:%1140=587%:%
%:%1141=588%:%
%:%1142=588%:%
%:%1143=589%:%
%:%1144=589%:%
%:%1145=590%:%
%:%1146=590%:%
%:%1147=591%:%
%:%1148=591%:%
%:%1149=591%:%
%:%1150=592%:%
%:%1151=592%:%
%:%1152=593%:%
%:%1153=593%:%
%:%1154=594%:%
%:%1155=594%:%
%:%1156=595%:%
%:%1157=595%:%
%:%1158=596%:%
%:%1159=596%:%
%:%1160=596%:%
%:%1161=597%:%
%:%1162=597%:%
%:%1163=598%:%
%:%1164=598%:%
%:%1165=599%:%
%:%1166=599%:%
%:%1167=600%:%
%:%1168=600%:%
%:%1169=600%:%
%:%1170=601%:%
%:%1171=601%:%
%:%1172=602%:%
%:%1173=602%:%
%:%1174=602%:%
%:%1175=603%:%
%:%1176=603%:%
%:%1177=604%:%
%:%1178=604%:%
%:%1179=604%:%
%:%1180=605%:%
%:%1181=605%:%
%:%1182=606%:%
%:%1188=606%:%
%:%1191=607%:%
%:%1192=608%:%
%:%1193=608%:%
%:%1200=609%:%
%:%1201=609%:%
%:%1202=610%:%
%:%1203=610%:%
%:%1204=611%:%
%:%1205=611%:%
%:%1206=611%:%
%:%1207=611%:%
%:%1208=612%:%
%:%1209=612%:%
%:%1210=613%:%
%:%1211=613%:%
%:%1212=614%:%
%:%1213=614%:%
%:%1214=614%:%
%:%1215=614%:%
%:%1216=615%:%
%:%1217=615%:%
%:%1218=616%:%
%:%1219=616%:%
%:%1220=617%:%
%:%1221=617%:%
%:%1222=617%:%
%:%1223=617%:%
%:%1224=618%:%
%:%1225=618%:%
%:%1226=619%:%
%:%1227=619%:%
%:%1228=620%:%
%:%1229=620%:%
%:%1230=620%:%
%:%1231=620%:%
%:%1232=621%:%
%:%1233=621%:%
%:%1234=622%:%
%:%1235=622%:%
%:%1236=623%:%
%:%1237=623%:%
%:%1238=623%:%
%:%1239=623%:%
%:%1240=624%:%
%:%1241=624%:%
%:%1242=625%:%
%:%1243=625%:%
%:%1244=626%:%
%:%1245=626%:%
%:%1246=626%:%
%:%1247=626%:%
%:%1248=627%:%
%:%1258=629%:%
%:%1260=631%:%
%:%1261=631%:%
%:%1262=632%:%
%:%1265=633%:%
%:%1269=633%:%
%:%1270=633%:%
%:%1279=635%:%
%:%1280=636%:%
%:%1281=637%:%
%:%1282=638%:%
%:%1283=639%:%
%:%1284=640%:%
%:%1285=641%:%
%:%1286=642%:%
%:%1287=643%:%
%:%1289=645%:%
%:%1290=645%:%
%:%1293=646%:%
%:%1297=646%:%
%:%1307=648%:%
%:%1308=649%:%
%:%1309=650%:%
%:%1310=651%:%
%:%1311=652%:%
%:%1312=653%:%
%:%1313=654%:%
%:%1314=655%:%
%:%1315=656%:%
%:%1316=657%:%
%:%1317=658%:%
%:%1318=659%:%
%:%1319=660%:%
%:%1320=661%:%
%:%1321=662%:%
%:%1322=663%:%
%:%1323=664%:%
%:%1324=665%:%
%:%1325=666%:%
%:%1326=667%:%
%:%1327=668%:%
%:%1328=669%:%
%:%1329=670%:%
%:%1330=671%:%
%:%1331=672%:%
%:%1332=673%:%
%:%1333=674%:%
%:%1334=675%:%
%:%1335=676%:%
%:%1336=677%:%
%:%1337=678%:%
%:%1338=679%:%
%:%1339=680%:%
%:%1340=681%:%
%:%1341=682%:%
%:%1342=683%:%
%:%1343=684%:%
%:%1344=685%:%
%:%1345=686%:%
%:%1346=687%:%
%:%1347=688%:%
%:%1348=689%:%
%:%1349=690%:%
%:%1350=691%:%
%:%1351=692%:%
%:%1352=693%:%
%:%1353=694%:%
%:%1354=695%:%
%:%1355=696%:%
%:%1356=697%:%
%:%1357=698%:%
%:%1358=699%:%
%:%1359=700%:%
%:%1360=701%:%
%:%1361=702%:%
%:%1362=703%:%
%:%1363=704%:%
%:%1364=705%:%
%:%1365=706%:%
%:%1366=707%:%
%:%1367=708%:%
%:%1368=709%:%
%:%1369=710%:%
%:%1370=711%:%
%:%1371=712%:%
%:%1372=713%:%
%:%1373=714%:%
%:%1374=715%:%
%:%1375=716%:%
%:%1376=717%:%
%:%1377=718%:%
%:%1378=719%:%
%:%1379=720%:%
%:%1380=721%:%
%:%1381=722%:%
%:%1382=723%:%
%:%1383=724%:%
%:%1384=725%:%
%:%1385=726%:%
%:%1386=727%:%
%:%1387=728%:%
%:%1388=729%:%
%:%1389=730%:%
%:%1390=731%:%
%:%1391=732%:%
%:%1392=733%:%
%:%1393=734%:%
%:%1394=735%:%
%:%1395=736%:%
%:%1396=737%:%
%:%1397=738%:%
%:%1398=739%:%
%:%1399=740%:%
%:%1400=741%:%
%:%1401=742%:%
%:%1402=743%:%
%:%1403=744%:%
%:%1404=745%:%
%:%1405=746%:%
%:%1406=747%:%
%:%1407=748%:%
%:%1408=749%:%
%:%1409=750%:%
%:%1410=751%:%
%:%1411=752%:%
%:%1412=753%:%
%:%1414=755%:%
%:%1415=755%:%
%:%1416=756%:%
%:%1423=757%:%
%:%1424=757%:%
%:%1425=758%:%
%:%1426=758%:%
%:%1427=759%:%
%:%1428=759%:%
%:%1429=760%:%
%:%1430=760%:%
%:%1431=761%:%
%:%1432=761%:%
%:%1433=762%:%
%:%1434=762%:%
%:%1435=763%:%
%:%1441=763%:%
%:%1444=764%:%
%:%1445=765%:%
%:%1446=765%:%
%:%1447=766%:%
%:%1448=767%:%
%:%1455=768%:%
%:%1456=768%:%
%:%1457=769%:%
%:%1458=769%:%
%:%1459=770%:%
%:%1460=770%:%
%:%1461=771%:%
%:%1462=771%:%
%:%1463=771%:%
%:%1464=772%:%
%:%1465=772%:%
%:%1466=773%:%
%:%1467=773%:%
%:%1468=774%:%
%:%1469=774%:%
%:%1470=774%:%
%:%1471=775%:%
%:%1472=775%:%
%:%1473=776%:%
%:%1474=776%:%
%:%1475=776%:%
%:%1476=777%:%
%:%1477=777%:%
%:%1478=778%:%
%:%1488=780%:%
%:%1489=781%:%
%:%1490=782%:%
%:%1492=784%:%
%:%1493=784%:%
%:%1494=785%:%
%:%1495=786%:%
%:%1498=787%:%
%:%1502=787%:%
%:%1503=787%:%
%:%1508=787%:%
%:%1511=788%:%
%:%1512=789%:%
%:%1513=789%:%
%:%1514=790%:%
%:%1515=791%:%
%:%1516=792%:%
%:%1523=793%:%
%:%1524=793%:%
%:%1525=794%:%
%:%1526=794%:%
%:%1527=794%:%
%:%1528=795%:%
%:%1529=795%:%
%:%1530=796%:%
%:%1531=796%:%
%:%1532=796%:%
%:%1533=797%:%
%:%1534=797%:%
%:%1535=798%:%
%:%1536=798%:%
%:%1537=799%:%
%:%1538=799%:%
%:%1539=800%:%
%:%1540=800%:%
%:%1541=800%:%
%:%1542=801%:%
%:%1543=801%:%
%:%1544=801%:%
%:%1545=802%:%
%:%1546=802%:%
%:%1547=802%:%
%:%1548=803%:%
%:%1549=803%:%
%:%1550=804%:%
%:%1551=804%:%
%:%1552=804%:%
%:%1553=805%:%
%:%1554=805%:%
%:%1555=806%:%
%:%1565=808%:%
%:%1566=809%:%
%:%1567=810%:%
%:%1568=811%:%
%:%1569=812%:%
%:%1570=813%:%
%:%1572=815%:%
%:%1573=815%:%
%:%1574=816%:%
%:%1575=817%:%
%:%1582=818%:%
%:%1583=818%:%
%:%1584=819%:%
%:%1585=819%:%
%:%1586=820%:%
%:%1587=820%:%
%:%1588=821%:%
%:%1589=821%:%
%:%1590=821%:%
%:%1591=822%:%
%:%1592=822%:%
%:%1593=823%:%
%:%1594=823%:%
%:%1595=823%:%
%:%1596=824%:%
%:%1597=824%:%
%:%1598=825%:%
%:%1604=825%:%
%:%1607=826%:%
%:%1608=827%:%
%:%1609=827%:%
%:%1610=828%:%
%:%1611=829%:%
%:%1618=830%:%
%:%1619=830%:%
%:%1620=831%:%
%:%1621=831%:%
%:%1622=832%:%
%:%1623=832%:%
%:%1624=833%:%
%:%1625=833%:%
%:%1626=833%:%
%:%1627=834%:%
%:%1628=834%:%
%:%1629=835%:%
%:%1630=835%:%
%:%1631=835%:%
%:%1632=836%:%
%:%1633=836%:%
%:%1634=837%:%
%:%1644=839%:%
%:%1645=840%:%
%:%1646=841%:%
%:%1648=843%:%
%:%1649=843%:%
%:%1650=844%:%
%:%1651=845%:%
%:%1652=846%:%
%:%1653=847%:%
%:%1660=848%:%
%:%1661=848%:%
%:%1662=849%:%
%:%1663=849%:%
%:%1664=849%:%
%:%1665=850%:%
%:%1666=850%:%
%:%1667=851%:%
%:%1668=851%:%
%:%1669=851%:%
%:%1670=852%:%
%:%1671=852%:%
%:%1672=853%:%
%:%1673=853%:%
%:%1674=853%:%
%:%1675=854%:%
%:%1676=854%:%
%:%1677=855%:%
%:%1678=855%:%
%:%1679=856%:%
%:%1680=856%:%
%:%1681=856%:%
%:%1682=857%:%
%:%1683=857%:%
%:%1684=857%:%
%:%1685=858%:%
%:%1686=858%:%
%:%1687=859%:%
%:%1688=859%:%
%:%1689=860%:%
%:%1690=860%:%
%:%1691=861%:%
%:%1692=861%:%
%:%1693=861%:%
%:%1694=862%:%
%:%1695=862%:%
%:%1696=862%:%
%:%1697=863%:%
%:%1698=863%:%
%:%1699=863%:%
%:%1700=864%:%
%:%1701=864%:%
%:%1702=865%:%
%:%1703=865%:%
%:%1704=865%:%
%:%1705=866%:%
%:%1706=866%:%
%:%1707=867%:%
%:%1713=867%:%
%:%1716=868%:%
%:%1717=869%:%
%:%1718=869%:%
%:%1719=870%:%
%:%1720=871%:%
%:%1721=872%:%
%:%1722=873%:%
%:%1729=874%:%
%:%1730=874%:%
%:%1731=875%:%
%:%1732=875%:%
%:%1733=875%:%
%:%1734=876%:%
%:%1735=876%:%
%:%1736=877%:%
%:%1737=877%:%
%:%1738=877%:%
%:%1739=878%:%
%:%1740=878%:%
%:%1741=879%:%
%:%1742=879%:%
%:%1743=879%:%
%:%1744=880%:%
%:%1745=880%:%
%:%1746=881%:%
%:%1747=881%:%
%:%1748=882%:%
%:%1749=882%:%
%:%1750=882%:%
%:%1751=883%:%
%:%1752=883%:%
%:%1753=883%:%
%:%1754=884%:%
%:%1755=884%:%
%:%1756=885%:%
%:%1757=885%:%
%:%1758=886%:%
%:%1759=886%:%
%:%1760=887%:%
%:%1761=887%:%
%:%1762=887%:%
%:%1763=888%:%
%:%1764=888%:%
%:%1765=888%:%
%:%1766=889%:%
%:%1767=889%:%
%:%1768=889%:%
%:%1769=890%:%
%:%1770=890%:%
%:%1771=891%:%
%:%1772=891%:%
%:%1773=891%:%
%:%1774=892%:%
%:%1775=892%:%
%:%1776=893%:%
%:%1782=893%:%
%:%1785=894%:%
%:%1786=895%:%
%:%1787=895%:%
%:%1788=896%:%
%:%1789=897%:%
%:%1790=898%:%
%:%1791=899%:%
%:%1798=900%:%
%:%1799=900%:%
%:%1800=901%:%
%:%1801=901%:%
%:%1802=901%:%
%:%1803=902%:%
%:%1804=902%:%
%:%1805=903%:%
%:%1806=903%:%
%:%1807=903%:%
%:%1808=904%:%
%:%1809=904%:%
%:%1810=905%:%
%:%1811=905%:%
%:%1812=905%:%
%:%1813=906%:%
%:%1814=906%:%
%:%1815=907%:%
%:%1816=907%:%
%:%1817=908%:%
%:%1818=908%:%
%:%1819=908%:%
%:%1820=909%:%
%:%1821=909%:%
%:%1822=909%:%
%:%1823=910%:%
%:%1824=910%:%
%:%1825=911%:%
%:%1826=911%:%
%:%1827=912%:%
%:%1828=912%:%
%:%1829=913%:%
%:%1830=913%:%
%:%1831=913%:%
%:%1832=914%:%
%:%1833=914%:%
%:%1834=914%:%
%:%1835=915%:%
%:%1836=915%:%
%:%1837=915%:%
%:%1838=916%:%
%:%1839=916%:%
%:%1840=917%:%
%:%1841=917%:%
%:%1842=917%:%
%:%1843=918%:%
%:%1844=918%:%
%:%1845=919%:%
%:%1851=919%:%
%:%1854=920%:%
%:%1855=921%:%
%:%1856=921%:%
%:%1863=922%:%
%:%1864=922%:%
%:%1865=923%:%
%:%1866=923%:%
%:%1867=924%:%
%:%1868=924%:%
%:%1869=924%:%
%:%1870=924%:%
%:%1871=925%:%
%:%1872=925%:%
%:%1873=926%:%
%:%1874=926%:%
%:%1875=927%:%
%:%1876=927%:%
%:%1877=927%:%
%:%1878=927%:%
%:%1879=928%:%
%:%1880=928%:%
%:%1881=929%:%
%:%1882=929%:%
%:%1883=930%:%
%:%1884=930%:%
%:%1885=930%:%
%:%1886=930%:%
%:%1887=931%:%
%:%1888=931%:%
%:%1889=932%:%
%:%1890=932%:%
%:%1891=933%:%
%:%1892=933%:%
%:%1893=933%:%
%:%1894=933%:%
%:%1895=934%:%
%:%1896=934%:%
%:%1897=935%:%
%:%1898=935%:%
%:%1899=936%:%
%:%1900=936%:%
%:%1901=936%:%
%:%1902=936%:%
%:%1903=937%:%
%:%1904=937%:%
%:%1905=938%:%
%:%1906=938%:%
%:%1907=939%:%
%:%1908=939%:%
%:%1909=939%:%
%:%1910=939%:%
%:%1911=940%:%
%:%1921=942%:%
%:%1923=944%:%
%:%1924=944%:%
%:%1925=945%:%
%:%1928=946%:%
%:%1932=946%:%
%:%1933=946%:%
%:%1947=948%:%
%:%1959=950%:%
%:%1960=951%:%
%:%1961=952%:%
%:%1962=953%:%
%:%1963=954%:%
%:%1964=955%:%
%:%1965=956%:%
%:%1966=957%:%
%:%1967=958%:%
%:%1968=959%:%
%:%1970=961%:%
%:%1971=961%:%
%:%1974=962%:%
%:%1978=962%:%
%:%1988=964%:%
%:%1989=965%:%
%:%1990=966%:%
%:%1991=967%:%
%:%1992=968%:%
%:%1993=969%:%
%:%1994=970%:%
%:%1995=971%:%
%:%1996=972%:%
%:%1997=973%:%
%:%1998=974%:%
%:%1999=975%:%
%:%2000=976%:%
%:%2001=977%:%
%:%2002=978%:%
%:%2003=979%:%
%:%2004=980%:%
%:%2005=981%:%
%:%2006=982%:%
%:%2007=983%:%
%:%2008=984%:%
%:%2009=985%:%
%:%2010=986%:%
%:%2011=987%:%
%:%2012=988%:%
%:%2013=989%:%
%:%2014=990%:%
%:%2015=991%:%
%:%2016=992%:%
%:%2017=993%:%
%:%2018=994%:%
%:%2019=995%:%
%:%2020=996%:%
%:%2021=997%:%
%:%2022=998%:%
%:%2023=999%:%
%:%2024=1000%:%
%:%2025=1001%:%
%:%2026=1002%:%
%:%2027=1003%:%
%:%2028=1004%:%
%:%2029=1005%:%
%:%2030=1006%:%
%:%2031=1007%:%
%:%2032=1008%:%
%:%2033=1009%:%
%:%2034=1010%:%
%:%2035=1011%:%
%:%2036=1012%:%
%:%2037=1013%:%
%:%2039=1015%:%
%:%2040=1015%:%
%:%2047=1016%:%
%:%2048=1016%:%
%:%2049=1017%:%
%:%2050=1017%:%
%:%2051=1018%:%
%:%2052=1018%:%
%:%2053=1019%:%
%:%2054=1019%:%
%:%2055=1019%:%
%:%2056=1020%:%
%:%2057=1020%:%
%:%2058=1021%:%
%:%2059=1021%:%
%:%2060=1021%:%
%:%2061=1022%:%
%:%2062=1022%:%
%:%2063=1023%:%
%:%2064=1023%:%
%:%2065=1023%:%
%:%2066=1024%:%
%:%2067=1024%:%
%:%2068=1025%:%
%:%2069=1025%:%
%:%2070=1025%:%
%:%2071=1026%:%
%:%2072=1026%:%
%:%2073=1027%:%
%:%2074=1027%:%
%:%2075=1027%:%
%:%2076=1028%:%
%:%2077=1028%:%
%:%2078=1029%:%
%:%2079=1029%:%
%:%2080=1029%:%
%:%2081=1030%:%
%:%2082=1030%:%
%:%2083=1031%:%
%:%2093=1033%:%
%:%2094=1034%:%
%:%2096=1036%:%
%:%2097=1036%:%
%:%2100=1037%:%
%:%2104=1037%:%
%:%2105=1037%:%
%:%2106=1038%:%
%:%2115=1040%:%
%:%2116=1041%:%
%:%2117=1042%:%
%:%2118=1043%:%
%:%2119=1044%:%
%:%2120=1045%:%
%:%2121=1046%:%
%:%2122=1047%:%
%:%2123=1048%:%
%:%2124=1049%:%
%:%2125=1050%:%
%:%2126=1051%:%
%:%2127=1052%:%
%:%2128=1053%:%
%:%2130=1055%:%
%:%2131=1055%:%
%:%2132=1056%:%
%:%2133=1057%:%
%:%2140=1058%:%
%:%2141=1058%:%
%:%2142=1059%:%
%:%2143=1059%:%
%:%2144=1060%:%
%:%2145=1060%:%
%:%2146=1061%:%
%:%2147=1061%:%
%:%2148=1061%:%
%:%2149=1062%:%
%:%2150=1062%:%
%:%2151=1063%:%
%:%2152=1063%:%
%:%2153=1063%:%
%:%2154=1064%:%
%:%2155=1064%:%
%:%2156=1065%:%
%:%2157=1065%:%
%:%2158=1065%:%
%:%2159=1066%:%
%:%2160=1066%:%
%:%2161=1067%:%
%:%2162=1067%:%
%:%2163=1067%:%
%:%2164=1068%:%
%:%2165=1068%:%
%:%2166=1069%:%
%:%2167=1069%:%
%:%2168=1069%:%
%:%2169=1070%:%
%:%2170=1070%:%
%:%2171=1071%:%
%:%2177=1071%:%
%:%2180=1072%:%
%:%2181=1073%:%
%:%2182=1073%:%
%:%2189=1074%:%
%:%2190=1074%:%
%:%2191=1075%:%
%:%2192=1075%:%
%:%2193=1076%:%
%:%2194=1076%:%
%:%2195=1076%:%
%:%2196=1076%:%
%:%2197=1077%:%
%:%2198=1077%:%
%:%2199=1078%:%
%:%2200=1078%:%
%:%2201=1079%:%
%:%2202=1079%:%
%:%2203=1079%:%
%:%2204=1079%:%
%:%2205=1080%:%
%:%2215=1082%:%
%:%2217=1084%:%
%:%2218=1084%:%
%:%2219=1085%:%
%:%2222=1086%:%
%:%2226=1086%:%
%:%2227=1086%:%
%:%2236=1088%:%
%:%2237=1089%:%
%:%2238=1090%:%
%:%2239=1091%:%
%:%2240=1092%:%
%:%2241=1093%:%
%:%2242=1094%:%
%:%2243=1095%:%
%:%2244=1096%:%
%:%2246=1098%:%
%:%2247=1098%:%
%:%2250=1099%:%
%:%2254=1099%:%
%:%2264=1101%:%
%:%2265=1102%:%
%:%2266=1103%:%
%:%2267=1104%:%
%:%2268=1105%:%
%:%2269=1106%:%
%:%2270=1107%:%
%:%2271=1108%:%
%:%2272=1109%:%
%:%2273=1110%:%
%:%2274=1111%:%
%:%2275=1112%:%
%:%2276=1113%:%
%:%2277=1114%:%
%:%2278=1115%:%
%:%2279=1116%:%
%:%2280=1117%:%
%:%2281=1118%:%
%:%2282=1119%:%
%:%2283=1120%:%
%:%2284=1121%:%
%:%2285=1122%:%
%:%2286=1123%:%
%:%2287=1124%:%
%:%2288=1125%:%
%:%2289=1126%:%
%:%2290=1127%:%
%:%2291=1128%:%
%:%2292=1129%:%
%:%2293=1130%:%
%:%2294=1131%:%
%:%2295=1132%:%
%:%2296=1133%:%
%:%2297=1134%:%
%:%2299=1136%:%
%:%2300=1136%:%
%:%2307=1137%:%
%:%2308=1137%:%
%:%2309=1138%:%
%:%2310=1138%:%
%:%2311=1139%:%
%:%2312=1139%:%
%:%2313=1140%:%
%:%2314=1140%:%
%:%2315=1140%:%
%:%2316=1141%:%
%:%2317=1141%:%
%:%2318=1142%:%
%:%2319=1142%:%
%:%2320=1142%:%
%:%2321=1143%:%
%:%2322=1143%:%
%:%2323=1144%:%
%:%2324=1144%:%
%:%2325=1144%:%
%:%2326=1145%:%
%:%2327=1145%:%
%:%2328=1146%:%
%:%2329=1146%:%
%:%2330=1146%:%
%:%2331=1147%:%
%:%2332=1147%:%
%:%2333=1148%:%
%:%2343=1150%:%
%:%2344=1151%:%
%:%2346=1153%:%
%:%2347=1153%:%
%:%2350=1154%:%
%:%2354=1154%:%
%:%2355=1154%:%
%:%2356=1155%:%
%:%2365=1157%:%
%:%2366=1158%:%
%:%2367=1159%:%
%:%2368=1160%:%
%:%2369=1161%:%
%:%2370=1162%:%
%:%2371=1163%:%
%:%2372=1164%:%
%:%2373=1165%:%
%:%2374=1166%:%
%:%2375=1167%:%
%:%2376=1168%:%
%:%2377=1169%:%
%:%2378=1170%:%
%:%2380=1172%:%
%:%2381=1172%:%
%:%2382=1173%:%
%:%2383=1174%:%
%:%2390=1175%:%
%:%2391=1175%:%
%:%2392=1176%:%
%:%2393=1176%:%
%:%2394=1177%:%
%:%2395=1177%:%
%:%2396=1178%:%
%:%2397=1178%:%
%:%2398=1178%:%
%:%2399=1179%:%
%:%2400=1179%:%
%:%2401=1180%:%
%:%2402=1180%:%
%:%2403=1180%:%
%:%2404=1181%:%
%:%2405=1181%:%
%:%2406=1182%:%
%:%2407=1182%:%
%:%2408=1182%:%
%:%2409=1183%:%
%:%2410=1183%:%
%:%2411=1184%:%
%:%2412=1184%:%
%:%2413=1184%:%
%:%2414=1185%:%
%:%2415=1185%:%
%:%2416=1186%:%
%:%2417=1186%:%
%:%2418=1186%:%
%:%2419=1187%:%
%:%2420=1187%:%
%:%2421=1188%:%
%:%2427=1188%:%
%:%2430=1189%:%
%:%2431=1190%:%
%:%2432=1190%:%
%:%2439=1191%:%
%:%2440=1191%:%
%:%2441=1192%:%
%:%2442=1192%:%
%:%2443=1193%:%
%:%2444=1193%:%
%:%2445=1193%:%
%:%2446=1193%:%
%:%2447=1194%:%
%:%2448=1194%:%
%:%2449=1195%:%
%:%2450=1195%:%
%:%2451=1196%:%
%:%2452=1196%:%
%:%2453=1196%:%
%:%2454=1196%:%
%:%2455=1197%:%
%:%2461=1197%:%
%:%2464=1198%:%
%:%2465=1199%:%
%:%2466=1199%:%
%:%2467=1200%:%
%:%2470=1201%:%
%:%2474=1201%:%
%:%2475=1201%:%
%:%2489=1203%:%
%:%2501=1205%:%
%:%2502=1206%:%
%:%2503=1207%:%
%:%2504=1208%:%
%:%2505=1209%:%
%:%2506=1210%:%
%:%2507=1211%:%
%:%2508=1212%:%
%:%2509=1213%:%
%:%2510=1214%:%
%:%2511=1215%:%
%:%2512=1216%:%
%:%2514=1218%:%
%:%2515=1218%:%
%:%2517=1220%:%
%:%2518=1221%:%
%:%2520=1223%:%
%:%2521=1223%:%
%:%2522=1224%:%
%:%2525=1225%:%
%:%2529=1225%:%
%:%2530=1225%:%
%:%2531=1226%:%
%:%2532=1227%:%
%:%2533=1227%:%
%:%2534=1228%:%
%:%2535=1229%:%
%:%2536=1229%:%
%:%2541=1229%:%
%:%2544=1230%:%
%:%2545=1231%:%
%:%2548=1233%:%
%:%2549=1234%:%
%:%2550=1235%:%
%:%2551=1236%:%
%:%2553=1238%:%
%:%2554=1238%:%
%:%2555=1239%:%
%:%2558=1240%:%
%:%2562=1240%:%
%:%2563=1240%:%
%:%2572=1242%:%
%:%2573=1243%:%
%:%2574=1244%:%
%:%2575=1245%:%
%:%2576=1246%:%
%:%2577=1247%:%
%:%2578=1248%:%
%:%2580=1250%:%
%:%2581=1250%:%
%:%2583=1252%:%
%:%2584=1253%:%
%:%2585=1254%:%
%:%2586=1255%:%
%:%2587=1256%:%
%:%2588=1257%:%
%:%2589=1258%:%
%:%2591=1260%:%
%:%2592=1260%:%
%:%2595=1261%:%
%:%2599=1261%:%
%:%2609=1263%:%
%:%2610=1264%:%
%:%2612=1266%:%
%:%2613=1266%:%
%:%2614=1267%:%
%:%2621=1268%:%
%:%2622=1268%:%
%:%2623=1269%:%
%:%2624=1269%:%
%:%2625=1270%:%
%:%2626=1270%:%
%:%2627=1271%:%
%:%2628=1271%:%
%:%2629=1272%:%
%:%2630=1272%:%
%:%2631=1273%:%
%:%2632=1273%:%
%:%2633=1274%:%
%:%2634=1274%:%
%:%2635=1274%:%
%:%2636=1275%:%
%:%2637=1275%:%
%:%2638=1276%:%
%:%2639=1276%:%
%:%2640=1277%:%
%:%2641=1277%:%
%:%2642=1278%:%
%:%2643=1278%:%
%:%2644=1279%:%
%:%2645=1279%:%
%:%2646=1280%:%
%:%2647=1280%:%
%:%2648=1281%:%
%:%2649=1281%:%
%:%2650=1282%:%
%:%2651=1282%:%
%:%2652=1283%:%
%:%2653=1283%:%
%:%2654=1284%:%
%:%2655=1284%:%
%:%2656=1285%:%
%:%2657=1285%:%
%:%2658=1286%:%
%:%2659=1286%:%
%:%2660=1287%:%
%:%2661=1287%:%
%:%2662=1287%:%
%:%2663=1288%:%
%:%2664=1288%:%
%:%2665=1289%:%
%:%2666=1289%:%
%:%2667=1290%:%
%:%2673=1290%:%
%:%2676=1291%:%
%:%2677=1292%:%
%:%2678=1292%:%
%:%2685=1293%:%
%:%2686=1293%:%
%:%2687=1294%:%
%:%2688=1294%:%
%:%2689=1295%:%
%:%2690=1295%:%
%:%2691=1296%:%
%:%2692=1296%:%
%:%2693=1297%:%
%:%2694=1297%:%
%:%2695=1298%:%
%:%2696=1298%:%
%:%2697=1298%:%
%:%2698=1299%:%
%:%2699=1299%:%
%:%2700=1299%:%
%:%2701=1300%:%
%:%2702=1300%:%
%:%2703=1301%:%
%:%2704=1301%:%
%:%2705=1301%:%
%:%2706=1302%:%
%:%2707=1302%:%
%:%2708=1303%:%
%:%2709=1303%:%
%:%2710=1303%:%
%:%2711=1304%:%
%:%2712=1304%:%
%:%2713=1305%:%
%:%2714=1305%:%
%:%2715=1306%:%
%:%2716=1306%:%
%:%2717=1307%:%
%:%2718=1307%:%
%:%2719=1308%:%
%:%2720=1308%:%
%:%2721=1309%:%
%:%2722=1309%:%
%:%2723=1309%:%
%:%2724=1310%:%
%:%2725=1310%:%
%:%2726=1311%:%
%:%2727=1311%:%
%:%2728=1312%:%
%:%2729=1312%:%
%:%2730=1312%:%
%:%2731=1313%:%
%:%2732=1313%:%
%:%2733=1313%:%
%:%2734=1314%:%
%:%2735=1314%:%
%:%2736=1315%:%
%:%2737=1315%:%
%:%2738=1315%:%
%:%2739=1316%:%
%:%2740=1316%:%
%:%2741=1317%:%
%:%2742=1317%:%
%:%2743=1318%:%
%:%2744=1318%:%
%:%2745=1319%:%
%:%2746=1319%:%
%:%2747=1320%:%
%:%2748=1320%:%
%:%2749=1321%:%
%:%2750=1321%:%
%:%2751=1322%:%
%:%2757=1322%:%
%:%2760=1323%:%
%:%2761=1324%:%
%:%2762=1324%:%
%:%2763=1325%:%
%:%2766=1326%:%
%:%2770=1326%:%
%:%2771=1326%:%
%:%2780=1328%:%
%:%2781=1329%:%
%:%2782=1330%:%
%:%2783=1331%:%
%:%2784=1332%:%
%:%2785=1333%:%
%:%2786=1334%:%
%:%2787=1335%:%
%:%2788=1336%:%
%:%2789=1337%:%
%:%2790=1338%:%
%:%2791=1339%:%
%:%2792=1340%:%
%:%2793=1341%:%
%:%2794=1342%:%
%:%2796=1344%:%
%:%2797=1344%:%
%:%2799=1346%:%
%:%2800=1347%:%
%:%2801=1348%:%
%:%2802=1349%:%
%:%2803=1350%:%
%:%2804=1351%:%
%:%2805=1352%:%
%:%2806=1353%:%
%:%2807=1354%:%
%:%2808=1355%:%
%:%2810=1357%:%
%:%2811=1357%:%
%:%2812=1358%:%
%:%2813=1359%:%
%:%2814=1360%:%
%:%2816=1362%:%
%:%2817=1363%:%
%:%2818=1364%:%
%:%2819=1365%:%
%:%2820=1366%:%
%:%2821=1367%:%
%:%2822=1368%:%
%:%2823=1369%:%
%:%2824=1370%:%
%:%2825=1371%:%
%:%2826=1372%:%
%:%2828=1374%:%
%:%2829=1374%:%
%:%2830=1375%:%
%:%2833=1376%:%
%:%2837=1376%:%
%:%2838=1376%:%
%:%2839=1377%:%
%:%2840=1378%:%
%:%2841=1378%:%
%:%2842=1379%:%
%:%2843=1379%:%
%:%2848=1379%:%
%:%2851=1380%:%
%:%2852=1381%:%
%:%2855=1383%:%
%:%2856=1384%:%
%:%2857=1385%:%
%:%2858=1386%:%
%:%2859=1387%:%
%:%2860=1388%:%
%:%2861=1389%:%
%:%2862=1390%:%
%:%2864=1392%:%
%:%2865=1392%:%
%:%2868=1393%:%
%:%2872=1393%:%
%:%2882=1395%:%
%:%2883=1396%:%
%:%2884=1397%:%
%:%2885=1398%:%
%:%2886=1399%:%
%:%2887=1400%:%
%:%2888=1401%:%
%:%2889=1402%:%
%:%2890=1403%:%
%:%2891=1404%:%
%:%2892=1405%:%
%:%2893=1406%:%
%:%2894=1407%:%
%:%2895=1408%:%
%:%2896=1409%:%
%:%2897=1410%:%
%:%2898=1411%:%
%:%2899=1412%:%
%:%2900=1413%:%
%:%2902=1415%:%
%:%2903=1415%:%
%:%2910=1416%:%
%:%2911=1416%:%
%:%2912=1417%:%
%:%2913=1417%:%
%:%2914=1418%:%
%:%2915=1418%:%
%:%2916=1419%:%
%:%2917=1419%:%
%:%2918=1419%:%
%:%2919=1420%:%
%:%2920=1420%:%
%:%2921=1421%:%
%:%2922=1421%:%
%:%2923=1421%:%
%:%2924=1422%:%
%:%2925=1422%:%
%:%2926=1423%:%
%:%2927=1423%:%
%:%2928=1423%:%
%:%2929=1424%:%
%:%2930=1424%:%
%:%2931=1425%:%
%:%2932=1425%:%
%:%2933=1425%:%
%:%2934=1426%:%
%:%2935=1426%:%
%:%2936=1427%:%
%:%2937=1427%:%
%:%2938=1427%:%
%:%2939=1428%:%
%:%2940=1428%:%
%:%2941=1429%:%
%:%2951=1431%:%
%:%2953=1433%:%
%:%2954=1433%:%
%:%2955=1434%:%
%:%2958=1435%:%
%:%2962=1435%:%
%:%2963=1435%:%
%:%2964=1436%:%
%:%2965=1436%:%

%
\begin{isabellebody}%
\setisabellecontext{Hintikka}%
%
\isadelimtheory
%
\endisadelimtheory
%
\isatagtheory
%
\endisatagtheory
{\isafoldtheory}%
%
\isadelimtheory
%
\endisadelimtheory
%
\isadelimdocument
%
\endisadelimdocument
%
\isatagdocument
%
\isamarkupsection{Conjuntos de Hintikka y propiedades básicas%
}
\isamarkuptrue%
%
\endisatagdocument
{\isafolddocument}%
%
\isadelimdocument
%
\endisadelimdocument
%
\begin{isamarkuptext}%
En esta sección presentaremos un tipo de conjuntos de fórmulas:
  los conjuntos de Hintikka. Probaremos finalmente que todo conjunto 
  de Hintikka es satisfacible.

  \begin{definicion}
  Se llama \isa{conjunto\ de\ Hintikka} a todo conjunto de fórmulas \isa{S} que
  verifica las siguientes condiciones:
    \begin{enumerate}
      \item \isa{{\isasymbottom}\ {\isasymnotin}\ S}.
      \item Dada \isa{p} una fórmula atómica cualquiera, no se tiene 
        simultáneamente que\\ \isa{p\ {\isasymin}\ S} y \isa{{\isasymnot}\ p\ {\isasymin}\ S}.
      \item Si \isa{F\ {\isasymand}\ G\ {\isasymin}\ S}, entonces \isa{F\ {\isasymin}\ S} y \isa{G\ {\isasymin}\ S}.
      \item Si \isa{F\ {\isasymor}\ G\ {\isasymin}\ S}, entonces \isa{F\ {\isasymin}\ S} o \isa{G\ {\isasymin}\ S}.
      \item Si \isa{F\ {\isasymrightarrow}\ G\ {\isasymin}\ S}, entonces \isa{{\isasymnot}\ F\ {\isasymin}\ S} o \isa{G\ {\isasymin}\ S}.
      \item Si \isa{{\isasymnot}{\isacharparenleft}{\isasymnot}\ F{\isacharparenright}\ {\isasymin}\ S}, entonces \isa{F\ {\isasymin}\ S}.
      \item Si \isa{{\isasymnot}{\isacharparenleft}F\ {\isasymand}\ G{\isacharparenright}\ {\isasymin}\ S}, entonces \isa{{\isasymnot}\ F\ {\isasymin}\ S} o \isa{{\isasymnot}\ G\ {\isasymin}\ S}.
      \item Si \isa{{\isasymnot}{\isacharparenleft}F\ {\isasymor}\ G{\isacharparenright}\ {\isasymin}\ S}, entonces \isa{{\isasymnot}\ F\ {\isasymin}\ S} y \isa{{\isasymnot}\ G\ {\isasymin}\ S}. 
      \item Si \isa{{\isasymnot}{\isacharparenleft}F\ {\isasymrightarrow}\ G{\isacharparenright}\ {\isasymin}\ S}, entonces \isa{F\ {\isasymin}\ S} y \isa{{\isasymnot}\ G\ {\isasymin}\ S}. 
    \end{enumerate}  
  \end{definicion}

  En Isabelle se formaliza mediante el tipo \isa{definition} como sigue.%
\end{isamarkuptext}\isamarkuptrue%
\isacommand{definition}\isamarkupfalse%
\ {\isachardoublequoteopen}Hintikka\ S\ {\isasymequiv}\ \isanewline
{\isacharparenleft}{\isasymbottom}\ {\isasymnotin}\ S\isanewline
\ \ {\isasymand}\ {\isacharparenleft}{\isasymforall}k{\isachardot}\ Atom\ k\ {\isasymin}\ S\ {\isasymlongrightarrow}\ \isactrlbold {\isasymnot}\ {\isacharparenleft}Atom\ k{\isacharparenright}\ {\isasymin}\ S\ {\isasymlongrightarrow}\ False{\isacharparenright}\isanewline
\ \ {\isasymand}\ {\isacharparenleft}{\isasymforall}F\ G{\isachardot}\ F\ \isactrlbold {\isasymand}\ G\ {\isasymin}\ S\ {\isasymlongrightarrow}\ F\ {\isasymin}\ S\ {\isasymand}\ G\ {\isasymin}\ S{\isacharparenright}\isanewline
\ \ {\isasymand}\ {\isacharparenleft}{\isasymforall}F\ G{\isachardot}\ F\ \isactrlbold {\isasymor}\ G\ {\isasymin}\ S\ {\isasymlongrightarrow}\ F\ {\isasymin}\ S\ {\isasymor}\ G\ {\isasymin}\ S{\isacharparenright}\isanewline
\ \ {\isasymand}\ {\isacharparenleft}{\isasymforall}F\ G{\isachardot}\ F\ \isactrlbold {\isasymrightarrow}\ G\ {\isasymin}\ S\ {\isasymlongrightarrow}\ \isactrlbold {\isasymnot}F\ {\isasymin}\ S\ {\isasymor}\ G\ {\isasymin}\ S{\isacharparenright}\isanewline
\ \ {\isasymand}\ {\isacharparenleft}{\isasymforall}F{\isachardot}\ \isactrlbold {\isasymnot}\ {\isacharparenleft}\isactrlbold {\isasymnot}F{\isacharparenright}\ {\isasymin}\ S\ {\isasymlongrightarrow}\ F\ {\isasymin}\ S{\isacharparenright}\isanewline
\ \ {\isasymand}\ {\isacharparenleft}{\isasymforall}F\ G{\isachardot}\ \isactrlbold {\isasymnot}{\isacharparenleft}F\ \isactrlbold {\isasymand}\ G{\isacharparenright}\ {\isasymin}\ S\ {\isasymlongrightarrow}\ \isactrlbold {\isasymnot}\ F\ {\isasymin}\ S\ {\isasymor}\ \isactrlbold {\isasymnot}\ G\ {\isasymin}\ S{\isacharparenright}\isanewline
\ \ {\isasymand}\ {\isacharparenleft}{\isasymforall}F\ G{\isachardot}\ \isactrlbold {\isasymnot}{\isacharparenleft}F\ \isactrlbold {\isasymor}\ G{\isacharparenright}\ {\isasymin}\ S\ {\isasymlongrightarrow}\ \isactrlbold {\isasymnot}\ F\ {\isasymin}\ S\ {\isasymand}\ \isactrlbold {\isasymnot}\ G\ {\isasymin}\ S{\isacharparenright}\isanewline
\ \ {\isasymand}\ {\isacharparenleft}{\isasymforall}F\ G{\isachardot}\ \isactrlbold {\isasymnot}{\isacharparenleft}F\ \isactrlbold {\isasymrightarrow}\ G{\isacharparenright}\ {\isasymin}\ S\ {\isasymlongrightarrow}\ F\ {\isasymin}\ S\ {\isasymand}\ \isactrlbold {\isasymnot}\ G\ {\isasymin}\ S{\isacharparenright}{\isacharparenright}{\isachardoublequoteclose}%
\begin{isamarkuptext}%
A continuación ilustramos con un ejemplo de conjunto de fórmulas
  de Hintikka.%
\end{isamarkuptext}\isamarkuptrue%
\isacommand{notepad}\isamarkupfalse%
\isanewline
\isakeyword{begin}\isanewline
%
\isadelimproof
\isanewline
\ \ %
\endisadelimproof
%
\isatagproof
\isacommand{have}\isamarkupfalse%
\ {\isachardoublequoteopen}Hintikka\ {\isacharbraceleft}Atom\ {\isadigit{0}}\ \isactrlbold {\isasymand}\ {\isacharparenleft}{\isacharparenleft}\isactrlbold {\isasymnot}\ {\isacharparenleft}Atom\ {\isadigit{1}}{\isacharparenright}{\isacharparenright}\ \isactrlbold {\isasymrightarrow}\ Atom\ {\isadigit{2}}{\isacharparenright}{\isacharcomma}\ \isanewline
\ \ \ \ \ \ \ \ \ \ \ \ \ \ \ \ \ {\isacharparenleft}{\isacharparenleft}\isactrlbold {\isasymnot}\ {\isacharparenleft}Atom\ {\isadigit{1}}{\isacharparenright}{\isacharparenright}\ \isactrlbold {\isasymrightarrow}\ Atom\ {\isadigit{2}}{\isacharparenright}{\isacharcomma}\ \isanewline
\ \ \ \ \ \ \ \ \ \ \ \ \ \ \ \ \ Atom\ {\isadigit{0}}{\isacharcomma}\ \isactrlbold {\isasymnot}{\isacharparenleft}\isactrlbold {\isasymnot}\ {\isacharparenleft}Atom\ {\isadigit{1}}{\isacharparenright}{\isacharparenright}{\isacharcomma}\ Atom\ {\isadigit{1}}{\isacharbraceright}{\isachardoublequoteclose}\isanewline
\ \ \ \ \isacommand{unfolding}\isamarkupfalse%
\ Hintikka{\isacharunderscore}def\ \isacommand{by}\isamarkupfalse%
\ simp%
\endisatagproof
{\isafoldproof}%
%
\isadelimproof
\isanewline
%
\endisadelimproof
\isanewline
\isacommand{end}\isamarkupfalse%
%
\begin{isamarkuptext}%
En contraposición, el siguiente conjunto de fórmulas no es
  de Hintikka, pues no cumple la segunda condición de la definición 
  anterior.%
\end{isamarkuptext}\isamarkuptrue%
\isacommand{notepad}\isamarkupfalse%
\isanewline
\isakeyword{begin}\isanewline
%
\isadelimproof
\isanewline
\ \ %
\endisadelimproof
%
\isatagproof
\isacommand{have}\isamarkupfalse%
\ {\isachardoublequoteopen}{\isasymnot}\ Hintikka\ {\isacharbraceleft}Atom\ {\isadigit{0}}\ \isactrlbold {\isasymand}\ {\isacharparenleft}{\isacharparenleft}\isactrlbold {\isasymnot}\ {\isacharparenleft}Atom\ {\isadigit{1}}{\isacharparenright}{\isacharparenright}\ \isactrlbold {\isasymrightarrow}\ Atom\ {\isadigit{2}}{\isacharparenright}{\isacharcomma}\ \isanewline
\ \ \ \ \ \ \ \ \ \ \ \ \ \ \ \ \ {\isacharparenleft}{\isacharparenleft}\isactrlbold {\isasymnot}\ {\isacharparenleft}Atom\ {\isadigit{1}}{\isacharparenright}{\isacharparenright}\ \isactrlbold {\isasymrightarrow}\ Atom\ {\isadigit{2}}{\isacharparenright}{\isacharcomma}\ \isanewline
\ \ \ \ \ \ \ \ \ \ \ \ \ \ \ \ \ Atom\ {\isadigit{0}}{\isacharcomma}\ \isactrlbold {\isasymnot}{\isacharparenleft}\isactrlbold {\isasymnot}\ {\isacharparenleft}Atom\ {\isadigit{1}}{\isacharparenright}{\isacharparenright}{\isacharcomma}\ Atom\ {\isadigit{1}}{\isacharcomma}\ \isactrlbold {\isasymnot}{\isacharparenleft}Atom\ {\isadigit{1}}{\isacharparenright}{\isacharbraceright}{\isachardoublequoteclose}\isanewline
\ \ \ \ \isacommand{unfolding}\isamarkupfalse%
\ Hintikka{\isacharunderscore}def\ \isacommand{by}\isamarkupfalse%
\ auto%
\endisatagproof
{\isafoldproof}%
%
\isadelimproof
\isanewline
%
\endisadelimproof
\isanewline
\isacommand{end}\isamarkupfalse%
%
\begin{isamarkuptext}%
En adelante presentaremos una serie de lemas auxiliares
  derivados de la definición de conjunto de Hintikka que nos facilitarán
  posteriormente las demostraciones en Isabelle/HOL.

  El primer lema expresa que la conjunción de las nueve condiciones de 
  la definición anterior es una condición necesaria para que un conjunto 
  sea de Hintikka.%
\end{isamarkuptext}\isamarkuptrue%
\isacommand{lemma}\isamarkupfalse%
\ auxEq{\isacharcolon}\ \isanewline
\ \ \isakeyword{assumes}\ {\isachardoublequoteopen}Hintikka\ S{\isachardoublequoteclose}\isanewline
\ \ \isakeyword{shows}\ {\isachardoublequoteopen}{\isasymbottom}\ {\isasymnotin}\ S\isanewline
\ \ {\isasymand}\ {\isacharparenleft}{\isasymforall}k{\isachardot}\ Atom\ k\ {\isasymin}\ S\ {\isasymlongrightarrow}\ \isactrlbold {\isasymnot}\ {\isacharparenleft}Atom\ k{\isacharparenright}\ {\isasymin}\ S\ {\isasymlongrightarrow}\ False{\isacharparenright}\isanewline
\ \ {\isasymand}\ {\isacharparenleft}{\isasymforall}F\ G{\isachardot}\ F\ \isactrlbold {\isasymand}\ G\ {\isasymin}\ S\ {\isasymlongrightarrow}\ F\ {\isasymin}\ S\ {\isasymand}\ G\ {\isasymin}\ S{\isacharparenright}\isanewline
\ \ {\isasymand}\ {\isacharparenleft}{\isasymforall}F\ G{\isachardot}\ F\ \isactrlbold {\isasymor}\ G\ {\isasymin}\ S\ {\isasymlongrightarrow}\ F\ {\isasymin}\ S\ {\isasymor}\ G\ {\isasymin}\ S{\isacharparenright}\isanewline
\ \ {\isasymand}\ {\isacharparenleft}{\isasymforall}F\ G{\isachardot}\ F\ \isactrlbold {\isasymrightarrow}\ G\ {\isasymin}\ S\ {\isasymlongrightarrow}\ \isactrlbold {\isasymnot}F\ {\isasymin}\ S\ {\isasymor}\ G\ {\isasymin}\ S{\isacharparenright}\isanewline
\ \ {\isasymand}\ {\isacharparenleft}{\isasymforall}F{\isachardot}\ \isactrlbold {\isasymnot}\ {\isacharparenleft}\isactrlbold {\isasymnot}F{\isacharparenright}\ {\isasymin}\ S\ {\isasymlongrightarrow}\ F\ {\isasymin}\ S{\isacharparenright}\isanewline
\ \ {\isasymand}\ {\isacharparenleft}{\isasymforall}F\ G{\isachardot}\ \isactrlbold {\isasymnot}{\isacharparenleft}F\ \isactrlbold {\isasymand}\ G{\isacharparenright}\ {\isasymin}\ S\ {\isasymlongrightarrow}\ \isactrlbold {\isasymnot}\ F\ {\isasymin}\ S\ {\isasymor}\ \isactrlbold {\isasymnot}\ G\ {\isasymin}\ S{\isacharparenright}\isanewline
\ \ {\isasymand}\ {\isacharparenleft}{\isasymforall}F\ G{\isachardot}\ \isactrlbold {\isasymnot}{\isacharparenleft}F\ \isactrlbold {\isasymor}\ G{\isacharparenright}\ {\isasymin}\ S\ {\isasymlongrightarrow}\ \isactrlbold {\isasymnot}\ F\ {\isasymin}\ S\ {\isasymand}\ \isactrlbold {\isasymnot}\ G\ {\isasymin}\ S{\isacharparenright}\isanewline
\ \ {\isasymand}\ {\isacharparenleft}{\isasymforall}F\ G{\isachardot}\ \isactrlbold {\isasymnot}{\isacharparenleft}F\ \isactrlbold {\isasymrightarrow}\ G{\isacharparenright}\ {\isasymin}\ S\ {\isasymlongrightarrow}\ F\ {\isasymin}\ S\ {\isasymand}\ \isactrlbold {\isasymnot}\ G\ {\isasymin}\ S{\isacharparenright}{\isachardoublequoteclose}\isanewline
%
\isadelimproof
%
\endisadelimproof
%
\isatagproof
\isacommand{proof}\isamarkupfalse%
\ {\isacharminus}\isanewline
\ \ \isacommand{have}\isamarkupfalse%
\ {\isachardoublequoteopen}Hintikka\ S\ {\isacharequal}\ {\isacharparenleft}\ {\isasymbottom}\ {\isasymnotin}\ S\isanewline
\ \ {\isasymand}\ {\isacharparenleft}{\isasymforall}k{\isachardot}\ Atom\ k\ {\isasymin}\ S\ {\isasymlongrightarrow}\ \isactrlbold {\isasymnot}\ {\isacharparenleft}Atom\ k{\isacharparenright}\ {\isasymin}\ S\ {\isasymlongrightarrow}\ False{\isacharparenright}\isanewline
\ \ {\isasymand}\ {\isacharparenleft}{\isasymforall}F\ G{\isachardot}\ F\ \isactrlbold {\isasymand}\ G\ {\isasymin}\ S\ {\isasymlongrightarrow}\ F\ {\isasymin}\ S\ {\isasymand}\ G\ {\isasymin}\ S{\isacharparenright}\isanewline
\ \ {\isasymand}\ {\isacharparenleft}{\isasymforall}F\ G{\isachardot}\ F\ \isactrlbold {\isasymor}\ G\ {\isasymin}\ S\ {\isasymlongrightarrow}\ F\ {\isasymin}\ S\ {\isasymor}\ G\ {\isasymin}\ S{\isacharparenright}\isanewline
\ \ {\isasymand}\ {\isacharparenleft}{\isasymforall}F\ G{\isachardot}\ F\ \isactrlbold {\isasymrightarrow}\ G\ {\isasymin}\ S\ {\isasymlongrightarrow}\ \isactrlbold {\isasymnot}F\ {\isasymin}\ S\ {\isasymor}\ G\ {\isasymin}\ S{\isacharparenright}\isanewline
\ \ {\isasymand}\ {\isacharparenleft}{\isasymforall}F{\isachardot}\ \isactrlbold {\isasymnot}\ {\isacharparenleft}\isactrlbold {\isasymnot}F{\isacharparenright}\ {\isasymin}\ S\ {\isasymlongrightarrow}\ F\ {\isasymin}\ S{\isacharparenright}\isanewline
\ \ {\isasymand}\ {\isacharparenleft}{\isasymforall}F\ G{\isachardot}\ \isactrlbold {\isasymnot}{\isacharparenleft}F\ \isactrlbold {\isasymand}\ G{\isacharparenright}\ {\isasymin}\ S\ {\isasymlongrightarrow}\ \isactrlbold {\isasymnot}\ F\ {\isasymin}\ S\ {\isasymor}\ \isactrlbold {\isasymnot}\ G\ {\isasymin}\ S{\isacharparenright}\isanewline
\ \ {\isasymand}\ {\isacharparenleft}{\isasymforall}F\ G{\isachardot}\ \isactrlbold {\isasymnot}{\isacharparenleft}F\ \isactrlbold {\isasymor}\ G{\isacharparenright}\ {\isasymin}\ S\ {\isasymlongrightarrow}\ \isactrlbold {\isasymnot}\ F\ {\isasymin}\ S\ {\isasymand}\ \isactrlbold {\isasymnot}\ G\ {\isasymin}\ S{\isacharparenright}\isanewline
\ \ {\isasymand}\ {\isacharparenleft}{\isasymforall}F\ G{\isachardot}\ \isactrlbold {\isasymnot}{\isacharparenleft}F\ \isactrlbold {\isasymrightarrow}\ G{\isacharparenright}\ {\isasymin}\ S\ {\isasymlongrightarrow}\ F\ {\isasymin}\ S\ {\isasymand}\ \isactrlbold {\isasymnot}\ G\ {\isasymin}\ S{\isacharparenright}{\isacharparenright}{\isachardoublequoteclose}\ \isanewline
\ \ \ \ \isacommand{using}\isamarkupfalse%
\ Hintikka{\isacharunderscore}def\ \isacommand{by}\isamarkupfalse%
\ {\isacharparenleft}simp\ only{\isacharcolon}\ atomize{\isacharunderscore}eq{\isacharparenright}\isanewline
\ \ \isacommand{then}\isamarkupfalse%
\ \isacommand{show}\isamarkupfalse%
\ {\isacharquery}thesis\isanewline
\ \ \ \ \isacommand{using}\isamarkupfalse%
\ assms\ \isacommand{by}\isamarkupfalse%
\ {\isacharparenleft}rule\ iffD{\isadigit{1}}{\isacharparenright}\isanewline
\isacommand{qed}\isamarkupfalse%
%
\endisatagproof
{\isafoldproof}%
%
\isadelimproof
%
\endisadelimproof
%
\begin{isamarkuptext}%
Asimismo presentaremos nueve lemas correspondientes a cada
  condición de la definición de conjunto de Hintikka. 

  \begin{lema}
    Si \isa{S} es un conjunto de Hintikka, \isa{{\isasymbottom}\ {\isasymnotin}\ S}.
  \end{lema}

  \begin{lema}
    Sea \isa{S} un conjunto de Hintikka. Si \isa{p} es una fórmula atómica tal 
    que \isa{p\ {\isasymin}\ S}, entonces \isa{{\isasymnot}\ p\ {\isasymnotin}\ S}.
  \end{lema}

  \begin{lema}
    Sea \isa{S} un conjunto de Hintikka. Si \isa{F\ {\isasymand}\ G\ {\isasymin}\ S}, entonces 
    \isa{F\ {\isasymin}\ S} y \isa{G\ {\isasymin}\ S}.
  \end{lema}

  \begin{lema}
    Sea \isa{S} un conjunto de Hintikka. Si \isa{F\ {\isasymor}\ G\ {\isasymin}\ S}, entonces 
    \isa{F\ {\isasymin}\ S} o \isa{G\ {\isasymin}\ S}.
  \end{lema}

  \begin{lema}
    Sea \isa{S} un conjunto de Hintikka. Si \isa{F\ {\isasymrightarrow}\ G\ {\isasymin}\ S}, entonces 
    \isa{{\isasymnot}\ F\ {\isasymin}\ S} o \isa{G\ {\isasymin}\ S}.
  \end{lema}

  \begin{lema}
    Sea \isa{S} un conjunto de Hintikka. Si \isa{{\isasymnot}{\isacharparenleft}{\isasymnot}\ F{\isacharparenright}\ {\isasymin}\ S}, entonces \isa{F\ {\isasymin}\ S}.
  \end{lema}

  \begin{lema}
    Sea \isa{S} un conjunto de Hintikka. Si \isa{{\isasymnot}{\isacharparenleft}F\ {\isasymand}\ G{\isacharparenright}\ {\isasymin}\ S}, entonces 
    \isa{{\isasymnot}\ F\ {\isasymin}\ S} o \isa{{\isasymnot}\ G\ {\isasymin}\ S}.
  \end{lema}

  \begin{lema}
    Sea \isa{S} un conjunto de Hintikka. Si \isa{{\isasymnot}{\isacharparenleft}F\ {\isasymor}\ G{\isacharparenright}\ {\isasymin}\ S}, entonces 
    \isa{{\isasymnot}\ F\ {\isasymin}\ S} y \isa{{\isasymnot}\ G\ {\isasymin}\ S}.
  \end{lema}

  \begin{lema}
    Sea \isa{S} un conjunto de Hintikka. Si \isa{{\isasymnot}{\isacharparenleft}F\ {\isasymrightarrow}\ G{\isacharparenright}\ {\isasymin}\ S}, entonces 
    \isa{F\ {\isasymin}\ S} y \isa{{\isasymnot}\ G\ {\isasymin}\ S}.
  \end{lema}

  Como se puede observar, los lemas anteriores se corresponden con 
  cada condición necesaria de la definición de conjunto de Hintikka. 
  De este modo, la prueba de estos resultados se obtiene directamente 
  de dicha definición al suponer que \isa{S} es un conjunto de Hintikka.

  Su formalización y demostración en Isabelle/HOL son las siguientes.%
\end{isamarkuptext}\isamarkuptrue%
\isacommand{lemma}\isamarkupfalse%
\isanewline
\ \ \isakeyword{assumes}\ {\isachardoublequoteopen}Hintikka\ S{\isachardoublequoteclose}\ \isanewline
\ \ \isakeyword{shows}\ {\isachardoublequoteopen}{\isasymbottom}\ {\isasymnotin}\ S{\isachardoublequoteclose}\isanewline
%
\isadelimproof
%
\endisadelimproof
%
\isatagproof
\isacommand{proof}\isamarkupfalse%
\ {\isacharminus}\isanewline
\ \isacommand{have}\isamarkupfalse%
\ {\isachardoublequoteopen}{\isasymbottom}\ {\isasymnotin}\ S\isanewline
\ \ {\isasymand}\ {\isacharparenleft}{\isasymforall}k{\isachardot}\ Atom\ k\ {\isasymin}\ S\ {\isasymlongrightarrow}\ \isactrlbold {\isasymnot}\ {\isacharparenleft}Atom\ k{\isacharparenright}\ {\isasymin}\ S\ {\isasymlongrightarrow}\ False{\isacharparenright}\isanewline
\ \ {\isasymand}\ {\isacharparenleft}{\isasymforall}F\ G{\isachardot}\ F\ \isactrlbold {\isasymand}\ G\ {\isasymin}\ S\ {\isasymlongrightarrow}\ F\ {\isasymin}\ S\ {\isasymand}\ G\ {\isasymin}\ S{\isacharparenright}\isanewline
\ \ {\isasymand}\ {\isacharparenleft}{\isasymforall}F\ G{\isachardot}\ F\ \isactrlbold {\isasymor}\ G\ {\isasymin}\ S\ {\isasymlongrightarrow}\ F\ {\isasymin}\ S\ {\isasymor}\ G\ {\isasymin}\ S{\isacharparenright}\isanewline
\ \ {\isasymand}\ {\isacharparenleft}{\isasymforall}F\ G{\isachardot}\ F\ \isactrlbold {\isasymrightarrow}\ G\ {\isasymin}\ S\ {\isasymlongrightarrow}\ \isactrlbold {\isasymnot}F\ {\isasymin}\ S\ {\isasymor}\ G\ {\isasymin}\ S{\isacharparenright}\isanewline
\ \ {\isasymand}\ {\isacharparenleft}{\isasymforall}F{\isachardot}\ \isactrlbold {\isasymnot}\ {\isacharparenleft}\isactrlbold {\isasymnot}F{\isacharparenright}\ {\isasymin}\ S\ {\isasymlongrightarrow}\ F\ {\isasymin}\ S{\isacharparenright}\isanewline
\ \ {\isasymand}\ {\isacharparenleft}{\isasymforall}F\ G{\isachardot}\ \isactrlbold {\isasymnot}{\isacharparenleft}F\ \isactrlbold {\isasymand}\ G{\isacharparenright}\ {\isasymin}\ S\ {\isasymlongrightarrow}\ \isactrlbold {\isasymnot}\ F\ {\isasymin}\ S\ {\isasymor}\ \isactrlbold {\isasymnot}\ G\ {\isasymin}\ S{\isacharparenright}\isanewline
\ \ {\isasymand}\ {\isacharparenleft}{\isasymforall}F\ G{\isachardot}\ \isactrlbold {\isasymnot}{\isacharparenleft}F\ \isactrlbold {\isasymor}\ G{\isacharparenright}\ {\isasymin}\ S\ {\isasymlongrightarrow}\ \isactrlbold {\isasymnot}\ F\ {\isasymin}\ S\ {\isasymand}\ \isactrlbold {\isasymnot}\ G\ {\isasymin}\ S{\isacharparenright}\isanewline
\ \ {\isasymand}\ {\isacharparenleft}{\isasymforall}F\ G{\isachardot}\ \isactrlbold {\isasymnot}{\isacharparenleft}F\ \isactrlbold {\isasymrightarrow}\ G{\isacharparenright}\ {\isasymin}\ S\ {\isasymlongrightarrow}\ F\ {\isasymin}\ S\ {\isasymand}\ \isactrlbold {\isasymnot}\ G\ {\isasymin}\ S{\isacharparenright}{\isachardoublequoteclose}\isanewline
\ \ \ \ \isacommand{using}\isamarkupfalse%
\ assms\ \isacommand{by}\isamarkupfalse%
\ {\isacharparenleft}rule\ auxEq{\isacharparenright}\isanewline
\ \ \isacommand{thus}\isamarkupfalse%
\ {\isachardoublequoteopen}{\isasymbottom}\ {\isasymnotin}\ S{\isachardoublequoteclose}\ \isacommand{by}\isamarkupfalse%
\ {\isacharparenleft}rule\ conjunct{\isadigit{1}}{\isacharparenright}\isanewline
\isacommand{qed}\isamarkupfalse%
%
\endisatagproof
{\isafoldproof}%
%
\isadelimproof
\isanewline
%
\endisadelimproof
\isanewline
\isacommand{lemma}\isamarkupfalse%
\ Hintikka{\isacharunderscore}l{\isadigit{1}}{\isacharcolon}\ \isanewline
\ {\isachardoublequoteopen}Hintikka\ S\ {\isasymLongrightarrow}\ {\isasymbottom}\ {\isasymnotin}\ S{\isachardoublequoteclose}\isanewline
%
\isadelimproof
\ \ %
\endisadelimproof
%
\isatagproof
\isacommand{using}\isamarkupfalse%
\ Hintikka{\isacharunderscore}def\ \isacommand{by}\isamarkupfalse%
\ blast%
\endisatagproof
{\isafoldproof}%
%
\isadelimproof
\isanewline
%
\endisadelimproof
\isanewline
\isacommand{lemma}\isamarkupfalse%
\isanewline
\ \ \isakeyword{assumes}\ {\isachardoublequoteopen}Hintikka\ S{\isachardoublequoteclose}\ \isanewline
\ \ \isakeyword{shows}\ {\isachardoublequoteopen}Atom\ k\ {\isasymin}\ S\ {\isasymlongrightarrow}\ \isactrlbold {\isasymnot}\ {\isacharparenleft}Atom\ k{\isacharparenright}\ {\isasymnotin}\ S{\isachardoublequoteclose}\isanewline
%
\isadelimproof
%
\endisadelimproof
%
\isatagproof
\isacommand{proof}\isamarkupfalse%
\ {\isacharparenleft}rule\ impI{\isacharparenright}\isanewline
\ \ \isacommand{assume}\isamarkupfalse%
\ H{\isacharcolon}{\isachardoublequoteopen}Atom\ k\ {\isasymin}\ S{\isachardoublequoteclose}\isanewline
\ \isacommand{have}\isamarkupfalse%
\ {\isachardoublequoteopen}{\isasymbottom}\ {\isasymnotin}\ S\isanewline
\ \ {\isasymand}\ {\isacharparenleft}{\isasymforall}k{\isachardot}\ Atom\ k\ {\isasymin}\ S\ {\isasymlongrightarrow}\ \isactrlbold {\isasymnot}\ {\isacharparenleft}Atom\ k{\isacharparenright}\ {\isasymin}\ S\ {\isasymlongrightarrow}\ False{\isacharparenright}\isanewline
\ \ {\isasymand}\ {\isacharparenleft}{\isasymforall}F\ G{\isachardot}\ F\ \isactrlbold {\isasymand}\ G\ {\isasymin}\ S\ {\isasymlongrightarrow}\ F\ {\isasymin}\ S\ {\isasymand}\ G\ {\isasymin}\ S{\isacharparenright}\isanewline
\ \ {\isasymand}\ {\isacharparenleft}{\isasymforall}F\ G{\isachardot}\ F\ \isactrlbold {\isasymor}\ G\ {\isasymin}\ S\ {\isasymlongrightarrow}\ F\ {\isasymin}\ S\ {\isasymor}\ G\ {\isasymin}\ S{\isacharparenright}\isanewline
\ \ {\isasymand}\ {\isacharparenleft}{\isasymforall}F\ G{\isachardot}\ F\ \isactrlbold {\isasymrightarrow}\ G\ {\isasymin}\ S\ {\isasymlongrightarrow}\ \isactrlbold {\isasymnot}F\ {\isasymin}\ S\ {\isasymor}\ G\ {\isasymin}\ S{\isacharparenright}\isanewline
\ \ {\isasymand}\ {\isacharparenleft}{\isasymforall}F{\isachardot}\ \isactrlbold {\isasymnot}\ {\isacharparenleft}\isactrlbold {\isasymnot}F{\isacharparenright}\ {\isasymin}\ S\ {\isasymlongrightarrow}\ F\ {\isasymin}\ S{\isacharparenright}\isanewline
\ \ {\isasymand}\ {\isacharparenleft}{\isasymforall}F\ G{\isachardot}\ \isactrlbold {\isasymnot}{\isacharparenleft}F\ \isactrlbold {\isasymand}\ G{\isacharparenright}\ {\isasymin}\ S\ {\isasymlongrightarrow}\ \isactrlbold {\isasymnot}\ F\ {\isasymin}\ S\ {\isasymor}\ \isactrlbold {\isasymnot}\ G\ {\isasymin}\ S{\isacharparenright}\isanewline
\ \ {\isasymand}\ {\isacharparenleft}{\isasymforall}F\ G{\isachardot}\ \isactrlbold {\isasymnot}{\isacharparenleft}F\ \isactrlbold {\isasymor}\ G{\isacharparenright}\ {\isasymin}\ S\ {\isasymlongrightarrow}\ \isactrlbold {\isasymnot}\ F\ {\isasymin}\ S\ {\isasymand}\ \isactrlbold {\isasymnot}\ G\ {\isasymin}\ S{\isacharparenright}\isanewline
\ \ {\isasymand}\ {\isacharparenleft}{\isasymforall}F\ G{\isachardot}\ \isactrlbold {\isasymnot}{\isacharparenleft}F\ \isactrlbold {\isasymrightarrow}\ G{\isacharparenright}\ {\isasymin}\ S\ {\isasymlongrightarrow}\ F\ {\isasymin}\ S\ {\isasymand}\ \isactrlbold {\isasymnot}\ G\ {\isasymin}\ S{\isacharparenright}{\isachardoublequoteclose}\isanewline
\ \ \ \isacommand{using}\isamarkupfalse%
\ assms\ \isacommand{by}\isamarkupfalse%
\ {\isacharparenleft}rule\ auxEq{\isacharparenright}\isanewline
\ \ \isacommand{then}\isamarkupfalse%
\ \isacommand{have}\isamarkupfalse%
\ {\isachardoublequoteopen}{\isacharparenleft}{\isasymforall}k{\isachardot}\ Atom\ k\ {\isasymin}\ S\ {\isasymlongrightarrow}\ \isactrlbold {\isasymnot}\ {\isacharparenleft}Atom\ k{\isacharparenright}\ {\isasymin}\ S\ {\isasymlongrightarrow}\ False{\isacharparenright}{\isachardoublequoteclose}\isanewline
\ \ \ \ \isacommand{by}\isamarkupfalse%
\ {\isacharparenleft}iprover\ elim{\isacharcolon}\ conjunct{\isadigit{2}}\ conjunct{\isadigit{1}}{\isacharparenright}\isanewline
\ \ \isacommand{then}\isamarkupfalse%
\ \isacommand{have}\isamarkupfalse%
\ {\isachardoublequoteopen}{\isasymforall}k{\isachardot}\ Atom\ k\ {\isasymin}\ S\ {\isasymlongrightarrow}\ \isactrlbold {\isasymnot}\ {\isacharparenleft}Atom\ k{\isacharparenright}\ {\isasymnotin}\ S{\isachardoublequoteclose}\isanewline
\ \ \ \ \isacommand{by}\isamarkupfalse%
\ {\isacharparenleft}simp\ only{\isacharcolon}\ not{\isacharunderscore}def{\isacharparenright}\isanewline
\ \ \isacommand{then}\isamarkupfalse%
\ \isacommand{have}\isamarkupfalse%
\ {\isachardoublequoteopen}Atom\ k\ {\isasymin}\ S\ {\isasymlongrightarrow}\ \isactrlbold {\isasymnot}\ {\isacharparenleft}Atom\ k{\isacharparenright}\ {\isasymnotin}\ S{\isachardoublequoteclose}\isanewline
\ \ \ \ \isacommand{by}\isamarkupfalse%
\ {\isacharparenleft}rule\ allE{\isacharparenright}\isanewline
\ \ \isacommand{thus}\isamarkupfalse%
\ {\isachardoublequoteopen}\isactrlbold {\isasymnot}\ {\isacharparenleft}Atom\ k{\isacharparenright}\ {\isasymnotin}\ S{\isachardoublequoteclose}\isanewline
\ \ \ \ \isacommand{using}\isamarkupfalse%
\ H\ \isacommand{by}\isamarkupfalse%
\ {\isacharparenleft}rule\ mp{\isacharparenright}\isanewline
\isacommand{qed}\isamarkupfalse%
%
\endisatagproof
{\isafoldproof}%
%
\isadelimproof
\isanewline
%
\endisadelimproof
\isanewline
\isacommand{lemma}\isamarkupfalse%
\ Hintikka{\isacharunderscore}l{\isadigit{2}}{\isacharcolon}\ \isanewline
\ {\isachardoublequoteopen}Hintikka\ S\ {\isasymLongrightarrow}\ \ {\isacharparenleft}Atom\ k\ {\isasymin}\ S\ {\isasymlongrightarrow}\ \isactrlbold {\isasymnot}\ {\isacharparenleft}Atom\ k{\isacharparenright}\ {\isasymnotin}\ S{\isacharparenright}{\isachardoublequoteclose}\isanewline
%
\isadelimproof
\ \ %
\endisadelimproof
%
\isatagproof
\isacommand{by}\isamarkupfalse%
\ {\isacharparenleft}smt\ Hintikka{\isacharunderscore}def{\isacharparenright}%
\endisatagproof
{\isafoldproof}%
%
\isadelimproof
\isanewline
%
\endisadelimproof
\isanewline
\isacommand{lemma}\isamarkupfalse%
\ \isanewline
\ \ \isakeyword{assumes}\ {\isachardoublequoteopen}Hintikka\ S{\isachardoublequoteclose}\isanewline
\ \ \isakeyword{shows}\ {\isachardoublequoteopen}F\ \isactrlbold {\isasymand}\ G\ {\isasymin}\ S\ {\isasymlongrightarrow}\ F\ {\isasymin}\ S\ {\isasymand}\ G\ {\isasymin}\ S{\isachardoublequoteclose}\isanewline
%
\isadelimproof
%
\endisadelimproof
%
\isatagproof
\isacommand{proof}\isamarkupfalse%
\ {\isacharparenleft}rule\ impI{\isacharparenright}\isanewline
\ \ \isacommand{assume}\isamarkupfalse%
\ {\isachardoublequoteopen}F\ \isactrlbold {\isasymand}\ G\ {\isasymin}\ S{\isachardoublequoteclose}\isanewline
\ \isacommand{have}\isamarkupfalse%
\ {\isachardoublequoteopen}{\isasymbottom}\ {\isasymnotin}\ S\isanewline
\ \ {\isasymand}\ {\isacharparenleft}{\isasymforall}k{\isachardot}\ Atom\ k\ {\isasymin}\ S\ {\isasymlongrightarrow}\ \isactrlbold {\isasymnot}\ {\isacharparenleft}Atom\ k{\isacharparenright}\ {\isasymin}\ S\ {\isasymlongrightarrow}\ False{\isacharparenright}\isanewline
\ \ {\isasymand}\ {\isacharparenleft}{\isasymforall}F\ G{\isachardot}\ F\ \isactrlbold {\isasymand}\ G\ {\isasymin}\ S\ {\isasymlongrightarrow}\ F\ {\isasymin}\ S\ {\isasymand}\ G\ {\isasymin}\ S{\isacharparenright}\isanewline
\ \ {\isasymand}\ {\isacharparenleft}{\isasymforall}F\ G{\isachardot}\ F\ \isactrlbold {\isasymor}\ G\ {\isasymin}\ S\ {\isasymlongrightarrow}\ F\ {\isasymin}\ S\ {\isasymor}\ G\ {\isasymin}\ S{\isacharparenright}\isanewline
\ \ {\isasymand}\ {\isacharparenleft}{\isasymforall}F\ G{\isachardot}\ F\ \isactrlbold {\isasymrightarrow}\ G\ {\isasymin}\ S\ {\isasymlongrightarrow}\ \isactrlbold {\isasymnot}F\ {\isasymin}\ S\ {\isasymor}\ G\ {\isasymin}\ S{\isacharparenright}\isanewline
\ \ {\isasymand}\ {\isacharparenleft}{\isasymforall}F{\isachardot}\ \isactrlbold {\isasymnot}\ {\isacharparenleft}\isactrlbold {\isasymnot}F{\isacharparenright}\ {\isasymin}\ S\ {\isasymlongrightarrow}\ F\ {\isasymin}\ S{\isacharparenright}\isanewline
\ \ {\isasymand}\ {\isacharparenleft}{\isasymforall}F\ G{\isachardot}\ \isactrlbold {\isasymnot}{\isacharparenleft}F\ \isactrlbold {\isasymand}\ G{\isacharparenright}\ {\isasymin}\ S\ {\isasymlongrightarrow}\ \isactrlbold {\isasymnot}\ F\ {\isasymin}\ S\ {\isasymor}\ \isactrlbold {\isasymnot}\ G\ {\isasymin}\ S{\isacharparenright}\isanewline
\ \ {\isasymand}\ {\isacharparenleft}{\isasymforall}F\ G{\isachardot}\ \isactrlbold {\isasymnot}{\isacharparenleft}F\ \isactrlbold {\isasymor}\ G{\isacharparenright}\ {\isasymin}\ S\ {\isasymlongrightarrow}\ \isactrlbold {\isasymnot}\ F\ {\isasymin}\ S\ {\isasymand}\ \isactrlbold {\isasymnot}\ G\ {\isasymin}\ S{\isacharparenright}\isanewline
\ \ {\isasymand}\ {\isacharparenleft}{\isasymforall}F\ G{\isachardot}\ \isactrlbold {\isasymnot}{\isacharparenleft}F\ \isactrlbold {\isasymrightarrow}\ G{\isacharparenright}\ {\isasymin}\ S\ {\isasymlongrightarrow}\ F\ {\isasymin}\ S\ {\isasymand}\ \isactrlbold {\isasymnot}\ G\ {\isasymin}\ S{\isacharparenright}{\isachardoublequoteclose}\isanewline
\ \ \ \isacommand{using}\isamarkupfalse%
\ assms\ \isacommand{by}\isamarkupfalse%
\ {\isacharparenleft}rule\ auxEq{\isacharparenright}\isanewline
\ \ \isacommand{then}\isamarkupfalse%
\ \isacommand{have}\isamarkupfalse%
\ {\isachardoublequoteopen}{\isasymforall}F\ G{\isachardot}\ F\ \isactrlbold {\isasymand}\ G\ {\isasymin}\ S\ {\isasymlongrightarrow}\ F\ {\isasymin}\ S\ {\isasymand}\ G\ {\isasymin}\ S{\isachardoublequoteclose}\isanewline
\ \ \ \ \isacommand{by}\isamarkupfalse%
\ {\isacharparenleft}iprover\ elim{\isacharcolon}\ conjunct{\isadigit{2}}\ conjunct{\isadigit{1}}{\isacharparenright}\ \isanewline
\ \ \isacommand{then}\isamarkupfalse%
\ \isacommand{have}\isamarkupfalse%
\ {\isachardoublequoteopen}F\ \isactrlbold {\isasymand}\ G\ {\isasymin}\ S\ {\isasymlongrightarrow}\ F\ {\isasymin}\ S\ {\isasymand}\ G\ {\isasymin}\ S{\isachardoublequoteclose}\isanewline
\ \ \ \ \isacommand{by}\isamarkupfalse%
\ {\isacharparenleft}iprover\ elim{\isacharcolon}\ allE{\isacharparenright}\isanewline
\ \ \isacommand{thus}\isamarkupfalse%
\ {\isachardoublequoteopen}F\ {\isasymin}\ S\ {\isasymand}\ G\ {\isasymin}\ S{\isachardoublequoteclose}\isanewline
\ \ \ \ \isacommand{using}\isamarkupfalse%
\ {\isacartoucheopen}F\ \isactrlbold {\isasymand}\ G\ {\isasymin}\ S{\isacartoucheclose}\ \isacommand{by}\isamarkupfalse%
\ {\isacharparenleft}rule\ mp{\isacharparenright}\isanewline
\isacommand{qed}\isamarkupfalse%
%
\endisatagproof
{\isafoldproof}%
%
\isadelimproof
\isanewline
%
\endisadelimproof
\isanewline
\isacommand{lemma}\isamarkupfalse%
\ Hintikka{\isacharunderscore}l{\isadigit{3}}{\isacharcolon}\ \isanewline
\ {\isachardoublequoteopen}Hintikka\ S\ {\isasymLongrightarrow}\ \ {\isacharparenleft}F\ \isactrlbold {\isasymand}\ G\ {\isasymin}\ S\ {\isasymlongrightarrow}\ F\ {\isasymin}\ S\ {\isasymand}\ G\ {\isasymin}\ S{\isacharparenright}{\isachardoublequoteclose}\isanewline
%
\isadelimproof
\ \ %
\endisadelimproof
%
\isatagproof
\isacommand{by}\isamarkupfalse%
\ {\isacharparenleft}smt\ Hintikka{\isacharunderscore}def{\isacharparenright}%
\endisatagproof
{\isafoldproof}%
%
\isadelimproof
\isanewline
%
\endisadelimproof
\isanewline
\isacommand{lemma}\isamarkupfalse%
\isanewline
\ \ \isakeyword{assumes}\ {\isachardoublequoteopen}Hintikka\ S{\isachardoublequoteclose}\isanewline
\ \ \isakeyword{shows}\ {\isachardoublequoteopen}F\ \isactrlbold {\isasymor}\ G\ {\isasymin}\ S\ {\isasymlongrightarrow}\ F\ {\isasymin}\ S\ {\isasymor}\ G\ {\isasymin}\ S{\isachardoublequoteclose}\isanewline
%
\isadelimproof
%
\endisadelimproof
%
\isatagproof
\isacommand{proof}\isamarkupfalse%
\ {\isacharparenleft}rule\ impI{\isacharparenright}\isanewline
\ \ \isacommand{assume}\isamarkupfalse%
\ H{\isacharcolon}{\isachardoublequoteopen}F\ \isactrlbold {\isasymor}\ G\ {\isasymin}\ S{\isachardoublequoteclose}\isanewline
\ \isacommand{have}\isamarkupfalse%
\ {\isachardoublequoteopen}{\isasymbottom}\ {\isasymnotin}\ S\isanewline
\ \ {\isasymand}\ {\isacharparenleft}{\isasymforall}k{\isachardot}\ Atom\ k\ {\isasymin}\ S\ {\isasymlongrightarrow}\ \isactrlbold {\isasymnot}\ {\isacharparenleft}Atom\ k{\isacharparenright}\ {\isasymin}\ S\ {\isasymlongrightarrow}\ False{\isacharparenright}\isanewline
\ \ {\isasymand}\ {\isacharparenleft}{\isasymforall}F\ G{\isachardot}\ F\ \isactrlbold {\isasymand}\ G\ {\isasymin}\ S\ {\isasymlongrightarrow}\ F\ {\isasymin}\ S\ {\isasymand}\ G\ {\isasymin}\ S{\isacharparenright}\isanewline
\ \ {\isasymand}\ {\isacharparenleft}{\isasymforall}F\ G{\isachardot}\ F\ \isactrlbold {\isasymor}\ G\ {\isasymin}\ S\ {\isasymlongrightarrow}\ F\ {\isasymin}\ S\ {\isasymor}\ G\ {\isasymin}\ S{\isacharparenright}\isanewline
\ \ {\isasymand}\ {\isacharparenleft}{\isasymforall}F\ G{\isachardot}\ F\ \isactrlbold {\isasymrightarrow}\ G\ {\isasymin}\ S\ {\isasymlongrightarrow}\ \isactrlbold {\isasymnot}F\ {\isasymin}\ S\ {\isasymor}\ G\ {\isasymin}\ S{\isacharparenright}\isanewline
\ \ {\isasymand}\ {\isacharparenleft}{\isasymforall}F{\isachardot}\ \isactrlbold {\isasymnot}\ {\isacharparenleft}\isactrlbold {\isasymnot}F{\isacharparenright}\ {\isasymin}\ S\ {\isasymlongrightarrow}\ F\ {\isasymin}\ S{\isacharparenright}\isanewline
\ \ {\isasymand}\ {\isacharparenleft}{\isasymforall}F\ G{\isachardot}\ \isactrlbold {\isasymnot}{\isacharparenleft}F\ \isactrlbold {\isasymand}\ G{\isacharparenright}\ {\isasymin}\ S\ {\isasymlongrightarrow}\ \isactrlbold {\isasymnot}\ F\ {\isasymin}\ S\ {\isasymor}\ \isactrlbold {\isasymnot}\ G\ {\isasymin}\ S{\isacharparenright}\isanewline
\ \ {\isasymand}\ {\isacharparenleft}{\isasymforall}F\ G{\isachardot}\ \isactrlbold {\isasymnot}{\isacharparenleft}F\ \isactrlbold {\isasymor}\ G{\isacharparenright}\ {\isasymin}\ S\ {\isasymlongrightarrow}\ \isactrlbold {\isasymnot}\ F\ {\isasymin}\ S\ {\isasymand}\ \isactrlbold {\isasymnot}\ G\ {\isasymin}\ S{\isacharparenright}\isanewline
\ \ {\isasymand}\ {\isacharparenleft}{\isasymforall}F\ G{\isachardot}\ \isactrlbold {\isasymnot}{\isacharparenleft}F\ \isactrlbold {\isasymrightarrow}\ G{\isacharparenright}\ {\isasymin}\ S\ {\isasymlongrightarrow}\ F\ {\isasymin}\ S\ {\isasymand}\ \isactrlbold {\isasymnot}\ G\ {\isasymin}\ S{\isacharparenright}{\isachardoublequoteclose}\isanewline
\ \ \ \isacommand{using}\isamarkupfalse%
\ assms\ \isacommand{by}\isamarkupfalse%
\ {\isacharparenleft}rule\ auxEq{\isacharparenright}\isanewline
\ \ \isacommand{then}\isamarkupfalse%
\ \isacommand{have}\isamarkupfalse%
\ {\isachardoublequoteopen}{\isasymforall}F\ G{\isachardot}\ F\ \isactrlbold {\isasymor}\ G\ {\isasymin}\ S\ {\isasymlongrightarrow}\ F\ {\isasymin}\ S\ {\isasymor}\ G\ {\isasymin}\ S{\isachardoublequoteclose}\ \isanewline
\ \ \ \ \isacommand{by}\isamarkupfalse%
\ {\isacharparenleft}iprover\ elim{\isacharcolon}\ conjunct{\isadigit{2}}\ conjunct{\isadigit{1}}{\isacharparenright}\isanewline
\ \ \isacommand{then}\isamarkupfalse%
\ \isacommand{have}\isamarkupfalse%
\ {\isachardoublequoteopen}F\ \isactrlbold {\isasymor}\ G\ {\isasymin}\ S\ {\isasymlongrightarrow}\ F\ {\isasymin}\ S\ {\isasymor}\ G\ {\isasymin}\ S{\isachardoublequoteclose}\isanewline
\ \ \ \ \isacommand{by}\isamarkupfalse%
\ {\isacharparenleft}iprover\ elim{\isacharcolon}\ allE{\isacharparenright}\isanewline
\ \ \ \ \isacommand{thus}\isamarkupfalse%
\ {\isachardoublequoteopen}F\ {\isasymin}\ S\ {\isasymor}\ G\ {\isasymin}\ S{\isachardoublequoteclose}\isanewline
\ \ \ \ \ \ \isacommand{using}\isamarkupfalse%
\ H\ \isacommand{by}\isamarkupfalse%
\ {\isacharparenleft}rule\ mp{\isacharparenright}\isanewline
\ \ \isacommand{qed}\isamarkupfalse%
%
\endisatagproof
{\isafoldproof}%
%
\isadelimproof
\isanewline
%
\endisadelimproof
\isanewline
\isacommand{lemma}\isamarkupfalse%
\ Hintikka{\isacharunderscore}l{\isadigit{4}}{\isacharcolon}\ \isanewline
\ {\isachardoublequoteopen}Hintikka\ S\ {\isasymLongrightarrow}\ \ {\isacharparenleft}F\ \isactrlbold {\isasymor}\ G\ {\isasymin}\ S\ {\isasymlongrightarrow}\ F\ {\isasymin}\ S\ {\isasymor}\ G\ {\isasymin}\ S{\isacharparenright}{\isachardoublequoteclose}\isanewline
%
\isadelimproof
\ \ %
\endisadelimproof
%
\isatagproof
\isacommand{by}\isamarkupfalse%
\ {\isacharparenleft}smt\ Hintikka{\isacharunderscore}def{\isacharparenright}%
\endisatagproof
{\isafoldproof}%
%
\isadelimproof
\isanewline
%
\endisadelimproof
\isanewline
\isacommand{lemma}\isamarkupfalse%
\isanewline
\ \ \isakeyword{assumes}\ {\isachardoublequoteopen}Hintikka\ S{\isachardoublequoteclose}\ \isanewline
\ \ \isakeyword{shows}\ {\isachardoublequoteopen}F\ \isactrlbold {\isasymrightarrow}\ G\ {\isasymin}\ S\ {\isasymlongrightarrow}\ \isactrlbold {\isasymnot}F\ {\isasymin}\ S\ {\isasymor}\ G\ {\isasymin}\ S{\isachardoublequoteclose}\isanewline
%
\isadelimproof
%
\endisadelimproof
%
\isatagproof
\isacommand{proof}\isamarkupfalse%
\ {\isacharparenleft}rule\ impI{\isacharparenright}\isanewline
\ \ \isacommand{assume}\isamarkupfalse%
\ H{\isacharcolon}{\isachardoublequoteopen}F\ \isactrlbold {\isasymrightarrow}\ G\ {\isasymin}\ S{\isachardoublequoteclose}\isanewline
\ \isacommand{have}\isamarkupfalse%
\ {\isachardoublequoteopen}{\isasymbottom}\ {\isasymnotin}\ S\isanewline
\ \ {\isasymand}\ {\isacharparenleft}{\isasymforall}k{\isachardot}\ Atom\ k\ {\isasymin}\ S\ {\isasymlongrightarrow}\ \isactrlbold {\isasymnot}\ {\isacharparenleft}Atom\ k{\isacharparenright}\ {\isasymin}\ S\ {\isasymlongrightarrow}\ False{\isacharparenright}\isanewline
\ \ {\isasymand}\ {\isacharparenleft}{\isasymforall}F\ G{\isachardot}\ F\ \isactrlbold {\isasymand}\ G\ {\isasymin}\ S\ {\isasymlongrightarrow}\ F\ {\isasymin}\ S\ {\isasymand}\ G\ {\isasymin}\ S{\isacharparenright}\isanewline
\ \ {\isasymand}\ {\isacharparenleft}{\isasymforall}F\ G{\isachardot}\ F\ \isactrlbold {\isasymor}\ G\ {\isasymin}\ S\ {\isasymlongrightarrow}\ F\ {\isasymin}\ S\ {\isasymor}\ G\ {\isasymin}\ S{\isacharparenright}\isanewline
\ \ {\isasymand}\ {\isacharparenleft}{\isasymforall}F\ G{\isachardot}\ F\ \isactrlbold {\isasymrightarrow}\ G\ {\isasymin}\ S\ {\isasymlongrightarrow}\ \isactrlbold {\isasymnot}F\ {\isasymin}\ S\ {\isasymor}\ G\ {\isasymin}\ S{\isacharparenright}\isanewline
\ \ {\isasymand}\ {\isacharparenleft}{\isasymforall}F{\isachardot}\ \isactrlbold {\isasymnot}\ {\isacharparenleft}\isactrlbold {\isasymnot}F{\isacharparenright}\ {\isasymin}\ S\ {\isasymlongrightarrow}\ F\ {\isasymin}\ S{\isacharparenright}\isanewline
\ \ {\isasymand}\ {\isacharparenleft}{\isasymforall}F\ G{\isachardot}\ \isactrlbold {\isasymnot}{\isacharparenleft}F\ \isactrlbold {\isasymand}\ G{\isacharparenright}\ {\isasymin}\ S\ {\isasymlongrightarrow}\ \isactrlbold {\isasymnot}\ F\ {\isasymin}\ S\ {\isasymor}\ \isactrlbold {\isasymnot}\ G\ {\isasymin}\ S{\isacharparenright}\isanewline
\ \ {\isasymand}\ {\isacharparenleft}{\isasymforall}F\ G{\isachardot}\ \isactrlbold {\isasymnot}{\isacharparenleft}F\ \isactrlbold {\isasymor}\ G{\isacharparenright}\ {\isasymin}\ S\ {\isasymlongrightarrow}\ \isactrlbold {\isasymnot}\ F\ {\isasymin}\ S\ {\isasymand}\ \isactrlbold {\isasymnot}\ G\ {\isasymin}\ S{\isacharparenright}\isanewline
\ \ {\isasymand}\ {\isacharparenleft}{\isasymforall}F\ G{\isachardot}\ \isactrlbold {\isasymnot}{\isacharparenleft}F\ \isactrlbold {\isasymrightarrow}\ G{\isacharparenright}\ {\isasymin}\ S\ {\isasymlongrightarrow}\ F\ {\isasymin}\ S\ {\isasymand}\ \isactrlbold {\isasymnot}\ G\ {\isasymin}\ S{\isacharparenright}{\isachardoublequoteclose}\isanewline
\ \ \ \isacommand{using}\isamarkupfalse%
\ assms\ \isacommand{by}\isamarkupfalse%
\ {\isacharparenleft}rule\ auxEq{\isacharparenright}\isanewline
\ \ \isacommand{then}\isamarkupfalse%
\ \isacommand{have}\isamarkupfalse%
\ {\isachardoublequoteopen}{\isasymforall}F\ G{\isachardot}\ F\ \isactrlbold {\isasymrightarrow}\ G\ {\isasymin}\ S\ {\isasymlongrightarrow}\ \isactrlbold {\isasymnot}F\ {\isasymin}\ S\ {\isasymor}\ G\ {\isasymin}\ S{\isachardoublequoteclose}\isanewline
\ \ \ \ \isacommand{by}\isamarkupfalse%
\ {\isacharparenleft}iprover\ elim{\isacharcolon}\ conjunct{\isadigit{2}}\ conjunct{\isadigit{1}}{\isacharparenright}\isanewline
\ \ \isacommand{then}\isamarkupfalse%
\ \isacommand{have}\isamarkupfalse%
\ {\isachardoublequoteopen}F\ \isactrlbold {\isasymrightarrow}\ G\ {\isasymin}\ S\ {\isasymlongrightarrow}\ \isactrlbold {\isasymnot}F\ {\isasymin}\ S\ {\isasymor}\ G\ {\isasymin}\ S{\isachardoublequoteclose}\isanewline
\ \ \ \ \isacommand{by}\isamarkupfalse%
\ {\isacharparenleft}iprover\ elim{\isacharcolon}\ allE{\isacharparenright}\isanewline
\ \ \isacommand{thus}\isamarkupfalse%
\ {\isachardoublequoteopen}\isactrlbold {\isasymnot}F\ {\isasymin}\ S\ {\isasymor}\ G\ {\isasymin}\ S{\isachardoublequoteclose}\isanewline
\ \ \ \ \isacommand{using}\isamarkupfalse%
\ H\ \isacommand{by}\isamarkupfalse%
\ {\isacharparenleft}rule\ mp{\isacharparenright}\isanewline
\isacommand{qed}\isamarkupfalse%
%
\endisatagproof
{\isafoldproof}%
%
\isadelimproof
\isanewline
%
\endisadelimproof
\isanewline
\isacommand{lemma}\isamarkupfalse%
\ Hintikka{\isacharunderscore}l{\isadigit{5}}{\isacharcolon}\ \isanewline
\ {\isachardoublequoteopen}Hintikka\ S\ {\isasymLongrightarrow}\ \ \ {\isacharparenleft}F\ \isactrlbold {\isasymrightarrow}\ G\ {\isasymin}\ S\ {\isasymlongrightarrow}\ \isactrlbold {\isasymnot}F\ {\isasymin}\ S\ {\isasymor}\ G\ {\isasymin}\ S{\isacharparenright}{\isachardoublequoteclose}\isanewline
%
\isadelimproof
\ \ %
\endisadelimproof
%
\isatagproof
\isacommand{by}\isamarkupfalse%
\ {\isacharparenleft}smt\ Hintikka{\isacharunderscore}def{\isacharparenright}%
\endisatagproof
{\isafoldproof}%
%
\isadelimproof
\isanewline
%
\endisadelimproof
\isanewline
\isacommand{lemma}\isamarkupfalse%
\isanewline
\ \ \isakeyword{assumes}\ {\isachardoublequoteopen}Hintikka\ S{\isachardoublequoteclose}\isanewline
\ \ \isakeyword{shows}\ {\isachardoublequoteopen}{\isacharparenleft}\isactrlbold {\isasymnot}\ {\isacharparenleft}\isactrlbold {\isasymnot}F{\isacharparenright}\ {\isasymin}\ S\ {\isasymlongrightarrow}\ F\ {\isasymin}\ S{\isacharparenright}{\isachardoublequoteclose}\isanewline
%
\isadelimproof
%
\endisadelimproof
%
\isatagproof
\isacommand{proof}\isamarkupfalse%
\ {\isacharparenleft}rule\ impI{\isacharparenright}\isanewline
\ \ \isacommand{assume}\isamarkupfalse%
\ H{\isacharcolon}{\isachardoublequoteopen}\isactrlbold {\isasymnot}\ {\isacharparenleft}\isactrlbold {\isasymnot}F{\isacharparenright}\ {\isasymin}\ S{\isachardoublequoteclose}\isanewline
\ \isacommand{have}\isamarkupfalse%
\ {\isachardoublequoteopen}{\isasymbottom}\ {\isasymnotin}\ S\isanewline
\ \ {\isasymand}\ {\isacharparenleft}{\isasymforall}k{\isachardot}\ Atom\ k\ {\isasymin}\ S\ {\isasymlongrightarrow}\ \isactrlbold {\isasymnot}\ {\isacharparenleft}Atom\ k{\isacharparenright}\ {\isasymin}\ S\ {\isasymlongrightarrow}\ False{\isacharparenright}\isanewline
\ \ {\isasymand}\ {\isacharparenleft}{\isasymforall}F\ G{\isachardot}\ F\ \isactrlbold {\isasymand}\ G\ {\isasymin}\ S\ {\isasymlongrightarrow}\ F\ {\isasymin}\ S\ {\isasymand}\ G\ {\isasymin}\ S{\isacharparenright}\isanewline
\ \ {\isasymand}\ {\isacharparenleft}{\isasymforall}F\ G{\isachardot}\ F\ \isactrlbold {\isasymor}\ G\ {\isasymin}\ S\ {\isasymlongrightarrow}\ F\ {\isasymin}\ S\ {\isasymor}\ G\ {\isasymin}\ S{\isacharparenright}\isanewline
\ \ {\isasymand}\ {\isacharparenleft}{\isasymforall}F\ G{\isachardot}\ F\ \isactrlbold {\isasymrightarrow}\ G\ {\isasymin}\ S\ {\isasymlongrightarrow}\ \isactrlbold {\isasymnot}F\ {\isasymin}\ S\ {\isasymor}\ G\ {\isasymin}\ S{\isacharparenright}\isanewline
\ \ {\isasymand}\ {\isacharparenleft}{\isasymforall}F{\isachardot}\ \isactrlbold {\isasymnot}\ {\isacharparenleft}\isactrlbold {\isasymnot}F{\isacharparenright}\ {\isasymin}\ S\ {\isasymlongrightarrow}\ F\ {\isasymin}\ S{\isacharparenright}\isanewline
\ \ {\isasymand}\ {\isacharparenleft}{\isasymforall}F\ G{\isachardot}\ \isactrlbold {\isasymnot}{\isacharparenleft}F\ \isactrlbold {\isasymand}\ G{\isacharparenright}\ {\isasymin}\ S\ {\isasymlongrightarrow}\ \isactrlbold {\isasymnot}\ F\ {\isasymin}\ S\ {\isasymor}\ \isactrlbold {\isasymnot}\ G\ {\isasymin}\ S{\isacharparenright}\isanewline
\ \ {\isasymand}\ {\isacharparenleft}{\isasymforall}F\ G{\isachardot}\ \isactrlbold {\isasymnot}{\isacharparenleft}F\ \isactrlbold {\isasymor}\ G{\isacharparenright}\ {\isasymin}\ S\ {\isasymlongrightarrow}\ \isactrlbold {\isasymnot}\ F\ {\isasymin}\ S\ {\isasymand}\ \isactrlbold {\isasymnot}\ G\ {\isasymin}\ S{\isacharparenright}\isanewline
\ \ {\isasymand}\ {\isacharparenleft}{\isasymforall}F\ G{\isachardot}\ \isactrlbold {\isasymnot}{\isacharparenleft}F\ \isactrlbold {\isasymrightarrow}\ G{\isacharparenright}\ {\isasymin}\ S\ {\isasymlongrightarrow}\ F\ {\isasymin}\ S\ {\isasymand}\ \isactrlbold {\isasymnot}\ G\ {\isasymin}\ S{\isacharparenright}{\isachardoublequoteclose}\isanewline
\ \ \ \isacommand{using}\isamarkupfalse%
\ assms\ \isacommand{by}\isamarkupfalse%
\ {\isacharparenleft}rule\ auxEq{\isacharparenright}\isanewline
\ \ \isacommand{then}\isamarkupfalse%
\ \isacommand{have}\isamarkupfalse%
\ {\isachardoublequoteopen}{\isasymforall}F{\isachardot}\ \isactrlbold {\isasymnot}\ {\isacharparenleft}\isactrlbold {\isasymnot}F{\isacharparenright}\ {\isasymin}\ S\ {\isasymlongrightarrow}\ F\ {\isasymin}\ S{\isachardoublequoteclose}\isanewline
\ \ \ \ \isacommand{by}\isamarkupfalse%
\ {\isacharparenleft}iprover\ elim{\isacharcolon}\ conjunct{\isadigit{2}}\ conjunct{\isadigit{1}}{\isacharparenright}\isanewline
\ \ \isacommand{then}\isamarkupfalse%
\ \isacommand{have}\isamarkupfalse%
\ {\isachardoublequoteopen}\isactrlbold {\isasymnot}\ {\isacharparenleft}\isactrlbold {\isasymnot}F{\isacharparenright}\ {\isasymin}\ S\ {\isasymlongrightarrow}\ F\ {\isasymin}\ S{\isachardoublequoteclose}\isanewline
\ \ \ \ \isacommand{by}\isamarkupfalse%
\ {\isacharparenleft}rule\ allE{\isacharparenright}\isanewline
\ \ \isacommand{thus}\isamarkupfalse%
\ {\isachardoublequoteopen}F\ {\isasymin}\ S{\isachardoublequoteclose}\isanewline
\ \ \ \ \isacommand{using}\isamarkupfalse%
\ H\ \isacommand{by}\isamarkupfalse%
\ {\isacharparenleft}rule\ mp{\isacharparenright}\isanewline
\isacommand{qed}\isamarkupfalse%
%
\endisatagproof
{\isafoldproof}%
%
\isadelimproof
\isanewline
%
\endisadelimproof
\isanewline
\isacommand{lemma}\isamarkupfalse%
\ Hintikka{\isacharunderscore}l{\isadigit{6}}{\isacharcolon}\ \isanewline
\ {\isachardoublequoteopen}Hintikka\ S\ {\isasymLongrightarrow}\ \isactrlbold {\isasymnot}\ {\isacharparenleft}\isactrlbold {\isasymnot}F{\isacharparenright}\ {\isasymin}\ S\ {\isasymlongrightarrow}\ F\ {\isasymin}\ S{\isachardoublequoteclose}\isanewline
%
\isadelimproof
\ \ %
\endisadelimproof
%
\isatagproof
\isacommand{by}\isamarkupfalse%
\ {\isacharparenleft}smt\ Hintikka{\isacharunderscore}def{\isacharparenright}%
\endisatagproof
{\isafoldproof}%
%
\isadelimproof
\isanewline
%
\endisadelimproof
\isanewline
\isacommand{lemma}\isamarkupfalse%
\ \isanewline
\ \ \isakeyword{assumes}\ {\isachardoublequoteopen}Hintikka\ S{\isachardoublequoteclose}\ \isanewline
\ \ \isakeyword{shows}\ {\isachardoublequoteopen}\isactrlbold {\isasymnot}{\isacharparenleft}F\ \isactrlbold {\isasymand}\ G{\isacharparenright}\ {\isasymin}\ S\ {\isasymlongrightarrow}\ \isactrlbold {\isasymnot}\ F\ {\isasymin}\ S\ {\isasymor}\ \isactrlbold {\isasymnot}\ G\ {\isasymin}\ S{\isachardoublequoteclose}\isanewline
%
\isadelimproof
%
\endisadelimproof
%
\isatagproof
\isacommand{proof}\isamarkupfalse%
\ {\isacharparenleft}rule\ impI{\isacharparenright}\isanewline
\ \ \isacommand{assume}\isamarkupfalse%
\ H{\isacharcolon}{\isachardoublequoteopen}\isactrlbold {\isasymnot}{\isacharparenleft}F\ \isactrlbold {\isasymand}\ G{\isacharparenright}\ {\isasymin}\ S{\isachardoublequoteclose}\isanewline
\ \isacommand{have}\isamarkupfalse%
\ {\isachardoublequoteopen}{\isasymbottom}\ {\isasymnotin}\ S\isanewline
\ \ {\isasymand}\ {\isacharparenleft}{\isasymforall}k{\isachardot}\ Atom\ k\ {\isasymin}\ S\ {\isasymlongrightarrow}\ \isactrlbold {\isasymnot}\ {\isacharparenleft}Atom\ k{\isacharparenright}\ {\isasymin}\ S\ {\isasymlongrightarrow}\ False{\isacharparenright}\isanewline
\ \ {\isasymand}\ {\isacharparenleft}{\isasymforall}F\ G{\isachardot}\ F\ \isactrlbold {\isasymand}\ G\ {\isasymin}\ S\ {\isasymlongrightarrow}\ F\ {\isasymin}\ S\ {\isasymand}\ G\ {\isasymin}\ S{\isacharparenright}\isanewline
\ \ {\isasymand}\ {\isacharparenleft}{\isasymforall}F\ G{\isachardot}\ F\ \isactrlbold {\isasymor}\ G\ {\isasymin}\ S\ {\isasymlongrightarrow}\ F\ {\isasymin}\ S\ {\isasymor}\ G\ {\isasymin}\ S{\isacharparenright}\isanewline
\ \ {\isasymand}\ {\isacharparenleft}{\isasymforall}F\ G{\isachardot}\ F\ \isactrlbold {\isasymrightarrow}\ G\ {\isasymin}\ S\ {\isasymlongrightarrow}\ \isactrlbold {\isasymnot}F\ {\isasymin}\ S\ {\isasymor}\ G\ {\isasymin}\ S{\isacharparenright}\isanewline
\ \ {\isasymand}\ {\isacharparenleft}{\isasymforall}F{\isachardot}\ \isactrlbold {\isasymnot}\ {\isacharparenleft}\isactrlbold {\isasymnot}F{\isacharparenright}\ {\isasymin}\ S\ {\isasymlongrightarrow}\ F\ {\isasymin}\ S{\isacharparenright}\isanewline
\ \ {\isasymand}\ {\isacharparenleft}{\isasymforall}F\ G{\isachardot}\ \isactrlbold {\isasymnot}{\isacharparenleft}F\ \isactrlbold {\isasymand}\ G{\isacharparenright}\ {\isasymin}\ S\ {\isasymlongrightarrow}\ \isactrlbold {\isasymnot}\ F\ {\isasymin}\ S\ {\isasymor}\ \isactrlbold {\isasymnot}\ G\ {\isasymin}\ S{\isacharparenright}\isanewline
\ \ {\isasymand}\ {\isacharparenleft}{\isasymforall}F\ G{\isachardot}\ \isactrlbold {\isasymnot}{\isacharparenleft}F\ \isactrlbold {\isasymor}\ G{\isacharparenright}\ {\isasymin}\ S\ {\isasymlongrightarrow}\ \isactrlbold {\isasymnot}\ F\ {\isasymin}\ S\ {\isasymand}\ \isactrlbold {\isasymnot}\ G\ {\isasymin}\ S{\isacharparenright}\isanewline
\ \ {\isasymand}\ {\isacharparenleft}{\isasymforall}F\ G{\isachardot}\ \isactrlbold {\isasymnot}{\isacharparenleft}F\ \isactrlbold {\isasymrightarrow}\ G{\isacharparenright}\ {\isasymin}\ S\ {\isasymlongrightarrow}\ F\ {\isasymin}\ S\ {\isasymand}\ \isactrlbold {\isasymnot}\ G\ {\isasymin}\ S{\isacharparenright}{\isachardoublequoteclose}\isanewline
\ \ \ \isacommand{using}\isamarkupfalse%
\ assms\ \isacommand{by}\isamarkupfalse%
\ {\isacharparenleft}rule\ auxEq{\isacharparenright}\isanewline
\ \ \isacommand{then}\isamarkupfalse%
\ \isacommand{have}\isamarkupfalse%
\ {\isachardoublequoteopen}{\isasymforall}F\ G{\isachardot}\ \isactrlbold {\isasymnot}{\isacharparenleft}F\ \isactrlbold {\isasymand}\ G{\isacharparenright}\ {\isasymin}\ S\ {\isasymlongrightarrow}\ \isactrlbold {\isasymnot}\ F\ {\isasymin}\ S\ {\isasymor}\ \isactrlbold {\isasymnot}\ G\ {\isasymin}\ S{\isachardoublequoteclose}\isanewline
\ \ \ \ \isacommand{by}\isamarkupfalse%
\ {\isacharparenleft}iprover\ elim{\isacharcolon}\ conjunct{\isadigit{2}}\ conjunct{\isadigit{1}}{\isacharparenright}\isanewline
\ \ \isacommand{then}\isamarkupfalse%
\ \isacommand{have}\isamarkupfalse%
\ {\isachardoublequoteopen}\isactrlbold {\isasymnot}{\isacharparenleft}F\ \isactrlbold {\isasymand}\ G{\isacharparenright}\ {\isasymin}\ S\ {\isasymlongrightarrow}\ \isactrlbold {\isasymnot}\ F\ {\isasymin}\ S\ {\isasymor}\ \isactrlbold {\isasymnot}\ G\ {\isasymin}\ S{\isachardoublequoteclose}\isanewline
\ \ \ \ \isacommand{by}\isamarkupfalse%
\ {\isacharparenleft}iprover\ elim{\isacharcolon}\ allE{\isacharparenright}\isanewline
\ \ \ \ \isacommand{thus}\isamarkupfalse%
\ {\isachardoublequoteopen}\isactrlbold {\isasymnot}\ F\ {\isasymin}\ S\ {\isasymor}\ \isactrlbold {\isasymnot}\ G\ {\isasymin}\ S{\isachardoublequoteclose}\isanewline
\ \ \ \ \ \ \isacommand{using}\isamarkupfalse%
\ H\ \isacommand{by}\isamarkupfalse%
\ {\isacharparenleft}rule\ mp{\isacharparenright}\isanewline
\ \ \isacommand{qed}\isamarkupfalse%
%
\endisatagproof
{\isafoldproof}%
%
\isadelimproof
\isanewline
%
\endisadelimproof
\isanewline
\isacommand{lemma}\isamarkupfalse%
\ Hintikka{\isacharunderscore}l{\isadigit{7}}{\isacharcolon}\ \isanewline
\ {\isachardoublequoteopen}Hintikka\ S\ {\isasymLongrightarrow}\ \isactrlbold {\isasymnot}{\isacharparenleft}F\ \isactrlbold {\isasymand}\ G{\isacharparenright}\ {\isasymin}\ S\ {\isasymlongrightarrow}\ \isactrlbold {\isasymnot}\ F\ {\isasymin}\ S\ {\isasymor}\ \isactrlbold {\isasymnot}\ G\ {\isasymin}\ S{\isachardoublequoteclose}\isanewline
%
\isadelimproof
\ \ %
\endisadelimproof
%
\isatagproof
\isacommand{by}\isamarkupfalse%
\ {\isacharparenleft}smt\ Hintikka{\isacharunderscore}def{\isacharparenright}%
\endisatagproof
{\isafoldproof}%
%
\isadelimproof
\isanewline
%
\endisadelimproof
\isanewline
\isacommand{lemma}\isamarkupfalse%
\isanewline
\ \ \isakeyword{assumes}\ {\isachardoublequoteopen}Hintikka\ S{\isachardoublequoteclose}\ \isanewline
\ \ \isakeyword{shows}\ {\isachardoublequoteopen}\isactrlbold {\isasymnot}{\isacharparenleft}F\ \isactrlbold {\isasymor}\ G{\isacharparenright}\ {\isasymin}\ S\ {\isasymlongrightarrow}\ \isactrlbold {\isasymnot}\ F\ {\isasymin}\ S\ {\isasymand}\ \isactrlbold {\isasymnot}\ G\ {\isasymin}\ S{\isachardoublequoteclose}\isanewline
%
\isadelimproof
%
\endisadelimproof
%
\isatagproof
\isacommand{proof}\isamarkupfalse%
\ {\isacharparenleft}rule\ impI{\isacharparenright}\isanewline
\ \ \isacommand{assume}\isamarkupfalse%
\ H{\isacharcolon}{\isachardoublequoteopen}\isactrlbold {\isasymnot}{\isacharparenleft}F\ \isactrlbold {\isasymor}\ G{\isacharparenright}\ {\isasymin}\ S{\isachardoublequoteclose}\isanewline
\ \isacommand{have}\isamarkupfalse%
\ {\isachardoublequoteopen}{\isasymbottom}\ {\isasymnotin}\ S\isanewline
\ \ {\isasymand}\ {\isacharparenleft}{\isasymforall}k{\isachardot}\ Atom\ k\ {\isasymin}\ S\ {\isasymlongrightarrow}\ \isactrlbold {\isasymnot}\ {\isacharparenleft}Atom\ k{\isacharparenright}\ {\isasymin}\ S\ {\isasymlongrightarrow}\ False{\isacharparenright}\isanewline
\ \ {\isasymand}\ {\isacharparenleft}{\isasymforall}F\ G{\isachardot}\ F\ \isactrlbold {\isasymand}\ G\ {\isasymin}\ S\ {\isasymlongrightarrow}\ F\ {\isasymin}\ S\ {\isasymand}\ G\ {\isasymin}\ S{\isacharparenright}\isanewline
\ \ {\isasymand}\ {\isacharparenleft}{\isasymforall}F\ G{\isachardot}\ F\ \isactrlbold {\isasymor}\ G\ {\isasymin}\ S\ {\isasymlongrightarrow}\ F\ {\isasymin}\ S\ {\isasymor}\ G\ {\isasymin}\ S{\isacharparenright}\isanewline
\ \ {\isasymand}\ {\isacharparenleft}{\isasymforall}F\ G{\isachardot}\ F\ \isactrlbold {\isasymrightarrow}\ G\ {\isasymin}\ S\ {\isasymlongrightarrow}\ \isactrlbold {\isasymnot}F\ {\isasymin}\ S\ {\isasymor}\ G\ {\isasymin}\ S{\isacharparenright}\isanewline
\ \ {\isasymand}\ {\isacharparenleft}{\isasymforall}F{\isachardot}\ \isactrlbold {\isasymnot}\ {\isacharparenleft}\isactrlbold {\isasymnot}F{\isacharparenright}\ {\isasymin}\ S\ {\isasymlongrightarrow}\ F\ {\isasymin}\ S{\isacharparenright}\isanewline
\ \ {\isasymand}\ {\isacharparenleft}{\isasymforall}F\ G{\isachardot}\ \isactrlbold {\isasymnot}{\isacharparenleft}F\ \isactrlbold {\isasymand}\ G{\isacharparenright}\ {\isasymin}\ S\ {\isasymlongrightarrow}\ \isactrlbold {\isasymnot}\ F\ {\isasymin}\ S\ {\isasymor}\ \isactrlbold {\isasymnot}\ G\ {\isasymin}\ S{\isacharparenright}\isanewline
\ \ {\isasymand}\ {\isacharparenleft}{\isasymforall}F\ G{\isachardot}\ \isactrlbold {\isasymnot}{\isacharparenleft}F\ \isactrlbold {\isasymor}\ G{\isacharparenright}\ {\isasymin}\ S\ {\isasymlongrightarrow}\ \isactrlbold {\isasymnot}\ F\ {\isasymin}\ S\ {\isasymand}\ \isactrlbold {\isasymnot}\ G\ {\isasymin}\ S{\isacharparenright}\isanewline
\ \ {\isasymand}\ {\isacharparenleft}{\isasymforall}F\ G{\isachardot}\ \isactrlbold {\isasymnot}{\isacharparenleft}F\ \isactrlbold {\isasymrightarrow}\ G{\isacharparenright}\ {\isasymin}\ S\ {\isasymlongrightarrow}\ F\ {\isasymin}\ S\ {\isasymand}\ \isactrlbold {\isasymnot}\ G\ {\isasymin}\ S{\isacharparenright}{\isachardoublequoteclose}\isanewline
\ \ \ \isacommand{using}\isamarkupfalse%
\ assms\ \isacommand{by}\isamarkupfalse%
\ {\isacharparenleft}rule\ auxEq{\isacharparenright}\isanewline
\ \ \isacommand{then}\isamarkupfalse%
\ \isacommand{have}\isamarkupfalse%
\ {\isachardoublequoteopen}{\isasymforall}F\ G{\isachardot}\ \isactrlbold {\isasymnot}{\isacharparenleft}F\ \isactrlbold {\isasymor}\ G{\isacharparenright}\ {\isasymin}\ S\ {\isasymlongrightarrow}\ \isactrlbold {\isasymnot}\ F\ {\isasymin}\ S\ {\isasymand}\ \isactrlbold {\isasymnot}\ G\ {\isasymin}\ S{\isachardoublequoteclose}\isanewline
\ \ \ \ \isacommand{by}\isamarkupfalse%
\ {\isacharparenleft}iprover\ elim{\isacharcolon}\ conjunct{\isadigit{2}}\ conjunct{\isadigit{1}}{\isacharparenright}\isanewline
\ \ \isacommand{then}\isamarkupfalse%
\ \isacommand{have}\isamarkupfalse%
\ {\isachardoublequoteopen}\isactrlbold {\isasymnot}{\isacharparenleft}F\ \isactrlbold {\isasymor}\ G{\isacharparenright}\ {\isasymin}\ S\ {\isasymlongrightarrow}\ \isactrlbold {\isasymnot}\ F\ {\isasymin}\ S\ {\isasymand}\ \isactrlbold {\isasymnot}\ G\ {\isasymin}\ S{\isachardoublequoteclose}\isanewline
\ \ \ \ \isacommand{by}\isamarkupfalse%
\ {\isacharparenleft}iprover\ elim{\isacharcolon}\ allE{\isacharparenright}\isanewline
\ \ \isacommand{thus}\isamarkupfalse%
\ {\isachardoublequoteopen}\isactrlbold {\isasymnot}\ F\ {\isasymin}\ S\ {\isasymand}\ \isactrlbold {\isasymnot}\ G\ {\isasymin}\ S{\isachardoublequoteclose}\isanewline
\ \ \ \ \isacommand{using}\isamarkupfalse%
\ H\ \isacommand{by}\isamarkupfalse%
\ {\isacharparenleft}rule\ mp{\isacharparenright}\isanewline
\isacommand{qed}\isamarkupfalse%
%
\endisatagproof
{\isafoldproof}%
%
\isadelimproof
\isanewline
%
\endisadelimproof
\isanewline
\isacommand{lemma}\isamarkupfalse%
\ Hintikka{\isacharunderscore}l{\isadigit{8}}{\isacharcolon}\ \isanewline
\ {\isachardoublequoteopen}Hintikka\ S\ {\isasymLongrightarrow}\ {\isacharparenleft}\ \isactrlbold {\isasymnot}{\isacharparenleft}F\ \isactrlbold {\isasymor}\ G{\isacharparenright}\ {\isasymin}\ S\ {\isasymlongrightarrow}\ \isactrlbold {\isasymnot}\ F\ {\isasymin}\ S\ {\isasymand}\ \isactrlbold {\isasymnot}\ G\ {\isasymin}\ S{\isacharparenright}{\isachardoublequoteclose}\isanewline
%
\isadelimproof
\ \ %
\endisadelimproof
%
\isatagproof
\isacommand{by}\isamarkupfalse%
\ {\isacharparenleft}smt\ Hintikka{\isacharunderscore}def{\isacharparenright}%
\endisatagproof
{\isafoldproof}%
%
\isadelimproof
\isanewline
%
\endisadelimproof
\isanewline
\isacommand{lemma}\isamarkupfalse%
\ \isanewline
\ \ \isakeyword{assumes}\ {\isachardoublequoteopen}Hintikka\ S{\isachardoublequoteclose}\ \isanewline
\ \ \isakeyword{shows}\ {\isachardoublequoteopen}\isactrlbold {\isasymnot}{\isacharparenleft}F\ \isactrlbold {\isasymrightarrow}\ G{\isacharparenright}\ {\isasymin}\ S\ {\isasymlongrightarrow}\ F\ {\isasymin}\ S\ {\isasymand}\ \isactrlbold {\isasymnot}\ G\ {\isasymin}\ S{\isachardoublequoteclose}\isanewline
%
\isadelimproof
%
\endisadelimproof
%
\isatagproof
\isacommand{proof}\isamarkupfalse%
\ {\isacharparenleft}rule\ impI{\isacharparenright}\isanewline
\ \ \isacommand{assume}\isamarkupfalse%
\ H{\isacharcolon}{\isachardoublequoteopen}\isactrlbold {\isasymnot}{\isacharparenleft}F\ \isactrlbold {\isasymrightarrow}\ G{\isacharparenright}\ {\isasymin}\ S{\isachardoublequoteclose}\isanewline
\ \isacommand{have}\isamarkupfalse%
\ {\isachardoublequoteopen}{\isasymbottom}\ {\isasymnotin}\ S\isanewline
\ \ {\isasymand}\ {\isacharparenleft}{\isasymforall}k{\isachardot}\ Atom\ k\ {\isasymin}\ S\ {\isasymlongrightarrow}\ \isactrlbold {\isasymnot}\ {\isacharparenleft}Atom\ k{\isacharparenright}\ {\isasymin}\ S\ {\isasymlongrightarrow}\ False{\isacharparenright}\isanewline
\ \ {\isasymand}\ {\isacharparenleft}{\isasymforall}F\ G{\isachardot}\ F\ \isactrlbold {\isasymand}\ G\ {\isasymin}\ S\ {\isasymlongrightarrow}\ F\ {\isasymin}\ S\ {\isasymand}\ G\ {\isasymin}\ S{\isacharparenright}\isanewline
\ \ {\isasymand}\ {\isacharparenleft}{\isasymforall}F\ G{\isachardot}\ F\ \isactrlbold {\isasymor}\ G\ {\isasymin}\ S\ {\isasymlongrightarrow}\ F\ {\isasymin}\ S\ {\isasymor}\ G\ {\isasymin}\ S{\isacharparenright}\isanewline
\ \ {\isasymand}\ {\isacharparenleft}{\isasymforall}F\ G{\isachardot}\ F\ \isactrlbold {\isasymrightarrow}\ G\ {\isasymin}\ S\ {\isasymlongrightarrow}\ \isactrlbold {\isasymnot}F\ {\isasymin}\ S\ {\isasymor}\ G\ {\isasymin}\ S{\isacharparenright}\isanewline
\ \ {\isasymand}\ {\isacharparenleft}{\isasymforall}F{\isachardot}\ \isactrlbold {\isasymnot}\ {\isacharparenleft}\isactrlbold {\isasymnot}F{\isacharparenright}\ {\isasymin}\ S\ {\isasymlongrightarrow}\ F\ {\isasymin}\ S{\isacharparenright}\isanewline
\ \ {\isasymand}\ {\isacharparenleft}{\isasymforall}F\ G{\isachardot}\ \isactrlbold {\isasymnot}{\isacharparenleft}F\ \isactrlbold {\isasymand}\ G{\isacharparenright}\ {\isasymin}\ S\ {\isasymlongrightarrow}\ \isactrlbold {\isasymnot}\ F\ {\isasymin}\ S\ {\isasymor}\ \isactrlbold {\isasymnot}\ G\ {\isasymin}\ S{\isacharparenright}\isanewline
\ \ {\isasymand}\ {\isacharparenleft}{\isasymforall}F\ G{\isachardot}\ \isactrlbold {\isasymnot}{\isacharparenleft}F\ \isactrlbold {\isasymor}\ G{\isacharparenright}\ {\isasymin}\ S\ {\isasymlongrightarrow}\ \isactrlbold {\isasymnot}\ F\ {\isasymin}\ S\ {\isasymand}\ \isactrlbold {\isasymnot}\ G\ {\isasymin}\ S{\isacharparenright}\isanewline
\ \ {\isasymand}\ {\isacharparenleft}{\isasymforall}F\ G{\isachardot}\ \isactrlbold {\isasymnot}{\isacharparenleft}F\ \isactrlbold {\isasymrightarrow}\ G{\isacharparenright}\ {\isasymin}\ S\ {\isasymlongrightarrow}\ F\ {\isasymin}\ S\ {\isasymand}\ \isactrlbold {\isasymnot}\ G\ {\isasymin}\ S{\isacharparenright}{\isachardoublequoteclose}\isanewline
\ \ \ \isacommand{using}\isamarkupfalse%
\ assms\ \isacommand{by}\isamarkupfalse%
\ {\isacharparenleft}rule\ auxEq{\isacharparenright}\isanewline
\ \ \isacommand{then}\isamarkupfalse%
\ \isacommand{have}\isamarkupfalse%
\ {\isachardoublequoteopen}{\isasymforall}F\ G{\isachardot}\ \isactrlbold {\isasymnot}{\isacharparenleft}F\ \isactrlbold {\isasymrightarrow}\ G{\isacharparenright}\ {\isasymin}\ S\ {\isasymlongrightarrow}\ F\ {\isasymin}\ S\ {\isasymand}\ \isactrlbold {\isasymnot}\ G\ {\isasymin}\ S{\isachardoublequoteclose}\isanewline
\ \ \ \ \isacommand{by}\isamarkupfalse%
\ {\isacharparenleft}iprover\ elim{\isacharcolon}\ conjunct{\isadigit{2}}{\isacharparenright}\isanewline
\ \ \isacommand{then}\isamarkupfalse%
\ \isacommand{have}\isamarkupfalse%
\ {\isachardoublequoteopen}\isactrlbold {\isasymnot}{\isacharparenleft}F\ \isactrlbold {\isasymrightarrow}\ G{\isacharparenright}\ {\isasymin}\ S\ {\isasymlongrightarrow}\ F\ {\isasymin}\ S\ {\isasymand}\ \isactrlbold {\isasymnot}\ G\ {\isasymin}\ S{\isachardoublequoteclose}\isanewline
\ \ \ \ \isacommand{by}\isamarkupfalse%
\ {\isacharparenleft}iprover\ elim{\isacharcolon}\ allE{\isacharparenright}\isanewline
\ \ \isacommand{thus}\isamarkupfalse%
\ {\isachardoublequoteopen}F\ {\isasymin}\ S\ {\isasymand}\ \isactrlbold {\isasymnot}\ G\ {\isasymin}\ S{\isachardoublequoteclose}\isanewline
\ \ \ \ \isacommand{using}\isamarkupfalse%
\ H\ \isacommand{by}\isamarkupfalse%
\ {\isacharparenleft}rule\ mp{\isacharparenright}\isanewline
\isacommand{qed}\isamarkupfalse%
%
\endisatagproof
{\isafoldproof}%
%
\isadelimproof
\isanewline
%
\endisadelimproof
\isanewline
\isacommand{lemma}\isamarkupfalse%
\ Hintikka{\isacharunderscore}l{\isadigit{9}}{\isacharcolon}\ \isanewline
\ {\isachardoublequoteopen}Hintikka\ S\ {\isasymLongrightarrow}\ \isactrlbold {\isasymnot}{\isacharparenleft}F\ \isactrlbold {\isasymrightarrow}\ G{\isacharparenright}\ {\isasymin}\ S\ {\isasymlongrightarrow}\ F\ {\isasymin}\ S\ {\isasymand}\ \isactrlbold {\isasymnot}\ G\ {\isasymin}\ S{\isachardoublequoteclose}\isanewline
%
\isadelimproof
\ \ %
\endisadelimproof
%
\isatagproof
\isacommand{by}\isamarkupfalse%
\ {\isacharparenleft}smt\ Hintikka{\isacharunderscore}def{\isacharparenright}%
\endisatagproof
{\isafoldproof}%
%
\isadelimproof
%
\endisadelimproof
%
\begin{isamarkuptext}%
Las pruebas anteriores siguen un esquema similar en Isabelle. 
  En primer lugar, mediante el lema \isa{auxEq}, al suponer inicialmente
  que el conjunto es de Hintikka, se verifica la conjunción de las 
  condiciones de su definición. Posteriormente se emplean distintos
  métodos para disgregar estas condiciones en los distintos lemas. Para 
  este propósito se utiliza, en particular, la táctica de 
  demostración \isa{{\isacharparenleft}iprover\ elim{\isacharcolon}\ {\isasymopen}rules{\isasymclose}{\isacharparenright}}. Con esta táctica aplicamos
  reiteradamente una o varias reglas y reducimos pasos en la prueba de 
  Isabelle/HOL. Para ello, nos hemos servido del método de demostración 
  \isa{elim} que permite aplicar repetidamente reglas de 
  eliminación especificadas. En nuestro caso, hemos utilizado las reglas 
  de eliminación de la conjunción y la regla de eliminación del 
  cuantificador existencial. Por otro lado, \isa{iprover} es uno de los 
  métodos automáticos de demostración en Isabelle/HOL que 
  depende del contexto y de las reglas o métodos específicamente 
  declarados a continuación del mismo. 

  Finalmente, veamos un resultado derivado de las condiciones
  exigidas a los conjuntos de Hintikka.

  \begin{lema}
    Dado un conjunto de Hintikka, una fórmula no pertenece al conjunto 
    si su negación sí pertenece al mismo. Es decir, si \isa{S} es un 
    conjunto de Hintikka y \isa{{\isasymnot}\ F\ {\isasymin}\ S}, entonces\\ \isa{F\ {\isasymnotin}\ S}.
  \end{lema}

  Antes de pasar a la demostración del resultado, cabe añadir que
  para su prueba utilizaremos la regla lógica \isa{modus\ tollens}. 

  \begin{lema}[Modus tollens]
   Si \isa{P} implica \isa{Q} y \isa{Q} es falso, entonces \isa{P} es también falso.
  \end{lema}

  Teniendo esto en cuenta, la demostración del lema es la siguiente.

\begin{demostracion}
   Sea \isa{S} un conjunto de Hintikka y \isa{F} un fórmula. Hay que probar
   que si \isa{{\isasymnot}F\ {\isasymin}\ S}, entonces \isa{F\ {\isasymnotin}\ S}. 
   La prueba se realiza por inducción sobre la estructura de las 
   fórmulas proposicionales. Veamos los distintos casos.

\begin{enumerate}
   \item[Caso 1:] \isa{F\ {\isacharequal}\ p}, fórmula atómica.

    Supongamos que \isa{{\isasymnot}\ p\ {\isasymin}\ S}.  Si \isa{p\ {\isasymin}\ S} por definición de conjunto 
    de Hintikka,\\ \isa{{\isasymnot}\ p\ {\isasymnotin}\ S}, en contra de la hipótesis. 

   \item[Caso 2:] \isa{F\ {\isacharequal}\ {\isasymbottom}}   

    Supongamos que \isa{{\isasymnot}\ {\isasymbottom}\ {\isasymin}\ S}. Como \isa{S} es un conjunto de Hintikka, por 
    definición se tiene que \isa{{\isasymbottom}\ {\isasymnotin}\ S}, como queríamos demostrar.

   \item[Caso 3:] \isa{F\ {\isacharequal}\ {\isasymnot}\ G}, y \isa{G} verifica la hipótesis de inducción. 

     Es decir, \isa{G} verifica \isa{HI}: \isa{{\isasymnot}\ G\ {\isasymin}\ S\ {\isasymLongrightarrow}\ G\ {\isasymnotin}\ S}. 

     Probemos que  \isa{{\isasymnot}\ F\ {\isasymin}\ S\ {\isasymLongrightarrow}\ F\ {\isasymnotin}\ S}. 

    En efecto,

$$\begin{array}{lrl}
 & \isa{{\isasymnot}\ F\ {\isasymin}\ S} & \\
\isa{{\isasymLongrightarrow}} & \isa{{\isasymnot}{\isasymnot}\ G\ {\isasymin}\ S} & \\
\isa{{\isasymLongrightarrow}} & \isa{G\ {\isasymin}\ S} & \isa{{\isacharparenleft}S\ es\ conjunto\ de\ Hintikka{\isacharparenright}}\\
\isa{{\isasymLongrightarrow}} & \isa{{\isasymnot}\ G\ {\isasymnotin}\ S} & \isa{{\isacharparenleft}HI\ y\ contraposición{\isacharparenright}}\\
\isa{{\isasymLongrightarrow}} & \isa{F\ {\isasymnotin}\ S} &
      \end{array}$$ 

   
    \item[Caso 4:] \isa{F\ {\isacharequal}\ G\ {\isasymand}\ H} y tanto \isa{G} como \isa{H} verifican la 
    hipótesis de inducción. 

    Es decir, se verifica \isa{HI}: \isa{{\isasymnot}\ G\ {\isasymin}\ S\ {\isasymLongrightarrow}\ G\ {\isasymnotin}\ S} y 
    \isa{{\isasymnot}\ H\ {\isasymin}\ S\ {\isasymLongrightarrow}\ H\ {\isasymnotin}\ S}. 

    Probemos que  \isa{{\isasymnot}\ F\ {\isasymin}\ S\ {\isasymLongrightarrow}\ F\ {\isasymnotin}\ S}. 

    En efecto,
$$\begin{array}{lrl}
 & \isa{{\isasymnot}\ F\ {\isasymin}\ S} & \\
\isa{{\isasymLongrightarrow}} & \isa{{\isasymnot}\ {\isacharparenleft}G\ {\isasymand}\ H{\isacharparenright}\ {\isasymin}\ S} & \\
\isa{{\isasymLongrightarrow}} & \isa{{\isasymnot}\ G\ {\isasymin}\ S\ {\isasymor}\ {\isasymnot}\ H\ {\isasymin}\ S} & \isa{{\isacharparenleft}S\ es\ conjunto\ de\ Hintikka{\isacharparenright}}\\
\isa{{\isasymLongrightarrow}} & \isa{G\ {\isasymnotin}\ S\ {\isasymor}\ H\ {\isasymnotin}\ S} & \isa{{\isacharparenleft}HI{\isacharparenright}}
      \end{array}$$ 

  Por otra parte, si \isa{F\ {\isasymin}\ S} se tiene que \isa{G\ {\isasymin}\ S\ {\isasymand}\ H\ {\isasymin}\ S}, por ser \isa{S} 
  conjunto de Hintikka, lo que contradice lo anterior. Por tanto, 
  \isa{F\ {\isasymnotin}\ S}.

  \item[Caso 5:] \isa{F\ {\isacharequal}\ G\ {\isasymor}\ H} y tanto \isa{G} como \isa{H} verifican la hipótesis 
  de inducción. 

  Es decir, se verifica \isa{HI}: \isa{{\isasymnot}\ G\ {\isasymin}\ S\ {\isasymLongrightarrow}\ G\ {\isasymnotin}\ S} y 
  \isa{{\isasymnot}\ H\ {\isasymin}\ S\ {\isasymLongrightarrow}\ H\ {\isasymnotin}\ S}. 

  Probemos que  \isa{{\isasymnot}\ F\ {\isasymin}\ S\ {\isasymLongrightarrow}\ F\ {\isasymnotin}\ S}. 

  En efecto,
$$\begin{array}{lrl}
 & \isa{{\isasymnot}\ F\ {\isasymin}\ S} & \\
\isa{{\isasymLongrightarrow}} & \isa{{\isasymnot}\ {\isacharparenleft}G\ {\isasymor}\ H{\isacharparenright}\ {\isasymin}\ S} & \\
\isa{{\isasymLongrightarrow}} & \isa{{\isasymnot}\ G\ {\isasymin}\ S\ {\isasymand}\ {\isasymnot}\ H\ {\isasymin}\ S} & \isa{{\isacharparenleft}S\ es\ conjunto\ de\ Hintikka{\isacharparenright}}\\
\isa{{\isasymLongrightarrow}} & \isa{G\ {\isasymnotin}\ S\ {\isasymand}\ H\ {\isasymnotin}\ S} & \isa{{\isacharparenleft}HI{\isacharparenright}}
      \end{array}$$ 

  Por otra parte, si \isa{F\ {\isasymin}\ S} se tiene que \isa{G\ {\isasymin}\ S\ {\isasymor}\ H\ {\isasymin}\ S}, por ser \isa{S}
  conjunto de Hintikka, en contra de lo obtenido anteriormente. Por
  tanto, \isa{F\ {\isasymnotin}\ S}.

  \item[Caso 6:] \isa{F\ {\isacharequal}\ G\ {\isasymlongrightarrow}\ H} y tanto \isa{G} como \isa{H} verifican la 
  hipótesis de inducción. 

  Es decir, se verifica \isa{HI}: \isa{{\isasymnot}\ G\ {\isasymin}\ S\ {\isasymLongrightarrow}\ G\ {\isasymnotin}\ S} y \isa{{\isasymnot}\ H\ {\isasymin}\ S\ {\isasymLongrightarrow}\ H\ {\isasymnotin}\ S}. 

  Probemos que  \isa{{\isasymnot}\ F\ {\isasymin}\ S\ {\isasymLongrightarrow}\ F\ {\isasymnotin}\ S}. 

  En efecto,
$$\begin{array}{lrl}
 & \isa{{\isasymnot}\ F\ {\isasymin}\ S} & \\
\isa{{\isasymLongrightarrow}} & \isa{{\isasymnot}\ {\isacharparenleft}G\ {\isasymlongrightarrow}\ H{\isacharparenright}\ {\isasymin}\ S} & \\
\isa{{\isasymLongrightarrow}} & \isa{G\ {\isasymin}\ S\ {\isasymand}\ {\isasymnot}\ H\ {\isasymin}\ S} & \isa{{\isacharparenleft}S\ es\ conjunto\ de\ Hintikka{\isacharparenright}}\\
\isa{{\isasymLongrightarrow}} & \isa{{\isasymnot}\ G\ {\isasymnotin}\ S\ {\isasymand}\ {\isasymnot}\ H\ {\isasymin}\ S} & \isa{{\isacharparenleft}HI\ y\ contraposición{\isacharparenright}}\\
\isa{{\isasymLongrightarrow}} & \isa{{\isasymnot}\ G\ {\isasymnotin}\ S\ {\isasymand}\ H\ {\isasymnotin}\ S} & \isa{{\isacharparenleft}HI{\isacharparenright}}
      \end{array}$$ 

  Por otra parte, si \isa{F\ {\isasymin}\ S} se tiene que \isa{{\isasymnot}\ G\ {\isasymin}\ S\ {\isasymor}\ H\ {\isasymin}\ S}, por ser
  \isa{S} conjunto de Hintikka, lo que contradice lo anterior. Por lo
  tanto, \isa{F\ {\isasymnotin}\ S}.
  \end{enumerate}

  Con lo que termina la demostración.
  \end{demostracion} 

  Por otra parte, su enunciado se formaliza en Isabelle de la siguiente 
  forma.%
\end{isamarkuptext}\isamarkuptrue%
\isacommand{lemma}\isamarkupfalse%
\ {\isachardoublequoteopen}Hintikka\ S\ {\isasymLongrightarrow}\ {\isacharparenleft}\isactrlbold {\isasymnot}\ F\ {\isasymin}\ S\ {\isasymlongrightarrow}\ F\ {\isasymnotin}\ S{\isacharparenright}{\isachardoublequoteclose}\isanewline
%
\isadelimproof
\ \ %
\endisadelimproof
%
\isatagproof
\isacommand{oops}\isamarkupfalse%
%
\endisatagproof
{\isafoldproof}%
%
\isadelimproof
%
\endisadelimproof
%
\begin{isamarkuptext}%
Antes de proceder con las distintas pruebas en Isabelle/HOL, 
  vamos a formalizar la regla \isa{modus\ tollens} usada en las 
  demostraciones. Esta regla no está definida en Isabelle, de modo que 
  se introducirá a continuación como lema auxiliar. Además, mostraremos
  su prueba detallada.%
\end{isamarkuptext}\isamarkuptrue%
\isacommand{lemma}\isamarkupfalse%
\ mt{\isacharcolon}\ \isanewline
\ \ \isakeyword{assumes}\ {\isachardoublequoteopen}F\ {\isasymlongrightarrow}\ G{\isachardoublequoteclose}\ \isanewline
\ \ \ \ \ \ \ \ \ \ {\isachardoublequoteopen}{\isasymnot}\ G{\isachardoublequoteclose}\isanewline
\ \ \isakeyword{shows}\ {\isachardoublequoteopen}{\isasymnot}\ F{\isachardoublequoteclose}\isanewline
%
\isadelimproof
%
\endisadelimproof
%
\isatagproof
\isacommand{proof}\isamarkupfalse%
\ {\isacharminus}\isanewline
\ \ \isacommand{have}\isamarkupfalse%
\ {\isachardoublequoteopen}{\isasymnot}\ G\ {\isasymlongrightarrow}\ {\isasymnot}\ F{\isachardoublequoteclose}\isanewline
\ \ \ \ \isacommand{using}\isamarkupfalse%
\ assms{\isacharparenleft}{\isadigit{1}}{\isacharparenright}\ \isacommand{by}\isamarkupfalse%
\ {\isacharparenleft}rule\ not{\isacharunderscore}mono{\isacharparenright}\isanewline
\ \ \isacommand{thus}\isamarkupfalse%
\ {\isachardoublequoteopen}{\isasymnot}\ F{\isachardoublequoteclose}\isanewline
\ \ \ \ \isacommand{using}\isamarkupfalse%
\ assms{\isacharparenleft}{\isadigit{2}}{\isacharparenright}\ \isacommand{by}\isamarkupfalse%
\ {\isacharparenleft}rule\ mp{\isacharparenright}\isanewline
\isacommand{qed}\isamarkupfalse%
%
\endisatagproof
{\isafoldproof}%
%
\isadelimproof
%
\endisadelimproof
%
\begin{isamarkuptext}%
Procedamos con la demostración del lema en Isabelle/HOL de
  manera detallada. Como es habitual para facilitar dicha prueba, se 
  hará cada caso de la estructura de fórmulas por separado.%
\end{isamarkuptext}\isamarkuptrue%
\isacommand{lemma}\isamarkupfalse%
\ Hintikka{\isacharunderscore}l{\isadigit{1}}{\isadigit{0}}{\isacharunderscore}atom{\isacharcolon}\ \isanewline
\ \ \isakeyword{assumes}\ {\isachardoublequoteopen}Hintikka\ S{\isachardoublequoteclose}\ \isanewline
\ \ \isakeyword{shows}\ {\isachardoublequoteopen}\isactrlbold {\isasymnot}\ {\isacharparenleft}Atom\ x{\isacharparenright}\ {\isasymin}\ S\ {\isasymlongrightarrow}\ Atom\ x\ {\isasymnotin}\ S{\isachardoublequoteclose}\isanewline
%
\isadelimproof
%
\endisadelimproof
%
\isatagproof
\isacommand{proof}\isamarkupfalse%
\ {\isacharparenleft}rule\ impI{\isacharparenright}\isanewline
\ \ \isacommand{assume}\isamarkupfalse%
\ {\isachardoublequoteopen}\isactrlbold {\isasymnot}\ {\isacharparenleft}Atom\ x{\isacharparenright}\ {\isasymin}\ S{\isachardoublequoteclose}\isanewline
\ \ \isacommand{then}\isamarkupfalse%
\ \isacommand{have}\isamarkupfalse%
\ {\isachardoublequoteopen}{\isasymnot}\ {\isacharparenleft}\isactrlbold {\isasymnot}\ {\isacharparenleft}Atom\ x{\isacharparenright}\ {\isasymnotin}\ S{\isacharparenright}{\isachardoublequoteclose}\isanewline
\ \ \ \ \isacommand{by}\isamarkupfalse%
\ {\isacharparenleft}rule\ contrapos{\isacharunderscore}pn{\isacharparenright}\isanewline
\ \ \isacommand{have}\isamarkupfalse%
\ {\isachardoublequoteopen}Atom\ x\ {\isasymin}\ S\ {\isasymlongrightarrow}\ \isactrlbold {\isasymnot}\ {\isacharparenleft}Atom\ x{\isacharparenright}\ {\isasymnotin}\ S{\isachardoublequoteclose}\isanewline
\ \ \ \ \isacommand{using}\isamarkupfalse%
\ assms\ \isacommand{by}\isamarkupfalse%
\ {\isacharparenleft}rule\ Hintikka{\isacharunderscore}l{\isadigit{2}}{\isacharparenright}\isanewline
\ \ \isacommand{thus}\isamarkupfalse%
\ {\isachardoublequoteopen}Atom\ x\ {\isasymnotin}\ S{\isachardoublequoteclose}\isanewline
\ \ \ \ \isacommand{using}\isamarkupfalse%
\ {\isacartoucheopen}{\isasymnot}{\isacharparenleft}\isactrlbold {\isasymnot}\ {\isacharparenleft}Atom\ x{\isacharparenright}\ {\isasymnotin}\ S{\isacharparenright}{\isacartoucheclose}\ \isacommand{by}\isamarkupfalse%
\ {\isacharparenleft}rule\ mt{\isacharparenright}\isanewline
\isacommand{qed}\isamarkupfalse%
%
\endisatagproof
{\isafoldproof}%
%
\isadelimproof
\isanewline
%
\endisadelimproof
\isanewline
\isacommand{lemma}\isamarkupfalse%
\ Hintikka{\isacharunderscore}l{\isadigit{1}}{\isadigit{0}}{\isacharunderscore}bot{\isacharcolon}\ \isanewline
\ \ \isakeyword{assumes}\ {\isachardoublequoteopen}Hintikka\ S{\isachardoublequoteclose}\ \isanewline
\ \ \isakeyword{shows}\ {\isachardoublequoteopen}\isactrlbold {\isasymnot}\ {\isasymbottom}\ {\isasymin}\ S\ {\isasymlongrightarrow}\ {\isasymbottom}\ {\isasymnotin}\ S{\isachardoublequoteclose}\isanewline
%
\isadelimproof
%
\endisadelimproof
%
\isatagproof
\isacommand{proof}\isamarkupfalse%
\ {\isacharparenleft}rule\ impI{\isacharparenright}\isanewline
\ \ \isacommand{assume}\isamarkupfalse%
\ {\isachardoublequoteopen}\isactrlbold {\isasymnot}\ {\isasymbottom}\ {\isasymin}\ S{\isachardoublequoteclose}\ \isanewline
\ \ \isacommand{show}\isamarkupfalse%
\ {\isachardoublequoteopen}{\isasymbottom}\ {\isasymnotin}\ S{\isachardoublequoteclose}\isanewline
\ \ \ \ \isacommand{using}\isamarkupfalse%
\ assms\ \isacommand{by}\isamarkupfalse%
\ {\isacharparenleft}rule\ Hintikka{\isacharunderscore}l{\isadigit{1}}{\isacharparenright}\isanewline
\isacommand{qed}\isamarkupfalse%
%
\endisatagproof
{\isafoldproof}%
%
\isadelimproof
\isanewline
%
\endisadelimproof
\isanewline
\isacommand{lemma}\isamarkupfalse%
\ Hintikka{\isacharunderscore}l{\isadigit{1}}{\isadigit{0}}{\isacharunderscore}not{\isacharcolon}\ \isanewline
\ \ \isakeyword{assumes}\ {\isachardoublequoteopen}Hintikka\ S\ {\isasymLongrightarrow}\ \isactrlbold {\isasymnot}\ F\ {\isasymin}\ S\ {\isasymlongrightarrow}\ F\ {\isasymnotin}\ S{\isachardoublequoteclose}\isanewline
\ \ \ \ \ \ \ \ \ \ {\isachardoublequoteopen}Hintikka\ S{\isachardoublequoteclose}\isanewline
\ \ \ \ \ \ \ \ \isakeyword{shows}\ {\isachardoublequoteopen}\isactrlbold {\isasymnot}\ {\isacharparenleft}\isactrlbold {\isasymnot}\ F{\isacharparenright}\ {\isasymin}\ S\ {\isasymlongrightarrow}\ \isactrlbold {\isasymnot}\ F\ {\isasymnotin}\ S{\isachardoublequoteclose}\isanewline
%
\isadelimproof
%
\endisadelimproof
%
\isatagproof
\isacommand{proof}\isamarkupfalse%
\ {\isacharparenleft}rule\ impI{\isacharparenright}\isanewline
\ \ \isacommand{assume}\isamarkupfalse%
\ {\isachardoublequoteopen}\isactrlbold {\isasymnot}\ {\isacharparenleft}\isactrlbold {\isasymnot}\ F{\isacharparenright}\ {\isasymin}\ S{\isachardoublequoteclose}\isanewline
\ \ \isacommand{have}\isamarkupfalse%
\ {\isachardoublequoteopen}\isactrlbold {\isasymnot}\ {\isacharparenleft}\isactrlbold {\isasymnot}\ F{\isacharparenright}\ {\isasymin}\ S\ {\isasymlongrightarrow}\ F\ {\isasymin}\ S{\isachardoublequoteclose}\isanewline
\ \ \ \ \isacommand{using}\isamarkupfalse%
\ assms{\isacharparenleft}{\isadigit{2}}{\isacharparenright}\ \isacommand{by}\isamarkupfalse%
\ {\isacharparenleft}rule\ Hintikka{\isacharunderscore}l{\isadigit{6}}{\isacharparenright}\isanewline
\ \ \isacommand{then}\isamarkupfalse%
\ \isacommand{have}\isamarkupfalse%
\ {\isachardoublequoteopen}F\ {\isasymin}\ S{\isachardoublequoteclose}\isanewline
\ \ \ \ \isacommand{using}\isamarkupfalse%
\ {\isacartoucheopen}\isactrlbold {\isasymnot}\ {\isacharparenleft}\isactrlbold {\isasymnot}\ F{\isacharparenright}\ {\isasymin}\ S{\isacartoucheclose}\ \isacommand{by}\isamarkupfalse%
\ {\isacharparenleft}rule\ mp{\isacharparenright}\isanewline
\ \ \isacommand{then}\isamarkupfalse%
\ \isacommand{have}\isamarkupfalse%
\ {\isachardoublequoteopen}{\isasymnot}\ {\isacharparenleft}F\ {\isasymnotin}\ S{\isacharparenright}{\isachardoublequoteclose}\isanewline
\ \ \ \ \isacommand{by}\isamarkupfalse%
\ {\isacharparenleft}rule\ contrapos{\isacharunderscore}pn{\isacharparenright}\isanewline
\ \ \isacommand{have}\isamarkupfalse%
\ {\isachardoublequoteopen}\isactrlbold {\isasymnot}\ F\ {\isasymin}\ S\ {\isasymlongrightarrow}\ F\ {\isasymnotin}\ S{\isachardoublequoteclose}\isanewline
\ \ \ \ \isacommand{using}\isamarkupfalse%
\ assms\ \isacommand{by}\isamarkupfalse%
\ this\isanewline
\ \ \isacommand{thus}\isamarkupfalse%
\ {\isachardoublequoteopen}\isactrlbold {\isasymnot}\ F\ {\isasymnotin}\ S{\isachardoublequoteclose}\isanewline
\ \ \ \ \isacommand{using}\isamarkupfalse%
\ {\isacartoucheopen}{\isasymnot}\ {\isacharparenleft}F\ {\isasymnotin}\ S{\isacharparenright}{\isacartoucheclose}\ \isacommand{by}\isamarkupfalse%
\ {\isacharparenleft}rule\ mt{\isacharparenright}\isanewline
\isacommand{qed}\isamarkupfalse%
%
\endisatagproof
{\isafoldproof}%
%
\isadelimproof
%
\endisadelimproof
%
\begin{isamarkuptext}%
Para facilitar las pruebas de los casos correspondientes a las
  conectivas binarias emplearemos los siguientes lemas auxiliares. Estos
  nos permitirán, a partir de la negación de un predicado, introducir 
  la negación de una conjunción o disyunción de dicho predicado con otro
  cualquiera.%
\end{isamarkuptext}\isamarkuptrue%
\isacommand{lemma}\isamarkupfalse%
\ notConj{\isadigit{1}}{\isacharcolon}\ \isanewline
\ \ \isakeyword{assumes}\ {\isachardoublequoteopen}{\isasymnot}\ P{\isachardoublequoteclose}\isanewline
\ \ \isakeyword{shows}\ {\isachardoublequoteopen}{\isasymnot}\ {\isacharparenleft}P\ {\isasymand}\ Q{\isacharparenright}{\isachardoublequoteclose}\isanewline
%
\isadelimproof
%
\endisadelimproof
%
\isatagproof
\isacommand{proof}\isamarkupfalse%
\ {\isacharparenleft}rule\ notI{\isacharparenright}\isanewline
\ \ \isacommand{assume}\isamarkupfalse%
\ {\isachardoublequoteopen}P\ {\isasymand}\ Q{\isachardoublequoteclose}\isanewline
\ \ \isacommand{then}\isamarkupfalse%
\ \isacommand{have}\isamarkupfalse%
\ {\isachardoublequoteopen}P{\isachardoublequoteclose}\isanewline
\ \ \ \ \isacommand{by}\isamarkupfalse%
\ {\isacharparenleft}rule\ conjunct{\isadigit{1}}{\isacharparenright}\isanewline
\ \ \isacommand{show}\isamarkupfalse%
\ {\isachardoublequoteopen}False{\isachardoublequoteclose}\isanewline
\ \ \ \ \isacommand{using}\isamarkupfalse%
\ assms\ {\isacartoucheopen}P{\isacartoucheclose}\ \isacommand{by}\isamarkupfalse%
\ {\isacharparenleft}rule\ notE{\isacharparenright}\isanewline
\isacommand{qed}\isamarkupfalse%
%
\endisatagproof
{\isafoldproof}%
%
\isadelimproof
\isanewline
%
\endisadelimproof
\isanewline
\isacommand{lemma}\isamarkupfalse%
\ notConj{\isadigit{2}}{\isacharcolon}\ \isanewline
\ \ \isakeyword{assumes}\ {\isachardoublequoteopen}{\isasymnot}\ Q{\isachardoublequoteclose}\isanewline
\ \ \isakeyword{shows}\ {\isachardoublequoteopen}{\isasymnot}\ {\isacharparenleft}P\ {\isasymand}\ Q{\isacharparenright}{\isachardoublequoteclose}\isanewline
%
\isadelimproof
%
\endisadelimproof
%
\isatagproof
\isacommand{proof}\isamarkupfalse%
\ {\isacharparenleft}rule\ notI{\isacharparenright}\isanewline
\ \ \isacommand{assume}\isamarkupfalse%
\ {\isachardoublequoteopen}P\ {\isasymand}\ Q{\isachardoublequoteclose}\isanewline
\ \ \isacommand{then}\isamarkupfalse%
\ \isacommand{have}\isamarkupfalse%
\ {\isachardoublequoteopen}Q{\isachardoublequoteclose}\isanewline
\ \ \ \ \isacommand{by}\isamarkupfalse%
\ {\isacharparenleft}rule\ conjunct{\isadigit{2}}{\isacharparenright}\isanewline
\ \ \isacommand{show}\isamarkupfalse%
\ {\isachardoublequoteopen}False{\isachardoublequoteclose}\isanewline
\ \ \ \ \isacommand{using}\isamarkupfalse%
\ assms\ {\isacartoucheopen}Q{\isacartoucheclose}\ \isacommand{by}\isamarkupfalse%
\ {\isacharparenleft}rule\ notE{\isacharparenright}\isanewline
\isacommand{qed}\isamarkupfalse%
%
\endisatagproof
{\isafoldproof}%
%
\isadelimproof
\isanewline
%
\endisadelimproof
\isanewline
\isacommand{lemma}\isamarkupfalse%
\ notDisj{\isacharcolon}\isanewline
\ \ \isakeyword{assumes}\ {\isachardoublequoteopen}{\isasymnot}\ P{\isachardoublequoteclose}\isanewline
\ \ \ \ \ \ \ \ \ \ {\isachardoublequoteopen}{\isasymnot}\ Q{\isachardoublequoteclose}\isanewline
\ \ \ \ \ \ \ \ \isakeyword{shows}\ {\isachardoublequoteopen}{\isasymnot}\ {\isacharparenleft}P\ {\isasymor}\ Q{\isacharparenright}{\isachardoublequoteclose}\isanewline
%
\isadelimproof
%
\endisadelimproof
%
\isatagproof
\isacommand{proof}\isamarkupfalse%
\ {\isacharparenleft}rule\ notI{\isacharparenright}\isanewline
\ \ \isacommand{assume}\isamarkupfalse%
\ {\isachardoublequoteopen}P\ {\isasymor}\ Q{\isachardoublequoteclose}\isanewline
\ \ \isacommand{then}\isamarkupfalse%
\ \isacommand{show}\isamarkupfalse%
\ {\isachardoublequoteopen}False{\isachardoublequoteclose}\isanewline
\ \ \isacommand{proof}\isamarkupfalse%
\ {\isacharparenleft}rule\ disjE{\isacharparenright}\isanewline
\ \ \ \ \isacommand{assume}\isamarkupfalse%
\ {\isachardoublequoteopen}P{\isachardoublequoteclose}\isanewline
\ \ \ \ \isacommand{show}\isamarkupfalse%
\ {\isachardoublequoteopen}False{\isachardoublequoteclose}\isanewline
\ \ \ \ \ \ \isacommand{using}\isamarkupfalse%
\ assms{\isacharparenleft}{\isadigit{1}}{\isacharparenright}\ {\isacartoucheopen}P{\isacartoucheclose}\ \isacommand{by}\isamarkupfalse%
\ {\isacharparenleft}rule\ notE{\isacharparenright}\isanewline
\ \ \isacommand{next}\isamarkupfalse%
\isanewline
\ \ \ \ \isacommand{assume}\isamarkupfalse%
\ {\isachardoublequoteopen}Q{\isachardoublequoteclose}\isanewline
\ \ \ \ \isacommand{show}\isamarkupfalse%
\ {\isachardoublequoteopen}False{\isachardoublequoteclose}\isanewline
\ \ \ \ \ \ \isacommand{using}\isamarkupfalse%
\ assms{\isacharparenleft}{\isadigit{2}}{\isacharparenright}\ {\isacartoucheopen}Q{\isacartoucheclose}\ \isacommand{by}\isamarkupfalse%
\ {\isacharparenleft}rule\ notE{\isacharparenright}\isanewline
\ \ \isacommand{qed}\isamarkupfalse%
\isanewline
\isacommand{qed}\isamarkupfalse%
%
\endisatagproof
{\isafoldproof}%
%
\isadelimproof
%
\endisadelimproof
%
\begin{isamarkuptext}%
Comencemos las pruebas detalladas de los casos correspondientes a 
  las conectivas binarias. Una vez terminadas, se mostrará la prueba 
  detallada del lema completo.%
\end{isamarkuptext}\isamarkuptrue%
\isacommand{lemma}\isamarkupfalse%
\ Hintikka{\isacharunderscore}l{\isadigit{1}}{\isadigit{0}}{\isacharunderscore}and{\isacharcolon}\ \isanewline
\ \ \isakeyword{assumes}\ {\isachardoublequoteopen}Hintikka\ S\ {\isasymLongrightarrow}\ \isactrlbold {\isasymnot}\ G\ {\isasymin}\ S\ {\isasymlongrightarrow}\ G\ {\isasymnotin}\ S{\isachardoublequoteclose}\isanewline
\ \ \ \ \ \ \ \ \ \ {\isachardoublequoteopen}Hintikka\ S\ {\isasymLongrightarrow}\ \isactrlbold {\isasymnot}\ H\ {\isasymin}\ S\ {\isasymlongrightarrow}\ H\ {\isasymnotin}\ S{\isachardoublequoteclose}\isanewline
\ \ \ \ \ \ \ \ \ \ {\isachardoublequoteopen}Hintikka\ S{\isachardoublequoteclose}\isanewline
\ \ \isakeyword{shows}\ {\isachardoublequoteopen}\isactrlbold {\isasymnot}\ {\isacharparenleft}G\ \isactrlbold {\isasymand}\ H{\isacharparenright}\ {\isasymin}\ S\ {\isasymlongrightarrow}\ G\ \isactrlbold {\isasymand}\ H\ {\isasymnotin}\ S{\isachardoublequoteclose}\isanewline
%
\isadelimproof
%
\endisadelimproof
%
\isatagproof
\isacommand{proof}\isamarkupfalse%
\ {\isacharparenleft}rule\ impI{\isacharparenright}\isanewline
\ \ \isacommand{assume}\isamarkupfalse%
\ {\isachardoublequoteopen}\isactrlbold {\isasymnot}\ {\isacharparenleft}G\ \isactrlbold {\isasymand}\ H{\isacharparenright}\ {\isasymin}\ S{\isachardoublequoteclose}\isanewline
\ \ \isacommand{have}\isamarkupfalse%
\ {\isachardoublequoteopen}\isactrlbold {\isasymnot}\ {\isacharparenleft}G\ \isactrlbold {\isasymand}\ H{\isacharparenright}\ {\isasymin}\ S\ {\isasymlongrightarrow}\ \isactrlbold {\isasymnot}\ G\ {\isasymin}\ S\ {\isasymor}\ \isactrlbold {\isasymnot}\ H\ {\isasymin}\ S{\isachardoublequoteclose}\isanewline
\ \ \ \ \isacommand{using}\isamarkupfalse%
\ assms{\isacharparenleft}{\isadigit{3}}{\isacharparenright}\ \isacommand{by}\isamarkupfalse%
\ {\isacharparenleft}rule\ Hintikka{\isacharunderscore}l{\isadigit{7}}{\isacharparenright}\isanewline
\ \ \isacommand{then}\isamarkupfalse%
\ \isacommand{have}\isamarkupfalse%
\ {\isachardoublequoteopen}\isactrlbold {\isasymnot}\ G\ {\isasymin}\ S\ {\isasymor}\ \isactrlbold {\isasymnot}\ H\ {\isasymin}\ S{\isachardoublequoteclose}\isanewline
\ \ \ \ \isacommand{using}\isamarkupfalse%
\ {\isacartoucheopen}\isactrlbold {\isasymnot}\ {\isacharparenleft}G\ \isactrlbold {\isasymand}\ H{\isacharparenright}\ {\isasymin}\ S{\isacartoucheclose}\ \isacommand{by}\isamarkupfalse%
\ {\isacharparenleft}rule\ mp{\isacharparenright}\isanewline
\ \ \isacommand{then}\isamarkupfalse%
\ \isacommand{show}\isamarkupfalse%
\ {\isachardoublequoteopen}G\ \isactrlbold {\isasymand}\ H\ {\isasymnotin}\ S{\isachardoublequoteclose}\isanewline
\ \ \isacommand{proof}\isamarkupfalse%
\ {\isacharparenleft}rule\ disjE{\isacharparenright}\isanewline
\ \ \ \ \isacommand{assume}\isamarkupfalse%
\ {\isachardoublequoteopen}\isactrlbold {\isasymnot}\ G\ {\isasymin}\ S{\isachardoublequoteclose}\isanewline
\ \ \ \ \isacommand{have}\isamarkupfalse%
\ {\isachardoublequoteopen}\isactrlbold {\isasymnot}\ G\ {\isasymin}\ S\ {\isasymlongrightarrow}\ G\ {\isasymnotin}\ S{\isachardoublequoteclose}\isanewline
\ \ \ \ \ \ \isacommand{using}\isamarkupfalse%
\ assms{\isacharparenleft}{\isadigit{1}}{\isacharparenright}\ assms{\isacharparenleft}{\isadigit{3}}{\isacharparenright}\ \isacommand{by}\isamarkupfalse%
\ this\isanewline
\ \ \ \ \isacommand{then}\isamarkupfalse%
\ \isacommand{have}\isamarkupfalse%
\ {\isachardoublequoteopen}G\ {\isasymnotin}\ S{\isachardoublequoteclose}\isanewline
\ \ \ \ \ \ \isacommand{using}\isamarkupfalse%
\ {\isacartoucheopen}\isactrlbold {\isasymnot}\ G\ {\isasymin}\ S{\isacartoucheclose}\ \isacommand{by}\isamarkupfalse%
\ {\isacharparenleft}rule\ mp{\isacharparenright}\isanewline
\ \ \ \ \isacommand{then}\isamarkupfalse%
\ \isacommand{have}\isamarkupfalse%
\ {\isachardoublequoteopen}{\isasymnot}\ {\isacharparenleft}G\ {\isasymin}\ S\ {\isasymand}\ H\ {\isasymin}\ S{\isacharparenright}{\isachardoublequoteclose}\isanewline
\ \ \ \ \ \ \isacommand{by}\isamarkupfalse%
\ {\isacharparenleft}rule\ notConj{\isadigit{1}}{\isacharparenright}\isanewline
\ \ \ \ \isacommand{have}\isamarkupfalse%
\ {\isachardoublequoteopen}G\ \isactrlbold {\isasymand}\ H\ {\isasymin}\ S\ {\isasymlongrightarrow}\ G\ {\isasymin}\ S\ {\isasymand}\ H\ {\isasymin}\ S{\isachardoublequoteclose}\isanewline
\ \ \ \ \ \ \isacommand{using}\isamarkupfalse%
\ assms{\isacharparenleft}{\isadigit{3}}{\isacharparenright}\ \isacommand{by}\isamarkupfalse%
\ {\isacharparenleft}rule\ Hintikka{\isacharunderscore}l{\isadigit{3}}{\isacharparenright}\isanewline
\ \ \ \ \isacommand{thus}\isamarkupfalse%
\ {\isachardoublequoteopen}G\ \isactrlbold {\isasymand}\ H\ {\isasymnotin}\ S{\isachardoublequoteclose}\isanewline
\ \ \ \ \ \ \isacommand{using}\isamarkupfalse%
\ {\isacartoucheopen}{\isasymnot}\ {\isacharparenleft}G\ {\isasymin}\ S\ {\isasymand}\ H\ {\isasymin}\ S{\isacharparenright}{\isacartoucheclose}\ \ \isacommand{by}\isamarkupfalse%
\ {\isacharparenleft}rule\ mt{\isacharparenright}\isanewline
\ \ \isacommand{next}\isamarkupfalse%
\isanewline
\ \ \ \ \isacommand{assume}\isamarkupfalse%
\ {\isachardoublequoteopen}\isactrlbold {\isasymnot}\ H\ {\isasymin}\ S{\isachardoublequoteclose}\isanewline
\ \ \ \ \isacommand{have}\isamarkupfalse%
\ {\isachardoublequoteopen}\isactrlbold {\isasymnot}\ H\ {\isasymin}\ S\ {\isasymlongrightarrow}\ H\ {\isasymnotin}\ S{\isachardoublequoteclose}\isanewline
\ \ \ \ \ \ \isacommand{using}\isamarkupfalse%
\ assms{\isacharparenleft}{\isadigit{2}}{\isacharparenright}\ assms{\isacharparenleft}{\isadigit{3}}{\isacharparenright}\ \isacommand{by}\isamarkupfalse%
\ this\isanewline
\ \ \ \ \isacommand{then}\isamarkupfalse%
\ \isacommand{have}\isamarkupfalse%
\ {\isachardoublequoteopen}H\ {\isasymnotin}\ S{\isachardoublequoteclose}\isanewline
\ \ \ \ \ \ \isacommand{using}\isamarkupfalse%
\ {\isacartoucheopen}\isactrlbold {\isasymnot}\ H\ {\isasymin}\ S{\isacartoucheclose}\ \isacommand{by}\isamarkupfalse%
\ {\isacharparenleft}rule\ mp{\isacharparenright}\isanewline
\ \ \ \ \isacommand{then}\isamarkupfalse%
\ \isacommand{have}\isamarkupfalse%
\ {\isachardoublequoteopen}{\isasymnot}\ {\isacharparenleft}G\ {\isasymin}\ S\ {\isasymand}\ H\ {\isasymin}\ S{\isacharparenright}{\isachardoublequoteclose}\ \isanewline
\ \ \ \ \ \ \isacommand{by}\isamarkupfalse%
\ {\isacharparenleft}rule\ notConj{\isadigit{2}}{\isacharparenright}\isanewline
\ \ \ \ \isacommand{have}\isamarkupfalse%
\ {\isachardoublequoteopen}G\ \isactrlbold {\isasymand}\ H\ {\isasymin}\ S\ {\isasymlongrightarrow}\ G\ {\isasymin}\ S\ {\isasymand}\ H\ {\isasymin}\ S{\isachardoublequoteclose}\isanewline
\ \ \ \ \ \ \isacommand{using}\isamarkupfalse%
\ assms{\isacharparenleft}{\isadigit{3}}{\isacharparenright}\ \isacommand{by}\isamarkupfalse%
\ {\isacharparenleft}rule\ Hintikka{\isacharunderscore}l{\isadigit{3}}{\isacharparenright}\isanewline
\ \ \ \ \isacommand{thus}\isamarkupfalse%
\ {\isachardoublequoteopen}G\ \isactrlbold {\isasymand}\ H\ {\isasymnotin}\ S{\isachardoublequoteclose}\isanewline
\ \ \ \ \ \ \isacommand{using}\isamarkupfalse%
\ {\isacartoucheopen}{\isasymnot}\ {\isacharparenleft}G\ {\isasymin}\ S\ {\isasymand}\ H\ {\isasymin}\ S{\isacharparenright}{\isacartoucheclose}\ \isacommand{by}\isamarkupfalse%
\ {\isacharparenleft}rule\ mt{\isacharparenright}\isanewline
\ \ \isacommand{qed}\isamarkupfalse%
\isanewline
\isacommand{qed}\isamarkupfalse%
%
\endisatagproof
{\isafoldproof}%
%
\isadelimproof
\isanewline
%
\endisadelimproof
\isanewline
\isacommand{lemma}\isamarkupfalse%
\ Hintikka{\isacharunderscore}l{\isadigit{1}}{\isadigit{0}}{\isacharunderscore}or{\isacharcolon}\ \isanewline
\ \ \isakeyword{assumes}\ {\isachardoublequoteopen}Hintikka\ S\ {\isasymLongrightarrow}\ \isactrlbold {\isasymnot}\ G\ {\isasymin}\ S\ {\isasymlongrightarrow}\ G\ {\isasymnotin}\ S{\isachardoublequoteclose}\isanewline
\ \ \ \ \ \ \ \ \ \ {\isachardoublequoteopen}Hintikka\ S\ {\isasymLongrightarrow}\ \isactrlbold {\isasymnot}\ H\ {\isasymin}\ S\ {\isasymlongrightarrow}\ H\ {\isasymnotin}\ S{\isachardoublequoteclose}\isanewline
\ \ \ \ \ \ \ \ \ \ {\isachardoublequoteopen}Hintikka\ S{\isachardoublequoteclose}\isanewline
\ \ \isakeyword{shows}\ {\isachardoublequoteopen}\isactrlbold {\isasymnot}\ {\isacharparenleft}G\ \isactrlbold {\isasymor}\ H{\isacharparenright}\ {\isasymin}\ S\ {\isasymlongrightarrow}\ G\ \isactrlbold {\isasymor}\ H\ {\isasymnotin}\ S{\isachardoublequoteclose}\isanewline
%
\isadelimproof
%
\endisadelimproof
%
\isatagproof
\isacommand{proof}\isamarkupfalse%
\ {\isacharparenleft}rule\ impI{\isacharparenright}\isanewline
\ \ \isacommand{assume}\isamarkupfalse%
\ {\isachardoublequoteopen}\isactrlbold {\isasymnot}\ {\isacharparenleft}G\ \isactrlbold {\isasymor}\ H{\isacharparenright}\ {\isasymin}\ S{\isachardoublequoteclose}\isanewline
\ \ \isacommand{have}\isamarkupfalse%
\ {\isachardoublequoteopen}\isactrlbold {\isasymnot}\ {\isacharparenleft}G\ \isactrlbold {\isasymor}\ H{\isacharparenright}\ {\isasymin}\ S\ {\isasymlongrightarrow}\ {\isacharparenleft}\isactrlbold {\isasymnot}\ G\ {\isasymin}\ S\ {\isasymand}\ \isactrlbold {\isasymnot}\ H\ {\isasymin}\ S{\isacharparenright}{\isachardoublequoteclose}\isanewline
\ \ \ \ \isacommand{using}\isamarkupfalse%
\ assms{\isacharparenleft}{\isadigit{3}}{\isacharparenright}\ \isacommand{by}\isamarkupfalse%
\ {\isacharparenleft}rule\ Hintikka{\isacharunderscore}l{\isadigit{8}}{\isacharparenright}\isanewline
\ \ \isacommand{then}\isamarkupfalse%
\ \isacommand{have}\isamarkupfalse%
\ C{\isacharcolon}{\isachardoublequoteopen}\isactrlbold {\isasymnot}\ G\ {\isasymin}\ S\ {\isasymand}\ \isactrlbold {\isasymnot}\ H\ {\isasymin}\ S{\isachardoublequoteclose}\isanewline
\ \ \ \ \isacommand{using}\isamarkupfalse%
\ {\isacartoucheopen}\isactrlbold {\isasymnot}\ {\isacharparenleft}G\ \isactrlbold {\isasymor}\ H{\isacharparenright}\ {\isasymin}\ S{\isacartoucheclose}\ \isacommand{by}\isamarkupfalse%
\ {\isacharparenleft}rule\ mp{\isacharparenright}\isanewline
\ \ \isacommand{then}\isamarkupfalse%
\ \isacommand{have}\isamarkupfalse%
\ {\isachardoublequoteopen}\isactrlbold {\isasymnot}\ G\ {\isasymin}\ S{\isachardoublequoteclose}\isanewline
\ \ \ \ \isacommand{by}\isamarkupfalse%
\ {\isacharparenleft}rule\ conjunct{\isadigit{1}}{\isacharparenright}\isanewline
\ \ \isacommand{have}\isamarkupfalse%
\ {\isachardoublequoteopen}\isactrlbold {\isasymnot}\ G\ {\isasymin}\ S\ {\isasymlongrightarrow}\ G\ {\isasymnotin}\ S{\isachardoublequoteclose}\isanewline
\ \ \ \ \isacommand{using}\isamarkupfalse%
\ assms{\isacharparenleft}{\isadigit{1}}{\isacharparenright}\ assms{\isacharparenleft}{\isadigit{3}}{\isacharparenright}\ \isacommand{by}\isamarkupfalse%
\ this\isanewline
\ \ \isacommand{then}\isamarkupfalse%
\ \isacommand{have}\isamarkupfalse%
\ {\isachardoublequoteopen}G\ {\isasymnotin}\ S{\isachardoublequoteclose}\ \isanewline
\ \ \ \ \isacommand{using}\isamarkupfalse%
\ {\isacartoucheopen}\isactrlbold {\isasymnot}\ G\ {\isasymin}\ S{\isacartoucheclose}\ \isacommand{by}\isamarkupfalse%
\ {\isacharparenleft}rule\ mp{\isacharparenright}\ \isanewline
\ \ \isacommand{have}\isamarkupfalse%
\ {\isachardoublequoteopen}\isactrlbold {\isasymnot}\ H\ {\isasymin}\ S{\isachardoublequoteclose}\isanewline
\ \ \ \ \isacommand{using}\isamarkupfalse%
\ C\ \isacommand{by}\isamarkupfalse%
\ {\isacharparenleft}rule\ conjunct{\isadigit{2}}{\isacharparenright}\isanewline
\ \ \isacommand{have}\isamarkupfalse%
\ {\isachardoublequoteopen}\isactrlbold {\isasymnot}\ H\ {\isasymin}\ S\ {\isasymlongrightarrow}\ H\ {\isasymnotin}\ S{\isachardoublequoteclose}\isanewline
\ \ \ \ \isacommand{using}\isamarkupfalse%
\ assms{\isacharparenleft}{\isadigit{2}}{\isacharparenright}\ assms{\isacharparenleft}{\isadigit{3}}{\isacharparenright}\ \isacommand{by}\isamarkupfalse%
\ this\isanewline
\ \ \isacommand{then}\isamarkupfalse%
\ \isacommand{have}\isamarkupfalse%
\ {\isachardoublequoteopen}H\ {\isasymnotin}\ S{\isachardoublequoteclose}\ \isanewline
\ \ \ \ \isacommand{using}\isamarkupfalse%
\ {\isacartoucheopen}\isactrlbold {\isasymnot}\ H\ {\isasymin}\ S{\isacartoucheclose}\ \isacommand{by}\isamarkupfalse%
\ {\isacharparenleft}rule\ mp{\isacharparenright}\isanewline
\ \ \isacommand{have}\isamarkupfalse%
\ {\isachardoublequoteopen}{\isasymnot}\ {\isacharparenleft}G\ {\isasymin}\ S\ {\isasymor}\ H\ {\isasymin}\ S{\isacharparenright}{\isachardoublequoteclose}\isanewline
\ \ \ \ \isacommand{using}\isamarkupfalse%
\ {\isacartoucheopen}G\ {\isasymnotin}\ S{\isacartoucheclose}\ {\isacartoucheopen}H\ {\isasymnotin}\ S{\isacartoucheclose}\ \isacommand{by}\isamarkupfalse%
\ {\isacharparenleft}rule\ notDisj{\isacharparenright}\isanewline
\ \ \isacommand{have}\isamarkupfalse%
\ {\isachardoublequoteopen}{\isacharparenleft}G\ \isactrlbold {\isasymor}\ H\ {\isasymin}\ S\ {\isasymlongrightarrow}\ G\ {\isasymin}\ S\ {\isasymor}\ H\ {\isasymin}\ S{\isacharparenright}{\isachardoublequoteclose}\isanewline
\ \ \ \ \isacommand{using}\isamarkupfalse%
\ assms{\isacharparenleft}{\isadigit{3}}{\isacharparenright}\ \isacommand{by}\isamarkupfalse%
\ {\isacharparenleft}rule\ Hintikka{\isacharunderscore}l{\isadigit{4}}{\isacharparenright}\isanewline
\ \ \isacommand{thus}\isamarkupfalse%
\ {\isachardoublequoteopen}G\ \isactrlbold {\isasymor}\ H\ {\isasymnotin}\ S{\isachardoublequoteclose}\isanewline
\ \ \ \ \isacommand{using}\isamarkupfalse%
\ {\isacartoucheopen}{\isasymnot}\ {\isacharparenleft}G\ {\isasymin}\ S\ {\isasymor}\ H\ {\isasymin}\ S{\isacharparenright}{\isacartoucheclose}\ \isacommand{by}\isamarkupfalse%
\ {\isacharparenleft}rule\ mt{\isacharparenright}\isanewline
\isacommand{qed}\isamarkupfalse%
%
\endisatagproof
{\isafoldproof}%
%
\isadelimproof
\isanewline
%
\endisadelimproof
\isanewline
\isacommand{lemma}\isamarkupfalse%
\ Hintikka{\isacharunderscore}l{\isadigit{1}}{\isadigit{0}}{\isacharunderscore}imp{\isacharcolon}\ \isanewline
\ \ \isakeyword{assumes}\ {\isachardoublequoteopen}Hintikka\ S\ {\isasymLongrightarrow}\ \isactrlbold {\isasymnot}\ G\ {\isasymin}\ S\ {\isasymlongrightarrow}\ G\ {\isasymnotin}\ S{\isachardoublequoteclose}\isanewline
\ \ \ \ \ \ \ \ \ \ {\isachardoublequoteopen}Hintikka\ S\ {\isasymLongrightarrow}\ \isactrlbold {\isasymnot}\ H\ {\isasymin}\ S\ {\isasymlongrightarrow}\ H\ {\isasymnotin}\ S{\isachardoublequoteclose}\isanewline
\ \ \ \ \ \ \ \ \ \ {\isachardoublequoteopen}Hintikka\ S{\isachardoublequoteclose}\isanewline
\ \ \isakeyword{shows}\ {\isachardoublequoteopen}\isactrlbold {\isasymnot}\ {\isacharparenleft}G\ \isactrlbold {\isasymrightarrow}\ H{\isacharparenright}\ {\isasymin}\ S\ {\isasymlongrightarrow}\ G\ \isactrlbold {\isasymrightarrow}\ H\ {\isasymnotin}\ S{\isachardoublequoteclose}\isanewline
%
\isadelimproof
%
\endisadelimproof
%
\isatagproof
\isacommand{proof}\isamarkupfalse%
\ {\isacharparenleft}rule\ impI{\isacharparenright}\isanewline
\ \ \isacommand{assume}\isamarkupfalse%
\ {\isachardoublequoteopen}\isactrlbold {\isasymnot}\ {\isacharparenleft}G\ \isactrlbold {\isasymrightarrow}\ H{\isacharparenright}\ {\isasymin}\ S{\isachardoublequoteclose}\isanewline
\ \ \isacommand{have}\isamarkupfalse%
\ {\isachardoublequoteopen}\isactrlbold {\isasymnot}\ {\isacharparenleft}G\ \isactrlbold {\isasymrightarrow}\ H{\isacharparenright}\ {\isasymin}\ S\ {\isasymlongrightarrow}\ {\isacharparenleft}G\ {\isasymin}\ S\ {\isasymand}\ \isactrlbold {\isasymnot}\ H\ {\isasymin}\ S{\isacharparenright}{\isachardoublequoteclose}\isanewline
\ \ \ \ \isacommand{using}\isamarkupfalse%
\ assms{\isacharparenleft}{\isadigit{3}}{\isacharparenright}\ \isacommand{by}\isamarkupfalse%
\ {\isacharparenleft}rule\ Hintikka{\isacharunderscore}l{\isadigit{9}}{\isacharparenright}\isanewline
\ \ \isacommand{then}\isamarkupfalse%
\ \isacommand{have}\isamarkupfalse%
\ C{\isacharcolon}{\isachardoublequoteopen}G\ {\isasymin}\ S\ {\isasymand}\ \isactrlbold {\isasymnot}\ H\ {\isasymin}\ S{\isachardoublequoteclose}\isanewline
\ \ \ \ \isacommand{using}\isamarkupfalse%
\ {\isacartoucheopen}\isactrlbold {\isasymnot}\ {\isacharparenleft}G\ \isactrlbold {\isasymrightarrow}\ H{\isacharparenright}\ {\isasymin}\ S{\isacartoucheclose}\ \isacommand{by}\isamarkupfalse%
\ {\isacharparenleft}rule\ mp{\isacharparenright}\isanewline
\ \ \isacommand{then}\isamarkupfalse%
\ \isacommand{have}\isamarkupfalse%
\ {\isachardoublequoteopen}G\ {\isasymin}\ S{\isachardoublequoteclose}\isanewline
\ \ \ \ \isacommand{by}\isamarkupfalse%
\ {\isacharparenleft}rule\ conjunct{\isadigit{1}}{\isacharparenright}\isanewline
\ \ \isacommand{then}\isamarkupfalse%
\ \isacommand{have}\isamarkupfalse%
\ {\isachardoublequoteopen}{\isasymnot}\ {\isacharparenleft}G\ {\isasymnotin}\ S{\isacharparenright}{\isachardoublequoteclose}\isanewline
\ \ \ \ \isacommand{by}\isamarkupfalse%
\ {\isacharparenleft}rule\ contrapos{\isacharunderscore}pn{\isacharparenright}\isanewline
\ \ \isacommand{have}\isamarkupfalse%
\ {\isachardoublequoteopen}\isactrlbold {\isasymnot}\ G\ {\isasymin}\ S\ {\isasymlongrightarrow}\ G\ {\isasymnotin}\ S{\isachardoublequoteclose}\isanewline
\ \ \ \ \isacommand{using}\isamarkupfalse%
\ assms{\isacharparenleft}{\isadigit{1}}{\isacharparenright}\ assms{\isacharparenleft}{\isadigit{3}}{\isacharparenright}\ \isacommand{by}\isamarkupfalse%
\ this\isanewline
\ \ \isacommand{then}\isamarkupfalse%
\ \isacommand{have}\isamarkupfalse%
\ {\isachardoublequoteopen}\isactrlbold {\isasymnot}\ G\ {\isasymnotin}\ S{\isachardoublequoteclose}\isanewline
\ \ \ \ \isacommand{using}\isamarkupfalse%
\ {\isacartoucheopen}{\isasymnot}\ {\isacharparenleft}G\ {\isasymnotin}\ S{\isacharparenright}{\isacartoucheclose}\ \isacommand{by}\isamarkupfalse%
\ {\isacharparenleft}rule\ mt{\isacharparenright}\isanewline
\ \ \isacommand{have}\isamarkupfalse%
\ {\isachardoublequoteopen}\isactrlbold {\isasymnot}\ H\ {\isasymin}\ S{\isachardoublequoteclose}\isanewline
\ \ \ \ \isacommand{using}\isamarkupfalse%
\ C\ \isacommand{by}\isamarkupfalse%
\ {\isacharparenleft}rule\ conjunct{\isadigit{2}}{\isacharparenright}\isanewline
\ \ \isacommand{have}\isamarkupfalse%
\ {\isachardoublequoteopen}\isactrlbold {\isasymnot}\ H\ {\isasymin}\ S\ {\isasymlongrightarrow}\ H\ {\isasymnotin}\ S{\isachardoublequoteclose}\isanewline
\ \ \ \ \isacommand{using}\isamarkupfalse%
\ assms{\isacharparenleft}{\isadigit{2}}{\isacharparenright}\ assms{\isacharparenleft}{\isadigit{3}}{\isacharparenright}\ \isacommand{by}\isamarkupfalse%
\ this\isanewline
\ \ \isacommand{then}\isamarkupfalse%
\ \isacommand{have}\isamarkupfalse%
\ {\isachardoublequoteopen}H\ {\isasymnotin}\ S{\isachardoublequoteclose}\isanewline
\ \ \ \ \isacommand{using}\isamarkupfalse%
\ {\isacartoucheopen}\isactrlbold {\isasymnot}\ H\ {\isasymin}\ S{\isacartoucheclose}\ \isacommand{by}\isamarkupfalse%
\ {\isacharparenleft}rule\ mp{\isacharparenright}\isanewline
\ \ \isacommand{have}\isamarkupfalse%
\ {\isachardoublequoteopen}{\isasymnot}\ {\isacharparenleft}\isactrlbold {\isasymnot}\ G\ {\isasymin}\ S\ {\isasymor}\ H\ {\isasymin}\ S{\isacharparenright}{\isachardoublequoteclose}\isanewline
\ \ \ \ \isacommand{using}\isamarkupfalse%
\ {\isacartoucheopen}\isactrlbold {\isasymnot}\ G\ {\isasymnotin}\ S{\isacartoucheclose}\ {\isacartoucheopen}H\ {\isasymnotin}\ S{\isacartoucheclose}\ \isacommand{by}\isamarkupfalse%
\ {\isacharparenleft}rule\ notDisj{\isacharparenright}\isanewline
\ \ \isacommand{have}\isamarkupfalse%
\ {\isachardoublequoteopen}G\ \isactrlbold {\isasymrightarrow}\ H\ {\isasymin}\ S\ {\isasymlongrightarrow}\ \isactrlbold {\isasymnot}\ G\ {\isasymin}\ S\ {\isasymor}\ H\ {\isasymin}\ S{\isachardoublequoteclose}\isanewline
\ \ \ \ \isacommand{using}\isamarkupfalse%
\ assms{\isacharparenleft}{\isadigit{3}}{\isacharparenright}\ \isacommand{by}\isamarkupfalse%
\ {\isacharparenleft}rule\ Hintikka{\isacharunderscore}l{\isadigit{5}}{\isacharparenright}\isanewline
\ \ \isacommand{thus}\isamarkupfalse%
\ {\isachardoublequoteopen}G\ \isactrlbold {\isasymrightarrow}\ H\ {\isasymnotin}\ S{\isachardoublequoteclose}\isanewline
\ \ \ \ \isacommand{using}\isamarkupfalse%
\ {\isacartoucheopen}{\isasymnot}\ {\isacharparenleft}\isactrlbold {\isasymnot}\ G\ {\isasymin}\ S\ {\isasymor}\ H\ {\isasymin}\ S{\isacharparenright}{\isacartoucheclose}\ \isacommand{by}\isamarkupfalse%
\ {\isacharparenleft}rule\ mt{\isacharparenright}\isanewline
\isacommand{qed}\isamarkupfalse%
%
\endisatagproof
{\isafoldproof}%
%
\isadelimproof
\isanewline
%
\endisadelimproof
\isanewline
\isacommand{lemma}\isamarkupfalse%
\ {\isachardoublequoteopen}Hintikka\ S\ {\isasymLongrightarrow}\ \isactrlbold {\isasymnot}\ F\ {\isasymin}\ S\ {\isasymlongrightarrow}\ F\ {\isasymnotin}\ S{\isachardoublequoteclose}\isanewline
%
\isadelimproof
%
\endisadelimproof
%
\isatagproof
\isacommand{proof}\isamarkupfalse%
\ {\isacharparenleft}induct\ F{\isacharparenright}\isanewline
\isacommand{case}\isamarkupfalse%
\ {\isacharparenleft}Atom\ x{\isacharparenright}\isanewline
\ \ \isacommand{then}\isamarkupfalse%
\ \isacommand{show}\isamarkupfalse%
\ {\isacharquery}case\ \isacommand{by}\isamarkupfalse%
\ {\isacharparenleft}rule\ Hintikka{\isacharunderscore}l{\isadigit{1}}{\isadigit{0}}{\isacharunderscore}atom{\isacharparenright}\isanewline
\isacommand{next}\isamarkupfalse%
\isanewline
\ \ \isacommand{case}\isamarkupfalse%
\ Bot\isanewline
\ \ \isacommand{then}\isamarkupfalse%
\ \isacommand{show}\isamarkupfalse%
\ {\isacharquery}case\ \isacommand{by}\isamarkupfalse%
\ {\isacharparenleft}rule\ Hintikka{\isacharunderscore}l{\isadigit{1}}{\isadigit{0}}{\isacharunderscore}bot{\isacharparenright}\isanewline
\isacommand{next}\isamarkupfalse%
\isanewline
\ \ \isacommand{case}\isamarkupfalse%
\ {\isacharparenleft}Not\ F{\isacharparenright}\isanewline
\ \ \isacommand{then}\isamarkupfalse%
\ \isacommand{show}\isamarkupfalse%
\ {\isacharquery}case\ \isacommand{by}\isamarkupfalse%
\ {\isacharparenleft}rule\ Hintikka{\isacharunderscore}l{\isadigit{1}}{\isadigit{0}}{\isacharunderscore}not{\isacharparenright}\isanewline
\isacommand{next}\isamarkupfalse%
\isanewline
\ \ \isacommand{case}\isamarkupfalse%
\ {\isacharparenleft}And\ F{\isadigit{1}}\ F{\isadigit{2}}{\isacharparenright}\isanewline
\ \ \isacommand{then}\isamarkupfalse%
\ \isacommand{show}\isamarkupfalse%
\ {\isacharquery}case\ \isacommand{by}\isamarkupfalse%
\ {\isacharparenleft}rule\ Hintikka{\isacharunderscore}l{\isadigit{1}}{\isadigit{0}}{\isacharunderscore}and{\isacharparenright}\isanewline
\isacommand{next}\isamarkupfalse%
\isanewline
\ \ \isacommand{case}\isamarkupfalse%
\ {\isacharparenleft}Or\ F{\isadigit{1}}\ F{\isadigit{2}}{\isacharparenright}\isanewline
\ \ \isacommand{then}\isamarkupfalse%
\ \isacommand{show}\isamarkupfalse%
\ {\isacharquery}case\ \isacommand{by}\isamarkupfalse%
\ {\isacharparenleft}rule\ Hintikka{\isacharunderscore}l{\isadigit{1}}{\isadigit{0}}{\isacharunderscore}or{\isacharparenright}\isanewline
\isacommand{next}\isamarkupfalse%
\isanewline
\ \ \isacommand{case}\isamarkupfalse%
\ {\isacharparenleft}Imp\ F{\isadigit{1}}\ F{\isadigit{2}}{\isacharparenright}\isanewline
\ \ \isacommand{then}\isamarkupfalse%
\ \isacommand{show}\isamarkupfalse%
\ {\isacharquery}case\ \isacommand{by}\isamarkupfalse%
\ {\isacharparenleft}rule\ Hintikka{\isacharunderscore}l{\isadigit{1}}{\isadigit{0}}{\isacharunderscore}imp{\isacharparenright}\isanewline
\isacommand{qed}\isamarkupfalse%
%
\endisatagproof
{\isafoldproof}%
%
\isadelimproof
%
\endisadelimproof
%
\begin{isamarkuptext}%
Por último, su demostración automática es la que sigue.%
\end{isamarkuptext}\isamarkuptrue%
\isacommand{lemma}\isamarkupfalse%
\ Hintikka{\isacharunderscore}l{\isadigit{1}}{\isadigit{0}}{\isacharcolon}\ \isanewline
\ {\isachardoublequoteopen}Hintikka\ S\ {\isasymLongrightarrow}\ \isactrlbold {\isasymnot}\ F\ {\isasymin}\ S\ {\isasymlongrightarrow}\ F\ {\isasymnotin}\ S{\isachardoublequoteclose}\isanewline
%
\isadelimproof
\ \ %
\endisadelimproof
%
\isatagproof
\isacommand{apply}\isamarkupfalse%
\ {\isacharparenleft}induct\ F{\isacharparenright}\isanewline
\ \ \isacommand{apply}\isamarkupfalse%
\ {\isacharparenleft}meson\ Hintikka{\isacharunderscore}l{\isadigit{2}}{\isacharparenright}\isanewline
\ \ \isacommand{apply}\isamarkupfalse%
\ {\isacharparenleft}simp\ add{\isacharcolon}\ Hintikka{\isacharunderscore}l{\isadigit{1}}{\isacharparenright}\isanewline
\ \ \isacommand{using}\isamarkupfalse%
\ Hintikka{\isacharunderscore}l{\isadigit{6}}\ \isacommand{apply}\isamarkupfalse%
\ blast\isanewline
\ \ \isacommand{using}\isamarkupfalse%
\ Hintikka{\isacharunderscore}l{\isadigit{3}}\ Hintikka{\isacharunderscore}l{\isadigit{7}}\ \isacommand{apply}\isamarkupfalse%
\ blast\isanewline
\ \ \isacommand{apply}\isamarkupfalse%
\ {\isacharparenleft}smt\ Hintikka{\isacharunderscore}def{\isacharparenright}\isanewline
\ \ \isacommand{using}\isamarkupfalse%
\ Hintikka{\isacharunderscore}l{\isadigit{5}}\ Hintikka{\isacharunderscore}l{\isadigit{9}}\ \isacommand{by}\isamarkupfalse%
\ blast%
\endisatagproof
{\isafoldproof}%
%
\isadelimproof
%
\endisadelimproof
%
\isadelimdocument
%
\endisadelimdocument
%
\isatagdocument
%
\isamarkupsection{Lema de Hintikka%
}
\isamarkuptrue%
%
\endisatagdocument
{\isafolddocument}%
%
\isadelimdocument
%
\endisadelimdocument
%
\begin{isamarkuptext}%
Una vez definida la noción de conjunto de Hintikka y conocidas las
  propiedades que se deducen de ella, nuestro objetivo será demostrar
  que todo conjunto de Hintikka es satisfacible. Por definición, para 
  probar que un conjunto es satisfacible basta hallar una interpretación 
  que sea modelo suyo. De este modo, definimos el siguiente tipo de 
  interpretaciones.

  \begin{definicion}
    Sea un conjunto de fórmulas cualquiera. Se define la \isa{interpretación\ asociada\ al\ conjunto} como aquella que devuelve \isa{Verdadero} sobre 
    las variables proposicionales cuya correspondiente fórmula 
    atómica pertence al conjunto, y \isa{Falso} en caso contrario.
  \end{definicion}

  En Isabelle se formalizará mediante el tipo \isa{definition} como se
  expone a\\ continuación.%
\end{isamarkuptext}\isamarkuptrue%
\isacommand{definition}\isamarkupfalse%
\ setValuation\ {\isacharcolon}{\isacharcolon}\ \isanewline
\ \ \ {\isachardoublequoteopen}{\isacharparenleft}{\isacharprime}a\ formula{\isacharparenright}\ set\ {\isasymRightarrow}\ {\isacharprime}a\ valuation{\isachardoublequoteclose}\ \isakeyword{where}\isanewline
\ \ \ \ {\isachardoublequoteopen}setValuation\ S\ \ {\isasymequiv}\ {\isasymlambda}k{\isachardot}\ Atom\ k\ {\isasymin}\ S{\isachardoublequoteclose}%
\begin{isamarkuptext}%
Presentemos ahora ejemplos del valor de ciertas fórmulas 
  en la interpretación asociada a los conjuntos siguientes.%
\end{isamarkuptext}\isamarkuptrue%
\isacommand{notepad}\isamarkupfalse%
\isanewline
\isakeyword{begin}\isanewline
%
\isadelimproof
\isanewline
\ \ %
\endisadelimproof
%
\isatagproof
\isacommand{have}\isamarkupfalse%
\ {\isachardoublequoteopen}{\isacharparenleft}setValuation\ {\isacharbraceleft}Atom\ {\isadigit{0}}\ \isactrlbold {\isasymand}\ {\isacharparenleft}{\isacharparenleft}\isactrlbold {\isasymnot}\ {\isacharparenleft}Atom\ {\isadigit{1}}{\isacharparenright}{\isacharparenright}\ \isactrlbold {\isasymrightarrow}\ Atom\ {\isadigit{2}}{\isacharparenright}{\isacharcomma}\ \isanewline
\ \ \ \ \ \ \ \ \ \ \ \ {\isacharparenleft}{\isacharparenleft}\isactrlbold {\isasymnot}\ {\isacharparenleft}Atom\ {\isadigit{1}}{\isacharparenright}{\isacharparenright}\ \isactrlbold {\isasymrightarrow}\ Atom\ {\isadigit{2}}{\isacharparenright}{\isacharcomma}\ Atom\ {\isadigit{0}}{\isacharcomma}\isanewline
\ \ \ \ \ \ \ \ \ \ \ \ \isactrlbold {\isasymnot}{\isacharparenleft}\isactrlbold {\isasymnot}\ {\isacharparenleft}Atom\ {\isadigit{1}}{\isacharparenright}{\isacharparenright}{\isacharcomma}\ Atom\ {\isadigit{1}}{\isacharbraceright}{\isacharparenright}\ {\isasymTurnstile}\ Atom\ {\isadigit{1}}\ \isactrlbold {\isasymrightarrow}\ Atom\ {\isadigit{0}}\ {\isacharequal}\ True{\isachardoublequoteclose}\isanewline
\ \ \ \ \isacommand{unfolding}\isamarkupfalse%
\ setValuation{\isacharunderscore}def\ \isacommand{by}\isamarkupfalse%
\ simp\isanewline
\isanewline
\ \ \isacommand{have}\isamarkupfalse%
\ {\isachardoublequoteopen}{\isacharparenleft}setValuation\ {\isacharbraceleft}Atom\ {\isadigit{3}}\ \isactrlbold {\isasymor}\ {\isacharparenleft}\isactrlbold {\isasymnot}\ {\isacharparenleft}Atom\ {\isadigit{1}}{\isacharparenright}{\isacharparenright}{\isacharcomma}\ \isanewline
\ \ \ \ \ \ \ \ \ \ \ \ \isactrlbold {\isasymnot}\ {\isacharparenleft}\isactrlbold {\isasymnot}\ {\isacharparenleft}Atom\ {\isadigit{6}}{\isacharparenright}{\isacharparenright}{\isacharbraceright}{\isacharparenright}\ {\isasymTurnstile}\ Atom\ {\isadigit{2}}\ \isactrlbold {\isasymor}\ Atom\ {\isadigit{6}}\ {\isacharequal}\ False{\isachardoublequoteclose}\isanewline
\ \ \ \ \isacommand{unfolding}\isamarkupfalse%
\ setValuation{\isacharunderscore}def\ \isacommand{by}\isamarkupfalse%
\ simp%
\endisatagproof
{\isafoldproof}%
%
\isadelimproof
\isanewline
%
\endisadelimproof
\isanewline
\isacommand{end}\isamarkupfalse%
%
\begin{isamarkuptext}%
Previamente a probar que los conjuntos de Hintikka son 
  satisfacibles y con el fin de facilitar dicha demostración, 
  introducimos el siguiente resultado.

  \begin{lema}
    La interpretación asociada a un conjunto de Hintikka es modelo de
    una fórmula si esta pertenece al conjunto. Además, dicha 
    interpretación no es modelo de una fórmula si su negación 
    pertenece al conjunto.
  \end{lema}

  Su formalización en Isabelle es la siguiente.%
\end{isamarkuptext}\isamarkuptrue%
\isacommand{lemma}\isamarkupfalse%
\ {\isachardoublequoteopen}Hintikka\ S\ {\isasymLongrightarrow}\ {\isacharparenleft}F\ {\isasymin}\ S\ {\isasymlongrightarrow}\ isModel\ {\isacharparenleft}setValuation\ S{\isacharparenright}\ F{\isacharparenright}\isanewline
\ \ \ \ \ \ \ \ \ \ \ {\isasymand}\ {\isacharparenleft}\isactrlbold {\isasymnot}\ F\ {\isasymin}\ S\ {\isasymlongrightarrow}\ {\isacharparenleft}{\isasymnot}\ {\isacharparenleft}isModel\ {\isacharparenleft}setValuation\ S{\isacharparenright}\ F{\isacharparenright}{\isacharparenright}{\isacharparenright}{\isachardoublequoteclose}\isanewline
%
\isadelimproof
\ \ %
\endisadelimproof
%
\isatagproof
\isacommand{oops}\isamarkupfalse%
%
\endisatagproof
{\isafoldproof}%
%
\isadelimproof
%
\endisadelimproof
%
\begin{isamarkuptext}%
Procedamos a la demostración del resultado.

\begin{demostracion}
  Sea \isa{S} un conjunto de Hintikka y denotemos por \isa{{\isasymI}\isactrlsub S} a la 
  interpretación asociada a \isa{S}. Sea \isa{F} una fórmula, hay que probar lo 
  siguiente:

  \isa{{\isacharparenleft}F\ {\isasymin}\ S\ {\isasymLongrightarrow}\ {\isasymI}\isactrlsub S\ {\isasymTurnstile}\ F{\isacharparenright}\ {\isasymand}\ {\isacharparenleft}{\isasymnot}\ F\ {\isasymin}\ S\ {\isasymLongrightarrow}\ \ {\isasymI}\isactrlsub S}  $\not\models$ \isa{F{\isacharparenright}}

  La prueba se realiza por inducción sobre la estructura de las 
  fórmulas proposicionales. Veamos los distintos casos.

\begin{enumerate}
   \item[Caso 1:] \isa{F\ {\isacharequal}\ p}, fórmula atómica.

    Si \isa{p\ {\isasymin}\ S}, entonces \isa{{\isasymI}\isactrlsub S{\isacharparenleft}p{\isacharparenright}\ {\isacharequal}\ True} por definición de la 
    interpretación asociada a \isa{S}. 

    Por otro lado, si \isa{{\isasymnot}\ p\ {\isasymin}\ S}, entonces
    \isa{p\ {\isasymnotin}\ S} por ser \isa{S} de Hintikka. Por tanto,\\ \isa{{\isasymI}\isactrlsub S{\isacharparenleft}p{\isacharparenright}\ {\isacharequal}\ False} 
    por definición de \isa{{\isasymI}\isactrlsub S}. 

   \item[Caso 2:] \isa{F\ {\isacharequal}\ {\isasymbottom}}   

    Si \isa{{\isasymbottom}\ {\isasymin}\ S}, como por definición de conjunto de Hintikka sabemos que 
    \isa{{\isasymbottom}\ {\isasymnotin}\ S}, se tendría una contradicción. Luego, en particular, 
    tenemos el resultado.

    Por otra parte, si \isa{{\isasymnot}\ {\isasymbottom}\ {\isasymin}\ S}, \isa{{\isasymI}\isactrlsub S} no es modelo de \isa{{\isasymbottom}} pues el 
    valor de \isa{{\isasymbottom}} es \isa{Falso} en cualquier interpretación.

   \item[Caso 3:] \isa{F\ {\isacharequal}\ {\isasymnot}\ G}, y \isa{G} verifica la hipótesis de inducción. 

    Es decir, \isa{HI}: \isa{{\isacharparenleft}G\ {\isasymin}\ S\ {\isasymLongrightarrow}\ {\isasymI}\isactrlsub S\ {\isasymTurnstile}\ G{\isacharparenright}\ {\isasymand}\ {\isacharparenleft}{\isasymnot}\ G\ {\isasymin}\ S\ {\isasymLongrightarrow}\ {\isasymI}\isactrlsub S}  
        $\not\models$ \isa{G{\isacharparenright}}

    Probemos que  \isa{{\isacharparenleft}F\ {\isasymin}\ S\ {\isasymLongrightarrow}\ {\isasymI}\isactrlsub S\ {\isasymTurnstile}\ F{\isacharparenright}\ {\isasymand}\ {\isacharparenleft}{\isasymnot}\ F\ {\isasymin}\ S\ {\isasymLongrightarrow}\ \ {\isasymI}\isactrlsub S}  
        $\not\models$ \isa{F{\isacharparenright}}

    En efecto,
    $$\begin{array}{lrl}
    & \isa{F\ {\isasymin}\ S} & \\
\isa{{\isasymLongrightarrow}} & \isa{{\isasymnot}\ G\ {\isasymin}\ S} & \\
\isa{{\isasymLongrightarrow}} &  \isa{{\isasymI}\isactrlsub S{\isacharparenleft}G{\isacharparenright}}  \isa{{\isacharequal}\ False}& \isa{{\isacharparenleft}HI{\isacharparenright}}\\
\isa{{\isasymLongrightarrow}} & \isa{{\isasymI}\isactrlsub S{\isacharparenleft}{\isasymnot}\ G{\isacharparenright}} \isa{{\isacharequal}\ True} & \\
\isa{{\isasymLongrightarrow}} & \isa{{\isasymI}\isactrlsub S\ {\isasymTurnstile}\ F} &
      \end{array}$$ 

 Análogamente,

$$\begin{array}{lrl}
 & \isa{{\isasymnot}\ F\ {\isasymin}\ S} & \\
\isa{{\isasymLongrightarrow}} & \isa{{\isasymnot}{\isasymnot}\ G\ {\isasymin}\ S} & \\
\isa{{\isasymLongrightarrow}} & \isa{G\ {\isasymin}\ S}  & \isa{{\isacharparenleft}S\ es\ conjunto\ de\ Hintikka{\isacharparenright}}\\
\isa{{\isasymLongrightarrow}} &  \isa{{\isasymI}\isactrlsub S{\isacharparenleft}G{\isacharparenright}}  \isa{{\isacharequal}\ True} & \isa{{\isacharparenleft}HI{\isacharparenright}}\\
\isa{{\isasymLongrightarrow}} & \isa{{\isasymI}\isactrlsub S{\isacharparenleft}{\isasymnot}\ G{\isacharparenright}}  \isa{{\isacharequal}\ False} & \\
\isa{{\isasymLongrightarrow}} & \isa{{\isasymI}\isactrlsub S{\isacharparenleft}F{\isacharparenright}}  \isa{{\isacharequal}\ False} & \\
\isa{{\isasymLongrightarrow}} & \isa{{\isasymI}\isactrlsub S} \not\models \isa{F} & 
      \end{array}$$
  
 \item[Caso 4:] \isa{F\ {\isacharequal}\ G\ {\isasymand}\ H} y tanto \isa{G} como \isa{H} verifican la 
 hipótesis de inducción. 

 Es decir, se verifican 
  \begin{enumerate}
    \item [HI1:]  \isa{{\isacharparenleft}G\ {\isasymin}\ S\ {\isasymLongrightarrow}\ {\isasymI}\isactrlsub S\ {\isasymTurnstile}\ G{\isacharparenright}\ {\isasymand}\ {\isacharparenleft}{\isasymnot}\ G\ {\isasymin}\ S\ {\isasymLongrightarrow}\ \ {\isasymI}\isactrlsub S}  
        $\not\models$ \isa{G{\isacharparenright}}
    \item [HI2:]  \isa{{\isacharparenleft}H\ {\isasymin}\ S\ {\isasymLongrightarrow}\ {\isasymI}\isactrlsub S\ {\isasymTurnstile}\ H{\isacharparenright}\ {\isasymand}\ {\isacharparenleft}{\isasymnot}\ H\ {\isasymin}\ S\ {\isasymLongrightarrow}\ \ {\isasymI}\isactrlsub S}  
        $\not\models$ \isa{H{\isacharparenright}}
  \end{enumerate}
 
 Probemos que  \isa{{\isacharparenleft}F\ {\isasymin}\ S\ {\isasymLongrightarrow}\ {\isasymI}\isactrlsub S\ {\isasymTurnstile}\ F{\isacharparenright}\ {\isasymand}\ {\isacharparenleft}{\isasymnot}\ F\ {\isasymin}\ S\ {\isasymLongrightarrow}\ \ {\isasymI}\isactrlsub S}  
    $\not\models$ \isa{F{\isacharparenright}}

 En efecto,
$$\begin{array}{lrl}
 & \isa{F\ {\isasymin}\ S} & \\
\isa{{\isasymLongrightarrow}} & \isa{G\ {\isasymand}\ H\ {\isasymin}\ S} & \\
\isa{{\isasymLongrightarrow}} & \isa{G\ {\isasymin}\ S\ {\isasymand}\ H\ {\isasymin}\ S} & \isa{{\isacharparenleft}S\ es\ conjunto\ de\ Hintikka{\isacharparenright}} \\
\isa{{\isasymLongrightarrow}} & \isa{{\isasymI}\isactrlsub S\ {\isasymTurnstile}\ G\ {\isasymand}\ {\isasymI}\isactrlsub S\ {\isasymTurnstile}\ H} & \isa{{\isacharparenleft}HI{\isadigit{1}}\ y\ HI{\isadigit{2}}{\isacharparenright}} \\
\isa{{\isasymLongrightarrow}} & \isa{{\isasymI}\isactrlsub S{\isacharparenleft}G{\isacharparenright}\ {\isasymand}\ {\isasymI}\isactrlsub S{\isacharparenleft}H{\isacharparenright}} \isa{{\isacharequal}\ True}& \\
\isa{{\isasymLongrightarrow}} & \isa{{\isasymI}\isactrlsub S{\isacharparenleft}G\ {\isasymand}\ H{\isacharparenright}} \isa{{\isacharequal}\ True}& \\
\isa{{\isasymLongrightarrow}} & \isa{{\isasymI}\isactrlsub S{\isacharparenleft}F{\isacharparenright}} \isa{{\isacharequal}\ True}& \\
\isa{{\isasymLongrightarrow}} & \isa{{\isasymI}\isactrlsub S\ {\isasymTurnstile}\ F} & \\
      \end{array}$$ 

Por otra parte, 
$$\begin{array}{lrl}
 & \isa{{\isasymnot}\ F\ {\isasymin}\ S} & \\
\isa{{\isasymLongrightarrow}} & \isa{{\isasymnot}\ {\isacharparenleft}G\ {\isasymand}\ H{\isacharparenright}\ {\isasymin}\ S} & \\
\isa{{\isasymLongrightarrow}} & \isa{{\isasymnot}\ G\ {\isasymin}\ S\ {\isasymor}\ {\isasymnot}\ H\ {\isasymin}\ S} & \isa{{\isacharparenleft}S\ es\ conjunto\ de\ Hintikka{\isacharparenright}}\\
\isa{{\isasymLongrightarrow}} & \isa{{\isasymI}\isactrlsub S} \not\models \isa{G\ {\isasymor}\ {\isasymI}\isactrlsub S} \not\models \isa{H} & \isa{{\isacharparenleft}HI{\isadigit{1}}\ y\ HI{\isadigit{2}}{\isacharparenright}}\\
\isa{{\isasymLongrightarrow}} & \isa{{\isasymnot}\ {\isacharparenleft}{\isasymI}\isactrlsub S\ {\isasymTurnstile}\ G\ {\isasymand}\ {\isasymI}\isactrlsub S\ {\isasymTurnstile}\ H{\isacharparenright}} & \\
\isa{{\isasymLongrightarrow}} & \isa{{\isasymI}\isactrlsub S{\isacharparenleft}G{\isacharparenright}\ {\isasymand}\ {\isasymI}\isactrlsub S{\isacharparenleft}H{\isacharparenright}} \isa{{\isacharequal}\ False} & \\
\isa{{\isasymLongrightarrow}} & \isa{{\isasymI}\isactrlsub S{\isacharparenleft}G\ {\isasymand}\ H{\isacharparenright}} \isa{{\isacharequal}\ False} & \\
\isa{{\isasymLongrightarrow}} & \isa{{\isasymI}\isactrlsub S{\isacharparenleft}F{\isacharparenright}}  \isa{{\isacharequal}\ False}& \\ 
\isa{{\isasymLongrightarrow}} & \isa{{\isasymI}\isactrlsub S} \not\models \isa{F} & 
      \end{array}$$ 

  \item[Caso 5:] \isa{F\ {\isacharequal}\ G\ {\isasymor}\ H} y tanto \isa{G} como \isa{H} verifican la hipótesis 
  de inducción. 

  Es decir, se verifican 
   \begin{enumerate}
      \item [HI1:]  \isa{{\isacharparenleft}G\ {\isasymin}\ S\ {\isasymLongrightarrow}\ {\isasymI}\isactrlsub S\ {\isasymTurnstile}\ G{\isacharparenright}\ {\isasymand}\ {\isacharparenleft}{\isasymnot}\ G\ {\isasymin}\ S\ {\isasymLongrightarrow}\ \ {\isasymI}\isactrlsub S}  
          $\not\models$ \isa{G{\isacharparenright}}
      \item [HI2:]  \isa{{\isacharparenleft}H\ {\isasymin}\ S\ {\isasymLongrightarrow}\ {\isasymI}\isactrlsub S\ {\isasymTurnstile}\ H{\isacharparenright}\ {\isasymand}\ {\isacharparenleft}{\isasymnot}\ H\ {\isasymin}\ S\ {\isasymLongrightarrow}\ \ {\isasymI}\isactrlsub S}  
          $\not\models$ \isa{H{\isacharparenright}}
   \end{enumerate}
 
  Probemos que  \isa{{\isacharparenleft}F\ {\isasymin}\ S\ {\isasymLongrightarrow}\ {\isasymI}\isactrlsub S\ {\isasymTurnstile}\ F{\isacharparenright}\ {\isasymand}\ {\isacharparenleft}{\isasymnot}\ F\ {\isasymin}\ S\ {\isasymLongrightarrow}\ {\isasymI}\isactrlsub S}  
  $\not\models$ \isa{F{\isacharparenright}} 

  En efecto,
$$\begin{array}{lrl}
 & \isa{F\ {\isasymin}\ S} & \\
\isa{{\isasymLongrightarrow}} & \isa{G\ {\isasymor}\ H\ {\isasymin}\ S} & \\
\isa{{\isasymLongrightarrow}} & \isa{G\ {\isasymin}\ S\ {\isasymor}\ H\ {\isasymin}\ S} & \isa{{\isacharparenleft}S\ es\ conjunto\ de\ Hintikka{\isacharparenright}} \\
\isa{{\isasymLongrightarrow}} & \isa{{\isasymI}\isactrlsub S\ {\isasymTurnstile}\ G\ {\isasymor}\ {\isasymI}\isactrlsub S\ {\isasymTurnstile}\ H} & \isa{{\isacharparenleft}HI{\isadigit{1}}\ y\ HI{\isadigit{2}}{\isacharparenright}} \\
\isa{{\isasymLongrightarrow}} & \isa{{\isasymI}\isactrlsub S{\isacharparenleft}G{\isacharparenright}\ {\isasymor}\ {\isasymI}\isactrlsub S{\isacharparenleft}H{\isacharparenright}} \isa{{\isacharequal}\ True}& \\
\isa{{\isasymLongrightarrow}} & \isa{{\isasymI}\isactrlsub S{\isacharparenleft}G\ {\isasymor}\ H{\isacharparenright}} \isa{{\isacharequal}\ True}& \\
\isa{{\isasymLongrightarrow}} & \isa{{\isasymI}\isactrlsub S{\isacharparenleft}F{\isacharparenright}} \isa{{\isacharequal}\ True}& \\
\isa{{\isasymLongrightarrow}} & \isa{{\isasymI}\isactrlsub S\ {\isasymTurnstile}\ F} & \\
      \end{array}$$ 

Por otra parte, 
$$\begin{array}{lrl}
 & \isa{{\isasymnot}\ F\ {\isasymin}\ S} & \\
\isa{{\isasymLongrightarrow}} & \isa{{\isasymnot}\ {\isacharparenleft}G\ {\isasymor}\ H{\isacharparenright}\ {\isasymin}\ S} & \\
\isa{{\isasymLongrightarrow}} & \isa{{\isasymnot}\ G\ {\isasymin}\ S\ {\isasymand}\ {\isasymnot}\ H\ {\isasymin}\ S} & \isa{{\isacharparenleft}S\ es\ conjunto\ de\ Hintikka{\isacharparenright}}\\
\isa{{\isasymLongrightarrow}} & \isa{{\isasymI}\isactrlsub S} \not\models \isa{G\ {\isasymand}\ {\isasymI}\isactrlsub S} \not\models \isa{H} & \isa{{\isacharparenleft}HI{\isadigit{1}}\ y\ HI{\isadigit{2}}{\isacharparenright}}\\
\isa{{\isasymLongrightarrow}} & \isa{{\isasymnot}\ {\isacharparenleft}{\isasymI}\isactrlsub S\ {\isasymTurnstile}\ G\ {\isasymor}\ {\isasymI}\isactrlsub S\ {\isasymTurnstile}\ H{\isacharparenright}} & \\
\isa{{\isasymLongrightarrow}} & \isa{{\isasymI}\isactrlsub S{\isacharparenleft}G{\isacharparenright}\ {\isasymor}\ {\isasymI}\isactrlsub S{\isacharparenleft}H{\isacharparenright}} \isa{{\isacharequal}\ False} & \\
\isa{{\isasymLongrightarrow}} & \isa{{\isasymI}\isactrlsub S{\isacharparenleft}G\ {\isasymor}\ H{\isacharparenright}} \isa{{\isacharequal}\ False} & \\
\isa{{\isasymLongrightarrow}} & \isa{{\isasymI}\isactrlsub S{\isacharparenleft}F{\isacharparenright}}  \isa{{\isacharequal}\ False}& \\ 
\isa{{\isasymLongrightarrow}} & \isa{{\isasymI}\isactrlsub S} \not\models \isa{F} & 
      \end{array}$$ 

  \item[Caso 6:] \isa{F\ {\isacharequal}\ G\ {\isasymlongrightarrow}\ H} y tanto \isa{G} como \isa{H} verifican la 
  hipótesis de inducción. 

  Es decir, se verifican 
   \begin{enumerate}
     \item [HI1:]  \isa{{\isacharparenleft}G\ {\isasymin}\ S\ {\isasymLongrightarrow}\ {\isasymI}\isactrlsub S\ {\isasymTurnstile}\ G{\isacharparenright}\ {\isasymand}\ {\isacharparenleft}{\isasymnot}\ G\ {\isasymin}\ S\ {\isasymLongrightarrow}\ \ {\isasymI}\isactrlsub S}  
        $\not\models$ \isa{G{\isacharparenright}}
     \item [HI2:]  \isa{{\isacharparenleft}H\ {\isasymin}\ S\ {\isasymLongrightarrow}\ {\isasymI}\isactrlsub S\ {\isasymTurnstile}\ H{\isacharparenright}\ {\isasymand}\ {\isacharparenleft}{\isasymnot}\ H\ {\isasymin}\ S\ {\isasymLongrightarrow}\ \ {\isasymI}\isactrlsub S}  
        $\not\models$ \isa{H{\isacharparenright}}
   \end{enumerate}
 
   Probemos que  \isa{{\isacharparenleft}F\ {\isasymin}\ S\ {\isasymLongrightarrow}\ {\isasymI}\isactrlsub S\ {\isasymTurnstile}\ F{\isacharparenright}\ {\isasymand}\ {\isacharparenleft}{\isasymnot}\ F\ {\isasymin}\ S\ {\isasymLongrightarrow}\ \ {\isasymI}\isactrlsub S}  
   $\not\models$ \isa{F{\isacharparenright}}

   En efecto,
$$\begin{array}{lrl}
 & \isa{F\ {\isasymin}\ S} & \\
\isa{{\isasymLongrightarrow}} & \isa{G\ {\isasymlongrightarrow}\ H\ {\isasymin}\ S} & \\
\isa{{\isasymLongrightarrow}} & \isa{{\isasymnot}\ G\ {\isasymin}\ S\ {\isasymor}\ H\ {\isasymin}\ S} & \isa{{\isacharparenleft}S\ es\ conjunto\ de\ Hintikka{\isacharparenright}} \\
  \end{array}$$ 

  Veamos que \isa{{\isasymI}\isactrlsub S\ {\isasymTurnstile}\ F} considerando ambos casos de la disyunción
  anterior.
$$\begin{array}{lrl}
 & \isa{{\isasymnot}\ G\ {\isasymin}\ S} & \\
\isa{{\isasymLongrightarrow}} & \isa{{\isasymI}\isactrlsub S} \not\models \isa{G} & \isa{{\isacharparenleft}HI{\isadigit{1}}{\isacharparenright}} \\
\isa{{\isasymLongrightarrow}} & \isa{{\isasymI}\isactrlsub S\ {\isasymTurnstile}\ G\ {\isasymlongrightarrow}\ {\isasymI}\isactrlsub S\ {\isasymTurnstile}\ H} & \\
\isa{{\isasymLongrightarrow}} & \isa{{\isasymI}\isactrlsub S{\isacharparenleft}G{\isacharparenright}\ {\isasymlongrightarrow}\ {\isasymI}\isactrlsub S{\isacharparenleft}H{\isacharparenright}} \isa{{\isacharequal}\ True}& \\
\isa{{\isasymLongrightarrow}} & \isa{{\isasymI}\isactrlsub S{\isacharparenleft}G\ {\isasymlongrightarrow}\ H{\isacharparenright}} \isa{{\isacharequal}\ True}& \\
\isa{{\isasymLongrightarrow}} & \isa{{\isasymI}\isactrlsub S{\isacharparenleft}F{\isacharparenright}} \isa{{\isacharequal}\ True}& \\
\isa{{\isasymLongrightarrow}} & \isa{{\isasymI}\isactrlsub S\ {\isasymTurnstile}\ F} & \\
  \end{array}$$ 

$$\begin{array}{lrl}
 & \isa{H\ {\isasymin}\ S} & \\
\isa{{\isasymLongrightarrow}} & \isa{{\isasymI}\isactrlsub S\ {\isasymTurnstile}\ H} & \isa{{\isacharparenleft}HI{\isadigit{2}}{\isacharparenright}} \\
\isa{{\isasymLongrightarrow}} & \isa{{\isasymI}\isactrlsub S\ {\isasymTurnstile}\ G\ {\isasymlongrightarrow}\ {\isasymI}\isactrlsub S\ {\isasymTurnstile}\ H} & \\
\isa{{\isasymLongrightarrow}} & \isa{{\isasymI}\isactrlsub S{\isacharparenleft}G{\isacharparenright}\ {\isasymlongrightarrow}\ {\isasymI}\isactrlsub S{\isacharparenleft}H{\isacharparenright}} \isa{{\isacharequal}\ True}& \\
\isa{{\isasymLongrightarrow}} & \isa{{\isasymI}\isactrlsub S{\isacharparenleft}G\ {\isasymlongrightarrow}\ H{\isacharparenright}} \isa{{\isacharequal}\ True}& \\
\isa{{\isasymLongrightarrow}} & \isa{{\isasymI}\isactrlsub S{\isacharparenleft}F{\isacharparenright}} \isa{{\isacharequal}\ True}& \\
\isa{{\isasymLongrightarrow}} & \isa{{\isasymI}\isactrlsub S\ {\isasymTurnstile}\ F} & \\
  \end{array}$$ 

  Probando el resultado para este caso.

Por otra parte, 
$$\begin{array}{lrl}
 & \isa{{\isasymnot}\ F\ {\isasymin}\ S} & \\
\isa{{\isasymLongrightarrow}} & \isa{{\isasymnot}\ {\isacharparenleft}G\ {\isasymlongrightarrow}\ H{\isacharparenright}\ {\isasymin}\ S} & \\
\isa{{\isasymLongrightarrow}} & \isa{G\ {\isasymin}\ S\ {\isasymand}\ {\isasymnot}\ H\ {\isasymin}\ S} & \isa{{\isacharparenleft}S\ es\ conjunto\ de\ Hintikka{\isacharparenright}}\\
\isa{{\isasymLongrightarrow}} & \isa{{\isasymI}\isactrlsub S\ {\isasymTurnstile}\ G\ {\isasymand}\ {\isasymI}\isactrlsub S} \not\models \isa{H} & \isa{{\isacharparenleft}HI{\isadigit{1}}\ y\ HI{\isadigit{2}}{\isacharparenright}}\\
\isa{{\isasymLongrightarrow}} & \isa{{\isasymnot}\ {\isacharparenleft}{\isasymI}\isactrlsub S\ {\isasymTurnstile}\ G\ {\isasymlongrightarrow}\ {\isasymI}\isactrlsub S\ {\isasymTurnstile}\ H{\isacharparenright}} & \\
\isa{{\isasymLongrightarrow}} & \isa{{\isasymI}\isactrlsub S{\isacharparenleft}G{\isacharparenright}\ {\isasymlongrightarrow}\ {\isasymI}\isactrlsub S{\isacharparenleft}H{\isacharparenright}} \isa{{\isacharequal}\ False} & \\
\isa{{\isasymLongrightarrow}} & \isa{{\isasymI}\isactrlsub S{\isacharparenleft}G\ {\isasymlongrightarrow}\ H{\isacharparenright}} \isa{{\isacharequal}\ False} & \\
\isa{{\isasymLongrightarrow}} & \isa{{\isasymI}\isactrlsub S{\isacharparenleft}F{\isacharparenright}}  \isa{{\isacharequal}\ False}& \\ 
\isa{{\isasymLongrightarrow}} & \isa{{\isasymI}\isactrlsub S} \not\models \isa{F} & 
      \end{array}$$ 
 \end{enumerate}

  Queda probada la segunda afirmación.

    Con lo que termina la demostración.
 \end{demostracion} 

  Una vez terminada la prueba anterior, procedemos a las distintas
  demostraciones del lema mediante Isabelle/HOL. En primer lugar
  aparecerán las demostraciones detalladas de cada caso de la estructura
  de las fórmulas por separado. Posteriormente se mostrará la prueba
  detallada del lema completo.%
\end{isamarkuptext}\isamarkuptrue%
\isacommand{lemma}\isamarkupfalse%
\isanewline
\ \ \isakeyword{assumes}\ \ {\isachardoublequoteopen}Hintikka\ S{\isachardoublequoteclose}\isanewline
\ \ \isakeyword{shows}\ {\isachardoublequoteopen}{\isasymAnd}x{\isachardot}\ {\isacharparenleft}Atom\ x\ {\isasymin}\ S\ {\isasymlongrightarrow}\ isModel\ {\isacharparenleft}setValuation\ S{\isacharparenright}\ {\isacharparenleft}Atom\ x{\isacharparenright}{\isacharparenright}\ {\isasymand}\isanewline
\ \ \ \ \ \ \ \ \ {\isacharparenleft}\isactrlbold {\isasymnot}\ {\isacharparenleft}Atom\ x{\isacharparenright}\ {\isasymin}\ S\ {\isasymlongrightarrow}\ {\isasymnot}\ isModel\ {\isacharparenleft}setValuation\ S{\isacharparenright}\ {\isacharparenleft}Atom\ x{\isacharparenright}{\isacharparenright}{\isachardoublequoteclose}\isanewline
%
\isadelimproof
%
\endisadelimproof
%
\isatagproof
\isacommand{proof}\isamarkupfalse%
\ {\isacharparenleft}rule\ conjI{\isacharparenright}\isanewline
\ \ \isacommand{show}\isamarkupfalse%
\ {\isachardoublequoteopen}{\isasymAnd}x{\isachardot}\ Atom\ x\ {\isasymin}\ S\ {\isasymlongrightarrow}\ isModel\ {\isacharparenleft}setValuation\ S{\isacharparenright}\ {\isacharparenleft}Atom\ x{\isacharparenright}{\isachardoublequoteclose}\ \isanewline
\ \ \isacommand{proof}\isamarkupfalse%
\isanewline
\ \ \ \ \isacommand{fix}\isamarkupfalse%
\ x\isanewline
\ \ \ \ \isacommand{assume}\isamarkupfalse%
\ {\isachardoublequoteopen}Atom\ x\ {\isasymin}\ S{\isachardoublequoteclose}\isanewline
\ \ \ \ \isacommand{hence}\isamarkupfalse%
\ {\isachardoublequoteopen}{\isacharparenleft}setValuation\ S{\isacharparenright}\ x{\isachardoublequoteclose}\isanewline
\ \ \ \ \ \ \isacommand{by}\isamarkupfalse%
\ {\isacharparenleft}simp\ only{\isacharcolon}\ setValuation{\isacharunderscore}def{\isacharparenright}\isanewline
\ \ \ \ \isacommand{hence}\isamarkupfalse%
\ {\isachardoublequoteopen}setValuation\ S\ {\isasymTurnstile}\ Atom\ x{\isachardoublequoteclose}\isanewline
\ \ \ \ \ \ \isacommand{by}\isamarkupfalse%
\ {\isacharparenleft}simp\ only{\isacharcolon}\ formula{\isacharunderscore}semantics{\isachardot}simps{\isacharparenleft}{\isadigit{1}}{\isacharparenright}{\isacharparenright}\isanewline
\ \ \ \ \isacommand{thus}\isamarkupfalse%
\ {\isachardoublequoteopen}isModel\ {\isacharparenleft}setValuation\ S{\isacharparenright}\ {\isacharparenleft}Atom\ x{\isacharparenright}{\isachardoublequoteclose}\isanewline
\ \ \ \ \ \ \isacommand{by}\isamarkupfalse%
\ {\isacharparenleft}simp\ only{\isacharcolon}\ isModel{\isacharunderscore}def{\isacharparenright}\isanewline
\ \ \isacommand{qed}\isamarkupfalse%
\isanewline
\isacommand{next}\isamarkupfalse%
\ \isanewline
\ \ \isacommand{show}\isamarkupfalse%
\ \isanewline
\ \ {\isachardoublequoteopen}{\isasymAnd}x{\isachardot}\ \isactrlbold {\isasymnot}\ {\isacharparenleft}Atom\ x{\isacharparenright}\ {\isasymin}\ S\ {\isasymlongrightarrow}\ {\isasymnot}\ isModel\ {\isacharparenleft}setValuation\ S{\isacharparenright}\ {\isacharparenleft}Atom\ x{\isacharparenright}{\isachardoublequoteclose}\ \isanewline
\ \ \isacommand{proof}\isamarkupfalse%
\isanewline
\ \ \ \ \isacommand{fix}\isamarkupfalse%
\ x\isanewline
\ \ \ \ \isacommand{assume}\isamarkupfalse%
\ {\isachardoublequoteopen}\isactrlbold {\isasymnot}\ {\isacharparenleft}Atom\ x{\isacharparenright}\ {\isasymin}\ S{\isachardoublequoteclose}\ \isanewline
\ \ \ \ \isacommand{have}\isamarkupfalse%
\ {\isachardoublequoteopen}\isactrlbold {\isasymnot}\ {\isacharparenleft}Atom\ x{\isacharparenright}\ {\isasymin}\ S\ {\isasymlongrightarrow}\ Atom\ x\ {\isasymnotin}\ S{\isachardoublequoteclose}\isanewline
\ \ \ \ \ \ \isacommand{using}\isamarkupfalse%
\ assms\ \isacommand{by}\isamarkupfalse%
\ {\isacharparenleft}rule\ Hintikka{\isacharunderscore}l{\isadigit{1}}{\isadigit{0}}{\isacharparenright}\isanewline
\ \ \ \ \isacommand{then}\isamarkupfalse%
\ \isacommand{have}\isamarkupfalse%
\ {\isachardoublequoteopen}Atom\ x\ {\isasymnotin}\ S{\isachardoublequoteclose}\isanewline
\ \ \ \ \ \ \isacommand{using}\isamarkupfalse%
\ {\isacartoucheopen}\isactrlbold {\isasymnot}\ {\isacharparenleft}Atom\ x{\isacharparenright}\ {\isasymin}\ S{\isacartoucheclose}\ \isacommand{by}\isamarkupfalse%
\ {\isacharparenleft}rule\ mp{\isacharparenright}\isanewline
\ \ \ \ \isacommand{also}\isamarkupfalse%
\ \isacommand{have}\isamarkupfalse%
\ {\isachardoublequoteopen}{\isacharparenleft}{\isasymnot}\ {\isacharparenleft}Atom\ x\ {\isasymin}\ S{\isacharparenright}{\isacharparenright}\ {\isacharequal}\ {\isacharparenleft}{\isasymnot}\ {\isacharparenleft}setValuation\ S{\isacharparenright}\ x{\isacharparenright}{\isachardoublequoteclose}\isanewline
\ \ \ \ \ \ \isacommand{by}\isamarkupfalse%
\ {\isacharparenleft}simp\ only{\isacharcolon}\ setValuation{\isacharunderscore}def{\isacharparenright}\isanewline
\ \ \ \ \isacommand{also}\isamarkupfalse%
\ \isacommand{have}\isamarkupfalse%
\ {\isachardoublequoteopen}{\isasymdots}\ {\isacharequal}\ {\isacharparenleft}{\isasymnot}\ {\isacharparenleft}{\isacharparenleft}setValuation\ S{\isacharparenright}\ {\isasymTurnstile}\ {\isacharparenleft}Atom\ x{\isacharparenright}{\isacharparenright}{\isacharparenright}{\isachardoublequoteclose}\isanewline
\ \ \ \ \ \ \isacommand{by}\isamarkupfalse%
\ {\isacharparenleft}simp\ only{\isacharcolon}\ formula{\isacharunderscore}semantics{\isachardot}simps{\isacharparenleft}{\isadigit{1}}{\isacharparenright}{\isacharparenright}\isanewline
\ \ \ \ \isacommand{also}\isamarkupfalse%
\ \isacommand{have}\isamarkupfalse%
\ {\isachardoublequoteopen}{\isasymdots}\ {\isacharequal}\ {\isacharparenleft}{\isasymnot}\ isModel\ {\isacharparenleft}setValuation\ S{\isacharparenright}\ {\isacharparenleft}Atom\ x{\isacharparenright}{\isacharparenright}{\isachardoublequoteclose}\ \isanewline
\ \ \ \ \ \ \isacommand{by}\isamarkupfalse%
\ {\isacharparenleft}simp\ only{\isacharcolon}\ isModel{\isacharunderscore}def{\isacharparenright}\isanewline
\ \ \ \ \isacommand{finally}\isamarkupfalse%
\ \isacommand{show}\isamarkupfalse%
\ {\isachardoublequoteopen}{\isasymnot}\ isModel\ {\isacharparenleft}setValuation\ S{\isacharparenright}\ {\isacharparenleft}Atom\ x{\isacharparenright}{\isachardoublequoteclose}\isanewline
\ \ \ \ \ \ \isacommand{by}\isamarkupfalse%
\ this\isanewline
\ \ \isacommand{qed}\isamarkupfalse%
\isanewline
\isacommand{qed}\isamarkupfalse%
%
\endisatagproof
{\isafoldproof}%
%
\isadelimproof
\isanewline
%
\endisadelimproof
\isanewline
\isacommand{lemma}\isamarkupfalse%
\ Hl{\isadigit{2}}{\isacharunderscore}{\isadigit{1}}{\isacharcolon}\isanewline
\ \ \isakeyword{assumes}\ \ {\isachardoublequoteopen}Hintikka\ S{\isachardoublequoteclose}\isanewline
\ \ \isakeyword{shows}\ {\isachardoublequoteopen}{\isasymAnd}x{\isachardot}\ {\isacharparenleft}Atom\ x\ {\isasymin}\ S\ {\isasymlongrightarrow}\ isModel\ {\isacharparenleft}setValuation\ S{\isacharparenright}\ {\isacharparenleft}Atom\ x{\isacharparenright}{\isacharparenright}\ {\isasymand}\isanewline
\ \ \ \ \ \ \ \ \ {\isacharparenleft}\isactrlbold {\isasymnot}\ {\isacharparenleft}Atom\ x{\isacharparenright}\ {\isasymin}\ S\ {\isasymlongrightarrow}\ {\isasymnot}\ isModel\ {\isacharparenleft}setValuation\ S{\isacharparenright}\ {\isacharparenleft}Atom\ x{\isacharparenright}{\isacharparenright}{\isachardoublequoteclose}\isanewline
%
\isadelimproof
\ \ %
\endisadelimproof
%
\isatagproof
\isacommand{by}\isamarkupfalse%
\ {\isacharparenleft}simp\ add{\isacharcolon}\ Hintikka{\isacharunderscore}l{\isadigit{1}}{\isadigit{0}}\ assms\ isModel{\isacharunderscore}def\ setValuation{\isacharunderscore}def{\isacharparenright}%
\endisatagproof
{\isafoldproof}%
%
\isadelimproof
\isanewline
%
\endisadelimproof
\isanewline
\isacommand{lemma}\isamarkupfalse%
\ \isanewline
\ \ \isakeyword{assumes}\ {\isachardoublequoteopen}Hintikka\ S{\isachardoublequoteclose}\isanewline
\ \ \isakeyword{shows}\ {\isachardoublequoteopen}{\isacharparenleft}{\isasymbottom}\ {\isasymin}\ S\ {\isasymlongrightarrow}\ isModel\ {\isacharparenleft}setValuation\ S{\isacharparenright}\ {\isasymbottom}{\isacharparenright}\isanewline
\ \ \ \ \ \ \ \ \ \ \ {\isasymand}\ {\isacharparenleft}\isactrlbold {\isasymnot}\ {\isasymbottom}\ {\isasymin}\ S\ {\isasymlongrightarrow}\ {\isacharparenleft}{\isasymnot}{\isacharparenleft}isModel\ {\isacharparenleft}setValuation\ S{\isacharparenright}\ {\isasymbottom}{\isacharparenright}{\isacharparenright}{\isacharparenright}{\isachardoublequoteclose}\isanewline
%
\isadelimproof
%
\endisadelimproof
%
\isatagproof
\isacommand{proof}\isamarkupfalse%
\ {\isacharparenleft}rule\ conjI{\isacharparenright}\isanewline
\ \ \isacommand{show}\isamarkupfalse%
\ {\isachardoublequoteopen}{\isasymbottom}\ {\isasymin}\ S\ {\isasymlongrightarrow}\ isModel\ {\isacharparenleft}setValuation\ S{\isacharparenright}\ {\isasymbottom}{\isachardoublequoteclose}\isanewline
\ \ \isacommand{proof}\isamarkupfalse%
\ {\isacharparenleft}rule\ impI{\isacharparenright}\isanewline
\ \ \ \ \isacommand{assume}\isamarkupfalse%
\ {\isachardoublequoteopen}{\isasymbottom}\ {\isasymin}\ S{\isachardoublequoteclose}\isanewline
\ \ \ \ \isacommand{have}\isamarkupfalse%
\ {\isachardoublequoteopen}{\isasymbottom}\ {\isasymnotin}\ S{\isachardoublequoteclose}\ \isanewline
\ \ \ \ \ \ \isacommand{using}\isamarkupfalse%
\ assms\ \isacommand{by}\isamarkupfalse%
\ {\isacharparenleft}rule\ Hintikka{\isacharunderscore}l{\isadigit{1}}{\isacharparenright}\isanewline
\ \ \ \ \isacommand{thus}\isamarkupfalse%
\ {\isachardoublequoteopen}isModel\ {\isacharparenleft}setValuation\ S{\isacharparenright}\ {\isasymbottom}{\isachardoublequoteclose}\isanewline
\ \ \ \ \ \ \isacommand{using}\isamarkupfalse%
\ {\isacartoucheopen}{\isasymbottom}\ {\isasymin}\ S{\isacartoucheclose}\ \isacommand{by}\isamarkupfalse%
\ {\isacharparenleft}rule\ notE{\isacharparenright}\isanewline
\ \ \isacommand{qed}\isamarkupfalse%
\isanewline
\isacommand{next}\isamarkupfalse%
\isanewline
\ \ \isacommand{show}\isamarkupfalse%
\ {\isachardoublequoteopen}\isactrlbold {\isasymnot}\ {\isasymbottom}\ {\isasymin}\ S\ {\isasymlongrightarrow}\ {\isasymnot}\ isModel\ {\isacharparenleft}setValuation\ S{\isacharparenright}\ {\isasymbottom}{\isachardoublequoteclose}\isanewline
\ \ \isacommand{proof}\isamarkupfalse%
\ {\isacharparenleft}rule\ impI{\isacharparenright}\isanewline
\ \ \ \ \isacommand{assume}\isamarkupfalse%
\ {\isachardoublequoteopen}\isactrlbold {\isasymnot}\ {\isasymbottom}\ {\isasymin}\ S{\isachardoublequoteclose}\isanewline
\ \ \ \ \isacommand{have}\isamarkupfalse%
\ {\isachardoublequoteopen}{\isasymnot}\ {\isacharparenleft}setValuation\ S{\isacharparenright}\ {\isasymTurnstile}\ {\isasymbottom}{\isachardoublequoteclose}\isanewline
\ \ \ \ \isacommand{proof}\isamarkupfalse%
\ {\isacharparenleft}rule\ notI{\isacharparenright}\isanewline
\ \ \ \ \ \isacommand{assume}\isamarkupfalse%
\ {\isachardoublequoteopen}setValuation\ S\ {\isasymTurnstile}\ {\isasymbottom}{\isachardoublequoteclose}\isanewline
\ \ \ \ \ \ \isacommand{thus}\isamarkupfalse%
\ {\isachardoublequoteopen}False{\isachardoublequoteclose}\isanewline
\ \ \ \ \ \ \ \ \isacommand{by}\isamarkupfalse%
\ {\isacharparenleft}simp\ only{\isacharcolon}\ formula{\isacharunderscore}semantics{\isachardot}simps{\isacharparenleft}{\isadigit{2}}{\isacharparenright}{\isacharparenright}\isanewline
\ \ \ \ \isacommand{qed}\isamarkupfalse%
\isanewline
\ \ \ \ \isacommand{also}\isamarkupfalse%
\ \isacommand{have}\isamarkupfalse%
\ {\isachardoublequoteopen}{\isacharparenleft}{\isasymnot}\ {\isacharparenleft}setValuation\ S{\isacharparenright}\ {\isasymTurnstile}\ {\isasymbottom}{\isacharparenright}\ {\isacharequal}\ {\isacharparenleft}{\isasymnot}\ isModel\ {\isacharparenleft}setValuation\ S{\isacharparenright}\ {\isasymbottom}{\isacharparenright}{\isachardoublequoteclose}\isanewline
\ \ \ \ \ \ \isacommand{by}\isamarkupfalse%
\ {\isacharparenleft}simp\ only{\isacharcolon}\ isModel{\isacharunderscore}def{\isacharparenright}\isanewline
\ \ \ \ \isacommand{finally}\isamarkupfalse%
\ \isacommand{show}\isamarkupfalse%
\ {\isachardoublequoteopen}{\isasymnot}\ isModel\ {\isacharparenleft}setValuation\ S{\isacharparenright}\ {\isasymbottom}{\isachardoublequoteclose}\isanewline
\ \ \ \ \ \ \isacommand{by}\isamarkupfalse%
\ this\isanewline
\ \ \isacommand{qed}\isamarkupfalse%
\isanewline
\isacommand{qed}\isamarkupfalse%
%
\endisatagproof
{\isafoldproof}%
%
\isadelimproof
\isanewline
%
\endisadelimproof
\isanewline
\isacommand{lemma}\isamarkupfalse%
\ Hl{\isadigit{2}}{\isacharunderscore}{\isadigit{2}}{\isacharcolon}\isanewline
\ \ \isakeyword{assumes}\ {\isachardoublequoteopen}Hintikka\ S{\isachardoublequoteclose}\isanewline
\ \ \isakeyword{shows}\ {\isachardoublequoteopen}{\isacharparenleft}{\isasymbottom}\ {\isasymin}\ S\ {\isasymlongrightarrow}\ isModel\ {\isacharparenleft}setValuation\ S{\isacharparenright}\ {\isasymbottom}{\isacharparenright}\isanewline
\ \ \ \ \ \ \ \ \ \ \ {\isasymand}\ {\isacharparenleft}\isactrlbold {\isasymnot}\ {\isasymbottom}\ {\isasymin}\ S\ {\isasymlongrightarrow}\ {\isacharparenleft}{\isasymnot}\ {\isacharparenleft}isModel\ {\isacharparenleft}setValuation\ S{\isacharparenright}\ {\isasymbottom}{\isacharparenright}{\isacharparenright}{\isacharparenright}{\isachardoublequoteclose}\isanewline
%
\isadelimproof
\ \ %
\endisadelimproof
%
\isatagproof
\isacommand{by}\isamarkupfalse%
\ {\isacharparenleft}simp\ add{\isacharcolon}\ Hintikka{\isacharunderscore}l{\isadigit{1}}\ assms\ isModel{\isacharunderscore}def{\isacharparenright}%
\endisatagproof
{\isafoldproof}%
%
\isadelimproof
\isanewline
%
\endisadelimproof
\isanewline
\isacommand{lemma}\isamarkupfalse%
\isanewline
\ \ \isakeyword{assumes}\ {\isachardoublequoteopen}Hintikka\ S{\isachardoublequoteclose}\isanewline
\ \ \ \ \ \ \ \ \ \ {\isachardoublequoteopen}{\isasymAnd}F{\isachardot}\ {\isacharparenleft}F\ {\isasymin}\ S\ {\isasymlongrightarrow}\ isModel\ {\isacharparenleft}setValuation\ S{\isacharparenright}\ F{\isacharparenright}\ {\isasymand}\isanewline
\ \ \ \ \ \ \ \ \ {\isacharparenleft}\isactrlbold {\isasymnot}\ F\ {\isasymin}\ S\ {\isasymlongrightarrow}\ {\isasymnot}\ isModel\ {\isacharparenleft}setValuation\ S{\isacharparenright}\ F{\isacharparenright}{\isachardoublequoteclose}\isanewline
\ \ \isakeyword{shows}\ {\isachardoublequoteopen}{\isasymAnd}F{\isachardot}\ {\isacharparenleft}\isactrlbold {\isasymnot}\ F\ {\isasymin}\ S\ {\isasymlongrightarrow}\ isModel\ {\isacharparenleft}setValuation\ S{\isacharparenright}\ {\isacharparenleft}\isactrlbold {\isasymnot}\ F{\isacharparenright}{\isacharparenright}\ {\isasymand}\isanewline
\ \ \ \ \ \ \ \ \ {\isacharparenleft}\isactrlbold {\isasymnot}\ {\isacharparenleft}\isactrlbold {\isasymnot}\ F{\isacharparenright}\ {\isasymin}\ S\ {\isasymlongrightarrow}\ {\isasymnot}\ isModel\ {\isacharparenleft}setValuation\ S{\isacharparenright}\ {\isacharparenleft}\isactrlbold {\isasymnot}\ F{\isacharparenright}{\isacharparenright}{\isachardoublequoteclose}\isanewline
%
\isadelimproof
%
\endisadelimproof
%
\isatagproof
\isacommand{proof}\isamarkupfalse%
\ {\isacharparenleft}rule\ conjI{\isacharparenright}\ \isanewline
\ \ \isacommand{show}\isamarkupfalse%
\ {\isachardoublequoteopen}{\isasymAnd}F{\isachardot}\ \isactrlbold {\isasymnot}\ F\ {\isasymin}\ S\ {\isasymlongrightarrow}\ isModel\ {\isacharparenleft}setValuation\ S{\isacharparenright}\ {\isacharparenleft}\isactrlbold {\isasymnot}\ F{\isacharparenright}{\isachardoublequoteclose}\isanewline
\ \ \isacommand{proof}\isamarkupfalse%
\ \isanewline
\ \ \ \ \isacommand{fix}\isamarkupfalse%
\ F\isanewline
\ \ \ \ \isacommand{assume}\isamarkupfalse%
\ {\isachardoublequoteopen}\isactrlbold {\isasymnot}\ F\ {\isasymin}\ S{\isachardoublequoteclose}\isanewline
\ \ \ \ \isacommand{have}\isamarkupfalse%
\ {\isachardoublequoteopen}\isactrlbold {\isasymnot}\ F\ {\isasymin}\ S\ {\isasymlongrightarrow}\ {\isasymnot}\ isModel\ {\isacharparenleft}setValuation\ S{\isacharparenright}\ F{\isachardoublequoteclose}\isanewline
\ \ \ \ \ \ \isacommand{using}\isamarkupfalse%
\ assms{\isacharparenleft}{\isadigit{2}}{\isacharparenright}\ \isacommand{by}\isamarkupfalse%
\ {\isacharparenleft}rule\ conjunct{\isadigit{2}}{\isacharparenright}\isanewline
\ \ \ \ \isacommand{then}\isamarkupfalse%
\ \isacommand{have}\isamarkupfalse%
\ {\isachardoublequoteopen}{\isasymnot}\ isModel\ {\isacharparenleft}setValuation\ S{\isacharparenright}\ F{\isachardoublequoteclose}\isanewline
\ \ \ \ \ \ \isacommand{using}\isamarkupfalse%
\ \ {\isacartoucheopen}\isactrlbold {\isasymnot}\ F\ {\isasymin}\ S{\isacartoucheclose}\ \isacommand{by}\isamarkupfalse%
\ {\isacharparenleft}rule\ mp{\isacharparenright}\isanewline
\ \ \ \ \isacommand{also}\isamarkupfalse%
\ \isacommand{have}\isamarkupfalse%
\ {\isachardoublequoteopen}{\isacharparenleft}{\isasymnot}\ isModel\ {\isacharparenleft}setValuation\ S{\isacharparenright}\ F{\isacharparenright}\ {\isacharequal}\ {\isacharparenleft}{\isasymnot}\ {\isacharparenleft}setValuation\ S{\isacharparenright}\ {\isasymTurnstile}\ F{\isacharparenright}{\isachardoublequoteclose}\isanewline
\ \ \ \ \ \ \isacommand{by}\isamarkupfalse%
\ {\isacharparenleft}simp\ only{\isacharcolon}\ isModel{\isacharunderscore}def{\isacharparenright}\isanewline
\ \ \ \ \isacommand{also}\isamarkupfalse%
\ \isacommand{have}\isamarkupfalse%
\ {\isachardoublequoteopen}{\isasymdots}\ {\isacharequal}\ setValuation\ S\ {\isasymTurnstile}\ {\isacharparenleft}\isactrlbold {\isasymnot}\ F{\isacharparenright}{\isachardoublequoteclose}\isanewline
\ \ \ \ \ \ \isacommand{by}\isamarkupfalse%
\ {\isacharparenleft}simp\ only{\isacharcolon}\ formula{\isacharunderscore}semantics{\isachardot}simps{\isacharparenleft}{\isadigit{3}}{\isacharparenright}{\isacharparenright}\isanewline
\ \ \ \ \isacommand{also}\isamarkupfalse%
\ \isacommand{have}\isamarkupfalse%
\ {\isachardoublequoteopen}{\isasymdots}\ {\isacharequal}\ isModel\ {\isacharparenleft}setValuation\ S{\isacharparenright}\ {\isacharparenleft}\isactrlbold {\isasymnot}\ F{\isacharparenright}{\isachardoublequoteclose}\isanewline
\ \ \ \ \ \ \isacommand{by}\isamarkupfalse%
\ {\isacharparenleft}simp\ only{\isacharcolon}\ isModel{\isacharunderscore}def{\isacharparenright}\isanewline
\ \ \ \ \isacommand{finally}\isamarkupfalse%
\ \isacommand{show}\isamarkupfalse%
\ {\isachardoublequoteopen}isModel\ {\isacharparenleft}setValuation\ S{\isacharparenright}\ {\isacharparenleft}\isactrlbold {\isasymnot}\ F{\isacharparenright}{\isachardoublequoteclose}\isanewline
\ \ \ \ \ \ \isacommand{by}\isamarkupfalse%
\ this\isanewline
\ \ \isacommand{qed}\isamarkupfalse%
\isanewline
\isacommand{next}\isamarkupfalse%
\isanewline
\ \ \isacommand{show}\isamarkupfalse%
\ {\isachardoublequoteopen}{\isasymAnd}F{\isachardot}\ \isactrlbold {\isasymnot}\ {\isacharparenleft}\isactrlbold {\isasymnot}\ F{\isacharparenright}\ {\isasymin}\ S\ {\isasymlongrightarrow}\ {\isasymnot}\ isModel\ {\isacharparenleft}setValuation\ S{\isacharparenright}\ {\isacharparenleft}\isactrlbold {\isasymnot}\ F{\isacharparenright}{\isachardoublequoteclose}\isanewline
\ \ \isacommand{proof}\isamarkupfalse%
\isanewline
\ \ \ \ \isacommand{fix}\isamarkupfalse%
\ F\isanewline
\ \ \ \ \isacommand{assume}\isamarkupfalse%
\ {\isachardoublequoteopen}\isactrlbold {\isasymnot}\ {\isacharparenleft}\isactrlbold {\isasymnot}\ F{\isacharparenright}\ {\isasymin}\ S{\isachardoublequoteclose}\isanewline
\ \ \ \ \isacommand{have}\isamarkupfalse%
\ {\isachardoublequoteopen}\isactrlbold {\isasymnot}\ {\isacharparenleft}\isactrlbold {\isasymnot}\ F{\isacharparenright}\ {\isasymin}\ S\ {\isasymlongrightarrow}\ F\ {\isasymin}\ S{\isachardoublequoteclose}\ \isanewline
\ \ \ \ \ \ \isacommand{using}\isamarkupfalse%
\ assms{\isacharparenleft}{\isadigit{1}}{\isacharparenright}\ \isacommand{by}\isamarkupfalse%
\ {\isacharparenleft}rule\ Hintikka{\isacharunderscore}l{\isadigit{6}}{\isacharparenright}\isanewline
\ \ \ \ \isacommand{then}\isamarkupfalse%
\ \isacommand{have}\isamarkupfalse%
\ {\isachardoublequoteopen}F\ {\isasymin}\ S{\isachardoublequoteclose}\isanewline
\ \ \ \ \ \ \isacommand{using}\isamarkupfalse%
\ {\isacartoucheopen}\isactrlbold {\isasymnot}\ {\isacharparenleft}\isactrlbold {\isasymnot}\ F{\isacharparenright}\ {\isasymin}\ S{\isacartoucheclose}\ \isacommand{by}\isamarkupfalse%
\ {\isacharparenleft}rule\ mp{\isacharparenright}\isanewline
\ \ \ \ \isacommand{have}\isamarkupfalse%
\ {\isachardoublequoteopen}F\ {\isasymin}\ S\ {\isasymlongrightarrow}\ isModel\ {\isacharparenleft}setValuation\ S{\isacharparenright}\ F{\isachardoublequoteclose}\ \isanewline
\ \ \ \ \ \ \isacommand{using}\isamarkupfalse%
\ assms{\isacharparenleft}{\isadigit{2}}{\isacharparenright}\ \isacommand{by}\isamarkupfalse%
\ {\isacharparenleft}rule\ conjunct{\isadigit{1}}{\isacharparenright}\isanewline
\ \ \ \ \isacommand{then}\isamarkupfalse%
\ \isacommand{have}\isamarkupfalse%
\ {\isachardoublequoteopen}isModel\ {\isacharparenleft}setValuation\ S{\isacharparenright}\ F{\isachardoublequoteclose}\isanewline
\ \ \ \ \ \ \isacommand{using}\isamarkupfalse%
\ {\isacartoucheopen}F\ {\isasymin}\ S{\isacartoucheclose}\ \isacommand{by}\isamarkupfalse%
\ {\isacharparenleft}rule\ mp{\isacharparenright}\isanewline
\ \ \ \ \isacommand{then}\isamarkupfalse%
\ \isacommand{have}\isamarkupfalse%
\ {\isachardoublequoteopen}{\isacharparenleft}{\isasymnot}\ {\isacharparenleft}{\isasymnot}\ isModel\ {\isacharparenleft}setValuation\ S{\isacharparenright}\ F{\isacharparenright}{\isacharparenright}{\isachardoublequoteclose}\isanewline
\ \ \ \ \ \ \isacommand{by}\isamarkupfalse%
\ {\isacharparenleft}rule\ contrapos{\isacharunderscore}pn{\isacharparenright}\isanewline
\ \ \ \ \isacommand{also}\isamarkupfalse%
\ \isacommand{have}\isamarkupfalse%
\ {\isachardoublequoteopen}{\isacharparenleft}{\isasymnot}\ {\isacharparenleft}{\isasymnot}\ isModel\ {\isacharparenleft}setValuation\ S{\isacharparenright}\ F{\isacharparenright}{\isacharparenright}\ {\isacharequal}\ \isanewline
\ \ \ \ \ \ \ \ {\isacharparenleft}{\isasymnot}\ {\isacharparenleft}{\isasymnot}\ {\isacharparenleft}setValuation\ S{\isacharparenright}\ {\isasymTurnstile}\ F{\isacharparenright}{\isacharparenright}{\isachardoublequoteclose}\isanewline
\ \ \ \ \ \ \isacommand{by}\isamarkupfalse%
\ {\isacharparenleft}simp\ only{\isacharcolon}\ isModel{\isacharunderscore}def{\isacharparenright}\isanewline
\ \ \ \ \isacommand{also}\isamarkupfalse%
\ \isacommand{have}\isamarkupfalse%
\ {\isachardoublequoteopen}{\isasymdots}\ {\isacharequal}\ {\isacharparenleft}{\isasymnot}\ {\isacharparenleft}setValuation\ S{\isacharparenright}\ {\isasymTurnstile}\ {\isacharparenleft}\isactrlbold {\isasymnot}\ F{\isacharparenright}{\isacharparenright}{\isachardoublequoteclose}\isanewline
\ \ \ \ \ \ \isacommand{by}\isamarkupfalse%
\ {\isacharparenleft}simp\ only{\isacharcolon}\ formula{\isacharunderscore}semantics{\isachardot}simps{\isacharparenleft}{\isadigit{3}}{\isacharparenright}{\isacharparenright}\isanewline
\ \ \ \ \isacommand{also}\isamarkupfalse%
\ \isacommand{have}\isamarkupfalse%
\ {\isachardoublequoteopen}{\isasymdots}\ {\isacharequal}\ {\isacharparenleft}{\isasymnot}\ isModel\ {\isacharparenleft}setValuation\ S{\isacharparenright}\ {\isacharparenleft}\isactrlbold {\isasymnot}\ F{\isacharparenright}{\isacharparenright}{\isachardoublequoteclose}\isanewline
\ \ \ \ \ \ \isacommand{by}\isamarkupfalse%
\ {\isacharparenleft}simp\ only{\isacharcolon}\ isModel{\isacharunderscore}def{\isacharparenright}\isanewline
\ \ \ \ \isacommand{finally}\isamarkupfalse%
\ \isacommand{show}\isamarkupfalse%
\ {\isachardoublequoteopen}{\isasymnot}\ isModel\ {\isacharparenleft}setValuation\ S{\isacharparenright}\ {\isacharparenleft}\isactrlbold {\isasymnot}\ F{\isacharparenright}{\isachardoublequoteclose}\isanewline
\ \ \ \ \ \ \isacommand{by}\isamarkupfalse%
\ this\isanewline
\ \ \isacommand{qed}\isamarkupfalse%
\isanewline
\isacommand{qed}\isamarkupfalse%
%
\endisatagproof
{\isafoldproof}%
%
\isadelimproof
\isanewline
%
\endisadelimproof
\isanewline
\isacommand{lemma}\isamarkupfalse%
\ Hl{\isadigit{2}}{\isacharunderscore}{\isadigit{3}}{\isacharcolon}\isanewline
\ \ \isakeyword{assumes}\ {\isachardoublequoteopen}Hintikka\ S{\isachardoublequoteclose}\isanewline
\ \ \isakeyword{shows}\ {\isachardoublequoteopen}\ {\isasymAnd}F{\isachardot}\ {\isacharparenleft}F\ {\isasymin}\ S\ {\isasymlongrightarrow}\ isModel\ {\isacharparenleft}setValuation\ S{\isacharparenright}\ F{\isacharparenright}\ {\isasymand}\isanewline
\ \ \ \ \ \ \ \ \ {\isacharparenleft}\isactrlbold {\isasymnot}\ F\ {\isasymin}\ S\ {\isasymlongrightarrow}\ {\isasymnot}\ isModel\ {\isacharparenleft}setValuation\ S{\isacharparenright}\ F{\isacharparenright}\ {\isasymLongrightarrow}\isanewline
\ \ \ \ \ \ \ \ \ {\isacharparenleft}\isactrlbold {\isasymnot}\ F\ {\isasymin}\ S\ {\isasymlongrightarrow}\ isModel\ {\isacharparenleft}setValuation\ S{\isacharparenright}\ {\isacharparenleft}\isactrlbold {\isasymnot}\ F{\isacharparenright}{\isacharparenright}\ {\isasymand}\isanewline
\ \ \ \ \ \ \ \ \ {\isacharparenleft}\isactrlbold {\isasymnot}\ {\isacharparenleft}\isactrlbold {\isasymnot}\ F{\isacharparenright}\ {\isasymin}\ S\ {\isasymlongrightarrow}\ {\isasymnot}\ isModel\ {\isacharparenleft}setValuation\ S{\isacharparenright}\ {\isacharparenleft}\isactrlbold {\isasymnot}\ F{\isacharparenright}{\isacharparenright}{\isachardoublequoteclose}\isanewline
%
\isadelimproof
\ \ %
\endisadelimproof
%
\isatagproof
\isacommand{using}\isamarkupfalse%
\ Hintikka{\isacharunderscore}l{\isadigit{6}}\ assms\ isModel{\isacharunderscore}def\ formula{\isacharunderscore}semantics{\isachardot}simps{\isacharparenleft}{\isadigit{3}}{\isacharparenright}\ \isacommand{by}\isamarkupfalse%
\ blast%
\endisatagproof
{\isafoldproof}%
%
\isadelimproof
\isanewline
%
\endisadelimproof
\isanewline
\isacommand{lemma}\isamarkupfalse%
\isanewline
\ \ \isakeyword{assumes}\ {\isachardoublequoteopen}Hintikka\ S{\isachardoublequoteclose}\isanewline
\ \ \ \ \ \ \ \ \ \ {\isachardoublequoteopen}{\isasymAnd}F{\isadigit{1}}{\isachardot}\ {\isacharparenleft}F{\isadigit{1}}\ {\isasymin}\ S\ {\isasymlongrightarrow}\ isModel\ {\isacharparenleft}setValuation\ S{\isacharparenright}\ F{\isadigit{1}}{\isacharparenright}\ {\isasymand}\isanewline
\ \ \ \ \ \ \ \ \ \ \ {\isacharparenleft}\isactrlbold {\isasymnot}\ F{\isadigit{1}}\ {\isasymin}\ S\ {\isasymlongrightarrow}\ {\isasymnot}\ isModel\ {\isacharparenleft}setValuation\ S{\isacharparenright}\ F{\isadigit{1}}{\isacharparenright}{\isachardoublequoteclose}\ \isanewline
\ \ \ \ \ \ \ \ \ \ {\isachardoublequoteopen}{\isasymAnd}F{\isadigit{2}}{\isachardot}\ {\isacharparenleft}F{\isadigit{2}}\ {\isasymin}\ S\ {\isasymlongrightarrow}\ isModel\ {\isacharparenleft}setValuation\ S{\isacharparenright}\ F{\isadigit{2}}{\isacharparenright}\ {\isasymand}\isanewline
\ \ \ \ \ \ \ \ \ \ \ {\isacharparenleft}\isactrlbold {\isasymnot}\ F{\isadigit{2}}\ {\isasymin}\ S\ {\isasymlongrightarrow}\ {\isasymnot}\ isModel\ {\isacharparenleft}setValuation\ S{\isacharparenright}\ F{\isadigit{2}}{\isacharparenright}{\isachardoublequoteclose}\isanewline
\ \ \isakeyword{shows}\ {\isachardoublequoteopen}{\isasymAnd}F{\isadigit{1}}\ F{\isadigit{2}}{\isachardot}\ {\isacharparenleft}F{\isadigit{1}}\ \isactrlbold {\isasymand}\ F{\isadigit{2}}\ {\isasymin}\ S\ {\isasymlongrightarrow}\ isModel\ {\isacharparenleft}setValuation\ S{\isacharparenright}\ {\isacharparenleft}F{\isadigit{1}}\ \isactrlbold {\isasymand}\ F{\isadigit{2}}{\isacharparenright}{\isacharparenright}\ {\isasymand}\isanewline
\ \ \ \ \ \ \ {\isacharparenleft}\isactrlbold {\isasymnot}\ {\isacharparenleft}F{\isadigit{1}}\ \isactrlbold {\isasymand}\ F{\isadigit{2}}{\isacharparenright}\ {\isasymin}\ S\ {\isasymlongrightarrow}\ {\isasymnot}\ isModel\ {\isacharparenleft}setValuation\ S{\isacharparenright}\ {\isacharparenleft}F{\isadigit{1}}\ \isactrlbold {\isasymand}\ F{\isadigit{2}}{\isacharparenright}{\isacharparenright}{\isachardoublequoteclose}\isanewline
%
\isadelimproof
%
\endisadelimproof
%
\isatagproof
\isacommand{proof}\isamarkupfalse%
\ {\isacharparenleft}rule\ conjI{\isacharparenright}\isanewline
\ \ \isacommand{show}\isamarkupfalse%
\ {\isachardoublequoteopen}{\isasymAnd}F{\isadigit{1}}\ F{\isadigit{2}}{\isachardot}\ F{\isadigit{1}}\ \isactrlbold {\isasymand}\ F{\isadigit{2}}\ {\isasymin}\ S\ {\isasymlongrightarrow}\ isModel\ {\isacharparenleft}setValuation\ S{\isacharparenright}\ {\isacharparenleft}F{\isadigit{1}}\ \isactrlbold {\isasymand}\ F{\isadigit{2}}{\isacharparenright}{\isachardoublequoteclose}\isanewline
\ \ \isacommand{proof}\isamarkupfalse%
\ \isanewline
\ \ \ \ \isacommand{fix}\isamarkupfalse%
\ F{\isadigit{1}}\ F{\isadigit{2}}\isanewline
\ \ \ \ \isacommand{assume}\isamarkupfalse%
\ {\isachardoublequoteopen}F{\isadigit{1}}\ \isactrlbold {\isasymand}\ F{\isadigit{2}}\ {\isasymin}\ S{\isachardoublequoteclose}\isanewline
\ \ \ \ \isacommand{have}\isamarkupfalse%
\ {\isachardoublequoteopen}F{\isadigit{1}}\ \isactrlbold {\isasymand}\ F{\isadigit{2}}\ {\isasymin}\ S\ {\isasymlongrightarrow}\ F{\isadigit{1}}\ {\isasymin}\ S\ {\isasymand}\ F{\isadigit{2}}\ {\isasymin}\ S{\isachardoublequoteclose}\isanewline
\ \ \ \ \ \ \isacommand{using}\isamarkupfalse%
\ assms{\isacharparenleft}{\isadigit{1}}{\isacharparenright}\ \isacommand{by}\isamarkupfalse%
\ {\isacharparenleft}rule\ Hintikka{\isacharunderscore}l{\isadigit{3}}{\isacharparenright}\isanewline
\ \ \ \ \isacommand{then}\isamarkupfalse%
\ \isacommand{have}\isamarkupfalse%
\ C{\isacharcolon}{\isachardoublequoteopen}F{\isadigit{1}}\ {\isasymin}\ S\ {\isasymand}\ F{\isadigit{2}}\ {\isasymin}\ S{\isachardoublequoteclose}\isanewline
\ \ \ \ \ \ \isacommand{using}\isamarkupfalse%
\ {\isacartoucheopen}F{\isadigit{1}}\ \isactrlbold {\isasymand}\ F{\isadigit{2}}\ {\isasymin}\ S{\isacartoucheclose}\ \isacommand{by}\isamarkupfalse%
\ {\isacharparenleft}rule\ mp{\isacharparenright}\isanewline
\ \ \ \ \isacommand{then}\isamarkupfalse%
\ \isacommand{have}\isamarkupfalse%
\ {\isachardoublequoteopen}F{\isadigit{1}}\ {\isasymin}\ S{\isachardoublequoteclose}\ \isanewline
\ \ \ \ \ \ \isacommand{by}\isamarkupfalse%
\ {\isacharparenleft}rule\ conjunct{\isadigit{1}}{\isacharparenright}\isanewline
\ \ \ \ \isacommand{have}\isamarkupfalse%
\ {\isachardoublequoteopen}F{\isadigit{1}}\ {\isasymin}\ S\ {\isasymlongrightarrow}\ isModel\ {\isacharparenleft}setValuation\ S{\isacharparenright}\ F{\isadigit{1}}{\isachardoublequoteclose}\isanewline
\ \ \ \ \ \ \isacommand{using}\isamarkupfalse%
\ assms{\isacharparenleft}{\isadigit{2}}{\isacharparenright}\ \isacommand{by}\isamarkupfalse%
\ {\isacharparenleft}rule\ conjunct{\isadigit{1}}{\isacharparenright}\isanewline
\ \ \ \ \isacommand{then}\isamarkupfalse%
\ \isacommand{have}\isamarkupfalse%
\ {\isachardoublequoteopen}isModel\ {\isacharparenleft}setValuation\ S{\isacharparenright}\ F{\isadigit{1}}{\isachardoublequoteclose}\isanewline
\ \ \ \ \ \ \isacommand{using}\isamarkupfalse%
\ {\isacartoucheopen}F{\isadigit{1}}\ {\isasymin}\ S{\isacartoucheclose}\ \isacommand{by}\isamarkupfalse%
\ {\isacharparenleft}rule\ mp{\isacharparenright}\isanewline
\ \ \ \ \isacommand{then}\isamarkupfalse%
\ \isacommand{have}\isamarkupfalse%
\ I{\isadigit{1}}{\isacharcolon}{\isachardoublequoteopen}{\isacharparenleft}setValuation\ S{\isacharparenright}\ {\isasymTurnstile}\ F{\isadigit{1}}{\isachardoublequoteclose}\isanewline
\ \ \ \ \ \ \isacommand{by}\isamarkupfalse%
\ {\isacharparenleft}simp\ only{\isacharcolon}\ isModel{\isacharunderscore}def{\isacharparenright}\isanewline
\ \ \ \ \isacommand{have}\isamarkupfalse%
\ {\isachardoublequoteopen}F{\isadigit{2}}\ {\isasymin}\ S{\isachardoublequoteclose}\isanewline
\ \ \ \ \ \ \isacommand{using}\isamarkupfalse%
\ C\ \isacommand{by}\isamarkupfalse%
\ {\isacharparenleft}rule\ conjunct{\isadigit{2}}{\isacharparenright}\isanewline
\ \ \ \ \isacommand{have}\isamarkupfalse%
\ {\isachardoublequoteopen}F{\isadigit{2}}\ {\isasymin}\ S\ {\isasymlongrightarrow}\ isModel\ {\isacharparenleft}setValuation\ S{\isacharparenright}\ F{\isadigit{2}}{\isachardoublequoteclose}\isanewline
\ \ \ \ \ \ \isacommand{using}\isamarkupfalse%
\ assms{\isacharparenleft}{\isadigit{3}}{\isacharparenright}\ \isacommand{by}\isamarkupfalse%
\ {\isacharparenleft}rule\ conjunct{\isadigit{1}}{\isacharparenright}\isanewline
\ \ \ \ \isacommand{then}\isamarkupfalse%
\ \isacommand{have}\isamarkupfalse%
\ {\isachardoublequoteopen}isModel\ {\isacharparenleft}setValuation\ S{\isacharparenright}\ F{\isadigit{2}}{\isachardoublequoteclose}\isanewline
\ \ \ \ \ \ \isacommand{using}\isamarkupfalse%
\ {\isacartoucheopen}F{\isadigit{2}}\ {\isasymin}\ S{\isacartoucheclose}\ \isacommand{by}\isamarkupfalse%
\ {\isacharparenleft}rule\ mp{\isacharparenright}\isanewline
\ \ \ \ \isacommand{then}\isamarkupfalse%
\ \isacommand{have}\isamarkupfalse%
\ I{\isadigit{2}}{\isacharcolon}{\isachardoublequoteopen}{\isacharparenleft}setValuation\ S{\isacharparenright}\ {\isasymTurnstile}\ F{\isadigit{2}}{\isachardoublequoteclose}\isanewline
\ \ \ \ \ \ \isacommand{by}\isamarkupfalse%
\ {\isacharparenleft}simp\ only{\isacharcolon}\ isModel{\isacharunderscore}def{\isacharparenright}\ \isanewline
\ \ \ \ \isacommand{have}\isamarkupfalse%
\ {\isachardoublequoteopen}{\isacharparenleft}{\isacharparenleft}setValuation\ S{\isacharparenright}\ {\isasymTurnstile}\ F{\isadigit{1}}{\isacharparenright}\ {\isasymand}\ {\isacharparenleft}{\isacharparenleft}setValuation\ S{\isacharparenright}\ {\isasymTurnstile}\ F{\isadigit{2}}{\isacharparenright}{\isachardoublequoteclose}\isanewline
\ \ \ \ \ \ \isacommand{using}\isamarkupfalse%
\ I{\isadigit{1}}\ I{\isadigit{2}}\ \isacommand{by}\isamarkupfalse%
\ {\isacharparenleft}rule\ conjI{\isacharparenright}\isanewline
\ \ \ \ \isacommand{then}\isamarkupfalse%
\ \isacommand{have}\isamarkupfalse%
\ {\isachardoublequoteopen}{\isacharparenleft}setValuation\ S{\isacharparenright}\ {\isasymTurnstile}\ {\isacharparenleft}F{\isadigit{1}}\ \isactrlbold {\isasymand}\ F{\isadigit{2}}{\isacharparenright}{\isachardoublequoteclose}\isanewline
\ \ \ \ \ \ \isacommand{by}\isamarkupfalse%
\ {\isacharparenleft}simp\ only{\isacharcolon}\ formula{\isacharunderscore}semantics{\isachardot}simps{\isacharparenleft}{\isadigit{4}}{\isacharparenright}{\isacharparenright}\isanewline
\ \ \ \ \isacommand{thus}\isamarkupfalse%
\ {\isachardoublequoteopen}isModel\ {\isacharparenleft}setValuation\ S{\isacharparenright}\ {\isacharparenleft}F{\isadigit{1}}\ \isactrlbold {\isasymand}\ F{\isadigit{2}}{\isacharparenright}{\isachardoublequoteclose}\isanewline
\ \ \ \ \ \ \isacommand{by}\isamarkupfalse%
\ {\isacharparenleft}simp\ only{\isacharcolon}\ isModel{\isacharunderscore}def{\isacharparenright}\ \isanewline
\ \ \isacommand{qed}\isamarkupfalse%
\isanewline
\isacommand{next}\isamarkupfalse%
\isanewline
\ \ \isacommand{show}\isamarkupfalse%
\ {\isachardoublequoteopen}{\isasymAnd}F{\isadigit{1}}\ F{\isadigit{2}}{\isachardot}\ \isactrlbold {\isasymnot}\ {\isacharparenleft}F{\isadigit{1}}\ \isactrlbold {\isasymand}\ F{\isadigit{2}}{\isacharparenright}\ {\isasymin}\ S\ {\isasymlongrightarrow}\ {\isasymnot}\ isModel\ {\isacharparenleft}setValuation\ S{\isacharparenright}\ {\isacharparenleft}F{\isadigit{1}}\ \isactrlbold {\isasymand}\ F{\isadigit{2}}{\isacharparenright}{\isachardoublequoteclose}\isanewline
\ \ \isacommand{proof}\isamarkupfalse%
\isanewline
\ \ \ \ \isacommand{fix}\isamarkupfalse%
\ F{\isadigit{1}}\ F{\isadigit{2}}\isanewline
\ \ \ \ \isacommand{assume}\isamarkupfalse%
\ {\isachardoublequoteopen}\isactrlbold {\isasymnot}\ {\isacharparenleft}F{\isadigit{1}}\ \isactrlbold {\isasymand}\ F{\isadigit{2}}{\isacharparenright}\ {\isasymin}\ S{\isachardoublequoteclose}\isanewline
\ \ \ \ \isacommand{have}\isamarkupfalse%
\ {\isachardoublequoteopen}\isactrlbold {\isasymnot}\ {\isacharparenleft}F{\isadigit{1}}\ \isactrlbold {\isasymand}\ F{\isadigit{2}}{\isacharparenright}\ {\isasymin}\ S\ {\isasymlongrightarrow}\ \isactrlbold {\isasymnot}\ F{\isadigit{1}}\ {\isasymin}\ S\ {\isasymor}\ \isactrlbold {\isasymnot}\ F{\isadigit{2}}\ {\isasymin}\ S{\isachardoublequoteclose}\isanewline
\ \ \ \ \ \ \isacommand{using}\isamarkupfalse%
\ assms{\isacharparenleft}{\isadigit{1}}{\isacharparenright}\ \isacommand{by}\isamarkupfalse%
\ {\isacharparenleft}rule\ Hintikka{\isacharunderscore}l{\isadigit{7}}{\isacharparenright}\isanewline
\ \ \ \ \isacommand{then}\isamarkupfalse%
\ \isacommand{have}\isamarkupfalse%
\ {\isachardoublequoteopen}\isactrlbold {\isasymnot}\ F{\isadigit{1}}\ {\isasymin}\ S\ {\isasymor}\ \isactrlbold {\isasymnot}\ F{\isadigit{2}}\ {\isasymin}\ S{\isachardoublequoteclose}\isanewline
\ \ \ \ \ \ \isacommand{using}\isamarkupfalse%
\ {\isacartoucheopen}\isactrlbold {\isasymnot}\ {\isacharparenleft}F{\isadigit{1}}\ \isactrlbold {\isasymand}\ F{\isadigit{2}}{\isacharparenright}\ {\isasymin}\ S{\isacartoucheclose}\ \isacommand{by}\isamarkupfalse%
\ {\isacharparenleft}rule\ mp{\isacharparenright}\isanewline
\ \ \ \ \isacommand{then}\isamarkupfalse%
\ \isacommand{show}\isamarkupfalse%
\ {\isachardoublequoteopen}{\isasymnot}\ isModel\ {\isacharparenleft}setValuation\ S{\isacharparenright}\ {\isacharparenleft}F{\isadigit{1}}\ \isactrlbold {\isasymand}\ F{\isadigit{2}}{\isacharparenright}{\isachardoublequoteclose}\isanewline
\ \ \ \ \isacommand{proof}\isamarkupfalse%
\ {\isacharparenleft}rule\ disjE{\isacharparenright}\isanewline
\ \ \ \ \ \ \isacommand{assume}\isamarkupfalse%
\ {\isachardoublequoteopen}\isactrlbold {\isasymnot}\ F{\isadigit{1}}\ {\isasymin}\ S{\isachardoublequoteclose}\isanewline
\ \ \ \ \ \ \isacommand{have}\isamarkupfalse%
\ {\isachardoublequoteopen}\isactrlbold {\isasymnot}\ F{\isadigit{1}}\ {\isasymin}\ S\ {\isasymlongrightarrow}\ {\isasymnot}\ isModel\ {\isacharparenleft}setValuation\ S{\isacharparenright}\ F{\isadigit{1}}{\isachardoublequoteclose}\isanewline
\ \ \ \ \ \ \ \ \isacommand{using}\isamarkupfalse%
\ assms{\isacharparenleft}{\isadigit{2}}{\isacharparenright}\ \isacommand{by}\isamarkupfalse%
\ {\isacharparenleft}rule\ conjunct{\isadigit{2}}{\isacharparenright}\isanewline
\ \ \ \ \ \ \isacommand{then}\isamarkupfalse%
\ \isacommand{have}\isamarkupfalse%
\ {\isachardoublequoteopen}{\isasymnot}\ isModel\ {\isacharparenleft}setValuation\ S{\isacharparenright}\ F{\isadigit{1}}{\isachardoublequoteclose}\isanewline
\ \ \ \ \ \ \ \ \isacommand{using}\isamarkupfalse%
\ {\isacartoucheopen}\isactrlbold {\isasymnot}\ F{\isadigit{1}}\ {\isasymin}\ S{\isacartoucheclose}\ \isacommand{by}\isamarkupfalse%
\ {\isacharparenleft}rule\ mp{\isacharparenright}\isanewline
\ \ \ \ \ \ \isacommand{also}\isamarkupfalse%
\ \isacommand{have}\isamarkupfalse%
\ {\isachardoublequoteopen}{\isacharparenleft}{\isasymnot}\ isModel\ {\isacharparenleft}setValuation\ S{\isacharparenright}\ F{\isadigit{1}}{\isacharparenright}\ {\isacharequal}\ {\isacharparenleft}{\isasymnot}\ {\isacharparenleft}setValuation\ S{\isacharparenright}\ {\isasymTurnstile}\ F{\isadigit{1}}{\isacharparenright}{\isachardoublequoteclose}\isanewline
\ \ \ \ \ \ \ \ \isacommand{by}\isamarkupfalse%
\ {\isacharparenleft}simp\ only{\isacharcolon}\ isModel{\isacharunderscore}def{\isacharparenright}\isanewline
\ \ \ \ \ \ \isacommand{finally}\isamarkupfalse%
\ \isacommand{have}\isamarkupfalse%
\ {\isachardoublequoteopen}{\isasymnot}\ {\isacharparenleft}setValuation\ S{\isacharparenright}\ {\isasymTurnstile}\ F{\isadigit{1}}{\isachardoublequoteclose}\isanewline
\ \ \ \ \ \ \ \ \isacommand{by}\isamarkupfalse%
\ this\isanewline
\ \ \ \ \ \ \isacommand{then}\isamarkupfalse%
\ \isacommand{have}\isamarkupfalse%
\ {\isachardoublequoteopen}{\isasymnot}\ {\isacharparenleft}{\isacharparenleft}setValuation\ S{\isacharparenright}\ {\isasymTurnstile}\ F{\isadigit{1}}\ {\isasymand}\ {\isacharparenleft}setValuation\ S{\isacharparenright}\ {\isasymTurnstile}\ F{\isadigit{2}}{\isacharparenright}{\isachardoublequoteclose}\ \isanewline
\ \ \ \ \ \ \ \ \isacommand{by}\isamarkupfalse%
\ {\isacharparenleft}rule\ notConj{\isadigit{1}}{\isacharparenright}\isanewline
\ \ \ \ \ \ \isacommand{also}\isamarkupfalse%
\ \isacommand{have}\isamarkupfalse%
\ {\isachardoublequoteopen}{\isacharparenleft}{\isasymnot}\ {\isacharparenleft}{\isacharparenleft}setValuation\ S{\isacharparenright}\ {\isasymTurnstile}\ F{\isadigit{1}}\ {\isasymand}\ {\isacharparenleft}setValuation\ S{\isacharparenright}\ {\isasymTurnstile}\ F{\isadigit{2}}{\isacharparenright}{\isacharparenright}\ {\isacharequal}\ \isanewline
\ \ \ \ \ \ \ \ \ \ {\isacharparenleft}{\isasymnot}\ {\isacharparenleft}{\isacharparenleft}setValuation\ S{\isacharparenright}\ {\isasymTurnstile}\ F{\isadigit{1}}\ \isactrlbold {\isasymand}\ F{\isadigit{2}}{\isacharparenright}{\isacharparenright}{\isachardoublequoteclose}\isanewline
\ \ \ \ \ \ \ \ \isacommand{by}\isamarkupfalse%
\ {\isacharparenleft}simp\ only{\isacharcolon}\ formula{\isacharunderscore}semantics{\isachardot}simps{\isacharparenleft}{\isadigit{4}}{\isacharparenright}{\isacharparenright}\isanewline
\ \ \ \ \ \ \isacommand{also}\isamarkupfalse%
\ \isacommand{have}\isamarkupfalse%
\ {\isachardoublequoteopen}{\isasymdots}\ {\isacharequal}\ {\isacharparenleft}{\isasymnot}\ isModel\ {\isacharparenleft}setValuation\ S{\isacharparenright}\ {\isacharparenleft}F{\isadigit{1}}\ \isactrlbold {\isasymand}\ F{\isadigit{2}}{\isacharparenright}{\isacharparenright}{\isachardoublequoteclose}\isanewline
\ \ \ \ \ \ \ \ \isacommand{by}\isamarkupfalse%
\ {\isacharparenleft}simp\ only{\isacharcolon}\ isModel{\isacharunderscore}def{\isacharparenright}\isanewline
\ \ \ \ \ \ \isacommand{finally}\isamarkupfalse%
\ \isacommand{show}\isamarkupfalse%
\ {\isachardoublequoteopen}{\isacharparenleft}{\isasymnot}\ isModel\ {\isacharparenleft}setValuation\ S{\isacharparenright}\ {\isacharparenleft}F{\isadigit{1}}\ \isactrlbold {\isasymand}\ F{\isadigit{2}}{\isacharparenright}{\isacharparenright}{\isachardoublequoteclose}\isanewline
\ \ \ \ \ \ \ \ \isacommand{by}\isamarkupfalse%
\ this\isanewline
\ \ \ \ \isacommand{next}\isamarkupfalse%
\isanewline
\ \ \ \ \ \ \isacommand{assume}\isamarkupfalse%
\ {\isachardoublequoteopen}\isactrlbold {\isasymnot}\ F{\isadigit{2}}\ {\isasymin}\ S{\isachardoublequoteclose}\isanewline
\ \ \ \ \ \ \isacommand{have}\isamarkupfalse%
\ {\isachardoublequoteopen}\isactrlbold {\isasymnot}\ F{\isadigit{2}}\ {\isasymin}\ S\ {\isasymlongrightarrow}\ {\isasymnot}\ isModel\ {\isacharparenleft}setValuation\ S{\isacharparenright}\ F{\isadigit{2}}{\isachardoublequoteclose}\isanewline
\ \ \ \ \ \ \ \ \isacommand{using}\isamarkupfalse%
\ assms{\isacharparenleft}{\isadigit{3}}{\isacharparenright}\ \isacommand{by}\isamarkupfalse%
\ {\isacharparenleft}rule\ conjunct{\isadigit{2}}{\isacharparenright}\isanewline
\ \ \ \ \ \ \isacommand{then}\isamarkupfalse%
\ \isacommand{have}\isamarkupfalse%
\ {\isachardoublequoteopen}{\isasymnot}\ isModel\ {\isacharparenleft}setValuation\ S{\isacharparenright}\ F{\isadigit{2}}{\isachardoublequoteclose}\isanewline
\ \ \ \ \ \ \ \ \isacommand{using}\isamarkupfalse%
\ {\isacartoucheopen}\isactrlbold {\isasymnot}\ F{\isadigit{2}}\ {\isasymin}\ S{\isacartoucheclose}\ \isacommand{by}\isamarkupfalse%
\ {\isacharparenleft}rule\ mp{\isacharparenright}\isanewline
\ \ \ \ \ \ \isacommand{also}\isamarkupfalse%
\ \isacommand{have}\isamarkupfalse%
\ {\isachardoublequoteopen}{\isacharparenleft}{\isasymnot}\ isModel\ {\isacharparenleft}setValuation\ S{\isacharparenright}\ F{\isadigit{2}}{\isacharparenright}\ {\isacharequal}\ {\isacharparenleft}{\isasymnot}\ {\isacharparenleft}setValuation\ S{\isacharparenright}\ {\isasymTurnstile}\ F{\isadigit{2}}{\isacharparenright}{\isachardoublequoteclose}\isanewline
\ \ \ \ \ \ \ \ \isacommand{by}\isamarkupfalse%
\ {\isacharparenleft}simp\ only{\isacharcolon}\ isModel{\isacharunderscore}def{\isacharparenright}\isanewline
\ \ \ \ \ \ \isacommand{finally}\isamarkupfalse%
\ \isacommand{have}\isamarkupfalse%
\ {\isachardoublequoteopen}{\isasymnot}\ {\isacharparenleft}setValuation\ S{\isacharparenright}\ {\isasymTurnstile}\ F{\isadigit{2}}{\isachardoublequoteclose}\isanewline
\ \ \ \ \ \ \ \ \isacommand{by}\isamarkupfalse%
\ this\isanewline
\ \ \ \ \ \ \isacommand{then}\isamarkupfalse%
\ \isacommand{have}\isamarkupfalse%
\ {\isachardoublequoteopen}{\isasymnot}\ {\isacharparenleft}{\isacharparenleft}setValuation\ S{\isacharparenright}\ {\isasymTurnstile}\ F{\isadigit{1}}\ {\isasymand}\ {\isacharparenleft}setValuation\ S{\isacharparenright}\ {\isasymTurnstile}\ F{\isadigit{2}}{\isacharparenright}{\isachardoublequoteclose}\ \isanewline
\ \ \ \ \ \ \ \ \isacommand{by}\isamarkupfalse%
\ {\isacharparenleft}rule\ notConj{\isadigit{2}}{\isacharparenright}\isanewline
\ \ \ \ \ \ \isacommand{also}\isamarkupfalse%
\ \isacommand{have}\isamarkupfalse%
\ {\isachardoublequoteopen}{\isacharparenleft}{\isasymnot}\ {\isacharparenleft}{\isacharparenleft}setValuation\ S{\isacharparenright}\ {\isasymTurnstile}\ F{\isadigit{1}}\ {\isasymand}\ {\isacharparenleft}setValuation\ S{\isacharparenright}\ {\isasymTurnstile}\ F{\isadigit{2}}{\isacharparenright}{\isacharparenright}\ {\isacharequal}\ \isanewline
\ \ \ \ \ \ \ \ \ \ {\isacharparenleft}{\isasymnot}\ {\isacharparenleft}{\isacharparenleft}setValuation\ S{\isacharparenright}\ {\isasymTurnstile}\ {\isacharparenleft}F{\isadigit{1}}\ \isactrlbold {\isasymand}\ F{\isadigit{2}}{\isacharparenright}{\isacharparenright}{\isacharparenright}{\isachardoublequoteclose}\isanewline
\ \ \ \ \ \ \ \ \isacommand{by}\isamarkupfalse%
\ {\isacharparenleft}simp\ only{\isacharcolon}\ formula{\isacharunderscore}semantics{\isachardot}simps{\isacharparenleft}{\isadigit{4}}{\isacharparenright}{\isacharparenright}\isanewline
\ \ \ \ \ \ \isacommand{also}\isamarkupfalse%
\ \isacommand{have}\isamarkupfalse%
\ {\isachardoublequoteopen}{\isasymdots}\ {\isacharequal}\ {\isacharparenleft}{\isasymnot}\ isModel\ {\isacharparenleft}setValuation\ S{\isacharparenright}\ {\isacharparenleft}F{\isadigit{1}}\ \isactrlbold {\isasymand}\ F{\isadigit{2}}{\isacharparenright}{\isacharparenright}{\isachardoublequoteclose}\isanewline
\ \ \ \ \ \ \ \ \isacommand{by}\isamarkupfalse%
\ {\isacharparenleft}simp\ only{\isacharcolon}\ isModel{\isacharunderscore}def{\isacharparenright}\isanewline
\ \ \ \ \ \ \isacommand{finally}\isamarkupfalse%
\ \isacommand{show}\isamarkupfalse%
\ {\isachardoublequoteopen}{\isasymnot}\ isModel\ {\isacharparenleft}setValuation\ S{\isacharparenright}\ {\isacharparenleft}F{\isadigit{1}}\ \isactrlbold {\isasymand}\ F{\isadigit{2}}{\isacharparenright}{\isachardoublequoteclose}\isanewline
\ \ \ \ \ \ \ \ \isacommand{by}\isamarkupfalse%
\ this\isanewline
\ \ \ \ \isacommand{qed}\isamarkupfalse%
\isanewline
\ \ \isacommand{qed}\isamarkupfalse%
\isanewline
\isacommand{qed}\isamarkupfalse%
%
\endisatagproof
{\isafoldproof}%
%
\isadelimproof
\isanewline
%
\endisadelimproof
\isanewline
\isacommand{lemma}\isamarkupfalse%
\ Hl{\isadigit{2}}{\isacharunderscore}{\isadigit{4}}{\isacharcolon}\isanewline
\ \ \isakeyword{assumes}\ {\isachardoublequoteopen}Hintikka\ S{\isachardoublequoteclose}\isanewline
\ \ \isakeyword{shows}\ {\isachardoublequoteopen}{\isasymAnd}F{\isadigit{1}}\ F{\isadigit{2}}{\isachardot}\isanewline
\ \ \ \ \ \ \ {\isacharparenleft}F{\isadigit{1}}\ {\isasymin}\ S\ {\isasymlongrightarrow}\ isModel\ {\isacharparenleft}setValuation\ S{\isacharparenright}\ F{\isadigit{1}}{\isacharparenright}\ {\isasymand}\isanewline
\ \ \ \ \ \ \ {\isacharparenleft}\isactrlbold {\isasymnot}\ F{\isadigit{1}}\ {\isasymin}\ S\ {\isasymlongrightarrow}\ {\isasymnot}\ isModel\ {\isacharparenleft}setValuation\ S{\isacharparenright}\ F{\isadigit{1}}{\isacharparenright}\ {\isasymLongrightarrow}\isanewline
\ \ \ \ \ \ \ {\isacharparenleft}F{\isadigit{2}}\ {\isasymin}\ S\ {\isasymlongrightarrow}\ isModel\ {\isacharparenleft}setValuation\ S{\isacharparenright}\ F{\isadigit{2}}{\isacharparenright}\ {\isasymand}\isanewline
\ \ \ \ \ \ \ {\isacharparenleft}\isactrlbold {\isasymnot}\ F{\isadigit{2}}\ {\isasymin}\ S\ {\isasymlongrightarrow}\ {\isasymnot}\ isModel\ {\isacharparenleft}setValuation\ S{\isacharparenright}\ F{\isadigit{2}}{\isacharparenright}\ {\isasymLongrightarrow}\isanewline
\ \ \ \ \ \ \ {\isacharparenleft}F{\isadigit{1}}\ \isactrlbold {\isasymand}\ F{\isadigit{2}}\ {\isasymin}\ S\ {\isasymlongrightarrow}\ isModel\ {\isacharparenleft}setValuation\ S{\isacharparenright}\ {\isacharparenleft}F{\isadigit{1}}\ \isactrlbold {\isasymand}\ F{\isadigit{2}}{\isacharparenright}{\isacharparenright}\ {\isasymand}\isanewline
\ \ \ \ \ \ \ {\isacharparenleft}\isactrlbold {\isasymnot}\ {\isacharparenleft}F{\isadigit{1}}\ \isactrlbold {\isasymand}\ F{\isadigit{2}}{\isacharparenright}\ {\isasymin}\ S\ {\isasymlongrightarrow}\ {\isasymnot}\ isModel\ {\isacharparenleft}setValuation\ S{\isacharparenright}\ {\isacharparenleft}F{\isadigit{1}}\ \isactrlbold {\isasymand}\ F{\isadigit{2}}{\isacharparenright}{\isacharparenright}{\isachardoublequoteclose}\isanewline
%
\isadelimproof
\ \ %
\endisadelimproof
%
\isatagproof
\isacommand{by}\isamarkupfalse%
\ {\isacharparenleft}meson\ Hintikka{\isacharunderscore}l{\isadigit{3}}\ Hintikka{\isacharunderscore}l{\isadigit{7}}\ assms\ isModel{\isacharunderscore}def\ formula{\isacharunderscore}semantics{\isachardot}simps{\isacharparenleft}{\isadigit{4}}{\isacharparenright}{\isacharparenright}%
\endisatagproof
{\isafoldproof}%
%
\isadelimproof
\isanewline
%
\endisadelimproof
\isanewline
\isacommand{lemma}\isamarkupfalse%
\isanewline
\ \ \isakeyword{assumes}\ {\isachardoublequoteopen}Hintikka\ S{\isachardoublequoteclose}\isanewline
\ \ \ \ \ \ \ \ \ \ {\isachardoublequoteopen}{\isasymAnd}F{\isadigit{1}}{\isachardot}\ {\isacharparenleft}F{\isadigit{1}}\ {\isasymin}\ S\ {\isasymlongrightarrow}\ isModel\ {\isacharparenleft}setValuation\ S{\isacharparenright}\ F{\isadigit{1}}{\isacharparenright}\ {\isasymand}\isanewline
\ \ \ \ \ \ \ \ \ \ \ {\isacharparenleft}\isactrlbold {\isasymnot}\ F{\isadigit{1}}\ {\isasymin}\ S\ {\isasymlongrightarrow}\ {\isasymnot}\ isModel\ {\isacharparenleft}setValuation\ S{\isacharparenright}\ F{\isadigit{1}}{\isacharparenright}{\isachardoublequoteclose}\ \isanewline
\ \ \ \ \ \ \ \ \ \ {\isachardoublequoteopen}{\isasymAnd}F{\isadigit{2}}{\isachardot}\ {\isacharparenleft}F{\isadigit{2}}\ {\isasymin}\ S\ {\isasymlongrightarrow}\ isModel\ {\isacharparenleft}setValuation\ S{\isacharparenright}\ F{\isadigit{2}}{\isacharparenright}\ {\isasymand}\isanewline
\ \ \ \ \ \ \ \ \ \ \ {\isacharparenleft}\isactrlbold {\isasymnot}\ F{\isadigit{2}}\ {\isasymin}\ S\ {\isasymlongrightarrow}\ {\isasymnot}\ isModel\ {\isacharparenleft}setValuation\ S{\isacharparenright}\ F{\isadigit{2}}{\isacharparenright}{\isachardoublequoteclose}\isanewline
\ \ \isakeyword{shows}\ {\isachardoublequoteopen}{\isasymAnd}F{\isadigit{1}}\ F{\isadigit{2}}{\isachardot}\ {\isacharparenleft}F{\isadigit{1}}\ \isactrlbold {\isasymor}\ F{\isadigit{2}}\ {\isasymin}\ S\ {\isasymlongrightarrow}\ isModel\ {\isacharparenleft}setValuation\ S{\isacharparenright}\ {\isacharparenleft}F{\isadigit{1}}\ \isactrlbold {\isasymor}\ F{\isadigit{2}}{\isacharparenright}{\isacharparenright}\ \isanewline
\ \ {\isasymand}\ {\isacharparenleft}\isactrlbold {\isasymnot}\ {\isacharparenleft}F{\isadigit{1}}\ \isactrlbold {\isasymor}\ F{\isadigit{2}}{\isacharparenright}\ {\isasymin}\ S\ {\isasymlongrightarrow}\ {\isasymnot}\ isModel\ {\isacharparenleft}setValuation\ S{\isacharparenright}\ {\isacharparenleft}F{\isadigit{1}}\ \isactrlbold {\isasymor}\ F{\isadigit{2}}{\isacharparenright}{\isacharparenright}{\isachardoublequoteclose}\isanewline
%
\isadelimproof
%
\endisadelimproof
%
\isatagproof
\isacommand{proof}\isamarkupfalse%
\ {\isacharparenleft}rule\ conjI{\isacharparenright}\isanewline
\ \ \isacommand{show}\isamarkupfalse%
\ {\isachardoublequoteopen}{\isasymAnd}\ F{\isadigit{1}}\ F{\isadigit{2}}{\isachardot}\ F{\isadigit{1}}\ \isactrlbold {\isasymor}\ F{\isadigit{2}}\ {\isasymin}\ S\ {\isasymlongrightarrow}\ isModel\ {\isacharparenleft}setValuation\ S{\isacharparenright}\ {\isacharparenleft}F{\isadigit{1}}\ \isactrlbold {\isasymor}\ F{\isadigit{2}}{\isacharparenright}{\isachardoublequoteclose}\isanewline
\ \ \isacommand{proof}\isamarkupfalse%
\isanewline
\ \ \ \ \isacommand{fix}\isamarkupfalse%
\ F{\isadigit{1}}\ F{\isadigit{2}}\isanewline
\ \ \ \ \isacommand{assume}\isamarkupfalse%
\ {\isachardoublequoteopen}F{\isadigit{1}}\ \isactrlbold {\isasymor}\ F{\isadigit{2}}\ {\isasymin}\ S{\isachardoublequoteclose}\isanewline
\ \ \ \ \isacommand{have}\isamarkupfalse%
\ {\isachardoublequoteopen}F{\isadigit{1}}\ \isactrlbold {\isasymor}\ F{\isadigit{2}}\ {\isasymin}\ S\ {\isasymlongrightarrow}\ F{\isadigit{1}}\ {\isasymin}\ S\ {\isasymor}\ F{\isadigit{2}}\ {\isasymin}\ S{\isachardoublequoteclose}\isanewline
\ \ \ \ \ \ \isacommand{using}\isamarkupfalse%
\ assms{\isacharparenleft}{\isadigit{1}}{\isacharparenright}\ \isacommand{by}\isamarkupfalse%
\ {\isacharparenleft}rule\ Hintikka{\isacharunderscore}l{\isadigit{4}}{\isacharparenright}\isanewline
\ \ \ \ \isacommand{then}\isamarkupfalse%
\ \isacommand{have}\isamarkupfalse%
\ {\isachardoublequoteopen}F{\isadigit{1}}\ {\isasymin}\ S\ {\isasymor}\ F{\isadigit{2}}\ {\isasymin}\ S{\isachardoublequoteclose}\isanewline
\ \ \ \ \ \ \isacommand{using}\isamarkupfalse%
\ {\isacartoucheopen}F{\isadigit{1}}\ \isactrlbold {\isasymor}\ F{\isadigit{2}}\ {\isasymin}\ S{\isacartoucheclose}\ \isacommand{by}\isamarkupfalse%
\ {\isacharparenleft}rule\ mp{\isacharparenright}\isanewline
\ \ \ \ \isacommand{then}\isamarkupfalse%
\ \isacommand{show}\isamarkupfalse%
\ {\isachardoublequoteopen}isModel\ {\isacharparenleft}setValuation\ S{\isacharparenright}\ {\isacharparenleft}F{\isadigit{1}}\ \isactrlbold {\isasymor}\ F{\isadigit{2}}{\isacharparenright}{\isachardoublequoteclose}\isanewline
\ \ \ \ \isacommand{proof}\isamarkupfalse%
\ {\isacharparenleft}rule\ disjE{\isacharparenright}\isanewline
\ \ \ \ \ \ \isacommand{assume}\isamarkupfalse%
\ {\isachardoublequoteopen}F{\isadigit{1}}\ {\isasymin}\ S{\isachardoublequoteclose}\isanewline
\ \ \ \ \ \ \isacommand{have}\isamarkupfalse%
\ {\isachardoublequoteopen}F{\isadigit{1}}\ {\isasymin}\ S\ {\isasymlongrightarrow}\ isModel\ {\isacharparenleft}setValuation\ S{\isacharparenright}\ F{\isadigit{1}}{\isachardoublequoteclose}\isanewline
\ \ \ \ \ \ \ \ \isacommand{using}\isamarkupfalse%
\ assms{\isacharparenleft}{\isadigit{2}}{\isacharparenright}\ \isacommand{by}\isamarkupfalse%
\ {\isacharparenleft}rule\ conjunct{\isadigit{1}}{\isacharparenright}\isanewline
\ \ \ \ \ \ \isacommand{then}\isamarkupfalse%
\ \isacommand{have}\isamarkupfalse%
\ {\isachardoublequoteopen}isModel\ {\isacharparenleft}setValuation\ S{\isacharparenright}\ F{\isadigit{1}}{\isachardoublequoteclose}\ \isanewline
\ \ \ \ \ \ \ \ \isacommand{using}\isamarkupfalse%
\ {\isacartoucheopen}F{\isadigit{1}}\ {\isasymin}\ S{\isacartoucheclose}\ \isacommand{by}\isamarkupfalse%
\ {\isacharparenleft}rule\ mp{\isacharparenright}\isanewline
\ \ \ \ \ \ \isacommand{then}\isamarkupfalse%
\ \isacommand{have}\isamarkupfalse%
\ {\isachardoublequoteopen}{\isacharparenleft}setValuation\ S{\isacharparenright}\ {\isasymTurnstile}\ F{\isadigit{1}}{\isachardoublequoteclose}\isanewline
\ \ \ \ \ \ \ \ \isacommand{by}\isamarkupfalse%
\ {\isacharparenleft}simp\ only{\isacharcolon}\ isModel{\isacharunderscore}def{\isacharparenright}\isanewline
\ \ \ \ \ \ \isacommand{then}\isamarkupfalse%
\ \isacommand{have}\isamarkupfalse%
\ {\isachardoublequoteopen}{\isacharparenleft}setValuation\ S{\isacharparenright}\ {\isasymTurnstile}\ F{\isadigit{1}}\ {\isasymor}\ {\isacharparenleft}setValuation\ S{\isacharparenright}\ {\isasymTurnstile}\ F{\isadigit{2}}{\isachardoublequoteclose}\isanewline
\ \ \ \ \ \ \ \ \isacommand{by}\isamarkupfalse%
\ {\isacharparenleft}rule\ disjI{\isadigit{1}}{\isacharparenright}\isanewline
\ \ \ \ \ \ \isacommand{then}\isamarkupfalse%
\ \isacommand{have}\isamarkupfalse%
\ {\isachardoublequoteopen}{\isacharparenleft}setValuation\ S{\isacharparenright}\ {\isasymTurnstile}\ {\isacharparenleft}F{\isadigit{1}}\ \isactrlbold {\isasymor}\ F{\isadigit{2}}{\isacharparenright}{\isachardoublequoteclose}\isanewline
\ \ \ \ \ \ \ \ \isacommand{by}\isamarkupfalse%
\ {\isacharparenleft}simp\ only{\isacharcolon}\ formula{\isacharunderscore}semantics{\isachardot}simps{\isacharparenleft}{\isadigit{5}}{\isacharparenright}{\isacharparenright}\isanewline
\ \ \ \ \ \ \isacommand{thus}\isamarkupfalse%
\ {\isachardoublequoteopen}isModel\ {\isacharparenleft}setValuation\ S{\isacharparenright}\ {\isacharparenleft}F{\isadigit{1}}\ \isactrlbold {\isasymor}\ F{\isadigit{2}}{\isacharparenright}{\isachardoublequoteclose}\isanewline
\ \ \ \ \ \ \ \ \isacommand{by}\isamarkupfalse%
\ {\isacharparenleft}simp\ only{\isacharcolon}\ isModel{\isacharunderscore}def{\isacharparenright}\isanewline
\ \ \ \ \isacommand{next}\isamarkupfalse%
\isanewline
\ \ \ \ \ \ \isacommand{assume}\isamarkupfalse%
\ {\isachardoublequoteopen}F{\isadigit{2}}\ {\isasymin}\ S{\isachardoublequoteclose}\isanewline
\ \ \ \ \ \ \isacommand{have}\isamarkupfalse%
\ {\isachardoublequoteopen}F{\isadigit{2}}\ {\isasymin}\ S\ {\isasymlongrightarrow}\ isModel\ {\isacharparenleft}setValuation\ S{\isacharparenright}\ F{\isadigit{2}}{\isachardoublequoteclose}\isanewline
\ \ \ \ \ \ \ \ \isacommand{using}\isamarkupfalse%
\ assms{\isacharparenleft}{\isadigit{3}}{\isacharparenright}\ \isacommand{by}\isamarkupfalse%
\ {\isacharparenleft}rule\ conjunct{\isadigit{1}}{\isacharparenright}\isanewline
\ \ \ \ \ \ \isacommand{then}\isamarkupfalse%
\ \isacommand{have}\isamarkupfalse%
\ {\isachardoublequoteopen}isModel\ {\isacharparenleft}setValuation\ S{\isacharparenright}\ F{\isadigit{2}}{\isachardoublequoteclose}\ \isanewline
\ \ \ \ \ \ \ \ \isacommand{using}\isamarkupfalse%
\ {\isacartoucheopen}F{\isadigit{2}}\ {\isasymin}\ S{\isacartoucheclose}\ \isacommand{by}\isamarkupfalse%
\ {\isacharparenleft}rule\ mp{\isacharparenright}\isanewline
\ \ \ \ \ \ \isacommand{then}\isamarkupfalse%
\ \isacommand{have}\isamarkupfalse%
\ {\isachardoublequoteopen}{\isacharparenleft}setValuation\ S{\isacharparenright}\ {\isasymTurnstile}\ F{\isadigit{2}}{\isachardoublequoteclose}\isanewline
\ \ \ \ \ \ \ \ \isacommand{by}\isamarkupfalse%
\ {\isacharparenleft}simp\ only{\isacharcolon}\ isModel{\isacharunderscore}def{\isacharparenright}\isanewline
\ \ \ \ \ \ \isacommand{then}\isamarkupfalse%
\ \isacommand{have}\isamarkupfalse%
\ {\isachardoublequoteopen}{\isacharparenleft}setValuation\ S{\isacharparenright}\ {\isasymTurnstile}\ F{\isadigit{1}}\ {\isasymor}\ {\isacharparenleft}setValuation\ S{\isacharparenright}\ {\isasymTurnstile}\ F{\isadigit{2}}{\isachardoublequoteclose}\isanewline
\ \ \ \ \ \ \ \ \isacommand{by}\isamarkupfalse%
\ {\isacharparenleft}rule\ disjI{\isadigit{2}}{\isacharparenright}\isanewline
\ \ \ \ \ \ \isacommand{then}\isamarkupfalse%
\ \isacommand{have}\isamarkupfalse%
\ {\isachardoublequoteopen}{\isacharparenleft}setValuation\ S{\isacharparenright}\ {\isasymTurnstile}\ {\isacharparenleft}F{\isadigit{1}}\ \isactrlbold {\isasymor}\ F{\isadigit{2}}{\isacharparenright}{\isachardoublequoteclose}\isanewline
\ \ \ \ \ \ \ \ \isacommand{by}\isamarkupfalse%
\ {\isacharparenleft}simp\ only{\isacharcolon}\ formula{\isacharunderscore}semantics{\isachardot}simps{\isacharparenleft}{\isadigit{5}}{\isacharparenright}{\isacharparenright}\isanewline
\ \ \ \ \ \ \isacommand{thus}\isamarkupfalse%
\ {\isachardoublequoteopen}isModel\ {\isacharparenleft}setValuation\ S{\isacharparenright}\ {\isacharparenleft}F{\isadigit{1}}\ \isactrlbold {\isasymor}\ F{\isadigit{2}}{\isacharparenright}{\isachardoublequoteclose}\isanewline
\ \ \ \ \ \ \ \ \isacommand{by}\isamarkupfalse%
\ {\isacharparenleft}simp\ only{\isacharcolon}\ isModel{\isacharunderscore}def{\isacharparenright}\isanewline
\ \ \ \ \isacommand{qed}\isamarkupfalse%
\isanewline
\ \ \isacommand{qed}\isamarkupfalse%
\isanewline
\isacommand{next}\isamarkupfalse%
\isanewline
\ \ \isacommand{show}\isamarkupfalse%
\ {\isachardoublequoteopen}{\isasymAnd}F{\isadigit{1}}\ F{\isadigit{2}}{\isachardot}\ \isactrlbold {\isasymnot}\ {\isacharparenleft}F{\isadigit{1}}\ \isactrlbold {\isasymor}\ F{\isadigit{2}}{\isacharparenright}\ {\isasymin}\ S\ {\isasymlongrightarrow}\ {\isasymnot}\ isModel\ {\isacharparenleft}setValuation\ S{\isacharparenright}\ {\isacharparenleft}F{\isadigit{1}}\ \isactrlbold {\isasymor}\ F{\isadigit{2}}{\isacharparenright}{\isachardoublequoteclose}\isanewline
\ \ \isacommand{proof}\isamarkupfalse%
\ {\isacharparenleft}rule\ impI{\isacharparenright}\isanewline
\ \ \ \ \isacommand{fix}\isamarkupfalse%
\ F{\isadigit{1}}\ F{\isadigit{2}}\isanewline
\ \ \ \ \isacommand{assume}\isamarkupfalse%
\ {\isachardoublequoteopen}\isactrlbold {\isasymnot}\ {\isacharparenleft}F{\isadigit{1}}\ \isactrlbold {\isasymor}\ F{\isadigit{2}}{\isacharparenright}\ {\isasymin}\ S{\isachardoublequoteclose}\isanewline
\ \ \ \ \isacommand{have}\isamarkupfalse%
\ {\isachardoublequoteopen}\isactrlbold {\isasymnot}\ {\isacharparenleft}F{\isadigit{1}}\ \isactrlbold {\isasymor}\ F{\isadigit{2}}{\isacharparenright}\ {\isasymin}\ S\ {\isasymlongrightarrow}\ \isactrlbold {\isasymnot}\ F{\isadigit{1}}\ {\isasymin}\ S\ {\isasymand}\ \isactrlbold {\isasymnot}\ F{\isadigit{2}}\ {\isasymin}\ S{\isachardoublequoteclose}\isanewline
\ \ \ \ \ \ \isacommand{using}\isamarkupfalse%
\ assms{\isacharparenleft}{\isadigit{1}}{\isacharparenright}\ \isacommand{by}\isamarkupfalse%
\ {\isacharparenleft}rule\ Hintikka{\isacharunderscore}l{\isadigit{8}}{\isacharparenright}\isanewline
\ \ \ \ \isacommand{then}\isamarkupfalse%
\ \isacommand{have}\isamarkupfalse%
\ C{\isacharcolon}{\isachardoublequoteopen}\isactrlbold {\isasymnot}\ F{\isadigit{1}}\ {\isasymin}\ S\ {\isasymand}\ \isactrlbold {\isasymnot}\ F{\isadigit{2}}\ {\isasymin}\ S{\isachardoublequoteclose}\isanewline
\ \ \ \ \ \ \isacommand{using}\isamarkupfalse%
\ {\isacartoucheopen}\isactrlbold {\isasymnot}\ {\isacharparenleft}F{\isadigit{1}}\ \isactrlbold {\isasymor}\ F{\isadigit{2}}{\isacharparenright}\ {\isasymin}\ S{\isacartoucheclose}\ \isacommand{by}\isamarkupfalse%
\ {\isacharparenleft}rule\ mp{\isacharparenright}\isanewline
\ \ \ \ \isacommand{then}\isamarkupfalse%
\ \isacommand{have}\isamarkupfalse%
\ {\isachardoublequoteopen}\isactrlbold {\isasymnot}\ F{\isadigit{1}}\ {\isasymin}\ S{\isachardoublequoteclose}\isanewline
\ \ \ \ \ \ \isacommand{by}\isamarkupfalse%
\ {\isacharparenleft}rule\ conjunct{\isadigit{1}}{\isacharparenright}\isanewline
\ \ \ \ \isacommand{have}\isamarkupfalse%
\ {\isachardoublequoteopen}\isactrlbold {\isasymnot}\ F{\isadigit{1}}\ {\isasymin}\ S\ {\isasymlongrightarrow}\ {\isasymnot}\ isModel\ {\isacharparenleft}setValuation\ S{\isacharparenright}\ F{\isadigit{1}}{\isachardoublequoteclose}\isanewline
\ \ \ \ \ \ \isacommand{using}\isamarkupfalse%
\ assms{\isacharparenleft}{\isadigit{2}}{\isacharparenright}\ \isacommand{by}\isamarkupfalse%
\ {\isacharparenleft}rule\ conjunct{\isadigit{2}}{\isacharparenright}\isanewline
\ \ \ \ \isacommand{then}\isamarkupfalse%
\ \isacommand{have}\isamarkupfalse%
\ {\isachardoublequoteopen}{\isasymnot}\ isModel\ {\isacharparenleft}setValuation\ S{\isacharparenright}\ F{\isadigit{1}}{\isachardoublequoteclose}\isanewline
\ \ \ \ \ \ \isacommand{using}\isamarkupfalse%
\ {\isacartoucheopen}\isactrlbold {\isasymnot}\ F{\isadigit{1}}\ {\isasymin}\ S{\isacartoucheclose}\ \isacommand{by}\isamarkupfalse%
\ {\isacharparenleft}rule\ mp{\isacharparenright}\isanewline
\ \ \ \ \isacommand{also}\isamarkupfalse%
\ \isacommand{have}\isamarkupfalse%
\ {\isachardoublequoteopen}{\isacharparenleft}{\isasymnot}\ isModel\ {\isacharparenleft}setValuation\ S{\isacharparenright}\ F{\isadigit{1}}{\isacharparenright}\ {\isacharequal}\ {\isacharparenleft}{\isasymnot}\ {\isacharparenleft}setValuation\ S{\isacharparenright}\ {\isasymTurnstile}\ F{\isadigit{1}}{\isacharparenright}{\isachardoublequoteclose}\isanewline
\ \ \ \ \ \ \isacommand{by}\isamarkupfalse%
\ {\isacharparenleft}simp\ only{\isacharcolon}\ isModel{\isacharunderscore}def{\isacharparenright}\isanewline
\ \ \ \ \isacommand{finally}\isamarkupfalse%
\ \isacommand{have}\isamarkupfalse%
\ D{\isadigit{1}}{\isacharcolon}{\isachardoublequoteopen}{\isasymnot}\ {\isacharparenleft}setValuation\ S{\isacharparenright}\ {\isasymTurnstile}\ F{\isadigit{1}}{\isachardoublequoteclose}\isanewline
\ \ \ \ \ \ \isacommand{by}\isamarkupfalse%
\ this\isanewline
\ \ \ \ \isacommand{have}\isamarkupfalse%
\ {\isachardoublequoteopen}\isactrlbold {\isasymnot}\ F{\isadigit{2}}\ {\isasymin}\ S{\isachardoublequoteclose}\isanewline
\ \ \ \ \ \ \isacommand{using}\isamarkupfalse%
\ C\ \isacommand{by}\isamarkupfalse%
\ {\isacharparenleft}rule\ conjunct{\isadigit{2}}{\isacharparenright}\isanewline
\ \ \ \ \isacommand{have}\isamarkupfalse%
\ {\isachardoublequoteopen}\isactrlbold {\isasymnot}\ F{\isadigit{2}}\ {\isasymin}\ S\ {\isasymlongrightarrow}\ {\isasymnot}\ isModel\ {\isacharparenleft}setValuation\ S{\isacharparenright}\ F{\isadigit{2}}{\isachardoublequoteclose}\isanewline
\ \ \ \ \ \ \isacommand{using}\isamarkupfalse%
\ assms{\isacharparenleft}{\isadigit{3}}{\isacharparenright}\ \isacommand{by}\isamarkupfalse%
\ {\isacharparenleft}rule\ conjunct{\isadigit{2}}{\isacharparenright}\isanewline
\ \ \ \ \isacommand{then}\isamarkupfalse%
\ \isacommand{have}\isamarkupfalse%
\ {\isachardoublequoteopen}{\isasymnot}\ isModel\ {\isacharparenleft}setValuation\ S{\isacharparenright}\ F{\isadigit{2}}{\isachardoublequoteclose}\isanewline
\ \ \ \ \ \ \isacommand{using}\isamarkupfalse%
\ {\isacartoucheopen}\isactrlbold {\isasymnot}\ F{\isadigit{2}}\ {\isasymin}\ S{\isacartoucheclose}\ \isacommand{by}\isamarkupfalse%
\ {\isacharparenleft}rule\ mp{\isacharparenright}\isanewline
\ \ \ \ \isacommand{also}\isamarkupfalse%
\ \isacommand{have}\isamarkupfalse%
\ {\isachardoublequoteopen}{\isacharparenleft}{\isasymnot}\ isModel\ {\isacharparenleft}setValuation\ S{\isacharparenright}\ F{\isadigit{2}}{\isacharparenright}\ {\isacharequal}\ {\isacharparenleft}{\isasymnot}\ {\isacharparenleft}setValuation\ S{\isacharparenright}\ {\isasymTurnstile}\ F{\isadigit{2}}{\isacharparenright}{\isachardoublequoteclose}\isanewline
\ \ \ \ \ \ \isacommand{by}\isamarkupfalse%
\ {\isacharparenleft}simp\ only{\isacharcolon}\ isModel{\isacharunderscore}def{\isacharparenright}\isanewline
\ \ \ \ \isacommand{finally}\isamarkupfalse%
\ \isacommand{have}\isamarkupfalse%
\ D{\isadigit{2}}{\isacharcolon}{\isachardoublequoteopen}{\isasymnot}\ {\isacharparenleft}setValuation\ S{\isacharparenright}\ {\isasymTurnstile}\ F{\isadigit{2}}{\isachardoublequoteclose}\isanewline
\ \ \ \ \ \ \isacommand{by}\isamarkupfalse%
\ this\isanewline
\ \ \ \ \isacommand{have}\isamarkupfalse%
\ {\isachardoublequoteopen}{\isasymnot}\ {\isacharparenleft}{\isacharparenleft}setValuation\ S{\isacharparenright}\ {\isasymTurnstile}\ F{\isadigit{1}}\ {\isasymor}\ {\isacharparenleft}setValuation\ S{\isacharparenright}\ {\isasymTurnstile}\ F{\isadigit{2}}{\isacharparenright}{\isachardoublequoteclose}\isanewline
\ \ \ \ \ \ \isacommand{using}\isamarkupfalse%
\ D{\isadigit{1}}\ D{\isadigit{2}}\ \isacommand{by}\isamarkupfalse%
\ {\isacharparenleft}rule\ notDisj{\isacharparenright}\isanewline
\ \ \ \ \isacommand{also}\isamarkupfalse%
\ \isacommand{have}\isamarkupfalse%
\ {\isachardoublequoteopen}{\isacharparenleft}{\isasymnot}\ {\isacharparenleft}{\isacharparenleft}setValuation\ S{\isacharparenright}\ {\isasymTurnstile}\ F{\isadigit{1}}\ {\isasymor}\ {\isacharparenleft}setValuation\ S{\isacharparenright}\ {\isasymTurnstile}\ F{\isadigit{2}}{\isacharparenright}{\isacharparenright}\ {\isacharequal}\ \isanewline
\ \ \ \ \ \ \ \ \ \ {\isacharparenleft}{\isasymnot}\ {\isacharparenleft}setValuation\ S{\isacharparenright}\ {\isasymTurnstile}\ {\isacharparenleft}F{\isadigit{1}}\ \isactrlbold {\isasymor}\ F{\isadigit{2}}{\isacharparenright}{\isacharparenright}{\isachardoublequoteclose}\isanewline
\ \ \ \ \ \ \isacommand{by}\isamarkupfalse%
\ {\isacharparenleft}simp\ only{\isacharcolon}\ formula{\isacharunderscore}semantics{\isachardot}simps{\isacharparenleft}{\isadigit{5}}{\isacharparenright}{\isacharparenright}\isanewline
\ \ \ \ \isacommand{also}\isamarkupfalse%
\ \isacommand{have}\isamarkupfalse%
\ {\isachardoublequoteopen}{\isasymdots}\ {\isacharequal}\ {\isacharparenleft}{\isasymnot}\ isModel\ {\isacharparenleft}setValuation\ S{\isacharparenright}\ {\isacharparenleft}F{\isadigit{1}}\ \isactrlbold {\isasymor}\ F{\isadigit{2}}{\isacharparenright}{\isacharparenright}{\isachardoublequoteclose}\isanewline
\ \ \ \ \ \ \isacommand{by}\isamarkupfalse%
\ {\isacharparenleft}simp\ only{\isacharcolon}\ isModel{\isacharunderscore}def{\isacharparenright}\isanewline
\ \ \ \ \isacommand{finally}\isamarkupfalse%
\ \isacommand{show}\isamarkupfalse%
\ {\isachardoublequoteopen}{\isasymnot}\ isModel\ {\isacharparenleft}setValuation\ S{\isacharparenright}\ {\isacharparenleft}F{\isadigit{1}}\ \isactrlbold {\isasymor}\ F{\isadigit{2}}{\isacharparenright}{\isachardoublequoteclose}\isanewline
\ \ \ \ \ \ \isacommand{by}\isamarkupfalse%
\ this\isanewline
\ \ \isacommand{qed}\isamarkupfalse%
\isanewline
\isacommand{qed}\isamarkupfalse%
%
\endisatagproof
{\isafoldproof}%
%
\isadelimproof
\isanewline
%
\endisadelimproof
\isanewline
\isanewline
\isacommand{lemma}\isamarkupfalse%
\ Hl{\isadigit{2}}{\isacharunderscore}{\isadigit{5}}{\isacharcolon}\isanewline
\ \ \isakeyword{assumes}\ {\isachardoublequoteopen}Hintikka\ S{\isachardoublequoteclose}\isanewline
\ \ \isakeyword{shows}\ {\isachardoublequoteopen}{\isasymAnd}F{\isadigit{1}}\ F{\isadigit{2}}{\isachardot}\isanewline
\ \ \ \ \ \ \ {\isacharparenleft}F{\isadigit{1}}\ {\isasymin}\ S\ {\isasymlongrightarrow}\ isModel\ {\isacharparenleft}setValuation\ S{\isacharparenright}\ F{\isadigit{1}}{\isacharparenright}\ {\isasymand}\isanewline
\ \ \ \ \ \ \ {\isacharparenleft}\isactrlbold {\isasymnot}\ F{\isadigit{1}}\ {\isasymin}\ S\ {\isasymlongrightarrow}\ {\isasymnot}\ isModel\ {\isacharparenleft}setValuation\ S{\isacharparenright}\ F{\isadigit{1}}{\isacharparenright}\ {\isasymLongrightarrow}\isanewline
\ \ \ \ \ \ \ {\isacharparenleft}F{\isadigit{2}}\ {\isasymin}\ S\ {\isasymlongrightarrow}\ isModel\ {\isacharparenleft}setValuation\ S{\isacharparenright}\ F{\isadigit{2}}{\isacharparenright}\ {\isasymand}\isanewline
\ \ \ \ \ \ \ {\isacharparenleft}\isactrlbold {\isasymnot}\ F{\isadigit{2}}\ {\isasymin}\ S\ {\isasymlongrightarrow}\ {\isasymnot}\ isModel\ {\isacharparenleft}setValuation\ S{\isacharparenright}\ F{\isadigit{2}}{\isacharparenright}\ {\isasymLongrightarrow}\isanewline
\ \ \ \ \ \ \ {\isacharparenleft}F{\isadigit{1}}\ \isactrlbold {\isasymor}\ F{\isadigit{2}}\ {\isasymin}\ S\ {\isasymlongrightarrow}\ isModel\ {\isacharparenleft}setValuation\ S{\isacharparenright}\ {\isacharparenleft}F{\isadigit{1}}\ \isactrlbold {\isasymor}\ F{\isadigit{2}}{\isacharparenright}{\isacharparenright}\ {\isasymand}\isanewline
\ \ \ \ \ \ \ {\isacharparenleft}\isactrlbold {\isasymnot}\ {\isacharparenleft}F{\isadigit{1}}\ \isactrlbold {\isasymor}\ F{\isadigit{2}}{\isacharparenright}\ {\isasymin}\ S\ {\isasymlongrightarrow}\ {\isasymnot}\ isModel\ {\isacharparenleft}setValuation\ S{\isacharparenright}\ {\isacharparenleft}F{\isadigit{1}}\ \isactrlbold {\isasymor}\ F{\isadigit{2}}{\isacharparenright}{\isacharparenright}{\isachardoublequoteclose}\isanewline
%
\isadelimproof
\ \ %
\endisadelimproof
%
\isatagproof
\isacommand{by}\isamarkupfalse%
\ {\isacharparenleft}smt\ Hintikka{\isacharunderscore}def\ assms\ isModel{\isacharunderscore}def\ formula{\isacharunderscore}semantics{\isachardot}simps{\isacharparenleft}{\isadigit{5}}{\isacharparenright}{\isacharparenright}%
\endisatagproof
{\isafoldproof}%
%
\isadelimproof
\isanewline
%
\endisadelimproof
\isanewline
\isacommand{lemma}\isamarkupfalse%
\isanewline
\ \ \isakeyword{assumes}\ {\isachardoublequoteopen}Hintikka\ S{\isachardoublequoteclose}\isanewline
\ \ \ \ \ \ \ \ \ \ {\isachardoublequoteopen}{\isasymAnd}F{\isadigit{1}}{\isachardot}\ {\isacharparenleft}F{\isadigit{1}}\ {\isasymin}\ S\ {\isasymlongrightarrow}\ isModel\ {\isacharparenleft}setValuation\ S{\isacharparenright}\ F{\isadigit{1}}{\isacharparenright}\ {\isasymand}\isanewline
\ \ \ \ \ \ \ \ \ \ \ {\isacharparenleft}\isactrlbold {\isasymnot}\ F{\isadigit{1}}\ {\isasymin}\ S\ {\isasymlongrightarrow}\ {\isasymnot}\ isModel\ {\isacharparenleft}setValuation\ S{\isacharparenright}\ F{\isadigit{1}}{\isacharparenright}{\isachardoublequoteclose}\ \isanewline
\ \ \ \ \ \ \ \ \ \ {\isachardoublequoteopen}{\isasymAnd}F{\isadigit{2}}{\isachardot}\ {\isacharparenleft}F{\isadigit{2}}\ {\isasymin}\ S\ {\isasymlongrightarrow}\ isModel\ {\isacharparenleft}setValuation\ S{\isacharparenright}\ F{\isadigit{2}}{\isacharparenright}\ {\isasymand}\isanewline
\ \ \ \ \ \ \ \ \ \ \ {\isacharparenleft}\isactrlbold {\isasymnot}\ F{\isadigit{2}}\ {\isasymin}\ S\ {\isasymlongrightarrow}\ {\isasymnot}\ isModel\ {\isacharparenleft}setValuation\ S{\isacharparenright}\ F{\isadigit{2}}{\isacharparenright}{\isachardoublequoteclose}\isanewline
\ \ \isakeyword{shows}\ {\isachardoublequoteopen}{\isasymAnd}F{\isadigit{1}}\ F{\isadigit{2}}{\isachardot}\ {\isacharparenleft}F{\isadigit{1}}\ \isactrlbold {\isasymrightarrow}\ F{\isadigit{2}}\ {\isasymin}\ S\ {\isasymlongrightarrow}\ isModel\ {\isacharparenleft}setValuation\ S{\isacharparenright}\ {\isacharparenleft}F{\isadigit{1}}\ \isactrlbold {\isasymrightarrow}\ F{\isadigit{2}}{\isacharparenright}{\isacharparenright}\ \isanewline
\ \ \ \ \ \ {\isasymand}\ {\isacharparenleft}\isactrlbold {\isasymnot}\ {\isacharparenleft}F{\isadigit{1}}\ \isactrlbold {\isasymrightarrow}\ F{\isadigit{2}}{\isacharparenright}\ {\isasymin}\ S\ {\isasymlongrightarrow}\ {\isasymnot}\ isModel\ {\isacharparenleft}setValuation\ S{\isacharparenright}\ {\isacharparenleft}F{\isadigit{1}}\ \isactrlbold {\isasymrightarrow}\ F{\isadigit{2}}{\isacharparenright}{\isacharparenright}{\isachardoublequoteclose}\isanewline
%
\isadelimproof
%
\endisadelimproof
%
\isatagproof
\isacommand{proof}\isamarkupfalse%
\ {\isacharparenleft}rule\ conjI{\isacharparenright}\isanewline
\ \ \isacommand{show}\isamarkupfalse%
\ {\isachardoublequoteopen}{\isasymAnd}F{\isadigit{1}}\ F{\isadigit{2}}{\isachardot}\ F{\isadigit{1}}\ \isactrlbold {\isasymrightarrow}\ F{\isadigit{2}}\ {\isasymin}\ S\ {\isasymlongrightarrow}\ isModel\ {\isacharparenleft}setValuation\ S{\isacharparenright}\ {\isacharparenleft}F{\isadigit{1}}\ \isactrlbold {\isasymrightarrow}\ F{\isadigit{2}}{\isacharparenright}{\isachardoublequoteclose}\isanewline
\ \ \isacommand{proof}\isamarkupfalse%
\ {\isacharparenleft}rule\ impI{\isacharparenright}\isanewline
\ \ \ \ \isacommand{fix}\isamarkupfalse%
\ F{\isadigit{1}}\ F{\isadigit{2}}\isanewline
\ \ \ \ \isacommand{assume}\isamarkupfalse%
\ {\isachardoublequoteopen}F{\isadigit{1}}\ \isactrlbold {\isasymrightarrow}\ F{\isadigit{2}}\ {\isasymin}\ S{\isachardoublequoteclose}\isanewline
\ \ \ \ \isacommand{have}\isamarkupfalse%
\ {\isachardoublequoteopen}F{\isadigit{1}}\ \isactrlbold {\isasymrightarrow}\ F{\isadigit{2}}\ {\isasymin}\ S\ {\isasymlongrightarrow}\ \isactrlbold {\isasymnot}\ F{\isadigit{1}}\ {\isasymin}\ S\ {\isasymor}\ F{\isadigit{2}}\ {\isasymin}\ S{\isachardoublequoteclose}\isanewline
\ \ \ \ \ \ \isacommand{using}\isamarkupfalse%
\ assms{\isacharparenleft}{\isadigit{1}}{\isacharparenright}\ \isacommand{by}\isamarkupfalse%
\ {\isacharparenleft}rule\ Hintikka{\isacharunderscore}l{\isadigit{5}}{\isacharparenright}\isanewline
\ \ \ \ \isacommand{then}\isamarkupfalse%
\ \isacommand{have}\isamarkupfalse%
\ {\isachardoublequoteopen}\isactrlbold {\isasymnot}\ F{\isadigit{1}}\ {\isasymin}\ S\ {\isasymor}\ F{\isadigit{2}}\ {\isasymin}\ S{\isachardoublequoteclose}\isanewline
\ \ \ \ \ \ \isacommand{using}\isamarkupfalse%
\ {\isacartoucheopen}F{\isadigit{1}}\ \isactrlbold {\isasymrightarrow}\ F{\isadigit{2}}\ {\isasymin}\ S{\isacartoucheclose}\ \isacommand{by}\isamarkupfalse%
\ {\isacharparenleft}rule\ mp{\isacharparenright}\isanewline
\ \ \ \ \isacommand{then}\isamarkupfalse%
\ \isacommand{show}\isamarkupfalse%
\ {\isachardoublequoteopen}isModel\ {\isacharparenleft}setValuation\ S{\isacharparenright}\ {\isacharparenleft}F{\isadigit{1}}\ \isactrlbold {\isasymrightarrow}\ F{\isadigit{2}}{\isacharparenright}{\isachardoublequoteclose}\isanewline
\ \ \ \ \isacommand{proof}\isamarkupfalse%
\ {\isacharparenleft}rule\ disjE{\isacharparenright}\isanewline
\ \ \ \ \ \ \isacommand{assume}\isamarkupfalse%
\ {\isachardoublequoteopen}\isactrlbold {\isasymnot}\ F{\isadigit{1}}\ {\isasymin}\ S{\isachardoublequoteclose}\isanewline
\ \ \ \ \ \ \isacommand{have}\isamarkupfalse%
\ {\isachardoublequoteopen}\isactrlbold {\isasymnot}\ F{\isadigit{1}}\ {\isasymin}\ S\ {\isasymlongrightarrow}\ {\isasymnot}\ isModel\ {\isacharparenleft}setValuation\ S{\isacharparenright}\ F{\isadigit{1}}{\isachardoublequoteclose}\isanewline
\ \ \ \ \ \ \ \ \isacommand{using}\isamarkupfalse%
\ assms{\isacharparenleft}{\isadigit{2}}{\isacharparenright}\ \isacommand{by}\isamarkupfalse%
\ {\isacharparenleft}rule\ conjunct{\isadigit{2}}{\isacharparenright}\isanewline
\ \ \ \ \ \ \isacommand{then}\isamarkupfalse%
\ \isacommand{have}\isamarkupfalse%
\ {\isachardoublequoteopen}{\isasymnot}\ isModel\ {\isacharparenleft}setValuation\ S{\isacharparenright}\ F{\isadigit{1}}{\isachardoublequoteclose}\isanewline
\ \ \ \ \ \ \ \ \isacommand{using}\isamarkupfalse%
\ {\isacartoucheopen}\isactrlbold {\isasymnot}\ F{\isadigit{1}}\ {\isasymin}\ S{\isacartoucheclose}\ \isacommand{by}\isamarkupfalse%
\ {\isacharparenleft}rule\ mp{\isacharparenright}\isanewline
\ \ \ \ \ \ \isacommand{also}\isamarkupfalse%
\ \isacommand{have}\isamarkupfalse%
\ {\isachardoublequoteopen}{\isacharparenleft}{\isasymnot}\ isModel\ {\isacharparenleft}setValuation\ S{\isacharparenright}\ F{\isadigit{1}}{\isacharparenright}\ {\isacharequal}\ {\isacharparenleft}{\isasymnot}\ {\isacharparenleft}setValuation\ S{\isacharparenright}\ {\isasymTurnstile}\ F{\isadigit{1}}{\isacharparenright}{\isachardoublequoteclose}\isanewline
\ \ \ \ \ \ \ \ \isacommand{by}\isamarkupfalse%
\ {\isacharparenleft}simp\ only{\isacharcolon}\ isModel{\isacharunderscore}def{\isacharparenright}\isanewline
\ \ \ \ \ \ \isacommand{finally}\isamarkupfalse%
\ \isacommand{have}\isamarkupfalse%
\ {\isachardoublequoteopen}{\isasymnot}\ {\isacharparenleft}setValuation\ S{\isacharparenright}\ {\isasymTurnstile}\ F{\isadigit{1}}{\isachardoublequoteclose}\isanewline
\ \ \ \ \ \ \ \ \isacommand{by}\isamarkupfalse%
\ this\isanewline
\ \ \ \ \ \ \isacommand{have}\isamarkupfalse%
\ {\isachardoublequoteopen}{\isacharparenleft}setValuation\ S{\isacharparenright}\ {\isasymTurnstile}\ F{\isadigit{1}}\ {\isasymlongrightarrow}\ {\isacharparenleft}setValuation\ S{\isacharparenright}\ {\isasymTurnstile}\ F{\isadigit{2}}{\isachardoublequoteclose}\isanewline
\ \ \ \ \ \ \isacommand{proof}\isamarkupfalse%
\ {\isacharparenleft}rule\ impI{\isacharparenright}\isanewline
\ \ \ \ \ \ \ \ \isacommand{assume}\isamarkupfalse%
\ {\isachardoublequoteopen}{\isacharparenleft}setValuation\ S{\isacharparenright}\ {\isasymTurnstile}\ F{\isadigit{1}}{\isachardoublequoteclose}\isanewline
\ \ \ \ \ \ \ \ \isacommand{show}\isamarkupfalse%
\ {\isachardoublequoteopen}{\isacharparenleft}setValuation\ S{\isacharparenright}\ {\isasymTurnstile}\ F{\isadigit{2}}{\isachardoublequoteclose}\isanewline
\ \ \ \ \ \ \ \ \ \ \isacommand{using}\isamarkupfalse%
\ {\isacartoucheopen}{\isasymnot}\ {\isacharparenleft}setValuation\ S{\isacharparenright}\ {\isasymTurnstile}\ F{\isadigit{1}}{\isacartoucheclose}\ {\isacartoucheopen}{\isacharparenleft}setValuation\ S{\isacharparenright}\ {\isasymTurnstile}\ F{\isadigit{1}}{\isacartoucheclose}\ \isacommand{by}\isamarkupfalse%
\ {\isacharparenleft}rule\ notE{\isacharparenright}\ \isanewline
\ \ \ \ \ \ \isacommand{qed}\isamarkupfalse%
\isanewline
\ \ \ \ \ \ \isacommand{then}\isamarkupfalse%
\ \isacommand{have}\isamarkupfalse%
\ {\isachardoublequoteopen}{\isacharparenleft}setValuation\ S{\isacharparenright}\ {\isasymTurnstile}\ {\isacharparenleft}F{\isadigit{1}}\ \isactrlbold {\isasymrightarrow}\ F{\isadigit{2}}{\isacharparenright}{\isachardoublequoteclose}\isanewline
\ \ \ \ \ \ \ \ \isacommand{by}\isamarkupfalse%
\ {\isacharparenleft}simp\ only{\isacharcolon}\ formula{\isacharunderscore}semantics{\isachardot}simps{\isacharparenleft}{\isadigit{6}}{\isacharparenright}{\isacharparenright}\isanewline
\ \ \ \ \ \ \isacommand{thus}\isamarkupfalse%
\ {\isacharquery}thesis\isanewline
\ \ \ \ \ \ \ \ \isacommand{by}\isamarkupfalse%
\ {\isacharparenleft}simp\ only{\isacharcolon}\ isModel{\isacharunderscore}def{\isacharparenright}\isanewline
\ \ \ \ \isacommand{next}\isamarkupfalse%
\isanewline
\ \ \ \ \ \ \isacommand{assume}\isamarkupfalse%
\ {\isachardoublequoteopen}F{\isadigit{2}}\ {\isasymin}\ S{\isachardoublequoteclose}\isanewline
\ \ \ \ \ \ \isacommand{have}\isamarkupfalse%
\ {\isachardoublequoteopen}F{\isadigit{2}}\ {\isasymin}\ S\ {\isasymlongrightarrow}\ isModel\ {\isacharparenleft}setValuation\ S{\isacharparenright}\ F{\isadigit{2}}{\isachardoublequoteclose}\isanewline
\ \ \ \ \ \ \ \ \isacommand{using}\isamarkupfalse%
\ assms{\isacharparenleft}{\isadigit{3}}{\isacharparenright}\ \isacommand{by}\isamarkupfalse%
\ {\isacharparenleft}rule\ conjunct{\isadigit{1}}{\isacharparenright}\isanewline
\ \ \ \ \ \ \isacommand{then}\isamarkupfalse%
\ \isacommand{have}\isamarkupfalse%
\ {\isachardoublequoteopen}isModel\ {\isacharparenleft}setValuation\ S{\isacharparenright}\ F{\isadigit{2}}{\isachardoublequoteclose}\isanewline
\ \ \ \ \ \ \ \ \isacommand{using}\isamarkupfalse%
\ {\isacartoucheopen}F{\isadigit{2}}\ {\isasymin}\ S{\isacartoucheclose}\ \isacommand{by}\isamarkupfalse%
\ {\isacharparenleft}rule\ mp{\isacharparenright}\isanewline
\ \ \ \ \ \ \isacommand{then}\isamarkupfalse%
\ \isacommand{have}\isamarkupfalse%
\ {\isachardoublequoteopen}{\isacharparenleft}setValuation\ S{\isacharparenright}\ {\isasymTurnstile}\ F{\isadigit{2}}{\isachardoublequoteclose}\isanewline
\ \ \ \ \ \ \ \ \isacommand{by}\isamarkupfalse%
\ {\isacharparenleft}simp\ only{\isacharcolon}\ isModel{\isacharunderscore}def{\isacharparenright}\isanewline
\ \ \ \ \ \ \isacommand{have}\isamarkupfalse%
\ {\isachardoublequoteopen}{\isacharparenleft}setValuation\ S{\isacharparenright}\ {\isasymTurnstile}\ F{\isadigit{1}}\ {\isasymlongrightarrow}\ {\isacharparenleft}setValuation\ S{\isacharparenright}\ {\isasymTurnstile}\ F{\isadigit{2}}{\isachardoublequoteclose}\isanewline
\ \ \ \ \ \ \isacommand{proof}\isamarkupfalse%
\ {\isacharparenleft}rule\ impI{\isacharparenright}\isanewline
\ \ \ \ \ \ \ \ \isacommand{assume}\isamarkupfalse%
\ {\isachardoublequoteopen}{\isacharparenleft}setValuation\ S{\isacharparenright}\ {\isasymTurnstile}\ F{\isadigit{1}}{\isachardoublequoteclose}\isanewline
\ \ \ \ \ \ \ \ \isacommand{show}\isamarkupfalse%
\ {\isachardoublequoteopen}{\isacharparenleft}setValuation\ S{\isacharparenright}\ {\isasymTurnstile}\ F{\isadigit{2}}{\isachardoublequoteclose}\isanewline
\ \ \ \ \ \ \ \ \ \ \isacommand{using}\isamarkupfalse%
\ {\isacartoucheopen}{\isacharparenleft}setValuation\ S{\isacharparenright}\ {\isasymTurnstile}\ F{\isadigit{2}}{\isacartoucheclose}\ \isacommand{by}\isamarkupfalse%
\ this\ \isanewline
\ \ \ \ \ \ \isacommand{qed}\isamarkupfalse%
\isanewline
\ \ \ \ \ \ \isacommand{then}\isamarkupfalse%
\ \isacommand{have}\isamarkupfalse%
\ {\isachardoublequoteopen}{\isacharparenleft}setValuation\ S{\isacharparenright}\ {\isasymTurnstile}\ {\isacharparenleft}F{\isadigit{1}}\ \isactrlbold {\isasymrightarrow}\ F{\isadigit{2}}{\isacharparenright}{\isachardoublequoteclose}\isanewline
\ \ \ \ \ \ \ \ \isacommand{by}\isamarkupfalse%
\ {\isacharparenleft}simp\ only{\isacharcolon}\ formula{\isacharunderscore}semantics{\isachardot}simps{\isacharparenleft}{\isadigit{6}}{\isacharparenright}{\isacharparenright}\isanewline
\ \ \ \ \ \ \isacommand{thus}\isamarkupfalse%
\ {\isacharquery}thesis\isanewline
\ \ \ \ \ \ \ \ \isacommand{by}\isamarkupfalse%
\ {\isacharparenleft}simp\ only{\isacharcolon}\ isModel{\isacharunderscore}def{\isacharparenright}\isanewline
\ \ \ \ \isacommand{qed}\isamarkupfalse%
\isanewline
\ \ \isacommand{qed}\isamarkupfalse%
\isanewline
\isacommand{next}\isamarkupfalse%
\isanewline
\ \ \isacommand{show}\isamarkupfalse%
\ {\isachardoublequoteopen}{\isasymAnd}F{\isadigit{1}}\ F{\isadigit{2}}{\isachardot}\ \isactrlbold {\isasymnot}\ {\isacharparenleft}F{\isadigit{1}}\ \isactrlbold {\isasymrightarrow}\ F{\isadigit{2}}{\isacharparenright}\ {\isasymin}\ S\ {\isasymlongrightarrow}\ {\isasymnot}\ isModel\ {\isacharparenleft}setValuation\ S{\isacharparenright}\ {\isacharparenleft}F{\isadigit{1}}\ \isactrlbold {\isasymrightarrow}\ F{\isadigit{2}}{\isacharparenright}{\isachardoublequoteclose}\isanewline
\ \ \isacommand{proof}\isamarkupfalse%
\ {\isacharparenleft}rule\ impI{\isacharparenright}\isanewline
\ \ \ \ \isacommand{fix}\isamarkupfalse%
\ F{\isadigit{1}}\ F{\isadigit{2}}\isanewline
\ \ \ \ \isacommand{assume}\isamarkupfalse%
\ {\isachardoublequoteopen}\isactrlbold {\isasymnot}\ {\isacharparenleft}F{\isadigit{1}}\ \isactrlbold {\isasymrightarrow}\ F{\isadigit{2}}{\isacharparenright}\ {\isasymin}\ S{\isachardoublequoteclose}\isanewline
\ \ \ \ \isacommand{have}\isamarkupfalse%
\ {\isachardoublequoteopen}\isactrlbold {\isasymnot}\ {\isacharparenleft}F{\isadigit{1}}\ \isactrlbold {\isasymrightarrow}\ F{\isadigit{2}}{\isacharparenright}\ {\isasymin}\ S\ {\isasymlongrightarrow}\ F{\isadigit{1}}\ {\isasymin}\ S\ {\isasymand}\ \isactrlbold {\isasymnot}\ F{\isadigit{2}}\ {\isasymin}\ S{\isachardoublequoteclose}\isanewline
\ \ \ \ \ \ \isacommand{using}\isamarkupfalse%
\ assms{\isacharparenleft}{\isadigit{1}}{\isacharparenright}\ \isacommand{by}\isamarkupfalse%
\ {\isacharparenleft}rule\ Hintikka{\isacharunderscore}l{\isadigit{9}}{\isacharparenright}\isanewline
\ \ \ \ \isacommand{then}\isamarkupfalse%
\ \isacommand{have}\isamarkupfalse%
\ C{\isacharcolon}{\isachardoublequoteopen}F{\isadigit{1}}\ {\isasymin}\ S\ {\isasymand}\ \isactrlbold {\isasymnot}\ F{\isadigit{2}}\ {\isasymin}\ S{\isachardoublequoteclose}\isanewline
\ \ \ \ \ \ \isacommand{using}\isamarkupfalse%
\ {\isacartoucheopen}\isactrlbold {\isasymnot}\ {\isacharparenleft}F{\isadigit{1}}\ \isactrlbold {\isasymrightarrow}\ F{\isadigit{2}}{\isacharparenright}\ {\isasymin}\ S{\isacartoucheclose}\ \isacommand{by}\isamarkupfalse%
\ {\isacharparenleft}rule\ mp{\isacharparenright}\isanewline
\ \ \ \ \isacommand{then}\isamarkupfalse%
\ \isacommand{have}\isamarkupfalse%
\ {\isachardoublequoteopen}F{\isadigit{1}}\ {\isasymin}\ S{\isachardoublequoteclose}\isanewline
\ \ \ \ \ \ \isacommand{by}\isamarkupfalse%
\ {\isacharparenleft}rule\ conjunct{\isadigit{1}}{\isacharparenright}\isanewline
\ \ \ \ \isacommand{have}\isamarkupfalse%
\ {\isachardoublequoteopen}F{\isadigit{1}}\ {\isasymin}\ S\ {\isasymlongrightarrow}\ isModel\ {\isacharparenleft}setValuation\ S{\isacharparenright}\ F{\isadigit{1}}{\isachardoublequoteclose}\isanewline
\ \ \ \ \ \ \isacommand{using}\isamarkupfalse%
\ assms{\isacharparenleft}{\isadigit{2}}{\isacharparenright}\ \isacommand{by}\isamarkupfalse%
\ {\isacharparenleft}rule\ conjunct{\isadigit{1}}{\isacharparenright}\isanewline
\ \ \ \ \isacommand{then}\isamarkupfalse%
\ \isacommand{have}\isamarkupfalse%
\ {\isachardoublequoteopen}isModel\ {\isacharparenleft}setValuation\ S{\isacharparenright}\ F{\isadigit{1}}{\isachardoublequoteclose}\isanewline
\ \ \ \ \ \ \isacommand{using}\isamarkupfalse%
\ {\isacartoucheopen}F{\isadigit{1}}\ {\isasymin}\ S{\isacartoucheclose}\ \isacommand{by}\isamarkupfalse%
\ {\isacharparenleft}rule\ mp{\isacharparenright}\isanewline
\ \ \ \ \isacommand{then}\isamarkupfalse%
\ \isacommand{have}\isamarkupfalse%
\ {\isachardoublequoteopen}{\isacharparenleft}setValuation\ S{\isacharparenright}\ {\isasymTurnstile}\ F{\isadigit{1}}{\isachardoublequoteclose}\isanewline
\ \ \ \ \ \ \isacommand{by}\isamarkupfalse%
\ {\isacharparenleft}simp\ only{\isacharcolon}\ isModel{\isacharunderscore}def{\isacharparenright}\isanewline
\ \ \ \ \isacommand{have}\isamarkupfalse%
\ {\isachardoublequoteopen}\isactrlbold {\isasymnot}\ F{\isadigit{2}}\ {\isasymin}\ S{\isachardoublequoteclose}\isanewline
\ \ \ \ \ \ \isacommand{using}\isamarkupfalse%
\ C\ \isacommand{by}\isamarkupfalse%
\ {\isacharparenleft}rule\ conjunct{\isadigit{2}}{\isacharparenright}\isanewline
\ \ \ \ \isacommand{have}\isamarkupfalse%
\ {\isachardoublequoteopen}\isactrlbold {\isasymnot}\ F{\isadigit{2}}\ {\isasymin}\ S\ {\isasymlongrightarrow}\ {\isasymnot}\ isModel\ {\isacharparenleft}setValuation\ S{\isacharparenright}\ F{\isadigit{2}}{\isachardoublequoteclose}\isanewline
\ \ \ \ \ \ \isacommand{using}\isamarkupfalse%
\ assms{\isacharparenleft}{\isadigit{3}}{\isacharparenright}\ \isacommand{by}\isamarkupfalse%
\ {\isacharparenleft}rule\ conjunct{\isadigit{2}}{\isacharparenright}\isanewline
\ \ \ \ \isacommand{then}\isamarkupfalse%
\ \isacommand{have}\isamarkupfalse%
\ {\isachardoublequoteopen}{\isasymnot}\ isModel\ {\isacharparenleft}setValuation\ S{\isacharparenright}\ F{\isadigit{2}}{\isachardoublequoteclose}\isanewline
\ \ \ \ \ \ \isacommand{using}\isamarkupfalse%
\ {\isacartoucheopen}\isactrlbold {\isasymnot}\ F{\isadigit{2}}\ {\isasymin}\ S{\isacartoucheclose}\ \isacommand{by}\isamarkupfalse%
\ {\isacharparenleft}rule\ mp{\isacharparenright}\isanewline
\ \ \ \ \isacommand{also}\isamarkupfalse%
\ \isacommand{have}\isamarkupfalse%
\ {\isachardoublequoteopen}{\isacharparenleft}{\isasymnot}\ isModel\ {\isacharparenleft}setValuation\ S{\isacharparenright}\ F{\isadigit{2}}{\isacharparenright}\ {\isacharequal}\ {\isacharparenleft}{\isasymnot}\ {\isacharparenleft}setValuation\ S{\isacharparenright}\ {\isasymTurnstile}\ F{\isadigit{2}}{\isacharparenright}{\isachardoublequoteclose}\isanewline
\ \ \ \ \ \ \isacommand{by}\isamarkupfalse%
\ {\isacharparenleft}simp\ only{\isacharcolon}\ isModel{\isacharunderscore}def{\isacharparenright}\isanewline
\ \ \ \ \isacommand{finally}\isamarkupfalse%
\ \isacommand{have}\isamarkupfalse%
\ {\isachardoublequoteopen}{\isasymnot}\ {\isacharparenleft}setValuation\ S{\isacharparenright}\ {\isasymTurnstile}\ F{\isadigit{2}}{\isachardoublequoteclose}\isanewline
\ \ \ \ \ \ \isacommand{by}\isamarkupfalse%
\ this\isanewline
\ \ \ \ \isacommand{have}\isamarkupfalse%
\ {\isachardoublequoteopen}{\isasymnot}\ {\isacharparenleft}{\isacharparenleft}setValuation\ S{\isacharparenright}\ {\isasymTurnstile}\ F{\isadigit{1}}\ {\isasymlongrightarrow}\ {\isacharparenleft}setValuation\ S{\isacharparenright}\ {\isasymTurnstile}\ F{\isadigit{2}}{\isacharparenright}{\isachardoublequoteclose}\isanewline
\ \ \ \ \isacommand{proof}\isamarkupfalse%
\ {\isacharparenleft}rule\ notI{\isacharparenright}\isanewline
\ \ \ \ \ \ \isacommand{assume}\isamarkupfalse%
\ {\isachardoublequoteopen}{\isacharparenleft}setValuation\ S{\isacharparenright}\ {\isasymTurnstile}\ F{\isadigit{1}}\ {\isasymlongrightarrow}\ {\isacharparenleft}setValuation\ S{\isacharparenright}\ {\isasymTurnstile}\ F{\isadigit{2}}{\isachardoublequoteclose}\isanewline
\ \ \ \ \ \ \isacommand{then}\isamarkupfalse%
\ \isacommand{have}\isamarkupfalse%
\ {\isachardoublequoteopen}{\isacharparenleft}setValuation\ S{\isacharparenright}\ {\isasymTurnstile}\ F{\isadigit{2}}{\isachardoublequoteclose}\isanewline
\ \ \ \ \ \ \ \ \isacommand{using}\isamarkupfalse%
\ {\isacartoucheopen}{\isacharparenleft}setValuation\ S{\isacharparenright}\ {\isasymTurnstile}\ F{\isadigit{1}}{\isacartoucheclose}\ \isacommand{by}\isamarkupfalse%
\ {\isacharparenleft}rule\ mp{\isacharparenright}\isanewline
\ \ \ \ \ \ \isacommand{show}\isamarkupfalse%
\ {\isachardoublequoteopen}False{\isachardoublequoteclose}\ \isanewline
\ \ \ \ \ \ \ \ \isacommand{using}\isamarkupfalse%
\ {\isacartoucheopen}{\isasymnot}\ {\isacharparenleft}setValuation\ S{\isacharparenright}\ {\isasymTurnstile}\ F{\isadigit{2}}{\isacartoucheclose}\ {\isacartoucheopen}{\isacharparenleft}setValuation\ S{\isacharparenright}\ {\isasymTurnstile}\ F{\isadigit{2}}{\isacartoucheclose}\ \isacommand{by}\isamarkupfalse%
\ {\isacharparenleft}rule\ notE{\isacharparenright}\isanewline
\ \ \ \ \isacommand{qed}\isamarkupfalse%
\isanewline
\ \ \ \ \isacommand{also}\isamarkupfalse%
\ \isacommand{have}\isamarkupfalse%
\ {\isachardoublequoteopen}{\isacharparenleft}{\isasymnot}\ {\isacharparenleft}{\isacharparenleft}setValuation\ S{\isacharparenright}\ {\isasymTurnstile}\ F{\isadigit{1}}\ {\isasymlongrightarrow}\ {\isacharparenleft}setValuation\ S{\isacharparenright}\ {\isasymTurnstile}\ F{\isadigit{2}}{\isacharparenright}{\isacharparenright}\ {\isacharequal}\ \isanewline
\ \ \ \ \ \ \ \ {\isacharparenleft}{\isasymnot}\ {\isacharparenleft}setValuation\ S{\isacharparenright}\ {\isasymTurnstile}\ {\isacharparenleft}F{\isadigit{1}}\ \isactrlbold {\isasymrightarrow}\ F{\isadigit{2}}{\isacharparenright}{\isacharparenright}{\isachardoublequoteclose}\isanewline
\ \ \ \ \ \ \isacommand{by}\isamarkupfalse%
\ {\isacharparenleft}simp\ only{\isacharcolon}\ formula{\isacharunderscore}semantics{\isachardot}simps{\isacharparenleft}{\isadigit{6}}{\isacharparenright}{\isacharparenright}\isanewline
\ \ \ \ \isacommand{also}\isamarkupfalse%
\ \isacommand{have}\isamarkupfalse%
\ {\isachardoublequoteopen}{\isasymdots}\ {\isacharequal}\ {\isacharparenleft}{\isasymnot}\ isModel\ {\isacharparenleft}setValuation\ S{\isacharparenright}\ {\isacharparenleft}F{\isadigit{1}}\ \isactrlbold {\isasymrightarrow}\ F{\isadigit{2}}{\isacharparenright}{\isacharparenright}{\isachardoublequoteclose}\isanewline
\ \ \ \ \ \ \isacommand{by}\isamarkupfalse%
\ {\isacharparenleft}simp\ only{\isacharcolon}\ isModel{\isacharunderscore}def{\isacharparenright}\isanewline
\ \ \ \ \isacommand{finally}\isamarkupfalse%
\ \isacommand{show}\isamarkupfalse%
\ {\isachardoublequoteopen}{\isasymnot}\ isModel\ {\isacharparenleft}setValuation\ S{\isacharparenright}\ {\isacharparenleft}F{\isadigit{1}}\ \isactrlbold {\isasymrightarrow}\ F{\isadigit{2}}{\isacharparenright}{\isachardoublequoteclose}\isanewline
\ \ \ \ \ \ \isacommand{by}\isamarkupfalse%
\ this\isanewline
\ \ \isacommand{qed}\isamarkupfalse%
\isanewline
\isacommand{qed}\isamarkupfalse%
%
\endisatagproof
{\isafoldproof}%
%
\isadelimproof
\isanewline
%
\endisadelimproof
\isanewline
\isacommand{lemma}\isamarkupfalse%
\ Hl{\isadigit{2}}{\isacharunderscore}{\isadigit{6}}{\isacharcolon}\isanewline
\ \ \isakeyword{assumes}\ {\isachardoublequoteopen}Hintikka\ S{\isachardoublequoteclose}\isanewline
\ \ \isakeyword{shows}\ {\isachardoublequoteopen}\ {\isasymAnd}F{\isadigit{1}}\ F{\isadigit{2}}{\isachardot}\isanewline
\ \ \ \ \ \ \ {\isacharparenleft}F{\isadigit{1}}\ {\isasymin}\ S\ {\isasymlongrightarrow}\ isModel\ {\isacharparenleft}setValuation\ S{\isacharparenright}\ F{\isadigit{1}}{\isacharparenright}\ {\isasymand}\isanewline
\ \ \ \ \ \ \ {\isacharparenleft}\isactrlbold {\isasymnot}\ F{\isadigit{1}}\ {\isasymin}\ S\ {\isasymlongrightarrow}\ {\isasymnot}\ isModel\ {\isacharparenleft}setValuation\ S{\isacharparenright}\ F{\isadigit{1}}{\isacharparenright}\ {\isasymLongrightarrow}\isanewline
\ \ \ \ \ \ \ {\isacharparenleft}F{\isadigit{2}}\ {\isasymin}\ S\ {\isasymlongrightarrow}\ isModel\ {\isacharparenleft}setValuation\ S{\isacharparenright}\ F{\isadigit{2}}{\isacharparenright}\ {\isasymand}\isanewline
\ \ \ \ \ \ \ {\isacharparenleft}\isactrlbold {\isasymnot}\ F{\isadigit{2}}\ {\isasymin}\ S\ {\isasymlongrightarrow}\ {\isasymnot}\ isModel\ {\isacharparenleft}setValuation\ S{\isacharparenright}\ F{\isadigit{2}}{\isacharparenright}\ {\isasymLongrightarrow}\isanewline
\ \ \ \ \ \ \ {\isacharparenleft}F{\isadigit{1}}\ \isactrlbold {\isasymrightarrow}\ F{\isadigit{2}}\ {\isasymin}\ S\ {\isasymlongrightarrow}\ isModel\ {\isacharparenleft}setValuation\ S{\isacharparenright}\ {\isacharparenleft}F{\isadigit{1}}\ \isactrlbold {\isasymrightarrow}\ F{\isadigit{2}}{\isacharparenright}{\isacharparenright}\ {\isasymand}\isanewline
\ \ \ \ \ \ \ {\isacharparenleft}\isactrlbold {\isasymnot}\ {\isacharparenleft}F{\isadigit{1}}\ \isactrlbold {\isasymrightarrow}\ F{\isadigit{2}}{\isacharparenright}\ {\isasymin}\ S\ {\isasymlongrightarrow}\ {\isasymnot}\ isModel\ {\isacharparenleft}setValuation\ S{\isacharparenright}\ {\isacharparenleft}F{\isadigit{1}}\ \isactrlbold {\isasymrightarrow}\ F{\isadigit{2}}{\isacharparenright}{\isacharparenright}{\isachardoublequoteclose}\isanewline
%
\isadelimproof
\ \ %
\endisadelimproof
%
\isatagproof
\isacommand{by}\isamarkupfalse%
\ {\isacharparenleft}meson\ Hintikka{\isacharunderscore}l{\isadigit{5}}\ Hintikka{\isacharunderscore}l{\isadigit{9}}\ assms\ isModel{\isacharunderscore}def\ formula{\isacharunderscore}semantics{\isachardot}simps{\isacharparenleft}{\isadigit{6}}{\isacharparenright}{\isacharparenright}%
\endisatagproof
{\isafoldproof}%
%
\isadelimproof
\isanewline
%
\endisadelimproof
\isanewline
\isacommand{lemma}\isamarkupfalse%
\ Hintikkas{\isacharunderscore}lemma{\isacharunderscore}l{\isadigit{2}}{\isacharcolon}\isanewline
\ \ \isakeyword{assumes}\ {\isachardoublequoteopen}Hintikka\ S{\isachardoublequoteclose}\isanewline
\ \ \isakeyword{shows}\ {\isachardoublequoteopen}{\isacharparenleft}F\ {\isasymin}\ S\ {\isasymlongrightarrow}\ isModel\ {\isacharparenleft}setValuation\ S{\isacharparenright}\ F{\isacharparenright}\isanewline
\ \ \ \ \ \ \ \ \ \ \ {\isasymand}\ {\isacharparenleft}\isactrlbold {\isasymnot}\ F\ {\isasymin}\ S\ {\isasymlongrightarrow}\ {\isacharparenleft}{\isasymnot}{\isacharparenleft}isModel\ {\isacharparenleft}setValuation\ S{\isacharparenright}\ F{\isacharparenright}{\isacharparenright}{\isacharparenright}{\isachardoublequoteclose}\isanewline
%
\isadelimproof
%
\endisadelimproof
%
\isatagproof
\isacommand{proof}\isamarkupfalse%
\ {\isacharparenleft}induct\ F{\isacharparenright}\isanewline
\ \ \isacommand{fix}\isamarkupfalse%
\ x\isanewline
\ \ \isacommand{show}\isamarkupfalse%
\ {\isachardoublequoteopen}{\isacharparenleft}Atom\ x\ {\isasymin}\ S\ {\isasymlongrightarrow}\ isModel\ {\isacharparenleft}setValuation\ S{\isacharparenright}\ {\isacharparenleft}Atom\ x{\isacharparenright}{\isacharparenright}\ {\isasymand}\isanewline
\ \ \ \ \ \ \ \ \ {\isacharparenleft}\isactrlbold {\isasymnot}\ {\isacharparenleft}Atom\ x{\isacharparenright}\ {\isasymin}\ S\ {\isasymlongrightarrow}\ {\isasymnot}\ isModel\ {\isacharparenleft}setValuation\ S{\isacharparenright}\ {\isacharparenleft}Atom\ x{\isacharparenright}{\isacharparenright}{\isachardoublequoteclose}\isanewline
\ \ \ \ \isacommand{using}\isamarkupfalse%
\ assms\ \isacommand{by}\isamarkupfalse%
\ {\isacharparenleft}rule\ Hl{\isadigit{2}}{\isacharunderscore}{\isadigit{1}}{\isacharparenright}\isanewline
\isacommand{next}\isamarkupfalse%
\isanewline
\ \ \isacommand{show}\isamarkupfalse%
\ {\isachardoublequoteopen}{\isacharparenleft}{\isasymbottom}\ {\isasymin}\ S\ {\isasymlongrightarrow}\ isModel\ {\isacharparenleft}setValuation\ S{\isacharparenright}\ {\isasymbottom}{\isacharparenright}\ {\isasymand}\isanewline
\ \ \ \ {\isacharparenleft}\isactrlbold {\isasymnot}\ {\isasymbottom}\ {\isasymin}\ S\ {\isasymlongrightarrow}\ {\isasymnot}\ isModel\ {\isacharparenleft}setValuation\ S{\isacharparenright}\ {\isasymbottom}{\isacharparenright}{\isachardoublequoteclose}\ \isanewline
\ \ \ \ \isacommand{using}\isamarkupfalse%
\ assms\ \isacommand{by}\isamarkupfalse%
\ {\isacharparenleft}rule\ Hl{\isadigit{2}}{\isacharunderscore}{\isadigit{2}}{\isacharparenright}\isanewline
\isacommand{next}\isamarkupfalse%
\isanewline
\ \ \isacommand{fix}\isamarkupfalse%
\ F\isanewline
\ \ \isacommand{show}\isamarkupfalse%
\ {\isachardoublequoteopen}{\isacharparenleft}F\ {\isasymin}\ S\ {\isasymlongrightarrow}\ isModel\ {\isacharparenleft}setValuation\ S{\isacharparenright}\ F{\isacharparenright}\ {\isasymand}\isanewline
\ \ \ \ \ \ \ \ \ {\isacharparenleft}\isactrlbold {\isasymnot}\ F\ {\isasymin}\ S\ {\isasymlongrightarrow}\ {\isasymnot}\ isModel\ {\isacharparenleft}setValuation\ S{\isacharparenright}\ F{\isacharparenright}\ {\isasymLongrightarrow}\isanewline
\ \ \ \ \ \ \ \ \ {\isacharparenleft}\isactrlbold {\isasymnot}\ F\ {\isasymin}\ S\ {\isasymlongrightarrow}\ isModel\ {\isacharparenleft}setValuation\ S{\isacharparenright}\ {\isacharparenleft}\isactrlbold {\isasymnot}\ F{\isacharparenright}{\isacharparenright}\ {\isasymand}\isanewline
\ \ \ \ \ \ \ \ \ {\isacharparenleft}\isactrlbold {\isasymnot}\ {\isacharparenleft}\isactrlbold {\isasymnot}\ F{\isacharparenright}\ {\isasymin}\ S\ {\isasymlongrightarrow}\ {\isasymnot}\ isModel\ {\isacharparenleft}setValuation\ S{\isacharparenright}\ {\isacharparenleft}\isactrlbold {\isasymnot}\ F{\isacharparenright}{\isacharparenright}{\isachardoublequoteclose}\isanewline
\ \ \ \ \isacommand{using}\isamarkupfalse%
\ assms\ \isacommand{by}\isamarkupfalse%
\ {\isacharparenleft}rule\ Hl{\isadigit{2}}{\isacharunderscore}{\isadigit{3}}{\isacharparenright}\isanewline
\isacommand{next}\isamarkupfalse%
\isanewline
\ \ \isacommand{fix}\isamarkupfalse%
\ F{\isadigit{1}}\ F{\isadigit{2}}\isanewline
\ \ \isacommand{show}\isamarkupfalse%
\ {\isachardoublequoteopen}{\isacharparenleft}F{\isadigit{1}}\ {\isasymin}\ S\ {\isasymlongrightarrow}\ isModel\ {\isacharparenleft}setValuation\ S{\isacharparenright}\ F{\isadigit{1}}{\isacharparenright}\ {\isasymand}\isanewline
\ \ \ \ \ \ \ {\isacharparenleft}\isactrlbold {\isasymnot}\ F{\isadigit{1}}\ {\isasymin}\ S\ {\isasymlongrightarrow}\ {\isasymnot}\ isModel\ {\isacharparenleft}setValuation\ S{\isacharparenright}\ F{\isadigit{1}}{\isacharparenright}\ {\isasymLongrightarrow}\isanewline
\ \ \ \ \ \ \ {\isacharparenleft}F{\isadigit{2}}\ {\isasymin}\ S\ {\isasymlongrightarrow}\ isModel\ {\isacharparenleft}setValuation\ S{\isacharparenright}\ F{\isadigit{2}}{\isacharparenright}\ {\isasymand}\isanewline
\ \ \ \ \ \ \ {\isacharparenleft}\isactrlbold {\isasymnot}\ F{\isadigit{2}}\ {\isasymin}\ S\ {\isasymlongrightarrow}\ {\isasymnot}\ isModel\ {\isacharparenleft}setValuation\ S{\isacharparenright}\ F{\isadigit{2}}{\isacharparenright}\ {\isasymLongrightarrow}\isanewline
\ \ \ \ \ \ \ {\isacharparenleft}F{\isadigit{1}}\ \isactrlbold {\isasymand}\ F{\isadigit{2}}\ {\isasymin}\ S\ {\isasymlongrightarrow}\ isModel\ {\isacharparenleft}setValuation\ S{\isacharparenright}\ {\isacharparenleft}F{\isadigit{1}}\ \isactrlbold {\isasymand}\ F{\isadigit{2}}{\isacharparenright}{\isacharparenright}\ {\isasymand}\isanewline
\ \ \ \ \ \ \ {\isacharparenleft}\isactrlbold {\isasymnot}\ {\isacharparenleft}F{\isadigit{1}}\ \isactrlbold {\isasymand}\ F{\isadigit{2}}{\isacharparenright}\ {\isasymin}\ S\ {\isasymlongrightarrow}\ {\isasymnot}\ isModel\ {\isacharparenleft}setValuation\ S{\isacharparenright}\ {\isacharparenleft}F{\isadigit{1}}\ \isactrlbold {\isasymand}\ F{\isadigit{2}}{\isacharparenright}{\isacharparenright}{\isachardoublequoteclose}\isanewline
\ \ \ \ \isacommand{using}\isamarkupfalse%
\ assms\ \isacommand{by}\isamarkupfalse%
\ {\isacharparenleft}rule\ Hl{\isadigit{2}}{\isacharunderscore}{\isadigit{4}}{\isacharparenright}\isanewline
\isacommand{next}\isamarkupfalse%
\isanewline
\ \ \isacommand{fix}\isamarkupfalse%
\ F{\isadigit{1}}\ F{\isadigit{2}}\isanewline
\ \ \isacommand{show}\isamarkupfalse%
\ {\isachardoublequoteopen}{\isacharparenleft}F{\isadigit{1}}\ {\isasymin}\ S\ {\isasymlongrightarrow}\ isModel\ {\isacharparenleft}setValuation\ S{\isacharparenright}\ F{\isadigit{1}}{\isacharparenright}\ {\isasymand}\isanewline
\ \ \ \ \ \ \ {\isacharparenleft}\isactrlbold {\isasymnot}\ F{\isadigit{1}}\ {\isasymin}\ S\ {\isasymlongrightarrow}\ {\isasymnot}\ isModel\ {\isacharparenleft}setValuation\ S{\isacharparenright}\ F{\isadigit{1}}{\isacharparenright}\ {\isasymLongrightarrow}\isanewline
\ \ \ \ \ \ \ {\isacharparenleft}F{\isadigit{2}}\ {\isasymin}\ S\ {\isasymlongrightarrow}\ isModel\ {\isacharparenleft}setValuation\ S{\isacharparenright}\ F{\isadigit{2}}{\isacharparenright}\ {\isasymand}\isanewline
\ \ \ \ \ \ \ {\isacharparenleft}\isactrlbold {\isasymnot}\ F{\isadigit{2}}\ {\isasymin}\ S\ {\isasymlongrightarrow}\ {\isasymnot}\ isModel\ {\isacharparenleft}setValuation\ S{\isacharparenright}\ F{\isadigit{2}}{\isacharparenright}\ {\isasymLongrightarrow}\isanewline
\ \ \ \ \ \ \ {\isacharparenleft}F{\isadigit{1}}\ \isactrlbold {\isasymor}\ F{\isadigit{2}}\ {\isasymin}\ S\ {\isasymlongrightarrow}\ isModel\ {\isacharparenleft}setValuation\ S{\isacharparenright}\ {\isacharparenleft}F{\isadigit{1}}\ \isactrlbold {\isasymor}\ F{\isadigit{2}}{\isacharparenright}{\isacharparenright}\ {\isasymand}\isanewline
\ \ \ \ \ \ \ {\isacharparenleft}\isactrlbold {\isasymnot}\ {\isacharparenleft}F{\isadigit{1}}\ \isactrlbold {\isasymor}\ F{\isadigit{2}}{\isacharparenright}\ {\isasymin}\ S\ {\isasymlongrightarrow}\ {\isasymnot}\ isModel\ {\isacharparenleft}setValuation\ S{\isacharparenright}\ {\isacharparenleft}F{\isadigit{1}}\ \isactrlbold {\isasymor}\ F{\isadigit{2}}{\isacharparenright}{\isacharparenright}{\isachardoublequoteclose}\isanewline
\ \ \ \ \isacommand{using}\isamarkupfalse%
\ assms\ \isacommand{by}\isamarkupfalse%
\ {\isacharparenleft}rule\ Hl{\isadigit{2}}{\isacharunderscore}{\isadigit{5}}{\isacharparenright}\isanewline
\isacommand{next}\isamarkupfalse%
\isanewline
\ \ \isacommand{fix}\isamarkupfalse%
\ F{\isadigit{1}}\ F{\isadigit{2}}\isanewline
\ \ \isacommand{show}\isamarkupfalse%
\ {\isachardoublequoteopen}{\isacharparenleft}F{\isadigit{1}}\ {\isasymin}\ S\ {\isasymlongrightarrow}\ isModel\ {\isacharparenleft}setValuation\ S{\isacharparenright}\ F{\isadigit{1}}{\isacharparenright}\ {\isasymand}\isanewline
\ \ \ \ \ \ \ {\isacharparenleft}\isactrlbold {\isasymnot}\ F{\isadigit{1}}\ {\isasymin}\ S\ {\isasymlongrightarrow}\ {\isasymnot}\ isModel\ {\isacharparenleft}setValuation\ S{\isacharparenright}\ F{\isadigit{1}}{\isacharparenright}\ {\isasymLongrightarrow}\isanewline
\ \ \ \ \ \ \ {\isacharparenleft}F{\isadigit{2}}\ {\isasymin}\ S\ {\isasymlongrightarrow}\ isModel\ {\isacharparenleft}setValuation\ S{\isacharparenright}\ F{\isadigit{2}}{\isacharparenright}\ {\isasymand}\isanewline
\ \ \ \ \ \ \ {\isacharparenleft}\isactrlbold {\isasymnot}\ F{\isadigit{2}}\ {\isasymin}\ S\ {\isasymlongrightarrow}\ {\isasymnot}\ isModel\ {\isacharparenleft}setValuation\ S{\isacharparenright}\ F{\isadigit{2}}{\isacharparenright}\ {\isasymLongrightarrow}\isanewline
\ \ \ \ \ \ \ {\isacharparenleft}F{\isadigit{1}}\ \isactrlbold {\isasymrightarrow}\ F{\isadigit{2}}\ {\isasymin}\ S\ {\isasymlongrightarrow}\ isModel\ {\isacharparenleft}setValuation\ S{\isacharparenright}\ {\isacharparenleft}F{\isadigit{1}}\ \isactrlbold {\isasymrightarrow}\ F{\isadigit{2}}{\isacharparenright}{\isacharparenright}\ {\isasymand}\isanewline
\ \ \ \ \ \ \ {\isacharparenleft}\isactrlbold {\isasymnot}\ {\isacharparenleft}F{\isadigit{1}}\ \isactrlbold {\isasymrightarrow}\ F{\isadigit{2}}{\isacharparenright}\ {\isasymin}\ S\ {\isasymlongrightarrow}\ {\isasymnot}\ isModel\ {\isacharparenleft}setValuation\ S{\isacharparenright}\ {\isacharparenleft}F{\isadigit{1}}\ \isactrlbold {\isasymrightarrow}\ F{\isadigit{2}}{\isacharparenright}{\isacharparenright}{\isachardoublequoteclose}\isanewline
\ \ \ \ \isacommand{using}\isamarkupfalse%
\ assms\ \isacommand{by}\isamarkupfalse%
\ {\isacharparenleft}rule\ Hl{\isadigit{2}}{\isacharunderscore}{\isadigit{6}}{\isacharparenright}\isanewline
\isacommand{qed}\isamarkupfalse%
%
\endisatagproof
{\isafoldproof}%
%
\isadelimproof
%
\endisadelimproof
%
\begin{isamarkuptext}%
Para concluir, demostremos el \isa{Lema\ de\ Hintikka} empleando el
  resultado anterior.

  \begin{teorema}[Lema de Hintikka]
    Todo conjunto de Hintikka es satisfacible.
  \end{teorema}

  \begin{demostracion}
    Consideremos un conjunto de fórmulas \isa{S} tal que sea un conjunto de 
    Hintikka. Queremos demostrar que \isa{S} es satisfacible, es decir, que 
    tiene algún modelo. En otras palabras, debemos hallar una 
    interpretación que sea modelo de \isa{S}.

    En primer lugar, probemos que la interpretación asociada a \isa{S} es 
    modelo de \isa{S}. Por definición de modelo de un conjunto, basta 
    comprobar que es modelo de toda fórmula perteneciente al mismo. 
    Fijada una fórmula cualquiera, hemos visto anteriormente que la 
    interpretación asociada a \isa{S} es modelo de la fórmula si esta 
    pertenece al conjunto. Por tanto, dicha interpretación es, en 
    efecto, modelo de todas las fórmulas que pertenecen a \isa{S}. Luego la 
    interpretación asociada a \isa{S} es modelo de \isa{S}.

    En conclusión, hemos hallado una interpretación que es modelo del
    conjunto. Por lo tanto, \isa{S} es satisfacible, como se quería probar.
  \end{demostracion}

  Por su parte, la prueba detallada en Isabelle emplea el lema
  auxiliar \isa{Hintikka{\isacharunderscore}model}. Con él se demuestra la primera parte del 
  lema de Hintikka: dado un conjunto de Hintikka, la interpretación 
  asociada al conjunto es modelo del mismo.%
\end{isamarkuptext}\isamarkuptrue%
\isacommand{lemma}\isamarkupfalse%
\ Hintikka{\isacharunderscore}model{\isacharcolon}\isanewline
\ \ \isakeyword{assumes}\ {\isachardoublequoteopen}Hintikka\ S{\isachardoublequoteclose}\isanewline
\ \ \isakeyword{shows}\ {\isachardoublequoteopen}isModelSet\ {\isacharparenleft}setValuation\ S{\isacharparenright}\ S{\isachardoublequoteclose}\isanewline
%
\isadelimproof
%
\endisadelimproof
%
\isatagproof
\isacommand{proof}\isamarkupfalse%
\ {\isacharminus}\isanewline
\ \ \isacommand{have}\isamarkupfalse%
\ {\isachardoublequoteopen}{\isasymforall}F{\isachardot}\ {\isacharparenleft}F\ {\isasymin}\ S\ {\isasymlongrightarrow}\ isModel\ {\isacharparenleft}setValuation\ S{\isacharparenright}\ F{\isacharparenright}{\isachardoublequoteclose}\isanewline
\ \ \isacommand{proof}\isamarkupfalse%
\ {\isacharparenleft}rule\ allI{\isacharparenright}\isanewline
\ \ \ \ \isacommand{fix}\isamarkupfalse%
\ F\isanewline
\ \ \ \ \isacommand{have}\isamarkupfalse%
\ {\isachardoublequoteopen}{\isacharparenleft}F\ {\isasymin}\ S\ {\isasymlongrightarrow}\ isModel\ {\isacharparenleft}setValuation\ S{\isacharparenright}\ F{\isacharparenright}\isanewline
\ \ \ \ \ \ \ \ \ \ \ {\isasymand}\ {\isacharparenleft}\isactrlbold {\isasymnot}\ F\ {\isasymin}\ S\ {\isasymlongrightarrow}\ {\isacharparenleft}{\isasymnot}{\isacharparenleft}isModel\ {\isacharparenleft}setValuation\ S{\isacharparenright}\ F{\isacharparenright}{\isacharparenright}{\isacharparenright}{\isachardoublequoteclose}\isanewline
\ \ \ \ \ \ \isacommand{using}\isamarkupfalse%
\ assms\ \isacommand{by}\isamarkupfalse%
\ {\isacharparenleft}rule\ Hintikkas{\isacharunderscore}lemma{\isacharunderscore}l{\isadigit{2}}{\isacharparenright}\isanewline
\ \ \ \ \isacommand{thus}\isamarkupfalse%
\ {\isachardoublequoteopen}F\ {\isasymin}\ S\ {\isasymlongrightarrow}\ isModel\ {\isacharparenleft}setValuation\ S{\isacharparenright}\ F{\isachardoublequoteclose}\isanewline
\ \ \ \ \ \ \isacommand{by}\isamarkupfalse%
\ {\isacharparenleft}rule\ conjunct{\isadigit{1}}{\isacharparenright}\isanewline
\ \ \isacommand{qed}\isamarkupfalse%
\isanewline
\ \ \isacommand{thus}\isamarkupfalse%
\ {\isachardoublequoteopen}isModelSet\ {\isacharparenleft}setValuation\ S{\isacharparenright}\ S{\isachardoublequoteclose}\isanewline
\ \ \ \ \isacommand{by}\isamarkupfalse%
\ {\isacharparenleft}simp\ only{\isacharcolon}\ modelSet{\isacharparenright}\isanewline
\isacommand{qed}\isamarkupfalse%
%
\endisatagproof
{\isafoldproof}%
%
\isadelimproof
%
\endisadelimproof
%
\begin{isamarkuptext}%
Finalmente, las pruebas detallada y automática del 
  \isa{Lema\ de\ Hintikka} en Isabelle/HOL.%
\end{isamarkuptext}\isamarkuptrue%
\isacommand{theorem}\isamarkupfalse%
\isanewline
\ \ \isakeyword{assumes}\ {\isachardoublequoteopen}Hintikka\ S{\isachardoublequoteclose}\isanewline
\ \ \isakeyword{shows}\ {\isachardoublequoteopen}sat\ S{\isachardoublequoteclose}\isanewline
%
\isadelimproof
%
\endisadelimproof
%
\isatagproof
\isacommand{proof}\isamarkupfalse%
\ {\isacharminus}\isanewline
\ \ \isacommand{have}\isamarkupfalse%
\ {\isachardoublequoteopen}isModelSet\ {\isacharparenleft}setValuation\ S{\isacharparenright}\ S{\isachardoublequoteclose}\isanewline
\ \ \ \ \isacommand{using}\isamarkupfalse%
\ assms\ \isacommand{by}\isamarkupfalse%
\ {\isacharparenleft}rule\ Hintikka{\isacharunderscore}model{\isacharparenright}\isanewline
\ \ \isacommand{then}\isamarkupfalse%
\ \isacommand{have}\isamarkupfalse%
\ {\isachardoublequoteopen}{\isasymexists}{\isasymA}{\isachardot}\ isModelSet\ {\isasymA}\ S{\isachardoublequoteclose}\isanewline
\ \ \ \ \isacommand{by}\isamarkupfalse%
\ {\isacharparenleft}simp\ only{\isacharcolon}\ exI{\isacharparenright}\isanewline
\ \ \isacommand{thus}\isamarkupfalse%
\ {\isachardoublequoteopen}sat\ S{\isachardoublequoteclose}\ \isanewline
\ \ \ \ \isacommand{by}\isamarkupfalse%
\ {\isacharparenleft}simp\ only{\isacharcolon}\ satAlt{\isacharparenright}\isanewline
\isacommand{qed}\isamarkupfalse%
%
\endisatagproof
{\isafoldproof}%
%
\isadelimproof
\isanewline
%
\endisadelimproof
\isanewline
\isacommand{theorem}\isamarkupfalse%
\ Hintikkaslemma{\isacharcolon}\isanewline
\ \ \isakeyword{assumes}\ {\isachardoublequoteopen}Hintikka\ S{\isachardoublequoteclose}\isanewline
\ \ \isakeyword{shows}\ {\isachardoublequoteopen}sat\ S{\isachardoublequoteclose}\isanewline
%
\isadelimproof
\ \ %
\endisadelimproof
%
\isatagproof
\isacommand{using}\isamarkupfalse%
\ Hintikka{\isacharunderscore}model\ assms\ satAlt\ \isacommand{by}\isamarkupfalse%
\ blast%
\endisatagproof
{\isafoldproof}%
%
\isadelimproof
%
\endisadelimproof
%
\isadelimtheory
%
\endisadelimtheory
%
\isatagtheory
%
\endisatagtheory
{\isafoldtheory}%
%
\isadelimtheory
%
\endisadelimtheory
%
\end{isabellebody}%
\endinput
%:%file=~/Desktop/TFM/TFM1/Hintikka.thy%:%
%:%24=9%:%
%:%36=11%:%
%:%37=12%:%
%:%38=13%:%
%:%39=14%:%
%:%40=15%:%
%:%41=16%:%
%:%42=17%:%
%:%43=18%:%
%:%44=19%:%
%:%45=20%:%
%:%46=21%:%
%:%47=22%:%
%:%48=23%:%
%:%49=24%:%
%:%50=25%:%
%:%51=26%:%
%:%52=27%:%
%:%53=28%:%
%:%54=29%:%
%:%55=30%:%
%:%56=31%:%
%:%57=32%:%
%:%59=34%:%
%:%60=34%:%
%:%71=45%:%
%:%72=46%:%
%:%74=48%:%
%:%75=48%:%
%:%76=49%:%
%:%79=50%:%
%:%80=51%:%
%:%84=51%:%
%:%85=51%:%
%:%87=53%:%
%:%88=54%:%
%:%89=54%:%
%:%90=54%:%
%:%95=54%:%
%:%98=55%:%
%:%99=56%:%
%:%102=58%:%
%:%103=59%:%
%:%104=60%:%
%:%106=62%:%
%:%107=62%:%
%:%108=63%:%
%:%111=64%:%
%:%112=65%:%
%:%116=65%:%
%:%117=65%:%
%:%119=67%:%
%:%120=68%:%
%:%121=68%:%
%:%122=68%:%
%:%127=68%:%
%:%130=69%:%
%:%131=70%:%
%:%134=72%:%
%:%135=73%:%
%:%136=74%:%
%:%137=75%:%
%:%138=76%:%
%:%139=77%:%
%:%140=78%:%
%:%142=80%:%
%:%143=80%:%
%:%144=81%:%
%:%145=82%:%
%:%153=90%:%
%:%160=91%:%
%:%161=91%:%
%:%162=92%:%
%:%163=92%:%
%:%171=100%:%
%:%172=101%:%
%:%173=101%:%
%:%174=101%:%
%:%175=102%:%
%:%176=102%:%
%:%177=102%:%
%:%178=103%:%
%:%179=103%:%
%:%180=103%:%
%:%181=104%:%
%:%191=106%:%
%:%192=107%:%
%:%193=108%:%
%:%194=109%:%
%:%195=110%:%
%:%196=111%:%
%:%197=112%:%
%:%198=113%:%
%:%199=114%:%
%:%200=115%:%
%:%201=116%:%
%:%202=117%:%
%:%203=118%:%
%:%204=119%:%
%:%205=120%:%
%:%206=121%:%
%:%207=122%:%
%:%208=123%:%
%:%209=124%:%
%:%210=125%:%
%:%211=126%:%
%:%212=127%:%
%:%213=128%:%
%:%214=129%:%
%:%215=130%:%
%:%216=131%:%
%:%217=132%:%
%:%218=133%:%
%:%219=134%:%
%:%220=135%:%
%:%221=136%:%
%:%222=137%:%
%:%223=138%:%
%:%224=139%:%
%:%225=140%:%
%:%226=141%:%
%:%227=142%:%
%:%228=143%:%
%:%229=144%:%
%:%230=145%:%
%:%231=146%:%
%:%232=147%:%
%:%233=148%:%
%:%234=149%:%
%:%235=150%:%
%:%236=151%:%
%:%237=152%:%
%:%238=153%:%
%:%239=154%:%
%:%240=155%:%
%:%241=156%:%
%:%242=157%:%
%:%244=159%:%
%:%245=159%:%
%:%246=160%:%
%:%247=161%:%
%:%254=162%:%
%:%255=162%:%
%:%256=163%:%
%:%257=163%:%
%:%265=171%:%
%:%266=172%:%
%:%267=172%:%
%:%268=172%:%
%:%269=173%:%
%:%270=173%:%
%:%271=173%:%
%:%272=174%:%
%:%278=174%:%
%:%281=175%:%
%:%282=176%:%
%:%283=176%:%
%:%284=177%:%
%:%287=178%:%
%:%291=178%:%
%:%292=178%:%
%:%293=178%:%
%:%298=178%:%
%:%301=179%:%
%:%302=180%:%
%:%303=180%:%
%:%304=181%:%
%:%305=182%:%
%:%312=183%:%
%:%313=183%:%
%:%314=184%:%
%:%315=184%:%
%:%316=185%:%
%:%317=185%:%
%:%325=193%:%
%:%326=194%:%
%:%327=194%:%
%:%328=194%:%
%:%329=195%:%
%:%330=195%:%
%:%331=195%:%
%:%332=196%:%
%:%333=196%:%
%:%334=197%:%
%:%335=197%:%
%:%336=197%:%
%:%337=198%:%
%:%338=198%:%
%:%339=199%:%
%:%340=199%:%
%:%341=199%:%
%:%342=200%:%
%:%343=200%:%
%:%344=201%:%
%:%345=201%:%
%:%346=202%:%
%:%347=202%:%
%:%348=202%:%
%:%349=203%:%
%:%355=203%:%
%:%358=204%:%
%:%359=205%:%
%:%360=205%:%
%:%361=206%:%
%:%364=207%:%
%:%368=207%:%
%:%369=207%:%
%:%374=207%:%
%:%377=208%:%
%:%378=209%:%
%:%379=209%:%
%:%380=210%:%
%:%381=211%:%
%:%388=212%:%
%:%389=212%:%
%:%390=213%:%
%:%391=213%:%
%:%392=214%:%
%:%393=214%:%
%:%401=222%:%
%:%402=223%:%
%:%403=223%:%
%:%404=223%:%
%:%405=224%:%
%:%406=224%:%
%:%407=224%:%
%:%408=225%:%
%:%409=225%:%
%:%410=226%:%
%:%411=226%:%
%:%412=226%:%
%:%413=227%:%
%:%414=227%:%
%:%415=228%:%
%:%416=228%:%
%:%417=229%:%
%:%418=229%:%
%:%419=229%:%
%:%420=230%:%
%:%426=230%:%
%:%429=231%:%
%:%430=232%:%
%:%431=232%:%
%:%432=233%:%
%:%435=234%:%
%:%439=234%:%
%:%440=234%:%
%:%445=234%:%
%:%448=235%:%
%:%449=236%:%
%:%450=236%:%
%:%451=237%:%
%:%452=238%:%
%:%459=239%:%
%:%460=239%:%
%:%461=240%:%
%:%462=240%:%
%:%463=241%:%
%:%464=241%:%
%:%472=249%:%
%:%473=250%:%
%:%474=250%:%
%:%475=250%:%
%:%476=251%:%
%:%477=251%:%
%:%478=251%:%
%:%479=252%:%
%:%480=252%:%
%:%481=253%:%
%:%482=253%:%
%:%483=253%:%
%:%484=254%:%
%:%485=254%:%
%:%486=255%:%
%:%487=255%:%
%:%488=256%:%
%:%489=256%:%
%:%490=256%:%
%:%491=257%:%
%:%497=257%:%
%:%500=258%:%
%:%501=259%:%
%:%502=259%:%
%:%503=260%:%
%:%506=261%:%
%:%510=261%:%
%:%511=261%:%
%:%516=261%:%
%:%519=262%:%
%:%520=263%:%
%:%521=263%:%
%:%522=264%:%
%:%523=265%:%
%:%530=266%:%
%:%531=266%:%
%:%532=267%:%
%:%533=267%:%
%:%534=268%:%
%:%535=268%:%
%:%543=276%:%
%:%544=277%:%
%:%545=277%:%
%:%546=277%:%
%:%547=278%:%
%:%548=278%:%
%:%549=278%:%
%:%550=279%:%
%:%551=279%:%
%:%552=280%:%
%:%553=280%:%
%:%554=280%:%
%:%555=281%:%
%:%556=281%:%
%:%557=282%:%
%:%558=282%:%
%:%559=283%:%
%:%560=283%:%
%:%561=283%:%
%:%562=284%:%
%:%568=284%:%
%:%571=285%:%
%:%572=286%:%
%:%573=286%:%
%:%574=287%:%
%:%577=288%:%
%:%581=288%:%
%:%582=288%:%
%:%587=288%:%
%:%590=289%:%
%:%591=290%:%
%:%592=290%:%
%:%593=291%:%
%:%594=292%:%
%:%601=293%:%
%:%602=293%:%
%:%603=294%:%
%:%604=294%:%
%:%605=295%:%
%:%606=295%:%
%:%614=303%:%
%:%615=304%:%
%:%616=304%:%
%:%617=304%:%
%:%618=305%:%
%:%619=305%:%
%:%620=305%:%
%:%621=306%:%
%:%622=306%:%
%:%623=307%:%
%:%624=307%:%
%:%625=307%:%
%:%626=308%:%
%:%627=308%:%
%:%628=309%:%
%:%629=309%:%
%:%630=310%:%
%:%631=310%:%
%:%632=310%:%
%:%633=311%:%
%:%639=311%:%
%:%642=312%:%
%:%643=313%:%
%:%644=313%:%
%:%645=314%:%
%:%648=315%:%
%:%652=315%:%
%:%653=315%:%
%:%658=315%:%
%:%661=316%:%
%:%662=317%:%
%:%663=317%:%
%:%664=318%:%
%:%665=319%:%
%:%672=320%:%
%:%673=320%:%
%:%674=321%:%
%:%675=321%:%
%:%676=322%:%
%:%677=322%:%
%:%685=330%:%
%:%686=331%:%
%:%687=331%:%
%:%688=331%:%
%:%689=332%:%
%:%690=332%:%
%:%691=332%:%
%:%692=333%:%
%:%693=333%:%
%:%694=334%:%
%:%695=334%:%
%:%696=334%:%
%:%697=335%:%
%:%698=335%:%
%:%699=336%:%
%:%700=336%:%
%:%701=337%:%
%:%702=337%:%
%:%703=337%:%
%:%704=338%:%
%:%710=338%:%
%:%713=339%:%
%:%714=340%:%
%:%715=340%:%
%:%716=341%:%
%:%719=342%:%
%:%723=342%:%
%:%724=342%:%
%:%729=342%:%
%:%732=343%:%
%:%733=344%:%
%:%734=344%:%
%:%735=345%:%
%:%736=346%:%
%:%743=347%:%
%:%744=347%:%
%:%745=348%:%
%:%746=348%:%
%:%747=349%:%
%:%748=349%:%
%:%756=357%:%
%:%757=358%:%
%:%758=358%:%
%:%759=358%:%
%:%760=359%:%
%:%761=359%:%
%:%762=359%:%
%:%763=360%:%
%:%764=360%:%
%:%765=361%:%
%:%766=361%:%
%:%767=361%:%
%:%768=362%:%
%:%769=362%:%
%:%770=363%:%
%:%771=363%:%
%:%772=364%:%
%:%773=364%:%
%:%774=364%:%
%:%775=365%:%
%:%781=365%:%
%:%784=366%:%
%:%785=367%:%
%:%786=367%:%
%:%787=368%:%
%:%790=369%:%
%:%794=369%:%
%:%795=369%:%
%:%800=369%:%
%:%803=370%:%
%:%804=371%:%
%:%805=371%:%
%:%806=372%:%
%:%807=373%:%
%:%814=374%:%
%:%815=374%:%
%:%816=375%:%
%:%817=375%:%
%:%818=376%:%
%:%819=376%:%
%:%827=384%:%
%:%828=385%:%
%:%829=385%:%
%:%830=385%:%
%:%831=386%:%
%:%832=386%:%
%:%833=386%:%
%:%834=387%:%
%:%835=387%:%
%:%836=388%:%
%:%837=388%:%
%:%838=388%:%
%:%839=389%:%
%:%840=389%:%
%:%841=390%:%
%:%842=390%:%
%:%843=391%:%
%:%844=391%:%
%:%845=391%:%
%:%846=392%:%
%:%852=392%:%
%:%855=393%:%
%:%856=394%:%
%:%857=394%:%
%:%858=395%:%
%:%861=396%:%
%:%865=396%:%
%:%866=396%:%
%:%875=398%:%
%:%876=399%:%
%:%877=400%:%
%:%878=401%:%
%:%879=402%:%
%:%880=403%:%
%:%881=404%:%
%:%882=405%:%
%:%883=406%:%
%:%884=407%:%
%:%885=408%:%
%:%886=409%:%
%:%887=410%:%
%:%888=411%:%
%:%889=412%:%
%:%890=413%:%
%:%891=414%:%
%:%892=415%:%
%:%893=416%:%
%:%894=417%:%
%:%895=418%:%
%:%896=419%:%
%:%897=420%:%
%:%898=421%:%
%:%899=422%:%
%:%900=423%:%
%:%901=424%:%
%:%902=425%:%
%:%903=426%:%
%:%904=427%:%
%:%905=428%:%
%:%906=429%:%
%:%907=430%:%
%:%908=431%:%
%:%909=432%:%
%:%910=433%:%
%:%911=434%:%
%:%912=435%:%
%:%913=436%:%
%:%914=437%:%
%:%915=438%:%
%:%916=439%:%
%:%917=440%:%
%:%918=441%:%
%:%919=442%:%
%:%920=443%:%
%:%921=444%:%
%:%922=445%:%
%:%923=446%:%
%:%924=447%:%
%:%925=448%:%
%:%926=449%:%
%:%927=450%:%
%:%928=451%:%
%:%929=452%:%
%:%930=453%:%
%:%931=454%:%
%:%932=455%:%
%:%933=456%:%
%:%934=457%:%
%:%935=458%:%
%:%936=459%:%
%:%937=460%:%
%:%938=461%:%
%:%939=462%:%
%:%940=463%:%
%:%941=464%:%
%:%942=465%:%
%:%943=466%:%
%:%944=467%:%
%:%945=468%:%
%:%946=469%:%
%:%947=470%:%
%:%948=471%:%
%:%949=472%:%
%:%950=473%:%
%:%951=474%:%
%:%952=475%:%
%:%953=476%:%
%:%954=477%:%
%:%955=478%:%
%:%956=479%:%
%:%957=480%:%
%:%958=481%:%
%:%959=482%:%
%:%960=483%:%
%:%961=484%:%
%:%962=485%:%
%:%963=486%:%
%:%964=487%:%
%:%965=488%:%
%:%966=489%:%
%:%967=490%:%
%:%968=491%:%
%:%969=492%:%
%:%970=493%:%
%:%971=494%:%
%:%972=495%:%
%:%973=496%:%
%:%974=497%:%
%:%975=498%:%
%:%976=499%:%
%:%977=500%:%
%:%978=501%:%
%:%979=502%:%
%:%980=503%:%
%:%981=504%:%
%:%982=505%:%
%:%983=506%:%
%:%984=507%:%
%:%985=508%:%
%:%986=509%:%
%:%987=510%:%
%:%988=511%:%
%:%989=512%:%
%:%990=513%:%
%:%991=514%:%
%:%992=515%:%
%:%993=516%:%
%:%994=517%:%
%:%995=518%:%
%:%996=519%:%
%:%997=520%:%
%:%998=521%:%
%:%999=522%:%
%:%1000=523%:%
%:%1001=524%:%
%:%1002=525%:%
%:%1003=526%:%
%:%1004=527%:%
%:%1005=528%:%
%:%1006=529%:%
%:%1007=530%:%
%:%1008=531%:%
%:%1009=532%:%
%:%1011=534%:%
%:%1012=534%:%
%:%1015=535%:%
%:%1019=535%:%
%:%1029=537%:%
%:%1030=538%:%
%:%1031=539%:%
%:%1032=540%:%
%:%1033=541%:%
%:%1035=543%:%
%:%1036=543%:%
%:%1037=544%:%
%:%1038=545%:%
%:%1039=546%:%
%:%1046=547%:%
%:%1047=547%:%
%:%1048=548%:%
%:%1049=548%:%
%:%1050=549%:%
%:%1051=549%:%
%:%1052=549%:%
%:%1053=550%:%
%:%1054=550%:%
%:%1055=551%:%
%:%1056=551%:%
%:%1057=551%:%
%:%1058=552%:%
%:%1068=554%:%
%:%1069=555%:%
%:%1070=556%:%
%:%1072=558%:%
%:%1073=558%:%
%:%1074=559%:%
%:%1075=560%:%
%:%1082=561%:%
%:%1083=561%:%
%:%1084=562%:%
%:%1085=562%:%
%:%1086=563%:%
%:%1087=563%:%
%:%1088=563%:%
%:%1089=564%:%
%:%1090=564%:%
%:%1091=565%:%
%:%1092=565%:%
%:%1093=566%:%
%:%1094=566%:%
%:%1095=566%:%
%:%1096=567%:%
%:%1097=567%:%
%:%1098=568%:%
%:%1099=568%:%
%:%1100=568%:%
%:%1101=569%:%
%:%1107=569%:%
%:%1110=570%:%
%:%1111=571%:%
%:%1112=571%:%
%:%1113=572%:%
%:%1114=573%:%
%:%1121=574%:%
%:%1122=574%:%
%:%1123=575%:%
%:%1124=575%:%
%:%1125=576%:%
%:%1126=576%:%
%:%1127=577%:%
%:%1128=577%:%
%:%1129=577%:%
%:%1130=578%:%
%:%1136=578%:%
%:%1139=579%:%
%:%1140=580%:%
%:%1141=580%:%
%:%1142=581%:%
%:%1143=582%:%
%:%1144=583%:%
%:%1151=584%:%
%:%1152=584%:%
%:%1153=585%:%
%:%1154=585%:%
%:%1155=586%:%
%:%1156=586%:%
%:%1157=587%:%
%:%1158=587%:%
%:%1159=587%:%
%:%1160=588%:%
%:%1161=588%:%
%:%1162=588%:%
%:%1163=589%:%
%:%1164=589%:%
%:%1165=589%:%
%:%1166=590%:%
%:%1167=590%:%
%:%1168=590%:%
%:%1169=591%:%
%:%1170=591%:%
%:%1171=592%:%
%:%1172=592%:%
%:%1173=593%:%
%:%1174=593%:%
%:%1175=593%:%
%:%1176=594%:%
%:%1177=594%:%
%:%1178=595%:%
%:%1179=595%:%
%:%1180=595%:%
%:%1181=596%:%
%:%1191=598%:%
%:%1192=599%:%
%:%1193=600%:%
%:%1194=601%:%
%:%1195=602%:%
%:%1197=604%:%
%:%1198=604%:%
%:%1199=605%:%
%:%1200=606%:%
%:%1207=607%:%
%:%1208=607%:%
%:%1209=608%:%
%:%1210=608%:%
%:%1211=609%:%
%:%1212=609%:%
%:%1213=609%:%
%:%1214=610%:%
%:%1215=610%:%
%:%1216=611%:%
%:%1217=611%:%
%:%1218=612%:%
%:%1219=612%:%
%:%1220=612%:%
%:%1221=613%:%
%:%1227=613%:%
%:%1230=614%:%
%:%1231=615%:%
%:%1232=615%:%
%:%1233=616%:%
%:%1234=617%:%
%:%1241=618%:%
%:%1242=618%:%
%:%1243=619%:%
%:%1244=619%:%
%:%1245=620%:%
%:%1246=620%:%
%:%1247=620%:%
%:%1248=621%:%
%:%1249=621%:%
%:%1250=622%:%
%:%1251=622%:%
%:%1252=623%:%
%:%1253=623%:%
%:%1254=623%:%
%:%1255=624%:%
%:%1261=624%:%
%:%1264=625%:%
%:%1265=626%:%
%:%1266=626%:%
%:%1267=627%:%
%:%1268=628%:%
%:%1269=629%:%
%:%1276=630%:%
%:%1277=630%:%
%:%1278=631%:%
%:%1279=631%:%
%:%1280=632%:%
%:%1281=632%:%
%:%1282=632%:%
%:%1283=633%:%
%:%1284=633%:%
%:%1285=634%:%
%:%1286=634%:%
%:%1287=635%:%
%:%1288=635%:%
%:%1289=636%:%
%:%1290=636%:%
%:%1291=636%:%
%:%1292=637%:%
%:%1293=637%:%
%:%1294=638%:%
%:%1295=638%:%
%:%1296=639%:%
%:%1297=639%:%
%:%1298=640%:%
%:%1299=640%:%
%:%1300=640%:%
%:%1301=641%:%
%:%1302=641%:%
%:%1303=642%:%
%:%1313=644%:%
%:%1314=645%:%
%:%1315=646%:%
%:%1317=648%:%
%:%1318=648%:%
%:%1319=649%:%
%:%1320=650%:%
%:%1321=651%:%
%:%1322=652%:%
%:%1329=653%:%
%:%1330=653%:%
%:%1331=654%:%
%:%1332=654%:%
%:%1333=655%:%
%:%1334=655%:%
%:%1335=656%:%
%:%1336=656%:%
%:%1337=656%:%
%:%1338=657%:%
%:%1339=657%:%
%:%1340=657%:%
%:%1341=658%:%
%:%1342=658%:%
%:%1343=658%:%
%:%1344=659%:%
%:%1345=659%:%
%:%1346=659%:%
%:%1347=660%:%
%:%1348=660%:%
%:%1349=661%:%
%:%1350=661%:%
%:%1351=662%:%
%:%1352=662%:%
%:%1353=663%:%
%:%1354=663%:%
%:%1355=663%:%
%:%1356=664%:%
%:%1357=664%:%
%:%1358=664%:%
%:%1359=665%:%
%:%1360=665%:%
%:%1361=665%:%
%:%1362=666%:%
%:%1363=666%:%
%:%1364=666%:%
%:%1365=667%:%
%:%1366=667%:%
%:%1367=668%:%
%:%1368=668%:%
%:%1369=669%:%
%:%1370=669%:%
%:%1371=669%:%
%:%1372=670%:%
%:%1373=670%:%
%:%1374=671%:%
%:%1375=671%:%
%:%1376=671%:%
%:%1377=672%:%
%:%1378=672%:%
%:%1379=673%:%
%:%1380=673%:%
%:%1381=674%:%
%:%1382=674%:%
%:%1383=675%:%
%:%1384=675%:%
%:%1385=675%:%
%:%1386=676%:%
%:%1387=676%:%
%:%1388=676%:%
%:%1389=677%:%
%:%1390=677%:%
%:%1391=677%:%
%:%1392=678%:%
%:%1393=678%:%
%:%1394=678%:%
%:%1395=679%:%
%:%1396=679%:%
%:%1397=680%:%
%:%1398=680%:%
%:%1399=681%:%
%:%1400=681%:%
%:%1401=681%:%
%:%1402=682%:%
%:%1403=682%:%
%:%1404=683%:%
%:%1405=683%:%
%:%1406=683%:%
%:%1407=684%:%
%:%1408=684%:%
%:%1409=685%:%
%:%1415=685%:%
%:%1418=686%:%
%:%1419=687%:%
%:%1420=687%:%
%:%1421=688%:%
%:%1422=689%:%
%:%1423=690%:%
%:%1424=691%:%
%:%1431=692%:%
%:%1432=692%:%
%:%1433=693%:%
%:%1434=693%:%
%:%1435=694%:%
%:%1436=694%:%
%:%1437=695%:%
%:%1438=695%:%
%:%1439=695%:%
%:%1440=696%:%
%:%1441=696%:%
%:%1442=696%:%
%:%1443=697%:%
%:%1444=697%:%
%:%1445=697%:%
%:%1446=698%:%
%:%1447=698%:%
%:%1448=698%:%
%:%1449=699%:%
%:%1450=699%:%
%:%1451=700%:%
%:%1452=700%:%
%:%1453=701%:%
%:%1454=701%:%
%:%1455=701%:%
%:%1456=702%:%
%:%1457=702%:%
%:%1458=702%:%
%:%1459=703%:%
%:%1460=703%:%
%:%1461=703%:%
%:%1462=704%:%
%:%1463=704%:%
%:%1464=705%:%
%:%1465=705%:%
%:%1466=705%:%
%:%1467=706%:%
%:%1468=706%:%
%:%1469=707%:%
%:%1470=707%:%
%:%1471=707%:%
%:%1472=708%:%
%:%1473=708%:%
%:%1474=708%:%
%:%1475=709%:%
%:%1476=709%:%
%:%1477=709%:%
%:%1478=710%:%
%:%1479=710%:%
%:%1480=711%:%
%:%1481=711%:%
%:%1482=711%:%
%:%1483=712%:%
%:%1484=712%:%
%:%1485=713%:%
%:%1486=713%:%
%:%1487=713%:%
%:%1488=714%:%
%:%1489=714%:%
%:%1490=715%:%
%:%1491=715%:%
%:%1492=715%:%
%:%1493=716%:%
%:%1499=716%:%
%:%1502=717%:%
%:%1503=718%:%
%:%1504=718%:%
%:%1505=719%:%
%:%1506=720%:%
%:%1507=721%:%
%:%1508=722%:%
%:%1515=723%:%
%:%1516=723%:%
%:%1517=724%:%
%:%1518=724%:%
%:%1519=725%:%
%:%1520=725%:%
%:%1521=726%:%
%:%1522=726%:%
%:%1523=726%:%
%:%1524=727%:%
%:%1525=727%:%
%:%1526=727%:%
%:%1527=728%:%
%:%1528=728%:%
%:%1529=728%:%
%:%1530=729%:%
%:%1531=729%:%
%:%1532=729%:%
%:%1533=730%:%
%:%1534=730%:%
%:%1535=731%:%
%:%1536=731%:%
%:%1537=731%:%
%:%1538=732%:%
%:%1539=732%:%
%:%1540=733%:%
%:%1541=733%:%
%:%1542=734%:%
%:%1543=734%:%
%:%1544=734%:%
%:%1545=735%:%
%:%1546=735%:%
%:%1547=735%:%
%:%1548=736%:%
%:%1549=736%:%
%:%1550=736%:%
%:%1551=737%:%
%:%1552=737%:%
%:%1553=738%:%
%:%1554=738%:%
%:%1555=738%:%
%:%1556=739%:%
%:%1557=739%:%
%:%1558=740%:%
%:%1559=740%:%
%:%1560=740%:%
%:%1561=741%:%
%:%1562=741%:%
%:%1563=741%:%
%:%1564=742%:%
%:%1565=742%:%
%:%1566=742%:%
%:%1567=743%:%
%:%1568=743%:%
%:%1569=744%:%
%:%1570=744%:%
%:%1571=744%:%
%:%1572=745%:%
%:%1573=745%:%
%:%1574=746%:%
%:%1575=746%:%
%:%1576=746%:%
%:%1577=747%:%
%:%1578=747%:%
%:%1579=748%:%
%:%1580=748%:%
%:%1581=748%:%
%:%1582=749%:%
%:%1588=749%:%
%:%1591=750%:%
%:%1592=751%:%
%:%1593=751%:%
%:%1600=752%:%
%:%1601=752%:%
%:%1602=753%:%
%:%1603=753%:%
%:%1604=754%:%
%:%1605=754%:%
%:%1606=754%:%
%:%1607=754%:%
%:%1608=755%:%
%:%1609=755%:%
%:%1610=756%:%
%:%1611=756%:%
%:%1612=757%:%
%:%1613=757%:%
%:%1614=757%:%
%:%1615=757%:%
%:%1616=758%:%
%:%1617=758%:%
%:%1618=759%:%
%:%1619=759%:%
%:%1620=760%:%
%:%1621=760%:%
%:%1622=760%:%
%:%1623=760%:%
%:%1624=761%:%
%:%1625=761%:%
%:%1626=762%:%
%:%1627=762%:%
%:%1628=763%:%
%:%1629=763%:%
%:%1630=763%:%
%:%1631=763%:%
%:%1632=764%:%
%:%1633=764%:%
%:%1634=765%:%
%:%1635=765%:%
%:%1636=766%:%
%:%1637=766%:%
%:%1638=766%:%
%:%1639=766%:%
%:%1640=767%:%
%:%1641=767%:%
%:%1642=768%:%
%:%1643=768%:%
%:%1644=769%:%
%:%1645=769%:%
%:%1646=769%:%
%:%1647=769%:%
%:%1648=770%:%
%:%1658=772%:%
%:%1660=774%:%
%:%1661=774%:%
%:%1662=775%:%
%:%1665=776%:%
%:%1669=776%:%
%:%1670=776%:%
%:%1671=777%:%
%:%1672=777%:%
%:%1673=778%:%
%:%1674=778%:%
%:%1675=779%:%
%:%1676=779%:%
%:%1677=779%:%
%:%1678=780%:%
%:%1679=780%:%
%:%1680=780%:%
%:%1681=781%:%
%:%1682=781%:%
%:%1683=782%:%
%:%1684=782%:%
%:%1685=782%:%
%:%1699=784%:%
%:%1711=786%:%
%:%1712=787%:%
%:%1713=788%:%
%:%1714=789%:%
%:%1715=790%:%
%:%1716=791%:%
%:%1717=792%:%
%:%1718=793%:%
%:%1719=794%:%
%:%1719=795%:%
%:%1720=796%:%
%:%1721=797%:%
%:%1722=798%:%
%:%1723=799%:%
%:%1724=800%:%
%:%1725=801%:%
%:%1727=803%:%
%:%1728=803%:%
%:%1729=804%:%
%:%1730=805%:%
%:%1732=807%:%
%:%1733=808%:%
%:%1735=810%:%
%:%1736=810%:%
%:%1737=811%:%
%:%1740=812%:%
%:%1741=813%:%
%:%1745=813%:%
%:%1746=813%:%
%:%1748=815%:%
%:%1749=816%:%
%:%1750=816%:%
%:%1751=816%:%
%:%1752=817%:%
%:%1753=818%:%
%:%1754=818%:%
%:%1755=819%:%
%:%1756=820%:%
%:%1757=820%:%
%:%1758=820%:%
%:%1763=820%:%
%:%1766=821%:%
%:%1767=822%:%
%:%1770=824%:%
%:%1771=825%:%
%:%1772=826%:%
%:%1773=827%:%
%:%1774=828%:%
%:%1775=829%:%
%:%1776=830%:%
%:%1777=831%:%
%:%1778=832%:%
%:%1779=833%:%
%:%1780=834%:%
%:%1781=835%:%
%:%1783=837%:%
%:%1784=837%:%
%:%1785=838%:%
%:%1788=839%:%
%:%1792=839%:%
%:%1802=841%:%
%:%1803=842%:%
%:%1804=843%:%
%:%1805=844%:%
%:%1806=845%:%
%:%1807=846%:%
%:%1808=847%:%
%:%1809=848%:%
%:%1810=849%:%
%:%1811=850%:%
%:%1812=851%:%
%:%1813=852%:%
%:%1814=853%:%
%:%1815=854%:%
%:%1816=855%:%
%:%1817=856%:%
%:%1818=857%:%
%:%1819=858%:%
%:%1820=859%:%
%:%1821=860%:%
%:%1822=861%:%
%:%1823=862%:%
%:%1824=863%:%
%:%1825=864%:%
%:%1826=865%:%
%:%1827=866%:%
%:%1828=867%:%
%:%1829=868%:%
%:%1830=869%:%
%:%1831=870%:%
%:%1832=871%:%
%:%1833=872%:%
%:%1834=873%:%
%:%1835=874%:%
%:%1836=875%:%
%:%1837=876%:%
%:%1838=877%:%
%:%1839=878%:%
%:%1840=879%:%
%:%1841=880%:%
%:%1842=881%:%
%:%1843=882%:%
%:%1844=883%:%
%:%1845=884%:%
%:%1846=885%:%
%:%1847=886%:%
%:%1848=887%:%
%:%1849=888%:%
%:%1850=889%:%
%:%1851=890%:%
%:%1852=891%:%
%:%1853=892%:%
%:%1854=893%:%
%:%1855=894%:%
%:%1856=895%:%
%:%1857=896%:%
%:%1858=897%:%
%:%1859=898%:%
%:%1860=899%:%
%:%1861=900%:%
%:%1862=901%:%
%:%1863=902%:%
%:%1864=903%:%
%:%1865=904%:%
%:%1866=905%:%
%:%1867=906%:%
%:%1868=907%:%
%:%1869=908%:%
%:%1870=909%:%
%:%1871=910%:%
%:%1872=911%:%
%:%1873=912%:%
%:%1874=913%:%
%:%1875=914%:%
%:%1876=915%:%
%:%1877=916%:%
%:%1878=917%:%
%:%1879=918%:%
%:%1880=919%:%
%:%1881=920%:%
%:%1882=921%:%
%:%1883=922%:%
%:%1884=923%:%
%:%1885=924%:%
%:%1886=925%:%
%:%1887=926%:%
%:%1888=927%:%
%:%1889=928%:%
%:%1890=929%:%
%:%1891=930%:%
%:%1892=931%:%
%:%1893=932%:%
%:%1893=933%:%
%:%1894=934%:%
%:%1895=935%:%
%:%1896=936%:%
%:%1897=937%:%
%:%1898=938%:%
%:%1899=939%:%
%:%1900=940%:%
%:%1901=941%:%
%:%1902=942%:%
%:%1903=943%:%
%:%1904=944%:%
%:%1905=945%:%
%:%1906=946%:%
%:%1907=947%:%
%:%1908=948%:%
%:%1909=949%:%
%:%1910=950%:%
%:%1911=951%:%
%:%1912=952%:%
%:%1913=953%:%
%:%1914=954%:%
%:%1915=955%:%
%:%1916=956%:%
%:%1917=957%:%
%:%1918=958%:%
%:%1919=959%:%
%:%1920=960%:%
%:%1921=961%:%
%:%1922=962%:%
%:%1923=963%:%
%:%1924=964%:%
%:%1925=965%:%
%:%1926=966%:%
%:%1927=967%:%
%:%1928=968%:%
%:%1929=969%:%
%:%1930=970%:%
%:%1931=971%:%
%:%1932=972%:%
%:%1932=973%:%
%:%1933=974%:%
%:%1934=975%:%
%:%1935=976%:%
%:%1936=977%:%
%:%1937=978%:%
%:%1938=979%:%
%:%1939=980%:%
%:%1940=981%:%
%:%1941=982%:%
%:%1942=983%:%
%:%1943=984%:%
%:%1944=985%:%
%:%1945=986%:%
%:%1946=987%:%
%:%1947=988%:%
%:%1948=989%:%
%:%1949=990%:%
%:%1950=991%:%
%:%1951=992%:%
%:%1952=993%:%
%:%1953=994%:%
%:%1954=995%:%
%:%1955=996%:%
%:%1956=997%:%
%:%1957=998%:%
%:%1958=999%:%
%:%1959=1000%:%
%:%1960=1001%:%
%:%1961=1002%:%
%:%1962=1003%:%
%:%1963=1004%:%
%:%1964=1005%:%
%:%1965=1006%:%
%:%1966=1007%:%
%:%1967=1008%:%
%:%1968=1009%:%
%:%1969=1010%:%
%:%1970=1011%:%
%:%1971=1012%:%
%:%1972=1013%:%
%:%1973=1014%:%
%:%1974=1015%:%
%:%1975=1016%:%
%:%1976=1017%:%
%:%1977=1018%:%
%:%1978=1019%:%
%:%1979=1020%:%
%:%1980=1021%:%
%:%1981=1022%:%
%:%1982=1023%:%
%:%1983=1024%:%
%:%1984=1025%:%
%:%1985=1026%:%
%:%1986=1027%:%
%:%1987=1028%:%
%:%1988=1029%:%
%:%1989=1030%:%
%:%1990=1031%:%
%:%1991=1032%:%
%:%1992=1033%:%
%:%1993=1034%:%
%:%1994=1035%:%
%:%1995=1036%:%
%:%1996=1037%:%
%:%1997=1038%:%
%:%1998=1039%:%
%:%1999=1040%:%
%:%2000=1041%:%
%:%2001=1042%:%
%:%2002=1043%:%
%:%2003=1044%:%
%:%2004=1045%:%
%:%2005=1046%:%
%:%2006=1047%:%
%:%2007=1048%:%
%:%2008=1049%:%
%:%2010=1051%:%
%:%2011=1051%:%
%:%2012=1052%:%
%:%2013=1053%:%
%:%2014=1054%:%
%:%2021=1055%:%
%:%2022=1055%:%
%:%2023=1056%:%
%:%2024=1056%:%
%:%2025=1057%:%
%:%2026=1057%:%
%:%2027=1058%:%
%:%2028=1058%:%
%:%2029=1059%:%
%:%2030=1059%:%
%:%2031=1060%:%
%:%2032=1060%:%
%:%2033=1061%:%
%:%2034=1061%:%
%:%2035=1062%:%
%:%2036=1062%:%
%:%2037=1063%:%
%:%2038=1063%:%
%:%2039=1064%:%
%:%2040=1064%:%
%:%2041=1065%:%
%:%2042=1065%:%
%:%2043=1066%:%
%:%2044=1066%:%
%:%2045=1067%:%
%:%2046=1067%:%
%:%2047=1068%:%
%:%2048=1068%:%
%:%2049=1069%:%
%:%2050=1070%:%
%:%2051=1070%:%
%:%2052=1071%:%
%:%2053=1071%:%
%:%2054=1072%:%
%:%2055=1072%:%
%:%2056=1073%:%
%:%2057=1073%:%
%:%2058=1074%:%
%:%2059=1074%:%
%:%2060=1074%:%
%:%2061=1075%:%
%:%2062=1075%:%
%:%2063=1075%:%
%:%2064=1076%:%
%:%2065=1076%:%
%:%2066=1076%:%
%:%2067=1077%:%
%:%2068=1077%:%
%:%2069=1077%:%
%:%2070=1078%:%
%:%2071=1078%:%
%:%2072=1079%:%
%:%2073=1079%:%
%:%2074=1079%:%
%:%2075=1080%:%
%:%2076=1080%:%
%:%2077=1081%:%
%:%2078=1081%:%
%:%2079=1081%:%
%:%2080=1082%:%
%:%2081=1082%:%
%:%2082=1083%:%
%:%2083=1083%:%
%:%2084=1083%:%
%:%2085=1084%:%
%:%2086=1084%:%
%:%2087=1085%:%
%:%2088=1085%:%
%:%2089=1086%:%
%:%2095=1086%:%
%:%2098=1087%:%
%:%2099=1088%:%
%:%2100=1088%:%
%:%2101=1089%:%
%:%2102=1090%:%
%:%2103=1091%:%
%:%2106=1092%:%
%:%2110=1092%:%
%:%2111=1092%:%
%:%2116=1092%:%
%:%2119=1093%:%
%:%2120=1094%:%
%:%2121=1094%:%
%:%2122=1095%:%
%:%2123=1096%:%
%:%2124=1097%:%
%:%2131=1098%:%
%:%2132=1098%:%
%:%2133=1099%:%
%:%2134=1099%:%
%:%2135=1100%:%
%:%2136=1100%:%
%:%2137=1101%:%
%:%2138=1101%:%
%:%2139=1102%:%
%:%2140=1102%:%
%:%2141=1103%:%
%:%2142=1103%:%
%:%2143=1103%:%
%:%2144=1104%:%
%:%2145=1104%:%
%:%2146=1105%:%
%:%2147=1105%:%
%:%2148=1105%:%
%:%2149=1106%:%
%:%2150=1106%:%
%:%2151=1107%:%
%:%2152=1107%:%
%:%2153=1108%:%
%:%2154=1108%:%
%:%2155=1109%:%
%:%2156=1109%:%
%:%2157=1110%:%
%:%2158=1110%:%
%:%2159=1111%:%
%:%2160=1111%:%
%:%2161=1112%:%
%:%2162=1112%:%
%:%2163=1113%:%
%:%2164=1113%:%
%:%2165=1114%:%
%:%2166=1114%:%
%:%2167=1115%:%
%:%2168=1115%:%
%:%2169=1116%:%
%:%2170=1116%:%
%:%2171=1117%:%
%:%2172=1117%:%
%:%2173=1117%:%
%:%2174=1118%:%
%:%2175=1118%:%
%:%2176=1119%:%
%:%2177=1119%:%
%:%2178=1119%:%
%:%2179=1120%:%
%:%2180=1120%:%
%:%2181=1121%:%
%:%2182=1121%:%
%:%2183=1122%:%
%:%2189=1122%:%
%:%2192=1123%:%
%:%2193=1124%:%
%:%2194=1124%:%
%:%2195=1125%:%
%:%2196=1126%:%
%:%2197=1127%:%
%:%2200=1128%:%
%:%2204=1128%:%
%:%2205=1128%:%
%:%2210=1128%:%
%:%2213=1129%:%
%:%2214=1130%:%
%:%2215=1130%:%
%:%2216=1131%:%
%:%2217=1132%:%
%:%2218=1133%:%
%:%2219=1134%:%
%:%2220=1135%:%
%:%2227=1136%:%
%:%2228=1136%:%
%:%2229=1137%:%
%:%2230=1137%:%
%:%2231=1138%:%
%:%2232=1138%:%
%:%2233=1139%:%
%:%2234=1139%:%
%:%2235=1140%:%
%:%2236=1140%:%
%:%2237=1141%:%
%:%2238=1141%:%
%:%2239=1142%:%
%:%2240=1142%:%
%:%2241=1142%:%
%:%2242=1143%:%
%:%2243=1143%:%
%:%2244=1143%:%
%:%2245=1144%:%
%:%2246=1144%:%
%:%2247=1144%:%
%:%2248=1145%:%
%:%2249=1145%:%
%:%2250=1145%:%
%:%2251=1146%:%
%:%2252=1146%:%
%:%2253=1147%:%
%:%2254=1147%:%
%:%2255=1147%:%
%:%2256=1148%:%
%:%2257=1148%:%
%:%2258=1149%:%
%:%2259=1149%:%
%:%2260=1149%:%
%:%2261=1150%:%
%:%2262=1150%:%
%:%2263=1151%:%
%:%2264=1151%:%
%:%2265=1151%:%
%:%2266=1152%:%
%:%2267=1152%:%
%:%2268=1153%:%
%:%2269=1153%:%
%:%2270=1154%:%
%:%2271=1154%:%
%:%2272=1155%:%
%:%2273=1155%:%
%:%2274=1156%:%
%:%2275=1156%:%
%:%2276=1157%:%
%:%2277=1157%:%
%:%2278=1158%:%
%:%2279=1158%:%
%:%2280=1159%:%
%:%2281=1159%:%
%:%2282=1160%:%
%:%2283=1160%:%
%:%2284=1160%:%
%:%2285=1161%:%
%:%2286=1161%:%
%:%2287=1161%:%
%:%2288=1162%:%
%:%2289=1162%:%
%:%2290=1162%:%
%:%2291=1163%:%
%:%2292=1163%:%
%:%2293=1164%:%
%:%2294=1164%:%
%:%2295=1164%:%
%:%2296=1165%:%
%:%2297=1165%:%
%:%2298=1165%:%
%:%2299=1166%:%
%:%2300=1166%:%
%:%2301=1166%:%
%:%2302=1167%:%
%:%2303=1167%:%
%:%2304=1167%:%
%:%2305=1168%:%
%:%2306=1168%:%
%:%2307=1169%:%
%:%2308=1169%:%
%:%2309=1169%:%
%:%2310=1170%:%
%:%2311=1171%:%
%:%2312=1171%:%
%:%2313=1172%:%
%:%2314=1172%:%
%:%2315=1172%:%
%:%2316=1173%:%
%:%2317=1173%:%
%:%2318=1174%:%
%:%2319=1174%:%
%:%2320=1174%:%
%:%2321=1175%:%
%:%2322=1175%:%
%:%2323=1176%:%
%:%2324=1176%:%
%:%2325=1176%:%
%:%2326=1177%:%
%:%2327=1177%:%
%:%2328=1178%:%
%:%2329=1178%:%
%:%2330=1179%:%
%:%2336=1179%:%
%:%2339=1180%:%
%:%2340=1181%:%
%:%2341=1181%:%
%:%2342=1182%:%
%:%2343=1183%:%
%:%2346=1186%:%
%:%2349=1187%:%
%:%2353=1187%:%
%:%2354=1187%:%
%:%2355=1187%:%
%:%2360=1187%:%
%:%2363=1188%:%
%:%2364=1189%:%
%:%2365=1189%:%
%:%2366=1190%:%
%:%2367=1191%:%
%:%2368=1192%:%
%:%2369=1193%:%
%:%2370=1194%:%
%:%2371=1195%:%
%:%2372=1196%:%
%:%2379=1197%:%
%:%2380=1197%:%
%:%2381=1198%:%
%:%2382=1198%:%
%:%2383=1199%:%
%:%2384=1199%:%
%:%2385=1200%:%
%:%2386=1200%:%
%:%2387=1201%:%
%:%2388=1201%:%
%:%2389=1202%:%
%:%2390=1202%:%
%:%2391=1203%:%
%:%2392=1203%:%
%:%2393=1203%:%
%:%2394=1204%:%
%:%2395=1204%:%
%:%2396=1204%:%
%:%2397=1205%:%
%:%2398=1205%:%
%:%2399=1205%:%
%:%2400=1206%:%
%:%2401=1206%:%
%:%2402=1206%:%
%:%2403=1207%:%
%:%2404=1207%:%
%:%2405=1208%:%
%:%2406=1208%:%
%:%2407=1209%:%
%:%2408=1209%:%
%:%2409=1209%:%
%:%2410=1210%:%
%:%2411=1210%:%
%:%2412=1210%:%
%:%2413=1211%:%
%:%2414=1211%:%
%:%2415=1211%:%
%:%2416=1212%:%
%:%2417=1212%:%
%:%2418=1212%:%
%:%2419=1213%:%
%:%2420=1213%:%
%:%2421=1214%:%
%:%2422=1214%:%
%:%2423=1215%:%
%:%2424=1215%:%
%:%2425=1215%:%
%:%2426=1216%:%
%:%2427=1216%:%
%:%2428=1217%:%
%:%2429=1217%:%
%:%2430=1217%:%
%:%2431=1218%:%
%:%2432=1218%:%
%:%2433=1218%:%
%:%2434=1219%:%
%:%2435=1219%:%
%:%2436=1219%:%
%:%2437=1220%:%
%:%2438=1220%:%
%:%2439=1220%:%
%:%2440=1221%:%
%:%2441=1221%:%
%:%2442=1222%:%
%:%2443=1222%:%
%:%2444=1223%:%
%:%2445=1223%:%
%:%2446=1223%:%
%:%2447=1224%:%
%:%2448=1224%:%
%:%2449=1224%:%
%:%2450=1225%:%
%:%2451=1225%:%
%:%2452=1226%:%
%:%2453=1226%:%
%:%2454=1227%:%
%:%2455=1227%:%
%:%2456=1228%:%
%:%2457=1228%:%
%:%2458=1229%:%
%:%2459=1229%:%
%:%2460=1230%:%
%:%2461=1230%:%
%:%2462=1231%:%
%:%2463=1231%:%
%:%2464=1232%:%
%:%2465=1232%:%
%:%2466=1233%:%
%:%2467=1233%:%
%:%2468=1234%:%
%:%2469=1234%:%
%:%2470=1235%:%
%:%2471=1235%:%
%:%2472=1235%:%
%:%2473=1236%:%
%:%2474=1236%:%
%:%2475=1236%:%
%:%2476=1237%:%
%:%2477=1237%:%
%:%2478=1237%:%
%:%2479=1238%:%
%:%2480=1238%:%
%:%2481=1238%:%
%:%2482=1239%:%
%:%2483=1239%:%
%:%2484=1240%:%
%:%2485=1240%:%
%:%2486=1241%:%
%:%2487=1241%:%
%:%2488=1242%:%
%:%2489=1242%:%
%:%2490=1242%:%
%:%2491=1243%:%
%:%2492=1243%:%
%:%2493=1243%:%
%:%2494=1244%:%
%:%2495=1244%:%
%:%2496=1244%:%
%:%2497=1245%:%
%:%2498=1245%:%
%:%2499=1245%:%
%:%2500=1246%:%
%:%2501=1246%:%
%:%2502=1247%:%
%:%2503=1247%:%
%:%2504=1247%:%
%:%2505=1248%:%
%:%2506=1248%:%
%:%2507=1249%:%
%:%2508=1249%:%
%:%2509=1249%:%
%:%2510=1250%:%
%:%2511=1250%:%
%:%2512=1251%:%
%:%2513=1251%:%
%:%2514=1251%:%
%:%2515=1252%:%
%:%2516=1253%:%
%:%2517=1253%:%
%:%2518=1254%:%
%:%2519=1254%:%
%:%2520=1254%:%
%:%2521=1255%:%
%:%2522=1255%:%
%:%2523=1256%:%
%:%2524=1256%:%
%:%2525=1256%:%
%:%2526=1257%:%
%:%2527=1257%:%
%:%2528=1258%:%
%:%2529=1258%:%
%:%2530=1259%:%
%:%2531=1259%:%
%:%2532=1260%:%
%:%2533=1260%:%
%:%2534=1261%:%
%:%2535=1261%:%
%:%2536=1261%:%
%:%2537=1262%:%
%:%2538=1262%:%
%:%2539=1262%:%
%:%2540=1263%:%
%:%2541=1263%:%
%:%2542=1263%:%
%:%2543=1264%:%
%:%2544=1264%:%
%:%2545=1264%:%
%:%2546=1265%:%
%:%2547=1265%:%
%:%2548=1266%:%
%:%2549=1266%:%
%:%2550=1266%:%
%:%2551=1267%:%
%:%2552=1267%:%
%:%2553=1268%:%
%:%2554=1268%:%
%:%2555=1268%:%
%:%2556=1269%:%
%:%2557=1269%:%
%:%2558=1270%:%
%:%2559=1270%:%
%:%2560=1270%:%
%:%2561=1271%:%
%:%2562=1272%:%
%:%2563=1272%:%
%:%2564=1273%:%
%:%2565=1273%:%
%:%2566=1273%:%
%:%2567=1274%:%
%:%2568=1274%:%
%:%2569=1275%:%
%:%2570=1275%:%
%:%2571=1275%:%
%:%2572=1276%:%
%:%2573=1276%:%
%:%2574=1277%:%
%:%2575=1277%:%
%:%2576=1278%:%
%:%2577=1278%:%
%:%2578=1279%:%
%:%2584=1279%:%
%:%2587=1280%:%
%:%2588=1281%:%
%:%2589=1281%:%
%:%2590=1282%:%
%:%2591=1283%:%
%:%2597=1289%:%
%:%2600=1290%:%
%:%2604=1290%:%
%:%2605=1290%:%
%:%2610=1290%:%
%:%2613=1291%:%
%:%2614=1292%:%
%:%2615=1292%:%
%:%2616=1293%:%
%:%2617=1294%:%
%:%2618=1295%:%
%:%2619=1296%:%
%:%2620=1297%:%
%:%2621=1298%:%
%:%2622=1299%:%
%:%2629=1300%:%
%:%2630=1300%:%
%:%2631=1301%:%
%:%2632=1301%:%
%:%2633=1302%:%
%:%2634=1302%:%
%:%2635=1303%:%
%:%2636=1303%:%
%:%2637=1304%:%
%:%2638=1304%:%
%:%2639=1305%:%
%:%2640=1305%:%
%:%2641=1306%:%
%:%2642=1306%:%
%:%2643=1306%:%
%:%2644=1307%:%
%:%2645=1307%:%
%:%2646=1307%:%
%:%2647=1308%:%
%:%2648=1308%:%
%:%2649=1308%:%
%:%2650=1309%:%
%:%2651=1309%:%
%:%2652=1309%:%
%:%2653=1310%:%
%:%2654=1310%:%
%:%2655=1311%:%
%:%2656=1311%:%
%:%2657=1312%:%
%:%2658=1312%:%
%:%2659=1313%:%
%:%2660=1313%:%
%:%2661=1313%:%
%:%2662=1314%:%
%:%2663=1314%:%
%:%2664=1314%:%
%:%2665=1315%:%
%:%2666=1315%:%
%:%2667=1315%:%
%:%2668=1316%:%
%:%2669=1316%:%
%:%2670=1316%:%
%:%2671=1317%:%
%:%2672=1317%:%
%:%2673=1318%:%
%:%2674=1318%:%
%:%2675=1318%:%
%:%2676=1319%:%
%:%2677=1319%:%
%:%2678=1320%:%
%:%2679=1320%:%
%:%2680=1320%:%
%:%2681=1321%:%
%:%2682=1321%:%
%:%2683=1322%:%
%:%2684=1322%:%
%:%2685=1323%:%
%:%2686=1323%:%
%:%2687=1324%:%
%:%2688=1324%:%
%:%2689=1325%:%
%:%2690=1325%:%
%:%2691=1326%:%
%:%2692=1326%:%
%:%2693=1327%:%
%:%2694=1327%:%
%:%2695=1327%:%
%:%2696=1328%:%
%:%2697=1328%:%
%:%2698=1328%:%
%:%2699=1329%:%
%:%2700=1329%:%
%:%2701=1329%:%
%:%2702=1330%:%
%:%2703=1330%:%
%:%2704=1330%:%
%:%2705=1331%:%
%:%2706=1331%:%
%:%2707=1332%:%
%:%2708=1332%:%
%:%2709=1332%:%
%:%2710=1333%:%
%:%2711=1333%:%
%:%2712=1334%:%
%:%2713=1334%:%
%:%2714=1334%:%
%:%2715=1335%:%
%:%2716=1335%:%
%:%2717=1336%:%
%:%2718=1336%:%
%:%2719=1337%:%
%:%2720=1337%:%
%:%2721=1338%:%
%:%2722=1338%:%
%:%2723=1339%:%
%:%2724=1339%:%
%:%2725=1340%:%
%:%2726=1340%:%
%:%2727=1341%:%
%:%2728=1341%:%
%:%2729=1342%:%
%:%2730=1342%:%
%:%2731=1343%:%
%:%2732=1343%:%
%:%2733=1344%:%
%:%2734=1344%:%
%:%2735=1345%:%
%:%2736=1345%:%
%:%2737=1346%:%
%:%2738=1346%:%
%:%2739=1346%:%
%:%2740=1347%:%
%:%2741=1347%:%
%:%2742=1347%:%
%:%2743=1348%:%
%:%2744=1348%:%
%:%2745=1348%:%
%:%2746=1349%:%
%:%2747=1349%:%
%:%2748=1349%:%
%:%2749=1350%:%
%:%2750=1350%:%
%:%2751=1351%:%
%:%2752=1351%:%
%:%2753=1352%:%
%:%2754=1352%:%
%:%2755=1352%:%
%:%2756=1353%:%
%:%2757=1353%:%
%:%2758=1353%:%
%:%2759=1354%:%
%:%2760=1354%:%
%:%2761=1354%:%
%:%2762=1355%:%
%:%2763=1355%:%
%:%2764=1355%:%
%:%2765=1356%:%
%:%2766=1356%:%
%:%2767=1357%:%
%:%2768=1357%:%
%:%2769=1357%:%
%:%2770=1358%:%
%:%2771=1358%:%
%:%2772=1359%:%
%:%2773=1359%:%
%:%2774=1360%:%
%:%2775=1360%:%
%:%2776=1360%:%
%:%2777=1361%:%
%:%2778=1361%:%
%:%2779=1362%:%
%:%2780=1362%:%
%:%2781=1362%:%
%:%2782=1363%:%
%:%2783=1363%:%
%:%2784=1363%:%
%:%2785=1364%:%
%:%2786=1364%:%
%:%2787=1364%:%
%:%2788=1365%:%
%:%2789=1365%:%
%:%2790=1365%:%
%:%2791=1366%:%
%:%2792=1366%:%
%:%2793=1367%:%
%:%2794=1367%:%
%:%2795=1367%:%
%:%2796=1368%:%
%:%2797=1368%:%
%:%2798=1369%:%
%:%2799=1369%:%
%:%2800=1370%:%
%:%2801=1370%:%
%:%2802=1370%:%
%:%2803=1371%:%
%:%2804=1371%:%
%:%2805=1371%:%
%:%2806=1372%:%
%:%2807=1373%:%
%:%2808=1373%:%
%:%2809=1374%:%
%:%2810=1374%:%
%:%2811=1374%:%
%:%2812=1375%:%
%:%2813=1375%:%
%:%2814=1376%:%
%:%2815=1376%:%
%:%2816=1376%:%
%:%2817=1377%:%
%:%2818=1377%:%
%:%2819=1378%:%
%:%2820=1378%:%
%:%2821=1379%:%
%:%2827=1379%:%
%:%2830=1380%:%
%:%2831=1381%:%
%:%2832=1382%:%
%:%2833=1382%:%
%:%2834=1383%:%
%:%2835=1384%:%
%:%2841=1390%:%
%:%2844=1391%:%
%:%2848=1391%:%
%:%2849=1391%:%
%:%2854=1391%:%
%:%2857=1392%:%
%:%2858=1393%:%
%:%2859=1393%:%
%:%2860=1394%:%
%:%2861=1395%:%
%:%2862=1396%:%
%:%2863=1397%:%
%:%2864=1398%:%
%:%2865=1399%:%
%:%2866=1400%:%
%:%2873=1401%:%
%:%2874=1401%:%
%:%2875=1402%:%
%:%2876=1402%:%
%:%2877=1403%:%
%:%2878=1403%:%
%:%2879=1404%:%
%:%2880=1404%:%
%:%2881=1405%:%
%:%2882=1405%:%
%:%2883=1406%:%
%:%2884=1406%:%
%:%2885=1407%:%
%:%2886=1407%:%
%:%2887=1407%:%
%:%2888=1408%:%
%:%2889=1408%:%
%:%2890=1408%:%
%:%2891=1409%:%
%:%2892=1409%:%
%:%2893=1409%:%
%:%2894=1410%:%
%:%2895=1410%:%
%:%2896=1410%:%
%:%2897=1411%:%
%:%2898=1411%:%
%:%2899=1412%:%
%:%2900=1412%:%
%:%2901=1413%:%
%:%2902=1413%:%
%:%2903=1414%:%
%:%2904=1414%:%
%:%2905=1414%:%
%:%2906=1415%:%
%:%2907=1415%:%
%:%2908=1415%:%
%:%2909=1416%:%
%:%2910=1416%:%
%:%2911=1416%:%
%:%2912=1417%:%
%:%2913=1417%:%
%:%2914=1417%:%
%:%2915=1418%:%
%:%2916=1418%:%
%:%2917=1419%:%
%:%2918=1419%:%
%:%2919=1419%:%
%:%2920=1420%:%
%:%2921=1420%:%
%:%2922=1421%:%
%:%2923=1421%:%
%:%2924=1422%:%
%:%2925=1422%:%
%:%2926=1423%:%
%:%2927=1423%:%
%:%2928=1424%:%
%:%2929=1424%:%
%:%2930=1425%:%
%:%2931=1425%:%
%:%2932=1425%:%
%:%2933=1426%:%
%:%2934=1426%:%
%:%2935=1427%:%
%:%2936=1427%:%
%:%2937=1427%:%
%:%2938=1428%:%
%:%2939=1428%:%
%:%2940=1429%:%
%:%2941=1429%:%
%:%2942=1430%:%
%:%2943=1430%:%
%:%2944=1431%:%
%:%2945=1431%:%
%:%2946=1432%:%
%:%2947=1432%:%
%:%2948=1433%:%
%:%2949=1433%:%
%:%2950=1434%:%
%:%2951=1434%:%
%:%2952=1434%:%
%:%2953=1435%:%
%:%2954=1435%:%
%:%2955=1435%:%
%:%2956=1436%:%
%:%2957=1436%:%
%:%2958=1436%:%
%:%2959=1437%:%
%:%2960=1437%:%
%:%2961=1437%:%
%:%2962=1438%:%
%:%2963=1438%:%
%:%2964=1439%:%
%:%2965=1439%:%
%:%2966=1440%:%
%:%2967=1440%:%
%:%2968=1441%:%
%:%2969=1441%:%
%:%2970=1442%:%
%:%2971=1442%:%
%:%2972=1443%:%
%:%2973=1443%:%
%:%2974=1443%:%
%:%2975=1444%:%
%:%2976=1444%:%
%:%2977=1445%:%
%:%2978=1445%:%
%:%2979=1445%:%
%:%2980=1446%:%
%:%2981=1446%:%
%:%2982=1447%:%
%:%2983=1447%:%
%:%2984=1448%:%
%:%2985=1448%:%
%:%2986=1449%:%
%:%2987=1449%:%
%:%2988=1450%:%
%:%2989=1450%:%
%:%2990=1451%:%
%:%2991=1451%:%
%:%2992=1452%:%
%:%2993=1452%:%
%:%2994=1453%:%
%:%2995=1453%:%
%:%2996=1454%:%
%:%2997=1454%:%
%:%2998=1455%:%
%:%2999=1455%:%
%:%3000=1456%:%
%:%3001=1456%:%
%:%3002=1457%:%
%:%3003=1457%:%
%:%3004=1457%:%
%:%3005=1458%:%
%:%3006=1458%:%
%:%3007=1458%:%
%:%3008=1459%:%
%:%3009=1459%:%
%:%3010=1459%:%
%:%3011=1460%:%
%:%3012=1460%:%
%:%3013=1460%:%
%:%3014=1461%:%
%:%3015=1461%:%
%:%3016=1462%:%
%:%3017=1462%:%
%:%3018=1463%:%
%:%3019=1463%:%
%:%3020=1463%:%
%:%3021=1464%:%
%:%3022=1464%:%
%:%3023=1464%:%
%:%3024=1465%:%
%:%3025=1465%:%
%:%3026=1465%:%
%:%3027=1466%:%
%:%3028=1466%:%
%:%3029=1466%:%
%:%3030=1467%:%
%:%3031=1467%:%
%:%3032=1468%:%
%:%3033=1468%:%
%:%3034=1469%:%
%:%3035=1469%:%
%:%3036=1469%:%
%:%3037=1470%:%
%:%3038=1470%:%
%:%3039=1471%:%
%:%3040=1471%:%
%:%3041=1471%:%
%:%3042=1472%:%
%:%3043=1472%:%
%:%3044=1472%:%
%:%3045=1473%:%
%:%3046=1473%:%
%:%3047=1473%:%
%:%3048=1474%:%
%:%3049=1474%:%
%:%3050=1474%:%
%:%3051=1475%:%
%:%3052=1475%:%
%:%3053=1476%:%
%:%3054=1476%:%
%:%3055=1476%:%
%:%3056=1477%:%
%:%3057=1477%:%
%:%3058=1478%:%
%:%3059=1478%:%
%:%3060=1479%:%
%:%3061=1479%:%
%:%3062=1480%:%
%:%3063=1480%:%
%:%3064=1481%:%
%:%3065=1481%:%
%:%3066=1481%:%
%:%3067=1482%:%
%:%3068=1482%:%
%:%3069=1482%:%
%:%3070=1483%:%
%:%3071=1483%:%
%:%3072=1484%:%
%:%3073=1484%:%
%:%3074=1484%:%
%:%3075=1485%:%
%:%3076=1485%:%
%:%3077=1486%:%
%:%3078=1486%:%
%:%3079=1486%:%
%:%3080=1487%:%
%:%3081=1488%:%
%:%3082=1488%:%
%:%3083=1489%:%
%:%3084=1489%:%
%:%3085=1489%:%
%:%3086=1490%:%
%:%3087=1490%:%
%:%3088=1491%:%
%:%3089=1491%:%
%:%3090=1491%:%
%:%3091=1492%:%
%:%3092=1492%:%
%:%3093=1493%:%
%:%3094=1493%:%
%:%3095=1494%:%
%:%3101=1494%:%
%:%3104=1495%:%
%:%3105=1496%:%
%:%3106=1496%:%
%:%3107=1497%:%
%:%3108=1498%:%
%:%3114=1504%:%
%:%3117=1505%:%
%:%3121=1505%:%
%:%3122=1505%:%
%:%3127=1505%:%
%:%3130=1506%:%
%:%3131=1507%:%
%:%3132=1507%:%
%:%3133=1508%:%
%:%3134=1509%:%
%:%3135=1510%:%
%:%3142=1511%:%
%:%3143=1511%:%
%:%3144=1512%:%
%:%3145=1512%:%
%:%3146=1513%:%
%:%3147=1513%:%
%:%3148=1514%:%
%:%3149=1515%:%
%:%3150=1515%:%
%:%3151=1515%:%
%:%3152=1516%:%
%:%3153=1516%:%
%:%3154=1517%:%
%:%3155=1517%:%
%:%3156=1518%:%
%:%3157=1519%:%
%:%3158=1519%:%
%:%3159=1519%:%
%:%3160=1520%:%
%:%3161=1520%:%
%:%3162=1521%:%
%:%3163=1521%:%
%:%3164=1522%:%
%:%3165=1522%:%
%:%3168=1525%:%
%:%3169=1526%:%
%:%3170=1526%:%
%:%3171=1526%:%
%:%3172=1527%:%
%:%3173=1527%:%
%:%3174=1528%:%
%:%3175=1528%:%
%:%3176=1529%:%
%:%3177=1529%:%
%:%3182=1534%:%
%:%3183=1535%:%
%:%3184=1535%:%
%:%3185=1535%:%
%:%3186=1536%:%
%:%3187=1536%:%
%:%3188=1537%:%
%:%3189=1537%:%
%:%3190=1538%:%
%:%3191=1538%:%
%:%3196=1543%:%
%:%3197=1544%:%
%:%3198=1544%:%
%:%3199=1544%:%
%:%3200=1545%:%
%:%3201=1545%:%
%:%3202=1546%:%
%:%3203=1546%:%
%:%3204=1547%:%
%:%3205=1547%:%
%:%3210=1552%:%
%:%3211=1553%:%
%:%3212=1553%:%
%:%3213=1553%:%
%:%3214=1554%:%
%:%3224=1556%:%
%:%3225=1557%:%
%:%3226=1558%:%
%:%3227=1559%:%
%:%3228=1560%:%
%:%3229=1561%:%
%:%3230=1562%:%
%:%3231=1563%:%
%:%3232=1564%:%
%:%3233=1565%:%
%:%3234=1566%:%
%:%3235=1567%:%
%:%3236=1568%:%
%:%3237=1569%:%
%:%3238=1570%:%
%:%3239=1571%:%
%:%3240=1572%:%
%:%3241=1573%:%
%:%3242=1574%:%
%:%3243=1575%:%
%:%3244=1576%:%
%:%3245=1577%:%
%:%3246=1578%:%
%:%3247=1579%:%
%:%3248=1580%:%
%:%3249=1581%:%
%:%3250=1582%:%
%:%3251=1583%:%
%:%3252=1584%:%
%:%3253=1585%:%
%:%3255=1587%:%
%:%3256=1587%:%
%:%3257=1588%:%
%:%3258=1589%:%
%:%3265=1590%:%
%:%3266=1590%:%
%:%3267=1591%:%
%:%3268=1591%:%
%:%3269=1592%:%
%:%3270=1592%:%
%:%3271=1593%:%
%:%3272=1593%:%
%:%3273=1594%:%
%:%3274=1594%:%
%:%3275=1595%:%
%:%3276=1596%:%
%:%3277=1596%:%
%:%3278=1596%:%
%:%3279=1597%:%
%:%3280=1597%:%
%:%3281=1598%:%
%:%3282=1598%:%
%:%3283=1599%:%
%:%3284=1599%:%
%:%3285=1600%:%
%:%3286=1600%:%
%:%3287=1601%:%
%:%3288=1601%:%
%:%3289=1602%:%
%:%3299=1604%:%
%:%3300=1605%:%
%:%3302=1607%:%
%:%3303=1607%:%
%:%3304=1608%:%
%:%3305=1609%:%
%:%3312=1610%:%
%:%3313=1610%:%
%:%3314=1611%:%
%:%3315=1611%:%
%:%3316=1612%:%
%:%3317=1612%:%
%:%3318=1612%:%
%:%3319=1613%:%
%:%3320=1613%:%
%:%3321=1613%:%
%:%3322=1614%:%
%:%3323=1614%:%
%:%3324=1615%:%
%:%3325=1615%:%
%:%3326=1616%:%
%:%3327=1616%:%
%:%3328=1617%:%
%:%3334=1617%:%
%:%3337=1618%:%
%:%3338=1619%:%
%:%3339=1619%:%
%:%3340=1620%:%
%:%3341=1621%:%
%:%3344=1622%:%
%:%3348=1622%:%
%:%3349=1622%:%
%:%3350=1622%:%

%
\begin{isabellebody}%
\setisabellecontext{Notunif}%
%
\isadelimtheory
%
\endisadelimtheory
%
\isatagtheory
%
\endisatagtheory
{\isafoldtheory}%
%
\isadelimtheory
%
\endisadelimtheory
%
\begin{isamarkuptext}%
\comentario{Localización de sello.png.}
\comentario{Cambiar los directores}
\comentario{Introducción. Mirar fitting p. 53 y 54}%
\end{isamarkuptext}\isamarkuptrue%
%
\begin{isamarkuptext}%
En esta sección introduciremos la notación uniforme inicialmente 
  desarrollada por \isa{R{\isachardot}\ M{\isachardot}\ Smullyan} (añadir referencia bibliográfica). La finalidad
  de dicha notación es reducir el número de casos a considerar sobre la estructura de 
  las fórmulas al clasificar éstas en dos categorías, facilitando las demostraciones
  y métodos empleados en adelante.

  \comentario{Añadir referencia bibliográfica.}

  De este modo, las fórmulas proposicionales pueden ser de dos tipos: aquellas que 
  de tipo conjuntivo (las fórmulas \isa{{\isasymalpha}}) y las de tipo disyuntivo (las fórmulas \isa{{\isasymbeta}}). 
  Cada fórmula de tipo \isa{{\isasymalpha}}, o \isa{{\isasymbeta}} respectivamente, tiene asociada sus  
  dos componentes \isa{{\isasymalpha}\isactrlsub {\isadigit{1}}} y \isa{{\isasymalpha}\isactrlsub {\isadigit{2}}}, o \isa{{\isasymbeta}\isactrlsub {\isadigit{1}}} y \isa{{\isasymbeta}\isactrlsub {\isadigit{2}}} respectivamente. Para justificar dicha clasificación,
  introduzcamos inicialmente la definición de fórmulas semánticamente equivalentes.

  \begin{definicion}
    Dos fórmulas son \isa{semánticamente\ equivalentes} si tienen el mismo valor para toda 
    interpretación.
  \end{definicion}

  En Isabelle podemos formalizar la definición de la siguiente manera.%
\end{isamarkuptext}\isamarkuptrue%
\isacommand{definition}\isamarkupfalse%
\ {\isachardoublequoteopen}semanticEq\ F\ G\ {\isasymequiv}\ {\isasymforall}{\isasymA}{\isachardot}\ {\isacharparenleft}{\isasymA}\ {\isasymTurnstile}\ F{\isacharparenright}\ {\isasymlongleftrightarrow}\ {\isacharparenleft}{\isasymA}\ {\isasymTurnstile}\ G{\isacharparenright}{\isachardoublequoteclose}%
\begin{isamarkuptext}%
Según la definición del valor de verdad de una fórmula proposicional en una 
  interpretación dada, podemos ver los siguientes ejemplos de fórmulas semánticamente equivalentes.%
\end{isamarkuptext}\isamarkuptrue%
\isacommand{lemma}\isamarkupfalse%
\ {\isachardoublequoteopen}semanticEq\ {\isacharparenleft}Atom\ p{\isacharparenright}\ {\isacharparenleft}{\isacharparenleft}Atom\ p{\isacharparenright}\ \isactrlbold {\isasymor}\ {\isacharparenleft}Atom\ p{\isacharparenright}{\isacharparenright}{\isachardoublequoteclose}\ \isanewline
%
\isadelimproof
\ \ %
\endisadelimproof
%
\isatagproof
\isacommand{by}\isamarkupfalse%
\ {\isacharparenleft}simp\ add{\isacharcolon}\ semanticEq{\isacharunderscore}def{\isacharparenright}%
\endisatagproof
{\isafoldproof}%
%
\isadelimproof
\isanewline
%
\endisadelimproof
\isanewline
\isacommand{lemma}\isamarkupfalse%
\ {\isachardoublequoteopen}semanticEq\ {\isacharparenleft}Atom\ p{\isacharparenright}\ {\isacharparenleft}{\isacharparenleft}Atom\ p{\isacharparenright}\ \isactrlbold {\isasymand}\ {\isacharparenleft}Atom\ p{\isacharparenright}{\isacharparenright}{\isachardoublequoteclose}\ \isanewline
%
\isadelimproof
\ \ %
\endisadelimproof
%
\isatagproof
\isacommand{by}\isamarkupfalse%
\ {\isacharparenleft}simp\ add{\isacharcolon}\ semanticEq{\isacharunderscore}def{\isacharparenright}%
\endisatagproof
{\isafoldproof}%
%
\isadelimproof
\isanewline
%
\endisadelimproof
\isanewline
\isacommand{lemma}\isamarkupfalse%
\ {\isachardoublequoteopen}semanticEq\ {\isasymbottom}\ {\isacharparenleft}{\isasymbottom}\ \isactrlbold {\isasymand}\ {\isasymbottom}{\isacharparenright}{\isachardoublequoteclose}\ \isanewline
%
\isadelimproof
\ \ %
\endisadelimproof
%
\isatagproof
\isacommand{by}\isamarkupfalse%
\ {\isacharparenleft}simp\ add{\isacharcolon}\ semanticEq{\isacharunderscore}def{\isacharparenright}%
\endisatagproof
{\isafoldproof}%
%
\isadelimproof
\isanewline
%
\endisadelimproof
\isanewline
\isacommand{lemma}\isamarkupfalse%
\ {\isachardoublequoteopen}semanticEq\ {\isasymbottom}\ {\isacharparenleft}{\isasymbottom}\ \isactrlbold {\isasymor}\ {\isasymbottom}{\isacharparenright}{\isachardoublequoteclose}\ \isanewline
%
\isadelimproof
\ \ %
\endisadelimproof
%
\isatagproof
\isacommand{by}\isamarkupfalse%
\ {\isacharparenleft}simp\ add{\isacharcolon}\ semanticEq{\isacharunderscore}def{\isacharparenright}%
\endisatagproof
{\isafoldproof}%
%
\isadelimproof
\isanewline
%
\endisadelimproof
\isanewline
\isacommand{lemma}\isamarkupfalse%
\ {\isachardoublequoteopen}semanticEq\ {\isasymbottom}\ {\isacharparenleft}\isactrlbold {\isasymnot}\ {\isasymtop}{\isacharparenright}{\isachardoublequoteclose}\isanewline
%
\isadelimproof
\ \ %
\endisadelimproof
%
\isatagproof
\isacommand{by}\isamarkupfalse%
\ {\isacharparenleft}simp\ add{\isacharcolon}\ semanticEq{\isacharunderscore}def\ top{\isacharunderscore}semantics{\isacharparenright}%
\endisatagproof
{\isafoldproof}%
%
\isadelimproof
\isanewline
%
\endisadelimproof
\isanewline
\isacommand{lemma}\isamarkupfalse%
\ {\isachardoublequoteopen}semanticEq\ F\ {\isacharparenleft}\isactrlbold {\isasymnot}{\isacharparenleft}\isactrlbold {\isasymnot}\ F{\isacharparenright}{\isacharparenright}{\isachardoublequoteclose}\isanewline
%
\isadelimproof
\ \ %
\endisadelimproof
%
\isatagproof
\isacommand{by}\isamarkupfalse%
\ {\isacharparenleft}simp\ add{\isacharcolon}\ semanticEq{\isacharunderscore}def{\isacharparenright}%
\endisatagproof
{\isafoldproof}%
%
\isadelimproof
\isanewline
%
\endisadelimproof
\isanewline
\isacommand{lemma}\isamarkupfalse%
\ {\isachardoublequoteopen}semanticEq\ {\isacharparenleft}\isactrlbold {\isasymnot}{\isacharparenleft}\isactrlbold {\isasymnot}\ F{\isacharparenright}{\isacharparenright}\ {\isacharparenleft}F\ \isactrlbold {\isasymor}\ F{\isacharparenright}{\isachardoublequoteclose}\isanewline
%
\isadelimproof
\ \ %
\endisadelimproof
%
\isatagproof
\isacommand{by}\isamarkupfalse%
\ {\isacharparenleft}simp\ add{\isacharcolon}\ semanticEq{\isacharunderscore}def{\isacharparenright}%
\endisatagproof
{\isafoldproof}%
%
\isadelimproof
\isanewline
%
\endisadelimproof
\isanewline
\isacommand{lemma}\isamarkupfalse%
\ {\isachardoublequoteopen}semanticEq\ {\isacharparenleft}\isactrlbold {\isasymnot}{\isacharparenleft}\isactrlbold {\isasymnot}\ F{\isacharparenright}{\isacharparenright}\ {\isacharparenleft}F\ \isactrlbold {\isasymand}\ F{\isacharparenright}{\isachardoublequoteclose}\isanewline
%
\isadelimproof
\ \ %
\endisadelimproof
%
\isatagproof
\isacommand{by}\isamarkupfalse%
\ {\isacharparenleft}simp\ add{\isacharcolon}\ semanticEq{\isacharunderscore}def{\isacharparenright}%
\endisatagproof
{\isafoldproof}%
%
\isadelimproof
\isanewline
%
\endisadelimproof
\isanewline
\isacommand{lemma}\isamarkupfalse%
\ {\isachardoublequoteopen}semanticEq\ {\isacharparenleft}\isactrlbold {\isasymnot}\ F\ \isactrlbold {\isasymand}\ \isactrlbold {\isasymnot}\ G{\isacharparenright}\ {\isacharparenleft}\isactrlbold {\isasymnot}{\isacharparenleft}F\ \isactrlbold {\isasymor}\ G{\isacharparenright}{\isacharparenright}{\isachardoublequoteclose}\isanewline
%
\isadelimproof
\ \ %
\endisadelimproof
%
\isatagproof
\isacommand{by}\isamarkupfalse%
\ {\isacharparenleft}simp\ add{\isacharcolon}\ semanticEq{\isacharunderscore}def{\isacharparenright}%
\endisatagproof
{\isafoldproof}%
%
\isadelimproof
\isanewline
%
\endisadelimproof
\isanewline
\isacommand{lemma}\isamarkupfalse%
\ {\isachardoublequoteopen}semanticEq\ {\isacharparenleft}F\ \isactrlbold {\isasymrightarrow}\ G{\isacharparenright}\ {\isacharparenleft}\isactrlbold {\isasymnot}\ F\ \isactrlbold {\isasymor}\ G{\isacharparenright}{\isachardoublequoteclose}\isanewline
%
\isadelimproof
\ \ %
\endisadelimproof
%
\isatagproof
\isacommand{by}\isamarkupfalse%
\ {\isacharparenleft}simp\ add{\isacharcolon}\ semanticEq{\isacharunderscore}def{\isacharparenright}%
\endisatagproof
{\isafoldproof}%
%
\isadelimproof
%
\endisadelimproof
%
\begin{isamarkuptext}%
En contraposición, también podemos dar ejemplos de fórmulas que no son semánticamente 
  equivalentes.%
\end{isamarkuptext}\isamarkuptrue%
\isacommand{lemma}\isamarkupfalse%
\ {\isachardoublequoteopen}{\isasymnot}\ semanticEq\ {\isacharparenleft}Atom\ p{\isacharparenright}\ {\isacharparenleft}\isactrlbold {\isasymnot}{\isacharparenleft}Atom\ p{\isacharparenright}{\isacharparenright}{\isachardoublequoteclose}\isanewline
%
\isadelimproof
\ \ %
\endisadelimproof
%
\isatagproof
\isacommand{by}\isamarkupfalse%
\ {\isacharparenleft}simp\ add{\isacharcolon}\ semanticEq{\isacharunderscore}def{\isacharparenright}%
\endisatagproof
{\isafoldproof}%
%
\isadelimproof
\isanewline
%
\endisadelimproof
\isanewline
\isacommand{lemma}\isamarkupfalse%
\ {\isachardoublequoteopen}{\isasymnot}\ semanticEq\ {\isasymbottom}\ {\isasymtop}{\isachardoublequoteclose}\isanewline
%
\isadelimproof
\ \ %
\endisadelimproof
%
\isatagproof
\isacommand{by}\isamarkupfalse%
\ {\isacharparenleft}simp\ add{\isacharcolon}\ semanticEq{\isacharunderscore}def\ top{\isacharunderscore}semantics{\isacharparenright}%
\endisatagproof
{\isafoldproof}%
%
\isadelimproof
%
\endisadelimproof
%
\begin{isamarkuptext}%
Por tanto, diremos intuitivamente que una fórmula es de tipo \isa{{\isasymalpha}} con componentes \isa{{\isasymalpha}\isactrlsub {\isadigit{1}}} y \isa{{\isasymalpha}\isactrlsub {\isadigit{2}}}
  si es semánticamente equivalente a la fórmula \isa{{\isasymalpha}\isactrlsub {\isadigit{1}}\ {\isasymand}\ {\isasymalpha}\isactrlsub {\isadigit{2}}}. Del mismo modo, una fórmula será de tipo
  \isa{{\isasymbeta}} con componentes \isa{{\isasymbeta}\isactrlsub {\isadigit{1}}} y \isa{{\isasymbeta}\isactrlsub {\isadigit{2}}} si es semánticamente equivalente a la fórmula \isa{{\isasymbeta}\isactrlsub {\isadigit{1}}\ {\isasymor}\ {\isasymbeta}\isactrlsub {\isadigit{2}}}.

  \begin{definicion}
    Las fórmulas de tipo \isa{{\isasymalpha}} (\isa{fórmulas\ conjuntivas}) y sus correspondientes componentes
    \isa{{\isasymalpha}\isactrlsub {\isadigit{1}}} y \isa{{\isasymalpha}\isactrlsub {\isadigit{2}}} se definen como sigue: dadas \isa{F} y \isa{G} fórmulas cualesquiera,
    \begin{enumerate}
      \item \isa{F\ {\isasymand}\ G} es una fórmula de tipo \isa{{\isasymalpha}} cuyas componentes son \isa{F} y \isa{G}.
      \item \isa{{\isasymnot}{\isacharparenleft}F\ {\isasymor}\ G{\isacharparenright}} es una fórmula de tipo \isa{{\isasymalpha}} cuyas componentes son \isa{{\isasymnot}\ F} y \isa{{\isasymnot}\ G}.
      \item \isa{{\isasymnot}{\isacharparenleft}F\ {\isasymlongrightarrow}\ G{\isacharparenright}} es una fórmula de tipo \isa{{\isasymalpha}} cuyas componentes son \isa{F} y \isa{{\isasymnot}\ G}.
    \end{enumerate} 
  \end{definicion}

  De este modo, de los ejemplos anteriores podemos deducir que las fórmulas atómicas son de tipo \isa{{\isasymalpha}}
  y sus componentes \isa{{\isasymalpha}\isactrlsub {\isadigit{1}}} y \isa{{\isasymalpha}\isactrlsub {\isadigit{2}}} son la propia fórmula. Del mismo modo, la constante \isa{{\isasymbottom}} también es 
  una fórmula conjuntiva cuyas componentes son ella misma. Por último, podemos observar que dada
  una fórmula cualquiera \isa{F}, su doble negación \isa{{\isasymnot}{\isacharparenleft}{\isasymnot}\ F{\isacharparenright}} es una fórmula de tipo \isa{{\isasymalpha}} y componentes
  \isa{F} y \isa{F}.

  Formalizaremos en Isabelle el conjunto de fórmulas \isa{{\isasymalpha}} como un predicato inductivo. De este modo,
  las reglas anteriores que construyen el conjunto de fórmulas de tipo \isa{{\isasymalpha}} se formalizan en Isabelle 
  como reglas de introducción. Además, añadiremos explícitamente una cuarta regla que introduce la 
  doble negación de una fórmula como fórmula de tipo \isa{{\isasymalpha}}. De este modo, facilitaremos la prueba de 
  resultados posteriores relacionados con la definición de conjunto de Hintikka, que constituyen una
  base para la demostración del \isa{teorema\ de\ existencia\ de\ modelo}.%
\end{isamarkuptext}\isamarkuptrue%
\isacommand{inductive}\isamarkupfalse%
\ Con\ {\isacharcolon}{\isacharcolon}\ {\isachardoublequoteopen}{\isacharprime}a\ formula\ {\isacharequal}{\isachargreater}\ {\isacharprime}a\ formula\ {\isacharequal}{\isachargreater}\ {\isacharprime}a\ formula\ {\isacharequal}{\isachargreater}\ bool{\isachardoublequoteclose}\ \isakeyword{where}\isanewline
{\isachardoublequoteopen}Con\ {\isacharparenleft}And\ F\ G{\isacharparenright}\ F\ G{\isachardoublequoteclose}\ {\isacharbar}\isanewline
{\isachardoublequoteopen}Con\ {\isacharparenleft}Not\ {\isacharparenleft}Or\ F\ G{\isacharparenright}{\isacharparenright}\ {\isacharparenleft}Not\ F{\isacharparenright}\ {\isacharparenleft}Not\ G{\isacharparenright}{\isachardoublequoteclose}\ {\isacharbar}\isanewline
{\isachardoublequoteopen}Con\ {\isacharparenleft}Not\ {\isacharparenleft}Imp\ F\ G{\isacharparenright}{\isacharparenright}\ F\ {\isacharparenleft}Not\ G{\isacharparenright}{\isachardoublequoteclose}\ {\isacharbar}\isanewline
{\isachardoublequoteopen}Con\ {\isacharparenleft}Not\ {\isacharparenleft}Not\ F{\isacharparenright}{\isacharparenright}\ F\ F{\isachardoublequoteclose}%
\begin{isamarkuptext}%
Las reglas de introducción que proporciona la definición anterior son
  las siguientes.

  \begin{itemize}
    \item[] \isa{Con\ {\isacharparenleft}F\ \isactrlbold {\isasymand}\ G{\isacharparenright}\ F\ G\isasep\isanewline%
Con\ {\isacharparenleft}\isactrlbold {\isasymnot}\ {\isacharparenleft}F\ \isactrlbold {\isasymor}\ G{\isacharparenright}{\isacharparenright}\ {\isacharparenleft}\isactrlbold {\isasymnot}\ F{\isacharparenright}\ {\isacharparenleft}\isactrlbold {\isasymnot}\ G{\isacharparenright}\isasep\isanewline%
Con\ {\isacharparenleft}\isactrlbold {\isasymnot}\ {\isacharparenleft}F\ \isactrlbold {\isasymrightarrow}\ G{\isacharparenright}{\isacharparenright}\ F\ {\isacharparenleft}\isactrlbold {\isasymnot}\ G{\isacharparenright}\isasep\isanewline%
Con\ {\isacharparenleft}\isactrlbold {\isasymnot}\ {\isacharparenleft}\isactrlbold {\isasymnot}\ F{\isacharparenright}{\isacharparenright}\ F\ F} 
      \hfill (\isa{Con{\isachardot}intros})
  \end{itemize}
  
  Por otro lado, definamos las fórmulas disyuntivas.

  \begin{definicion}
    Las fórmulas de tipo \isa{{\isasymbeta}} (\isa{fórmulas\ disyuntivas}) y sus correspondientes componentes
    \isa{{\isasymbeta}\isactrlsub {\isadigit{1}}} y \isa{{\isasymbeta}\isactrlsub {\isadigit{2}}} se definen como sigue: dadas \isa{F} y \isa{G} fórmulas cualesquiera,
    \begin{enumerate}
      \item \isa{F\ {\isasymor}\ G} es una fórmula de tipo \isa{{\isasymbeta}} cuyas componentes son \isa{F} y \isa{G}.
      \item \isa{F\ {\isasymlongrightarrow}\ G} es una fórmula de tipo \isa{{\isasymbeta}} cuyas componentes son \isa{{\isasymnot}\ F} y \isa{G}.
      \item \isa{{\isasymnot}{\isacharparenleft}F\ {\isasymand}\ G{\isacharparenright}} es una fórmula de tipo \isa{{\isasymbeta}} cuyas componentes son \isa{{\isasymnot}\ F} y \isa{{\isasymnot}\ G}.
    \end{enumerate} 
  \end{definicion}

  De los ejemplos dados anteriormente, podemos deducir análogamente que las fórmulas atómicas, la
  constante \isa{{\isasymbottom}} y la doble negación sob también fórmulas disyuntivas con las mismas componentes que
  las dadas para el tipo conjuntivo.

  Del mismo modo, su formalización se realiza como un predicado inductivo, de manera que las reglas 
  que definen el conjunto de fórmulas de tipo \isa{{\isasymbeta}} se formalizan en Isabelle como reglas de 
  introducción. Análogamente, introduciremos de manera explícita una regla que señala que la doble 
  negación de una fórmula es una fórmula de tipo disyuntivo.%
\end{isamarkuptext}\isamarkuptrue%
\isacommand{inductive}\isamarkupfalse%
\ Dis\ {\isacharcolon}{\isacharcolon}\ {\isachardoublequoteopen}{\isacharprime}a\ formula\ {\isacharequal}{\isachargreater}\ {\isacharprime}a\ formula\ {\isacharequal}{\isachargreater}\ {\isacharprime}a\ formula\ {\isacharequal}{\isachargreater}\ bool{\isachardoublequoteclose}\ \isakeyword{where}\isanewline
{\isachardoublequoteopen}Dis\ {\isacharparenleft}Or\ F\ G{\isacharparenright}\ F\ G{\isachardoublequoteclose}\ {\isacharbar}\isanewline
{\isachardoublequoteopen}Dis\ {\isacharparenleft}Imp\ F\ G{\isacharparenright}\ {\isacharparenleft}Not\ F{\isacharparenright}\ G{\isachardoublequoteclose}\ {\isacharbar}\isanewline
{\isachardoublequoteopen}Dis\ {\isacharparenleft}Not\ {\isacharparenleft}And\ F\ G{\isacharparenright}{\isacharparenright}\ {\isacharparenleft}Not\ F{\isacharparenright}\ {\isacharparenleft}Not\ G{\isacharparenright}{\isachardoublequoteclose}\ {\isacharbar}\isanewline
{\isachardoublequoteopen}Dis\ {\isacharparenleft}Not\ {\isacharparenleft}Not\ F{\isacharparenright}{\isacharparenright}\ F\ F{\isachardoublequoteclose}%
\begin{isamarkuptext}%
Las reglas de introducción que proporciona esta formalización se muestran a continuación.

  \begin{itemize}
    \item[] \isa{Dis\ {\isacharparenleft}F\ \isactrlbold {\isasymor}\ G{\isacharparenright}\ F\ G\isasep\isanewline%
Dis\ {\isacharparenleft}F\ \isactrlbold {\isasymrightarrow}\ G{\isacharparenright}\ {\isacharparenleft}\isactrlbold {\isasymnot}\ F{\isacharparenright}\ G\isasep\isanewline%
Dis\ {\isacharparenleft}\isactrlbold {\isasymnot}\ {\isacharparenleft}F\ \isactrlbold {\isasymand}\ G{\isacharparenright}{\isacharparenright}\ {\isacharparenleft}\isactrlbold {\isasymnot}\ F{\isacharparenright}\ {\isacharparenleft}\isactrlbold {\isasymnot}\ G{\isacharparenright}\isasep\isanewline%
Dis\ {\isacharparenleft}\isactrlbold {\isasymnot}\ {\isacharparenleft}\isactrlbold {\isasymnot}\ F{\isacharparenright}{\isacharparenright}\ F\ F} 
      \hfill (\isa{Dis{\isachardot}intros})
  \end{itemize}

  Cabe observar que las formalizaciones de la definiciones de fórmulas de tipo \isa{{\isasymalpha}} y \isa{{\isasymbeta}} son 
  definiciones sintácticas, pues construyen los correspondientes conjuntos de fórmulas a partir de 
  una reglas sintácticas concretas. Se trata de una simplificación de la intuición original de la 
  clasificación de las fórmulas mediante notación uniforme, ya que se prescinde de la noción de 
  equivalencia semántica que permite clasificar la totalidad de las fórmulas proposicionales. 

  Veamos la clasificación de casos concretos de fórmulas. Por ejemplo, según hemos definido la 
  fórmula \isa{{\isasymtop}}, es sencillo comprobar que se trata de una fórmula disyuntiva.%
\end{isamarkuptext}\isamarkuptrue%
\isacommand{lemma}\isamarkupfalse%
\ {\isachardoublequoteopen}Dis\ {\isasymtop}\ {\isacharparenleft}\isactrlbold {\isasymnot}\ {\isasymbottom}{\isacharparenright}\ {\isasymbottom}{\isachardoublequoteclose}\ \isanewline
%
\isadelimproof
\ \ %
\endisadelimproof
%
\isatagproof
\isacommand{unfolding}\isamarkupfalse%
\ Top{\isacharunderscore}def\ \isacommand{by}\isamarkupfalse%
\ {\isacharparenleft}simp\ only{\isacharcolon}\ Dis{\isachardot}intros{\isacharparenleft}{\isadigit{2}}{\isacharparenright}{\isacharparenright}%
\endisatagproof
{\isafoldproof}%
%
\isadelimproof
%
\endisadelimproof
%
\begin{isamarkuptext}%
Por otro lado, se observa a partir de las correspondientes definiciones que la conjunción
  generalizada de una lista de fórmulas es una fórmula de tipo \isa{{\isasymalpha}} y la disyunción generalizada de
  una lista de fórmulas es una fórmula de tipo \isa{{\isasymbeta}}.%
\end{isamarkuptext}\isamarkuptrue%
\isacommand{lemma}\isamarkupfalse%
\ {\isachardoublequoteopen}Con\ {\isacharparenleft}\isactrlbold {\isasymAnd}{\isacharparenleft}F{\isacharhash}Fs{\isacharparenright}{\isacharparenright}\ F\ {\isacharparenleft}\isactrlbold {\isasymAnd}Fs{\isacharparenright}{\isachardoublequoteclose}\isanewline
%
\isadelimproof
\ \ %
\endisadelimproof
%
\isatagproof
\isacommand{by}\isamarkupfalse%
\ {\isacharparenleft}simp\ only{\isacharcolon}\ BigAnd{\isachardot}simps\ Con{\isachardot}intros{\isacharparenleft}{\isadigit{1}}{\isacharparenright}{\isacharparenright}%
\endisatagproof
{\isafoldproof}%
%
\isadelimproof
\isanewline
%
\endisadelimproof
\isanewline
\isacommand{lemma}\isamarkupfalse%
\ {\isachardoublequoteopen}Dis\ {\isacharparenleft}\isactrlbold {\isasymOr}{\isacharparenleft}F{\isacharhash}Fs{\isacharparenright}{\isacharparenright}\ F\ {\isacharparenleft}\isactrlbold {\isasymOr}Fs{\isacharparenright}{\isachardoublequoteclose}\isanewline
%
\isadelimproof
\ \ %
\endisadelimproof
%
\isatagproof
\isacommand{by}\isamarkupfalse%
\ {\isacharparenleft}simp\ only{\isacharcolon}\ BigOr{\isachardot}simps\ Dis{\isachardot}intros{\isacharparenleft}{\isadigit{1}}{\isacharparenright}{\isacharparenright}%
\endisatagproof
{\isafoldproof}%
%
\isadelimproof
%
\endisadelimproof
%
\begin{isamarkuptext}%
Finalmente, de las reglas que definen las fórmulas conjuntivas y disyuntivas se deduce que
  la doble negación de una fórmula es una fórmula perteneciente a ambos tipos.%
\end{isamarkuptext}\isamarkuptrue%
\isacommand{lemma}\isamarkupfalse%
\ notDisCon{\isacharcolon}\ {\isachardoublequoteopen}Con\ {\isacharparenleft}Not\ {\isacharparenleft}Not\ F{\isacharparenright}{\isacharparenright}\ F\ F{\isachardoublequoteclose}\ {\isachardoublequoteopen}Dis\ {\isacharparenleft}Not\ {\isacharparenleft}Not\ F{\isacharparenright}{\isacharparenright}\ F\ F{\isachardoublequoteclose}\ \isanewline
%
\isadelimproof
\ \ %
\endisadelimproof
%
\isatagproof
\isacommand{by}\isamarkupfalse%
\ {\isacharparenleft}simp\ only{\isacharcolon}\ Con{\isachardot}intros{\isacharparenleft}{\isadigit{4}}{\isacharparenright}\ Dis{\isachardot}intros{\isacharparenleft}{\isadigit{4}}{\isacharparenright}{\isacharparenright}{\isacharplus}%
\endisatagproof
{\isafoldproof}%
%
\isadelimproof
%
\endisadelimproof
%
\begin{isamarkuptext}%
A continuación vamos a introducir el siguiente lema que caracteriza las fórmulas de tipo \isa{{\isasymalpha}} 
  y \isa{{\isasymbeta}}, facilitando el uso de la notación uniforme en Isabelle.%
\end{isamarkuptext}\isamarkuptrue%
\isacommand{lemma}\isamarkupfalse%
\ con{\isacharunderscore}dis{\isacharunderscore}simps{\isacharcolon}\isanewline
\ \ {\isachardoublequoteopen}Con\ a{\isadigit{1}}\ a{\isadigit{2}}\ a{\isadigit{3}}\ {\isacharequal}\ {\isacharparenleft}a{\isadigit{1}}\ {\isacharequal}\ a{\isadigit{2}}\ \isactrlbold {\isasymand}\ a{\isadigit{3}}\ {\isasymor}\ \isanewline
\ \ \ \ {\isacharparenleft}{\isasymexists}F\ G{\isachardot}\ a{\isadigit{1}}\ {\isacharequal}\ \isactrlbold {\isasymnot}\ {\isacharparenleft}F\ \isactrlbold {\isasymor}\ G{\isacharparenright}\ {\isasymand}\ a{\isadigit{2}}\ {\isacharequal}\ \isactrlbold {\isasymnot}\ F\ {\isasymand}\ a{\isadigit{3}}\ {\isacharequal}\ \isactrlbold {\isasymnot}\ G{\isacharparenright}\ {\isasymor}\ \isanewline
\ \ \ \ {\isacharparenleft}{\isasymexists}G{\isachardot}\ a{\isadigit{1}}\ {\isacharequal}\ \isactrlbold {\isasymnot}\ {\isacharparenleft}a{\isadigit{2}}\ \isactrlbold {\isasymrightarrow}\ G{\isacharparenright}\ {\isasymand}\ a{\isadigit{3}}\ {\isacharequal}\ \isactrlbold {\isasymnot}\ G{\isacharparenright}\ {\isasymor}\ \isanewline
\ \ \ \ a{\isadigit{1}}\ {\isacharequal}\ \isactrlbold {\isasymnot}\ {\isacharparenleft}\isactrlbold {\isasymnot}\ a{\isadigit{2}}{\isacharparenright}\ {\isasymand}\ a{\isadigit{3}}\ {\isacharequal}\ a{\isadigit{2}}{\isacharparenright}{\isachardoublequoteclose}\isanewline
\ \ {\isachardoublequoteopen}Dis\ a{\isadigit{1}}\ a{\isadigit{2}}\ a{\isadigit{3}}\ {\isacharequal}\ {\isacharparenleft}a{\isadigit{1}}\ {\isacharequal}\ a{\isadigit{2}}\ \isactrlbold {\isasymor}\ a{\isadigit{3}}\ {\isasymor}\ \isanewline
\ \ \ \ {\isacharparenleft}{\isasymexists}F\ G{\isachardot}\ a{\isadigit{1}}\ {\isacharequal}\ F\ \isactrlbold {\isasymrightarrow}\ G\ {\isasymand}\ a{\isadigit{2}}\ {\isacharequal}\ \isactrlbold {\isasymnot}\ F\ {\isasymand}\ a{\isadigit{3}}\ {\isacharequal}\ G{\isacharparenright}\ {\isasymor}\ \isanewline
\ \ \ \ {\isacharparenleft}{\isasymexists}F\ G{\isachardot}\ a{\isadigit{1}}\ {\isacharequal}\ \isactrlbold {\isasymnot}\ {\isacharparenleft}F\ \isactrlbold {\isasymand}\ G{\isacharparenright}\ {\isasymand}\ a{\isadigit{2}}\ {\isacharequal}\ \isactrlbold {\isasymnot}\ F\ {\isasymand}\ a{\isadigit{3}}\ {\isacharequal}\ \isactrlbold {\isasymnot}\ G{\isacharparenright}\ {\isasymor}\ \isanewline
\ \ \ \ a{\isadigit{1}}\ {\isacharequal}\ \isactrlbold {\isasymnot}\ {\isacharparenleft}\isactrlbold {\isasymnot}\ a{\isadigit{2}}{\isacharparenright}\ {\isasymand}\ a{\isadigit{3}}\ {\isacharequal}\ a{\isadigit{2}}{\isacharparenright}{\isachardoublequoteclose}\ \isanewline
%
\isadelimproof
\ \ %
\endisadelimproof
%
\isatagproof
\isacommand{by}\isamarkupfalse%
\ {\isacharparenleft}simp{\isacharunderscore}all\ add{\isacharcolon}\ Con{\isachardot}simps\ Dis{\isachardot}simps{\isacharparenright}%
\endisatagproof
{\isafoldproof}%
%
\isadelimproof
%
\endisadelimproof
%
\begin{isamarkuptext}%
Por último, introduzcamos un resultado que permite caracterizar los conjuntos de Hintikka 
  empleando la notación uniforme.

  \begin{lema}[Caracterización de los conjuntos de Hintikka mediante la notación uniforme]
    Dado un conjunto de fórmulas proposicionales \isa{S}, son equivalentes:
    \begin{enumerate}
      \item \isa{S} es un conjunto de Hintikka.
      \item Se verifican las condiciones siguientes:
      \begin{itemize}
        \item \isa{{\isasymbottom}} no pertenece a \isa{S}.
        \item Dada \isa{p} una fórmula atómica cualquiera, no se tiene 
        simultáneamente que\\ \isa{p\ {\isasymin}\ S} y \isa{{\isasymnot}\ p\ {\isasymin}\ S}.
        \item Para toda fórmula de tipo \isa{{\isasymalpha}} con componentes \isa{{\isasymalpha}\isactrlsub {\isadigit{1}}} y \isa{{\isasymalpha}\isactrlsub {\isadigit{2}}} se verifica 
        que si la fórmula pertenece a \isa{S}, entonces \isa{{\isasymalpha}\isactrlsub {\isadigit{1}}} y \isa{{\isasymalpha}\isactrlsub {\isadigit{2}}} también.
        \item Para toda fórmula de tipo \isa{{\isasymbeta}} con componentes \isa{{\isasymbeta}\isactrlsub {\isadigit{1}}} y \isa{{\isasymbeta}\isactrlsub {\isadigit{2}}} se verifica 
        que si la fórmula pertenece a \isa{S}, entonces o bien \isa{{\isasymbeta}\isactrlsub {\isadigit{1}}} pertenece
        a \isa{S} o bien \isa{{\isasymbeta}\isactrlsub {\isadigit{2}}} pertenece a \isa{S}.
      \end{itemize} 
    \end{enumerate}
  \end{lema}

  En Isabelle/HOL se formaliza del siguiente modo.%
\end{isamarkuptext}\isamarkuptrue%
\isacommand{lemma}\isamarkupfalse%
\ {\isachardoublequoteopen}Hintikka\ S\ {\isacharequal}\ {\isacharparenleft}{\isasymbottom}\ {\isasymnotin}\ S\isanewline
{\isasymand}\ {\isacharparenleft}{\isasymforall}k{\isachardot}\ Atom\ k\ {\isasymin}\ S\ {\isasymlongrightarrow}\ \isactrlbold {\isasymnot}\ {\isacharparenleft}Atom\ k{\isacharparenright}\ {\isasymin}\ S\ {\isasymlongrightarrow}\ False{\isacharparenright}\isanewline
{\isasymand}\ {\isacharparenleft}{\isasymforall}F\ G\ H{\isachardot}\ Con\ F\ G\ H\ {\isasymlongrightarrow}\ F\ {\isasymin}\ S\ {\isasymlongrightarrow}\ G\ {\isasymin}\ S\ {\isasymand}\ H\ {\isasymin}\ S{\isacharparenright}\isanewline
{\isasymand}\ {\isacharparenleft}{\isasymforall}F\ G\ H{\isachardot}\ Dis\ F\ G\ H\ {\isasymlongrightarrow}\ F\ {\isasymin}\ S\ {\isasymlongrightarrow}\ G\ {\isasymin}\ S\ {\isasymor}\ H\ {\isasymin}\ S{\isacharparenright}{\isacharparenright}{\isachardoublequoteclose}\ \isanewline
%
\isadelimproof
\ \ %
\endisadelimproof
%
\isatagproof
\isacommand{oops}\isamarkupfalse%
%
\endisatagproof
{\isafoldproof}%
%
\isadelimproof
%
\endisadelimproof
%
\begin{isamarkuptext}%
Procedamos a la demostración del resultado.

\begin{demostracion}
  Para probar la equivalencia, veamos cada una de las implicaciones por separado.

\textbf{\isa{{\isadigit{1}}{\isacharparenright}\ {\isasymLongrightarrow}\ {\isadigit{2}}{\isacharparenright}}}

  Supongamos que \isa{S} es un conjunto de Hintikka. Vamos a probar que, en efecto, se 
  verifican las condiciones del enunciado del lema.

  Por definición de conjunto de Hintikka, \isa{S} verifica las siguientes condiciones:
  \begin{enumerate}
    \item \isa{{\isasymbottom}\ {\isasymnotin}\ S}.
    \item Dada \isa{p} una fórmula atómica cualquiera, no se tiene 
      simultáneamente que\\ \isa{p\ {\isasymin}\ S} y \isa{{\isasymnot}\ p\ {\isasymin}\ S}.
    \item Si \isa{G\ {\isasymand}\ H\ {\isasymin}\ S}, entonces \isa{G\ {\isasymin}\ S} y \isa{H\ {\isasymin}\ S}.
    \item Si \isa{G\ {\isasymor}\ H\ {\isasymin}\ S}, entonces \isa{G\ {\isasymin}\ S} o \isa{H\ {\isasymin}\ S}.
    \item Si \isa{G\ {\isasymrightarrow}\ H\ {\isasymin}\ S}, entonces \isa{{\isasymnot}\ G\ {\isasymin}\ S} o \isa{H\ {\isasymin}\ S}.
    \item Si \isa{{\isasymnot}{\isacharparenleft}{\isasymnot}\ G{\isacharparenright}\ {\isasymin}\ S}, entonces \isa{G\ {\isasymin}\ S}.
    \item Si \isa{{\isasymnot}{\isacharparenleft}G\ {\isasymand}\ H{\isacharparenright}\ {\isasymin}\ S}, entonces \isa{{\isasymnot}\ G\ {\isasymin}\ S} o \isa{{\isasymnot}\ H\ {\isasymin}\ S}.
    \item Si \isa{{\isasymnot}{\isacharparenleft}G\ {\isasymor}\ H{\isacharparenright}\ {\isasymin}\ S}, entonces \isa{{\isasymnot}\ G\ {\isasymin}\ S} y \isa{{\isasymnot}\ H\ {\isasymin}\ S}. 
    \item Si \isa{{\isasymnot}{\isacharparenleft}G\ {\isasymrightarrow}\ H{\isacharparenright}\ {\isasymin}\ S}, entonces \isa{G\ {\isasymin}\ S} y \isa{{\isasymnot}\ H\ {\isasymin}\ S}. 
  \end{enumerate}  
  De este modo, el conjunto \isa{S} cumple la primera y la segunda condición del
  enunciado del lema, que se corresponden con las dos primeras condiciones
  de la definición de conjunto de Hintikka. Veamos que, además, verifica las
  dos últimas condiciones del resultado.

  En primer lugar, probemos que para toda fórmula de tipo \isa{{\isasymalpha}} con 
  componentes \isa{{\isasymalpha}\isactrlsub {\isadigit{1}}} y \isa{{\isasymalpha}\isactrlsub {\isadigit{2}}} se verifica que si la fórmula pertenece al conjunto 
  \isa{S}, entonces \isa{{\isasymalpha}\isactrlsub {\isadigit{1}}} y \isa{{\isasymalpha}\isactrlsub {\isadigit{2}}} también. Para ello, supongamos que una fórmula 
  cualquiera de tipo \isa{{\isasymalpha}} pertence a \isa{S}. Por definición de este tipo de
  fórmulas, tenemos que \isa{{\isasymalpha}} puede ser de la forma \isa{G\ {\isasymand}\ H}, \isa{{\isasymnot}{\isacharparenleft}{\isasymnot}\ G{\isacharparenright}},\\ \isa{{\isasymnot}{\isacharparenleft}G\ {\isasymor}\ H{\isacharparenright}} 
  o \isa{{\isasymnot}{\isacharparenleft}G\ {\isasymlongrightarrow}\ H{\isacharparenright}} para fórmulas \isa{G} y \isa{H} cualesquiera. Probemos que, para cada
  tipo de fórmula \isa{{\isasymalpha}} perteneciente a \isa{S}, sus componentes \isa{{\isasymalpha}\isactrlsub {\isadigit{1}}} y \isa{{\isasymalpha}\isactrlsub {\isadigit{2}}} están en
  \isa{S}.

  \isa{{\isasymsqdot}\ Fórmula\ del\ tipo\ G\ {\isasymand}\ H}: Sus componentes conjuntivas son \isa{G} y \isa{H}. 
  Por la tercera condición de la definición de conjunto de Hintikka, obtenemos 
  que si \isa{G\ {\isasymand}\ H} pertenece a \isa{S}, entonces \isa{G} y \isa{H} están ambas en el conjunto,
  lo que prueba este caso.
    
  \isa{{\isasymsqdot}\ Fórmula\ del\ tipo\ {\isasymnot}{\isacharparenleft}{\isasymnot}\ G{\isacharparenright}}: Sus componentes conjuntivas son ambas \isa{G}.
  Por la sexta condición de la definición de conjunto de Hintikka, obtenemos que
  si \isa{{\isasymnot}{\isacharparenleft}{\isasymnot}\ G{\isacharparenright}} pertenece a \isa{S}, entonces \isa{G} pertenece al conjunto, lo que prueba
  este caso.

  \isa{{\isasymsqdot}\ Fórmula\ del\ tipo\ {\isasymnot}{\isacharparenleft}G\ {\isasymor}\ H{\isacharparenright}}: Sus componentes conjuntivas son \isa{{\isasymnot}\ G} y \isa{{\isasymnot}\ H}. 
  Por la octava condición de la definición de conjunto de Hintikka, obtenemos 
  que si \isa{{\isasymnot}{\isacharparenleft}G\ {\isasymor}\ H{\isacharparenright}} pertenece a \isa{S}, entonces \isa{{\isasymnot}\ G} y \isa{{\isasymnot}\ H} están ambas en el conjunto,
  lo que prueba este caso.

  \isa{{\isasymsqdot}\ Fórmula\ del\ tipo\ {\isasymnot}{\isacharparenleft}G\ {\isasymlongrightarrow}\ H{\isacharparenright}}: Sus componentes conjuntivas son \isa{G} y \isa{{\isasymnot}\ H}. 
  Por la novena condición de la definición de conjunto de Hintikka, obtenemos 
  que si\\ \isa{{\isasymnot}{\isacharparenleft}G\ {\isasymlongrightarrow}\ H{\isacharparenright}} pertenece a \isa{S}, entonces \isa{G} y \isa{{\isasymnot}\ H} están ambas en el conjunto,
  lo que prueba este caso.

  Finalmente, probemos que para toda fórmula de tipo \isa{{\isasymbeta}} con componentes \isa{{\isasymbeta}\isactrlsub {\isadigit{1}}} y 
  \isa{{\isasymbeta}\isactrlsub {\isadigit{2}}} se verifica que si la fórmula pertenece al conjunto \isa{S}, entonces o bien \isa{{\isasymbeta}\isactrlsub {\isadigit{1}}} 
  pertenece al conjunto o bien \isa{{\isasymbeta}\isactrlsub {\isadigit{2}}} pertenece a conjunto. Para ello, supongamos que 
  una fórmula cualquiera de tipo \isa{{\isasymbeta}} pertence a \isa{S}. Por definición de este tipo de
  fórmulas, tenemos que \isa{{\isasymbeta}} puede ser de la forma \isa{G\ {\isasymor}\ H}, \isa{G\ {\isasymlongrightarrow}\ H}, \isa{{\isasymnot}{\isacharparenleft}{\isasymnot}\ G{\isacharparenright}} 
  o \isa{{\isasymnot}{\isacharparenleft}G\ {\isasymand}\ H{\isacharparenright}} para fórmulas \isa{G} y \isa{H} cualesquiera. Probemos que, para cada
  tipo de fórmula \isa{{\isasymbeta}} perteneciente a \isa{S}, o bien su componente \isa{{\isasymbeta}\isactrlsub {\isadigit{1}}} pertenece a \isa{S} 
  o bien su componente \isa{{\isasymbeta}\isactrlsub {\isadigit{2}}} pertenece a \isa{S}.

  \isa{{\isasymsqdot}\ Fórmula\ del\ tipo\ G\ {\isasymor}\ H}: Sus componentes disyuntivas son \isa{G} y \isa{H}. 
    Por la cuarta condición de la definición de conjunto de Hintikka, obtenemos 
    que si \isa{G\ {\isasymor}\ H} pertenece a \isa{S}, entonces o bien \isa{G} está en \isa{S} o bien \isa{H} está
    en \isa{S}, lo que prueba este caso.

  \isa{{\isasymsqdot}\ Fórmula\ del\ tipo\ G\ {\isasymlongrightarrow}\ H}: Sus componentes disyuntivas son \isa{{\isasymnot}\ G} y \isa{H}.
    Por la quinta condición de la definición de conjunto de Hintikka, obtenemos que
    si \isa{G\ {\isasymlongrightarrow}\ H} pertenece a \isa{S}, entonces o bien \isa{{\isasymnot}\ G} pertenece al conjunto o bien
    \isa{H} pertenece al conjunto, lo que prueba este caso.

  \isa{{\isasymsqdot}\ Fórmula\ del\ tipo\ {\isasymnot}{\isacharparenleft}{\isasymnot}\ G{\isacharparenright}}: Sus componentes conjuntivas son ambas \isa{G}.
    Por la sexta condición de la definición de conjunto de Hintikka, obtenemos 
    que si \isa{{\isasymnot}{\isacharparenleft}{\isasymnot}\ G{\isacharparenright}} pertenece a \isa{S}, entonces \isa{G} pertenece al conjunto. De este modo,
    por la regla de introducción a la disyunción, se prueba que o bien una de las 
    componentes está en el conjunto o bien lo está la otra pues, en este caso,
    coinciden.

  \isa{{\isasymsqdot}\ Fórmula\ del\ tipo\ {\isasymnot}{\isacharparenleft}G\ {\isasymand}\ H{\isacharparenright}}: Sus componentes conjuntivas son \isa{{\isasymnot}\ G} y \isa{{\isasymnot}\ H}. 
    Por la séptima condición de la definición de conjunto de Hintikka, obtenemos 
    que si\\ \isa{{\isasymnot}{\isacharparenleft}G\ {\isasymand}\ H{\isacharparenright}} pertenece a \isa{S}, entonces o bien \isa{{\isasymnot}\ G} pertenece al conjunto
    o bien \isa{{\isasymnot}\ H} pertenece al conjunto, lo que prueba este caso.

\textbf{\isa{{\isadigit{2}}{\isacharparenright}\ {\isasymLongrightarrow}\ {\isadigit{1}}{\isacharparenright}}}

  Supongamos que se verifican las condiciones del enunciado del lema:

  \begin{itemize}
    \item \isa{{\isasymbottom}} no pertenece a \isa{S}.
    \item Dada \isa{p} una fórmula atómica cualquiera, no se tiene 
    simultáneamente que\\ \isa{p\ {\isasymin}\ S} y \isa{{\isasymnot}\ p\ {\isasymin}\ S}.
    \item Para toda fórmula de tipo \isa{{\isasymalpha}} con componentes \isa{{\isasymalpha}\isactrlsub {\isadigit{1}}} y \isa{{\isasymalpha}\isactrlsub {\isadigit{2}}} se verifica 
    que si la fórmula pertenece a \isa{S}, entonces \isa{{\isasymalpha}\isactrlsub {\isadigit{1}}} y \isa{{\isasymalpha}\isactrlsub {\isadigit{2}}} también.
    \item Para toda fórmula de tipo \isa{{\isasymbeta}} con componentes \isa{{\isasymbeta}\isactrlsub {\isadigit{1}}} y \isa{{\isasymbeta}\isactrlsub {\isadigit{2}}} se verifica 
    que si la fórmula pertenece a \isa{S}, entonces o bien \isa{{\isasymbeta}\isactrlsub {\isadigit{1}}} pertenece
    a \isa{S} o bien \isa{{\isasymbeta}\isactrlsub {\isadigit{2}}} pertenece a \isa{S}.
  \end{itemize}  

  Vamos a probar que \isa{S} es un conjunto de Hintikka.

  Por la definición de conjunto de Hintikka, es suficiente probar las siguientes
  condiciones:

  \begin{enumerate}
    \item \isa{{\isasymbottom}\ {\isasymnotin}\ S}.
    \item Dada \isa{p} una fórmula atómica cualquiera, no se tiene 
      simultáneamente que\\ \isa{p\ {\isasymin}\ S} y \isa{{\isasymnot}\ p\ {\isasymin}\ S}.
    \item Si \isa{G\ {\isasymand}\ H\ {\isasymin}\ S}, entonces \isa{G\ {\isasymin}\ S} y \isa{H\ {\isasymin}\ S}.
    \item Si \isa{G\ {\isasymor}\ H\ {\isasymin}\ S}, entonces \isa{G\ {\isasymin}\ S} o \isa{H\ {\isasymin}\ S}.
    \item Si \isa{G\ {\isasymrightarrow}\ H\ {\isasymin}\ S}, entonces \isa{{\isasymnot}\ G\ {\isasymin}\ S} o \isa{H\ {\isasymin}\ S}.
    \item Si \isa{{\isasymnot}{\isacharparenleft}{\isasymnot}\ G{\isacharparenright}\ {\isasymin}\ S}, entonces \isa{G\ {\isasymin}\ S}.
    \item Si \isa{{\isasymnot}{\isacharparenleft}G\ {\isasymand}\ H{\isacharparenright}\ {\isasymin}\ S}, entonces \isa{{\isasymnot}\ G\ {\isasymin}\ S} o \isa{{\isasymnot}\ H\ {\isasymin}\ S}.
    \item Si \isa{{\isasymnot}{\isacharparenleft}G\ {\isasymor}\ H{\isacharparenright}\ {\isasymin}\ S}, entonces \isa{{\isasymnot}\ G\ {\isasymin}\ S} y \isa{{\isasymnot}\ H\ {\isasymin}\ S}. 
    \item Si \isa{{\isasymnot}{\isacharparenleft}G\ {\isasymrightarrow}\ H{\isacharparenright}\ {\isasymin}\ S}, entonces \isa{G\ {\isasymin}\ S} y \isa{{\isasymnot}\ H\ {\isasymin}\ S}. 
  \end{enumerate} 

  En primer lugar se observa que, por hipótesis, se verifican las dos primeras
  condiciones de la definición de conjunto de Hintikka. Veamos que, en efecto, se
  cumplen las demás.

  \begin{enumerate}
    \item[\isa{{\isadigit{3}}{\isacharparenright}}] Supongamos que \isa{G\ {\isasymand}\ H} está en \isa{S} para fórmulas \isa{G} y \isa{H} cualesquiera.
    Por definición, \isa{G\ {\isasymand}\ H} es una fórmula de tipo \isa{{\isasymalpha}} con componentes \isa{G} y \isa{H}. 
    Por lo tanto, por hipótesis se cumple que \isa{G} y \isa{H} están en \isa{S}.
    \item[\isa{{\isadigit{4}}{\isacharparenright}}] Supongamos que \isa{G\ {\isasymor}\ H} está en \isa{S} para fórmulas \isa{G} y \isa{H} cualesquiera.
    Por definición, \isa{G\ {\isasymor}\ H} es una fórmula de tipo \isa{{\isasymbeta}} con componentes \isa{G} y \isa{H}. 
    Por lo tanto, por hipótesis se cumple que o bien \isa{G} está en \isa{S} o bien \isa{H} está 
    en \isa{S}.
    \item[\isa{{\isadigit{5}}{\isacharparenright}}] Supongamos que \isa{G\ {\isasymlongrightarrow}\ H} está en \isa{S} para fórmulas \isa{G} y \isa{H} cualesquiera.
    Por definición, \isa{G\ {\isasymlongrightarrow}\ H} es una fórmula de tipo \isa{{\isasymbeta}} con componentes \isa{{\isasymnot}\ G} y \isa{H}. 
    Por lo tanto, por hipótesis se cumple que o bien \isa{{\isasymnot}\ G} está en \isa{S} o bien \isa{H} está 
    en \isa{S}.
    \item[\isa{{\isadigit{6}}{\isacharparenright}}] Supongamos que \isa{{\isasymnot}{\isacharparenleft}{\isasymnot}\ G{\isacharparenright}} está en \isa{S} para una fórmula \isa{G} cualquiera.
    Por definición, \isa{{\isasymnot}{\isacharparenleft}{\isasymnot}\ G{\isacharparenright}} es una fórmula de tipo \isa{{\isasymalpha}} cuyas componentes son ambas \isa{G}. 
    Por lo tanto, por hipótesis se cumple que \isa{G} está en \isa{S}.
    \item[\isa{{\isadigit{7}}{\isacharparenright}}] Supongamos que \isa{{\isasymnot}{\isacharparenleft}G\ {\isasymand}\ H{\isacharparenright}} está en \isa{S} para fórmulas \isa{G} y \isa{H} cualesquiera.
    Por definición, \isa{{\isasymnot}{\isacharparenleft}G\ {\isasymand}\ H{\isacharparenright}} es una fórmula de tipo \isa{{\isasymbeta}} con componentes \isa{{\isasymnot}\ G} y \isa{{\isasymnot}\ H}. 
    Por lo tanto, por hipótesis se cumple que o bien \isa{{\isasymnot}\ G} está en \isa{S} o bien \isa{{\isasymnot}\ H} está 
    en \isa{S}.
    \item[\isa{{\isadigit{8}}{\isacharparenright}}] Supongamos que \isa{{\isasymnot}{\isacharparenleft}G\ {\isasymor}\ H{\isacharparenright}} está en \isa{S} para fórmulas \isa{G} y \isa{H} cualesquiera.
    Por definición, \isa{{\isasymnot}{\isacharparenleft}G\ {\isasymor}\ H{\isacharparenright}} es una fórmula de tipo \isa{{\isasymalpha}} con componentes \isa{{\isasymnot}\ G} y \isa{{\isasymnot}\ H}. 
    Por lo tanto, por hipótesis se cumple que \isa{{\isasymnot}\ G} y \isa{{\isasymnot}\ H} están en \isa{S}.
    \item[\isa{{\isadigit{9}}{\isacharparenright}}] Supongamos que \isa{{\isasymnot}{\isacharparenleft}G\ {\isasymlongrightarrow}\ H{\isacharparenright}} está en \isa{S} para fórmulas \isa{G} y \isa{H} cualesquiera. 
    Por definición, \isa{{\isasymnot}{\isacharparenleft}G\ {\isasymlongrightarrow}\ H{\isacharparenright}} es una fórmula de tipo \isa{{\isasymalpha}} con componentes \isa{G} y \isa{{\isasymnot}\ H}.
    Por lo tanto, por hipótesis se cumple que \isa{G} y \isa{{\isasymnot}\ H} están en \isa{S}.
  \end{enumerate}

  Por tanto, queda probado el resultado.
\end{demostracion}

  Para probar de manera detallada el lema en Isabelle vamos a demostrar
  cada una de las implicaciones de la equivalencia por separado. 

  La primera implicación del lema se basa en dos lemas auxiliares. El primero de ellos 
  prueba que la tercera, sexta, octava y novena condición de la definición de conjunto de 
  Hintikka son suficientes para probar que para toda fórmula de tipo \isa{{\isasymalpha}} con componentes 
  \isa{{\isasymalpha}\isactrlsub {\isadigit{1}}} y \isa{{\isasymalpha}\isactrlsub {\isadigit{2}}} se verifica que si la fórmula pertenece al conjunto \isa{S}, entonces \isa{{\isasymalpha}\isactrlsub {\isadigit{1}}} y 
  \isa{{\isasymalpha}\isactrlsub {\isadigit{2}}} también. Su demostración detallada en Isabelle se muestra a continuación.%
\end{isamarkuptext}\isamarkuptrue%
\isacommand{lemma}\isamarkupfalse%
\ Hintikka{\isacharunderscore}alt{\isadigit{1}}Con{\isacharcolon}\isanewline
\ \ \isakeyword{assumes}\ {\isachardoublequoteopen}{\isacharparenleft}{\isasymforall}G\ H{\isachardot}\ G\ \isactrlbold {\isasymand}\ H\ {\isasymin}\ S\ {\isasymlongrightarrow}\ G\ {\isasymin}\ S\ {\isasymand}\ H\ {\isasymin}\ S{\isacharparenright}\isanewline
\ \ {\isasymand}\ {\isacharparenleft}{\isasymforall}G{\isachardot}\ \isactrlbold {\isasymnot}\ {\isacharparenleft}\isactrlbold {\isasymnot}\ G{\isacharparenright}\ {\isasymin}\ S\ {\isasymlongrightarrow}\ G\ {\isasymin}\ S{\isacharparenright}\isanewline
\ \ {\isasymand}\ {\isacharparenleft}{\isasymforall}G\ H{\isachardot}\ \isactrlbold {\isasymnot}{\isacharparenleft}G\ \isactrlbold {\isasymor}\ H{\isacharparenright}\ {\isasymin}\ S\ {\isasymlongrightarrow}\ \isactrlbold {\isasymnot}\ G\ {\isasymin}\ S\ {\isasymand}\ \isactrlbold {\isasymnot}\ H\ {\isasymin}\ S{\isacharparenright}\isanewline
\ \ {\isasymand}\ {\isacharparenleft}{\isasymforall}G\ H{\isachardot}\ \isactrlbold {\isasymnot}{\isacharparenleft}G\ \isactrlbold {\isasymrightarrow}\ H{\isacharparenright}\ {\isasymin}\ S\ {\isasymlongrightarrow}\ G\ {\isasymin}\ S\ {\isasymand}\ \isactrlbold {\isasymnot}\ H\ {\isasymin}\ S{\isacharparenright}{\isachardoublequoteclose}\isanewline
\ \ \isakeyword{shows}\ {\isachardoublequoteopen}Con\ F\ G\ H\ {\isasymlongrightarrow}\ F\ {\isasymin}\ S\ {\isasymlongrightarrow}\ G\ {\isasymin}\ S\ {\isasymand}\ H\ {\isasymin}\ S{\isachardoublequoteclose}\isanewline
%
\isadelimproof
%
\endisadelimproof
%
\isatagproof
\isacommand{proof}\isamarkupfalse%
\ {\isacharparenleft}rule\ impI{\isacharparenright}\isanewline
\ \ \isacommand{assume}\isamarkupfalse%
\ {\isachardoublequoteopen}Con\ F\ G\ H{\isachardoublequoteclose}\isanewline
\ \ \isacommand{then}\isamarkupfalse%
\ \isacommand{have}\isamarkupfalse%
\ {\isachardoublequoteopen}F\ {\isacharequal}\ G\ \isactrlbold {\isasymand}\ H\ {\isasymor}\ \isanewline
\ \ \ \ {\isacharparenleft}{\isacharparenleft}{\isasymexists}G{\isadigit{1}}\ H{\isadigit{1}}{\isachardot}\ F\ {\isacharequal}\ \isactrlbold {\isasymnot}\ {\isacharparenleft}G{\isadigit{1}}\ \isactrlbold {\isasymor}\ H{\isadigit{1}}{\isacharparenright}\ {\isasymand}\ G\ {\isacharequal}\ \isactrlbold {\isasymnot}\ G{\isadigit{1}}\ {\isasymand}\ H\ {\isacharequal}\ \isactrlbold {\isasymnot}\ H{\isadigit{1}}{\isacharparenright}\ {\isasymor}\ \isanewline
\ \ \ \ {\isacharparenleft}{\isasymexists}H{\isadigit{2}}{\isachardot}\ F\ {\isacharequal}\ \isactrlbold {\isasymnot}\ {\isacharparenleft}G\ \isactrlbold {\isasymrightarrow}\ H{\isadigit{2}}{\isacharparenright}\ {\isasymand}\ H\ {\isacharequal}\ \isactrlbold {\isasymnot}\ H{\isadigit{2}}{\isacharparenright}\ {\isasymor}\ \isanewline
\ \ \ \ F\ {\isacharequal}\ \isactrlbold {\isasymnot}\ {\isacharparenleft}\isactrlbold {\isasymnot}\ G{\isacharparenright}\ {\isasymand}\ H\ {\isacharequal}\ G{\isacharparenright}{\isachardoublequoteclose}\isanewline
\ \ \ \ \isacommand{by}\isamarkupfalse%
\ {\isacharparenleft}simp\ only{\isacharcolon}\ con{\isacharunderscore}dis{\isacharunderscore}simps{\isacharparenleft}{\isadigit{1}}{\isacharparenright}{\isacharparenright}\isanewline
\ \ \isacommand{thus}\isamarkupfalse%
\ {\isachardoublequoteopen}F\ {\isasymin}\ S\ {\isasymlongrightarrow}\ G\ {\isasymin}\ S\ {\isasymand}\ H\ {\isasymin}\ S{\isachardoublequoteclose}\isanewline
\ \ \isacommand{proof}\isamarkupfalse%
\ {\isacharparenleft}rule\ disjE{\isacharparenright}\isanewline
\ \ \ \ \isacommand{assume}\isamarkupfalse%
\ {\isachardoublequoteopen}F\ {\isacharequal}\ G\ \isactrlbold {\isasymand}\ H{\isachardoublequoteclose}\isanewline
\ \ \ \ \isacommand{have}\isamarkupfalse%
\ {\isachardoublequoteopen}{\isasymforall}G\ H{\isachardot}\ G\ \isactrlbold {\isasymand}\ H\ {\isasymin}\ S\ {\isasymlongrightarrow}\ G\ {\isasymin}\ S\ {\isasymand}\ H\ {\isasymin}\ S{\isachardoublequoteclose}\isanewline
\ \ \ \ \ \ \isacommand{using}\isamarkupfalse%
\ assms\ \isacommand{by}\isamarkupfalse%
\ {\isacharparenleft}rule\ conjunct{\isadigit{1}}{\isacharparenright}\isanewline
\ \ \ \ \isacommand{thus}\isamarkupfalse%
\ {\isachardoublequoteopen}F\ {\isasymin}\ S\ {\isasymlongrightarrow}\ G\ {\isasymin}\ S\ {\isasymand}\ H\ {\isasymin}\ S{\isachardoublequoteclose}\isanewline
\ \ \ \ \ \ \isacommand{using}\isamarkupfalse%
\ {\isacartoucheopen}F\ {\isacharequal}\ G\ \isactrlbold {\isasymand}\ H{\isacartoucheclose}\ \isacommand{by}\isamarkupfalse%
\ {\isacharparenleft}iprover\ elim{\isacharcolon}\ allE{\isacharparenright}\isanewline
\ \ \isacommand{next}\isamarkupfalse%
\ \isanewline
\ \ \ \ \isacommand{assume}\isamarkupfalse%
\ {\isachardoublequoteopen}{\isacharparenleft}{\isasymexists}G{\isadigit{1}}\ H{\isadigit{1}}{\isachardot}\ F\ {\isacharequal}\ \isactrlbold {\isasymnot}\ {\isacharparenleft}G{\isadigit{1}}\ \isactrlbold {\isasymor}\ H{\isadigit{1}}{\isacharparenright}\ {\isasymand}\ G\ {\isacharequal}\ \isactrlbold {\isasymnot}\ G{\isadigit{1}}\ {\isasymand}\ H\ {\isacharequal}\ \isactrlbold {\isasymnot}\ H{\isadigit{1}}{\isacharparenright}\ {\isasymor}\ \isanewline
\ \ \ \ {\isacharparenleft}{\isacharparenleft}{\isasymexists}H{\isadigit{2}}{\isachardot}\ F\ {\isacharequal}\ \isactrlbold {\isasymnot}\ {\isacharparenleft}G\ \isactrlbold {\isasymrightarrow}\ H{\isadigit{2}}{\isacharparenright}\ {\isasymand}\ H\ {\isacharequal}\ \isactrlbold {\isasymnot}\ H{\isadigit{2}}{\isacharparenright}\ {\isasymor}\ \isanewline
\ \ \ \ F\ {\isacharequal}\ \isactrlbold {\isasymnot}\ {\isacharparenleft}\isactrlbold {\isasymnot}\ G{\isacharparenright}\ {\isasymand}\ H\ {\isacharequal}\ G{\isacharparenright}{\isachardoublequoteclose}\isanewline
\ \ \ \ \isacommand{thus}\isamarkupfalse%
\ {\isachardoublequoteopen}F\ {\isasymin}\ S\ {\isasymlongrightarrow}\ G\ {\isasymin}\ S\ {\isasymand}\ H\ {\isasymin}\ S{\isachardoublequoteclose}\ \isanewline
\ \ \ \ \isacommand{proof}\isamarkupfalse%
\ {\isacharparenleft}rule\ disjE{\isacharparenright}\isanewline
\ \ \ \ \ \ \isacommand{assume}\isamarkupfalse%
\ E{\isadigit{1}}{\isacharcolon}{\isachardoublequoteopen}{\isasymexists}G{\isadigit{1}}\ H{\isadigit{1}}{\isachardot}\ F\ {\isacharequal}\ \isactrlbold {\isasymnot}\ {\isacharparenleft}G{\isadigit{1}}\ \isactrlbold {\isasymor}\ H{\isadigit{1}}{\isacharparenright}\ {\isasymand}\ G\ {\isacharequal}\ \isactrlbold {\isasymnot}\ G{\isadigit{1}}\ {\isasymand}\ H\ {\isacharequal}\ \isactrlbold {\isasymnot}\ H{\isadigit{1}}{\isachardoublequoteclose}\isanewline
\ \ \ \ \ \ \isacommand{obtain}\isamarkupfalse%
\ G{\isadigit{1}}\ H{\isadigit{1}}\ \isakeyword{where}\ A{\isadigit{1}}{\isacharcolon}{\isachardoublequoteopen}F\ {\isacharequal}\ \isactrlbold {\isasymnot}\ {\isacharparenleft}G{\isadigit{1}}\ \isactrlbold {\isasymor}\ H{\isadigit{1}}{\isacharparenright}\ {\isasymand}\ G\ {\isacharequal}\ \isactrlbold {\isasymnot}\ G{\isadigit{1}}\ {\isasymand}\ H\ {\isacharequal}\ \isactrlbold {\isasymnot}\ H{\isadigit{1}}{\isachardoublequoteclose}\isanewline
\ \ \ \ \ \ \ \ \isacommand{using}\isamarkupfalse%
\ E{\isadigit{1}}\ \isacommand{by}\isamarkupfalse%
\ {\isacharparenleft}iprover\ elim{\isacharcolon}\ exE{\isacharparenright}\isanewline
\ \ \ \ \ \ \isacommand{then}\isamarkupfalse%
\ \isacommand{have}\isamarkupfalse%
\ {\isachardoublequoteopen}F\ {\isacharequal}\ \isactrlbold {\isasymnot}\ {\isacharparenleft}G{\isadigit{1}}\ \isactrlbold {\isasymor}\ H{\isadigit{1}}{\isacharparenright}{\isachardoublequoteclose}\isanewline
\ \ \ \ \ \ \ \ \isacommand{by}\isamarkupfalse%
\ {\isacharparenleft}rule\ conjunct{\isadigit{1}}{\isacharparenright}\isanewline
\ \ \ \ \ \ \isacommand{have}\isamarkupfalse%
\ {\isachardoublequoteopen}G\ {\isacharequal}\ \isactrlbold {\isasymnot}\ G{\isadigit{1}}{\isachardoublequoteclose}\isanewline
\ \ \ \ \ \ \ \ \isacommand{using}\isamarkupfalse%
\ A{\isadigit{1}}\ \isacommand{by}\isamarkupfalse%
\ {\isacharparenleft}iprover\ elim{\isacharcolon}\ conjunct{\isadigit{1}}{\isacharparenright}\isanewline
\ \ \ \ \ \ \isacommand{have}\isamarkupfalse%
\ {\isachardoublequoteopen}H\ {\isacharequal}\ \isactrlbold {\isasymnot}\ H{\isadigit{1}}{\isachardoublequoteclose}\isanewline
\ \ \ \ \ \ \ \ \isacommand{using}\isamarkupfalse%
\ A{\isadigit{1}}\ \isacommand{by}\isamarkupfalse%
\ {\isacharparenleft}iprover\ elim{\isacharcolon}\ conjunct{\isadigit{1}}{\isacharparenright}\isanewline
\ \ \ \ \ \ \isacommand{have}\isamarkupfalse%
\ {\isachardoublequoteopen}{\isasymforall}G\ H{\isachardot}\ \isactrlbold {\isasymnot}{\isacharparenleft}G\ \isactrlbold {\isasymor}\ H{\isacharparenright}\ {\isasymin}\ S\ {\isasymlongrightarrow}\ \isactrlbold {\isasymnot}\ G\ {\isasymin}\ S\ {\isasymand}\ \isactrlbold {\isasymnot}\ H\ {\isasymin}\ S{\isachardoublequoteclose}\isanewline
\ \ \ \ \ \ \ \ \isacommand{using}\isamarkupfalse%
\ assms\ \isacommand{by}\isamarkupfalse%
\ {\isacharparenleft}iprover\ elim{\isacharcolon}\ conjunct{\isadigit{2}}\ conjunct{\isadigit{1}}{\isacharparenright}\isanewline
\ \ \ \ \ \ \isacommand{thus}\isamarkupfalse%
\ {\isachardoublequoteopen}F\ {\isasymin}\ S\ {\isasymlongrightarrow}\ G\ {\isasymin}\ S\ {\isasymand}\ H\ {\isasymin}\ S{\isachardoublequoteclose}\isanewline
\ \ \ \ \ \ \ \ \isacommand{using}\isamarkupfalse%
\ {\isacartoucheopen}F\ {\isacharequal}\ \isactrlbold {\isasymnot}\ {\isacharparenleft}G{\isadigit{1}}\ \isactrlbold {\isasymor}\ H{\isadigit{1}}{\isacharparenright}{\isacartoucheclose}\ {\isacartoucheopen}G\ {\isacharequal}\ \isactrlbold {\isasymnot}\ G{\isadigit{1}}{\isacartoucheclose}\ {\isacartoucheopen}H\ {\isacharequal}\ \isactrlbold {\isasymnot}\ H{\isadigit{1}}{\isacartoucheclose}\ \isacommand{by}\isamarkupfalse%
\ {\isacharparenleft}iprover\ elim{\isacharcolon}\ allE{\isacharparenright}\isanewline
\ \ \ \ \isacommand{next}\isamarkupfalse%
\isanewline
\ \ \ \ \ \ \isacommand{assume}\isamarkupfalse%
\ {\isachardoublequoteopen}{\isacharparenleft}{\isasymexists}H{\isadigit{2}}{\isachardot}\ F\ {\isacharequal}\ \isactrlbold {\isasymnot}\ {\isacharparenleft}G\ \isactrlbold {\isasymrightarrow}\ H{\isadigit{2}}{\isacharparenright}\ {\isasymand}\ H\ {\isacharequal}\ \isactrlbold {\isasymnot}\ H{\isadigit{2}}{\isacharparenright}\ {\isasymor}\ \isanewline
\ \ \ \ \ \ F\ {\isacharequal}\ \isactrlbold {\isasymnot}\ {\isacharparenleft}\isactrlbold {\isasymnot}\ G{\isacharparenright}\ {\isasymand}\ H\ {\isacharequal}\ G{\isachardoublequoteclose}\isanewline
\ \ \ \ \ \ \isacommand{thus}\isamarkupfalse%
\ {\isachardoublequoteopen}F\ {\isasymin}\ S\ {\isasymlongrightarrow}\ G\ {\isasymin}\ S\ {\isasymand}\ H\ {\isasymin}\ S{\isachardoublequoteclose}\ \isanewline
\ \ \ \ \ \ \isacommand{proof}\isamarkupfalse%
\ {\isacharparenleft}rule\ disjE{\isacharparenright}\isanewline
\ \ \ \ \ \ \ \ \isacommand{assume}\isamarkupfalse%
\ E{\isadigit{2}}{\isacharcolon}{\isachardoublequoteopen}{\isasymexists}H{\isadigit{2}}{\isachardot}\ F\ {\isacharequal}\ \isactrlbold {\isasymnot}\ {\isacharparenleft}G\ \isactrlbold {\isasymrightarrow}\ H{\isadigit{2}}{\isacharparenright}\ {\isasymand}\ H\ {\isacharequal}\ \isactrlbold {\isasymnot}\ H{\isadigit{2}}{\isachardoublequoteclose}\isanewline
\ \ \ \ \ \ \ \ \isacommand{obtain}\isamarkupfalse%
\ H{\isadigit{2}}\ \isakeyword{where}\ A{\isadigit{2}}{\isacharcolon}{\isachardoublequoteopen}F\ {\isacharequal}\ \isactrlbold {\isasymnot}\ {\isacharparenleft}G\ \isactrlbold {\isasymrightarrow}\ H{\isadigit{2}}{\isacharparenright}\ {\isasymand}\ H\ {\isacharequal}\ \isactrlbold {\isasymnot}\ H{\isadigit{2}}{\isachardoublequoteclose}\isanewline
\ \ \ \ \ \ \ \ \ \ \isacommand{using}\isamarkupfalse%
\ E{\isadigit{2}}\ \isacommand{by}\isamarkupfalse%
\ {\isacharparenleft}rule\ exE{\isacharparenright}\isanewline
\ \ \ \ \ \ \ \ \isacommand{have}\isamarkupfalse%
\ {\isachardoublequoteopen}F\ {\isacharequal}\ \isactrlbold {\isasymnot}\ {\isacharparenleft}G\ \isactrlbold {\isasymrightarrow}\ H{\isadigit{2}}{\isacharparenright}{\isachardoublequoteclose}\isanewline
\ \ \ \ \ \ \ \ \ \ \isacommand{using}\isamarkupfalse%
\ A{\isadigit{2}}\ \isacommand{by}\isamarkupfalse%
\ {\isacharparenleft}rule\ conjunct{\isadigit{1}}{\isacharparenright}\isanewline
\ \ \ \ \ \ \ \ \isacommand{have}\isamarkupfalse%
\ {\isachardoublequoteopen}H\ {\isacharequal}\ \isactrlbold {\isasymnot}\ H{\isadigit{2}}{\isachardoublequoteclose}\isanewline
\ \ \ \ \ \ \ \ \ \ \isacommand{using}\isamarkupfalse%
\ A{\isadigit{2}}\ \isacommand{by}\isamarkupfalse%
\ {\isacharparenleft}rule\ conjunct{\isadigit{2}}{\isacharparenright}\isanewline
\ \ \ \ \ \ \ \ \isacommand{have}\isamarkupfalse%
\ {\isachardoublequoteopen}{\isasymforall}G\ H{\isachardot}\ \isactrlbold {\isasymnot}{\isacharparenleft}G\ \isactrlbold {\isasymrightarrow}\ H{\isacharparenright}\ {\isasymin}\ S\ {\isasymlongrightarrow}\ G\ {\isasymin}\ S\ {\isasymand}\ \isactrlbold {\isasymnot}\ H\ {\isasymin}\ S{\isachardoublequoteclose}\isanewline
\ \ \ \ \ \ \ \ \ \ \isacommand{using}\isamarkupfalse%
\ assms\ \isacommand{by}\isamarkupfalse%
\ {\isacharparenleft}iprover\ elim{\isacharcolon}\ conjunct{\isadigit{2}}\ conjunct{\isadigit{1}}{\isacharparenright}\isanewline
\ \ \ \ \ \ \ \ \isacommand{thus}\isamarkupfalse%
\ {\isachardoublequoteopen}F\ {\isasymin}\ S\ {\isasymlongrightarrow}\ G\ {\isasymin}\ S\ {\isasymand}\ H\ {\isasymin}\ S{\isachardoublequoteclose}\isanewline
\ \ \ \ \ \ \ \ \ \ \isacommand{using}\isamarkupfalse%
\ {\isacartoucheopen}F\ {\isacharequal}\ \isactrlbold {\isasymnot}\ {\isacharparenleft}G\ \isactrlbold {\isasymrightarrow}\ H{\isadigit{2}}{\isacharparenright}{\isacartoucheclose}\ {\isacartoucheopen}H\ {\isacharequal}\ \isactrlbold {\isasymnot}\ H{\isadigit{2}}{\isacartoucheclose}\ \isacommand{by}\isamarkupfalse%
\ {\isacharparenleft}iprover\ elim{\isacharcolon}\ allE{\isacharparenright}\isanewline
\ \ \ \ \ \ \isacommand{next}\isamarkupfalse%
\ \isanewline
\ \ \ \ \ \ \ \ \isacommand{assume}\isamarkupfalse%
\ {\isachardoublequoteopen}F\ {\isacharequal}\ \isactrlbold {\isasymnot}\ {\isacharparenleft}\isactrlbold {\isasymnot}\ G{\isacharparenright}\ {\isasymand}\ H\ {\isacharequal}\ G{\isachardoublequoteclose}\isanewline
\ \ \ \ \ \ \ \ \isacommand{then}\isamarkupfalse%
\ \isacommand{have}\isamarkupfalse%
\ {\isachardoublequoteopen}F\ {\isacharequal}\ \isactrlbold {\isasymnot}\ {\isacharparenleft}\isactrlbold {\isasymnot}\ G{\isacharparenright}{\isachardoublequoteclose}\isanewline
\ \ \ \ \ \ \ \ \ \ \isacommand{by}\isamarkupfalse%
\ {\isacharparenleft}rule\ conjunct{\isadigit{1}}{\isacharparenright}\isanewline
\ \ \ \ \ \ \ \ \isacommand{have}\isamarkupfalse%
\ {\isachardoublequoteopen}H\ {\isacharequal}\ G{\isachardoublequoteclose}\isanewline
\ \ \ \ \ \ \ \ \ \ \isacommand{using}\isamarkupfalse%
\ {\isacartoucheopen}F\ {\isacharequal}\ \isactrlbold {\isasymnot}\ {\isacharparenleft}\isactrlbold {\isasymnot}\ G{\isacharparenright}\ {\isasymand}\ H\ {\isacharequal}\ G{\isacartoucheclose}\ \isacommand{by}\isamarkupfalse%
\ {\isacharparenleft}rule\ conjunct{\isadigit{2}}{\isacharparenright}\isanewline
\ \ \ \ \ \ \ \ \isacommand{have}\isamarkupfalse%
\ {\isachardoublequoteopen}{\isasymforall}G{\isachardot}\ \isactrlbold {\isasymnot}\ {\isacharparenleft}\isactrlbold {\isasymnot}\ G{\isacharparenright}\ {\isasymin}\ S\ {\isasymlongrightarrow}\ G\ {\isasymin}\ S{\isachardoublequoteclose}\isanewline
\ \ \ \ \ \ \ \ \ \ \isacommand{using}\isamarkupfalse%
\ assms\ \isacommand{by}\isamarkupfalse%
\ {\isacharparenleft}iprover\ elim{\isacharcolon}\ conjunct{\isadigit{2}}\ conjunct{\isadigit{1}}{\isacharparenright}\isanewline
\ \ \ \ \ \ \ \ \isacommand{then}\isamarkupfalse%
\ \isacommand{have}\isamarkupfalse%
\ {\isachardoublequoteopen}\isactrlbold {\isasymnot}\ {\isacharparenleft}\isactrlbold {\isasymnot}\ G{\isacharparenright}\ {\isasymin}\ S\ {\isasymlongrightarrow}\ G\ {\isasymin}\ S{\isachardoublequoteclose}\isanewline
\ \ \ \ \ \ \ \ \ \ \isacommand{by}\isamarkupfalse%
\ {\isacharparenleft}rule\ allE{\isacharparenright}\isanewline
\ \ \ \ \ \ \ \ \isacommand{then}\isamarkupfalse%
\ \isacommand{have}\isamarkupfalse%
\ {\isachardoublequoteopen}F\ {\isasymin}\ S\ {\isasymlongrightarrow}\ G\ {\isasymin}\ S{\isachardoublequoteclose}\isanewline
\ \ \ \ \ \ \ \ \ \ \isacommand{by}\isamarkupfalse%
\ {\isacharparenleft}simp\ only{\isacharcolon}\ {\isacartoucheopen}F\ {\isacharequal}\ \isactrlbold {\isasymnot}\ {\isacharparenleft}\isactrlbold {\isasymnot}\ G{\isacharparenright}{\isacartoucheclose}{\isacharparenright}\ \isanewline
\ \ \ \ \ \ \ \ \isacommand{then}\isamarkupfalse%
\ \isacommand{have}\isamarkupfalse%
\ {\isachardoublequoteopen}F\ {\isasymin}\ S\ {\isasymlongrightarrow}\ G\ {\isasymin}\ S\ {\isasymand}\ G\ {\isasymin}\ S{\isachardoublequoteclose}\isanewline
\ \ \ \ \ \ \ \ \ \ \isacommand{by}\isamarkupfalse%
\ {\isacharparenleft}simp\ only{\isacharcolon}\ conj{\isacharunderscore}absorb{\isacharparenright}\isanewline
\ \ \ \ \ \ \ \ \isacommand{thus}\isamarkupfalse%
\ {\isachardoublequoteopen}F\ {\isasymin}\ S\ {\isasymlongrightarrow}\ G\ {\isasymin}\ S\ {\isasymand}\ H\ {\isasymin}\ S{\isachardoublequoteclose}\isanewline
\ \ \ \ \ \ \ \ \ \ \isacommand{by}\isamarkupfalse%
\ {\isacharparenleft}simp\ only{\isacharcolon}\ {\isacartoucheopen}H{\isacharequal}G{\isacartoucheclose}{\isacharparenright}\isanewline
\ \ \ \ \ \ \isacommand{qed}\isamarkupfalse%
\isanewline
\ \ \ \ \isacommand{qed}\isamarkupfalse%
\isanewline
\ \ \isacommand{qed}\isamarkupfalse%
\isanewline
\isacommand{qed}\isamarkupfalse%
%
\endisatagproof
{\isafoldproof}%
%
\isadelimproof
%
\endisadelimproof
%
\begin{isamarkuptext}%
Por otro lado, el segundo lema auxiliar prueba que la cuarta, quinta, sexta
  y séptima condición de la definición de conjunto de Hintikka son suficientes para
  probar que para toda fórmula de tipo \isa{{\isasymbeta}} con componentes \isa{{\isasymbeta}\isactrlsub {\isadigit{1}}} y \isa{{\isasymbeta}\isactrlsub {\isadigit{2}}} se verifica 
  que si la fórmula pertenece al conjunto \isa{S}, entonces o bien \isa{{\isasymbeta}\isactrlsub {\isadigit{1}}} pertenece al
  conjunto o bien \isa{{\isasymbeta}\isactrlsub {\isadigit{2}}} pertenece al conjunto. Veamos su prueba detallada en 
  Isabelle/HOL.%
\end{isamarkuptext}\isamarkuptrue%
\isacommand{lemma}\isamarkupfalse%
\ Hintikka{\isacharunderscore}alt{\isadigit{1}}Dis{\isacharcolon}\isanewline
\ \ \isakeyword{assumes}\ \ {\isachardoublequoteopen}{\isacharparenleft}{\isasymforall}\ G\ H{\isachardot}\ G\ \isactrlbold {\isasymor}\ H\ {\isasymin}\ S\ {\isasymlongrightarrow}\ G\ {\isasymin}\ S\ {\isasymor}\ H\ {\isasymin}\ S{\isacharparenright}\isanewline
\ \ {\isasymand}\ {\isacharparenleft}{\isasymforall}\ G\ H{\isachardot}\ G\ \isactrlbold {\isasymrightarrow}\ H\ {\isasymin}\ S\ {\isasymlongrightarrow}\ \isactrlbold {\isasymnot}\ G\ {\isasymin}\ S\ {\isasymor}\ H\ {\isasymin}\ S{\isacharparenright}\isanewline
\ \ {\isasymand}\ {\isacharparenleft}{\isasymforall}\ G{\isachardot}\ \isactrlbold {\isasymnot}\ {\isacharparenleft}\isactrlbold {\isasymnot}\ G{\isacharparenright}\ {\isasymin}\ S\ {\isasymlongrightarrow}\ G\ {\isasymin}\ S{\isacharparenright}\isanewline
\ \ {\isasymand}\ {\isacharparenleft}{\isasymforall}\ G\ H{\isachardot}\ \isactrlbold {\isasymnot}{\isacharparenleft}G\ \isactrlbold {\isasymand}\ H{\isacharparenright}\ {\isasymin}\ S\ {\isasymlongrightarrow}\ \isactrlbold {\isasymnot}\ G\ {\isasymin}\ S\ {\isasymor}\ \isactrlbold {\isasymnot}\ H\ {\isasymin}\ S{\isacharparenright}{\isachardoublequoteclose}\isanewline
\ \ \isakeyword{shows}\ {\isachardoublequoteopen}Dis\ F\ G\ H\ {\isasymlongrightarrow}\ F\ {\isasymin}\ S\ {\isasymlongrightarrow}\ G\ {\isasymin}\ S\ {\isasymor}\ H\ {\isasymin}\ S{\isachardoublequoteclose}\isanewline
%
\isadelimproof
%
\endisadelimproof
%
\isatagproof
\isacommand{proof}\isamarkupfalse%
\ {\isacharparenleft}rule\ impI{\isacharparenright}\isanewline
\ \ \isacommand{assume}\isamarkupfalse%
\ {\isachardoublequoteopen}Dis\ F\ G\ H{\isachardoublequoteclose}\isanewline
\ \ \isacommand{then}\isamarkupfalse%
\ \isacommand{have}\isamarkupfalse%
\ {\isachardoublequoteopen}F\ {\isacharequal}\ G\ \isactrlbold {\isasymor}\ H\ {\isasymor}\ \isanewline
\ \ \ \ {\isacharparenleft}{\isasymexists}G{\isadigit{1}}\ H{\isadigit{1}}{\isachardot}\ F\ {\isacharequal}\ G{\isadigit{1}}\ \isactrlbold {\isasymrightarrow}\ H{\isadigit{1}}\ {\isasymand}\ G\ {\isacharequal}\ \isactrlbold {\isasymnot}\ G{\isadigit{1}}\ {\isasymand}\ H\ {\isacharequal}\ H{\isadigit{1}}{\isacharparenright}\ {\isasymor}\ \isanewline
\ \ \ \ {\isacharparenleft}{\isasymexists}G{\isadigit{2}}\ H{\isadigit{2}}{\isachardot}\ F\ {\isacharequal}\ \isactrlbold {\isasymnot}\ {\isacharparenleft}G{\isadigit{2}}\ \isactrlbold {\isasymand}\ H{\isadigit{2}}{\isacharparenright}\ {\isasymand}\ G\ {\isacharequal}\ \isactrlbold {\isasymnot}\ G{\isadigit{2}}\ {\isasymand}\ H\ {\isacharequal}\ \isactrlbold {\isasymnot}\ H{\isadigit{2}}{\isacharparenright}\ {\isasymor}\ \isanewline
\ \ \ \ F\ {\isacharequal}\ \isactrlbold {\isasymnot}\ {\isacharparenleft}\isactrlbold {\isasymnot}\ G{\isacharparenright}\ {\isasymand}\ H\ {\isacharequal}\ G{\isachardoublequoteclose}\ \isanewline
\ \ \ \ \isacommand{by}\isamarkupfalse%
\ {\isacharparenleft}simp\ only{\isacharcolon}\ con{\isacharunderscore}dis{\isacharunderscore}simps{\isacharparenleft}{\isadigit{2}}{\isacharparenright}{\isacharparenright}\isanewline
\ \ \isacommand{thus}\isamarkupfalse%
\ {\isachardoublequoteopen}F\ {\isasymin}\ S\ {\isasymlongrightarrow}\ G\ {\isasymin}\ S\ {\isasymor}\ H\ {\isasymin}\ S{\isachardoublequoteclose}\ \isanewline
\ \ \isacommand{proof}\isamarkupfalse%
\ {\isacharparenleft}rule\ disjE{\isacharparenright}\isanewline
\ \ \ \ \isacommand{assume}\isamarkupfalse%
\ {\isachardoublequoteopen}F\ {\isacharequal}\ G\ \isactrlbold {\isasymor}\ H{\isachardoublequoteclose}\isanewline
\ \ \ \ \isacommand{have}\isamarkupfalse%
\ {\isachardoublequoteopen}{\isasymforall}G\ H{\isachardot}\ G\ \isactrlbold {\isasymor}\ H\ {\isasymin}\ S\ {\isasymlongrightarrow}\ G\ {\isasymin}\ S\ {\isasymor}\ H\ {\isasymin}\ S{\isachardoublequoteclose}\isanewline
\ \ \ \ \ \ \isacommand{using}\isamarkupfalse%
\ assms\ \isacommand{by}\isamarkupfalse%
\ {\isacharparenleft}rule\ conjunct{\isadigit{1}}{\isacharparenright}\isanewline
\ \ \ \ \isacommand{thus}\isamarkupfalse%
\ {\isachardoublequoteopen}F\ {\isasymin}\ S\ {\isasymlongrightarrow}\ G\ {\isasymin}\ S\ {\isasymor}\ H\ {\isasymin}\ S{\isachardoublequoteclose}\ \isanewline
\ \ \ \ \ \ \isacommand{using}\isamarkupfalse%
\ {\isacartoucheopen}F\ {\isacharequal}\ G\ \isactrlbold {\isasymor}\ H{\isacartoucheclose}\ \isacommand{by}\isamarkupfalse%
\ {\isacharparenleft}iprover\ elim{\isacharcolon}\ allE{\isacharparenright}\isanewline
\ \ \isacommand{next}\isamarkupfalse%
\isanewline
\ \ \ \ \isacommand{assume}\isamarkupfalse%
\ {\isachardoublequoteopen}{\isacharparenleft}{\isasymexists}G{\isadigit{1}}\ H{\isadigit{1}}{\isachardot}\ F\ {\isacharequal}\ G{\isadigit{1}}\ \isactrlbold {\isasymrightarrow}\ H{\isadigit{1}}\ {\isasymand}\ G\ {\isacharequal}\ \isactrlbold {\isasymnot}\ G{\isadigit{1}}\ {\isasymand}\ H\ {\isacharequal}\ H{\isadigit{1}}{\isacharparenright}\ {\isasymor}\ \isanewline
\ \ \ \ {\isacharparenleft}{\isasymexists}G{\isadigit{2}}\ H{\isadigit{2}}{\isachardot}\ F\ {\isacharequal}\ \isactrlbold {\isasymnot}\ {\isacharparenleft}G{\isadigit{2}}\ \isactrlbold {\isasymand}\ H{\isadigit{2}}{\isacharparenright}\ {\isasymand}\ G\ {\isacharequal}\ \isactrlbold {\isasymnot}\ G{\isadigit{2}}\ {\isasymand}\ H\ {\isacharequal}\ \isactrlbold {\isasymnot}\ H{\isadigit{2}}{\isacharparenright}\ {\isasymor}\ \isanewline
\ \ \ \ F\ {\isacharequal}\ \isactrlbold {\isasymnot}\ {\isacharparenleft}\isactrlbold {\isasymnot}\ G{\isacharparenright}\ {\isasymand}\ H\ {\isacharequal}\ G{\isachardoublequoteclose}\isanewline
\ \ \ \ \isacommand{thus}\isamarkupfalse%
\ {\isachardoublequoteopen}F\ {\isasymin}\ S\ {\isasymlongrightarrow}\ G\ {\isasymin}\ S\ {\isasymor}\ H\ {\isasymin}\ S{\isachardoublequoteclose}\isanewline
\ \ \ \ \isacommand{proof}\isamarkupfalse%
\ {\isacharparenleft}rule\ disjE{\isacharparenright}\isanewline
\ \ \ \ \ \ \isacommand{assume}\isamarkupfalse%
\ E{\isadigit{1}}{\isacharcolon}{\isachardoublequoteopen}{\isasymexists}G{\isadigit{1}}\ H{\isadigit{1}}{\isachardot}\ F\ {\isacharequal}\ G{\isadigit{1}}\ \isactrlbold {\isasymrightarrow}\ H{\isadigit{1}}\ {\isasymand}\ G\ {\isacharequal}\ \isactrlbold {\isasymnot}\ G{\isadigit{1}}\ {\isasymand}\ H\ {\isacharequal}\ H{\isadigit{1}}{\isachardoublequoteclose}\isanewline
\ \ \ \ \ \ \isacommand{obtain}\isamarkupfalse%
\ G{\isadigit{1}}\ H{\isadigit{1}}\ \isakeyword{where}\ A{\isadigit{1}}{\isacharcolon}{\isachardoublequoteopen}F\ {\isacharequal}\ G{\isadigit{1}}\ \isactrlbold {\isasymrightarrow}\ H{\isadigit{1}}\ {\isasymand}\ G\ {\isacharequal}\ \isactrlbold {\isasymnot}\ G{\isadigit{1}}\ {\isasymand}\ H\ {\isacharequal}\ H{\isadigit{1}}{\isachardoublequoteclose}\isanewline
\ \ \ \ \ \ \ \ \isacommand{using}\isamarkupfalse%
\ E{\isadigit{1}}\ \isacommand{by}\isamarkupfalse%
\ {\isacharparenleft}iprover\ elim{\isacharcolon}\ exE{\isacharparenright}\isanewline
\ \ \ \ \ \ \isacommand{have}\isamarkupfalse%
\ {\isachardoublequoteopen}F\ {\isacharequal}\ G{\isadigit{1}}\ \isactrlbold {\isasymrightarrow}\ H{\isadigit{1}}{\isachardoublequoteclose}\isanewline
\ \ \ \ \ \ \ \ \isacommand{using}\isamarkupfalse%
\ A{\isadigit{1}}\ \isacommand{by}\isamarkupfalse%
\ {\isacharparenleft}rule\ conjunct{\isadigit{1}}{\isacharparenright}\isanewline
\ \ \ \ \ \ \isacommand{have}\isamarkupfalse%
\ {\isachardoublequoteopen}G\ {\isacharequal}\ \isactrlbold {\isasymnot}\ G{\isadigit{1}}{\isachardoublequoteclose}\isanewline
\ \ \ \ \ \ \ \ \isacommand{using}\isamarkupfalse%
\ A{\isadigit{1}}\ \isacommand{by}\isamarkupfalse%
\ {\isacharparenleft}iprover\ elim{\isacharcolon}\ conjunct{\isadigit{1}}{\isacharparenright}\isanewline
\ \ \ \ \ \ \isacommand{have}\isamarkupfalse%
\ {\isachardoublequoteopen}H\ {\isacharequal}\ H{\isadigit{1}}{\isachardoublequoteclose}\isanewline
\ \ \ \ \ \ \ \ \isacommand{using}\isamarkupfalse%
\ A{\isadigit{1}}\ \isacommand{by}\isamarkupfalse%
\ {\isacharparenleft}iprover\ elim{\isacharcolon}\ conjunct{\isadigit{2}}\ conjunct{\isadigit{1}}{\isacharparenright}\isanewline
\ \ \ \ \ \ \isacommand{have}\isamarkupfalse%
\ {\isachardoublequoteopen}{\isasymforall}G\ H{\isachardot}\ G\ \isactrlbold {\isasymrightarrow}\ H\ {\isasymin}\ S\ {\isasymlongrightarrow}\ \isactrlbold {\isasymnot}\ G\ {\isasymin}\ S\ {\isasymor}\ H\ {\isasymin}\ S{\isachardoublequoteclose}\isanewline
\ \ \ \ \ \ \ \ \isacommand{using}\isamarkupfalse%
\ assms\ \isacommand{by}\isamarkupfalse%
\ {\isacharparenleft}iprover\ elim{\isacharcolon}\ conjunct{\isadigit{2}}\ conjunct{\isadigit{1}}{\isacharparenright}\isanewline
\ \ \ \ \ \ \isacommand{thus}\isamarkupfalse%
\ {\isachardoublequoteopen}F\ {\isasymin}\ S\ {\isasymlongrightarrow}\ G\ {\isasymin}\ S\ {\isasymor}\ H\ {\isasymin}\ S{\isachardoublequoteclose}\isanewline
\ \ \ \ \ \ \ \ \isacommand{using}\isamarkupfalse%
\ {\isacartoucheopen}F\ {\isacharequal}\ G{\isadigit{1}}\ \isactrlbold {\isasymrightarrow}\ H{\isadigit{1}}{\isacartoucheclose}\ {\isacartoucheopen}G\ {\isacharequal}\ \isactrlbold {\isasymnot}\ G{\isadigit{1}}{\isacartoucheclose}\ {\isacartoucheopen}H\ {\isacharequal}\ H{\isadigit{1}}{\isacartoucheclose}\ \isacommand{by}\isamarkupfalse%
\ {\isacharparenleft}iprover\ elim{\isacharcolon}\ allE{\isacharparenright}\isanewline
\ \ \ \ \isacommand{next}\isamarkupfalse%
\isanewline
\ \ \ \ \ \ \isacommand{assume}\isamarkupfalse%
\ {\isachardoublequoteopen}{\isacharparenleft}{\isasymexists}G{\isadigit{2}}\ H{\isadigit{2}}{\isachardot}\ F\ {\isacharequal}\ \isactrlbold {\isasymnot}\ {\isacharparenleft}G{\isadigit{2}}\ \isactrlbold {\isasymand}\ H{\isadigit{2}}{\isacharparenright}\ {\isasymand}\ G\ {\isacharequal}\ \isactrlbold {\isasymnot}\ G{\isadigit{2}}\ {\isasymand}\ H\ {\isacharequal}\ \isactrlbold {\isasymnot}\ H{\isadigit{2}}{\isacharparenright}\ {\isasymor}\ \isanewline
\ \ \ \ \ \ F\ {\isacharequal}\ \isactrlbold {\isasymnot}\ {\isacharparenleft}\isactrlbold {\isasymnot}\ G{\isacharparenright}\ {\isasymand}\ H\ {\isacharequal}\ G{\isachardoublequoteclose}\isanewline
\ \ \ \ \ \ \isacommand{thus}\isamarkupfalse%
\ {\isachardoublequoteopen}F\ {\isasymin}\ S\ {\isasymlongrightarrow}\ G\ {\isasymin}\ S\ {\isasymor}\ H\ {\isasymin}\ S{\isachardoublequoteclose}\isanewline
\ \ \ \ \ \ \isacommand{proof}\isamarkupfalse%
\ {\isacharparenleft}rule\ disjE{\isacharparenright}\isanewline
\ \ \ \ \ \ \ \ \isacommand{assume}\isamarkupfalse%
\ E{\isadigit{2}}{\isacharcolon}{\isachardoublequoteopen}{\isasymexists}G{\isadigit{2}}\ H{\isadigit{2}}{\isachardot}\ F\ {\isacharequal}\ \isactrlbold {\isasymnot}\ {\isacharparenleft}G{\isadigit{2}}\ \isactrlbold {\isasymand}\ H{\isadigit{2}}{\isacharparenright}\ {\isasymand}\ G\ {\isacharequal}\ \isactrlbold {\isasymnot}\ G{\isadigit{2}}\ {\isasymand}\ H\ {\isacharequal}\ \isactrlbold {\isasymnot}\ H{\isadigit{2}}{\isachardoublequoteclose}\isanewline
\ \ \ \ \ \ \ \ \isacommand{obtain}\isamarkupfalse%
\ G{\isadigit{2}}\ H{\isadigit{2}}\ \isakeyword{where}\ A{\isadigit{2}}{\isacharcolon}{\isachardoublequoteopen}F\ {\isacharequal}\ \isactrlbold {\isasymnot}\ {\isacharparenleft}G{\isadigit{2}}\ \isactrlbold {\isasymand}\ H{\isadigit{2}}{\isacharparenright}\ {\isasymand}\ G\ {\isacharequal}\ \isactrlbold {\isasymnot}\ G{\isadigit{2}}\ {\isasymand}\ H\ {\isacharequal}\ \isactrlbold {\isasymnot}\ H{\isadigit{2}}{\isachardoublequoteclose}\ \isanewline
\ \ \ \ \ \ \ \ \ \ \isacommand{using}\isamarkupfalse%
\ E{\isadigit{2}}\ \isacommand{by}\isamarkupfalse%
\ {\isacharparenleft}iprover\ elim{\isacharcolon}\ exE{\isacharparenright}\isanewline
\ \ \ \ \ \ \ \ \isacommand{have}\isamarkupfalse%
\ {\isachardoublequoteopen}F\ {\isacharequal}\ \isactrlbold {\isasymnot}\ {\isacharparenleft}G{\isadigit{2}}\ \isactrlbold {\isasymand}\ H{\isadigit{2}}{\isacharparenright}{\isachardoublequoteclose}\ \isanewline
\ \ \ \ \ \ \ \ \ \ \isacommand{using}\isamarkupfalse%
\ A{\isadigit{2}}\ \isacommand{by}\isamarkupfalse%
\ {\isacharparenleft}rule\ conjunct{\isadigit{1}}{\isacharparenright}\isanewline
\ \ \ \ \ \ \ \ \isacommand{have}\isamarkupfalse%
\ {\isachardoublequoteopen}G\ {\isacharequal}\ \isactrlbold {\isasymnot}\ G{\isadigit{2}}{\isachardoublequoteclose}\isanewline
\ \ \ \ \ \ \ \ \ \ \isacommand{using}\isamarkupfalse%
\ A{\isadigit{2}}\ \isacommand{by}\isamarkupfalse%
\ {\isacharparenleft}iprover\ elim{\isacharcolon}\ conjunct{\isadigit{2}}\ conjunct{\isadigit{1}}{\isacharparenright}\isanewline
\ \ \ \ \ \ \ \ \isacommand{have}\isamarkupfalse%
\ {\isachardoublequoteopen}H\ {\isacharequal}\ \isactrlbold {\isasymnot}\ H{\isadigit{2}}{\isachardoublequoteclose}\isanewline
\ \ \ \ \ \ \ \ \ \ \isacommand{using}\isamarkupfalse%
\ A{\isadigit{2}}\ \isacommand{by}\isamarkupfalse%
\ {\isacharparenleft}iprover\ elim{\isacharcolon}\ conjunct{\isadigit{1}}{\isacharparenright}\isanewline
\ \ \ \ \ \ \ \ \isacommand{have}\isamarkupfalse%
\ {\isachardoublequoteopen}{\isasymforall}\ G\ H{\isachardot}\ \isactrlbold {\isasymnot}{\isacharparenleft}G\ \isactrlbold {\isasymand}\ H{\isacharparenright}\ {\isasymin}\ S\ {\isasymlongrightarrow}\ \isactrlbold {\isasymnot}\ G\ {\isasymin}\ S\ {\isasymor}\ \isactrlbold {\isasymnot}\ H\ {\isasymin}\ S{\isachardoublequoteclose}\isanewline
\ \ \ \ \ \ \ \ \ \ \isacommand{using}\isamarkupfalse%
\ assms\ \isacommand{by}\isamarkupfalse%
\ {\isacharparenleft}iprover\ elim{\isacharcolon}\ conjunct{\isadigit{2}}\ conjunct{\isadigit{1}}{\isacharparenright}\isanewline
\ \ \ \ \ \ \ \ \isacommand{thus}\isamarkupfalse%
\ {\isachardoublequoteopen}F\ {\isasymin}\ S\ {\isasymlongrightarrow}\ G\ {\isasymin}\ S\ {\isasymor}\ H\ {\isasymin}\ S{\isachardoublequoteclose}\isanewline
\ \ \ \ \ \ \ \ \ \ \isacommand{using}\isamarkupfalse%
\ {\isacartoucheopen}F\ {\isacharequal}\ \isactrlbold {\isasymnot}{\isacharparenleft}G{\isadigit{2}}\ \isactrlbold {\isasymand}\ H{\isadigit{2}}{\isacharparenright}{\isacartoucheclose}\ {\isacartoucheopen}G\ {\isacharequal}\ \isactrlbold {\isasymnot}\ G{\isadigit{2}}{\isacartoucheclose}\ {\isacartoucheopen}H\ {\isacharequal}\ \isactrlbold {\isasymnot}\ H{\isadigit{2}}{\isacartoucheclose}\ \isacommand{by}\isamarkupfalse%
\ {\isacharparenleft}iprover\ elim{\isacharcolon}\ allE{\isacharparenright}\isanewline
\ \ \ \ \ \ \isacommand{next}\isamarkupfalse%
\isanewline
\ \ \ \ \ \ \ \ \isacommand{assume}\isamarkupfalse%
\ {\isachardoublequoteopen}F\ {\isacharequal}\ \isactrlbold {\isasymnot}\ {\isacharparenleft}\isactrlbold {\isasymnot}\ G{\isacharparenright}\ {\isasymand}\ H\ {\isacharequal}\ G{\isachardoublequoteclose}\isanewline
\ \ \ \ \ \ \ \ \isacommand{then}\isamarkupfalse%
\ \isacommand{have}\isamarkupfalse%
\ {\isachardoublequoteopen}F\ {\isacharequal}\ \isactrlbold {\isasymnot}\ {\isacharparenleft}\isactrlbold {\isasymnot}\ G{\isacharparenright}{\isachardoublequoteclose}\ \isanewline
\ \ \ \ \ \ \ \ \ \ \isacommand{by}\isamarkupfalse%
\ {\isacharparenleft}rule\ conjunct{\isadigit{1}}{\isacharparenright}\isanewline
\ \ \ \ \ \ \ \ \isacommand{have}\isamarkupfalse%
\ {\isachardoublequoteopen}H\ {\isacharequal}\ G{\isachardoublequoteclose}\isanewline
\ \ \ \ \ \ \ \ \ \ \isacommand{using}\isamarkupfalse%
\ {\isacartoucheopen}F\ {\isacharequal}\ \isactrlbold {\isasymnot}\ {\isacharparenleft}\isactrlbold {\isasymnot}\ G{\isacharparenright}\ {\isasymand}\ H\ {\isacharequal}\ G{\isacartoucheclose}\ \isacommand{by}\isamarkupfalse%
\ {\isacharparenleft}rule\ conjunct{\isadigit{2}}{\isacharparenright}\isanewline
\ \ \ \ \ \ \ \ \isacommand{have}\isamarkupfalse%
\ {\isachardoublequoteopen}{\isasymforall}\ G{\isachardot}\ \isactrlbold {\isasymnot}\ {\isacharparenleft}\isactrlbold {\isasymnot}\ G{\isacharparenright}\ {\isasymin}\ S\ {\isasymlongrightarrow}\ G\ {\isasymin}\ S{\isachardoublequoteclose}\isanewline
\ \ \ \ \ \ \ \ \ \ \isacommand{using}\isamarkupfalse%
\ assms\ \isacommand{by}\isamarkupfalse%
\ {\isacharparenleft}iprover\ elim{\isacharcolon}\ conjunct{\isadigit{2}}\ conjunct{\isadigit{1}}{\isacharparenright}\isanewline
\ \ \ \ \ \ \ \ \isacommand{then}\isamarkupfalse%
\ \isacommand{have}\isamarkupfalse%
\ {\isachardoublequoteopen}\isactrlbold {\isasymnot}\ {\isacharparenleft}\isactrlbold {\isasymnot}\ G{\isacharparenright}\ {\isasymin}\ S\ {\isasymlongrightarrow}\ G\ {\isasymin}\ S{\isachardoublequoteclose}\isanewline
\ \ \ \ \ \ \ \ \ \ \isacommand{by}\isamarkupfalse%
\ {\isacharparenleft}rule\ allE{\isacharparenright}\isanewline
\ \ \ \ \ \ \ \ \isacommand{then}\isamarkupfalse%
\ \isacommand{have}\isamarkupfalse%
\ {\isachardoublequoteopen}F\ {\isasymin}\ S\ {\isasymlongrightarrow}\ G\ {\isasymin}\ S{\isachardoublequoteclose}\isanewline
\ \ \ \ \ \ \ \ \ \ \isacommand{by}\isamarkupfalse%
\ {\isacharparenleft}simp\ only{\isacharcolon}\ {\isacartoucheopen}F\ {\isacharequal}\ \isactrlbold {\isasymnot}\ {\isacharparenleft}\isactrlbold {\isasymnot}\ G{\isacharparenright}{\isacartoucheclose}{\isacharparenright}\isanewline
\ \ \ \ \ \ \ \ \isacommand{then}\isamarkupfalse%
\ \isacommand{have}\isamarkupfalse%
\ {\isachardoublequoteopen}F\ {\isasymin}\ S\ {\isasymlongrightarrow}\ G\ {\isasymin}\ S\ {\isasymor}\ G\ {\isasymin}\ S{\isachardoublequoteclose}\isanewline
\ \ \ \ \ \ \ \ \ \ \isacommand{by}\isamarkupfalse%
\ {\isacharparenleft}simp\ only{\isacharcolon}\ disj{\isacharunderscore}absorb{\isacharparenright}\isanewline
\ \ \ \ \ \ \ \ \isacommand{thus}\isamarkupfalse%
\ {\isachardoublequoteopen}F\ {\isasymin}\ S\ {\isasymlongrightarrow}\ G\ {\isasymin}\ S\ {\isasymor}\ H\ {\isasymin}\ S{\isachardoublequoteclose}\isanewline
\ \ \ \ \ \ \ \ \isacommand{by}\isamarkupfalse%
\ {\isacharparenleft}simp\ only{\isacharcolon}\ {\isacartoucheopen}H\ {\isacharequal}\ G{\isacartoucheclose}{\isacharparenright}\isanewline
\ \ \ \ \ \ \isacommand{qed}\isamarkupfalse%
\isanewline
\ \ \ \ \isacommand{qed}\isamarkupfalse%
\isanewline
\ \ \isacommand{qed}\isamarkupfalse%
\isanewline
\isacommand{qed}\isamarkupfalse%
%
\endisatagproof
{\isafoldproof}%
%
\isadelimproof
%
\endisadelimproof
%
\begin{isamarkuptext}%
Finalmente, podemos demostrar detalladamente esta primera implicación de la
  equivalencia del lema en Isabelle.%
\end{isamarkuptext}\isamarkuptrue%
\isacommand{lemma}\isamarkupfalse%
\ Hintikka{\isacharunderscore}alt{\isadigit{1}}{\isacharcolon}\isanewline
\ \ \isakeyword{assumes}\ {\isachardoublequoteopen}Hintikka\ S{\isachardoublequoteclose}\isanewline
\ \ \isakeyword{shows}\ {\isachardoublequoteopen}{\isasymbottom}\ {\isasymnotin}\ S\isanewline
{\isasymand}\ {\isacharparenleft}{\isasymforall}k{\isachardot}\ Atom\ k\ {\isasymin}\ S\ {\isasymlongrightarrow}\ \isactrlbold {\isasymnot}\ {\isacharparenleft}Atom\ k{\isacharparenright}\ {\isasymin}\ S\ {\isasymlongrightarrow}\ False{\isacharparenright}\isanewline
{\isasymand}\ {\isacharparenleft}{\isasymforall}F\ G\ H{\isachardot}\ Con\ F\ G\ H\ {\isasymlongrightarrow}\ F\ {\isasymin}\ S\ {\isasymlongrightarrow}\ G\ {\isasymin}\ S\ {\isasymand}\ H\ {\isasymin}\ S{\isacharparenright}\isanewline
{\isasymand}\ {\isacharparenleft}{\isasymforall}F\ G\ H{\isachardot}\ Dis\ F\ G\ H\ {\isasymlongrightarrow}\ F\ {\isasymin}\ S\ {\isasymlongrightarrow}\ G\ {\isasymin}\ S\ {\isasymor}\ H\ {\isasymin}\ S{\isacharparenright}{\isachardoublequoteclose}\isanewline
%
\isadelimproof
%
\endisadelimproof
%
\isatagproof
\isacommand{proof}\isamarkupfalse%
\ {\isacharminus}\isanewline
\ \ \isacommand{have}\isamarkupfalse%
\ Hk{\isacharcolon}{\isachardoublequoteopen}{\isacharparenleft}{\isasymbottom}\ {\isasymnotin}\ S\isanewline
\ \ {\isasymand}\ {\isacharparenleft}{\isasymforall}k{\isachardot}\ Atom\ k\ {\isasymin}\ S\ {\isasymlongrightarrow}\ \isactrlbold {\isasymnot}\ {\isacharparenleft}Atom\ k{\isacharparenright}\ {\isasymin}\ S\ {\isasymlongrightarrow}\ False{\isacharparenright}\isanewline
\ \ {\isasymand}\ {\isacharparenleft}{\isasymforall}G\ H{\isachardot}\ G\ \isactrlbold {\isasymand}\ H\ {\isasymin}\ S\ {\isasymlongrightarrow}\ G\ {\isasymin}\ S\ {\isasymand}\ H\ {\isasymin}\ S{\isacharparenright}\isanewline
\ \ {\isasymand}\ {\isacharparenleft}{\isasymforall}G\ H{\isachardot}\ G\ \isactrlbold {\isasymor}\ H\ {\isasymin}\ S\ {\isasymlongrightarrow}\ G\ {\isasymin}\ S\ {\isasymor}\ H\ {\isasymin}\ S{\isacharparenright}\isanewline
\ \ {\isasymand}\ {\isacharparenleft}{\isasymforall}G\ H{\isachardot}\ G\ \isactrlbold {\isasymrightarrow}\ H\ {\isasymin}\ S\ {\isasymlongrightarrow}\ \isactrlbold {\isasymnot}G\ {\isasymin}\ S\ {\isasymor}\ H\ {\isasymin}\ S{\isacharparenright}\isanewline
\ \ {\isasymand}\ {\isacharparenleft}{\isasymforall}G{\isachardot}\ \isactrlbold {\isasymnot}\ {\isacharparenleft}\isactrlbold {\isasymnot}G{\isacharparenright}\ {\isasymin}\ S\ {\isasymlongrightarrow}\ G\ {\isasymin}\ S{\isacharparenright}\isanewline
\ \ {\isasymand}\ {\isacharparenleft}{\isasymforall}G\ H{\isachardot}\ \isactrlbold {\isasymnot}{\isacharparenleft}G\ \isactrlbold {\isasymand}\ H{\isacharparenright}\ {\isasymin}\ S\ {\isasymlongrightarrow}\ \isactrlbold {\isasymnot}\ G\ {\isasymin}\ S\ {\isasymor}\ \isactrlbold {\isasymnot}\ H\ {\isasymin}\ S{\isacharparenright}\isanewline
\ \ {\isasymand}\ {\isacharparenleft}{\isasymforall}G\ H{\isachardot}\ \isactrlbold {\isasymnot}{\isacharparenleft}G\ \isactrlbold {\isasymor}\ H{\isacharparenright}\ {\isasymin}\ S\ {\isasymlongrightarrow}\ \isactrlbold {\isasymnot}\ G\ {\isasymin}\ S\ {\isasymand}\ \isactrlbold {\isasymnot}\ H\ {\isasymin}\ S{\isacharparenright}\isanewline
\ \ {\isasymand}\ {\isacharparenleft}{\isasymforall}G\ H{\isachardot}\ \isactrlbold {\isasymnot}{\isacharparenleft}G\ \isactrlbold {\isasymrightarrow}\ H{\isacharparenright}\ {\isasymin}\ S\ {\isasymlongrightarrow}\ G\ {\isasymin}\ S\ {\isasymand}\ \isactrlbold {\isasymnot}\ H\ {\isasymin}\ S{\isacharparenright}{\isacharparenright}{\isachardoublequoteclose}\isanewline
\ \ \ \ \isacommand{using}\isamarkupfalse%
\ assms\ \isacommand{by}\isamarkupfalse%
\ {\isacharparenleft}rule\ auxEq{\isacharparenright}\isanewline
\ \ \isacommand{then}\isamarkupfalse%
\ \isacommand{have}\isamarkupfalse%
\ C{\isadigit{1}}{\isacharcolon}\ {\isachardoublequoteopen}{\isasymbottom}\ {\isasymnotin}\ S{\isachardoublequoteclose}\isanewline
\ \ \ \ \isacommand{by}\isamarkupfalse%
\ {\isacharparenleft}rule\ conjunct{\isadigit{1}}{\isacharparenright}\isanewline
\ \ \isacommand{have}\isamarkupfalse%
\ C{\isadigit{2}}{\isacharcolon}\ {\isachardoublequoteopen}{\isasymforall}k{\isachardot}\ Atom\ k\ {\isasymin}\ S\ {\isasymlongrightarrow}\ \isactrlbold {\isasymnot}\ {\isacharparenleft}Atom\ k{\isacharparenright}\ {\isasymin}\ S\ {\isasymlongrightarrow}\ False{\isachardoublequoteclose}\isanewline
\ \ \ \ \isacommand{using}\isamarkupfalse%
\ Hk\ \isacommand{by}\isamarkupfalse%
\ {\isacharparenleft}iprover\ elim{\isacharcolon}\ conjunct{\isadigit{2}}\ conjunct{\isadigit{1}}{\isacharparenright}\isanewline
\ \ \isacommand{have}\isamarkupfalse%
\ C{\isadigit{3}}{\isacharcolon}\ {\isachardoublequoteopen}{\isasymforall}F\ G\ H{\isachardot}\ Con\ F\ G\ H\ {\isasymlongrightarrow}\ F\ {\isasymin}\ S\ {\isasymlongrightarrow}\ G\ {\isasymin}\ S\ {\isasymand}\ H\ {\isasymin}\ S{\isachardoublequoteclose}\isanewline
\ \ \isacommand{proof}\isamarkupfalse%
\ {\isacharparenleft}rule\ allI{\isacharparenright}{\isacharplus}\isanewline
\ \ \ \ \isacommand{fix}\isamarkupfalse%
\ F\ G\ H\isanewline
\ \ \ \ \isacommand{have}\isamarkupfalse%
\ C{\isadigit{3}}{\isadigit{1}}{\isacharcolon}{\isachardoublequoteopen}{\isasymforall}G\ H{\isachardot}\ G\ \isactrlbold {\isasymand}\ H\ {\isasymin}\ S\ {\isasymlongrightarrow}\ G\ {\isasymin}\ S\ {\isasymand}\ H\ {\isasymin}\ S{\isachardoublequoteclose}\isanewline
\ \ \ \ \ \ \isacommand{using}\isamarkupfalse%
\ Hk\ \isacommand{by}\isamarkupfalse%
\ {\isacharparenleft}iprover\ elim{\isacharcolon}\ conjunct{\isadigit{2}}\ conjunct{\isadigit{1}}{\isacharparenright}\isanewline
\ \ \ \ \isacommand{have}\isamarkupfalse%
\ C{\isadigit{3}}{\isadigit{2}}{\isacharcolon}{\isachardoublequoteopen}{\isasymforall}G{\isachardot}\ \isactrlbold {\isasymnot}\ {\isacharparenleft}\isactrlbold {\isasymnot}\ G{\isacharparenright}\ {\isasymin}\ S\ {\isasymlongrightarrow}\ G\ {\isasymin}\ S{\isachardoublequoteclose}\isanewline
\ \ \ \ \ \ \isacommand{using}\isamarkupfalse%
\ Hk\ \isacommand{by}\isamarkupfalse%
\ {\isacharparenleft}iprover\ elim{\isacharcolon}\ conjunct{\isadigit{2}}\ conjunct{\isadigit{1}}{\isacharparenright}\isanewline
\ \ \ \ \isacommand{have}\isamarkupfalse%
\ C{\isadigit{3}}{\isadigit{3}}{\isacharcolon}{\isachardoublequoteopen}{\isasymforall}G\ H{\isachardot}\ \isactrlbold {\isasymnot}{\isacharparenleft}G\ \isactrlbold {\isasymor}\ H{\isacharparenright}\ {\isasymin}\ S\ {\isasymlongrightarrow}\ \isactrlbold {\isasymnot}\ G\ {\isasymin}\ S\ {\isasymand}\ \isactrlbold {\isasymnot}\ H\ {\isasymin}\ S{\isachardoublequoteclose}\isanewline
\ \ \ \ \ \ \isacommand{using}\isamarkupfalse%
\ Hk\ \isacommand{by}\isamarkupfalse%
\ {\isacharparenleft}iprover\ elim{\isacharcolon}\ conjunct{\isadigit{2}}\ conjunct{\isadigit{1}}{\isacharparenright}\isanewline
\ \ \ \ \isacommand{have}\isamarkupfalse%
\ C{\isadigit{3}}{\isadigit{4}}{\isacharcolon}{\isachardoublequoteopen}{\isasymforall}G\ H{\isachardot}\ \isactrlbold {\isasymnot}{\isacharparenleft}G\ \isactrlbold {\isasymrightarrow}\ H{\isacharparenright}\ {\isasymin}\ S\ {\isasymlongrightarrow}\ G\ {\isasymin}\ S\ {\isasymand}\ \isactrlbold {\isasymnot}\ H\ {\isasymin}\ S{\isachardoublequoteclose}\isanewline
\ \ \ \ \ \ \isacommand{using}\isamarkupfalse%
\ Hk\ \isacommand{by}\isamarkupfalse%
\ {\isacharparenleft}iprover\ elim{\isacharcolon}\ conjunct{\isadigit{2}}\ conjunct{\isadigit{1}}{\isacharparenright}\isanewline
\ \ \ \ \isacommand{have}\isamarkupfalse%
\ {\isachardoublequoteopen}{\isacharparenleft}{\isasymforall}G\ H{\isachardot}\ G\ \isactrlbold {\isasymand}\ H\ {\isasymin}\ S\ {\isasymlongrightarrow}\ G\ {\isasymin}\ S\ {\isasymand}\ H\ {\isasymin}\ S{\isacharparenright}\isanewline
\ \ \ \ \ \ \ \ \ \ {\isasymand}\ {\isacharparenleft}{\isasymforall}G{\isachardot}\ \isactrlbold {\isasymnot}\ {\isacharparenleft}\isactrlbold {\isasymnot}\ G{\isacharparenright}\ {\isasymin}\ S\ {\isasymlongrightarrow}\ G\ {\isasymin}\ S{\isacharparenright}\isanewline
\ \ \ \ \ \ \ \ \ \ {\isasymand}\ {\isacharparenleft}{\isasymforall}G\ H{\isachardot}\ \isactrlbold {\isasymnot}{\isacharparenleft}G\ \isactrlbold {\isasymor}\ H{\isacharparenright}\ {\isasymin}\ S\ {\isasymlongrightarrow}\ \isactrlbold {\isasymnot}\ G\ {\isasymin}\ S\ {\isasymand}\ \isactrlbold {\isasymnot}\ H\ {\isasymin}\ S{\isacharparenright}\isanewline
\ \ \ \ \ \ \ \ \ \ {\isasymand}\ {\isacharparenleft}{\isasymforall}G\ H{\isachardot}\ \isactrlbold {\isasymnot}{\isacharparenleft}G\ \isactrlbold {\isasymrightarrow}\ H{\isacharparenright}\ {\isasymin}\ S\ {\isasymlongrightarrow}\ G\ {\isasymin}\ S\ {\isasymand}\ \isactrlbold {\isasymnot}\ H\ {\isasymin}\ S{\isacharparenright}{\isachardoublequoteclose}\ \isanewline
\ \ \ \ \ \ \isacommand{using}\isamarkupfalse%
\ C{\isadigit{3}}{\isadigit{1}}\ C{\isadigit{3}}{\isadigit{2}}\ C{\isadigit{3}}{\isadigit{3}}\ C{\isadigit{3}}{\isadigit{4}}\ \isacommand{by}\isamarkupfalse%
\ {\isacharparenleft}iprover\ intro{\isacharcolon}\ conjI{\isacharparenright}\isanewline
\ \ \ \ \isacommand{thus}\isamarkupfalse%
\ {\isachardoublequoteopen}Con\ F\ G\ H\ {\isasymlongrightarrow}\ F\ {\isasymin}\ S\ {\isasymlongrightarrow}\ G\ {\isasymin}\ S\ {\isasymand}\ H\ {\isasymin}\ S{\isachardoublequoteclose}\isanewline
\ \ \ \ \ \ \isacommand{by}\isamarkupfalse%
\ {\isacharparenleft}rule\ Hintikka{\isacharunderscore}alt{\isadigit{1}}Con{\isacharparenright}\isanewline
\ \ \isacommand{qed}\isamarkupfalse%
\isanewline
\ \ \isacommand{have}\isamarkupfalse%
\ C{\isadigit{4}}{\isacharcolon}{\isachardoublequoteopen}{\isasymforall}F\ G\ H{\isachardot}\ Dis\ F\ G\ H\ {\isasymlongrightarrow}\ F\ {\isasymin}\ S\ {\isasymlongrightarrow}\ G\ {\isasymin}\ S\ {\isasymor}\ H\ {\isasymin}\ S{\isachardoublequoteclose}\isanewline
\ \ \isacommand{proof}\isamarkupfalse%
\ {\isacharparenleft}rule\ allI{\isacharparenright}{\isacharplus}\isanewline
\ \ \ \ \isacommand{fix}\isamarkupfalse%
\ F\ G\ H\isanewline
\ \ \ \ \isacommand{have}\isamarkupfalse%
\ C{\isadigit{4}}{\isadigit{1}}{\isacharcolon}{\isachardoublequoteopen}{\isasymforall}G\ H{\isachardot}\ G\ \isactrlbold {\isasymor}\ H\ {\isasymin}\ S\ {\isasymlongrightarrow}\ G\ {\isasymin}\ S\ {\isasymor}\ H\ {\isasymin}\ S{\isachardoublequoteclose}\isanewline
\ \ \ \ \ \ \isacommand{using}\isamarkupfalse%
\ Hk\ \isacommand{by}\isamarkupfalse%
\ {\isacharparenleft}iprover\ elim{\isacharcolon}\ conjunct{\isadigit{2}}\ conjunct{\isadigit{1}}{\isacharparenright}\isanewline
\ \ \ \ \isacommand{have}\isamarkupfalse%
\ C{\isadigit{4}}{\isadigit{2}}{\isacharcolon}{\isachardoublequoteopen}{\isasymforall}G\ H{\isachardot}\ G\ \isactrlbold {\isasymrightarrow}\ H\ {\isasymin}\ S\ {\isasymlongrightarrow}\ \isactrlbold {\isasymnot}\ G\ {\isasymin}\ S\ {\isasymor}\ H\ {\isasymin}\ S{\isachardoublequoteclose}\isanewline
\ \ \ \ \ \ \isacommand{using}\isamarkupfalse%
\ Hk\ \isacommand{by}\isamarkupfalse%
\ {\isacharparenleft}iprover\ elim{\isacharcolon}\ conjunct{\isadigit{2}}\ conjunct{\isadigit{1}}{\isacharparenright}\isanewline
\ \ \ \ \isacommand{have}\isamarkupfalse%
\ C{\isadigit{4}}{\isadigit{3}}{\isacharcolon}{\isachardoublequoteopen}{\isasymforall}G{\isachardot}\ \isactrlbold {\isasymnot}\ {\isacharparenleft}\isactrlbold {\isasymnot}\ G{\isacharparenright}\ {\isasymin}\ S\ {\isasymlongrightarrow}\ G\ {\isasymin}\ S{\isachardoublequoteclose}\isanewline
\ \ \ \ \ \ \isacommand{using}\isamarkupfalse%
\ Hk\ \isacommand{by}\isamarkupfalse%
\ {\isacharparenleft}iprover\ elim{\isacharcolon}\ conjunct{\isadigit{2}}\ conjunct{\isadigit{1}}{\isacharparenright}\isanewline
\ \ \ \ \isacommand{have}\isamarkupfalse%
\ C{\isadigit{4}}{\isadigit{4}}{\isacharcolon}{\isachardoublequoteopen}{\isasymforall}G\ H{\isachardot}\ \isactrlbold {\isasymnot}{\isacharparenleft}G\ \isactrlbold {\isasymand}\ H{\isacharparenright}\ {\isasymin}\ S\ {\isasymlongrightarrow}\ \isactrlbold {\isasymnot}\ G\ {\isasymin}\ S\ {\isasymor}\ \isactrlbold {\isasymnot}\ H\ {\isasymin}\ S{\isachardoublequoteclose}\isanewline
\ \ \ \ \ \ \isacommand{using}\isamarkupfalse%
\ Hk\ \isacommand{by}\isamarkupfalse%
\ {\isacharparenleft}iprover\ elim{\isacharcolon}\ conjunct{\isadigit{2}}\ conjunct{\isadigit{1}}{\isacharparenright}\isanewline
\ \ \ \ \isacommand{have}\isamarkupfalse%
\ {\isachardoublequoteopen}{\isacharparenleft}{\isasymforall}G\ H{\isachardot}\ G\ \isactrlbold {\isasymor}\ H\ {\isasymin}\ S\ {\isasymlongrightarrow}\ G\ {\isasymin}\ S\ {\isasymor}\ H\ {\isasymin}\ S{\isacharparenright}\isanewline
\ \ \ \ \ \ \ \ \ \ {\isasymand}\ {\isacharparenleft}{\isasymforall}G\ H{\isachardot}\ G\ \isactrlbold {\isasymrightarrow}\ H\ {\isasymin}\ S\ {\isasymlongrightarrow}\ \isactrlbold {\isasymnot}\ G\ {\isasymin}\ S\ {\isasymor}\ H\ {\isasymin}\ S{\isacharparenright}\isanewline
\ \ \ \ \ \ \ \ \ \ {\isasymand}\ {\isacharparenleft}{\isasymforall}G{\isachardot}\ \isactrlbold {\isasymnot}\ {\isacharparenleft}\isactrlbold {\isasymnot}\ G{\isacharparenright}\ {\isasymin}\ S\ {\isasymlongrightarrow}\ G\ {\isasymin}\ S{\isacharparenright}\isanewline
\ \ \ \ \ \ \ \ \ \ {\isasymand}\ {\isacharparenleft}{\isasymforall}G\ H{\isachardot}\ \isactrlbold {\isasymnot}{\isacharparenleft}G\ \isactrlbold {\isasymand}\ H{\isacharparenright}\ {\isasymin}\ S\ {\isasymlongrightarrow}\ \isactrlbold {\isasymnot}\ G\ {\isasymin}\ S\ {\isasymor}\ \isactrlbold {\isasymnot}\ H\ {\isasymin}\ S{\isacharparenright}{\isachardoublequoteclose}\isanewline
\ \ \ \ \ \ \isacommand{using}\isamarkupfalse%
\ C{\isadigit{4}}{\isadigit{1}}\ C{\isadigit{4}}{\isadigit{2}}\ C{\isadigit{4}}{\isadigit{3}}\ C{\isadigit{4}}{\isadigit{4}}\ \isacommand{by}\isamarkupfalse%
\ {\isacharparenleft}iprover\ intro{\isacharcolon}\ conjI{\isacharparenright}\isanewline
\ \ \ \ \isacommand{thus}\isamarkupfalse%
\ {\isachardoublequoteopen}Dis\ F\ G\ H\ {\isasymlongrightarrow}\ F\ {\isasymin}\ S\ {\isasymlongrightarrow}\ G\ {\isasymin}\ S\ {\isasymor}\ H\ {\isasymin}\ S{\isachardoublequoteclose}\isanewline
\ \ \ \ \ \ \isacommand{by}\isamarkupfalse%
\ {\isacharparenleft}rule\ Hintikka{\isacharunderscore}alt{\isadigit{1}}Dis{\isacharparenright}\isanewline
\ \ \isacommand{qed}\isamarkupfalse%
\isanewline
\ \ \isacommand{show}\isamarkupfalse%
\ {\isachardoublequoteopen}{\isasymbottom}\ {\isasymnotin}\ S\isanewline
\ \ {\isasymand}\ {\isacharparenleft}{\isasymforall}k{\isachardot}\ Atom\ k\ {\isasymin}\ S\ {\isasymlongrightarrow}\ \isactrlbold {\isasymnot}\ {\isacharparenleft}Atom\ k{\isacharparenright}\ {\isasymin}\ S\ {\isasymlongrightarrow}\ False{\isacharparenright}\isanewline
\ \ {\isasymand}\ {\isacharparenleft}{\isasymforall}F\ G\ H{\isachardot}\ Con\ F\ G\ H\ {\isasymlongrightarrow}\ F\ {\isasymin}\ S\ {\isasymlongrightarrow}\ G\ {\isasymin}\ S\ {\isasymand}\ H\ {\isasymin}\ S{\isacharparenright}\isanewline
\ \ {\isasymand}\ {\isacharparenleft}{\isasymforall}F\ G\ H{\isachardot}\ Dis\ F\ G\ H\ {\isasymlongrightarrow}\ F\ {\isasymin}\ S\ {\isasymlongrightarrow}\ G\ {\isasymin}\ S\ {\isasymor}\ H\ {\isasymin}\ S{\isacharparenright}{\isachardoublequoteclose}\isanewline
\ \ \ \ \isacommand{using}\isamarkupfalse%
\ C{\isadigit{1}}\ C{\isadigit{2}}\ C{\isadigit{3}}\ C{\isadigit{4}}\ \isacommand{by}\isamarkupfalse%
\ {\isacharparenleft}iprover\ intro{\isacharcolon}\ conjI{\isacharparenright}\isanewline
\isacommand{qed}\isamarkupfalse%
%
\endisatagproof
{\isafoldproof}%
%
\isadelimproof
%
\endisadelimproof
%
\begin{isamarkuptext}%
Por último, probamos la implicación recíproca de forma detallada en Isabelle mediante
  el siguiente lema.%
\end{isamarkuptext}\isamarkuptrue%
\isacommand{lemma}\isamarkupfalse%
\ Hintikka{\isacharunderscore}alt{\isadigit{2}}{\isacharcolon}\isanewline
\ \ \isakeyword{assumes}\ {\isachardoublequoteopen}{\isasymbottom}\ {\isasymnotin}\ S\isanewline
{\isasymand}\ {\isacharparenleft}{\isasymforall}k{\isachardot}\ Atom\ k\ {\isasymin}\ S\ {\isasymlongrightarrow}\ \isactrlbold {\isasymnot}\ {\isacharparenleft}Atom\ k{\isacharparenright}\ {\isasymin}\ S\ {\isasymlongrightarrow}\ False{\isacharparenright}\isanewline
{\isasymand}\ {\isacharparenleft}{\isasymforall}F\ G\ H{\isachardot}\ Con\ F\ G\ H\ {\isasymlongrightarrow}\ F\ {\isasymin}\ S\ {\isasymlongrightarrow}\ G\ {\isasymin}\ S\ {\isasymand}\ H\ {\isasymin}\ S{\isacharparenright}\ \isanewline
{\isasymand}\ {\isacharparenleft}{\isasymforall}F\ G\ H{\isachardot}\ Dis\ F\ G\ H\ {\isasymlongrightarrow}\ F\ {\isasymin}\ S\ {\isasymlongrightarrow}\ G\ {\isasymin}\ S\ {\isasymor}\ H\ {\isasymin}\ S{\isacharparenright}{\isachardoublequoteclose}\ \ \isanewline
\ \ \isakeyword{shows}\ {\isachardoublequoteopen}Hintikka\ S{\isachardoublequoteclose}\isanewline
%
\isadelimproof
%
\endisadelimproof
%
\isatagproof
\isacommand{proof}\isamarkupfalse%
\ {\isacharminus}\isanewline
\ \ \isacommand{have}\isamarkupfalse%
\ Con{\isacharcolon}{\isachardoublequoteopen}{\isasymforall}F\ G\ H{\isachardot}\ Con\ F\ G\ H\ {\isasymlongrightarrow}\ F\ {\isasymin}\ S\ {\isasymlongrightarrow}\ G\ {\isasymin}\ S\ {\isasymand}\ H\ {\isasymin}\ S{\isachardoublequoteclose}\isanewline
\ \ \ \ \isacommand{using}\isamarkupfalse%
\ assms\ \isacommand{by}\isamarkupfalse%
\ {\isacharparenleft}iprover\ elim{\isacharcolon}\ conjunct{\isadigit{2}}\ conjunct{\isadigit{1}}{\isacharparenright}\isanewline
\ \ \isacommand{have}\isamarkupfalse%
\ Dis{\isacharcolon}{\isachardoublequoteopen}{\isasymforall}F\ G\ H{\isachardot}\ Dis\ F\ G\ H\ {\isasymlongrightarrow}\ F\ {\isasymin}\ S\ {\isasymlongrightarrow}\ G\ {\isasymin}\ S\ {\isasymor}\ H\ {\isasymin}\ S{\isachardoublequoteclose}\isanewline
\ \ \ \ \isacommand{using}\isamarkupfalse%
\ assms\ \isacommand{by}\isamarkupfalse%
\ {\isacharparenleft}iprover\ elim{\isacharcolon}\ conjunct{\isadigit{2}}\ conjunct{\isadigit{1}}{\isacharparenright}\isanewline
\ \ \isacommand{have}\isamarkupfalse%
\ {\isachardoublequoteopen}{\isasymbottom}\ {\isasymnotin}\ S\isanewline
\ \ {\isasymand}\ {\isacharparenleft}{\isasymforall}k{\isachardot}\ Atom\ k\ {\isasymin}\ S\ {\isasymlongrightarrow}\ \isactrlbold {\isasymnot}\ {\isacharparenleft}Atom\ k{\isacharparenright}\ {\isasymin}\ S\ {\isasymlongrightarrow}\ False{\isacharparenright}\isanewline
\ \ {\isasymand}\ {\isacharparenleft}{\isasymforall}G\ H{\isachardot}\ G\ \isactrlbold {\isasymand}\ H\ {\isasymin}\ S\ {\isasymlongrightarrow}\ G\ {\isasymin}\ S\ {\isasymand}\ H\ {\isasymin}\ S{\isacharparenright}\isanewline
\ \ {\isasymand}\ {\isacharparenleft}{\isasymforall}G\ H{\isachardot}\ G\ \isactrlbold {\isasymor}\ H\ {\isasymin}\ S\ {\isasymlongrightarrow}\ G\ {\isasymin}\ S\ {\isasymor}\ H\ {\isasymin}\ S{\isacharparenright}\isanewline
\ \ {\isasymand}\ {\isacharparenleft}{\isasymforall}G\ H{\isachardot}\ G\ \isactrlbold {\isasymrightarrow}\ H\ {\isasymin}\ S\ {\isasymlongrightarrow}\ \isactrlbold {\isasymnot}G\ {\isasymin}\ S\ {\isasymor}\ H\ {\isasymin}\ S{\isacharparenright}\isanewline
\ \ {\isasymand}\ {\isacharparenleft}{\isasymforall}G{\isachardot}\ \isactrlbold {\isasymnot}\ {\isacharparenleft}\isactrlbold {\isasymnot}G{\isacharparenright}\ {\isasymin}\ S\ {\isasymlongrightarrow}\ G\ {\isasymin}\ S{\isacharparenright}\isanewline
\ \ {\isasymand}\ {\isacharparenleft}{\isasymforall}G\ H{\isachardot}\ \isactrlbold {\isasymnot}{\isacharparenleft}G\ \isactrlbold {\isasymand}\ H{\isacharparenright}\ {\isasymin}\ S\ {\isasymlongrightarrow}\ \isactrlbold {\isasymnot}\ G\ {\isasymin}\ S\ {\isasymor}\ \isactrlbold {\isasymnot}\ H\ {\isasymin}\ S{\isacharparenright}\isanewline
\ \ {\isasymand}\ {\isacharparenleft}{\isasymforall}G\ H{\isachardot}\ \isactrlbold {\isasymnot}{\isacharparenleft}G\ \isactrlbold {\isasymor}\ H{\isacharparenright}\ {\isasymin}\ S\ {\isasymlongrightarrow}\ \isactrlbold {\isasymnot}\ G\ {\isasymin}\ S\ {\isasymand}\ \isactrlbold {\isasymnot}\ H\ {\isasymin}\ S{\isacharparenright}\isanewline
\ \ {\isasymand}\ {\isacharparenleft}{\isasymforall}G\ H{\isachardot}\ \isactrlbold {\isasymnot}{\isacharparenleft}G\ \isactrlbold {\isasymrightarrow}\ H{\isacharparenright}\ {\isasymin}\ S\ {\isasymlongrightarrow}\ G\ {\isasymin}\ S\ {\isasymand}\ \isactrlbold {\isasymnot}\ H\ {\isasymin}\ S{\isacharparenright}{\isachardoublequoteclose}\isanewline
\ \ \isacommand{proof}\isamarkupfalse%
\ {\isacharminus}\isanewline
\ \ \ \ \isacommand{have}\isamarkupfalse%
\ C{\isadigit{1}}{\isacharcolon}{\isachardoublequoteopen}{\isasymbottom}\ {\isasymnotin}\ S{\isachardoublequoteclose}\isanewline
\ \ \ \ \ \ \isacommand{using}\isamarkupfalse%
\ assms\ \isacommand{by}\isamarkupfalse%
\ {\isacharparenleft}rule\ conjunct{\isadigit{1}}{\isacharparenright}\isanewline
\ \ \ \ \isacommand{have}\isamarkupfalse%
\ C{\isadigit{2}}{\isacharcolon}{\isachardoublequoteopen}{\isasymforall}k{\isachardot}\ Atom\ k\ {\isasymin}\ S\ {\isasymlongrightarrow}\ \isactrlbold {\isasymnot}\ {\isacharparenleft}Atom\ k{\isacharparenright}\ {\isasymin}\ S\ {\isasymlongrightarrow}\ False{\isachardoublequoteclose}\isanewline
\ \ \ \ \ \ \isacommand{using}\isamarkupfalse%
\ assms\ \isacommand{by}\isamarkupfalse%
\ {\isacharparenleft}iprover\ elim{\isacharcolon}\ conjunct{\isadigit{2}}\ conjunct{\isadigit{1}}{\isacharparenright}\isanewline
\ \ \ \ \isacommand{have}\isamarkupfalse%
\ C{\isadigit{3}}{\isacharcolon}{\isachardoublequoteopen}{\isasymforall}G\ H{\isachardot}\ G\ \isactrlbold {\isasymand}\ H\ {\isasymin}\ S\ {\isasymlongrightarrow}\ G\ {\isasymin}\ S\ {\isasymand}\ H\ {\isasymin}\ S{\isachardoublequoteclose}\isanewline
\ \ \ \ \isacommand{proof}\isamarkupfalse%
\ {\isacharparenleft}rule\ allI{\isacharparenright}{\isacharplus}\isanewline
\ \ \ \ \ \ \isacommand{fix}\isamarkupfalse%
\ G\ H\isanewline
\ \ \ \ \ \ \isacommand{show}\isamarkupfalse%
\ {\isachardoublequoteopen}G\ \isactrlbold {\isasymand}\ H\ {\isasymin}\ S\ {\isasymlongrightarrow}\ G\ {\isasymin}\ S\ {\isasymand}\ H\ {\isasymin}\ S{\isachardoublequoteclose}\isanewline
\ \ \ \ \ \ \isacommand{proof}\isamarkupfalse%
\ {\isacharparenleft}rule\ impI{\isacharparenright}\isanewline
\ \ \ \ \ \ \ \ \isacommand{assume}\isamarkupfalse%
\ {\isachardoublequoteopen}G\ \isactrlbold {\isasymand}\ H\ {\isasymin}\ S{\isachardoublequoteclose}\isanewline
\ \ \ \ \ \ \ \ \isacommand{have}\isamarkupfalse%
\ {\isachardoublequoteopen}Con\ {\isacharparenleft}G\ \isactrlbold {\isasymand}\ H{\isacharparenright}\ G\ H{\isachardoublequoteclose}\isanewline
\ \ \ \ \ \ \ \ \ \ \isacommand{by}\isamarkupfalse%
\ {\isacharparenleft}simp\ only{\isacharcolon}\ Con{\isachardot}intros{\isacharparenleft}{\isadigit{1}}{\isacharparenright}{\isacharparenright}\isanewline
\ \ \ \ \ \ \ \ \isacommand{have}\isamarkupfalse%
\ {\isachardoublequoteopen}Con\ {\isacharparenleft}G\ \isactrlbold {\isasymand}\ H{\isacharparenright}\ G\ H\ {\isasymlongrightarrow}\ G\ \isactrlbold {\isasymand}\ H\ {\isasymin}\ S\ {\isasymlongrightarrow}\ G\ {\isasymin}\ S\ {\isasymand}\ H\ {\isasymin}\ S{\isachardoublequoteclose}\isanewline
\ \ \ \ \ \ \ \ \ \ \isacommand{using}\isamarkupfalse%
\ Con\ \isacommand{by}\isamarkupfalse%
\ {\isacharparenleft}iprover\ elim{\isacharcolon}\ allE{\isacharparenright}\isanewline
\ \ \ \ \ \ \ \ \isacommand{then}\isamarkupfalse%
\ \isacommand{have}\isamarkupfalse%
\ {\isachardoublequoteopen}G\ \isactrlbold {\isasymand}\ H\ {\isasymin}\ S\ {\isasymlongrightarrow}\ G\ {\isasymin}\ S\ {\isasymand}\ H\ {\isasymin}\ S{\isachardoublequoteclose}\isanewline
\ \ \ \ \ \ \ \ \ \ \isacommand{using}\isamarkupfalse%
\ {\isacartoucheopen}Con\ {\isacharparenleft}G\ \isactrlbold {\isasymand}\ H{\isacharparenright}\ G\ H{\isacartoucheclose}\ \isacommand{by}\isamarkupfalse%
\ {\isacharparenleft}rule\ mp{\isacharparenright}\isanewline
\ \ \ \ \ \ \ \ \isacommand{thus}\isamarkupfalse%
\ {\isachardoublequoteopen}G\ {\isasymin}\ S\ {\isasymand}\ H\ {\isasymin}\ S{\isachardoublequoteclose}\isanewline
\ \ \ \ \ \ \ \ \ \ \isacommand{using}\isamarkupfalse%
\ {\isacartoucheopen}G\ \isactrlbold {\isasymand}\ H\ {\isasymin}\ S{\isacartoucheclose}\ \isacommand{by}\isamarkupfalse%
\ {\isacharparenleft}rule\ mp{\isacharparenright}\isanewline
\ \ \ \ \ \ \isacommand{qed}\isamarkupfalse%
\isanewline
\ \ \ \ \isacommand{qed}\isamarkupfalse%
\isanewline
\ \ \ \ \isacommand{have}\isamarkupfalse%
\ C{\isadigit{4}}{\isacharcolon}{\isachardoublequoteopen}{\isasymforall}G\ H{\isachardot}\ G\ \isactrlbold {\isasymor}\ H\ {\isasymin}\ S\ {\isasymlongrightarrow}\ G\ {\isasymin}\ S\ {\isasymor}\ H\ {\isasymin}\ S{\isachardoublequoteclose}\isanewline
\ \ \ \ \isacommand{proof}\isamarkupfalse%
\ {\isacharparenleft}rule\ allI{\isacharparenright}{\isacharplus}\isanewline
\ \ \ \ \ \ \isacommand{fix}\isamarkupfalse%
\ G\ H\isanewline
\ \ \ \ \ \ \isacommand{show}\isamarkupfalse%
\ {\isachardoublequoteopen}G\ \isactrlbold {\isasymor}\ H\ {\isasymin}\ S\ {\isasymlongrightarrow}\ G\ {\isasymin}\ S\ {\isasymor}\ H\ {\isasymin}\ S{\isachardoublequoteclose}\isanewline
\ \ \ \ \ \ \isacommand{proof}\isamarkupfalse%
\ {\isacharparenleft}rule\ impI{\isacharparenright}\isanewline
\ \ \ \ \ \ \ \ \isacommand{assume}\isamarkupfalse%
\ {\isachardoublequoteopen}G\ \isactrlbold {\isasymor}\ H\ {\isasymin}\ S{\isachardoublequoteclose}\isanewline
\ \ \ \ \ \ \ \ \isacommand{have}\isamarkupfalse%
\ {\isachardoublequoteopen}Dis\ {\isacharparenleft}G\ \isactrlbold {\isasymor}\ H{\isacharparenright}\ G\ H{\isachardoublequoteclose}\isanewline
\ \ \ \ \ \ \ \ \ \ \isacommand{by}\isamarkupfalse%
\ {\isacharparenleft}simp\ only{\isacharcolon}\ Dis{\isachardot}intros{\isacharparenleft}{\isadigit{1}}{\isacharparenright}{\isacharparenright}\isanewline
\ \ \ \ \ \ \ \ \isacommand{have}\isamarkupfalse%
\ {\isachardoublequoteopen}Dis\ {\isacharparenleft}G\ \isactrlbold {\isasymor}\ H{\isacharparenright}\ G\ H\ {\isasymlongrightarrow}\ G\ \isactrlbold {\isasymor}\ H\ {\isasymin}\ S\ {\isasymlongrightarrow}\ G\ {\isasymin}\ S\ {\isasymor}\ H\ {\isasymin}\ S{\isachardoublequoteclose}\isanewline
\ \ \ \ \ \ \ \ \ \ \isacommand{using}\isamarkupfalse%
\ Dis\ \isacommand{by}\isamarkupfalse%
\ {\isacharparenleft}iprover\ elim{\isacharcolon}\ allE{\isacharparenright}\isanewline
\ \ \ \ \ \ \ \ \isacommand{then}\isamarkupfalse%
\ \isacommand{have}\isamarkupfalse%
\ {\isachardoublequoteopen}G\ \isactrlbold {\isasymor}\ H\ {\isasymin}\ S\ {\isasymlongrightarrow}\ G\ {\isasymin}\ S\ {\isasymor}\ H\ {\isasymin}\ S{\isachardoublequoteclose}\isanewline
\ \ \ \ \ \ \ \ \ \ \isacommand{using}\isamarkupfalse%
\ {\isacartoucheopen}Dis\ {\isacharparenleft}G\ \isactrlbold {\isasymor}\ H{\isacharparenright}\ G\ H{\isacartoucheclose}\ \isacommand{by}\isamarkupfalse%
\ {\isacharparenleft}rule\ mp{\isacharparenright}\isanewline
\ \ \ \ \ \ \ \ \isacommand{thus}\isamarkupfalse%
\ {\isachardoublequoteopen}G\ {\isasymin}\ S\ {\isasymor}\ H\ {\isasymin}\ S{\isachardoublequoteclose}\isanewline
\ \ \ \ \ \ \ \ \ \ \isacommand{using}\isamarkupfalse%
\ {\isacartoucheopen}G\ \isactrlbold {\isasymor}\ H\ {\isasymin}\ S{\isacartoucheclose}\ \isacommand{by}\isamarkupfalse%
\ {\isacharparenleft}rule\ mp{\isacharparenright}\isanewline
\ \ \ \ \ \ \isacommand{qed}\isamarkupfalse%
\isanewline
\ \ \ \ \isacommand{qed}\isamarkupfalse%
\isanewline
\ \ \ \ \isacommand{have}\isamarkupfalse%
\ C{\isadigit{5}}{\isacharcolon}{\isachardoublequoteopen}{\isasymforall}G\ H{\isachardot}\ G\ \isactrlbold {\isasymrightarrow}\ H\ {\isasymin}\ S\ {\isasymlongrightarrow}\ \isactrlbold {\isasymnot}\ G\ {\isasymin}\ S\ {\isasymor}\ H\ {\isasymin}\ S{\isachardoublequoteclose}\isanewline
\ \ \ \ \isacommand{proof}\isamarkupfalse%
\ {\isacharparenleft}rule\ allI{\isacharparenright}{\isacharplus}\isanewline
\ \ \ \ \ \ \isacommand{fix}\isamarkupfalse%
\ G\ H\isanewline
\ \ \ \ \ \ \isacommand{show}\isamarkupfalse%
\ {\isachardoublequoteopen}G\ \isactrlbold {\isasymrightarrow}\ H\ {\isasymin}\ S\ {\isasymlongrightarrow}\ \isactrlbold {\isasymnot}\ G\ {\isasymin}\ S\ {\isasymor}\ H\ {\isasymin}\ S{\isachardoublequoteclose}\isanewline
\ \ \ \ \ \ \isacommand{proof}\isamarkupfalse%
\ {\isacharparenleft}rule\ impI{\isacharparenright}\isanewline
\ \ \ \ \ \ \ \ \isacommand{assume}\isamarkupfalse%
\ {\isachardoublequoteopen}G\ \isactrlbold {\isasymrightarrow}\ H\ {\isasymin}\ S{\isachardoublequoteclose}\ \isanewline
\ \ \ \ \ \ \ \ \isacommand{have}\isamarkupfalse%
\ {\isachardoublequoteopen}Dis\ {\isacharparenleft}G\ \isactrlbold {\isasymrightarrow}\ H{\isacharparenright}\ {\isacharparenleft}\isactrlbold {\isasymnot}\ G{\isacharparenright}\ H{\isachardoublequoteclose}\isanewline
\ \ \ \ \ \ \ \ \ \ \isacommand{by}\isamarkupfalse%
\ {\isacharparenleft}simp\ only{\isacharcolon}\ Dis{\isachardot}intros{\isacharparenleft}{\isadigit{2}}{\isacharparenright}{\isacharparenright}\isanewline
\ \ \ \ \ \ \ \ \isacommand{have}\isamarkupfalse%
\ {\isachardoublequoteopen}Dis\ {\isacharparenleft}G\ \isactrlbold {\isasymrightarrow}\ H{\isacharparenright}\ {\isacharparenleft}\isactrlbold {\isasymnot}\ G{\isacharparenright}\ H\ {\isasymlongrightarrow}\ G\ \isactrlbold {\isasymrightarrow}\ H\ {\isasymin}\ S\ {\isasymlongrightarrow}\ \isactrlbold {\isasymnot}\ G\ {\isasymin}\ S\ {\isasymor}\ H\ {\isasymin}\ S{\isachardoublequoteclose}\isanewline
\ \ \ \ \ \ \ \ \ \ \isacommand{using}\isamarkupfalse%
\ Dis\ \isacommand{by}\isamarkupfalse%
\ {\isacharparenleft}iprover\ elim{\isacharcolon}\ allE{\isacharparenright}\isanewline
\ \ \ \ \ \ \ \ \isacommand{then}\isamarkupfalse%
\ \isacommand{have}\isamarkupfalse%
\ {\isachardoublequoteopen}G\ \isactrlbold {\isasymrightarrow}\ H\ {\isasymin}\ S\ {\isasymlongrightarrow}\ \isactrlbold {\isasymnot}\ G\ {\isasymin}\ S\ {\isasymor}\ H\ {\isasymin}\ S{\isachardoublequoteclose}\ \isanewline
\ \ \ \ \ \ \ \ \ \ \isacommand{using}\isamarkupfalse%
\ {\isacartoucheopen}Dis\ {\isacharparenleft}G\ \isactrlbold {\isasymrightarrow}\ H{\isacharparenright}\ {\isacharparenleft}\isactrlbold {\isasymnot}\ G{\isacharparenright}\ H{\isacartoucheclose}\ \isacommand{by}\isamarkupfalse%
\ {\isacharparenleft}rule\ mp{\isacharparenright}\isanewline
\ \ \ \ \ \ \ \ \isacommand{thus}\isamarkupfalse%
\ {\isachardoublequoteopen}\isactrlbold {\isasymnot}\ G\ {\isasymin}\ S\ {\isasymor}\ H\ {\isasymin}\ S{\isachardoublequoteclose}\isanewline
\ \ \ \ \ \ \ \ \ \ \isacommand{using}\isamarkupfalse%
\ {\isacartoucheopen}G\ \isactrlbold {\isasymrightarrow}\ H\ {\isasymin}\ S{\isacartoucheclose}\ \isacommand{by}\isamarkupfalse%
\ {\isacharparenleft}rule\ mp{\isacharparenright}\isanewline
\ \ \ \ \ \ \isacommand{qed}\isamarkupfalse%
\isanewline
\ \ \ \ \isacommand{qed}\isamarkupfalse%
\isanewline
\ \ \ \ \isacommand{have}\isamarkupfalse%
\ C{\isadigit{6}}{\isacharcolon}{\isachardoublequoteopen}{\isasymforall}G{\isachardot}\ \isactrlbold {\isasymnot}{\isacharparenleft}\isactrlbold {\isasymnot}\ G{\isacharparenright}\ {\isasymin}\ S\ {\isasymlongrightarrow}\ G\ {\isasymin}\ S{\isachardoublequoteclose}\isanewline
\ \ \ \ \isacommand{proof}\isamarkupfalse%
\ {\isacharparenleft}rule\ allI{\isacharparenright}\isanewline
\ \ \ \ \ \ \isacommand{fix}\isamarkupfalse%
\ G\isanewline
\ \ \ \ \ \ \isacommand{show}\isamarkupfalse%
\ {\isachardoublequoteopen}\isactrlbold {\isasymnot}{\isacharparenleft}\isactrlbold {\isasymnot}\ G{\isacharparenright}\ {\isasymin}\ S\ {\isasymlongrightarrow}\ G\ {\isasymin}\ S{\isachardoublequoteclose}\isanewline
\ \ \ \ \ \ \isacommand{proof}\isamarkupfalse%
\ {\isacharparenleft}rule\ impI{\isacharparenright}\isanewline
\ \ \ \ \ \ \ \ \isacommand{assume}\isamarkupfalse%
\ {\isachardoublequoteopen}\isactrlbold {\isasymnot}\ {\isacharparenleft}\isactrlbold {\isasymnot}\ G{\isacharparenright}\ {\isasymin}\ S{\isachardoublequoteclose}\ \isanewline
\ \ \ \ \ \ \ \ \isacommand{have}\isamarkupfalse%
\ {\isachardoublequoteopen}Con\ {\isacharparenleft}\isactrlbold {\isasymnot}\ {\isacharparenleft}\isactrlbold {\isasymnot}\ G{\isacharparenright}{\isacharparenright}\ G\ G{\isachardoublequoteclose}\isanewline
\ \ \ \ \ \ \ \ \ \ \isacommand{by}\isamarkupfalse%
\ {\isacharparenleft}simp\ only{\isacharcolon}\ Con{\isachardot}intros{\isacharparenleft}{\isadigit{4}}{\isacharparenright}{\isacharparenright}\isanewline
\ \ \ \ \ \ \ \ \isacommand{have}\isamarkupfalse%
\ {\isachardoublequoteopen}Con\ {\isacharparenleft}\isactrlbold {\isasymnot}{\isacharparenleft}\isactrlbold {\isasymnot}\ G{\isacharparenright}{\isacharparenright}\ G\ G\ {\isasymlongrightarrow}\ {\isacharparenleft}\isactrlbold {\isasymnot}{\isacharparenleft}\isactrlbold {\isasymnot}\ G{\isacharparenright}{\isacharparenright}\ {\isasymin}\ S\ {\isasymlongrightarrow}\ G\ {\isasymin}\ S\ {\isasymand}\ G\ {\isasymin}\ S{\isachardoublequoteclose}\isanewline
\ \ \ \ \ \ \ \ \ \ \isacommand{using}\isamarkupfalse%
\ Con\ \isacommand{by}\isamarkupfalse%
\ {\isacharparenleft}iprover\ elim{\isacharcolon}\ allE{\isacharparenright}\isanewline
\ \ \ \ \ \ \ \ \isacommand{then}\isamarkupfalse%
\ \isacommand{have}\isamarkupfalse%
\ {\isachardoublequoteopen}{\isacharparenleft}\isactrlbold {\isasymnot}{\isacharparenleft}\isactrlbold {\isasymnot}\ G{\isacharparenright}{\isacharparenright}\ {\isasymin}\ S\ {\isasymlongrightarrow}\ G\ {\isasymin}\ S\ {\isasymand}\ G\ {\isasymin}\ S{\isachardoublequoteclose}\isanewline
\ \ \ \ \ \ \ \ \ \ \isacommand{using}\isamarkupfalse%
\ {\isacartoucheopen}Con\ {\isacharparenleft}\isactrlbold {\isasymnot}\ {\isacharparenleft}\isactrlbold {\isasymnot}\ G{\isacharparenright}{\isacharparenright}\ G\ G{\isacartoucheclose}\ \isacommand{by}\isamarkupfalse%
\ {\isacharparenleft}rule\ mp{\isacharparenright}\isanewline
\ \ \ \ \ \ \ \ \isacommand{then}\isamarkupfalse%
\ \isacommand{have}\isamarkupfalse%
\ {\isachardoublequoteopen}G\ {\isasymin}\ S\ {\isasymand}\ G\ {\isasymin}\ S{\isachardoublequoteclose}\isanewline
\ \ \ \ \ \ \ \ \ \ \isacommand{using}\isamarkupfalse%
\ {\isacartoucheopen}\isactrlbold {\isasymnot}\ {\isacharparenleft}\isactrlbold {\isasymnot}\ G{\isacharparenright}\ {\isasymin}\ S{\isacartoucheclose}\ \isacommand{by}\isamarkupfalse%
\ {\isacharparenleft}rule\ mp{\isacharparenright}\isanewline
\ \ \ \ \ \ \ \ \isacommand{thus}\isamarkupfalse%
\ {\isachardoublequoteopen}G\ {\isasymin}\ S{\isachardoublequoteclose}\isanewline
\ \ \ \ \ \ \ \ \ \ \isacommand{by}\isamarkupfalse%
\ {\isacharparenleft}simp\ only{\isacharcolon}\ conj{\isacharunderscore}absorb{\isacharparenright}\isanewline
\ \ \ \ \ \ \isacommand{qed}\isamarkupfalse%
\isanewline
\ \ \ \ \isacommand{qed}\isamarkupfalse%
\isanewline
\ \ \ \ \isacommand{have}\isamarkupfalse%
\ C{\isadigit{7}}{\isacharcolon}{\isachardoublequoteopen}{\isasymforall}G\ H{\isachardot}\ \isactrlbold {\isasymnot}{\isacharparenleft}G\ \isactrlbold {\isasymand}\ H{\isacharparenright}\ {\isasymin}\ S\ {\isasymlongrightarrow}\ \isactrlbold {\isasymnot}\ G\ {\isasymin}\ S\ {\isasymor}\ \isactrlbold {\isasymnot}\ H\ {\isasymin}\ S{\isachardoublequoteclose}\isanewline
\ \ \ \ \isacommand{proof}\isamarkupfalse%
\ {\isacharparenleft}rule\ allI{\isacharparenright}{\isacharplus}\isanewline
\ \ \ \ \ \ \isacommand{fix}\isamarkupfalse%
\ G\ H\isanewline
\ \ \ \ \ \ \isacommand{show}\isamarkupfalse%
\ {\isachardoublequoteopen}\isactrlbold {\isasymnot}{\isacharparenleft}G\ \isactrlbold {\isasymand}\ H{\isacharparenright}\ {\isasymin}\ S\ {\isasymlongrightarrow}\ \isactrlbold {\isasymnot}\ G\ {\isasymin}\ S\ {\isasymor}\ \isactrlbold {\isasymnot}\ H\ {\isasymin}\ S{\isachardoublequoteclose}\isanewline
\ \ \ \ \ \ \isacommand{proof}\isamarkupfalse%
\ {\isacharparenleft}rule\ impI{\isacharparenright}\isanewline
\ \ \ \ \ \ \ \ \isacommand{assume}\isamarkupfalse%
\ {\isachardoublequoteopen}\isactrlbold {\isasymnot}{\isacharparenleft}G\ \isactrlbold {\isasymand}\ H{\isacharparenright}\ {\isasymin}\ S{\isachardoublequoteclose}\isanewline
\ \ \ \ \ \ \ \ \isacommand{have}\isamarkupfalse%
\ {\isachardoublequoteopen}Dis\ {\isacharparenleft}\isactrlbold {\isasymnot}{\isacharparenleft}G\ \isactrlbold {\isasymand}\ H{\isacharparenright}{\isacharparenright}\ {\isacharparenleft}\isactrlbold {\isasymnot}\ G{\isacharparenright}\ {\isacharparenleft}\isactrlbold {\isasymnot}\ H{\isacharparenright}{\isachardoublequoteclose}\isanewline
\ \ \ \ \ \ \ \ \ \ \isacommand{by}\isamarkupfalse%
\ {\isacharparenleft}simp\ only{\isacharcolon}\ Dis{\isachardot}intros{\isacharparenleft}{\isadigit{3}}{\isacharparenright}{\isacharparenright}\isanewline
\ \ \ \ \ \ \ \ \isacommand{have}\isamarkupfalse%
\ {\isachardoublequoteopen}Dis\ {\isacharparenleft}\isactrlbold {\isasymnot}{\isacharparenleft}G\ \isactrlbold {\isasymand}\ H{\isacharparenright}{\isacharparenright}\ {\isacharparenleft}\isactrlbold {\isasymnot}\ G{\isacharparenright}\ {\isacharparenleft}\isactrlbold {\isasymnot}\ H{\isacharparenright}\ {\isasymlongrightarrow}\ \isactrlbold {\isasymnot}{\isacharparenleft}G\ \isactrlbold {\isasymand}\ H{\isacharparenright}\ {\isasymin}\ S\ {\isasymlongrightarrow}\ \isactrlbold {\isasymnot}\ G\ {\isasymin}\ S\ {\isasymor}\ \isactrlbold {\isasymnot}\ H\ {\isasymin}\ S{\isachardoublequoteclose}\isanewline
\ \ \ \ \ \ \ \ \ \ \isacommand{using}\isamarkupfalse%
\ Dis\ \isacommand{by}\isamarkupfalse%
\ {\isacharparenleft}iprover\ elim{\isacharcolon}\ allE{\isacharparenright}\isanewline
\ \ \ \ \ \ \ \ \isacommand{then}\isamarkupfalse%
\ \isacommand{have}\isamarkupfalse%
\ {\isachardoublequoteopen}\isactrlbold {\isasymnot}{\isacharparenleft}G\ \isactrlbold {\isasymand}\ H{\isacharparenright}\ {\isasymin}\ S\ {\isasymlongrightarrow}\ \isactrlbold {\isasymnot}\ G\ {\isasymin}\ S\ {\isasymor}\ \isactrlbold {\isasymnot}\ H\ {\isasymin}\ S{\isachardoublequoteclose}\isanewline
\ \ \ \ \ \ \ \ \ \ \isacommand{using}\isamarkupfalse%
\ {\isacartoucheopen}Dis\ {\isacharparenleft}\isactrlbold {\isasymnot}{\isacharparenleft}G\ \isactrlbold {\isasymand}\ H{\isacharparenright}{\isacharparenright}\ {\isacharparenleft}\isactrlbold {\isasymnot}\ G{\isacharparenright}\ {\isacharparenleft}\isactrlbold {\isasymnot}\ H{\isacharparenright}{\isacartoucheclose}\ \isacommand{by}\isamarkupfalse%
\ {\isacharparenleft}rule\ mp{\isacharparenright}\isanewline
\ \ \ \ \ \ \ \ \isacommand{thus}\isamarkupfalse%
\ {\isachardoublequoteopen}\isactrlbold {\isasymnot}\ G\ {\isasymin}\ S\ {\isasymor}\ \isactrlbold {\isasymnot}\ H\ {\isasymin}\ S{\isachardoublequoteclose}\isanewline
\ \ \ \ \ \ \ \ \ \ \isacommand{using}\isamarkupfalse%
\ {\isacartoucheopen}\isactrlbold {\isasymnot}{\isacharparenleft}G\ \isactrlbold {\isasymand}\ H{\isacharparenright}\ {\isasymin}\ S{\isacartoucheclose}\ \isacommand{by}\isamarkupfalse%
\ {\isacharparenleft}rule\ mp{\isacharparenright}\isanewline
\ \ \ \ \ \ \isacommand{qed}\isamarkupfalse%
\isanewline
\ \ \ \ \isacommand{qed}\isamarkupfalse%
\isanewline
\ \ \ \ \isacommand{have}\isamarkupfalse%
\ C{\isadigit{8}}{\isacharcolon}{\isachardoublequoteopen}{\isasymforall}G\ H{\isachardot}\ \isactrlbold {\isasymnot}{\isacharparenleft}G\ \isactrlbold {\isasymor}\ H{\isacharparenright}\ {\isasymin}\ S\ {\isasymlongrightarrow}\ \isactrlbold {\isasymnot}\ G\ {\isasymin}\ S\ {\isasymand}\ \isactrlbold {\isasymnot}\ H\ {\isasymin}\ S{\isachardoublequoteclose}\isanewline
\ \ \ \ \isacommand{proof}\isamarkupfalse%
\ {\isacharparenleft}rule\ allI{\isacharparenright}{\isacharplus}\isanewline
\ \ \ \ \ \ \isacommand{fix}\isamarkupfalse%
\ G\ H\isanewline
\ \ \ \ \ \ \isacommand{show}\isamarkupfalse%
\ {\isachardoublequoteopen}\isactrlbold {\isasymnot}{\isacharparenleft}G\ \isactrlbold {\isasymor}\ H{\isacharparenright}\ {\isasymin}\ S\ {\isasymlongrightarrow}\ \isactrlbold {\isasymnot}\ G\ {\isasymin}\ S\ {\isasymand}\ \isactrlbold {\isasymnot}\ H\ {\isasymin}\ S{\isachardoublequoteclose}\isanewline
\ \ \ \ \ \ \isacommand{proof}\isamarkupfalse%
\ {\isacharparenleft}rule\ impI{\isacharparenright}\isanewline
\ \ \ \ \ \ \ \ \isacommand{assume}\isamarkupfalse%
\ {\isachardoublequoteopen}\isactrlbold {\isasymnot}{\isacharparenleft}G\ \isactrlbold {\isasymor}\ H{\isacharparenright}\ {\isasymin}\ S{\isachardoublequoteclose}\isanewline
\ \ \ \ \ \ \ \ \isacommand{have}\isamarkupfalse%
\ {\isachardoublequoteopen}Con\ {\isacharparenleft}\isactrlbold {\isasymnot}{\isacharparenleft}G\ \isactrlbold {\isasymor}\ H{\isacharparenright}{\isacharparenright}\ {\isacharparenleft}\isactrlbold {\isasymnot}\ G{\isacharparenright}\ {\isacharparenleft}\isactrlbold {\isasymnot}\ H{\isacharparenright}{\isachardoublequoteclose}\isanewline
\ \ \ \ \ \ \ \ \ \ \isacommand{by}\isamarkupfalse%
\ {\isacharparenleft}simp\ only{\isacharcolon}\ Con{\isachardot}intros{\isacharparenleft}{\isadigit{2}}{\isacharparenright}{\isacharparenright}\isanewline
\ \ \ \ \ \ \ \ \isacommand{have}\isamarkupfalse%
\ {\isachardoublequoteopen}Con\ {\isacharparenleft}\isactrlbold {\isasymnot}{\isacharparenleft}G\ \isactrlbold {\isasymor}\ H{\isacharparenright}{\isacharparenright}\ {\isacharparenleft}\isactrlbold {\isasymnot}\ G{\isacharparenright}\ {\isacharparenleft}\isactrlbold {\isasymnot}\ H{\isacharparenright}\ {\isasymlongrightarrow}\ \isactrlbold {\isasymnot}{\isacharparenleft}G\ \isactrlbold {\isasymor}\ H{\isacharparenright}\ {\isasymin}\ S\ {\isasymlongrightarrow}\ \isactrlbold {\isasymnot}\ G\ {\isasymin}\ S\ {\isasymand}\ \isactrlbold {\isasymnot}\ H\ {\isasymin}\ S{\isachardoublequoteclose}\isanewline
\ \ \ \ \ \ \ \ \ \ \isacommand{using}\isamarkupfalse%
\ Con\ \isacommand{by}\isamarkupfalse%
\ {\isacharparenleft}iprover\ elim{\isacharcolon}\ allE{\isacharparenright}\isanewline
\ \ \ \ \ \ \ \ \isacommand{then}\isamarkupfalse%
\ \isacommand{have}\isamarkupfalse%
\ {\isachardoublequoteopen}\isactrlbold {\isasymnot}{\isacharparenleft}G\ \isactrlbold {\isasymor}\ H{\isacharparenright}\ {\isasymin}\ S\ {\isasymlongrightarrow}\ \isactrlbold {\isasymnot}\ G\ {\isasymin}\ S\ {\isasymand}\ \isactrlbold {\isasymnot}\ H\ {\isasymin}\ S{\isachardoublequoteclose}\isanewline
\ \ \ \ \ \ \ \ \ \ \isacommand{using}\isamarkupfalse%
\ {\isacartoucheopen}Con\ {\isacharparenleft}\isactrlbold {\isasymnot}{\isacharparenleft}G\ \isactrlbold {\isasymor}\ H{\isacharparenright}{\isacharparenright}\ {\isacharparenleft}\isactrlbold {\isasymnot}\ G{\isacharparenright}\ {\isacharparenleft}\isactrlbold {\isasymnot}\ H{\isacharparenright}{\isacartoucheclose}\ \isacommand{by}\isamarkupfalse%
\ {\isacharparenleft}rule\ mp{\isacharparenright}\isanewline
\ \ \ \ \ \ \ \ \isacommand{thus}\isamarkupfalse%
\ {\isachardoublequoteopen}\isactrlbold {\isasymnot}\ G\ {\isasymin}\ S\ {\isasymand}\ \isactrlbold {\isasymnot}\ H\ {\isasymin}\ S{\isachardoublequoteclose}\isanewline
\ \ \ \ \ \ \ \ \ \ \isacommand{using}\isamarkupfalse%
\ {\isacartoucheopen}\isactrlbold {\isasymnot}{\isacharparenleft}G\ \isactrlbold {\isasymor}\ H{\isacharparenright}\ {\isasymin}\ S{\isacartoucheclose}\ \isacommand{by}\isamarkupfalse%
\ {\isacharparenleft}rule\ mp{\isacharparenright}\isanewline
\ \ \ \ \ \ \isacommand{qed}\isamarkupfalse%
\isanewline
\ \ \ \ \isacommand{qed}\isamarkupfalse%
\isanewline
\ \ \ \ \isacommand{have}\isamarkupfalse%
\ C{\isadigit{9}}{\isacharcolon}{\isachardoublequoteopen}{\isasymforall}G\ H{\isachardot}\ \isactrlbold {\isasymnot}{\isacharparenleft}G\ \isactrlbold {\isasymrightarrow}\ H{\isacharparenright}\ {\isasymin}\ S\ {\isasymlongrightarrow}\ G\ {\isasymin}\ S\ {\isasymand}\ \isactrlbold {\isasymnot}\ H\ {\isasymin}\ S{\isachardoublequoteclose}\isanewline
\ \ \ \ \isacommand{proof}\isamarkupfalse%
\ {\isacharparenleft}rule\ allI{\isacharparenright}{\isacharplus}\isanewline
\ \ \ \ \ \ \isacommand{fix}\isamarkupfalse%
\ G\ H\isanewline
\ \ \ \ \ \ \isacommand{show}\isamarkupfalse%
\ {\isachardoublequoteopen}\isactrlbold {\isasymnot}{\isacharparenleft}G\ \isactrlbold {\isasymrightarrow}\ H{\isacharparenright}\ {\isasymin}\ S\ {\isasymlongrightarrow}\ G\ {\isasymin}\ S\ {\isasymand}\ \isactrlbold {\isasymnot}\ H\ {\isasymin}\ S{\isachardoublequoteclose}\isanewline
\ \ \ \ \ \ \isacommand{proof}\isamarkupfalse%
\ {\isacharparenleft}rule\ impI{\isacharparenright}\isanewline
\ \ \ \ \ \ \ \ \isacommand{assume}\isamarkupfalse%
\ {\isachardoublequoteopen}\isactrlbold {\isasymnot}{\isacharparenleft}G\ \isactrlbold {\isasymrightarrow}\ H{\isacharparenright}\ {\isasymin}\ S{\isachardoublequoteclose}\isanewline
\ \ \ \ \ \ \ \ \isacommand{have}\isamarkupfalse%
\ {\isachardoublequoteopen}Con\ {\isacharparenleft}\isactrlbold {\isasymnot}{\isacharparenleft}G\ \isactrlbold {\isasymrightarrow}\ H{\isacharparenright}{\isacharparenright}\ G\ {\isacharparenleft}\isactrlbold {\isasymnot}\ H{\isacharparenright}{\isachardoublequoteclose}\isanewline
\ \ \ \ \ \ \ \ \ \ \isacommand{by}\isamarkupfalse%
\ {\isacharparenleft}simp\ only{\isacharcolon}\ Con{\isachardot}intros{\isacharparenleft}{\isadigit{3}}{\isacharparenright}{\isacharparenright}\isanewline
\ \ \ \ \ \ \ \ \isacommand{have}\isamarkupfalse%
\ {\isachardoublequoteopen}Con\ {\isacharparenleft}\isactrlbold {\isasymnot}{\isacharparenleft}G\ \isactrlbold {\isasymrightarrow}\ H{\isacharparenright}{\isacharparenright}\ G\ {\isacharparenleft}\isactrlbold {\isasymnot}\ H{\isacharparenright}\ {\isasymlongrightarrow}\ \isactrlbold {\isasymnot}{\isacharparenleft}G\ \isactrlbold {\isasymrightarrow}\ H{\isacharparenright}\ {\isasymin}\ S\ {\isasymlongrightarrow}\ G\ {\isasymin}\ S\ {\isasymand}\ \isactrlbold {\isasymnot}\ H\ {\isasymin}\ S{\isachardoublequoteclose}\isanewline
\ \ \ \ \ \ \ \ \ \ \isacommand{using}\isamarkupfalse%
\ Con\ \isacommand{by}\isamarkupfalse%
\ {\isacharparenleft}iprover\ elim{\isacharcolon}\ allE{\isacharparenright}\isanewline
\ \ \ \ \ \ \ \ \isacommand{then}\isamarkupfalse%
\ \isacommand{have}\isamarkupfalse%
\ {\isachardoublequoteopen}\isactrlbold {\isasymnot}{\isacharparenleft}G\ \isactrlbold {\isasymrightarrow}\ H{\isacharparenright}\ {\isasymin}\ S\ {\isasymlongrightarrow}\ G\ {\isasymin}\ S\ {\isasymand}\ \isactrlbold {\isasymnot}\ H\ {\isasymin}\ S{\isachardoublequoteclose}\isanewline
\ \ \ \ \ \ \ \ \ \ \isacommand{using}\isamarkupfalse%
\ {\isacartoucheopen}Con\ {\isacharparenleft}\isactrlbold {\isasymnot}{\isacharparenleft}G\ \isactrlbold {\isasymrightarrow}\ H{\isacharparenright}{\isacharparenright}\ G\ {\isacharparenleft}\isactrlbold {\isasymnot}\ H{\isacharparenright}{\isacartoucheclose}\ \isacommand{by}\isamarkupfalse%
\ {\isacharparenleft}rule\ mp{\isacharparenright}\isanewline
\ \ \ \ \ \ \ \ \isacommand{thus}\isamarkupfalse%
\ {\isachardoublequoteopen}G\ {\isasymin}\ S\ {\isasymand}\ \isactrlbold {\isasymnot}\ H\ {\isasymin}\ S{\isachardoublequoteclose}\isanewline
\ \ \ \ \ \ \ \ \ \ \isacommand{using}\isamarkupfalse%
\ {\isacartoucheopen}\isactrlbold {\isasymnot}{\isacharparenleft}G\ \isactrlbold {\isasymrightarrow}\ H{\isacharparenright}\ {\isasymin}\ S{\isacartoucheclose}\ \isacommand{by}\isamarkupfalse%
\ {\isacharparenleft}rule\ mp{\isacharparenright}\isanewline
\ \ \ \ \ \ \isacommand{qed}\isamarkupfalse%
\isanewline
\ \ \ \ \isacommand{qed}\isamarkupfalse%
\isanewline
\ \ \ \ \isacommand{have}\isamarkupfalse%
\ A{\isacharcolon}{\isachardoublequoteopen}{\isasymbottom}\ {\isasymnotin}\ S\isanewline
\ \ \ \ {\isasymand}\ {\isacharparenleft}{\isasymforall}k{\isachardot}\ Atom\ k\ {\isasymin}\ S\ {\isasymlongrightarrow}\ \isactrlbold {\isasymnot}\ {\isacharparenleft}Atom\ k{\isacharparenright}\ {\isasymin}\ S\ {\isasymlongrightarrow}\ False{\isacharparenright}\isanewline
\ \ \ \ {\isasymand}\ {\isacharparenleft}{\isasymforall}G\ H{\isachardot}\ G\ \isactrlbold {\isasymand}\ H\ {\isasymin}\ S\ {\isasymlongrightarrow}\ G\ {\isasymin}\ S\ {\isasymand}\ H\ {\isasymin}\ S{\isacharparenright}\isanewline
\ \ \ \ {\isasymand}\ {\isacharparenleft}{\isasymforall}G\ H{\isachardot}\ G\ \isactrlbold {\isasymor}\ H\ {\isasymin}\ S\ {\isasymlongrightarrow}\ G\ {\isasymin}\ S\ {\isasymor}\ H\ {\isasymin}\ S{\isacharparenright}\isanewline
\ \ \ \ {\isasymand}\ {\isacharparenleft}{\isasymforall}G\ H{\isachardot}\ G\ \isactrlbold {\isasymrightarrow}\ H\ {\isasymin}\ S\ {\isasymlongrightarrow}\ \isactrlbold {\isasymnot}G\ {\isasymin}\ S\ {\isasymor}\ H\ {\isasymin}\ S{\isacharparenright}{\isachardoublequoteclose}\isanewline
\ \ \ \ \ \ \isacommand{using}\isamarkupfalse%
\ C{\isadigit{1}}\ C{\isadigit{2}}\ C{\isadigit{3}}\ C{\isadigit{4}}\ C{\isadigit{5}}\ \isacommand{by}\isamarkupfalse%
\ {\isacharparenleft}iprover\ intro{\isacharcolon}\ conjI{\isacharparenright}\isanewline
\ \ \ \ \isacommand{have}\isamarkupfalse%
\ B{\isacharcolon}{\isachardoublequoteopen}{\isacharparenleft}{\isasymforall}G{\isachardot}\ \isactrlbold {\isasymnot}\ {\isacharparenleft}\isactrlbold {\isasymnot}G{\isacharparenright}\ {\isasymin}\ S\ {\isasymlongrightarrow}\ G\ {\isasymin}\ S{\isacharparenright}\isanewline
\ \ \ \ {\isasymand}\ {\isacharparenleft}{\isasymforall}G\ H{\isachardot}\ \isactrlbold {\isasymnot}{\isacharparenleft}G\ \isactrlbold {\isasymand}\ H{\isacharparenright}\ {\isasymin}\ S\ {\isasymlongrightarrow}\ \isactrlbold {\isasymnot}\ G\ {\isasymin}\ S\ {\isasymor}\ \isactrlbold {\isasymnot}\ H\ {\isasymin}\ S{\isacharparenright}\isanewline
\ \ \ \ {\isasymand}\ {\isacharparenleft}{\isasymforall}G\ H{\isachardot}\ \isactrlbold {\isasymnot}{\isacharparenleft}G\ \isactrlbold {\isasymor}\ H{\isacharparenright}\ {\isasymin}\ S\ {\isasymlongrightarrow}\ \isactrlbold {\isasymnot}\ G\ {\isasymin}\ S\ {\isasymand}\ \isactrlbold {\isasymnot}\ H\ {\isasymin}\ S{\isacharparenright}\isanewline
\ \ \ \ {\isasymand}\ {\isacharparenleft}{\isasymforall}G\ H{\isachardot}\ \isactrlbold {\isasymnot}{\isacharparenleft}G\ \isactrlbold {\isasymrightarrow}\ H{\isacharparenright}\ {\isasymin}\ S\ {\isasymlongrightarrow}\ G\ {\isasymin}\ S\ {\isasymand}\ \isactrlbold {\isasymnot}\ H\ {\isasymin}\ S{\isacharparenright}{\isachardoublequoteclose}\isanewline
\ \ \ \ \ \ \isacommand{using}\isamarkupfalse%
\ C{\isadigit{6}}\ C{\isadigit{7}}\ C{\isadigit{8}}\ C{\isadigit{9}}\ \isacommand{by}\isamarkupfalse%
\ {\isacharparenleft}iprover\ intro{\isacharcolon}\ conjI{\isacharparenright}\isanewline
\ \ \ \ \isacommand{have}\isamarkupfalse%
\ {\isachardoublequoteopen}{\isacharparenleft}{\isasymbottom}\ {\isasymnotin}\ S\isanewline
\ \ \ \ {\isasymand}\ {\isacharparenleft}{\isasymforall}k{\isachardot}\ Atom\ k\ {\isasymin}\ S\ {\isasymlongrightarrow}\ \isactrlbold {\isasymnot}\ {\isacharparenleft}Atom\ k{\isacharparenright}\ {\isasymin}\ S\ {\isasymlongrightarrow}\ False{\isacharparenright}\isanewline
\ \ \ \ {\isasymand}\ {\isacharparenleft}{\isasymforall}G\ H{\isachardot}\ G\ \isactrlbold {\isasymand}\ H\ {\isasymin}\ S\ {\isasymlongrightarrow}\ G\ {\isasymin}\ S\ {\isasymand}\ H\ {\isasymin}\ S{\isacharparenright}\isanewline
\ \ \ \ {\isasymand}\ {\isacharparenleft}{\isasymforall}G\ H{\isachardot}\ G\ \isactrlbold {\isasymor}\ H\ {\isasymin}\ S\ {\isasymlongrightarrow}\ G\ {\isasymin}\ S\ {\isasymor}\ H\ {\isasymin}\ S{\isacharparenright}\isanewline
\ \ \ \ {\isasymand}\ {\isacharparenleft}{\isasymforall}G\ H{\isachardot}\ G\ \isactrlbold {\isasymrightarrow}\ H\ {\isasymin}\ S\ {\isasymlongrightarrow}\ \isactrlbold {\isasymnot}G\ {\isasymin}\ S\ {\isasymor}\ H\ {\isasymin}\ S{\isacharparenright}{\isacharparenright}\isanewline
\ \ \ \ {\isasymand}\ {\isacharparenleft}{\isacharparenleft}{\isasymforall}G{\isachardot}\ \isactrlbold {\isasymnot}\ {\isacharparenleft}\isactrlbold {\isasymnot}G{\isacharparenright}\ {\isasymin}\ S\ {\isasymlongrightarrow}\ G\ {\isasymin}\ S{\isacharparenright}\isanewline
\ \ \ \ {\isasymand}\ {\isacharparenleft}{\isasymforall}G\ H{\isachardot}\ \isactrlbold {\isasymnot}{\isacharparenleft}G\ \isactrlbold {\isasymand}\ H{\isacharparenright}\ {\isasymin}\ S\ {\isasymlongrightarrow}\ \isactrlbold {\isasymnot}\ G\ {\isasymin}\ S\ {\isasymor}\ \isactrlbold {\isasymnot}\ H\ {\isasymin}\ S{\isacharparenright}\isanewline
\ \ \ \ {\isasymand}\ {\isacharparenleft}{\isasymforall}G\ H{\isachardot}\ \isactrlbold {\isasymnot}{\isacharparenleft}G\ \isactrlbold {\isasymor}\ H{\isacharparenright}\ {\isasymin}\ S\ {\isasymlongrightarrow}\ \isactrlbold {\isasymnot}\ G\ {\isasymin}\ S\ {\isasymand}\ \isactrlbold {\isasymnot}\ H\ {\isasymin}\ S{\isacharparenright}\isanewline
\ \ \ \ {\isasymand}\ {\isacharparenleft}{\isasymforall}G\ H{\isachardot}\ \isactrlbold {\isasymnot}{\isacharparenleft}G\ \isactrlbold {\isasymrightarrow}\ H{\isacharparenright}\ {\isasymin}\ S\ {\isasymlongrightarrow}\ G\ {\isasymin}\ S\ {\isasymand}\ \isactrlbold {\isasymnot}\ H\ {\isasymin}\ S{\isacharparenright}{\isacharparenright}{\isachardoublequoteclose}\isanewline
\ \ \ \ \ \ \isacommand{using}\isamarkupfalse%
\ A\ B\ \isacommand{by}\isamarkupfalse%
\ {\isacharparenleft}rule\ conjI{\isacharparenright}\isanewline
\ \ \ \ \isacommand{thus}\isamarkupfalse%
\ {\isachardoublequoteopen}{\isasymbottom}\ {\isasymnotin}\ S\isanewline
\ \ \ \ {\isasymand}\ {\isacharparenleft}{\isasymforall}k{\isachardot}\ Atom\ k\ {\isasymin}\ S\ {\isasymlongrightarrow}\ \isactrlbold {\isasymnot}\ {\isacharparenleft}Atom\ k{\isacharparenright}\ {\isasymin}\ S\ {\isasymlongrightarrow}\ False{\isacharparenright}\isanewline
\ \ \ \ {\isasymand}\ {\isacharparenleft}{\isasymforall}G\ H{\isachardot}\ G\ \isactrlbold {\isasymand}\ H\ {\isasymin}\ S\ {\isasymlongrightarrow}\ G\ {\isasymin}\ S\ {\isasymand}\ H\ {\isasymin}\ S{\isacharparenright}\isanewline
\ \ \ \ {\isasymand}\ {\isacharparenleft}{\isasymforall}G\ H{\isachardot}\ G\ \isactrlbold {\isasymor}\ H\ {\isasymin}\ S\ {\isasymlongrightarrow}\ G\ {\isasymin}\ S\ {\isasymor}\ H\ {\isasymin}\ S{\isacharparenright}\isanewline
\ \ \ \ {\isasymand}\ {\isacharparenleft}{\isasymforall}G\ H{\isachardot}\ G\ \isactrlbold {\isasymrightarrow}\ H\ {\isasymin}\ S\ {\isasymlongrightarrow}\ \isactrlbold {\isasymnot}G\ {\isasymin}\ S\ {\isasymor}\ H\ {\isasymin}\ S{\isacharparenright}\isanewline
\ \ \ \ {\isasymand}\ {\isacharparenleft}{\isasymforall}G{\isachardot}\ \isactrlbold {\isasymnot}\ {\isacharparenleft}\isactrlbold {\isasymnot}G{\isacharparenright}\ {\isasymin}\ S\ {\isasymlongrightarrow}\ G\ {\isasymin}\ S{\isacharparenright}\isanewline
\ \ \ \ {\isasymand}\ {\isacharparenleft}{\isasymforall}G\ H{\isachardot}\ \isactrlbold {\isasymnot}{\isacharparenleft}G\ \isactrlbold {\isasymand}\ H{\isacharparenright}\ {\isasymin}\ S\ {\isasymlongrightarrow}\ \isactrlbold {\isasymnot}\ G\ {\isasymin}\ S\ {\isasymor}\ \isactrlbold {\isasymnot}\ H\ {\isasymin}\ S{\isacharparenright}\isanewline
\ \ \ \ {\isasymand}\ {\isacharparenleft}{\isasymforall}G\ H{\isachardot}\ \isactrlbold {\isasymnot}{\isacharparenleft}G\ \isactrlbold {\isasymor}\ H{\isacharparenright}\ {\isasymin}\ S\ {\isasymlongrightarrow}\ \isactrlbold {\isasymnot}\ G\ {\isasymin}\ S\ {\isasymand}\ \isactrlbold {\isasymnot}\ H\ {\isasymin}\ S{\isacharparenright}\isanewline
\ \ \ \ {\isasymand}\ {\isacharparenleft}{\isasymforall}G\ H{\isachardot}\ \isactrlbold {\isasymnot}{\isacharparenleft}G\ \isactrlbold {\isasymrightarrow}\ H{\isacharparenright}\ {\isasymin}\ S\ {\isasymlongrightarrow}\ G\ {\isasymin}\ S\ {\isasymand}\ \isactrlbold {\isasymnot}\ H\ {\isasymin}\ S{\isacharparenright}{\isachardoublequoteclose}\ \isanewline
\ \ \ \ \ \ \isacommand{by}\isamarkupfalse%
\ {\isacharparenleft}iprover\ intro{\isacharcolon}\ conj{\isacharunderscore}assoc{\isacharparenright}\isanewline
\ \ \isacommand{qed}\isamarkupfalse%
\isanewline
\ \ \isacommand{thus}\isamarkupfalse%
\ {\isachardoublequoteopen}Hintikka\ S{\isachardoublequoteclose}\isanewline
\ \ \ \ \isacommand{unfolding}\isamarkupfalse%
\ Hintikka{\isacharunderscore}def\ \isacommand{by}\isamarkupfalse%
\ this\isanewline
\isacommand{qed}\isamarkupfalse%
%
\endisatagproof
{\isafoldproof}%
%
\isadelimproof
%
\endisadelimproof
%
\begin{isamarkuptext}%
En conclusión, el lema completo se demuestra detalladamente en Isabelle/HOL como sigue.%
\end{isamarkuptext}\isamarkuptrue%
\isacommand{lemma}\isamarkupfalse%
\ {\isachardoublequoteopen}Hintikka\ S\ {\isacharequal}\ {\isacharparenleft}{\isasymbottom}\ {\isasymnotin}\ S\isanewline
{\isasymand}\ {\isacharparenleft}{\isasymforall}k{\isachardot}\ Atom\ k\ {\isasymin}\ S\ {\isasymlongrightarrow}\ \isactrlbold {\isasymnot}\ {\isacharparenleft}Atom\ k{\isacharparenright}\ {\isasymin}\ S\ {\isasymlongrightarrow}\ False{\isacharparenright}\isanewline
{\isasymand}\ {\isacharparenleft}{\isasymforall}F\ G\ H{\isachardot}\ Con\ F\ G\ H\ {\isasymlongrightarrow}\ F\ {\isasymin}\ S\ {\isasymlongrightarrow}\ G\ {\isasymin}\ S\ {\isasymand}\ H\ {\isasymin}\ S{\isacharparenright}\isanewline
{\isasymand}\ {\isacharparenleft}{\isasymforall}F\ G\ H{\isachardot}\ Dis\ F\ G\ H\ {\isasymlongrightarrow}\ F\ {\isasymin}\ S\ {\isasymlongrightarrow}\ G\ {\isasymin}\ S\ {\isasymor}\ H\ {\isasymin}\ S{\isacharparenright}{\isacharparenright}{\isachardoublequoteclose}\ \ \isanewline
%
\isadelimproof
%
\endisadelimproof
%
\isatagproof
\isacommand{proof}\isamarkupfalse%
\ {\isacharparenleft}rule\ iffI{\isacharparenright}\isanewline
\ \ \isacommand{assume}\isamarkupfalse%
\ {\isachardoublequoteopen}Hintikka\ S{\isachardoublequoteclose}\isanewline
\ \ \isacommand{thus}\isamarkupfalse%
\ {\isachardoublequoteopen}{\isacharparenleft}{\isasymbottom}\ {\isasymnotin}\ S\isanewline
\ \ {\isasymand}\ {\isacharparenleft}{\isasymforall}k{\isachardot}\ Atom\ k\ {\isasymin}\ S\ {\isasymlongrightarrow}\ \isactrlbold {\isasymnot}\ {\isacharparenleft}Atom\ k{\isacharparenright}\ {\isasymin}\ S\ {\isasymlongrightarrow}\ False{\isacharparenright}\isanewline
\ \ {\isasymand}\ {\isacharparenleft}{\isasymforall}F\ G\ H{\isachardot}\ Con\ F\ G\ H\ {\isasymlongrightarrow}\ F\ {\isasymin}\ S\ {\isasymlongrightarrow}\ G\ {\isasymin}\ S\ {\isasymand}\ H\ {\isasymin}\ S{\isacharparenright}\isanewline
\ \ {\isasymand}\ {\isacharparenleft}{\isasymforall}F\ G\ H{\isachardot}\ Dis\ F\ G\ H\ {\isasymlongrightarrow}\ F\ {\isasymin}\ S\ {\isasymlongrightarrow}\ G\ {\isasymin}\ S\ {\isasymor}\ H\ {\isasymin}\ S{\isacharparenright}{\isacharparenright}{\isachardoublequoteclose}\isanewline
\ \ \ \ \isacommand{by}\isamarkupfalse%
\ {\isacharparenleft}rule\ Hintikka{\isacharunderscore}alt{\isadigit{1}}{\isacharparenright}\isanewline
\isacommand{next}\isamarkupfalse%
\isanewline
\ \ \isacommand{assume}\isamarkupfalse%
\ {\isachardoublequoteopen}{\isacharparenleft}{\isasymbottom}\ {\isasymnotin}\ S\isanewline
\ \ {\isasymand}\ {\isacharparenleft}{\isasymforall}k{\isachardot}\ Atom\ k\ {\isasymin}\ S\ {\isasymlongrightarrow}\ \isactrlbold {\isasymnot}\ {\isacharparenleft}Atom\ k{\isacharparenright}\ {\isasymin}\ S\ {\isasymlongrightarrow}\ False{\isacharparenright}\isanewline
\ \ {\isasymand}\ {\isacharparenleft}{\isasymforall}F\ G\ H{\isachardot}\ Con\ F\ G\ H\ {\isasymlongrightarrow}\ F\ {\isasymin}\ S\ {\isasymlongrightarrow}\ G\ {\isasymin}\ S\ {\isasymand}\ H\ {\isasymin}\ S{\isacharparenright}\isanewline
\ \ {\isasymand}\ {\isacharparenleft}{\isasymforall}F\ G\ H{\isachardot}\ Dis\ F\ G\ H\ {\isasymlongrightarrow}\ F\ {\isasymin}\ S\ {\isasymlongrightarrow}\ G\ {\isasymin}\ S\ {\isasymor}\ H\ {\isasymin}\ S{\isacharparenright}{\isacharparenright}{\isachardoublequoteclose}\isanewline
\ \ \isacommand{thus}\isamarkupfalse%
\ {\isachardoublequoteopen}Hintikka\ S{\isachardoublequoteclose}\isanewline
\ \ \ \ \isacommand{by}\isamarkupfalse%
\ {\isacharparenleft}rule\ Hintikka{\isacharunderscore}alt{\isadigit{2}}{\isacharparenright}\isanewline
\isacommand{qed}\isamarkupfalse%
%
\endisatagproof
{\isafoldproof}%
%
\isadelimproof
%
\endisadelimproof
%
\begin{isamarkuptext}%
Por último, veamos su demostración automática.%
\end{isamarkuptext}\isamarkuptrue%
\isacommand{lemma}\isamarkupfalse%
\ Hintikka{\isacharunderscore}alt{\isacharcolon}\ {\isachardoublequoteopen}Hintikka\ S\ {\isacharequal}\ {\isacharparenleft}{\isasymbottom}\ {\isasymnotin}\ S\isanewline
{\isasymand}\ {\isacharparenleft}{\isasymforall}k{\isachardot}\ Atom\ k\ {\isasymin}\ S\ {\isasymlongrightarrow}\ \isactrlbold {\isasymnot}\ {\isacharparenleft}Atom\ k{\isacharparenright}\ {\isasymin}\ S\ {\isasymlongrightarrow}\ False{\isacharparenright}\isanewline
{\isasymand}\ {\isacharparenleft}{\isasymforall}F\ G\ H{\isachardot}\ Con\ F\ G\ H\ {\isasymlongrightarrow}\ F\ {\isasymin}\ S\ {\isasymlongrightarrow}\ G\ {\isasymin}\ S\ {\isasymand}\ H\ {\isasymin}\ S{\isacharparenright}\isanewline
{\isasymand}\ {\isacharparenleft}{\isasymforall}F\ G\ H{\isachardot}\ Dis\ F\ G\ H\ {\isasymlongrightarrow}\ F\ {\isasymin}\ S\ {\isasymlongrightarrow}\ G\ {\isasymin}\ S\ {\isasymor}\ H\ {\isasymin}\ S{\isacharparenright}{\isacharparenright}{\isachardoublequoteclose}\ \ \isanewline
%
\isadelimproof
\ \ %
\endisadelimproof
%
\isatagproof
\isacommand{apply}\isamarkupfalse%
{\isacharparenleft}simp\ add{\isacharcolon}\ Hintikka{\isacharunderscore}def\ con{\isacharunderscore}dis{\isacharunderscore}simps{\isacharparenright}\isanewline
\ \ \isacommand{apply}\isamarkupfalse%
{\isacharparenleft}rule\ iffI{\isacharparenright}\isanewline
\ \ \ \isacommand{subgoal}\isamarkupfalse%
\ \isacommand{by}\isamarkupfalse%
\ blast\isanewline
\ \ \isacommand{subgoal}\isamarkupfalse%
\ \isacommand{by}\isamarkupfalse%
\ safe\ metis{\isacharplus}\isanewline
\ \ \isacommand{done}\isamarkupfalse%
\isanewline
%
\endisatagproof
{\isafoldproof}%
%
\isadelimproof
%
\endisadelimproof
%
\isadelimtheory
%
\endisadelimtheory
%
\isatagtheory
%
\endisatagtheory
{\isafoldtheory}%
%
\isadelimtheory
%
\endisadelimtheory
%
\end{isabellebody}%
\endinput
%:%file=~/TFM/TFM/Notunif.thy%:%
%:%19=11%:%
%:%20=12%:%
%:%21=13%:%
%:%25=17%:%
%:%26=18%:%
%:%27=19%:%
%:%28=20%:%
%:%29=21%:%
%:%30=22%:%
%:%31=23%:%
%:%32=24%:%
%:%33=25%:%
%:%34=26%:%
%:%35=27%:%
%:%36=28%:%
%:%37=29%:%
%:%38=30%:%
%:%39=31%:%
%:%40=32%:%
%:%41=33%:%
%:%42=34%:%
%:%43=35%:%
%:%44=36%:%
%:%46=38%:%
%:%47=38%:%
%:%49=40%:%
%:%50=41%:%
%:%52=43%:%
%:%53=43%:%
%:%56=44%:%
%:%60=44%:%
%:%61=44%:%
%:%66=44%:%
%:%69=45%:%
%:%70=46%:%
%:%71=46%:%
%:%74=47%:%
%:%78=47%:%
%:%79=47%:%
%:%84=47%:%
%:%87=48%:%
%:%88=49%:%
%:%89=49%:%
%:%92=50%:%
%:%96=50%:%
%:%97=50%:%
%:%102=50%:%
%:%105=51%:%
%:%106=52%:%
%:%107=52%:%
%:%110=53%:%
%:%114=53%:%
%:%115=53%:%
%:%120=53%:%
%:%123=54%:%
%:%124=55%:%
%:%125=55%:%
%:%128=56%:%
%:%132=56%:%
%:%133=56%:%
%:%138=56%:%
%:%141=57%:%
%:%142=58%:%
%:%143=58%:%
%:%146=59%:%
%:%150=59%:%
%:%151=59%:%
%:%156=59%:%
%:%159=60%:%
%:%160=61%:%
%:%161=61%:%
%:%164=62%:%
%:%168=62%:%
%:%169=62%:%
%:%174=62%:%
%:%177=63%:%
%:%178=64%:%
%:%179=64%:%
%:%182=65%:%
%:%186=65%:%
%:%187=65%:%
%:%192=65%:%
%:%195=66%:%
%:%196=67%:%
%:%197=67%:%
%:%200=68%:%
%:%204=68%:%
%:%205=68%:%
%:%210=68%:%
%:%213=69%:%
%:%214=70%:%
%:%215=70%:%
%:%218=71%:%
%:%222=71%:%
%:%223=71%:%
%:%232=73%:%
%:%233=74%:%
%:%235=76%:%
%:%236=76%:%
%:%239=77%:%
%:%243=77%:%
%:%244=77%:%
%:%249=77%:%
%:%252=78%:%
%:%253=79%:%
%:%254=79%:%
%:%257=80%:%
%:%261=80%:%
%:%262=80%:%
%:%271=82%:%
%:%272=83%:%
%:%273=84%:%
%:%274=85%:%
%:%275=86%:%
%:%276=87%:%
%:%277=88%:%
%:%278=89%:%
%:%279=90%:%
%:%280=91%:%
%:%281=92%:%
%:%282=93%:%
%:%283=94%:%
%:%284=95%:%
%:%285=96%:%
%:%286=97%:%
%:%287=98%:%
%:%288=99%:%
%:%289=100%:%
%:%290=101%:%
%:%291=102%:%
%:%292=103%:%
%:%293=104%:%
%:%294=105%:%
%:%295=106%:%
%:%296=107%:%
%:%298=109%:%
%:%299=109%:%
%:%300=110%:%
%:%301=111%:%
%:%302=112%:%
%:%303=113%:%
%:%305=115%:%
%:%306=116%:%
%:%307=117%:%
%:%308=118%:%
%:%309=119%:%
%:%312=119%:%
%:%313=120%:%
%:%314=121%:%
%:%315=122%:%
%:%316=123%:%
%:%317=124%:%
%:%318=125%:%
%:%319=126%:%
%:%320=127%:%
%:%321=128%:%
%:%322=129%:%
%:%323=130%:%
%:%324=131%:%
%:%325=132%:%
%:%326=133%:%
%:%327=134%:%
%:%328=135%:%
%:%329=136%:%
%:%330=137%:%
%:%331=138%:%
%:%332=139%:%
%:%333=140%:%
%:%334=141%:%
%:%335=142%:%
%:%337=144%:%
%:%338=144%:%
%:%339=145%:%
%:%340=146%:%
%:%341=147%:%
%:%342=148%:%
%:%344=150%:%
%:%345=151%:%
%:%346=152%:%
%:%347=153%:%
%:%350=153%:%
%:%351=154%:%
%:%352=155%:%
%:%353=156%:%
%:%354=157%:%
%:%355=158%:%
%:%356=159%:%
%:%357=160%:%
%:%358=161%:%
%:%359=162%:%
%:%360=163%:%
%:%361=164%:%
%:%363=166%:%
%:%364=166%:%
%:%367=167%:%
%:%371=167%:%
%:%372=167%:%
%:%373=167%:%
%:%382=169%:%
%:%383=170%:%
%:%384=171%:%
%:%386=173%:%
%:%387=173%:%
%:%390=174%:%
%:%394=174%:%
%:%395=174%:%
%:%400=174%:%
%:%403=175%:%
%:%404=176%:%
%:%405=176%:%
%:%408=177%:%
%:%412=177%:%
%:%413=177%:%
%:%422=179%:%
%:%423=180%:%
%:%425=182%:%
%:%426=182%:%
%:%429=183%:%
%:%433=183%:%
%:%434=183%:%
%:%443=185%:%
%:%444=186%:%
%:%446=188%:%
%:%447=188%:%
%:%448=189%:%
%:%451=192%:%
%:%452=193%:%
%:%455=196%:%
%:%458=197%:%
%:%462=197%:%
%:%463=197%:%
%:%472=199%:%
%:%473=200%:%
%:%474=201%:%
%:%475=202%:%
%:%476=203%:%
%:%477=204%:%
%:%478=205%:%
%:%479=206%:%
%:%480=207%:%
%:%481=208%:%
%:%482=209%:%
%:%483=210%:%
%:%484=211%:%
%:%485=212%:%
%:%486=213%:%
%:%487=214%:%
%:%488=215%:%
%:%489=216%:%
%:%490=217%:%
%:%491=218%:%
%:%492=219%:%
%:%493=220%:%
%:%495=222%:%
%:%496=222%:%
%:%499=225%:%
%:%502=226%:%
%:%506=226%:%
%:%516=228%:%
%:%517=229%:%
%:%518=230%:%
%:%519=231%:%
%:%520=232%:%
%:%521=233%:%
%:%522=234%:%
%:%523=235%:%
%:%524=236%:%
%:%525=237%:%
%:%526=238%:%
%:%527=239%:%
%:%528=240%:%
%:%529=241%:%
%:%530=242%:%
%:%531=243%:%
%:%532=244%:%
%:%533=245%:%
%:%534=246%:%
%:%535=247%:%
%:%536=248%:%
%:%537=249%:%
%:%538=250%:%
%:%539=251%:%
%:%540=252%:%
%:%541=253%:%
%:%542=254%:%
%:%543=255%:%
%:%544=256%:%
%:%545=257%:%
%:%546=258%:%
%:%547=259%:%
%:%548=260%:%
%:%549=261%:%
%:%550=262%:%
%:%551=263%:%
%:%552=264%:%
%:%553=265%:%
%:%554=266%:%
%:%555=267%:%
%:%556=268%:%
%:%557=269%:%
%:%558=270%:%
%:%559=271%:%
%:%560=272%:%
%:%561=273%:%
%:%562=274%:%
%:%563=275%:%
%:%564=276%:%
%:%565=277%:%
%:%566=278%:%
%:%567=279%:%
%:%568=280%:%
%:%569=281%:%
%:%570=282%:%
%:%571=283%:%
%:%572=284%:%
%:%573=285%:%
%:%574=286%:%
%:%575=287%:%
%:%576=288%:%
%:%577=289%:%
%:%578=290%:%
%:%579=291%:%
%:%580=292%:%
%:%581=293%:%
%:%582=294%:%
%:%583=295%:%
%:%584=296%:%
%:%585=297%:%
%:%586=298%:%
%:%587=299%:%
%:%588=300%:%
%:%589=301%:%
%:%590=302%:%
%:%591=303%:%
%:%592=304%:%
%:%593=305%:%
%:%594=306%:%
%:%595=307%:%
%:%596=308%:%
%:%597=309%:%
%:%598=310%:%
%:%599=311%:%
%:%600=312%:%
%:%601=313%:%
%:%602=314%:%
%:%603=315%:%
%:%604=316%:%
%:%605=317%:%
%:%606=318%:%
%:%607=319%:%
%:%608=320%:%
%:%609=321%:%
%:%610=322%:%
%:%611=323%:%
%:%612=324%:%
%:%613=325%:%
%:%614=326%:%
%:%615=327%:%
%:%616=328%:%
%:%617=329%:%
%:%618=330%:%
%:%619=331%:%
%:%620=332%:%
%:%621=333%:%
%:%622=334%:%
%:%623=335%:%
%:%624=336%:%
%:%625=337%:%
%:%626=338%:%
%:%627=339%:%
%:%628=340%:%
%:%629=341%:%
%:%630=342%:%
%:%631=343%:%
%:%632=344%:%
%:%633=345%:%
%:%634=346%:%
%:%635=347%:%
%:%636=348%:%
%:%637=349%:%
%:%638=350%:%
%:%639=351%:%
%:%640=352%:%
%:%641=353%:%
%:%642=354%:%
%:%643=355%:%
%:%644=356%:%
%:%645=357%:%
%:%646=358%:%
%:%647=359%:%
%:%648=360%:%
%:%649=361%:%
%:%650=362%:%
%:%651=363%:%
%:%652=364%:%
%:%653=365%:%
%:%654=366%:%
%:%655=367%:%
%:%656=368%:%
%:%657=369%:%
%:%658=370%:%
%:%659=371%:%
%:%660=372%:%
%:%661=373%:%
%:%662=374%:%
%:%663=375%:%
%:%664=376%:%
%:%665=377%:%
%:%666=378%:%
%:%667=379%:%
%:%668=380%:%
%:%669=381%:%
%:%670=382%:%
%:%671=383%:%
%:%672=384%:%
%:%673=385%:%
%:%674=386%:%
%:%675=387%:%
%:%676=388%:%
%:%677=389%:%
%:%678=390%:%
%:%680=392%:%
%:%681=392%:%
%:%682=393%:%
%:%685=396%:%
%:%686=397%:%
%:%693=398%:%
%:%694=398%:%
%:%695=399%:%
%:%696=399%:%
%:%697=400%:%
%:%698=400%:%
%:%699=400%:%
%:%702=403%:%
%:%703=404%:%
%:%704=404%:%
%:%705=405%:%
%:%706=405%:%
%:%707=406%:%
%:%708=406%:%
%:%709=407%:%
%:%710=407%:%
%:%711=408%:%
%:%712=408%:%
%:%713=409%:%
%:%714=409%:%
%:%715=409%:%
%:%716=410%:%
%:%717=410%:%
%:%718=411%:%
%:%719=411%:%
%:%720=411%:%
%:%721=412%:%
%:%722=412%:%
%:%723=413%:%
%:%724=413%:%
%:%726=415%:%
%:%727=416%:%
%:%728=416%:%
%:%729=417%:%
%:%730=417%:%
%:%731=418%:%
%:%732=418%:%
%:%733=419%:%
%:%734=419%:%
%:%735=420%:%
%:%736=420%:%
%:%737=420%:%
%:%738=421%:%
%:%739=421%:%
%:%740=421%:%
%:%741=422%:%
%:%742=422%:%
%:%743=423%:%
%:%744=423%:%
%:%745=424%:%
%:%746=424%:%
%:%747=424%:%
%:%748=425%:%
%:%749=425%:%
%:%750=426%:%
%:%751=426%:%
%:%752=426%:%
%:%753=427%:%
%:%754=427%:%
%:%755=428%:%
%:%756=428%:%
%:%757=428%:%
%:%758=429%:%
%:%759=429%:%
%:%760=430%:%
%:%761=430%:%
%:%762=430%:%
%:%763=431%:%
%:%764=431%:%
%:%765=432%:%
%:%766=432%:%
%:%767=433%:%
%:%768=434%:%
%:%769=434%:%
%:%770=435%:%
%:%771=435%:%
%:%772=436%:%
%:%773=436%:%
%:%774=437%:%
%:%775=437%:%
%:%776=438%:%
%:%777=438%:%
%:%778=438%:%
%:%779=439%:%
%:%780=439%:%
%:%781=440%:%
%:%782=440%:%
%:%783=440%:%
%:%784=441%:%
%:%785=441%:%
%:%786=442%:%
%:%787=442%:%
%:%788=442%:%
%:%789=443%:%
%:%790=443%:%
%:%791=444%:%
%:%792=444%:%
%:%793=444%:%
%:%794=445%:%
%:%795=445%:%
%:%796=446%:%
%:%797=446%:%
%:%798=446%:%
%:%799=447%:%
%:%800=447%:%
%:%801=448%:%
%:%802=448%:%
%:%803=449%:%
%:%804=449%:%
%:%805=449%:%
%:%806=450%:%
%:%807=450%:%
%:%808=451%:%
%:%809=451%:%
%:%810=452%:%
%:%811=452%:%
%:%812=452%:%
%:%813=453%:%
%:%814=453%:%
%:%815=454%:%
%:%816=454%:%
%:%817=454%:%
%:%818=455%:%
%:%819=455%:%
%:%820=455%:%
%:%821=456%:%
%:%822=456%:%
%:%823=457%:%
%:%824=457%:%
%:%825=457%:%
%:%826=458%:%
%:%827=458%:%
%:%828=459%:%
%:%829=459%:%
%:%830=459%:%
%:%831=460%:%
%:%832=460%:%
%:%833=461%:%
%:%834=461%:%
%:%835=462%:%
%:%836=462%:%
%:%837=463%:%
%:%838=463%:%
%:%839=464%:%
%:%840=464%:%
%:%841=465%:%
%:%842=465%:%
%:%843=466%:%
%:%853=468%:%
%:%854=469%:%
%:%855=470%:%
%:%856=471%:%
%:%857=472%:%
%:%858=473%:%
%:%860=475%:%
%:%861=475%:%
%:%862=476%:%
%:%865=479%:%
%:%866=480%:%
%:%873=481%:%
%:%874=481%:%
%:%875=482%:%
%:%876=482%:%
%:%877=483%:%
%:%878=483%:%
%:%879=483%:%
%:%882=486%:%
%:%883=487%:%
%:%884=487%:%
%:%885=488%:%
%:%886=488%:%
%:%887=489%:%
%:%888=489%:%
%:%889=490%:%
%:%890=490%:%
%:%891=491%:%
%:%892=491%:%
%:%893=492%:%
%:%894=492%:%
%:%895=492%:%
%:%896=493%:%
%:%897=493%:%
%:%898=494%:%
%:%899=494%:%
%:%900=494%:%
%:%901=495%:%
%:%902=495%:%
%:%903=496%:%
%:%904=496%:%
%:%906=498%:%
%:%907=499%:%
%:%908=499%:%
%:%909=500%:%
%:%910=500%:%
%:%911=501%:%
%:%912=501%:%
%:%913=502%:%
%:%914=502%:%
%:%915=503%:%
%:%916=503%:%
%:%917=503%:%
%:%918=504%:%
%:%919=504%:%
%:%920=505%:%
%:%921=505%:%
%:%922=505%:%
%:%923=506%:%
%:%924=506%:%
%:%925=507%:%
%:%926=507%:%
%:%927=507%:%
%:%928=508%:%
%:%929=508%:%
%:%930=509%:%
%:%931=509%:%
%:%932=509%:%
%:%933=510%:%
%:%934=510%:%
%:%935=511%:%
%:%936=511%:%
%:%937=511%:%
%:%938=512%:%
%:%939=512%:%
%:%940=513%:%
%:%941=513%:%
%:%942=513%:%
%:%943=514%:%
%:%944=514%:%
%:%945=515%:%
%:%946=515%:%
%:%947=516%:%
%:%948=517%:%
%:%949=517%:%
%:%950=518%:%
%:%951=518%:%
%:%952=519%:%
%:%953=519%:%
%:%954=520%:%
%:%955=520%:%
%:%956=521%:%
%:%957=521%:%
%:%958=521%:%
%:%959=522%:%
%:%960=522%:%
%:%961=523%:%
%:%962=523%:%
%:%963=523%:%
%:%964=524%:%
%:%965=524%:%
%:%966=525%:%
%:%967=525%:%
%:%968=525%:%
%:%969=526%:%
%:%970=526%:%
%:%971=527%:%
%:%972=527%:%
%:%973=527%:%
%:%974=528%:%
%:%975=528%:%
%:%976=529%:%
%:%977=529%:%
%:%978=529%:%
%:%979=530%:%
%:%980=530%:%
%:%981=531%:%
%:%982=531%:%
%:%983=531%:%
%:%984=532%:%
%:%985=532%:%
%:%986=533%:%
%:%987=533%:%
%:%988=534%:%
%:%989=534%:%
%:%990=534%:%
%:%991=535%:%
%:%992=535%:%
%:%993=536%:%
%:%994=536%:%
%:%995=537%:%
%:%996=537%:%
%:%997=537%:%
%:%998=538%:%
%:%999=538%:%
%:%1000=539%:%
%:%1001=539%:%
%:%1002=539%:%
%:%1003=540%:%
%:%1004=540%:%
%:%1005=540%:%
%:%1006=541%:%
%:%1007=541%:%
%:%1008=542%:%
%:%1009=542%:%
%:%1010=542%:%
%:%1011=543%:%
%:%1012=543%:%
%:%1013=544%:%
%:%1014=544%:%
%:%1015=544%:%
%:%1016=545%:%
%:%1017=545%:%
%:%1018=546%:%
%:%1019=546%:%
%:%1020=547%:%
%:%1021=547%:%
%:%1022=548%:%
%:%1023=548%:%
%:%1024=549%:%
%:%1025=549%:%
%:%1026=550%:%
%:%1027=550%:%
%:%1028=551%:%
%:%1038=553%:%
%:%1039=554%:%
%:%1041=556%:%
%:%1042=556%:%
%:%1043=557%:%
%:%1044=558%:%
%:%1047=561%:%
%:%1054=562%:%
%:%1055=562%:%
%:%1056=563%:%
%:%1057=563%:%
%:%1065=571%:%
%:%1066=572%:%
%:%1067=572%:%
%:%1068=572%:%
%:%1069=573%:%
%:%1070=573%:%
%:%1071=573%:%
%:%1072=574%:%
%:%1073=574%:%
%:%1074=575%:%
%:%1075=575%:%
%:%1076=576%:%
%:%1077=576%:%
%:%1078=576%:%
%:%1079=577%:%
%:%1080=577%:%
%:%1081=578%:%
%:%1082=578%:%
%:%1083=579%:%
%:%1084=579%:%
%:%1085=580%:%
%:%1086=580%:%
%:%1087=581%:%
%:%1088=581%:%
%:%1089=581%:%
%:%1090=582%:%
%:%1091=582%:%
%:%1092=583%:%
%:%1093=583%:%
%:%1094=583%:%
%:%1095=584%:%
%:%1096=584%:%
%:%1097=585%:%
%:%1098=585%:%
%:%1099=585%:%
%:%1100=586%:%
%:%1101=586%:%
%:%1102=587%:%
%:%1103=587%:%
%:%1104=587%:%
%:%1105=588%:%
%:%1106=588%:%
%:%1109=591%:%
%:%1110=592%:%
%:%1111=592%:%
%:%1112=592%:%
%:%1113=593%:%
%:%1114=593%:%
%:%1115=594%:%
%:%1116=594%:%
%:%1117=595%:%
%:%1118=595%:%
%:%1119=596%:%
%:%1120=596%:%
%:%1121=597%:%
%:%1122=597%:%
%:%1123=598%:%
%:%1124=598%:%
%:%1125=599%:%
%:%1126=599%:%
%:%1127=600%:%
%:%1128=600%:%
%:%1129=600%:%
%:%1130=601%:%
%:%1131=601%:%
%:%1132=602%:%
%:%1133=602%:%
%:%1134=602%:%
%:%1135=603%:%
%:%1136=603%:%
%:%1137=604%:%
%:%1138=604%:%
%:%1139=604%:%
%:%1140=605%:%
%:%1141=605%:%
%:%1142=606%:%
%:%1143=606%:%
%:%1144=606%:%
%:%1145=607%:%
%:%1146=607%:%
%:%1149=610%:%
%:%1150=611%:%
%:%1151=611%:%
%:%1152=611%:%
%:%1153=612%:%
%:%1154=612%:%
%:%1155=613%:%
%:%1156=613%:%
%:%1157=614%:%
%:%1158=614%:%
%:%1159=615%:%
%:%1160=615%:%
%:%1163=618%:%
%:%1164=619%:%
%:%1165=619%:%
%:%1166=619%:%
%:%1167=620%:%
%:%1177=622%:%
%:%1178=623%:%
%:%1180=625%:%
%:%1181=625%:%
%:%1182=626%:%
%:%1185=629%:%
%:%1186=630%:%
%:%1193=631%:%
%:%1194=631%:%
%:%1195=632%:%
%:%1196=632%:%
%:%1197=633%:%
%:%1198=633%:%
%:%1199=633%:%
%:%1200=634%:%
%:%1201=634%:%
%:%1202=635%:%
%:%1203=635%:%
%:%1204=635%:%
%:%1205=636%:%
%:%1206=636%:%
%:%1214=644%:%
%:%1215=645%:%
%:%1216=645%:%
%:%1217=646%:%
%:%1218=646%:%
%:%1219=647%:%
%:%1220=647%:%
%:%1221=647%:%
%:%1222=648%:%
%:%1223=648%:%
%:%1224=649%:%
%:%1225=649%:%
%:%1226=649%:%
%:%1227=650%:%
%:%1228=650%:%
%:%1229=651%:%
%:%1230=651%:%
%:%1231=652%:%
%:%1232=652%:%
%:%1233=653%:%
%:%1234=653%:%
%:%1235=654%:%
%:%1236=654%:%
%:%1237=655%:%
%:%1238=655%:%
%:%1239=656%:%
%:%1240=656%:%
%:%1241=657%:%
%:%1242=657%:%
%:%1243=658%:%
%:%1244=658%:%
%:%1245=659%:%
%:%1246=659%:%
%:%1247=659%:%
%:%1248=660%:%
%:%1249=660%:%
%:%1250=660%:%
%:%1251=661%:%
%:%1252=661%:%
%:%1253=661%:%
%:%1254=662%:%
%:%1255=662%:%
%:%1256=663%:%
%:%1257=663%:%
%:%1258=663%:%
%:%1259=664%:%
%:%1260=664%:%
%:%1261=665%:%
%:%1262=665%:%
%:%1263=666%:%
%:%1264=666%:%
%:%1265=667%:%
%:%1266=667%:%
%:%1267=668%:%
%:%1268=668%:%
%:%1269=669%:%
%:%1270=669%:%
%:%1271=670%:%
%:%1272=670%:%
%:%1273=671%:%
%:%1274=671%:%
%:%1275=672%:%
%:%1276=672%:%
%:%1277=673%:%
%:%1278=673%:%
%:%1279=674%:%
%:%1280=674%:%
%:%1281=675%:%
%:%1282=675%:%
%:%1283=675%:%
%:%1284=676%:%
%:%1285=676%:%
%:%1286=676%:%
%:%1287=677%:%
%:%1288=677%:%
%:%1289=677%:%
%:%1290=678%:%
%:%1291=678%:%
%:%1292=679%:%
%:%1293=679%:%
%:%1294=679%:%
%:%1295=680%:%
%:%1296=680%:%
%:%1297=681%:%
%:%1298=681%:%
%:%1299=682%:%
%:%1300=682%:%
%:%1301=683%:%
%:%1302=683%:%
%:%1303=684%:%
%:%1304=684%:%
%:%1305=685%:%
%:%1306=685%:%
%:%1307=686%:%
%:%1308=686%:%
%:%1309=687%:%
%:%1310=687%:%
%:%1311=688%:%
%:%1312=688%:%
%:%1313=689%:%
%:%1314=689%:%
%:%1315=690%:%
%:%1316=690%:%
%:%1317=691%:%
%:%1318=691%:%
%:%1319=691%:%
%:%1320=692%:%
%:%1321=692%:%
%:%1322=692%:%
%:%1323=693%:%
%:%1324=693%:%
%:%1325=693%:%
%:%1326=694%:%
%:%1327=694%:%
%:%1328=695%:%
%:%1329=695%:%
%:%1330=695%:%
%:%1331=696%:%
%:%1332=696%:%
%:%1333=697%:%
%:%1334=697%:%
%:%1335=698%:%
%:%1336=698%:%
%:%1337=699%:%
%:%1338=699%:%
%:%1339=700%:%
%:%1340=700%:%
%:%1341=701%:%
%:%1342=701%:%
%:%1343=702%:%
%:%1344=702%:%
%:%1345=703%:%
%:%1346=703%:%
%:%1347=704%:%
%:%1348=704%:%
%:%1349=705%:%
%:%1350=705%:%
%:%1351=706%:%
%:%1352=706%:%
%:%1353=707%:%
%:%1354=707%:%
%:%1355=707%:%
%:%1356=708%:%
%:%1357=708%:%
%:%1358=708%:%
%:%1359=709%:%
%:%1360=709%:%
%:%1361=709%:%
%:%1362=710%:%
%:%1363=710%:%
%:%1364=710%:%
%:%1365=711%:%
%:%1366=711%:%
%:%1367=711%:%
%:%1368=712%:%
%:%1369=712%:%
%:%1370=713%:%
%:%1371=713%:%
%:%1372=714%:%
%:%1373=714%:%
%:%1374=715%:%
%:%1375=715%:%
%:%1376=716%:%
%:%1377=716%:%
%:%1378=717%:%
%:%1379=717%:%
%:%1380=718%:%
%:%1381=718%:%
%:%1382=719%:%
%:%1383=719%:%
%:%1384=720%:%
%:%1385=720%:%
%:%1386=721%:%
%:%1387=721%:%
%:%1388=722%:%
%:%1389=722%:%
%:%1390=723%:%
%:%1391=723%:%
%:%1392=724%:%
%:%1393=724%:%
%:%1394=725%:%
%:%1395=725%:%
%:%1396=725%:%
%:%1397=726%:%
%:%1398=726%:%
%:%1399=726%:%
%:%1400=727%:%
%:%1401=727%:%
%:%1402=727%:%
%:%1403=728%:%
%:%1404=728%:%
%:%1405=729%:%
%:%1406=729%:%
%:%1407=729%:%
%:%1408=730%:%
%:%1409=730%:%
%:%1410=731%:%
%:%1411=731%:%
%:%1412=732%:%
%:%1413=732%:%
%:%1414=733%:%
%:%1415=733%:%
%:%1416=734%:%
%:%1417=734%:%
%:%1418=735%:%
%:%1419=735%:%
%:%1420=736%:%
%:%1421=736%:%
%:%1422=737%:%
%:%1423=737%:%
%:%1424=738%:%
%:%1425=738%:%
%:%1426=739%:%
%:%1427=739%:%
%:%1428=740%:%
%:%1429=740%:%
%:%1430=741%:%
%:%1431=741%:%
%:%1432=741%:%
%:%1433=742%:%
%:%1434=742%:%
%:%1435=742%:%
%:%1436=743%:%
%:%1437=743%:%
%:%1438=743%:%
%:%1439=744%:%
%:%1440=744%:%
%:%1441=745%:%
%:%1442=745%:%
%:%1443=745%:%
%:%1444=746%:%
%:%1445=746%:%
%:%1446=747%:%
%:%1447=747%:%
%:%1448=748%:%
%:%1449=748%:%
%:%1450=749%:%
%:%1451=749%:%
%:%1452=750%:%
%:%1453=750%:%
%:%1454=751%:%
%:%1455=751%:%
%:%1456=752%:%
%:%1457=752%:%
%:%1458=753%:%
%:%1459=753%:%
%:%1460=754%:%
%:%1461=754%:%
%:%1462=755%:%
%:%1463=755%:%
%:%1464=756%:%
%:%1465=756%:%
%:%1466=757%:%
%:%1467=757%:%
%:%1468=757%:%
%:%1469=758%:%
%:%1470=758%:%
%:%1471=758%:%
%:%1472=759%:%
%:%1473=759%:%
%:%1474=759%:%
%:%1475=760%:%
%:%1476=760%:%
%:%1477=761%:%
%:%1478=761%:%
%:%1479=761%:%
%:%1480=762%:%
%:%1481=762%:%
%:%1482=763%:%
%:%1483=763%:%
%:%1484=764%:%
%:%1485=764%:%
%:%1489=768%:%
%:%1490=769%:%
%:%1491=769%:%
%:%1492=769%:%
%:%1493=770%:%
%:%1494=770%:%
%:%1497=773%:%
%:%1498=774%:%
%:%1499=774%:%
%:%1500=774%:%
%:%1501=775%:%
%:%1502=775%:%
%:%1510=783%:%
%:%1511=784%:%
%:%1512=784%:%
%:%1513=784%:%
%:%1514=785%:%
%:%1515=785%:%
%:%1523=793%:%
%:%1524=794%:%
%:%1525=794%:%
%:%1526=795%:%
%:%1527=795%:%
%:%1528=796%:%
%:%1529=796%:%
%:%1530=797%:%
%:%1531=797%:%
%:%1532=797%:%
%:%1533=798%:%
%:%1543=800%:%
%:%1545=802%:%
%:%1546=802%:%
%:%1549=805%:%
%:%1556=806%:%
%:%1557=806%:%
%:%1558=807%:%
%:%1559=807%:%
%:%1560=808%:%
%:%1561=808%:%
%:%1564=811%:%
%:%1565=812%:%
%:%1566=812%:%
%:%1567=813%:%
%:%1568=813%:%
%:%1569=814%:%
%:%1570=814%:%
%:%1573=817%:%
%:%1574=818%:%
%:%1575=818%:%
%:%1576=819%:%
%:%1577=819%:%
%:%1578=820%:%
%:%1588=822%:%
%:%1590=824%:%
%:%1591=824%:%
%:%1594=827%:%
%:%1597=828%:%
%:%1601=828%:%
%:%1602=828%:%
%:%1603=829%:%
%:%1604=829%:%
%:%1605=830%:%
%:%1606=830%:%
%:%1607=830%:%
%:%1608=831%:%
%:%1609=831%:%
%:%1610=831%:%
%:%1611=832%:%
%:%1612=832%:%

%
\begin{isabellebody}%
\setisabellecontext{Pcp}%
%
\isadelimtheory
%
\endisadelimtheory
%
\isatagtheory
%
\endisatagtheory
{\isafoldtheory}%
%
\isadelimtheory
%
\endisadelimtheory
%
\begin{isamarkuptext}%
En este capítulo nos centraremos en demostrar el \isa{teorema\ de\ existencia\ de\ modelos}.
  Dicho teorema prueba la satisfacibilidad de un conjunto de fórmulas \isa{S} si este pertenece a una 
  colección de conjuntos \isa{C} que verifica la \isa{propiedad\ de\ consistencia\ proposicional}. Para su 
  prueba, definiremos las propiedades de \isa{carácter\ finito} y \isa{ser\ cerrada\ bajo\ subconjuntos} para
  colecciones de conjuntos de fórmulas. De este modo, mediante distintos resultados que relacionan
  estas propiedades con la \isa{propiedad\ de\ consistencia\ proposicional}, dada una colección \isa{C} 
  cualquiera en las condiciones anteriormente descritas, podemos encontrar una colección \isa{C{\isacharprime}} que la 
  contenga que verifique la \isa{propiedad\ de\ consistencia\ proposicional}, sea \isa{cerrada\ bajo\ subconjuntos} y de \isa{carácter\ finito}. Por otro lado, definiremos una sucesión de conjuntos de
  fórmulas a partir de la colección \isa{C{\isacharprime}} y el conjunto \isa{S}. Además, definiremos el límite de dicha
  sucesión que, en particular, contendrá al conjunto \isa{S}. Finalmente probaremos que dicho límite es 
  un conjunto satisfacible por el \isa{lema\ de\ Hintikka} y, por contención, quedará probada la 
  satisfacibilidad del conjunto \isa{S}.

  \comentario{Modificar dicho párrafo al final y ver cómo reubicar explicación completa del
  teorema de existencia de modelo.}%
\end{isamarkuptext}\isamarkuptrue%
%
\begin{isamarkuptext}%
En esta sección vamos a definir la \isa{propiedad\ de\ consistencia\ proposicional} para una 
  colección de conjuntos de fórmulas proposicionales. Finalmente mostraremos un lema
  que caracteriza dicha propiedad mediante la notación uniforme.%
\end{isamarkuptext}\isamarkuptrue%
%
\begin{isamarkuptext}%
\begin{definicion}
    Sea \isa{C} una colección de conjuntos de fórmulas proposicionales. Decimos que
    \isa{C} verifica la \isa{propiedad\ de\ consistencia\ proposicional} si, para todo
    conjunto \isa{S} perteneciente a la colección, se verifica:
    \begin{enumerate}
      \item \isa{{\isasymbottom}\ {\isasymnotin}\ S}.
      \item Dada \isa{p} una fórmula atómica cualquiera, no se tiene 
        simultáneamente que\\ \isa{p\ {\isasymin}\ S} y \isa{{\isasymnot}\ p\ {\isasymin}\ S}.
      \item Si \isa{F\ {\isasymand}\ G\ {\isasymin}\ S}, entonces el conjunto \isa{{\isacharbraceleft}F{\isacharcomma}G{\isacharbraceright}\ {\isasymunion}\ S} pertenece a \isa{C}.
      \item Si \isa{F\ {\isasymor}\ G\ {\isasymin}\ S}, entonces o bien el conjunto \isa{{\isacharbraceleft}F{\isacharbraceright}\ {\isasymunion}\ S} pertenece a \isa{C}, o bien el 
        conjunto \isa{{\isacharbraceleft}G{\isacharbraceright}\ {\isasymunion}\ S} pertenece a \isa{C}.
      \item Si \isa{F\ {\isasymrightarrow}\ G\ {\isasymin}\ S}, entonces o bien el conjunto \isa{{\isacharbraceleft}{\isasymnot}\ F{\isacharbraceright}\ {\isasymunion}\ S} pertenece a \isa{C}, o bien el 
        conjunto \isa{{\isacharbraceleft}G{\isacharbraceright}\ {\isasymunion}\ S} pertenece a \isa{C}.
      \item Si \isa{{\isasymnot}{\isacharparenleft}{\isasymnot}\ F{\isacharparenright}\ {\isasymin}\ S}, entonces el conjunto \isa{{\isacharbraceleft}F{\isacharbraceright}\ {\isasymunion}\ S} pertenece a \isa{C}.
      \item Si \isa{{\isasymnot}{\isacharparenleft}F\ {\isasymand}\ G{\isacharparenright}\ {\isasymin}\ S}, entonces o bien el conjunto \isa{{\isacharbraceleft}{\isasymnot}\ F{\isacharbraceright}\ {\isasymunion}\ S} pertenece a \isa{C}, o bien el 
        conjunto \isa{{\isacharbraceleft}{\isasymnot}\ G{\isacharbraceright}\ {\isasymunion}\ S} pertenece a \isa{C}.
      \item Si \isa{{\isasymnot}{\isacharparenleft}F\ {\isasymor}\ G{\isacharparenright}\ {\isasymin}\ S}, entonces el conjunto \isa{{\isacharbraceleft}{\isasymnot}\ F{\isacharcomma}\ {\isasymnot}\ G{\isacharbraceright}\ {\isasymunion}\ S} pertenece a \isa{C}.
      \item Si \isa{{\isasymnot}{\isacharparenleft}F\ {\isasymrightarrow}\ G{\isacharparenright}\ {\isasymin}\ S}, entonces el conjunto \isa{{\isacharbraceleft}F{\isacharcomma}\ {\isasymnot}\ G{\isacharbraceright}\ {\isasymunion}\ S} pertenece a \isa{C}.
    \end{enumerate}
  \end{definicion}

  Veamos, a continuación, su formalización en Isabelle mediante el tipo \isa{definition}.%
\end{isamarkuptext}\isamarkuptrue%
\isacommand{definition}\isamarkupfalse%
\ {\isachardoublequoteopen}pcp\ C\ {\isasymequiv}\ {\isacharparenleft}{\isasymforall}S\ {\isasymin}\ C{\isachardot}\isanewline
\ \ {\isasymbottom}\ {\isasymnotin}\ S\isanewline
{\isasymand}\ {\isacharparenleft}{\isasymforall}k{\isachardot}\ Atom\ k\ {\isasymin}\ S\ {\isasymlongrightarrow}\ \isactrlbold {\isasymnot}\ {\isacharparenleft}Atom\ k{\isacharparenright}\ {\isasymin}\ S\ {\isasymlongrightarrow}\ False{\isacharparenright}\isanewline
{\isasymand}\ {\isacharparenleft}{\isasymforall}F\ G{\isachardot}\ F\ \isactrlbold {\isasymand}\ G\ {\isasymin}\ S\ {\isasymlongrightarrow}\ {\isacharbraceleft}F{\isacharcomma}G{\isacharbraceright}\ {\isasymunion}\ S\ {\isasymin}\ C{\isacharparenright}\isanewline
{\isasymand}\ {\isacharparenleft}{\isasymforall}F\ G{\isachardot}\ F\ \isactrlbold {\isasymor}\ G\ {\isasymin}\ S\ {\isasymlongrightarrow}\ {\isacharbraceleft}F{\isacharbraceright}\ {\isasymunion}\ S\ {\isasymin}\ C\ {\isasymor}\ {\isacharbraceleft}G{\isacharbraceright}\ {\isasymunion}\ S\ {\isasymin}\ C{\isacharparenright}\isanewline
{\isasymand}\ {\isacharparenleft}{\isasymforall}F\ G{\isachardot}\ F\ \isactrlbold {\isasymrightarrow}\ G\ {\isasymin}\ S\ {\isasymlongrightarrow}\ {\isacharbraceleft}\isactrlbold {\isasymnot}F{\isacharbraceright}\ {\isasymunion}\ S\ {\isasymin}\ C\ {\isasymor}\ {\isacharbraceleft}G{\isacharbraceright}\ {\isasymunion}\ S\ {\isasymin}\ C{\isacharparenright}\isanewline
{\isasymand}\ {\isacharparenleft}{\isasymforall}F{\isachardot}\ \isactrlbold {\isasymnot}\ {\isacharparenleft}\isactrlbold {\isasymnot}F{\isacharparenright}\ {\isasymin}\ S\ {\isasymlongrightarrow}\ {\isacharbraceleft}F{\isacharbraceright}\ {\isasymunion}\ S\ {\isasymin}\ C{\isacharparenright}\isanewline
{\isasymand}\ {\isacharparenleft}{\isasymforall}F\ G{\isachardot}\ \isactrlbold {\isasymnot}{\isacharparenleft}F\ \isactrlbold {\isasymand}\ G{\isacharparenright}\ {\isasymin}\ S\ {\isasymlongrightarrow}\ {\isacharbraceleft}\isactrlbold {\isasymnot}\ F{\isacharbraceright}\ {\isasymunion}\ S\ {\isasymin}\ C\ {\isasymor}\ {\isacharbraceleft}\isactrlbold {\isasymnot}\ G{\isacharbraceright}\ {\isasymunion}\ S\ {\isasymin}\ C{\isacharparenright}\isanewline
{\isasymand}\ {\isacharparenleft}{\isasymforall}F\ G{\isachardot}\ \isactrlbold {\isasymnot}{\isacharparenleft}F\ \isactrlbold {\isasymor}\ G{\isacharparenright}\ {\isasymin}\ S\ {\isasymlongrightarrow}\ {\isacharbraceleft}\isactrlbold {\isasymnot}\ F{\isacharcomma}\ \isactrlbold {\isasymnot}\ G{\isacharbraceright}\ {\isasymunion}\ S\ {\isasymin}\ C{\isacharparenright}\isanewline
{\isasymand}\ {\isacharparenleft}{\isasymforall}F\ G{\isachardot}\ \isactrlbold {\isasymnot}{\isacharparenleft}F\ \isactrlbold {\isasymrightarrow}\ G{\isacharparenright}\ {\isasymin}\ S\ {\isasymlongrightarrow}\ {\isacharbraceleft}F{\isacharcomma}\isactrlbold {\isasymnot}\ G{\isacharbraceright}\ {\isasymunion}\ S\ {\isasymin}\ C{\isacharparenright}{\isacharparenright}{\isachardoublequoteclose}%
\begin{isamarkuptext}%
Observando la definición anterior, se prueba fácilmente que la colección trivial
  formada por el conjunto vacío de fórmulas verifica la propiedad de consistencia 
  proposicional.%
\end{isamarkuptext}\isamarkuptrue%
\isacommand{lemma}\isamarkupfalse%
\ {\isachardoublequoteopen}pcp\ {\isacharbraceleft}{\isacharbraceleft}{\isacharbraceright}{\isacharbraceright}{\isachardoublequoteclose}\isanewline
%
\isadelimproof
\ \ %
\endisadelimproof
%
\isatagproof
\isacommand{unfolding}\isamarkupfalse%
\ pcp{\isacharunderscore}def\ \isacommand{by}\isamarkupfalse%
\ simp%
\endisatagproof
{\isafoldproof}%
%
\isadelimproof
%
\endisadelimproof
%
\begin{isamarkuptext}%
Del mismo modo, aplicando la definición, se demuestra que los siguientes ejemplos
  de colecciones de conjuntos de fórmulas proposicionales verifican igualmente la 
  propiedad.%
\end{isamarkuptext}\isamarkuptrue%
\isacommand{lemma}\isamarkupfalse%
\ {\isachardoublequoteopen}pcp\ {\isacharbraceleft}{\isacharbraceleft}Atom\ {\isadigit{0}}{\isacharbraceright}{\isacharbraceright}{\isachardoublequoteclose}\isanewline
%
\isadelimproof
\ \ %
\endisadelimproof
%
\isatagproof
\isacommand{unfolding}\isamarkupfalse%
\ pcp{\isacharunderscore}def\ \isacommand{by}\isamarkupfalse%
\ simp%
\endisatagproof
{\isafoldproof}%
%
\isadelimproof
\isanewline
%
\endisadelimproof
\isanewline
\isacommand{lemma}\isamarkupfalse%
\ {\isachardoublequoteopen}pcp\ {\isacharbraceleft}{\isacharbraceleft}{\isacharparenleft}\isactrlbold {\isasymnot}\ {\isacharparenleft}Atom\ {\isadigit{1}}{\isacharparenright}{\isacharparenright}\ \isactrlbold {\isasymrightarrow}\ Atom\ {\isadigit{2}}{\isacharbraceright}{\isacharcomma}\isanewline
\ \ \ {\isacharbraceleft}{\isacharparenleft}{\isacharparenleft}\isactrlbold {\isasymnot}\ {\isacharparenleft}Atom\ {\isadigit{1}}{\isacharparenright}{\isacharparenright}\ \isactrlbold {\isasymrightarrow}\ Atom\ {\isadigit{2}}{\isacharparenright}{\isacharcomma}\ \isactrlbold {\isasymnot}{\isacharparenleft}\isactrlbold {\isasymnot}\ {\isacharparenleft}Atom\ {\isadigit{1}}{\isacharparenright}{\isacharparenright}{\isacharbraceright}{\isacharcomma}\isanewline
\ \ {\isacharbraceleft}{\isacharparenleft}{\isacharparenleft}\isactrlbold {\isasymnot}\ {\isacharparenleft}Atom\ {\isadigit{1}}{\isacharparenright}{\isacharparenright}\ \isactrlbold {\isasymrightarrow}\ Atom\ {\isadigit{2}}{\isacharparenright}{\isacharcomma}\ \isactrlbold {\isasymnot}{\isacharparenleft}\isactrlbold {\isasymnot}\ {\isacharparenleft}Atom\ {\isadigit{1}}{\isacharparenright}{\isacharparenright}{\isacharcomma}\ \ Atom\ {\isadigit{1}}{\isacharbraceright}{\isacharbraceright}{\isachardoublequoteclose}\ \isanewline
%
\isadelimproof
\ \ %
\endisadelimproof
%
\isatagproof
\isacommand{unfolding}\isamarkupfalse%
\ pcp{\isacharunderscore}def\ \isacommand{by}\isamarkupfalse%
\ auto%
\endisatagproof
{\isafoldproof}%
%
\isadelimproof
%
\endisadelimproof
%
\begin{isamarkuptext}%
Por último, en contraposición podemos ilustrar un caso de colección que no verifique la 
  propiedad con la siguiente colección obtenida al modificar el último ejemplo. De
  esta manera, aunque la colección verifique correctamente la quinta condición de la
  definición, no cumplirá la sexta.%
\end{isamarkuptext}\isamarkuptrue%
\isacommand{lemma}\isamarkupfalse%
\ {\isachardoublequoteopen}{\isasymnot}\ pcp\ {\isacharbraceleft}{\isacharbraceleft}{\isacharparenleft}\isactrlbold {\isasymnot}\ {\isacharparenleft}Atom\ {\isadigit{1}}{\isacharparenright}{\isacharparenright}\ \isactrlbold {\isasymrightarrow}\ Atom\ {\isadigit{2}}{\isacharbraceright}{\isacharcomma}\isanewline
\ \ \ {\isacharbraceleft}{\isacharparenleft}{\isacharparenleft}\isactrlbold {\isasymnot}\ {\isacharparenleft}Atom\ {\isadigit{1}}{\isacharparenright}{\isacharparenright}\ \isactrlbold {\isasymrightarrow}\ Atom\ {\isadigit{2}}{\isacharparenright}{\isacharcomma}\ \isactrlbold {\isasymnot}{\isacharparenleft}\isactrlbold {\isasymnot}\ {\isacharparenleft}Atom\ {\isadigit{1}}{\isacharparenright}{\isacharparenright}{\isacharbraceright}{\isacharbraceright}{\isachardoublequoteclose}\ \isanewline
%
\isadelimproof
\ \ %
\endisadelimproof
%
\isatagproof
\isacommand{unfolding}\isamarkupfalse%
\ pcp{\isacharunderscore}def\ \isacommand{by}\isamarkupfalse%
\ auto%
\endisatagproof
{\isafoldproof}%
%
\isadelimproof
%
\endisadelimproof
%
\begin{isamarkuptext}%
Por otra parte, veamos un resultado que permite la caracterización de la 
  propiedad de consistencia proposicional mediante la notación uniforme.

  \begin{lema}[Caracterización de \isa{P{\isachardot}C{\isachardot}P} mediante la notación uniforme]
    Dada una colección \isa{C} de conjuntos de fórmulas proposicionales, son equivalentes:
    \begin{enumerate}
      \item \isa{C} verifica la propiedad de consistencia proposicional.
      \item Para cualquier conjunto de fórmulas \isa{S} de la colección, se verifican las 
      condiciones:
      \begin{itemize}
        \item \isa{{\isasymbottom}} no pertenece a \isa{S}.
        \item Dada \isa{p} una fórmula atómica cualquiera, no se tiene 
        simultáneamente que\\ \isa{p\ {\isasymin}\ S} y \isa{{\isasymnot}\ p\ {\isasymin}\ S}.
        \item Para toda fórmula de tipo \isa{{\isasymalpha}} con componentes \isa{{\isasymalpha}\isactrlsub {\isadigit{1}}} y \isa{{\isasymalpha}\isactrlsub {\isadigit{2}}} tal que \isa{{\isasymalpha}}
        pertenece a \isa{S}, se tiene que \isa{{\isacharbraceleft}{\isasymalpha}\isactrlsub {\isadigit{1}}{\isacharcomma}{\isasymalpha}\isactrlsub {\isadigit{2}}{\isacharbraceright}\ {\isasymunion}\ S} pertenece a \isa{C}.
        \item Para toda fórmula de tipo \isa{{\isasymbeta}} con componentes \isa{{\isasymbeta}\isactrlsub {\isadigit{1}}} y \isa{{\isasymbeta}\isactrlsub {\isadigit{2}}} tal que \isa{{\isasymbeta}}
        pertenece a \isa{S}, se tiene que o bien \isa{{\isacharbraceleft}{\isasymbeta}\isactrlsub {\isadigit{1}}{\isacharbraceright}\ {\isasymunion}\ S} pertenece a \isa{C} o 
        bien \isa{{\isacharbraceleft}{\isasymbeta}\isactrlsub {\isadigit{2}}{\isacharbraceright}\ {\isasymunion}\ S} pertenece a \isa{C}.
      \end{itemize} 
    \end{enumerate}
  \end{lema}

  En Isabelle/HOL se formaliza el resultado como sigue.%
\end{isamarkuptext}\isamarkuptrue%
\isacommand{lemma}\isamarkupfalse%
\ {\isachardoublequoteopen}pcp\ C\ {\isacharequal}\ {\isacharparenleft}{\isasymforall}S\ {\isasymin}\ C{\isachardot}\ {\isasymbottom}\ {\isasymnotin}\ S\isanewline
{\isasymand}\ {\isacharparenleft}{\isasymforall}k{\isachardot}\ Atom\ k\ {\isasymin}\ S\ {\isasymlongrightarrow}\ \isactrlbold {\isasymnot}\ {\isacharparenleft}Atom\ k{\isacharparenright}\ {\isasymin}\ S\ {\isasymlongrightarrow}\ False{\isacharparenright}\isanewline
{\isasymand}\ {\isacharparenleft}{\isasymforall}F\ G\ H{\isachardot}\ Con\ F\ G\ H\ {\isasymlongrightarrow}\ F\ {\isasymin}\ S\ {\isasymlongrightarrow}\ {\isacharbraceleft}G{\isacharcomma}H{\isacharbraceright}\ {\isasymunion}\ S\ {\isasymin}\ C{\isacharparenright}\isanewline
{\isasymand}\ {\isacharparenleft}{\isasymforall}F\ G\ H{\isachardot}\ Dis\ F\ G\ H\ {\isasymlongrightarrow}\ F\ {\isasymin}\ S\ {\isasymlongrightarrow}\ {\isacharbraceleft}G{\isacharbraceright}\ {\isasymunion}\ S\ {\isasymin}\ C\ {\isasymor}\ {\isacharbraceleft}H{\isacharbraceright}\ {\isasymunion}\ S\ {\isasymin}\ C{\isacharparenright}{\isacharparenright}{\isachardoublequoteclose}\isanewline
%
\isadelimproof
\ \ %
\endisadelimproof
%
\isatagproof
\isacommand{oops}\isamarkupfalse%
%
\endisatagproof
{\isafoldproof}%
%
\isadelimproof
%
\endisadelimproof
%
\begin{isamarkuptext}%
En primer lugar, veamos la demostración del lema.

\begin{demostracion}
  Para probar la equivalencia, veamos cada una de las implicaciones por separado.

\textbf{\isa{{\isadigit{1}}{\isacharparenright}\ {\isasymLongrightarrow}\ {\isadigit{2}}{\isacharparenright}}}
  
  Supongamos que \isa{C} es una colección de conjuntos de fórmulas proposicionales que
  verifica la propiedad de consistencia proposicional. Vamos a probar que, en efecto,
  cumple las condiciones de \isa{{\isadigit{2}}{\isacharparenright}}. 

  Consideremos un conjunto de fórmulas \isa{S} perteneciente a la colección \isa{C}.
  Por hipótesis, de la definición de propiedad de consistencia proposicional obtenemos
  que \isa{S} verifica las siguientes condiciones:
 \begin{enumerate}
      \item \isa{{\isasymbottom}\ {\isasymnotin}\ S}.
      \item Dada \isa{p} una fórmula atómica cualquiera, no se tiene 
        simultáneamente que\\ \isa{p\ {\isasymin}\ S} y \isa{{\isasymnot}\ p\ {\isasymin}\ S}.
      \item Si \isa{G\ {\isasymand}\ H\ {\isasymin}\ S}, entonces el conjunto \isa{{\isacharbraceleft}G{\isacharcomma}H{\isacharbraceright}\ {\isasymunion}\ S} pertenece a \isa{C}.
      \item Si \isa{G\ {\isasymor}\ H\ {\isasymin}\ S}, entonces o bien el conjunto \isa{{\isacharbraceleft}G{\isacharbraceright}\ {\isasymunion}\ S} pertenece a \isa{C}, o bien el 
        conjunto \isa{{\isacharbraceleft}H{\isacharbraceright}\ {\isasymunion}\ S} pertenece a \isa{C}.
      \item Si \isa{G\ {\isasymrightarrow}\ H\ {\isasymin}\ S}, entonces o bien el conjunto \isa{{\isacharbraceleft}{\isasymnot}\ G{\isacharbraceright}\ {\isasymunion}\ S} pertenece a \isa{C}, o bien el 
        conjunto \isa{{\isacharbraceleft}H{\isacharbraceright}\ {\isasymunion}\ S} pertenece a \isa{C}.
      \item Si \isa{{\isasymnot}{\isacharparenleft}{\isasymnot}\ G{\isacharparenright}\ {\isasymin}\ S}, entonces el conjunto \isa{{\isacharbraceleft}G{\isacharbraceright}\ {\isasymunion}\ S} pertenece a \isa{C}.
      \item Si \isa{{\isasymnot}{\isacharparenleft}G\ {\isasymand}\ H{\isacharparenright}\ {\isasymin}\ S}, entonces o bien el conjunto \isa{{\isacharbraceleft}{\isasymnot}\ G{\isacharbraceright}\ {\isasymunion}\ S} pertenece a \isa{C}, o bien el 
        conjunto \isa{{\isacharbraceleft}{\isasymnot}\ H{\isacharbraceright}\ {\isasymunion}\ S} pertenece a \isa{C}.
      \item Si \isa{{\isasymnot}{\isacharparenleft}G\ {\isasymor}\ H{\isacharparenright}\ {\isasymin}\ S}, entonces el conjunto \isa{{\isacharbraceleft}{\isasymnot}\ G{\isacharcomma}\ {\isasymnot}\ H{\isacharbraceright}\ {\isasymunion}\ S} pertenece a \isa{C}.
      \item Si \isa{{\isasymnot}{\isacharparenleft}G\ {\isasymrightarrow}\ H{\isacharparenright}\ {\isasymin}\ S}, entonces el conjunto \isa{{\isacharbraceleft}G{\isacharcomma}\ {\isasymnot}\ H{\isacharbraceright}\ {\isasymunion}\ S} pertenece a \isa{C}.
 \end{enumerate}

  Las dos primeras condiciones se corresponden con los dos primeros resultados que queríamos
  demostrar. De este modo, falta probar:
  \begin{itemize}
     \item Para toda fórmula de tipo \isa{{\isasymalpha}} con componentes \isa{{\isasymalpha}\isactrlsub {\isadigit{1}}} y \isa{{\isasymalpha}\isactrlsub {\isadigit{2}}} tal que \isa{{\isasymalpha}}
     pertenece a \isa{S}, se tiene que \isa{{\isacharbraceleft}{\isasymalpha}\isactrlsub {\isadigit{1}}{\isacharcomma}{\isasymalpha}\isactrlsub {\isadigit{2}}{\isacharbraceright}\ {\isasymunion}\ S} pertenece a \isa{C}.
     \item Para toda fórmula de tipo \isa{{\isasymbeta}} con componentes \isa{{\isasymbeta}\isactrlsub {\isadigit{1}}} y \isa{{\isasymbeta}\isactrlsub {\isadigit{2}}} tal que \isa{{\isasymbeta}}
     pertenece a \isa{S}, se tiene que o bien \isa{{\isacharbraceleft}{\isasymbeta}\isactrlsub {\isadigit{1}}{\isacharbraceright}\ {\isasymunion}\ S} pertenece a \isa{C} o 
     bien \isa{{\isacharbraceleft}{\isasymbeta}\isactrlsub {\isadigit{2}}{\isacharbraceright}\ {\isasymunion}\ S} pertenece a \isa{C}.   
  \end{itemize} 

  En primer lugar, vamos a deducir el primer resultado correspondiente a las fórmulas
  de tipo \isa{{\isasymalpha}} de las condiciones tercera, sexta, octava y novena de la definición de 
  propiedad de consistencia proposicional. En efecto, consideremos una fórmula de tipo 
  \isa{{\isasymalpha}} cualquiera con componentes \isa{{\isasymalpha}\isactrlsub {\isadigit{1}}} y \isa{{\isasymalpha}\isactrlsub {\isadigit{2}}} tal que \isa{{\isasymalpha}} pertenece a \isa{S}. Sabemos que 
  la fórmula es de la forma \isa{G\ {\isasymand}\ H}, \isa{{\isasymnot}\ {\isacharparenleft}{\isasymnot}\ G{\isacharparenright}}, \isa{{\isasymnot}\ {\isacharparenleft}G\ {\isasymor}\ H{\isacharparenright}} o 
  \isa{{\isasymnot}{\isacharparenleft}G\ {\isasymlongrightarrow}\ H{\isacharparenright}} para ciertas fórmulas \isa{G} y \isa{H}. Vamos a probar que para cada caso se cumple que 
  \isa{{\isacharbraceleft}{\isasymalpha}\isactrlsub {\isadigit{1}}{\isacharcomma}\ {\isasymalpha}\isactrlsub {\isadigit{2}}{\isacharbraceright}\ {\isasymunion}\ S} pertenece a la colección:

  \isa{{\isasymsqdot}\ Fórmula\ de\ tipo\ G\ {\isasymand}\ H}: En este caso, sus componentes conjuntivas \isa{{\isasymalpha}\isactrlsub {\isadigit{1}}} y \isa{{\isasymalpha}\isactrlsub {\isadigit{2}}} son \isa{G} 
    y \isa{H} respectivamente. Luego tenemos que \isa{{\isacharbraceleft}{\isasymalpha}\isactrlsub {\isadigit{1}}{\isacharcomma}\ {\isasymalpha}\isactrlsub {\isadigit{2}}{\isacharbraceright}\ {\isasymunion}\ S}  pertenece a \isa{C} por
    la tercera condición de la definición de propiedad de consistencia
    proposicional.

  \isa{{\isasymsqdot}\ Fórmula\ de\ tipo\ {\isasymnot}\ {\isacharparenleft}{\isasymnot}\ G{\isacharparenright}}: En este caso, sus componentes conjuntivas \isa{{\isasymalpha}\isactrlsub {\isadigit{1}}} y \isa{{\isasymalpha}\isactrlsub {\isadigit{2}}} son 
    ambas \isa{G}. Como el conjunto \isa{{\isacharbraceleft}{\isasymalpha}\isactrlsub {\isadigit{1}}{\isacharbraceright}\ {\isasymunion}\ S} es equivalente a \isa{{\isacharbraceleft}{\isasymalpha}\isactrlsub {\isadigit{1}}{\isacharcomma}\ {\isasymalpha}\isactrlsub {\isadigit{2}}{\isacharbraceright}\ {\isasymunion}\ S} ya
    que \isa{{\isasymalpha}\isactrlsub {\isadigit{1}}} y \isa{{\isasymalpha}\isactrlsub {\isadigit{2}}} son iguales, tenemos que este último pertenece a \isa{C} por la sexta 
    condición de la definición de propiedad de consistencia proposicional.

  \isa{{\isasymsqdot}\ Fórmula\ de\ tipo\ {\isasymnot}{\isacharparenleft}G\ {\isasymor}\ H{\isacharparenright}}: En este caso, sus componentes conjuntivas \isa{{\isasymalpha}\isactrlsub {\isadigit{1}}} y \isa{{\isasymalpha}\isactrlsub {\isadigit{2}}} son\\ 
    \isa{{\isasymnot}\ G} y \isa{{\isasymnot}\ H} respectivamente. Luego tenemos que \isa{{\isacharbraceleft}{\isasymalpha}\isactrlsub {\isadigit{1}}{\isacharcomma}\ {\isasymalpha}\isactrlsub {\isadigit{2}}{\isacharbraceright}\ {\isasymunion}\ S}  pertenece a \isa{C} por
    la octava condición de la definición de propiedad de consistencia proposicional.

  \isa{{\isasymsqdot}\ Fórmula\ de\ tipo\ {\isasymnot}{\isacharparenleft}G\ {\isasymlongrightarrow}\ H{\isacharparenright}}: En este caso, sus componentes conjuntivas \isa{{\isasymalpha}\isactrlsub {\isadigit{1}}} y \isa{{\isasymalpha}\isactrlsub {\isadigit{2}}} son \isa{G} y 
    \isa{{\isasymnot}\ H} respectivamente. Luego tenemos que \isa{{\isacharbraceleft}{\isasymalpha}\isactrlsub {\isadigit{1}}{\isacharcomma}\ {\isasymalpha}\isactrlsub {\isadigit{2}}{\isacharbraceright}\ {\isasymunion}\ S}  pertenece a \isa{C} por la novena 
    condición de la definición de propiedad de consistencia proposicional.

  Finalmente, el resultado correspondiente a las fórmulas de tipo \isa{{\isasymbeta}} se obtiene de las 
  condiciones cuarta, quinta, sexta y séptima de la definición de propiedad de consistencia 
  proposicional. Para probarlo, consideremos una fórmula cualquiera de tipo \isa{{\isasymbeta}} perteneciente
  al conjunto \isa{S} y cuyas componentes disyuntivas son \isa{{\isasymbeta}\isactrlsub {\isadigit{1}}} y \isa{{\isasymbeta}\isactrlsub {\isadigit{2}}}. Por simplificación, sabemos 
  que dicha fórmula es de la forma \isa{G\ {\isasymor}\ H}, \isa{G\ {\isasymlongrightarrow}\ H}, \isa{{\isasymnot}\ {\isacharparenleft}{\isasymnot}\ G{\isacharparenright}} o \isa{{\isasymnot}{\isacharparenleft}G\ {\isasymand}\ H{\isacharparenright}} para ciertas 
  fórmulas \isa{G} y \isa{H}. Deduzcamos que, en efecto, tenemos que o bien \isa{{\isacharbraceleft}{\isasymbeta}\isactrlsub {\isadigit{1}}{\isacharbraceright}\ {\isasymunion}\ S} está en \isa{C} o bien 
  \isa{{\isacharbraceleft}{\isasymbeta}\isactrlsub {\isadigit{2}}{\isacharbraceright}\ {\isasymunion}\ S} está en \isa{C}.

  \isa{{\isasymsqdot}\ Fórmula\ de\ tipo\ G\ {\isasymor}\ H}: En este caso, sus componentes disyuntivas \isa{{\isasymbeta}\isactrlsub {\isadigit{1}}} y \isa{{\isasymbeta}\isactrlsub {\isadigit{2}}} son \isa{G} y 
    \isa{H} respectivamente. Luego tenemos que o bien \isa{{\isacharbraceleft}{\isasymbeta}\isactrlsub {\isadigit{1}}{\isacharbraceright}\ {\isasymunion}\ S}  pertenece a \isa{C} o bien\\
    \isa{{\isacharbraceleft}{\isasymbeta}\isactrlsub {\isadigit{2}}{\isacharbraceright}\ {\isasymunion}\ S} pertenece a \isa{C} por la cuarta condición de la definición de propiedad de 
    consistencia proposicional.

  \isa{{\isasymsqdot}\ Fórmula\ de\ tipo\ G\ {\isasymlongrightarrow}\ H}: En este caso, sus componentes disyuntivas \isa{{\isasymbeta}\isactrlsub {\isadigit{1}}} y \isa{{\isasymbeta}\isactrlsub {\isadigit{2}}} son\\ 
    \isa{{\isasymnot}\ G} y \isa{H} respectivamente. Luego tenemos que o bien \isa{{\isacharbraceleft}{\isasymbeta}\isactrlsub {\isadigit{1}}{\isacharbraceright}\ {\isasymunion}\ S}  pertenece a \isa{C} o 
    bien\\ \isa{{\isacharbraceleft}{\isasymbeta}\isactrlsub {\isadigit{2}}{\isacharbraceright}\ {\isasymunion}\ S} pertenece a \isa{C} por la quinta condición de la definición de propiedad 
    de consistencia proposicional.

  \isa{{\isasymsqdot}\ Fórmula\ de\ tipo\ {\isasymnot}{\isacharparenleft}{\isasymnot}\ G{\isacharparenright}}: En este caso, sus componentes disyuntivas \isa{{\isasymbeta}\isactrlsub {\isadigit{1}}} y \isa{{\isasymbeta}\isactrlsub {\isadigit{2}}} son ambas 
    \isa{G}. Luego tenemos que, en particular, el conjunto \isa{{\isacharbraceleft}{\isasymbeta}\isactrlsub {\isadigit{1}}{\isacharbraceright}\ {\isasymunion}\ S} pertenece a \isa{C} por la 
    sexta condición de la definición de propiedad de consistencia proposicional. Por tanto, se 
    verifica que o bien \isa{{\isacharbraceleft}{\isasymbeta}\isactrlsub {\isadigit{1}}{\isacharbraceright}\ {\isasymunion}\ S} está en \isa{C} o bien \isa{{\isacharbraceleft}{\isasymbeta}\isactrlsub {\isadigit{2}}{\isacharbraceright}\ {\isasymunion}\ S} está en \isa{C}.

  \isa{{\isasymsqdot}\ Fórmula\ de\ tipo\ {\isasymnot}{\isacharparenleft}G\ {\isasymand}\ H{\isacharparenright}}: En este caso, sus componentes disyuntivas \isa{{\isasymbeta}\isactrlsub {\isadigit{1}}} y \isa{{\isasymbeta}\isactrlsub {\isadigit{2}}} son \\ 
    \isa{{\isasymnot}\ G} y \isa{{\isasymnot}\ H} respectivamente. Luego tenemos que o bien \isa{{\isacharbraceleft}{\isasymbeta}\isactrlsub {\isadigit{1}}{\isacharbraceright}\ {\isasymunion}\ S} pertenece a \isa{C} o 
    bien \isa{{\isacharbraceleft}{\isasymbeta}\isactrlsub {\isadigit{2}}{\isacharbraceright}\ {\isasymunion}\ S} pertenece a \isa{C} por la séptima condición de la definición de propiedad 
    de consistencia proposicional.

  De este modo, queda probada la primera implicación de la equivalencia. Veamos la prueba de la 
  implicación contraria.

\textbf{\isa{{\isadigit{2}}{\isacharparenright}\ {\isasymLongrightarrow}\ {\isadigit{1}}{\isacharparenright}}}

  Supongamos que, dada una colección de conjuntos de fórmulas proposicionales \isa{C}, para cualquier
  conjunto \isa{S} de la colección se verifica:
  \begin{itemize}
    \item \isa{{\isasymbottom}} no pertenece a \isa{S}.
    \item Dada \isa{p} una fórmula atómica cualquiera, no se tiene 
    simultáneamente que\\ \isa{p\ {\isasymin}\ S} y \isa{{\isasymnot}\ p\ {\isasymin}\ S}.
    \item Para toda fórmula de tipo \isa{{\isasymalpha}} con componentes \isa{{\isasymalpha}\isactrlsub {\isadigit{1}}} y \isa{{\isasymalpha}\isactrlsub {\isadigit{2}}} tal que \isa{{\isasymalpha}}
    pertenece a \isa{S}, se tiene que \isa{{\isacharbraceleft}{\isasymalpha}\isactrlsub {\isadigit{1}}{\isacharcomma}{\isasymalpha}\isactrlsub {\isadigit{2}}{\isacharbraceright}\ {\isasymunion}\ S} pertenece a \isa{C}.
    \item Para toda fórmula de tipo \isa{{\isasymbeta}} con componentes \isa{{\isasymbeta}\isactrlsub {\isadigit{1}}} y \isa{{\isasymbeta}\isactrlsub {\isadigit{2}}} tal que \isa{{\isasymbeta}}
    pertenece a \isa{S}, se tiene que o bien \isa{{\isacharbraceleft}{\isasymbeta}\isactrlsub {\isadigit{1}}{\isacharbraceright}\ {\isasymunion}\ S} pertenece a \isa{C} o 
    bien \isa{{\isacharbraceleft}{\isasymbeta}\isactrlsub {\isadigit{2}}{\isacharbraceright}\ {\isasymunion}\ S} pertenece a \isa{C}.
  \end{itemize}

  Probemos que \isa{C} verifica la propiedad de consistencia proposicional. Por la definición
  de la propiedad basta probar que, dado un conjunto cualquiera \isa{S} perteneciente a \isa{C}, se
  verifican las siguientes condiciones:
  \begin{enumerate}
    \item \isa{{\isasymbottom}\ {\isasymnotin}\ S}.
    \item Dada \isa{p} una fórmula atómica cualquiera, no se tiene 
      simultáneamente que\\ \isa{p\ {\isasymin}\ S} y \isa{{\isasymnot}\ p\ {\isasymin}\ S}.
    \item Si \isa{G\ {\isasymand}\ H\ {\isasymin}\ S}, entonces el conjunto \isa{{\isacharbraceleft}G{\isacharcomma}H{\isacharbraceright}\ {\isasymunion}\ S} pertenece a \isa{C}.
    \item Si \isa{G\ {\isasymor}\ H\ {\isasymin}\ S}, entonces o bien el conjunto \isa{{\isacharbraceleft}G{\isacharbraceright}\ {\isasymunion}\ S} pertenece a \isa{C}, o bien el conjunto 
      \isa{{\isacharbraceleft}H{\isacharbraceright}\ {\isasymunion}\ S} pertenece a \isa{C}.
    \item Si \isa{G\ {\isasymrightarrow}\ H\ {\isasymin}\ S}, entonces o bien el conjunto \isa{{\isacharbraceleft}{\isasymnot}\ G{\isacharbraceright}\ {\isasymunion}\ S} pertenece a \isa{C}, o bien el 
      conjunto \isa{{\isacharbraceleft}H{\isacharbraceright}\ {\isasymunion}\ S} pertenece a \isa{C}.
    \item Si \isa{{\isasymnot}{\isacharparenleft}{\isasymnot}\ G{\isacharparenright}\ {\isasymin}\ S}, entonces el conjunto \isa{{\isacharbraceleft}G{\isacharbraceright}\ {\isasymunion}\ S} pertenece a \isa{C}.
    \item Si \isa{{\isasymnot}{\isacharparenleft}G\ {\isasymand}\ H{\isacharparenright}\ {\isasymin}\ S}, entonces o bien el conjunto \isa{{\isacharbraceleft}{\isasymnot}\ G{\isacharbraceright}\ {\isasymunion}\ S} pertenece a \isa{C}, o bien el 
      conjunto \isa{{\isacharbraceleft}{\isasymnot}\ H{\isacharbraceright}\ {\isasymunion}\ S} pertenece a \isa{C}.
    \item Si \isa{{\isasymnot}{\isacharparenleft}G\ {\isasymor}\ H{\isacharparenright}\ {\isasymin}\ S}, entonces el conjunto \isa{{\isacharbraceleft}{\isasymnot}\ G{\isacharcomma}\ {\isasymnot}\ H{\isacharbraceright}\ {\isasymunion}\ S} pertenece a \isa{C}.
    \item Si \isa{{\isasymnot}{\isacharparenleft}G\ {\isasymrightarrow}\ H{\isacharparenright}\ {\isasymin}\ S}, entonces el conjunto \isa{{\isacharbraceleft}G{\isacharcomma}\ {\isasymnot}\ H{\isacharbraceright}\ {\isasymunion}\ S} pertenece a \isa{C}.
  \end{enumerate}

  En primer lugar, se observa que por hipótesis se cumplen las dos primeras condiciones de
  la definición.

  Por otra parte, vamos a deducir las condiciones tercera, sexta, octava y novena de la
  definición de la propiedad de consistencia proposicional a partir de la hipótesis sobre las 
  fórmulas de tipo \isa{{\isasymalpha}}.
  \begin{enumerate}
    \item[\isa{{\isadigit{3}}{\isacharparenright}}:] Supongamos que la fórmula \isa{G\ {\isasymand}\ H} pertenece a \isa{S} para fórmulas \isa{G} y \isa{H}
    cualesquiera. Observemos que se trata de una fórmula de tipo \isa{{\isasymalpha}} de componentes conjuntivas
    \isa{G} y \isa{H}. Luego, por hipótesis, tenemos que \isa{{\isacharbraceleft}G{\isacharcomma}\ H{\isacharbraceright}\ {\isasymunion}\ S} pertenece a \isa{C}.
    \item[\isa{{\isadigit{6}}{\isacharparenright}}:] Supongamos que la fórmula \isa{{\isasymnot}{\isacharparenleft}{\isasymnot}\ G{\isacharparenright}} pertenece a \isa{S} para la fórmula \isa{G} 
    cualquiera. Observemos que se trata de una fórmula de tipo \isa{{\isasymalpha}} cuyas componentes conjuntivas 
    son ambas la fórmula \isa{G}. Por hipótesis, tenemos que el conjunto \isa{{\isacharbraceleft}G{\isacharcomma}G{\isacharbraceright}\ {\isasymunion}\ S} pertence a \isa{C}
    y, puesto que dicho conjunto es equivalente a \isa{{\isacharbraceleft}G{\isacharbraceright}\ {\isasymunion}\ S}, tenemos el resultado.
    \item[\isa{{\isadigit{8}}{\isacharparenright}}:] Supongamos que la fórmula \isa{{\isasymnot}{\isacharparenleft}G\ {\isasymor}\ H{\isacharparenright}} pertenece a \isa{S} para fórmulas \isa{G} y \isa{H}
    cualesquiera. Observemos que se trata de una fórmula de tipo \isa{{\isasymalpha}} de componentes conjuntivas
    \isa{{\isasymnot}\ G} y \isa{{\isasymnot}\ H}. Luego, por hipótesis, tenemos que \isa{{\isacharbraceleft}{\isasymnot}\ G{\isacharcomma}\ {\isasymnot}\ H{\isacharbraceright}\ {\isasymunion}\ S} pertenece a \isa{C}.
    \item[\isa{{\isadigit{9}}{\isacharparenright}}:] Supongamos que la fórmula \isa{{\isasymnot}{\isacharparenleft}G\ {\isasymlongrightarrow}\ H{\isacharparenright}} pertenece a \isa{S} para fórmulas \isa{G} y \isa{H}
    cualesquiera. Observemos que se trata de una fórmula de tipo \isa{{\isasymalpha}} de componentes conjuntivas
    \isa{G} y \isa{{\isasymnot}\ H}. Luego, por hipótesis, tenemos que \isa{{\isacharbraceleft}G{\isacharcomma}\ {\isasymnot}\ H{\isacharbraceright}\ {\isasymunion}\ S} pertenece a \isa{C}.
  \end{enumerate} 

  Finalmente, deduzcamos el resto de condiciones de la definición de propiedad de consistencia
  proposicional a partir de la hipótesis referente a las fórmulas de tipo \isa{{\isasymbeta}}.
  \begin{enumerate}
    \item[\isa{{\isadigit{4}}{\isacharparenright}}:] Supongamos que la fórmula \isa{G\ {\isasymor}\ H} pertenece a \isa{S} para fórmulas \isa{G} y \isa{H}
    cualesquiera. Observemos que se trata de una fórmula de tipo \isa{{\isasymbeta}} de componentes disyuntivas
    \isa{G} y \isa{H}. Luego, por hipótesis, tenemos que o bien \isa{{\isacharbraceleft}G{\isacharbraceright}\ {\isasymunion}\ S} pertenece a \isa{C} o bien\\
    \isa{{\isacharbraceleft}H{\isacharbraceright}\ {\isasymunion}\ S} pertenece a \isa{C}.
    \item[\isa{{\isadigit{5}}{\isacharparenright}}:] Supongamos que la fórmula \isa{G\ {\isasymlongrightarrow}\ H} pertenece a \isa{S} para fórmulas \isa{G} y \isa{H}
    cualesquiera. Observemos que se trata de una fórmula de tipo \isa{{\isasymbeta}} de componentes disyuntivas
    \isa{{\isasymnot}\ G} y \isa{H}. Luego, por hipótesis, tenemos que o bien \isa{{\isacharbraceleft}{\isasymnot}\ G{\isacharbraceright}\ {\isasymunion}\ S} pertenece a \isa{C} o
    bien \isa{{\isacharbraceleft}H{\isacharbraceright}\ {\isasymunion}\ S} pertenece a \isa{C}.
    \item[\isa{{\isadigit{7}}{\isacharparenright}}:] Supongamos que la fórmula \isa{{\isasymnot}{\isacharparenleft}G\ {\isasymand}\ H{\isacharparenright}} pertenece a \isa{S} para fórmulas \isa{G} y \isa{H}
    cualesquiera. Observemos que se trata de una fórmula de tipo \isa{{\isasymbeta}} de componentes disyuntivas
    \isa{{\isasymnot}\ G} y \isa{{\isasymnot}\ H}. Luego, por hipótesis, tenemos que o bien \isa{{\isacharbraceleft}{\isasymnot}\ G{\isacharbraceright}\ {\isasymunion}\ S} pertenece a \isa{C} o
    bien \isa{{\isacharbraceleft}{\isasymnot}\ H{\isacharbraceright}\ {\isasymunion}\ S} pertenece \isa{C}.
  \end{enumerate} 

  De este modo, hemos probado a partir de la hipótesis todas las condiciones que garantizan que la
  colección \isa{C} cumple la propiedad de consistencia proposicional. Por lo tanto, queda demostrado el
  resultado.
\end{demostracion}

  Para probar este resultado de manera detallada en Isabelle vamos a demostrar cada una de las 
  implicaciones de la equivalencia por separado. La primera implicación del lema se basa en dos 
  lemas auxiliares. El primero de ellos deduce la condición de \isa{{\isadigit{2}}{\isacharparenright}} sobre fórmulas de tipo \isa{{\isasymalpha}} a 
  partir de las condiciones tercera, sexta, octava y novena de la definición de propiedad de 
  consistencia proposicional. Su demostración detallada en Isabelle se muestra a continuación.%
\end{isamarkuptext}\isamarkuptrue%
\isacommand{lemma}\isamarkupfalse%
\ pcp{\isacharunderscore}alt{\isadigit{1}}Con{\isacharcolon}\isanewline
\ \ \isakeyword{assumes}\ {\isachardoublequoteopen}{\isacharparenleft}{\isasymforall}G\ H{\isachardot}\ G\ \isactrlbold {\isasymand}\ H\ {\isasymin}\ S\ {\isasymlongrightarrow}\ {\isacharbraceleft}G{\isacharcomma}H{\isacharbraceright}\ {\isasymunion}\ S\ {\isasymin}\ C{\isacharparenright}\isanewline
\ \ {\isasymand}\ {\isacharparenleft}{\isasymforall}G{\isachardot}\ \isactrlbold {\isasymnot}\ {\isacharparenleft}\isactrlbold {\isasymnot}G{\isacharparenright}\ {\isasymin}\ S\ {\isasymlongrightarrow}\ {\isacharbraceleft}G{\isacharbraceright}\ {\isasymunion}\ S\ {\isasymin}\ C{\isacharparenright}\isanewline
\ \ {\isasymand}\ {\isacharparenleft}{\isasymforall}G\ H{\isachardot}\ \isactrlbold {\isasymnot}{\isacharparenleft}G\ \isactrlbold {\isasymor}\ H{\isacharparenright}\ {\isasymin}\ S\ {\isasymlongrightarrow}\ {\isacharbraceleft}\isactrlbold {\isasymnot}\ G{\isacharcomma}\ \isactrlbold {\isasymnot}\ H{\isacharbraceright}\ {\isasymunion}\ S\ {\isasymin}\ C{\isacharparenright}\isanewline
\ \ {\isasymand}\ {\isacharparenleft}{\isasymforall}G\ H{\isachardot}\ \isactrlbold {\isasymnot}{\isacharparenleft}G\ \isactrlbold {\isasymrightarrow}\ H{\isacharparenright}\ {\isasymin}\ S\ {\isasymlongrightarrow}\ {\isacharbraceleft}G{\isacharcomma}\isactrlbold {\isasymnot}\ H{\isacharbraceright}\ {\isasymunion}\ S\ {\isasymin}\ C{\isacharparenright}{\isachardoublequoteclose}\isanewline
\ \ \isakeyword{shows}\ {\isachardoublequoteopen}{\isasymforall}F\ G\ H{\isachardot}\ Con\ F\ G\ H\ {\isasymlongrightarrow}\ F\ {\isasymin}\ S\ {\isasymlongrightarrow}\ {\isacharbraceleft}G{\isacharcomma}H{\isacharbraceright}\ {\isasymunion}\ S\ {\isasymin}\ C{\isachardoublequoteclose}\isanewline
%
\isadelimproof
%
\endisadelimproof
%
\isatagproof
\isacommand{proof}\isamarkupfalse%
\ {\isacharminus}\isanewline
\ \ \isacommand{have}\isamarkupfalse%
\ C{\isadigit{1}}{\isacharcolon}{\isachardoublequoteopen}{\isasymforall}G\ H{\isachardot}\ G\ \isactrlbold {\isasymand}\ H\ {\isasymin}\ S\ {\isasymlongrightarrow}\ {\isacharbraceleft}G{\isacharcomma}H{\isacharbraceright}\ {\isasymunion}\ S\ {\isasymin}\ C{\isachardoublequoteclose}\isanewline
\ \ \ \ \isacommand{using}\isamarkupfalse%
\ assms\ \isacommand{by}\isamarkupfalse%
\ {\isacharparenleft}rule\ conjunct{\isadigit{1}}{\isacharparenright}\isanewline
\ \ \isacommand{have}\isamarkupfalse%
\ C{\isadigit{2}}{\isacharcolon}{\isachardoublequoteopen}{\isasymforall}G{\isachardot}\ \isactrlbold {\isasymnot}\ {\isacharparenleft}\isactrlbold {\isasymnot}G{\isacharparenright}\ {\isasymin}\ S\ {\isasymlongrightarrow}\ {\isacharbraceleft}G{\isacharbraceright}\ {\isasymunion}\ S\ {\isasymin}\ C{\isachardoublequoteclose}\isanewline
\ \ \ \ \isacommand{using}\isamarkupfalse%
\ assms\ \isacommand{by}\isamarkupfalse%
\ {\isacharparenleft}iprover\ elim{\isacharcolon}\ conjunct{\isadigit{2}}\ conjunct{\isadigit{1}}{\isacharparenright}\isanewline
\ \ \isacommand{have}\isamarkupfalse%
\ C{\isadigit{3}}{\isacharcolon}{\isachardoublequoteopen}{\isasymforall}G\ H{\isachardot}\ \isactrlbold {\isasymnot}{\isacharparenleft}G\ \isactrlbold {\isasymor}\ H{\isacharparenright}\ {\isasymin}\ S\ {\isasymlongrightarrow}\ {\isacharbraceleft}\isactrlbold {\isasymnot}\ G{\isacharcomma}\ \isactrlbold {\isasymnot}\ H{\isacharbraceright}\ {\isasymunion}\ S\ {\isasymin}\ C{\isachardoublequoteclose}\isanewline
\ \ \ \ \isacommand{using}\isamarkupfalse%
\ assms\ \isacommand{by}\isamarkupfalse%
\ {\isacharparenleft}iprover\ elim{\isacharcolon}\ conjunct{\isadigit{2}}\ conjunct{\isadigit{1}}{\isacharparenright}\isanewline
\ \ \isacommand{have}\isamarkupfalse%
\ C{\isadigit{4}}{\isacharcolon}{\isachardoublequoteopen}{\isasymforall}G\ H{\isachardot}\ \isactrlbold {\isasymnot}{\isacharparenleft}G\ \isactrlbold {\isasymrightarrow}\ H{\isacharparenright}\ {\isasymin}\ S\ {\isasymlongrightarrow}\ {\isacharbraceleft}G{\isacharcomma}\isactrlbold {\isasymnot}\ H{\isacharbraceright}\ {\isasymunion}\ S\ {\isasymin}\ C{\isachardoublequoteclose}\isanewline
\ \ \ \ \isacommand{using}\isamarkupfalse%
\ assms\ \isacommand{by}\isamarkupfalse%
\ {\isacharparenleft}iprover\ elim{\isacharcolon}\ conjunct{\isadigit{2}}{\isacharparenright}\ \isanewline
\ \ \isacommand{show}\isamarkupfalse%
\ {\isachardoublequoteopen}{\isasymforall}F\ G\ H{\isachardot}\ Con\ F\ G\ H\ {\isasymlongrightarrow}\ F\ {\isasymin}\ S\ {\isasymlongrightarrow}\ {\isacharbraceleft}G{\isacharcomma}H{\isacharbraceright}\ {\isasymunion}\ S\ {\isasymin}\ C{\isachardoublequoteclose}\isanewline
\ \ \isacommand{proof}\isamarkupfalse%
\ {\isacharparenleft}rule\ allI{\isacharparenright}{\isacharplus}\isanewline
\ \ \ \ \isacommand{fix}\isamarkupfalse%
\ F\ G\ H\isanewline
\ \ \ \ \isacommand{show}\isamarkupfalse%
\ {\isachardoublequoteopen}Con\ F\ G\ H\ {\isasymlongrightarrow}\ F\ {\isasymin}\ S\ {\isasymlongrightarrow}\ {\isacharbraceleft}G{\isacharcomma}H{\isacharbraceright}\ {\isasymunion}\ S\ {\isasymin}\ C{\isachardoublequoteclose}\isanewline
\ \ \ \ \isacommand{proof}\isamarkupfalse%
\ {\isacharparenleft}rule\ impI{\isacharparenright}\isanewline
\ \ \ \ \ \ \isacommand{assume}\isamarkupfalse%
\ {\isachardoublequoteopen}Con\ F\ G\ H{\isachardoublequoteclose}\isanewline
\ \ \ \ \ \ \isacommand{then}\isamarkupfalse%
\ \isacommand{have}\isamarkupfalse%
\ {\isachardoublequoteopen}F\ {\isacharequal}\ G\ \isactrlbold {\isasymand}\ H\ {\isasymor}\ \isanewline
\ \ \ \ \ \ \ \ \ \ \ \ \ \ \ \ {\isacharparenleft}{\isacharparenleft}{\isasymexists}G{\isadigit{1}}\ H{\isadigit{1}}{\isachardot}\ F\ {\isacharequal}\ \isactrlbold {\isasymnot}\ {\isacharparenleft}G{\isadigit{1}}\ \isactrlbold {\isasymor}\ H{\isadigit{1}}{\isacharparenright}\ {\isasymand}\ G\ {\isacharequal}\ \isactrlbold {\isasymnot}\ G{\isadigit{1}}\ {\isasymand}\ H\ {\isacharequal}\ \isactrlbold {\isasymnot}\ H{\isadigit{1}}{\isacharparenright}\ {\isasymor}\ \isanewline
\ \ \ \ \ \ \ \ \ \ \ \ \ \ \ \ {\isacharparenleft}{\isasymexists}H{\isadigit{2}}{\isachardot}\ F\ {\isacharequal}\ \isactrlbold {\isasymnot}\ {\isacharparenleft}G\ \isactrlbold {\isasymrightarrow}\ H{\isadigit{2}}{\isacharparenright}\ {\isasymand}\ H\ {\isacharequal}\ \isactrlbold {\isasymnot}\ H{\isadigit{2}}{\isacharparenright}\ {\isasymor}\ \isanewline
\ \ \ \ \ \ \ \ \ \ \ \ \ \ \ \ F\ {\isacharequal}\ \isactrlbold {\isasymnot}\ {\isacharparenleft}\isactrlbold {\isasymnot}\ G{\isacharparenright}\ {\isasymand}\ H\ {\isacharequal}\ G{\isacharparenright}{\isachardoublequoteclose}\isanewline
\ \ \ \ \ \ \ \ \isacommand{by}\isamarkupfalse%
\ {\isacharparenleft}simp\ only{\isacharcolon}\ con{\isacharunderscore}dis{\isacharunderscore}simps{\isacharparenleft}{\isadigit{1}}{\isacharparenright}{\isacharparenright}\isanewline
\ \ \ \ \ \ \isacommand{thus}\isamarkupfalse%
\ {\isachardoublequoteopen}F\ {\isasymin}\ S\ {\isasymlongrightarrow}\ {\isacharbraceleft}G{\isacharcomma}H{\isacharbraceright}\ {\isasymunion}\ S\ {\isasymin}\ C{\isachardoublequoteclose}\isanewline
\ \ \ \ \ \ \isacommand{proof}\isamarkupfalse%
\ {\isacharparenleft}rule\ disjE{\isacharparenright}\isanewline
\ \ \ \ \ \ \ \ \isacommand{assume}\isamarkupfalse%
\ {\isachardoublequoteopen}F\ {\isacharequal}\ G\ \isactrlbold {\isasymand}\ H{\isachardoublequoteclose}\isanewline
\ \ \ \ \ \ \ \ \isacommand{show}\isamarkupfalse%
\ {\isachardoublequoteopen}F\ {\isasymin}\ S\ {\isasymlongrightarrow}\ {\isacharbraceleft}G{\isacharcomma}H{\isacharbraceright}\ {\isasymunion}\ S\ {\isasymin}\ C{\isachardoublequoteclose}\isanewline
\ \ \ \ \ \ \ \ \ \ \isacommand{using}\isamarkupfalse%
\ C{\isadigit{1}}\ {\isacartoucheopen}F\ {\isacharequal}\ G\ \isactrlbold {\isasymand}\ H{\isacartoucheclose}\ \isacommand{by}\isamarkupfalse%
\ {\isacharparenleft}iprover\ elim{\isacharcolon}\ allE{\isacharparenright}\isanewline
\ \ \ \ \ \ \isacommand{next}\isamarkupfalse%
\isanewline
\ \ \ \ \ \ \ \ \isacommand{assume}\isamarkupfalse%
\ {\isachardoublequoteopen}{\isacharparenleft}{\isasymexists}G{\isadigit{1}}\ H{\isadigit{1}}{\isachardot}\ F\ {\isacharequal}\ \isactrlbold {\isasymnot}\ {\isacharparenleft}G{\isadigit{1}}\ \isactrlbold {\isasymor}\ H{\isadigit{1}}{\isacharparenright}\ {\isasymand}\ G\ {\isacharequal}\ \isactrlbold {\isasymnot}\ G{\isadigit{1}}\ {\isasymand}\ H\ {\isacharequal}\ \isactrlbold {\isasymnot}\ H{\isadigit{1}}{\isacharparenright}\ {\isasymor}\ \isanewline
\ \ \ \ \ \ \ \ \ \ \ \ \ \ \ \ {\isacharparenleft}{\isasymexists}H{\isadigit{2}}{\isachardot}\ F\ {\isacharequal}\ \isactrlbold {\isasymnot}\ {\isacharparenleft}G\ \isactrlbold {\isasymrightarrow}\ H{\isadigit{2}}{\isacharparenright}\ {\isasymand}\ H\ {\isacharequal}\ \isactrlbold {\isasymnot}\ H{\isadigit{2}}{\isacharparenright}\ {\isasymor}\ \isanewline
\ \ \ \ \ \ \ \ \ \ \ \ \ \ \ \ F\ {\isacharequal}\ \isactrlbold {\isasymnot}\ {\isacharparenleft}\isactrlbold {\isasymnot}\ G{\isacharparenright}\ {\isasymand}\ H\ {\isacharequal}\ G{\isachardoublequoteclose}\isanewline
\ \ \ \ \ \ \ \ \isacommand{thus}\isamarkupfalse%
\ {\isachardoublequoteopen}F\ {\isasymin}\ S\ {\isasymlongrightarrow}\ {\isacharbraceleft}G{\isacharcomma}H{\isacharbraceright}\ {\isasymunion}\ S\ {\isasymin}\ C{\isachardoublequoteclose}\isanewline
\ \ \ \ \ \ \ \ \isacommand{proof}\isamarkupfalse%
\ {\isacharparenleft}rule\ disjE{\isacharparenright}\isanewline
\ \ \ \ \ \ \ \ \ \ \isacommand{assume}\isamarkupfalse%
\ E{\isadigit{1}}{\isacharcolon}{\isachardoublequoteopen}{\isasymexists}G{\isadigit{1}}\ H{\isadigit{1}}{\isachardot}\ F\ {\isacharequal}\ \isactrlbold {\isasymnot}\ {\isacharparenleft}G{\isadigit{1}}\ \isactrlbold {\isasymor}\ H{\isadigit{1}}{\isacharparenright}\ {\isasymand}\ G\ {\isacharequal}\ \isactrlbold {\isasymnot}\ G{\isadigit{1}}\ {\isasymand}\ H\ {\isacharequal}\ \isactrlbold {\isasymnot}\ H{\isadigit{1}}{\isachardoublequoteclose}\isanewline
\ \ \ \ \ \ \ \ \ \ \isacommand{obtain}\isamarkupfalse%
\ G{\isadigit{1}}\ H{\isadigit{1}}\ \isakeyword{where}\ A{\isadigit{1}}{\isacharcolon}{\isachardoublequoteopen}F\ {\isacharequal}\ \isactrlbold {\isasymnot}\ {\isacharparenleft}G{\isadigit{1}}\ \isactrlbold {\isasymor}\ H{\isadigit{1}}{\isacharparenright}\ {\isasymand}\ G\ {\isacharequal}\ \isactrlbold {\isasymnot}\ G{\isadigit{1}}\ {\isasymand}\ H\ {\isacharequal}\ \isactrlbold {\isasymnot}\ H{\isadigit{1}}{\isachardoublequoteclose}\isanewline
\ \ \ \ \ \ \ \ \ \ \ \ \isacommand{using}\isamarkupfalse%
\ E{\isadigit{1}}\ \isacommand{by}\isamarkupfalse%
\ {\isacharparenleft}iprover\ elim{\isacharcolon}\ exE{\isacharparenright}\isanewline
\ \ \ \ \ \ \ \ \ \ \isacommand{have}\isamarkupfalse%
\ {\isachardoublequoteopen}F\ {\isacharequal}\ \isactrlbold {\isasymnot}\ {\isacharparenleft}G{\isadigit{1}}\ \isactrlbold {\isasymor}\ H{\isadigit{1}}{\isacharparenright}{\isachardoublequoteclose}\isanewline
\ \ \ \ \ \ \ \ \ \ \ \ \isacommand{using}\isamarkupfalse%
\ A{\isadigit{1}}\ \isacommand{by}\isamarkupfalse%
\ {\isacharparenleft}rule\ conjunct{\isadigit{1}}{\isacharparenright}\isanewline
\ \ \ \ \ \ \ \ \ \ \isacommand{have}\isamarkupfalse%
\ {\isachardoublequoteopen}G\ {\isacharequal}\ \isactrlbold {\isasymnot}\ G{\isadigit{1}}{\isachardoublequoteclose}\isanewline
\ \ \ \ \ \ \ \ \ \ \ \ \isacommand{using}\isamarkupfalse%
\ A{\isadigit{1}}\ \isacommand{by}\isamarkupfalse%
\ {\isacharparenleft}iprover\ elim{\isacharcolon}\ conjunct{\isadigit{2}}\ conjunct{\isadigit{1}}{\isacharparenright}\isanewline
\ \ \ \ \ \ \ \ \ \ \isacommand{have}\isamarkupfalse%
\ {\isachardoublequoteopen}H\ {\isacharequal}\ \isactrlbold {\isasymnot}\ H{\isadigit{1}}{\isachardoublequoteclose}\isanewline
\ \ \ \ \ \ \ \ \ \ \ \ \isacommand{using}\isamarkupfalse%
\ A{\isadigit{1}}\ \isacommand{by}\isamarkupfalse%
\ {\isacharparenleft}iprover\ elim{\isacharcolon}\ conjunct{\isadigit{2}}{\isacharparenright}\isanewline
\ \ \ \ \ \ \ \ \ \ \isacommand{show}\isamarkupfalse%
\ {\isachardoublequoteopen}F\ {\isasymin}\ S\ {\isasymlongrightarrow}\ {\isacharbraceleft}G{\isacharcomma}H{\isacharbraceright}\ {\isasymunion}\ S\ {\isasymin}\ C{\isachardoublequoteclose}\isanewline
\ \ \ \ \ \ \ \ \ \ \ \ \isacommand{using}\isamarkupfalse%
\ C{\isadigit{3}}\ {\isacartoucheopen}F\ {\isacharequal}\ \isactrlbold {\isasymnot}\ {\isacharparenleft}G{\isadigit{1}}\ \isactrlbold {\isasymor}\ H{\isadigit{1}}{\isacharparenright}{\isacartoucheclose}\ {\isacartoucheopen}G\ {\isacharequal}\ \isactrlbold {\isasymnot}\ G{\isadigit{1}}{\isacartoucheclose}\ {\isacartoucheopen}H\ {\isacharequal}\ \isactrlbold {\isasymnot}\ H{\isadigit{1}}{\isacartoucheclose}\ \isacommand{by}\isamarkupfalse%
\ {\isacharparenleft}iprover\ elim{\isacharcolon}\ allE{\isacharparenright}\isanewline
\ \ \ \ \ \ \ \ \isacommand{next}\isamarkupfalse%
\isanewline
\ \ \ \ \ \ \ \ \ \ \isacommand{assume}\isamarkupfalse%
\ {\isachardoublequoteopen}{\isacharparenleft}{\isasymexists}H{\isadigit{2}}{\isachardot}\ F\ {\isacharequal}\ \isactrlbold {\isasymnot}\ {\isacharparenleft}G\ \isactrlbold {\isasymrightarrow}\ H{\isadigit{2}}{\isacharparenright}\ {\isasymand}\ H\ {\isacharequal}\ \isactrlbold {\isasymnot}\ H{\isadigit{2}}{\isacharparenright}\ {\isasymor}\ \isanewline
\ \ \ \ \ \ \ \ \ \ \ \ \ \ \ \ \ \ \ F\ {\isacharequal}\ \isactrlbold {\isasymnot}\ {\isacharparenleft}\isactrlbold {\isasymnot}\ G{\isacharparenright}\ {\isasymand}\ H\ {\isacharequal}\ G{\isachardoublequoteclose}\ \isanewline
\ \ \ \ \ \ \ \ \ \ \isacommand{thus}\isamarkupfalse%
\ {\isachardoublequoteopen}F\ {\isasymin}\ S\ {\isasymlongrightarrow}\ {\isacharbraceleft}G{\isacharcomma}H{\isacharbraceright}\ {\isasymunion}\ S\ {\isasymin}\ C{\isachardoublequoteclose}\isanewline
\ \ \ \ \ \ \ \ \ \ \isacommand{proof}\isamarkupfalse%
\ {\isacharparenleft}rule\ disjE{\isacharparenright}\isanewline
\ \ \ \ \ \ \ \ \ \ \ \ \isacommand{assume}\isamarkupfalse%
\ E{\isadigit{2}}{\isacharcolon}{\isachardoublequoteopen}{\isasymexists}H{\isadigit{2}}{\isachardot}\ F\ {\isacharequal}\ \isactrlbold {\isasymnot}\ {\isacharparenleft}G\ \isactrlbold {\isasymrightarrow}\ H{\isadigit{2}}{\isacharparenright}\ {\isasymand}\ H\ {\isacharequal}\ \isactrlbold {\isasymnot}\ H{\isadigit{2}}{\isachardoublequoteclose}\isanewline
\ \ \ \ \ \ \ \ \ \ \ \ \isacommand{obtain}\isamarkupfalse%
\ H{\isadigit{2}}\ \isakeyword{where}\ A{\isadigit{2}}{\isacharcolon}{\isachardoublequoteopen}F\ {\isacharequal}\ \isactrlbold {\isasymnot}\ {\isacharparenleft}G\ \isactrlbold {\isasymrightarrow}\ H{\isadigit{2}}{\isacharparenright}\ {\isasymand}\ H\ {\isacharequal}\ \isactrlbold {\isasymnot}\ H{\isadigit{2}}{\isachardoublequoteclose}\isanewline
\ \ \ \ \ \ \ \ \ \ \ \ \ \ \isacommand{using}\isamarkupfalse%
\ E{\isadigit{2}}\ \isacommand{by}\isamarkupfalse%
\ {\isacharparenleft}rule\ exE{\isacharparenright}\isanewline
\ \ \ \ \ \ \ \ \ \ \ \ \isacommand{have}\isamarkupfalse%
\ {\isachardoublequoteopen}F\ {\isacharequal}\ \isactrlbold {\isasymnot}\ {\isacharparenleft}G\ \isactrlbold {\isasymrightarrow}\ H{\isadigit{2}}{\isacharparenright}{\isachardoublequoteclose}\isanewline
\ \ \ \ \ \ \ \ \ \ \ \ \ \ \isacommand{using}\isamarkupfalse%
\ A{\isadigit{2}}\ \isacommand{by}\isamarkupfalse%
\ {\isacharparenleft}rule\ conjunct{\isadigit{1}}{\isacharparenright}\isanewline
\ \ \ \ \ \ \ \ \ \ \ \ \isacommand{have}\isamarkupfalse%
\ {\isachardoublequoteopen}H\ {\isacharequal}\ \isactrlbold {\isasymnot}\ H{\isadigit{2}}{\isachardoublequoteclose}\isanewline
\ \ \ \ \ \ \ \ \ \ \ \ \ \ \isacommand{using}\isamarkupfalse%
\ A{\isadigit{2}}\ \isacommand{by}\isamarkupfalse%
\ {\isacharparenleft}rule\ conjunct{\isadigit{2}}{\isacharparenright}\isanewline
\ \ \ \ \ \ \ \ \ \ \ \ \isacommand{show}\isamarkupfalse%
\ {\isachardoublequoteopen}F\ {\isasymin}\ S\ {\isasymlongrightarrow}\ {\isacharbraceleft}G{\isacharcomma}H{\isacharbraceright}\ {\isasymunion}\ S\ {\isasymin}\ C{\isachardoublequoteclose}\isanewline
\ \ \ \ \ \ \ \ \ \ \ \ \ \ \isacommand{using}\isamarkupfalse%
\ C{\isadigit{4}}\ {\isacartoucheopen}F\ {\isacharequal}\ \isactrlbold {\isasymnot}\ {\isacharparenleft}G\ \isactrlbold {\isasymrightarrow}\ H{\isadigit{2}}{\isacharparenright}{\isacartoucheclose}\ {\isacartoucheopen}H\ {\isacharequal}\ \isactrlbold {\isasymnot}\ H{\isadigit{2}}{\isacartoucheclose}\ \isacommand{by}\isamarkupfalse%
\ {\isacharparenleft}iprover\ elim{\isacharcolon}\ allE{\isacharparenright}\isanewline
\ \ \ \ \ \ \ \ \ \ \isacommand{next}\isamarkupfalse%
\isanewline
\ \ \ \ \ \ \ \ \ \ \ \ \isacommand{assume}\isamarkupfalse%
\ A{\isadigit{3}}{\isacharcolon}{\isachardoublequoteopen}F\ {\isacharequal}\ \isactrlbold {\isasymnot}{\isacharparenleft}\isactrlbold {\isasymnot}\ G{\isacharparenright}\ {\isasymand}\ H\ {\isacharequal}\ G{\isachardoublequoteclose}\isanewline
\ \ \ \ \ \ \ \ \ \ \ \ \isacommand{then}\isamarkupfalse%
\ \isacommand{have}\isamarkupfalse%
\ {\isachardoublequoteopen}F\ {\isacharequal}\ \isactrlbold {\isasymnot}{\isacharparenleft}\isactrlbold {\isasymnot}\ G{\isacharparenright}{\isachardoublequoteclose}\isanewline
\ \ \ \ \ \ \ \ \ \ \ \ \ \ \isacommand{by}\isamarkupfalse%
\ {\isacharparenleft}rule\ conjunct{\isadigit{1}}{\isacharparenright}\isanewline
\ \ \ \ \ \ \ \ \ \ \ \ \isacommand{have}\isamarkupfalse%
\ {\isachardoublequoteopen}H\ {\isacharequal}\ G{\isachardoublequoteclose}\isanewline
\ \ \ \ \ \ \ \ \ \ \ \ \ \ \isacommand{using}\isamarkupfalse%
\ A{\isadigit{3}}\ \isacommand{by}\isamarkupfalse%
\ {\isacharparenleft}rule\ conjunct{\isadigit{2}}{\isacharparenright}\isanewline
\ \ \ \ \ \ \ \ \ \ \ \ \isacommand{have}\isamarkupfalse%
\ {\isachardoublequoteopen}F\ {\isasymin}\ S\ {\isasymlongrightarrow}\ {\isacharbraceleft}G{\isacharbraceright}\ {\isasymunion}\ S\ {\isasymin}\ C{\isachardoublequoteclose}\isanewline
\ \ \ \ \ \ \ \ \ \ \ \ \ \ \isacommand{using}\isamarkupfalse%
\ C{\isadigit{2}}\ {\isacartoucheopen}F\ {\isacharequal}\ \isactrlbold {\isasymnot}{\isacharparenleft}\isactrlbold {\isasymnot}\ G{\isacharparenright}{\isacartoucheclose}\ \isacommand{by}\isamarkupfalse%
\ {\isacharparenleft}iprover\ elim{\isacharcolon}\ allE{\isacharparenright}\isanewline
\ \ \ \ \ \ \ \ \ \ \ \ \isacommand{then}\isamarkupfalse%
\ \isacommand{have}\isamarkupfalse%
\ {\isachardoublequoteopen}F\ {\isasymin}\ S\ {\isasymlongrightarrow}\ {\isacharbraceleft}G{\isacharcomma}G{\isacharbraceright}\ {\isasymunion}\ S\ {\isasymin}\ C{\isachardoublequoteclose}\isanewline
\ \ \ \ \ \ \ \ \ \ \ \ \ \ \isacommand{by}\isamarkupfalse%
\ {\isacharparenleft}simp\ only{\isacharcolon}\ insert{\isacharunderscore}absorb{\isadigit{2}}{\isacharparenright}\isanewline
\ \ \ \ \ \ \ \ \ \ \ \ \isacommand{thus}\isamarkupfalse%
\ {\isachardoublequoteopen}F\ {\isasymin}\ S\ {\isasymlongrightarrow}\ {\isacharbraceleft}G{\isacharcomma}H{\isacharbraceright}\ {\isasymunion}\ S\ {\isasymin}\ C{\isachardoublequoteclose}\ \isanewline
\ \ \ \ \ \ \ \ \ \ \ \ \ \ \isacommand{by}\isamarkupfalse%
\ {\isacharparenleft}simp\ only{\isacharcolon}\ {\isacartoucheopen}H\ {\isacharequal}\ G{\isacartoucheclose}{\isacharparenright}\isanewline
\ \ \ \ \ \ \ \ \ \ \isacommand{qed}\isamarkupfalse%
\isanewline
\ \ \ \ \ \ \ \ \isacommand{qed}\isamarkupfalse%
\isanewline
\ \ \ \ \ \ \isacommand{qed}\isamarkupfalse%
\isanewline
\ \ \ \ \isacommand{qed}\isamarkupfalse%
\isanewline
\ \ \isacommand{qed}\isamarkupfalse%
\isanewline
\isacommand{qed}\isamarkupfalse%
%
\endisatagproof
{\isafoldproof}%
%
\isadelimproof
%
\endisadelimproof
%
\begin{isamarkuptext}%
Finalmente, el siguiente lema auxiliar deduce la condición de \isa{{\isadigit{2}}{\isacharparenright}} sobre fórmulas de tipo \isa{{\isasymbeta}} 
  a partir de las condiciones cuarta, quinta, sexta y séptima de la definición de propiedad de 
  consistencia proposicional.%
\end{isamarkuptext}\isamarkuptrue%
\isacommand{lemma}\isamarkupfalse%
\ pcp{\isacharunderscore}alt{\isadigit{1}}Dis{\isacharcolon}\isanewline
\ \ \isakeyword{assumes}\ {\isachardoublequoteopen}{\isacharparenleft}{\isasymforall}G\ H{\isachardot}\ G\ \isactrlbold {\isasymor}\ H\ {\isasymin}\ S\ {\isasymlongrightarrow}\ {\isacharbraceleft}G{\isacharbraceright}\ {\isasymunion}\ S\ {\isasymin}\ C\ {\isasymor}\ {\isacharbraceleft}H{\isacharbraceright}\ {\isasymunion}\ S\ {\isasymin}\ C{\isacharparenright}\isanewline
\ \ {\isasymand}\ {\isacharparenleft}{\isasymforall}G\ H{\isachardot}\ G\ \isactrlbold {\isasymrightarrow}\ H\ {\isasymin}\ S\ {\isasymlongrightarrow}\ {\isacharbraceleft}\isactrlbold {\isasymnot}\ G{\isacharbraceright}\ {\isasymunion}\ S\ {\isasymin}\ C\ {\isasymor}\ {\isacharbraceleft}H{\isacharbraceright}\ {\isasymunion}\ S\ {\isasymin}\ C{\isacharparenright}\isanewline
\ \ {\isasymand}\ {\isacharparenleft}{\isasymforall}G{\isachardot}\ \isactrlbold {\isasymnot}\ {\isacharparenleft}\isactrlbold {\isasymnot}G{\isacharparenright}\ {\isasymin}\ S\ {\isasymlongrightarrow}\ {\isacharbraceleft}G{\isacharbraceright}\ {\isasymunion}\ S\ {\isasymin}\ C{\isacharparenright}\isanewline
\ \ {\isasymand}\ {\isacharparenleft}{\isasymforall}G\ H{\isachardot}\ \isactrlbold {\isasymnot}{\isacharparenleft}G\ \isactrlbold {\isasymand}\ H{\isacharparenright}\ {\isasymin}\ S\ {\isasymlongrightarrow}\ {\isacharbraceleft}\isactrlbold {\isasymnot}\ G{\isacharbraceright}\ {\isasymunion}\ S\ {\isasymin}\ C\ {\isasymor}\ {\isacharbraceleft}\isactrlbold {\isasymnot}\ H{\isacharbraceright}\ {\isasymunion}\ S\ {\isasymin}\ C{\isacharparenright}{\isachardoublequoteclose}\isanewline
\ \ \isakeyword{shows}\ {\isachardoublequoteopen}{\isasymforall}F\ G\ H{\isachardot}\ Dis\ F\ G\ H\ {\isasymlongrightarrow}\ F\ {\isasymin}\ S\ {\isasymlongrightarrow}\ {\isacharbraceleft}G{\isacharbraceright}\ {\isasymunion}\ S\ {\isasymin}\ C\ {\isasymor}\ {\isacharbraceleft}H{\isacharbraceright}\ {\isasymunion}\ S\ {\isasymin}\ C{\isachardoublequoteclose}\isanewline
%
\isadelimproof
%
\endisadelimproof
%
\isatagproof
\isacommand{proof}\isamarkupfalse%
\ {\isacharminus}\isanewline
\ \ \isacommand{have}\isamarkupfalse%
\ C{\isadigit{1}}{\isacharcolon}{\isachardoublequoteopen}{\isasymforall}G\ H{\isachardot}\ G\ \isactrlbold {\isasymor}\ H\ {\isasymin}\ S\ {\isasymlongrightarrow}\ {\isacharbraceleft}G{\isacharbraceright}\ {\isasymunion}\ S\ {\isasymin}\ C\ {\isasymor}\ {\isacharbraceleft}H{\isacharbraceright}\ {\isasymunion}\ S\ {\isasymin}\ C{\isachardoublequoteclose}\isanewline
\ \ \ \ \isacommand{using}\isamarkupfalse%
\ assms\ \isacommand{by}\isamarkupfalse%
\ {\isacharparenleft}rule\ conjunct{\isadigit{1}}{\isacharparenright}\isanewline
\ \ \isacommand{have}\isamarkupfalse%
\ C{\isadigit{2}}{\isacharcolon}{\isachardoublequoteopen}{\isasymforall}G\ H{\isachardot}\ G\ \isactrlbold {\isasymrightarrow}\ H\ {\isasymin}\ S\ {\isasymlongrightarrow}\ {\isacharbraceleft}\isactrlbold {\isasymnot}\ G{\isacharbraceright}\ {\isasymunion}\ S\ {\isasymin}\ C\ {\isasymor}\ {\isacharbraceleft}H{\isacharbraceright}\ {\isasymunion}\ S\ {\isasymin}\ C{\isachardoublequoteclose}\isanewline
\ \ \ \ \isacommand{using}\isamarkupfalse%
\ assms\ \isacommand{by}\isamarkupfalse%
\ {\isacharparenleft}iprover\ elim{\isacharcolon}\ conjunct{\isadigit{2}}\ conjunct{\isadigit{1}}{\isacharparenright}\isanewline
\ \ \isacommand{have}\isamarkupfalse%
\ C{\isadigit{3}}{\isacharcolon}{\isachardoublequoteopen}{\isasymforall}G{\isachardot}\ \isactrlbold {\isasymnot}\ {\isacharparenleft}\isactrlbold {\isasymnot}G{\isacharparenright}\ {\isasymin}\ S\ {\isasymlongrightarrow}\ {\isacharbraceleft}G{\isacharbraceright}\ {\isasymunion}\ S\ {\isasymin}\ C{\isachardoublequoteclose}\isanewline
\ \ \ \ \isacommand{using}\isamarkupfalse%
\ assms\ \isacommand{by}\isamarkupfalse%
\ {\isacharparenleft}iprover\ elim{\isacharcolon}\ conjunct{\isadigit{2}}\ conjunct{\isadigit{1}}{\isacharparenright}\isanewline
\ \ \isacommand{have}\isamarkupfalse%
\ C{\isadigit{4}}{\isacharcolon}{\isachardoublequoteopen}{\isasymforall}G\ H{\isachardot}\ \isactrlbold {\isasymnot}{\isacharparenleft}G\ \isactrlbold {\isasymand}\ H{\isacharparenright}\ {\isasymin}\ S\ {\isasymlongrightarrow}\ {\isacharbraceleft}\isactrlbold {\isasymnot}\ G{\isacharbraceright}\ {\isasymunion}\ S\ {\isasymin}\ C\ {\isasymor}\ {\isacharbraceleft}\isactrlbold {\isasymnot}\ H{\isacharbraceright}\ {\isasymunion}\ S\ {\isasymin}\ C{\isachardoublequoteclose}\isanewline
\ \ \ \ \isacommand{using}\isamarkupfalse%
\ assms\ \isacommand{by}\isamarkupfalse%
\ {\isacharparenleft}iprover\ elim{\isacharcolon}\ conjunct{\isadigit{2}}{\isacharparenright}\ \isanewline
\ \ \isacommand{show}\isamarkupfalse%
\ {\isachardoublequoteopen}{\isasymforall}F\ G\ H{\isachardot}\ Dis\ F\ G\ H\ {\isasymlongrightarrow}\ F\ {\isasymin}\ S\ {\isasymlongrightarrow}\ {\isacharbraceleft}G{\isacharbraceright}\ {\isasymunion}\ S\ {\isasymin}\ C\ {\isasymor}\ {\isacharbraceleft}H{\isacharbraceright}\ {\isasymunion}\ S\ {\isasymin}\ C{\isachardoublequoteclose}\isanewline
\ \ \isacommand{proof}\isamarkupfalse%
\ {\isacharparenleft}rule\ allI{\isacharparenright}{\isacharplus}\isanewline
\ \ \ \ \isacommand{fix}\isamarkupfalse%
\ F\ G\ H\isanewline
\ \ \ \ \isacommand{show}\isamarkupfalse%
\ {\isachardoublequoteopen}Dis\ F\ G\ H\ {\isasymlongrightarrow}\ F\ {\isasymin}\ S\ {\isasymlongrightarrow}\ {\isacharbraceleft}G{\isacharbraceright}\ {\isasymunion}\ S\ {\isasymin}\ C\ {\isasymor}\ {\isacharbraceleft}H{\isacharbraceright}\ {\isasymunion}\ S\ {\isasymin}\ C{\isachardoublequoteclose}\isanewline
\ \ \ \ \isacommand{proof}\isamarkupfalse%
\ {\isacharparenleft}rule\ impI{\isacharparenright}\isanewline
\ \ \ \ \ \ \isacommand{assume}\isamarkupfalse%
\ {\isachardoublequoteopen}Dis\ F\ G\ H{\isachardoublequoteclose}\isanewline
\ \ \ \ \ \ \isacommand{then}\isamarkupfalse%
\ \isacommand{have}\isamarkupfalse%
\ {\isachardoublequoteopen}F\ {\isacharequal}\ G\ \isactrlbold {\isasymor}\ H\ {\isasymor}\ \isanewline
\ \ \ \ \ \ \ \ \ \ \ \ \ \ \ \ {\isacharparenleft}{\isasymexists}G{\isadigit{1}}\ H{\isadigit{1}}{\isachardot}\ F\ {\isacharequal}\ G{\isadigit{1}}\ \isactrlbold {\isasymrightarrow}\ H{\isadigit{1}}\ {\isasymand}\ G\ {\isacharequal}\ \isactrlbold {\isasymnot}\ G{\isadigit{1}}\ {\isasymand}\ H\ {\isacharequal}\ H{\isadigit{1}}{\isacharparenright}\ {\isasymor}\ \isanewline
\ \ \ \ \ \ \ \ \ \ \ \ \ \ \ \ {\isacharparenleft}{\isasymexists}G{\isadigit{2}}\ H{\isadigit{2}}{\isachardot}\ F\ {\isacharequal}\ \isactrlbold {\isasymnot}\ {\isacharparenleft}G{\isadigit{2}}\ \isactrlbold {\isasymand}\ H{\isadigit{2}}{\isacharparenright}\ {\isasymand}\ G\ {\isacharequal}\ \isactrlbold {\isasymnot}\ G{\isadigit{2}}\ {\isasymand}\ H\ {\isacharequal}\ \isactrlbold {\isasymnot}\ H{\isadigit{2}}{\isacharparenright}\ {\isasymor}\ \isanewline
\ \ \ \ \ \ \ \ \ \ \ \ \ \ \ \ F\ {\isacharequal}\ \isactrlbold {\isasymnot}\ {\isacharparenleft}\isactrlbold {\isasymnot}\ G{\isacharparenright}\ {\isasymand}\ H\ {\isacharequal}\ G{\isachardoublequoteclose}\ \isanewline
\ \ \ \ \ \ \ \ \isacommand{by}\isamarkupfalse%
\ {\isacharparenleft}simp\ only{\isacharcolon}\ con{\isacharunderscore}dis{\isacharunderscore}simps{\isacharparenleft}{\isadigit{2}}{\isacharparenright}{\isacharparenright}\isanewline
\ \ \ \ \ \ \isacommand{thus}\isamarkupfalse%
\ {\isachardoublequoteopen}F\ {\isasymin}\ S\ {\isasymlongrightarrow}\ {\isacharbraceleft}G{\isacharbraceright}\ {\isasymunion}\ S\ {\isasymin}\ C\ {\isasymor}\ {\isacharbraceleft}H{\isacharbraceright}\ {\isasymunion}\ S\ {\isasymin}\ C{\isachardoublequoteclose}\isanewline
\ \ \ \ \ \ \isacommand{proof}\isamarkupfalse%
\ {\isacharparenleft}rule\ disjE{\isacharparenright}\isanewline
\ \ \ \ \ \ \ \ \isacommand{assume}\isamarkupfalse%
\ {\isachardoublequoteopen}F\ {\isacharequal}\ G\ \isactrlbold {\isasymor}\ H{\isachardoublequoteclose}\isanewline
\ \ \ \ \ \ \ \ \isacommand{show}\isamarkupfalse%
\ {\isachardoublequoteopen}F\ {\isasymin}\ S\ {\isasymlongrightarrow}\ {\isacharbraceleft}G{\isacharbraceright}\ {\isasymunion}\ S\ {\isasymin}\ C\ {\isasymor}\ {\isacharbraceleft}H{\isacharbraceright}\ {\isasymunion}\ S\ {\isasymin}\ C{\isachardoublequoteclose}\isanewline
\ \ \ \ \ \ \ \ \ \ \isacommand{using}\isamarkupfalse%
\ C{\isadigit{1}}\ {\isacartoucheopen}F\ {\isacharequal}\ G\ \isactrlbold {\isasymor}\ H{\isacartoucheclose}\ \isacommand{by}\isamarkupfalse%
\ {\isacharparenleft}iprover\ elim{\isacharcolon}\ allE{\isacharparenright}\isanewline
\ \ \ \ \ \ \isacommand{next}\isamarkupfalse%
\isanewline
\ \ \ \ \ \ \ \ \isacommand{assume}\isamarkupfalse%
\ {\isachardoublequoteopen}{\isacharparenleft}{\isasymexists}G{\isadigit{1}}\ H{\isadigit{1}}{\isachardot}\ F\ {\isacharequal}\ G{\isadigit{1}}\ \isactrlbold {\isasymrightarrow}\ H{\isadigit{1}}\ {\isasymand}\ G\ {\isacharequal}\ \isactrlbold {\isasymnot}\ G{\isadigit{1}}\ {\isasymand}\ H\ {\isacharequal}\ H{\isadigit{1}}{\isacharparenright}\ {\isasymor}\ \isanewline
\ \ \ \ \ \ \ \ \ \ \ \ \ \ {\isacharparenleft}{\isasymexists}G{\isadigit{2}}\ H{\isadigit{2}}{\isachardot}\ F\ {\isacharequal}\ \isactrlbold {\isasymnot}\ {\isacharparenleft}G{\isadigit{2}}\ \isactrlbold {\isasymand}\ H{\isadigit{2}}{\isacharparenright}\ {\isasymand}\ G\ {\isacharequal}\ \isactrlbold {\isasymnot}\ G{\isadigit{2}}\ {\isasymand}\ H\ {\isacharequal}\ \isactrlbold {\isasymnot}\ H{\isadigit{2}}{\isacharparenright}\ {\isasymor}\ \isanewline
\ \ \ \ \ \ \ \ \ \ \ \ \ \ F\ {\isacharequal}\ \isactrlbold {\isasymnot}\ {\isacharparenleft}\isactrlbold {\isasymnot}\ G{\isacharparenright}\ {\isasymand}\ H\ {\isacharequal}\ G{\isachardoublequoteclose}\isanewline
\ \ \ \ \ \ \ \ \isacommand{thus}\isamarkupfalse%
\ {\isachardoublequoteopen}F\ {\isasymin}\ S\ {\isasymlongrightarrow}\ {\isacharbraceleft}G{\isacharbraceright}\ {\isasymunion}\ S\ {\isasymin}\ C\ {\isasymor}\ {\isacharbraceleft}H{\isacharbraceright}\ {\isasymunion}\ S\ {\isasymin}\ C{\isachardoublequoteclose}\isanewline
\ \ \ \ \ \ \ \ \isacommand{proof}\isamarkupfalse%
\ {\isacharparenleft}rule\ disjE{\isacharparenright}\isanewline
\ \ \ \ \ \ \ \ \ \ \isacommand{assume}\isamarkupfalse%
\ E{\isadigit{1}}{\isacharcolon}{\isachardoublequoteopen}{\isasymexists}G{\isadigit{1}}\ H{\isadigit{1}}{\isachardot}\ F\ {\isacharequal}\ {\isacharparenleft}G{\isadigit{1}}\ \isactrlbold {\isasymrightarrow}\ H{\isadigit{1}}{\isacharparenright}\ {\isasymand}\ G\ {\isacharequal}\ \isactrlbold {\isasymnot}\ G{\isadigit{1}}\ {\isasymand}\ H\ {\isacharequal}\ H{\isadigit{1}}{\isachardoublequoteclose}\isanewline
\ \ \ \ \ \ \ \ \ \ \isacommand{obtain}\isamarkupfalse%
\ G{\isadigit{1}}\ H{\isadigit{1}}\ \isakeyword{where}\ A{\isadigit{1}}{\isacharcolon}{\isachardoublequoteopen}\ F\ {\isacharequal}\ {\isacharparenleft}G{\isadigit{1}}\ \isactrlbold {\isasymrightarrow}\ H{\isadigit{1}}{\isacharparenright}\ {\isasymand}\ G\ {\isacharequal}\ \isactrlbold {\isasymnot}\ G{\isadigit{1}}\ {\isasymand}\ H\ {\isacharequal}\ H{\isadigit{1}}{\isachardoublequoteclose}\isanewline
\ \ \ \ \ \ \ \ \ \ \ \ \isacommand{using}\isamarkupfalse%
\ E{\isadigit{1}}\ \isacommand{by}\isamarkupfalse%
\ {\isacharparenleft}iprover\ elim{\isacharcolon}\ exE{\isacharparenright}\isanewline
\ \ \ \ \ \ \ \ \ \ \isacommand{have}\isamarkupfalse%
\ {\isachardoublequoteopen}F\ {\isacharequal}\ {\isacharparenleft}G{\isadigit{1}}\ \isactrlbold {\isasymrightarrow}\ H{\isadigit{1}}{\isacharparenright}{\isachardoublequoteclose}\isanewline
\ \ \ \ \ \ \ \ \ \ \ \ \isacommand{using}\isamarkupfalse%
\ A{\isadigit{1}}\ \isacommand{by}\isamarkupfalse%
\ {\isacharparenleft}rule\ conjunct{\isadigit{1}}{\isacharparenright}\isanewline
\ \ \ \ \ \ \ \ \ \ \isacommand{have}\isamarkupfalse%
\ {\isachardoublequoteopen}G\ {\isacharequal}\ \isactrlbold {\isasymnot}\ G{\isadigit{1}}{\isachardoublequoteclose}\isanewline
\ \ \ \ \ \ \ \ \ \ \ \ \isacommand{using}\isamarkupfalse%
\ A{\isadigit{1}}\ \isacommand{by}\isamarkupfalse%
\ {\isacharparenleft}iprover\ elim{\isacharcolon}\ conjunct{\isadigit{2}}\ conjunct{\isadigit{1}}{\isacharparenright}\isanewline
\ \ \ \ \ \ \ \ \ \ \isacommand{have}\isamarkupfalse%
\ {\isachardoublequoteopen}H\ {\isacharequal}\ H{\isadigit{1}}{\isachardoublequoteclose}\isanewline
\ \ \ \ \ \ \ \ \ \ \ \ \isacommand{using}\isamarkupfalse%
\ A{\isadigit{1}}\ \isacommand{by}\isamarkupfalse%
\ {\isacharparenleft}iprover\ elim{\isacharcolon}\ conjunct{\isadigit{2}}{\isacharparenright}\isanewline
\ \ \ \ \ \ \ \ \ \ \isacommand{show}\isamarkupfalse%
\ {\isachardoublequoteopen}F\ {\isasymin}\ S\ {\isasymlongrightarrow}\ {\isacharbraceleft}G{\isacharbraceright}\ {\isasymunion}\ S\ {\isasymin}\ C\ {\isasymor}\ {\isacharbraceleft}H{\isacharbraceright}\ {\isasymunion}\ S\ {\isasymin}\ C{\isachardoublequoteclose}\isanewline
\ \ \ \ \ \ \ \ \ \ \ \ \isacommand{using}\isamarkupfalse%
\ C{\isadigit{2}}\ {\isacartoucheopen}F\ {\isacharequal}\ {\isacharparenleft}G{\isadigit{1}}\ \isactrlbold {\isasymrightarrow}\ H{\isadigit{1}}{\isacharparenright}{\isacartoucheclose}\ {\isacartoucheopen}G\ {\isacharequal}\ \isactrlbold {\isasymnot}\ G{\isadigit{1}}{\isacartoucheclose}\ {\isacartoucheopen}H\ {\isacharequal}\ H{\isadigit{1}}{\isacartoucheclose}\ \isacommand{by}\isamarkupfalse%
\ {\isacharparenleft}iprover\ elim{\isacharcolon}\ allE{\isacharparenright}\isanewline
\ \ \ \ \ \ \ \ \isacommand{next}\isamarkupfalse%
\isanewline
\ \ \ \ \ \ \ \ \ \ \isacommand{assume}\isamarkupfalse%
\ {\isachardoublequoteopen}{\isacharparenleft}{\isasymexists}G{\isadigit{2}}\ H{\isadigit{2}}{\isachardot}\ F\ {\isacharequal}\ \isactrlbold {\isasymnot}\ {\isacharparenleft}G{\isadigit{2}}\ \isactrlbold {\isasymand}\ H{\isadigit{2}}{\isacharparenright}\ {\isasymand}\ G\ {\isacharequal}\ \isactrlbold {\isasymnot}\ G{\isadigit{2}}\ {\isasymand}\ H\ {\isacharequal}\ \isactrlbold {\isasymnot}\ H{\isadigit{2}}{\isacharparenright}\ {\isasymor}\ \isanewline
\ \ \ \ \ \ \ \ \ \ \ \ \ \ \ \ \ \ F\ {\isacharequal}\ \isactrlbold {\isasymnot}\ {\isacharparenleft}\isactrlbold {\isasymnot}\ G{\isacharparenright}\ {\isasymand}\ H\ {\isacharequal}\ G{\isachardoublequoteclose}\ \isanewline
\ \ \ \ \ \ \ \ \ \ \isacommand{thus}\isamarkupfalse%
\ {\isachardoublequoteopen}F\ {\isasymin}\ S\ {\isasymlongrightarrow}\ {\isacharbraceleft}G{\isacharbraceright}\ {\isasymunion}\ S\ {\isasymin}\ C\ {\isasymor}\ {\isacharbraceleft}H{\isacharbraceright}\ {\isasymunion}\ S\ {\isasymin}\ C{\isachardoublequoteclose}\isanewline
\ \ \ \ \ \ \ \ \ \ \isacommand{proof}\isamarkupfalse%
\ {\isacharparenleft}rule\ disjE{\isacharparenright}\isanewline
\ \ \ \ \ \ \ \ \ \ \ \ \isacommand{assume}\isamarkupfalse%
\ E{\isadigit{2}}{\isacharcolon}{\isachardoublequoteopen}{\isasymexists}G{\isadigit{2}}\ H{\isadigit{2}}{\isachardot}\ F\ {\isacharequal}\ \isactrlbold {\isasymnot}\ {\isacharparenleft}G{\isadigit{2}}\ \isactrlbold {\isasymand}\ H{\isadigit{2}}{\isacharparenright}\ {\isasymand}\ G\ {\isacharequal}\ \isactrlbold {\isasymnot}\ G{\isadigit{2}}\ {\isasymand}\ H\ {\isacharequal}\ \isactrlbold {\isasymnot}\ H{\isadigit{2}}{\isachardoublequoteclose}\isanewline
\ \ \ \ \ \ \ \ \ \ \ \ \isacommand{obtain}\isamarkupfalse%
\ G{\isadigit{2}}\ H{\isadigit{2}}\ \isakeyword{where}\ A{\isadigit{2}}{\isacharcolon}{\isachardoublequoteopen}F\ {\isacharequal}\ \isactrlbold {\isasymnot}\ {\isacharparenleft}G{\isadigit{2}}\ \isactrlbold {\isasymand}\ H{\isadigit{2}}{\isacharparenright}\ {\isasymand}\ G\ {\isacharequal}\ \isactrlbold {\isasymnot}\ G{\isadigit{2}}\ {\isasymand}\ H\ {\isacharequal}\ \isactrlbold {\isasymnot}\ H{\isadigit{2}}{\isachardoublequoteclose}\isanewline
\ \ \ \ \ \ \ \ \ \ \ \ \ \ \isacommand{using}\isamarkupfalse%
\ E{\isadigit{2}}\ \isacommand{by}\isamarkupfalse%
\ {\isacharparenleft}iprover\ elim{\isacharcolon}\ exE{\isacharparenright}\isanewline
\ \ \ \ \ \ \ \ \ \ \ \ \isacommand{have}\isamarkupfalse%
\ {\isachardoublequoteopen}F\ {\isacharequal}\ \isactrlbold {\isasymnot}\ {\isacharparenleft}G{\isadigit{2}}\ \isactrlbold {\isasymand}\ H{\isadigit{2}}{\isacharparenright}{\isachardoublequoteclose}\isanewline
\ \ \ \ \ \ \ \ \ \ \ \ \ \ \isacommand{using}\isamarkupfalse%
\ A{\isadigit{2}}\ \isacommand{by}\isamarkupfalse%
\ {\isacharparenleft}rule\ conjunct{\isadigit{1}}{\isacharparenright}\isanewline
\ \ \ \ \ \ \ \ \ \ \ \ \isacommand{have}\isamarkupfalse%
\ {\isachardoublequoteopen}G\ {\isacharequal}\ \isactrlbold {\isasymnot}\ G{\isadigit{2}}{\isachardoublequoteclose}\isanewline
\ \ \ \ \ \ \ \ \ \ \ \ \ \ \isacommand{using}\isamarkupfalse%
\ A{\isadigit{2}}\ \isacommand{by}\isamarkupfalse%
\ {\isacharparenleft}iprover\ elim{\isacharcolon}\ conjunct{\isadigit{2}}\ conjunct{\isadigit{1}}{\isacharparenright}\isanewline
\ \ \ \ \ \ \ \ \ \ \ \ \isacommand{have}\isamarkupfalse%
\ {\isachardoublequoteopen}H\ {\isacharequal}\ \isactrlbold {\isasymnot}\ H{\isadigit{2}}{\isachardoublequoteclose}\isanewline
\ \ \ \ \ \ \ \ \ \ \ \ \ \ \isacommand{using}\isamarkupfalse%
\ A{\isadigit{2}}\ \isacommand{by}\isamarkupfalse%
\ {\isacharparenleft}iprover\ elim{\isacharcolon}\ conjunct{\isadigit{2}}{\isacharparenright}\isanewline
\ \ \ \ \ \ \ \ \ \ \ \ \isacommand{show}\isamarkupfalse%
\ {\isachardoublequoteopen}F\ {\isasymin}\ S\ {\isasymlongrightarrow}\ {\isacharbraceleft}G{\isacharbraceright}\ {\isasymunion}\ S\ {\isasymin}\ C\ {\isasymor}\ {\isacharbraceleft}H{\isacharbraceright}\ {\isasymunion}\ S\ {\isasymin}\ C{\isachardoublequoteclose}\isanewline
\ \ \ \ \ \ \ \ \ \ \ \ \ \ \isacommand{using}\isamarkupfalse%
\ C{\isadigit{4}}\ {\isacartoucheopen}F\ {\isacharequal}\ \isactrlbold {\isasymnot}\ {\isacharparenleft}G{\isadigit{2}}\ \isactrlbold {\isasymand}\ H{\isadigit{2}}{\isacharparenright}{\isacartoucheclose}\ {\isacartoucheopen}G\ {\isacharequal}\ \isactrlbold {\isasymnot}\ G{\isadigit{2}}{\isacartoucheclose}\ {\isacartoucheopen}H\ {\isacharequal}\ \isactrlbold {\isasymnot}\ H{\isadigit{2}}{\isacartoucheclose}\ \isacommand{by}\isamarkupfalse%
\ {\isacharparenleft}iprover\ elim{\isacharcolon}\ allE{\isacharparenright}\isanewline
\ \ \ \ \ \ \ \ \ \ \isacommand{next}\isamarkupfalse%
\isanewline
\ \ \ \ \ \ \ \ \ \ \ \ \isacommand{assume}\isamarkupfalse%
\ A{\isadigit{3}}{\isacharcolon}{\isachardoublequoteopen}F\ {\isacharequal}\ \isactrlbold {\isasymnot}{\isacharparenleft}\isactrlbold {\isasymnot}\ G{\isacharparenright}\ {\isasymand}\ H\ {\isacharequal}\ G{\isachardoublequoteclose}\isanewline
\ \ \ \ \ \ \ \ \ \ \ \ \isacommand{then}\isamarkupfalse%
\ \isacommand{have}\isamarkupfalse%
\ {\isachardoublequoteopen}F\ {\isacharequal}\ \isactrlbold {\isasymnot}{\isacharparenleft}\isactrlbold {\isasymnot}\ G{\isacharparenright}{\isachardoublequoteclose}\isanewline
\ \ \ \ \ \ \ \ \ \ \ \ \ \ \isacommand{by}\isamarkupfalse%
\ {\isacharparenleft}rule\ conjunct{\isadigit{1}}{\isacharparenright}\isanewline
\ \ \ \ \ \ \ \ \ \ \ \ \isacommand{have}\isamarkupfalse%
\ {\isachardoublequoteopen}H\ {\isacharequal}\ G{\isachardoublequoteclose}\isanewline
\ \ \ \ \ \ \ \ \ \ \ \ \ \ \isacommand{using}\isamarkupfalse%
\ A{\isadigit{3}}\ \isacommand{by}\isamarkupfalse%
\ {\isacharparenleft}rule\ conjunct{\isadigit{2}}{\isacharparenright}\isanewline
\ \ \ \ \ \ \ \ \ \ \ \ \isacommand{have}\isamarkupfalse%
\ {\isachardoublequoteopen}F\ {\isasymin}\ S\ {\isasymlongrightarrow}\ {\isacharbraceleft}G{\isacharbraceright}\ {\isasymunion}\ S\ {\isasymin}\ C{\isachardoublequoteclose}\isanewline
\ \ \ \ \ \ \ \ \ \ \ \ \ \ \isacommand{using}\isamarkupfalse%
\ C{\isadigit{3}}\ {\isacartoucheopen}F\ {\isacharequal}\ \isactrlbold {\isasymnot}{\isacharparenleft}\isactrlbold {\isasymnot}\ G{\isacharparenright}{\isacartoucheclose}\ \isacommand{by}\isamarkupfalse%
\ {\isacharparenleft}iprover\ elim{\isacharcolon}\ allE{\isacharparenright}\isanewline
\ \ \ \ \ \ \ \ \ \ \ \ \isacommand{then}\isamarkupfalse%
\ \isacommand{have}\isamarkupfalse%
\ {\isachardoublequoteopen}F\ {\isasymin}\ S\ {\isasymlongrightarrow}\ {\isacharbraceleft}G{\isacharbraceright}\ {\isasymunion}\ S\ {\isasymin}\ C\ {\isasymor}\ {\isacharbraceleft}G{\isacharbraceright}\ {\isasymunion}\ S\ {\isasymin}\ C{\isachardoublequoteclose}\isanewline
\ \ \ \ \ \ \ \ \ \ \ \ \ \ \isacommand{by}\isamarkupfalse%
\ {\isacharparenleft}simp\ only{\isacharcolon}\ disj{\isacharunderscore}absorb{\isacharparenright}\isanewline
\ \ \ \ \ \ \ \ \ \ \ \ \isacommand{thus}\isamarkupfalse%
\ {\isachardoublequoteopen}F\ {\isasymin}\ S\ {\isasymlongrightarrow}\ {\isacharbraceleft}G{\isacharbraceright}\ {\isasymunion}\ S\ {\isasymin}\ C\ {\isasymor}\ {\isacharbraceleft}H{\isacharbraceright}\ {\isasymunion}\ S\ {\isasymin}\ C{\isachardoublequoteclose}\isanewline
\ \ \ \ \ \ \ \ \ \ \ \ \ \ \isacommand{by}\isamarkupfalse%
\ {\isacharparenleft}simp\ only{\isacharcolon}\ {\isacartoucheopen}H\ {\isacharequal}\ G{\isacartoucheclose}{\isacharparenright}\isanewline
\ \ \ \ \ \ \ \ \ \ \isacommand{qed}\isamarkupfalse%
\isanewline
\ \ \ \ \ \ \ \ \isacommand{qed}\isamarkupfalse%
\isanewline
\ \ \ \ \ \ \isacommand{qed}\isamarkupfalse%
\isanewline
\ \ \ \ \isacommand{qed}\isamarkupfalse%
\isanewline
\ \ \isacommand{qed}\isamarkupfalse%
\isanewline
\isacommand{qed}\isamarkupfalse%
%
\endisatagproof
{\isafoldproof}%
%
\isadelimproof
%
\endisadelimproof
%
\begin{isamarkuptext}%
De esta manera, mediante los anteriores lemas auxiliares podemos probar la primera
  implicación detalladamente en Isabelle.%
\end{isamarkuptext}\isamarkuptrue%
\isacommand{lemma}\isamarkupfalse%
\ pcp{\isacharunderscore}alt{\isadigit{1}}{\isacharcolon}\ \isanewline
\ \ \isakeyword{assumes}\ {\isachardoublequoteopen}pcp\ C{\isachardoublequoteclose}\isanewline
\ \ \isakeyword{shows}\ {\isachardoublequoteopen}{\isasymforall}S\ {\isasymin}\ C{\isachardot}\ {\isasymbottom}\ {\isasymnotin}\ S\isanewline
\ \ {\isasymand}\ {\isacharparenleft}{\isasymforall}k{\isachardot}\ Atom\ k\ {\isasymin}\ S\ {\isasymlongrightarrow}\ \isactrlbold {\isasymnot}\ {\isacharparenleft}Atom\ k{\isacharparenright}\ {\isasymin}\ S\ {\isasymlongrightarrow}\ False{\isacharparenright}\isanewline
\ \ {\isasymand}\ {\isacharparenleft}{\isasymforall}F\ G\ H{\isachardot}\ Con\ F\ G\ H\ {\isasymlongrightarrow}\ F\ {\isasymin}\ S\ {\isasymlongrightarrow}\ {\isacharbraceleft}G{\isacharcomma}H{\isacharbraceright}\ {\isasymunion}\ S\ {\isasymin}\ C{\isacharparenright}\isanewline
\ \ {\isasymand}\ {\isacharparenleft}{\isasymforall}F\ G\ H{\isachardot}\ Dis\ F\ G\ H\ {\isasymlongrightarrow}\ F\ {\isasymin}\ S\ {\isasymlongrightarrow}\ {\isacharbraceleft}G{\isacharbraceright}\ {\isasymunion}\ S\ {\isasymin}\ C\ {\isasymor}\ {\isacharbraceleft}H{\isacharbraceright}\ {\isasymunion}\ S\ {\isasymin}\ C{\isacharparenright}{\isachardoublequoteclose}\isanewline
%
\isadelimproof
%
\endisadelimproof
%
\isatagproof
\isacommand{proof}\isamarkupfalse%
\ {\isacharparenleft}rule\ ballI{\isacharparenright}\isanewline
\ \ \isacommand{fix}\isamarkupfalse%
\ S\isanewline
\ \ \isacommand{assume}\isamarkupfalse%
\ {\isachardoublequoteopen}S\ {\isasymin}\ C{\isachardoublequoteclose}\isanewline
\ \ \isacommand{have}\isamarkupfalse%
\ {\isachardoublequoteopen}{\isacharparenleft}{\isasymforall}S\ {\isasymin}\ C{\isachardot}\isanewline
\ \ {\isasymbottom}\ {\isasymnotin}\ S\isanewline
\ \ {\isasymand}\ {\isacharparenleft}{\isasymforall}k{\isachardot}\ Atom\ k\ {\isasymin}\ S\ {\isasymlongrightarrow}\ \isactrlbold {\isasymnot}\ {\isacharparenleft}Atom\ k{\isacharparenright}\ {\isasymin}\ S\ {\isasymlongrightarrow}\ False{\isacharparenright}\isanewline
\ \ {\isasymand}\ {\isacharparenleft}{\isasymforall}G\ H{\isachardot}\ G\ \isactrlbold {\isasymand}\ H\ {\isasymin}\ S\ {\isasymlongrightarrow}\ {\isacharbraceleft}G{\isacharcomma}H{\isacharbraceright}\ {\isasymunion}\ S\ {\isasymin}\ C{\isacharparenright}\isanewline
\ \ {\isasymand}\ {\isacharparenleft}{\isasymforall}G\ H{\isachardot}\ G\ \isactrlbold {\isasymor}\ H\ {\isasymin}\ S\ {\isasymlongrightarrow}\ {\isacharbraceleft}G{\isacharbraceright}\ {\isasymunion}\ S\ {\isasymin}\ C\ {\isasymor}\ {\isacharbraceleft}H{\isacharbraceright}\ {\isasymunion}\ S\ {\isasymin}\ C{\isacharparenright}\isanewline
\ \ {\isasymand}\ {\isacharparenleft}{\isasymforall}G\ H{\isachardot}\ G\ \isactrlbold {\isasymrightarrow}\ H\ {\isasymin}\ S\ {\isasymlongrightarrow}\ {\isacharbraceleft}\isactrlbold {\isasymnot}G{\isacharbraceright}\ {\isasymunion}\ S\ {\isasymin}\ C\ {\isasymor}\ {\isacharbraceleft}H{\isacharbraceright}\ {\isasymunion}\ S\ {\isasymin}\ C{\isacharparenright}\isanewline
\ \ {\isasymand}\ {\isacharparenleft}{\isasymforall}G{\isachardot}\ \isactrlbold {\isasymnot}\ {\isacharparenleft}\isactrlbold {\isasymnot}G{\isacharparenright}\ {\isasymin}\ S\ {\isasymlongrightarrow}\ {\isacharbraceleft}G{\isacharbraceright}\ {\isasymunion}\ S\ {\isasymin}\ C{\isacharparenright}\isanewline
\ \ {\isasymand}\ {\isacharparenleft}{\isasymforall}G\ H{\isachardot}\ \isactrlbold {\isasymnot}{\isacharparenleft}G\ \isactrlbold {\isasymand}\ H{\isacharparenright}\ {\isasymin}\ S\ {\isasymlongrightarrow}\ {\isacharbraceleft}\isactrlbold {\isasymnot}\ G{\isacharbraceright}\ {\isasymunion}\ S\ {\isasymin}\ C\ {\isasymor}\ {\isacharbraceleft}\isactrlbold {\isasymnot}\ H{\isacharbraceright}\ {\isasymunion}\ S\ {\isasymin}\ C{\isacharparenright}\isanewline
\ \ {\isasymand}\ {\isacharparenleft}{\isasymforall}G\ H{\isachardot}\ \isactrlbold {\isasymnot}{\isacharparenleft}G\ \isactrlbold {\isasymor}\ H{\isacharparenright}\ {\isasymin}\ S\ {\isasymlongrightarrow}\ {\isacharbraceleft}\isactrlbold {\isasymnot}\ G{\isacharcomma}\ \isactrlbold {\isasymnot}\ H{\isacharbraceright}\ {\isasymunion}\ S\ {\isasymin}\ C{\isacharparenright}\isanewline
\ \ {\isasymand}\ {\isacharparenleft}{\isasymforall}G\ H{\isachardot}\ \isactrlbold {\isasymnot}{\isacharparenleft}G\ \isactrlbold {\isasymrightarrow}\ H{\isacharparenright}\ {\isasymin}\ S\ {\isasymlongrightarrow}\ {\isacharbraceleft}G{\isacharcomma}\isactrlbold {\isasymnot}\ H{\isacharbraceright}\ {\isasymunion}\ S\ {\isasymin}\ C{\isacharparenright}{\isacharparenright}{\isachardoublequoteclose}\isanewline
\ \ \ \ \isacommand{using}\isamarkupfalse%
\ assms\ \isacommand{by}\isamarkupfalse%
\ {\isacharparenleft}simp\ only{\isacharcolon}\ pcp{\isacharunderscore}def{\isacharparenright}\isanewline
\ \ \isacommand{then}\isamarkupfalse%
\ \isacommand{have}\isamarkupfalse%
\ pcpS{\isacharcolon}{\isachardoublequoteopen}{\isasymbottom}\ {\isasymnotin}\ S\isanewline
\ \ {\isasymand}\ {\isacharparenleft}{\isasymforall}k{\isachardot}\ Atom\ k\ {\isasymin}\ S\ {\isasymlongrightarrow}\ \isactrlbold {\isasymnot}\ {\isacharparenleft}Atom\ k{\isacharparenright}\ {\isasymin}\ S\ {\isasymlongrightarrow}\ False{\isacharparenright}\isanewline
\ \ {\isasymand}\ {\isacharparenleft}{\isasymforall}G\ H{\isachardot}\ G\ \isactrlbold {\isasymand}\ H\ {\isasymin}\ S\ {\isasymlongrightarrow}\ {\isacharbraceleft}G{\isacharcomma}H{\isacharbraceright}\ {\isasymunion}\ S\ {\isasymin}\ C{\isacharparenright}\isanewline
\ \ {\isasymand}\ {\isacharparenleft}{\isasymforall}G\ H{\isachardot}\ G\ \isactrlbold {\isasymor}\ H\ {\isasymin}\ S\ {\isasymlongrightarrow}\ {\isacharbraceleft}G{\isacharbraceright}\ {\isasymunion}\ S\ {\isasymin}\ C\ {\isasymor}\ {\isacharbraceleft}H{\isacharbraceright}\ {\isasymunion}\ S\ {\isasymin}\ C{\isacharparenright}\isanewline
\ \ {\isasymand}\ {\isacharparenleft}{\isasymforall}G\ H{\isachardot}\ G\ \isactrlbold {\isasymrightarrow}\ H\ {\isasymin}\ S\ {\isasymlongrightarrow}\ {\isacharbraceleft}\isactrlbold {\isasymnot}G{\isacharbraceright}\ {\isasymunion}\ S\ {\isasymin}\ C\ {\isasymor}\ {\isacharbraceleft}H{\isacharbraceright}\ {\isasymunion}\ S\ {\isasymin}\ C{\isacharparenright}\isanewline
\ \ {\isasymand}\ {\isacharparenleft}{\isasymforall}G{\isachardot}\ \isactrlbold {\isasymnot}\ {\isacharparenleft}\isactrlbold {\isasymnot}G{\isacharparenright}\ {\isasymin}\ S\ {\isasymlongrightarrow}\ {\isacharbraceleft}G{\isacharbraceright}\ {\isasymunion}\ S\ {\isasymin}\ C{\isacharparenright}\isanewline
\ \ {\isasymand}\ {\isacharparenleft}{\isasymforall}G\ H{\isachardot}\ \isactrlbold {\isasymnot}{\isacharparenleft}G\ \isactrlbold {\isasymand}\ H{\isacharparenright}\ {\isasymin}\ S\ {\isasymlongrightarrow}\ {\isacharbraceleft}\isactrlbold {\isasymnot}\ G{\isacharbraceright}\ {\isasymunion}\ S\ {\isasymin}\ C\ {\isasymor}\ {\isacharbraceleft}\isactrlbold {\isasymnot}\ H{\isacharbraceright}\ {\isasymunion}\ S\ {\isasymin}\ C{\isacharparenright}\isanewline
\ \ {\isasymand}\ {\isacharparenleft}{\isasymforall}G\ H{\isachardot}\ \isactrlbold {\isasymnot}{\isacharparenleft}G\ \isactrlbold {\isasymor}\ H{\isacharparenright}\ {\isasymin}\ S\ {\isasymlongrightarrow}\ {\isacharbraceleft}\isactrlbold {\isasymnot}\ G{\isacharcomma}\ \isactrlbold {\isasymnot}\ H{\isacharbraceright}\ {\isasymunion}\ S\ {\isasymin}\ C{\isacharparenright}\isanewline
\ \ {\isasymand}\ {\isacharparenleft}{\isasymforall}G\ H{\isachardot}\ \isactrlbold {\isasymnot}{\isacharparenleft}G\ \isactrlbold {\isasymrightarrow}\ H{\isacharparenright}\ {\isasymin}\ S\ {\isasymlongrightarrow}\ {\isacharbraceleft}G{\isacharcomma}\isactrlbold {\isasymnot}\ H{\isacharbraceright}\ {\isasymunion}\ S\ {\isasymin}\ C{\isacharparenright}{\isachardoublequoteclose}\isanewline
\ \ \ \ \isacommand{using}\isamarkupfalse%
\ {\isacartoucheopen}S\ {\isasymin}\ C{\isacartoucheclose}\ \isacommand{by}\isamarkupfalse%
\ {\isacharparenleft}rule\ bspec{\isacharparenright}\isanewline
\ \ \isacommand{then}\isamarkupfalse%
\ \isacommand{have}\isamarkupfalse%
\ C{\isadigit{1}}{\isacharcolon}{\isachardoublequoteopen}{\isasymbottom}\ {\isasymnotin}\ S{\isachardoublequoteclose}\isanewline
\ \ \ \ \isacommand{by}\isamarkupfalse%
\ {\isacharparenleft}rule\ conjunct{\isadigit{1}}{\isacharparenright}\isanewline
\ \ \isacommand{have}\isamarkupfalse%
\ C{\isadigit{2}}{\isacharcolon}{\isachardoublequoteopen}{\isasymforall}k{\isachardot}\ Atom\ k\ {\isasymin}\ S\ {\isasymlongrightarrow}\ \isactrlbold {\isasymnot}\ {\isacharparenleft}Atom\ k{\isacharparenright}\ {\isasymin}\ S\ {\isasymlongrightarrow}\ False{\isachardoublequoteclose}\isanewline
\ \ \ \ \isacommand{using}\isamarkupfalse%
\ pcpS\ \isacommand{by}\isamarkupfalse%
\ {\isacharparenleft}iprover\ elim{\isacharcolon}\ conjunct{\isadigit{2}}\ conjunct{\isadigit{1}}{\isacharparenright}\isanewline
\ \ \isacommand{have}\isamarkupfalse%
\ C{\isadigit{3}}{\isacharcolon}{\isachardoublequoteopen}{\isasymforall}G\ H{\isachardot}\ G\ \isactrlbold {\isasymand}\ H\ {\isasymin}\ S\ {\isasymlongrightarrow}\ {\isacharbraceleft}G{\isacharcomma}H{\isacharbraceright}\ {\isasymunion}\ S\ {\isasymin}\ C{\isachardoublequoteclose}\isanewline
\ \ \ \ \isacommand{using}\isamarkupfalse%
\ pcpS\ \isacommand{by}\isamarkupfalse%
\ {\isacharparenleft}iprover\ elim{\isacharcolon}\ conjunct{\isadigit{2}}\ conjunct{\isadigit{1}}{\isacharparenright}\isanewline
\ \ \isacommand{have}\isamarkupfalse%
\ C{\isadigit{4}}{\isacharcolon}{\isachardoublequoteopen}{\isasymforall}G\ H{\isachardot}\ G\ \isactrlbold {\isasymor}\ H\ {\isasymin}\ S\ {\isasymlongrightarrow}\ {\isacharbraceleft}G{\isacharbraceright}\ {\isasymunion}\ S\ {\isasymin}\ C\ {\isasymor}\ {\isacharbraceleft}H{\isacharbraceright}\ {\isasymunion}\ S\ {\isasymin}\ C{\isachardoublequoteclose}\isanewline
\ \ \ \ \isacommand{using}\isamarkupfalse%
\ pcpS\ \isacommand{by}\isamarkupfalse%
\ {\isacharparenleft}iprover\ elim{\isacharcolon}\ conjunct{\isadigit{2}}\ conjunct{\isadigit{1}}{\isacharparenright}\isanewline
\ \ \isacommand{have}\isamarkupfalse%
\ C{\isadigit{5}}{\isacharcolon}{\isachardoublequoteopen}{\isasymforall}G\ H{\isachardot}\ G\ \isactrlbold {\isasymrightarrow}\ H\ {\isasymin}\ S\ {\isasymlongrightarrow}\ {\isacharbraceleft}\isactrlbold {\isasymnot}G{\isacharbraceright}\ {\isasymunion}\ S\ {\isasymin}\ C\ {\isasymor}\ {\isacharbraceleft}H{\isacharbraceright}\ {\isasymunion}\ S\ {\isasymin}\ C{\isachardoublequoteclose}\isanewline
\ \ \ \ \isacommand{using}\isamarkupfalse%
\ pcpS\ \isacommand{by}\isamarkupfalse%
\ {\isacharparenleft}iprover\ elim{\isacharcolon}\ conjunct{\isadigit{2}}\ conjunct{\isadigit{1}}{\isacharparenright}\isanewline
\ \ \isacommand{have}\isamarkupfalse%
\ C{\isadigit{6}}{\isacharcolon}{\isachardoublequoteopen}{\isasymforall}G{\isachardot}\ \isactrlbold {\isasymnot}\ {\isacharparenleft}\isactrlbold {\isasymnot}G{\isacharparenright}\ {\isasymin}\ S\ {\isasymlongrightarrow}\ {\isacharbraceleft}G{\isacharbraceright}\ {\isasymunion}\ S\ {\isasymin}\ C{\isachardoublequoteclose}\isanewline
\ \ \ \ \isacommand{using}\isamarkupfalse%
\ pcpS\ \isacommand{by}\isamarkupfalse%
\ {\isacharparenleft}iprover\ elim{\isacharcolon}\ conjunct{\isadigit{2}}\ conjunct{\isadigit{1}}{\isacharparenright}\isanewline
\ \ \isacommand{have}\isamarkupfalse%
\ C{\isadigit{7}}{\isacharcolon}{\isachardoublequoteopen}{\isasymforall}G\ H{\isachardot}\ \isactrlbold {\isasymnot}{\isacharparenleft}G\ \isactrlbold {\isasymand}\ H{\isacharparenright}\ {\isasymin}\ S\ {\isasymlongrightarrow}\ {\isacharbraceleft}\isactrlbold {\isasymnot}\ G{\isacharbraceright}\ {\isasymunion}\ S\ {\isasymin}\ C\ {\isasymor}\ {\isacharbraceleft}\isactrlbold {\isasymnot}\ H{\isacharbraceright}\ {\isasymunion}\ S\ {\isasymin}\ C{\isachardoublequoteclose}\isanewline
\ \ \ \ \isacommand{using}\isamarkupfalse%
\ pcpS\ \isacommand{by}\isamarkupfalse%
\ {\isacharparenleft}iprover\ elim{\isacharcolon}\ conjunct{\isadigit{2}}\ conjunct{\isadigit{1}}{\isacharparenright}\isanewline
\ \ \isacommand{have}\isamarkupfalse%
\ C{\isadigit{8}}{\isacharcolon}{\isachardoublequoteopen}{\isasymforall}G\ H{\isachardot}\ \isactrlbold {\isasymnot}{\isacharparenleft}G\ \isactrlbold {\isasymor}\ H{\isacharparenright}\ {\isasymin}\ S\ {\isasymlongrightarrow}\ {\isacharbraceleft}\isactrlbold {\isasymnot}\ G{\isacharcomma}\ \isactrlbold {\isasymnot}\ H{\isacharbraceright}\ {\isasymunion}\ S\ {\isasymin}\ C{\isachardoublequoteclose}\isanewline
\ \ \ \ \isacommand{using}\isamarkupfalse%
\ pcpS\ \isacommand{by}\isamarkupfalse%
\ {\isacharparenleft}iprover\ elim{\isacharcolon}\ conjunct{\isadigit{2}}\ conjunct{\isadigit{1}}{\isacharparenright}\isanewline
\ \ \isacommand{have}\isamarkupfalse%
\ C{\isadigit{9}}{\isacharcolon}{\isachardoublequoteopen}{\isasymforall}G\ H{\isachardot}\ \isactrlbold {\isasymnot}{\isacharparenleft}G\ \isactrlbold {\isasymrightarrow}\ H{\isacharparenright}\ {\isasymin}\ S\ {\isasymlongrightarrow}\ {\isacharbraceleft}G{\isacharcomma}\isactrlbold {\isasymnot}\ H{\isacharbraceright}\ {\isasymunion}\ S\ {\isasymin}\ C{\isachardoublequoteclose}\isanewline
\ \ \ \ \isacommand{using}\isamarkupfalse%
\ pcpS\ \isacommand{by}\isamarkupfalse%
\ {\isacharparenleft}iprover\ elim{\isacharcolon}\ conjunct{\isadigit{2}}{\isacharparenright}\isanewline
\ \ \isacommand{have}\isamarkupfalse%
\ {\isachardoublequoteopen}{\isacharparenleft}{\isasymforall}G\ H{\isachardot}\ G\ \isactrlbold {\isasymand}\ H\ {\isasymin}\ S\ {\isasymlongrightarrow}\ {\isacharbraceleft}G{\isacharcomma}H{\isacharbraceright}\ {\isasymunion}\ S\ {\isasymin}\ C{\isacharparenright}\isanewline
\ \ {\isasymand}\ {\isacharparenleft}{\isasymforall}G{\isachardot}\ \isactrlbold {\isasymnot}\ {\isacharparenleft}\isactrlbold {\isasymnot}G{\isacharparenright}\ {\isasymin}\ S\ {\isasymlongrightarrow}\ {\isacharbraceleft}G{\isacharbraceright}\ {\isasymunion}\ S\ {\isasymin}\ C{\isacharparenright}\isanewline
\ \ {\isasymand}\ {\isacharparenleft}{\isasymforall}G\ H{\isachardot}\ \isactrlbold {\isasymnot}{\isacharparenleft}G\ \isactrlbold {\isasymor}\ H{\isacharparenright}\ {\isasymin}\ S\ {\isasymlongrightarrow}\ {\isacharbraceleft}\isactrlbold {\isasymnot}\ G{\isacharcomma}\ \isactrlbold {\isasymnot}\ H{\isacharbraceright}\ {\isasymunion}\ S\ {\isasymin}\ C{\isacharparenright}\isanewline
\ \ {\isasymand}\ {\isacharparenleft}{\isasymforall}G\ H{\isachardot}\ \isactrlbold {\isasymnot}{\isacharparenleft}G\ \isactrlbold {\isasymrightarrow}\ H{\isacharparenright}\ {\isasymin}\ S\ {\isasymlongrightarrow}\ {\isacharbraceleft}G{\isacharcomma}\isactrlbold {\isasymnot}\ H{\isacharbraceright}\ {\isasymunion}\ S\ {\isasymin}\ C{\isacharparenright}{\isachardoublequoteclose}\isanewline
\ \ \ \ \isacommand{using}\isamarkupfalse%
\ C{\isadigit{3}}\ C{\isadigit{6}}\ C{\isadigit{8}}\ C{\isadigit{9}}\ \isacommand{by}\isamarkupfalse%
\ {\isacharparenleft}iprover\ intro{\isacharcolon}\ conjI{\isacharparenright}\isanewline
\ \ \isacommand{then}\isamarkupfalse%
\ \isacommand{have}\isamarkupfalse%
\ Con{\isacharcolon}{\isachardoublequoteopen}{\isasymforall}F\ G\ H{\isachardot}\ Con\ F\ G\ H\ {\isasymlongrightarrow}\ F\ {\isasymin}\ S\ {\isasymlongrightarrow}\ {\isacharbraceleft}G{\isacharcomma}H{\isacharbraceright}\ {\isasymunion}\ S\ {\isasymin}\ C{\isachardoublequoteclose}\isanewline
\ \ \ \ \isacommand{by}\isamarkupfalse%
\ {\isacharparenleft}rule\ pcp{\isacharunderscore}alt{\isadigit{1}}Con{\isacharparenright}\isanewline
\ \ \isacommand{have}\isamarkupfalse%
\ {\isachardoublequoteopen}{\isacharparenleft}{\isasymforall}G\ H{\isachardot}\ G\ \isactrlbold {\isasymor}\ H\ {\isasymin}\ S\ {\isasymlongrightarrow}\ {\isacharbraceleft}G{\isacharbraceright}\ {\isasymunion}\ S\ {\isasymin}\ C\ {\isasymor}\ {\isacharbraceleft}H{\isacharbraceright}\ {\isasymunion}\ S\ {\isasymin}\ C{\isacharparenright}\isanewline
\ \ {\isasymand}\ {\isacharparenleft}{\isasymforall}G\ H{\isachardot}\ G\ \isactrlbold {\isasymrightarrow}\ H\ {\isasymin}\ S\ {\isasymlongrightarrow}\ {\isacharbraceleft}\isactrlbold {\isasymnot}\ G{\isacharbraceright}\ {\isasymunion}\ S\ {\isasymin}\ C\ {\isasymor}\ {\isacharbraceleft}H{\isacharbraceright}\ {\isasymunion}\ S\ {\isasymin}\ C{\isacharparenright}\isanewline
\ \ {\isasymand}\ {\isacharparenleft}{\isasymforall}G{\isachardot}\ \isactrlbold {\isasymnot}\ {\isacharparenleft}\isactrlbold {\isasymnot}G{\isacharparenright}\ {\isasymin}\ S\ {\isasymlongrightarrow}\ {\isacharbraceleft}G{\isacharbraceright}\ {\isasymunion}\ S\ {\isasymin}\ C{\isacharparenright}\isanewline
\ \ {\isasymand}\ {\isacharparenleft}{\isasymforall}G\ H{\isachardot}\ \isactrlbold {\isasymnot}{\isacharparenleft}G\ \isactrlbold {\isasymand}\ H{\isacharparenright}\ {\isasymin}\ S\ {\isasymlongrightarrow}\ {\isacharbraceleft}\isactrlbold {\isasymnot}\ G{\isacharbraceright}\ {\isasymunion}\ S\ {\isasymin}\ C\ {\isasymor}\ {\isacharbraceleft}\isactrlbold {\isasymnot}\ H{\isacharbraceright}\ {\isasymunion}\ S\ {\isasymin}\ C{\isacharparenright}{\isachardoublequoteclose}\isanewline
\ \ \ \ \isacommand{using}\isamarkupfalse%
\ C{\isadigit{4}}\ C{\isadigit{5}}\ C{\isadigit{6}}\ C{\isadigit{7}}\ \isacommand{by}\isamarkupfalse%
\ {\isacharparenleft}iprover\ intro{\isacharcolon}\ conjI{\isacharparenright}\isanewline
\ \ \isacommand{then}\isamarkupfalse%
\ \isacommand{have}\isamarkupfalse%
\ Dis{\isacharcolon}{\isachardoublequoteopen}{\isasymforall}F\ G\ H{\isachardot}\ Dis\ F\ G\ H\ {\isasymlongrightarrow}\ F\ {\isasymin}\ S\ {\isasymlongrightarrow}\ {\isacharbraceleft}G{\isacharbraceright}\ {\isasymunion}\ S\ {\isasymin}\ C\ {\isasymor}\ {\isacharbraceleft}H{\isacharbraceright}\ {\isasymunion}\ S\ {\isasymin}\ C{\isachardoublequoteclose}\isanewline
\ \ \ \ \isacommand{by}\isamarkupfalse%
\ {\isacharparenleft}rule\ pcp{\isacharunderscore}alt{\isadigit{1}}Dis{\isacharparenright}\isanewline
\ \ \isacommand{thus}\isamarkupfalse%
\ {\isachardoublequoteopen}{\isasymbottom}\ {\isasymnotin}\ S\isanewline
\ \ {\isasymand}\ {\isacharparenleft}{\isasymforall}k{\isachardot}\ Atom\ k\ {\isasymin}\ S\ {\isasymlongrightarrow}\ \isactrlbold {\isasymnot}\ {\isacharparenleft}Atom\ k{\isacharparenright}\ {\isasymin}\ S\ {\isasymlongrightarrow}\ False{\isacharparenright}\isanewline
\ \ {\isasymand}\ {\isacharparenleft}{\isasymforall}F\ G\ H{\isachardot}\ Con\ F\ G\ H\ {\isasymlongrightarrow}\ F\ {\isasymin}\ S\ {\isasymlongrightarrow}\ {\isacharbraceleft}G{\isacharcomma}H{\isacharbraceright}\ {\isasymunion}\ S\ {\isasymin}\ C{\isacharparenright}\isanewline
\ \ {\isasymand}\ {\isacharparenleft}{\isasymforall}F\ G\ H{\isachardot}\ Dis\ F\ G\ H\ {\isasymlongrightarrow}\ F\ {\isasymin}\ S\ {\isasymlongrightarrow}\ {\isacharbraceleft}G{\isacharbraceright}\ {\isasymunion}\ S\ {\isasymin}\ C\ {\isasymor}\ {\isacharbraceleft}H{\isacharbraceright}\ {\isasymunion}\ S\ {\isasymin}\ C{\isacharparenright}{\isachardoublequoteclose}\isanewline
\ \ \ \ \isacommand{using}\isamarkupfalse%
\ C{\isadigit{1}}\ C{\isadigit{2}}\ Con\ Dis\ \isacommand{by}\isamarkupfalse%
\ {\isacharparenleft}iprover\ intro{\isacharcolon}\ conjI{\isacharparenright}\isanewline
\isacommand{qed}\isamarkupfalse%
%
\endisatagproof
{\isafoldproof}%
%
\isadelimproof
%
\endisadelimproof
%
\begin{isamarkuptext}%
Por otro lado, veamos la demostración detallada de la implicación recíproca de la
  equivalencia. Para ello, utilizaremos distintos lemas auxiliares para deducir cada una de las 
  condiciones de la definición de propiedad de consistencia proposicional a partir de las
  hipótesis sobre las fórmulas de tipo \isa{{\isasymalpha}} y \isa{{\isasymbeta}}. En primer lugar, veamos los lemas que se deducen
  condiciones a partir de la hipótesis referente a las fórmulas de tipo \isa{{\isasymalpha}}.%
\end{isamarkuptext}\isamarkuptrue%
\isacommand{lemma}\isamarkupfalse%
\ pcp{\isacharunderscore}alt{\isadigit{2}}Con{\isadigit{1}}{\isacharcolon}\isanewline
\ \ \isakeyword{assumes}\ {\isachardoublequoteopen}{\isasymforall}F\ G\ H{\isachardot}\ Con\ F\ G\ H\ {\isasymlongrightarrow}\ F\ {\isasymin}\ S\ {\isasymlongrightarrow}\ {\isacharbraceleft}G{\isacharcomma}H{\isacharbraceright}\ {\isasymunion}\ S\ {\isasymin}\ C{\isachardoublequoteclose}\isanewline
\ \ \isakeyword{shows}\ {\isachardoublequoteopen}{\isasymforall}G\ H{\isachardot}\ G\ \isactrlbold {\isasymand}\ H\ {\isasymin}\ S\ {\isasymlongrightarrow}\ {\isacharbraceleft}G{\isacharcomma}H{\isacharbraceright}\ {\isasymunion}\ S\ {\isasymin}\ C{\isachardoublequoteclose}\isanewline
%
\isadelimproof
%
\endisadelimproof
%
\isatagproof
\isacommand{proof}\isamarkupfalse%
\ {\isacharparenleft}rule\ allI{\isacharparenright}{\isacharplus}\isanewline
\ \ \isacommand{fix}\isamarkupfalse%
\ G\ H\isanewline
\ \ \isacommand{show}\isamarkupfalse%
\ {\isachardoublequoteopen}G\ \isactrlbold {\isasymand}\ H\ {\isasymin}\ S\ {\isasymlongrightarrow}\ {\isacharbraceleft}G{\isacharcomma}H{\isacharbraceright}\ {\isasymunion}\ S\ {\isasymin}\ C{\isachardoublequoteclose}\isanewline
\ \ \isacommand{proof}\isamarkupfalse%
\ {\isacharparenleft}rule\ impI{\isacharparenright}\isanewline
\ \ \ \ \isacommand{assume}\isamarkupfalse%
\ {\isachardoublequoteopen}G\ \isactrlbold {\isasymand}\ H\ {\isasymin}\ S{\isachardoublequoteclose}\isanewline
\ \ \ \ \isacommand{then}\isamarkupfalse%
\ \isacommand{have}\isamarkupfalse%
\ {\isachardoublequoteopen}Con\ {\isacharparenleft}G\ \isactrlbold {\isasymand}\ H{\isacharparenright}\ G\ H{\isachardoublequoteclose}\isanewline
\ \ \ \ \ \ \isacommand{by}\isamarkupfalse%
\ {\isacharparenleft}simp\ only{\isacharcolon}\ Con{\isachardot}intros{\isacharparenleft}{\isadigit{1}}{\isacharparenright}{\isacharparenright}\isanewline
\ \ \ \ \isacommand{let}\isamarkupfalse%
\ {\isacharquery}F{\isacharequal}{\isachardoublequoteopen}G\ \isactrlbold {\isasymand}\ H{\isachardoublequoteclose}\isanewline
\ \ \ \ \isacommand{have}\isamarkupfalse%
\ {\isachardoublequoteopen}Con\ {\isacharquery}F\ G\ H\ {\isasymlongrightarrow}\ {\isacharquery}F\ {\isasymin}\ S\ {\isasymlongrightarrow}\ {\isacharbraceleft}G{\isacharcomma}H{\isacharbraceright}\ {\isasymunion}\ S\ {\isasymin}\ C{\isachardoublequoteclose}\isanewline
\ \ \ \ \ \ \isacommand{using}\isamarkupfalse%
\ assms\ \isacommand{by}\isamarkupfalse%
\ {\isacharparenleft}iprover\ elim{\isacharcolon}\ allE{\isacharparenright}\isanewline
\ \ \ \ \isacommand{then}\isamarkupfalse%
\ \isacommand{have}\isamarkupfalse%
\ {\isachardoublequoteopen}{\isacharquery}F\ {\isasymin}\ S\ {\isasymlongrightarrow}\ {\isacharbraceleft}G{\isacharcomma}H{\isacharbraceright}\ {\isasymunion}\ S\ {\isasymin}\ C{\isachardoublequoteclose}\isanewline
\ \ \ \ \ \ \isacommand{using}\isamarkupfalse%
\ {\isacartoucheopen}Con\ {\isacharparenleft}G\ \isactrlbold {\isasymand}\ H{\isacharparenright}\ G\ H{\isacartoucheclose}\ \isacommand{by}\isamarkupfalse%
\ {\isacharparenleft}rule\ mp{\isacharparenright}\isanewline
\ \ \ \ \isacommand{thus}\isamarkupfalse%
\ {\isachardoublequoteopen}{\isacharbraceleft}G{\isacharcomma}H{\isacharbraceright}\ {\isasymunion}\ S\ {\isasymin}\ C{\isachardoublequoteclose}\isanewline
\ \ \ \ \ \ \isacommand{using}\isamarkupfalse%
\ {\isacartoucheopen}{\isacharparenleft}G\ \isactrlbold {\isasymand}\ H{\isacharparenright}\ {\isasymin}\ S{\isacartoucheclose}\ \isacommand{by}\isamarkupfalse%
\ {\isacharparenleft}rule\ mp{\isacharparenright}\isanewline
\ \ \isacommand{qed}\isamarkupfalse%
\isanewline
\isacommand{qed}\isamarkupfalse%
%
\endisatagproof
{\isafoldproof}%
%
\isadelimproof
\isanewline
%
\endisadelimproof
\isanewline
\isacommand{lemma}\isamarkupfalse%
\ pcp{\isacharunderscore}alt{\isadigit{2}}Con{\isadigit{2}}{\isacharcolon}\isanewline
\ \ \isakeyword{assumes}\ {\isachardoublequoteopen}{\isasymforall}F\ G\ H{\isachardot}\ Con\ F\ G\ H\ {\isasymlongrightarrow}\ F\ {\isasymin}\ S\ {\isasymlongrightarrow}\ {\isacharbraceleft}G{\isacharcomma}H{\isacharbraceright}\ {\isasymunion}\ S\ {\isasymin}\ C{\isachardoublequoteclose}\isanewline
\ \ \isakeyword{shows}\ {\isachardoublequoteopen}{\isasymforall}G{\isachardot}\ \isactrlbold {\isasymnot}\ {\isacharparenleft}\isactrlbold {\isasymnot}G{\isacharparenright}\ {\isasymin}\ S\ {\isasymlongrightarrow}\ {\isacharbraceleft}G{\isacharbraceright}\ {\isasymunion}\ S\ {\isasymin}\ C{\isachardoublequoteclose}\isanewline
%
\isadelimproof
%
\endisadelimproof
%
\isatagproof
\isacommand{proof}\isamarkupfalse%
\ {\isacharparenleft}rule\ allI{\isacharparenright}\isanewline
\ \ \isacommand{fix}\isamarkupfalse%
\ G\ \isanewline
\ \ \isacommand{show}\isamarkupfalse%
\ {\isachardoublequoteopen}\isactrlbold {\isasymnot}\ {\isacharparenleft}\isactrlbold {\isasymnot}G{\isacharparenright}\ {\isasymin}\ S\ {\isasymlongrightarrow}\ {\isacharbraceleft}G{\isacharbraceright}\ {\isasymunion}\ S\ {\isasymin}\ C{\isachardoublequoteclose}\isanewline
\ \ \isacommand{proof}\isamarkupfalse%
\ {\isacharparenleft}rule\ impI{\isacharparenright}\isanewline
\ \ \ \ \isacommand{assume}\isamarkupfalse%
\ {\isachardoublequoteopen}\isactrlbold {\isasymnot}{\isacharparenleft}\isactrlbold {\isasymnot}G{\isacharparenright}\ {\isasymin}\ S{\isachardoublequoteclose}\isanewline
\ \ \ \ \isacommand{then}\isamarkupfalse%
\ \isacommand{have}\isamarkupfalse%
\ {\isachardoublequoteopen}Con\ {\isacharparenleft}\isactrlbold {\isasymnot}{\isacharparenleft}\isactrlbold {\isasymnot}G{\isacharparenright}{\isacharparenright}\ G\ G{\isachardoublequoteclose}\isanewline
\ \ \ \ \ \ \isacommand{by}\isamarkupfalse%
\ {\isacharparenleft}simp\ only{\isacharcolon}\ Con{\isachardot}intros{\isacharparenleft}{\isadigit{4}}{\isacharparenright}{\isacharparenright}\isanewline
\ \ \ \ \isacommand{let}\isamarkupfalse%
\ {\isacharquery}F{\isacharequal}{\isachardoublequoteopen}\isactrlbold {\isasymnot}{\isacharparenleft}\isactrlbold {\isasymnot}\ G{\isacharparenright}{\isachardoublequoteclose}\isanewline
\ \ \ \ \isacommand{have}\isamarkupfalse%
\ {\isachardoublequoteopen}{\isasymforall}G\ H{\isachardot}\ Con\ {\isacharquery}F\ G\ H\ {\isasymlongrightarrow}\ {\isacharquery}F\ {\isasymin}\ S\ {\isasymlongrightarrow}\ {\isacharbraceleft}G{\isacharcomma}H{\isacharbraceright}\ {\isasymunion}\ S\ {\isasymin}\ C{\isachardoublequoteclose}\isanewline
\ \ \ \ \ \ \isacommand{using}\isamarkupfalse%
\ assms\ \isacommand{by}\isamarkupfalse%
\ {\isacharparenleft}rule\ allE{\isacharparenright}\isanewline
\ \ \ \ \isacommand{then}\isamarkupfalse%
\ \isacommand{have}\isamarkupfalse%
\ {\isachardoublequoteopen}{\isasymforall}H{\isachardot}\ Con\ {\isacharquery}F\ G\ H\ {\isasymlongrightarrow}\ {\isacharquery}F\ {\isasymin}\ S\ {\isasymlongrightarrow}\ {\isacharbraceleft}G{\isacharcomma}H{\isacharbraceright}\ {\isasymunion}\ S\ {\isasymin}\ C{\isachardoublequoteclose}\isanewline
\ \ \ \ \ \ \isacommand{by}\isamarkupfalse%
\ {\isacharparenleft}rule\ allE{\isacharparenright}\isanewline
\ \ \ \ \isacommand{then}\isamarkupfalse%
\ \isacommand{have}\isamarkupfalse%
\ {\isachardoublequoteopen}Con\ {\isacharquery}F\ G\ G\ {\isasymlongrightarrow}\ {\isacharquery}F\ {\isasymin}\ S\ {\isasymlongrightarrow}\ {\isacharbraceleft}G{\isacharcomma}G{\isacharbraceright}\ {\isasymunion}\ S\ {\isasymin}\ C{\isachardoublequoteclose}\isanewline
\ \ \ \ \ \ \isacommand{by}\isamarkupfalse%
\ {\isacharparenleft}rule\ allE{\isacharparenright}\isanewline
\ \ \ \ \isacommand{then}\isamarkupfalse%
\ \isacommand{have}\isamarkupfalse%
\ {\isachardoublequoteopen}{\isacharquery}F\ {\isasymin}\ S\ {\isasymlongrightarrow}\ {\isacharbraceleft}G{\isacharcomma}G{\isacharbraceright}\ {\isasymunion}\ S\ {\isasymin}\ C{\isachardoublequoteclose}\isanewline
\ \ \ \ \ \ \isacommand{using}\isamarkupfalse%
\ {\isacartoucheopen}Con\ {\isacharparenleft}\isactrlbold {\isasymnot}{\isacharparenleft}\isactrlbold {\isasymnot}G{\isacharparenright}{\isacharparenright}\ G\ G{\isacartoucheclose}\ \isacommand{by}\isamarkupfalse%
\ {\isacharparenleft}rule\ mp{\isacharparenright}\isanewline
\ \ \ \ \isacommand{then}\isamarkupfalse%
\ \isacommand{have}\isamarkupfalse%
\ {\isachardoublequoteopen}{\isacharbraceleft}G{\isacharcomma}G{\isacharbraceright}\ {\isasymunion}\ S\ {\isasymin}\ C{\isachardoublequoteclose}\isanewline
\ \ \ \ \ \ \isacommand{using}\isamarkupfalse%
\ {\isacartoucheopen}{\isacharparenleft}\isactrlbold {\isasymnot}{\isacharparenleft}\isactrlbold {\isasymnot}G{\isacharparenright}{\isacharparenright}\ {\isasymin}\ S{\isacartoucheclose}\ \isacommand{by}\isamarkupfalse%
\ {\isacharparenleft}rule\ mp{\isacharparenright}\isanewline
\ \ \ \ \isacommand{thus}\isamarkupfalse%
\ {\isachardoublequoteopen}{\isacharbraceleft}G{\isacharbraceright}\ {\isasymunion}\ S\ {\isasymin}\ C{\isachardoublequoteclose}\isanewline
\ \ \ \ \ \ \isacommand{by}\isamarkupfalse%
\ {\isacharparenleft}simp\ only{\isacharcolon}\ insert{\isacharunderscore}absorb{\isadigit{2}}{\isacharparenright}\isanewline
\ \ \isacommand{qed}\isamarkupfalse%
\isanewline
\isacommand{qed}\isamarkupfalse%
%
\endisatagproof
{\isafoldproof}%
%
\isadelimproof
\isanewline
%
\endisadelimproof
\isanewline
\isacommand{lemma}\isamarkupfalse%
\ pcp{\isacharunderscore}alt{\isadigit{2}}Con{\isadigit{3}}{\isacharcolon}\isanewline
\ \ \isakeyword{assumes}\ {\isachardoublequoteopen}{\isasymforall}F\ G\ H{\isachardot}\ Con\ F\ G\ H\ {\isasymlongrightarrow}\ F\ {\isasymin}\ S\ {\isasymlongrightarrow}\ {\isacharbraceleft}G{\isacharcomma}H{\isacharbraceright}\ {\isasymunion}\ S\ {\isasymin}\ C{\isachardoublequoteclose}\isanewline
\ \ \isakeyword{shows}\ {\isachardoublequoteopen}{\isasymforall}G\ H{\isachardot}\ \isactrlbold {\isasymnot}{\isacharparenleft}G\ \isactrlbold {\isasymor}\ H{\isacharparenright}\ {\isasymin}\ S\ {\isasymlongrightarrow}\ {\isacharbraceleft}\isactrlbold {\isasymnot}\ G{\isacharcomma}\ \isactrlbold {\isasymnot}\ H{\isacharbraceright}\ {\isasymunion}\ S\ {\isasymin}\ C{\isachardoublequoteclose}\isanewline
%
\isadelimproof
%
\endisadelimproof
%
\isatagproof
\isacommand{proof}\isamarkupfalse%
\ {\isacharparenleft}rule\ allI{\isacharparenright}{\isacharplus}\isanewline
\ \ \isacommand{fix}\isamarkupfalse%
\ G\ H\isanewline
\ \ \isacommand{show}\isamarkupfalse%
\ {\isachardoublequoteopen}\isactrlbold {\isasymnot}{\isacharparenleft}G\ \isactrlbold {\isasymor}\ H{\isacharparenright}\ {\isasymin}\ S\ {\isasymlongrightarrow}\ {\isacharbraceleft}\isactrlbold {\isasymnot}\ G{\isacharcomma}\ \isactrlbold {\isasymnot}\ H{\isacharbraceright}\ {\isasymunion}\ S\ {\isasymin}\ C{\isachardoublequoteclose}\isanewline
\ \ \isacommand{proof}\isamarkupfalse%
\ {\isacharparenleft}rule\ impI{\isacharparenright}\isanewline
\ \ \ \ \isacommand{assume}\isamarkupfalse%
\ {\isachardoublequoteopen}\isactrlbold {\isasymnot}{\isacharparenleft}G\ \isactrlbold {\isasymor}\ H{\isacharparenright}\ {\isasymin}\ S{\isachardoublequoteclose}\isanewline
\ \ \ \ \isacommand{then}\isamarkupfalse%
\ \isacommand{have}\isamarkupfalse%
\ {\isachardoublequoteopen}Con\ {\isacharparenleft}\isactrlbold {\isasymnot}{\isacharparenleft}G\ \isactrlbold {\isasymor}\ H{\isacharparenright}{\isacharparenright}\ {\isacharparenleft}\isactrlbold {\isasymnot}G{\isacharparenright}\ {\isacharparenleft}\isactrlbold {\isasymnot}H{\isacharparenright}{\isachardoublequoteclose}\isanewline
\ \ \ \ \ \ \isacommand{by}\isamarkupfalse%
\ {\isacharparenleft}simp\ only{\isacharcolon}\ Con{\isachardot}intros{\isacharparenleft}{\isadigit{2}}{\isacharparenright}{\isacharparenright}\isanewline
\ \ \ \ \isacommand{let}\isamarkupfalse%
\ {\isacharquery}F\ {\isacharequal}\ {\isachardoublequoteopen}\isactrlbold {\isasymnot}{\isacharparenleft}G\ \isactrlbold {\isasymor}\ H{\isacharparenright}{\isachardoublequoteclose}\isanewline
\ \ \ \ \isacommand{have}\isamarkupfalse%
\ {\isachardoublequoteopen}Con\ {\isacharquery}F\ {\isacharparenleft}\isactrlbold {\isasymnot}G{\isacharparenright}\ {\isacharparenleft}\isactrlbold {\isasymnot}H{\isacharparenright}\ {\isasymlongrightarrow}\ {\isacharquery}F\ {\isasymin}\ S\ {\isasymlongrightarrow}\ {\isacharbraceleft}\isactrlbold {\isasymnot}G{\isacharcomma}\isactrlbold {\isasymnot}H{\isacharbraceright}\ {\isasymunion}\ S\ {\isasymin}\ C{\isachardoublequoteclose}\isanewline
\ \ \ \ \ \ \isacommand{using}\isamarkupfalse%
\ assms\ \isacommand{by}\isamarkupfalse%
\ {\isacharparenleft}iprover\ elim{\isacharcolon}\ allE{\isacharparenright}\isanewline
\ \ \ \ \isacommand{then}\isamarkupfalse%
\ \isacommand{have}\isamarkupfalse%
\ {\isachardoublequoteopen}{\isacharquery}F\ {\isasymin}\ S\ {\isasymlongrightarrow}\ {\isacharbraceleft}\isactrlbold {\isasymnot}G{\isacharcomma}\isactrlbold {\isasymnot}H{\isacharbraceright}\ {\isasymunion}\ S\ {\isasymin}\ C{\isachardoublequoteclose}\isanewline
\ \ \ \ \ \ \isacommand{using}\isamarkupfalse%
\ {\isacartoucheopen}Con\ {\isacharparenleft}\isactrlbold {\isasymnot}{\isacharparenleft}G\ \isactrlbold {\isasymor}\ H{\isacharparenright}{\isacharparenright}\ {\isacharparenleft}\isactrlbold {\isasymnot}G{\isacharparenright}\ {\isacharparenleft}\isactrlbold {\isasymnot}H{\isacharparenright}{\isacartoucheclose}\ \isacommand{by}\isamarkupfalse%
\ {\isacharparenleft}rule\ mp{\isacharparenright}\isanewline
\ \ \ \ \isacommand{thus}\isamarkupfalse%
\ {\isachardoublequoteopen}{\isacharbraceleft}\isactrlbold {\isasymnot}G{\isacharcomma}\isactrlbold {\isasymnot}H{\isacharbraceright}\ {\isasymunion}\ S\ {\isasymin}\ C{\isachardoublequoteclose}\isanewline
\ \ \ \ \ \ \isacommand{using}\isamarkupfalse%
\ {\isacartoucheopen}\isactrlbold {\isasymnot}{\isacharparenleft}G\ \isactrlbold {\isasymor}\ H{\isacharparenright}\ {\isasymin}\ S{\isacartoucheclose}\ \isacommand{by}\isamarkupfalse%
\ {\isacharparenleft}rule\ mp{\isacharparenright}\isanewline
\ \ \isacommand{qed}\isamarkupfalse%
\isanewline
\isacommand{qed}\isamarkupfalse%
%
\endisatagproof
{\isafoldproof}%
%
\isadelimproof
\isanewline
%
\endisadelimproof
\isanewline
\isacommand{lemma}\isamarkupfalse%
\ pcp{\isacharunderscore}alt{\isadigit{2}}Con{\isadigit{4}}{\isacharcolon}\isanewline
\ \ \isakeyword{assumes}\ {\isachardoublequoteopen}{\isasymforall}F\ G\ H{\isachardot}\ Con\ F\ G\ H\ {\isasymlongrightarrow}\ F\ {\isasymin}\ S\ {\isasymlongrightarrow}\ {\isacharbraceleft}G{\isacharcomma}H{\isacharbraceright}\ {\isasymunion}\ S\ {\isasymin}\ C{\isachardoublequoteclose}\isanewline
\ \ \isakeyword{shows}\ {\isachardoublequoteopen}{\isasymforall}G\ H{\isachardot}\ \isactrlbold {\isasymnot}{\isacharparenleft}G\ \isactrlbold {\isasymrightarrow}\ H{\isacharparenright}\ {\isasymin}\ S\ {\isasymlongrightarrow}\ {\isacharbraceleft}G{\isacharcomma}\isactrlbold {\isasymnot}\ H{\isacharbraceright}\ {\isasymunion}\ S\ {\isasymin}\ C{\isachardoublequoteclose}\isanewline
%
\isadelimproof
%
\endisadelimproof
%
\isatagproof
\isacommand{proof}\isamarkupfalse%
\ {\isacharparenleft}rule\ allI{\isacharparenright}{\isacharplus}\isanewline
\ \ \isacommand{fix}\isamarkupfalse%
\ G\ H\isanewline
\ \ \isacommand{show}\isamarkupfalse%
\ {\isachardoublequoteopen}\isactrlbold {\isasymnot}{\isacharparenleft}G\ \isactrlbold {\isasymrightarrow}\ H{\isacharparenright}\ {\isasymin}\ S\ {\isasymlongrightarrow}\ {\isacharbraceleft}G{\isacharcomma}\isactrlbold {\isasymnot}\ H{\isacharbraceright}\ {\isasymunion}\ S\ {\isasymin}\ C{\isachardoublequoteclose}\isanewline
\ \ \isacommand{proof}\isamarkupfalse%
\ {\isacharparenleft}rule\ impI{\isacharparenright}\isanewline
\ \ \ \ \isacommand{assume}\isamarkupfalse%
\ {\isachardoublequoteopen}\isactrlbold {\isasymnot}{\isacharparenleft}G\ \isactrlbold {\isasymrightarrow}\ H{\isacharparenright}\ {\isasymin}\ S{\isachardoublequoteclose}\isanewline
\ \ \ \ \isacommand{then}\isamarkupfalse%
\ \isacommand{have}\isamarkupfalse%
\ {\isachardoublequoteopen}Con\ {\isacharparenleft}\isactrlbold {\isasymnot}{\isacharparenleft}G\ \isactrlbold {\isasymrightarrow}\ H{\isacharparenright}{\isacharparenright}\ G\ {\isacharparenleft}\isactrlbold {\isasymnot}H{\isacharparenright}{\isachardoublequoteclose}\isanewline
\ \ \ \ \ \ \isacommand{by}\isamarkupfalse%
\ {\isacharparenleft}simp\ only{\isacharcolon}\ Con{\isachardot}intros{\isacharparenleft}{\isadigit{3}}{\isacharparenright}{\isacharparenright}\isanewline
\ \ \ \ \isacommand{let}\isamarkupfalse%
\ {\isacharquery}F\ {\isacharequal}\ {\isachardoublequoteopen}\isactrlbold {\isasymnot}{\isacharparenleft}G\ \isactrlbold {\isasymrightarrow}\ H{\isacharparenright}{\isachardoublequoteclose}\isanewline
\ \ \ \ \isacommand{have}\isamarkupfalse%
\ {\isachardoublequoteopen}Con\ {\isacharquery}F\ G\ {\isacharparenleft}\isactrlbold {\isasymnot}H{\isacharparenright}\ {\isasymlongrightarrow}\ {\isacharquery}F\ {\isasymin}\ S\ {\isasymlongrightarrow}\ {\isacharbraceleft}G{\isacharcomma}\isactrlbold {\isasymnot}H{\isacharbraceright}\ {\isasymunion}\ S\ {\isasymin}\ C{\isachardoublequoteclose}\isanewline
\ \ \ \ \ \ \isacommand{using}\isamarkupfalse%
\ assms\ \isacommand{by}\isamarkupfalse%
\ {\isacharparenleft}iprover\ elim{\isacharcolon}\ allE{\isacharparenright}\isanewline
\ \ \ \ \isacommand{then}\isamarkupfalse%
\ \isacommand{have}\isamarkupfalse%
\ {\isachardoublequoteopen}{\isacharquery}F\ {\isasymin}\ S\ {\isasymlongrightarrow}\ {\isacharbraceleft}G{\isacharcomma}\isactrlbold {\isasymnot}H{\isacharbraceright}\ {\isasymunion}\ S\ {\isasymin}\ C{\isachardoublequoteclose}\ \ \isanewline
\ \ \ \ \ \ \isacommand{using}\isamarkupfalse%
\ {\isacartoucheopen}Con\ {\isacharparenleft}\isactrlbold {\isasymnot}{\isacharparenleft}G\ \isactrlbold {\isasymrightarrow}\ H{\isacharparenright}{\isacharparenright}\ G\ {\isacharparenleft}\isactrlbold {\isasymnot}H{\isacharparenright}{\isacartoucheclose}\ \isacommand{by}\isamarkupfalse%
\ {\isacharparenleft}rule\ mp{\isacharparenright}\isanewline
\ \ \ \ \isacommand{thus}\isamarkupfalse%
\ {\isachardoublequoteopen}{\isacharbraceleft}G{\isacharcomma}\isactrlbold {\isasymnot}H{\isacharbraceright}\ {\isasymunion}\ S\ {\isasymin}\ C{\isachardoublequoteclose}\isanewline
\ \ \ \ \ \ \isacommand{using}\isamarkupfalse%
\ {\isacartoucheopen}\isactrlbold {\isasymnot}{\isacharparenleft}G\ \isactrlbold {\isasymrightarrow}\ H{\isacharparenright}\ {\isasymin}\ S{\isacartoucheclose}\ \isacommand{by}\isamarkupfalse%
\ {\isacharparenleft}rule\ mp{\isacharparenright}\isanewline
\ \ \isacommand{qed}\isamarkupfalse%
\isanewline
\isacommand{qed}\isamarkupfalse%
%
\endisatagproof
{\isafoldproof}%
%
\isadelimproof
%
\endisadelimproof
%
\begin{isamarkuptext}%
Por otro lado, los siguientes lemas auxiliares prueban el resto de condiciones de la
  definición de propiedad de consistencia proposicional a partir de la hipótesis referente a 
  fórmulas de tipo \isa{{\isasymbeta}}.%
\end{isamarkuptext}\isamarkuptrue%
\isacommand{lemma}\isamarkupfalse%
\ pcp{\isacharunderscore}alt{\isadigit{2}}Dis{\isadigit{1}}{\isacharcolon}\isanewline
\ \ \isakeyword{assumes}\ {\isachardoublequoteopen}{\isasymforall}F\ G\ H{\isachardot}\ Dis\ F\ G\ H\ {\isasymlongrightarrow}\ F\ {\isasymin}\ S\ {\isasymlongrightarrow}\ {\isacharbraceleft}G{\isacharbraceright}\ {\isasymunion}\ S\ {\isasymin}\ C\ {\isasymor}\ {\isacharbraceleft}H{\isacharbraceright}\ {\isasymunion}\ S\ {\isasymin}\ C{\isachardoublequoteclose}\isanewline
\ \ \isakeyword{shows}\ {\isachardoublequoteopen}{\isasymforall}G\ H{\isachardot}\ G\ \isactrlbold {\isasymor}\ H\ {\isasymin}\ S\ {\isasymlongrightarrow}\ {\isacharbraceleft}G{\isacharbraceright}\ {\isasymunion}\ S\ {\isasymin}\ C\ {\isasymor}\ {\isacharbraceleft}H{\isacharbraceright}\ {\isasymunion}\ S\ {\isasymin}\ C{\isachardoublequoteclose}\isanewline
%
\isadelimproof
%
\endisadelimproof
%
\isatagproof
\isacommand{proof}\isamarkupfalse%
\ {\isacharparenleft}rule\ allI{\isacharparenright}{\isacharplus}\isanewline
\ \ \isacommand{fix}\isamarkupfalse%
\ G\ H\isanewline
\ \ \isacommand{show}\isamarkupfalse%
\ {\isachardoublequoteopen}G\ \isactrlbold {\isasymor}\ H\ {\isasymin}\ S\ {\isasymlongrightarrow}\ {\isacharbraceleft}G{\isacharbraceright}\ {\isasymunion}\ S\ {\isasymin}\ C\ {\isasymor}\ {\isacharbraceleft}H{\isacharbraceright}\ {\isasymunion}\ S\ {\isasymin}\ C{\isachardoublequoteclose}\isanewline
\ \ \isacommand{proof}\isamarkupfalse%
\ {\isacharparenleft}rule\ impI{\isacharparenright}\isanewline
\ \ \ \ \isacommand{assume}\isamarkupfalse%
\ {\isachardoublequoteopen}G\ \isactrlbold {\isasymor}\ H\ {\isasymin}\ S{\isachardoublequoteclose}\isanewline
\ \ \ \ \isacommand{then}\isamarkupfalse%
\ \isacommand{have}\isamarkupfalse%
\ {\isachardoublequoteopen}Dis\ {\isacharparenleft}G\ \isactrlbold {\isasymor}\ H{\isacharparenright}\ G\ H{\isachardoublequoteclose}\isanewline
\ \ \ \ \ \ \isacommand{by}\isamarkupfalse%
\ {\isacharparenleft}simp\ only{\isacharcolon}\ Dis{\isachardot}intros{\isacharparenleft}{\isadigit{1}}{\isacharparenright}{\isacharparenright}\isanewline
\ \ \ \ \isacommand{let}\isamarkupfalse%
\ {\isacharquery}F{\isacharequal}{\isachardoublequoteopen}G\ \isactrlbold {\isasymor}\ H{\isachardoublequoteclose}\isanewline
\ \ \ \ \isacommand{have}\isamarkupfalse%
\ {\isachardoublequoteopen}Dis\ {\isacharquery}F\ G\ H\ {\isasymlongrightarrow}\ {\isacharquery}F\ {\isasymin}\ S\ {\isasymlongrightarrow}\ {\isacharbraceleft}G{\isacharbraceright}\ {\isasymunion}\ S\ {\isasymin}\ C\ {\isasymor}\ {\isacharbraceleft}H{\isacharbraceright}\ {\isasymunion}\ S\ {\isasymin}\ C{\isachardoublequoteclose}\isanewline
\ \ \ \ \ \ \isacommand{using}\isamarkupfalse%
\ assms\ \isacommand{by}\isamarkupfalse%
\ {\isacharparenleft}iprover\ elim{\isacharcolon}\ allE{\isacharparenright}\isanewline
\ \ \ \ \isacommand{then}\isamarkupfalse%
\ \isacommand{have}\isamarkupfalse%
\ {\isachardoublequoteopen}{\isacharquery}F\ {\isasymin}\ S\ {\isasymlongrightarrow}\ {\isacharbraceleft}G{\isacharbraceright}\ {\isasymunion}\ S\ {\isasymin}\ C\ {\isasymor}\ {\isacharbraceleft}H{\isacharbraceright}\ {\isasymunion}\ S\ {\isasymin}\ C{\isachardoublequoteclose}\isanewline
\ \ \ \ \ \ \isacommand{using}\isamarkupfalse%
\ {\isacartoucheopen}Dis\ {\isacharparenleft}G\ \isactrlbold {\isasymor}\ H{\isacharparenright}\ G\ H{\isacartoucheclose}\ \isacommand{by}\isamarkupfalse%
\ {\isacharparenleft}rule\ mp{\isacharparenright}\isanewline
\ \ \ \ \isacommand{thus}\isamarkupfalse%
\ {\isachardoublequoteopen}{\isacharbraceleft}G{\isacharbraceright}\ {\isasymunion}\ S\ {\isasymin}\ C\ {\isasymor}\ {\isacharbraceleft}H{\isacharbraceright}\ {\isasymunion}\ S\ {\isasymin}\ C{\isachardoublequoteclose}\isanewline
\ \ \ \ \ \ \isacommand{using}\isamarkupfalse%
\ {\isacartoucheopen}{\isacharparenleft}G\ \isactrlbold {\isasymor}\ H{\isacharparenright}\ {\isasymin}\ S{\isacartoucheclose}\ \isacommand{by}\isamarkupfalse%
\ {\isacharparenleft}rule\ mp{\isacharparenright}\isanewline
\ \ \isacommand{qed}\isamarkupfalse%
\isanewline
\isacommand{qed}\isamarkupfalse%
%
\endisatagproof
{\isafoldproof}%
%
\isadelimproof
\isanewline
%
\endisadelimproof
\isanewline
\isacommand{lemma}\isamarkupfalse%
\ pcp{\isacharunderscore}alt{\isadigit{2}}Dis{\isadigit{2}}{\isacharcolon}\isanewline
\ \ \isakeyword{assumes}\ {\isachardoublequoteopen}{\isasymforall}F\ G\ H{\isachardot}\ Dis\ F\ G\ H\ {\isasymlongrightarrow}\ F\ {\isasymin}\ S\ {\isasymlongrightarrow}\ {\isacharbraceleft}G{\isacharbraceright}\ {\isasymunion}\ S\ {\isasymin}\ C\ {\isasymor}\ {\isacharbraceleft}H{\isacharbraceright}\ {\isasymunion}\ S\ {\isasymin}\ C{\isachardoublequoteclose}\isanewline
\ \ \isakeyword{shows}\ {\isachardoublequoteopen}{\isasymforall}G\ H{\isachardot}\ G\ \isactrlbold {\isasymrightarrow}\ H\ {\isasymin}\ S\ {\isasymlongrightarrow}\ {\isacharbraceleft}\isactrlbold {\isasymnot}\ G{\isacharbraceright}\ {\isasymunion}\ S\ {\isasymin}\ C\ {\isasymor}\ {\isacharbraceleft}H{\isacharbraceright}\ {\isasymunion}\ S\ {\isasymin}\ C{\isachardoublequoteclose}\isanewline
%
\isadelimproof
%
\endisadelimproof
%
\isatagproof
\isacommand{proof}\isamarkupfalse%
\ {\isacharparenleft}rule\ allI{\isacharparenright}{\isacharplus}\isanewline
\ \ \isacommand{fix}\isamarkupfalse%
\ G\ H\isanewline
\ \ \isacommand{show}\isamarkupfalse%
\ {\isachardoublequoteopen}G\ \isactrlbold {\isasymrightarrow}\ H\ {\isasymin}\ S\ {\isasymlongrightarrow}\ {\isacharbraceleft}\isactrlbold {\isasymnot}\ G{\isacharbraceright}\ {\isasymunion}\ S\ {\isasymin}\ C\ {\isasymor}\ {\isacharbraceleft}H{\isacharbraceright}\ {\isasymunion}\ S\ {\isasymin}\ C{\isachardoublequoteclose}\isanewline
\ \ \isacommand{proof}\isamarkupfalse%
\ {\isacharparenleft}rule\ impI{\isacharparenright}\isanewline
\ \ \ \ \isacommand{assume}\isamarkupfalse%
\ {\isachardoublequoteopen}G\ \isactrlbold {\isasymrightarrow}\ H\ {\isasymin}\ S{\isachardoublequoteclose}\isanewline
\ \ \ \ \isacommand{then}\isamarkupfalse%
\ \isacommand{have}\isamarkupfalse%
\ {\isachardoublequoteopen}Dis\ {\isacharparenleft}G\ \isactrlbold {\isasymrightarrow}\ H{\isacharparenright}\ {\isacharparenleft}\isactrlbold {\isasymnot}G{\isacharparenright}\ H{\isachardoublequoteclose}\isanewline
\ \ \ \ \ \ \isacommand{by}\isamarkupfalse%
\ {\isacharparenleft}simp\ only{\isacharcolon}\ Dis{\isachardot}intros{\isacharparenleft}{\isadigit{2}}{\isacharparenright}{\isacharparenright}\isanewline
\ \ \ \ \isacommand{let}\isamarkupfalse%
\ {\isacharquery}F{\isacharequal}{\isachardoublequoteopen}G\ \isactrlbold {\isasymrightarrow}\ H{\isachardoublequoteclose}\ \isanewline
\ \ \ \ \isacommand{have}\isamarkupfalse%
\ {\isachardoublequoteopen}Dis\ {\isacharquery}F\ {\isacharparenleft}\isactrlbold {\isasymnot}G{\isacharparenright}\ H\ {\isasymlongrightarrow}\ {\isacharquery}F\ {\isasymin}\ S\ {\isasymlongrightarrow}\ {\isacharbraceleft}\isactrlbold {\isasymnot}G{\isacharbraceright}\ {\isasymunion}\ S\ {\isasymin}\ C\ {\isasymor}\ {\isacharbraceleft}H{\isacharbraceright}\ {\isasymunion}\ S\ {\isasymin}\ C{\isachardoublequoteclose}\isanewline
\ \ \ \ \ \ \isacommand{using}\isamarkupfalse%
\ assms\ \isacommand{by}\isamarkupfalse%
\ {\isacharparenleft}iprover\ elim{\isacharcolon}\ allE{\isacharparenright}\isanewline
\ \ \ \ \isacommand{then}\isamarkupfalse%
\ \isacommand{have}\isamarkupfalse%
\ {\isachardoublequoteopen}{\isacharquery}F\ {\isasymin}\ S\ {\isasymlongrightarrow}\ {\isacharbraceleft}\isactrlbold {\isasymnot}G{\isacharbraceright}\ {\isasymunion}\ S\ {\isasymin}\ C\ {\isasymor}\ {\isacharbraceleft}H{\isacharbraceright}\ {\isasymunion}\ S\ {\isasymin}\ C{\isachardoublequoteclose}\isanewline
\ \ \ \ \ \ \isacommand{using}\isamarkupfalse%
\ {\isacartoucheopen}Dis\ {\isacharparenleft}G\ \isactrlbold {\isasymrightarrow}\ H{\isacharparenright}\ {\isacharparenleft}\isactrlbold {\isasymnot}G{\isacharparenright}\ H{\isacartoucheclose}\ \isacommand{by}\isamarkupfalse%
\ {\isacharparenleft}rule\ mp{\isacharparenright}\isanewline
\ \ \ \ \isacommand{thus}\isamarkupfalse%
\ {\isachardoublequoteopen}{\isacharbraceleft}\isactrlbold {\isasymnot}G{\isacharbraceright}\ {\isasymunion}\ S\ {\isasymin}\ C\ {\isasymor}\ {\isacharbraceleft}H{\isacharbraceright}\ {\isasymunion}\ S\ {\isasymin}\ C{\isachardoublequoteclose}\isanewline
\ \ \ \ \ \ \isacommand{using}\isamarkupfalse%
\ {\isacartoucheopen}{\isacharparenleft}G\ \isactrlbold {\isasymrightarrow}\ H{\isacharparenright}\ {\isasymin}\ S{\isacartoucheclose}\ \isacommand{by}\isamarkupfalse%
\ {\isacharparenleft}rule\ mp{\isacharparenright}\isanewline
\ \ \isacommand{qed}\isamarkupfalse%
\isanewline
\isacommand{qed}\isamarkupfalse%
%
\endisatagproof
{\isafoldproof}%
%
\isadelimproof
\isanewline
%
\endisadelimproof
\isanewline
\isacommand{lemma}\isamarkupfalse%
\ pcp{\isacharunderscore}alt{\isadigit{2}}Dis{\isadigit{3}}{\isacharcolon}\isanewline
\ \ \isakeyword{assumes}\ {\isachardoublequoteopen}{\isasymforall}F\ G\ H{\isachardot}\ Dis\ F\ G\ H\ {\isasymlongrightarrow}\ F\ {\isasymin}\ S\ {\isasymlongrightarrow}\ {\isacharbraceleft}G{\isacharbraceright}\ {\isasymunion}\ S\ {\isasymin}\ C\ {\isasymor}\ {\isacharbraceleft}H{\isacharbraceright}\ {\isasymunion}\ S\ {\isasymin}\ C{\isachardoublequoteclose}\isanewline
\ \ \isakeyword{shows}\ {\isachardoublequoteopen}{\isasymforall}G\ H{\isachardot}\ \isactrlbold {\isasymnot}{\isacharparenleft}G\ \isactrlbold {\isasymand}\ H{\isacharparenright}\ {\isasymin}\ S\ {\isasymlongrightarrow}\ {\isacharbraceleft}\isactrlbold {\isasymnot}\ G{\isacharbraceright}\ {\isasymunion}\ S\ {\isasymin}\ C\ {\isasymor}\ {\isacharbraceleft}\isactrlbold {\isasymnot}\ H{\isacharbraceright}\ {\isasymunion}\ S\ {\isasymin}\ C{\isachardoublequoteclose}\isanewline
%
\isadelimproof
%
\endisadelimproof
%
\isatagproof
\isacommand{proof}\isamarkupfalse%
\ {\isacharparenleft}rule\ allI{\isacharparenright}{\isacharplus}\isanewline
\ \ \isacommand{fix}\isamarkupfalse%
\ G\ H\isanewline
\ \ \isacommand{show}\isamarkupfalse%
\ {\isachardoublequoteopen}\isactrlbold {\isasymnot}{\isacharparenleft}G\ \isactrlbold {\isasymand}\ H{\isacharparenright}\ {\isasymin}\ S\ {\isasymlongrightarrow}\ {\isacharbraceleft}\isactrlbold {\isasymnot}\ G{\isacharbraceright}\ {\isasymunion}\ S\ {\isasymin}\ C\ {\isasymor}\ {\isacharbraceleft}\isactrlbold {\isasymnot}\ H{\isacharbraceright}\ {\isasymunion}\ S\ {\isasymin}\ C{\isachardoublequoteclose}\isanewline
\ \ \isacommand{proof}\isamarkupfalse%
\ {\isacharparenleft}rule\ impI{\isacharparenright}\isanewline
\ \ \ \ \isacommand{assume}\isamarkupfalse%
\ {\isachardoublequoteopen}\isactrlbold {\isasymnot}{\isacharparenleft}G\ \isactrlbold {\isasymand}\ H{\isacharparenright}\ {\isasymin}\ S{\isachardoublequoteclose}\isanewline
\ \ \ \ \isacommand{then}\isamarkupfalse%
\ \isacommand{have}\isamarkupfalse%
\ {\isachardoublequoteopen}Dis\ {\isacharparenleft}\isactrlbold {\isasymnot}{\isacharparenleft}G\ \isactrlbold {\isasymand}\ H{\isacharparenright}{\isacharparenright}\ {\isacharparenleft}\isactrlbold {\isasymnot}G{\isacharparenright}\ {\isacharparenleft}\isactrlbold {\isasymnot}H{\isacharparenright}{\isachardoublequoteclose}\isanewline
\ \ \ \ \ \ \isacommand{by}\isamarkupfalse%
\ {\isacharparenleft}simp\ only{\isacharcolon}\ Dis{\isachardot}intros{\isacharparenleft}{\isadigit{3}}{\isacharparenright}{\isacharparenright}\isanewline
\ \ \ \ \isacommand{let}\isamarkupfalse%
\ {\isacharquery}F{\isacharequal}{\isachardoublequoteopen}\isactrlbold {\isasymnot}{\isacharparenleft}G\ \isactrlbold {\isasymand}\ H{\isacharparenright}{\isachardoublequoteclose}\isanewline
\ \ \ \ \isacommand{have}\isamarkupfalse%
\ {\isachardoublequoteopen}Dis\ {\isacharquery}F\ {\isacharparenleft}\isactrlbold {\isasymnot}G{\isacharparenright}\ {\isacharparenleft}\isactrlbold {\isasymnot}H{\isacharparenright}\ {\isasymlongrightarrow}\ {\isacharquery}F\ {\isasymin}\ S\ {\isasymlongrightarrow}\ {\isacharbraceleft}\isactrlbold {\isasymnot}G{\isacharbraceright}\ {\isasymunion}\ S\ {\isasymin}\ C\ {\isasymor}\ {\isacharbraceleft}\isactrlbold {\isasymnot}H{\isacharbraceright}\ {\isasymunion}\ S\ {\isasymin}\ C{\isachardoublequoteclose}\isanewline
\ \ \ \ \ \ \isacommand{using}\isamarkupfalse%
\ assms\ \isacommand{by}\isamarkupfalse%
\ {\isacharparenleft}iprover\ elim{\isacharcolon}\ allE{\isacharparenright}\isanewline
\ \ \ \ \isacommand{then}\isamarkupfalse%
\ \isacommand{have}\isamarkupfalse%
\ {\isachardoublequoteopen}{\isacharquery}F\ {\isasymin}\ S\ {\isasymlongrightarrow}\ {\isacharbraceleft}\isactrlbold {\isasymnot}G{\isacharbraceright}\ {\isasymunion}\ S\ {\isasymin}\ C\ {\isasymor}\ {\isacharbraceleft}\isactrlbold {\isasymnot}H{\isacharbraceright}\ {\isasymunion}\ S\ {\isasymin}\ C{\isachardoublequoteclose}\isanewline
\ \ \ \ \ \ \isacommand{using}\isamarkupfalse%
\ {\isacartoucheopen}Dis\ {\isacharparenleft}\isactrlbold {\isasymnot}{\isacharparenleft}G\ \isactrlbold {\isasymand}\ H{\isacharparenright}{\isacharparenright}\ {\isacharparenleft}\isactrlbold {\isasymnot}G{\isacharparenright}\ {\isacharparenleft}\isactrlbold {\isasymnot}H{\isacharparenright}{\isacartoucheclose}\ \isacommand{by}\isamarkupfalse%
\ {\isacharparenleft}rule\ mp{\isacharparenright}\isanewline
\ \ \ \ \isacommand{thus}\isamarkupfalse%
\ {\isachardoublequoteopen}{\isacharbraceleft}\isactrlbold {\isasymnot}G{\isacharbraceright}\ {\isasymunion}\ S\ {\isasymin}\ C\ {\isasymor}\ {\isacharbraceleft}\isactrlbold {\isasymnot}H{\isacharbraceright}\ {\isasymunion}\ S\ {\isasymin}\ C{\isachardoublequoteclose}\isanewline
\ \ \ \ \ \ \isacommand{using}\isamarkupfalse%
\ {\isacartoucheopen}\isactrlbold {\isasymnot}{\isacharparenleft}G\ \isactrlbold {\isasymand}\ H{\isacharparenright}\ {\isasymin}\ S{\isacartoucheclose}\ \isacommand{by}\isamarkupfalse%
\ {\isacharparenleft}rule\ mp{\isacharparenright}\isanewline
\ \ \isacommand{qed}\isamarkupfalse%
\isanewline
\isacommand{qed}\isamarkupfalse%
%
\endisatagproof
{\isafoldproof}%
%
\isadelimproof
%
\endisadelimproof
%
\begin{isamarkuptext}%
De este modo, procedemos a la demostración detallada de esta implicación en Isabelle.%
\end{isamarkuptext}\isamarkuptrue%
\isacommand{lemma}\isamarkupfalse%
\ pcp{\isacharunderscore}alt{\isadigit{2}}{\isacharcolon}\ \isanewline
\ \ \isakeyword{assumes}\ {\isachardoublequoteopen}{\isasymforall}S\ {\isasymin}\ C{\isachardot}\ {\isasymbottom}\ {\isasymnotin}\ S\isanewline
{\isasymand}\ {\isacharparenleft}{\isasymforall}k{\isachardot}\ Atom\ k\ {\isasymin}\ S\ {\isasymlongrightarrow}\ \isactrlbold {\isasymnot}\ {\isacharparenleft}Atom\ k{\isacharparenright}\ {\isasymin}\ S\ {\isasymlongrightarrow}\ False{\isacharparenright}\isanewline
{\isasymand}\ {\isacharparenleft}{\isasymforall}F\ G\ H{\isachardot}\ Con\ F\ G\ H\ {\isasymlongrightarrow}\ F\ {\isasymin}\ S\ {\isasymlongrightarrow}\ {\isacharbraceleft}G{\isacharcomma}H{\isacharbraceright}\ {\isasymunion}\ S\ {\isasymin}\ C{\isacharparenright}\isanewline
{\isasymand}\ {\isacharparenleft}{\isasymforall}F\ G\ H{\isachardot}\ Dis\ F\ G\ H\ {\isasymlongrightarrow}\ F\ {\isasymin}\ S\ {\isasymlongrightarrow}\ {\isacharbraceleft}G{\isacharbraceright}\ {\isasymunion}\ S\ {\isasymin}\ C\ {\isasymor}\ {\isacharbraceleft}H{\isacharbraceright}\ {\isasymunion}\ S\ {\isasymin}\ C{\isacharparenright}{\isachardoublequoteclose}\isanewline
\ \ \isakeyword{shows}\ {\isachardoublequoteopen}pcp\ C{\isachardoublequoteclose}\isanewline
%
\isadelimproof
\ \ %
\endisadelimproof
%
\isatagproof
\isacommand{unfolding}\isamarkupfalse%
\ pcp{\isacharunderscore}def\isanewline
\isacommand{proof}\isamarkupfalse%
\ {\isacharparenleft}rule\ ballI{\isacharparenright}\isanewline
\ \ \isacommand{fix}\isamarkupfalse%
\ S\isanewline
\ \ \isacommand{assume}\isamarkupfalse%
\ {\isachardoublequoteopen}S\ {\isasymin}\ C{\isachardoublequoteclose}\isanewline
\ \ \isacommand{have}\isamarkupfalse%
\ H{\isacharcolon}{\isachardoublequoteopen}{\isasymbottom}\ {\isasymnotin}\ S\isanewline
\ \ \ \ {\isasymand}\ {\isacharparenleft}{\isasymforall}k{\isachardot}\ Atom\ k\ {\isasymin}\ S\ {\isasymlongrightarrow}\ \isactrlbold {\isasymnot}\ {\isacharparenleft}Atom\ k{\isacharparenright}\ {\isasymin}\ S\ {\isasymlongrightarrow}\ False{\isacharparenright}\isanewline
\ \ \ \ {\isasymand}\ {\isacharparenleft}{\isasymforall}F\ G\ H{\isachardot}\ Con\ F\ G\ H\ {\isasymlongrightarrow}\ F\ {\isasymin}\ S\ {\isasymlongrightarrow}\ {\isacharbraceleft}G{\isacharcomma}H{\isacharbraceright}\ {\isasymunion}\ S\ {\isasymin}\ C{\isacharparenright}\isanewline
\ \ \ \ {\isasymand}\ {\isacharparenleft}{\isasymforall}F\ G\ H{\isachardot}\ Dis\ F\ G\ H\ {\isasymlongrightarrow}\ F\ {\isasymin}\ S\ {\isasymlongrightarrow}\ {\isacharbraceleft}G{\isacharbraceright}\ {\isasymunion}\ S\ {\isasymin}\ C\ {\isasymor}\ {\isacharbraceleft}H{\isacharbraceright}\ {\isasymunion}\ S\ {\isasymin}\ C{\isacharparenright}{\isachardoublequoteclose}\isanewline
\ \ \ \ \isacommand{using}\isamarkupfalse%
\ assms\ {\isacartoucheopen}S\ {\isasymin}\ C{\isacartoucheclose}\ \isacommand{by}\isamarkupfalse%
\ {\isacharparenleft}rule\ bspec{\isacharparenright}\isanewline
\ \ \isacommand{then}\isamarkupfalse%
\ \isacommand{have}\isamarkupfalse%
\ Con{\isacharcolon}{\isachardoublequoteopen}{\isasymforall}F\ G\ H{\isachardot}\ Con\ F\ G\ H\ {\isasymlongrightarrow}\ F\ {\isasymin}\ S\ {\isasymlongrightarrow}\ {\isacharbraceleft}G{\isacharcomma}H{\isacharbraceright}\ {\isasymunion}\ S\ {\isasymin}\ C{\isachardoublequoteclose}\isanewline
\ \ \ \ \isacommand{by}\isamarkupfalse%
\ {\isacharparenleft}iprover\ elim{\isacharcolon}\ conjunct{\isadigit{1}}\ conjunct{\isadigit{2}}{\isacharparenright}\isanewline
\ \ \isacommand{have}\isamarkupfalse%
\ Dis{\isacharcolon}{\isachardoublequoteopen}{\isasymforall}F\ G\ H{\isachardot}\ Dis\ F\ G\ H\ {\isasymlongrightarrow}\ F\ {\isasymin}\ S\ {\isasymlongrightarrow}\ {\isacharbraceleft}G{\isacharbraceright}\ {\isasymunion}\ S\ {\isasymin}\ C\ {\isasymor}\ {\isacharbraceleft}H{\isacharbraceright}\ {\isasymunion}\ S\ {\isasymin}\ C{\isachardoublequoteclose}\isanewline
\ \ \ \ \isacommand{using}\isamarkupfalse%
\ H\ \isacommand{by}\isamarkupfalse%
\ {\isacharparenleft}iprover\ elim{\isacharcolon}\ conjunct{\isadigit{1}}\ conjunct{\isadigit{2}}{\isacharparenright}\isanewline
\ \ \isacommand{have}\isamarkupfalse%
\ {\isadigit{1}}{\isacharcolon}{\isachardoublequoteopen}{\isasymbottom}\ {\isasymnotin}\ S\isanewline
\ \ \ \ {\isasymand}\ {\isacharparenleft}{\isasymforall}k{\isachardot}\ Atom\ k\ {\isasymin}\ S\ {\isasymlongrightarrow}\ \isactrlbold {\isasymnot}\ {\isacharparenleft}Atom\ k{\isacharparenright}\ {\isasymin}\ S\ {\isasymlongrightarrow}\ False{\isacharparenright}{\isachardoublequoteclose}\isanewline
\ \ \ \ \isacommand{using}\isamarkupfalse%
\ H\ \isacommand{by}\isamarkupfalse%
\ {\isacharparenleft}iprover\ elim{\isacharcolon}\ conjunct{\isadigit{1}}{\isacharparenright}\isanewline
\ \ \isacommand{have}\isamarkupfalse%
\ {\isadigit{2}}{\isacharcolon}{\isachardoublequoteopen}{\isasymforall}G\ H{\isachardot}\ G\ \isactrlbold {\isasymand}\ H\ {\isasymin}\ S\ {\isasymlongrightarrow}\ {\isacharbraceleft}G{\isacharcomma}H{\isacharbraceright}\ {\isasymunion}\ S\ {\isasymin}\ C{\isachardoublequoteclose}\isanewline
\ \ \ \ \isacommand{using}\isamarkupfalse%
\ Con\ \isacommand{by}\isamarkupfalse%
\ {\isacharparenleft}rule\ pcp{\isacharunderscore}alt{\isadigit{2}}Con{\isadigit{1}}{\isacharparenright}\isanewline
\ \ \isacommand{have}\isamarkupfalse%
\ {\isadigit{3}}{\isacharcolon}{\isachardoublequoteopen}{\isasymforall}G\ H{\isachardot}\ G\ \isactrlbold {\isasymor}\ H\ {\isasymin}\ S\ {\isasymlongrightarrow}\ {\isacharbraceleft}G{\isacharbraceright}\ {\isasymunion}\ S\ {\isasymin}\ C\ {\isasymor}\ {\isacharbraceleft}H{\isacharbraceright}\ {\isasymunion}\ S\ {\isasymin}\ C{\isachardoublequoteclose}\isanewline
\ \ \ \ \isacommand{using}\isamarkupfalse%
\ Dis\ \isacommand{by}\isamarkupfalse%
\ {\isacharparenleft}rule\ pcp{\isacharunderscore}alt{\isadigit{2}}Dis{\isadigit{1}}{\isacharparenright}\isanewline
\ \ \isacommand{have}\isamarkupfalse%
\ {\isadigit{4}}{\isacharcolon}{\isachardoublequoteopen}{\isasymforall}G\ H{\isachardot}\ G\ \isactrlbold {\isasymrightarrow}\ H\ {\isasymin}\ S\ {\isasymlongrightarrow}\ {\isacharbraceleft}\isactrlbold {\isasymnot}G{\isacharbraceright}\ {\isasymunion}\ S\ {\isasymin}\ C\ {\isasymor}\ {\isacharbraceleft}H{\isacharbraceright}\ {\isasymunion}\ S\ {\isasymin}\ C{\isachardoublequoteclose}\isanewline
\ \ \ \ \isacommand{using}\isamarkupfalse%
\ Dis\ \isacommand{by}\isamarkupfalse%
\ {\isacharparenleft}rule\ pcp{\isacharunderscore}alt{\isadigit{2}}Dis{\isadigit{2}}{\isacharparenright}\isanewline
\ \ \isacommand{have}\isamarkupfalse%
\ {\isadigit{5}}{\isacharcolon}{\isachardoublequoteopen}{\isasymforall}G{\isachardot}\ \isactrlbold {\isasymnot}\ {\isacharparenleft}\isactrlbold {\isasymnot}G{\isacharparenright}\ {\isasymin}\ S\ {\isasymlongrightarrow}\ {\isacharbraceleft}G{\isacharbraceright}\ {\isasymunion}\ S\ {\isasymin}\ C{\isachardoublequoteclose}\isanewline
\ \ \ \ \isacommand{using}\isamarkupfalse%
\ Con\ \isacommand{by}\isamarkupfalse%
\ {\isacharparenleft}rule\ pcp{\isacharunderscore}alt{\isadigit{2}}Con{\isadigit{2}}{\isacharparenright}\isanewline
\ \ \isacommand{have}\isamarkupfalse%
\ {\isadigit{6}}{\isacharcolon}{\isachardoublequoteopen}{\isasymforall}G\ H{\isachardot}\ \isactrlbold {\isasymnot}{\isacharparenleft}G\ \isactrlbold {\isasymand}\ H{\isacharparenright}\ {\isasymin}\ S\ {\isasymlongrightarrow}\ {\isacharbraceleft}\isactrlbold {\isasymnot}\ G{\isacharbraceright}\ {\isasymunion}\ S\ {\isasymin}\ C\ {\isasymor}\ {\isacharbraceleft}\isactrlbold {\isasymnot}\ H{\isacharbraceright}\ {\isasymunion}\ S\ {\isasymin}\ C{\isachardoublequoteclose}\isanewline
\ \ \ \ \isacommand{using}\isamarkupfalse%
\ Dis\ \isacommand{by}\isamarkupfalse%
\ {\isacharparenleft}rule\ pcp{\isacharunderscore}alt{\isadigit{2}}Dis{\isadigit{3}}{\isacharparenright}\isanewline
\ \ \isacommand{have}\isamarkupfalse%
\ {\isadigit{7}}{\isacharcolon}{\isachardoublequoteopen}{\isasymforall}G\ H{\isachardot}\ \isactrlbold {\isasymnot}{\isacharparenleft}G\ \isactrlbold {\isasymor}\ H{\isacharparenright}\ {\isasymin}\ S\ {\isasymlongrightarrow}\ {\isacharbraceleft}\isactrlbold {\isasymnot}\ G{\isacharcomma}\ \isactrlbold {\isasymnot}\ H{\isacharbraceright}\ {\isasymunion}\ S\ {\isasymin}\ C{\isachardoublequoteclose}\isanewline
\ \ \ \ \isacommand{using}\isamarkupfalse%
\ Con\ \isacommand{by}\isamarkupfalse%
\ {\isacharparenleft}rule\ pcp{\isacharunderscore}alt{\isadigit{2}}Con{\isadigit{3}}{\isacharparenright}\isanewline
\ \ \isacommand{have}\isamarkupfalse%
\ {\isadigit{8}}{\isacharcolon}{\isachardoublequoteopen}{\isasymforall}G\ H{\isachardot}\ \isactrlbold {\isasymnot}{\isacharparenleft}G\ \isactrlbold {\isasymrightarrow}\ H{\isacharparenright}\ {\isasymin}\ S\ {\isasymlongrightarrow}\ {\isacharbraceleft}G{\isacharcomma}\isactrlbold {\isasymnot}\ H{\isacharbraceright}\ {\isasymunion}\ S\ {\isasymin}\ C{\isachardoublequoteclose}\isanewline
\ \ \ \ \isacommand{using}\isamarkupfalse%
\ Con\ \isacommand{by}\isamarkupfalse%
\ {\isacharparenleft}rule\ pcp{\isacharunderscore}alt{\isadigit{2}}Con{\isadigit{4}}{\isacharparenright}\isanewline
\ \ \isacommand{have}\isamarkupfalse%
\ A{\isacharcolon}{\isachardoublequoteopen}{\isasymbottom}\ {\isasymnotin}\ S\isanewline
\ \ \ \ {\isasymand}\ {\isacharparenleft}{\isasymforall}k{\isachardot}\ Atom\ k\ {\isasymin}\ S\ {\isasymlongrightarrow}\ \isactrlbold {\isasymnot}\ {\isacharparenleft}Atom\ k{\isacharparenright}\ {\isasymin}\ S\ {\isasymlongrightarrow}\ False{\isacharparenright}\isanewline
\ \ \ \ {\isasymand}\ {\isacharparenleft}{\isasymforall}G\ H{\isachardot}\ G\ \isactrlbold {\isasymand}\ H\ {\isasymin}\ S\ {\isasymlongrightarrow}\ {\isacharbraceleft}G{\isacharcomma}H{\isacharbraceright}\ {\isasymunion}\ S\ {\isasymin}\ C{\isacharparenright}\isanewline
\ \ \ \ {\isasymand}\ {\isacharparenleft}{\isasymforall}G\ H{\isachardot}\ G\ \isactrlbold {\isasymor}\ H\ {\isasymin}\ S\ {\isasymlongrightarrow}\ {\isacharbraceleft}G{\isacharbraceright}\ {\isasymunion}\ S\ {\isasymin}\ C\ {\isasymor}\ {\isacharbraceleft}H{\isacharbraceright}\ {\isasymunion}\ S\ {\isasymin}\ C{\isacharparenright}\isanewline
\ \ \ \ {\isasymand}\ {\isacharparenleft}{\isasymforall}G\ H{\isachardot}\ G\ \isactrlbold {\isasymrightarrow}\ H\ {\isasymin}\ S\ {\isasymlongrightarrow}\ {\isacharbraceleft}\isactrlbold {\isasymnot}G{\isacharbraceright}\ {\isasymunion}\ S\ {\isasymin}\ C\ {\isasymor}\ {\isacharbraceleft}H{\isacharbraceright}\ {\isasymunion}\ S\ {\isasymin}\ C{\isacharparenright}{\isachardoublequoteclose}\isanewline
\ \ \ \ \isacommand{using}\isamarkupfalse%
\ {\isadigit{1}}\ {\isadigit{2}}\ {\isadigit{3}}\ {\isadigit{4}}\ \isacommand{by}\isamarkupfalse%
\ {\isacharparenleft}iprover\ intro{\isacharcolon}\ conjI{\isacharparenright}\isanewline
\ \ \isacommand{have}\isamarkupfalse%
\ B{\isacharcolon}{\isachardoublequoteopen}{\isacharparenleft}{\isasymforall}G{\isachardot}\ \isactrlbold {\isasymnot}\ {\isacharparenleft}\isactrlbold {\isasymnot}G{\isacharparenright}\ {\isasymin}\ S\ {\isasymlongrightarrow}\ {\isacharbraceleft}G{\isacharbraceright}\ {\isasymunion}\ S\ {\isasymin}\ C{\isacharparenright}\isanewline
\ \ \ \ {\isasymand}\ {\isacharparenleft}{\isasymforall}G\ H{\isachardot}\ \isactrlbold {\isasymnot}{\isacharparenleft}G\ \isactrlbold {\isasymand}\ H{\isacharparenright}\ {\isasymin}\ S\ {\isasymlongrightarrow}\ {\isacharbraceleft}\isactrlbold {\isasymnot}\ G{\isacharbraceright}\ {\isasymunion}\ S\ {\isasymin}\ C\ {\isasymor}\ {\isacharbraceleft}\isactrlbold {\isasymnot}\ H{\isacharbraceright}\ {\isasymunion}\ S\ {\isasymin}\ C{\isacharparenright}\isanewline
\ \ \ \ {\isasymand}\ {\isacharparenleft}{\isasymforall}G\ H{\isachardot}\ \isactrlbold {\isasymnot}{\isacharparenleft}G\ \isactrlbold {\isasymor}\ H{\isacharparenright}\ {\isasymin}\ S\ {\isasymlongrightarrow}\ {\isacharbraceleft}\isactrlbold {\isasymnot}\ G{\isacharcomma}\ \isactrlbold {\isasymnot}\ H{\isacharbraceright}\ {\isasymunion}\ S\ {\isasymin}\ C{\isacharparenright}\isanewline
\ \ \ \ {\isasymand}\ {\isacharparenleft}{\isasymforall}G\ H{\isachardot}\ \isactrlbold {\isasymnot}{\isacharparenleft}G\ \isactrlbold {\isasymrightarrow}\ H{\isacharparenright}\ {\isasymin}\ S\ {\isasymlongrightarrow}\ {\isacharbraceleft}G{\isacharcomma}\isactrlbold {\isasymnot}\ H{\isacharbraceright}\ {\isasymunion}\ S\ {\isasymin}\ C{\isacharparenright}{\isachardoublequoteclose}\isanewline
\ \ \ \ \isacommand{using}\isamarkupfalse%
\ {\isadigit{5}}\ {\isadigit{6}}\ {\isadigit{7}}\ {\isadigit{8}}\ \isacommand{by}\isamarkupfalse%
\ {\isacharparenleft}iprover\ intro{\isacharcolon}\ conjI{\isacharparenright}\isanewline
\ \ \isacommand{have}\isamarkupfalse%
\ {\isachardoublequoteopen}{\isacharparenleft}{\isasymbottom}\ {\isasymnotin}\ S\isanewline
\ \ \ \ {\isasymand}\ {\isacharparenleft}{\isasymforall}k{\isachardot}\ Atom\ k\ {\isasymin}\ S\ {\isasymlongrightarrow}\ \isactrlbold {\isasymnot}\ {\isacharparenleft}Atom\ k{\isacharparenright}\ {\isasymin}\ S\ {\isasymlongrightarrow}\ False{\isacharparenright}\isanewline
\ \ \ \ {\isasymand}\ {\isacharparenleft}{\isasymforall}G\ H{\isachardot}\ G\ \isactrlbold {\isasymand}\ H\ {\isasymin}\ S\ {\isasymlongrightarrow}\ {\isacharbraceleft}G{\isacharcomma}H{\isacharbraceright}\ {\isasymunion}\ S\ {\isasymin}\ C{\isacharparenright}\isanewline
\ \ \ \ {\isasymand}\ {\isacharparenleft}{\isasymforall}G\ H{\isachardot}\ G\ \isactrlbold {\isasymor}\ H\ {\isasymin}\ S\ {\isasymlongrightarrow}\ {\isacharbraceleft}G{\isacharbraceright}\ {\isasymunion}\ S\ {\isasymin}\ C\ {\isasymor}\ {\isacharbraceleft}H{\isacharbraceright}\ {\isasymunion}\ S\ {\isasymin}\ C{\isacharparenright}\isanewline
\ \ \ \ {\isasymand}\ {\isacharparenleft}{\isasymforall}G\ H{\isachardot}\ G\ \isactrlbold {\isasymrightarrow}\ H\ {\isasymin}\ S\ {\isasymlongrightarrow}\ {\isacharbraceleft}\isactrlbold {\isasymnot}G{\isacharbraceright}\ {\isasymunion}\ S\ {\isasymin}\ C\ {\isasymor}\ {\isacharbraceleft}H{\isacharbraceright}\ {\isasymunion}\ S\ {\isasymin}\ C{\isacharparenright}{\isacharparenright}\isanewline
\ \ \ \ {\isasymand}\ {\isacharparenleft}{\isacharparenleft}{\isasymforall}G{\isachardot}\ \isactrlbold {\isasymnot}\ {\isacharparenleft}\isactrlbold {\isasymnot}G{\isacharparenright}\ {\isasymin}\ S\ {\isasymlongrightarrow}\ {\isacharbraceleft}G{\isacharbraceright}\ {\isasymunion}\ S\ {\isasymin}\ C{\isacharparenright}\isanewline
\ \ \ \ {\isasymand}\ {\isacharparenleft}{\isasymforall}G\ H{\isachardot}\ \isactrlbold {\isasymnot}{\isacharparenleft}G\ \isactrlbold {\isasymand}\ H{\isacharparenright}\ {\isasymin}\ S\ {\isasymlongrightarrow}\ {\isacharbraceleft}\isactrlbold {\isasymnot}\ G{\isacharbraceright}\ {\isasymunion}\ S\ {\isasymin}\ C\ {\isasymor}\ {\isacharbraceleft}\isactrlbold {\isasymnot}\ H{\isacharbraceright}\ {\isasymunion}\ S\ {\isasymin}\ C{\isacharparenright}\isanewline
\ \ \ \ {\isasymand}\ {\isacharparenleft}{\isasymforall}G\ H{\isachardot}\ \isactrlbold {\isasymnot}{\isacharparenleft}G\ \isactrlbold {\isasymor}\ H{\isacharparenright}\ {\isasymin}\ S\ {\isasymlongrightarrow}\ {\isacharbraceleft}\isactrlbold {\isasymnot}\ G{\isacharcomma}\ \isactrlbold {\isasymnot}\ H{\isacharbraceright}\ {\isasymunion}\ S\ {\isasymin}\ C{\isacharparenright}\isanewline
\ \ \ \ {\isasymand}\ {\isacharparenleft}{\isasymforall}G\ H{\isachardot}\ \isactrlbold {\isasymnot}{\isacharparenleft}G\ \isactrlbold {\isasymrightarrow}\ H{\isacharparenright}\ {\isasymin}\ S\ {\isasymlongrightarrow}\ {\isacharbraceleft}G{\isacharcomma}\isactrlbold {\isasymnot}\ H{\isacharbraceright}\ {\isasymunion}\ S\ {\isasymin}\ C{\isacharparenright}{\isacharparenright}{\isachardoublequoteclose}\isanewline
\ \ \ \ \isacommand{using}\isamarkupfalse%
\ A\ B\ \isacommand{by}\isamarkupfalse%
\ {\isacharparenleft}rule\ conjI{\isacharparenright}\isanewline
\ \ \isacommand{thus}\isamarkupfalse%
\ {\isachardoublequoteopen}{\isasymbottom}\ {\isasymnotin}\ S\isanewline
\ \ \ \ {\isasymand}\ {\isacharparenleft}{\isasymforall}k{\isachardot}\ Atom\ k\ {\isasymin}\ S\ {\isasymlongrightarrow}\ \isactrlbold {\isasymnot}\ {\isacharparenleft}Atom\ k{\isacharparenright}\ {\isasymin}\ S\ {\isasymlongrightarrow}\ False{\isacharparenright}\isanewline
\ \ \ \ {\isasymand}\ {\isacharparenleft}{\isasymforall}G\ H{\isachardot}\ G\ \isactrlbold {\isasymand}\ H\ {\isasymin}\ S\ {\isasymlongrightarrow}\ {\isacharbraceleft}G{\isacharcomma}H{\isacharbraceright}\ {\isasymunion}\ S\ {\isasymin}\ C{\isacharparenright}\isanewline
\ \ \ \ {\isasymand}\ {\isacharparenleft}{\isasymforall}G\ H{\isachardot}\ G\ \isactrlbold {\isasymor}\ H\ {\isasymin}\ S\ {\isasymlongrightarrow}\ {\isacharbraceleft}G{\isacharbraceright}\ {\isasymunion}\ S\ {\isasymin}\ C\ {\isasymor}\ {\isacharbraceleft}H{\isacharbraceright}\ {\isasymunion}\ S\ {\isasymin}\ C{\isacharparenright}\isanewline
\ \ \ \ {\isasymand}\ {\isacharparenleft}{\isasymforall}G\ H{\isachardot}\ G\ \isactrlbold {\isasymrightarrow}\ H\ {\isasymin}\ S\ {\isasymlongrightarrow}\ {\isacharbraceleft}\isactrlbold {\isasymnot}G{\isacharbraceright}\ {\isasymunion}\ S\ {\isasymin}\ C\ {\isasymor}\ {\isacharbraceleft}H{\isacharbraceright}\ {\isasymunion}\ S\ {\isasymin}\ C{\isacharparenright}\isanewline
\ \ \ \ {\isasymand}\ {\isacharparenleft}{\isasymforall}G{\isachardot}\ \isactrlbold {\isasymnot}\ {\isacharparenleft}\isactrlbold {\isasymnot}G{\isacharparenright}\ {\isasymin}\ S\ {\isasymlongrightarrow}\ {\isacharbraceleft}G{\isacharbraceright}\ {\isasymunion}\ S\ {\isasymin}\ C{\isacharparenright}\isanewline
\ \ \ \ {\isasymand}\ {\isacharparenleft}{\isasymforall}G\ H{\isachardot}\ \isactrlbold {\isasymnot}{\isacharparenleft}G\ \isactrlbold {\isasymand}\ H{\isacharparenright}\ {\isasymin}\ S\ {\isasymlongrightarrow}\ {\isacharbraceleft}\isactrlbold {\isasymnot}\ G{\isacharbraceright}\ {\isasymunion}\ S\ {\isasymin}\ C\ {\isasymor}\ {\isacharbraceleft}\isactrlbold {\isasymnot}\ H{\isacharbraceright}\ {\isasymunion}\ S\ {\isasymin}\ C{\isacharparenright}\isanewline
\ \ \ \ {\isasymand}\ {\isacharparenleft}{\isasymforall}G\ H{\isachardot}\ \isactrlbold {\isasymnot}{\isacharparenleft}G\ \isactrlbold {\isasymor}\ H{\isacharparenright}\ {\isasymin}\ S\ {\isasymlongrightarrow}\ {\isacharbraceleft}\isactrlbold {\isasymnot}\ G{\isacharcomma}\ \isactrlbold {\isasymnot}\ H{\isacharbraceright}\ {\isasymunion}\ S\ {\isasymin}\ C{\isacharparenright}\isanewline
\ \ \ \ {\isasymand}\ {\isacharparenleft}{\isasymforall}G\ H{\isachardot}\ \isactrlbold {\isasymnot}{\isacharparenleft}G\ \isactrlbold {\isasymrightarrow}\ H{\isacharparenright}\ {\isasymin}\ S\ {\isasymlongrightarrow}\ {\isacharbraceleft}G{\isacharcomma}\isactrlbold {\isasymnot}\ H{\isacharbraceright}\ {\isasymunion}\ S\ {\isasymin}\ C{\isacharparenright}{\isachardoublequoteclose}\isanewline
\ \ \ \ \isacommand{by}\isamarkupfalse%
\ {\isacharparenleft}iprover\ intro{\isacharcolon}\ conj{\isacharunderscore}assoc{\isacharparenright}\isanewline
\isacommand{qed}\isamarkupfalse%
%
\endisatagproof
{\isafoldproof}%
%
\isadelimproof
%
\endisadelimproof
%
\begin{isamarkuptext}%
Una vez probadas detalladamente en Isabelle cada una de las implicaciones de la
  equivalencia, podemos finalmente concluir con la demostración del lema completo.%
\end{isamarkuptext}\isamarkuptrue%
\isacommand{lemma}\isamarkupfalse%
\ {\isachardoublequoteopen}pcp\ C\ {\isacharequal}\ {\isacharparenleft}{\isasymforall}S\ {\isasymin}\ C{\isachardot}\ {\isasymbottom}\ {\isasymnotin}\ S\isanewline
{\isasymand}\ {\isacharparenleft}{\isasymforall}k{\isachardot}\ Atom\ k\ {\isasymin}\ S\ {\isasymlongrightarrow}\ \isactrlbold {\isasymnot}\ {\isacharparenleft}Atom\ k{\isacharparenright}\ {\isasymin}\ S\ {\isasymlongrightarrow}\ False{\isacharparenright}\isanewline
{\isasymand}\ {\isacharparenleft}{\isasymforall}F\ G\ H{\isachardot}\ Con\ F\ G\ H\ {\isasymlongrightarrow}\ F\ {\isasymin}\ S\ {\isasymlongrightarrow}\ {\isacharbraceleft}G{\isacharcomma}H{\isacharbraceright}\ {\isasymunion}\ S\ {\isasymin}\ C{\isacharparenright}\isanewline
{\isasymand}\ {\isacharparenleft}{\isasymforall}F\ G\ H{\isachardot}\ Dis\ F\ G\ H\ {\isasymlongrightarrow}\ F\ {\isasymin}\ S\ {\isasymlongrightarrow}\ {\isacharbraceleft}G{\isacharbraceright}\ {\isasymunion}\ S\ {\isasymin}\ C\ {\isasymor}\ {\isacharbraceleft}H{\isacharbraceright}\ {\isasymunion}\ S\ {\isasymin}\ C{\isacharparenright}{\isacharparenright}{\isachardoublequoteclose}\isanewline
%
\isadelimproof
%
\endisadelimproof
%
\isatagproof
\isacommand{proof}\isamarkupfalse%
\ {\isacharparenleft}rule\ iffI{\isacharparenright}\isanewline
\ \ \isacommand{assume}\isamarkupfalse%
\ {\isachardoublequoteopen}pcp\ C{\isachardoublequoteclose}\isanewline
\ \ \isacommand{thus}\isamarkupfalse%
\ {\isachardoublequoteopen}{\isasymforall}S\ {\isasymin}\ C{\isachardot}\ {\isasymbottom}\ {\isasymnotin}\ S\isanewline
{\isasymand}\ {\isacharparenleft}{\isasymforall}k{\isachardot}\ Atom\ k\ {\isasymin}\ S\ {\isasymlongrightarrow}\ \isactrlbold {\isasymnot}\ {\isacharparenleft}Atom\ k{\isacharparenright}\ {\isasymin}\ S\ {\isasymlongrightarrow}\ False{\isacharparenright}\isanewline
{\isasymand}\ {\isacharparenleft}{\isasymforall}F\ G\ H{\isachardot}\ Con\ F\ G\ H\ {\isasymlongrightarrow}\ F\ {\isasymin}\ S\ {\isasymlongrightarrow}\ {\isacharbraceleft}G{\isacharcomma}H{\isacharbraceright}\ {\isasymunion}\ S\ {\isasymin}\ C{\isacharparenright}\isanewline
{\isasymand}\ {\isacharparenleft}{\isasymforall}F\ G\ H{\isachardot}\ Dis\ F\ G\ H\ {\isasymlongrightarrow}\ F\ {\isasymin}\ S\ {\isasymlongrightarrow}\ {\isacharbraceleft}G{\isacharbraceright}\ {\isasymunion}\ S\ {\isasymin}\ C\ {\isasymor}\ {\isacharbraceleft}H{\isacharbraceright}\ {\isasymunion}\ S\ {\isasymin}\ C{\isacharparenright}{\isachardoublequoteclose}\isanewline
\ \ \ \ \isacommand{by}\isamarkupfalse%
\ {\isacharparenleft}rule\ pcp{\isacharunderscore}alt{\isadigit{1}}{\isacharparenright}\isanewline
\isacommand{next}\isamarkupfalse%
\isanewline
\ \ \isacommand{assume}\isamarkupfalse%
\ {\isachardoublequoteopen}{\isasymforall}S\ {\isasymin}\ C{\isachardot}\ {\isasymbottom}\ {\isasymnotin}\ S\isanewline
{\isasymand}\ {\isacharparenleft}{\isasymforall}k{\isachardot}\ Atom\ k\ {\isasymin}\ S\ {\isasymlongrightarrow}\ \isactrlbold {\isasymnot}\ {\isacharparenleft}Atom\ k{\isacharparenright}\ {\isasymin}\ S\ {\isasymlongrightarrow}\ False{\isacharparenright}\isanewline
{\isasymand}\ {\isacharparenleft}{\isasymforall}F\ G\ H{\isachardot}\ Con\ F\ G\ H\ {\isasymlongrightarrow}\ F\ {\isasymin}\ S\ {\isasymlongrightarrow}\ {\isacharbraceleft}G{\isacharcomma}H{\isacharbraceright}\ {\isasymunion}\ S\ {\isasymin}\ C{\isacharparenright}\isanewline
{\isasymand}\ {\isacharparenleft}{\isasymforall}F\ G\ H{\isachardot}\ Dis\ F\ G\ H\ {\isasymlongrightarrow}\ F\ {\isasymin}\ S\ {\isasymlongrightarrow}\ {\isacharbraceleft}G{\isacharbraceright}\ {\isasymunion}\ S\ {\isasymin}\ C\ {\isasymor}\ {\isacharbraceleft}H{\isacharbraceright}\ {\isasymunion}\ S\ {\isasymin}\ C{\isacharparenright}{\isachardoublequoteclose}\isanewline
\ \ \isacommand{thus}\isamarkupfalse%
\ {\isachardoublequoteopen}pcp\ C{\isachardoublequoteclose}\isanewline
\ \ \ \ \isacommand{by}\isamarkupfalse%
\ {\isacharparenleft}rule\ pcp{\isacharunderscore}alt{\isadigit{2}}{\isacharparenright}\isanewline
\isacommand{qed}\isamarkupfalse%
%
\endisatagproof
{\isafoldproof}%
%
\isadelimproof
%
\endisadelimproof
%
\begin{isamarkuptext}%
La demostración automática del resultado en Isabelle/HOL se muestra finalmente a 
  continuación.%
\end{isamarkuptext}\isamarkuptrue%
\isacommand{lemma}\isamarkupfalse%
\ pcp{\isacharunderscore}alt{\isacharcolon}\ {\isachardoublequoteopen}pcp\ C\ {\isacharequal}\ {\isacharparenleft}{\isasymforall}S\ {\isasymin}\ C{\isachardot}\isanewline
\ \ {\isasymbottom}\ {\isasymnotin}\ S\isanewline
{\isasymand}\ {\isacharparenleft}{\isasymforall}k{\isachardot}\ Atom\ k\ {\isasymin}\ S\ {\isasymlongrightarrow}\ \isactrlbold {\isasymnot}\ {\isacharparenleft}Atom\ k{\isacharparenright}\ {\isasymin}\ S\ {\isasymlongrightarrow}\ False{\isacharparenright}\isanewline
{\isasymand}\ {\isacharparenleft}{\isasymforall}F\ G\ H{\isachardot}\ Con\ F\ G\ H\ {\isasymlongrightarrow}\ F\ {\isasymin}\ S\ {\isasymlongrightarrow}\ {\isacharbraceleft}G{\isacharcomma}H{\isacharbraceright}\ {\isasymunion}\ S\ {\isasymin}\ C{\isacharparenright}\isanewline
{\isasymand}\ {\isacharparenleft}{\isasymforall}F\ G\ H{\isachardot}\ Dis\ F\ G\ H\ {\isasymlongrightarrow}\ F\ {\isasymin}\ S\ {\isasymlongrightarrow}\ {\isacharbraceleft}G{\isacharbraceright}\ {\isasymunion}\ S\ {\isasymin}\ C\ {\isasymor}\ {\isacharbraceleft}H{\isacharbraceright}\ {\isasymunion}\ S\ {\isasymin}\ C{\isacharparenright}{\isacharparenright}{\isachardoublequoteclose}\isanewline
%
\isadelimproof
\ \ %
\endisadelimproof
%
\isatagproof
\isacommand{apply}\isamarkupfalse%
{\isacharparenleft}simp\ add{\isacharcolon}\ pcp{\isacharunderscore}def\ con{\isacharunderscore}dis{\isacharunderscore}simps{\isacharparenright}\isanewline
\ \ \isacommand{apply}\isamarkupfalse%
{\isacharparenleft}rule\ iffI{\isacharsemicolon}\ unfold\ Ball{\isacharunderscore}def{\isacharsemicolon}\ elim\ all{\isacharunderscore}forward{\isacharparenright}\isanewline
\ \ \isacommand{by}\isamarkupfalse%
\ {\isacharparenleft}auto\ simp\ add{\isacharcolon}\ insert{\isacharunderscore}absorb\ split{\isacharcolon}\ formula{\isachardot}splits{\isacharparenright}\isanewline
%
\endisatagproof
{\isafoldproof}%
%
\isadelimproof
%
\endisadelimproof
%
\isadelimtheory
%
\endisadelimtheory
%
\isatagtheory
%
\endisatagtheory
{\isafoldtheory}%
%
\isadelimtheory
%
\endisadelimtheory
%
\end{isabellebody}%
\endinput
%:%file=~/TFM/TFM/Pcp.thy%:%
%:%19=11%:%
%:%20=12%:%
%:%21=13%:%
%:%22=14%:%
%:%23=15%:%
%:%24=16%:%
%:%25=17%:%
%:%26=18%:%
%:%26=19%:%
%:%27=20%:%
%:%28=21%:%
%:%29=22%:%
%:%30=23%:%
%:%31=24%:%
%:%32=25%:%
%:%33=26%:%
%:%37=29%:%
%:%38=30%:%
%:%39=31%:%
%:%43=34%:%
%:%44=35%:%
%:%45=36%:%
%:%46=37%:%
%:%47=38%:%
%:%48=39%:%
%:%49=40%:%
%:%50=41%:%
%:%51=42%:%
%:%52=43%:%
%:%53=44%:%
%:%54=45%:%
%:%55=46%:%
%:%56=47%:%
%:%57=48%:%
%:%58=49%:%
%:%59=50%:%
%:%60=51%:%
%:%61=52%:%
%:%62=53%:%
%:%63=54%:%
%:%64=55%:%
%:%66=57%:%
%:%67=57%:%
%:%78=68%:%
%:%79=69%:%
%:%80=70%:%
%:%82=72%:%
%:%83=72%:%
%:%86=73%:%
%:%90=73%:%
%:%91=73%:%
%:%92=73%:%
%:%101=75%:%
%:%102=76%:%
%:%103=77%:%
%:%105=79%:%
%:%106=79%:%
%:%109=80%:%
%:%113=80%:%
%:%114=80%:%
%:%115=80%:%
%:%120=80%:%
%:%123=81%:%
%:%124=82%:%
%:%125=82%:%
%:%127=84%:%
%:%130=85%:%
%:%134=85%:%
%:%135=85%:%
%:%136=85%:%
%:%145=87%:%
%:%146=88%:%
%:%147=89%:%
%:%148=90%:%
%:%150=92%:%
%:%151=92%:%
%:%152=93%:%
%:%155=94%:%
%:%159=94%:%
%:%160=94%:%
%:%161=94%:%
%:%170=96%:%
%:%171=97%:%
%:%172=98%:%
%:%173=99%:%
%:%174=100%:%
%:%175=101%:%
%:%176=102%:%
%:%177=103%:%
%:%178=104%:%
%:%179=105%:%
%:%180=106%:%
%:%181=107%:%
%:%182=108%:%
%:%183=109%:%
%:%184=110%:%
%:%185=111%:%
%:%186=112%:%
%:%187=113%:%
%:%188=114%:%
%:%189=115%:%
%:%190=116%:%
%:%191=117%:%
%:%192=118%:%
%:%194=120%:%
%:%195=120%:%
%:%198=123%:%
%:%201=124%:%
%:%205=124%:%
%:%215=126%:%
%:%216=127%:%
%:%217=128%:%
%:%218=129%:%
%:%219=130%:%
%:%220=131%:%
%:%221=132%:%
%:%222=133%:%
%:%223=134%:%
%:%224=135%:%
%:%225=136%:%
%:%226=137%:%
%:%227=138%:%
%:%228=139%:%
%:%229=140%:%
%:%230=141%:%
%:%231=142%:%
%:%232=143%:%
%:%233=144%:%
%:%234=145%:%
%:%235=146%:%
%:%236=147%:%
%:%237=148%:%
%:%238=149%:%
%:%239=150%:%
%:%240=151%:%
%:%241=152%:%
%:%242=153%:%
%:%243=154%:%
%:%244=155%:%
%:%245=156%:%
%:%246=157%:%
%:%247=158%:%
%:%248=159%:%
%:%249=160%:%
%:%250=161%:%
%:%251=162%:%
%:%252=163%:%
%:%253=164%:%
%:%254=165%:%
%:%255=166%:%
%:%256=167%:%
%:%257=168%:%
%:%258=169%:%
%:%259=170%:%
%:%260=171%:%
%:%261=172%:%
%:%262=173%:%
%:%263=174%:%
%:%264=175%:%
%:%265=176%:%
%:%266=177%:%
%:%267=178%:%
%:%268=179%:%
%:%269=180%:%
%:%270=181%:%
%:%271=182%:%
%:%272=183%:%
%:%273=184%:%
%:%274=185%:%
%:%275=186%:%
%:%276=187%:%
%:%277=188%:%
%:%278=189%:%
%:%279=190%:%
%:%280=191%:%
%:%281=192%:%
%:%282=193%:%
%:%283=194%:%
%:%284=195%:%
%:%285=196%:%
%:%286=197%:%
%:%287=198%:%
%:%288=199%:%
%:%289=200%:%
%:%290=201%:%
%:%291=202%:%
%:%292=203%:%
%:%293=204%:%
%:%294=205%:%
%:%295=206%:%
%:%296=207%:%
%:%297=208%:%
%:%298=209%:%
%:%299=210%:%
%:%300=211%:%
%:%301=212%:%
%:%302=213%:%
%:%303=214%:%
%:%304=215%:%
%:%305=216%:%
%:%306=217%:%
%:%307=218%:%
%:%308=219%:%
%:%309=220%:%
%:%310=221%:%
%:%311=222%:%
%:%312=223%:%
%:%313=224%:%
%:%314=225%:%
%:%315=226%:%
%:%316=227%:%
%:%317=228%:%
%:%318=229%:%
%:%319=230%:%
%:%320=231%:%
%:%321=232%:%
%:%322=233%:%
%:%323=234%:%
%:%324=235%:%
%:%325=236%:%
%:%326=237%:%
%:%327=238%:%
%:%328=239%:%
%:%329=240%:%
%:%330=241%:%
%:%331=242%:%
%:%332=243%:%
%:%333=244%:%
%:%334=245%:%
%:%335=246%:%
%:%336=247%:%
%:%337=248%:%
%:%338=249%:%
%:%339=250%:%
%:%340=251%:%
%:%341=252%:%
%:%342=253%:%
%:%343=254%:%
%:%344=255%:%
%:%345=256%:%
%:%346=257%:%
%:%347=258%:%
%:%348=259%:%
%:%349=260%:%
%:%350=261%:%
%:%351=262%:%
%:%352=263%:%
%:%353=264%:%
%:%354=265%:%
%:%355=266%:%
%:%356=267%:%
%:%357=268%:%
%:%358=269%:%
%:%359=270%:%
%:%360=271%:%
%:%361=272%:%
%:%362=273%:%
%:%363=274%:%
%:%364=275%:%
%:%365=276%:%
%:%366=277%:%
%:%367=278%:%
%:%368=279%:%
%:%369=280%:%
%:%370=281%:%
%:%371=282%:%
%:%372=283%:%
%:%373=284%:%
%:%374=285%:%
%:%375=286%:%
%:%376=287%:%
%:%377=288%:%
%:%378=289%:%
%:%379=290%:%
%:%380=291%:%
%:%381=292%:%
%:%382=293%:%
%:%383=294%:%
%:%384=295%:%
%:%385=296%:%
%:%386=297%:%
%:%387=298%:%
%:%388=299%:%
%:%389=300%:%
%:%390=301%:%
%:%391=302%:%
%:%392=303%:%
%:%393=304%:%
%:%394=305%:%
%:%396=307%:%
%:%397=307%:%
%:%398=308%:%
%:%401=311%:%
%:%402=312%:%
%:%409=313%:%
%:%410=313%:%
%:%411=314%:%
%:%412=314%:%
%:%413=315%:%
%:%414=315%:%
%:%415=315%:%
%:%416=316%:%
%:%417=316%:%
%:%418=317%:%
%:%419=317%:%
%:%420=317%:%
%:%421=318%:%
%:%422=318%:%
%:%423=319%:%
%:%424=319%:%
%:%425=319%:%
%:%426=320%:%
%:%427=320%:%
%:%428=321%:%
%:%429=321%:%
%:%430=321%:%
%:%431=322%:%
%:%432=322%:%
%:%433=323%:%
%:%434=323%:%
%:%435=324%:%
%:%436=324%:%
%:%437=325%:%
%:%438=325%:%
%:%439=326%:%
%:%440=326%:%
%:%441=327%:%
%:%442=327%:%
%:%443=328%:%
%:%444=328%:%
%:%445=328%:%
%:%448=331%:%
%:%449=332%:%
%:%450=332%:%
%:%451=333%:%
%:%452=333%:%
%:%453=334%:%
%:%454=334%:%
%:%455=335%:%
%:%456=335%:%
%:%457=336%:%
%:%458=336%:%
%:%459=337%:%
%:%460=337%:%
%:%461=337%:%
%:%462=338%:%
%:%463=338%:%
%:%464=339%:%
%:%465=339%:%
%:%467=341%:%
%:%468=342%:%
%:%469=342%:%
%:%470=343%:%
%:%471=343%:%
%:%472=344%:%
%:%473=344%:%
%:%474=345%:%
%:%475=345%:%
%:%476=346%:%
%:%477=346%:%
%:%478=346%:%
%:%479=347%:%
%:%480=347%:%
%:%481=348%:%
%:%482=348%:%
%:%483=348%:%
%:%484=349%:%
%:%485=349%:%
%:%486=350%:%
%:%487=350%:%
%:%488=350%:%
%:%489=351%:%
%:%490=351%:%
%:%491=352%:%
%:%492=352%:%
%:%493=352%:%
%:%494=353%:%
%:%495=353%:%
%:%496=354%:%
%:%497=354%:%
%:%498=354%:%
%:%499=355%:%
%:%500=355%:%
%:%501=356%:%
%:%502=356%:%
%:%503=357%:%
%:%504=358%:%
%:%505=358%:%
%:%506=359%:%
%:%507=359%:%
%:%508=360%:%
%:%509=360%:%
%:%510=361%:%
%:%511=361%:%
%:%512=362%:%
%:%513=362%:%
%:%514=362%:%
%:%515=363%:%
%:%516=363%:%
%:%517=364%:%
%:%518=364%:%
%:%519=364%:%
%:%520=365%:%
%:%521=365%:%
%:%522=366%:%
%:%523=366%:%
%:%524=366%:%
%:%525=367%:%
%:%526=367%:%
%:%527=368%:%
%:%528=368%:%
%:%529=368%:%
%:%530=369%:%
%:%531=369%:%
%:%532=370%:%
%:%533=370%:%
%:%534=371%:%
%:%535=371%:%
%:%536=371%:%
%:%537=372%:%
%:%538=372%:%
%:%539=373%:%
%:%540=373%:%
%:%541=374%:%
%:%542=374%:%
%:%543=374%:%
%:%544=375%:%
%:%545=375%:%
%:%546=376%:%
%:%547=376%:%
%:%548=376%:%
%:%549=377%:%
%:%550=377%:%
%:%551=377%:%
%:%552=378%:%
%:%553=378%:%
%:%554=379%:%
%:%555=379%:%
%:%556=380%:%
%:%557=380%:%
%:%558=381%:%
%:%559=381%:%
%:%560=382%:%
%:%561=382%:%
%:%562=383%:%
%:%563=383%:%
%:%564=384%:%
%:%565=384%:%
%:%566=385%:%
%:%567=385%:%
%:%568=386%:%
%:%578=388%:%
%:%579=389%:%
%:%580=390%:%
%:%582=392%:%
%:%583=392%:%
%:%584=393%:%
%:%587=396%:%
%:%588=397%:%
%:%595=398%:%
%:%596=398%:%
%:%597=399%:%
%:%598=399%:%
%:%599=400%:%
%:%600=400%:%
%:%601=400%:%
%:%602=401%:%
%:%603=401%:%
%:%604=402%:%
%:%605=402%:%
%:%606=402%:%
%:%607=403%:%
%:%608=403%:%
%:%609=404%:%
%:%610=404%:%
%:%611=404%:%
%:%612=405%:%
%:%613=405%:%
%:%614=406%:%
%:%615=406%:%
%:%616=406%:%
%:%617=407%:%
%:%618=407%:%
%:%619=408%:%
%:%620=408%:%
%:%621=409%:%
%:%622=409%:%
%:%623=410%:%
%:%624=410%:%
%:%625=411%:%
%:%626=411%:%
%:%627=412%:%
%:%628=412%:%
%:%629=413%:%
%:%630=413%:%
%:%631=413%:%
%:%634=416%:%
%:%635=417%:%
%:%636=417%:%
%:%637=418%:%
%:%638=418%:%
%:%639=419%:%
%:%640=419%:%
%:%641=420%:%
%:%642=420%:%
%:%643=421%:%
%:%644=421%:%
%:%645=422%:%
%:%646=422%:%
%:%647=422%:%
%:%648=423%:%
%:%649=423%:%
%:%650=424%:%
%:%651=424%:%
%:%653=426%:%
%:%654=427%:%
%:%655=427%:%
%:%656=428%:%
%:%657=428%:%
%:%658=429%:%
%:%659=429%:%
%:%660=430%:%
%:%661=430%:%
%:%662=431%:%
%:%663=431%:%
%:%664=431%:%
%:%665=432%:%
%:%666=432%:%
%:%667=433%:%
%:%668=433%:%
%:%669=433%:%
%:%670=434%:%
%:%671=434%:%
%:%672=435%:%
%:%673=435%:%
%:%674=435%:%
%:%675=436%:%
%:%676=436%:%
%:%677=437%:%
%:%678=437%:%
%:%679=437%:%
%:%680=438%:%
%:%681=438%:%
%:%682=439%:%
%:%683=439%:%
%:%684=439%:%
%:%685=440%:%
%:%686=440%:%
%:%687=441%:%
%:%688=441%:%
%:%689=442%:%
%:%690=443%:%
%:%691=443%:%
%:%692=444%:%
%:%693=444%:%
%:%694=445%:%
%:%695=445%:%
%:%696=446%:%
%:%697=446%:%
%:%698=447%:%
%:%699=447%:%
%:%700=447%:%
%:%701=448%:%
%:%702=448%:%
%:%703=449%:%
%:%704=449%:%
%:%705=449%:%
%:%706=450%:%
%:%707=450%:%
%:%708=451%:%
%:%709=451%:%
%:%710=451%:%
%:%711=452%:%
%:%712=452%:%
%:%713=453%:%
%:%714=453%:%
%:%715=453%:%
%:%716=454%:%
%:%717=454%:%
%:%718=455%:%
%:%719=455%:%
%:%720=455%:%
%:%721=456%:%
%:%722=456%:%
%:%723=457%:%
%:%724=457%:%
%:%725=458%:%
%:%726=458%:%
%:%727=458%:%
%:%728=459%:%
%:%729=459%:%
%:%730=460%:%
%:%731=460%:%
%:%732=461%:%
%:%733=461%:%
%:%734=461%:%
%:%735=462%:%
%:%736=462%:%
%:%737=463%:%
%:%738=463%:%
%:%739=463%:%
%:%740=464%:%
%:%741=464%:%
%:%742=464%:%
%:%743=465%:%
%:%744=465%:%
%:%745=466%:%
%:%746=466%:%
%:%747=467%:%
%:%748=467%:%
%:%749=468%:%
%:%750=468%:%
%:%751=469%:%
%:%752=469%:%
%:%753=470%:%
%:%754=470%:%
%:%755=471%:%
%:%756=471%:%
%:%757=472%:%
%:%758=472%:%
%:%759=473%:%
%:%769=475%:%
%:%770=476%:%
%:%772=478%:%
%:%773=478%:%
%:%774=479%:%
%:%775=480%:%
%:%778=483%:%
%:%785=484%:%
%:%786=484%:%
%:%787=485%:%
%:%788=485%:%
%:%789=486%:%
%:%790=486%:%
%:%791=487%:%
%:%792=487%:%
%:%801=496%:%
%:%802=497%:%
%:%803=497%:%
%:%804=497%:%
%:%805=498%:%
%:%806=498%:%
%:%807=498%:%
%:%815=506%:%
%:%816=507%:%
%:%817=507%:%
%:%818=507%:%
%:%819=508%:%
%:%820=508%:%
%:%821=508%:%
%:%822=509%:%
%:%823=509%:%
%:%824=510%:%
%:%825=510%:%
%:%826=511%:%
%:%827=511%:%
%:%828=511%:%
%:%829=512%:%
%:%830=512%:%
%:%831=513%:%
%:%832=513%:%
%:%833=513%:%
%:%834=514%:%
%:%835=514%:%
%:%836=515%:%
%:%837=515%:%
%:%838=515%:%
%:%839=516%:%
%:%840=516%:%
%:%841=517%:%
%:%842=517%:%
%:%843=517%:%
%:%844=518%:%
%:%845=518%:%
%:%846=519%:%
%:%847=519%:%
%:%848=519%:%
%:%849=520%:%
%:%850=520%:%
%:%851=521%:%
%:%852=521%:%
%:%853=521%:%
%:%854=522%:%
%:%855=522%:%
%:%856=523%:%
%:%857=523%:%
%:%858=523%:%
%:%859=524%:%
%:%860=524%:%
%:%861=525%:%
%:%862=525%:%
%:%863=525%:%
%:%864=526%:%
%:%865=526%:%
%:%868=529%:%
%:%869=530%:%
%:%870=530%:%
%:%871=530%:%
%:%872=531%:%
%:%873=531%:%
%:%874=531%:%
%:%875=532%:%
%:%876=532%:%
%:%877=533%:%
%:%878=533%:%
%:%881=536%:%
%:%882=537%:%
%:%883=537%:%
%:%884=537%:%
%:%885=538%:%
%:%886=538%:%
%:%887=538%:%
%:%888=539%:%
%:%889=539%:%
%:%890=540%:%
%:%891=540%:%
%:%894=543%:%
%:%895=544%:%
%:%896=544%:%
%:%897=544%:%
%:%898=545%:%
%:%908=547%:%
%:%909=548%:%
%:%910=549%:%
%:%911=550%:%
%:%912=551%:%
%:%914=553%:%
%:%915=553%:%
%:%916=554%:%
%:%917=555%:%
%:%924=556%:%
%:%925=556%:%
%:%926=557%:%
%:%927=557%:%
%:%928=558%:%
%:%929=558%:%
%:%930=559%:%
%:%931=559%:%
%:%932=560%:%
%:%933=560%:%
%:%934=561%:%
%:%935=561%:%
%:%936=561%:%
%:%937=562%:%
%:%938=562%:%
%:%939=563%:%
%:%940=563%:%
%:%941=564%:%
%:%942=564%:%
%:%943=565%:%
%:%944=565%:%
%:%945=565%:%
%:%946=566%:%
%:%947=566%:%
%:%948=566%:%
%:%949=567%:%
%:%950=567%:%
%:%951=567%:%
%:%952=568%:%
%:%953=568%:%
%:%954=569%:%
%:%955=569%:%
%:%956=569%:%
%:%957=570%:%
%:%958=570%:%
%:%959=571%:%
%:%965=571%:%
%:%968=572%:%
%:%969=573%:%
%:%970=573%:%
%:%971=574%:%
%:%972=575%:%
%:%979=576%:%
%:%980=576%:%
%:%981=577%:%
%:%982=577%:%
%:%983=578%:%
%:%984=578%:%
%:%985=579%:%
%:%986=579%:%
%:%987=580%:%
%:%988=580%:%
%:%989=581%:%
%:%990=581%:%
%:%991=581%:%
%:%992=582%:%
%:%993=582%:%
%:%994=583%:%
%:%995=583%:%
%:%996=584%:%
%:%997=584%:%
%:%998=585%:%
%:%999=585%:%
%:%1000=585%:%
%:%1001=586%:%
%:%1002=586%:%
%:%1003=586%:%
%:%1004=587%:%
%:%1005=587%:%
%:%1006=588%:%
%:%1007=588%:%
%:%1008=588%:%
%:%1009=589%:%
%:%1010=589%:%
%:%1011=590%:%
%:%1012=590%:%
%:%1013=590%:%
%:%1014=591%:%
%:%1015=591%:%
%:%1016=591%:%
%:%1017=592%:%
%:%1018=592%:%
%:%1019=592%:%
%:%1020=593%:%
%:%1021=593%:%
%:%1022=593%:%
%:%1023=594%:%
%:%1024=594%:%
%:%1025=595%:%
%:%1026=595%:%
%:%1027=596%:%
%:%1028=596%:%
%:%1029=597%:%
%:%1035=597%:%
%:%1038=598%:%
%:%1039=599%:%
%:%1040=599%:%
%:%1041=600%:%
%:%1042=601%:%
%:%1049=602%:%
%:%1050=602%:%
%:%1051=603%:%
%:%1052=603%:%
%:%1053=604%:%
%:%1054=604%:%
%:%1055=605%:%
%:%1056=605%:%
%:%1057=606%:%
%:%1058=606%:%
%:%1059=607%:%
%:%1060=607%:%
%:%1061=607%:%
%:%1062=608%:%
%:%1063=608%:%
%:%1064=609%:%
%:%1065=609%:%
%:%1066=610%:%
%:%1067=610%:%
%:%1068=611%:%
%:%1069=611%:%
%:%1070=611%:%
%:%1071=612%:%
%:%1072=612%:%
%:%1073=612%:%
%:%1074=613%:%
%:%1075=613%:%
%:%1076=613%:%
%:%1077=614%:%
%:%1078=614%:%
%:%1079=615%:%
%:%1080=615%:%
%:%1081=615%:%
%:%1082=616%:%
%:%1083=616%:%
%:%1084=617%:%
%:%1090=617%:%
%:%1093=618%:%
%:%1094=619%:%
%:%1095=619%:%
%:%1096=620%:%
%:%1097=621%:%
%:%1104=622%:%
%:%1105=622%:%
%:%1106=623%:%
%:%1107=623%:%
%:%1108=624%:%
%:%1109=624%:%
%:%1110=625%:%
%:%1111=625%:%
%:%1112=626%:%
%:%1113=626%:%
%:%1114=627%:%
%:%1115=627%:%
%:%1116=627%:%
%:%1117=628%:%
%:%1118=628%:%
%:%1119=629%:%
%:%1120=629%:%
%:%1121=630%:%
%:%1122=630%:%
%:%1123=631%:%
%:%1124=631%:%
%:%1125=631%:%
%:%1126=632%:%
%:%1127=632%:%
%:%1128=632%:%
%:%1129=633%:%
%:%1130=633%:%
%:%1131=633%:%
%:%1132=634%:%
%:%1133=634%:%
%:%1134=635%:%
%:%1135=635%:%
%:%1136=635%:%
%:%1137=636%:%
%:%1138=636%:%
%:%1139=637%:%
%:%1149=639%:%
%:%1150=640%:%
%:%1151=641%:%
%:%1153=643%:%
%:%1154=643%:%
%:%1155=644%:%
%:%1156=645%:%
%:%1163=646%:%
%:%1164=646%:%
%:%1165=647%:%
%:%1166=647%:%
%:%1167=648%:%
%:%1168=648%:%
%:%1169=649%:%
%:%1170=649%:%
%:%1171=650%:%
%:%1172=650%:%
%:%1173=651%:%
%:%1174=651%:%
%:%1175=651%:%
%:%1176=652%:%
%:%1177=652%:%
%:%1178=653%:%
%:%1179=653%:%
%:%1180=654%:%
%:%1181=654%:%
%:%1182=655%:%
%:%1183=655%:%
%:%1184=655%:%
%:%1185=656%:%
%:%1186=656%:%
%:%1187=656%:%
%:%1188=657%:%
%:%1189=657%:%
%:%1190=657%:%
%:%1191=658%:%
%:%1192=658%:%
%:%1193=659%:%
%:%1194=659%:%
%:%1195=659%:%
%:%1196=660%:%
%:%1197=660%:%
%:%1198=661%:%
%:%1204=661%:%
%:%1207=662%:%
%:%1208=663%:%
%:%1209=663%:%
%:%1210=664%:%
%:%1211=665%:%
%:%1218=666%:%
%:%1219=666%:%
%:%1220=667%:%
%:%1221=667%:%
%:%1222=668%:%
%:%1223=668%:%
%:%1224=669%:%
%:%1225=669%:%
%:%1226=670%:%
%:%1227=670%:%
%:%1228=671%:%
%:%1229=671%:%
%:%1230=671%:%
%:%1231=672%:%
%:%1232=672%:%
%:%1233=673%:%
%:%1234=673%:%
%:%1235=674%:%
%:%1236=674%:%
%:%1237=675%:%
%:%1238=675%:%
%:%1239=675%:%
%:%1240=676%:%
%:%1241=676%:%
%:%1242=676%:%
%:%1243=677%:%
%:%1244=677%:%
%:%1245=677%:%
%:%1246=678%:%
%:%1247=678%:%
%:%1248=679%:%
%:%1249=679%:%
%:%1250=679%:%
%:%1251=680%:%
%:%1252=680%:%
%:%1253=681%:%
%:%1259=681%:%
%:%1262=682%:%
%:%1263=683%:%
%:%1264=683%:%
%:%1265=684%:%
%:%1266=685%:%
%:%1273=686%:%
%:%1274=686%:%
%:%1275=687%:%
%:%1276=687%:%
%:%1277=688%:%
%:%1278=688%:%
%:%1279=689%:%
%:%1280=689%:%
%:%1281=690%:%
%:%1282=690%:%
%:%1283=691%:%
%:%1284=691%:%
%:%1285=691%:%
%:%1286=692%:%
%:%1287=692%:%
%:%1288=693%:%
%:%1289=693%:%
%:%1290=694%:%
%:%1291=694%:%
%:%1292=695%:%
%:%1293=695%:%
%:%1294=695%:%
%:%1295=696%:%
%:%1296=696%:%
%:%1297=696%:%
%:%1298=697%:%
%:%1299=697%:%
%:%1300=697%:%
%:%1301=698%:%
%:%1302=698%:%
%:%1303=699%:%
%:%1304=699%:%
%:%1305=699%:%
%:%1306=700%:%
%:%1307=700%:%
%:%1308=701%:%
%:%1318=703%:%
%:%1320=705%:%
%:%1321=705%:%
%:%1322=706%:%
%:%1325=709%:%
%:%1326=710%:%
%:%1329=711%:%
%:%1333=711%:%
%:%1334=711%:%
%:%1335=712%:%
%:%1336=712%:%
%:%1337=713%:%
%:%1338=713%:%
%:%1339=714%:%
%:%1340=714%:%
%:%1341=715%:%
%:%1342=715%:%
%:%1345=718%:%
%:%1346=719%:%
%:%1347=719%:%
%:%1348=719%:%
%:%1349=720%:%
%:%1350=720%:%
%:%1351=720%:%
%:%1352=721%:%
%:%1353=721%:%
%:%1354=722%:%
%:%1355=722%:%
%:%1356=723%:%
%:%1357=723%:%
%:%1358=723%:%
%:%1359=724%:%
%:%1360=724%:%
%:%1361=725%:%
%:%1362=726%:%
%:%1363=726%:%
%:%1364=726%:%
%:%1365=727%:%
%:%1366=727%:%
%:%1367=728%:%
%:%1368=728%:%
%:%1369=728%:%
%:%1370=729%:%
%:%1371=729%:%
%:%1372=730%:%
%:%1373=730%:%
%:%1374=730%:%
%:%1375=731%:%
%:%1376=731%:%
%:%1377=732%:%
%:%1378=732%:%
%:%1379=732%:%
%:%1380=733%:%
%:%1381=733%:%
%:%1382=734%:%
%:%1383=734%:%
%:%1384=734%:%
%:%1385=735%:%
%:%1386=735%:%
%:%1387=736%:%
%:%1388=736%:%
%:%1389=736%:%
%:%1390=737%:%
%:%1391=737%:%
%:%1392=738%:%
%:%1393=738%:%
%:%1394=738%:%
%:%1395=739%:%
%:%1396=739%:%
%:%1397=740%:%
%:%1398=740%:%
%:%1399=740%:%
%:%1400=741%:%
%:%1401=741%:%
%:%1405=745%:%
%:%1406=746%:%
%:%1407=746%:%
%:%1408=746%:%
%:%1409=747%:%
%:%1410=747%:%
%:%1413=750%:%
%:%1414=751%:%
%:%1415=751%:%
%:%1416=751%:%
%:%1417=752%:%
%:%1418=752%:%
%:%1426=760%:%
%:%1427=761%:%
%:%1428=761%:%
%:%1429=761%:%
%:%1430=762%:%
%:%1431=762%:%
%:%1439=770%:%
%:%1440=771%:%
%:%1441=771%:%
%:%1442=772%:%
%:%1452=774%:%
%:%1453=775%:%
%:%1455=777%:%
%:%1456=777%:%
%:%1459=780%:%
%:%1466=781%:%
%:%1467=781%:%
%:%1468=782%:%
%:%1469=782%:%
%:%1470=783%:%
%:%1471=783%:%
%:%1474=786%:%
%:%1475=787%:%
%:%1476=787%:%
%:%1477=788%:%
%:%1478=788%:%
%:%1479=789%:%
%:%1480=789%:%
%:%1483=792%:%
%:%1484=793%:%
%:%1485=793%:%
%:%1486=794%:%
%:%1487=794%:%
%:%1488=795%:%
%:%1498=797%:%
%:%1499=798%:%
%:%1501=800%:%
%:%1502=800%:%
%:%1506=804%:%
%:%1509=805%:%
%:%1513=805%:%
%:%1514=805%:%
%:%1515=806%:%
%:%1516=806%:%
%:%1517=807%:%
%:%1518=807%:%

%
\begin{isabellebody}%
\setisabellecontext{ColecScfc}%
%
\isadelimtheory
%
\endisadelimtheory
%
\isatagtheory
%
\endisatagtheory
{\isafoldtheory}%
%
\isadelimtheory
%
\endisadelimtheory
%
\begin{isamarkuptext}%
En este apartado definiremos colecciones de conjuntos \isa{cerradas\ bajo\ subconjuntos} y de 
  \isa{carácter\ finito}, junto con tres resultados sobre las mismas. El primero de ellos permite
  extender una colección que verifique la propiedad de consistencia proposicional a otra que 
  también la verifique y sea cerrada bajo subconjuntos. Posteriormente probaremos que toda colección
  de carácter finito es cerrada bajo subconjuntos. Finalmente, mostraremos un resultado que 
  permite extender una colección cerrada bajo subconjuntos que verifique la propiedad de
  consistencia proposicional a otra que también verifique dicha propiedad y sea de carácter 
  finito.

  \begin{definicion}
    Una colección de conjuntos es \isa{cerrada\ bajo\ subconjuntos} si todo subconjunto de cada conjunto 
    de la colección pertenece a la colección.
  \end{definicion}

  En Isabelle se formaliza de la siguiente manera.%
\end{isamarkuptext}\isamarkuptrue%
\isacommand{definition}\isamarkupfalse%
\ {\isachardoublequoteopen}subset{\isacharunderscore}closed\ C\ {\isasymequiv}\ {\isacharparenleft}{\isasymforall}S\ {\isasymin}\ C{\isachardot}\ {\isasymforall}S{\isacharprime}{\isasymsubseteq}S{\isachardot}\ S{\isacharprime}\ {\isasymin}\ C{\isacharparenright}{\isachardoublequoteclose}%
\begin{isamarkuptext}%
Mostremos algunos ejemplos para ilustrar la definición. Para ello, veamos si las colecciones
  de conjuntos de fórmulas proposicionales expuestas en los ejemplos anteriores son cerradas bajo 
  subconjuntos.%
\end{isamarkuptext}\isamarkuptrue%
\isacommand{lemma}\isamarkupfalse%
\ {\isachardoublequoteopen}subset{\isacharunderscore}closed\ {\isacharbraceleft}{\isacharbraceleft}{\isacharbraceright}{\isacharbraceright}{\isachardoublequoteclose}\isanewline
%
\isadelimproof
\ \ %
\endisadelimproof
%
\isatagproof
\isacommand{unfolding}\isamarkupfalse%
\ subset{\isacharunderscore}closed{\isacharunderscore}def\ \isacommand{by}\isamarkupfalse%
\ simp%
\endisatagproof
{\isafoldproof}%
%
\isadelimproof
\isanewline
%
\endisadelimproof
\isanewline
\isacommand{lemma}\isamarkupfalse%
\ {\isachardoublequoteopen}{\isasymnot}\ subset{\isacharunderscore}closed\ {\isacharbraceleft}{\isacharbraceleft}Atom\ {\isadigit{0}}{\isacharbraceright}{\isacharbraceright}{\isachardoublequoteclose}\isanewline
%
\isadelimproof
\ \ %
\endisadelimproof
%
\isatagproof
\isacommand{unfolding}\isamarkupfalse%
\ subset{\isacharunderscore}closed{\isacharunderscore}def\ \isacommand{by}\isamarkupfalse%
\ auto%
\endisatagproof
{\isafoldproof}%
%
\isadelimproof
%
\endisadelimproof
%
\begin{isamarkuptext}%
Observemos que, puesto que el conjunto vacío es subconjunto de todo conjunto, una
  condición necesaria para que una colección sea cerrada bajo subconjuntos es que contenga al
  conjunto vacío.%
\end{isamarkuptext}\isamarkuptrue%
\isacommand{lemma}\isamarkupfalse%
\ {\isachardoublequoteopen}subset{\isacharunderscore}closed\ {\isacharbraceleft}{\isacharbraceleft}Atom\ {\isadigit{0}}{\isacharbraceright}{\isacharcomma}{\isacharbraceleft}{\isacharbraceright}{\isacharbraceright}{\isachardoublequoteclose}\isanewline
%
\isadelimproof
\ \ %
\endisadelimproof
%
\isatagproof
\isacommand{unfolding}\isamarkupfalse%
\ subset{\isacharunderscore}closed{\isacharunderscore}def\ \isacommand{by}\isamarkupfalse%
\ auto%
\endisatagproof
{\isafoldproof}%
%
\isadelimproof
%
\endisadelimproof
%
\begin{isamarkuptext}%
De este modo, se deduce fácilmente que el resto de colecciones expuestas en los ejemplos
  anteriores no son cerradas bajo subconjuntos.%
\end{isamarkuptext}\isamarkuptrue%
\isacommand{lemma}\isamarkupfalse%
\ {\isachardoublequoteopen}{\isasymnot}\ subset{\isacharunderscore}closed\ {\isacharbraceleft}{\isacharbraceleft}{\isacharparenleft}\isactrlbold {\isasymnot}\ {\isacharparenleft}Atom\ {\isadigit{1}}{\isacharparenright}{\isacharparenright}\ \isactrlbold {\isasymrightarrow}\ Atom\ {\isadigit{2}}{\isacharbraceright}{\isacharcomma}\isanewline
\ \ \ {\isacharbraceleft}{\isacharparenleft}{\isacharparenleft}\isactrlbold {\isasymnot}\ {\isacharparenleft}Atom\ {\isadigit{1}}{\isacharparenright}{\isacharparenright}\ \isactrlbold {\isasymrightarrow}\ Atom\ {\isadigit{2}}{\isacharparenright}{\isacharcomma}\ \isactrlbold {\isasymnot}{\isacharparenleft}\isactrlbold {\isasymnot}\ {\isacharparenleft}Atom\ {\isadigit{1}}{\isacharparenright}{\isacharparenright}{\isacharbraceright}{\isacharcomma}\isanewline
\ \ {\isacharbraceleft}{\isacharparenleft}{\isacharparenleft}\isactrlbold {\isasymnot}\ {\isacharparenleft}Atom\ {\isadigit{1}}{\isacharparenright}{\isacharparenright}\ \isactrlbold {\isasymrightarrow}\ Atom\ {\isadigit{2}}{\isacharparenright}{\isacharcomma}\ \isactrlbold {\isasymnot}{\isacharparenleft}\isactrlbold {\isasymnot}\ {\isacharparenleft}Atom\ {\isadigit{1}}{\isacharparenright}{\isacharparenright}{\isacharcomma}\ \ Atom\ {\isadigit{1}}{\isacharbraceright}{\isacharbraceright}{\isachardoublequoteclose}\ \isanewline
%
\isadelimproof
\ \ %
\endisadelimproof
%
\isatagproof
\isacommand{unfolding}\isamarkupfalse%
\ subset{\isacharunderscore}closed{\isacharunderscore}def\ \isacommand{by}\isamarkupfalse%
\ auto%
\endisatagproof
{\isafoldproof}%
%
\isadelimproof
\isanewline
%
\endisadelimproof
\isanewline
\isacommand{lemma}\isamarkupfalse%
\ {\isachardoublequoteopen}{\isasymnot}\ subset{\isacharunderscore}closed\ {\isacharbraceleft}{\isacharbraceleft}{\isacharparenleft}\isactrlbold {\isasymnot}\ {\isacharparenleft}Atom\ {\isadigit{1}}{\isacharparenright}{\isacharparenright}\ \isactrlbold {\isasymrightarrow}\ Atom\ {\isadigit{2}}{\isacharbraceright}{\isacharcomma}\isanewline
\ \ \ {\isacharbraceleft}{\isacharparenleft}{\isacharparenleft}\isactrlbold {\isasymnot}\ {\isacharparenleft}Atom\ {\isadigit{1}}{\isacharparenright}{\isacharparenright}\ \isactrlbold {\isasymrightarrow}\ Atom\ {\isadigit{2}}{\isacharparenright}{\isacharcomma}\ \isactrlbold {\isasymnot}{\isacharparenleft}\isactrlbold {\isasymnot}\ {\isacharparenleft}Atom\ {\isadigit{1}}{\isacharparenright}{\isacharparenright}{\isacharbraceright}{\isacharbraceright}{\isachardoublequoteclose}\ \isanewline
%
\isadelimproof
\ \ %
\endisadelimproof
%
\isatagproof
\isacommand{unfolding}\isamarkupfalse%
\ subset{\isacharunderscore}closed{\isacharunderscore}def\ \isacommand{by}\isamarkupfalse%
\ auto%
\endisatagproof
{\isafoldproof}%
%
\isadelimproof
%
\endisadelimproof
%
\begin{isamarkuptext}%
Continuemos con la noción de propiedad de carácter finito.

\begin{definicion}
  Una colección de conjuntos tiene la \isa{propiedad\ de\ carácter\ finito} si para cualquier conjunto
  son equivalentes:
  \begin{enumerate}
    \item El conjunto pertenece a la colección.
    \item Todo subconjunto finito suyo pertenece a la colección.
  \end{enumerate}
\end{definicion}

  La formalización en Isabelle/HOL de dicha definición se muestra a continuación.%
\end{isamarkuptext}\isamarkuptrue%
\isacommand{definition}\isamarkupfalse%
\ {\isachardoublequoteopen}finite{\isacharunderscore}character\ C\ {\isasymequiv}\ \isanewline
\ \ \ \ \ \ \ \ \ \ \ \ {\isacharparenleft}{\isasymforall}S{\isachardot}\ S\ {\isasymin}\ C\ {\isasymlongleftrightarrow}\ {\isacharparenleft}{\isasymforall}S{\isacharprime}\ {\isasymsubseteq}\ S{\isachardot}\ finite\ S{\isacharprime}\ {\isasymlongrightarrow}\ S{\isacharprime}\ {\isasymin}\ C{\isacharparenright}{\isacharparenright}{\isachardoublequoteclose}%
\begin{isamarkuptext}%
Distingamos las colecciones de los ejemplos anteriores que tengan la propiedad de carácter 
  finito. Análogamente, puesto que el conjunto vacío es finito y subconjunto de cualquier conjunto, 
  se observa que una condición necesaria para que una colección tenga la propiedad de carácter 
  finito es que contenga al conjunto vacío.%
\end{isamarkuptext}\isamarkuptrue%
\isacommand{lemma}\isamarkupfalse%
\ {\isachardoublequoteopen}finite{\isacharunderscore}character\ {\isacharbraceleft}{\isacharbraceleft}{\isacharbraceright}{\isacharbraceright}{\isachardoublequoteclose}\isanewline
%
\isadelimproof
\ \ %
\endisadelimproof
%
\isatagproof
\isacommand{unfolding}\isamarkupfalse%
\ finite{\isacharunderscore}character{\isacharunderscore}def\ \isacommand{by}\isamarkupfalse%
\ auto%
\endisatagproof
{\isafoldproof}%
%
\isadelimproof
\isanewline
%
\endisadelimproof
\isanewline
\isacommand{lemma}\isamarkupfalse%
\ {\isachardoublequoteopen}{\isasymnot}\ finite{\isacharunderscore}character\ {\isacharbraceleft}{\isacharbraceleft}Atom\ {\isadigit{0}}{\isacharbraceright}{\isacharbraceright}{\isachardoublequoteclose}\isanewline
%
\isadelimproof
\ \ %
\endisadelimproof
%
\isatagproof
\isacommand{unfolding}\isamarkupfalse%
\ finite{\isacharunderscore}character{\isacharunderscore}def\ \isacommand{by}\isamarkupfalse%
\ auto%
\endisatagproof
{\isafoldproof}%
%
\isadelimproof
\isanewline
%
\endisadelimproof
\isanewline
\isacommand{lemma}\isamarkupfalse%
\ {\isachardoublequoteopen}finite{\isacharunderscore}character\ {\isacharbraceleft}{\isacharbraceleft}Atom\ {\isadigit{0}}{\isacharbraceright}{\isacharcomma}{\isacharbraceleft}{\isacharbraceright}{\isacharbraceright}{\isachardoublequoteclose}\isanewline
%
\isadelimproof
\ \ %
\endisadelimproof
%
\isatagproof
\isacommand{unfolding}\isamarkupfalse%
\ finite{\isacharunderscore}character{\isacharunderscore}def\ \isacommand{by}\isamarkupfalse%
\ auto%
\endisatagproof
{\isafoldproof}%
%
\isadelimproof
%
\endisadelimproof
%
\begin{isamarkuptext}%
Una vez introducidas las definiciones anteriores, veamos los resultados que las relacionan
  con la propiedad de consistencia proposicional. De este modo, combinándolos en la prueba del 
  \isa{teorema\ de\ existencia\ de\ modelo}, dada una colección \isa{C} cualquiera que verifique la propiedad 
  de consistencia proposicional, podemos extenderla a una colección \isa{C{\isacharprime}} que también la verifique y 
  además sea cerrada bajo subconjuntos y de carácter finito.

\comentario{Volver a revisar el párrafo anterior al final de la
redacción de la sección.}

  \begin{lema}
    Toda colección de conjuntos con la propiedad de consistencia proposicional se puede extender a
    una colección que también verifique la propiedad de consistencia proposicional y sea cerrada 
    bajo subconjuntos.
  \end{lema}

  En Isabelle se formaliza el resultado de la siguiente manera.%
\end{isamarkuptext}\isamarkuptrue%
\isacommand{lemma}\isamarkupfalse%
\ {\isachardoublequoteopen}pcp\ C\ {\isasymLongrightarrow}\ {\isasymexists}C{\isacharprime}{\isachardot}\ C\ {\isasymsubseteq}\ C{\isacharprime}\ {\isasymand}\ pcp\ C{\isacharprime}\ {\isasymand}\ subset{\isacharunderscore}closed\ C{\isacharprime}{\isachardoublequoteclose}\isanewline
%
\isadelimproof
\ \ %
\endisadelimproof
%
\isatagproof
\isacommand{oops}\isamarkupfalse%
%
\endisatagproof
{\isafoldproof}%
%
\isadelimproof
%
\endisadelimproof
%
\begin{isamarkuptext}%
Procedamos con su demostración.

\begin{demostracion}
  Dada una colección de conjuntos cualquiera \isa{C}, consideremos la colección formada por los 
  conjuntos tales que son subconjuntos de algún conjunto de \isa{C}. Notemos esta colección por 
  \isa{C{\isacharprime}\ {\isacharequal}\ {\isacharbraceleft}S{\isacharprime}{\isachardot}\ {\isasymexists}S{\isasymin}C{\isachardot}\ S{\isacharprime}\ {\isasymsubseteq}\ S{\isacharbraceright}}. Vamos a probar que, en efecto, \isa{C{\isacharprime}} verifica  las condiciones del lema.

  En primer lugar, veamos que \isa{C} está contenida en \isa{C{\isacharprime}}. Para ello, consideremos un conjunto
  cualquiera perteneciente a \isa{C}. Puesto que la propiedad de contención es reflexiva, dicho conjunto 
  es subconjunto de sí mismo. De este modo, por definición de \isa{C{\isacharprime}}, se verifica que el conjunto 
  pertenece a \isa{C{\isacharprime}}.

  Por otro lado, comprobemos que \isa{C{\isacharprime}} tiene la propiedad de consistencia proposicional.
  Por el lema de caracterización de la propiedad de consistencia proposicional mediante la
  notación uniforme basta probar que, para cualquier conjunto de fórmulas \isa{S} de \isa{C{\isacharprime}}, se 
  verifican las condiciones:
  \begin{itemize}
    \item \isa{{\isasymbottom}} no pertenece a \isa{S}.
    \item Dada \isa{p} una fórmula atómica cualquiera, no se tiene 
    simultáneamente que\\ \isa{p\ {\isasymin}\ S} y \isa{{\isasymnot}\ p\ {\isasymin}\ S}.
    \item Para toda fórmula de tipo \isa{{\isasymalpha}} con componentes \isa{{\isasymalpha}\isactrlsub {\isadigit{1}}} y \isa{{\isasymalpha}\isactrlsub {\isadigit{2}}} tal que \isa{{\isasymalpha}}
    pertenece a \isa{S}, se tiene que \isa{{\isacharbraceleft}{\isasymalpha}\isactrlsub {\isadigit{1}}{\isacharcomma}{\isasymalpha}\isactrlsub {\isadigit{2}}{\isacharbraceright}\ {\isasymunion}\ S} pertenece a \isa{C{\isacharprime}}.
    \item Para toda fórmula de tipo \isa{{\isasymbeta}} con componentes \isa{{\isasymbeta}\isactrlsub {\isadigit{1}}} y \isa{{\isasymbeta}\isactrlsub {\isadigit{2}}} tal que \isa{{\isasymbeta}}
    pertenece a \isa{S}, se tiene que o bien \isa{{\isacharbraceleft}{\isasymbeta}\isactrlsub {\isadigit{1}}{\isacharbraceright}\ {\isasymunion}\ S} pertenece a \isa{C{\isacharprime}} o 
    bien \isa{{\isacharbraceleft}{\isasymbeta}\isactrlsub {\isadigit{2}}{\isacharbraceright}\ {\isasymunion}\ S} pertenece a \isa{C{\isacharprime}}.
  \end{itemize} 

  De este modo, sea \isa{S} un conjunto de fórmulas cualquiera de la colección \isa{C{\isacharprime}}. Por definición de
  dicha colección, existe un conjunto \isa{S{\isacharprime}} pertenciente a \isa{C} tal que \isa{S} está contenido en \isa{S{\isacharprime}}.
  Como \isa{C} tiene la propiedad de consistencia proposicional por hipótesis, verifica las condiciones
  del lema de caracterización de la propiedad de consistencia proposicional mediante la notación 
  uniforme. En particular, puesto que \isa{S{\isacharprime}} pertenece a \isa{C}, se verifica: 
  \begin{itemize}
    \item \isa{{\isasymbottom}} no pertenece a \isa{S{\isacharprime}}.
    \item Dada \isa{p} una fórmula atómica cualquiera, no se tiene 
    simultáneamente que\\ \isa{p\ {\isasymin}\ S{\isacharprime}} y \isa{{\isasymnot}\ p\ {\isasymin}\ S{\isacharprime}}.
    \item Para toda fórmula de tipo \isa{{\isasymalpha}} con componentes \isa{{\isasymalpha}\isactrlsub {\isadigit{1}}} y \isa{{\isasymalpha}\isactrlsub {\isadigit{2}}} tal que \isa{{\isasymalpha}}
    pertenece a \isa{S{\isacharprime}}, se tiene que \isa{{\isacharbraceleft}{\isasymalpha}\isactrlsub {\isadigit{1}}{\isacharcomma}{\isasymalpha}\isactrlsub {\isadigit{2}}{\isacharbraceright}\ {\isasymunion}\ S{\isacharprime}} pertenece a \isa{C}.
    \item Para toda fórmula de tipo \isa{{\isasymbeta}} con componentes \isa{{\isasymbeta}\isactrlsub {\isadigit{1}}} y \isa{{\isasymbeta}\isactrlsub {\isadigit{2}}} tal que \isa{{\isasymbeta}}
    pertenece a \isa{S{\isacharprime}}, se tiene que o bien \isa{{\isacharbraceleft}{\isasymbeta}\isactrlsub {\isadigit{1}}{\isacharbraceright}\ {\isasymunion}\ S{\isacharprime}} pertenece a \isa{C} o 
    bien \isa{{\isacharbraceleft}{\isasymbeta}\isactrlsub {\isadigit{2}}{\isacharbraceright}\ {\isasymunion}\ S{\isacharprime}} pertenece a \isa{C}.
  \end{itemize} 

  Por tanto, como \isa{S} está contenida en \isa{S{\isacharprime}}, se verifica análogamente que \isa{{\isasymbottom}} no pertence a \isa{S}
  y que dada una fórmula atómica cualquiera \isa{p}, no se tiene simultáneamente que\\ \isa{p\ {\isasymin}\ S} y 
  \isa{{\isasymnot}\ p\ {\isasymin}\ S{\isachardot}} Veamos que se verifican el resto de condiciones del lema de caracterización:

  \isa{{\isasymsqdot}\ Condición\ para\ fórmulas\ de\ tipo\ {\isasymalpha}}: Sea una fórmula de tipo \isa{{\isasymalpha}} con componentes \isa{{\isasymalpha}\isactrlsub {\isadigit{1}}} y 
    \isa{{\isasymalpha}\isactrlsub {\isadigit{2}}} tal que \isa{{\isasymalpha}} pertenece a \isa{S}. Como \isa{S} está contenida en \isa{S{\isacharprime}}, tenemos que la fórmula 
    pertence también a \isa{S{\isacharprime}}. De este modo, se verifica que \isa{{\isacharbraceleft}{\isasymalpha}\isactrlsub {\isadigit{1}}{\isacharcomma}{\isasymalpha}\isactrlsub {\isadigit{2}}{\isacharbraceright}\ {\isasymunion}\ S{\isacharprime}} pertenece a la colección 
    \isa{C}. Por otro lado, como el conjunto \isa{S} está contenido en \isa{S{\isacharprime}}, se observa fácilmente que\\
    \isa{{\isacharbraceleft}{\isasymalpha}\isactrlsub {\isadigit{1}}{\isacharcomma}{\isasymalpha}\isactrlsub {\isadigit{2}}{\isacharbraceright}\ {\isasymunion}\ S} está contenido en \isa{{\isacharbraceleft}{\isasymalpha}\isactrlsub {\isadigit{1}}{\isacharcomma}{\isasymalpha}\isactrlsub {\isadigit{2}}{\isacharbraceright}\ {\isasymunion}\ S{\isacharprime}}. Por lo tanto, el conjunto \isa{{\isacharbraceleft}{\isasymalpha}\isactrlsub {\isadigit{1}}{\isacharcomma}{\isasymalpha}\isactrlsub {\isadigit{2}}{\isacharbraceright}\ {\isasymunion}\ S} está en 
    \isa{C{\isacharprime}} por definición de esta, ya que es subconjunto de \isa{{\isacharbraceleft}{\isasymalpha}\isactrlsub {\isadigit{1}}{\isacharcomma}{\isasymalpha}\isactrlsub {\isadigit{2}}{\isacharbraceright}\ {\isasymunion}\ S{\isacharprime}} que pertence a \isa{C}.

  \isa{{\isasymsqdot}\ Condición\ para\ fórmulas\ de\ tipo\ {\isasymbeta}}: Sea una fórmula de tipo \isa{{\isasymbeta}} con componentes \isa{{\isasymbeta}\isactrlsub {\isadigit{1}}} y
    \isa{{\isasymbeta}\isactrlsub {\isadigit{2}}} tal que la fórmula pertenece a \isa{S}. Como el conjunto \isa{S} está contenido en \isa{S{\isacharprime}}, tenemos 
    que la fórmula pertence, a su vez, a \isa{S{\isacharprime}}. De este modo, se verifica que o bien \isa{{\isacharbraceleft}{\isasymbeta}\isactrlsub {\isadigit{1}}{\isacharbraceright}\ {\isasymunion}\ S{\isacharprime}}
    pertenece a \isa{C} o bien \isa{{\isacharbraceleft}{\isasymbeta}\isactrlsub {\isadigit{2}}{\isacharbraceright}\ {\isasymunion}\ S{\isacharprime}} pertence a \isa{C}. Por eliminación de la disyunción anterior, 
    vamos a probar que o bien \isa{{\isacharbraceleft}{\isasymbeta}\isactrlsub {\isadigit{1}}{\isacharbraceright}\ {\isasymunion}\ S} pertenece a \isa{C{\isacharprime}} o bien \isa{{\isacharbraceleft}{\isasymbeta}\isactrlsub {\isadigit{2}}{\isacharbraceright}\ {\isasymunion}\ S} pertenece a \isa{C{\isacharprime}}.
    \begin{itemize}
      \item Supongamos, en primer lugar, que \isa{{\isacharbraceleft}{\isasymbeta}\isactrlsub {\isadigit{1}}{\isacharbraceright}\ {\isasymunion}\ S{\isacharprime}} pertenece a \isa{C}. Puesto que el conjunto \isa{S}
      está contenido en \isa{S{\isacharprime}}, se observa fácilmente que \isa{{\isacharbraceleft}{\isasymbeta}\isactrlsub {\isadigit{1}}{\isacharbraceright}\ {\isasymunion}\ S} está contenido en\\ \isa{{\isacharbraceleft}{\isasymbeta}\isactrlsub {\isadigit{1}}{\isacharbraceright}\ {\isasymunion}\ S{\isacharprime}}.
      Por definición de la colección \isa{C{\isacharprime}}, tenemos que \isa{{\isacharbraceleft}{\isasymbeta}\isactrlsub {\isadigit{1}}{\isacharbraceright}\ {\isasymunion}\ S} pertenece a \isa{C{\isacharprime}}, ya que es
      subconjunto de \isa{{\isacharbraceleft}{\isasymbeta}\isactrlsub {\isadigit{1}}{\isacharbraceright}\ {\isasymunion}\ S{\isacharprime}} que pertenece a \isa{C}. Por tanto, hemos probado que o bien \isa{{\isacharbraceleft}{\isasymbeta}\isactrlsub {\isadigit{1}}{\isacharbraceright}\ {\isasymunion}\ S} 
      pertenece a \isa{C{\isacharprime}} o bien \isa{{\isacharbraceleft}{\isasymbeta}\isactrlsub {\isadigit{2}}{\isacharbraceright}\ {\isasymunion}\ S} pertenece a \isa{C{\isacharprime}}.
      \item Supongamos, finalmente, que \isa{{\isacharbraceleft}{\isasymbeta}\isactrlsub {\isadigit{2}}{\isacharbraceright}\ {\isasymunion}\ S{\isacharprime}} pertenece a \isa{C}. Análogamente obtenemos que
      \isa{{\isacharbraceleft}{\isasymbeta}\isactrlsub {\isadigit{2}}{\isacharbraceright}\ {\isasymunion}\ S} está contenido en \isa{{\isacharbraceleft}{\isasymbeta}\isactrlsub {\isadigit{2}}{\isacharbraceright}\ {\isasymunion}\ S{\isacharprime}}, luego \isa{{\isacharbraceleft}{\isasymbeta}\isactrlsub {\isadigit{2}}{\isacharbraceright}\ {\isasymunion}\ S} pertenece a \isa{C{\isacharprime}} por definición.
      Por tanto, o bien \isa{{\isacharbraceleft}{\isasymbeta}\isactrlsub {\isadigit{1}}{\isacharbraceright}\ {\isasymunion}\ S} pertenece a \isa{C{\isacharprime}} o bien \isa{{\isacharbraceleft}{\isasymbeta}\isactrlsub {\isadigit{2}}{\isacharbraceright}\ {\isasymunion}\ S} pertenece a \isa{C{\isacharprime}}.
    \end{itemize}
    De esta manera, queda probado que dada una fórmula de tipo \isa{{\isasymbeta}} y componentes \isa{{\isasymbeta}\isactrlsub {\isadigit{1}}} y \isa{{\isasymbeta}\isactrlsub {\isadigit{2}}} tal que
    pertenezca al conjunto \isa{S}, se verifica que o bien \isa{{\isacharbraceleft}{\isasymbeta}\isactrlsub {\isadigit{1}}{\isacharbraceright}\ {\isasymunion}\ S} pertenece a \isa{C{\isacharprime}} o bien \isa{{\isacharbraceleft}{\isasymbeta}\isactrlsub {\isadigit{2}}{\isacharbraceright}\ {\isasymunion}\ S}
    pertenece a \isa{C{\isacharprime}}.

  En conclusión, por el lema de caracterización de la propiedad de consistencia proposicional
  mediante la notación uniforme, queda probado que \isa{C{\isacharprime}} tiene la propiedad de consistencia
  proposicional. 

  Finalmente probemos que, además, \isa{C{\isacharprime}} es cerrada bajo subconjuntos. Por definición de ser cerrado
  bajo subconjuntos, basta probar que dado un conjunto perteneciente a \isa{C{\isacharprime}} verifica que todo 
  subconjunto suyo pertenece a \isa{C{\isacharprime}}. Consideremos \isa{S} un conjunto cualquiera de \isa{C{\isacharprime}}. Por
  definición de \isa{C{\isacharprime}}, existe un conjunto \isa{S{\isacharprime}} perteneciente a la colección \isa{C} tal que \isa{S} es
  subconjunto de \isa{S{\isacharprime}}. Sea \isa{S{\isacharprime}{\isacharprime}} un subconjunto cualquiera de \isa{S}. Como \isa{S} es subconjunto de \isa{S{\isacharprime}},
  se tiene que \isa{S{\isacharprime}{\isacharprime}} es, a su vez, subconjunto de \isa{S{\isacharprime}}. De este modo, existe un conjunto 
  perteneciente a la colección \isa{C} del cual \isa{S{\isacharprime}{\isacharprime}} es subconjunto. Por tanto, por definición de \isa{C{\isacharprime}}, 
  \isa{S{\isacharprime}{\isacharprime}} pertenece a la colección \isa{C{\isacharprime}}, como quería demostrar.
\end{demostracion}

  Procedamos con las demostraciones del lema en Isabelle/HOL.

  En primer lugar, vamos a introducir dos lemas auxiliares que emplearemos a lo largo de
  esta sección. El primero se trata de un lema similar al lema \isa{ballI} definido en Isabelle pero 
  considerando la relación de contención en lugar de la de pertenencia.%
\end{isamarkuptext}\isamarkuptrue%
\isacommand{lemma}\isamarkupfalse%
\ sallI{\isacharcolon}\ {\isachardoublequoteopen}{\isacharparenleft}{\isasymAnd}S{\isachardot}\ S\ {\isasymsubseteq}\ A\ {\isasymLongrightarrow}\ P\ S{\isacharparenright}\ {\isasymLongrightarrow}\ {\isasymforall}S\ {\isasymsubseteq}\ A{\isachardot}\ P\ S{\isachardoublequoteclose}\isanewline
%
\isadelimproof
\ \ %
\endisadelimproof
%
\isatagproof
\isacommand{by}\isamarkupfalse%
\ simp%
\endisatagproof
{\isafoldproof}%
%
\isadelimproof
%
\endisadelimproof
%
\begin{isamarkuptext}%
Por último definimos el siguiente lema auxiliar similar al lema \isa{bspec} de Isabelle/HOL
  considerando, análogamente, la relación de contención en lugar de la de pertenencia.%
\end{isamarkuptext}\isamarkuptrue%
\isacommand{lemma}\isamarkupfalse%
\ sspec{\isacharcolon}\ {\isachardoublequoteopen}{\isasymforall}S\ {\isasymsubseteq}\ A{\isachardot}\ P\ S\ {\isasymLongrightarrow}\ S\ {\isasymsubseteq}\ A\ {\isasymLongrightarrow}\ P\ S{\isachardoublequoteclose}\isanewline
%
\isadelimproof
\ \ %
\endisadelimproof
%
\isatagproof
\isacommand{by}\isamarkupfalse%
\ simp%
\endisatagproof
{\isafoldproof}%
%
\isadelimproof
%
\endisadelimproof
%
\begin{isamarkuptext}%
Veamos la prueba detallada del lema en Isabelle/HOL. Esta se fundamenta en tres lemas
  auxiliares: el primero prueba que la colección \isa{C} está contenida en \isa{C{\isacharprime}}, el segundo que
  \isa{C{\isacharprime}} tiene la propiedad de consistencia proposicional y, finalmente, el tercer lema demuestra que
  \isa{C{\isacharprime}} es cerrada bajo subconjuntos. En primer lugar, dada una colección cualquiera \isa{C}, definiremos 
  en Isabelle su extensión \isa{C{\isacharprime}} como sigue.%
\end{isamarkuptext}\isamarkuptrue%
\isacommand{definition}\isamarkupfalse%
\ extensionSC\ {\isacharcolon}{\isacharcolon}\ {\isachardoublequoteopen}{\isacharparenleft}{\isacharparenleft}{\isacharprime}a\ formula{\isacharparenright}\ set{\isacharparenright}\ set\ {\isasymRightarrow}\ {\isacharparenleft}{\isacharparenleft}{\isacharprime}a\ formula{\isacharparenright}\ set{\isacharparenright}\ set{\isachardoublequoteclose}\isanewline
\ \ \isakeyword{where}\ extensionSC{\isacharcolon}\ {\isachardoublequoteopen}extensionSC\ C\ {\isacharequal}\ {\isacharbraceleft}s{\isachardot}\ {\isasymexists}S{\isasymin}C{\isachardot}\ s\ {\isasymsubseteq}\ S{\isacharbraceright}{\isachardoublequoteclose}%
\begin{isamarkuptext}%
Una vez formalizada la extensión en Isabelle, comencemos probando de manera detallada que toda
  colección está contenida en su extensión así definida.%
\end{isamarkuptext}\isamarkuptrue%
\isacommand{lemma}\isamarkupfalse%
\ ex{\isadigit{1}}{\isacharunderscore}subset{\isacharcolon}\ {\isachardoublequoteopen}C\ {\isasymsubseteq}\ {\isacharparenleft}extensionSC\ C{\isacharparenright}{\isachardoublequoteclose}\isanewline
%
\isadelimproof
%
\endisadelimproof
%
\isatagproof
\isacommand{proof}\isamarkupfalse%
\ {\isacharparenleft}rule\ subsetI{\isacharparenright}\isanewline
\ \ \isacommand{fix}\isamarkupfalse%
\ s\isanewline
\ \ \isacommand{assume}\isamarkupfalse%
\ {\isachardoublequoteopen}s\ {\isasymin}\ C{\isachardoublequoteclose}\isanewline
\ \ \isacommand{have}\isamarkupfalse%
\ {\isachardoublequoteopen}s\ {\isasymsubseteq}\ s{\isachardoublequoteclose}\isanewline
\ \ \ \ \isacommand{by}\isamarkupfalse%
\ {\isacharparenleft}rule\ subset{\isacharunderscore}refl{\isacharparenright}\isanewline
\ \ \isacommand{then}\isamarkupfalse%
\ \isacommand{have}\isamarkupfalse%
\ {\isachardoublequoteopen}{\isasymexists}S{\isasymin}C{\isachardot}\ s\ {\isasymsubseteq}\ S{\isachardoublequoteclose}\isanewline
\ \ \ \ \isacommand{using}\isamarkupfalse%
\ {\isacartoucheopen}s\ {\isasymin}\ C{\isacartoucheclose}\ \isacommand{by}\isamarkupfalse%
\ {\isacharparenleft}rule\ bexI{\isacharparenright}\isanewline
\ \ \isacommand{thus}\isamarkupfalse%
\ {\isachardoublequoteopen}s\ {\isasymin}\ {\isacharparenleft}extensionSC\ C{\isacharparenright}{\isachardoublequoteclose}\isanewline
\ \ \ \ \isacommand{by}\isamarkupfalse%
\ {\isacharparenleft}simp\ only{\isacharcolon}\ mem{\isacharunderscore}Collect{\isacharunderscore}eq\ extensionSC{\isacharparenright}\isanewline
\isacommand{qed}\isamarkupfalse%
%
\endisatagproof
{\isafoldproof}%
%
\isadelimproof
%
\endisadelimproof
%
\begin{isamarkuptext}%
Prosigamos con la prueba del lema auxiliar que demuestra que \isa{C{\isacharprime}} tiene la propiedad
  de consistencia proposicional. Para ello, emplearemos un lema auxiliar que amplia el lema de 
  Isabelle \isa{insert{\isacharunderscore}is{\isacharunderscore}Un} para la unión de dos elementos y un conjunto, como se muestra a 
  continuación.%
\end{isamarkuptext}\isamarkuptrue%
\isacommand{lemma}\isamarkupfalse%
\ insertSetElem{\isacharcolon}\ {\isachardoublequoteopen}insert\ a\ {\isacharparenleft}insert\ b\ C{\isacharparenright}\ {\isacharequal}\ {\isacharbraceleft}a{\isacharcomma}b{\isacharbraceright}\ {\isasymunion}\ C{\isachardoublequoteclose}\isanewline
%
\isadelimproof
\ \ %
\endisadelimproof
%
\isatagproof
\isacommand{by}\isamarkupfalse%
\ simp%
\endisatagproof
{\isafoldproof}%
%
\isadelimproof
%
\endisadelimproof
%
\begin{isamarkuptext}%
Una vez introducido dicho lema auxiliar, podemos dar la prueba detallada del lema que 
  demuestra que \isa{C{\isacharprime}} tiene la propiedad de consistencia proposicional.%
\end{isamarkuptext}\isamarkuptrue%
\isacommand{lemma}\isamarkupfalse%
\ ex{\isadigit{1}}{\isacharunderscore}pcp{\isacharcolon}\ \isanewline
\ \ \isakeyword{assumes}\ {\isachardoublequoteopen}pcp\ C{\isachardoublequoteclose}\isanewline
\ \ \isakeyword{shows}\ {\isachardoublequoteopen}pcp\ {\isacharparenleft}extensionSC\ C{\isacharparenright}{\isachardoublequoteclose}\isanewline
%
\isadelimproof
%
\endisadelimproof
%
\isatagproof
\isacommand{proof}\isamarkupfalse%
\ {\isacharminus}\isanewline
\ \ \isacommand{have}\isamarkupfalse%
\ C{\isadigit{1}}{\isacharcolon}\ {\isachardoublequoteopen}C\ {\isasymsubseteq}\ {\isacharparenleft}extensionSC\ C{\isacharparenright}{\isachardoublequoteclose}\isanewline
\ \ \ \ \isacommand{by}\isamarkupfalse%
\ {\isacharparenleft}rule\ ex{\isadigit{1}}{\isacharunderscore}subset{\isacharparenright}\isanewline
\ \ \isacommand{show}\isamarkupfalse%
\ {\isachardoublequoteopen}pcp\ {\isacharparenleft}extensionSC\ C{\isacharparenright}{\isachardoublequoteclose}\isanewline
\ \ \isacommand{proof}\isamarkupfalse%
\ {\isacharparenleft}rule\ pcp{\isacharunderscore}alt{\isadigit{2}}{\isacharparenright}\isanewline
\ \ \ \ \isacommand{show}\isamarkupfalse%
\ {\isachardoublequoteopen}{\isasymforall}S\ {\isasymin}\ {\isacharparenleft}extensionSC\ C{\isacharparenright}{\isachardot}\ {\isacharparenleft}{\isasymbottom}\ {\isasymnotin}\ S\isanewline
\ \ \ \ {\isasymand}\ {\isacharparenleft}{\isasymforall}k{\isachardot}\ Atom\ k\ {\isasymin}\ S\ {\isasymlongrightarrow}\ \isactrlbold {\isasymnot}\ {\isacharparenleft}Atom\ k{\isacharparenright}\ {\isasymin}\ S\ {\isasymlongrightarrow}\ False{\isacharparenright}\isanewline
\ \ \ \ {\isasymand}\ {\isacharparenleft}{\isasymforall}F\ G\ H{\isachardot}\ Con\ F\ G\ H\ {\isasymlongrightarrow}\ F\ {\isasymin}\ S\ {\isasymlongrightarrow}\ {\isacharbraceleft}G{\isacharcomma}H{\isacharbraceright}\ {\isasymunion}\ S\ {\isasymin}\ {\isacharparenleft}extensionSC\ C{\isacharparenright}{\isacharparenright}\isanewline
\ \ \ \ {\isasymand}\ {\isacharparenleft}{\isasymforall}F\ G\ H{\isachardot}\ Dis\ F\ G\ H\ {\isasymlongrightarrow}\ F\ {\isasymin}\ S\ {\isasymlongrightarrow}\ {\isacharbraceleft}G{\isacharbraceright}\ {\isasymunion}\ S\ {\isasymin}\ {\isacharparenleft}extensionSC\ C{\isacharparenright}\ {\isasymor}\ {\isacharbraceleft}H{\isacharbraceright}\ {\isasymunion}\ S\ {\isasymin}\ {\isacharparenleft}extensionSC\ C{\isacharparenright}{\isacharparenright}{\isacharparenright}{\isachardoublequoteclose}\isanewline
\ \ \ \ \isacommand{proof}\isamarkupfalse%
\ {\isacharparenleft}rule\ ballI{\isacharparenright}\isanewline
\ \ \ \ \ \ \isacommand{fix}\isamarkupfalse%
\ S{\isacharprime}\isanewline
\ \ \ \ \ \ \isacommand{assume}\isamarkupfalse%
\ {\isachardoublequoteopen}S{\isacharprime}\ {\isasymin}\ {\isacharparenleft}extensionSC\ C{\isacharparenright}{\isachardoublequoteclose}\isanewline
\ \ \ \ \ \ \isacommand{then}\isamarkupfalse%
\ \isacommand{have}\isamarkupfalse%
\ {\isadigit{1}}{\isacharcolon}{\isachardoublequoteopen}{\isasymexists}S\ {\isasymin}\ C{\isachardot}\ S{\isacharprime}\ {\isasymsubseteq}\ S{\isachardoublequoteclose}\isanewline
\ \ \ \ \ \ \ \ \isacommand{unfolding}\isamarkupfalse%
\ extensionSC\ \isacommand{by}\isamarkupfalse%
\ {\isacharparenleft}rule\ CollectD{\isacharparenright}\ \ \isanewline
\ \ \ \ \ \ \isacommand{obtain}\isamarkupfalse%
\ S\ \isakeyword{where}\ {\isachardoublequoteopen}S\ {\isasymin}\ C{\isachardoublequoteclose}\ {\isachardoublequoteopen}S{\isacharprime}\ {\isasymsubseteq}\ S{\isachardoublequoteclose}\isanewline
\ \ \ \ \ \ \ \ \isacommand{using}\isamarkupfalse%
\ {\isadigit{1}}\ \isacommand{by}\isamarkupfalse%
\ {\isacharparenleft}rule\ bexE{\isacharparenright}\isanewline
\ \ \ \ \ \ \isacommand{have}\isamarkupfalse%
\ {\isachardoublequoteopen}{\isasymforall}S\ {\isasymin}\ C{\isachardot}\isanewline
\ \ \ \ \ \ {\isasymbottom}\ {\isasymnotin}\ S\isanewline
\ \ \ \ \ \ {\isasymand}\ {\isacharparenleft}{\isasymforall}k{\isachardot}\ Atom\ k\ {\isasymin}\ S\ {\isasymlongrightarrow}\ \isactrlbold {\isasymnot}\ {\isacharparenleft}Atom\ k{\isacharparenright}\ {\isasymin}\ S\ {\isasymlongrightarrow}\ False{\isacharparenright}\isanewline
\ \ \ \ \ \ {\isasymand}\ {\isacharparenleft}{\isasymforall}F\ G\ H{\isachardot}\ Con\ F\ G\ H\ {\isasymlongrightarrow}\ F\ {\isasymin}\ S\ {\isasymlongrightarrow}\ {\isacharbraceleft}G{\isacharcomma}H{\isacharbraceright}\ {\isasymunion}\ S\ {\isasymin}\ C{\isacharparenright}\isanewline
\ \ \ \ \ \ {\isasymand}\ {\isacharparenleft}{\isasymforall}F\ G\ H{\isachardot}\ Dis\ F\ G\ H\ {\isasymlongrightarrow}\ F\ {\isasymin}\ S\ {\isasymlongrightarrow}\ {\isacharbraceleft}G{\isacharbraceright}\ {\isasymunion}\ S\ {\isasymin}\ C\ {\isasymor}\ {\isacharbraceleft}H{\isacharbraceright}\ {\isasymunion}\ S\ {\isasymin}\ C{\isacharparenright}{\isachardoublequoteclose}\isanewline
\ \ \ \ \ \ \ \ \isacommand{using}\isamarkupfalse%
\ assms\ \isacommand{by}\isamarkupfalse%
\ {\isacharparenleft}rule\ pcp{\isacharunderscore}alt{\isadigit{1}}{\isacharparenright}\isanewline
\ \ \ \ \ \ \isacommand{then}\isamarkupfalse%
\ \isacommand{have}\isamarkupfalse%
\ H{\isacharcolon}{\isachardoublequoteopen}{\isasymbottom}\ {\isasymnotin}\ S\isanewline
\ \ \ \ \ \ {\isasymand}\ {\isacharparenleft}{\isasymforall}k{\isachardot}\ Atom\ k\ {\isasymin}\ S\ {\isasymlongrightarrow}\ \isactrlbold {\isasymnot}\ {\isacharparenleft}Atom\ k{\isacharparenright}\ {\isasymin}\ S\ {\isasymlongrightarrow}\ False{\isacharparenright}\isanewline
\ \ \ \ \ \ {\isasymand}\ {\isacharparenleft}{\isasymforall}F\ G\ H{\isachardot}\ Con\ F\ G\ H\ {\isasymlongrightarrow}\ F\ {\isasymin}\ S\ {\isasymlongrightarrow}\ {\isacharbraceleft}G{\isacharcomma}H{\isacharbraceright}\ {\isasymunion}\ S\ {\isasymin}\ C{\isacharparenright}\isanewline
\ \ \ \ \ \ {\isasymand}\ {\isacharparenleft}{\isasymforall}F\ G\ H{\isachardot}\ Dis\ F\ G\ H\ {\isasymlongrightarrow}\ F\ {\isasymin}\ S\ {\isasymlongrightarrow}\ {\isacharbraceleft}G{\isacharbraceright}\ {\isasymunion}\ S\ {\isasymin}\ C\ {\isasymor}\ {\isacharbraceleft}H{\isacharbraceright}\ {\isasymunion}\ S\ {\isasymin}\ C{\isacharparenright}{\isachardoublequoteclose}\isanewline
\ \ \ \ \ \ \ \ \isacommand{using}\isamarkupfalse%
\ {\isacartoucheopen}S\ {\isasymin}\ C{\isacartoucheclose}\ \isacommand{by}\isamarkupfalse%
\ {\isacharparenleft}rule\ bspec{\isacharparenright}\isanewline
\ \ \ \ \ \ \isacommand{then}\isamarkupfalse%
\ \isacommand{have}\isamarkupfalse%
\ {\isachardoublequoteopen}{\isasymbottom}\ {\isasymnotin}\ S{\isachardoublequoteclose}\isanewline
\ \ \ \ \ \ \ \ \isacommand{by}\isamarkupfalse%
\ {\isacharparenleft}rule\ conjunct{\isadigit{1}}{\isacharparenright}\isanewline
\ \ \ \ \ \ \isacommand{have}\isamarkupfalse%
\ S{\isadigit{1}}{\isacharcolon}{\isachardoublequoteopen}{\isasymbottom}\ {\isasymnotin}\ S{\isacharprime}{\isachardoublequoteclose}\isanewline
\ \ \ \ \ \ \ \ \isacommand{using}\isamarkupfalse%
\ {\isacartoucheopen}S{\isacharprime}\ {\isasymsubseteq}\ S{\isacartoucheclose}\ {\isacartoucheopen}{\isasymbottom}\ {\isasymnotin}\ S{\isacartoucheclose}\ \isacommand{by}\isamarkupfalse%
\ {\isacharparenleft}rule\ contra{\isacharunderscore}subsetD{\isacharparenright}\isanewline
\ \ \ \ \ \ \isacommand{have}\isamarkupfalse%
\ Atom{\isacharcolon}{\isachardoublequoteopen}{\isasymforall}k{\isachardot}\ Atom\ k\ {\isasymin}\ S\ {\isasymlongrightarrow}\ \isactrlbold {\isasymnot}\ {\isacharparenleft}Atom\ k{\isacharparenright}\ {\isasymin}\ S\ {\isasymlongrightarrow}\ False{\isachardoublequoteclose}\isanewline
\ \ \ \ \ \ \ \ \isacommand{using}\isamarkupfalse%
\ H\ \isacommand{by}\isamarkupfalse%
\ {\isacharparenleft}iprover\ elim{\isacharcolon}\ conjunct{\isadigit{1}}\ conjunct{\isadigit{2}}{\isacharparenright}\isanewline
\ \ \ \ \ \ \isacommand{have}\isamarkupfalse%
\ S{\isadigit{2}}{\isacharcolon}{\isachardoublequoteopen}{\isasymforall}k{\isachardot}\ Atom\ k\ {\isasymin}\ S{\isacharprime}\ {\isasymlongrightarrow}\ \isactrlbold {\isasymnot}\ {\isacharparenleft}Atom\ k{\isacharparenright}\ {\isasymin}\ S{\isacharprime}\ {\isasymlongrightarrow}\ False{\isachardoublequoteclose}\isanewline
\ \ \ \ \ \ \isacommand{proof}\isamarkupfalse%
\ {\isacharparenleft}rule\ allI{\isacharparenright}\isanewline
\ \ \ \ \ \ \ \ \isacommand{fix}\isamarkupfalse%
\ k\isanewline
\ \ \ \ \ \ \ \ \isacommand{show}\isamarkupfalse%
\ {\isachardoublequoteopen}Atom\ k\ {\isasymin}\ S{\isacharprime}\ {\isasymlongrightarrow}\ \isactrlbold {\isasymnot}\ {\isacharparenleft}Atom\ k{\isacharparenright}\ {\isasymin}\ S{\isacharprime}\ {\isasymlongrightarrow}\ False{\isachardoublequoteclose}\isanewline
\ \ \ \ \ \ \ \ \isacommand{proof}\isamarkupfalse%
\ {\isacharparenleft}rule\ impI{\isacharparenright}{\isacharplus}\isanewline
\ \ \ \ \ \ \ \ \ \ \isacommand{assume}\isamarkupfalse%
\ {\isachardoublequoteopen}Atom\ k\ {\isasymin}\ S{\isacharprime}{\isachardoublequoteclose}\isanewline
\ \ \ \ \ \ \ \ \ \ \isacommand{assume}\isamarkupfalse%
\ {\isachardoublequoteopen}\isactrlbold {\isasymnot}\ {\isacharparenleft}Atom\ k{\isacharparenright}\ {\isasymin}\ S{\isacharprime}{\isachardoublequoteclose}\isanewline
\ \ \ \ \ \ \ \ \ \ \isacommand{have}\isamarkupfalse%
\ {\isachardoublequoteopen}Atom\ k\ {\isasymin}\ S{\isachardoublequoteclose}\ \isanewline
\ \ \ \ \ \ \ \ \ \ \ \ \isacommand{using}\isamarkupfalse%
\ {\isacartoucheopen}S{\isacharprime}\ {\isasymsubseteq}\ S{\isacartoucheclose}\ {\isacartoucheopen}Atom\ k\ {\isasymin}\ S{\isacharprime}{\isacartoucheclose}\ \isacommand{by}\isamarkupfalse%
\ {\isacharparenleft}rule\ set{\isacharunderscore}mp{\isacharparenright}\isanewline
\ \ \ \ \ \ \ \ \ \ \isacommand{have}\isamarkupfalse%
\ {\isachardoublequoteopen}\isactrlbold {\isasymnot}\ {\isacharparenleft}Atom\ k{\isacharparenright}\ {\isasymin}\ S{\isachardoublequoteclose}\isanewline
\ \ \ \ \ \ \ \ \ \ \ \ \isacommand{using}\isamarkupfalse%
\ {\isacartoucheopen}S{\isacharprime}\ {\isasymsubseteq}\ S{\isacartoucheclose}\ {\isacartoucheopen}\isactrlbold {\isasymnot}\ {\isacharparenleft}Atom\ k{\isacharparenright}\ {\isasymin}\ S{\isacharprime}{\isacartoucheclose}\ \isacommand{by}\isamarkupfalse%
\ {\isacharparenleft}rule\ set{\isacharunderscore}mp{\isacharparenright}\isanewline
\ \ \ \ \ \ \ \ \ \ \isacommand{have}\isamarkupfalse%
\ {\isachardoublequoteopen}Atom\ k\ {\isasymin}\ S\ {\isasymlongrightarrow}\ \isactrlbold {\isasymnot}\ {\isacharparenleft}Atom\ k{\isacharparenright}\ {\isasymin}\ S\ {\isasymlongrightarrow}\ False{\isachardoublequoteclose}\isanewline
\ \ \ \ \ \ \ \ \ \ \ \ \isacommand{using}\isamarkupfalse%
\ Atom\ \isacommand{by}\isamarkupfalse%
\ {\isacharparenleft}rule\ allE{\isacharparenright}\isanewline
\ \ \ \ \ \ \ \ \ \ \isacommand{then}\isamarkupfalse%
\ \isacommand{have}\isamarkupfalse%
\ {\isachardoublequoteopen}\isactrlbold {\isasymnot}\ {\isacharparenleft}Atom\ k{\isacharparenright}\ {\isasymin}\ S\ {\isasymlongrightarrow}\ False{\isachardoublequoteclose}\isanewline
\ \ \ \ \ \ \ \ \ \ \ \ \isacommand{using}\isamarkupfalse%
\ {\isacartoucheopen}Atom\ k\ {\isasymin}\ S{\isacartoucheclose}\ \isacommand{by}\isamarkupfalse%
\ {\isacharparenleft}rule\ mp{\isacharparenright}\isanewline
\ \ \ \ \ \ \ \ \ \ \isacommand{thus}\isamarkupfalse%
\ {\isachardoublequoteopen}False{\isachardoublequoteclose}\isanewline
\ \ \ \ \ \ \ \ \ \ \ \ \isacommand{using}\isamarkupfalse%
\ {\isacartoucheopen}\isactrlbold {\isasymnot}\ {\isacharparenleft}Atom\ k{\isacharparenright}\ {\isasymin}\ S{\isacartoucheclose}\ \isacommand{by}\isamarkupfalse%
\ {\isacharparenleft}rule\ mp{\isacharparenright}\isanewline
\ \ \ \ \ \ \ \ \isacommand{qed}\isamarkupfalse%
\isanewline
\ \ \ \ \ \ \isacommand{qed}\isamarkupfalse%
\isanewline
\ \ \ \ \ \ \isacommand{have}\isamarkupfalse%
\ Con{\isacharcolon}{\isachardoublequoteopen}{\isasymforall}F\ G\ H{\isachardot}\ Con\ F\ G\ H\ {\isasymlongrightarrow}\ F\ {\isasymin}\ S\ {\isasymlongrightarrow}\ {\isacharbraceleft}G{\isacharcomma}H{\isacharbraceright}\ {\isasymunion}\ S\ {\isasymin}\ C{\isachardoublequoteclose}\isanewline
\ \ \ \ \ \ \ \ \isacommand{using}\isamarkupfalse%
\ H\ \isacommand{by}\isamarkupfalse%
\ {\isacharparenleft}iprover\ elim{\isacharcolon}\ conjunct{\isadigit{1}}\ conjunct{\isadigit{2}}{\isacharparenright}\isanewline
\ \ \ \ \ \ \isacommand{have}\isamarkupfalse%
\ S{\isadigit{3}}{\isacharcolon}{\isachardoublequoteopen}{\isasymforall}F\ G\ H{\isachardot}\ Con\ F\ G\ H\ {\isasymlongrightarrow}\ F\ {\isasymin}\ S{\isacharprime}\ {\isasymlongrightarrow}\ {\isacharbraceleft}G{\isacharcomma}H{\isacharbraceright}\ {\isasymunion}\ S{\isacharprime}\ {\isasymin}\ {\isacharparenleft}extensionSC\ C{\isacharparenright}{\isachardoublequoteclose}\isanewline
\ \ \ \ \ \ \isacommand{proof}\isamarkupfalse%
\ {\isacharparenleft}rule\ allI{\isacharparenright}{\isacharplus}\isanewline
\ \ \ \ \ \ \ \ \isacommand{fix}\isamarkupfalse%
\ F\ G\ H\isanewline
\ \ \ \ \ \ \ \ \isacommand{show}\isamarkupfalse%
\ {\isachardoublequoteopen}Con\ F\ G\ H\ {\isasymlongrightarrow}\ F\ {\isasymin}\ S{\isacharprime}\ {\isasymlongrightarrow}\ {\isacharbraceleft}G{\isacharcomma}H{\isacharbraceright}\ {\isasymunion}\ S{\isacharprime}\ {\isasymin}\ {\isacharparenleft}extensionSC\ C{\isacharparenright}{\isachardoublequoteclose}\isanewline
\ \ \ \ \ \ \ \ \isacommand{proof}\isamarkupfalse%
\ {\isacharparenleft}rule\ impI{\isacharparenright}{\isacharplus}\isanewline
\ \ \ \ \ \ \ \ \ \ \isacommand{assume}\isamarkupfalse%
\ {\isachardoublequoteopen}Con\ F\ G\ H{\isachardoublequoteclose}\isanewline
\ \ \ \ \ \ \ \ \ \ \isacommand{assume}\isamarkupfalse%
\ {\isachardoublequoteopen}F\ {\isasymin}\ S{\isacharprime}{\isachardoublequoteclose}\isanewline
\ \ \ \ \ \ \ \ \ \ \isacommand{have}\isamarkupfalse%
\ {\isachardoublequoteopen}F\ {\isasymin}\ S{\isachardoublequoteclose}\isanewline
\ \ \ \ \ \ \ \ \ \ \ \ \isacommand{using}\isamarkupfalse%
\ {\isacartoucheopen}S{\isacharprime}\ {\isasymsubseteq}\ S{\isacartoucheclose}\ {\isacartoucheopen}F\ {\isasymin}\ S{\isacharprime}{\isacartoucheclose}\ \isacommand{by}\isamarkupfalse%
\ {\isacharparenleft}rule\ set{\isacharunderscore}mp{\isacharparenright}\isanewline
\ \ \ \ \ \ \ \ \ \ \isacommand{have}\isamarkupfalse%
\ {\isachardoublequoteopen}Con\ F\ G\ H\ {\isasymlongrightarrow}\ F\ {\isasymin}\ S\ {\isasymlongrightarrow}\ {\isacharbraceleft}G{\isacharcomma}H{\isacharbraceright}\ {\isasymunion}\ S\ {\isasymin}\ C{\isachardoublequoteclose}\isanewline
\ \ \ \ \ \ \ \ \ \ \ \ \isacommand{using}\isamarkupfalse%
\ Con\ \isacommand{by}\isamarkupfalse%
\ {\isacharparenleft}iprover\ elim{\isacharcolon}\ allE{\isacharparenright}\isanewline
\ \ \ \ \ \ \ \ \ \ \isacommand{then}\isamarkupfalse%
\ \isacommand{have}\isamarkupfalse%
\ {\isachardoublequoteopen}F\ {\isasymin}\ S\ {\isasymlongrightarrow}\ {\isacharbraceleft}G{\isacharcomma}H{\isacharbraceright}\ {\isasymunion}\ S\ {\isasymin}\ C{\isachardoublequoteclose}\isanewline
\ \ \ \ \ \ \ \ \ \ \ \ \isacommand{using}\isamarkupfalse%
\ {\isacartoucheopen}Con\ F\ G\ H{\isacartoucheclose}\ \isacommand{by}\isamarkupfalse%
\ {\isacharparenleft}rule\ mp{\isacharparenright}\isanewline
\ \ \ \ \ \ \ \ \ \ \isacommand{then}\isamarkupfalse%
\ \isacommand{have}\isamarkupfalse%
\ {\isachardoublequoteopen}{\isacharbraceleft}G{\isacharcomma}H{\isacharbraceright}\ {\isasymunion}\ S\ {\isasymin}\ C{\isachardoublequoteclose}\isanewline
\ \ \ \ \ \ \ \ \ \ \ \ \isacommand{using}\isamarkupfalse%
\ {\isacartoucheopen}F\ {\isasymin}\ S{\isacartoucheclose}\ \isacommand{by}\isamarkupfalse%
\ {\isacharparenleft}rule\ mp{\isacharparenright}\isanewline
\ \ \ \ \ \ \ \ \ \ \isacommand{have}\isamarkupfalse%
\ {\isachardoublequoteopen}S{\isacharprime}\ {\isasymsubseteq}\ insert\ H\ S{\isachardoublequoteclose}\isanewline
\ \ \ \ \ \ \ \ \ \ \ \ \isacommand{using}\isamarkupfalse%
\ {\isacartoucheopen}S{\isacharprime}\ {\isasymsubseteq}\ S{\isacartoucheclose}\ \isacommand{by}\isamarkupfalse%
\ {\isacharparenleft}rule\ subset{\isacharunderscore}insertI{\isadigit{2}}{\isacharparenright}\ \isanewline
\ \ \ \ \ \ \ \ \ \ \isacommand{then}\isamarkupfalse%
\ \isacommand{have}\isamarkupfalse%
\ {\isachardoublequoteopen}insert\ H\ S{\isacharprime}\ {\isasymsubseteq}\ insert\ H\ {\isacharparenleft}insert\ H\ S{\isacharparenright}{\isachardoublequoteclose}\isanewline
\ \ \ \ \ \ \ \ \ \ \ \ \isacommand{by}\isamarkupfalse%
\ {\isacharparenleft}simp\ only{\isacharcolon}\ insert{\isacharunderscore}mono{\isacharparenright}\isanewline
\ \ \ \ \ \ \ \ \ \ \isacommand{then}\isamarkupfalse%
\ \isacommand{have}\isamarkupfalse%
\ {\isachardoublequoteopen}insert\ H\ S{\isacharprime}\ {\isasymsubseteq}\ insert\ H\ S{\isachardoublequoteclose}\isanewline
\ \ \ \ \ \ \ \ \ \ \ \ \isacommand{by}\isamarkupfalse%
\ {\isacharparenleft}simp\ only{\isacharcolon}\ insert{\isacharunderscore}absorb{\isadigit{2}}{\isacharparenright}\isanewline
\ \ \ \ \ \ \ \ \ \ \isacommand{then}\isamarkupfalse%
\ \isacommand{have}\isamarkupfalse%
\ {\isachardoublequoteopen}insert\ G\ {\isacharparenleft}insert\ H\ S{\isacharprime}{\isacharparenright}\ {\isasymsubseteq}\ insert\ G\ {\isacharparenleft}insert\ H\ S{\isacharparenright}{\isachardoublequoteclose}\isanewline
\ \ \ \ \ \ \ \ \ \ \ \ \isacommand{by}\isamarkupfalse%
\ {\isacharparenleft}simp\ only{\isacharcolon}\ insert{\isacharunderscore}mono{\isacharparenright}\isanewline
\ \ \ \ \ \ \ \ \ \ \isacommand{have}\isamarkupfalse%
\ A{\isacharcolon}{\isachardoublequoteopen}insert\ G\ {\isacharparenleft}insert\ H\ S{\isacharprime}{\isacharparenright}\ {\isacharequal}\ {\isacharbraceleft}G{\isacharcomma}H{\isacharbraceright}\ {\isasymunion}\ S{\isacharprime}{\isachardoublequoteclose}\isanewline
\ \ \ \ \ \ \ \ \ \ \ \ \isacommand{by}\isamarkupfalse%
\ {\isacharparenleft}rule\ insertSetElem{\isacharparenright}\ \isanewline
\ \ \ \ \ \ \ \ \ \ \isacommand{have}\isamarkupfalse%
\ B{\isacharcolon}{\isachardoublequoteopen}insert\ G\ {\isacharparenleft}insert\ H\ S{\isacharparenright}\ {\isacharequal}\ {\isacharbraceleft}G{\isacharcomma}H{\isacharbraceright}\ {\isasymunion}\ S{\isachardoublequoteclose}\isanewline
\ \ \ \ \ \ \ \ \ \ \ \ \isacommand{by}\isamarkupfalse%
\ {\isacharparenleft}rule\ insertSetElem{\isacharparenright}\isanewline
\ \ \ \ \ \ \ \ \ \ \isacommand{have}\isamarkupfalse%
\ {\isachardoublequoteopen}{\isacharbraceleft}G{\isacharcomma}H{\isacharbraceright}\ {\isasymunion}\ S{\isacharprime}\ {\isasymsubseteq}\ {\isacharbraceleft}G{\isacharcomma}H{\isacharbraceright}\ {\isasymunion}\ S{\isachardoublequoteclose}\ \isanewline
\ \ \ \ \ \ \ \ \ \ \ \ \isacommand{using}\isamarkupfalse%
\ {\isacartoucheopen}insert\ G\ {\isacharparenleft}insert\ H\ S{\isacharprime}{\isacharparenright}\ {\isasymsubseteq}\ insert\ G\ {\isacharparenleft}insert\ H\ S{\isacharparenright}{\isacartoucheclose}\ \isacommand{by}\isamarkupfalse%
\ {\isacharparenleft}simp\ only{\isacharcolon}\ A\ B{\isacharparenright}\isanewline
\ \ \ \ \ \ \ \ \ \ \isacommand{then}\isamarkupfalse%
\ \isacommand{have}\isamarkupfalse%
\ {\isachardoublequoteopen}{\isasymexists}S\ {\isasymin}\ C{\isachardot}\ {\isacharbraceleft}G{\isacharcomma}H{\isacharbraceright}\ {\isasymunion}\ S{\isacharprime}\ {\isasymsubseteq}\ S{\isachardoublequoteclose}\isanewline
\ \ \ \ \ \ \ \ \ \ \ \ \isacommand{using}\isamarkupfalse%
\ {\isacartoucheopen}{\isacharbraceleft}G{\isacharcomma}H{\isacharbraceright}\ {\isasymunion}\ S\ {\isasymin}\ C{\isacartoucheclose}\ \isacommand{by}\isamarkupfalse%
\ {\isacharparenleft}rule\ bexI{\isacharparenright}\isanewline
\ \ \ \ \ \ \ \ \ \ \isacommand{thus}\isamarkupfalse%
\ {\isachardoublequoteopen}{\isacharbraceleft}G{\isacharcomma}H{\isacharbraceright}\ {\isasymunion}\ S{\isacharprime}\ {\isasymin}\ {\isacharparenleft}extensionSC\ C{\isacharparenright}{\isachardoublequoteclose}\ \isanewline
\ \ \ \ \ \ \ \ \ \ \ \ \isacommand{unfolding}\isamarkupfalse%
\ extensionSC\ \isacommand{by}\isamarkupfalse%
\ {\isacharparenleft}rule\ CollectI{\isacharparenright}\isanewline
\ \ \ \ \ \ \ \ \isacommand{qed}\isamarkupfalse%
\isanewline
\ \ \ \ \ \ \isacommand{qed}\isamarkupfalse%
\isanewline
\ \ \ \ \ \ \isacommand{have}\isamarkupfalse%
\ Dis{\isacharcolon}{\isachardoublequoteopen}{\isasymforall}F\ G\ H{\isachardot}\ Dis\ F\ G\ H\ {\isasymlongrightarrow}\ F\ {\isasymin}\ S\ {\isasymlongrightarrow}\ {\isacharbraceleft}G{\isacharbraceright}\ {\isasymunion}\ S\ {\isasymin}\ C\ {\isasymor}\ {\isacharbraceleft}H{\isacharbraceright}\ {\isasymunion}\ S\ {\isasymin}\ C{\isachardoublequoteclose}\isanewline
\ \ \ \ \ \ \ \ \isacommand{using}\isamarkupfalse%
\ H\ \isacommand{by}\isamarkupfalse%
\ {\isacharparenleft}iprover\ elim{\isacharcolon}\ conjunct{\isadigit{2}}{\isacharparenright}\isanewline
\ \ \ \ \ \ \isacommand{have}\isamarkupfalse%
\ S{\isadigit{4}}{\isacharcolon}{\isachardoublequoteopen}{\isasymforall}F\ G\ H{\isachardot}\ Dis\ F\ G\ H\ {\isasymlongrightarrow}\ F\ {\isasymin}\ S{\isacharprime}\ {\isasymlongrightarrow}\ {\isacharbraceleft}G{\isacharbraceright}\ {\isasymunion}\ S{\isacharprime}\ {\isasymin}\ {\isacharparenleft}extensionSC\ C{\isacharparenright}\ {\isasymor}\ {\isacharbraceleft}H{\isacharbraceright}\ {\isasymunion}\ S{\isacharprime}\ {\isasymin}\ {\isacharparenleft}extensionSC\ C{\isacharparenright}{\isachardoublequoteclose}\isanewline
\ \ \ \ \ \ \isacommand{proof}\isamarkupfalse%
\ {\isacharparenleft}rule\ allI{\isacharparenright}{\isacharplus}\isanewline
\ \ \ \ \ \ \ \ \isacommand{fix}\isamarkupfalse%
\ F\ G\ H\isanewline
\ \ \ \ \ \ \ \ \isacommand{show}\isamarkupfalse%
\ {\isachardoublequoteopen}Dis\ F\ G\ H\ {\isasymlongrightarrow}\ F\ {\isasymin}\ S{\isacharprime}\ {\isasymlongrightarrow}\ {\isacharbraceleft}G{\isacharbraceright}\ {\isasymunion}\ S{\isacharprime}\ {\isasymin}\ {\isacharparenleft}extensionSC\ C{\isacharparenright}\ {\isasymor}\ {\isacharbraceleft}H{\isacharbraceright}\ {\isasymunion}\ S{\isacharprime}\ {\isasymin}\ {\isacharparenleft}extensionSC\ C{\isacharparenright}{\isachardoublequoteclose}\isanewline
\ \ \ \ \ \ \ \ \isacommand{proof}\isamarkupfalse%
\ {\isacharparenleft}rule\ impI{\isacharparenright}{\isacharplus}\isanewline
\ \ \ \ \ \ \ \ \ \ \isacommand{assume}\isamarkupfalse%
\ {\isachardoublequoteopen}Dis\ F\ G\ H{\isachardoublequoteclose}\isanewline
\ \ \ \ \ \ \ \ \ \ \isacommand{assume}\isamarkupfalse%
\ {\isachardoublequoteopen}F\ {\isasymin}\ S{\isacharprime}{\isachardoublequoteclose}\isanewline
\ \ \ \ \ \ \ \ \ \ \isacommand{have}\isamarkupfalse%
\ {\isachardoublequoteopen}F\ {\isasymin}\ S{\isachardoublequoteclose}\isanewline
\ \ \ \ \ \ \ \ \ \ \ \ \isacommand{using}\isamarkupfalse%
\ {\isacartoucheopen}S{\isacharprime}\ {\isasymsubseteq}\ S{\isacartoucheclose}\ {\isacartoucheopen}F\ {\isasymin}\ S{\isacharprime}{\isacartoucheclose}\ \isacommand{by}\isamarkupfalse%
\ {\isacharparenleft}rule\ set{\isacharunderscore}mp{\isacharparenright}\isanewline
\ \ \ \ \ \ \ \ \ \ \isacommand{have}\isamarkupfalse%
\ {\isachardoublequoteopen}Dis\ F\ G\ H\ {\isasymlongrightarrow}\ F\ {\isasymin}\ S\ {\isasymlongrightarrow}\ {\isacharbraceleft}G{\isacharbraceright}\ {\isasymunion}\ S\ {\isasymin}\ C\ {\isasymor}\ {\isacharbraceleft}H{\isacharbraceright}\ {\isasymunion}\ S\ {\isasymin}\ C{\isachardoublequoteclose}\isanewline
\ \ \ \ \ \ \ \ \ \ \ \ \isacommand{using}\isamarkupfalse%
\ Dis\ \isacommand{by}\isamarkupfalse%
\ {\isacharparenleft}iprover\ elim{\isacharcolon}\ allE{\isacharparenright}\isanewline
\ \ \ \ \ \ \ \ \ \ \isacommand{then}\isamarkupfalse%
\ \isacommand{have}\isamarkupfalse%
\ {\isachardoublequoteopen}F\ {\isasymin}\ S\ {\isasymlongrightarrow}\ {\isacharbraceleft}G{\isacharbraceright}\ {\isasymunion}\ S\ {\isasymin}\ C\ {\isasymor}\ {\isacharbraceleft}H{\isacharbraceright}\ {\isasymunion}\ S\ {\isasymin}\ C{\isachardoublequoteclose}\isanewline
\ \ \ \ \ \ \ \ \ \ \ \ \isacommand{using}\isamarkupfalse%
\ {\isacartoucheopen}Dis\ F\ G\ H{\isacartoucheclose}\ \isacommand{by}\isamarkupfalse%
\ {\isacharparenleft}rule\ mp{\isacharparenright}\isanewline
\ \ \ \ \ \ \ \ \ \ \isacommand{then}\isamarkupfalse%
\ \isacommand{have}\isamarkupfalse%
\ {\isadigit{9}}{\isacharcolon}{\isachardoublequoteopen}{\isacharbraceleft}G{\isacharbraceright}\ {\isasymunion}\ S\ {\isasymin}\ C\ {\isasymor}\ {\isacharbraceleft}H{\isacharbraceright}\ {\isasymunion}\ S\ {\isasymin}\ C{\isachardoublequoteclose}\isanewline
\ \ \ \ \ \ \ \ \ \ \ \ \isacommand{using}\isamarkupfalse%
\ {\isacartoucheopen}F\ {\isasymin}\ S{\isacartoucheclose}\ \isacommand{by}\isamarkupfalse%
\ {\isacharparenleft}rule\ mp{\isacharparenright}\isanewline
\ \ \ \ \ \ \ \ \ \ \isacommand{show}\isamarkupfalse%
\ {\isachardoublequoteopen}{\isacharbraceleft}G{\isacharbraceright}\ {\isasymunion}\ S{\isacharprime}\ {\isasymin}\ {\isacharparenleft}extensionSC\ C{\isacharparenright}\ {\isasymor}\ {\isacharbraceleft}H{\isacharbraceright}\ {\isasymunion}\ S{\isacharprime}\ {\isasymin}\ {\isacharparenleft}extensionSC\ C{\isacharparenright}{\isachardoublequoteclose}\isanewline
\ \ \ \ \ \ \ \ \ \ \ \ \isacommand{using}\isamarkupfalse%
\ {\isadigit{9}}\isanewline
\ \ \ \ \ \ \ \ \ \ \isacommand{proof}\isamarkupfalse%
\ {\isacharparenleft}rule\ disjE{\isacharparenright}\isanewline
\ \ \ \ \ \ \ \ \ \ \ \ \isacommand{assume}\isamarkupfalse%
\ {\isachardoublequoteopen}{\isacharbraceleft}G{\isacharbraceright}\ {\isasymunion}\ S\ {\isasymin}\ C{\isachardoublequoteclose}\isanewline
\ \ \ \ \ \ \ \ \ \ \ \ \isacommand{have}\isamarkupfalse%
\ {\isachardoublequoteopen}insert\ G\ S{\isacharprime}\ {\isasymsubseteq}\ insert\ G\ S{\isachardoublequoteclose}\isanewline
\ \ \ \ \ \ \ \ \ \ \ \ \ \ \isacommand{using}\isamarkupfalse%
\ {\isacartoucheopen}S{\isacharprime}\ {\isasymsubseteq}\ S{\isacartoucheclose}\ \isacommand{by}\isamarkupfalse%
\ {\isacharparenleft}simp\ only{\isacharcolon}\ insert{\isacharunderscore}mono{\isacharparenright}\isanewline
\ \ \ \ \ \ \ \ \ \ \ \ \isacommand{have}\isamarkupfalse%
\ C{\isacharcolon}{\isachardoublequoteopen}insert\ G\ S{\isacharprime}\ {\isacharequal}\ {\isacharbraceleft}G{\isacharbraceright}\ {\isasymunion}\ S{\isacharprime}{\isachardoublequoteclose}\isanewline
\ \ \ \ \ \ \ \ \ \ \ \ \ \ \isacommand{by}\isamarkupfalse%
\ {\isacharparenleft}rule\ insert{\isacharunderscore}is{\isacharunderscore}Un{\isacharparenright}\isanewline
\ \ \ \ \ \ \ \ \ \ \ \ \isacommand{have}\isamarkupfalse%
\ D{\isacharcolon}{\isachardoublequoteopen}insert\ G\ S\ {\isacharequal}\ {\isacharbraceleft}G{\isacharbraceright}\ {\isasymunion}\ S{\isachardoublequoteclose}\isanewline
\ \ \ \ \ \ \ \ \ \ \ \ \ \ \isacommand{by}\isamarkupfalse%
\ {\isacharparenleft}rule\ insert{\isacharunderscore}is{\isacharunderscore}Un{\isacharparenright}\isanewline
\ \ \ \ \ \ \ \ \ \ \ \ \isacommand{have}\isamarkupfalse%
\ {\isachardoublequoteopen}{\isacharbraceleft}G{\isacharbraceright}\ {\isasymunion}\ S{\isacharprime}\ {\isasymsubseteq}\ {\isacharbraceleft}G{\isacharbraceright}\ {\isasymunion}\ S{\isachardoublequoteclose}\isanewline
\ \ \ \ \ \ \ \ \ \ \ \ \ \ \isacommand{using}\isamarkupfalse%
\ {\isacartoucheopen}insert\ G\ S{\isacharprime}\ {\isasymsubseteq}\ insert\ G\ S{\isacartoucheclose}\ \isacommand{by}\isamarkupfalse%
\ {\isacharparenleft}simp\ only{\isacharcolon}\ C\ D{\isacharparenright}\isanewline
\ \ \ \ \ \ \ \ \ \ \ \ \isacommand{then}\isamarkupfalse%
\ \isacommand{have}\isamarkupfalse%
\ {\isachardoublequoteopen}{\isasymexists}S\ {\isasymin}\ C{\isachardot}\ {\isacharbraceleft}G{\isacharbraceright}\ {\isasymunion}\ S{\isacharprime}\ {\isasymsubseteq}\ S{\isachardoublequoteclose}\isanewline
\ \ \ \ \ \ \ \ \ \ \ \ \ \ \isacommand{using}\isamarkupfalse%
\ {\isacartoucheopen}{\isacharbraceleft}G{\isacharbraceright}\ {\isasymunion}\ S\ {\isasymin}\ C{\isacartoucheclose}\ \isacommand{by}\isamarkupfalse%
\ {\isacharparenleft}rule\ bexI{\isacharparenright}\isanewline
\ \ \ \ \ \ \ \ \ \ \ \ \isacommand{then}\isamarkupfalse%
\ \isacommand{have}\isamarkupfalse%
\ {\isachardoublequoteopen}{\isacharbraceleft}G{\isacharbraceright}\ {\isasymunion}\ S{\isacharprime}\ {\isasymin}\ {\isacharparenleft}extensionSC\ C{\isacharparenright}{\isachardoublequoteclose}\isanewline
\ \ \ \ \ \ \ \ \ \ \ \ \ \ \isacommand{unfolding}\isamarkupfalse%
\ extensionSC\ \isacommand{by}\isamarkupfalse%
\ {\isacharparenleft}rule\ CollectI{\isacharparenright}\isanewline
\ \ \ \ \ \ \ \ \ \ \ \ \isacommand{thus}\isamarkupfalse%
\ {\isachardoublequoteopen}{\isacharbraceleft}G{\isacharbraceright}\ {\isasymunion}\ S{\isacharprime}\ {\isasymin}\ {\isacharparenleft}extensionSC\ C{\isacharparenright}\ {\isasymor}\ {\isacharbraceleft}H{\isacharbraceright}\ {\isasymunion}\ S{\isacharprime}\ {\isasymin}\ {\isacharparenleft}extensionSC\ C{\isacharparenright}{\isachardoublequoteclose}\isanewline
\ \ \ \ \ \ \ \ \ \ \ \ \ \ \isacommand{by}\isamarkupfalse%
\ {\isacharparenleft}rule\ disjI{\isadigit{1}}{\isacharparenright}\isanewline
\ \ \ \ \ \ \ \ \ \ \isacommand{next}\isamarkupfalse%
\isanewline
\ \ \ \ \ \ \ \ \ \ \ \ \isacommand{assume}\isamarkupfalse%
\ {\isachardoublequoteopen}{\isacharbraceleft}H{\isacharbraceright}\ {\isasymunion}\ S\ {\isasymin}\ C{\isachardoublequoteclose}\isanewline
\ \ \ \ \ \ \ \ \ \ \ \ \isacommand{have}\isamarkupfalse%
\ {\isachardoublequoteopen}insert\ H\ S{\isacharprime}\ {\isasymsubseteq}\ insert\ H\ S{\isachardoublequoteclose}\isanewline
\ \ \ \ \ \ \ \ \ \ \ \ \ \ \isacommand{using}\isamarkupfalse%
\ {\isacartoucheopen}S{\isacharprime}\ {\isasymsubseteq}\ S{\isacartoucheclose}\ \isacommand{by}\isamarkupfalse%
\ {\isacharparenleft}simp\ only{\isacharcolon}\ insert{\isacharunderscore}mono{\isacharparenright}\isanewline
\ \ \ \ \ \ \ \ \ \ \ \ \isacommand{have}\isamarkupfalse%
\ E{\isacharcolon}{\isachardoublequoteopen}insert\ H\ S{\isacharprime}\ {\isacharequal}\ {\isacharbraceleft}H{\isacharbraceright}\ {\isasymunion}\ S{\isacharprime}{\isachardoublequoteclose}\isanewline
\ \ \ \ \ \ \ \ \ \ \ \ \ \ \isacommand{by}\isamarkupfalse%
\ {\isacharparenleft}rule\ insert{\isacharunderscore}is{\isacharunderscore}Un{\isacharparenright}\isanewline
\ \ \ \ \ \ \ \ \ \ \ \ \isacommand{have}\isamarkupfalse%
\ F{\isacharcolon}{\isachardoublequoteopen}insert\ H\ S\ {\isacharequal}\ {\isacharbraceleft}H{\isacharbraceright}\ {\isasymunion}\ S{\isachardoublequoteclose}\isanewline
\ \ \ \ \ \ \ \ \ \ \ \ \ \ \isacommand{by}\isamarkupfalse%
\ {\isacharparenleft}rule\ insert{\isacharunderscore}is{\isacharunderscore}Un{\isacharparenright}\isanewline
\ \ \ \ \ \ \ \ \ \ \ \ \isacommand{then}\isamarkupfalse%
\ \isacommand{have}\isamarkupfalse%
\ {\isachardoublequoteopen}{\isacharbraceleft}H{\isacharbraceright}\ {\isasymunion}\ S{\isacharprime}\ {\isasymsubseteq}\ {\isacharbraceleft}H{\isacharbraceright}\ {\isasymunion}\ S{\isachardoublequoteclose}\isanewline
\ \ \ \ \ \ \ \ \ \ \ \ \ \ \isacommand{using}\isamarkupfalse%
\ {\isacartoucheopen}insert\ H\ S{\isacharprime}\ {\isasymsubseteq}\ insert\ H\ S{\isacartoucheclose}\ \isacommand{by}\isamarkupfalse%
\ {\isacharparenleft}simp\ only{\isacharcolon}\ E\ F{\isacharparenright}\isanewline
\ \ \ \ \ \ \ \ \ \ \ \ \isacommand{then}\isamarkupfalse%
\ \isacommand{have}\isamarkupfalse%
\ {\isachardoublequoteopen}{\isasymexists}S\ {\isasymin}\ C{\isachardot}\ {\isacharbraceleft}H{\isacharbraceright}\ {\isasymunion}\ S{\isacharprime}\ {\isasymsubseteq}\ S{\isachardoublequoteclose}\isanewline
\ \ \ \ \ \ \ \ \ \ \ \ \ \ \isacommand{using}\isamarkupfalse%
\ {\isacartoucheopen}{\isacharbraceleft}H{\isacharbraceright}\ {\isasymunion}\ S\ {\isasymin}\ C{\isacartoucheclose}\ \isacommand{by}\isamarkupfalse%
\ {\isacharparenleft}rule\ bexI{\isacharparenright}\isanewline
\ \ \ \ \ \ \ \ \ \ \ \ \isacommand{then}\isamarkupfalse%
\ \isacommand{have}\isamarkupfalse%
\ {\isachardoublequoteopen}{\isacharbraceleft}H{\isacharbraceright}\ {\isasymunion}\ S{\isacharprime}\ {\isasymin}\ {\isacharparenleft}extensionSC\ C{\isacharparenright}{\isachardoublequoteclose}\isanewline
\ \ \ \ \ \ \ \ \ \ \ \ \ \ \isacommand{unfolding}\isamarkupfalse%
\ extensionSC\ \isacommand{by}\isamarkupfalse%
\ {\isacharparenleft}rule\ CollectI{\isacharparenright}\isanewline
\ \ \ \ \ \ \ \ \ \ \ \ \isacommand{thus}\isamarkupfalse%
\ {\isachardoublequoteopen}{\isacharbraceleft}G{\isacharbraceright}\ {\isasymunion}\ S{\isacharprime}\ {\isasymin}\ {\isacharparenleft}extensionSC\ C{\isacharparenright}\ {\isasymor}\ {\isacharbraceleft}H{\isacharbraceright}\ {\isasymunion}\ S{\isacharprime}\ {\isasymin}\ {\isacharparenleft}extensionSC\ C{\isacharparenright}{\isachardoublequoteclose}\isanewline
\ \ \ \ \ \ \ \ \ \ \ \ \ \ \isacommand{by}\isamarkupfalse%
\ {\isacharparenleft}rule\ disjI{\isadigit{2}}{\isacharparenright}\isanewline
\ \ \ \ \ \ \ \ \ \ \isacommand{qed}\isamarkupfalse%
\isanewline
\ \ \ \ \ \ \ \ \isacommand{qed}\isamarkupfalse%
\isanewline
\ \ \ \ \ \ \isacommand{qed}\isamarkupfalse%
\isanewline
\ \ \ \ \ \ \isacommand{show}\isamarkupfalse%
\ {\isachardoublequoteopen}{\isasymbottom}\ {\isasymnotin}\ S{\isacharprime}\isanewline
\ \ \ \ {\isasymand}\ {\isacharparenleft}{\isasymforall}k{\isachardot}\ Atom\ k\ {\isasymin}\ S{\isacharprime}\ {\isasymlongrightarrow}\ \isactrlbold {\isasymnot}\ {\isacharparenleft}Atom\ k{\isacharparenright}\ {\isasymin}\ S{\isacharprime}\ {\isasymlongrightarrow}\ False{\isacharparenright}\isanewline
\ \ \ \ {\isasymand}\ {\isacharparenleft}{\isasymforall}F\ G\ H{\isachardot}\ Con\ F\ G\ H\ {\isasymlongrightarrow}\ F\ {\isasymin}\ S{\isacharprime}\ {\isasymlongrightarrow}\ {\isacharbraceleft}G{\isacharcomma}H{\isacharbraceright}\ {\isasymunion}\ S{\isacharprime}\ {\isasymin}\ {\isacharparenleft}extensionSC\ C{\isacharparenright}{\isacharparenright}\isanewline
\ \ \ \ {\isasymand}\ {\isacharparenleft}{\isasymforall}F\ G\ H{\isachardot}\ Dis\ F\ G\ H\ {\isasymlongrightarrow}\ F\ {\isasymin}\ S{\isacharprime}\ {\isasymlongrightarrow}\ {\isacharbraceleft}G{\isacharbraceright}\ {\isasymunion}\ S{\isacharprime}\ {\isasymin}\ {\isacharparenleft}extensionSC\ C{\isacharparenright}\ {\isasymor}\ {\isacharbraceleft}H{\isacharbraceright}\ {\isasymunion}\ S{\isacharprime}\ {\isasymin}\ {\isacharparenleft}extensionSC\ C{\isacharparenright}{\isacharparenright}{\isachardoublequoteclose}\isanewline
\ \ \ \ \ \ \ \ \isacommand{using}\isamarkupfalse%
\ S{\isadigit{1}}\ S{\isadigit{2}}\ S{\isadigit{3}}\ S{\isadigit{4}}\ \isacommand{by}\isamarkupfalse%
\ {\isacharparenleft}iprover\ intro{\isacharcolon}\ conjI{\isacharparenright}\isanewline
\ \ \ \ \isacommand{qed}\isamarkupfalse%
\isanewline
\ \ \isacommand{qed}\isamarkupfalse%
\isanewline
\isacommand{qed}\isamarkupfalse%
%
\endisatagproof
{\isafoldproof}%
%
\isadelimproof
%
\endisadelimproof
%
\begin{isamarkuptext}%
Finalmente, el siguiente lema auxiliar prueba que \isa{C{\isacharprime}} es cerrada bajo subconjuntos.%
\end{isamarkuptext}\isamarkuptrue%
\isacommand{lemma}\isamarkupfalse%
\ ex{\isadigit{1}}{\isacharunderscore}subset{\isacharunderscore}closed{\isacharcolon}\isanewline
\ \ \isakeyword{assumes}\ {\isachardoublequoteopen}pcp\ C{\isachardoublequoteclose}\isanewline
\ \ \isakeyword{shows}\ {\isachardoublequoteopen}subset{\isacharunderscore}closed\ {\isacharparenleft}extensionSC\ C{\isacharparenright}{\isachardoublequoteclose}\isanewline
%
\isadelimproof
\ \ %
\endisadelimproof
%
\isatagproof
\isacommand{unfolding}\isamarkupfalse%
\ subset{\isacharunderscore}closed{\isacharunderscore}def\isanewline
\isacommand{proof}\isamarkupfalse%
\ {\isacharparenleft}rule\ ballI{\isacharparenright}\isanewline
\ \ \isacommand{fix}\isamarkupfalse%
\ S{\isacharprime}\isanewline
\ \ \isacommand{assume}\isamarkupfalse%
\ {\isachardoublequoteopen}S{\isacharprime}\ {\isasymin}\ {\isacharparenleft}extensionSC\ C{\isacharparenright}{\isachardoublequoteclose}\isanewline
\ \ \isacommand{then}\isamarkupfalse%
\ \isacommand{have}\isamarkupfalse%
\ H{\isacharcolon}{\isachardoublequoteopen}{\isasymexists}S\ {\isasymin}\ C{\isachardot}\ S{\isacharprime}\ {\isasymsubseteq}\ S{\isachardoublequoteclose}\isanewline
\ \ \ \ \isacommand{unfolding}\isamarkupfalse%
\ extensionSC\ \isacommand{by}\isamarkupfalse%
\ {\isacharparenleft}rule\ CollectD{\isacharparenright}\isanewline
\ \ \isacommand{obtain}\isamarkupfalse%
\ S\ \isakeyword{where}\ {\isacartoucheopen}S\ {\isasymin}\ C{\isacartoucheclose}\ \isakeyword{and}\ {\isacartoucheopen}S{\isacharprime}\ {\isasymsubseteq}\ S{\isacartoucheclose}\ \isanewline
\ \ \ \ \isacommand{using}\isamarkupfalse%
\ H\ \isacommand{by}\isamarkupfalse%
\ {\isacharparenleft}rule\ bexE{\isacharparenright}\ \isanewline
\ \ \isacommand{show}\isamarkupfalse%
\ {\isachardoublequoteopen}{\isasymforall}S{\isacharprime}{\isacharprime}\ {\isasymsubseteq}\ S{\isacharprime}{\isachardot}\ S{\isacharprime}{\isacharprime}\ {\isasymin}\ {\isacharparenleft}extensionSC\ C{\isacharparenright}{\isachardoublequoteclose}\isanewline
\ \ \isacommand{proof}\isamarkupfalse%
\ {\isacharparenleft}rule\ sallI{\isacharparenright}\isanewline
\ \ \ \ \isacommand{fix}\isamarkupfalse%
\ S{\isacharprime}{\isacharprime}\isanewline
\ \ \ \ \isacommand{assume}\isamarkupfalse%
\ {\isachardoublequoteopen}S{\isacharprime}{\isacharprime}\ {\isasymsubseteq}\ S{\isacharprime}{\isachardoublequoteclose}\ \isanewline
\ \ \ \ \isacommand{then}\isamarkupfalse%
\ \isacommand{have}\isamarkupfalse%
\ {\isachardoublequoteopen}S{\isacharprime}{\isacharprime}\ {\isasymsubseteq}\ S{\isachardoublequoteclose}\isanewline
\ \ \ \ \ \ \isacommand{using}\isamarkupfalse%
\ {\isacartoucheopen}S{\isacharprime}\ {\isasymsubseteq}\ S{\isacartoucheclose}\ \isacommand{by}\isamarkupfalse%
\ {\isacharparenleft}rule\ subset{\isacharunderscore}trans{\isacharparenright}\isanewline
\ \ \ \ \isacommand{then}\isamarkupfalse%
\ \isacommand{have}\isamarkupfalse%
\ {\isachardoublequoteopen}{\isasymexists}S\ {\isasymin}\ C{\isachardot}\ S{\isacharprime}{\isacharprime}\ {\isasymsubseteq}\ S{\isachardoublequoteclose}\isanewline
\ \ \ \ \ \ \isacommand{using}\isamarkupfalse%
\ {\isacartoucheopen}S\ {\isasymin}\ C{\isacartoucheclose}\ \isacommand{by}\isamarkupfalse%
\ {\isacharparenleft}rule\ bexI{\isacharparenright}\isanewline
\ \ \ \ \isacommand{thus}\isamarkupfalse%
\ {\isachardoublequoteopen}S{\isacharprime}{\isacharprime}\ {\isasymin}\ {\isacharparenleft}extensionSC\ C{\isacharparenright}{\isachardoublequoteclose}\isanewline
\ \ \ \ \ \ \isacommand{unfolding}\isamarkupfalse%
\ extensionSC\ \isacommand{by}\isamarkupfalse%
\ {\isacharparenleft}rule\ CollectI{\isacharparenright}\isanewline
\ \ \isacommand{qed}\isamarkupfalse%
\isanewline
\isacommand{qed}\isamarkupfalse%
%
\endisatagproof
{\isafoldproof}%
%
\isadelimproof
%
\endisadelimproof
%
\begin{isamarkuptext}%
En conclusión, la prueba detallada del lema completo se muestra a continuación.%
\end{isamarkuptext}\isamarkuptrue%
\isacommand{lemma}\isamarkupfalse%
\ ex{\isadigit{1}}{\isacharcolon}\ \isanewline
\ \ \isakeyword{assumes}\ {\isachardoublequoteopen}pcp\ C{\isachardoublequoteclose}\isanewline
\ \ \isakeyword{shows}\ {\isachardoublequoteopen}{\isasymexists}C{\isacharprime}{\isachardot}\ C\ {\isasymsubseteq}\ C{\isacharprime}\ {\isasymand}\ pcp\ C{\isacharprime}\ {\isasymand}\ subset{\isacharunderscore}closed\ C{\isacharprime}{\isachardoublequoteclose}\isanewline
%
\isadelimproof
%
\endisadelimproof
%
\isatagproof
\isacommand{proof}\isamarkupfalse%
\ {\isacharminus}\isanewline
\ \ \isacommand{have}\isamarkupfalse%
\ C{\isadigit{1}}{\isacharcolon}{\isachardoublequoteopen}C\ {\isasymsubseteq}\ {\isacharparenleft}extensionSC\ C{\isacharparenright}{\isachardoublequoteclose}\isanewline
\ \ \ \ \isacommand{by}\isamarkupfalse%
\ {\isacharparenleft}rule\ ex{\isadigit{1}}{\isacharunderscore}subset{\isacharparenright}\isanewline
\ \ \isacommand{have}\isamarkupfalse%
\ C{\isadigit{2}}{\isacharcolon}{\isachardoublequoteopen}pcp\ {\isacharparenleft}extensionSC\ C{\isacharparenright}{\isachardoublequoteclose}\isanewline
\ \ \ \ \isacommand{using}\isamarkupfalse%
\ assms\ \isacommand{by}\isamarkupfalse%
\ {\isacharparenleft}rule\ ex{\isadigit{1}}{\isacharunderscore}pcp{\isacharparenright}\isanewline
\ \ \isacommand{have}\isamarkupfalse%
\ C{\isadigit{3}}{\isacharcolon}{\isachardoublequoteopen}subset{\isacharunderscore}closed\ {\isacharparenleft}extensionSC\ C{\isacharparenright}{\isachardoublequoteclose}\isanewline
\ \ \ \ \isacommand{using}\isamarkupfalse%
\ assms\ \isacommand{by}\isamarkupfalse%
\ {\isacharparenleft}rule\ ex{\isadigit{1}}{\isacharunderscore}subset{\isacharunderscore}closed{\isacharparenright}\isanewline
\ \ \isacommand{have}\isamarkupfalse%
\ {\isachardoublequoteopen}C\ {\isasymsubseteq}\ {\isacharparenleft}extensionSC\ C{\isacharparenright}\ {\isasymand}\ pcp\ {\isacharparenleft}extensionSC\ C{\isacharparenright}\ {\isasymand}\ subset{\isacharunderscore}closed\ {\isacharparenleft}extensionSC\ C{\isacharparenright}{\isachardoublequoteclose}\ \isanewline
\ \ \ \ \isacommand{using}\isamarkupfalse%
\ C{\isadigit{1}}\ C{\isadigit{2}}\ C{\isadigit{3}}\ \isacommand{by}\isamarkupfalse%
\ {\isacharparenleft}iprover\ intro{\isacharcolon}\ conjI{\isacharparenright}\isanewline
\ \ \isacommand{thus}\isamarkupfalse%
\ {\isacharquery}thesis\isanewline
\ \ \ \ \isacommand{by}\isamarkupfalse%
\ {\isacharparenleft}rule\ exI{\isacharparenright}\isanewline
\isacommand{qed}\isamarkupfalse%
%
\endisatagproof
{\isafoldproof}%
%
\isadelimproof
%
\endisadelimproof
%
\begin{isamarkuptext}%
Continuemos con el segundo resultado de este apartado.

  \begin{lema}
  Toda colección de conjuntos con la propiedad de carácter finito es cerrada bajo subconjuntos.
  \end{lema}

  En Isabelle, se formaliza como sigue.%
\end{isamarkuptext}\isamarkuptrue%
\isacommand{lemma}\isamarkupfalse%
\ \isanewline
\ \ \isakeyword{assumes}\ {\isachardoublequoteopen}finite{\isacharunderscore}character\ C{\isachardoublequoteclose}\isanewline
\ \ \isakeyword{shows}\ {\isachardoublequoteopen}subset{\isacharunderscore}closed\ C{\isachardoublequoteclose}\isanewline
%
\isadelimproof
\ \ %
\endisadelimproof
%
\isatagproof
\isacommand{oops}\isamarkupfalse%
%
\endisatagproof
{\isafoldproof}%
%
\isadelimproof
%
\endisadelimproof
%
\begin{isamarkuptext}%
Procedamos con la demostración del resultado.

  \begin{demostracion}
    Consideremos una colección de conjuntos \isa{C} con la propiedad de carácter finito. Probemos que, 
    en efecto, es cerrada bajo subconjuntos. Por definición de esta última propiedad, basta 
    demostrar que todo subconjunto de cada conjunto de \isa{C} pertenece también a \isa{C}.

    Para ello, tomemos un conjunto \isa{S} cualquiera perteneciente a \isa{C} y un subconjunto cualquiera 
    \isa{S{\isacharprime}} de \isa{S}. Probemos que \isa{S{\isacharprime}} está en \isa{C}. Por hipótesis, como \isa{C} tiene la propiedad de carácter 
    finito, verifica que, para cualquier conjunto \isa{A}, son equivalentes:
    \begin{enumerate}
      \item \isa{A} pertenece a \isa{C}.
      \item Todo subconjunto finito de \isa{A} pertenece a \isa{C}.
    \end{enumerate}

    Para probar que el subconjunto \isa{S{\isacharprime}} pertenece a \isa{C}, vamos a demostrar que todo subconjunto 
    finito de \isa{S{\isacharprime}} pertenece a \isa{C}.

    De este modo, consideremos un subconjunto cualquiera \isa{S{\isacharprime}{\isacharprime}} de \isa{S{\isacharprime}}. Como \isa{S{\isacharprime}} es subconjunto de \isa{S}, 
    por la transitividad de la relación de contención de conjuntos, se tiene que \isa{S{\isacharprime}{\isacharprime}} es subconjunto 
    de \isa{S}. Aplicando la definición de propiedad de carácter finito de \isa{C} para el conjunto \isa{S}, 
    como este pertenece a \isa{C}, verifica que todo subconjunto finito de \isa{S} pertenece a \isa{C}. En
    particular, como \isa{S{\isacharprime}{\isacharprime}} es subconjunto de \isa{S}, verifica que, si \isa{S{\isacharprime}{\isacharprime}} es finito, entonces \isa{S{\isacharprime}{\isacharprime}} 
    pertenece a \isa{C}. Por tanto, hemos probado que cualquier conjunto finito de \isa{S{\isacharprime}} pertenece a la
    colección. Finalmente por la propiedad de carácter finito de \isa{C}, se verifica que \isa{S{\isacharprime}} pertenece 
    a \isa{C}, como queríamos demostrar.
  \end{demostracion}

  Veamos, a continuación, la demostración detallada del resultado en Isabelle.%
\end{isamarkuptext}\isamarkuptrue%
\isacommand{lemma}\isamarkupfalse%
\isanewline
\ \ \isakeyword{assumes}\ {\isachardoublequoteopen}finite{\isacharunderscore}character\ C{\isachardoublequoteclose}\isanewline
\ \ \isakeyword{shows}\ {\isachardoublequoteopen}subset{\isacharunderscore}closed\ C{\isachardoublequoteclose}\isanewline
%
\isadelimproof
\ \ %
\endisadelimproof
%
\isatagproof
\isacommand{unfolding}\isamarkupfalse%
\ subset{\isacharunderscore}closed{\isacharunderscore}def\isanewline
\isacommand{proof}\isamarkupfalse%
\ {\isacharparenleft}intro\ ballI\ sallI{\isacharparenright}\isanewline
\ \ \isacommand{fix}\isamarkupfalse%
\ S{\isacharprime}\ S\isanewline
\ \ \isacommand{assume}\isamarkupfalse%
\ \ {\isacartoucheopen}S\ {\isasymin}\ C{\isacartoucheclose}\ \isakeyword{and}\ {\isacartoucheopen}S{\isacharprime}\ {\isasymsubseteq}\ S{\isacartoucheclose}\isanewline
\ \ \isacommand{have}\isamarkupfalse%
\ H{\isacharcolon}{\isachardoublequoteopen}{\isasymforall}A{\isachardot}\ A\ {\isasymin}\ C\ {\isasymlongleftrightarrow}\ {\isacharparenleft}{\isasymforall}A{\isacharprime}\ {\isasymsubseteq}\ A{\isachardot}\ finite\ A{\isacharprime}\ {\isasymlongrightarrow}\ A{\isacharprime}\ {\isasymin}\ C{\isacharparenright}{\isachardoublequoteclose}\isanewline
\ \ \ \ \isacommand{using}\isamarkupfalse%
\ assms\ \isacommand{unfolding}\isamarkupfalse%
\ finite{\isacharunderscore}character{\isacharunderscore}def\ \isacommand{by}\isamarkupfalse%
\ this\isanewline
\ \ \isacommand{have}\isamarkupfalse%
\ QPQ{\isacharcolon}{\isachardoublequoteopen}{\isasymforall}S{\isacharprime}{\isacharprime}\ {\isasymsubseteq}\ S{\isacharprime}{\isachardot}\ finite\ S{\isacharprime}{\isacharprime}\ {\isasymlongrightarrow}\ S{\isacharprime}{\isacharprime}\ {\isasymin}\ C{\isachardoublequoteclose}\isanewline
\ \ \isacommand{proof}\isamarkupfalse%
\ {\isacharparenleft}rule\ sallI{\isacharparenright}\isanewline
\ \ \ \ \isacommand{fix}\isamarkupfalse%
\ S{\isacharprime}{\isacharprime}\isanewline
\ \ \ \ \isacommand{assume}\isamarkupfalse%
\ {\isachardoublequoteopen}S{\isacharprime}{\isacharprime}\ {\isasymsubseteq}\ S{\isacharprime}{\isachardoublequoteclose}\isanewline
\ \ \ \ \isacommand{then}\isamarkupfalse%
\ \isacommand{have}\isamarkupfalse%
\ {\isachardoublequoteopen}S{\isacharprime}{\isacharprime}\ {\isasymsubseteq}\ S{\isachardoublequoteclose}\isanewline
\ \ \ \ \ \ \isacommand{using}\isamarkupfalse%
\ {\isacartoucheopen}S{\isacharprime}\ {\isasymsubseteq}\ S{\isacartoucheclose}\ \isacommand{by}\isamarkupfalse%
\ {\isacharparenleft}simp\ only{\isacharcolon}\ subset{\isacharunderscore}trans{\isacharparenright}\isanewline
\ \ \ \ \isacommand{have}\isamarkupfalse%
\ {\isadigit{1}}{\isacharcolon}{\isachardoublequoteopen}S\ {\isasymin}\ C\ {\isasymlongleftrightarrow}\ {\isacharparenleft}{\isasymforall}S{\isacharprime}\ {\isasymsubseteq}\ S{\isachardot}\ finite\ S{\isacharprime}\ {\isasymlongrightarrow}\ S{\isacharprime}\ {\isasymin}\ C{\isacharparenright}{\isachardoublequoteclose}\isanewline
\ \ \ \ \ \ \isacommand{using}\isamarkupfalse%
\ H\ \isacommand{by}\isamarkupfalse%
\ {\isacharparenleft}rule\ allE{\isacharparenright}\isanewline
\ \ \ \ \isacommand{have}\isamarkupfalse%
\ {\isachardoublequoteopen}{\isasymforall}S{\isacharprime}\ {\isasymsubseteq}\ S{\isachardot}\ finite\ S{\isacharprime}\ {\isasymlongrightarrow}\ S{\isacharprime}\ {\isasymin}\ C{\isachardoublequoteclose}\isanewline
\ \ \ \ \ \ \isacommand{using}\isamarkupfalse%
\ {\isacartoucheopen}S\ {\isasymin}\ C{\isacartoucheclose}\ {\isadigit{1}}\ \isacommand{by}\isamarkupfalse%
\ {\isacharparenleft}rule\ back{\isacharunderscore}subst{\isacharparenright}\isanewline
\ \ \ \ \isacommand{thus}\isamarkupfalse%
\ {\isachardoublequoteopen}finite\ S{\isacharprime}{\isacharprime}\ {\isasymlongrightarrow}\ S{\isacharprime}{\isacharprime}\ {\isasymin}\ C{\isachardoublequoteclose}\isanewline
\ \ \ \ \ \ \isacommand{using}\isamarkupfalse%
\ {\isacartoucheopen}S{\isacharprime}{\isacharprime}\ {\isasymsubseteq}\ S{\isacartoucheclose}\ \isacommand{by}\isamarkupfalse%
\ {\isacharparenleft}rule\ sspec{\isacharparenright}\isanewline
\ \ \isacommand{qed}\isamarkupfalse%
\isanewline
\ \ \isacommand{have}\isamarkupfalse%
\ {\isachardoublequoteopen}S{\isacharprime}\ {\isasymin}\ C\ {\isasymlongleftrightarrow}\ {\isacharparenleft}{\isasymforall}S{\isacharprime}{\isacharprime}\ {\isasymsubseteq}\ S{\isacharprime}{\isachardot}\ finite\ S{\isacharprime}{\isacharprime}\ {\isasymlongrightarrow}\ S{\isacharprime}{\isacharprime}\ {\isasymin}\ C{\isacharparenright}{\isachardoublequoteclose}\isanewline
\ \ \ \ \isacommand{using}\isamarkupfalse%
\ H\ \isacommand{by}\isamarkupfalse%
\ {\isacharparenleft}rule\ allE{\isacharparenright}\isanewline
\ \ \isacommand{thus}\isamarkupfalse%
\ {\isachardoublequoteopen}S{\isacharprime}\ {\isasymin}\ C{\isachardoublequoteclose}\isanewline
\ \ \ \ \isacommand{using}\isamarkupfalse%
\ QPQ\ \isacommand{by}\isamarkupfalse%
\ {\isacharparenleft}rule\ forw{\isacharunderscore}subst{\isacharparenright}\isanewline
\isacommand{qed}\isamarkupfalse%
%
\endisatagproof
{\isafoldproof}%
%
\isadelimproof
%
\endisadelimproof
%
\begin{isamarkuptext}%
Finalmente, su prueba automática en Isabelle/HOL es la siguiente.%
\end{isamarkuptext}\isamarkuptrue%
\isacommand{lemma}\isamarkupfalse%
\ ex{\isadigit{2}}{\isacharcolon}\ \isanewline
\ \ \isakeyword{assumes}\ fc{\isacharcolon}\ {\isachardoublequoteopen}finite{\isacharunderscore}character\ C{\isachardoublequoteclose}\isanewline
\ \ \isakeyword{shows}\ {\isachardoublequoteopen}subset{\isacharunderscore}closed\ C{\isachardoublequoteclose}\isanewline
%
\isadelimproof
\ \ %
\endisadelimproof
%
\isatagproof
\isacommand{unfolding}\isamarkupfalse%
\ subset{\isacharunderscore}closed{\isacharunderscore}def\isanewline
\isacommand{proof}\isamarkupfalse%
\ {\isacharparenleft}intro\ ballI\ sallI{\isacharparenright}\isanewline
\ \ \isacommand{fix}\isamarkupfalse%
\ S{\isacharprime}\ S\isanewline
\ \ \isacommand{assume}\isamarkupfalse%
\ e{\isacharcolon}\ {\isacartoucheopen}S\ {\isasymin}\ C{\isacartoucheclose}\ \isakeyword{and}\ s{\isacharcolon}\ {\isacartoucheopen}S{\isacharprime}\ {\isasymsubseteq}\ S{\isacartoucheclose}\isanewline
\ \ \isacommand{hence}\isamarkupfalse%
\ {\isacharasterisk}{\isacharcolon}\ {\isachardoublequoteopen}S{\isacharprime}{\isacharprime}\ {\isasymsubseteq}\ S{\isacharprime}\ {\isasymLongrightarrow}\ S{\isacharprime}{\isacharprime}\ {\isasymsubseteq}\ S{\isachardoublequoteclose}\ \isakeyword{for}\ S{\isacharprime}{\isacharprime}\ \isacommand{by}\isamarkupfalse%
\ simp\isanewline
\ \ \isacommand{from}\isamarkupfalse%
\ fc\ \isacommand{have}\isamarkupfalse%
\ {\isachardoublequoteopen}S{\isacharprime}{\isacharprime}\ {\isasymsubseteq}\ S\ {\isasymLongrightarrow}\ finite\ S{\isacharprime}{\isacharprime}\ {\isasymLongrightarrow}\ S{\isacharprime}{\isacharprime}\ {\isasymin}\ C{\isachardoublequoteclose}\ \isakeyword{for}\ S{\isacharprime}{\isacharprime}\ \isanewline
\ \ \ \ \isacommand{unfolding}\isamarkupfalse%
\ finite{\isacharunderscore}character{\isacharunderscore}def\ \isacommand{using}\isamarkupfalse%
\ e\ \isacommand{by}\isamarkupfalse%
\ blast\isanewline
\ \ \isacommand{hence}\isamarkupfalse%
\ {\isachardoublequoteopen}S{\isacharprime}{\isacharprime}\ {\isasymsubseteq}\ S{\isacharprime}\ {\isasymLongrightarrow}\ finite\ S{\isacharprime}{\isacharprime}\ {\isasymLongrightarrow}\ S{\isacharprime}{\isacharprime}\ {\isasymin}\ C{\isachardoublequoteclose}\ \isakeyword{for}\ S{\isacharprime}{\isacharprime}\ \isacommand{using}\isamarkupfalse%
\ {\isacharasterisk}\ \isacommand{by}\isamarkupfalse%
\ simp\isanewline
\ \ \isacommand{with}\isamarkupfalse%
\ fc\ \isacommand{show}\isamarkupfalse%
\ {\isacartoucheopen}S{\isacharprime}\ {\isasymin}\ C{\isacartoucheclose}\ \isacommand{unfolding}\isamarkupfalse%
\ finite{\isacharunderscore}character{\isacharunderscore}def\ \isacommand{by}\isamarkupfalse%
\ blast\isanewline
\isacommand{qed}\isamarkupfalse%
%
\endisatagproof
{\isafoldproof}%
%
\isadelimproof
%
\endisadelimproof
%
\begin{isamarkuptext}%
Introduzcamos el último resultado de la sección.

 \begin{lema}
    Toda colección de conjuntos con la propiedad de consistencia proposicional y cerrada bajo 
    subconjuntos se puede extender a una colección que también verifique la propiedad de 
    consistencia proposicional y sea de carácter finito.
 \end{lema}

 \begin{demostracion}
   Dada una colección de conjuntos \isa{C} en las condiciones del enunciado, vamos a considerar su 
   extensión \isa{C{\isacharprime}} definida como la unión de \isa{C} y la colección formada por aquellos conjuntos
   cuyos subconjuntos finitos pertenecen a \isa{C}. Es decir,\\ \isa{C{\isacharprime}\ {\isacharequal}\ C\ {\isasymunion}\ E} donde 
   \isa{E\ {\isacharequal}\ {\isacharbraceleft}S{\isachardot}\ {\isasymforall}S{\isacharprime}\ {\isasymsubseteq}\ S{\isachardot}\ finite\ S{\isacharprime}\ {\isasymlongrightarrow}\ S{\isacharprime}\ {\isasymin}\ C{\isacharbraceright}}. Es evidente que es extensión pues contiene 
   a la colección \isa{C}. Vamos a probar que, además es de carácter finito y verifica la 
   propiedad de consistencia proposicional.

   En primer lugar, demostremos que \isa{C{\isacharprime}} es de carácter finito. Por definición de dicha propiedad, 
   basta probar que, para cualquier conjunto, son equivalentes:
   \begin{enumerate}
    \item El conjunto pertenece \isa{C{\isacharprime}}.
    \item Todo subconjunto finito suyo pertenece a \isa{C{\isacharprime}}.
   \end{enumerate}

   Comencemos probando \isa{{\isadigit{1}}{\isacharparenright}\ {\isasymLongrightarrow}\ {\isadigit{2}}{\isacharparenright}}. Para ello, sea un conjunto \isa{S} de \isa{C{\isacharprime}} de modo que \isa{S{\isacharprime}} es un
   subconjunto finito suyo. Como \isa{S} pertenece a la extensión, por definición de la misma tenemos
   que o bien \isa{S} está en \isa{C} o bien \isa{S} está en \isa{E}. Vamos a probar que \isa{S{\isacharprime}} está en \isa{C{\isacharprime}} por
   eliminación de la disyunción anterior. En primer lugar, si suponemos que \isa{S} está en \isa{C}, como
   se trata de una colección cerrada bajo subconjuntos, tenemos que todo subconjunto de \isa{S} está en 
   \isa{C}. En particular, \isa{S{\isacharprime}} está en \isa{C} y, por definición de la extensión, se prueba
   que \isa{S{\isacharprime}} está en \isa{C{\isacharprime}}. Por otro lado, suponiendo que \isa{S} esté en \isa{E}, por definición de dicha 
   colección tenemos que todo subconjunto finito de \isa{S} está en \isa{C}. De este modo, por las hipótesis 
   se prueba que \isa{S{\isacharprime}} está en \isa{C} y, por tanto, pertenece a la extensión. 

   Por último, probemos la implicación \isa{{\isadigit{2}}{\isacharparenright}\ {\isasymLongrightarrow}\ {\isadigit{1}}{\isacharparenright}}. Sea un conjunto cualquiera \isa{S} tal que todo
   subconjunto finito suyo pertenece a \isa{C{\isacharprime}}. Vamos a probar que \isa{S} también pertenece a \isa{C{\isacharprime}}. En
   particular, probaremos que pertenece a \isa{E}. Luego basta probar que todo subconjunto finito de 
   \isa{S} pertenece a \isa{C}. Para ello, consideremos \isa{S{\isacharprime}} un subconjunto finito cualquiera de \isa{S}. Por
   hipótesis, tenemos que \isa{S{\isacharprime}} pertenece a \isa{C{\isacharprime}}. Por definición de la extensión, tenemos entonces
   que o bien \isa{S{\isacharprime}} está en \isa{C} (lo que daría por concluida la prueba) o bien \isa{S{\isacharprime}} está en \isa{E}. 
   De este modo, si suponemos que \isa{S{\isacharprime}} está en \isa{E}, por definición de dicha colección tenemos que
   todo subconjunto finito suyo está en \isa{C}. En particular, como todo conjunto es subconjunto de si
   mismo y como hemos supuesto que \isa{S{\isacharprime}} es finito, tenemos que \isa{S{\isacharprime}} está en \isa{C}, lo que prueba la
   implicación.

   Probemos, finalmente, que \isa{C{\isacharprime}} verifica la propiedad de consistencia proposicional. Para ello,
   vamos a considerar un conjunto cualquiera \isa{S} perteneciente a \isa{C{\isacharprime}} y probaremos que se verifican 
   las cuatro condiciones del lema de caracterización de la propiedad de consistencia proposicional
   mediante la notación uniforme. Como el conjunto \isa{S} pertenece a \isa{C{\isacharprime}}, se observa fácilmente por
   definición de la extensión que, o bien \isa{S} está en \isa{C} o bien \isa{S} está en \isa{E}. Veamos que, para 
   ambos casos, se verifican dichas condiciones.

   En primer lugar, supongamos que \isa{S} está en \isa{C}. Como \isa{C} verifica la propiedad de consistencia 
   proposicional por hipótesis, verifica el lema de caracterización en particular para el conjunto 
   \isa{S}. De este modo, se cumple:
   \begin{itemize}
     \item \isa{{\isasymbottom}} no pertenece a \isa{S}.
     \item Dada \isa{p} una fórmula atómica cualquiera, no se tiene 
      simultáneamente que\\ \isa{p\ {\isasymin}\ S} y \isa{{\isasymnot}\ p\ {\isasymin}\ S}.
     \item Para toda fórmula de tipo \isa{{\isasymalpha}} con componentes \isa{{\isasymalpha}\isactrlsub {\isadigit{1}}} y \isa{{\isasymalpha}\isactrlsub {\isadigit{2}}} tal que \isa{{\isasymalpha}}
      pertenece a \isa{S}, se tiene que \isa{{\isacharbraceleft}{\isasymalpha}\isactrlsub {\isadigit{1}}{\isacharcomma}{\isasymalpha}\isactrlsub {\isadigit{2}}{\isacharbraceright}\ {\isasymunion}\ S} pertenece a \isa{C}.
     \item Para toda fórmula de tipo \isa{{\isasymbeta}} con componentes \isa{{\isasymbeta}\isactrlsub {\isadigit{1}}} y \isa{{\isasymbeta}\isactrlsub {\isadigit{2}}} tal que \isa{{\isasymbeta}}
      pertenece a \isa{S}, se tiene que o bien \isa{{\isacharbraceleft}{\isasymbeta}\isactrlsub {\isadigit{1}}{\isacharbraceright}\ {\isasymunion}\ S} pertenece a \isa{C} o 
      bien \isa{{\isacharbraceleft}{\isasymbeta}\isactrlsub {\isadigit{2}}{\isacharbraceright}\ {\isasymunion}\ S} pertenece a \isa{C}.
   \end{itemize} 
  
  Por lo tanto, puesto que \isa{C} está contenida en la extensión \isa{C{\isacharprime}}, se verifican las cuatro
  condiciones del lema para \isa{C{\isacharprime}}.

  Supongamos ahora que \isa{S} está en \isa{E}. Probemos que, en efecto, verifica las condiciones del lema 
  de caracterización.

  En primer lugar vamos a demostrar que \isa{{\isasymbottom}\ {\isasymnotin}\ S} por reducción al absurdo. Si suponemos que \isa{{\isasymbottom}\ {\isasymin}\ S},
  se deduce que el conjunto \isa{{\isacharbraceleft}{\isasymbottom}{\isacharbraceright}} es un subconjunto finito de \isa{S}. Como \isa{S} está en \isa{E}, por
  definición tenemos que \isa{{\isacharbraceleft}{\isasymbottom}{\isacharbraceright}\ {\isasymin}\ C}. De este modo, aplicando el lema de\\ caracterización de la
  propiedad de consistencia proposicional para la colección \isa{C} y el conjunto \isa{{\isacharbraceleft}{\isasymbottom}{\isacharbraceright}}, por la primera
  condición obtenemos que \isa{{\isasymbottom}\ {\isasymnotin}\ {\isacharbraceleft}{\isasymbottom}{\isacharbraceright}}, llegando a una contradicción.

  Demostremos que se verifica la segunda condición del lema para las fórmulas atómicas. De este
  modo, vamos a probar que dada \isa{p} una fórmula atómica cualquiera, no se tiene simultáneamente que
  \isa{p\ {\isasymin}\ S} y \isa{{\isasymnot}\ p\ {\isasymin}\ S}. La prueba se realizará por reducción al absurdo, luego supongamos que para
  cierta fórmula atómica se verifica \isa{p\ {\isasymin}\ S} y\\ \isa{{\isasymnot}\ p\ {\isasymin}\ S}. Análogamente, se observa que el conjunto
  \isa{{\isacharbraceleft}p{\isacharcomma}\ {\isasymnot}\ p{\isacharbraceright}} es un subconjunto finito de \isa{S}, luego pertenece a \isa{C}. Aplicando el lema de
  caracterización de la propiedad de consistencia proposicional para la colección \isa{C} y el conjunto
  \isa{{\isacharbraceleft}p{\isacharcomma}\ {\isasymnot}\ p{\isacharbraceright}}, por la segunda condición obtenemos que no se tiene simultáneamente \isa{q\ {\isasymin}\ {\isacharbraceleft}p{\isacharcomma}\ {\isasymnot}\ p{\isacharbraceright}} y
  \isa{{\isasymnot}\ q\ {\isasymin}\ {\isacharbraceleft}p{\isacharcomma}\ {\isasymnot}\ p{\isacharbraceright}} para ninguna fórmula atómica \isa{q}, llegando así a una contradicción para la
  fórmula atómica \isa{p}.

  Por otro lado, vamos a probar que se verifica la tercera condición del lema de\\ caracterización
  sobre las fórmulas de tipo \isa{{\isasymalpha}}. Consideremos una fórmula cualquiera \isa{F} de tipo \isa{{\isasymalpha}} y componentes 
  \isa{{\isasymalpha}\isactrlsub {\isadigit{1}}} y \isa{{\isasymalpha}\isactrlsub {\isadigit{2}}}, y supongamos que \isa{F\ {\isasymin}\ S}. Demostraremos que\\ \isa{{\isacharbraceleft}{\isasymalpha}\isactrlsub {\isadigit{1}}{\isacharcomma}{\isasymalpha}\isactrlsub {\isadigit{2}}{\isacharbraceright}\ {\isasymunion}\ S\ {\isasymin}\ C{\isacharprime}}. 

  Para ello, probaremos inicialmente que todo subconjunto finito \isa{S{\isacharprime}} de \isa{S} tal que\\ \isa{F\ {\isasymin}\ S{\isacharprime}} 
  verifica \isa{{\isacharbraceleft}{\isasymalpha}\isactrlsub {\isadigit{1}}{\isacharcomma}{\isasymalpha}\isactrlsub {\isadigit{2}}{\isacharbraceright}\ {\isasymunion}\ S{\isacharprime}\ {\isasymin}\ C}. Consideremos \isa{S{\isacharprime}} subconjunto finito cualquiera de \isa{S} en las
  condiciones anteriores. Como \isa{S\ {\isasymin}\ E}, por definición tenemos que \isa{S{\isacharprime}\ {\isasymin}\ C}. Aplicando el lema de 
  caracterización de la propiedad de consistencia proposicional para la colección \isa{C} y el conjunto
  \isa{S{\isacharprime}}, por la tercera condición obtenemos que \isa{{\isacharbraceleft}{\isasymalpha}\isactrlsub {\isadigit{1}}{\isacharcomma}{\isasymalpha}\isactrlsub {\isadigit{2}}{\isacharbraceright}\ {\isasymunion}\ S{\isacharprime}\ {\isasymin}\ C} ya que hemos supuesto que 
  \isa{F\ {\isasymin}\ S{\isacharprime}}.

  Una vez probado el resultado anterior, demostremos que \isa{{\isacharbraceleft}{\isasymalpha}\isactrlsub {\isadigit{1}}{\isacharcomma}{\isasymalpha}\isactrlsub {\isadigit{2}}{\isacharbraceright}\ {\isasymunion}\ S\ {\isasymin}\ E} y, por definición de 
  \isa{C{\isacharprime}}, obtendremos \isa{{\isacharbraceleft}{\isasymalpha}\isactrlsub {\isadigit{1}}{\isacharcomma}{\isasymalpha}\isactrlsub {\isadigit{2}}{\isacharbraceright}\ {\isasymunion}\ S\ {\isasymin}\ C{\isacharprime}}. Además, por definición de \isa{E}, basta probar que todo 
  subconjunto finito de \isa{{\isacharbraceleft}{\isasymalpha}\isactrlsub {\isadigit{1}}{\isacharcomma}{\isasymalpha}\isactrlsub {\isadigit{2}}{\isacharbraceright}\ {\isasymunion}\ S} pertenece a \isa{C}. Consideremos \isa{S{\isacharprime}} un subconjunto finito 
  cualquiera de \isa{{\isacharbraceleft}{\isasymalpha}\isactrlsub {\isadigit{1}}{\isacharcomma}{\isasymalpha}\isactrlsub {\isadigit{2}}{\isacharbraceright}\ {\isasymunion}\ S}. Como \isa{F\ {\isasymin}\ S}, es sencillo comprobar que el conjunto 
  \isa{{\isacharbraceleft}F{\isacharbraceright}\ {\isasymunion}\ {\isacharparenleft}S{\isacharprime}\ {\isacharminus}\ {\isacharbraceleft}{\isasymalpha}\isactrlsub {\isadigit{1}}{\isacharcomma}{\isasymalpha}\isactrlsub {\isadigit{2}}{\isacharbraceright}{\isacharparenright}} es un subconjunto finito de \isa{S}. Por el resultado probado anteriormente, 
  tenemos que el conjunto \isa{{\isacharbraceleft}{\isasymalpha}\isactrlsub {\isadigit{1}}{\isacharcomma}{\isasymalpha}\isactrlsub {\isadigit{2}}{\isacharbraceright}\ {\isasymunion}\ {\isacharparenleft}{\isacharbraceleft}F{\isacharbraceright}\ {\isasymunion}\ {\isacharparenleft}S{\isacharprime}\ {\isacharminus}\ {\isacharbraceleft}{\isasymalpha}\isactrlsub {\isadigit{1}}{\isacharcomma}{\isasymalpha}\isactrlsub {\isadigit{2}}{\isacharbraceright}{\isacharparenright}{\isacharparenright}\ {\isacharequal}} \\ \isa{{\isacharequal}\ {\isacharbraceleft}F{\isacharcomma}{\isasymalpha}\isactrlsub {\isadigit{1}}{\isacharcomma}{\isasymalpha}\isactrlsub {\isadigit{2}}{\isacharbraceright}\ {\isasymunion}\ S{\isacharprime}} pertenece a \isa{C}. 
  Además, como \isa{C} es cerrada bajo subconjuntos, todo conjunto de \isa{C} verifica que cualquier 
  subconjunto suyo pertenece a la colección. Luego, como \isa{S{\isacharprime}} es un subconjunto de 
  \isa{{\isacharbraceleft}F{\isacharcomma}{\isasymalpha}\isactrlsub {\isadigit{1}}{\isacharcomma}{\isasymalpha}\isactrlsub {\isadigit{2}}{\isacharbraceright}\ {\isasymunion}\ S{\isacharprime}}, queda probado que \isa{S{\isacharprime}\ {\isasymin}\ C}.

  Finalmente, veamos que se verifica la última condición del lema de caracterización de la propiedad
  de consistencia proposicional referente a las fórmulas de tipo \isa{{\isasymbeta}}. Consideremos una fórmula 
  cualquiera \isa{F} de tipo \isa{{\isasymbeta}} con componentes \isa{{\isasymbeta}\isactrlsub {\isadigit{1}}} y \isa{{\isasymbeta}\isactrlsub {\isadigit{2}}} tal que \isa{F\ {\isasymin}\ S}. Vamos a probar que se
  tiene que o bien \isa{{\isacharbraceleft}{\isasymbeta}\isactrlsub {\isadigit{1}}{\isacharbraceright}\ {\isasymunion}\ S\ {\isasymin}\ E} o bien \isa{{\isacharbraceleft}{\isasymbeta}\isactrlsub {\isadigit{1}}{\isacharbraceright}\ {\isasymunion}\ S\ {\isasymin}\ E}. En tal caso, por definición de \isa{C{\isacharprime}} se
  cumple que o bien \isa{{\isacharbraceleft}{\isasymbeta}\isactrlsub {\isadigit{1}}{\isacharbraceright}\ {\isasymunion}\ S\ {\isasymin}\ C{\isacharprime}} o bien \isa{{\isacharbraceleft}{\isasymbeta}\isactrlsub {\isadigit{1}}{\isacharbraceright}\ {\isasymunion}\ S\ {\isasymin}\ C{\isacharprime}}. La prueba se realizará por reducción al
  absurdo. Para ello, probemos inicialmente dos resultados previos.

  \begin{description}
    \item[\isa{{\isasymone}{\isacharparenright}}] En las condiciones anteriores, si consideramos \isa{S\isactrlsub {\isadigit{1}}} y \isa{S\isactrlsub {\isadigit{2}}} subconjuntos finitos 
    cualesquiera de \isa{S} tales que \isa{F\ {\isasymin}\ S\isactrlsub {\isadigit{1}}} y \isa{F\ {\isasymin}\ S\isactrlsub {\isadigit{2}}}, entonces existe una fórmula \isa{I\ {\isasymin}\ {\isacharbraceleft}{\isasymbeta}\isactrlsub {\isadigit{1}}{\isacharcomma}{\isasymbeta}\isactrlsub {\isadigit{2}}{\isacharbraceright}} tal 
    que se verifica que tanto \isa{{\isacharbraceleft}I{\isacharbraceright}\ {\isasymunion}\ S\isactrlsub {\isadigit{1}}} como \isa{{\isacharbraceleft}I{\isacharbraceright}\ {\isasymunion}\ S\isactrlsub {\isadigit{2}}} están en \isa{C}.
  \end{description}
  
  Para probar \isa{{\isasymone}{\isacharparenright}}, consideremos el conjunto finito \isa{S\isactrlsub {\isadigit{1}}\ {\isasymunion}\ S\isactrlsub {\isadigit{2}}} que es subconjunto de \isa{S} por las 
  hipótesis. De este modo, como \isa{S\ {\isasymin}\ E}, tenemos que \isa{S\isactrlsub {\isadigit{1}}\ {\isasymunion}\ S\isactrlsub {\isadigit{2}}\ {\isasymin}\ C}. Aplicando el lema de 
  caracterización de la propiedad de consistencia proposicional para la colección \isa{C} y el conjunto 
  \isa{S\isactrlsub {\isadigit{1}}\ {\isasymunion}\ S\isactrlsub {\isadigit{2}}}, por la última condición sobre las fórmulas de tipo \isa{{\isasymbeta}}, como\\ \isa{F\ {\isasymin}\ S\isactrlsub {\isadigit{1}}\ {\isasymunion}\ S\isactrlsub {\isadigit{2}}} por las 
  hipótesis, se tiene que o bien \isa{{\isacharbraceleft}{\isasymbeta}\isactrlsub {\isadigit{1}}{\isacharbraceright}\ {\isasymunion}\ S\isactrlsub {\isadigit{1}}\ {\isasymunion}\ S\isactrlsub {\isadigit{2}}\ {\isasymin}\ C} o bien\\ \isa{{\isacharbraceleft}{\isasymbeta}\isactrlsub {\isadigit{2}}{\isacharbraceright}\ {\isasymunion}\ S\isactrlsub {\isadigit{1}}\ {\isasymunion}\ S\isactrlsub {\isadigit{2}}\ {\isasymin}\ C}. Por tanto, 
  existe una fórmula \isa{I\ {\isasymin}\ {\isacharbraceleft}{\isasymbeta}\isactrlsub {\isadigit{1}}{\isacharcomma}{\isasymbeta}\isactrlsub {\isadigit{2}}{\isacharbraceright}} tal que\\ \isa{{\isacharbraceleft}I{\isacharbraceright}\ {\isasymunion}\ S\isactrlsub {\isadigit{1}}\ {\isasymunion}\ S\isactrlsub {\isadigit{2}}\ {\isasymin}\ C}. Sea \isa{I} la fórmula que cumple lo 
  anterior. Como \isa{C} es cerrada bajo subconjuntos, los subconjuntos \isa{{\isacharbraceleft}I{\isacharbraceright}\ {\isasymunion}\ S\isactrlsub {\isadigit{1}}} y \isa{{\isacharbraceleft}I{\isacharbraceright}\ {\isasymunion}\ S\isactrlsub {\isadigit{2}}} de 
  \isa{{\isacharbraceleft}I{\isacharbraceright}\ {\isasymunion}\ S\isactrlsub {\isadigit{1}}\ {\isasymunion}\ S\isactrlsub {\isadigit{2}}} pertenecen también a \isa{C}. Por tanto, hemos probado que existe una fórmula 
  \isa{I\ {\isasymin}\ {\isacharbraceleft}{\isasymbeta}\isactrlsub {\isadigit{1}}{\isacharcomma}{\isasymbeta}\isactrlsub {\isadigit{2}}{\isacharbraceright}} tal que \isa{{\isacharbraceleft}I{\isacharbraceright}\ {\isasymunion}\ S\isactrlsub {\isadigit{1}}\ {\isasymin}\ C} y \isa{{\isacharbraceleft}I{\isacharbraceright}\ {\isasymunion}\ S\isactrlsub {\isadigit{2}}\ {\isasymin}\ C}.

  Por otra parte, veamos el segundo resultado. 

  \begin{description}
    \item[\isa{{\isasymtwo}{\isacharparenright}}] En las condiciones de \isa{{\isasymone}{\isacharparenright}} para conjuntos cualesquiera \isa{S\isactrlsub {\isadigit{1}}} y \isa{S\isactrlsub {\isadigit{2}}}, si además 
    suponemos que \isa{{\isacharbraceleft}{\isasymbeta}\isactrlsub {\isadigit{1}}{\isacharbraceright}\ {\isasymunion}\ S\isactrlsub {\isadigit{1}}\ {\isasymnotin}\ C} y \isa{{\isacharbraceleft}{\isasymbeta}\isactrlsub {\isadigit{2}}{\isacharbraceright}\ {\isasymunion}\ S\isactrlsub {\isadigit{2}}\ {\isasymnotin}\ C}, llegamos a una contradicción. 
  \end{description}

  Para probarlo, utilizaremos \isa{{\isasymone}{\isacharparenright}} para los conjuntos \isa{{\isacharbraceleft}F{\isacharbraceright}\ {\isasymunion}\ S\isactrlsub {\isadigit{1}}} y \isa{{\isacharbraceleft}F{\isacharbraceright}\ {\isasymunion}\ S\isactrlsub {\isadigit{2}}}. Como es evidente, 
  puesto que \isa{F\ {\isasymin}\ S}, se verifica que ambos conjuntos son subconjuntos de \isa{S}. Además, como \isa{S\isactrlsub {\isadigit{1}}} y 
  \isa{S\isactrlsub {\isadigit{2}}} son finitos, se tiene que \isa{{\isacharbraceleft}F{\isacharbraceright}\ {\isasymunion}\ S\isactrlsub {\isadigit{1}}} y \isa{{\isacharbraceleft}F{\isacharbraceright}\ {\isasymunion}\ S\isactrlsub {\isadigit{2}}} también lo son. Por último, es claro que 
  \isa{F} pertenece a ambos conjuntos. Por lo tanto, por \isa{{\isasymone}{\isacharparenright}} tenemos que existe una fórmula 
  \isa{I\ {\isasymin}\ {\isacharbraceleft}{\isasymbeta}\isactrlsub {\isadigit{1}}{\isacharcomma}{\isasymbeta}\isactrlsub {\isadigit{2}}{\isacharbraceright}} tal que \isa{{\isacharbraceleft}I{\isacharbraceright}\ {\isasymunion}\ {\isacharbraceleft}F{\isacharbraceright}\ {\isasymunion}\ S\isactrlsub {\isadigit{1}}\ {\isasymin}\ C} y \isa{{\isacharbraceleft}I{\isacharbraceright}\ {\isasymunion}\ {\isacharbraceleft}F{\isacharbraceright}\ {\isasymunion}\ S\isactrlsub {\isadigit{2}}\ {\isasymin}\ C}. Por otro lado, podemos probar 
  que \isa{{\isacharbraceleft}{\isasymbeta}\isactrlsub {\isadigit{1}}{\isacharbraceright}\ {\isasymunion}\ {\isacharbraceleft}F{\isacharbraceright}\ {\isasymunion}\ S\isactrlsub {\isadigit{1}}\ {\isasymnotin}\ C}. Esto se debe a que, en caso contrario, como \isa{C} es cerrado bajo 
  subconjuntos, tendríamos que el subconjunto\\ \isa{{\isacharbraceleft}{\isasymbeta}\isactrlsub {\isadigit{1}}{\isacharbraceright}\ {\isasymunion}\ S\isactrlsub {\isadigit{1}}} pertenecería a \isa{C}, lo que contradice las 
  hipótesis. Análogamente, obtenemos que \isa{{\isacharbraceleft}{\isasymbeta}\isactrlsub {\isadigit{2}}{\isacharbraceright}\ {\isasymunion}\ {\isacharbraceleft}F{\isacharbraceright}\ {\isasymunion}\ S\isactrlsub {\isadigit{2}}\ {\isasymnotin}\ C}. De este modo, obtenemos que para 
  toda fórmula \isa{I\ {\isasymin}\ {\isacharbraceleft}{\isasymbeta}\isactrlsub {\isadigit{1}}{\isacharcomma}{\isasymbeta}\isactrlsub {\isadigit{2}}{\isacharbraceright}} se cumple que o bien \isa{{\isacharbraceleft}I{\isacharbraceright}\ {\isasymunion}\ {\isacharbraceleft}F{\isacharbraceright}\ {\isasymunion}\ S\isactrlsub {\isadigit{1}}\ {\isasymnotin}\ C} o bien \isa{{\isacharbraceleft}I{\isacharbraceright}\ {\isasymunion}\ {\isacharbraceleft}F{\isacharbraceright}\ {\isasymunion}\ S\isactrlsub {\isadigit{2}}\ {\isasymnotin}\ C}. 
  Esto es equivalente a que no existe ninguna fórmula \isa{I\ {\isasymin}\ {\isacharbraceleft}{\isasymbeta}\isactrlsub {\isadigit{1}}{\isacharcomma}{\isasymbeta}\isactrlsub {\isadigit{2}}{\isacharbraceright}} tal que \isa{{\isacharbraceleft}I{\isacharbraceright}\ {\isasymunion}\ {\isacharbraceleft}F{\isacharbraceright}\ {\isasymunion}\ S\isactrlsub {\isadigit{1}}\ {\isasymin}\ C} y\\ 
  \isa{{\isacharbraceleft}I{\isacharbraceright}\ {\isasymunion}\ {\isacharbraceleft}F{\isacharbraceright}\ {\isasymunion}\ S\isactrlsub {\isadigit{2}}\ {\isasymin}\ C}, lo que contradice lo obtenido para los conjuntos \isa{{\isacharbraceleft}F{\isacharbraceright}\ {\isasymunion}\ S\isactrlsub {\isadigit{1}}} y\\ \isa{{\isacharbraceleft}F{\isacharbraceright}\ {\isasymunion}\ S\isactrlsub {\isadigit{2}}} 
  por \isa{{\isasymone}{\isacharparenright}}.

  Finalmente, con los resultados anteriores, podemos probar que o bien\\ \isa{{\isacharbraceleft}{\isasymbeta}\isactrlsub {\isadigit{1}}{\isacharbraceright}\ {\isasymunion}\ S\ {\isasymin}\ E} o bien 
  \isa{{\isacharbraceleft}{\isasymbeta}\isactrlsub {\isadigit{2}}{\isacharbraceright}\ {\isasymunion}\ S\ {\isasymin}\ E} por reducción al absurdo. Supongamos que\\ \isa{{\isacharbraceleft}{\isasymbeta}\isactrlsub {\isadigit{1}}{\isacharbraceright}\ {\isasymunion}\ S\ {\isasymnotin}\ E} y \isa{{\isacharbraceleft}{\isasymbeta}\isactrlsub {\isadigit{2}}{\isacharbraceright}\ {\isasymunion}\ S\ {\isasymnotin}\ E}. Por
  definición de \isa{E}, se verifica que existe algún subconjunto finito de \isa{{\isacharbraceleft}{\isasymbeta}\isactrlsub {\isadigit{1}}{\isacharbraceright}\ {\isasymunion}\ S} y existe algún 
  subconjunto finito de \isa{{\isacharbraceleft}{\isasymbeta}\isactrlsub {\isadigit{2}}{\isacharbraceright}\ {\isasymunion}\ S} tales que no pertenecen a \isa{C}. Notemos por \isa{S\isactrlsub {\isadigit{1}}} y \isa{S\isactrlsub {\isadigit{2}}} 
  respectivamente a los subconjuntos anteriores. Vamos a aplicar \isa{{\isasymtwo}{\isacharparenright}} para los conjuntos \isa{S\isactrlsub {\isadigit{1}}\ {\isacharminus}\ {\isacharbraceleft}{\isasymbeta}\isactrlsub {\isadigit{1}}{\isacharbraceright}} 
  y \isa{S\isactrlsub {\isadigit{2}}\ {\isacharminus}\ {\isacharbraceleft}{\isasymbeta}\isactrlsub {\isadigit{2}}{\isacharbraceright}} para llegar a la contradicción.

  Para ello, debemos probar que se verifican las hipótesis del resultado para los conjuntos
  señalados. Es claro que tanto \isa{S\isactrlsub {\isadigit{1}}\ {\isacharminus}\ {\isacharbraceleft}{\isasymbeta}\isactrlsub {\isadigit{1}}{\isacharbraceright}} como \isa{S\isactrlsub {\isadigit{2}}\ {\isacharminus}\ {\isacharbraceleft}{\isasymbeta}\isactrlsub {\isadigit{2}}{\isacharbraceright}} son subconjuntos de \isa{S}, ya que \isa{S\isactrlsub {\isadigit{1}}} y
  \isa{S\isactrlsub {\isadigit{2}}} son subconjuntos de \isa{{\isacharbraceleft}{\isasymbeta}\isactrlsub {\isadigit{1}}{\isacharbraceright}\ {\isasymunion}\ S} y \isa{{\isacharbraceleft}{\isasymbeta}\isactrlsub {\isadigit{2}}{\isacharbraceright}\ {\isasymunion}\ S} respectivamente. Además, como \isa{S\isactrlsub {\isadigit{1}}} y \isa{S\isactrlsub {\isadigit{2}}} son
  finitos, es evidente que \isa{S\isactrlsub {\isadigit{1}}\ {\isacharminus}\ {\isacharbraceleft}{\isasymbeta}\isactrlsub {\isadigit{1}}{\isacharbraceright}} y \isa{S\isactrlsub {\isadigit{2}}\ {\isacharminus}\ {\isacharbraceleft}{\isasymbeta}\isactrlsub {\isadigit{2}}{\isacharbraceright}} también lo son. Queda probar que los conjuntos 
  \isa{{\isacharbraceleft}{\isasymbeta}\isactrlsub {\isadigit{1}}{\isacharbraceright}\ {\isasymunion}\ {\isacharparenleft}S\isactrlsub {\isadigit{1}}\ {\isacharminus}\ {\isacharbraceleft}{\isasymbeta}\isactrlsub {\isadigit{1}}{\isacharbraceright}{\isacharparenright}\ {\isacharequal}\ {\isacharbraceleft}{\isasymbeta}\isactrlsub {\isadigit{1}}{\isacharbraceright}\ {\isasymunion}\ S\isactrlsub {\isadigit{1}}} y \isa{{\isacharbraceleft}{\isasymbeta}\isactrlsub {\isadigit{2}}{\isacharbraceright}\ {\isasymunion}\ {\isacharparenleft}S\isactrlsub {\isadigit{2}}\ {\isacharminus}\ {\isacharbraceleft}{\isasymbeta}\isactrlsub {\isadigit{2}}{\isacharbraceright}{\isacharparenright}\ {\isacharequal}\ {\isacharbraceleft}{\isasymbeta}\isactrlsub {\isadigit{2}}{\isacharbraceright}\ {\isasymunion}\ S\isactrlsub {\isadigit{2}}} no pertenecen a \isa{C}. Como ni 
  \isa{S\isactrlsub {\isadigit{1}}} ni \isa{S\isactrlsub {\isadigit{2}}} están en la colección \isa{C} cerrada bajo subconjuntos, se cumple que ninguno de ellos 
  son subconjuntos de \isa{S}. Sin embargo, se verifica que \isa{S\isactrlsub {\isadigit{1}}} es subconjunto de \isa{{\isacharbraceleft}{\isasymbeta}\isactrlsub {\isadigit{1}}{\isacharbraceright}\ {\isasymunion}\ S} y \isa{S\isactrlsub {\isadigit{2}}} es 
  subconjunto de \isa{{\isacharbraceleft}{\isasymbeta}\isactrlsub {\isadigit{2}}{\isacharbraceright}\ {\isasymunion}\ S}. Por tanto, se cumple que\\ \isa{{\isasymbeta}\isactrlsub {\isadigit{1}}\ {\isasymin}\ S\isactrlsub {\isadigit{1}}} y \isa{{\isasymbeta}\isactrlsub {\isadigit{2}}\ {\isasymin}\ S\isactrlsub {\isadigit{2}}}. Por lo tanto,
  tenemos finalmente que los conjuntos \isa{{\isacharbraceleft}{\isasymbeta}\isactrlsub {\isadigit{1}}{\isacharbraceright}\ {\isasymunion}\ S\isactrlsub {\isadigit{1}}\ {\isacharequal}\ S\isactrlsub {\isadigit{1}}} y\\ \isa{{\isacharbraceleft}{\isasymbeta}\isactrlsub {\isadigit{2}}{\isacharbraceright}\ {\isasymunion}\ S\isactrlsub {\isadigit{2}}\ {\isacharequal}\ S\isactrlsub {\isadigit{2}}} no pertenecen a \isa{C}.
  Finalmente, como se cumplen las condiciones del resultado \isa{{\isadigit{2}}{\isacharparenright}}, llegamos a una contradicción para 
  los conjuntos \isa{S\isactrlsub {\isadigit{1}}\ {\isacharminus}\ {\isacharbraceleft}{\isasymbeta}\isactrlsub {\isadigit{1}}{\isacharbraceright}} y \isa{S\isactrlsub {\isadigit{2}}\ {\isacharminus}\ {\isacharbraceleft}{\isasymbeta}\isactrlsub {\isadigit{2}}{\isacharbraceright}}, probando que o bien \isa{{\isacharbraceleft}{\isasymbeta}\isactrlsub {\isadigit{1}}{\isacharbraceright}\ {\isasymunion}\ S\ {\isasymin}\ E} o bien \isa{{\isacharbraceleft}{\isasymbeta}\isactrlsub {\isadigit{1}}{\isacharbraceright}\ {\isasymunion}\ S\ {\isasymin}\ E}. 
  Por lo tanto, obtenemos por definición de \isa{C{\isacharprime}} que o bien \isa{{\isacharbraceleft}{\isasymbeta}\isactrlsub {\isadigit{1}}{\isacharbraceright}\ {\isasymunion}\ S\ {\isasymin}\ C{\isacharprime}} o bien \isa{{\isacharbraceleft}{\isasymbeta}\isactrlsub {\isadigit{1}}{\isacharbraceright}\ {\isasymunion}\ S\ {\isasymin}\ C{\isacharprime}}.
 \end{demostracion}

  Finalmente, veamos la demostración detallada del lema en Isabelle. Debido a la cantidad de lemas
  auxiliares empleados en la prueba detallada, para facilitar la comprensión mostraremos a
  continuación un grafo que estructura las relaciones de necesidad de los lemas introducidos.
  
 \begin{tikzpicture}
  [
    grow                    = down,
    level 1/.style          = {sibling distance=7cm},
    level 2/.style          = {sibling distance=4cm},
    level 3/.style          = {sibling distance=5.7cm},
    level distance          = 1.5cm,
    edge from parent/.style = {draw},
    every node/.style       = {font=\tiny},
    sloped
  ]
  \node [root] {\isa{ex{\isadigit{3}}}\\ \isa{{\isacharparenleft}Lema\ {\isadigit{3}}{\isachardot}{\isadigit{0}}{\isachardot}{\isadigit{5}}{\isacharparenright}}}
    child { node [env] {\isa{ex{\isadigit{3}}{\isacharunderscore}finite{\isacharunderscore}character}\\ \isa{{\isacharparenleft}C{\isacharprime}\ tiene\ la\ propiedad\ de\ carácter\ finito{\isacharparenright}}}}
    child { node [env] {\isa{ex{\isadigit{3}}{\isacharunderscore}pcp}\\ \isa{{\isacharparenleft}C{\isacharprime}\ tiene\ la\ propiedad\ de\ consistencia\ proposicional{\isacharparenright}}}
      		child { node [env] {\isa{ex{\isadigit{3}}{\isacharunderscore}pcp{\isacharunderscore}SinC}\\ \isa{{\isacharparenleft}Caso\ del\ conjunto\ en\ C{\isacharparenright}}}}
      		child { node [env] {\isa{ex{\isadigit{3}}{\isacharunderscore}pcp{\isacharunderscore}SinE}\\ \isa{{\isacharparenleft}Caso\ del\ conjunto\ en\ E{\isacharparenright}}}
        				child { node [env] {\isa{ex{\isadigit{3}}{\isacharunderscore}pcp{\isacharunderscore}SinE{\isacharunderscore}CON}\\ \isa{{\isacharparenleft}Condición\ fórmulas\ de\ tipo\ {\isasymalpha}{\isacharparenright}}}}
        				child { node [env] {\isa{ex{\isadigit{3}}{\isacharunderscore}pcp{\isacharunderscore}SinE{\isacharunderscore}DIS}\\ \isa{{\isacharparenleft}Condición\ fórmulas\ de\ tipo\ {\isasymbeta}{\isacharparenright}}}
                      child { node [env] {\isa{ex{\isadigit{3}}{\isacharunderscore}pcp{\isacharunderscore}SinE{\isacharunderscore}DIS{\isacharunderscore}auxFalse}\\ \isa{{\isacharparenleft}Resultado\ {\isasymone}{\isacharparenright}}}
                            child { node [env] {\isa{ex{\isadigit{3}}{\isacharunderscore}pcp{\isacharunderscore}SinE{\isacharunderscore}DIS{\isacharunderscore}auxEx}\\ \isa{{\isacharparenleft}Resultado\ {\isasymtwo}{\isacharparenright}}}}}}}};
\end{tikzpicture}

  De este modo, la prueba del \isa{lema\ {\isadigit{1}}{\isachardot}{\isadigit{3}}{\isachardot}{\isadigit{5}}} se estructura fundamentalmente en dos lemas auxiliares. 
  El primero, formalizado como \isa{ex{\isadigit{3}}{\isacharunderscore}finite{\isacharunderscore}character} en Isabelle, prueba que la extensión 
  \isa{C{\isacharprime}\ {\isacharequal}\ C\ {\isasymunion}\ E}, donde \isa{E} es la colección formada por aquellos conjuntos cuyos subconjuntos finitos 
  pertenecen a \isa{C}, tiene la propiedad de carácter finito. El segundo, formalizado como \isa{ex{\isadigit{3}}{\isacharunderscore}pcp}, 
  demuestra que \isa{C{\isacharprime}} verifica la propiedad de consistencia proposicional demostrando que cumple las 
  condiciones suficientes de dicha propiedad por el lema de caracterización \isa{{\isadigit{1}}{\isachardot}{\isadigit{2}}{\isachardot}{\isadigit{5}}}. De este modo, 
  considerando un conjunto \isa{S\ {\isasymin}\ C{\isacharprime}}, \isa{ex{\isadigit{3}}{\isacharunderscore}pcp} precisa, a su vez, de dos lemas auxiliares que 
  prueben las condiciones suficientes de \isa{{\isadigit{1}}{\isachardot}{\isadigit{2}}{\isachardot}{\isadigit{5}}}: uno para el caso en que \isa{S\ {\isasymin}\ C} (\isa{ex{\isadigit{3}}{\isacharunderscore}pcp{\isacharunderscore}SinC}) y 
  otro para el caso en que \isa{S\ {\isasymin}\ E} (\isa{ex{\isadigit{3}}{\isacharunderscore}pcp{\isacharunderscore}SinE}). Por otro lado, para el último caso en que 
  \isa{S\ {\isasymin}\ E}, utilizaremos dos lemas auxiliares. El primero, formalizado como \isa{ex{\isadigit{3}}{\isacharunderscore}pcp{\isacharunderscore}SinE{\isacharunderscore}CON}, 
  prueba que para \isa{C} una colección con la propiedad de consistencia proposicional y cerrada bajo 
  subconjuntos, \isa{S\ {\isasymin}\ E} y sea \isa{F} una fórmula de tipo \isa{{\isasymalpha}} y componentes \isa{{\isasymalpha}\isactrlsub {\isadigit{1}}} y \isa{{\isasymalpha}\isactrlsub {\isadigit{2}}}, se tiene que\\ 
  \isa{{\isacharbraceleft}{\isasymalpha}\isactrlsub {\isadigit{1}}{\isacharcomma}{\isasymalpha}\isactrlsub {\isadigit{2}}{\isacharbraceright}\ {\isasymunion}\ S\ {\isasymin}\ C{\isacharprime}}. El segundo lema, formalizado como \isa{ex{\isadigit{3}}{\isacharunderscore}pcp{\isacharunderscore}SinE{\isacharunderscore}DIS}, prueba que para \isa{C} una 
  colección con la propiedad de consistencia proposicional y cerrada bajo subconjuntos, \isa{S\ {\isasymin}\ E} y 
  sea \isa{F} una fórmula de tipo \isa{{\isasymbeta}} y componentes \isa{{\isasymbeta}\isactrlsub {\isadigit{1}}} y \isa{{\isasymbeta}\isactrlsub {\isadigit{2}}}, se tiene que o bien \isa{{\isacharbraceleft}{\isasymbeta}\isactrlsub {\isadigit{1}}{\isacharbraceright}\ {\isasymunion}\ S\ {\isasymin}\ C{\isacharprime}} o 
  bien \isa{{\isacharbraceleft}{\isasymbeta}\isactrlsub {\isadigit{2}}{\isacharbraceright}\ {\isasymunion}\ S\ {\isasymin}\ C{\isacharprime}}. Por último, este segundo lema auxiliar se probará por reducción al absurdo, 
  precisando para ello de los siguientes resultados auxiliares:
  
  \begin{description}
    \item[\isa{Resultado\ {\isasymone}}] Formalizado como \isa{ex{\isadigit{3}}{\isacharunderscore}pcp{\isacharunderscore}SinE{\isacharunderscore}DIS{\isacharunderscore}auxEx}. Prueba que dada \isa{C} una 
    colección con la propiedad de consistencia proposicional y cerrada bajo subconjuntos,\\ \isa{S\ {\isasymin}\ E} y 
    sea \isa{F} es una fórmula de tipo \isa{{\isasymbeta}} de componentes \isa{{\isasymbeta}\isactrlsub {\isadigit{1}}} y \isa{{\isasymbeta}\isactrlsub {\isadigit{2}}}, si consideramos \isa{S\isactrlsub {\isadigit{1}}} y \isa{S\isactrlsub {\isadigit{2}}} 
    subconjuntos finitos cualesquiera de \isa{S} tales que \isa{F\ {\isasymin}\ S\isactrlsub {\isadigit{1}}} y \isa{F\ {\isasymin}\ S\isactrlsub {\isadigit{2}}}, entonces existe una 
    fórmula \isa{I\ {\isasymin}\ {\isacharbraceleft}{\isasymbeta}\isactrlsub {\isadigit{1}}{\isacharcomma}{\isasymbeta}\isactrlsub {\isadigit{2}}{\isacharbraceright}} tal que se verifica que tanto \isa{{\isacharbraceleft}I{\isacharbraceright}\ {\isasymunion}\ S\isactrlsub {\isadigit{1}}} como \isa{{\isacharbraceleft}I{\isacharbraceright}\ {\isasymunion}\ S\isactrlsub {\isadigit{2}}} están en \isa{C}. 
    \item[\isa{Resultado\ {\isasymtwo}}] Formalizado como \isa{ex{\isadigit{3}}{\isacharunderscore}pcp{\isacharunderscore}SinE{\isacharunderscore}DIS{\isacharunderscore}auxFalse}. Utiliza 
    \isa{ex{\isadigit{3}}{\isacharunderscore}pcp{\isacharunderscore}SinE{\isacharunderscore}DIS{\isacharunderscore}auxEx} como lema auxiliar. Prueba que, en las condiciones del \isa{Resultado\ {\isasymone}}, 
    si además suponemos que \isa{{\isacharbraceleft}{\isasymbeta}\isactrlsub {\isadigit{1}}{\isacharbraceright}\ {\isasymunion}\ S\isactrlsub {\isadigit{1}}\ {\isasymnotin}\ C} y \isa{{\isacharbraceleft}{\isasymbeta}\isactrlsub {\isadigit{2}}{\isacharbraceright}\ {\isasymunion}\ S\isactrlsub {\isadigit{2}}\ {\isasymnotin}\ C}, llegamos a una contradicción.
  \end{description} 

  Por otro lado, para facilitar la notación, dada una colección cualquiera \isa{C}, formalizamos las 
  colecciones \isa{E} y \isa{C{\isacharprime}} como \isa{extF\ C} y \isa{extensionFin\ C} respectivamente como se muestra a 
  continuación.%
\end{isamarkuptext}\isamarkuptrue%
\isacommand{definition}\isamarkupfalse%
\ extF\ {\isacharcolon}{\isacharcolon}\ {\isachardoublequoteopen}{\isacharparenleft}{\isacharparenleft}{\isacharprime}a\ formula{\isacharparenright}\ set{\isacharparenright}\ set\ {\isasymRightarrow}\ {\isacharparenleft}{\isacharparenleft}{\isacharprime}a\ formula{\isacharparenright}\ set{\isacharparenright}\ set{\isachardoublequoteclose}\isanewline
\ \ \isakeyword{where}\ extF{\isacharcolon}\ {\isachardoublequoteopen}extF\ C\ {\isacharequal}\ {\isacharbraceleft}S{\isachardot}\ {\isasymforall}S{\isacharprime}\ {\isasymsubseteq}\ S{\isachardot}\ finite\ S{\isacharprime}\ {\isasymlongrightarrow}\ S{\isacharprime}\ {\isasymin}\ C{\isacharbraceright}{\isachardoublequoteclose}\isanewline
\isanewline
\isacommand{definition}\isamarkupfalse%
\ extensionFin\ {\isacharcolon}{\isacharcolon}\ {\isachardoublequoteopen}{\isacharparenleft}{\isacharparenleft}{\isacharprime}a\ formula{\isacharparenright}\ set{\isacharparenright}\ set\ {\isasymRightarrow}\ {\isacharparenleft}{\isacharparenleft}{\isacharprime}a\ formula{\isacharparenright}\ set{\isacharparenright}\ set{\isachardoublequoteclose}\isanewline
\ \ \isakeyword{where}\ extensionFin{\isacharcolon}\ {\isachardoublequoteopen}extensionFin\ C\ {\isacharequal}\ C\ {\isasymunion}\ {\isacharparenleft}extF\ C{\isacharparenright}{\isachardoublequoteclose}%
\begin{isamarkuptext}%
Una vez hechas las aclaraciones anteriores, procedamos ordenadamente con la demostración 
  detallada de cada lema auxiliar que conforma la prueba del lema \isa{{\isadigit{1}}{\isachardot}{\isadigit{3}}{\isachardot}{\isadigit{5}}}. En primer lugar, probemos 
  detalladamente que la extensión \isa{C{\isacharprime}} tiene la propiedad de carácter finito.%
\end{isamarkuptext}\isamarkuptrue%
\isacommand{lemma}\isamarkupfalse%
\ ex{\isadigit{3}}{\isacharunderscore}finite{\isacharunderscore}character{\isacharcolon}\isanewline
\ \ \isakeyword{assumes}\ {\isachardoublequoteopen}subset{\isacharunderscore}closed\ C{\isachardoublequoteclose}\isanewline
\ \ \ \ \ \ \ \ \isakeyword{shows}\ {\isachardoublequoteopen}finite{\isacharunderscore}character\ {\isacharparenleft}extensionFin\ C{\isacharparenright}{\isachardoublequoteclose}\isanewline
%
\isadelimproof
%
\endisadelimproof
%
\isatagproof
\isacommand{proof}\isamarkupfalse%
\ {\isacharminus}\isanewline
\ \ \isacommand{show}\isamarkupfalse%
\ {\isachardoublequoteopen}finite{\isacharunderscore}character\ {\isacharparenleft}extensionFin\ C{\isacharparenright}{\isachardoublequoteclose}\isanewline
\ \ \ \ \isacommand{unfolding}\isamarkupfalse%
\ finite{\isacharunderscore}character{\isacharunderscore}def\isanewline
\ \ \isacommand{proof}\isamarkupfalse%
\ {\isacharparenleft}rule\ allI{\isacharparenright}\isanewline
\ \ \ \isacommand{fix}\isamarkupfalse%
\ S\isanewline
\ \ \ \isacommand{show}\isamarkupfalse%
\ {\isachardoublequoteopen}S\ {\isasymin}\ {\isacharparenleft}extensionFin\ C{\isacharparenright}\ {\isasymlongleftrightarrow}\ {\isacharparenleft}{\isasymforall}S{\isacharprime}\ {\isasymsubseteq}\ S{\isachardot}\ finite\ S{\isacharprime}\ {\isasymlongrightarrow}\ S{\isacharprime}\ {\isasymin}\ {\isacharparenleft}extensionFin\ C{\isacharparenright}{\isacharparenright}{\isachardoublequoteclose}\isanewline
\ \ \ \isacommand{proof}\isamarkupfalse%
\ {\isacharparenleft}rule\ iffI{\isacharparenright}\isanewline
\ \ \ \ \ \isacommand{assume}\isamarkupfalse%
\ {\isachardoublequoteopen}S\ {\isasymin}\ {\isacharparenleft}extensionFin\ C{\isacharparenright}{\isachardoublequoteclose}\isanewline
\ \ \ \ \ \isacommand{show}\isamarkupfalse%
\ {\isachardoublequoteopen}{\isasymforall}S{\isacharprime}\ {\isasymsubseteq}\ S{\isachardot}\ finite\ S{\isacharprime}\ {\isasymlongrightarrow}\ S{\isacharprime}\ {\isasymin}\ {\isacharparenleft}extensionFin\ C{\isacharparenright}{\isachardoublequoteclose}\isanewline
\ \ \ \ \ \isacommand{proof}\isamarkupfalse%
\ {\isacharparenleft}intro\ sallI\ impI{\isacharparenright}\isanewline
\ \ \ \ \ \ \ \isacommand{fix}\isamarkupfalse%
\ S{\isacharprime}\isanewline
\ \ \ \ \ \ \ \isacommand{assume}\isamarkupfalse%
\ {\isachardoublequoteopen}S{\isacharprime}\ {\isasymsubseteq}\ S{\isachardoublequoteclose}\isanewline
\ \ \ \ \ \ \ \isacommand{assume}\isamarkupfalse%
\ {\isachardoublequoteopen}finite\ S{\isacharprime}{\isachardoublequoteclose}\isanewline
\ \ \ \ \ \ \ \isacommand{have}\isamarkupfalse%
\ {\isachardoublequoteopen}S\ {\isasymin}\ C\ {\isasymor}\ S\ {\isasymin}\ {\isacharparenleft}extF\ C{\isacharparenright}{\isachardoublequoteclose}\isanewline
\ \ \ \ \ \ \ \ \ \isacommand{using}\isamarkupfalse%
\ {\isacartoucheopen}S\ {\isasymin}\ {\isacharparenleft}extensionFin\ C{\isacharparenright}{\isacartoucheclose}\ \isacommand{by}\isamarkupfalse%
\ {\isacharparenleft}simp\ only{\isacharcolon}\ extensionFin\ Un{\isacharunderscore}iff{\isacharparenright}\isanewline
\ \ \ \ \ \ \ \isacommand{thus}\isamarkupfalse%
\ {\isachardoublequoteopen}S{\isacharprime}\ {\isasymin}\ {\isacharparenleft}extensionFin\ C{\isacharparenright}{\isachardoublequoteclose}\isanewline
\ \ \ \ \ \ \ \isacommand{proof}\isamarkupfalse%
\ {\isacharparenleft}rule\ disjE{\isacharparenright}\isanewline
\ \ \ \ \ \ \ \ \ \isacommand{assume}\isamarkupfalse%
\ {\isachardoublequoteopen}S\ {\isasymin}\ C{\isachardoublequoteclose}\isanewline
\ \ \ \ \ \ \ \ \ \isacommand{have}\isamarkupfalse%
\ {\isachardoublequoteopen}{\isasymforall}S\ {\isasymin}\ C{\isachardot}\ {\isasymforall}S{\isacharprime}\ {\isasymsubseteq}\ S{\isachardot}\ S{\isacharprime}\ {\isasymin}\ C{\isachardoublequoteclose}\isanewline
\ \ \ \ \ \ \ \ \ \ \ \isacommand{using}\isamarkupfalse%
\ assms\ \isacommand{by}\isamarkupfalse%
\ {\isacharparenleft}simp\ only{\isacharcolon}\ subset{\isacharunderscore}closed{\isacharunderscore}def{\isacharparenright}\isanewline
\ \ \ \ \ \ \ \ \ \isacommand{then}\isamarkupfalse%
\ \isacommand{have}\isamarkupfalse%
\ {\isachardoublequoteopen}{\isasymforall}S{\isacharprime}\ {\isasymsubseteq}\ S{\isachardot}\ S{\isacharprime}\ {\isasymin}\ C{\isachardoublequoteclose}\isanewline
\ \ \ \ \ \ \ \ \ \ \ \isacommand{using}\isamarkupfalse%
\ {\isacartoucheopen}S\ {\isasymin}\ C{\isacartoucheclose}\ \isacommand{by}\isamarkupfalse%
\ {\isacharparenleft}rule\ bspec{\isacharparenright}\isanewline
\ \ \ \ \ \ \ \ \ \isacommand{then}\isamarkupfalse%
\ \isacommand{have}\isamarkupfalse%
\ {\isachardoublequoteopen}S{\isacharprime}\ {\isasymin}\ C{\isachardoublequoteclose}\isanewline
\ \ \ \ \ \ \ \ \ \ \ \isacommand{using}\isamarkupfalse%
\ {\isacartoucheopen}S{\isacharprime}\ {\isasymsubseteq}\ S{\isacartoucheclose}\ \isacommand{by}\isamarkupfalse%
\ {\isacharparenleft}rule\ sspec{\isacharparenright}\isanewline
\ \ \ \ \ \ \ \ \ \isacommand{thus}\isamarkupfalse%
\ {\isachardoublequoteopen}S{\isacharprime}\ {\isasymin}\ {\isacharparenleft}extensionFin\ C{\isacharparenright}{\isachardoublequoteclose}\ \isanewline
\ \ \ \ \ \ \ \ \ \ \ \isacommand{by}\isamarkupfalse%
\ {\isacharparenleft}simp\ only{\isacharcolon}\ extensionFin\ UnI{\isadigit{1}}{\isacharparenright}\isanewline
\ \ \ \ \ \ \ \isacommand{next}\isamarkupfalse%
\isanewline
\ \ \ \ \ \ \ \ \ \isacommand{assume}\isamarkupfalse%
\ {\isachardoublequoteopen}S\ {\isasymin}\ {\isacharparenleft}extF\ C{\isacharparenright}{\isachardoublequoteclose}\isanewline
\ \ \ \ \ \ \ \ \ \isacommand{then}\isamarkupfalse%
\ \isacommand{have}\isamarkupfalse%
\ {\isachardoublequoteopen}{\isasymforall}S{\isacharprime}\ {\isasymsubseteq}\ S{\isachardot}\ finite\ S{\isacharprime}\ {\isasymlongrightarrow}\ S{\isacharprime}\ {\isasymin}\ C{\isachardoublequoteclose}\isanewline
\ \ \ \ \ \ \ \ \ \ \ \isacommand{unfolding}\isamarkupfalse%
\ extF\ \isacommand{by}\isamarkupfalse%
\ {\isacharparenleft}rule\ CollectD{\isacharparenright}\isanewline
\ \ \ \ \ \ \ \ \ \isacommand{then}\isamarkupfalse%
\ \isacommand{have}\isamarkupfalse%
\ {\isachardoublequoteopen}finite\ S{\isacharprime}\ {\isasymlongrightarrow}\ S{\isacharprime}\ {\isasymin}\ C{\isachardoublequoteclose}\isanewline
\ \ \ \ \ \ \ \ \ \ \ \isacommand{using}\isamarkupfalse%
\ {\isacartoucheopen}S{\isacharprime}\ {\isasymsubseteq}\ S{\isacartoucheclose}\ \isacommand{by}\isamarkupfalse%
\ {\isacharparenleft}rule\ sspec{\isacharparenright}\isanewline
\ \ \ \ \ \ \ \ \ \isacommand{then}\isamarkupfalse%
\ \isacommand{have}\isamarkupfalse%
\ {\isachardoublequoteopen}S{\isacharprime}\ {\isasymin}\ C{\isachardoublequoteclose}\isanewline
\ \ \ \ \ \ \ \ \ \ \ \isacommand{using}\isamarkupfalse%
\ {\isacartoucheopen}finite\ S{\isacharprime}{\isacartoucheclose}\ \isacommand{by}\isamarkupfalse%
\ {\isacharparenleft}rule\ mp{\isacharparenright}\isanewline
\ \ \ \ \ \ \ \ \ \isacommand{thus}\isamarkupfalse%
\ {\isachardoublequoteopen}S{\isacharprime}\ {\isasymin}\ {\isacharparenleft}extensionFin\ C{\isacharparenright}{\isachardoublequoteclose}\isanewline
\ \ \ \ \ \ \ \ \ \ \ \isacommand{by}\isamarkupfalse%
\ {\isacharparenleft}simp\ only{\isacharcolon}\ extensionFin\ UnI{\isadigit{1}}{\isacharparenright}\isanewline
\ \ \ \ \ \ \ \isacommand{qed}\isamarkupfalse%
\isanewline
\ \ \ \ \ \isacommand{qed}\isamarkupfalse%
\isanewline
\ \ \ \isacommand{next}\isamarkupfalse%
\isanewline
\ \ \ \ \ \isacommand{assume}\isamarkupfalse%
\ {\isachardoublequoteopen}{\isasymforall}S{\isacharprime}\ {\isasymsubseteq}\ S{\isachardot}\ finite\ S{\isacharprime}\ {\isasymlongrightarrow}\ S{\isacharprime}\ {\isasymin}\ {\isacharparenleft}extensionFin\ C{\isacharparenright}{\isachardoublequoteclose}\isanewline
\ \ \ \ \ \isacommand{then}\isamarkupfalse%
\ \isacommand{have}\isamarkupfalse%
\ F{\isacharcolon}{\isachardoublequoteopen}{\isasymforall}S{\isacharprime}\ {\isasymsubseteq}\ S{\isachardot}\ finite\ S{\isacharprime}\ {\isasymlongrightarrow}\ S{\isacharprime}\ {\isasymin}\ C\ {\isasymor}\ S{\isacharprime}\ {\isasymin}\ {\isacharparenleft}extF\ C{\isacharparenright}{\isachardoublequoteclose}\isanewline
\ \ \ \ \ \ \ \isacommand{by}\isamarkupfalse%
\ {\isacharparenleft}simp\ only{\isacharcolon}\ extensionFin\ Un{\isacharunderscore}iff{\isacharparenright}\isanewline
\ \ \ \ \ \isacommand{have}\isamarkupfalse%
\ {\isachardoublequoteopen}{\isasymforall}S{\isacharprime}\ {\isasymsubseteq}\ S{\isachardot}\ finite\ S{\isacharprime}\ {\isasymlongrightarrow}\ S{\isacharprime}\ {\isasymin}\ C{\isachardoublequoteclose}\isanewline
\ \ \ \ \ \isacommand{proof}\isamarkupfalse%
\ {\isacharparenleft}rule\ sallI{\isacharparenright}\isanewline
\ \ \ \ \ \ \ \isacommand{fix}\isamarkupfalse%
\ S{\isacharprime}\isanewline
\ \ \ \ \ \ \ \isacommand{assume}\isamarkupfalse%
\ {\isachardoublequoteopen}S{\isacharprime}\ {\isasymsubseteq}\ S{\isachardoublequoteclose}\isanewline
\ \ \ \ \ \ \ \isacommand{show}\isamarkupfalse%
\ {\isachardoublequoteopen}finite\ S{\isacharprime}\ {\isasymlongrightarrow}\ S{\isacharprime}\ {\isasymin}\ C{\isachardoublequoteclose}\isanewline
\ \ \ \ \ \ \ \isacommand{proof}\isamarkupfalse%
\ {\isacharparenleft}rule\ impI{\isacharparenright}\isanewline
\ \ \ \ \ \ \ \ \ \isacommand{assume}\isamarkupfalse%
\ {\isachardoublequoteopen}finite\ S{\isacharprime}{\isachardoublequoteclose}\isanewline
\ \ \ \ \ \ \ \ \ \isacommand{have}\isamarkupfalse%
\ {\isachardoublequoteopen}finite\ S{\isacharprime}\ {\isasymlongrightarrow}\ S{\isacharprime}\ {\isasymin}\ C\ {\isasymor}\ S{\isacharprime}\ {\isasymin}\ {\isacharparenleft}extF\ C{\isacharparenright}{\isachardoublequoteclose}\ \isanewline
\ \ \ \ \ \ \ \ \ \ \ \isacommand{using}\isamarkupfalse%
\ F\ {\isacartoucheopen}S{\isacharprime}\ {\isasymsubseteq}\ S{\isacartoucheclose}\ \isacommand{by}\isamarkupfalse%
\ {\isacharparenleft}rule\ sspec{\isacharparenright}\isanewline
\ \ \ \ \ \ \ \ \ \isacommand{then}\isamarkupfalse%
\ \isacommand{have}\isamarkupfalse%
\ {\isachardoublequoteopen}S{\isacharprime}\ {\isasymin}\ C\ {\isasymor}\ S{\isacharprime}\ {\isasymin}\ {\isacharparenleft}extF\ C{\isacharparenright}{\isachardoublequoteclose}\isanewline
\ \ \ \ \ \ \ \ \ \ \ \isacommand{using}\isamarkupfalse%
\ {\isacartoucheopen}finite\ S{\isacharprime}{\isacartoucheclose}\ \isacommand{by}\isamarkupfalse%
\ {\isacharparenleft}rule\ mp{\isacharparenright}\isanewline
\ \ \ \ \ \ \ \ \ \isacommand{thus}\isamarkupfalse%
\ {\isachardoublequoteopen}S{\isacharprime}\ {\isasymin}\ C{\isachardoublequoteclose}\isanewline
\ \ \ \ \ \ \ \ \ \isacommand{proof}\isamarkupfalse%
\ {\isacharparenleft}rule\ disjE{\isacharparenright}\isanewline
\ \ \ \ \ \ \ \ \ \ \ \isacommand{assume}\isamarkupfalse%
\ {\isachardoublequoteopen}S{\isacharprime}\ {\isasymin}\ C{\isachardoublequoteclose}\isanewline
\ \ \ \ \ \ \ \ \ \ \ \isacommand{thus}\isamarkupfalse%
\ {\isachardoublequoteopen}S{\isacharprime}\ {\isasymin}\ C{\isachardoublequoteclose}\isanewline
\ \ \ \ \ \ \ \ \ \ \ \ \ \isacommand{by}\isamarkupfalse%
\ this\isanewline
\ \ \ \ \ \ \ \ \ \isacommand{next}\isamarkupfalse%
\isanewline
\ \ \ \ \ \ \ \ \ \ \ \isacommand{assume}\isamarkupfalse%
\ {\isachardoublequoteopen}S{\isacharprime}\ {\isasymin}\ {\isacharparenleft}extF\ C{\isacharparenright}{\isachardoublequoteclose}\isanewline
\ \ \ \ \ \ \ \ \ \ \ \isacommand{then}\isamarkupfalse%
\ \isacommand{have}\isamarkupfalse%
\ S{\isacharprime}{\isacharcolon}{\isachardoublequoteopen}{\isasymforall}S{\isacharprime}{\isacharprime}\ {\isasymsubseteq}\ S{\isacharprime}{\isachardot}\ finite\ S{\isacharprime}{\isacharprime}\ {\isasymlongrightarrow}\ S{\isacharprime}{\isacharprime}\ {\isasymin}\ C{\isachardoublequoteclose}\isanewline
\ \ \ \ \ \ \ \ \ \ \ \ \ \isacommand{unfolding}\isamarkupfalse%
\ extF\ \isacommand{by}\isamarkupfalse%
\ {\isacharparenleft}rule\ CollectD{\isacharparenright}\isanewline
\ \ \ \ \ \ \ \ \ \ \ \isacommand{have}\isamarkupfalse%
\ {\isachardoublequoteopen}S{\isacharprime}\ {\isasymsubseteq}\ S{\isacharprime}{\isachardoublequoteclose}\isanewline
\ \ \ \ \ \ \ \ \ \ \ \ \ \isacommand{by}\isamarkupfalse%
\ {\isacharparenleft}simp\ only{\isacharcolon}\ subset{\isacharunderscore}refl{\isacharparenright}\isanewline
\ \ \ \ \ \ \ \ \ \ \ \isacommand{have}\isamarkupfalse%
\ {\isachardoublequoteopen}finite\ S{\isacharprime}\ {\isasymlongrightarrow}\ S{\isacharprime}\ {\isasymin}\ C{\isachardoublequoteclose}\isanewline
\ \ \ \ \ \ \ \ \ \ \ \ \ \isacommand{using}\isamarkupfalse%
\ S{\isacharprime}\ {\isacartoucheopen}S{\isacharprime}\ {\isasymsubseteq}\ S{\isacharprime}{\isacartoucheclose}\ \isacommand{by}\isamarkupfalse%
\ {\isacharparenleft}rule\ sspec{\isacharparenright}\isanewline
\ \ \ \ \ \ \ \ \ \ \ \isacommand{thus}\isamarkupfalse%
\ {\isachardoublequoteopen}S{\isacharprime}\ {\isasymin}\ C{\isachardoublequoteclose}\isanewline
\ \ \ \ \ \ \ \ \ \ \ \ \ \isacommand{using}\isamarkupfalse%
\ {\isacartoucheopen}finite\ S{\isacharprime}{\isacartoucheclose}\ \isacommand{by}\isamarkupfalse%
\ {\isacharparenleft}rule\ mp{\isacharparenright}\isanewline
\ \ \ \ \ \ \ \ \ \isacommand{qed}\isamarkupfalse%
\isanewline
\ \ \ \ \ \ \ \isacommand{qed}\isamarkupfalse%
\isanewline
\ \ \ \ \ \isacommand{qed}\isamarkupfalse%
\isanewline
\ \ \ \ \ \isacommand{then}\isamarkupfalse%
\ \isacommand{have}\isamarkupfalse%
\ {\isachardoublequoteopen}S\ {\isasymin}\ {\isacharbraceleft}S{\isachardot}\ {\isasymforall}S{\isacharprime}\ {\isasymsubseteq}\ S{\isachardot}\ finite\ S{\isacharprime}\ {\isasymlongrightarrow}\ S{\isacharprime}\ {\isasymin}\ C{\isacharbraceright}{\isachardoublequoteclose}\isanewline
\ \ \ \ \ \ \ \isacommand{by}\isamarkupfalse%
\ {\isacharparenleft}rule\ CollectI{\isacharparenright}\isanewline
\ \ \ \ \ \isacommand{thus}\isamarkupfalse%
\ {\isachardoublequoteopen}S\ {\isasymin}\ {\isacharparenleft}extensionFin\ C{\isacharparenright}{\isachardoublequoteclose}\isanewline
\ \ \ \ \ \ \ \isacommand{by}\isamarkupfalse%
\ {\isacharparenleft}simp\ only{\isacharcolon}\ extF\ extensionFin\ UnI{\isadigit{2}}{\isacharparenright}\isanewline
\ \ \ \isacommand{qed}\isamarkupfalse%
\isanewline
\ \isacommand{qed}\isamarkupfalse%
\isanewline
\isacommand{qed}\isamarkupfalse%
%
\endisatagproof
{\isafoldproof}%
%
\isadelimproof
%
\endisadelimproof
%
\begin{isamarkuptext}%
Por otro lado, para probar que  \isa{C{\isacharprime}\ {\isacharequal}\ C\ {\isasymunion}\ E}  verifica la propiedad de consistencia 
  proposicional, consideraremos un conjunto \isa{S\ {\isasymin}\ C{\isacharprime}} y utilizaremos fundamentalmente dos lemas 
  auxiliares: uno para el caso en que \isa{S\ {\isasymin}\ C} y otro para el caso en que \isa{S\ {\isasymin}\ E}. 

  En primer lugar, vamos a probar el primer lema auxiliar para el caso en que\\ \isa{S\ {\isasymin}\ C}, formalizado
  como \isa{ex{\isadigit{3}}{\isacharunderscore}pcp{\isacharunderscore}SinC}. Dicho lema prueba que, si \isa{C} es una colección con la propiedad de 
  consistencia proposicional y cerrada bajo subconjuntos, y sea \isa{S\ {\isasymin}\ C}, se verifican
  las condiciones del lema de caracterización de la propiedad de consistencia proposicional para
  la extensión \isa{C{\isacharprime}}:
  \begin{itemize}
    \item \isa{{\isasymbottom}\ {\isasymnotin}\ S}.
    \item Dada \isa{p} una fórmula atómica cualquiera, no se tiene 
    simultáneamente que\\ \isa{p\ {\isasymin}\ S} y \isa{{\isasymnot}\ p\ {\isasymin}\ S}.
    \item Para toda fórmula de tipo \isa{{\isasymalpha}} con componentes \isa{{\isasymalpha}\isactrlsub {\isadigit{1}}} y \isa{{\isasymalpha}\isactrlsub {\isadigit{2}}} tal que \isa{{\isasymalpha}}
    pertenece a \isa{S}, se tiene que \isa{{\isacharbraceleft}{\isasymalpha}\isactrlsub {\isadigit{1}}{\isacharcomma}{\isasymalpha}\isactrlsub {\isadigit{2}}{\isacharbraceright}\ {\isasymunion}\ S} pertenece a \isa{C{\isacharprime}}.
    \item Para toda fórmula de tipo \isa{{\isasymbeta}} con componentes \isa{{\isasymbeta}\isactrlsub {\isadigit{1}}} y \isa{{\isasymbeta}\isactrlsub {\isadigit{2}}} tal que \isa{{\isasymbeta}}
    pertenece a \isa{S}, se tiene que o bien \isa{{\isacharbraceleft}{\isasymbeta}\isactrlsub {\isadigit{1}}{\isacharbraceright}\ {\isasymunion}\ S} pertenece a \isa{C{\isacharprime}} o 
    bien \isa{{\isacharbraceleft}{\isasymbeta}\isactrlsub {\isadigit{2}}{\isacharbraceright}\ {\isasymunion}\ S} pertenece a \isa{C{\isacharprime}}.
  \end{itemize}%
\end{isamarkuptext}\isamarkuptrue%
\isacommand{lemma}\isamarkupfalse%
\ ex{\isadigit{3}}{\isacharunderscore}pcp{\isacharunderscore}SinC{\isacharcolon}\isanewline
\ \ \isakeyword{assumes}\ {\isachardoublequoteopen}pcp\ C{\isachardoublequoteclose}\isanewline
\ \ \ \ \ \ \ \ \ \ {\isachardoublequoteopen}subset{\isacharunderscore}closed\ C{\isachardoublequoteclose}\isanewline
\ \ \ \ \ \ \ \ \ \ {\isachardoublequoteopen}S\ {\isasymin}\ C{\isachardoublequoteclose}\ \isanewline
\ \ \isakeyword{shows}\ {\isachardoublequoteopen}{\isasymbottom}\ {\isasymnotin}\ S\ {\isasymand}\isanewline
\ \ \ \ \ \ \ \ \ {\isacharparenleft}{\isasymforall}k{\isachardot}\ Atom\ k\ {\isasymin}\ S\ {\isasymlongrightarrow}\ \isactrlbold {\isasymnot}\ {\isacharparenleft}Atom\ k{\isacharparenright}\ {\isasymin}\ S\ {\isasymlongrightarrow}\ False{\isacharparenright}\ {\isasymand}\isanewline
\ \ \ \ \ \ \ \ \ {\isacharparenleft}{\isasymforall}F\ G\ H{\isachardot}\ Con\ F\ G\ H\ {\isasymlongrightarrow}\ F\ {\isasymin}\ S\ {\isasymlongrightarrow}\ {\isacharbraceleft}G{\isacharcomma}\ H{\isacharbraceright}\ {\isasymunion}\ S\ {\isasymin}\ {\isacharparenleft}extensionFin\ C{\isacharparenright}{\isacharparenright}\ {\isasymand}\isanewline
\ \ \ \ \ \ \ \ \ {\isacharparenleft}{\isasymforall}F\ G\ H{\isachardot}\ Dis\ F\ G\ H\ {\isasymlongrightarrow}\ F\ {\isasymin}\ S\ {\isasymlongrightarrow}\ {\isacharbraceleft}G{\isacharbraceright}\ {\isasymunion}\ S\ {\isasymin}{\isacharparenleft}extensionFin\ C{\isacharparenright}\ {\isasymor}\ {\isacharbraceleft}H{\isacharbraceright}\ {\isasymunion}\ S\ {\isasymin}\ {\isacharparenleft}extensionFin\ C{\isacharparenright}{\isacharparenright}{\isachardoublequoteclose}\isanewline
%
\isadelimproof
%
\endisadelimproof
%
\isatagproof
\isacommand{proof}\isamarkupfalse%
\ {\isacharminus}\isanewline
\ \ \isacommand{have}\isamarkupfalse%
\ PCP{\isacharcolon}{\isachardoublequoteopen}{\isasymforall}S\ {\isasymin}\ C{\isachardot}\isanewline
\ \ \ \ {\isasymbottom}\ {\isasymnotin}\ S\isanewline
\ \ \ \ {\isasymand}\ {\isacharparenleft}{\isasymforall}k{\isachardot}\ Atom\ k\ {\isasymin}\ S\ {\isasymlongrightarrow}\ \isactrlbold {\isasymnot}\ {\isacharparenleft}Atom\ k{\isacharparenright}\ {\isasymin}\ S\ {\isasymlongrightarrow}\ False{\isacharparenright}\isanewline
\ \ \ \ {\isasymand}\ {\isacharparenleft}{\isasymforall}F\ G\ H{\isachardot}\ Con\ F\ G\ H\ {\isasymlongrightarrow}\ F\ {\isasymin}\ S\ {\isasymlongrightarrow}\ {\isacharbraceleft}G{\isacharcomma}H{\isacharbraceright}\ {\isasymunion}\ S\ {\isasymin}\ C{\isacharparenright}\isanewline
\ \ \ \ {\isasymand}\ {\isacharparenleft}{\isasymforall}F\ G\ H{\isachardot}\ Dis\ F\ G\ H\ {\isasymlongrightarrow}\ F\ {\isasymin}\ S\ {\isasymlongrightarrow}\ {\isacharbraceleft}G{\isacharbraceright}\ {\isasymunion}\ S\ {\isasymin}\ C\ {\isasymor}\ {\isacharbraceleft}H{\isacharbraceright}\ {\isasymunion}\ S\ {\isasymin}\ C{\isacharparenright}{\isachardoublequoteclose}\isanewline
\ \ \ \ \isacommand{using}\isamarkupfalse%
\ assms{\isacharparenleft}{\isadigit{1}}{\isacharparenright}\ \isacommand{by}\isamarkupfalse%
\ {\isacharparenleft}rule\ pcp{\isacharunderscore}alt{\isadigit{1}}{\isacharparenright}\isanewline
\ \ \isacommand{have}\isamarkupfalse%
\ H{\isacharcolon}{\isachardoublequoteopen}{\isasymbottom}\ {\isasymnotin}\ S\isanewline
\ \ \ \ {\isasymand}\ {\isacharparenleft}{\isasymforall}k{\isachardot}\ Atom\ k\ {\isasymin}\ S\ {\isasymlongrightarrow}\ \isactrlbold {\isasymnot}\ {\isacharparenleft}Atom\ k{\isacharparenright}\ {\isasymin}\ S\ {\isasymlongrightarrow}\ False{\isacharparenright}\isanewline
\ \ \ \ {\isasymand}\ {\isacharparenleft}{\isasymforall}F\ G\ H{\isachardot}\ Con\ F\ G\ H\ {\isasymlongrightarrow}\ F\ {\isasymin}\ S\ {\isasymlongrightarrow}\ {\isacharbraceleft}G{\isacharcomma}H{\isacharbraceright}\ {\isasymunion}\ S\ {\isasymin}\ C{\isacharparenright}\isanewline
\ \ \ \ {\isasymand}\ {\isacharparenleft}{\isasymforall}F\ G\ H{\isachardot}\ Dis\ F\ G\ H\ {\isasymlongrightarrow}\ F\ {\isasymin}\ S\ {\isasymlongrightarrow}\ {\isacharbraceleft}G{\isacharbraceright}\ {\isasymunion}\ S\ {\isasymin}\ C\ {\isasymor}\ {\isacharbraceleft}H{\isacharbraceright}\ {\isasymunion}\ S\ {\isasymin}\ C{\isacharparenright}{\isachardoublequoteclose}\isanewline
\ \ \ \ \ \isacommand{using}\isamarkupfalse%
\ PCP\ {\isacartoucheopen}S\ {\isasymin}\ C{\isacartoucheclose}\ \isacommand{by}\isamarkupfalse%
\ {\isacharparenleft}rule\ bspec{\isacharparenright}\isanewline
\ \ \isacommand{then}\isamarkupfalse%
\ \isacommand{have}\isamarkupfalse%
\ A{\isadigit{1}}{\isacharcolon}{\isachardoublequoteopen}{\isasymbottom}\ {\isasymnotin}\ S{\isachardoublequoteclose}\isanewline
\ \ \ \ \isacommand{by}\isamarkupfalse%
\ {\isacharparenleft}rule\ conjunct{\isadigit{1}}{\isacharparenright}\isanewline
\ \ \isacommand{have}\isamarkupfalse%
\ A{\isadigit{2}}{\isacharcolon}{\isachardoublequoteopen}{\isasymforall}k{\isachardot}\ Atom\ k\ {\isasymin}\ S\ {\isasymlongrightarrow}\ \isactrlbold {\isasymnot}\ {\isacharparenleft}Atom\ k{\isacharparenright}\ {\isasymin}\ S\ {\isasymlongrightarrow}\ False{\isachardoublequoteclose}\isanewline
\ \ \ \ \isacommand{using}\isamarkupfalse%
\ H\ \isacommand{by}\isamarkupfalse%
\ {\isacharparenleft}iprover\ elim{\isacharcolon}\ conjunct{\isadigit{2}}\ conjunct{\isadigit{1}}{\isacharparenright}\isanewline
\ \ \isacommand{have}\isamarkupfalse%
\ S{\isadigit{3}}{\isacharcolon}{\isachardoublequoteopen}{\isasymforall}F\ G\ H{\isachardot}\ Con\ F\ G\ H\ {\isasymlongrightarrow}\ F\ {\isasymin}\ S\ {\isasymlongrightarrow}\ {\isacharbraceleft}G{\isacharcomma}H{\isacharbraceright}\ {\isasymunion}\ S\ {\isasymin}\ C{\isachardoublequoteclose}\isanewline
\ \ \ \ \isacommand{using}\isamarkupfalse%
\ H\ \isacommand{by}\isamarkupfalse%
\ {\isacharparenleft}iprover\ elim{\isacharcolon}\ conjunct{\isadigit{2}}\ conjunct{\isadigit{1}}{\isacharparenright}\isanewline
\ \ \isacommand{have}\isamarkupfalse%
\ A{\isadigit{3}}{\isacharcolon}{\isachardoublequoteopen}{\isasymforall}F\ G\ H{\isachardot}\ Con\ F\ G\ H\ {\isasymlongrightarrow}\ F\ {\isasymin}\ S\ {\isasymlongrightarrow}\ {\isacharbraceleft}G{\isacharcomma}\ H{\isacharbraceright}\ {\isasymunion}\ S\ {\isasymin}\ {\isacharparenleft}extensionFin\ C{\isacharparenright}{\isachardoublequoteclose}\isanewline
\ \ \isacommand{proof}\isamarkupfalse%
\ {\isacharparenleft}rule\ allI{\isacharparenright}{\isacharplus}\isanewline
\ \ \ \ \isacommand{fix}\isamarkupfalse%
\ F\ G\ H\isanewline
\ \ \ \ \isacommand{show}\isamarkupfalse%
\ {\isachardoublequoteopen}Con\ F\ G\ H\ {\isasymlongrightarrow}\ F\ {\isasymin}\ S\ {\isasymlongrightarrow}\ {\isacharbraceleft}G{\isacharcomma}\ H{\isacharbraceright}\ {\isasymunion}\ S\ {\isasymin}\ {\isacharparenleft}extensionFin\ C{\isacharparenright}{\isachardoublequoteclose}\isanewline
\ \ \ \ \isacommand{proof}\isamarkupfalse%
\ {\isacharparenleft}rule\ impI{\isacharparenright}{\isacharplus}\isanewline
\ \ \ \ \ \ \isacommand{assume}\isamarkupfalse%
\ {\isachardoublequoteopen}Con\ F\ G\ H{\isachardoublequoteclose}\isanewline
\ \ \ \ \ \ \isacommand{assume}\isamarkupfalse%
\ {\isachardoublequoteopen}F\ {\isasymin}\ S{\isachardoublequoteclose}\ \isanewline
\ \ \ \ \ \ \isacommand{have}\isamarkupfalse%
\ {\isachardoublequoteopen}Con\ F\ G\ H\ {\isasymlongrightarrow}\ F\ {\isasymin}\ S\ {\isasymlongrightarrow}\ {\isacharbraceleft}G{\isacharcomma}H{\isacharbraceright}\ {\isasymunion}\ S\ {\isasymin}\ C{\isachardoublequoteclose}\isanewline
\ \ \ \ \ \ \ \ \isacommand{using}\isamarkupfalse%
\ S{\isadigit{3}}\ \isacommand{by}\isamarkupfalse%
\ {\isacharparenleft}iprover\ elim{\isacharcolon}\ allE{\isacharparenright}\isanewline
\ \ \ \ \ \ \isacommand{then}\isamarkupfalse%
\ \isacommand{have}\isamarkupfalse%
\ {\isachardoublequoteopen}F\ {\isasymin}\ S\ {\isasymlongrightarrow}\ {\isacharbraceleft}G{\isacharcomma}H{\isacharbraceright}\ {\isasymunion}\ S\ {\isasymin}\ C{\isachardoublequoteclose}\isanewline
\ \ \ \ \ \ \ \ \isacommand{using}\isamarkupfalse%
\ {\isacartoucheopen}Con\ F\ G\ H{\isacartoucheclose}\ \isacommand{by}\isamarkupfalse%
\ {\isacharparenleft}rule\ mp{\isacharparenright}\isanewline
\ \ \ \ \ \ \isacommand{then}\isamarkupfalse%
\ \isacommand{have}\isamarkupfalse%
\ {\isachardoublequoteopen}{\isacharbraceleft}G{\isacharcomma}H{\isacharbraceright}\ {\isasymunion}\ S\ {\isasymin}\ C{\isachardoublequoteclose}\isanewline
\ \ \ \ \ \ \ \ \isacommand{using}\isamarkupfalse%
\ {\isacartoucheopen}F\ {\isasymin}\ S{\isacartoucheclose}\ \isacommand{by}\isamarkupfalse%
\ {\isacharparenleft}rule\ mp{\isacharparenright}\isanewline
\ \ \ \ \ \ \isacommand{thus}\isamarkupfalse%
\ {\isachardoublequoteopen}{\isacharbraceleft}G{\isacharcomma}H{\isacharbraceright}\ {\isasymunion}\ S\ {\isasymin}\ {\isacharparenleft}extensionFin\ C{\isacharparenright}{\isachardoublequoteclose}\isanewline
\ \ \ \ \ \ \ \ \isacommand{unfolding}\isamarkupfalse%
\ extensionFin\ \isacommand{by}\isamarkupfalse%
\ {\isacharparenleft}rule\ UnI{\isadigit{1}}{\isacharparenright}\isanewline
\ \ \ \ \isacommand{qed}\isamarkupfalse%
\isanewline
\ \ \isacommand{qed}\isamarkupfalse%
\isanewline
\ \ \isacommand{have}\isamarkupfalse%
\ S{\isadigit{4}}{\isacharcolon}{\isachardoublequoteopen}{\isasymforall}F\ G\ H{\isachardot}\ Dis\ F\ G\ H\ {\isasymlongrightarrow}\ F\ {\isasymin}\ S\ {\isasymlongrightarrow}\ {\isacharbraceleft}G{\isacharbraceright}\ {\isasymunion}\ S\ {\isasymin}\ C\ {\isasymor}\ {\isacharbraceleft}H{\isacharbraceright}\ {\isasymunion}\ S\ {\isasymin}\ C{\isachardoublequoteclose}\isanewline
\ \ \ \ \isacommand{using}\isamarkupfalse%
\ H\ \isacommand{by}\isamarkupfalse%
\ {\isacharparenleft}iprover\ elim{\isacharcolon}\ conjunct{\isadigit{2}}{\isacharparenright}\isanewline
\ \ \isacommand{have}\isamarkupfalse%
\ A{\isadigit{4}}{\isacharcolon}{\isachardoublequoteopen}{\isasymforall}F\ G\ H{\isachardot}\ Dis\ F\ G\ H\ {\isasymlongrightarrow}\ F\ {\isasymin}\ S\ {\isasymlongrightarrow}\ {\isacharbraceleft}G{\isacharbraceright}\ {\isasymunion}\ S\ {\isasymin}\ {\isacharparenleft}extensionFin\ C{\isacharparenright}\ {\isasymor}\ {\isacharbraceleft}H{\isacharbraceright}\ {\isasymunion}\ S\ {\isasymin}\ {\isacharparenleft}extensionFin\ C{\isacharparenright}{\isachardoublequoteclose}\isanewline
\ \ \isacommand{proof}\isamarkupfalse%
\ {\isacharparenleft}rule\ allI{\isacharparenright}{\isacharplus}\isanewline
\ \ \ \ \isacommand{fix}\isamarkupfalse%
\ F\ G\ H\isanewline
\ \ \ \ \isacommand{show}\isamarkupfalse%
\ {\isachardoublequoteopen}Dis\ F\ G\ H\ {\isasymlongrightarrow}\ F\ {\isasymin}\ S\ {\isasymlongrightarrow}\ {\isacharbraceleft}G{\isacharbraceright}\ {\isasymunion}\ S\ {\isasymin}\ {\isacharparenleft}extensionFin\ C{\isacharparenright}\ {\isasymor}\ {\isacharbraceleft}H{\isacharbraceright}\ {\isasymunion}\ S\ {\isasymin}\ {\isacharparenleft}extensionFin\ C{\isacharparenright}{\isachardoublequoteclose}\isanewline
\ \ \ \ \isacommand{proof}\isamarkupfalse%
\ {\isacharparenleft}rule\ impI{\isacharparenright}{\isacharplus}\isanewline
\ \ \ \ \ \ \isacommand{assume}\isamarkupfalse%
\ {\isachardoublequoteopen}Dis\ F\ G\ H{\isachardoublequoteclose}\isanewline
\ \ \ \ \ \ \isacommand{assume}\isamarkupfalse%
\ {\isachardoublequoteopen}F\ {\isasymin}\ S{\isachardoublequoteclose}\ \isanewline
\ \ \ \ \ \ \isacommand{have}\isamarkupfalse%
\ {\isachardoublequoteopen}Dis\ F\ G\ H\ {\isasymlongrightarrow}\ F\ {\isasymin}\ S\ {\isasymlongrightarrow}\ {\isacharbraceleft}G{\isacharbraceright}\ {\isasymunion}\ S\ {\isasymin}\ C\ {\isasymor}\ {\isacharbraceleft}H{\isacharbraceright}\ {\isasymunion}\ S\ {\isasymin}\ C{\isachardoublequoteclose}\isanewline
\ \ \ \ \ \ \ \ \isacommand{using}\isamarkupfalse%
\ S{\isadigit{4}}\ \isacommand{by}\isamarkupfalse%
\ {\isacharparenleft}iprover\ elim{\isacharcolon}\ allE{\isacharparenright}\isanewline
\ \ \ \ \ \ \isacommand{then}\isamarkupfalse%
\ \isacommand{have}\isamarkupfalse%
\ {\isachardoublequoteopen}F\ {\isasymin}\ S\ {\isasymlongrightarrow}\ {\isacharbraceleft}G{\isacharbraceright}\ {\isasymunion}\ S\ {\isasymin}\ C\ {\isasymor}\ {\isacharbraceleft}H{\isacharbraceright}\ {\isasymunion}\ S\ {\isasymin}\ C{\isachardoublequoteclose}\isanewline
\ \ \ \ \ \ \ \ \isacommand{using}\isamarkupfalse%
\ {\isacartoucheopen}Dis\ F\ G\ H{\isacartoucheclose}\ \isacommand{by}\isamarkupfalse%
\ {\isacharparenleft}rule\ mp{\isacharparenright}\isanewline
\ \ \ \ \ \ \isacommand{then}\isamarkupfalse%
\ \isacommand{have}\isamarkupfalse%
\ {\isachardoublequoteopen}{\isacharbraceleft}G{\isacharbraceright}\ {\isasymunion}\ S\ {\isasymin}\ C\ {\isasymor}\ {\isacharbraceleft}H{\isacharbraceright}\ {\isasymunion}\ S\ {\isasymin}\ C{\isachardoublequoteclose}\isanewline
\ \ \ \ \ \ \ \ \isacommand{using}\isamarkupfalse%
\ {\isacartoucheopen}F\ {\isasymin}\ S{\isacartoucheclose}\ \isacommand{by}\isamarkupfalse%
\ {\isacharparenleft}rule\ mp{\isacharparenright}\isanewline
\ \ \ \ \ \ \isacommand{thus}\isamarkupfalse%
\ {\isachardoublequoteopen}{\isacharbraceleft}G{\isacharbraceright}\ {\isasymunion}\ S\ {\isasymin}\ {\isacharparenleft}extensionFin\ C{\isacharparenright}\ {\isasymor}\ {\isacharbraceleft}H{\isacharbraceright}\ {\isasymunion}\ S\ {\isasymin}\ {\isacharparenleft}extensionFin\ C{\isacharparenright}{\isachardoublequoteclose}\isanewline
\ \ \ \ \ \ \isacommand{proof}\isamarkupfalse%
\ {\isacharparenleft}rule\ disjE{\isacharparenright}\isanewline
\ \ \ \ \ \ \ \ \isacommand{assume}\isamarkupfalse%
\ {\isachardoublequoteopen}{\isacharbraceleft}G{\isacharbraceright}\ {\isasymunion}\ S\ {\isasymin}\ C{\isachardoublequoteclose}\isanewline
\ \ \ \ \ \ \ \ \isacommand{then}\isamarkupfalse%
\ \isacommand{have}\isamarkupfalse%
\ {\isachardoublequoteopen}{\isacharbraceleft}G{\isacharbraceright}\ {\isasymunion}\ S\ {\isasymin}\ {\isacharparenleft}extensionFin\ C{\isacharparenright}{\isachardoublequoteclose}\isanewline
\ \ \ \ \ \ \ \ \ \ \isacommand{unfolding}\isamarkupfalse%
\ extensionFin\ \isacommand{by}\isamarkupfalse%
\ {\isacharparenleft}rule\ UnI{\isadigit{1}}{\isacharparenright}\isanewline
\ \ \ \ \ \ \ \ \isacommand{thus}\isamarkupfalse%
\ {\isachardoublequoteopen}{\isacharbraceleft}G{\isacharbraceright}\ {\isasymunion}\ S\ {\isasymin}\ {\isacharparenleft}extensionFin\ C{\isacharparenright}\ {\isasymor}\ {\isacharbraceleft}H{\isacharbraceright}\ {\isasymunion}\ S\ {\isasymin}\ {\isacharparenleft}extensionFin\ C{\isacharparenright}{\isachardoublequoteclose}\isanewline
\ \ \ \ \ \ \ \ \ \ \isacommand{by}\isamarkupfalse%
\ {\isacharparenleft}rule\ disjI{\isadigit{1}}{\isacharparenright}\isanewline
\ \ \ \ \ \ \isacommand{next}\isamarkupfalse%
\isanewline
\ \ \ \ \ \ \ \ \isacommand{assume}\isamarkupfalse%
\ {\isachardoublequoteopen}{\isacharbraceleft}H{\isacharbraceright}\ {\isasymunion}\ S\ {\isasymin}\ C{\isachardoublequoteclose}\isanewline
\ \ \ \ \ \ \ \ \isacommand{then}\isamarkupfalse%
\ \isacommand{have}\isamarkupfalse%
\ {\isachardoublequoteopen}{\isacharbraceleft}H{\isacharbraceright}\ {\isasymunion}\ S\ {\isasymin}\ {\isacharparenleft}extensionFin\ C{\isacharparenright}{\isachardoublequoteclose}\isanewline
\ \ \ \ \ \ \ \ \ \ \isacommand{unfolding}\isamarkupfalse%
\ extensionFin\ \isacommand{by}\isamarkupfalse%
\ {\isacharparenleft}rule\ UnI{\isadigit{1}}{\isacharparenright}\isanewline
\ \ \ \ \ \ \ \ \isacommand{thus}\isamarkupfalse%
\ {\isachardoublequoteopen}{\isacharbraceleft}G{\isacharbraceright}\ {\isasymunion}\ S\ {\isasymin}\ {\isacharparenleft}extensionFin\ C{\isacharparenright}\ {\isasymor}\ {\isacharbraceleft}H{\isacharbraceright}\ {\isasymunion}\ S\ {\isasymin}\ {\isacharparenleft}extensionFin\ C{\isacharparenright}{\isachardoublequoteclose}\isanewline
\ \ \ \ \ \ \ \ \ \ \isacommand{by}\isamarkupfalse%
\ {\isacharparenleft}rule\ disjI{\isadigit{2}}{\isacharparenright}\isanewline
\ \ \ \ \ \ \isacommand{qed}\isamarkupfalse%
\isanewline
\ \ \ \ \isacommand{qed}\isamarkupfalse%
\isanewline
\ \ \isacommand{qed}\isamarkupfalse%
\isanewline
\ \ \isacommand{show}\isamarkupfalse%
\ {\isachardoublequoteopen}{\isasymbottom}\ {\isasymnotin}\ S\ {\isasymand}\isanewline
\ \ \ \ \ \ \ \ {\isacharparenleft}{\isasymforall}k{\isachardot}\ Atom\ k\ {\isasymin}\ S\ {\isasymlongrightarrow}\ \isactrlbold {\isasymnot}\ {\isacharparenleft}Atom\ k{\isacharparenright}\ {\isasymin}\ S\ {\isasymlongrightarrow}\ False{\isacharparenright}\ {\isasymand}\isanewline
\ \ \ \ \ \ \ \ {\isacharparenleft}{\isasymforall}F\ G\ H{\isachardot}\ Con\ F\ G\ H\ {\isasymlongrightarrow}\ F\ {\isasymin}\ S\ {\isasymlongrightarrow}\ {\isacharbraceleft}G{\isacharcomma}\ H{\isacharbraceright}\ {\isasymunion}\ S\ {\isasymin}\ {\isacharparenleft}extensionFin\ C{\isacharparenright}{\isacharparenright}\ {\isasymand}\isanewline
\ \ \ \ \ \ \ \ {\isacharparenleft}{\isasymforall}F\ G\ H{\isachardot}\ Dis\ F\ G\ H\ {\isasymlongrightarrow}\ F\ {\isasymin}\ S\ {\isasymlongrightarrow}\ {\isacharbraceleft}G{\isacharbraceright}\ {\isasymunion}\ S\ {\isasymin}\ {\isacharparenleft}extensionFin\ C{\isacharparenright}\ {\isasymor}\ {\isacharbraceleft}H{\isacharbraceright}\ {\isasymunion}\ S\ {\isasymin}\ {\isacharparenleft}extensionFin\ C{\isacharparenright}{\isacharparenright}{\isachardoublequoteclose}\isanewline
\ \ \ \ \isacommand{using}\isamarkupfalse%
\ A{\isadigit{1}}\ A{\isadigit{2}}\ A{\isadigit{3}}\ A{\isadigit{4}}\ \isacommand{by}\isamarkupfalse%
\ {\isacharparenleft}iprover\ intro{\isacharcolon}\ conjI{\isacharparenright}\isanewline
\isacommand{qed}\isamarkupfalse%
%
\endisatagproof
{\isafoldproof}%
%
\isadelimproof
%
\endisadelimproof
%
\begin{isamarkuptext}%
Como hemos señalado con anterioridad, para probar el caso en que \isa{S\ {\isasymin}\ E}, donde \isa{E} es la 
  colección formada por aquellos conjuntos cuyos subconjuntos finitos pertenecen a \isa{C}, precisaremos 
  de distintos lemas auxiliares. El primero de ellos demuestra detalladamente que si \isa{C} es una
  colección con la propiedad de consistencia proposicional y cerrada bajo subconjuntos, \isa{S\ {\isasymin}\ E}
  y sea \isa{F} una fórmula de tipo \isa{{\isasymalpha}} con componentes \isa{{\isasymalpha}\isactrlsub {\isadigit{1}}} y \isa{{\isasymalpha}\isactrlsub {\isadigit{2}}}, se verifica que \isa{{\isacharbraceleft}{\isasymalpha}\isactrlsub {\isadigit{1}}{\isacharcomma}{\isasymalpha}\isactrlsub {\isadigit{2}}{\isacharbraceright}\ {\isasymunion}\ S} 
  pertenece a la extensión \isa{C{\isacharprime}\ {\isacharequal}\ C\ {\isasymunion}\ E}.%
\end{isamarkuptext}\isamarkuptrue%
\isacommand{lemma}\isamarkupfalse%
\ ex{\isadigit{3}}{\isacharunderscore}pcp{\isacharunderscore}SinE{\isacharunderscore}CON{\isacharcolon}\isanewline
\ \ \isakeyword{assumes}\ {\isachardoublequoteopen}pcp\ C{\isachardoublequoteclose}\isanewline
\ \ \ \ \ \ \ \ \ \ {\isachardoublequoteopen}subset{\isacharunderscore}closed\ C{\isachardoublequoteclose}\isanewline
\ \ \ \ \ \ \ \ \ \ {\isachardoublequoteopen}S\ {\isasymin}\ {\isacharparenleft}extF\ C{\isacharparenright}{\isachardoublequoteclose}\isanewline
\ \ \ \ \ \ \ \ \ \ {\isachardoublequoteopen}Con\ F\ G\ H{\isachardoublequoteclose}\isanewline
\ \ \ \ \ \ \ \ \ \ {\isachardoublequoteopen}F\ {\isasymin}\ S{\isachardoublequoteclose}\isanewline
\ \ \isakeyword{shows}\ {\isachardoublequoteopen}{\isacharbraceleft}G{\isacharcomma}H{\isacharbraceright}\ {\isasymunion}\ S\ {\isasymin}\ {\isacharparenleft}extensionFin\ C{\isacharparenright}{\isachardoublequoteclose}\ \isanewline
%
\isadelimproof
%
\endisadelimproof
%
\isatagproof
\isacommand{proof}\isamarkupfalse%
\ {\isacharminus}\isanewline
\ \ \isacommand{have}\isamarkupfalse%
\ PCP{\isacharcolon}{\isachardoublequoteopen}{\isasymforall}S\ {\isasymin}\ C{\isachardot}\isanewline
\ \ {\isasymbottom}\ {\isasymnotin}\ S\isanewline
{\isasymand}\ {\isacharparenleft}{\isasymforall}k{\isachardot}\ Atom\ k\ {\isasymin}\ S\ {\isasymlongrightarrow}\ \isactrlbold {\isasymnot}\ {\isacharparenleft}Atom\ k{\isacharparenright}\ {\isasymin}\ S\ {\isasymlongrightarrow}\ False{\isacharparenright}\isanewline
{\isasymand}\ {\isacharparenleft}{\isasymforall}F\ G\ H{\isachardot}\ Con\ F\ G\ H\ {\isasymlongrightarrow}\ F\ {\isasymin}\ S\ {\isasymlongrightarrow}\ {\isacharbraceleft}G{\isacharcomma}H{\isacharbraceright}\ {\isasymunion}\ S\ {\isasymin}\ C{\isacharparenright}\isanewline
{\isasymand}\ {\isacharparenleft}{\isasymforall}F\ G\ H{\isachardot}\ Dis\ F\ G\ H\ {\isasymlongrightarrow}\ F\ {\isasymin}\ S\ {\isasymlongrightarrow}\ {\isacharbraceleft}G{\isacharbraceright}\ {\isasymunion}\ S\ {\isasymin}\ C\ {\isasymor}\ {\isacharbraceleft}H{\isacharbraceright}\ {\isasymunion}\ S\ {\isasymin}\ C{\isacharparenright}{\isachardoublequoteclose}\isanewline
\ \ \ \ \isacommand{using}\isamarkupfalse%
\ assms{\isacharparenleft}{\isadigit{1}}{\isacharparenright}\ \isacommand{by}\isamarkupfalse%
\ {\isacharparenleft}rule\ pcp{\isacharunderscore}alt{\isadigit{1}}{\isacharparenright}\isanewline
\ \ \isacommand{have}\isamarkupfalse%
\ {\isadigit{1}}{\isacharcolon}{\isachardoublequoteopen}{\isasymforall}S{\isacharprime}\ {\isasymsubseteq}\ S{\isachardot}\ finite\ S{\isacharprime}\ {\isasymlongrightarrow}\ F\ {\isasymin}\ S{\isacharprime}\ {\isasymlongrightarrow}\ {\isacharbraceleft}G{\isacharcomma}H{\isacharbraceright}\ {\isasymunion}\ S{\isacharprime}\ {\isasymin}\ C{\isachardoublequoteclose}\isanewline
\ \ \isacommand{proof}\isamarkupfalse%
\ {\isacharparenleft}rule\ sallI{\isacharparenright}\isanewline
\ \ \ \ \isacommand{fix}\isamarkupfalse%
\ S{\isacharprime}\isanewline
\ \ \ \ \isacommand{assume}\isamarkupfalse%
\ {\isachardoublequoteopen}S{\isacharprime}\ {\isasymsubseteq}\ S{\isachardoublequoteclose}\isanewline
\ \ \ \ \isacommand{show}\isamarkupfalse%
\ {\isachardoublequoteopen}finite\ S{\isacharprime}\ {\isasymlongrightarrow}\ F\ {\isasymin}\ S{\isacharprime}\ {\isasymlongrightarrow}\ {\isacharbraceleft}G{\isacharcomma}H{\isacharbraceright}\ {\isasymunion}\ S{\isacharprime}\ {\isasymin}\ C{\isachardoublequoteclose}\isanewline
\ \ \ \ \isacommand{proof}\isamarkupfalse%
\ {\isacharparenleft}rule\ impI{\isacharparenright}{\isacharplus}\isanewline
\ \ \ \ \ \ \isacommand{assume}\isamarkupfalse%
\ {\isachardoublequoteopen}finite\ S{\isacharprime}{\isachardoublequoteclose}\isanewline
\ \ \ \ \ \ \isacommand{assume}\isamarkupfalse%
\ {\isachardoublequoteopen}F\ {\isasymin}\ S{\isacharprime}{\isachardoublequoteclose}\isanewline
\ \ \ \ \ \ \isacommand{have}\isamarkupfalse%
\ E{\isacharcolon}{\isachardoublequoteopen}{\isasymforall}S{\isacharprime}\ {\isasymsubseteq}\ S{\isachardot}\ finite\ S{\isacharprime}\ {\isasymlongrightarrow}\ S{\isacharprime}\ {\isasymin}\ C{\isachardoublequoteclose}\isanewline
\ \ \ \ \ \ \ \ \isacommand{using}\isamarkupfalse%
\ assms{\isacharparenleft}{\isadigit{3}}{\isacharparenright}\ \isacommand{unfolding}\isamarkupfalse%
\ extF\ \isacommand{by}\isamarkupfalse%
\ {\isacharparenleft}rule\ CollectD{\isacharparenright}\isanewline
\ \ \ \ \ \ \isacommand{then}\isamarkupfalse%
\ \isacommand{have}\isamarkupfalse%
\ {\isachardoublequoteopen}finite\ S{\isacharprime}\ {\isasymlongrightarrow}\ S{\isacharprime}\ {\isasymin}\ C{\isachardoublequoteclose}\isanewline
\ \ \ \ \ \ \ \ \isacommand{using}\isamarkupfalse%
\ {\isacartoucheopen}S{\isacharprime}\ {\isasymsubseteq}\ S{\isacartoucheclose}\ \isacommand{by}\isamarkupfalse%
\ {\isacharparenleft}rule\ sspec{\isacharparenright}\isanewline
\ \ \ \ \ \ \isacommand{then}\isamarkupfalse%
\ \isacommand{have}\isamarkupfalse%
\ {\isachardoublequoteopen}S{\isacharprime}\ {\isasymin}\ C{\isachardoublequoteclose}\isanewline
\ \ \ \ \ \ \ \ \isacommand{using}\isamarkupfalse%
\ {\isacartoucheopen}finite\ S{\isacharprime}{\isacartoucheclose}\ \isacommand{by}\isamarkupfalse%
\ {\isacharparenleft}rule\ mp{\isacharparenright}\isanewline
\ \ \ \ \ \ \isacommand{have}\isamarkupfalse%
\ {\isachardoublequoteopen}{\isasymbottom}\ {\isasymnotin}\ S{\isacharprime}\isanewline
\ \ \ \ \ \ \ \ \ \ \ \ {\isasymand}\ {\isacharparenleft}{\isasymforall}k{\isachardot}\ Atom\ k\ {\isasymin}\ S{\isacharprime}\ {\isasymlongrightarrow}\ \isactrlbold {\isasymnot}\ {\isacharparenleft}Atom\ k{\isacharparenright}\ {\isasymin}\ S{\isacharprime}\ {\isasymlongrightarrow}\ False{\isacharparenright}\isanewline
\ \ \ \ \ \ \ \ \ \ \ \ {\isasymand}\ {\isacharparenleft}{\isasymforall}F\ G\ H{\isachardot}\ Con\ F\ G\ H\ {\isasymlongrightarrow}\ F\ {\isasymin}\ S{\isacharprime}\ {\isasymlongrightarrow}\ {\isacharbraceleft}G{\isacharcomma}H{\isacharbraceright}\ {\isasymunion}\ S{\isacharprime}\ {\isasymin}\ C{\isacharparenright}\isanewline
\ \ \ \ \ \ \ \ \ \ \ \ {\isasymand}\ {\isacharparenleft}{\isasymforall}F\ G\ H{\isachardot}\ Dis\ F\ G\ H\ {\isasymlongrightarrow}\ F\ {\isasymin}\ S{\isacharprime}\ {\isasymlongrightarrow}\ {\isacharbraceleft}G{\isacharbraceright}\ {\isasymunion}\ S{\isacharprime}\ {\isasymin}\ C\ {\isasymor}\ {\isacharbraceleft}H{\isacharbraceright}\ {\isasymunion}\ S{\isacharprime}\ {\isasymin}\ C{\isacharparenright}{\isachardoublequoteclose}\isanewline
\ \ \ \ \ \ \ \ \isacommand{using}\isamarkupfalse%
\ PCP\ {\isacartoucheopen}S{\isacharprime}\ {\isasymin}\ C{\isacartoucheclose}\ \isacommand{by}\isamarkupfalse%
\ {\isacharparenleft}rule\ bspec{\isacharparenright}\isanewline
\ \ \ \ \ \ \isacommand{then}\isamarkupfalse%
\ \isacommand{have}\isamarkupfalse%
\ {\isachardoublequoteopen}{\isasymforall}F\ G\ H{\isachardot}\ Con\ F\ G\ H\ {\isasymlongrightarrow}\ F\ {\isasymin}\ S{\isacharprime}\ {\isasymlongrightarrow}\ {\isacharbraceleft}G{\isacharcomma}\ H{\isacharbraceright}\ {\isasymunion}\ S{\isacharprime}\ {\isasymin}\ C{\isachardoublequoteclose}\isanewline
\ \ \ \ \ \ \ \ \isacommand{by}\isamarkupfalse%
\ {\isacharparenleft}iprover\ elim{\isacharcolon}\ conjunct{\isadigit{2}}\ conjunct{\isadigit{1}}{\isacharparenright}\isanewline
\ \ \ \ \ \ \isacommand{then}\isamarkupfalse%
\ \isacommand{have}\isamarkupfalse%
\ {\isachardoublequoteopen}Con\ F\ G\ H\ {\isasymlongrightarrow}\ F\ {\isasymin}\ S{\isacharprime}\ {\isasymlongrightarrow}\ {\isacharbraceleft}G{\isacharcomma}\ H{\isacharbraceright}\ {\isasymunion}\ S{\isacharprime}\ {\isasymin}\ C{\isachardoublequoteclose}\isanewline
\ \ \ \ \ \ \ \ \isacommand{by}\isamarkupfalse%
\ {\isacharparenleft}iprover\ elim{\isacharcolon}\ allE{\isacharparenright}\isanewline
\ \ \ \ \ \ \isacommand{then}\isamarkupfalse%
\ \isacommand{have}\isamarkupfalse%
\ {\isachardoublequoteopen}F\ {\isasymin}\ S{\isacharprime}\ {\isasymlongrightarrow}\ {\isacharbraceleft}G{\isacharcomma}H{\isacharbraceright}\ {\isasymunion}\ S{\isacharprime}\ {\isasymin}\ C{\isachardoublequoteclose}\isanewline
\ \ \ \ \ \ \ \ \isacommand{using}\isamarkupfalse%
\ assms{\isacharparenleft}{\isadigit{4}}{\isacharparenright}\ \isacommand{by}\isamarkupfalse%
\ {\isacharparenleft}rule\ mp{\isacharparenright}\isanewline
\ \ \ \ \ \ \isacommand{thus}\isamarkupfalse%
\ {\isachardoublequoteopen}{\isacharbraceleft}G{\isacharcomma}\ H{\isacharbraceright}\ {\isasymunion}\ S{\isacharprime}\ {\isasymin}\ C{\isachardoublequoteclose}\isanewline
\ \ \ \ \ \ \ \ \isacommand{using}\isamarkupfalse%
\ {\isacartoucheopen}F\ {\isasymin}\ S{\isacharprime}{\isacartoucheclose}\ \isacommand{by}\isamarkupfalse%
\ {\isacharparenleft}rule\ mp{\isacharparenright}\isanewline
\ \ \ \ \isacommand{qed}\isamarkupfalse%
\isanewline
\ \ \isacommand{qed}\isamarkupfalse%
\isanewline
\ \ \isacommand{have}\isamarkupfalse%
\ {\isachardoublequoteopen}{\isacharbraceleft}G{\isacharcomma}H{\isacharbraceright}\ {\isasymunion}\ S\ {\isasymin}\ {\isacharparenleft}extF\ C{\isacharparenright}{\isachardoublequoteclose}\isanewline
\ \ \ \ \isacommand{unfolding}\isamarkupfalse%
\ mem{\isacharunderscore}Collect{\isacharunderscore}eq\ Un{\isacharunderscore}iff\ extF\isanewline
\ \ \isacommand{proof}\isamarkupfalse%
\ {\isacharparenleft}rule\ sallI{\isacharparenright}\isanewline
\ \ \ \ \isacommand{fix}\isamarkupfalse%
\ S{\isacharprime}\isanewline
\ \ \ \ \isacommand{assume}\isamarkupfalse%
\ H{\isacharcolon}{\isachardoublequoteopen}S{\isacharprime}\ {\isasymsubseteq}\ {\isacharbraceleft}G{\isacharcomma}H{\isacharbraceright}\ {\isasymunion}\ S{\isachardoublequoteclose}\isanewline
\ \ \ \ \isacommand{show}\isamarkupfalse%
\ {\isachardoublequoteopen}finite\ S{\isacharprime}\ {\isasymlongrightarrow}\ S{\isacharprime}\ {\isasymin}\ C{\isachardoublequoteclose}\isanewline
\ \ \ \ \isacommand{proof}\isamarkupfalse%
\ {\isacharparenleft}rule\ impI{\isacharparenright}\isanewline
\ \ \ \ \ \ \isacommand{assume}\isamarkupfalse%
\ {\isachardoublequoteopen}finite\ S{\isacharprime}{\isachardoublequoteclose}\isanewline
\ \ \ \ \ \ \isacommand{have}\isamarkupfalse%
\ {\isachardoublequoteopen}S{\isacharprime}\ {\isacharminus}\ {\isacharbraceleft}G{\isacharcomma}H{\isacharbraceright}\ {\isasymsubseteq}\ S{\isachardoublequoteclose}\isanewline
\ \ \ \ \ \ \ \ \isacommand{using}\isamarkupfalse%
\ H\ \isacommand{by}\isamarkupfalse%
\ {\isacharparenleft}simp\ only{\isacharcolon}\ Diff{\isacharunderscore}subset{\isacharunderscore}conv{\isacharparenright}\isanewline
\ \ \ \ \ \ \isacommand{have}\isamarkupfalse%
\ {\isachardoublequoteopen}F\ {\isasymin}\ S\ {\isasymand}\ {\isacharparenleft}S{\isacharprime}\ {\isacharminus}\ {\isacharbraceleft}G{\isacharcomma}H{\isacharbraceright}\ {\isasymsubseteq}\ S{\isacharparenright}{\isachardoublequoteclose}\isanewline
\ \ \ \ \ \ \ \ \isacommand{using}\isamarkupfalse%
\ assms{\isacharparenleft}{\isadigit{5}}{\isacharparenright}\ {\isacartoucheopen}S{\isacharprime}\ {\isacharminus}\ {\isacharbraceleft}G{\isacharcomma}H{\isacharbraceright}\ {\isasymsubseteq}\ S{\isacartoucheclose}\ \isacommand{by}\isamarkupfalse%
\ {\isacharparenleft}rule\ conjI{\isacharparenright}\isanewline
\ \ \ \ \ \ \isacommand{then}\isamarkupfalse%
\ \isacommand{have}\isamarkupfalse%
\ {\isachardoublequoteopen}insert\ F\ \ {\isacharparenleft}S{\isacharprime}\ {\isacharminus}\ {\isacharbraceleft}G{\isacharcomma}H{\isacharbraceright}{\isacharparenright}\ {\isasymsubseteq}\ S{\isachardoublequoteclose}\ \isanewline
\ \ \ \ \ \ \ \ \isacommand{by}\isamarkupfalse%
\ {\isacharparenleft}simp\ only{\isacharcolon}\ insert{\isacharunderscore}subset{\isacharparenright}\isanewline
\ \ \ \ \ \ \isacommand{have}\isamarkupfalse%
\ F{\isadigit{1}}{\isacharcolon}{\isachardoublequoteopen}finite\ {\isacharparenleft}insert\ F\ \ {\isacharparenleft}S{\isacharprime}\ {\isacharminus}\ {\isacharbraceleft}G{\isacharcomma}H{\isacharbraceright}{\isacharparenright}{\isacharparenright}\ {\isasymlongrightarrow}\ F\ {\isasymin}\ {\isacharparenleft}insert\ F\ \ {\isacharparenleft}S{\isacharprime}\ {\isacharminus}\ {\isacharbraceleft}G{\isacharcomma}H{\isacharbraceright}{\isacharparenright}{\isacharparenright}\ {\isasymlongrightarrow}\ {\isacharbraceleft}G{\isacharcomma}H{\isacharbraceright}\ {\isasymunion}\ {\isacharparenleft}insert\ F\ \ {\isacharparenleft}S{\isacharprime}\ {\isacharminus}\ {\isacharbraceleft}G{\isacharcomma}H{\isacharbraceright}{\isacharparenright}{\isacharparenright}\ {\isasymin}\ C{\isachardoublequoteclose}\isanewline
\ \ \ \ \ \ \ \ \isacommand{using}\isamarkupfalse%
\ {\isadigit{1}}\ {\isacartoucheopen}insert\ F\ \ {\isacharparenleft}S{\isacharprime}\ {\isacharminus}\ {\isacharbraceleft}G{\isacharcomma}H{\isacharbraceright}{\isacharparenright}\ {\isasymsubseteq}\ S{\isacartoucheclose}\ \isacommand{by}\isamarkupfalse%
\ {\isacharparenleft}rule\ sspec{\isacharparenright}\isanewline
\ \ \ \ \ \ \isacommand{have}\isamarkupfalse%
\ {\isachardoublequoteopen}finite\ {\isacharparenleft}S{\isacharprime}\ {\isacharminus}\ {\isacharbraceleft}G{\isacharcomma}H{\isacharbraceright}{\isacharparenright}{\isachardoublequoteclose}\isanewline
\ \ \ \ \ \ \ \ \isacommand{using}\isamarkupfalse%
\ {\isacartoucheopen}finite\ S{\isacharprime}{\isacartoucheclose}\ \isacommand{by}\isamarkupfalse%
\ {\isacharparenleft}rule\ finite{\isacharunderscore}Diff{\isacharparenright}\isanewline
\ \ \ \ \ \ \isacommand{then}\isamarkupfalse%
\ \isacommand{have}\isamarkupfalse%
\ {\isachardoublequoteopen}finite\ {\isacharparenleft}insert\ F\ {\isacharparenleft}S{\isacharprime}\ {\isacharminus}\ {\isacharbraceleft}G{\isacharcomma}H{\isacharbraceright}{\isacharparenright}{\isacharparenright}{\isachardoublequoteclose}\ \isanewline
\ \ \ \ \ \ \ \ \isacommand{by}\isamarkupfalse%
\ {\isacharparenleft}rule\ finite{\isachardot}insertI{\isacharparenright}\isanewline
\ \ \ \ \ \ \isacommand{have}\isamarkupfalse%
\ F{\isadigit{2}}{\isacharcolon}{\isachardoublequoteopen}F\ {\isasymin}\ {\isacharparenleft}insert\ F\ \ {\isacharparenleft}S{\isacharprime}\ {\isacharminus}\ {\isacharbraceleft}G{\isacharcomma}H{\isacharbraceright}{\isacharparenright}{\isacharparenright}\ {\isasymlongrightarrow}\ {\isacharbraceleft}G{\isacharcomma}H{\isacharbraceright}\ {\isasymunion}\ {\isacharparenleft}insert\ F\ \ {\isacharparenleft}S{\isacharprime}\ {\isacharminus}\ {\isacharbraceleft}G{\isacharcomma}H{\isacharbraceright}{\isacharparenright}{\isacharparenright}\ {\isasymin}\ C{\isachardoublequoteclose}\isanewline
\ \ \ \ \ \ \ \ \isacommand{using}\isamarkupfalse%
\ F{\isadigit{1}}\ {\isacartoucheopen}finite\ {\isacharparenleft}insert\ F\ {\isacharparenleft}S{\isacharprime}\ {\isacharminus}\ {\isacharbraceleft}G{\isacharcomma}H{\isacharbraceright}{\isacharparenright}{\isacharparenright}{\isacartoucheclose}\ \isacommand{by}\isamarkupfalse%
\ {\isacharparenleft}rule\ mp{\isacharparenright}\isanewline
\ \ \ \ \ \ \isacommand{have}\isamarkupfalse%
\ {\isachardoublequoteopen}F\ {\isasymin}\ {\isacharparenleft}insert\ F\ \ {\isacharparenleft}S{\isacharprime}\ {\isacharminus}\ {\isacharbraceleft}G{\isacharcomma}H{\isacharbraceright}{\isacharparenright}{\isacharparenright}{\isachardoublequoteclose}\isanewline
\ \ \ \ \ \ \ \ \isacommand{by}\isamarkupfalse%
\ {\isacharparenleft}simp\ only{\isacharcolon}\ insertI{\isadigit{1}}{\isacharparenright}\isanewline
\ \ \ \ \ \ \isacommand{have}\isamarkupfalse%
\ F{\isadigit{3}}{\isacharcolon}{\isachardoublequoteopen}{\isacharbraceleft}G{\isacharcomma}H{\isacharbraceright}\ {\isasymunion}\ {\isacharparenleft}insert\ F\ {\isacharparenleft}S{\isacharprime}\ {\isacharminus}\ {\isacharbraceleft}G{\isacharcomma}H{\isacharbraceright}{\isacharparenright}{\isacharparenright}\ {\isasymin}\ C{\isachardoublequoteclose}\isanewline
\ \ \ \ \ \ \ \ \isacommand{using}\isamarkupfalse%
\ F{\isadigit{2}}\ {\isacartoucheopen}F\ {\isasymin}\ insert\ F\ {\isacharparenleft}S{\isacharprime}\ {\isacharminus}\ {\isacharbraceleft}G{\isacharcomma}H{\isacharbraceright}{\isacharparenright}{\isacartoucheclose}\ \isacommand{by}\isamarkupfalse%
\ {\isacharparenleft}rule\ mp{\isacharparenright}\isanewline
\ \ \ \ \ \ \isacommand{have}\isamarkupfalse%
\ IU{\isadigit{1}}{\isacharcolon}{\isachardoublequoteopen}insert\ F\ {\isacharparenleft}S{\isacharprime}\ {\isacharminus}\ {\isacharbraceleft}G{\isacharcomma}H{\isacharbraceright}{\isacharparenright}\ {\isacharequal}\ {\isacharbraceleft}F{\isacharbraceright}\ {\isasymunion}\ {\isacharparenleft}S{\isacharprime}\ {\isacharminus}\ {\isacharbraceleft}G{\isacharcomma}H{\isacharbraceright}{\isacharparenright}{\isachardoublequoteclose}\isanewline
\ \ \ \ \ \ \ \ \isacommand{by}\isamarkupfalse%
\ {\isacharparenleft}rule\ insert{\isacharunderscore}is{\isacharunderscore}Un{\isacharparenright}\isanewline
\ \ \ \ \ \ \isacommand{have}\isamarkupfalse%
\ IU{\isadigit{2}}{\isacharcolon}{\isachardoublequoteopen}insert\ F\ {\isacharparenleft}{\isacharbraceleft}G{\isacharcomma}H{\isacharbraceright}\ {\isasymunion}\ S{\isacharprime}{\isacharparenright}\ {\isacharequal}\ {\isacharbraceleft}F{\isacharbraceright}\ {\isasymunion}\ {\isacharparenleft}{\isacharbraceleft}G{\isacharcomma}H{\isacharbraceright}\ {\isasymunion}\ S{\isacharprime}{\isacharparenright}{\isachardoublequoteclose}\isanewline
\ \ \ \ \ \ \ \ \isacommand{by}\isamarkupfalse%
\ {\isacharparenleft}rule\ insert{\isacharunderscore}is{\isacharunderscore}Un{\isacharparenright}\isanewline
\ \ \ \ \ \ \isacommand{have}\isamarkupfalse%
\ GH{\isacharcolon}{\isachardoublequoteopen}insert\ G\ {\isacharparenleft}insert\ H\ S{\isacharprime}{\isacharparenright}\ {\isacharequal}\ {\isacharbraceleft}G{\isacharcomma}H{\isacharbraceright}\ {\isasymunion}\ S{\isacharprime}{\isachardoublequoteclose}\isanewline
\ \ \ \ \ \ \ \ \isacommand{by}\isamarkupfalse%
\ {\isacharparenleft}rule\ insertSetElem{\isacharparenright}\isanewline
\ \ \ \ \ \ \isacommand{have}\isamarkupfalse%
\ {\isachardoublequoteopen}{\isacharbraceleft}G{\isacharcomma}H{\isacharbraceright}\ {\isasymunion}\ {\isacharparenleft}insert\ F\ {\isacharparenleft}S{\isacharprime}\ {\isacharminus}\ {\isacharbraceleft}G{\isacharcomma}H{\isacharbraceright}{\isacharparenright}{\isacharparenright}\ {\isacharequal}\ {\isacharbraceleft}G{\isacharcomma}H{\isacharbraceright}\ {\isasymunion}\ {\isacharparenleft}{\isacharbraceleft}F{\isacharbraceright}\ {\isasymunion}\ {\isacharparenleft}S{\isacharprime}\ {\isacharminus}\ {\isacharbraceleft}G{\isacharcomma}H{\isacharbraceright}{\isacharparenright}{\isacharparenright}{\isachardoublequoteclose}\isanewline
\ \ \ \ \ \ \ \ \isacommand{by}\isamarkupfalse%
\ {\isacharparenleft}simp\ only{\isacharcolon}\ IU{\isadigit{1}}{\isacharparenright}\isanewline
\ \ \ \ \ \ \isacommand{also}\isamarkupfalse%
\ \isacommand{have}\isamarkupfalse%
\ {\isachardoublequoteopen}{\isasymdots}\ {\isacharequal}\ {\isacharbraceleft}F{\isacharbraceright}\ {\isasymunion}\ {\isacharparenleft}{\isacharbraceleft}G{\isacharcomma}H{\isacharbraceright}\ {\isasymunion}\ {\isacharparenleft}S{\isacharprime}\ {\isacharminus}\ {\isacharbraceleft}G{\isacharcomma}H{\isacharbraceright}{\isacharparenright}{\isacharparenright}{\isachardoublequoteclose}\isanewline
\ \ \ \ \ \ \ \ \isacommand{by}\isamarkupfalse%
\ {\isacharparenleft}simp\ only{\isacharcolon}\ Un{\isacharunderscore}left{\isacharunderscore}commute{\isacharparenright}\isanewline
\ \ \ \ \ \ \isacommand{also}\isamarkupfalse%
\ \isacommand{have}\isamarkupfalse%
\ {\isachardoublequoteopen}{\isasymdots}\ {\isacharequal}\ {\isacharbraceleft}F{\isacharbraceright}\ {\isasymunion}\ {\isacharparenleft}{\isacharbraceleft}G{\isacharcomma}H{\isacharbraceright}\ {\isasymunion}\ S{\isacharprime}{\isacharparenright}{\isachardoublequoteclose}\isanewline
\ \ \ \ \ \ \ \ \isacommand{by}\isamarkupfalse%
\ {\isacharparenleft}simp\ only{\isacharcolon}\ Un{\isacharunderscore}Diff{\isacharunderscore}cancel{\isacharparenright}\isanewline
\ \ \ \ \ \ \isacommand{also}\isamarkupfalse%
\ \isacommand{have}\isamarkupfalse%
\ {\isachardoublequoteopen}{\isasymdots}\ {\isacharequal}\ insert\ F\ {\isacharparenleft}{\isacharbraceleft}G{\isacharcomma}H{\isacharbraceright}\ {\isasymunion}\ S{\isacharprime}{\isacharparenright}{\isachardoublequoteclose}\isanewline
\ \ \ \ \ \ \ \ \isacommand{by}\isamarkupfalse%
\ {\isacharparenleft}simp\ only{\isacharcolon}\ IU{\isadigit{2}}{\isacharparenright}\isanewline
\ \ \ \ \ \ \isacommand{also}\isamarkupfalse%
\ \isacommand{have}\isamarkupfalse%
\ {\isachardoublequoteopen}{\isasymdots}\ {\isacharequal}\ insert\ F\ {\isacharparenleft}insert\ G\ {\isacharparenleft}insert\ H\ S{\isacharprime}{\isacharparenright}{\isacharparenright}{\isachardoublequoteclose}\isanewline
\ \ \ \ \ \ \ \ \isacommand{by}\isamarkupfalse%
\ {\isacharparenleft}simp\ only{\isacharcolon}\ GH{\isacharparenright}\isanewline
\ \ \ \ \ \ \isacommand{finally}\isamarkupfalse%
\ \isacommand{have}\isamarkupfalse%
\ F{\isadigit{4}}{\isacharcolon}{\isachardoublequoteopen}{\isacharbraceleft}G{\isacharcomma}H{\isacharbraceright}\ {\isasymunion}\ {\isacharparenleft}insert\ F\ {\isacharparenleft}S{\isacharprime}\ {\isacharminus}\ {\isacharbraceleft}G{\isacharcomma}H{\isacharbraceright}{\isacharparenright}{\isacharparenright}\ {\isacharequal}\ insert\ F\ {\isacharparenleft}insert\ G\ {\isacharparenleft}insert\ H\ S{\isacharprime}{\isacharparenright}{\isacharparenright}{\isachardoublequoteclose}\isanewline
\ \ \ \ \ \ \ \ \isacommand{by}\isamarkupfalse%
\ this\isanewline
\ \ \ \ \ \ \isacommand{have}\isamarkupfalse%
\ C{\isadigit{1}}{\isacharcolon}{\isachardoublequoteopen}insert\ F\ {\isacharparenleft}insert\ G\ {\isacharparenleft}insert\ H\ S{\isacharprime}{\isacharparenright}{\isacharparenright}\ {\isasymin}\ C{\isachardoublequoteclose}\isanewline
\ \ \ \ \ \ \ \ \isacommand{using}\isamarkupfalse%
\ F{\isadigit{3}}\ \isacommand{by}\isamarkupfalse%
\ {\isacharparenleft}simp\ only{\isacharcolon}\ F{\isadigit{4}}{\isacharparenright}\isanewline
\ \ \ \ \ \ \isacommand{have}\isamarkupfalse%
\ {\isachardoublequoteopen}S{\isacharprime}\ {\isasymsubseteq}\ insert\ F\ S{\isacharprime}{\isachardoublequoteclose}\isanewline
\ \ \ \ \ \ \ \ \isacommand{by}\isamarkupfalse%
\ {\isacharparenleft}rule\ subset{\isacharunderscore}insertI{\isacharparenright}\isanewline
\ \ \ \ \ \ \isacommand{then}\isamarkupfalse%
\ \isacommand{have}\isamarkupfalse%
\ C{\isadigit{2}}{\isacharcolon}{\isachardoublequoteopen}S{\isacharprime}\ {\isasymsubseteq}\ insert\ F\ {\isacharparenleft}insert\ G\ {\isacharparenleft}insert\ H\ S{\isacharprime}{\isacharparenright}{\isacharparenright}{\isachardoublequoteclose}\isanewline
\ \ \ \ \ \ \ \ \isacommand{by}\isamarkupfalse%
\ {\isacharparenleft}simp\ only{\isacharcolon}\ subset{\isacharunderscore}insertI{\isadigit{2}}{\isacharparenright}\isanewline
\ \ \ \ \ \ \isacommand{let}\isamarkupfalse%
\ {\isacharquery}S{\isacharequal}{\isachardoublequoteopen}insert\ F\ {\isacharparenleft}insert\ G\ {\isacharparenleft}insert\ H\ S{\isacharprime}{\isacharparenright}{\isacharparenright}{\isachardoublequoteclose}\isanewline
\ \ \ \ \ \ \isacommand{have}\isamarkupfalse%
\ {\isachardoublequoteopen}{\isasymforall}S\ {\isasymin}\ C{\isachardot}\ {\isasymforall}S{\isacharprime}\ {\isasymsubseteq}\ S{\isachardot}\ S{\isacharprime}\ {\isasymin}\ C{\isachardoublequoteclose}\isanewline
\ \ \ \ \ \ \ \ \isacommand{using}\isamarkupfalse%
\ assms{\isacharparenleft}{\isadigit{2}}{\isacharparenright}\ \isacommand{by}\isamarkupfalse%
\ {\isacharparenleft}simp\ only{\isacharcolon}\ subset{\isacharunderscore}closed{\isacharunderscore}def{\isacharparenright}\isanewline
\ \ \ \ \ \ \isacommand{then}\isamarkupfalse%
\ \isacommand{have}\isamarkupfalse%
\ {\isachardoublequoteopen}{\isasymforall}S{\isacharprime}\ {\isasymsubseteq}\ {\isacharquery}S{\isachardot}\ S{\isacharprime}\ {\isasymin}\ C{\isachardoublequoteclose}\isanewline
\ \ \ \ \ \ \ \ \isacommand{using}\isamarkupfalse%
\ C{\isadigit{1}}\ \isacommand{by}\isamarkupfalse%
\ {\isacharparenleft}rule\ bspec{\isacharparenright}\isanewline
\ \ \ \ \ \ \isacommand{thus}\isamarkupfalse%
\ {\isachardoublequoteopen}S{\isacharprime}\ {\isasymin}\ C{\isachardoublequoteclose}\isanewline
\ \ \ \ \ \ \ \ \isacommand{using}\isamarkupfalse%
\ C{\isadigit{2}}\ \isacommand{by}\isamarkupfalse%
\ {\isacharparenleft}rule\ sspec{\isacharparenright}\isanewline
\ \ \ \ \isacommand{qed}\isamarkupfalse%
\isanewline
\ \ \isacommand{qed}\isamarkupfalse%
\isanewline
\ \ \isacommand{thus}\isamarkupfalse%
\ {\isachardoublequoteopen}{\isacharbraceleft}G{\isacharcomma}H{\isacharbraceright}\ {\isasymunion}\ S\ {\isasymin}\ {\isacharparenleft}extensionFin\ C{\isacharparenright}{\isachardoublequoteclose}\isanewline
\ \ \ \ \isacommand{unfolding}\isamarkupfalse%
\ extensionFin\ \isacommand{by}\isamarkupfalse%
\ {\isacharparenleft}rule\ UnI{\isadigit{2}}{\isacharparenright}\isanewline
\isacommand{qed}\isamarkupfalse%
%
\endisatagproof
{\isafoldproof}%
%
\isadelimproof
%
\endisadelimproof
%
\begin{isamarkuptext}%
Seguidamente, vamos a probar el lema auxiliar \isa{ex{\isadigit{3}}{\isacharunderscore}pcp{\isacharunderscore}SinE{\isacharunderscore}DIS}. Este demuestra que si \isa{C} es 
  una colección con la propiedad de consistencia proposicional y cerrada bajo subconjuntos, \isa{S\ {\isasymin}\ E}
  y sea \isa{F} una fórmula de tipo \isa{{\isasymbeta}} con componentes \isa{{\isasymbeta}\isactrlsub {\isadigit{1}}} y \isa{{\isasymbeta}\isactrlsub {\isadigit{2}}}, se verifica que o bien 
  \isa{{\isacharbraceleft}{\isasymbeta}\isactrlsub {\isadigit{1}}{\isacharbraceright}\ {\isasymunion}\ S\ {\isasymin}\ C{\isacharprime}} o bien \isa{{\isacharbraceleft}{\isasymbeta}\isactrlsub {\isadigit{2}}{\isacharbraceright}\ {\isasymunion}\ S\ {\isasymin}\ C{\isacharprime}}. Dicha prueba se realizará por reducción al absurdo. Para
  ello precisaremos de dos lemas previos que nos permitan llegar a una contradicción: 
  \isa{ex{\isadigit{3}}{\isacharunderscore}pcp{\isacharunderscore}SinE{\isacharunderscore}DIS{\isacharunderscore}auxEx} y \isa{ex{\isadigit{3}}{\isacharunderscore}pcp{\isacharunderscore}SinE{\isacharunderscore}DIS{\isacharunderscore}auxFalse}.

  En primer lugar, veamos la demostración del lema \isa{ex{\isadigit{3}}{\isacharunderscore}pcp{\isacharunderscore}SinE{\isacharunderscore}DIS{\isacharunderscore}auxEx}. Este prueba que dada 
  \isa{C} una colección con la propiedad de consistencia proposicional y cerrada bajo subconjuntos, 
  \isa{S\ {\isasymin}\ E} y sea \isa{F} es una fórmula de tipo \isa{{\isasymbeta}} de componentes \isa{{\isasymbeta}\isactrlsub {\isadigit{1}}} y \isa{{\isasymbeta}\isactrlsub {\isadigit{2}}}, si consideramos \isa{S\isactrlsub {\isadigit{1}}} y 
  \isa{S\isactrlsub {\isadigit{2}}} subconjuntos finitos cualesquiera de \isa{S} tales que \isa{F\ {\isasymin}\ S\isactrlsub {\isadigit{1}}} y\\ \isa{F\ {\isasymin}\ S\isactrlsub {\isadigit{2}}}, entonces existe una 
  fórmula \isa{I\ {\isasymin}\ {\isacharbraceleft}{\isasymbeta}\isactrlsub {\isadigit{1}}{\isacharcomma}{\isasymbeta}\isactrlsub {\isadigit{2}}{\isacharbraceright}} tal que se verifica que tanto \isa{{\isacharbraceleft}I{\isacharbraceright}\ {\isasymunion}\ S\isactrlsub {\isadigit{1}}} como \isa{{\isacharbraceleft}I{\isacharbraceright}\ {\isasymunion}\ S\isactrlsub {\isadigit{2}}} están en \isa{C}.%
\end{isamarkuptext}\isamarkuptrue%
\isacommand{lemma}\isamarkupfalse%
\ ex{\isadigit{3}}{\isacharunderscore}pcp{\isacharunderscore}SinE{\isacharunderscore}DIS{\isacharunderscore}auxEx{\isacharcolon}\isanewline
\ \ \isakeyword{assumes}\ {\isachardoublequoteopen}pcp\ C{\isachardoublequoteclose}\isanewline
\ \ \ \ \ \ \ \ \ \ {\isachardoublequoteopen}subset{\isacharunderscore}closed\ C{\isachardoublequoteclose}\isanewline
\ \ \ \ \ \ \ \ \ \ {\isachardoublequoteopen}S\ {\isasymin}\ {\isacharparenleft}extF\ C{\isacharparenright}{\isachardoublequoteclose}\isanewline
\ \ \ \ \ \ \ \ \ \ {\isachardoublequoteopen}Dis\ F\ G\ H{\isachardoublequoteclose}\isanewline
\ \ \ \ \ \ \ \ \ \ {\isachardoublequoteopen}S{\isadigit{1}}\ {\isasymsubseteq}\ S{\isachardoublequoteclose}\isanewline
\ \ \ \ \ \ \ \ \ \ {\isachardoublequoteopen}finite\ S{\isadigit{1}}{\isachardoublequoteclose}\isanewline
\ \ \ \ \ \ \ \ \ \ {\isachardoublequoteopen}F\ {\isasymin}\ S{\isadigit{1}}{\isachardoublequoteclose}\isanewline
\ \ \ \ \ \ \ \ \ \ {\isachardoublequoteopen}S{\isadigit{2}}\ {\isasymsubseteq}\ S{\isachardoublequoteclose}\isanewline
\ \ \ \ \ \ \ \ \ \ {\isachardoublequoteopen}finite\ S{\isadigit{2}}{\isachardoublequoteclose}\isanewline
\ \ \ \ \ \ \ \ \ \ {\isachardoublequoteopen}F\ {\isasymin}\ S{\isadigit{2}}{\isachardoublequoteclose}\isanewline
\ \ \isakeyword{shows}\ {\isachardoublequoteopen}{\isasymexists}I{\isasymin}{\isacharbraceleft}G{\isacharcomma}H{\isacharbraceright}{\isachardot}\ insert\ I\ S{\isadigit{1}}\ {\isasymin}\ C\ {\isasymand}\ insert\ I\ S{\isadigit{2}}\ {\isasymin}\ C{\isachardoublequoteclose}\ \isanewline
%
\isadelimproof
%
\endisadelimproof
%
\isatagproof
\isacommand{proof}\isamarkupfalse%
\ {\isacharminus}\isanewline
\ \ \isacommand{let}\isamarkupfalse%
\ {\isacharquery}S\ {\isacharequal}\ {\isachardoublequoteopen}S{\isadigit{1}}\ {\isasymunion}\ S{\isadigit{2}}{\isachardoublequoteclose}\isanewline
\ \ \isacommand{have}\isamarkupfalse%
\ {\isachardoublequoteopen}S{\isadigit{1}}\ {\isasymsubseteq}\ {\isacharquery}S{\isachardoublequoteclose}\isanewline
\ \ \ \ \isacommand{by}\isamarkupfalse%
\ {\isacharparenleft}simp\ only{\isacharcolon}\ Un{\isacharunderscore}upper{\isadigit{1}}{\isacharparenright}\isanewline
\ \ \isacommand{have}\isamarkupfalse%
\ {\isachardoublequoteopen}S{\isadigit{2}}\ {\isasymsubseteq}\ {\isacharquery}S{\isachardoublequoteclose}\isanewline
\ \ \ \ \isacommand{by}\isamarkupfalse%
\ {\isacharparenleft}simp\ only{\isacharcolon}\ Un{\isacharunderscore}upper{\isadigit{2}}{\isacharparenright}\isanewline
\ \ \isacommand{have}\isamarkupfalse%
\ {\isachardoublequoteopen}finite\ {\isacharquery}S{\isachardoublequoteclose}\isanewline
\ \ \ \ \isacommand{using}\isamarkupfalse%
\ assms{\isacharparenleft}{\isadigit{6}}{\isacharparenright}\ assms{\isacharparenleft}{\isadigit{9}}{\isacharparenright}\ \isacommand{by}\isamarkupfalse%
\ {\isacharparenleft}rule\ finite{\isacharunderscore}UnI{\isacharparenright}\isanewline
\ \ \isacommand{have}\isamarkupfalse%
\ {\isachardoublequoteopen}{\isacharquery}S\ {\isasymsubseteq}\ S{\isachardoublequoteclose}\ \isanewline
\ \ \ \ \isacommand{using}\isamarkupfalse%
\ assms{\isacharparenleft}{\isadigit{5}}{\isacharparenright}\ assms{\isacharparenleft}{\isadigit{8}}{\isacharparenright}\ \isacommand{by}\isamarkupfalse%
\ {\isacharparenleft}simp\ only{\isacharcolon}\ Un{\isacharunderscore}subset{\isacharunderscore}iff{\isacharparenright}\isanewline
\ \ \isacommand{have}\isamarkupfalse%
\ {\isachardoublequoteopen}{\isasymforall}S{\isacharprime}\ {\isasymsubseteq}\ S{\isachardot}\ finite\ S{\isacharprime}\ {\isasymlongrightarrow}\ S{\isacharprime}\ {\isasymin}\ C{\isachardoublequoteclose}\isanewline
\ \ \ \ \isacommand{using}\isamarkupfalse%
\ assms{\isacharparenleft}{\isadigit{3}}{\isacharparenright}\ \isacommand{unfolding}\isamarkupfalse%
\ extF\ \isacommand{by}\isamarkupfalse%
\ {\isacharparenleft}rule\ CollectD{\isacharparenright}\isanewline
\ \ \isacommand{then}\isamarkupfalse%
\ \isacommand{have}\isamarkupfalse%
\ {\isachardoublequoteopen}finite\ {\isacharquery}S\ {\isasymlongrightarrow}\ {\isacharquery}S\ {\isasymin}\ C{\isachardoublequoteclose}\isanewline
\ \ \ \ \isacommand{using}\isamarkupfalse%
\ {\isacartoucheopen}{\isacharquery}S\ {\isasymsubseteq}\ S{\isacartoucheclose}\ \isacommand{by}\isamarkupfalse%
\ {\isacharparenleft}rule\ sspec{\isacharparenright}\isanewline
\ \ \isacommand{then}\isamarkupfalse%
\ \isacommand{have}\isamarkupfalse%
\ {\isachardoublequoteopen}{\isacharquery}S\ {\isasymin}\ C{\isachardoublequoteclose}\ \isanewline
\ \ \ \ \isacommand{using}\isamarkupfalse%
\ {\isacartoucheopen}finite\ {\isacharquery}S{\isacartoucheclose}\ \isacommand{by}\isamarkupfalse%
\ {\isacharparenleft}rule\ mp{\isacharparenright}\isanewline
\ \ \isacommand{have}\isamarkupfalse%
\ {\isachardoublequoteopen}F\ {\isasymin}\ {\isacharquery}S{\isachardoublequoteclose}\ \isanewline
\ \ \ \ \isacommand{using}\isamarkupfalse%
\ assms{\isacharparenleft}{\isadigit{7}}{\isacharparenright}\ \isacommand{by}\isamarkupfalse%
\ {\isacharparenleft}rule\ UnI{\isadigit{1}}{\isacharparenright}\isanewline
\ \ \isacommand{have}\isamarkupfalse%
\ {\isachardoublequoteopen}{\isasymforall}S\ {\isasymin}\ C{\isachardot}\ {\isasymbottom}\ {\isasymnotin}\ S\isanewline
\ \ {\isasymand}\ {\isacharparenleft}{\isasymforall}k{\isachardot}\ Atom\ k\ {\isasymin}\ S\ {\isasymlongrightarrow}\ \isactrlbold {\isasymnot}\ {\isacharparenleft}Atom\ k{\isacharparenright}\ {\isasymin}\ S\ {\isasymlongrightarrow}\ False{\isacharparenright}\isanewline
\ \ {\isasymand}\ {\isacharparenleft}{\isasymforall}F\ G\ H{\isachardot}\ Con\ F\ G\ H\ {\isasymlongrightarrow}\ F\ {\isasymin}\ S\ {\isasymlongrightarrow}\ {\isacharbraceleft}G{\isacharcomma}H{\isacharbraceright}\ {\isasymunion}\ S\ {\isasymin}\ C{\isacharparenright}\isanewline
\ \ {\isasymand}\ {\isacharparenleft}{\isasymforall}F\ G\ H{\isachardot}\ Dis\ F\ G\ H\ {\isasymlongrightarrow}\ F\ {\isasymin}\ S\ {\isasymlongrightarrow}\ {\isacharbraceleft}G{\isacharbraceright}\ {\isasymunion}\ S\ {\isasymin}\ C\ {\isasymor}\ {\isacharbraceleft}H{\isacharbraceright}\ {\isasymunion}\ S\ {\isasymin}\ C{\isacharparenright}{\isachardoublequoteclose}\isanewline
\ \ \ \ \isacommand{using}\isamarkupfalse%
\ assms{\isacharparenleft}{\isadigit{1}}{\isacharparenright}\ \isacommand{by}\isamarkupfalse%
\ {\isacharparenleft}rule\ pcp{\isacharunderscore}alt{\isadigit{1}}{\isacharparenright}\isanewline
\ \ \isacommand{then}\isamarkupfalse%
\ \isacommand{have}\isamarkupfalse%
\ {\isachardoublequoteopen}{\isasymbottom}\ {\isasymnotin}\ {\isacharquery}S\isanewline
\ \ \ \ \ \ \ \ {\isasymand}\ {\isacharparenleft}{\isasymforall}k{\isachardot}\ Atom\ k\ {\isasymin}\ {\isacharquery}S\ {\isasymlongrightarrow}\ \isactrlbold {\isasymnot}\ {\isacharparenleft}Atom\ k{\isacharparenright}\ {\isasymin}\ {\isacharquery}S\ {\isasymlongrightarrow}\ False{\isacharparenright}\isanewline
\ \ \ \ \ \ \ \ {\isasymand}\ {\isacharparenleft}{\isasymforall}F\ G\ H{\isachardot}\ Con\ F\ G\ H\ {\isasymlongrightarrow}\ F\ {\isasymin}\ {\isacharquery}S\ {\isasymlongrightarrow}\ {\isacharbraceleft}G{\isacharcomma}H{\isacharbraceright}\ {\isasymunion}\ {\isacharquery}S\ {\isasymin}\ C{\isacharparenright}\isanewline
\ \ \ \ \ \ \ \ {\isasymand}\ {\isacharparenleft}{\isasymforall}F\ G\ H{\isachardot}\ Dis\ F\ G\ H\ {\isasymlongrightarrow}\ F\ {\isasymin}\ {\isacharquery}S\ {\isasymlongrightarrow}\ {\isacharbraceleft}G{\isacharbraceright}\ {\isasymunion}\ {\isacharquery}S\ {\isasymin}\ C\ {\isasymor}\ {\isacharbraceleft}H{\isacharbraceright}\ {\isasymunion}\ {\isacharquery}S\ {\isasymin}\ C{\isacharparenright}{\isachardoublequoteclose}\isanewline
\ \ \ \ \isacommand{using}\isamarkupfalse%
\ {\isacartoucheopen}{\isacharquery}S\ {\isasymin}\ C{\isacartoucheclose}\ \isacommand{by}\isamarkupfalse%
\ {\isacharparenleft}rule\ bspec{\isacharparenright}\isanewline
\ \ \isacommand{then}\isamarkupfalse%
\ \isacommand{have}\isamarkupfalse%
\ {\isachardoublequoteopen}{\isasymforall}F\ G\ H{\isachardot}\ Dis\ F\ G\ H\ {\isasymlongrightarrow}\ F\ {\isasymin}\ {\isacharquery}S\ {\isasymlongrightarrow}\ {\isacharbraceleft}G{\isacharbraceright}\ {\isasymunion}\ {\isacharquery}S\ {\isasymin}\ C\ {\isasymor}\ {\isacharbraceleft}H{\isacharbraceright}\ {\isasymunion}\ {\isacharquery}S\ {\isasymin}\ C{\isachardoublequoteclose}\isanewline
\ \ \ \ \isacommand{by}\isamarkupfalse%
\ {\isacharparenleft}iprover\ elim{\isacharcolon}\ conjunct{\isadigit{2}}{\isacharparenright}\isanewline
\ \ \isacommand{then}\isamarkupfalse%
\ \isacommand{have}\isamarkupfalse%
\ {\isachardoublequoteopen}Dis\ F\ G\ H\ {\isasymlongrightarrow}\ F\ {\isasymin}\ {\isacharquery}S\ {\isasymlongrightarrow}\ {\isacharbraceleft}G{\isacharbraceright}\ {\isasymunion}\ {\isacharquery}S\ {\isasymin}\ C\ {\isasymor}\ {\isacharbraceleft}H{\isacharbraceright}\ {\isasymunion}\ {\isacharquery}S\ {\isasymin}\ C{\isachardoublequoteclose}\isanewline
\ \ \ \ \isacommand{by}\isamarkupfalse%
\ {\isacharparenleft}iprover\ elim{\isacharcolon}\ allE{\isacharparenright}\isanewline
\ \ \isacommand{then}\isamarkupfalse%
\ \isacommand{have}\isamarkupfalse%
\ {\isachardoublequoteopen}F\ {\isasymin}\ {\isacharquery}S\ {\isasymlongrightarrow}\ {\isacharbraceleft}G{\isacharbraceright}\ {\isasymunion}\ {\isacharquery}S\ {\isasymin}\ C\ {\isasymor}\ {\isacharbraceleft}H{\isacharbraceright}\ {\isasymunion}\ {\isacharquery}S\ {\isasymin}\ C{\isachardoublequoteclose}\isanewline
\ \ \ \ \isacommand{using}\isamarkupfalse%
\ assms{\isacharparenleft}{\isadigit{4}}{\isacharparenright}\ \isacommand{by}\isamarkupfalse%
\ {\isacharparenleft}rule\ mp{\isacharparenright}\isanewline
\ \ \isacommand{then}\isamarkupfalse%
\ \isacommand{have}\isamarkupfalse%
\ insIsUn{\isacharcolon}{\isachardoublequoteopen}{\isacharbraceleft}G{\isacharbraceright}\ {\isasymunion}\ {\isacharquery}S\ {\isasymin}\ C\ {\isasymor}\ {\isacharbraceleft}H{\isacharbraceright}\ {\isasymunion}\ {\isacharquery}S\ {\isasymin}\ C{\isachardoublequoteclose}\isanewline
\ \ \ \ \isacommand{using}\isamarkupfalse%
\ {\isacartoucheopen}F\ {\isasymin}\ {\isacharquery}S{\isacartoucheclose}\ \isacommand{by}\isamarkupfalse%
\ {\isacharparenleft}rule\ mp{\isacharparenright}\isanewline
\ \ \isacommand{have}\isamarkupfalse%
\ insG{\isacharcolon}{\isachardoublequoteopen}insert\ G\ {\isacharquery}S\ {\isacharequal}\ {\isacharbraceleft}G{\isacharbraceright}\ {\isasymunion}\ {\isacharquery}S{\isachardoublequoteclose}\ \isanewline
\ \ \ \ \isacommand{by}\isamarkupfalse%
\ {\isacharparenleft}rule\ insert{\isacharunderscore}is{\isacharunderscore}Un{\isacharparenright}\isanewline
\ \ \isacommand{have}\isamarkupfalse%
\ insH{\isacharcolon}{\isachardoublequoteopen}insert\ H\ {\isacharquery}S\ {\isacharequal}\ {\isacharbraceleft}H{\isacharbraceright}\ {\isasymunion}\ {\isacharquery}S{\isachardoublequoteclose}\isanewline
\ \ \ \ \isacommand{by}\isamarkupfalse%
\ {\isacharparenleft}rule\ insert{\isacharunderscore}is{\isacharunderscore}Un{\isacharparenright}\isanewline
\ \ \isacommand{have}\isamarkupfalse%
\ {\isachardoublequoteopen}insert\ G\ {\isacharquery}S\ {\isasymin}\ C\ {\isasymor}\ insert\ H\ {\isacharquery}S\ {\isasymin}\ C{\isachardoublequoteclose}\isanewline
\ \ \ \ \isacommand{using}\isamarkupfalse%
\ insG\ insH\ \isacommand{by}\isamarkupfalse%
\ {\isacharparenleft}simp\ only{\isacharcolon}\ insIsUn{\isacharparenright}\isanewline
\ \ \isacommand{then}\isamarkupfalse%
\ \isacommand{have}\isamarkupfalse%
\ {\isachardoublequoteopen}{\isacharparenleft}insert\ G\ {\isacharquery}S\ {\isasymin}\ C\ {\isasymor}\ insert\ H\ {\isacharquery}S\ {\isasymin}\ C{\isacharparenright}\ {\isasymor}\ {\isacharparenleft}{\isasymexists}I\ {\isasymin}\ {\isacharbraceleft}{\isacharbraceright}{\isachardot}\ insert\ I\ {\isacharquery}S\ {\isasymin}\ C{\isacharparenright}{\isachardoublequoteclose}\isanewline
\ \ \ \ \isacommand{by}\isamarkupfalse%
\ {\isacharparenleft}simp\ only{\isacharcolon}\ disjI{\isadigit{1}}{\isacharparenright}\isanewline
\ \ \isacommand{then}\isamarkupfalse%
\ \isacommand{have}\isamarkupfalse%
\ {\isachardoublequoteopen}insert\ G\ {\isacharquery}S\ {\isasymin}\ C\ {\isasymor}\ {\isacharparenleft}insert\ H\ {\isacharquery}S\ {\isasymin}\ C\ {\isasymor}\ {\isacharparenleft}{\isasymexists}I\ {\isasymin}\ {\isacharbraceleft}{\isacharbraceright}{\isachardot}\ insert\ I\ {\isacharquery}S\ {\isasymin}\ C{\isacharparenright}{\isacharparenright}{\isachardoublequoteclose}\isanewline
\ \ \ \ \isacommand{by}\isamarkupfalse%
\ {\isacharparenleft}simp\ only{\isacharcolon}\ disj{\isacharunderscore}assoc{\isacharparenright}\isanewline
\ \ \isacommand{then}\isamarkupfalse%
\ \isacommand{have}\isamarkupfalse%
\ {\isachardoublequoteopen}insert\ G\ {\isacharquery}S\ {\isasymin}\ C\ {\isasymor}\ {\isacharparenleft}{\isasymexists}I\ {\isasymin}\ {\isacharbraceleft}H{\isacharbraceright}{\isachardot}\ insert\ I\ {\isacharquery}S\ {\isasymin}\ C{\isacharparenright}{\isachardoublequoteclose}\isanewline
\ \ \ \ \isacommand{by}\isamarkupfalse%
\ {\isacharparenleft}simp\ only{\isacharcolon}\ bex{\isacharunderscore}simps{\isacharparenleft}{\isadigit{5}}{\isacharparenright}{\isacharparenright}\isanewline
\ \ \isacommand{then}\isamarkupfalse%
\ \isacommand{have}\isamarkupfalse%
\ {\isadigit{1}}{\isacharcolon}{\isachardoublequoteopen}{\isasymexists}I\ {\isasymin}\ {\isacharbraceleft}G{\isacharcomma}H{\isacharbraceright}{\isachardot}\ insert\ I\ {\isacharquery}S\ {\isasymin}\ C{\isachardoublequoteclose}\ \isanewline
\ \ \ \ \isacommand{by}\isamarkupfalse%
\ {\isacharparenleft}simp\ only{\isacharcolon}\ bex{\isacharunderscore}simps{\isacharparenleft}{\isadigit{5}}{\isacharparenright}{\isacharparenright}\isanewline
\ \ \isacommand{obtain}\isamarkupfalse%
\ I\ \isakeyword{where}\ {\isachardoublequoteopen}I\ {\isasymin}\ {\isacharbraceleft}G{\isacharcomma}H{\isacharbraceright}{\isachardoublequoteclose}\ \isakeyword{and}\ {\isachardoublequoteopen}insert\ I\ {\isacharquery}S\ {\isasymin}\ C{\isachardoublequoteclose}\isanewline
\ \ \ \ \isacommand{using}\isamarkupfalse%
\ {\isadigit{1}}\ \isacommand{by}\isamarkupfalse%
\ {\isacharparenleft}rule\ bexE{\isacharparenright}\isanewline
\ \ \isacommand{have}\isamarkupfalse%
\ SC{\isacharcolon}{\isachardoublequoteopen}{\isasymforall}S\ {\isasymin}\ C{\isachardot}\ {\isasymforall}S{\isacharprime}{\isasymsubseteq}S{\isachardot}\ S{\isacharprime}\ {\isasymin}\ C{\isachardoublequoteclose}\isanewline
\ \ \ \ \isacommand{using}\isamarkupfalse%
\ assms{\isacharparenleft}{\isadigit{2}}{\isacharparenright}\ \isacommand{by}\isamarkupfalse%
\ {\isacharparenleft}simp\ only{\isacharcolon}\ subset{\isacharunderscore}closed{\isacharunderscore}def{\isacharparenright}\isanewline
\ \ \isacommand{then}\isamarkupfalse%
\ \isacommand{have}\isamarkupfalse%
\ {\isadigit{2}}{\isacharcolon}{\isachardoublequoteopen}{\isasymforall}S{\isacharprime}\ {\isasymsubseteq}\ {\isacharparenleft}insert\ I\ {\isacharquery}S{\isacharparenright}{\isachardot}\ S{\isacharprime}\ {\isasymin}\ C{\isachardoublequoteclose}\isanewline
\ \ \ \ \isacommand{using}\isamarkupfalse%
\ {\isacartoucheopen}insert\ I\ {\isacharquery}S\ {\isasymin}\ C{\isacartoucheclose}\ \isacommand{by}\isamarkupfalse%
\ {\isacharparenleft}rule\ bspec{\isacharparenright}\isanewline
\ \ \isacommand{have}\isamarkupfalse%
\ {\isachardoublequoteopen}insert\ I\ S{\isadigit{1}}\ {\isasymsubseteq}\ insert\ I\ {\isacharquery}S{\isachardoublequoteclose}\ \isanewline
\ \ \ \ \isacommand{using}\isamarkupfalse%
\ {\isacartoucheopen}S{\isadigit{1}}\ {\isasymsubseteq}\ {\isacharquery}S{\isacartoucheclose}\ \isacommand{by}\isamarkupfalse%
\ {\isacharparenleft}rule\ insert{\isacharunderscore}mono{\isacharparenright}\isanewline
\ \ \isacommand{have}\isamarkupfalse%
\ {\isachardoublequoteopen}insert\ I\ S{\isadigit{1}}\ {\isasymin}\ C{\isachardoublequoteclose}\isanewline
\ \ \ \ \isacommand{using}\isamarkupfalse%
\ {\isadigit{2}}\ {\isacartoucheopen}insert\ I\ S{\isadigit{1}}\ {\isasymsubseteq}\ insert\ I\ {\isacharquery}S{\isacartoucheclose}\ \isacommand{by}\isamarkupfalse%
\ {\isacharparenleft}rule\ sspec{\isacharparenright}\isanewline
\ \ \isacommand{have}\isamarkupfalse%
\ {\isachardoublequoteopen}insert\ I\ S{\isadigit{2}}\ {\isasymsubseteq}\ insert\ I\ {\isacharquery}S{\isachardoublequoteclose}\isanewline
\ \ \ \ \isacommand{using}\isamarkupfalse%
\ {\isacartoucheopen}S{\isadigit{2}}\ {\isasymsubseteq}\ {\isacharquery}S{\isacartoucheclose}\ \isacommand{by}\isamarkupfalse%
\ {\isacharparenleft}rule\ insert{\isacharunderscore}mono{\isacharparenright}\isanewline
\ \ \isacommand{have}\isamarkupfalse%
\ {\isachardoublequoteopen}insert\ I\ S{\isadigit{2}}\ {\isasymin}\ C{\isachardoublequoteclose}\isanewline
\ \ \ \ \isacommand{using}\isamarkupfalse%
\ {\isadigit{2}}\ {\isacartoucheopen}insert\ I\ S{\isadigit{2}}\ {\isasymsubseteq}\ insert\ I\ {\isacharquery}S{\isacartoucheclose}\ \isacommand{by}\isamarkupfalse%
\ {\isacharparenleft}rule\ sspec{\isacharparenright}\isanewline
\ \ \isacommand{have}\isamarkupfalse%
\ {\isachardoublequoteopen}insert\ I\ S{\isadigit{1}}\ {\isasymin}\ C\ {\isasymand}\ insert\ I\ S{\isadigit{2}}\ {\isasymin}\ C{\isachardoublequoteclose}\isanewline
\ \ \ \ \isacommand{using}\isamarkupfalse%
\ {\isacartoucheopen}insert\ I\ S{\isadigit{1}}\ {\isasymin}\ C{\isacartoucheclose}\ {\isacartoucheopen}insert\ I\ S{\isadigit{2}}\ {\isasymin}\ C{\isacartoucheclose}\ \isacommand{by}\isamarkupfalse%
\ {\isacharparenleft}rule\ conjI{\isacharparenright}\isanewline
\ \ \isacommand{thus}\isamarkupfalse%
\ {\isachardoublequoteopen}{\isasymexists}I{\isasymin}{\isacharbraceleft}G{\isacharcomma}H{\isacharbraceright}{\isachardot}\ insert\ I\ S{\isadigit{1}}\ {\isasymin}\ C\ {\isasymand}\ insert\ I\ S{\isadigit{2}}\ {\isasymin}\ C{\isachardoublequoteclose}\isanewline
\ \ \ \ \isacommand{using}\isamarkupfalse%
\ {\isacartoucheopen}I\ {\isasymin}\ {\isacharbraceleft}G{\isacharcomma}H{\isacharbraceright}{\isacartoucheclose}\ \isacommand{by}\isamarkupfalse%
\ {\isacharparenleft}rule\ bexI{\isacharparenright}\isanewline
\isacommand{qed}\isamarkupfalse%
%
\endisatagproof
{\isafoldproof}%
%
\isadelimproof
%
\endisadelimproof
%
\begin{isamarkuptext}%
Finalmente, el lema \isa{ex{\isadigit{3}}{\isacharunderscore}pcp{\isacharunderscore}SinE{\isacharunderscore}DIS{\isacharunderscore}auxFalse} prueba que dada una colección \isa{C} con la 
  propiedad de consistencia proposicional y cerrada bajo subconjuntos, \isa{S\ {\isasymin}\ E} y sea \isa{F} es una 
  fórmula de tipo \isa{{\isasymbeta}} de componentes \isa{{\isasymbeta}\isactrlsub {\isadigit{1}}} y \isa{{\isasymbeta}\isactrlsub {\isadigit{2}}}, si consideramos \isa{S\isactrlsub {\isadigit{1}}} y \isa{S\isactrlsub {\isadigit{2}}} subconjuntos finitos 
  cualesquiera de \isa{S} tales que \isa{F\ {\isasymin}\ S\isactrlsub {\isadigit{1}}}, \isa{F\ {\isasymin}\ S\isactrlsub {\isadigit{2}}}, \isa{{\isacharbraceleft}{\isasymbeta}\isactrlsub {\isadigit{1}}{\isacharbraceright}\ {\isasymunion}\ S\isactrlsub {\isadigit{1}}\ {\isasymnotin}\ C} y \isa{{\isacharbraceleft}{\isasymbeta}\isactrlsub {\isadigit{2}}{\isacharbraceright}\ {\isasymunion}\ S\isactrlsub {\isadigit{2}}\ {\isasymnotin}\ C}, llegamos a 
  una contradicción.%
\end{isamarkuptext}\isamarkuptrue%
\isacommand{lemma}\isamarkupfalse%
\ ex{\isadigit{3}}{\isacharunderscore}pcp{\isacharunderscore}SinE{\isacharunderscore}DIS{\isacharunderscore}auxFalse{\isacharcolon}\isanewline
\ \ \isakeyword{assumes}\ {\isachardoublequoteopen}pcp\ C{\isachardoublequoteclose}\ \isanewline
\ \ \ \ \ \ \ \ \ \ {\isachardoublequoteopen}subset{\isacharunderscore}closed\ C{\isachardoublequoteclose}\isanewline
\ \ \ \ \ \ \ \ \ \ {\isachardoublequoteopen}S\ {\isasymin}\ {\isacharparenleft}extF\ C{\isacharparenright}{\isachardoublequoteclose}\isanewline
\ \ \ \ \ \ \ \ \ \ {\isachardoublequoteopen}Dis\ F\ G\ H{\isachardoublequoteclose}\isanewline
\ \ \ \ \ \ \ \ \ \ {\isachardoublequoteopen}F\ {\isasymin}\ S{\isachardoublequoteclose}\isanewline
\ \ \ \ \ \ \ \ \ \ {\isachardoublequoteopen}S{\isadigit{1}}\ {\isasymsubseteq}\ S{\isachardoublequoteclose}\ \isanewline
\ \ \ \ \ \ \ \ \ \ {\isachardoublequoteopen}finite\ S{\isadigit{1}}{\isachardoublequoteclose}\ \isanewline
\ \ \ \ \ \ \ \ \ \ {\isachardoublequoteopen}insert\ G\ S{\isadigit{1}}\ {\isasymnotin}\ C{\isachardoublequoteclose}\ \isanewline
\ \ \ \ \ \ \ \ \ \ {\isachardoublequoteopen}S{\isadigit{2}}\ {\isasymsubseteq}\ S{\isachardoublequoteclose}\ \isanewline
\ \ \ \ \ \ \ \ \ \ {\isachardoublequoteopen}finite\ S{\isadigit{2}}{\isachardoublequoteclose}\ \isanewline
\ \ \ \ \ \ \ \ \ \ {\isachardoublequoteopen}insert\ H\ S{\isadigit{2}}\ {\isasymnotin}\ C{\isachardoublequoteclose}\isanewline
\ \ \ \ \ \ \ \ \isakeyword{shows}\ {\isachardoublequoteopen}False{\isachardoublequoteclose}\isanewline
%
\isadelimproof
%
\endisadelimproof
%
\isatagproof
\isacommand{proof}\isamarkupfalse%
\ {\isacharminus}\isanewline
\ \ \isacommand{let}\isamarkupfalse%
\ {\isacharquery}S{\isadigit{1}}{\isacharequal}{\isachardoublequoteopen}insert\ F\ S{\isadigit{1}}{\isachardoublequoteclose}\isanewline
\ \ \isacommand{let}\isamarkupfalse%
\ {\isacharquery}S{\isadigit{2}}{\isacharequal}{\isachardoublequoteopen}insert\ F\ S{\isadigit{2}}{\isachardoublequoteclose}\isanewline
\ \ \isacommand{have}\isamarkupfalse%
\ SC{\isacharcolon}{\isachardoublequoteopen}{\isasymforall}S\ {\isasymin}\ C{\isachardot}\ {\isasymforall}S{\isacharprime}{\isasymsubseteq}S{\isachardot}\ S{\isacharprime}\ {\isasymin}\ C{\isachardoublequoteclose}\isanewline
\ \ \ \ \isacommand{using}\isamarkupfalse%
\ assms{\isacharparenleft}{\isadigit{2}}{\isacharparenright}\ \isacommand{by}\isamarkupfalse%
\ {\isacharparenleft}simp\ only{\isacharcolon}\ subset{\isacharunderscore}closed{\isacharunderscore}def{\isacharparenright}\isanewline
\ \ \isacommand{have}\isamarkupfalse%
\ {\isadigit{1}}{\isacharcolon}{\isachardoublequoteopen}{\isacharquery}S{\isadigit{1}}\ {\isasymsubseteq}\ S{\isachardoublequoteclose}\isanewline
\ \ \ \ \isacommand{using}\isamarkupfalse%
\ {\isacartoucheopen}F\ {\isasymin}\ S{\isacartoucheclose}\ {\isacartoucheopen}S{\isadigit{1}}\ {\isasymsubseteq}\ S{\isacartoucheclose}\ \isacommand{by}\isamarkupfalse%
\ {\isacharparenleft}simp\ only{\isacharcolon}\ insert{\isacharunderscore}subset{\isacharparenright}\ \isanewline
\ \ \isacommand{have}\isamarkupfalse%
\ {\isadigit{2}}{\isacharcolon}{\isachardoublequoteopen}finite\ {\isacharquery}S{\isadigit{1}}{\isachardoublequoteclose}\isanewline
\ \ \ \ \isacommand{using}\isamarkupfalse%
\ {\isacartoucheopen}finite\ S{\isadigit{1}}{\isacartoucheclose}\ \isacommand{by}\isamarkupfalse%
\ {\isacharparenleft}simp\ only{\isacharcolon}\ finite{\isacharunderscore}insert{\isacharparenright}\ \isanewline
\ \ \isacommand{have}\isamarkupfalse%
\ {\isadigit{3}}{\isacharcolon}{\isachardoublequoteopen}F\ {\isasymin}\ {\isacharquery}S{\isadigit{1}}{\isachardoublequoteclose}\isanewline
\ \ \ \ \isacommand{by}\isamarkupfalse%
\ {\isacharparenleft}simp\ only{\isacharcolon}\ insertI{\isadigit{1}}{\isacharparenright}\ \isanewline
\ \ \isacommand{have}\isamarkupfalse%
\ {\isadigit{4}}{\isacharcolon}{\isachardoublequoteopen}insert\ G\ {\isacharquery}S{\isadigit{1}}\ {\isasymnotin}\ C{\isachardoublequoteclose}\ \isanewline
\ \ \isacommand{proof}\isamarkupfalse%
\ {\isacharparenleft}rule\ ccontr{\isacharparenright}\isanewline
\ \ \ \ \isacommand{assume}\isamarkupfalse%
\ {\isachardoublequoteopen}{\isasymnot}{\isacharparenleft}insert\ G\ {\isacharquery}S{\isadigit{1}}\ {\isasymnotin}\ C{\isacharparenright}{\isachardoublequoteclose}\isanewline
\ \ \ \ \isacommand{then}\isamarkupfalse%
\ \isacommand{have}\isamarkupfalse%
\ {\isachardoublequoteopen}insert\ G\ {\isacharquery}S{\isadigit{1}}\ {\isasymin}\ C{\isachardoublequoteclose}\isanewline
\ \ \ \ \ \ \isacommand{by}\isamarkupfalse%
\ {\isacharparenleft}rule\ notnotD{\isacharparenright}\isanewline
\ \ \ \ \isacommand{have}\isamarkupfalse%
\ SC{\isadigit{1}}{\isacharcolon}{\isachardoublequoteopen}{\isasymforall}S{\isacharprime}\ {\isasymsubseteq}\ {\isacharparenleft}insert\ G\ {\isacharquery}S{\isadigit{1}}{\isacharparenright}{\isachardot}\ S{\isacharprime}\ {\isasymin}\ C{\isachardoublequoteclose}\isanewline
\ \ \ \ \ \ \isacommand{using}\isamarkupfalse%
\ SC\ {\isacartoucheopen}insert\ G\ {\isacharquery}S{\isadigit{1}}\ {\isasymin}\ C{\isacartoucheclose}\ \isacommand{by}\isamarkupfalse%
\ {\isacharparenleft}rule\ bspec{\isacharparenright}\isanewline
\ \ \ \ \isacommand{have}\isamarkupfalse%
\ {\isachardoublequoteopen}insert\ G\ S{\isadigit{1}}\ {\isasymsubseteq}\ insert\ F\ {\isacharparenleft}insert\ G\ S{\isadigit{1}}{\isacharparenright}{\isachardoublequoteclose}\isanewline
\ \ \ \ \ \ \isacommand{by}\isamarkupfalse%
\ {\isacharparenleft}rule\ subset{\isacharunderscore}insertI{\isacharparenright}\isanewline
\ \ \ \ \isacommand{then}\isamarkupfalse%
\ \isacommand{have}\isamarkupfalse%
\ {\isachardoublequoteopen}insert\ G\ S{\isadigit{1}}\ {\isasymsubseteq}\ insert\ G\ {\isacharquery}S{\isadigit{1}}{\isachardoublequoteclose}\isanewline
\ \ \ \ \ \ \isacommand{by}\isamarkupfalse%
\ {\isacharparenleft}simp\ only{\isacharcolon}\ insert{\isacharunderscore}commute{\isacharparenright}\isanewline
\ \ \ \ \isacommand{have}\isamarkupfalse%
\ {\isachardoublequoteopen}insert\ G\ S{\isadigit{1}}\ {\isasymin}\ C{\isachardoublequoteclose}\isanewline
\ \ \ \ \ \ \isacommand{using}\isamarkupfalse%
\ SC{\isadigit{1}}\ {\isacartoucheopen}insert\ G\ S{\isadigit{1}}\ {\isasymsubseteq}\ insert\ G\ {\isacharquery}S{\isadigit{1}}{\isacartoucheclose}\ \isacommand{by}\isamarkupfalse%
\ {\isacharparenleft}rule\ sspec{\isacharparenright}\isanewline
\ \ \ \ \isacommand{show}\isamarkupfalse%
\ {\isachardoublequoteopen}False{\isachardoublequoteclose}\isanewline
\ \ \ \ \ \ \isacommand{using}\isamarkupfalse%
\ assms{\isacharparenleft}{\isadigit{8}}{\isacharparenright}\ {\isacartoucheopen}insert\ G\ S{\isadigit{1}}\ {\isasymin}\ C{\isacartoucheclose}\ \isacommand{by}\isamarkupfalse%
\ {\isacharparenleft}rule\ notE{\isacharparenright}\isanewline
\ \ \isacommand{qed}\isamarkupfalse%
\isanewline
\ \ \isacommand{have}\isamarkupfalse%
\ {\isadigit{5}}{\isacharcolon}{\isachardoublequoteopen}{\isacharquery}S{\isadigit{2}}\ {\isasymsubseteq}\ S{\isachardoublequoteclose}\isanewline
\ \ \ \ \isacommand{using}\isamarkupfalse%
\ {\isacartoucheopen}F\ {\isasymin}\ S{\isacartoucheclose}\ {\isacartoucheopen}S{\isadigit{2}}\ {\isasymsubseteq}\ S{\isacartoucheclose}\ \isacommand{by}\isamarkupfalse%
\ {\isacharparenleft}simp\ only{\isacharcolon}\ insert{\isacharunderscore}subset{\isacharparenright}\isanewline
\ \ \isacommand{have}\isamarkupfalse%
\ {\isadigit{6}}{\isacharcolon}{\isachardoublequoteopen}finite\ {\isacharquery}S{\isadigit{2}}{\isachardoublequoteclose}\isanewline
\ \ \ \ \isacommand{using}\isamarkupfalse%
\ {\isacartoucheopen}finite\ S{\isadigit{2}}{\isacartoucheclose}\ \isacommand{by}\isamarkupfalse%
\ {\isacharparenleft}simp\ only{\isacharcolon}\ finite{\isacharunderscore}insert{\isacharparenright}\isanewline
\ \ \isacommand{have}\isamarkupfalse%
\ {\isadigit{7}}{\isacharcolon}{\isachardoublequoteopen}F\ {\isasymin}\ {\isacharquery}S{\isadigit{2}}{\isachardoublequoteclose}\isanewline
\ \ \ \ \isacommand{by}\isamarkupfalse%
\ {\isacharparenleft}simp\ only{\isacharcolon}\ insertI{\isadigit{1}}{\isacharparenright}\isanewline
\ \ \isacommand{have}\isamarkupfalse%
\ {\isadigit{8}}{\isacharcolon}{\isachardoublequoteopen}insert\ H\ {\isacharquery}S{\isadigit{2}}\ {\isasymnotin}\ C{\isachardoublequoteclose}\ \isanewline
\ \ \isacommand{proof}\isamarkupfalse%
\ {\isacharparenleft}rule\ ccontr{\isacharparenright}\isanewline
\ \ \ \ \isacommand{assume}\isamarkupfalse%
\ {\isachardoublequoteopen}{\isasymnot}{\isacharparenleft}insert\ H\ {\isacharquery}S{\isadigit{2}}\ {\isasymnotin}\ C{\isacharparenright}{\isachardoublequoteclose}\isanewline
\ \ \ \ \isacommand{then}\isamarkupfalse%
\ \isacommand{have}\isamarkupfalse%
\ {\isachardoublequoteopen}insert\ H\ {\isacharquery}S{\isadigit{2}}\ {\isasymin}\ C{\isachardoublequoteclose}\isanewline
\ \ \ \ \ \ \isacommand{by}\isamarkupfalse%
\ {\isacharparenleft}rule\ notnotD{\isacharparenright}\isanewline
\ \ \ \ \isacommand{have}\isamarkupfalse%
\ SC{\isadigit{2}}{\isacharcolon}{\isachardoublequoteopen}{\isasymforall}S{\isacharprime}\ {\isasymsubseteq}\ {\isacharparenleft}insert\ H\ {\isacharquery}S{\isadigit{2}}{\isacharparenright}{\isachardot}\ S{\isacharprime}\ {\isasymin}\ C{\isachardoublequoteclose}\isanewline
\ \ \ \ \ \ \isacommand{using}\isamarkupfalse%
\ SC\ {\isacartoucheopen}insert\ H\ {\isacharquery}S{\isadigit{2}}\ {\isasymin}\ C{\isacartoucheclose}\ \isacommand{by}\isamarkupfalse%
\ {\isacharparenleft}rule\ bspec{\isacharparenright}\isanewline
\ \ \ \ \isacommand{have}\isamarkupfalse%
\ {\isachardoublequoteopen}insert\ H\ S{\isadigit{2}}\ {\isasymsubseteq}\ insert\ F\ {\isacharparenleft}insert\ H\ S{\isadigit{2}}{\isacharparenright}{\isachardoublequoteclose}\isanewline
\ \ \ \ \ \ \isacommand{by}\isamarkupfalse%
\ {\isacharparenleft}rule\ subset{\isacharunderscore}insertI{\isacharparenright}\isanewline
\ \ \ \ \isacommand{then}\isamarkupfalse%
\ \isacommand{have}\isamarkupfalse%
\ {\isachardoublequoteopen}insert\ H\ S{\isadigit{2}}\ {\isasymsubseteq}\ insert\ H\ {\isacharquery}S{\isadigit{2}}{\isachardoublequoteclose}\isanewline
\ \ \ \ \ \ \isacommand{by}\isamarkupfalse%
\ {\isacharparenleft}simp\ only{\isacharcolon}\ insert{\isacharunderscore}commute{\isacharparenright}\isanewline
\ \ \ \ \isacommand{have}\isamarkupfalse%
\ {\isachardoublequoteopen}insert\ H\ S{\isadigit{2}}\ {\isasymin}\ C{\isachardoublequoteclose}\isanewline
\ \ \ \ \ \ \isacommand{using}\isamarkupfalse%
\ SC{\isadigit{2}}\ {\isacartoucheopen}insert\ H\ S{\isadigit{2}}\ {\isasymsubseteq}\ insert\ H\ {\isacharquery}S{\isadigit{2}}{\isacartoucheclose}\ \isacommand{by}\isamarkupfalse%
\ {\isacharparenleft}rule\ sspec{\isacharparenright}\isanewline
\ \ \ \ \isacommand{show}\isamarkupfalse%
\ {\isachardoublequoteopen}False{\isachardoublequoteclose}\isanewline
\ \ \ \ \ \ \isacommand{using}\isamarkupfalse%
\ assms{\isacharparenleft}{\isadigit{1}}{\isadigit{1}}{\isacharparenright}\ {\isacartoucheopen}insert\ H\ S{\isadigit{2}}\ {\isasymin}\ C{\isacartoucheclose}\ \isacommand{by}\isamarkupfalse%
\ {\isacharparenleft}rule\ notE{\isacharparenright}\isanewline
\ \ \isacommand{qed}\isamarkupfalse%
\isanewline
\ \ \isacommand{have}\isamarkupfalse%
\ Ex{\isacharcolon}{\isachardoublequoteopen}{\isasymexists}I\ {\isasymin}\ {\isacharbraceleft}G{\isacharcomma}H{\isacharbraceright}{\isachardot}\ insert\ I\ {\isacharquery}S{\isadigit{1}}\ {\isasymin}\ C\ {\isasymand}\ insert\ I\ {\isacharquery}S{\isadigit{2}}\ {\isasymin}\ C{\isachardoublequoteclose}\isanewline
\ \ \ \ \isacommand{using}\isamarkupfalse%
\ assms{\isacharparenleft}{\isadigit{1}}{\isacharparenright}\ assms{\isacharparenleft}{\isadigit{2}}{\isacharparenright}\ assms{\isacharparenleft}{\isadigit{3}}{\isacharparenright}\ assms{\isacharparenleft}{\isadigit{4}}{\isacharparenright}\ {\isadigit{1}}\ {\isadigit{2}}\ {\isadigit{3}}\ {\isadigit{5}}\ {\isadigit{6}}\ {\isadigit{7}}\ \isacommand{by}\isamarkupfalse%
\ {\isacharparenleft}rule\ ex{\isadigit{3}}{\isacharunderscore}pcp{\isacharunderscore}SinE{\isacharunderscore}DIS{\isacharunderscore}auxEx{\isacharparenright}\isanewline
\ \ \isacommand{have}\isamarkupfalse%
\ {\isachardoublequoteopen}{\isasymforall}I\ {\isasymin}\ {\isacharbraceleft}G{\isacharcomma}H{\isacharbraceright}{\isachardot}\ insert\ I\ {\isacharquery}S{\isadigit{1}}\ {\isasymnotin}\ C\ {\isasymor}\ insert\ I\ {\isacharquery}S{\isadigit{2}}\ {\isasymnotin}\ C{\isachardoublequoteclose}\isanewline
\ \ \ \ \isacommand{using}\isamarkupfalse%
\ {\isadigit{4}}\ {\isadigit{8}}\ \isacommand{by}\isamarkupfalse%
\ simp\isanewline
\ \ \isacommand{then}\isamarkupfalse%
\ \isacommand{have}\isamarkupfalse%
\ {\isachardoublequoteopen}{\isasymforall}I\ {\isasymin}\ {\isacharbraceleft}G{\isacharcomma}H{\isacharbraceright}{\isachardot}\ {\isasymnot}{\isacharparenleft}insert\ I\ {\isacharquery}S{\isadigit{1}}\ {\isasymin}\ C\ {\isasymand}\ insert\ I\ {\isacharquery}S{\isadigit{2}}\ {\isasymin}\ C{\isacharparenright}{\isachardoublequoteclose}\isanewline
\ \ \ \ \isacommand{by}\isamarkupfalse%
\ {\isacharparenleft}simp\ only{\isacharcolon}\ de{\isacharunderscore}Morgan{\isacharunderscore}conj{\isacharparenright}\isanewline
\ \ \isacommand{then}\isamarkupfalse%
\ \isacommand{have}\isamarkupfalse%
\ {\isachardoublequoteopen}{\isasymnot}{\isacharparenleft}{\isasymexists}I\ {\isasymin}\ {\isacharbraceleft}G{\isacharcomma}H{\isacharbraceright}{\isachardot}\ insert\ I\ {\isacharquery}S{\isadigit{1}}\ {\isasymin}\ C\ {\isasymand}\ insert\ I\ {\isacharquery}S{\isadigit{2}}\ {\isasymin}\ C{\isacharparenright}{\isachardoublequoteclose}\isanewline
\ \ \ \ \isacommand{by}\isamarkupfalse%
\ {\isacharparenleft}simp\ only{\isacharcolon}\ bex{\isacharunderscore}simps{\isacharparenleft}{\isadigit{8}}{\isacharparenright}{\isacharparenright}\ \isanewline
\ \ \isacommand{thus}\isamarkupfalse%
\ {\isachardoublequoteopen}False{\isachardoublequoteclose}\isanewline
\ \ \ \ \isacommand{using}\isamarkupfalse%
\ Ex\ \isacommand{by}\isamarkupfalse%
\ {\isacharparenleft}rule\ notE{\isacharparenright}\isanewline
\isacommand{qed}\isamarkupfalse%
%
\endisatagproof
{\isafoldproof}%
%
\isadelimproof
%
\endisadelimproof
%
\begin{isamarkuptext}%
Una vez introducidos los lemas anteriores, podemos probar el lema \isa{ex{\isadigit{3}}{\isacharunderscore}pcp{\isacharunderscore}SinE{\isacharunderscore}DIS} que
  demuestra que si \isa{C} es una colección con la propiedad de consistencia proposicional y cerrada 
  bajo subconjuntos, \isa{S\ {\isasymin}\ E} y sea \isa{F} una fórmula de tipo \isa{{\isasymbeta}} con componentes \isa{{\isasymbeta}\isactrlsub {\isadigit{1}}} y \isa{{\isasymbeta}\isactrlsub {\isadigit{2}}}, se 
  verifica que o bien \isa{{\isacharbraceleft}{\isasymbeta}\isactrlsub {\isadigit{1}}{\isacharbraceright}\ {\isasymunion}\ S\ {\isasymin}\ C{\isacharprime}} o bien \isa{{\isacharbraceleft}{\isasymbeta}\isactrlsub {\isadigit{2}}{\isacharbraceright}\ {\isasymunion}\ S\ {\isasymin}\ C{\isacharprime}}. Además, para dicha prueba 
  necesitaremos los siguientes lemas auxiliares en Isabelle.%
\end{isamarkuptext}\isamarkuptrue%
\isacommand{lemma}\isamarkupfalse%
\ sall{\isacharunderscore}simps{\isacharunderscore}not{\isacharunderscore}all{\isacharcolon}\isanewline
\ \ \isakeyword{assumes}\ {\isachardoublequoteopen}{\isasymnot}{\isacharparenleft}{\isasymforall}x\ {\isasymsubseteq}\ A{\isachardot}\ P\ x{\isacharparenright}{\isachardoublequoteclose}\isanewline
\ \ \isakeyword{shows}\ {\isachardoublequoteopen}{\isasymexists}x\ {\isasymsubseteq}\ A{\isachardot}\ {\isacharparenleft}{\isasymnot}\ P\ x{\isacharparenright}{\isachardoublequoteclose}\isanewline
%
\isadelimproof
\ \ %
\endisadelimproof
%
\isatagproof
\isacommand{using}\isamarkupfalse%
\ assms\ \isacommand{by}\isamarkupfalse%
\ blast%
\endisatagproof
{\isafoldproof}%
%
\isadelimproof
\isanewline
%
\endisadelimproof
\isanewline
\isacommand{lemma}\isamarkupfalse%
\ subexE{\isacharcolon}\ {\isachardoublequoteopen}{\isasymexists}x{\isasymsubseteq}A{\isachardot}\ P\ x\ {\isasymLongrightarrow}\ {\isacharparenleft}{\isasymAnd}x{\isachardot}\ x{\isasymsubseteq}A\ {\isasymLongrightarrow}\ P\ x\ {\isasymLongrightarrow}\ Q{\isacharparenright}\ {\isasymLongrightarrow}\ Q{\isachardoublequoteclose}\isanewline
%
\isadelimproof
\ \ %
\endisadelimproof
%
\isatagproof
\isacommand{by}\isamarkupfalse%
\ blast%
\endisatagproof
{\isafoldproof}%
%
\isadelimproof
%
\endisadelimproof
%
\begin{isamarkuptext}%
De este modo, procedamos con la demostración detallada de \isa{ex{\isadigit{3}}{\isacharunderscore}pcp{\isacharunderscore}SinE{\isacharunderscore}DIS}.%
\end{isamarkuptext}\isamarkuptrue%
\isacommand{lemma}\isamarkupfalse%
\ ex{\isadigit{3}}{\isacharunderscore}pcp{\isacharunderscore}SinE{\isacharunderscore}DIS{\isacharcolon}\isanewline
\ \ \isakeyword{assumes}\ {\isachardoublequoteopen}pcp\ C{\isachardoublequoteclose}\isanewline
\ \ \ \ \ \ \ \ \ \ {\isachardoublequoteopen}subset{\isacharunderscore}closed\ C{\isachardoublequoteclose}\isanewline
\ \ \ \ \ \ \ \ \ \ {\isachardoublequoteopen}S\ {\isasymin}\ {\isacharparenleft}extF\ C{\isacharparenright}{\isachardoublequoteclose}\isanewline
\ \ \ \ \ \ \ \ \ \ {\isachardoublequoteopen}Dis\ F\ G\ H{\isachardoublequoteclose}\isanewline
\ \ \ \ \ \ \ \ \ \ {\isachardoublequoteopen}F\ {\isasymin}\ S{\isachardoublequoteclose}\isanewline
\ \ \isakeyword{shows}\ {\isachardoublequoteopen}{\isacharbraceleft}G{\isacharbraceright}\ {\isasymunion}\ S\ {\isasymin}\ {\isacharparenleft}extensionFin\ C{\isacharparenright}\ {\isasymor}\ {\isacharbraceleft}H{\isacharbraceright}\ {\isasymunion}\ S\ {\isasymin}\ {\isacharparenleft}extensionFin\ C{\isacharparenright}{\isachardoublequoteclose}\isanewline
%
\isadelimproof
%
\endisadelimproof
%
\isatagproof
\isacommand{proof}\isamarkupfalse%
\ {\isacharminus}\isanewline
\ \ \isacommand{have}\isamarkupfalse%
\ {\isachardoublequoteopen}{\isacharparenleft}extF\ C{\isacharparenright}\ {\isasymsubseteq}\ {\isacharparenleft}extensionFin\ C{\isacharparenright}{\isachardoublequoteclose}\ \isanewline
\ \ \ \ \isacommand{unfolding}\isamarkupfalse%
\ extensionFin\ \isacommand{by}\isamarkupfalse%
\ {\isacharparenleft}rule\ Un{\isacharunderscore}upper{\isadigit{2}}{\isacharparenright}\ \isanewline
\ \ \isacommand{have}\isamarkupfalse%
\ PCP{\isacharcolon}{\isachardoublequoteopen}{\isasymforall}S\ {\isasymin}\ C{\isachardot}\isanewline
\ \ \ \ \ \ \ \ \ \ \ \ {\isasymbottom}\ {\isasymnotin}\ S\isanewline
\ \ \ \ \ \ \ \ \ \ \ \ {\isasymand}\ {\isacharparenleft}{\isasymforall}k{\isachardot}\ Atom\ k\ {\isasymin}\ S\ {\isasymlongrightarrow}\ \isactrlbold {\isasymnot}\ {\isacharparenleft}Atom\ k{\isacharparenright}\ {\isasymin}\ S\ {\isasymlongrightarrow}\ False{\isacharparenright}\isanewline
\ \ \ \ \ \ \ \ \ \ \ \ {\isasymand}\ {\isacharparenleft}{\isasymforall}F\ G\ H{\isachardot}\ Con\ F\ G\ H\ {\isasymlongrightarrow}\ F\ {\isasymin}\ S\ {\isasymlongrightarrow}\ {\isacharbraceleft}G{\isacharcomma}H{\isacharbraceright}\ {\isasymunion}\ S\ {\isasymin}\ C{\isacharparenright}\isanewline
\ \ \ \ \ \ \ \ \ \ \ \ {\isasymand}\ {\isacharparenleft}{\isasymforall}F\ G\ H{\isachardot}\ Dis\ F\ G\ H\ {\isasymlongrightarrow}\ F\ {\isasymin}\ S\ {\isasymlongrightarrow}\ {\isacharbraceleft}G{\isacharbraceright}\ {\isasymunion}\ S\ {\isasymin}\ C\ {\isasymor}\ {\isacharbraceleft}H{\isacharbraceright}\ {\isasymunion}\ S\ {\isasymin}\ C{\isacharparenright}{\isachardoublequoteclose}\isanewline
\ \ \ \ \isacommand{using}\isamarkupfalse%
\ assms{\isacharparenleft}{\isadigit{1}}{\isacharparenright}\ \isacommand{by}\isamarkupfalse%
\ {\isacharparenleft}rule\ pcp{\isacharunderscore}alt{\isadigit{1}}{\isacharparenright}\isanewline
\ \ \isacommand{have}\isamarkupfalse%
\ E{\isacharcolon}{\isachardoublequoteopen}{\isasymforall}S{\isacharprime}\ {\isasymsubseteq}\ S{\isachardot}\ finite\ S{\isacharprime}\ {\isasymlongrightarrow}\ S{\isacharprime}\ {\isasymin}\ C{\isachardoublequoteclose}\isanewline
\ \ \ \ \isacommand{using}\isamarkupfalse%
\ assms{\isacharparenleft}{\isadigit{3}}{\isacharparenright}\ \isacommand{unfolding}\isamarkupfalse%
\ extF\ \isacommand{by}\isamarkupfalse%
\ {\isacharparenleft}rule\ CollectD{\isacharparenright}\isanewline
\ \ \isacommand{then}\isamarkupfalse%
\ \isacommand{have}\isamarkupfalse%
\ E{\isacharprime}{\isacharcolon}{\isachardoublequoteopen}{\isasymforall}S{\isacharprime}{\isachardot}\ S{\isacharprime}\ {\isasymsubseteq}\ S\ {\isasymlongrightarrow}\ finite\ S{\isacharprime}\ {\isasymlongrightarrow}\ S{\isacharprime}\ {\isasymin}\ C{\isachardoublequoteclose}\isanewline
\ \ \ \ \isacommand{by}\isamarkupfalse%
\ blast\isanewline
\ \ \isacommand{have}\isamarkupfalse%
\ SC{\isacharcolon}{\isachardoublequoteopen}{\isasymforall}S\ {\isasymin}\ C{\isachardot}\ {\isasymforall}S{\isacharprime}{\isasymsubseteq}S{\isachardot}\ S{\isacharprime}\ {\isasymin}\ C{\isachardoublequoteclose}\isanewline
\ \ \ \ \isacommand{using}\isamarkupfalse%
\ assms{\isacharparenleft}{\isadigit{2}}{\isacharparenright}\ \isacommand{by}\isamarkupfalse%
\ {\isacharparenleft}simp\ only{\isacharcolon}\ subset{\isacharunderscore}closed{\isacharunderscore}def{\isacharparenright}\isanewline
\ \ \isacommand{have}\isamarkupfalse%
\ {\isachardoublequoteopen}insert\ G\ S\ {\isasymin}\ {\isacharparenleft}extF\ C{\isacharparenright}\ {\isasymor}\ insert\ H\ S\ {\isasymin}\ {\isacharparenleft}extF\ C{\isacharparenright}{\isachardoublequoteclose}\ \isanewline
\ \ \isacommand{proof}\isamarkupfalse%
\ {\isacharparenleft}rule\ ccontr{\isacharparenright}\isanewline
\ \ \ \ \isacommand{assume}\isamarkupfalse%
\ {\isachardoublequoteopen}{\isasymnot}{\isacharparenleft}insert\ G\ S\ {\isasymin}\ {\isacharparenleft}extF\ C{\isacharparenright}\ {\isasymor}\ insert\ H\ S\ {\isasymin}\ {\isacharparenleft}extF\ C{\isacharparenright}{\isacharparenright}{\isachardoublequoteclose}\ \ \isanewline
\ \ \ \ \isacommand{then}\isamarkupfalse%
\ \isacommand{have}\isamarkupfalse%
\ Conj{\isacharcolon}{\isachardoublequoteopen}{\isasymnot}{\isacharparenleft}insert\ G\ S\ {\isasymin}\ {\isacharparenleft}extF\ C{\isacharparenright}{\isacharparenright}\ {\isasymand}\ {\isasymnot}{\isacharparenleft}insert\ H\ S\ {\isasymin}\ {\isacharparenleft}extF\ C{\isacharparenright}{\isacharparenright}{\isachardoublequoteclose}\ \isanewline
\ \ \ \ \ \ \isacommand{by}\isamarkupfalse%
\ {\isacharparenleft}simp\ only{\isacharcolon}\ simp{\isacharunderscore}thms{\isacharparenleft}{\isadigit{8}}{\isacharcomma}{\isadigit{2}}{\isadigit{5}}{\isacharparenright}\ de{\isacharunderscore}Morgan{\isacharunderscore}disj{\isacharparenright}\isanewline
\ \ \ \ \isacommand{then}\isamarkupfalse%
\ \isacommand{have}\isamarkupfalse%
\ {\isachardoublequoteopen}{\isasymnot}{\isacharparenleft}insert\ G\ S\ {\isasymin}\ {\isacharparenleft}extF\ C{\isacharparenright}{\isacharparenright}{\isachardoublequoteclose}\isanewline
\ \ \ \ \ \ \isacommand{by}\isamarkupfalse%
\ {\isacharparenleft}rule\ conjunct{\isadigit{1}}{\isacharparenright}\isanewline
\ \ \ \ \isacommand{then}\isamarkupfalse%
\ \isacommand{have}\isamarkupfalse%
\ {\isachardoublequoteopen}{\isasymnot}{\isacharparenleft}{\isasymforall}S{\isacharprime}\ {\isasymsubseteq}\ {\isacharparenleft}insert\ G\ S{\isacharparenright}{\isachardot}\ finite\ S{\isacharprime}\ {\isasymlongrightarrow}\ S{\isacharprime}\ {\isasymin}\ C{\isacharparenright}{\isachardoublequoteclose}\isanewline
\ \ \ \ \ \ \isacommand{unfolding}\isamarkupfalse%
\ extF\ \isacommand{by}\isamarkupfalse%
\ {\isacharparenleft}simp\ add{\isacharcolon}\ mem{\isacharunderscore}Collect{\isacharunderscore}eq{\isacharparenright}\isanewline
\ \ \ \ \isacommand{then}\isamarkupfalse%
\ \isacommand{have}\isamarkupfalse%
\ Ex{\isadigit{1}}{\isacharcolon}{\isachardoublequoteopen}{\isasymexists}S{\isacharprime}{\isasymsubseteq}\ {\isacharparenleft}insert\ G\ S{\isacharparenright}{\isachardot}\ {\isasymnot}{\isacharparenleft}finite\ S{\isacharprime}\ {\isasymlongrightarrow}\ S{\isacharprime}\ {\isasymin}\ C{\isacharparenright}{\isachardoublequoteclose}\isanewline
\ \ \ \ \ \ \isacommand{by}\isamarkupfalse%
\ {\isacharparenleft}rule\ sall{\isacharunderscore}simps{\isacharunderscore}not{\isacharunderscore}all{\isacharparenright}\isanewline
\ \ \ \ \isacommand{obtain}\isamarkupfalse%
\ S{\isadigit{1}}\ \isakeyword{where}\ {\isachardoublequoteopen}S{\isadigit{1}}\ {\isasymsubseteq}\ insert\ G\ S{\isachardoublequoteclose}\ \isakeyword{and}\ {\isachardoublequoteopen}{\isasymnot}{\isacharparenleft}finite\ S{\isadigit{1}}\ {\isasymlongrightarrow}\ S{\isadigit{1}}\ {\isasymin}\ C{\isacharparenright}{\isachardoublequoteclose}\isanewline
\ \ \ \ \ \ \isacommand{using}\isamarkupfalse%
\ Ex{\isadigit{1}}\ \isacommand{by}\isamarkupfalse%
\ {\isacharparenleft}rule\ subexE{\isacharparenright}\isanewline
\ \ \ \ \isacommand{have}\isamarkupfalse%
\ {\isachardoublequoteopen}finite\ S{\isadigit{1}}\ {\isasymand}\ S{\isadigit{1}}\ {\isasymnotin}\ C{\isachardoublequoteclose}\ \isanewline
\ \ \ \ \ \ \isacommand{using}\isamarkupfalse%
\ {\isacartoucheopen}{\isasymnot}{\isacharparenleft}finite\ S{\isadigit{1}}\ {\isasymlongrightarrow}\ S{\isadigit{1}}\ {\isasymin}\ C{\isacharparenright}{\isacartoucheclose}\ \isacommand{by}\isamarkupfalse%
\ {\isacharparenleft}simp\ only{\isacharcolon}\ simp{\isacharunderscore}thms{\isacharparenleft}{\isadigit{8}}{\isacharparenright}\ not{\isacharunderscore}imp{\isacharparenright}\isanewline
\ \ \ \ \isacommand{then}\isamarkupfalse%
\ \isacommand{have}\isamarkupfalse%
\ {\isachardoublequoteopen}finite\ S{\isadigit{1}}{\isachardoublequoteclose}\isanewline
\ \ \ \ \ \ \isacommand{by}\isamarkupfalse%
\ {\isacharparenleft}rule\ conjunct{\isadigit{1}}{\isacharparenright}\isanewline
\ \ \ \ \isacommand{have}\isamarkupfalse%
\ {\isachardoublequoteopen}S{\isadigit{1}}\ {\isasymnotin}\ C{\isachardoublequoteclose}\isanewline
\ \ \ \ \ \ \isacommand{using}\isamarkupfalse%
\ {\isacartoucheopen}finite\ S{\isadigit{1}}\ {\isasymand}\ S{\isadigit{1}}\ {\isasymnotin}\ C{\isacartoucheclose}\ \isacommand{by}\isamarkupfalse%
\ {\isacharparenleft}rule\ conjunct{\isadigit{2}}{\isacharparenright}\isanewline
\ \ \ \ \isacommand{then}\isamarkupfalse%
\ \isacommand{have}\isamarkupfalse%
\ {\isachardoublequoteopen}insert\ G\ S{\isadigit{1}}\ {\isasymnotin}\ C{\isachardoublequoteclose}\isanewline
\ \ \ \ \isacommand{proof}\isamarkupfalse%
\ {\isacharminus}\ \isanewline
\ \ \ \ \ \ \isacommand{have}\isamarkupfalse%
\ {\isachardoublequoteopen}S{\isadigit{1}}\ {\isasymsubseteq}\ S\ {\isasymlongrightarrow}\ finite\ S{\isadigit{1}}\ {\isasymlongrightarrow}\ S{\isadigit{1}}\ {\isasymin}\ C{\isachardoublequoteclose}\isanewline
\ \ \ \ \ \ \ \ \isacommand{using}\isamarkupfalse%
\ E{\isacharprime}\ \isacommand{by}\isamarkupfalse%
\ {\isacharparenleft}rule\ allE{\isacharparenright}\isanewline
\ \ \ \ \ \ \isacommand{then}\isamarkupfalse%
\ \isacommand{have}\isamarkupfalse%
\ {\isachardoublequoteopen}{\isasymnot}\ S{\isadigit{1}}\ {\isasymsubseteq}\ S{\isachardoublequoteclose}\isanewline
\ \ \ \ \ \ \ \ \isacommand{using}\isamarkupfalse%
\ {\isacartoucheopen}{\isasymnot}\ {\isacharparenleft}finite\ S{\isadigit{1}}\ {\isasymlongrightarrow}\ S{\isadigit{1}}\ {\isasymin}\ C{\isacharparenright}{\isacartoucheclose}\ \isacommand{by}\isamarkupfalse%
\ {\isacharparenleft}rule\ mt{\isacharparenright}\isanewline
\ \ \ \ \ \ \isacommand{then}\isamarkupfalse%
\ \isacommand{have}\isamarkupfalse%
\ {\isachardoublequoteopen}{\isacharparenleft}S{\isadigit{1}}\ {\isasymsubseteq}\ insert\ G\ S{\isacharparenright}\ {\isasymnoteq}\ {\isacharparenleft}S{\isadigit{1}}\ {\isasymsubseteq}\ S{\isacharparenright}{\isachardoublequoteclose}\isanewline
\ \ \ \ \ \ \ \ \isacommand{using}\isamarkupfalse%
\ {\isacartoucheopen}S{\isadigit{1}}\ {\isasymsubseteq}\ insert\ G\ S{\isacartoucheclose}\ \isacommand{by}\isamarkupfalse%
\ simp\isanewline
\ \ \ \ \ \ \isacommand{then}\isamarkupfalse%
\ \isacommand{have}\isamarkupfalse%
\ notSI{\isacharcolon}{\isachardoublequoteopen}{\isasymnot}{\isacharparenleft}S{\isadigit{1}}\ {\isasymsubseteq}\ insert\ G\ S\ {\isasymlongleftrightarrow}\ S{\isadigit{1}}\ {\isasymsubseteq}\ S{\isacharparenright}{\isachardoublequoteclose}\isanewline
\ \ \ \ \ \ \ \ \isacommand{by}\isamarkupfalse%
\ blast\isanewline
\ \ \ \ \ \ \isacommand{have}\isamarkupfalse%
\ subsetInsert{\isacharcolon}{\isachardoublequoteopen}G\ {\isasymnotin}\ S{\isadigit{1}}\ {\isasymLongrightarrow}\ S{\isadigit{1}}\ {\isasymsubseteq}\ insert\ G\ S\ {\isasymlongleftrightarrow}\ S{\isadigit{1}}\ {\isasymsubseteq}\ S{\isachardoublequoteclose}\isanewline
\ \ \ \ \ \ \ \ \isacommand{by}\isamarkupfalse%
\ {\isacharparenleft}rule\ subset{\isacharunderscore}insert{\isacharparenright}\isanewline
\ \ \ \ \ \ \isacommand{have}\isamarkupfalse%
\ {\isachardoublequoteopen}{\isasymnot}{\isacharparenleft}G\ {\isasymnotin}\ S{\isadigit{1}}{\isacharparenright}{\isachardoublequoteclose}\isanewline
\ \ \ \ \ \ \ \ \isacommand{using}\isamarkupfalse%
\ notSI\ subsetInsert\ \isacommand{by}\isamarkupfalse%
\ {\isacharparenleft}rule\ contrapos{\isacharunderscore}nn{\isacharparenright}\isanewline
\ \ \ \ \ \ \isacommand{then}\isamarkupfalse%
\ \isacommand{have}\isamarkupfalse%
\ {\isachardoublequoteopen}G\ {\isasymin}\ S{\isadigit{1}}{\isachardoublequoteclose}\isanewline
\ \ \ \ \ \ \ \ \isacommand{by}\isamarkupfalse%
\ {\isacharparenleft}rule\ notnotD{\isacharparenright}\isanewline
\ \ \ \ \ \ \isacommand{then}\isamarkupfalse%
\ \isacommand{have}\isamarkupfalse%
\ {\isachardoublequoteopen}insert\ G\ S{\isadigit{1}}\ {\isacharequal}\ S{\isadigit{1}}{\isachardoublequoteclose}\isanewline
\ \ \ \ \ \ \ \ \isacommand{by}\isamarkupfalse%
\ {\isacharparenleft}rule\ insert{\isacharunderscore}absorb{\isacharparenright}\isanewline
\ \ \ \ \ \ \isacommand{show}\isamarkupfalse%
\ {\isacharquery}thesis\isanewline
\ \ \ \ \ \ \ \ \isacommand{using}\isamarkupfalse%
\ {\isacartoucheopen}S{\isadigit{1}}\ {\isasymnotin}\ C{\isacartoucheclose}\ \isacommand{by}\isamarkupfalse%
\ {\isacharparenleft}simp\ only{\isacharcolon}\ simp{\isacharunderscore}thms{\isacharparenleft}{\isadigit{8}}{\isacharparenright}\ {\isacartoucheopen}insert\ G\ S{\isadigit{1}}\ {\isacharequal}\ S{\isadigit{1}}{\isacartoucheclose}{\isacharparenright}\isanewline
\ \ \ \ \isacommand{qed}\isamarkupfalse%
\ \isanewline
\ \ \ \ \isacommand{let}\isamarkupfalse%
\ {\isacharquery}S{\isadigit{1}}{\isacharequal}{\isachardoublequoteopen}S{\isadigit{1}}\ {\isacharminus}\ {\isacharbraceleft}G{\isacharbraceright}{\isachardoublequoteclose}\isanewline
\ \ \ \ \isacommand{have}\isamarkupfalse%
\ {\isachardoublequoteopen}insert\ G\ S\ {\isacharequal}\ {\isacharbraceleft}G{\isacharbraceright}\ {\isasymunion}\ S{\isachardoublequoteclose}\isanewline
\ \ \ \ \ \ \isacommand{by}\isamarkupfalse%
\ {\isacharparenleft}rule\ insert{\isacharunderscore}is{\isacharunderscore}Un{\isacharparenright}\isanewline
\ \ \ \ \isacommand{have}\isamarkupfalse%
\ {\isachardoublequoteopen}S{\isadigit{1}}\ {\isasymsubseteq}\ {\isacharbraceleft}G{\isacharbraceright}\ {\isasymunion}\ S{\isachardoublequoteclose}\isanewline
\ \ \ \ \ \ \isacommand{using}\isamarkupfalse%
\ {\isacartoucheopen}S{\isadigit{1}}\ {\isasymsubseteq}\ insert\ G\ S{\isacartoucheclose}\ \isacommand{by}\isamarkupfalse%
\ {\isacharparenleft}simp\ only{\isacharcolon}\ {\isacartoucheopen}insert\ G\ S\ {\isacharequal}\ {\isacharbraceleft}G{\isacharbraceright}\ {\isasymunion}\ S{\isacartoucheclose}{\isacharparenright}\isanewline
\ \ \ \ \isacommand{have}\isamarkupfalse%
\ {\isadigit{1}}{\isacharcolon}{\isachardoublequoteopen}{\isacharquery}S{\isadigit{1}}\ {\isasymsubseteq}\ S{\isachardoublequoteclose}\ \isanewline
\ \ \ \ \ \ \isacommand{using}\isamarkupfalse%
\ {\isacartoucheopen}S{\isadigit{1}}\ {\isasymsubseteq}\ {\isacharbraceleft}G{\isacharbraceright}\ {\isasymunion}\ S{\isacartoucheclose}\ \isacommand{by}\isamarkupfalse%
\ {\isacharparenleft}simp\ only{\isacharcolon}\ Diff{\isacharunderscore}subset{\isacharunderscore}conv{\isacharparenright}\isanewline
\ \ \ \ \isacommand{have}\isamarkupfalse%
\ {\isadigit{2}}{\isacharcolon}{\isachardoublequoteopen}finite\ {\isacharquery}S{\isadigit{1}}{\isachardoublequoteclose}\isanewline
\ \ \ \ \ \ \isacommand{using}\isamarkupfalse%
\ {\isacartoucheopen}finite\ S{\isadigit{1}}{\isacartoucheclose}\ \isacommand{by}\isamarkupfalse%
\ {\isacharparenleft}simp\ only{\isacharcolon}\ finite{\isacharunderscore}Diff{\isacharparenright}\isanewline
\ \ \ \ \isacommand{have}\isamarkupfalse%
\ {\isachardoublequoteopen}insert\ G\ {\isacharquery}S{\isadigit{1}}\ {\isacharequal}\ insert\ G\ S{\isadigit{1}}{\isachardoublequoteclose}\isanewline
\ \ \ \ \ \ \isacommand{by}\isamarkupfalse%
\ {\isacharparenleft}simp\ only{\isacharcolon}\ insert{\isacharunderscore}Diff{\isacharunderscore}single{\isacharparenright}\isanewline
\ \ \ \ \isacommand{then}\isamarkupfalse%
\ \isacommand{have}\isamarkupfalse%
\ {\isadigit{3}}{\isacharcolon}{\isachardoublequoteopen}insert\ G\ {\isacharquery}S{\isadigit{1}}\ {\isasymnotin}\ C{\isachardoublequoteclose}\isanewline
\ \ \ \ \ \ \isacommand{using}\isamarkupfalse%
\ {\isacartoucheopen}insert\ G\ S{\isadigit{1}}\ {\isasymnotin}\ C{\isacartoucheclose}\ \isacommand{by}\isamarkupfalse%
\ {\isacharparenleft}simp\ only{\isacharcolon}\ simp{\isacharunderscore}thms{\isacharparenleft}{\isadigit{6}}{\isacharcomma}{\isadigit{8}}{\isacharparenright}\ {\isacartoucheopen}insert\ G\ {\isacharquery}S{\isadigit{1}}\ {\isacharequal}\ insert\ G\ S{\isadigit{1}}{\isacartoucheclose}{\isacharparenright}\isanewline
\ \ \ \ \isacommand{have}\isamarkupfalse%
\ {\isachardoublequoteopen}insert\ H\ S\ {\isasymnotin}\ {\isacharparenleft}extF\ C{\isacharparenright}{\isachardoublequoteclose}\isanewline
\ \ \ \ \ \ \isacommand{using}\isamarkupfalse%
\ Conj\ \isacommand{by}\isamarkupfalse%
\ {\isacharparenleft}rule\ conjunct{\isadigit{2}}{\isacharparenright}\isanewline
\ \ \ \ \isacommand{then}\isamarkupfalse%
\ \isacommand{have}\isamarkupfalse%
\ {\isachardoublequoteopen}{\isasymnot}{\isacharparenleft}{\isasymforall}S{\isacharprime}\ {\isasymsubseteq}\ {\isacharparenleft}insert\ H\ S{\isacharparenright}{\isachardot}\ finite\ S{\isacharprime}\ {\isasymlongrightarrow}\ S{\isacharprime}\ {\isasymin}\ C{\isacharparenright}{\isachardoublequoteclose}\isanewline
\ \ \ \ \ \ \isacommand{unfolding}\isamarkupfalse%
\ extF\ \isacommand{by}\isamarkupfalse%
\ {\isacharparenleft}simp\ add{\isacharcolon}\ mem{\isacharunderscore}Collect{\isacharunderscore}eq{\isacharparenright}\isanewline
\ \ \ \ \isacommand{then}\isamarkupfalse%
\ \isacommand{have}\isamarkupfalse%
\ Ex{\isadigit{2}}{\isacharcolon}{\isachardoublequoteopen}{\isasymexists}S{\isacharprime}{\isasymsubseteq}\ {\isacharparenleft}insert\ H\ S{\isacharparenright}{\isachardot}\ {\isasymnot}{\isacharparenleft}finite\ S{\isacharprime}\ {\isasymlongrightarrow}\ S{\isacharprime}\ {\isasymin}\ C{\isacharparenright}{\isachardoublequoteclose}\isanewline
\ \ \ \ \ \ \isacommand{by}\isamarkupfalse%
\ {\isacharparenleft}rule\ sall{\isacharunderscore}simps{\isacharunderscore}not{\isacharunderscore}all{\isacharparenright}\isanewline
\ \ \ \ \isacommand{obtain}\isamarkupfalse%
\ S{\isadigit{2}}\ \isakeyword{where}\ {\isachardoublequoteopen}S{\isadigit{2}}\ {\isasymsubseteq}\ insert\ H\ S{\isachardoublequoteclose}\ \isakeyword{and}\ {\isachardoublequoteopen}{\isasymnot}{\isacharparenleft}finite\ S{\isadigit{2}}\ {\isasymlongrightarrow}\ S{\isadigit{2}}\ {\isasymin}\ C{\isacharparenright}{\isachardoublequoteclose}\isanewline
\ \ \ \ \ \ \isacommand{using}\isamarkupfalse%
\ Ex{\isadigit{2}}\ \isacommand{by}\isamarkupfalse%
\ {\isacharparenleft}rule\ subexE{\isacharparenright}\isanewline
\ \ \ \ \isacommand{have}\isamarkupfalse%
\ {\isachardoublequoteopen}finite\ S{\isadigit{2}}\ {\isasymand}\ S{\isadigit{2}}\ {\isasymnotin}\ C{\isachardoublequoteclose}\isanewline
\ \ \ \ \ \ \isacommand{using}\isamarkupfalse%
\ {\isacartoucheopen}{\isasymnot}{\isacharparenleft}finite\ S{\isadigit{2}}\ {\isasymlongrightarrow}\ S{\isadigit{2}}\ {\isasymin}\ C{\isacharparenright}{\isacartoucheclose}\ \isacommand{by}\isamarkupfalse%
\ {\isacharparenleft}simp\ only{\isacharcolon}\ simp{\isacharunderscore}thms{\isacharparenleft}{\isadigit{8}}{\isacharcomma}{\isadigit{2}}{\isadigit{5}}{\isacharparenright}\ not{\isacharunderscore}imp{\isacharparenright}\isanewline
\ \ \ \ \isacommand{then}\isamarkupfalse%
\ \isacommand{have}\isamarkupfalse%
\ {\isachardoublequoteopen}finite\ S{\isadigit{2}}{\isachardoublequoteclose}\isanewline
\ \ \ \ \ \ \isacommand{by}\isamarkupfalse%
\ {\isacharparenleft}rule\ conjunct{\isadigit{1}}{\isacharparenright}\isanewline
\ \ \ \ \isacommand{have}\isamarkupfalse%
\ {\isachardoublequoteopen}S{\isadigit{2}}\ {\isasymnotin}\ C{\isachardoublequoteclose}\isanewline
\ \ \ \ \ \ \isacommand{using}\isamarkupfalse%
\ {\isacartoucheopen}finite\ S{\isadigit{2}}\ {\isasymand}\ S{\isadigit{2}}\ {\isasymnotin}\ C{\isacartoucheclose}\ \isacommand{by}\isamarkupfalse%
\ {\isacharparenleft}rule\ conjunct{\isadigit{2}}{\isacharparenright}\isanewline
\ \ \ \ \isacommand{then}\isamarkupfalse%
\ \isacommand{have}\isamarkupfalse%
\ {\isachardoublequoteopen}insert\ H\ S{\isadigit{2}}\ {\isasymnotin}\ C{\isachardoublequoteclose}\isanewline
\ \ \ \ \isacommand{proof}\isamarkupfalse%
\ {\isacharminus}\isanewline
\ \ \ \ \ \ \isacommand{have}\isamarkupfalse%
\ {\isachardoublequoteopen}S{\isadigit{2}}\ {\isasymsubseteq}\ S\ {\isasymlongrightarrow}\ finite\ S{\isadigit{2}}\ {\isasymlongrightarrow}\ S{\isadigit{2}}\ {\isasymin}\ C{\isachardoublequoteclose}\isanewline
\ \ \ \ \ \ \ \ \isacommand{using}\isamarkupfalse%
\ E{\isacharprime}\ \isacommand{by}\isamarkupfalse%
\ {\isacharparenleft}rule\ allE{\isacharparenright}\isanewline
\ \ \ \ \ \ \isacommand{then}\isamarkupfalse%
\ \isacommand{have}\isamarkupfalse%
\ {\isachardoublequoteopen}{\isasymnot}\ S{\isadigit{2}}\ {\isasymsubseteq}\ S{\isachardoublequoteclose}\isanewline
\ \ \ \ \ \ \ \ \isacommand{using}\isamarkupfalse%
\ {\isacartoucheopen}{\isasymnot}\ {\isacharparenleft}finite\ S{\isadigit{2}}\ {\isasymlongrightarrow}\ S{\isadigit{2}}\ {\isasymin}\ C{\isacharparenright}{\isacartoucheclose}\ \isacommand{by}\isamarkupfalse%
\ {\isacharparenleft}rule\ mt{\isacharparenright}\isanewline
\ \ \ \ \ \ \isacommand{then}\isamarkupfalse%
\ \isacommand{have}\isamarkupfalse%
\ {\isachardoublequoteopen}{\isacharparenleft}S{\isadigit{2}}\ {\isasymsubseteq}\ insert\ H\ S{\isacharparenright}\ {\isasymnoteq}\ {\isacharparenleft}S{\isadigit{2}}\ {\isasymsubseteq}\ S{\isacharparenright}{\isachardoublequoteclose}\isanewline
\ \ \ \ \ \ \ \ \isacommand{using}\isamarkupfalse%
\ {\isacartoucheopen}S{\isadigit{2}}\ {\isasymsubseteq}\ insert\ H\ S{\isacartoucheclose}\ \isacommand{by}\isamarkupfalse%
\ simp\ \isanewline
\ \ \ \ \ \ \isacommand{then}\isamarkupfalse%
\ \isacommand{have}\isamarkupfalse%
\ notSI{\isacharcolon}{\isachardoublequoteopen}{\isasymnot}{\isacharparenleft}S{\isadigit{2}}\ {\isasymsubseteq}\ insert\ H\ S\ {\isasymlongleftrightarrow}\ S{\isadigit{2}}\ {\isasymsubseteq}\ S{\isacharparenright}{\isachardoublequoteclose}\isanewline
\ \ \ \ \ \ \ \ \isacommand{by}\isamarkupfalse%
\ blast\ \isanewline
\ \ \ \ \ \ \isacommand{have}\isamarkupfalse%
\ subsetInsert{\isacharcolon}{\isachardoublequoteopen}H\ {\isasymnotin}\ S{\isadigit{2}}\ {\isasymLongrightarrow}\ S{\isadigit{2}}\ {\isasymsubseteq}\ insert\ H\ S\ {\isasymlongleftrightarrow}\ S{\isadigit{2}}\ {\isasymsubseteq}\ S{\isachardoublequoteclose}\isanewline
\ \ \ \ \ \ \ \ \isacommand{by}\isamarkupfalse%
\ {\isacharparenleft}rule\ subset{\isacharunderscore}insert{\isacharparenright}\isanewline
\ \ \ \ \ \ \isacommand{have}\isamarkupfalse%
\ {\isachardoublequoteopen}{\isasymnot}{\isacharparenleft}H\ {\isasymnotin}\ S{\isadigit{2}}{\isacharparenright}{\isachardoublequoteclose}\isanewline
\ \ \ \ \ \ \ \ \isacommand{using}\isamarkupfalse%
\ notSI\ subsetInsert\ \isacommand{by}\isamarkupfalse%
\ {\isacharparenleft}rule\ contrapos{\isacharunderscore}nn{\isacharparenright}\isanewline
\ \ \ \ \ \ \isacommand{then}\isamarkupfalse%
\ \isacommand{have}\isamarkupfalse%
\ {\isachardoublequoteopen}H\ {\isasymin}\ S{\isadigit{2}}{\isachardoublequoteclose}\isanewline
\ \ \ \ \ \ \ \ \isacommand{by}\isamarkupfalse%
\ {\isacharparenleft}rule\ notnotD{\isacharparenright}\isanewline
\ \ \ \ \ \ \isacommand{then}\isamarkupfalse%
\ \isacommand{have}\isamarkupfalse%
\ {\isachardoublequoteopen}insert\ H\ S{\isadigit{2}}\ {\isacharequal}\ S{\isadigit{2}}{\isachardoublequoteclose}\isanewline
\ \ \ \ \ \ \ \ \isacommand{by}\isamarkupfalse%
\ {\isacharparenleft}rule\ insert{\isacharunderscore}absorb{\isacharparenright}\isanewline
\ \ \ \ \ \ \isacommand{show}\isamarkupfalse%
\ {\isacharquery}thesis\isanewline
\ \ \ \ \ \ \ \ \isacommand{using}\isamarkupfalse%
\ {\isacartoucheopen}S{\isadigit{2}}\ {\isasymnotin}\ C{\isacartoucheclose}\ \isacommand{by}\isamarkupfalse%
\ {\isacharparenleft}simp\ only{\isacharcolon}\ simp{\isacharunderscore}thms{\isacharparenleft}{\isadigit{8}}{\isacharparenright}\ {\isacartoucheopen}insert\ H\ S{\isadigit{2}}\ {\isacharequal}\ S{\isadigit{2}}{\isacartoucheclose}{\isacharparenright}\isanewline
\ \ \ \ \isacommand{qed}\isamarkupfalse%
\ \isanewline
\ \ \ \ \isacommand{let}\isamarkupfalse%
\ {\isacharquery}S{\isadigit{2}}{\isacharequal}{\isachardoublequoteopen}S{\isadigit{2}}\ {\isacharminus}\ {\isacharbraceleft}H{\isacharbraceright}{\isachardoublequoteclose}\isanewline
\ \ \ \ \isacommand{have}\isamarkupfalse%
\ {\isachardoublequoteopen}insert\ H\ S\ {\isacharequal}\ {\isacharbraceleft}H{\isacharbraceright}\ {\isasymunion}\ S{\isachardoublequoteclose}\isanewline
\ \ \ \ \ \ \isacommand{by}\isamarkupfalse%
\ {\isacharparenleft}rule\ insert{\isacharunderscore}is{\isacharunderscore}Un{\isacharparenright}\isanewline
\ \ \ \ \isacommand{have}\isamarkupfalse%
\ {\isachardoublequoteopen}S{\isadigit{2}}\ {\isasymsubseteq}\ {\isacharbraceleft}H{\isacharbraceright}\ {\isasymunion}\ S{\isachardoublequoteclose}\isanewline
\ \ \ \ \ \ \isacommand{using}\isamarkupfalse%
\ {\isacartoucheopen}S{\isadigit{2}}\ {\isasymsubseteq}\ insert\ H\ S{\isacartoucheclose}\ \isacommand{by}\isamarkupfalse%
\ {\isacharparenleft}simp\ only{\isacharcolon}\ {\isacartoucheopen}insert\ H\ S\ {\isacharequal}\ {\isacharbraceleft}H{\isacharbraceright}\ {\isasymunion}\ S{\isacartoucheclose}{\isacharparenright}\isanewline
\ \ \ \ \isacommand{have}\isamarkupfalse%
\ {\isadigit{4}}{\isacharcolon}{\isachardoublequoteopen}{\isacharquery}S{\isadigit{2}}\ {\isasymsubseteq}\ S{\isachardoublequoteclose}\ \isanewline
\ \ \ \ \ \ \isacommand{using}\isamarkupfalse%
\ {\isacartoucheopen}S{\isadigit{2}}\ {\isasymsubseteq}\ {\isacharbraceleft}H{\isacharbraceright}\ {\isasymunion}\ S{\isacartoucheclose}\ \isacommand{by}\isamarkupfalse%
\ {\isacharparenleft}simp\ only{\isacharcolon}\ Diff{\isacharunderscore}subset{\isacharunderscore}conv{\isacharparenright}\isanewline
\ \ \ \ \isacommand{have}\isamarkupfalse%
\ {\isadigit{5}}{\isacharcolon}{\isachardoublequoteopen}finite\ {\isacharquery}S{\isadigit{2}}{\isachardoublequoteclose}\ \isanewline
\ \ \ \ \ \ \isacommand{using}\isamarkupfalse%
\ {\isacartoucheopen}finite\ S{\isadigit{2}}{\isacartoucheclose}\ \isacommand{by}\isamarkupfalse%
\ {\isacharparenleft}simp\ only{\isacharcolon}\ finite{\isacharunderscore}Diff{\isacharparenright}\isanewline
\ \ \ \ \isacommand{have}\isamarkupfalse%
\ {\isachardoublequoteopen}insert\ H\ {\isacharquery}S{\isadigit{2}}\ {\isacharequal}\ insert\ H\ S{\isadigit{2}}{\isachardoublequoteclose}\isanewline
\ \ \ \ \ \ \isacommand{by}\isamarkupfalse%
\ {\isacharparenleft}simp\ only{\isacharcolon}\ insert{\isacharunderscore}Diff{\isacharunderscore}single{\isacharparenright}\isanewline
\ \ \ \ \isacommand{then}\isamarkupfalse%
\ \isacommand{have}\isamarkupfalse%
\ {\isadigit{6}}{\isacharcolon}{\isachardoublequoteopen}insert\ H\ {\isacharquery}S{\isadigit{2}}\ {\isasymnotin}\ C{\isachardoublequoteclose}\isanewline
\ \ \ \ \ \ \isacommand{using}\isamarkupfalse%
\ {\isacartoucheopen}insert\ H\ S{\isadigit{2}}\ {\isasymnotin}\ C{\isacartoucheclose}\ \isacommand{by}\isamarkupfalse%
\ {\isacharparenleft}simp\ only{\isacharcolon}\ simp{\isacharunderscore}thms{\isacharparenleft}{\isadigit{6}}{\isacharcomma}{\isadigit{8}}{\isacharparenright}\ {\isacartoucheopen}insert\ H\ {\isacharquery}S{\isadigit{2}}\ {\isacharequal}\ insert\ H\ S{\isadigit{2}}{\isacartoucheclose}{\isacharparenright}\isanewline
\ \ \ \ \isacommand{show}\isamarkupfalse%
\ {\isachardoublequoteopen}False{\isachardoublequoteclose}\isanewline
\ \ \ \ \ \ \isacommand{using}\isamarkupfalse%
\ assms{\isacharparenleft}{\isadigit{1}}{\isacharparenright}\ assms{\isacharparenleft}{\isadigit{2}}{\isacharparenright}\ assms{\isacharparenleft}{\isadigit{3}}{\isacharparenright}\ assms{\isacharparenleft}{\isadigit{4}}{\isacharparenright}\ assms{\isacharparenleft}{\isadigit{5}}{\isacharparenright}\ {\isadigit{1}}\ {\isadigit{2}}\ {\isadigit{3}}\ {\isadigit{4}}\ {\isadigit{5}}\ {\isadigit{6}}\ \isacommand{by}\isamarkupfalse%
\ {\isacharparenleft}rule\ ex{\isadigit{3}}{\isacharunderscore}pcp{\isacharunderscore}SinE{\isacharunderscore}DIS{\isacharunderscore}auxFalse{\isacharparenright}\isanewline
\ \ \isacommand{qed}\isamarkupfalse%
\isanewline
\ \ \isacommand{thus}\isamarkupfalse%
\ {\isacharquery}thesis\isanewline
\ \ \isacommand{proof}\isamarkupfalse%
\ {\isacharparenleft}rule\ disjE{\isacharparenright}\isanewline
\ \ \ \ \isacommand{assume}\isamarkupfalse%
\ {\isachardoublequoteopen}insert\ G\ S\ {\isasymin}\ {\isacharparenleft}extF\ C{\isacharparenright}{\isachardoublequoteclose}\isanewline
\ \ \ \ \isacommand{have}\isamarkupfalse%
\ insG{\isacharcolon}{\isachardoublequoteopen}insert\ G\ S\ {\isasymin}\ {\isacharparenleft}extensionFin\ C{\isacharparenright}{\isachardoublequoteclose}\isanewline
\ \ \ \ \ \ \isacommand{using}\isamarkupfalse%
\ {\isacartoucheopen}{\isacharparenleft}extF\ C{\isacharparenright}\ {\isasymsubseteq}\ {\isacharparenleft}extensionFin\ C{\isacharparenright}{\isacartoucheclose}\ {\isacartoucheopen}insert\ G\ S\ {\isasymin}\ {\isacharparenleft}extF\ C{\isacharparenright}{\isacartoucheclose}\ \isacommand{by}\isamarkupfalse%
\ {\isacharparenleft}simp\ only{\isacharcolon}\ in{\isacharunderscore}mono{\isacharparenright}\isanewline
\ \ \ \ \isacommand{have}\isamarkupfalse%
\ {\isachardoublequoteopen}insert\ G\ S\ {\isacharequal}\ {\isacharbraceleft}G{\isacharbraceright}\ {\isasymunion}\ S{\isachardoublequoteclose}\isanewline
\ \ \ \ \ \ \isacommand{by}\isamarkupfalse%
\ {\isacharparenleft}rule\ insert{\isacharunderscore}is{\isacharunderscore}Un{\isacharparenright}\isanewline
\ \ \ \ \isacommand{then}\isamarkupfalse%
\ \isacommand{have}\isamarkupfalse%
\ {\isachardoublequoteopen}{\isacharbraceleft}G{\isacharbraceright}\ {\isasymunion}\ S\ {\isasymin}\ {\isacharparenleft}extensionFin\ C{\isacharparenright}{\isachardoublequoteclose}\isanewline
\ \ \ \ \ \ \isacommand{using}\isamarkupfalse%
\ insG\ {\isacartoucheopen}insert\ G\ S\ {\isacharequal}\ {\isacharbraceleft}G{\isacharbraceright}\ {\isasymunion}\ S{\isacartoucheclose}\ \isacommand{by}\isamarkupfalse%
\ {\isacharparenleft}simp\ only{\isacharcolon}\ insG{\isacharparenright}\isanewline
\ \ \ \ \isacommand{thus}\isamarkupfalse%
\ {\isacharquery}thesis\isanewline
\ \ \ \ \ \ \isacommand{by}\isamarkupfalse%
\ {\isacharparenleft}rule\ disjI{\isadigit{1}}{\isacharparenright}\isanewline
\ \ \isacommand{next}\isamarkupfalse%
\isanewline
\ \ \ \ \isacommand{assume}\isamarkupfalse%
\ {\isachardoublequoteopen}insert\ H\ S\ {\isasymin}\ {\isacharparenleft}extF\ C{\isacharparenright}{\isachardoublequoteclose}\isanewline
\ \ \ \ \isacommand{have}\isamarkupfalse%
\ insH{\isacharcolon}{\isachardoublequoteopen}insert\ H\ S\ {\isasymin}\ {\isacharparenleft}extensionFin\ C{\isacharparenright}{\isachardoublequoteclose}\isanewline
\ \ \ \ \ \ \isacommand{using}\isamarkupfalse%
\ {\isacartoucheopen}{\isacharparenleft}extF\ C{\isacharparenright}\ {\isasymsubseteq}\ {\isacharparenleft}extensionFin\ C{\isacharparenright}{\isacartoucheclose}\ {\isacartoucheopen}insert\ H\ S\ {\isasymin}\ {\isacharparenleft}extF\ C{\isacharparenright}{\isacartoucheclose}\ \isacommand{by}\isamarkupfalse%
\ {\isacharparenleft}simp\ only{\isacharcolon}\ in{\isacharunderscore}mono{\isacharparenright}\isanewline
\ \ \ \ \isacommand{have}\isamarkupfalse%
\ {\isachardoublequoteopen}insert\ H\ S\ {\isacharequal}\ {\isacharbraceleft}H{\isacharbraceright}\ {\isasymunion}\ S{\isachardoublequoteclose}\isanewline
\ \ \ \ \ \ \isacommand{by}\isamarkupfalse%
\ {\isacharparenleft}rule\ insert{\isacharunderscore}is{\isacharunderscore}Un{\isacharparenright}\isanewline
\ \ \ \ \isacommand{then}\isamarkupfalse%
\ \isacommand{have}\isamarkupfalse%
\ {\isachardoublequoteopen}{\isacharbraceleft}H{\isacharbraceright}\ {\isasymunion}\ S\ {\isasymin}\ {\isacharparenleft}extensionFin\ C{\isacharparenright}{\isachardoublequoteclose}\isanewline
\ \ \ \ \ \ \isacommand{using}\isamarkupfalse%
\ insH\ {\isacartoucheopen}insert\ H\ S\ {\isacharequal}\ {\isacharbraceleft}H{\isacharbraceright}\ {\isasymunion}\ S{\isacartoucheclose}\ \isacommand{by}\isamarkupfalse%
\ {\isacharparenleft}simp\ only{\isacharcolon}\ insH{\isacharparenright}\isanewline
\ \ \ \ \isacommand{thus}\isamarkupfalse%
\ {\isacharquery}thesis\isanewline
\ \ \ \ \ \ \isacommand{by}\isamarkupfalse%
\ {\isacharparenleft}rule\ disjI{\isadigit{2}}{\isacharparenright}\isanewline
\ \ \isacommand{qed}\isamarkupfalse%
\isanewline
\isacommand{qed}\isamarkupfalse%
%
\endisatagproof
{\isafoldproof}%
%
\isadelimproof
%
\endisadelimproof
%
\begin{isamarkuptext}%
Probados los lemas \isa{ex{\isadigit{3}}{\isacharunderscore}pcp{\isacharunderscore}SinE{\isacharunderscore}CON} y \isa{ex{\isadigit{3}}{\isacharunderscore}pcp{\isacharunderscore}SinE{\isacharunderscore}DIS}, podemos demostrar que \isa{C{\isacharprime}\ {\isacharequal}\ C\ {\isasymunion}\ E} 
  verifica las condiciones del lema de caracterización de la propiedad de consistencia proposicional 
  para el caso en que \isa{S\ {\isasymin}\ E}, formalizado como \isa{ex{\isadigit{3}}{\isacharunderscore}pcp{\isacharunderscore}SinE}. Dicho lema prueba que, si \isa{C} es 
  una colección con la propiedad de consistencia proposicional y cerrada bajo subconjuntos, y sea 
  \isa{S\ {\isasymin}\ E}, se verifican las condiciones:
  \begin{itemize}
    \item \isa{{\isasymbottom}\ {\isasymnotin}\ S}.
    \item Dada \isa{p} una fórmula atómica cualquiera, no se tiene 
    simultáneamente que\\ \isa{p\ {\isasymin}\ S} y \isa{{\isasymnot}\ p\ {\isasymin}\ S}.
    \item Para toda fórmula de tipo \isa{{\isasymalpha}} con componentes \isa{{\isasymalpha}\isactrlsub {\isadigit{1}}} y \isa{{\isasymalpha}\isactrlsub {\isadigit{2}}} tal que \isa{{\isasymalpha}}
    pertenece a \isa{S}, se tiene que \isa{{\isacharbraceleft}{\isasymalpha}\isactrlsub {\isadigit{1}}{\isacharcomma}{\isasymalpha}\isactrlsub {\isadigit{2}}{\isacharbraceright}\ {\isasymunion}\ S} pertenece a \isa{C{\isacharprime}}.
    \item Para toda fórmula de tipo \isa{{\isasymbeta}} con componentes \isa{{\isasymbeta}\isactrlsub {\isadigit{1}}} y \isa{{\isasymbeta}\isactrlsub {\isadigit{2}}} tal que \isa{{\isasymbeta}}
    pertenece a \isa{S}, se tiene que o bien \isa{{\isacharbraceleft}{\isasymbeta}\isactrlsub {\isadigit{1}}{\isacharbraceright}\ {\isasymunion}\ S} pertenece a \isa{C{\isacharprime}} o 
    bien \isa{{\isacharbraceleft}{\isasymbeta}\isactrlsub {\isadigit{2}}{\isacharbraceright}\ {\isasymunion}\ S} pertenece a \isa{C{\isacharprime}}.
  \end{itemize}%
\end{isamarkuptext}\isamarkuptrue%
\isacommand{lemma}\isamarkupfalse%
\ ex{\isadigit{3}}{\isacharunderscore}pcp{\isacharunderscore}SinE{\isacharcolon}\isanewline
\ \ \isakeyword{assumes}\ {\isachardoublequoteopen}pcp\ C{\isachardoublequoteclose}\isanewline
\ \ \ \ \ \ \ \ \ \ {\isachardoublequoteopen}subset{\isacharunderscore}closed\ C{\isachardoublequoteclose}\isanewline
\ \ \ \ \ \ \ \ \ \ {\isachardoublequoteopen}S\ {\isasymin}\ {\isacharparenleft}extF\ C{\isacharparenright}{\isachardoublequoteclose}\ \isanewline
\ \ \isakeyword{shows}\ {\isachardoublequoteopen}{\isasymbottom}\ {\isasymnotin}\ S\ {\isasymand}\isanewline
\ \ \ \ \ \ \ \ \ {\isacharparenleft}{\isasymforall}k{\isachardot}\ Atom\ k\ {\isasymin}\ S\ {\isasymlongrightarrow}\ \isactrlbold {\isasymnot}\ {\isacharparenleft}Atom\ k{\isacharparenright}\ {\isasymin}\ S\ {\isasymlongrightarrow}\ False{\isacharparenright}\ {\isasymand}\isanewline
\ \ \ \ \ \ \ \ \ {\isacharparenleft}{\isasymforall}F\ G\ H{\isachardot}\ Con\ F\ G\ H\ {\isasymlongrightarrow}\ F\ {\isasymin}\ S\ {\isasymlongrightarrow}\ {\isacharbraceleft}G{\isacharcomma}\ H{\isacharbraceright}\ {\isasymunion}\ S\ {\isasymin}\ {\isacharparenleft}extensionFin\ C{\isacharparenright}{\isacharparenright}\ {\isasymand}\isanewline
\ \ \ \ \ \ \ \ \ {\isacharparenleft}{\isasymforall}F\ G\ H{\isachardot}\ Dis\ F\ G\ H\ {\isasymlongrightarrow}\ F\ {\isasymin}\ S\ {\isasymlongrightarrow}\ {\isacharbraceleft}G{\isacharbraceright}\ {\isasymunion}\ S\ {\isasymin}\ {\isacharparenleft}extensionFin\ C{\isacharparenright}\ {\isasymor}\ {\isacharbraceleft}H{\isacharbraceright}\ {\isasymunion}\ S\ {\isasymin}\ {\isacharparenleft}extensionFin\ C{\isacharparenright}{\isacharparenright}{\isachardoublequoteclose}\isanewline
%
\isadelimproof
%
\endisadelimproof
%
\isatagproof
\isacommand{proof}\isamarkupfalse%
\ {\isacharminus}\isanewline
\ \ \isacommand{have}\isamarkupfalse%
\ PCP{\isacharcolon}{\isachardoublequoteopen}{\isasymforall}S\ {\isasymin}\ C{\isachardot}\isanewline
\ \ \ \ \ \ \ \ \ {\isasymbottom}\ {\isasymnotin}\ S\ {\isasymand}\isanewline
\ \ \ \ \ \ \ \ \ {\isacharparenleft}{\isasymforall}k{\isachardot}\ Atom\ k\ {\isasymin}\ S\ {\isasymlongrightarrow}\ \isactrlbold {\isasymnot}\ {\isacharparenleft}Atom\ k{\isacharparenright}\ {\isasymin}\ S\ {\isasymlongrightarrow}\ False{\isacharparenright}\ {\isasymand}\isanewline
\ \ \ \ \ \ \ \ \ {\isacharparenleft}{\isasymforall}F\ G\ H{\isachardot}\ Con\ F\ G\ H\ {\isasymlongrightarrow}\ F\ {\isasymin}\ S\ {\isasymlongrightarrow}\ {\isacharbraceleft}G{\isacharcomma}\ H{\isacharbraceright}\ {\isasymunion}\ S\ {\isasymin}\ C{\isacharparenright}\ {\isasymand}\isanewline
\ \ \ \ \ \ \ \ \ {\isacharparenleft}{\isasymforall}F\ G\ H{\isachardot}\ Dis\ F\ G\ H\ {\isasymlongrightarrow}\ F\ {\isasymin}\ S\ {\isasymlongrightarrow}\ {\isacharbraceleft}G{\isacharbraceright}\ {\isasymunion}\ S\ {\isasymin}\ C\ {\isasymor}\ {\isacharbraceleft}H{\isacharbraceright}\ {\isasymunion}\ S\ {\isasymin}\ C{\isacharparenright}{\isachardoublequoteclose}\isanewline
\ \ \ \ \isacommand{using}\isamarkupfalse%
\ assms{\isacharparenleft}{\isadigit{1}}{\isacharparenright}\ \isacommand{by}\isamarkupfalse%
\ {\isacharparenleft}rule\ pcp{\isacharunderscore}alt{\isadigit{1}}{\isacharparenright}\isanewline
\ \ \isacommand{have}\isamarkupfalse%
\ E{\isacharcolon}{\isachardoublequoteopen}{\isasymforall}S{\isacharprime}\ {\isasymsubseteq}\ S{\isachardot}\ finite\ S{\isacharprime}\ {\isasymlongrightarrow}\ S{\isacharprime}\ {\isasymin}\ C{\isachardoublequoteclose}\isanewline
\ \ \ \ \isacommand{using}\isamarkupfalse%
\ assms{\isacharparenleft}{\isadigit{3}}{\isacharparenright}\ \isacommand{unfolding}\isamarkupfalse%
\ extF\ \isacommand{by}\isamarkupfalse%
\ {\isacharparenleft}rule\ CollectD{\isacharparenright}\isanewline
\ \ \isacommand{have}\isamarkupfalse%
\ {\isachardoublequoteopen}{\isacharbraceleft}{\isacharbraceright}\ {\isasymsubseteq}\ S{\isachardoublequoteclose}\isanewline
\ \ \ \ \isacommand{by}\isamarkupfalse%
\ {\isacharparenleft}rule\ empty{\isacharunderscore}subsetI{\isacharparenright}\isanewline
\ \ \isacommand{have}\isamarkupfalse%
\ C{\isadigit{1}}{\isacharcolon}{\isachardoublequoteopen}{\isasymbottom}\ {\isasymnotin}\ S{\isachardoublequoteclose}\isanewline
\ \ \isacommand{proof}\isamarkupfalse%
\ {\isacharparenleft}rule\ ccontr{\isacharparenright}\isanewline
\ \ \ \ \isacommand{assume}\isamarkupfalse%
\ {\isachardoublequoteopen}{\isasymnot}{\isacharparenleft}{\isasymbottom}\ {\isasymnotin}\ S{\isacharparenright}{\isachardoublequoteclose}\isanewline
\ \ \ \ \isacommand{then}\isamarkupfalse%
\ \isacommand{have}\isamarkupfalse%
\ {\isachardoublequoteopen}{\isasymbottom}\ {\isasymin}\ S{\isachardoublequoteclose}\isanewline
\ \ \ \ \ \ \isacommand{by}\isamarkupfalse%
\ {\isacharparenleft}rule\ notnotD{\isacharparenright}\isanewline
\ \ \ \ \isacommand{then}\isamarkupfalse%
\ \isacommand{have}\isamarkupfalse%
\ {\isachardoublequoteopen}{\isasymbottom}\ {\isasymin}\ S\ {\isasymand}\ {\isacharbraceleft}{\isacharbraceright}\ {\isasymsubseteq}\ S{\isachardoublequoteclose}\isanewline
\ \ \ \ \ \ \isacommand{using}\isamarkupfalse%
\ {\isacartoucheopen}{\isacharbraceleft}{\isacharbraceright}\ {\isasymsubseteq}\ S{\isacartoucheclose}\ \isacommand{by}\isamarkupfalse%
\ {\isacharparenleft}rule\ conjI{\isacharparenright}\isanewline
\ \ \ \ \isacommand{then}\isamarkupfalse%
\ \isacommand{have}\isamarkupfalse%
\ {\isachardoublequoteopen}insert\ {\isasymbottom}\ {\isacharbraceleft}{\isacharbraceright}\ {\isasymsubseteq}\ S{\isachardoublequoteclose}\ \isanewline
\ \ \ \ \ \ \isacommand{by}\isamarkupfalse%
\ {\isacharparenleft}simp\ only{\isacharcolon}\ insert{\isacharunderscore}subset{\isacharparenright}\isanewline
\ \ \ \ \isacommand{have}\isamarkupfalse%
\ {\isachardoublequoteopen}finite\ {\isacharbraceleft}{\isacharbraceright}{\isachardoublequoteclose}\isanewline
\ \ \ \ \ \ \isacommand{by}\isamarkupfalse%
\ {\isacharparenleft}rule\ finite{\isachardot}emptyI{\isacharparenright}\isanewline
\ \ \ \ \isacommand{then}\isamarkupfalse%
\ \isacommand{have}\isamarkupfalse%
\ {\isachardoublequoteopen}finite\ {\isacharparenleft}insert\ {\isasymbottom}\ {\isacharbraceleft}{\isacharbraceright}{\isacharparenright}{\isachardoublequoteclose}\isanewline
\ \ \ \ \ \ \isacommand{by}\isamarkupfalse%
\ {\isacharparenleft}rule\ finite{\isachardot}insertI{\isacharparenright}\isanewline
\ \ \ \ \isacommand{have}\isamarkupfalse%
\ {\isachardoublequoteopen}finite\ {\isacharparenleft}insert\ {\isasymbottom}\ {\isacharbraceleft}{\isacharbraceright}{\isacharparenright}\ {\isasymlongrightarrow}\ {\isacharparenleft}insert\ {\isasymbottom}\ {\isacharbraceleft}{\isacharbraceright}{\isacharparenright}\ {\isasymin}\ C{\isachardoublequoteclose}\isanewline
\ \ \ \ \ \ \isacommand{using}\isamarkupfalse%
\ E\ {\isacartoucheopen}{\isacharparenleft}insert\ {\isasymbottom}\ {\isacharbraceleft}{\isacharbraceright}{\isacharparenright}\ {\isasymsubseteq}\ S{\isacartoucheclose}\ \isacommand{by}\isamarkupfalse%
\ simp\ \isanewline
\ \ \ \ \isacommand{then}\isamarkupfalse%
\ \isacommand{have}\isamarkupfalse%
\ {\isachardoublequoteopen}{\isacharparenleft}insert\ {\isasymbottom}\ {\isacharbraceleft}{\isacharbraceright}{\isacharparenright}\ {\isasymin}\ C{\isachardoublequoteclose}\isanewline
\ \ \ \ \ \ \isacommand{using}\isamarkupfalse%
\ {\isacartoucheopen}finite\ {\isacharparenleft}insert\ {\isasymbottom}\ {\isacharbraceleft}{\isacharbraceright}{\isacharparenright}{\isacartoucheclose}\ \isacommand{by}\isamarkupfalse%
\ {\isacharparenleft}rule\ mp{\isacharparenright}\isanewline
\ \ \ \ \isacommand{have}\isamarkupfalse%
\ {\isachardoublequoteopen}{\isasymbottom}\ {\isasymnotin}\ {\isacharparenleft}insert\ {\isasymbottom}\ {\isacharbraceleft}{\isacharbraceright}{\isacharparenright}\ {\isasymand}\isanewline
\ \ \ \ \ \ \ \ \ {\isacharparenleft}{\isasymforall}k{\isachardot}\ Atom\ k\ {\isasymin}\ {\isacharparenleft}insert\ {\isasymbottom}\ {\isacharbraceleft}{\isacharbraceright}{\isacharparenright}\ {\isasymlongrightarrow}\ \isactrlbold {\isasymnot}\ {\isacharparenleft}Atom\ k{\isacharparenright}\ {\isasymin}\ {\isacharparenleft}insert\ {\isasymbottom}\ {\isacharbraceleft}{\isacharbraceright}{\isacharparenright}\ {\isasymlongrightarrow}\ False{\isacharparenright}\ {\isasymand}\isanewline
\ \ \ \ \ \ \ \ \ {\isacharparenleft}{\isasymforall}F\ G\ H{\isachardot}\ Con\ F\ G\ H\ {\isasymlongrightarrow}\ F\ {\isasymin}\ {\isacharparenleft}insert\ {\isasymbottom}\ {\isacharbraceleft}{\isacharbraceright}{\isacharparenright}\ {\isasymlongrightarrow}\ {\isacharbraceleft}G{\isacharcomma}\ H{\isacharbraceright}\ {\isasymunion}\ {\isacharparenleft}insert\ {\isasymbottom}\ {\isacharbraceleft}{\isacharbraceright}{\isacharparenright}\ {\isasymin}\ C{\isacharparenright}\ {\isasymand}\isanewline
\ \ \ \ \ \ \ \ \ {\isacharparenleft}{\isasymforall}F\ G\ H{\isachardot}\ Dis\ F\ G\ H\ {\isasymlongrightarrow}\ F\ {\isasymin}\ {\isacharparenleft}insert\ {\isasymbottom}\ {\isacharbraceleft}{\isacharbraceright}{\isacharparenright}\ {\isasymlongrightarrow}\ {\isacharbraceleft}G{\isacharbraceright}\ {\isasymunion}\ {\isacharparenleft}insert\ {\isasymbottom}\ {\isacharbraceleft}{\isacharbraceright}{\isacharparenright}\ {\isasymin}\ C\ {\isasymor}\ {\isacharbraceleft}H{\isacharbraceright}\ {\isasymunion}\ {\isacharparenleft}insert\ {\isasymbottom}\ {\isacharbraceleft}{\isacharbraceright}{\isacharparenright}\ {\isasymin}\ C{\isacharparenright}{\isachardoublequoteclose}\isanewline
\ \ \ \ \ \ \isacommand{using}\isamarkupfalse%
\ PCP\ {\isacartoucheopen}{\isacharparenleft}insert\ {\isasymbottom}\ {\isacharbraceleft}{\isacharbraceright}{\isacharparenright}\ {\isasymin}\ C{\isacartoucheclose}\ \isacommand{by}\isamarkupfalse%
\ blast\ \isanewline
\ \ \ \ \isacommand{then}\isamarkupfalse%
\ \isacommand{have}\isamarkupfalse%
\ {\isachardoublequoteopen}{\isasymbottom}\ {\isasymnotin}\ {\isacharparenleft}insert\ {\isasymbottom}\ {\isacharbraceleft}{\isacharbraceright}{\isacharparenright}{\isachardoublequoteclose}\isanewline
\ \ \ \ \ \ \isacommand{by}\isamarkupfalse%
\ {\isacharparenleft}rule\ conjunct{\isadigit{1}}{\isacharparenright}\isanewline
\ \ \ \ \isacommand{have}\isamarkupfalse%
\ {\isachardoublequoteopen}{\isasymbottom}\ {\isasymin}\ {\isacharparenleft}insert\ {\isasymbottom}\ {\isacharbraceleft}{\isacharbraceright}{\isacharparenright}{\isachardoublequoteclose}\isanewline
\ \ \ \ \ \ \isacommand{by}\isamarkupfalse%
\ {\isacharparenleft}rule\ insertI{\isadigit{1}}{\isacharparenright}\isanewline
\ \ \ \ \isacommand{show}\isamarkupfalse%
\ {\isachardoublequoteopen}False{\isachardoublequoteclose}\isanewline
\ \ \ \ \ \ \isacommand{using}\isamarkupfalse%
\ {\isacartoucheopen}{\isasymbottom}\ {\isasymnotin}\ {\isacharparenleft}insert\ {\isasymbottom}\ {\isacharbraceleft}{\isacharbraceright}{\isacharparenright}{\isacartoucheclose}\ {\isacartoucheopen}{\isasymbottom}\ {\isasymin}\ {\isacharparenleft}insert\ {\isasymbottom}\ {\isacharbraceleft}{\isacharbraceright}{\isacharparenright}{\isacartoucheclose}\ \isacommand{by}\isamarkupfalse%
\ {\isacharparenleft}rule\ notE{\isacharparenright}\isanewline
\ \ \isacommand{qed}\isamarkupfalse%
\isanewline
\ \ \isacommand{have}\isamarkupfalse%
\ C{\isadigit{2}}{\isacharcolon}{\isachardoublequoteopen}{\isasymforall}k{\isachardot}\ Atom\ k\ {\isasymin}\ S\ {\isasymlongrightarrow}\ \isactrlbold {\isasymnot}\ {\isacharparenleft}Atom\ k{\isacharparenright}\ {\isasymin}\ S\ {\isasymlongrightarrow}\ False{\isachardoublequoteclose}\isanewline
\ \ \isacommand{proof}\isamarkupfalse%
\ {\isacharparenleft}rule\ allI{\isacharparenright}\isanewline
\ \ \ \ \isacommand{fix}\isamarkupfalse%
\ k\isanewline
\ \ \ \ \isacommand{show}\isamarkupfalse%
\ {\isachardoublequoteopen}Atom\ k\ {\isasymin}\ S\ {\isasymlongrightarrow}\ \isactrlbold {\isasymnot}{\isacharparenleft}Atom\ k{\isacharparenright}\ {\isasymin}\ S\ {\isasymlongrightarrow}\ False{\isachardoublequoteclose}\isanewline
\ \ \ \ \isacommand{proof}\isamarkupfalse%
\ {\isacharparenleft}rule\ impI{\isacharparenright}{\isacharplus}\isanewline
\ \ \ \ \ \ \isacommand{assume}\isamarkupfalse%
\ {\isachardoublequoteopen}Atom\ k\ {\isasymin}\ S{\isachardoublequoteclose}\isanewline
\ \ \ \ \ \ \isacommand{assume}\isamarkupfalse%
\ {\isachardoublequoteopen}\isactrlbold {\isasymnot}{\isacharparenleft}Atom\ k{\isacharparenright}\ {\isasymin}\ S{\isachardoublequoteclose}\isanewline
\ \ \ \ \ \ \isacommand{let}\isamarkupfalse%
\ {\isacharquery}A{\isacharequal}{\isachardoublequoteopen}insert\ {\isacharparenleft}Atom\ k{\isacharparenright}\ {\isacharparenleft}insert\ {\isacharparenleft}\isactrlbold {\isasymnot}{\isacharparenleft}Atom\ k{\isacharparenright}{\isacharparenright}\ {\isacharbraceleft}{\isacharbraceright}{\isacharparenright}{\isachardoublequoteclose}\isanewline
\ \ \ \ \ \ \isacommand{have}\isamarkupfalse%
\ {\isachardoublequoteopen}Atom\ k\ {\isasymin}\ {\isacharquery}A{\isachardoublequoteclose}\isanewline
\ \ \ \ \ \ \ \ \isacommand{by}\isamarkupfalse%
\ {\isacharparenleft}simp\ only{\isacharcolon}\ insert{\isacharunderscore}iff\ simp{\isacharunderscore}thms{\isacharparenright}\ \isanewline
\ \ \ \ \ \ \isacommand{have}\isamarkupfalse%
\ {\isachardoublequoteopen}\isactrlbold {\isasymnot}{\isacharparenleft}Atom\ k{\isacharparenright}\ {\isasymin}\ {\isacharquery}A{\isachardoublequoteclose}\isanewline
\ \ \ \ \ \ \ \ \isacommand{by}\isamarkupfalse%
\ {\isacharparenleft}simp\ only{\isacharcolon}\ insert{\isacharunderscore}iff\ simp{\isacharunderscore}thms{\isacharparenright}\ \isanewline
\ \ \ \ \ \ \isacommand{have}\isamarkupfalse%
\ inSubset{\isacharcolon}{\isachardoublequoteopen}insert\ {\isacharparenleft}\isactrlbold {\isasymnot}{\isacharparenleft}Atom\ k{\isacharparenright}{\isacharparenright}\ {\isacharbraceleft}{\isacharbraceright}\ {\isasymsubseteq}\ S{\isachardoublequoteclose}\isanewline
\ \ \ \ \ \ \ \ \isacommand{using}\isamarkupfalse%
\ {\isacartoucheopen}\isactrlbold {\isasymnot}{\isacharparenleft}Atom\ k{\isacharparenright}\ {\isasymin}\ S{\isacartoucheclose}\ {\isacartoucheopen}{\isacharbraceleft}{\isacharbraceright}\ {\isasymsubseteq}\ S{\isacartoucheclose}\ \isacommand{by}\isamarkupfalse%
\ {\isacharparenleft}simp\ only{\isacharcolon}\ insert{\isacharunderscore}subset{\isacharparenright}\isanewline
\ \ \ \ \ \ \isacommand{have}\isamarkupfalse%
\ {\isachardoublequoteopen}{\isacharquery}A\ {\isasymsubseteq}\ S{\isachardoublequoteclose}\isanewline
\ \ \ \ \ \ \ \ \isacommand{using}\isamarkupfalse%
\ inSubset\ {\isacartoucheopen}Atom\ k\ {\isasymin}\ S{\isacartoucheclose}\ \isacommand{by}\isamarkupfalse%
\ {\isacharparenleft}simp\ only{\isacharcolon}\ insert{\isacharunderscore}subset{\isacharparenright}\isanewline
\ \ \ \ \ \ \isacommand{have}\isamarkupfalse%
\ {\isachardoublequoteopen}finite\ {\isacharbraceleft}{\isacharbraceright}{\isachardoublequoteclose}\isanewline
\ \ \ \ \ \ \ \ \isacommand{by}\isamarkupfalse%
\ {\isacharparenleft}simp\ only{\isacharcolon}\ finite{\isachardot}emptyI{\isacharparenright}\isanewline
\ \ \ \ \ \ \isacommand{then}\isamarkupfalse%
\ \isacommand{have}\isamarkupfalse%
\ {\isachardoublequoteopen}finite\ {\isacharparenleft}insert\ {\isacharparenleft}\isactrlbold {\isasymnot}{\isacharparenleft}Atom\ k{\isacharparenright}{\isacharparenright}\ {\isacharbraceleft}{\isacharbraceright}{\isacharparenright}{\isachardoublequoteclose}\isanewline
\ \ \ \ \ \ \ \ \isacommand{by}\isamarkupfalse%
\ {\isacharparenleft}rule\ finite{\isachardot}insertI{\isacharparenright}\isanewline
\ \ \ \ \ \ \isacommand{then}\isamarkupfalse%
\ \isacommand{have}\isamarkupfalse%
\ {\isachardoublequoteopen}finite\ {\isacharquery}A{\isachardoublequoteclose}\isanewline
\ \ \ \ \ \ \ \ \isacommand{by}\isamarkupfalse%
\ {\isacharparenleft}rule\ finite{\isachardot}insertI{\isacharparenright}\isanewline
\ \ \ \ \ \ \isacommand{have}\isamarkupfalse%
\ {\isachardoublequoteopen}finite\ {\isacharquery}A\ {\isasymlongrightarrow}\ {\isacharquery}A\ {\isasymin}\ C{\isachardoublequoteclose}\isanewline
\ \ \ \ \ \ \ \ \isacommand{using}\isamarkupfalse%
\ E\ {\isacartoucheopen}{\isacharquery}A\ {\isasymsubseteq}\ S{\isacartoucheclose}\ \isacommand{by}\isamarkupfalse%
\ {\isacharparenleft}rule\ sspec{\isacharparenright}\isanewline
\ \ \ \ \ \ \isacommand{then}\isamarkupfalse%
\ \isacommand{have}\isamarkupfalse%
\ {\isachardoublequoteopen}{\isacharquery}A\ {\isasymin}\ C{\isachardoublequoteclose}\isanewline
\ \ \ \ \ \ \ \ \isacommand{using}\isamarkupfalse%
\ {\isacartoucheopen}finite\ {\isacharquery}A{\isacartoucheclose}\ \isacommand{by}\isamarkupfalse%
\ {\isacharparenleft}rule\ mp{\isacharparenright}\isanewline
\ \ \ \ \ \ \isacommand{have}\isamarkupfalse%
\ {\isachardoublequoteopen}{\isasymbottom}\ {\isasymnotin}\ {\isacharquery}A\isanewline
\ \ \ \ \ \ \ \ \ \ \ \ {\isasymand}\ {\isacharparenleft}{\isasymforall}k{\isachardot}\ Atom\ k\ {\isasymin}\ {\isacharquery}A\ {\isasymlongrightarrow}\ \isactrlbold {\isasymnot}\ {\isacharparenleft}Atom\ k{\isacharparenright}\ {\isasymin}\ {\isacharquery}A\ {\isasymlongrightarrow}\ False{\isacharparenright}\isanewline
\ \ \ \ \ \ \ \ \ \ \ \ {\isasymand}\ {\isacharparenleft}{\isasymforall}F\ G\ H{\isachardot}\ Con\ F\ G\ H\ {\isasymlongrightarrow}\ F\ {\isasymin}\ {\isacharquery}A\ {\isasymlongrightarrow}\ {\isacharbraceleft}G{\isacharcomma}H{\isacharbraceright}\ {\isasymunion}\ {\isacharquery}A\ {\isasymin}\ C{\isacharparenright}\isanewline
\ \ \ \ \ \ \ \ \ \ \ \ {\isasymand}\ {\isacharparenleft}{\isasymforall}F\ G\ H{\isachardot}\ Dis\ F\ G\ H\ {\isasymlongrightarrow}\ F\ {\isasymin}\ {\isacharquery}A\ {\isasymlongrightarrow}\ {\isacharbraceleft}G{\isacharbraceright}\ {\isasymunion}\ {\isacharquery}A\ {\isasymin}\ C\ {\isasymor}\ {\isacharbraceleft}H{\isacharbraceright}\ {\isasymunion}\ {\isacharquery}A\ {\isasymin}\ C{\isacharparenright}{\isachardoublequoteclose}\isanewline
\ \ \ \ \ \ \ \ \isacommand{using}\isamarkupfalse%
\ PCP\ {\isacartoucheopen}{\isacharquery}A\ {\isasymin}\ C{\isacartoucheclose}\ \isacommand{by}\isamarkupfalse%
\ {\isacharparenleft}rule\ bspec{\isacharparenright}\isanewline
\ \ \ \ \ \ \isacommand{then}\isamarkupfalse%
\ \isacommand{have}\isamarkupfalse%
\ {\isachardoublequoteopen}{\isasymforall}k{\isachardot}\ Atom\ k\ {\isasymin}\ {\isacharquery}A\ {\isasymlongrightarrow}\ \isactrlbold {\isasymnot}\ {\isacharparenleft}Atom\ k{\isacharparenright}\ {\isasymin}\ {\isacharquery}A\ {\isasymlongrightarrow}\ False{\isachardoublequoteclose}\isanewline
\ \ \ \ \ \ \ \ \isacommand{by}\isamarkupfalse%
\ {\isacharparenleft}iprover\ elim{\isacharcolon}\ conjunct{\isadigit{2}}\ conjunct{\isadigit{1}}{\isacharparenright}\isanewline
\ \ \ \ \ \ \isacommand{then}\isamarkupfalse%
\ \isacommand{have}\isamarkupfalse%
\ {\isachardoublequoteopen}Atom\ k\ {\isasymin}\ {\isacharquery}A\ {\isasymlongrightarrow}\ \isactrlbold {\isasymnot}\ {\isacharparenleft}Atom\ k{\isacharparenright}\ {\isasymin}\ {\isacharquery}A\ {\isasymlongrightarrow}\ False{\isachardoublequoteclose}\isanewline
\ \ \ \ \ \ \ \ \isacommand{by}\isamarkupfalse%
\ {\isacharparenleft}iprover\ elim{\isacharcolon}\ allE{\isacharparenright}\isanewline
\ \ \ \ \ \ \isacommand{then}\isamarkupfalse%
\ \isacommand{have}\isamarkupfalse%
\ {\isachardoublequoteopen}\isactrlbold {\isasymnot}{\isacharparenleft}Atom\ k{\isacharparenright}\ {\isasymin}\ {\isacharquery}A\ {\isasymlongrightarrow}\ False{\isachardoublequoteclose}\isanewline
\ \ \ \ \ \ \ \ \isacommand{using}\isamarkupfalse%
\ {\isacartoucheopen}Atom\ k\ {\isasymin}\ {\isacharquery}A{\isacartoucheclose}\ \isacommand{by}\isamarkupfalse%
\ {\isacharparenleft}rule\ mp{\isacharparenright}\isanewline
\ \ \ \ \ \ \isacommand{thus}\isamarkupfalse%
\ {\isachardoublequoteopen}False{\isachardoublequoteclose}\isanewline
\ \ \ \ \ \ \ \ \isacommand{using}\isamarkupfalse%
\ {\isacartoucheopen}\isactrlbold {\isasymnot}{\isacharparenleft}Atom\ k{\isacharparenright}\ {\isasymin}\ {\isacharquery}A{\isacartoucheclose}\ \isacommand{by}\isamarkupfalse%
\ {\isacharparenleft}rule\ mp{\isacharparenright}\isanewline
\ \ \ \ \isacommand{qed}\isamarkupfalse%
\isanewline
\ \ \isacommand{qed}\isamarkupfalse%
\isanewline
\ \ \isacommand{have}\isamarkupfalse%
\ C{\isadigit{3}}{\isacharcolon}{\isachardoublequoteopen}{\isasymforall}F\ G\ H{\isachardot}\ Con\ F\ G\ H\ {\isasymlongrightarrow}\ F\ {\isasymin}\ S\ {\isasymlongrightarrow}\ {\isacharbraceleft}G{\isacharcomma}H{\isacharbraceright}\ {\isasymunion}\ S\ {\isasymin}\ {\isacharparenleft}extensionFin\ C{\isacharparenright}{\isachardoublequoteclose}\isanewline
\ \ \isacommand{proof}\isamarkupfalse%
\ {\isacharparenleft}rule\ allI{\isacharparenright}{\isacharplus}\isanewline
\ \ \ \ \isacommand{fix}\isamarkupfalse%
\ F\ G\ H\isanewline
\ \ \ \ \isacommand{show}\isamarkupfalse%
\ {\isachardoublequoteopen}Con\ F\ G\ H\ {\isasymlongrightarrow}\ F\ {\isasymin}\ S\ {\isasymlongrightarrow}\ {\isacharbraceleft}G{\isacharcomma}H{\isacharbraceright}\ {\isasymunion}\ S\ {\isasymin}\ {\isacharparenleft}extensionFin\ C{\isacharparenright}{\isachardoublequoteclose}\isanewline
\ \ \ \ \isacommand{proof}\isamarkupfalse%
\ {\isacharparenleft}rule\ impI{\isacharparenright}{\isacharplus}\isanewline
\ \ \ \ \ \ \isacommand{assume}\isamarkupfalse%
\ {\isachardoublequoteopen}Con\ F\ G\ H{\isachardoublequoteclose}\isanewline
\ \ \ \ \ \ \isacommand{assume}\isamarkupfalse%
\ {\isachardoublequoteopen}F\ {\isasymin}\ S{\isachardoublequoteclose}\ \isanewline
\ \ \ \ \ \ \isacommand{show}\isamarkupfalse%
\ {\isachardoublequoteopen}{\isacharbraceleft}G{\isacharcomma}H{\isacharbraceright}\ {\isasymunion}\ S\ {\isasymin}\ {\isacharparenleft}extensionFin\ C{\isacharparenright}{\isachardoublequoteclose}\ \isanewline
\ \ \ \ \ \ \ \ \isacommand{using}\isamarkupfalse%
\ assms{\isacharparenleft}{\isadigit{1}}{\isacharparenright}\ assms{\isacharparenleft}{\isadigit{2}}{\isacharparenright}\ assms{\isacharparenleft}{\isadigit{3}}{\isacharparenright}\ {\isacartoucheopen}Con\ F\ G\ H{\isacartoucheclose}\ {\isacartoucheopen}F\ {\isasymin}\ S{\isacartoucheclose}\ \isacommand{by}\isamarkupfalse%
\ {\isacharparenleft}simp\ only{\isacharcolon}\ ex{\isadigit{3}}{\isacharunderscore}pcp{\isacharunderscore}SinE{\isacharunderscore}CON{\isacharparenright}\isanewline
\ \ \ \ \isacommand{qed}\isamarkupfalse%
\isanewline
\ \ \isacommand{qed}\isamarkupfalse%
\isanewline
\ \ \isacommand{have}\isamarkupfalse%
\ C{\isadigit{4}}{\isacharcolon}{\isachardoublequoteopen}{\isasymforall}F\ G\ H{\isachardot}\ Dis\ F\ G\ H\ {\isasymlongrightarrow}\ F\ {\isasymin}\ S\ {\isasymlongrightarrow}\ {\isacharbraceleft}G{\isacharbraceright}\ {\isasymunion}\ S\ {\isasymin}\ {\isacharparenleft}extensionFin\ C{\isacharparenright}\ {\isasymor}\ {\isacharbraceleft}H{\isacharbraceright}\ {\isasymunion}\ S\ {\isasymin}\ {\isacharparenleft}extensionFin\ C{\isacharparenright}{\isachardoublequoteclose}\isanewline
\ \ \isacommand{proof}\isamarkupfalse%
\ {\isacharparenleft}rule\ allI{\isacharparenright}{\isacharplus}\isanewline
\ \ \ \ \isacommand{fix}\isamarkupfalse%
\ F\ G\ H\isanewline
\ \ \ \ \isacommand{show}\isamarkupfalse%
\ {\isachardoublequoteopen}Dis\ F\ G\ H\ {\isasymlongrightarrow}\ F\ {\isasymin}\ S\ {\isasymlongrightarrow}\ {\isacharbraceleft}G{\isacharbraceright}\ {\isasymunion}\ S\ {\isasymin}\ {\isacharparenleft}extensionFin\ C{\isacharparenright}\ {\isasymor}\ {\isacharbraceleft}H{\isacharbraceright}\ {\isasymunion}\ S\ {\isasymin}\ {\isacharparenleft}extensionFin\ C{\isacharparenright}{\isachardoublequoteclose}\isanewline
\ \ \ \ \isacommand{proof}\isamarkupfalse%
\ {\isacharparenleft}rule\ impI{\isacharparenright}{\isacharplus}\isanewline
\ \ \ \ \ \ \isacommand{assume}\isamarkupfalse%
\ {\isachardoublequoteopen}Dis\ F\ G\ H{\isachardoublequoteclose}\isanewline
\ \ \ \ \ \ \isacommand{assume}\isamarkupfalse%
\ {\isachardoublequoteopen}F\ {\isasymin}\ S{\isachardoublequoteclose}\ \isanewline
\ \ \ \ \ \ \isacommand{show}\isamarkupfalse%
\ {\isachardoublequoteopen}{\isacharbraceleft}G{\isacharbraceright}\ {\isasymunion}\ S\ {\isasymin}\ {\isacharparenleft}extensionFin\ C{\isacharparenright}\ {\isasymor}\ {\isacharbraceleft}H{\isacharbraceright}\ {\isasymunion}\ S\ {\isasymin}\ {\isacharparenleft}extensionFin\ C{\isacharparenright}{\isachardoublequoteclose}\isanewline
\ \ \ \ \ \ \ \ \isacommand{using}\isamarkupfalse%
\ assms{\isacharparenleft}{\isadigit{1}}{\isacharparenright}\ assms{\isacharparenleft}{\isadigit{2}}{\isacharparenright}\ assms{\isacharparenleft}{\isadigit{3}}{\isacharparenright}\ {\isacartoucheopen}Dis\ F\ G\ H{\isacartoucheclose}\ {\isacartoucheopen}F\ {\isasymin}\ S{\isacartoucheclose}\ \isacommand{by}\isamarkupfalse%
\ {\isacharparenleft}rule\ ex{\isadigit{3}}{\isacharunderscore}pcp{\isacharunderscore}SinE{\isacharunderscore}DIS{\isacharparenright}\isanewline
\ \ \ \ \isacommand{qed}\isamarkupfalse%
\isanewline
\ \ \isacommand{qed}\isamarkupfalse%
\isanewline
\ \ \isacommand{show}\isamarkupfalse%
\ {\isacharquery}thesis\isanewline
\ \ \ \ \isacommand{using}\isamarkupfalse%
\ C{\isadigit{1}}\ C{\isadigit{2}}\ C{\isadigit{3}}\ C{\isadigit{4}}\ \isacommand{by}\isamarkupfalse%
\ {\isacharparenleft}iprover\ intro{\isacharcolon}\ conjI{\isacharparenright}\isanewline
\isacommand{qed}\isamarkupfalse%
%
\endisatagproof
{\isafoldproof}%
%
\isadelimproof
%
\endisadelimproof
%
\begin{isamarkuptext}%
En conclusión, la prueba detallada completa en Isabelle que demuestra que la extensión \isa{C{\isacharprime}} 
  verifica la propiedad de consistencia proposicional dada una colección \isa{C} que también la
  verifique y sea cerrada bajo subconjuntos es la siguiente.%
\end{isamarkuptext}\isamarkuptrue%
\isacommand{lemma}\isamarkupfalse%
\ ex{\isadigit{3}}{\isacharunderscore}pcp{\isacharcolon}\isanewline
\ \ \isakeyword{assumes}\ {\isachardoublequoteopen}pcp\ C{\isachardoublequoteclose}\isanewline
\ \ \ \ \ \ \ \ \ \ {\isachardoublequoteopen}subset{\isacharunderscore}closed\ C{\isachardoublequoteclose}\isanewline
\ \ \ \ \ \ \ \ \isakeyword{shows}\ {\isachardoublequoteopen}pcp\ {\isacharparenleft}extensionFin\ C{\isacharparenright}{\isachardoublequoteclose}\isanewline
%
\isadelimproof
\ \ %
\endisadelimproof
%
\isatagproof
\isacommand{unfolding}\isamarkupfalse%
\ pcp{\isacharunderscore}alt\isanewline
\isacommand{proof}\isamarkupfalse%
\ {\isacharparenleft}rule\ ballI{\isacharparenright}\isanewline
\ \ \isacommand{have}\isamarkupfalse%
\ PCP{\isacharcolon}{\isachardoublequoteopen}{\isasymforall}S\ {\isasymin}\ C{\isachardot}\isanewline
\ \ \ \ {\isasymbottom}\ {\isasymnotin}\ S\isanewline
\ \ \ \ {\isasymand}\ {\isacharparenleft}{\isasymforall}k{\isachardot}\ Atom\ k\ {\isasymin}\ S\ {\isasymlongrightarrow}\ \isactrlbold {\isasymnot}\ {\isacharparenleft}Atom\ k{\isacharparenright}\ {\isasymin}\ S\ {\isasymlongrightarrow}\ False{\isacharparenright}\isanewline
\ \ \ \ {\isasymand}\ {\isacharparenleft}{\isasymforall}F\ G\ H{\isachardot}\ Con\ F\ G\ H\ {\isasymlongrightarrow}\ F\ {\isasymin}\ S\ {\isasymlongrightarrow}\ {\isacharbraceleft}G{\isacharcomma}H{\isacharbraceright}\ {\isasymunion}\ S\ {\isasymin}\ C{\isacharparenright}\isanewline
\ \ \ \ {\isasymand}\ {\isacharparenleft}{\isasymforall}F\ G\ H{\isachardot}\ Dis\ F\ G\ H\ {\isasymlongrightarrow}\ F\ {\isasymin}\ S\ {\isasymlongrightarrow}\ {\isacharbraceleft}G{\isacharbraceright}\ {\isasymunion}\ S\ {\isasymin}\ C\ {\isasymor}\ {\isacharbraceleft}H{\isacharbraceright}\ {\isasymunion}\ S\ {\isasymin}\ C{\isacharparenright}{\isachardoublequoteclose}\isanewline
\ \ \ \ \isacommand{using}\isamarkupfalse%
\ assms{\isacharparenleft}{\isadigit{1}}{\isacharparenright}\ \isacommand{by}\isamarkupfalse%
\ {\isacharparenleft}rule\ pcp{\isacharunderscore}alt{\isadigit{1}}{\isacharparenright}\isanewline
\ \ \isacommand{fix}\isamarkupfalse%
\ S\isanewline
\ \ \isacommand{assume}\isamarkupfalse%
\ {\isachardoublequoteopen}S\ {\isasymin}\ {\isacharparenleft}extensionFin\ C{\isacharparenright}{\isachardoublequoteclose}\isanewline
\ \ \isacommand{then}\isamarkupfalse%
\ \isacommand{have}\isamarkupfalse%
\ {\isachardoublequoteopen}S\ {\isasymin}\ C\ {\isasymor}\ S\ {\isasymin}\ {\isacharparenleft}extF\ C{\isacharparenright}{\isachardoublequoteclose}\isanewline
\ \ \ \ \isacommand{unfolding}\isamarkupfalse%
\ extensionFin\ \isacommand{by}\isamarkupfalse%
\ {\isacharparenleft}simp\ only{\isacharcolon}\ Un{\isacharunderscore}iff{\isacharparenright}\isanewline
\ \ \isacommand{thus}\isamarkupfalse%
\ {\isachardoublequoteopen}{\isasymbottom}\ {\isasymnotin}\ S\ {\isasymand}\isanewline
\ \ \ \ \ \ \ \ \ {\isacharparenleft}{\isasymforall}k{\isachardot}\ Atom\ k\ {\isasymin}\ S\ {\isasymlongrightarrow}\ \isactrlbold {\isasymnot}\ {\isacharparenleft}Atom\ k{\isacharparenright}\ {\isasymin}\ S\ {\isasymlongrightarrow}\ False{\isacharparenright}\ {\isasymand}\isanewline
\ \ \ \ \ \ \ \ \ {\isacharparenleft}{\isasymforall}F\ G\ H{\isachardot}\ Con\ F\ G\ H\ {\isasymlongrightarrow}\ F\ {\isasymin}\ S\ {\isasymlongrightarrow}\ {\isacharbraceleft}G{\isacharcomma}\ H{\isacharbraceright}\ {\isasymunion}\ S\ {\isasymin}\ {\isacharparenleft}extensionFin\ C{\isacharparenright}{\isacharparenright}\ {\isasymand}\isanewline
\ \ \ \ \ \ \ \ \ {\isacharparenleft}{\isasymforall}F\ G\ H{\isachardot}\ Dis\ F\ G\ H\ {\isasymlongrightarrow}\ F\ {\isasymin}\ S\ {\isasymlongrightarrow}\ {\isacharbraceleft}G{\isacharbraceright}\ {\isasymunion}\ S\ {\isasymin}\ {\isacharparenleft}extensionFin\ C{\isacharparenright}\ {\isasymor}\ {\isacharbraceleft}H{\isacharbraceright}\ {\isasymunion}\ S\ {\isasymin}\ {\isacharparenleft}extensionFin\ C{\isacharparenright}{\isacharparenright}{\isachardoublequoteclose}\isanewline
\ \ \isacommand{proof}\isamarkupfalse%
\ {\isacharparenleft}rule\ disjE{\isacharparenright}\isanewline
\ \ \ \ \isacommand{assume}\isamarkupfalse%
\ {\isachardoublequoteopen}S\ {\isasymin}\ C{\isachardoublequoteclose}\isanewline
\ \ \ \ \isacommand{show}\isamarkupfalse%
\ {\isacharquery}thesis\isanewline
\ \ \ \ \ \ \isacommand{using}\isamarkupfalse%
\ assms\ {\isacartoucheopen}S\ {\isasymin}\ C{\isacartoucheclose}\ \isacommand{by}\isamarkupfalse%
\ {\isacharparenleft}rule\ ex{\isadigit{3}}{\isacharunderscore}pcp{\isacharunderscore}SinC{\isacharparenright}\isanewline
\ \ \isacommand{next}\isamarkupfalse%
\isanewline
\ \ \ \ \isacommand{assume}\isamarkupfalse%
\ {\isachardoublequoteopen}S\ {\isasymin}\ {\isacharparenleft}extF\ C{\isacharparenright}{\isachardoublequoteclose}\isanewline
\ \ \ \ \isacommand{show}\isamarkupfalse%
\ {\isacharquery}thesis\isanewline
\ \ \ \ \ \ \isacommand{using}\isamarkupfalse%
\ assms\ {\isacartoucheopen}S\ {\isasymin}\ {\isacharparenleft}extF\ C{\isacharparenright}{\isacartoucheclose}\ \isacommand{by}\isamarkupfalse%
\ {\isacharparenleft}rule\ ex{\isadigit{3}}{\isacharunderscore}pcp{\isacharunderscore}SinE{\isacharparenright}\isanewline
\ \ \isacommand{qed}\isamarkupfalse%
\isanewline
\isacommand{qed}\isamarkupfalse%
%
\endisatagproof
{\isafoldproof}%
%
\isadelimproof
%
\endisadelimproof
%
\begin{isamarkuptext}%
Por último, podemos dar la prueba completa del lema \isa{{\isadigit{3}}{\isachardot}{\isadigit{0}}{\isachardot}{\isadigit{5}}} en Isabelle.%
\end{isamarkuptext}\isamarkuptrue%
\isacommand{lemma}\isamarkupfalse%
\ ex{\isadigit{3}}{\isacharcolon}\isanewline
\ \ \isakeyword{assumes}\ {\isachardoublequoteopen}pcp\ C{\isachardoublequoteclose}\isanewline
\ \ \ \ \ \ \ \ \ \ {\isachardoublequoteopen}subset{\isacharunderscore}closed\ C{\isachardoublequoteclose}\isanewline
\ \ \isakeyword{shows}\ {\isachardoublequoteopen}{\isasymexists}C{\isacharprime}{\isachardot}\ C\ {\isasymsubseteq}\ C{\isacharprime}\ {\isasymand}\ pcp\ C{\isacharprime}\ {\isasymand}\ finite{\isacharunderscore}character\ C{\isacharprime}{\isachardoublequoteclose}\isanewline
%
\isadelimproof
%
\endisadelimproof
%
\isatagproof
\isacommand{proof}\isamarkupfalse%
\ {\isacharminus}\isanewline
\ \ \isacommand{let}\isamarkupfalse%
\ {\isacharquery}C{\isacharprime}{\isacharequal}{\isachardoublequoteopen}extensionFin\ C{\isachardoublequoteclose}\isanewline
\ \ \isacommand{have}\isamarkupfalse%
\ C{\isadigit{1}}{\isacharcolon}{\isachardoublequoteopen}C\ {\isasymsubseteq}\ {\isacharquery}C{\isacharprime}{\isachardoublequoteclose}\isanewline
\ \ \ \ \isacommand{unfolding}\isamarkupfalse%
\ extensionFin\ \isacommand{by}\isamarkupfalse%
\ {\isacharparenleft}simp\ only{\isacharcolon}\ Un{\isacharunderscore}upper{\isadigit{1}}{\isacharparenright}\isanewline
\ \ \isacommand{have}\isamarkupfalse%
\ C{\isadigit{2}}{\isacharcolon}{\isachardoublequoteopen}finite{\isacharunderscore}character\ {\isacharparenleft}{\isacharquery}C{\isacharprime}{\isacharparenright}{\isachardoublequoteclose}\isanewline
\ \ \ \ \isacommand{using}\isamarkupfalse%
\ assms{\isacharparenleft}{\isadigit{2}}{\isacharparenright}\ \isacommand{by}\isamarkupfalse%
\ {\isacharparenleft}rule\ ex{\isadigit{3}}{\isacharunderscore}finite{\isacharunderscore}character{\isacharparenright}\isanewline
\ \ \isacommand{have}\isamarkupfalse%
\ C{\isadigit{3}}{\isacharcolon}{\isachardoublequoteopen}pcp\ {\isacharparenleft}{\isacharquery}C{\isacharprime}{\isacharparenright}{\isachardoublequoteclose}\isanewline
\ \ \ \ \isacommand{using}\isamarkupfalse%
\ assms\ \isacommand{by}\isamarkupfalse%
\ {\isacharparenleft}rule\ ex{\isadigit{3}}{\isacharunderscore}pcp{\isacharparenright}\isanewline
\ \ \isacommand{have}\isamarkupfalse%
\ {\isachardoublequoteopen}C\ {\isasymsubseteq}\ {\isacharquery}C{\isacharprime}\ {\isasymand}\ pcp\ {\isacharquery}C{\isacharprime}\ {\isasymand}\ finite{\isacharunderscore}character\ {\isacharquery}C{\isacharprime}{\isachardoublequoteclose}\isanewline
\ \ \ \ \isacommand{using}\isamarkupfalse%
\ C{\isadigit{1}}\ C{\isadigit{2}}\ C{\isadigit{3}}\ \isacommand{by}\isamarkupfalse%
\ {\isacharparenleft}iprover\ intro{\isacharcolon}\ conjI{\isacharparenright}\isanewline
\ \ \isacommand{thus}\isamarkupfalse%
\ {\isacharquery}thesis\isanewline
\ \ \ \ \isacommand{by}\isamarkupfalse%
\ {\isacharparenleft}rule\ exI{\isacharparenright}\isanewline
\isacommand{qed}\isamarkupfalse%
\isanewline
%
\endisatagproof
{\isafoldproof}%
%
\isadelimproof
%
\endisadelimproof
%
\isadelimtheory
%
\endisadelimtheory
%
\isatagtheory
%
\endisatagtheory
{\isafoldtheory}%
%
\isadelimtheory
%
\endisadelimtheory
%
\end{isabellebody}%
\endinput
%:%file=~/TFM/TFM/ColecScfc.thy%:%
%:%19=12%:%
%:%20=13%:%
%:%21=14%:%
%:%22=15%:%
%:%23=16%:%
%:%24=17%:%
%:%25=18%:%
%:%26=19%:%
%:%27=20%:%
%:%28=21%:%
%:%29=22%:%
%:%30=23%:%
%:%31=24%:%
%:%32=25%:%
%:%33=26%:%
%:%35=28%:%
%:%36=28%:%
%:%38=30%:%
%:%39=31%:%
%:%40=32%:%
%:%42=34%:%
%:%43=34%:%
%:%46=35%:%
%:%50=35%:%
%:%51=35%:%
%:%52=35%:%
%:%57=35%:%
%:%60=36%:%
%:%61=37%:%
%:%62=37%:%
%:%65=38%:%
%:%69=38%:%
%:%70=38%:%
%:%71=38%:%
%:%80=40%:%
%:%81=41%:%
%:%82=42%:%
%:%84=44%:%
%:%85=44%:%
%:%88=45%:%
%:%92=45%:%
%:%93=45%:%
%:%94=45%:%
%:%103=47%:%
%:%104=48%:%
%:%106=50%:%
%:%107=50%:%
%:%109=52%:%
%:%112=53%:%
%:%116=53%:%
%:%117=53%:%
%:%118=53%:%
%:%123=53%:%
%:%126=54%:%
%:%127=55%:%
%:%128=55%:%
%:%129=56%:%
%:%132=57%:%
%:%136=57%:%
%:%137=57%:%
%:%138=57%:%
%:%147=59%:%
%:%148=60%:%
%:%149=61%:%
%:%150=62%:%
%:%151=63%:%
%:%152=64%:%
%:%153=65%:%
%:%154=66%:%
%:%155=67%:%
%:%156=68%:%
%:%157=69%:%
%:%158=70%:%
%:%160=72%:%
%:%161=72%:%
%:%164=75%:%
%:%165=76%:%
%:%166=77%:%
%:%167=78%:%
%:%169=80%:%
%:%170=80%:%
%:%173=81%:%
%:%177=81%:%
%:%178=81%:%
%:%179=81%:%
%:%184=81%:%
%:%187=82%:%
%:%188=83%:%
%:%189=83%:%
%:%192=84%:%
%:%196=84%:%
%:%197=84%:%
%:%198=84%:%
%:%203=84%:%
%:%206=85%:%
%:%207=86%:%
%:%208=86%:%
%:%211=87%:%
%:%215=87%:%
%:%216=87%:%
%:%217=87%:%
%:%226=89%:%
%:%227=90%:%
%:%228=91%:%
%:%229=92%:%
%:%230=93%:%
%:%231=94%:%
%:%232=95%:%
%:%233=96%:%
%:%234=97%:%
%:%235=98%:%
%:%236=99%:%
%:%237=100%:%
%:%238=101%:%
%:%239=102%:%
%:%240=103%:%
%:%241=104%:%
%:%243=106%:%
%:%244=106%:%
%:%247=107%:%
%:%251=107%:%
%:%261=109%:%
%:%262=110%:%
%:%263=111%:%
%:%264=112%:%
%:%265=113%:%
%:%266=114%:%
%:%267=115%:%
%:%268=116%:%
%:%269=117%:%
%:%270=118%:%
%:%271=119%:%
%:%272=120%:%
%:%273=121%:%
%:%274=122%:%
%:%275=123%:%
%:%276=124%:%
%:%277=125%:%
%:%278=126%:%
%:%279=127%:%
%:%280=128%:%
%:%281=129%:%
%:%282=130%:%
%:%283=131%:%
%:%284=132%:%
%:%285=133%:%
%:%286=134%:%
%:%287=135%:%
%:%288=136%:%
%:%289=137%:%
%:%290=138%:%
%:%291=139%:%
%:%292=140%:%
%:%293=141%:%
%:%294=142%:%
%:%295=143%:%
%:%296=144%:%
%:%297=145%:%
%:%298=146%:%
%:%299=147%:%
%:%300=148%:%
%:%301=149%:%
%:%302=150%:%
%:%303=151%:%
%:%304=152%:%
%:%305=153%:%
%:%306=154%:%
%:%307=155%:%
%:%308=156%:%
%:%309=157%:%
%:%310=158%:%
%:%311=159%:%
%:%312=160%:%
%:%313=161%:%
%:%314=162%:%
%:%315=163%:%
%:%316=164%:%
%:%317=165%:%
%:%318=166%:%
%:%319=167%:%
%:%320=168%:%
%:%321=169%:%
%:%322=170%:%
%:%323=171%:%
%:%324=172%:%
%:%325=173%:%
%:%326=174%:%
%:%327=175%:%
%:%328=176%:%
%:%329=177%:%
%:%330=178%:%
%:%331=179%:%
%:%332=180%:%
%:%333=181%:%
%:%334=182%:%
%:%335=183%:%
%:%336=184%:%
%:%337=185%:%
%:%338=186%:%
%:%339=187%:%
%:%340=188%:%
%:%341=189%:%
%:%342=190%:%
%:%343=191%:%
%:%344=192%:%
%:%345=193%:%
%:%346=194%:%
%:%347=195%:%
%:%348=196%:%
%:%349=197%:%
%:%350=198%:%
%:%351=199%:%
%:%352=200%:%
%:%354=202%:%
%:%355=202%:%
%:%358=203%:%
%:%362=203%:%
%:%363=203%:%
%:%372=205%:%
%:%373=206%:%
%:%375=208%:%
%:%376=208%:%
%:%379=209%:%
%:%383=209%:%
%:%384=209%:%
%:%393=211%:%
%:%394=212%:%
%:%395=213%:%
%:%396=214%:%
%:%397=215%:%
%:%399=217%:%
%:%400=217%:%
%:%401=218%:%
%:%403=220%:%
%:%404=221%:%
%:%406=223%:%
%:%407=223%:%
%:%414=224%:%
%:%415=224%:%
%:%416=225%:%
%:%417=225%:%
%:%418=226%:%
%:%419=226%:%
%:%420=227%:%
%:%421=227%:%
%:%422=228%:%
%:%423=228%:%
%:%424=229%:%
%:%425=229%:%
%:%426=229%:%
%:%427=230%:%
%:%428=230%:%
%:%429=230%:%
%:%430=231%:%
%:%431=231%:%
%:%432=232%:%
%:%433=232%:%
%:%434=233%:%
%:%444=235%:%
%:%445=236%:%
%:%446=237%:%
%:%447=238%:%
%:%449=240%:%
%:%450=240%:%
%:%453=241%:%
%:%457=241%:%
%:%458=241%:%
%:%467=243%:%
%:%468=244%:%
%:%470=246%:%
%:%471=246%:%
%:%472=247%:%
%:%473=248%:%
%:%480=249%:%
%:%481=249%:%
%:%482=250%:%
%:%483=250%:%
%:%484=251%:%
%:%485=251%:%
%:%486=252%:%
%:%487=252%:%
%:%488=253%:%
%:%489=253%:%
%:%490=254%:%
%:%491=254%:%
%:%494=257%:%
%:%495=258%:%
%:%496=258%:%
%:%497=259%:%
%:%498=259%:%
%:%499=260%:%
%:%500=260%:%
%:%501=261%:%
%:%502=261%:%
%:%503=261%:%
%:%504=262%:%
%:%505=262%:%
%:%506=262%:%
%:%507=263%:%
%:%508=263%:%
%:%509=264%:%
%:%510=264%:%
%:%511=264%:%
%:%512=265%:%
%:%513=265%:%
%:%517=269%:%
%:%518=270%:%
%:%519=270%:%
%:%520=270%:%
%:%521=271%:%
%:%522=271%:%
%:%523=271%:%
%:%526=274%:%
%:%527=275%:%
%:%528=275%:%
%:%529=275%:%
%:%530=276%:%
%:%531=276%:%
%:%532=276%:%
%:%533=277%:%
%:%534=277%:%
%:%535=278%:%
%:%536=278%:%
%:%537=279%:%
%:%538=279%:%
%:%539=279%:%
%:%540=280%:%
%:%541=280%:%
%:%542=281%:%
%:%543=281%:%
%:%544=281%:%
%:%545=282%:%
%:%546=282%:%
%:%547=283%:%
%:%548=283%:%
%:%549=284%:%
%:%550=284%:%
%:%551=285%:%
%:%552=285%:%
%:%553=286%:%
%:%554=286%:%
%:%555=287%:%
%:%556=287%:%
%:%557=288%:%
%:%558=288%:%
%:%559=289%:%
%:%560=289%:%
%:%561=290%:%
%:%562=290%:%
%:%563=290%:%
%:%564=291%:%
%:%565=291%:%
%:%566=292%:%
%:%567=292%:%
%:%568=292%:%
%:%569=293%:%
%:%570=293%:%
%:%571=294%:%
%:%572=294%:%
%:%573=294%:%
%:%574=295%:%
%:%575=295%:%
%:%576=295%:%
%:%577=296%:%
%:%578=296%:%
%:%579=296%:%
%:%580=297%:%
%:%581=297%:%
%:%582=298%:%
%:%583=298%:%
%:%584=298%:%
%:%585=299%:%
%:%586=299%:%
%:%587=300%:%
%:%588=300%:%
%:%589=301%:%
%:%590=301%:%
%:%591=302%:%
%:%592=302%:%
%:%593=302%:%
%:%594=303%:%
%:%595=303%:%
%:%596=304%:%
%:%597=304%:%
%:%598=305%:%
%:%599=305%:%
%:%600=306%:%
%:%601=306%:%
%:%602=307%:%
%:%603=307%:%
%:%604=308%:%
%:%605=308%:%
%:%606=309%:%
%:%607=309%:%
%:%608=310%:%
%:%609=310%:%
%:%610=311%:%
%:%611=311%:%
%:%612=311%:%
%:%613=312%:%
%:%614=312%:%
%:%615=313%:%
%:%616=313%:%
%:%617=313%:%
%:%618=314%:%
%:%619=314%:%
%:%620=314%:%
%:%621=315%:%
%:%622=315%:%
%:%623=315%:%
%:%624=316%:%
%:%625=316%:%
%:%626=316%:%
%:%627=317%:%
%:%628=317%:%
%:%629=317%:%
%:%630=318%:%
%:%631=318%:%
%:%632=319%:%
%:%633=319%:%
%:%634=319%:%
%:%635=320%:%
%:%636=320%:%
%:%637=320%:%
%:%638=321%:%
%:%639=321%:%
%:%640=322%:%
%:%641=322%:%
%:%642=322%:%
%:%643=323%:%
%:%644=323%:%
%:%645=324%:%
%:%646=324%:%
%:%647=324%:%
%:%648=325%:%
%:%649=325%:%
%:%650=326%:%
%:%651=326%:%
%:%652=327%:%
%:%653=327%:%
%:%654=328%:%
%:%655=328%:%
%:%656=329%:%
%:%657=329%:%
%:%658=330%:%
%:%659=330%:%
%:%660=331%:%
%:%661=331%:%
%:%662=331%:%
%:%663=332%:%
%:%664=332%:%
%:%665=332%:%
%:%666=333%:%
%:%667=333%:%
%:%668=333%:%
%:%669=334%:%
%:%670=334%:%
%:%671=335%:%
%:%672=335%:%
%:%673=335%:%
%:%674=336%:%
%:%675=336%:%
%:%676=337%:%
%:%677=337%:%
%:%678=338%:%
%:%679=338%:%
%:%680=339%:%
%:%681=339%:%
%:%682=339%:%
%:%683=340%:%
%:%684=340%:%
%:%685=341%:%
%:%686=341%:%
%:%687=342%:%
%:%688=342%:%
%:%689=343%:%
%:%690=343%:%
%:%691=344%:%
%:%692=344%:%
%:%693=345%:%
%:%694=345%:%
%:%695=346%:%
%:%696=346%:%
%:%697=347%:%
%:%698=347%:%
%:%699=348%:%
%:%700=348%:%
%:%701=348%:%
%:%702=349%:%
%:%703=349%:%
%:%704=350%:%
%:%705=350%:%
%:%706=350%:%
%:%707=351%:%
%:%708=351%:%
%:%709=351%:%
%:%710=352%:%
%:%711=352%:%
%:%712=352%:%
%:%713=353%:%
%:%714=353%:%
%:%715=353%:%
%:%716=354%:%
%:%717=354%:%
%:%718=354%:%
%:%719=355%:%
%:%720=355%:%
%:%721=356%:%
%:%722=356%:%
%:%723=357%:%
%:%724=357%:%
%:%725=358%:%
%:%726=358%:%
%:%727=359%:%
%:%728=359%:%
%:%729=360%:%
%:%730=360%:%
%:%731=360%:%
%:%732=361%:%
%:%733=361%:%
%:%734=362%:%
%:%735=362%:%
%:%736=363%:%
%:%737=363%:%
%:%738=364%:%
%:%739=364%:%
%:%740=365%:%
%:%741=365%:%
%:%742=366%:%
%:%743=366%:%
%:%744=366%:%
%:%745=367%:%
%:%746=367%:%
%:%747=367%:%
%:%748=368%:%
%:%749=368%:%
%:%750=368%:%
%:%751=369%:%
%:%752=369%:%
%:%753=369%:%
%:%754=370%:%
%:%755=370%:%
%:%756=370%:%
%:%757=371%:%
%:%758=371%:%
%:%759=372%:%
%:%760=372%:%
%:%761=373%:%
%:%762=373%:%
%:%763=374%:%
%:%764=374%:%
%:%765=375%:%
%:%766=375%:%
%:%767=376%:%
%:%768=376%:%
%:%769=376%:%
%:%770=377%:%
%:%771=377%:%
%:%772=378%:%
%:%773=378%:%
%:%774=379%:%
%:%775=379%:%
%:%776=380%:%
%:%777=380%:%
%:%778=381%:%
%:%779=381%:%
%:%780=381%:%
%:%781=382%:%
%:%782=382%:%
%:%783=382%:%
%:%784=383%:%
%:%785=383%:%
%:%786=383%:%
%:%787=384%:%
%:%788=384%:%
%:%789=384%:%
%:%790=385%:%
%:%791=385%:%
%:%792=385%:%
%:%793=386%:%
%:%794=386%:%
%:%795=386%:%
%:%796=387%:%
%:%797=387%:%
%:%798=388%:%
%:%799=388%:%
%:%800=389%:%
%:%801=389%:%
%:%802=390%:%
%:%803=390%:%
%:%804=391%:%
%:%805=391%:%
%:%806=392%:%
%:%807=392%:%
%:%810=395%:%
%:%811=396%:%
%:%812=396%:%
%:%813=396%:%
%:%814=397%:%
%:%815=397%:%
%:%816=398%:%
%:%817=398%:%
%:%818=399%:%
%:%828=401%:%
%:%830=403%:%
%:%831=403%:%
%:%832=404%:%
%:%833=405%:%
%:%836=406%:%
%:%840=406%:%
%:%841=406%:%
%:%842=407%:%
%:%843=407%:%
%:%844=408%:%
%:%845=408%:%
%:%846=409%:%
%:%847=409%:%
%:%848=410%:%
%:%849=410%:%
%:%850=410%:%
%:%851=411%:%
%:%852=411%:%
%:%853=411%:%
%:%854=412%:%
%:%855=412%:%
%:%856=413%:%
%:%857=413%:%
%:%858=413%:%
%:%859=414%:%
%:%860=414%:%
%:%861=415%:%
%:%862=415%:%
%:%863=416%:%
%:%864=416%:%
%:%865=417%:%
%:%866=417%:%
%:%867=418%:%
%:%868=418%:%
%:%869=418%:%
%:%870=419%:%
%:%871=419%:%
%:%872=419%:%
%:%873=420%:%
%:%874=420%:%
%:%875=420%:%
%:%876=421%:%
%:%877=421%:%
%:%878=421%:%
%:%879=422%:%
%:%880=422%:%
%:%881=423%:%
%:%882=423%:%
%:%883=423%:%
%:%884=424%:%
%:%885=424%:%
%:%886=425%:%
%:%896=427%:%
%:%898=429%:%
%:%899=429%:%
%:%900=430%:%
%:%901=431%:%
%:%908=432%:%
%:%909=432%:%
%:%910=433%:%
%:%911=433%:%
%:%912=434%:%
%:%913=434%:%
%:%914=435%:%
%:%915=435%:%
%:%916=436%:%
%:%917=436%:%
%:%918=436%:%
%:%919=437%:%
%:%920=437%:%
%:%921=438%:%
%:%922=438%:%
%:%923=438%:%
%:%924=439%:%
%:%925=439%:%
%:%926=440%:%
%:%927=440%:%
%:%928=440%:%
%:%929=441%:%
%:%930=441%:%
%:%931=442%:%
%:%932=442%:%
%:%933=443%:%
%:%943=445%:%
%:%944=446%:%
%:%945=447%:%
%:%946=448%:%
%:%947=449%:%
%:%948=450%:%
%:%949=451%:%
%:%951=453%:%
%:%952=453%:%
%:%953=454%:%
%:%954=455%:%
%:%957=456%:%
%:%961=456%:%
%:%971=458%:%
%:%972=459%:%
%:%973=460%:%
%:%974=461%:%
%:%975=462%:%
%:%976=463%:%
%:%977=464%:%
%:%978=465%:%
%:%979=466%:%
%:%980=467%:%
%:%981=468%:%
%:%982=469%:%
%:%983=470%:%
%:%984=471%:%
%:%985=472%:%
%:%986=473%:%
%:%987=474%:%
%:%988=475%:%
%:%989=476%:%
%:%990=477%:%
%:%991=478%:%
%:%992=479%:%
%:%993=480%:%
%:%994=481%:%
%:%995=482%:%
%:%996=483%:%
%:%997=484%:%
%:%998=485%:%
%:%999=486%:%
%:%1001=488%:%
%:%1002=488%:%
%:%1003=489%:%
%:%1004=490%:%
%:%1007=491%:%
%:%1011=491%:%
%:%1012=491%:%
%:%1013=492%:%
%:%1014=492%:%
%:%1015=493%:%
%:%1016=493%:%
%:%1017=494%:%
%:%1018=494%:%
%:%1019=495%:%
%:%1020=495%:%
%:%1021=496%:%
%:%1022=496%:%
%:%1023=496%:%
%:%1024=496%:%
%:%1025=497%:%
%:%1026=497%:%
%:%1027=498%:%
%:%1028=498%:%
%:%1029=499%:%
%:%1030=499%:%
%:%1031=500%:%
%:%1032=500%:%
%:%1033=501%:%
%:%1034=501%:%
%:%1035=501%:%
%:%1036=502%:%
%:%1037=502%:%
%:%1038=502%:%
%:%1039=503%:%
%:%1040=503%:%
%:%1041=504%:%
%:%1042=504%:%
%:%1043=504%:%
%:%1044=505%:%
%:%1045=505%:%
%:%1046=506%:%
%:%1047=506%:%
%:%1048=506%:%
%:%1049=507%:%
%:%1050=507%:%
%:%1051=508%:%
%:%1052=508%:%
%:%1053=508%:%
%:%1054=509%:%
%:%1055=509%:%
%:%1056=510%:%
%:%1057=510%:%
%:%1058=511%:%
%:%1059=511%:%
%:%1060=511%:%
%:%1061=512%:%
%:%1062=512%:%
%:%1063=513%:%
%:%1064=513%:%
%:%1065=513%:%
%:%1066=514%:%
%:%1076=516%:%
%:%1078=518%:%
%:%1079=518%:%
%:%1080=519%:%
%:%1081=520%:%
%:%1084=521%:%
%:%1088=521%:%
%:%1089=521%:%
%:%1090=522%:%
%:%1091=522%:%
%:%1092=523%:%
%:%1093=523%:%
%:%1094=524%:%
%:%1095=524%:%
%:%1096=525%:%
%:%1097=525%:%
%:%1098=525%:%
%:%1099=526%:%
%:%1100=526%:%
%:%1101=526%:%
%:%1102=527%:%
%:%1103=527%:%
%:%1104=527%:%
%:%1105=527%:%
%:%1106=528%:%
%:%1107=528%:%
%:%1108=528%:%
%:%1109=528%:%
%:%1110=529%:%
%:%1111=529%:%
%:%1112=529%:%
%:%1113=529%:%
%:%1114=529%:%
%:%1115=530%:%
%:%1125=532%:%
%:%1126=533%:%
%:%1127=534%:%
%:%1128=535%:%
%:%1129=536%:%
%:%1130=537%:%
%:%1131=538%:%
%:%1132=539%:%
%:%1133=540%:%
%:%1134=541%:%
%:%1135=542%:%
%:%1136=543%:%
%:%1137=544%:%
%:%1138=545%:%
%:%1139=546%:%
%:%1140=547%:%
%:%1141=548%:%
%:%1142=549%:%
%:%1143=550%:%
%:%1144=551%:%
%:%1145=552%:%
%:%1146=553%:%
%:%1147=554%:%
%:%1148=555%:%
%:%1149=556%:%
%:%1150=557%:%
%:%1151=558%:%
%:%1152=559%:%
%:%1153=560%:%
%:%1154=561%:%
%:%1155=562%:%
%:%1156=563%:%
%:%1157=564%:%
%:%1158=565%:%
%:%1159=566%:%
%:%1160=567%:%
%:%1161=568%:%
%:%1162=569%:%
%:%1163=570%:%
%:%1164=571%:%
%:%1165=572%:%
%:%1166=573%:%
%:%1167=574%:%
%:%1168=575%:%
%:%1169=576%:%
%:%1170=577%:%
%:%1171=578%:%
%:%1172=579%:%
%:%1173=580%:%
%:%1174=581%:%
%:%1175=582%:%
%:%1176=583%:%
%:%1177=584%:%
%:%1178=585%:%
%:%1179=586%:%
%:%1180=587%:%
%:%1181=588%:%
%:%1182=589%:%
%:%1183=590%:%
%:%1184=591%:%
%:%1185=592%:%
%:%1186=593%:%
%:%1187=594%:%
%:%1188=595%:%
%:%1189=596%:%
%:%1190=597%:%
%:%1191=598%:%
%:%1192=599%:%
%:%1193=600%:%
%:%1194=601%:%
%:%1195=602%:%
%:%1196=603%:%
%:%1197=604%:%
%:%1198=605%:%
%:%1199=606%:%
%:%1200=607%:%
%:%1201=608%:%
%:%1202=609%:%
%:%1203=610%:%
%:%1204=611%:%
%:%1205=612%:%
%:%1206=613%:%
%:%1207=614%:%
%:%1208=615%:%
%:%1209=616%:%
%:%1210=617%:%
%:%1211=618%:%
%:%1212=619%:%
%:%1213=620%:%
%:%1214=621%:%
%:%1215=622%:%
%:%1216=623%:%
%:%1217=624%:%
%:%1218=625%:%
%:%1219=626%:%
%:%1220=627%:%
%:%1221=628%:%
%:%1222=629%:%
%:%1223=630%:%
%:%1224=631%:%
%:%1225=632%:%
%:%1226=633%:%
%:%1227=634%:%
%:%1228=635%:%
%:%1229=636%:%
%:%1230=637%:%
%:%1231=638%:%
%:%1232=639%:%
%:%1233=640%:%
%:%1234=641%:%
%:%1235=642%:%
%:%1236=643%:%
%:%1237=644%:%
%:%1238=645%:%
%:%1239=646%:%
%:%1240=647%:%
%:%1241=648%:%
%:%1242=649%:%
%:%1243=650%:%
%:%1244=651%:%
%:%1245=652%:%
%:%1246=653%:%
%:%1247=654%:%
%:%1248=655%:%
%:%1249=656%:%
%:%1250=657%:%
%:%1251=658%:%
%:%1252=659%:%
%:%1253=660%:%
%:%1254=661%:%
%:%1255=662%:%
%:%1256=663%:%
%:%1257=664%:%
%:%1258=665%:%
%:%1259=666%:%
%:%1260=667%:%
%:%1261=668%:%
%:%1262=669%:%
%:%1263=670%:%
%:%1264=671%:%
%:%1265=672%:%
%:%1266=673%:%
%:%1267=674%:%
%:%1268=675%:%
%:%1269=676%:%
%:%1270=677%:%
%:%1271=678%:%
%:%1272=679%:%
%:%1273=680%:%
%:%1274=681%:%
%:%1275=682%:%
%:%1276=683%:%
%:%1277=684%:%
%:%1278=685%:%
%:%1279=686%:%
%:%1280=687%:%
%:%1281=688%:%
%:%1282=689%:%
%:%1283=690%:%
%:%1284=691%:%
%:%1285=692%:%
%:%1286=693%:%
%:%1287=694%:%
%:%1288=695%:%
%:%1289=696%:%
%:%1290=697%:%
%:%1291=698%:%
%:%1292=699%:%
%:%1293=700%:%
%:%1294=701%:%
%:%1295=702%:%
%:%1296=703%:%
%:%1297=704%:%
%:%1298=705%:%
%:%1299=706%:%
%:%1300=707%:%
%:%1301=708%:%
%:%1302=709%:%
%:%1303=710%:%
%:%1304=711%:%
%:%1305=712%:%
%:%1306=713%:%
%:%1307=714%:%
%:%1308=715%:%
%:%1309=716%:%
%:%1310=717%:%
%:%1311=718%:%
%:%1312=719%:%
%:%1313=720%:%
%:%1314=721%:%
%:%1315=722%:%
%:%1316=723%:%
%:%1317=724%:%
%:%1318=725%:%
%:%1319=726%:%
%:%1320=727%:%
%:%1321=728%:%
%:%1322=729%:%
%:%1323=730%:%
%:%1324=731%:%
%:%1325=732%:%
%:%1326=733%:%
%:%1327=734%:%
%:%1328=735%:%
%:%1329=736%:%
%:%1330=737%:%
%:%1331=738%:%
%:%1332=739%:%
%:%1333=740%:%
%:%1334=741%:%
%:%1335=742%:%
%:%1336=743%:%
%:%1337=744%:%
%:%1338=745%:%
%:%1339=746%:%
%:%1340=747%:%
%:%1341=748%:%
%:%1342=749%:%
%:%1343=750%:%
%:%1344=751%:%
%:%1345=752%:%
%:%1346=753%:%
%:%1347=754%:%
%:%1348=755%:%
%:%1349=756%:%
%:%1350=757%:%
%:%1351=758%:%
%:%1352=759%:%
%:%1353=760%:%
%:%1354=761%:%
%:%1356=763%:%
%:%1357=763%:%
%:%1358=764%:%
%:%1359=765%:%
%:%1360=766%:%
%:%1361=766%:%
%:%1362=767%:%
%:%1364=769%:%
%:%1365=770%:%
%:%1366=771%:%
%:%1368=773%:%
%:%1369=773%:%
%:%1370=774%:%
%:%1371=775%:%
%:%1378=776%:%
%:%1379=776%:%
%:%1380=777%:%
%:%1381=777%:%
%:%1382=778%:%
%:%1383=778%:%
%:%1384=779%:%
%:%1385=779%:%
%:%1386=780%:%
%:%1387=780%:%
%:%1388=781%:%
%:%1389=781%:%
%:%1390=782%:%
%:%1391=782%:%
%:%1392=783%:%
%:%1393=783%:%
%:%1394=784%:%
%:%1395=784%:%
%:%1396=785%:%
%:%1397=785%:%
%:%1398=786%:%
%:%1399=786%:%
%:%1400=787%:%
%:%1401=787%:%
%:%1402=788%:%
%:%1403=788%:%
%:%1404=789%:%
%:%1405=789%:%
%:%1406=790%:%
%:%1407=790%:%
%:%1408=790%:%
%:%1409=791%:%
%:%1410=791%:%
%:%1411=792%:%
%:%1412=792%:%
%:%1413=793%:%
%:%1414=793%:%
%:%1415=794%:%
%:%1416=794%:%
%:%1417=795%:%
%:%1418=795%:%
%:%1419=795%:%
%:%1420=796%:%
%:%1421=796%:%
%:%1422=796%:%
%:%1423=797%:%
%:%1424=797%:%
%:%1425=797%:%
%:%1426=798%:%
%:%1427=798%:%
%:%1428=798%:%
%:%1429=799%:%
%:%1430=799%:%
%:%1431=799%:%
%:%1432=800%:%
%:%1433=800%:%
%:%1434=801%:%
%:%1435=801%:%
%:%1436=802%:%
%:%1437=802%:%
%:%1438=803%:%
%:%1439=803%:%
%:%1440=804%:%
%:%1441=804%:%
%:%1442=804%:%
%:%1443=805%:%
%:%1444=805%:%
%:%1445=805%:%
%:%1446=806%:%
%:%1447=806%:%
%:%1448=806%:%
%:%1449=807%:%
%:%1450=807%:%
%:%1451=807%:%
%:%1452=808%:%
%:%1453=808%:%
%:%1454=808%:%
%:%1455=809%:%
%:%1456=809%:%
%:%1457=809%:%
%:%1458=810%:%
%:%1459=810%:%
%:%1460=811%:%
%:%1461=811%:%
%:%1462=812%:%
%:%1463=812%:%
%:%1464=813%:%
%:%1465=813%:%
%:%1466=814%:%
%:%1467=814%:%
%:%1468=815%:%
%:%1469=815%:%
%:%1470=816%:%
%:%1471=816%:%
%:%1472=816%:%
%:%1473=817%:%
%:%1474=817%:%
%:%1475=818%:%
%:%1476=818%:%
%:%1477=819%:%
%:%1478=819%:%
%:%1479=820%:%
%:%1480=820%:%
%:%1481=821%:%
%:%1482=821%:%
%:%1483=822%:%
%:%1484=822%:%
%:%1485=823%:%
%:%1486=823%:%
%:%1487=824%:%
%:%1488=824%:%
%:%1489=825%:%
%:%1490=825%:%
%:%1491=826%:%
%:%1492=826%:%
%:%1493=826%:%
%:%1494=827%:%
%:%1495=827%:%
%:%1496=827%:%
%:%1497=828%:%
%:%1498=828%:%
%:%1499=828%:%
%:%1500=829%:%
%:%1501=829%:%
%:%1502=830%:%
%:%1503=830%:%
%:%1504=831%:%
%:%1505=831%:%
%:%1506=832%:%
%:%1507=832%:%
%:%1508=833%:%
%:%1509=833%:%
%:%1510=834%:%
%:%1511=834%:%
%:%1512=835%:%
%:%1513=835%:%
%:%1514=836%:%
%:%1515=836%:%
%:%1516=836%:%
%:%1517=837%:%
%:%1518=837%:%
%:%1519=837%:%
%:%1520=838%:%
%:%1521=838%:%
%:%1522=839%:%
%:%1523=839%:%
%:%1524=840%:%
%:%1525=840%:%
%:%1526=841%:%
%:%1527=841%:%
%:%1528=841%:%
%:%1529=842%:%
%:%1530=842%:%
%:%1531=843%:%
%:%1532=843%:%
%:%1533=843%:%
%:%1534=844%:%
%:%1535=844%:%
%:%1536=845%:%
%:%1537=845%:%
%:%1538=846%:%
%:%1539=846%:%
%:%1540=847%:%
%:%1541=847%:%
%:%1542=847%:%
%:%1543=848%:%
%:%1544=848%:%
%:%1545=849%:%
%:%1546=849%:%
%:%1547=850%:%
%:%1548=850%:%
%:%1549=851%:%
%:%1550=851%:%
%:%1551=852%:%
%:%1552=852%:%
%:%1553=853%:%
%:%1563=855%:%
%:%1564=856%:%
%:%1565=857%:%
%:%1566=858%:%
%:%1567=859%:%
%:%1568=860%:%
%:%1569=861%:%
%:%1570=862%:%
%:%1571=863%:%
%:%1572=864%:%
%:%1573=865%:%
%:%1574=866%:%
%:%1575=867%:%
%:%1576=868%:%
%:%1577=869%:%
%:%1578=870%:%
%:%1579=871%:%
%:%1580=872%:%
%:%1581=873%:%
%:%1583=875%:%
%:%1584=875%:%
%:%1585=876%:%
%:%1586=877%:%
%:%1587=878%:%
%:%1588=879%:%
%:%1591=882%:%
%:%1598=883%:%
%:%1599=883%:%
%:%1600=884%:%
%:%1601=884%:%
%:%1605=888%:%
%:%1606=889%:%
%:%1607=889%:%
%:%1608=889%:%
%:%1609=890%:%
%:%1610=890%:%
%:%1613=893%:%
%:%1614=894%:%
%:%1615=894%:%
%:%1616=894%:%
%:%1617=895%:%
%:%1618=895%:%
%:%1619=895%:%
%:%1620=896%:%
%:%1621=896%:%
%:%1622=897%:%
%:%1623=897%:%
%:%1624=898%:%
%:%1625=898%:%
%:%1626=898%:%
%:%1627=899%:%
%:%1628=899%:%
%:%1629=900%:%
%:%1630=900%:%
%:%1631=900%:%
%:%1632=901%:%
%:%1633=901%:%
%:%1634=902%:%
%:%1635=902%:%
%:%1636=903%:%
%:%1637=903%:%
%:%1638=904%:%
%:%1639=904%:%
%:%1640=905%:%
%:%1641=905%:%
%:%1642=906%:%
%:%1643=906%:%
%:%1644=907%:%
%:%1645=907%:%
%:%1646=908%:%
%:%1647=908%:%
%:%1648=909%:%
%:%1649=909%:%
%:%1650=909%:%
%:%1651=910%:%
%:%1652=910%:%
%:%1653=910%:%
%:%1654=911%:%
%:%1655=911%:%
%:%1656=911%:%
%:%1657=912%:%
%:%1658=912%:%
%:%1659=912%:%
%:%1660=913%:%
%:%1661=913%:%
%:%1662=913%:%
%:%1663=914%:%
%:%1664=914%:%
%:%1665=915%:%
%:%1666=915%:%
%:%1667=915%:%
%:%1668=916%:%
%:%1669=916%:%
%:%1670=917%:%
%:%1671=917%:%
%:%1672=918%:%
%:%1673=918%:%
%:%1674=919%:%
%:%1675=919%:%
%:%1676=919%:%
%:%1677=920%:%
%:%1678=920%:%
%:%1679=921%:%
%:%1680=921%:%
%:%1681=922%:%
%:%1682=922%:%
%:%1683=923%:%
%:%1684=923%:%
%:%1685=924%:%
%:%1686=924%:%
%:%1687=925%:%
%:%1688=925%:%
%:%1689=926%:%
%:%1690=926%:%
%:%1691=927%:%
%:%1692=927%:%
%:%1693=928%:%
%:%1694=928%:%
%:%1695=928%:%
%:%1696=929%:%
%:%1697=929%:%
%:%1698=929%:%
%:%1699=930%:%
%:%1700=930%:%
%:%1701=930%:%
%:%1702=931%:%
%:%1703=931%:%
%:%1704=931%:%
%:%1705=932%:%
%:%1706=932%:%
%:%1707=932%:%
%:%1708=933%:%
%:%1709=933%:%
%:%1710=934%:%
%:%1711=934%:%
%:%1712=935%:%
%:%1713=935%:%
%:%1714=936%:%
%:%1715=936%:%
%:%1716=936%:%
%:%1717=937%:%
%:%1718=937%:%
%:%1719=937%:%
%:%1720=938%:%
%:%1721=938%:%
%:%1722=939%:%
%:%1723=939%:%
%:%1724=940%:%
%:%1725=940%:%
%:%1726=941%:%
%:%1727=941%:%
%:%1728=942%:%
%:%1729=942%:%
%:%1730=942%:%
%:%1731=943%:%
%:%1732=943%:%
%:%1733=943%:%
%:%1734=944%:%
%:%1735=944%:%
%:%1736=945%:%
%:%1737=945%:%
%:%1738=946%:%
%:%1739=946%:%
%:%1740=947%:%
%:%1741=947%:%
%:%1742=948%:%
%:%1743=948%:%
%:%1744=949%:%
%:%1745=949%:%
%:%1748=952%:%
%:%1749=953%:%
%:%1750=953%:%
%:%1751=953%:%
%:%1752=954%:%
%:%1762=956%:%
%:%1763=957%:%
%:%1764=958%:%
%:%1765=959%:%
%:%1766=960%:%
%:%1767=961%:%
%:%1769=963%:%
%:%1770=963%:%
%:%1771=964%:%
%:%1772=965%:%
%:%1773=966%:%
%:%1774=967%:%
%:%1775=968%:%
%:%1776=969%:%
%:%1783=970%:%
%:%1784=970%:%
%:%1785=971%:%
%:%1786=971%:%
%:%1790=975%:%
%:%1791=976%:%
%:%1792=976%:%
%:%1793=976%:%
%:%1794=977%:%
%:%1795=977%:%
%:%1796=978%:%
%:%1797=978%:%
%:%1798=979%:%
%:%1799=979%:%
%:%1800=980%:%
%:%1801=980%:%
%:%1802=981%:%
%:%1803=981%:%
%:%1804=982%:%
%:%1805=982%:%
%:%1806=983%:%
%:%1807=983%:%
%:%1808=984%:%
%:%1809=984%:%
%:%1810=985%:%
%:%1811=985%:%
%:%1812=986%:%
%:%1813=986%:%
%:%1814=986%:%
%:%1815=986%:%
%:%1816=987%:%
%:%1817=987%:%
%:%1818=987%:%
%:%1819=988%:%
%:%1820=988%:%
%:%1821=988%:%
%:%1822=989%:%
%:%1823=989%:%
%:%1824=989%:%
%:%1825=990%:%
%:%1826=990%:%
%:%1827=990%:%
%:%1828=991%:%
%:%1829=991%:%
%:%1832=994%:%
%:%1833=995%:%
%:%1834=995%:%
%:%1835=995%:%
%:%1836=996%:%
%:%1837=996%:%
%:%1838=996%:%
%:%1839=997%:%
%:%1840=997%:%
%:%1841=998%:%
%:%1842=998%:%
%:%1843=998%:%
%:%1844=999%:%
%:%1845=999%:%
%:%1846=1000%:%
%:%1847=1000%:%
%:%1848=1000%:%
%:%1849=1001%:%
%:%1850=1001%:%
%:%1851=1001%:%
%:%1852=1002%:%
%:%1853=1002%:%
%:%1854=1003%:%
%:%1855=1003%:%
%:%1856=1003%:%
%:%1857=1004%:%
%:%1858=1004%:%
%:%1859=1005%:%
%:%1860=1005%:%
%:%1861=1006%:%
%:%1862=1006%:%
%:%1863=1007%:%
%:%1864=1007%:%
%:%1865=1008%:%
%:%1866=1008%:%
%:%1867=1009%:%
%:%1868=1009%:%
%:%1869=1010%:%
%:%1870=1010%:%
%:%1871=1011%:%
%:%1872=1011%:%
%:%1873=1012%:%
%:%1874=1012%:%
%:%1875=1013%:%
%:%1876=1013%:%
%:%1877=1014%:%
%:%1878=1014%:%
%:%1879=1015%:%
%:%1880=1015%:%
%:%1881=1015%:%
%:%1882=1016%:%
%:%1883=1016%:%
%:%1884=1017%:%
%:%1885=1017%:%
%:%1886=1017%:%
%:%1887=1018%:%
%:%1888=1018%:%
%:%1889=1018%:%
%:%1890=1019%:%
%:%1891=1019%:%
%:%1892=1020%:%
%:%1893=1020%:%
%:%1894=1021%:%
%:%1895=1021%:%
%:%1896=1021%:%
%:%1897=1022%:%
%:%1898=1022%:%
%:%1899=1023%:%
%:%1900=1023%:%
%:%1901=1023%:%
%:%1902=1024%:%
%:%1903=1024%:%
%:%1904=1024%:%
%:%1905=1025%:%
%:%1906=1025%:%
%:%1907=1026%:%
%:%1908=1026%:%
%:%1909=1027%:%
%:%1910=1027%:%
%:%1911=1027%:%
%:%1912=1028%:%
%:%1913=1028%:%
%:%1914=1029%:%
%:%1915=1029%:%
%:%1916=1030%:%
%:%1917=1030%:%
%:%1918=1031%:%
%:%1919=1031%:%
%:%1920=1031%:%
%:%1921=1032%:%
%:%1922=1032%:%
%:%1923=1033%:%
%:%1924=1033%:%
%:%1925=1034%:%
%:%1926=1034%:%
%:%1927=1035%:%
%:%1928=1035%:%
%:%1929=1036%:%
%:%1930=1036%:%
%:%1931=1037%:%
%:%1932=1037%:%
%:%1933=1038%:%
%:%1934=1038%:%
%:%1935=1039%:%
%:%1936=1039%:%
%:%1937=1040%:%
%:%1938=1040%:%
%:%1939=1040%:%
%:%1940=1041%:%
%:%1941=1041%:%
%:%1942=1042%:%
%:%1943=1042%:%
%:%1944=1042%:%
%:%1945=1043%:%
%:%1946=1043%:%
%:%1947=1044%:%
%:%1948=1044%:%
%:%1949=1044%:%
%:%1950=1045%:%
%:%1951=1045%:%
%:%1952=1046%:%
%:%1953=1046%:%
%:%1954=1046%:%
%:%1955=1047%:%
%:%1956=1047%:%
%:%1957=1048%:%
%:%1958=1048%:%
%:%1959=1048%:%
%:%1960=1049%:%
%:%1961=1049%:%
%:%1962=1050%:%
%:%1963=1050%:%
%:%1964=1051%:%
%:%1965=1051%:%
%:%1966=1051%:%
%:%1967=1052%:%
%:%1968=1052%:%
%:%1969=1053%:%
%:%1970=1053%:%
%:%1971=1054%:%
%:%1972=1054%:%
%:%1973=1054%:%
%:%1974=1055%:%
%:%1975=1055%:%
%:%1976=1056%:%
%:%1977=1056%:%
%:%1978=1057%:%
%:%1979=1057%:%
%:%1980=1058%:%
%:%1981=1058%:%
%:%1982=1058%:%
%:%1983=1059%:%
%:%1984=1059%:%
%:%1985=1059%:%
%:%1986=1060%:%
%:%1987=1060%:%
%:%1988=1060%:%
%:%1989=1061%:%
%:%1990=1061%:%
%:%1991=1062%:%
%:%1992=1062%:%
%:%1993=1062%:%
%:%1994=1063%:%
%:%1995=1063%:%
%:%1996=1064%:%
%:%1997=1064%:%
%:%1998=1065%:%
%:%1999=1065%:%
%:%2000=1066%:%
%:%2001=1066%:%
%:%2002=1066%:%
%:%2003=1067%:%
%:%2013=1069%:%
%:%2014=1070%:%
%:%2015=1071%:%
%:%2016=1072%:%
%:%2017=1073%:%
%:%2018=1074%:%
%:%2019=1075%:%
%:%2020=1076%:%
%:%2021=1077%:%
%:%2022=1078%:%
%:%2023=1079%:%
%:%2024=1080%:%
%:%2026=1082%:%
%:%2027=1082%:%
%:%2028=1083%:%
%:%2029=1084%:%
%:%2030=1085%:%
%:%2031=1086%:%
%:%2032=1087%:%
%:%2033=1088%:%
%:%2034=1089%:%
%:%2035=1090%:%
%:%2036=1091%:%
%:%2037=1092%:%
%:%2038=1093%:%
%:%2045=1094%:%
%:%2046=1094%:%
%:%2047=1095%:%
%:%2048=1095%:%
%:%2049=1096%:%
%:%2050=1096%:%
%:%2051=1097%:%
%:%2052=1097%:%
%:%2053=1098%:%
%:%2054=1098%:%
%:%2055=1099%:%
%:%2056=1099%:%
%:%2057=1100%:%
%:%2058=1100%:%
%:%2059=1101%:%
%:%2060=1101%:%
%:%2061=1101%:%
%:%2062=1102%:%
%:%2063=1102%:%
%:%2064=1103%:%
%:%2065=1103%:%
%:%2066=1103%:%
%:%2067=1104%:%
%:%2068=1104%:%
%:%2069=1105%:%
%:%2070=1105%:%
%:%2071=1105%:%
%:%2072=1105%:%
%:%2073=1106%:%
%:%2074=1106%:%
%:%2075=1106%:%
%:%2076=1107%:%
%:%2077=1107%:%
%:%2078=1107%:%
%:%2079=1108%:%
%:%2080=1108%:%
%:%2081=1108%:%
%:%2082=1109%:%
%:%2083=1109%:%
%:%2084=1109%:%
%:%2085=1110%:%
%:%2086=1110%:%
%:%2087=1111%:%
%:%2088=1111%:%
%:%2089=1111%:%
%:%2090=1112%:%
%:%2091=1112%:%
%:%2094=1115%:%
%:%2095=1116%:%
%:%2096=1116%:%
%:%2097=1116%:%
%:%2098=1117%:%
%:%2099=1117%:%
%:%2100=1117%:%
%:%2103=1120%:%
%:%2104=1121%:%
%:%2105=1121%:%
%:%2106=1121%:%
%:%2107=1122%:%
%:%2108=1122%:%
%:%2109=1122%:%
%:%2110=1123%:%
%:%2111=1123%:%
%:%2112=1124%:%
%:%2113=1124%:%
%:%2114=1124%:%
%:%2115=1125%:%
%:%2116=1125%:%
%:%2117=1126%:%
%:%2118=1126%:%
%:%2119=1126%:%
%:%2120=1127%:%
%:%2121=1127%:%
%:%2122=1127%:%
%:%2123=1128%:%
%:%2124=1128%:%
%:%2125=1128%:%
%:%2126=1129%:%
%:%2127=1129%:%
%:%2128=1129%:%
%:%2129=1130%:%
%:%2130=1130%:%
%:%2131=1131%:%
%:%2132=1131%:%
%:%2133=1132%:%
%:%2134=1132%:%
%:%2135=1133%:%
%:%2136=1133%:%
%:%2137=1134%:%
%:%2138=1134%:%
%:%2139=1135%:%
%:%2140=1135%:%
%:%2141=1135%:%
%:%2142=1136%:%
%:%2143=1136%:%
%:%2144=1136%:%
%:%2145=1137%:%
%:%2146=1137%:%
%:%2147=1138%:%
%:%2148=1138%:%
%:%2149=1138%:%
%:%2150=1139%:%
%:%2151=1139%:%
%:%2152=1140%:%
%:%2153=1140%:%
%:%2154=1140%:%
%:%2155=1141%:%
%:%2156=1141%:%
%:%2157=1142%:%
%:%2158=1142%:%
%:%2159=1142%:%
%:%2160=1143%:%
%:%2161=1143%:%
%:%2162=1144%:%
%:%2163=1144%:%
%:%2164=1145%:%
%:%2165=1145%:%
%:%2166=1145%:%
%:%2167=1146%:%
%:%2168=1146%:%
%:%2169=1147%:%
%:%2170=1147%:%
%:%2171=1147%:%
%:%2172=1148%:%
%:%2173=1148%:%
%:%2174=1148%:%
%:%2175=1149%:%
%:%2176=1149%:%
%:%2177=1149%:%
%:%2178=1150%:%
%:%2179=1150%:%
%:%2180=1151%:%
%:%2181=1151%:%
%:%2182=1151%:%
%:%2183=1152%:%
%:%2184=1152%:%
%:%2185=1153%:%
%:%2186=1153%:%
%:%2187=1153%:%
%:%2188=1154%:%
%:%2189=1154%:%
%:%2190=1155%:%
%:%2191=1155%:%
%:%2192=1155%:%
%:%2193=1156%:%
%:%2194=1156%:%
%:%2195=1157%:%
%:%2196=1157%:%
%:%2197=1157%:%
%:%2198=1158%:%
%:%2199=1158%:%
%:%2200=1159%:%
%:%2201=1159%:%
%:%2202=1159%:%
%:%2203=1160%:%
%:%2204=1160%:%
%:%2205=1161%:%
%:%2206=1161%:%
%:%2207=1161%:%
%:%2208=1162%:%
%:%2218=1164%:%
%:%2219=1165%:%
%:%2220=1166%:%
%:%2221=1167%:%
%:%2222=1168%:%
%:%2224=1170%:%
%:%2225=1170%:%
%:%2226=1171%:%
%:%2227=1172%:%
%:%2228=1173%:%
%:%2229=1174%:%
%:%2230=1175%:%
%:%2231=1176%:%
%:%2232=1177%:%
%:%2233=1178%:%
%:%2234=1179%:%
%:%2235=1180%:%
%:%2236=1181%:%
%:%2237=1182%:%
%:%2244=1183%:%
%:%2245=1183%:%
%:%2246=1184%:%
%:%2247=1184%:%
%:%2248=1185%:%
%:%2249=1185%:%
%:%2250=1186%:%
%:%2251=1186%:%
%:%2252=1187%:%
%:%2253=1187%:%
%:%2254=1187%:%
%:%2255=1188%:%
%:%2256=1188%:%
%:%2257=1189%:%
%:%2258=1189%:%
%:%2259=1189%:%
%:%2260=1190%:%
%:%2261=1190%:%
%:%2262=1191%:%
%:%2263=1191%:%
%:%2264=1191%:%
%:%2265=1192%:%
%:%2266=1192%:%
%:%2267=1193%:%
%:%2268=1193%:%
%:%2269=1194%:%
%:%2270=1194%:%
%:%2271=1195%:%
%:%2272=1195%:%
%:%2273=1196%:%
%:%2274=1196%:%
%:%2275=1197%:%
%:%2276=1197%:%
%:%2277=1197%:%
%:%2278=1198%:%
%:%2279=1198%:%
%:%2280=1199%:%
%:%2281=1199%:%
%:%2282=1200%:%
%:%2283=1200%:%
%:%2284=1200%:%
%:%2285=1201%:%
%:%2286=1201%:%
%:%2287=1202%:%
%:%2288=1202%:%
%:%2289=1203%:%
%:%2290=1203%:%
%:%2291=1203%:%
%:%2292=1204%:%
%:%2293=1204%:%
%:%2294=1205%:%
%:%2295=1205%:%
%:%2296=1206%:%
%:%2297=1206%:%
%:%2298=1206%:%
%:%2299=1207%:%
%:%2300=1207%:%
%:%2301=1208%:%
%:%2302=1208%:%
%:%2303=1208%:%
%:%2304=1209%:%
%:%2305=1209%:%
%:%2306=1210%:%
%:%2307=1210%:%
%:%2308=1211%:%
%:%2309=1211%:%
%:%2310=1211%:%
%:%2311=1212%:%
%:%2312=1212%:%
%:%2313=1213%:%
%:%2314=1213%:%
%:%2315=1213%:%
%:%2316=1214%:%
%:%2317=1214%:%
%:%2318=1215%:%
%:%2319=1215%:%
%:%2320=1216%:%
%:%2321=1216%:%
%:%2322=1217%:%
%:%2323=1217%:%
%:%2324=1218%:%
%:%2325=1218%:%
%:%2326=1219%:%
%:%2327=1219%:%
%:%2328=1219%:%
%:%2329=1220%:%
%:%2330=1220%:%
%:%2331=1221%:%
%:%2332=1221%:%
%:%2333=1222%:%
%:%2334=1222%:%
%:%2335=1222%:%
%:%2336=1223%:%
%:%2337=1223%:%
%:%2338=1224%:%
%:%2339=1224%:%
%:%2340=1225%:%
%:%2341=1225%:%
%:%2342=1225%:%
%:%2343=1226%:%
%:%2344=1226%:%
%:%2345=1227%:%
%:%2346=1227%:%
%:%2347=1228%:%
%:%2348=1228%:%
%:%2349=1228%:%
%:%2350=1229%:%
%:%2351=1229%:%
%:%2352=1230%:%
%:%2353=1230%:%
%:%2354=1230%:%
%:%2355=1231%:%
%:%2356=1231%:%
%:%2357=1232%:%
%:%2358=1232%:%
%:%2359=1233%:%
%:%2360=1233%:%
%:%2361=1233%:%
%:%2362=1234%:%
%:%2363=1234%:%
%:%2364=1235%:%
%:%2365=1235%:%
%:%2366=1235%:%
%:%2367=1236%:%
%:%2368=1236%:%
%:%2369=1236%:%
%:%2370=1237%:%
%:%2371=1237%:%
%:%2372=1238%:%
%:%2373=1238%:%
%:%2374=1238%:%
%:%2375=1239%:%
%:%2376=1239%:%
%:%2377=1240%:%
%:%2378=1240%:%
%:%2379=1241%:%
%:%2380=1241%:%
%:%2381=1241%:%
%:%2382=1242%:%
%:%2392=1244%:%
%:%2393=1245%:%
%:%2394=1246%:%
%:%2395=1247%:%
%:%2396=1248%:%
%:%2398=1250%:%
%:%2399=1250%:%
%:%2400=1251%:%
%:%2401=1252%:%
%:%2404=1253%:%
%:%2408=1253%:%
%:%2409=1253%:%
%:%2410=1253%:%
%:%2415=1253%:%
%:%2418=1254%:%
%:%2419=1255%:%
%:%2420=1255%:%
%:%2423=1256%:%
%:%2427=1256%:%
%:%2428=1256%:%
%:%2437=1258%:%
%:%2439=1260%:%
%:%2440=1260%:%
%:%2441=1261%:%
%:%2442=1262%:%
%:%2443=1263%:%
%:%2444=1264%:%
%:%2445=1265%:%
%:%2446=1266%:%
%:%2453=1267%:%
%:%2454=1267%:%
%:%2455=1268%:%
%:%2456=1268%:%
%:%2457=1269%:%
%:%2458=1269%:%
%:%2459=1269%:%
%:%2460=1270%:%
%:%2461=1270%:%
%:%2465=1274%:%
%:%2466=1275%:%
%:%2467=1275%:%
%:%2468=1275%:%
%:%2469=1276%:%
%:%2470=1276%:%
%:%2471=1277%:%
%:%2472=1277%:%
%:%2473=1277%:%
%:%2474=1277%:%
%:%2475=1278%:%
%:%2476=1278%:%
%:%2477=1278%:%
%:%2478=1279%:%
%:%2479=1279%:%
%:%2480=1280%:%
%:%2481=1280%:%
%:%2482=1281%:%
%:%2483=1281%:%
%:%2484=1281%:%
%:%2485=1282%:%
%:%2486=1282%:%
%:%2487=1283%:%
%:%2488=1283%:%
%:%2489=1284%:%
%:%2490=1284%:%
%:%2491=1285%:%
%:%2492=1285%:%
%:%2493=1285%:%
%:%2494=1286%:%
%:%2495=1286%:%
%:%2496=1287%:%
%:%2497=1287%:%
%:%2498=1287%:%
%:%2499=1288%:%
%:%2500=1288%:%
%:%2501=1289%:%
%:%2502=1289%:%
%:%2503=1289%:%
%:%2504=1290%:%
%:%2505=1290%:%
%:%2506=1290%:%
%:%2507=1291%:%
%:%2508=1291%:%
%:%2509=1291%:%
%:%2510=1292%:%
%:%2511=1292%:%
%:%2512=1293%:%
%:%2513=1293%:%
%:%2514=1294%:%
%:%2515=1294%:%
%:%2516=1294%:%
%:%2517=1295%:%
%:%2518=1295%:%
%:%2519=1296%:%
%:%2520=1296%:%
%:%2521=1296%:%
%:%2522=1297%:%
%:%2523=1297%:%
%:%2524=1297%:%
%:%2525=1298%:%
%:%2526=1298%:%
%:%2527=1299%:%
%:%2528=1299%:%
%:%2529=1300%:%
%:%2530=1300%:%
%:%2531=1300%:%
%:%2532=1301%:%
%:%2533=1301%:%
%:%2534=1301%:%
%:%2535=1302%:%
%:%2536=1302%:%
%:%2537=1303%:%
%:%2538=1303%:%
%:%2539=1304%:%
%:%2540=1304%:%
%:%2541=1304%:%
%:%2542=1305%:%
%:%2543=1305%:%
%:%2544=1305%:%
%:%2545=1306%:%
%:%2546=1306%:%
%:%2547=1306%:%
%:%2548=1307%:%
%:%2549=1307%:%
%:%2550=1307%:%
%:%2551=1308%:%
%:%2552=1308%:%
%:%2553=1308%:%
%:%2554=1309%:%
%:%2555=1309%:%
%:%2556=1309%:%
%:%2557=1310%:%
%:%2558=1310%:%
%:%2559=1311%:%
%:%2560=1311%:%
%:%2561=1312%:%
%:%2562=1312%:%
%:%2563=1313%:%
%:%2564=1313%:%
%:%2565=1314%:%
%:%2566=1314%:%
%:%2567=1314%:%
%:%2568=1315%:%
%:%2569=1315%:%
%:%2570=1315%:%
%:%2571=1316%:%
%:%2572=1316%:%
%:%2573=1317%:%
%:%2574=1317%:%
%:%2575=1317%:%
%:%2576=1318%:%
%:%2577=1318%:%
%:%2578=1319%:%
%:%2579=1319%:%
%:%2580=1320%:%
%:%2581=1320%:%
%:%2582=1320%:%
%:%2583=1321%:%
%:%2584=1321%:%
%:%2585=1322%:%
%:%2586=1322%:%
%:%2587=1323%:%
%:%2588=1323%:%
%:%2589=1324%:%
%:%2590=1324%:%
%:%2591=1325%:%
%:%2592=1325%:%
%:%2593=1326%:%
%:%2594=1326%:%
%:%2595=1326%:%
%:%2596=1327%:%
%:%2597=1327%:%
%:%2598=1328%:%
%:%2599=1328%:%
%:%2600=1328%:%
%:%2601=1329%:%
%:%2602=1329%:%
%:%2603=1330%:%
%:%2604=1330%:%
%:%2605=1330%:%
%:%2606=1331%:%
%:%2607=1331%:%
%:%2608=1332%:%
%:%2609=1332%:%
%:%2610=1333%:%
%:%2611=1333%:%
%:%2612=1333%:%
%:%2613=1334%:%
%:%2614=1334%:%
%:%2615=1334%:%
%:%2616=1335%:%
%:%2617=1335%:%
%:%2618=1336%:%
%:%2619=1336%:%
%:%2620=1336%:%
%:%2621=1337%:%
%:%2622=1337%:%
%:%2623=1337%:%
%:%2624=1338%:%
%:%2625=1338%:%
%:%2626=1338%:%
%:%2627=1339%:%
%:%2628=1339%:%
%:%2629=1339%:%
%:%2630=1340%:%
%:%2631=1340%:%
%:%2632=1341%:%
%:%2633=1341%:%
%:%2634=1342%:%
%:%2635=1342%:%
%:%2636=1342%:%
%:%2637=1343%:%
%:%2638=1343%:%
%:%2639=1344%:%
%:%2640=1344%:%
%:%2641=1344%:%
%:%2642=1345%:%
%:%2643=1345%:%
%:%2644=1345%:%
%:%2645=1346%:%
%:%2646=1346%:%
%:%2647=1347%:%
%:%2648=1347%:%
%:%2649=1348%:%
%:%2650=1348%:%
%:%2651=1348%:%
%:%2652=1349%:%
%:%2653=1349%:%
%:%2654=1349%:%
%:%2655=1350%:%
%:%2656=1350%:%
%:%2657=1351%:%
%:%2658=1351%:%
%:%2659=1352%:%
%:%2660=1352%:%
%:%2661=1352%:%
%:%2662=1353%:%
%:%2663=1353%:%
%:%2664=1353%:%
%:%2665=1354%:%
%:%2666=1354%:%
%:%2667=1354%:%
%:%2668=1355%:%
%:%2669=1355%:%
%:%2670=1355%:%
%:%2671=1356%:%
%:%2672=1356%:%
%:%2673=1356%:%
%:%2674=1357%:%
%:%2675=1357%:%
%:%2676=1357%:%
%:%2677=1358%:%
%:%2678=1358%:%
%:%2679=1359%:%
%:%2680=1359%:%
%:%2681=1360%:%
%:%2682=1360%:%
%:%2683=1361%:%
%:%2684=1361%:%
%:%2685=1362%:%
%:%2686=1362%:%
%:%2687=1362%:%
%:%2688=1363%:%
%:%2689=1363%:%
%:%2690=1363%:%
%:%2691=1364%:%
%:%2692=1364%:%
%:%2693=1365%:%
%:%2694=1365%:%
%:%2695=1365%:%
%:%2696=1366%:%
%:%2697=1366%:%
%:%2698=1367%:%
%:%2699=1367%:%
%:%2700=1368%:%
%:%2701=1368%:%
%:%2702=1368%:%
%:%2703=1369%:%
%:%2704=1369%:%
%:%2705=1370%:%
%:%2706=1370%:%
%:%2707=1371%:%
%:%2708=1371%:%
%:%2709=1372%:%
%:%2710=1372%:%
%:%2711=1373%:%
%:%2712=1373%:%
%:%2713=1374%:%
%:%2714=1374%:%
%:%2715=1374%:%
%:%2716=1375%:%
%:%2717=1375%:%
%:%2718=1376%:%
%:%2719=1376%:%
%:%2720=1376%:%
%:%2721=1377%:%
%:%2722=1377%:%
%:%2723=1378%:%
%:%2724=1378%:%
%:%2725=1378%:%
%:%2726=1379%:%
%:%2727=1379%:%
%:%2728=1380%:%
%:%2729=1380%:%
%:%2730=1381%:%
%:%2731=1381%:%
%:%2732=1381%:%
%:%2733=1382%:%
%:%2734=1382%:%
%:%2735=1382%:%
%:%2736=1383%:%
%:%2737=1383%:%
%:%2738=1384%:%
%:%2739=1384%:%
%:%2740=1384%:%
%:%2741=1385%:%
%:%2742=1385%:%
%:%2743=1386%:%
%:%2744=1386%:%
%:%2745=1387%:%
%:%2746=1387%:%
%:%2747=1388%:%
%:%2748=1388%:%
%:%2749=1389%:%
%:%2750=1389%:%
%:%2751=1390%:%
%:%2752=1390%:%
%:%2753=1390%:%
%:%2754=1391%:%
%:%2755=1391%:%
%:%2756=1392%:%
%:%2757=1392%:%
%:%2758=1393%:%
%:%2759=1393%:%
%:%2760=1393%:%
%:%2761=1394%:%
%:%2762=1394%:%
%:%2763=1394%:%
%:%2764=1395%:%
%:%2765=1395%:%
%:%2766=1396%:%
%:%2767=1396%:%
%:%2768=1397%:%
%:%2769=1397%:%
%:%2770=1398%:%
%:%2771=1398%:%
%:%2772=1399%:%
%:%2773=1399%:%
%:%2774=1400%:%
%:%2775=1400%:%
%:%2776=1400%:%
%:%2777=1401%:%
%:%2778=1401%:%
%:%2779=1402%:%
%:%2780=1402%:%
%:%2781=1403%:%
%:%2782=1403%:%
%:%2783=1403%:%
%:%2784=1404%:%
%:%2785=1404%:%
%:%2786=1404%:%
%:%2787=1405%:%
%:%2788=1405%:%
%:%2789=1406%:%
%:%2790=1406%:%
%:%2791=1407%:%
%:%2792=1407%:%
%:%2793=1408%:%
%:%2803=1410%:%
%:%2804=1411%:%
%:%2805=1412%:%
%:%2806=1413%:%
%:%2807=1414%:%
%:%2808=1415%:%
%:%2809=1416%:%
%:%2810=1417%:%
%:%2811=1418%:%
%:%2812=1419%:%
%:%2813=1420%:%
%:%2814=1421%:%
%:%2815=1422%:%
%:%2816=1423%:%
%:%2817=1424%:%
%:%2819=1426%:%
%:%2820=1426%:%
%:%2821=1427%:%
%:%2822=1428%:%
%:%2823=1429%:%
%:%2824=1430%:%
%:%2827=1433%:%
%:%2834=1434%:%
%:%2835=1434%:%
%:%2836=1435%:%
%:%2837=1435%:%
%:%2841=1439%:%
%:%2842=1440%:%
%:%2843=1440%:%
%:%2844=1440%:%
%:%2845=1441%:%
%:%2846=1441%:%
%:%2847=1442%:%
%:%2848=1442%:%
%:%2849=1442%:%
%:%2850=1442%:%
%:%2851=1443%:%
%:%2852=1443%:%
%:%2853=1444%:%
%:%2854=1444%:%
%:%2855=1445%:%
%:%2856=1445%:%
%:%2857=1446%:%
%:%2858=1446%:%
%:%2859=1447%:%
%:%2860=1447%:%
%:%2861=1448%:%
%:%2862=1448%:%
%:%2863=1448%:%
%:%2864=1449%:%
%:%2865=1449%:%
%:%2866=1450%:%
%:%2867=1450%:%
%:%2868=1450%:%
%:%2869=1451%:%
%:%2870=1451%:%
%:%2871=1451%:%
%:%2872=1452%:%
%:%2873=1452%:%
%:%2874=1452%:%
%:%2875=1453%:%
%:%2876=1453%:%
%:%2877=1454%:%
%:%2878=1454%:%
%:%2879=1455%:%
%:%2880=1455%:%
%:%2881=1456%:%
%:%2882=1456%:%
%:%2883=1456%:%
%:%2884=1457%:%
%:%2885=1457%:%
%:%2886=1458%:%
%:%2887=1458%:%
%:%2888=1459%:%
%:%2889=1459%:%
%:%2890=1459%:%
%:%2891=1460%:%
%:%2892=1460%:%
%:%2893=1460%:%
%:%2894=1461%:%
%:%2895=1461%:%
%:%2896=1461%:%
%:%2897=1462%:%
%:%2898=1462%:%
%:%2901=1465%:%
%:%2902=1466%:%
%:%2903=1466%:%
%:%2904=1466%:%
%:%2905=1467%:%
%:%2906=1467%:%
%:%2907=1467%:%
%:%2908=1468%:%
%:%2909=1468%:%
%:%2910=1469%:%
%:%2911=1469%:%
%:%2912=1470%:%
%:%2913=1470%:%
%:%2914=1471%:%
%:%2915=1471%:%
%:%2916=1472%:%
%:%2917=1472%:%
%:%2918=1472%:%
%:%2919=1473%:%
%:%2920=1473%:%
%:%2921=1474%:%
%:%2922=1474%:%
%:%2923=1475%:%
%:%2924=1475%:%
%:%2925=1476%:%
%:%2926=1476%:%
%:%2927=1477%:%
%:%2928=1477%:%
%:%2929=1478%:%
%:%2930=1478%:%
%:%2931=1479%:%
%:%2932=1479%:%
%:%2933=1480%:%
%:%2934=1480%:%
%:%2935=1481%:%
%:%2936=1481%:%
%:%2937=1482%:%
%:%2938=1482%:%
%:%2939=1483%:%
%:%2940=1483%:%
%:%2941=1484%:%
%:%2942=1484%:%
%:%2943=1485%:%
%:%2944=1485%:%
%:%2945=1486%:%
%:%2946=1486%:%
%:%2947=1487%:%
%:%2948=1487%:%
%:%2949=1487%:%
%:%2950=1488%:%
%:%2951=1488%:%
%:%2952=1489%:%
%:%2953=1489%:%
%:%2954=1489%:%
%:%2955=1490%:%
%:%2956=1490%:%
%:%2957=1491%:%
%:%2958=1491%:%
%:%2959=1492%:%
%:%2960=1492%:%
%:%2961=1492%:%
%:%2962=1493%:%
%:%2963=1493%:%
%:%2964=1494%:%
%:%2965=1494%:%
%:%2966=1494%:%
%:%2967=1495%:%
%:%2968=1495%:%
%:%2969=1496%:%
%:%2970=1496%:%
%:%2971=1497%:%
%:%2972=1497%:%
%:%2973=1497%:%
%:%2974=1498%:%
%:%2975=1498%:%
%:%2976=1498%:%
%:%2977=1499%:%
%:%2978=1499%:%
%:%2979=1499%:%
%:%2980=1500%:%
%:%2981=1500%:%
%:%2984=1503%:%
%:%2985=1504%:%
%:%2986=1504%:%
%:%2987=1504%:%
%:%2988=1505%:%
%:%2989=1505%:%
%:%2990=1505%:%
%:%2991=1506%:%
%:%2992=1506%:%
%:%2993=1507%:%
%:%2994=1507%:%
%:%2995=1507%:%
%:%2996=1508%:%
%:%2997=1508%:%
%:%2998=1509%:%
%:%2999=1509%:%
%:%3000=1509%:%
%:%3001=1510%:%
%:%3002=1510%:%
%:%3003=1510%:%
%:%3004=1511%:%
%:%3005=1511%:%
%:%3006=1512%:%
%:%3007=1512%:%
%:%3008=1512%:%
%:%3009=1513%:%
%:%3010=1513%:%
%:%3011=1514%:%
%:%3012=1514%:%
%:%3013=1515%:%
%:%3014=1515%:%
%:%3015=1516%:%
%:%3016=1516%:%
%:%3017=1517%:%
%:%3018=1517%:%
%:%3019=1518%:%
%:%3020=1518%:%
%:%3021=1519%:%
%:%3022=1519%:%
%:%3023=1520%:%
%:%3024=1520%:%
%:%3025=1521%:%
%:%3026=1521%:%
%:%3027=1522%:%
%:%3028=1522%:%
%:%3029=1523%:%
%:%3030=1523%:%
%:%3031=1523%:%
%:%3032=1524%:%
%:%3033=1524%:%
%:%3034=1525%:%
%:%3035=1525%:%
%:%3036=1526%:%
%:%3037=1526%:%
%:%3038=1527%:%
%:%3039=1527%:%
%:%3040=1528%:%
%:%3041=1528%:%
%:%3042=1529%:%
%:%3043=1529%:%
%:%3044=1530%:%
%:%3045=1530%:%
%:%3046=1531%:%
%:%3047=1531%:%
%:%3048=1532%:%
%:%3049=1532%:%
%:%3050=1533%:%
%:%3051=1533%:%
%:%3052=1534%:%
%:%3053=1534%:%
%:%3054=1534%:%
%:%3055=1535%:%
%:%3056=1535%:%
%:%3057=1536%:%
%:%3058=1536%:%
%:%3059=1537%:%
%:%3060=1537%:%
%:%3061=1538%:%
%:%3062=1538%:%
%:%3063=1538%:%
%:%3064=1539%:%
%:%3074=1541%:%
%:%3075=1542%:%
%:%3076=1543%:%
%:%3078=1545%:%
%:%3079=1545%:%
%:%3080=1546%:%
%:%3081=1547%:%
%:%3082=1548%:%
%:%3085=1549%:%
%:%3089=1549%:%
%:%3090=1549%:%
%:%3091=1550%:%
%:%3092=1550%:%
%:%3093=1551%:%
%:%3094=1551%:%
%:%3098=1555%:%
%:%3099=1556%:%
%:%3100=1556%:%
%:%3101=1556%:%
%:%3102=1557%:%
%:%3103=1557%:%
%:%3104=1558%:%
%:%3105=1558%:%
%:%3106=1559%:%
%:%3107=1559%:%
%:%3108=1559%:%
%:%3109=1560%:%
%:%3110=1560%:%
%:%3111=1560%:%
%:%3112=1561%:%
%:%3113=1561%:%
%:%3116=1564%:%
%:%3117=1565%:%
%:%3118=1565%:%
%:%3119=1566%:%
%:%3120=1566%:%
%:%3121=1567%:%
%:%3122=1567%:%
%:%3123=1568%:%
%:%3124=1568%:%
%:%3125=1568%:%
%:%3126=1569%:%
%:%3127=1569%:%
%:%3128=1570%:%
%:%3129=1570%:%
%:%3130=1571%:%
%:%3131=1571%:%
%:%3132=1572%:%
%:%3133=1572%:%
%:%3134=1572%:%
%:%3135=1573%:%
%:%3136=1573%:%
%:%3137=1574%:%
%:%3147=1576%:%
%:%3149=1578%:%
%:%3150=1578%:%
%:%3151=1579%:%
%:%3152=1580%:%
%:%3153=1581%:%
%:%3160=1582%:%
%:%3161=1582%:%
%:%3162=1583%:%
%:%3163=1583%:%
%:%3164=1584%:%
%:%3165=1584%:%
%:%3166=1585%:%
%:%3167=1585%:%
%:%3168=1585%:%
%:%3169=1586%:%
%:%3170=1586%:%
%:%3171=1587%:%
%:%3172=1587%:%
%:%3173=1587%:%
%:%3174=1588%:%
%:%3175=1588%:%
%:%3176=1589%:%
%:%3177=1589%:%
%:%3178=1589%:%
%:%3179=1590%:%
%:%3180=1590%:%
%:%3181=1591%:%
%:%3182=1591%:%
%:%3183=1591%:%
%:%3184=1592%:%
%:%3185=1592%:%
%:%3186=1593%:%
%:%3187=1593%:%
%:%3188=1594%:%
%:%3189=1594%:%

%
\begin{isabellebody}%
\setisabellecontext{TeoremaEx}%
%
\isadelimtheory
%
\endisadelimtheory
%
\isatagtheory
%
\endisatagtheory
{\isafoldtheory}%
%
\isadelimtheory
%
\endisadelimtheory
%
\begin{isamarkuptext}%
\comentario{Añadir introducción.}%
\end{isamarkuptext}\isamarkuptrue%
%
\begin{isamarkuptext}%
\comentario{Cambiar referencias de los lemas tras el cambio de índice.}%
\end{isamarkuptext}\isamarkuptrue%
%
\isadelimdocument
%
\endisadelimdocument
%
\isatagdocument
%
\isamarkupsection{Sucesiones de conjuntos%
}
\isamarkuptrue%
%
\endisatagdocument
{\isafolddocument}%
%
\isadelimdocument
%
\endisadelimdocument
%
\begin{isamarkuptext}%
En este apartado daremos una introducción sobre sucesiones de conjuntos de fórmulas a 
  partir de una colección y un conjunto de la misma. De este modo, se mostrarán distintas 
  características sobre las sucesiones y se definirá su límite. En la siguiente sección 
  probaremos que dicho límite constituye un conjunto satisfacible por el lema de Hintikka.

\comentario{Revisar el párrafo anterior al final}

  Recordemos que el conjunto de las fórmulas proposicionales se define recursivamente a partir 
  de un alfabeto numerable de variables proposicionales. Por lo tanto, el conjunto de fórmulas 
  proposicionales es igualmente numerable, de modo que es posible enumerar sus elementos. Una vez 
  realizada esta observación, veamos la definición de sucesión de conjuntos de fórmulas 
  proposicionales a partir de una colección y un conjunto de la misma.

\begin{definicion}
  Sea \isa{C} una colección de conjuntos de fórmulas proposicionales, \isa{S\ {\isasymin}\ C} y \isa{F\isactrlsub {\isadigit{1}}{\isacharcomma}\ F\isactrlsub {\isadigit{2}}{\isacharcomma}\ F\isactrlsub {\isadigit{3}}\ {\isasymdots}} una 
  enumeración de las fórmulas proposicionales. Se define la \isa{sucesión\ de\ conjuntos\ de\ C\ a\ partir\ de\ S} como sigue:

  $S_{0} = S$

  $S_{n+1} = \left\{ \begin{array}{lcc} S_{n} \cup \{F_{n}\} &  si  & S_{n} \cup \{F_{n}\} \in C \\ \\ S_{n} & c.c \end{array} \right.$ 
\end{definicion}

  Para su formalización en Isabelle se ha introducido una instancia en la teoría de \isa{Sintaxis} que 
  indica explícitamente que el conjunto de las fórmulas proposicionales es numerable, probado
  mediante el método \isa{countable{\isacharunderscore}datatype} de Isabelle.

  \isa{instance\ formula\ {\isacharcolon}{\isacharcolon}\ {\isacharparenleft}countable{\isacharparenright}\ countable\ by\ countable{\isacharunderscore}datatype}

  De esta manera se genera en Isabelle una enumeración predeterminada de los elementos del conjunto,
  junto con herramientas para probar propiedades referentes a la numerabilidad. En particular, en la 
  formalización de la definición \isa{{\isadigit{4}}{\isachardot}{\isadigit{1}}{\isachardot}{\isadigit{1}}} se utilizará la función \isa{from{\isacharunderscore}nat} que, al aplicarla a un 
  número natural \isa{n}, nos devuelve la \isa{n}-ésima fórmula proposicional según una enumeración 
  predeterminada en Isabelle. 

  Puesto que la definición de las sucesiones en \isa{{\isadigit{4}}{\isachardot}{\isadigit{1}}{\isachardot}{\isadigit{1}}} se trata de una definición 
  recursiva sobre la estructura recursiva de los números naturales, se formalizará en Isabelle
  mediante el tipo de funciones primitivas recursivas de la siguiente manera.%
\end{isamarkuptext}\isamarkuptrue%
\isacommand{primrec}\isamarkupfalse%
\ pcp{\isacharunderscore}seq\ \isakeyword{where}\isanewline
{\isachardoublequoteopen}pcp{\isacharunderscore}seq\ C\ S\ {\isadigit{0}}\ {\isacharequal}\ S{\isachardoublequoteclose}\ {\isacharbar}\isanewline
{\isachardoublequoteopen}pcp{\isacharunderscore}seq\ C\ S\ {\isacharparenleft}Suc\ n{\isacharparenright}\ {\isacharequal}\ {\isacharparenleft}let\ Sn\ {\isacharequal}\ pcp{\isacharunderscore}seq\ C\ S\ n{\isacharsemicolon}\ Sn{\isadigit{1}}\ {\isacharequal}\ insert\ {\isacharparenleft}from{\isacharunderscore}nat\ n{\isacharparenright}\ Sn\ in\isanewline
\ \ \ \ \ \ \ \ \ \ \ \ \ \ \ \ \ \ \ \ \ \ \ \ if\ Sn{\isadigit{1}}\ {\isasymin}\ C\ then\ Sn{\isadigit{1}}\ else\ Sn{\isacharparenright}{\isachardoublequoteclose}%
\begin{isamarkuptext}%
Veamos el primer resultado sobre dichas sucesiones.

  \begin{lema}
    Sea \isa{C} una colección de conjuntos con la propiedad de consistencia proposicional,\\ \isa{S\ {\isasymin}\ C} y 
    \isa{{\isacharbraceleft}S\isactrlsub n{\isacharbraceright}} la sucesión de conjuntos de \isa{C} a partir de \isa{S} construida según la definición \isa{{\isadigit{4}}{\isachardot}{\isadigit{1}}{\isachardot}{\isadigit{1}}}. 
    Entonces, para todo \isa{n\ {\isasymin}\ {\isasymnat}} se verifica que \isa{S\isactrlsub n\ {\isasymin}\ C}.
  \end{lema}

  Procedamos con su demostración.

  \begin{demostracion}
    El resultado se prueba por inducción en los números naturales que conforman los subíndices de la 
    sucesión.

    En primer lugar, tenemos que \isa{S\isactrlsub {\isadigit{0}}\ {\isacharequal}\ S} pertenece a \isa{C} por hipótesis.

    Por otro lado, supongamos que \isa{S\isactrlsub n\ {\isasymin}\ C}. Probemos que \isa{S\isactrlsub n\isactrlsub {\isacharplus}\isactrlsub {\isadigit{1}}\ {\isasymin}\ C}. Si suponemos que \isa{S\isactrlsub n\ {\isasymunion}\ {\isacharbraceleft}F\isactrlsub n{\isacharbraceright}\ {\isasymin}\ C},
    por definición tenemos que \isa{S\isactrlsub n\isactrlsub {\isacharplus}\isactrlsub {\isadigit{1}}\ {\isacharequal}\ S\isactrlsub n\ {\isasymunion}\ {\isacharbraceleft}F\isactrlsub n{\isacharbraceright}}, luego pertenece a \isa{C}. En caso contrario, si
    suponemos que \isa{S\isactrlsub n\ {\isasymunion}\ {\isacharbraceleft}F\isactrlsub n{\isacharbraceright}\ {\isasymnotin}\ C}, por definición tenemos que \isa{S\isactrlsub n\isactrlsub {\isacharplus}\isactrlsub {\isadigit{1}}\ {\isacharequal}\ S\isactrlsub n}, que pertenece igualmente
    a \isa{C} por hipótesis de inducción. Por tanto, queda probado el resultado.
  \end{demostracion}

  La formalización y demostración detallada del lema en Isabelle son las siguientes.%
\end{isamarkuptext}\isamarkuptrue%
\isacommand{lemma}\isamarkupfalse%
\ \isanewline
\ \ \isakeyword{assumes}\ {\isachardoublequoteopen}pcp\ C{\isachardoublequoteclose}\ \isanewline
\ \ \ \ \ \ \ \ \ \ {\isachardoublequoteopen}S\ {\isasymin}\ C{\isachardoublequoteclose}\isanewline
\ \ \ \ \ \ \ \ \isakeyword{shows}\ {\isachardoublequoteopen}pcp{\isacharunderscore}seq\ C\ S\ n\ {\isasymin}\ C{\isachardoublequoteclose}\isanewline
%
\isadelimproof
%
\endisadelimproof
%
\isatagproof
\isacommand{proof}\isamarkupfalse%
\ {\isacharparenleft}induction\ n{\isacharparenright}\isanewline
\ \ \isacommand{show}\isamarkupfalse%
\ {\isachardoublequoteopen}pcp{\isacharunderscore}seq\ C\ S\ {\isadigit{0}}\ {\isasymin}\ C{\isachardoublequoteclose}\isanewline
\ \ \ \ \isacommand{by}\isamarkupfalse%
\ {\isacharparenleft}simp\ only{\isacharcolon}\ pcp{\isacharunderscore}seq{\isachardot}simps{\isacharparenleft}{\isadigit{1}}{\isacharparenright}\ {\isacartoucheopen}S\ {\isasymin}\ C{\isacartoucheclose}{\isacharparenright}\isanewline
\isacommand{next}\isamarkupfalse%
\isanewline
\ \ \isacommand{fix}\isamarkupfalse%
\ n\isanewline
\ \ \isacommand{assume}\isamarkupfalse%
\ HI{\isacharcolon}{\isachardoublequoteopen}pcp{\isacharunderscore}seq\ C\ S\ n\ {\isasymin}\ C{\isachardoublequoteclose}\isanewline
\ \ \isacommand{have}\isamarkupfalse%
\ {\isachardoublequoteopen}pcp{\isacharunderscore}seq\ C\ S\ {\isacharparenleft}Suc\ n{\isacharparenright}\ {\isacharequal}\ {\isacharparenleft}let\ Sn\ {\isacharequal}\ pcp{\isacharunderscore}seq\ C\ S\ n{\isacharsemicolon}\ Sn{\isadigit{1}}\ {\isacharequal}\ insert\ {\isacharparenleft}from{\isacharunderscore}nat\ n{\isacharparenright}\ Sn\ in\isanewline
\ \ \ \ \ \ \ \ \ \ \ \ \ \ \ \ \ \ \ \ \ \ \ \ if\ Sn{\isadigit{1}}\ {\isasymin}\ C\ then\ Sn{\isadigit{1}}\ else\ Sn{\isacharparenright}{\isachardoublequoteclose}\ \isanewline
\ \ \ \ \isacommand{by}\isamarkupfalse%
\ {\isacharparenleft}simp\ only{\isacharcolon}\ pcp{\isacharunderscore}seq{\isachardot}simps{\isacharparenleft}{\isadigit{2}}{\isacharparenright}{\isacharparenright}\isanewline
\ \ \isacommand{then}\isamarkupfalse%
\ \isacommand{have}\isamarkupfalse%
\ SucDef{\isacharcolon}{\isachardoublequoteopen}pcp{\isacharunderscore}seq\ C\ S\ {\isacharparenleft}Suc\ n{\isacharparenright}\ {\isacharequal}\ {\isacharparenleft}if\ insert\ {\isacharparenleft}from{\isacharunderscore}nat\ n{\isacharparenright}\ {\isacharparenleft}pcp{\isacharunderscore}seq\ C\ S\ n{\isacharparenright}\ {\isasymin}\ C\ then\ \isanewline
\ \ \ \ \ \ \ \ \ \ \ \ \ \ \ \ \ \ \ \ insert\ {\isacharparenleft}from{\isacharunderscore}nat\ n{\isacharparenright}\ {\isacharparenleft}pcp{\isacharunderscore}seq\ C\ S\ n{\isacharparenright}\ else\ pcp{\isacharunderscore}seq\ C\ S\ n{\isacharparenright}{\isachardoublequoteclose}\ \isanewline
\ \ \ \ \isacommand{by}\isamarkupfalse%
\ {\isacharparenleft}simp\ only{\isacharcolon}\ Let{\isacharunderscore}def{\isacharparenright}\isanewline
\ \ \isacommand{show}\isamarkupfalse%
\ {\isachardoublequoteopen}pcp{\isacharunderscore}seq\ C\ S\ {\isacharparenleft}Suc\ n{\isacharparenright}\ {\isasymin}\ C{\isachardoublequoteclose}\isanewline
\ \ \isacommand{proof}\isamarkupfalse%
\ {\isacharparenleft}cases{\isacharparenright}\isanewline
\ \ \ \ \isacommand{assume}\isamarkupfalse%
\ {\isadigit{1}}{\isacharcolon}{\isachardoublequoteopen}insert\ {\isacharparenleft}from{\isacharunderscore}nat\ n{\isacharparenright}\ {\isacharparenleft}pcp{\isacharunderscore}seq\ C\ S\ n{\isacharparenright}\ {\isasymin}\ C{\isachardoublequoteclose}\isanewline
\ \ \ \ \isacommand{have}\isamarkupfalse%
\ {\isachardoublequoteopen}pcp{\isacharunderscore}seq\ C\ S\ {\isacharparenleft}Suc\ n{\isacharparenright}\ {\isacharequal}\ insert\ {\isacharparenleft}from{\isacharunderscore}nat\ n{\isacharparenright}\ {\isacharparenleft}pcp{\isacharunderscore}seq\ C\ S\ n{\isacharparenright}{\isachardoublequoteclose}\isanewline
\ \ \ \ \ \ \isacommand{using}\isamarkupfalse%
\ SucDef\ {\isadigit{1}}\ \isacommand{by}\isamarkupfalse%
\ {\isacharparenleft}simp\ only{\isacharcolon}\ if{\isacharunderscore}True{\isacharparenright}\isanewline
\ \ \ \ \isacommand{thus}\isamarkupfalse%
\ {\isachardoublequoteopen}pcp{\isacharunderscore}seq\ C\ S\ {\isacharparenleft}Suc\ n{\isacharparenright}\ {\isasymin}\ C{\isachardoublequoteclose}\isanewline
\ \ \ \ \ \ \isacommand{by}\isamarkupfalse%
\ {\isacharparenleft}simp\ only{\isacharcolon}\ {\isadigit{1}}{\isacharparenright}\isanewline
\ \ \isacommand{next}\isamarkupfalse%
\isanewline
\ \ \ \ \isacommand{assume}\isamarkupfalse%
\ {\isadigit{2}}{\isacharcolon}{\isachardoublequoteopen}insert\ {\isacharparenleft}from{\isacharunderscore}nat\ n{\isacharparenright}\ {\isacharparenleft}pcp{\isacharunderscore}seq\ C\ S\ n{\isacharparenright}\ {\isasymnotin}\ C{\isachardoublequoteclose}\isanewline
\ \ \ \ \isacommand{have}\isamarkupfalse%
\ {\isachardoublequoteopen}pcp{\isacharunderscore}seq\ C\ S\ {\isacharparenleft}Suc\ n{\isacharparenright}\ {\isacharequal}\ pcp{\isacharunderscore}seq\ C\ S\ n{\isachardoublequoteclose}\isanewline
\ \ \ \ \ \ \isacommand{using}\isamarkupfalse%
\ SucDef\ {\isadigit{2}}\ \isacommand{by}\isamarkupfalse%
\ {\isacharparenleft}simp\ only{\isacharcolon}\ if{\isacharunderscore}False{\isacharparenright}\isanewline
\ \ \ \ \isacommand{thus}\isamarkupfalse%
\ {\isachardoublequoteopen}pcp{\isacharunderscore}seq\ C\ S\ {\isacharparenleft}Suc\ n{\isacharparenright}\ {\isasymin}\ C{\isachardoublequoteclose}\isanewline
\ \ \ \ \ \ \isacommand{by}\isamarkupfalse%
\ {\isacharparenleft}simp\ only{\isacharcolon}\ HI{\isacharparenright}\isanewline
\ \ \isacommand{qed}\isamarkupfalse%
\isanewline
\isacommand{qed}\isamarkupfalse%
%
\endisatagproof
{\isafoldproof}%
%
\isadelimproof
%
\endisadelimproof
%
\begin{isamarkuptext}%
Del mismo modo, podemos probar el lema de manera automática en Isabelle.%
\end{isamarkuptext}\isamarkuptrue%
\isacommand{lemma}\isamarkupfalse%
\ pcp{\isacharunderscore}seq{\isacharunderscore}in{\isacharcolon}\ {\isachardoublequoteopen}pcp\ C\ {\isasymLongrightarrow}\ S\ {\isasymin}\ C\ {\isasymLongrightarrow}\ pcp{\isacharunderscore}seq\ C\ S\ n\ {\isasymin}\ C{\isachardoublequoteclose}\isanewline
%
\isadelimproof
%
\endisadelimproof
%
\isatagproof
\isacommand{proof}\isamarkupfalse%
{\isacharparenleft}induction\ n{\isacharparenright}\isanewline
\ \ \isacommand{case}\isamarkupfalse%
\ {\isacharparenleft}Suc\ n{\isacharparenright}\ \ \isanewline
\ \ \isacommand{hence}\isamarkupfalse%
\ {\isachardoublequoteopen}pcp{\isacharunderscore}seq\ C\ S\ n\ {\isasymin}\ C{\isachardoublequoteclose}\ \isacommand{by}\isamarkupfalse%
\ simp\isanewline
\ \ \isacommand{thus}\isamarkupfalse%
\ {\isacharquery}case\ \isacommand{by}\isamarkupfalse%
\ {\isacharparenleft}simp\ add{\isacharcolon}\ Let{\isacharunderscore}def{\isacharparenright}\isanewline
\isacommand{qed}\isamarkupfalse%
\ simp%
\endisatagproof
{\isafoldproof}%
%
\isadelimproof
%
\endisadelimproof
%
\begin{isamarkuptext}%
Por otro lado, veamos la monotonía de dichas sucesiones.

  \begin{lema}
    Toda sucesión de conjuntos construida a partir de una colección y un conjunto según la
    definición \isa{{\isadigit{4}}{\isachardot}{\isadigit{1}}{\isachardot}{\isadigit{1}}} es monótona.
  \end{lema}

  En Isabelle, se formaliza de la siguiente forma.%
\end{isamarkuptext}\isamarkuptrue%
\isacommand{lemma}\isamarkupfalse%
\ {\isachardoublequoteopen}pcp{\isacharunderscore}seq\ C\ S\ n\ {\isasymsubseteq}\ pcp{\isacharunderscore}seq\ C\ S\ {\isacharparenleft}Suc\ n{\isacharparenright}{\isachardoublequoteclose}\isanewline
%
\isadelimproof
\ \ %
\endisadelimproof
%
\isatagproof
\isacommand{oops}\isamarkupfalse%
%
\endisatagproof
{\isafoldproof}%
%
\isadelimproof
%
\endisadelimproof
%
\begin{isamarkuptext}%
Procedamos con la demostración del lema.

  \begin{demostracion}
    Sea una colección de conjuntos \isa{C}, \isa{S\ {\isasymin}\ C} y \isa{{\isacharbraceleft}S\isactrlsub n{\isacharbraceright}} la sucesión de conjuntos de \isa{C} a partir de 
    \isa{S} según la definición \isa{{\isadigit{4}}{\isachardot}{\isadigit{1}}{\isachardot}{\isadigit{1}}}. Para probar que \isa{{\isacharbraceleft}S\isactrlsub n{\isacharbraceright}} es monótona, basta probar que \isa{S\isactrlsub n\ {\isasymsubseteq}\ S\isactrlsub n\isactrlsub {\isacharplus}\isactrlsub {\isadigit{1}}} 
    para todo \isa{n\ {\isasymin}\ {\isasymnat}}. En efecto, el resultado es inmediato al considerar dos casos para todo 
    \isa{n\ {\isasymin}\ {\isasymnat}}: \isa{S\isactrlsub n\ {\isasymunion}\ {\isacharbraceleft}F\isactrlsub n{\isacharbraceright}\ {\isasymin}\ C} o \isa{S\isactrlsub n\ {\isasymunion}\ {\isacharbraceleft}F\isactrlsub n{\isacharbraceright}\ {\isasymnotin}\ C}. Si suponemos que\\ \isa{S\isactrlsub n\ {\isasymunion}\ {\isacharbraceleft}F\isactrlsub n{\isacharbraceright}\ {\isasymin}\ C}, por definición 
    tenemos que \isa{S\isactrlsub n\isactrlsub {\isacharplus}\isactrlsub {\isadigit{1}}\ {\isacharequal}\ S\isactrlsub n\ {\isasymunion}\ {\isacharbraceleft}F\isactrlsub n{\isacharbraceright}}, luego es claro que\\ \isa{S\isactrlsub n\ {\isasymsubseteq}\ S\isactrlsub n\isactrlsub {\isacharplus}\isactrlsub {\isadigit{1}}}. En caso contrario, si 
    \isa{S\isactrlsub n\ {\isasymunion}\ {\isacharbraceleft}F\isactrlsub n{\isacharbraceright}\ {\isasymnotin}\ C}, por definición se tiene que \isa{S\isactrlsub n\isactrlsub {\isacharplus}\isactrlsub {\isadigit{1}}\ {\isacharequal}\ S\isactrlsub n}, obteniéndose igualmente el resultado
    por la propiedad reflexiva de la contención de conjuntos. 
  \end{demostracion}

  La prueba detallada en Isabelle se muestra a continuación.%
\end{isamarkuptext}\isamarkuptrue%
\isacommand{lemma}\isamarkupfalse%
\ {\isachardoublequoteopen}pcp{\isacharunderscore}seq\ C\ S\ n\ {\isasymsubseteq}\ pcp{\isacharunderscore}seq\ C\ S\ {\isacharparenleft}Suc\ n{\isacharparenright}{\isachardoublequoteclose}\isanewline
%
\isadelimproof
%
\endisadelimproof
%
\isatagproof
\isacommand{proof}\isamarkupfalse%
\ {\isacharminus}\isanewline
\ \ \isacommand{have}\isamarkupfalse%
\ {\isachardoublequoteopen}pcp{\isacharunderscore}seq\ C\ S\ {\isacharparenleft}Suc\ n{\isacharparenright}\ {\isacharequal}\ {\isacharparenleft}let\ Sn\ {\isacharequal}\ pcp{\isacharunderscore}seq\ C\ S\ n{\isacharsemicolon}\ Sn{\isadigit{1}}\ {\isacharequal}\ insert\ {\isacharparenleft}from{\isacharunderscore}nat\ n{\isacharparenright}\ Sn\ in\isanewline
\ \ \ \ \ \ \ \ \ \ \ \ \ \ \ \ \ \ \ \ \ \ \ \ if\ Sn{\isadigit{1}}\ {\isasymin}\ C\ then\ Sn{\isadigit{1}}\ else\ Sn{\isacharparenright}{\isachardoublequoteclose}\ \isanewline
\ \ \ \ \isacommand{by}\isamarkupfalse%
\ {\isacharparenleft}simp\ only{\isacharcolon}\ pcp{\isacharunderscore}seq{\isachardot}simps{\isacharparenleft}{\isadigit{2}}{\isacharparenright}{\isacharparenright}\isanewline
\ \ \isacommand{then}\isamarkupfalse%
\ \isacommand{have}\isamarkupfalse%
\ SucDef{\isacharcolon}{\isachardoublequoteopen}pcp{\isacharunderscore}seq\ C\ S\ {\isacharparenleft}Suc\ n{\isacharparenright}\ {\isacharequal}\ {\isacharparenleft}if\ insert\ {\isacharparenleft}from{\isacharunderscore}nat\ n{\isacharparenright}\ {\isacharparenleft}pcp{\isacharunderscore}seq\ C\ S\ n{\isacharparenright}\ {\isasymin}\ C\ then\ \isanewline
\ \ \ \ \ \ \ \ \ \ \ \ \ \ \ \ \ \ \ \ insert\ {\isacharparenleft}from{\isacharunderscore}nat\ n{\isacharparenright}\ {\isacharparenleft}pcp{\isacharunderscore}seq\ C\ S\ n{\isacharparenright}\ else\ pcp{\isacharunderscore}seq\ C\ S\ n{\isacharparenright}{\isachardoublequoteclose}\ \isanewline
\ \ \ \ \isacommand{by}\isamarkupfalse%
\ {\isacharparenleft}simp\ only{\isacharcolon}\ Let{\isacharunderscore}def{\isacharparenright}\isanewline
\ \ \isacommand{thus}\isamarkupfalse%
\ {\isachardoublequoteopen}pcp{\isacharunderscore}seq\ C\ S\ n\ {\isasymsubseteq}\ pcp{\isacharunderscore}seq\ C\ S\ {\isacharparenleft}Suc\ n{\isacharparenright}{\isachardoublequoteclose}\isanewline
\ \ \isacommand{proof}\isamarkupfalse%
\ {\isacharparenleft}cases{\isacharparenright}\isanewline
\ \ \ \ \isacommand{assume}\isamarkupfalse%
\ {\isadigit{1}}{\isacharcolon}{\isachardoublequoteopen}insert\ {\isacharparenleft}from{\isacharunderscore}nat\ n{\isacharparenright}\ {\isacharparenleft}pcp{\isacharunderscore}seq\ C\ S\ n{\isacharparenright}\ {\isasymin}\ C{\isachardoublequoteclose}\isanewline
\ \ \ \ \isacommand{have}\isamarkupfalse%
\ {\isachardoublequoteopen}pcp{\isacharunderscore}seq\ C\ S\ {\isacharparenleft}Suc\ n{\isacharparenright}\ {\isacharequal}\ insert\ {\isacharparenleft}from{\isacharunderscore}nat\ n{\isacharparenright}\ {\isacharparenleft}pcp{\isacharunderscore}seq\ C\ S\ n{\isacharparenright}{\isachardoublequoteclose}\isanewline
\ \ \ \ \ \ \isacommand{using}\isamarkupfalse%
\ SucDef\ {\isadigit{1}}\ \isacommand{by}\isamarkupfalse%
\ {\isacharparenleft}simp\ only{\isacharcolon}\ if{\isacharunderscore}True{\isacharparenright}\isanewline
\ \ \ \ \isacommand{thus}\isamarkupfalse%
\ {\isachardoublequoteopen}pcp{\isacharunderscore}seq\ C\ S\ n\ {\isasymsubseteq}\ pcp{\isacharunderscore}seq\ C\ S\ {\isacharparenleft}Suc\ n{\isacharparenright}{\isachardoublequoteclose}\isanewline
\ \ \ \ \ \ \isacommand{by}\isamarkupfalse%
\ {\isacharparenleft}simp\ only{\isacharcolon}\ subset{\isacharunderscore}insertI{\isacharparenright}\isanewline
\ \ \isacommand{next}\isamarkupfalse%
\isanewline
\ \ \ \ \isacommand{assume}\isamarkupfalse%
\ {\isadigit{2}}{\isacharcolon}{\isachardoublequoteopen}insert\ {\isacharparenleft}from{\isacharunderscore}nat\ n{\isacharparenright}\ {\isacharparenleft}pcp{\isacharunderscore}seq\ C\ S\ n{\isacharparenright}\ {\isasymnotin}\ C{\isachardoublequoteclose}\isanewline
\ \ \ \ \isacommand{have}\isamarkupfalse%
\ {\isachardoublequoteopen}pcp{\isacharunderscore}seq\ C\ S\ {\isacharparenleft}Suc\ n{\isacharparenright}\ {\isacharequal}\ pcp{\isacharunderscore}seq\ C\ S\ n{\isachardoublequoteclose}\isanewline
\ \ \ \ \ \ \isacommand{using}\isamarkupfalse%
\ SucDef\ {\isadigit{2}}\ \isacommand{by}\isamarkupfalse%
\ {\isacharparenleft}simp\ only{\isacharcolon}\ if{\isacharunderscore}False{\isacharparenright}\isanewline
\ \ \ \ \isacommand{thus}\isamarkupfalse%
\ {\isachardoublequoteopen}pcp{\isacharunderscore}seq\ C\ S\ n\ {\isasymsubseteq}\ pcp{\isacharunderscore}seq\ C\ S\ {\isacharparenleft}Suc\ n{\isacharparenright}{\isachardoublequoteclose}\isanewline
\ \ \ \ \ \ \isacommand{by}\isamarkupfalse%
\ {\isacharparenleft}simp\ only{\isacharcolon}\ subset{\isacharunderscore}refl{\isacharparenright}\isanewline
\ \ \isacommand{qed}\isamarkupfalse%
\isanewline
\isacommand{qed}\isamarkupfalse%
%
\endisatagproof
{\isafoldproof}%
%
\isadelimproof
%
\endisadelimproof
%
\begin{isamarkuptext}%
Del mismo modo, se puede probar automáticamente en Isabelle/HOL.%
\end{isamarkuptext}\isamarkuptrue%
\isacommand{lemma}\isamarkupfalse%
\ pcp{\isacharunderscore}seq{\isacharunderscore}monotonicity{\isacharcolon}{\isachardoublequoteopen}pcp{\isacharunderscore}seq\ C\ S\ n\ {\isasymsubseteq}\ pcp{\isacharunderscore}seq\ C\ S\ {\isacharparenleft}Suc\ n{\isacharparenright}{\isachardoublequoteclose}\isanewline
%
\isadelimproof
\ \ %
\endisadelimproof
%
\isatagproof
\isacommand{by}\isamarkupfalse%
\ {\isacharparenleft}smt\ eq{\isacharunderscore}iff\ pcp{\isacharunderscore}seq{\isachardot}simps{\isacharparenleft}{\isadigit{2}}{\isacharparenright}\ subset{\isacharunderscore}insertI{\isacharparenright}%
\endisatagproof
{\isafoldproof}%
%
\isadelimproof
%
\endisadelimproof
%
\begin{isamarkuptext}%
Por otra lado, para facilitar posteriores demostraciones en Isabelle/HOL, vamos a formalizar 
  el lema anterior empleando la siguiente definición generalizada de monotonía.%
\end{isamarkuptext}\isamarkuptrue%
\isacommand{lemma}\isamarkupfalse%
\ pcp{\isacharunderscore}seq{\isacharunderscore}mono{\isacharcolon}\isanewline
\ \ \isakeyword{assumes}\ {\isachardoublequoteopen}n\ {\isasymle}\ m{\isachardoublequoteclose}\ \isanewline
\ \ \isakeyword{shows}\ {\isachardoublequoteopen}pcp{\isacharunderscore}seq\ C\ S\ n\ {\isasymsubseteq}\ pcp{\isacharunderscore}seq\ C\ S\ m{\isachardoublequoteclose}\isanewline
%
\isadelimproof
\ \ %
\endisadelimproof
%
\isatagproof
\isacommand{using}\isamarkupfalse%
\ pcp{\isacharunderscore}seq{\isacharunderscore}monotonicity\ assms\ \isacommand{by}\isamarkupfalse%
\ {\isacharparenleft}rule\ lift{\isacharunderscore}Suc{\isacharunderscore}mono{\isacharunderscore}le{\isacharparenright}%
\endisatagproof
{\isafoldproof}%
%
\isadelimproof
%
\endisadelimproof
%
\begin{isamarkuptext}%
A continuación daremos un lema que permite caracterizar un elemento de la sucesión en función 
  de los anteriores.

\begin{lema}
  Sea \isa{C} una colección de conjuntos, \isa{S\ {\isasymin}\ C} y \isa{{\isacharbraceleft}S\isactrlsub n{\isacharbraceright}} la sucesión de conjuntos de \isa{C} a partir de 
  \isa{S} construida según la definición \isa{{\isadigit{4}}{\isachardot}{\isadigit{1}}{\isachardot}{\isadigit{1}}}. Entonces, para todos \isa{n}, \isa{m\ {\isasymin}\ {\isasymnat}} 
  se verifica $\bigcup_{n \leq m} S_{n} = S_{m}$
\end{lema}

\begin{demostracion}
  En las condiciones del enunciado, la prueba se realiza por inducción en \isa{m\ {\isasymin}\ {\isasymnat}}.

  En primer lugar, consideremos el caso base \isa{m\ {\isacharequal}\ {\isadigit{0}}}. El resultado se obtiene directamente:

  $\bigcup_{n \leq 0} S_{n} = \bigcup_{n = 0} S_{n} = S_{0} = S_{m}$

  Por otro lado, supongamos por hipótesis de inducción que $\bigcup_{n \leq m} S_{n} = S_{m}$.
  Veamos que se verifica $\bigcup_{n \leq m + 1} S_{n} = S_{m + 1}$. Observemos que si \isa{n\ {\isasymle}\ m\ {\isacharplus}\ {\isadigit{1}}},
  entonces se tiene que, o bien \isa{n\ {\isasymle}\ m}, o bien \isa{n\ {\isacharequal}\ m\ {\isacharplus}\ {\isadigit{1}}}. De este modo, aplicando la 
  hipótesis de inducción, deducimos lo siguiente.

  $\bigcup_{n \leq m + 1} S_{n} = \bigcup_{n \leq m} S_{n} \cup \bigcup_{n = m + 1} S_{n} = \bigcup_{n \leq m} S_{n} \cup S_{m + 1} = S_{m} \cup S_{m + 1}$

  Por la monotonía de la sucesión, se tiene que \isa{S\isactrlsub m\ {\isasymsubseteq}\ S\isactrlsub m\isactrlsub {\isacharplus}\isactrlsub {\isadigit{1}}}. Luego, se verifica:

  $\bigcup_{n \leq m + 1} S_{n} = S_{m} \cup S_{m + 1} = S_{m + 1}$

  Lo que prueba el resultado.
\end{demostracion}

  Procedamos a su formalización y demostración detallada. Para ello, emplearemos la unión 
  generalizada en Isabelle/HOL perteneciente a la teoría 
  \href{https://n9.cl/gtf5x}{Complete-Lattices.thy}, junto con distintas propiedades sobre la misma
  definidas en dicha teoría. El uso de teoría de retículos en este caso se debe a que, en Isabelle,
  los conjuntos se han formalizado como predicados según la teoría 
  \href{https://bit.ly/3ibCuje}{Set.thy}. De esta manera, un elemento pertenece a un conjunto si 
  verifica el predicado que lo caracteriza. Además, en dicha teoría se instancia que el tipo de los 
  conjuntos es un álgebra de \isa{Boole} acotada, es decir, es un retículo distributivo para las 
  operaciones unión e intersección que tiene un supremo y un ínfimo. En consecuencia, la unión 
  generalizada de conjuntos se formaliza en Isabelle como el supremo del retículo completo que 
  conforman.

  Veamos la prueba detallada del resultado en Isabelle/HOL.%
\end{isamarkuptext}\isamarkuptrue%
\isacommand{lemma}\isamarkupfalse%
\ {\isachardoublequoteopen}{\isasymUnion}{\isacharbraceleft}pcp{\isacharunderscore}seq\ C\ S\ n{\isacharbar}n{\isachardot}\ n\ {\isasymle}\ m{\isacharbraceright}\ {\isacharequal}\ pcp{\isacharunderscore}seq\ C\ S\ m{\isachardoublequoteclose}\isanewline
%
\isadelimproof
%
\endisadelimproof
%
\isatagproof
\isacommand{proof}\isamarkupfalse%
\ {\isacharparenleft}induct\ m{\isacharparenright}\isanewline
\ \ \isacommand{have}\isamarkupfalse%
\ \ {\isachardoublequoteopen}{\isasymUnion}{\isacharbraceleft}pcp{\isacharunderscore}seq\ C\ S\ n{\isacharbar}n{\isachardot}\ n\ {\isasymle}\ {\isadigit{0}}{\isacharbraceright}\ {\isacharequal}\ {\isasymUnion}{\isacharbraceleft}pcp{\isacharunderscore}seq\ C\ S\ n{\isacharbar}n{\isachardot}\ n\ {\isacharequal}\ {\isadigit{0}}{\isacharbraceright}{\isachardoublequoteclose}\isanewline
\ \ \ \ \isacommand{by}\isamarkupfalse%
\ {\isacharparenleft}simp\ only{\isacharcolon}\ le{\isacharunderscore}zero{\isacharunderscore}eq{\isacharparenright}\isanewline
\ \ \isacommand{also}\isamarkupfalse%
\ \isacommand{have}\isamarkupfalse%
\ {\isachardoublequoteopen}{\isasymdots}\ {\isacharequal}\ {\isasymUnion}{\isacharparenleft}{\isacharparenleft}pcp{\isacharunderscore}seq\ C\ S{\isacharparenright}{\isacharbackquote}{\isacharbraceleft}n{\isachardot}\ n\ {\isacharequal}\ {\isadigit{0}}{\isacharbraceright}{\isacharparenright}{\isachardoublequoteclose}\isanewline
\ \ \ \ \isacommand{by}\isamarkupfalse%
\ {\isacharparenleft}simp\ only{\isacharcolon}\ image{\isacharunderscore}Collect{\isacharparenright}\isanewline
\ \ \isacommand{also}\isamarkupfalse%
\ \isacommand{have}\isamarkupfalse%
\ {\isachardoublequoteopen}{\isasymdots}\ {\isacharequal}\ {\isasymUnion}{\isacharbraceleft}pcp{\isacharunderscore}seq\ C\ S\ {\isadigit{0}}{\isacharbraceright}{\isachardoublequoteclose}\isanewline
\ \ \ \ \isacommand{by}\isamarkupfalse%
\ {\isacharparenleft}simp\ only{\isacharcolon}\ singleton{\isacharunderscore}conv\ image{\isacharunderscore}insert\ image{\isacharunderscore}empty{\isacharparenright}\isanewline
\ \ \isacommand{also}\isamarkupfalse%
\ \isacommand{have}\isamarkupfalse%
\ {\isachardoublequoteopen}{\isasymdots}\ {\isacharequal}\ pcp{\isacharunderscore}seq\ C\ S\ {\isadigit{0}}{\isachardoublequoteclose}\ \isanewline
\ \ \ \ \isacommand{by}\isamarkupfalse%
\ \ {\isacharparenleft}simp\ only{\isacharcolon}cSup{\isacharunderscore}singleton{\isacharparenright}\isanewline
\ \ \isacommand{finally}\isamarkupfalse%
\ \isacommand{show}\isamarkupfalse%
\ {\isachardoublequoteopen}{\isasymUnion}{\isacharbraceleft}pcp{\isacharunderscore}seq\ C\ S\ n{\isacharbar}n{\isachardot}\ n\ {\isasymle}\ {\isadigit{0}}{\isacharbraceright}\ {\isacharequal}\ pcp{\isacharunderscore}seq\ C\ S\ {\isadigit{0}}{\isachardoublequoteclose}\ \isanewline
\ \ \ \ \isacommand{by}\isamarkupfalse%
\ this\isanewline
\isacommand{next}\isamarkupfalse%
\isanewline
\ \ \isacommand{fix}\isamarkupfalse%
\ m\isanewline
\ \ \isacommand{assume}\isamarkupfalse%
\ HI{\isacharcolon}{\isachardoublequoteopen}{\isasymUnion}{\isacharbraceleft}pcp{\isacharunderscore}seq\ C\ S\ n{\isacharbar}n{\isachardot}\ n\ {\isasymle}\ m{\isacharbraceright}\ {\isacharequal}\ pcp{\isacharunderscore}seq\ C\ S\ m{\isachardoublequoteclose}\isanewline
\ \ \isacommand{have}\isamarkupfalse%
\ {\isachardoublequoteopen}m\ {\isasymle}\ Suc\ m{\isachardoublequoteclose}\ \isanewline
\ \ \ \ \isacommand{by}\isamarkupfalse%
\ {\isacharparenleft}simp\ only{\isacharcolon}\ add{\isacharunderscore}{\isadigit{0}}{\isacharunderscore}right{\isacharparenright}\isanewline
\ \ \isacommand{then}\isamarkupfalse%
\ \isacommand{have}\isamarkupfalse%
\ Mon{\isacharcolon}{\isachardoublequoteopen}pcp{\isacharunderscore}seq\ C\ S\ m\ {\isasymsubseteq}\ pcp{\isacharunderscore}seq\ C\ S\ {\isacharparenleft}Suc\ m{\isacharparenright}{\isachardoublequoteclose}\isanewline
\ \ \ \ \isacommand{by}\isamarkupfalse%
\ {\isacharparenleft}rule\ pcp{\isacharunderscore}seq{\isacharunderscore}mono{\isacharparenright}\isanewline
\ \ \isacommand{have}\isamarkupfalse%
\ {\isachardoublequoteopen}{\isasymUnion}{\isacharbraceleft}pcp{\isacharunderscore}seq\ C\ S\ n\ {\isacharbar}\ n{\isachardot}\ n\ {\isasymle}\ Suc\ m{\isacharbraceright}\ {\isacharequal}\ {\isasymUnion}{\isacharparenleft}{\isacharparenleft}pcp{\isacharunderscore}seq\ C\ S{\isacharparenright}{\isacharbackquote}{\isacharparenleft}{\isacharbraceleft}n{\isachardot}\ n\ {\isasymle}\ Suc\ m{\isacharbraceright}{\isacharparenright}{\isacharparenright}{\isachardoublequoteclose}\isanewline
\ \ \ \ \isacommand{by}\isamarkupfalse%
\ {\isacharparenleft}simp\ only{\isacharcolon}\ image{\isacharunderscore}Collect{\isacharparenright}\isanewline
\ \ \isacommand{also}\isamarkupfalse%
\ \isacommand{have}\isamarkupfalse%
\ {\isachardoublequoteopen}{\isasymdots}\ {\isacharequal}\ {\isasymUnion}{\isacharparenleft}{\isacharparenleft}pcp{\isacharunderscore}seq\ C\ S{\isacharparenright}{\isacharbackquote}{\isacharparenleft}{\isacharbraceleft}Suc\ m{\isacharbraceright}\ {\isasymunion}\ {\isacharbraceleft}n{\isachardot}\ n\ {\isasymle}\ m{\isacharbraceright}{\isacharparenright}{\isacharparenright}{\isachardoublequoteclose}\isanewline
\ \ \ \ \isacommand{by}\isamarkupfalse%
\ {\isacharparenleft}simp\ only{\isacharcolon}\ le{\isacharunderscore}Suc{\isacharunderscore}eq\ Collect{\isacharunderscore}disj{\isacharunderscore}eq\ Un{\isacharunderscore}commute\ singleton{\isacharunderscore}conv{\isacharparenright}\isanewline
\ \ \isacommand{also}\isamarkupfalse%
\ \isacommand{have}\isamarkupfalse%
\ {\isachardoublequoteopen}{\isasymdots}\ {\isacharequal}\ {\isasymUnion}{\isacharparenleft}{\isacharbraceleft}pcp{\isacharunderscore}seq\ C\ S\ {\isacharparenleft}Suc\ m{\isacharparenright}{\isacharbraceright}\ {\isasymunion}\ {\isacharbraceleft}pcp{\isacharunderscore}seq\ C\ S\ n\ {\isacharbar}\ n{\isachardot}\ n\ {\isasymle}\ m{\isacharbraceright}{\isacharparenright}{\isachardoublequoteclose}\isanewline
\ \ \ \ \isacommand{by}\isamarkupfalse%
\ {\isacharparenleft}simp\ only{\isacharcolon}\ image{\isacharunderscore}Un\ image{\isacharunderscore}insert\ image{\isacharunderscore}empty\ image{\isacharunderscore}Collect{\isacharparenright}\isanewline
\ \ \isacommand{also}\isamarkupfalse%
\ \isacommand{have}\isamarkupfalse%
\ {\isachardoublequoteopen}{\isasymdots}\ {\isacharequal}\ {\isasymUnion}{\isacharbraceleft}pcp{\isacharunderscore}seq\ C\ S\ {\isacharparenleft}Suc\ m{\isacharparenright}{\isacharbraceright}\ {\isasymunion}\ {\isasymUnion}{\isacharbraceleft}pcp{\isacharunderscore}seq\ C\ S\ n\ {\isacharbar}\ n{\isachardot}\ n\ {\isasymle}\ m{\isacharbraceright}{\isachardoublequoteclose}\isanewline
\ \ \ \ \isacommand{by}\isamarkupfalse%
\ {\isacharparenleft}simp\ only{\isacharcolon}\ Union{\isacharunderscore}Un{\isacharunderscore}distrib{\isacharparenright}\isanewline
\ \ \isacommand{also}\isamarkupfalse%
\ \isacommand{have}\isamarkupfalse%
\ {\isachardoublequoteopen}{\isasymdots}\ {\isacharequal}\ {\isacharparenleft}pcp{\isacharunderscore}seq\ C\ S\ {\isacharparenleft}Suc\ m{\isacharparenright}{\isacharparenright}\ {\isasymunion}\ {\isasymUnion}{\isacharbraceleft}pcp{\isacharunderscore}seq\ C\ S\ n\ {\isacharbar}\ n{\isachardot}\ n\ {\isasymle}\ m{\isacharbraceright}{\isachardoublequoteclose}\isanewline
\ \ \ \ \isacommand{by}\isamarkupfalse%
\ {\isacharparenleft}simp\ only{\isacharcolon}\ cSup{\isacharunderscore}singleton{\isacharparenright}\isanewline
\ \ \isacommand{also}\isamarkupfalse%
\ \isacommand{have}\isamarkupfalse%
\ {\isachardoublequoteopen}{\isasymdots}\ {\isacharequal}\ {\isacharparenleft}pcp{\isacharunderscore}seq\ C\ S\ {\isacharparenleft}Suc\ m{\isacharparenright}{\isacharparenright}\ {\isasymunion}\ {\isacharparenleft}pcp{\isacharunderscore}seq\ C\ S\ m{\isacharparenright}{\isachardoublequoteclose}\isanewline
\ \ \ \ \isacommand{by}\isamarkupfalse%
\ {\isacharparenleft}simp\ only{\isacharcolon}\ HI{\isacharparenright}\isanewline
\ \ \isacommand{also}\isamarkupfalse%
\ \isacommand{have}\isamarkupfalse%
\ {\isachardoublequoteopen}{\isasymdots}\ {\isacharequal}\ pcp{\isacharunderscore}seq\ C\ S\ {\isacharparenleft}Suc\ m{\isacharparenright}{\isachardoublequoteclose}\isanewline
\ \ \ \ \isacommand{using}\isamarkupfalse%
\ Mon\ \isacommand{by}\isamarkupfalse%
\ {\isacharparenleft}simp\ only{\isacharcolon}\ Un{\isacharunderscore}absorb{\isadigit{2}}{\isacharparenright}\isanewline
\ \ \isacommand{finally}\isamarkupfalse%
\ \isacommand{show}\isamarkupfalse%
\ {\isachardoublequoteopen}{\isasymUnion}{\isacharbraceleft}pcp{\isacharunderscore}seq\ C\ S\ n{\isacharbar}n{\isachardot}\ n\ {\isasymle}\ {\isacharparenleft}Suc\ m{\isacharparenright}{\isacharbraceright}\ {\isacharequal}\ pcp{\isacharunderscore}seq\ C\ S\ {\isacharparenleft}Suc\ m{\isacharparenright}{\isachardoublequoteclose}\isanewline
\ \ \ \ \isacommand{by}\isamarkupfalse%
\ this\isanewline
\isacommand{qed}\isamarkupfalse%
%
\endisatagproof
{\isafoldproof}%
%
\isadelimproof
%
\endisadelimproof
%
\begin{isamarkuptext}%
Análogamente, podemos dar una prueba automática.%
\end{isamarkuptext}\isamarkuptrue%
\isacommand{lemma}\isamarkupfalse%
\ pcp{\isacharunderscore}seq{\isacharunderscore}UN{\isacharcolon}\ {\isachardoublequoteopen}{\isasymUnion}{\isacharbraceleft}pcp{\isacharunderscore}seq\ C\ S\ n{\isacharbar}n{\isachardot}\ n\ {\isasymle}\ m{\isacharbraceright}\ {\isacharequal}\ pcp{\isacharunderscore}seq\ C\ S\ m{\isachardoublequoteclose}\isanewline
%
\isadelimproof
%
\endisadelimproof
%
\isatagproof
\isacommand{proof}\isamarkupfalse%
{\isacharparenleft}induction\ m{\isacharparenright}\isanewline
\ \ \isacommand{case}\isamarkupfalse%
\ {\isacharparenleft}Suc\ m{\isacharparenright}\isanewline
\ \ \isacommand{have}\isamarkupfalse%
\ {\isachardoublequoteopen}{\isacharbraceleft}f\ n\ {\isacharbar}n{\isachardot}\ n\ {\isasymle}\ Suc\ m{\isacharbraceright}\ {\isacharequal}\ insert\ {\isacharparenleft}f\ {\isacharparenleft}Suc\ m{\isacharparenright}{\isacharparenright}\ {\isacharbraceleft}f\ n\ {\isacharbar}n{\isachardot}\ n\ {\isasymle}\ m{\isacharbraceright}{\isachardoublequoteclose}\ \isanewline
\ \ \ \ \isakeyword{for}\ f\ \isacommand{using}\isamarkupfalse%
\ le{\isacharunderscore}Suc{\isacharunderscore}eq\ \isacommand{by}\isamarkupfalse%
\ auto\isanewline
\ \ \isacommand{hence}\isamarkupfalse%
\ {\isachardoublequoteopen}{\isacharbraceleft}pcp{\isacharunderscore}seq\ C\ S\ n\ {\isacharbar}n{\isachardot}\ n\ {\isasymle}\ Suc\ m{\isacharbraceright}\ {\isacharequal}\ \isanewline
\ \ \ \ \ \ \ \ \ \ insert\ {\isacharparenleft}pcp{\isacharunderscore}seq\ C\ S\ {\isacharparenleft}Suc\ m{\isacharparenright}{\isacharparenright}\ {\isacharbraceleft}pcp{\isacharunderscore}seq\ C\ S\ n\ {\isacharbar}n{\isachardot}\ n\ {\isasymle}\ m{\isacharbraceright}{\isachardoublequoteclose}\ \isacommand{{\isachardot}}\isamarkupfalse%
\isanewline
\ \ \isacommand{hence}\isamarkupfalse%
\ {\isachardoublequoteopen}{\isasymUnion}{\isacharbraceleft}pcp{\isacharunderscore}seq\ C\ S\ n\ {\isacharbar}n{\isachardot}\ n\ {\isasymle}\ Suc\ m{\isacharbraceright}\ {\isacharequal}\ \isanewline
\ \ \ \ \ \ \ \ \ {\isasymUnion}{\isacharbraceleft}pcp{\isacharunderscore}seq\ C\ S\ n\ {\isacharbar}n{\isachardot}\ n\ {\isasymle}\ m{\isacharbraceright}\ {\isasymunion}\ pcp{\isacharunderscore}seq\ C\ S\ {\isacharparenleft}Suc\ m{\isacharparenright}{\isachardoublequoteclose}\ \isacommand{by}\isamarkupfalse%
\ auto\isanewline
\ \ \isacommand{thus}\isamarkupfalse%
\ {\isacharquery}case\ \isacommand{using}\isamarkupfalse%
\ Suc\ pcp{\isacharunderscore}seq{\isacharunderscore}mono\ \isacommand{by}\isamarkupfalse%
\ blast\isanewline
\isacommand{qed}\isamarkupfalse%
\ simp%
\endisatagproof
{\isafoldproof}%
%
\isadelimproof
%
\endisadelimproof
%
\begin{isamarkuptext}%
Finalmente, definamos el límite de las sucesiones presentadas en la definición \isa{{\isadigit{4}}{\isachardot}{\isadigit{1}}{\isachardot}{\isadigit{1}}}.

 \begin{definicion}
  Sea \isa{C} una colección, \isa{S\ {\isasymin}\ C} y \isa{{\isacharbraceleft}S\isactrlsub n{\isacharbraceright}} la sucesión de conjuntos de \isa{C} a partir de \isa{S} según la
  definición \isa{{\isadigit{4}}{\isachardot}{\isadigit{1}}{\isachardot}{\isadigit{1}}}. Se define el \isa{límite\ de\ la\ sucesión\ de\ conjuntos\ de\ C\ a\ partir\ de\ S} como 
  $\bigcup_{n = 0}^{\infty} S_{n}$
 \end{definicion}

  La definición del límite se formaliza utilizando la unión generalizada de Isabelle como sigue.%
\end{isamarkuptext}\isamarkuptrue%
\isacommand{definition}\isamarkupfalse%
\ {\isachardoublequoteopen}pcp{\isacharunderscore}lim\ C\ S\ {\isasymequiv}\ {\isasymUnion}{\isacharbraceleft}pcp{\isacharunderscore}seq\ C\ S\ n{\isacharbar}n{\isachardot}\ True{\isacharbraceright}{\isachardoublequoteclose}%
\begin{isamarkuptext}%
Veamos el primer resultado sobre el límite.

\begin{lema}
  Sea \isa{C} una colección de conjuntos, \isa{S\ {\isasymin}\ C} y \isa{{\isacharbraceleft}S\isactrlsub n{\isacharbraceright}} la sucesión de conjuntos de \isa{C} a partir de
  \isa{S} según la definición \isa{{\isadigit{4}}{\isachardot}{\isadigit{1}}{\isachardot}{\isadigit{1}}}. Entonces, para todo \isa{n\ {\isasymin}\ {\isasymnat}}, se verifica:

  $S_{n} \subseteq \bigcup_{n = 0}^{\infty} S_{n}$
\end{lema}

\begin{demostracion}
  El resultado se obtiene de manera inmediata ya que, para todo \isa{n\ {\isasymin}\ {\isasymnat}}, se verifica que 
  $S_{n} \in \{S_{n}\}_{n = 0}^{\infty}$. Por tanto, es claro que 
  $S_{n} \subseteq \bigcup_{n = 0}^{\infty} S_{n}$.
\end{demostracion}

  Su formalización y demostración detallada en Isabelle se muestran a continuación.%
\end{isamarkuptext}\isamarkuptrue%
\isacommand{lemma}\isamarkupfalse%
\ {\isachardoublequoteopen}pcp{\isacharunderscore}seq\ C\ S\ n\ {\isasymsubseteq}\ pcp{\isacharunderscore}lim\ C\ S{\isachardoublequoteclose}\isanewline
%
\isadelimproof
\ \ %
\endisadelimproof
%
\isatagproof
\isacommand{unfolding}\isamarkupfalse%
\ pcp{\isacharunderscore}lim{\isacharunderscore}def\isanewline
\isacommand{proof}\isamarkupfalse%
\ {\isacharminus}\isanewline
\ \ \isacommand{have}\isamarkupfalse%
\ {\isachardoublequoteopen}n\ {\isasymin}\ {\isacharbraceleft}n\ {\isacharbar}\ n{\isachardot}\ True{\isacharbraceright}{\isachardoublequoteclose}\ \isanewline
\ \ \ \ \isacommand{by}\isamarkupfalse%
\ {\isacharparenleft}simp\ only{\isacharcolon}\ simp{\isacharunderscore}thms{\isacharparenleft}{\isadigit{2}}{\isadigit{1}}{\isacharcomma}{\isadigit{3}}{\isadigit{8}}{\isacharparenright}\ Collect{\isacharunderscore}const\ if{\isacharunderscore}True\ UNIV{\isacharunderscore}I{\isacharparenright}\ \isanewline
\ \ \isacommand{then}\isamarkupfalse%
\ \isacommand{have}\isamarkupfalse%
\ {\isachardoublequoteopen}pcp{\isacharunderscore}seq\ C\ S\ n\ {\isasymin}\ {\isacharparenleft}pcp{\isacharunderscore}seq\ C\ S{\isacharparenright}{\isacharbackquote}{\isacharbraceleft}n\ {\isacharbar}\ n{\isachardot}\ True{\isacharbraceright}{\isachardoublequoteclose}\isanewline
\ \ \ \ \isacommand{by}\isamarkupfalse%
\ {\isacharparenleft}simp\ only{\isacharcolon}\ imageI{\isacharparenright}\isanewline
\ \ \isacommand{then}\isamarkupfalse%
\ \isacommand{have}\isamarkupfalse%
\ {\isachardoublequoteopen}pcp{\isacharunderscore}seq\ C\ S\ n\ {\isasymin}\ {\isacharbraceleft}pcp{\isacharunderscore}seq\ C\ S\ n\ {\isacharbar}\ n{\isachardot}\ True{\isacharbraceright}{\isachardoublequoteclose}\isanewline
\ \ \ \ \isacommand{by}\isamarkupfalse%
\ {\isacharparenleft}simp\ only{\isacharcolon}\ image{\isacharunderscore}Collect\ simp{\isacharunderscore}thms{\isacharparenleft}{\isadigit{4}}{\isadigit{0}}{\isacharparenright}{\isacharparenright}\isanewline
\ \ \isacommand{thus}\isamarkupfalse%
\ {\isachardoublequoteopen}pcp{\isacharunderscore}seq\ C\ S\ n\ {\isasymsubseteq}\ {\isasymUnion}{\isacharbraceleft}pcp{\isacharunderscore}seq\ C\ S\ n\ {\isacharbar}\ n{\isachardot}\ True{\isacharbraceright}{\isachardoublequoteclose}\isanewline
\ \ \ \ \isacommand{by}\isamarkupfalse%
\ {\isacharparenleft}simp\ only{\isacharcolon}\ Union{\isacharunderscore}upper{\isacharparenright}\isanewline
\isacommand{qed}\isamarkupfalse%
%
\endisatagproof
{\isafoldproof}%
%
\isadelimproof
%
\endisadelimproof
%
\begin{isamarkuptext}%
Podemos probarlo de manera automática como sigue.%
\end{isamarkuptext}\isamarkuptrue%
\isacommand{lemma}\isamarkupfalse%
\ pcp{\isacharunderscore}seq{\isacharunderscore}sub{\isacharcolon}\ {\isachardoublequoteopen}pcp{\isacharunderscore}seq\ C\ S\ n\ {\isasymsubseteq}\ pcp{\isacharunderscore}lim\ C\ S{\isachardoublequoteclose}\ \isanewline
%
\isadelimproof
\ \ %
\endisadelimproof
%
\isatagproof
\isacommand{unfolding}\isamarkupfalse%
\ pcp{\isacharunderscore}lim{\isacharunderscore}def\ \isacommand{by}\isamarkupfalse%
\ blast%
\endisatagproof
{\isafoldproof}%
%
\isadelimproof
%
\endisadelimproof
%
\begin{isamarkuptext}%
Mostremos otro resultado. 

  \begin{lema}
    Sea \isa{C} una colección de conjuntos de fórmulas proposicionales, \isa{S\ {\isasymin}\ C} y \isa{{\isacharbraceleft}S\isactrlsub n{\isacharbraceright}} la sucesión de 
    conjuntos de \isa{C} a partir de \isa{S} según la definición \isa{{\isadigit{4}}{\isachardot}{\isadigit{1}}{\isachardot}{\isadigit{1}}}. Si \isa{F} es una fórmula tal que
    $F \in \bigcup_{n = 0}^{\infty} S_{n}$, entonces existe un \isa{k\ {\isasymin}\ {\isasymnat}} tal que \isa{F\ {\isasymin}\ S\isactrlsub k}. 
  \end{lema}

  \begin{demostracion}
    La prueba es inmediata de la definición del límite de la sucesión de conjuntos \isa{{\isacharbraceleft}S\isactrlsub n{\isacharbraceright}}: si
    \isa{F} pertenece a la unión generalizada $\bigcup_{n = 0}^{\infty} S_{n}$, entonces existe algún
    conjunto \isa{S\isactrlsub k} tal que \isa{F\ {\isasymin}\ S\isactrlsub k}. Es decir, existe \isa{k\ {\isasymin}\ {\isasymnat}} tal que \isa{F\ {\isasymin}\ S\isactrlsub k}, como queríamos
    demostrar.
  \end{demostracion} 

  Su prueba detallada en Isabelle/HOL es la siguiente.%
\end{isamarkuptext}\isamarkuptrue%
\isacommand{lemma}\isamarkupfalse%
\ \isanewline
\ \ \isakeyword{assumes}\ {\isachardoublequoteopen}F\ {\isasymin}\ pcp{\isacharunderscore}lim\ C\ S{\isachardoublequoteclose}\isanewline
\ \ \isakeyword{shows}\ {\isachardoublequoteopen}{\isasymexists}k{\isachardot}\ F\ {\isasymin}\ pcp{\isacharunderscore}seq\ C\ S\ k{\isachardoublequoteclose}\ \isanewline
%
\isadelimproof
%
\endisadelimproof
%
\isatagproof
\isacommand{proof}\isamarkupfalse%
\ {\isacharminus}\isanewline
\ \ \isacommand{have}\isamarkupfalse%
\ {\isachardoublequoteopen}F\ {\isasymin}\ {\isasymUnion}{\isacharparenleft}{\isacharparenleft}pcp{\isacharunderscore}seq\ C\ S{\isacharparenright}\ {\isacharbackquote}\ {\isacharbraceleft}n\ {\isacharbar}\ n{\isachardot}\ True{\isacharbraceright}{\isacharparenright}{\isachardoublequoteclose}\isanewline
\ \ \ \ \isacommand{using}\isamarkupfalse%
\ assms\ \isacommand{by}\isamarkupfalse%
\ {\isacharparenleft}simp\ only{\isacharcolon}\ pcp{\isacharunderscore}lim{\isacharunderscore}def\ image{\isacharunderscore}Collect\ simp{\isacharunderscore}thms{\isacharparenleft}{\isadigit{4}}{\isadigit{0}}{\isacharparenright}{\isacharparenright}\isanewline
\ \ \isacommand{then}\isamarkupfalse%
\ \isacommand{have}\isamarkupfalse%
\ {\isachardoublequoteopen}{\isasymexists}k\ {\isasymin}\ {\isacharbraceleft}n{\isachardot}\ True{\isacharbraceright}{\isachardot}\ F\ {\isasymin}\ pcp{\isacharunderscore}seq\ C\ S\ k{\isachardoublequoteclose}\isanewline
\ \ \ \ \isacommand{by}\isamarkupfalse%
\ {\isacharparenleft}simp\ only{\isacharcolon}\ UN{\isacharunderscore}iff\ simp{\isacharunderscore}thms{\isacharparenleft}{\isadigit{4}}{\isadigit{0}}{\isacharparenright}{\isacharparenright}\isanewline
\ \ \isacommand{then}\isamarkupfalse%
\ \isacommand{have}\isamarkupfalse%
\ {\isachardoublequoteopen}{\isasymexists}k\ {\isasymin}\ UNIV{\isachardot}\ F\ {\isasymin}\ pcp{\isacharunderscore}seq\ C\ S\ k{\isachardoublequoteclose}\ \isanewline
\ \ \ \ \isacommand{by}\isamarkupfalse%
\ {\isacharparenleft}simp\ only{\isacharcolon}\ UNIV{\isacharunderscore}def{\isacharparenright}\isanewline
\ \ \isacommand{thus}\isamarkupfalse%
\ {\isachardoublequoteopen}{\isasymexists}k{\isachardot}\ F\ {\isasymin}\ pcp{\isacharunderscore}seq\ C\ S\ k{\isachardoublequoteclose}\ \isanewline
\ \ \ \ \isacommand{by}\isamarkupfalse%
\ {\isacharparenleft}simp\ only{\isacharcolon}\ bex{\isacharunderscore}UNIV{\isacharparenright}\isanewline
\isacommand{qed}\isamarkupfalse%
%
\endisatagproof
{\isafoldproof}%
%
\isadelimproof
%
\endisadelimproof
%
\begin{isamarkuptext}%
Mostremos, a continuación, la demostración automática del resultado.%
\end{isamarkuptext}\isamarkuptrue%
\isacommand{lemma}\isamarkupfalse%
\ pcp{\isacharunderscore}lim{\isacharunderscore}inserted{\isacharunderscore}at{\isacharunderscore}ex{\isacharcolon}\ \isanewline
\ \ \ \ {\isachardoublequoteopen}S{\isacharprime}\ {\isasymin}\ pcp{\isacharunderscore}lim\ C\ S\ {\isasymLongrightarrow}\ {\isasymexists}k{\isachardot}\ S{\isacharprime}\ {\isasymin}\ pcp{\isacharunderscore}seq\ C\ S\ k{\isachardoublequoteclose}\isanewline
%
\isadelimproof
\ \ %
\endisadelimproof
%
\isatagproof
\isacommand{unfolding}\isamarkupfalse%
\ pcp{\isacharunderscore}lim{\isacharunderscore}def\ \isacommand{by}\isamarkupfalse%
\ blast%
\endisatagproof
{\isafoldproof}%
%
\isadelimproof
%
\endisadelimproof
%
\begin{isamarkuptext}%
Por último, veamos la siguiente propiedad sobre conjuntos finitos contenidos en el límite de 
  las sucesiones definido en \isa{{\isadigit{4}}{\isachardot}{\isadigit{1}}{\isachardot}{\isadigit{5}}}.

\begin{lema}
  Sea \isa{C} una colección, \isa{S\ {\isasymin}\ C} y \isa{{\isacharbraceleft}S\isactrlsub n{\isacharbraceright}} la sucesión de conjuntos de \isa{C} a partir de \isa{S} según la
  definición \isa{{\isadigit{4}}{\isachardot}{\isadigit{1}}{\isachardot}{\isadigit{1}}}. Si \isa{S{\isacharprime}} es un conjunto finito tal que \isa{S{\isacharprime}\ {\isasymsubseteq}} $\bigcup_{n = 0}^{\infty} S_{n}$, 
  entonces existe un\\ \isa{k\ {\isasymin}\ {\isasymnat}} tal que \isa{S{\isacharprime}\ {\isasymsubseteq}\ S\isactrlsub k}.
\end{lema}

\begin{demostracion}
  La prueba del resultado se realiza por inducción sobre la estructura recursiva de los conjuntos 
  finitos.

  En primer lugar, probemos el caso base correspondiente al conjunto vacío. Supongamos que \isa{{\isacharbraceleft}{\isacharbraceright}} está 
  contenido en el límite de la sucesión de conjuntos de \isa{C} a partir de \isa{S}. Como \isa{{\isacharbraceleft}{\isacharbraceright}} es 
  subconjunto de todo conjunto, en particular lo es de \isa{S\ {\isacharequal}\ S\isactrlsub {\isadigit{0}}}, probando así el primer caso.

  Por otra parte, probemos el paso de inducción. Sea \isa{S{\isacharprime}} un conjunto finito verificando la 
  hipótesis de inducción: si \isa{S{\isacharprime}} está contenido en el límite de la sucesión de conjuntos de 
  \isa{C} a partir de \isa{S}, entonces también está contenido en algún \isa{S\isactrlsub k\isactrlsub {\isacharprime}} para cierto \isa{k{\isacharprime}\ {\isasymin}\ {\isasymnat}}. Sea 
  \isa{F} una fórmula tal que \isa{F\ {\isasymnotin}\ S{\isacharprime}}. Vamos a probar que si \isa{{\isacharbraceleft}F{\isacharbraceright}\ {\isasymunion}\ S{\isacharprime}} está contenido en el límite, 
  entonces está contenido en \isa{S\isactrlsub k} para algún \isa{k\ {\isasymin}\ {\isasymnat}}. 

  Como hemos supuesto que \isa{{\isacharbraceleft}F{\isacharbraceright}\ {\isasymunion}\ S{\isacharprime}} está contenido en el límite, entonces se verifica que \isa{F}
  pertenece al límite y \isa{S{\isacharprime}} está contenido en él. Por el lema \isa{{\isadigit{4}}{\isachardot}{\isadigit{1}}{\isachardot}{\isadigit{7}}}, como \isa{F} pertenece al 
  límite, deducimos que existe un \isa{k\ {\isasymin}\ {\isasymnat}} tal que \isa{F\ {\isasymin}\ S\isactrlsub k}. Por otro lado, como \isa{S{\isacharprime}} está contenido
  en el límite, por hipótesis de inducción existe algún \isa{k{\isacharprime}\ {\isasymin}\ {\isasymnat}} tal que \isa{S{\isacharprime}\ {\isasymsubseteq}\ S\isactrlsub k\isactrlsub {\isacharprime}}. El resultado 
  se obtiene considerando el máximo entre \isa{k} y \isa{k{\isacharprime}}, que notaremos por \isa{k{\isacharprime}{\isacharprime}}. En efecto, por la 
  monotonía de la sucesión, se verifica que tanto \isa{S\isactrlsub k} como \isa{S\isactrlsub k\isactrlsub {\isacharprime}} están contenidos en \isa{S\isactrlsub k\isactrlsub {\isacharprime}\isactrlsub {\isacharprime}}. De este 
  modo, como \isa{S{\isacharprime}\ {\isasymsubseteq}\ S\isactrlsub k\isactrlsub {\isacharprime}}, por la transitividad de la contención de conjuntos se tiene que 
  \isa{S{\isacharprime}\ {\isasymsubseteq}\ S\isactrlsub k\isactrlsub {\isacharprime}\isactrlsub {\isacharprime}}. Además, como \isa{F\ {\isasymin}\ S\isactrlsub k}, se tiene que \isa{F\ {\isasymin}\ S\isactrlsub k\isactrlsub {\isacharprime}\isactrlsub {\isacharprime}}. Por lo tanto, \isa{{\isacharbraceleft}F{\isacharbraceright}\ {\isasymunion}\ S{\isacharprime}\ {\isasymsubseteq}\ S\isactrlsub k\isactrlsub {\isacharprime}\isactrlsub {\isacharprime}}, como 
  queríamos demostrar. 
\end{demostracion}

  Procedamos con la demostración detallada en Isabelle.%
\end{isamarkuptext}\isamarkuptrue%
\isacommand{lemma}\isamarkupfalse%
\ \isanewline
\ \ \isakeyword{assumes}\ {\isachardoublequoteopen}finite\ S{\isacharprime}{\isachardoublequoteclose}\isanewline
\ \ \ \ \ \ \ \ \ \ {\isachardoublequoteopen}S{\isacharprime}\ {\isasymsubseteq}\ pcp{\isacharunderscore}lim\ C\ S{\isachardoublequoteclose}\isanewline
\ \ \ \ \ \ \ \ \isakeyword{shows}\ {\isachardoublequoteopen}{\isasymexists}k{\isachardot}\ S{\isacharprime}\ {\isasymsubseteq}\ pcp{\isacharunderscore}seq\ C\ S\ k{\isachardoublequoteclose}\isanewline
%
\isadelimproof
\ \ %
\endisadelimproof
%
\isatagproof
\isacommand{using}\isamarkupfalse%
\ assms\isanewline
\isacommand{proof}\isamarkupfalse%
\ {\isacharparenleft}induction\ S{\isacharprime}\ rule{\isacharcolon}\ finite{\isacharunderscore}induct{\isacharparenright}\isanewline
\ \ \isacommand{case}\isamarkupfalse%
\ empty\isanewline
\ \ \isacommand{have}\isamarkupfalse%
\ {\isachardoublequoteopen}pcp{\isacharunderscore}seq\ C\ S\ {\isadigit{0}}\ {\isacharequal}\ S{\isachardoublequoteclose}\isanewline
\ \ \ \ \isacommand{by}\isamarkupfalse%
\ {\isacharparenleft}simp\ only{\isacharcolon}\ pcp{\isacharunderscore}seq{\isachardot}simps{\isacharparenleft}{\isadigit{1}}{\isacharparenright}{\isacharparenright}\isanewline
\ \ \isacommand{have}\isamarkupfalse%
\ {\isachardoublequoteopen}{\isacharbraceleft}{\isacharbraceright}\ {\isasymsubseteq}\ S{\isachardoublequoteclose}\isanewline
\ \ \ \ \isacommand{by}\isamarkupfalse%
\ {\isacharparenleft}rule\ order{\isacharunderscore}bot{\isacharunderscore}class{\isachardot}bot{\isachardot}extremum{\isacharparenright}\isanewline
\ \ \isacommand{then}\isamarkupfalse%
\ \isacommand{have}\isamarkupfalse%
\ {\isachardoublequoteopen}{\isacharbraceleft}{\isacharbraceright}\ {\isasymsubseteq}\ pcp{\isacharunderscore}seq\ C\ S\ {\isadigit{0}}{\isachardoublequoteclose}\isanewline
\ \ \ \ \isacommand{by}\isamarkupfalse%
\ {\isacharparenleft}simp\ only{\isacharcolon}\ {\isacartoucheopen}pcp{\isacharunderscore}seq\ C\ S\ {\isadigit{0}}\ {\isacharequal}\ S{\isacartoucheclose}{\isacharparenright}\isanewline
\ \ \isacommand{then}\isamarkupfalse%
\ \isacommand{show}\isamarkupfalse%
\ {\isacharquery}case\ \isanewline
\ \ \ \ \isacommand{by}\isamarkupfalse%
\ {\isacharparenleft}rule\ exI{\isacharparenright}\isanewline
\isacommand{next}\isamarkupfalse%
\isanewline
\ \ \isacommand{case}\isamarkupfalse%
\ {\isacharparenleft}insert\ F\ S{\isacharprime}{\isacharparenright}\isanewline
\ \ \isacommand{then}\isamarkupfalse%
\ \isacommand{have}\isamarkupfalse%
\ {\isachardoublequoteopen}insert\ F\ S{\isacharprime}\ {\isasymsubseteq}\ pcp{\isacharunderscore}lim\ C\ S{\isachardoublequoteclose}\isanewline
\ \ \ \ \isacommand{by}\isamarkupfalse%
\ {\isacharparenleft}simp\ only{\isacharcolon}\ insert{\isachardot}prems{\isacharparenright}\isanewline
\ \ \isacommand{then}\isamarkupfalse%
\ \isacommand{have}\isamarkupfalse%
\ C{\isacharcolon}{\isachardoublequoteopen}F\ {\isasymin}\ {\isacharparenleft}pcp{\isacharunderscore}lim\ C\ S{\isacharparenright}\ {\isasymand}\ S{\isacharprime}\ {\isasymsubseteq}\ pcp{\isacharunderscore}lim\ C\ S{\isachardoublequoteclose}\isanewline
\ \ \ \ \isacommand{by}\isamarkupfalse%
\ {\isacharparenleft}simp\ only{\isacharcolon}\ insert{\isacharunderscore}subset{\isacharparenright}\ \isanewline
\ \ \isacommand{then}\isamarkupfalse%
\ \isacommand{have}\isamarkupfalse%
\ {\isachardoublequoteopen}S{\isacharprime}\ {\isasymsubseteq}\ pcp{\isacharunderscore}lim\ C\ S{\isachardoublequoteclose}\isanewline
\ \ \ \ \isacommand{by}\isamarkupfalse%
\ {\isacharparenleft}rule\ conjunct{\isadigit{2}}{\isacharparenright}\isanewline
\ \ \isacommand{then}\isamarkupfalse%
\ \isacommand{have}\isamarkupfalse%
\ EX{\isadigit{1}}{\isacharcolon}{\isachardoublequoteopen}{\isasymexists}k{\isachardot}\ S{\isacharprime}\ {\isasymsubseteq}\ pcp{\isacharunderscore}seq\ C\ S\ k{\isachardoublequoteclose}\isanewline
\ \ \ \ \isacommand{by}\isamarkupfalse%
\ {\isacharparenleft}simp\ only{\isacharcolon}\ insert{\isachardot}IH{\isacharparenright}\isanewline
\ \ \isacommand{obtain}\isamarkupfalse%
\ k{\isadigit{1}}\ \isakeyword{where}\ {\isachardoublequoteopen}S{\isacharprime}\ {\isasymsubseteq}\ pcp{\isacharunderscore}seq\ C\ S\ k{\isadigit{1}}{\isachardoublequoteclose}\isanewline
\ \ \ \ \isacommand{using}\isamarkupfalse%
\ EX{\isadigit{1}}\ \isacommand{by}\isamarkupfalse%
\ {\isacharparenleft}rule\ exE{\isacharparenright}\isanewline
\ \ \isacommand{have}\isamarkupfalse%
\ {\isachardoublequoteopen}F\ {\isasymin}\ pcp{\isacharunderscore}lim\ C\ S{\isachardoublequoteclose}\isanewline
\ \ \ \ \isacommand{using}\isamarkupfalse%
\ C\ \isacommand{by}\isamarkupfalse%
\ {\isacharparenleft}rule\ conjunct{\isadigit{1}}{\isacharparenright}\isanewline
\ \ \isacommand{then}\isamarkupfalse%
\ \isacommand{have}\isamarkupfalse%
\ EX{\isadigit{2}}{\isacharcolon}{\isachardoublequoteopen}{\isasymexists}k{\isachardot}\ F\ {\isasymin}\ pcp{\isacharunderscore}seq\ C\ S\ k{\isachardoublequoteclose}\isanewline
\ \ \ \ \isacommand{by}\isamarkupfalse%
\ {\isacharparenleft}rule\ pcp{\isacharunderscore}lim{\isacharunderscore}inserted{\isacharunderscore}at{\isacharunderscore}ex{\isacharparenright}\isanewline
\ \ \isacommand{obtain}\isamarkupfalse%
\ k{\isadigit{2}}\ \isakeyword{where}\ {\isachardoublequoteopen}F\ {\isasymin}\ pcp{\isacharunderscore}seq\ C\ S\ k{\isadigit{2}}{\isachardoublequoteclose}\ \isanewline
\ \ \ \ \isacommand{using}\isamarkupfalse%
\ EX{\isadigit{2}}\ \isacommand{by}\isamarkupfalse%
\ {\isacharparenleft}rule\ exE{\isacharparenright}\isanewline
\ \ \isacommand{have}\isamarkupfalse%
\ {\isachardoublequoteopen}k{\isadigit{1}}\ {\isasymle}\ max\ k{\isadigit{1}}\ k{\isadigit{2}}{\isachardoublequoteclose}\isanewline
\ \ \ \ \isacommand{by}\isamarkupfalse%
\ {\isacharparenleft}simp\ only{\isacharcolon}\ linorder{\isacharunderscore}class{\isachardot}max{\isachardot}cobounded{\isadigit{1}}{\isacharparenright}\isanewline
\ \ \isacommand{then}\isamarkupfalse%
\ \isacommand{have}\isamarkupfalse%
\ {\isachardoublequoteopen}pcp{\isacharunderscore}seq\ C\ S\ k{\isadigit{1}}\ {\isasymsubseteq}\ pcp{\isacharunderscore}seq\ C\ S\ {\isacharparenleft}max\ k{\isadigit{1}}\ k{\isadigit{2}}{\isacharparenright}{\isachardoublequoteclose}\isanewline
\ \ \ \ \isacommand{by}\isamarkupfalse%
\ {\isacharparenleft}rule\ pcp{\isacharunderscore}seq{\isacharunderscore}mono{\isacharparenright}\isanewline
\ \ \isacommand{have}\isamarkupfalse%
\ {\isachardoublequoteopen}k{\isadigit{2}}\ {\isasymle}\ max\ k{\isadigit{1}}\ k{\isadigit{2}}{\isachardoublequoteclose}\isanewline
\ \ \ \ \isacommand{by}\isamarkupfalse%
\ {\isacharparenleft}simp\ only{\isacharcolon}\ linorder{\isacharunderscore}class{\isachardot}max{\isachardot}cobounded{\isadigit{2}}{\isacharparenright}\isanewline
\ \ \isacommand{then}\isamarkupfalse%
\ \isacommand{have}\isamarkupfalse%
\ {\isachardoublequoteopen}pcp{\isacharunderscore}seq\ C\ S\ k{\isadigit{2}}\ {\isasymsubseteq}\ pcp{\isacharunderscore}seq\ C\ S\ {\isacharparenleft}max\ k{\isadigit{1}}\ k{\isadigit{2}}{\isacharparenright}{\isachardoublequoteclose}\isanewline
\ \ \ \ \isacommand{by}\isamarkupfalse%
\ {\isacharparenleft}rule\ pcp{\isacharunderscore}seq{\isacharunderscore}mono{\isacharparenright}\isanewline
\ \ \isacommand{have}\isamarkupfalse%
\ {\isachardoublequoteopen}S{\isacharprime}\ {\isasymsubseteq}\ pcp{\isacharunderscore}seq\ C\ S\ {\isacharparenleft}max\ k{\isadigit{1}}\ k{\isadigit{2}}{\isacharparenright}{\isachardoublequoteclose}\isanewline
\ \ \ \ \isacommand{using}\isamarkupfalse%
\ {\isacartoucheopen}S{\isacharprime}\ {\isasymsubseteq}\ pcp{\isacharunderscore}seq\ C\ S\ k{\isadigit{1}}{\isacartoucheclose}\ {\isacartoucheopen}pcp{\isacharunderscore}seq\ C\ S\ k{\isadigit{1}}\ {\isasymsubseteq}\ pcp{\isacharunderscore}seq\ C\ S\ {\isacharparenleft}max\ k{\isadigit{1}}\ k{\isadigit{2}}{\isacharparenright}{\isacartoucheclose}\ \isacommand{by}\isamarkupfalse%
\ {\isacharparenleft}rule\ subset{\isacharunderscore}trans{\isacharparenright}\isanewline
\ \ \isacommand{have}\isamarkupfalse%
\ {\isachardoublequoteopen}F\ {\isasymin}\ pcp{\isacharunderscore}seq\ C\ S\ {\isacharparenleft}max\ k{\isadigit{1}}\ k{\isadigit{2}}{\isacharparenright}{\isachardoublequoteclose}\isanewline
\ \ \ \ \isacommand{using}\isamarkupfalse%
\ {\isacartoucheopen}F\ {\isasymin}\ pcp{\isacharunderscore}seq\ C\ S\ k{\isadigit{2}}{\isacartoucheclose}\ {\isacartoucheopen}pcp{\isacharunderscore}seq\ C\ S\ k{\isadigit{2}}\ {\isasymsubseteq}\ pcp{\isacharunderscore}seq\ C\ S\ {\isacharparenleft}max\ k{\isadigit{1}}\ k{\isadigit{2}}{\isacharparenright}{\isacartoucheclose}\ \isacommand{by}\isamarkupfalse%
\ {\isacharparenleft}rule\ rev{\isacharunderscore}subsetD{\isacharparenright}\isanewline
\ \ \isacommand{then}\isamarkupfalse%
\ \isacommand{have}\isamarkupfalse%
\ {\isadigit{1}}{\isacharcolon}{\isachardoublequoteopen}insert\ F\ S{\isacharprime}\ {\isasymsubseteq}\ pcp{\isacharunderscore}seq\ C\ S\ {\isacharparenleft}max\ k{\isadigit{1}}\ k{\isadigit{2}}{\isacharparenright}{\isachardoublequoteclose}\isanewline
\ \ \ \ \isacommand{using}\isamarkupfalse%
\ {\isacartoucheopen}S{\isacharprime}\ {\isasymsubseteq}\ pcp{\isacharunderscore}seq\ C\ S\ {\isacharparenleft}max\ k{\isadigit{1}}\ k{\isadigit{2}}{\isacharparenright}{\isacartoucheclose}\ \isacommand{by}\isamarkupfalse%
\ {\isacharparenleft}simp\ only{\isacharcolon}\ insert{\isacharunderscore}subset{\isacharparenright}\isanewline
\ \ \isacommand{thus}\isamarkupfalse%
\ {\isacharquery}case\isanewline
\ \ \ \ \isacommand{by}\isamarkupfalse%
\ {\isacharparenleft}rule\ exI{\isacharparenright}\isanewline
\isacommand{qed}\isamarkupfalse%
%
\endisatagproof
{\isafoldproof}%
%
\isadelimproof
%
\endisadelimproof
%
\begin{isamarkuptext}%
Finalmente, su demostración automática en Isabelle/HOL es la siguiente.%
\end{isamarkuptext}\isamarkuptrue%
\isacommand{lemma}\isamarkupfalse%
\ finite{\isacharunderscore}pcp{\isacharunderscore}lim{\isacharunderscore}EX{\isacharcolon}\isanewline
\ \ \isakeyword{assumes}\ {\isachardoublequoteopen}finite\ S{\isacharprime}{\isachardoublequoteclose}\isanewline
\ \ \ \ \ \ \ \ \ \ {\isachardoublequoteopen}S{\isacharprime}\ {\isasymsubseteq}\ pcp{\isacharunderscore}lim\ C\ S{\isachardoublequoteclose}\isanewline
\ \ \ \ \ \ \ \ \isakeyword{shows}\ {\isachardoublequoteopen}{\isasymexists}k{\isachardot}\ S{\isacharprime}\ {\isasymsubseteq}\ pcp{\isacharunderscore}seq\ C\ S\ k{\isachardoublequoteclose}\isanewline
%
\isadelimproof
\ \ %
\endisadelimproof
%
\isatagproof
\isacommand{using}\isamarkupfalse%
\ assms\isanewline
\isacommand{proof}\isamarkupfalse%
{\isacharparenleft}induction\ S{\isacharprime}\ rule{\isacharcolon}\ finite{\isacharunderscore}induct{\isacharparenright}\ \isanewline
\ \ \isacommand{case}\isamarkupfalse%
\ {\isacharparenleft}insert\ F\ S{\isacharprime}{\isacharparenright}\isanewline
\ \ \isacommand{hence}\isamarkupfalse%
\ {\isachardoublequoteopen}{\isasymexists}k{\isachardot}\ S{\isacharprime}\ {\isasymsubseteq}\ pcp{\isacharunderscore}seq\ C\ S\ k{\isachardoublequoteclose}\ \isacommand{by}\isamarkupfalse%
\ fast\isanewline
\ \ \isacommand{then}\isamarkupfalse%
\ \isacommand{guess}\isamarkupfalse%
\ k{\isadigit{1}}\ \isacommand{{\isachardot}{\isachardot}}\isamarkupfalse%
\isanewline
\ \ \isacommand{moreover}\isamarkupfalse%
\ \isacommand{obtain}\isamarkupfalse%
\ k{\isadigit{2}}\ \isakeyword{where}\ {\isachardoublequoteopen}F\ {\isasymin}\ pcp{\isacharunderscore}seq\ C\ S\ k{\isadigit{2}}{\isachardoublequoteclose}\isanewline
\ \ \ \ \isacommand{by}\isamarkupfalse%
\ {\isacharparenleft}meson\ pcp{\isacharunderscore}lim{\isacharunderscore}inserted{\isacharunderscore}at{\isacharunderscore}ex\ insert{\isachardot}prems\ insert{\isacharunderscore}subset{\isacharparenright}\isanewline
\ \ \isacommand{ultimately}\isamarkupfalse%
\ \isacommand{have}\isamarkupfalse%
\ {\isachardoublequoteopen}insert\ F\ S{\isacharprime}\ {\isasymsubseteq}\ pcp{\isacharunderscore}seq\ C\ S\ {\isacharparenleft}max\ k{\isadigit{1}}\ k{\isadigit{2}}{\isacharparenright}{\isachardoublequoteclose}\isanewline
\ \ \ \ \isacommand{by}\isamarkupfalse%
\ {\isacharparenleft}meson\ pcp{\isacharunderscore}seq{\isacharunderscore}mono\ dual{\isacharunderscore}order{\isachardot}trans\ insert{\isacharunderscore}subset\ max{\isachardot}bounded{\isacharunderscore}iff\ order{\isacharunderscore}refl\ subsetCE{\isacharparenright}\isanewline
\ \ \isacommand{thus}\isamarkupfalse%
\ {\isacharquery}case\ \isacommand{by}\isamarkupfalse%
\ blast\isanewline
\isacommand{qed}\isamarkupfalse%
\ simp%
\endisatagproof
{\isafoldproof}%
%
\isadelimproof
%
\endisadelimproof
%
\isadelimdocument
%
\endisadelimdocument
%
\isatagdocument
%
\isamarkupsection{El Teorema de Existencia de Modelo%
}
\isamarkuptrue%
%
\endisatagdocument
{\isafolddocument}%
%
\isadelimdocument
%
\endisadelimdocument
%
\begin{isamarkuptext}%
En esta sección demostraremos finalmente el 
  \isa{teorema\ de\ existencia\ de\ modelo}, el cual prueba que todo conjunto de fórmulas perteneciente a 
  una colección que verifique la propiedad de consistencia proposicional es satisfacible. Para ello, 
  considerando una colección \isa{C} cualquiera y \isa{S\ {\isasymin}\ C}, empleando resultados anteriores extenderemos 
  la colección a una colección \isa{C{\isacharprime}{\isacharprime}} que tenga la propiedad de consistencia proposicional, sea
  cerrada bajo subconjuntos y sea de carácter finito. De este modo, en esta sección probaremos que el 
  límite de la sucesión formada a partir de una colección que tenga dichas condiciones y un conjunto
  cualquiera \isa{S} como se indica en la definición \isa{{\isadigit{1}}{\isachardot}{\isadigit{4}}{\isachardot}{\isadigit{1}}} pertenece a la colección. Es más, 
  demostraremos que dicho límite se trata de un conjunto de \isa{Hintikka} luego, por el \isa{teorema\ de\ Hintikka}, es satisfacible. Finalmente, como \isa{S} está contenido en el límite, quedará demostrada 
  la satisfacibilidad del conjunto \isa{S} al heredarla por contención.

  \comentario{Habrá que modificar el párrafo anterior al final.}%
\end{isamarkuptext}\isamarkuptrue%
%
\begin{isamarkuptext}%
En primer lugar, probemos que si \isa{C} es una colección que verifica la propiedad de 
  consistencia proposicional, es cerrada bajo subconjuntos y es de carácter finito, entonces el 
  límite de toda sucesión de conjuntos de \isa{C} según la definición \isa{{\isadigit{4}}{\isachardot}{\isadigit{1}}{\isachardot}{\isadigit{1}}} pertenece a \isa{C}.

  \begin{lema}
    Sea \isa{C} una colección de conjuntos que verifica la propiedad de consistencia proposicional, es 
    cerrada bajo subconjuntos y es de carácter finito. Sea \isa{S\ {\isasymin}\ C} y \isa{{\isacharbraceleft}S\isactrlsub n{\isacharbraceright}} la sucesión de conjuntos
    de \isa{C} a partir de \isa{S} según la definición \isa{{\isadigit{4}}{\isachardot}{\isadigit{1}}{\isachardot}{\isadigit{1}}}. Entonces, el límite de la sucesión está en
    \isa{C}.
  \end{lema}

  \begin{demostracion}
    Por definición, como \isa{C} es de carácter finito, para todo conjunto son equivalentes:
    \begin{enumerate}
      \item El conjunto pertenece a \isa{C}.
      \item Todo subconjunto finito suyo pertenece a \isa{C}.
    \end{enumerate}

    De este modo, para demostrar que el límite de la sucesión \isa{{\isacharbraceleft}S\isactrlsub n{\isacharbraceright}} pertenece a \isa{C}, basta probar
    que todo subconjunto finito suyo está en \isa{C}.

    Sea \isa{S{\isacharprime}} un subconjunto finito del límite de la sucesión. Por el lema \isa{{\isadigit{1}}{\isachardot}{\isadigit{4}}{\isachardot}{\isadigit{8}}}, existe un
    \isa{k\ {\isasymin}\ {\isasymnat}} tal que \isa{S{\isacharprime}\ {\isasymsubseteq}\ S\isactrlsub k}. Por tanto, como \isa{S\isactrlsub k\ {\isasymin}\ C} para todo \isa{k\ {\isasymin}\ {\isasymnat}} y \isa{C} es cerrada bajo
    subconjuntos, por definición se tiene que \isa{S{\isacharprime}\ {\isasymin}\ C}, como queríamos demostrar.
  \end{demostracion}

  En Isabelle se formaliza y demuestra detalladamente como sigue.%
\end{isamarkuptext}\isamarkuptrue%
\isacommand{lemma}\isamarkupfalse%
\isanewline
\ \ \isakeyword{assumes}\ {\isachardoublequoteopen}pcp\ C{\isachardoublequoteclose}\isanewline
\ \ \ \ \ \ \ \ \ \ {\isachardoublequoteopen}S\ {\isasymin}\ C{\isachardoublequoteclose}\isanewline
\ \ \ \ \ \ \ \ \ \ {\isachardoublequoteopen}subset{\isacharunderscore}closed\ C{\isachardoublequoteclose}\isanewline
\ \ \ \ \ \ \ \ \ \ {\isachardoublequoteopen}finite{\isacharunderscore}character\ C{\isachardoublequoteclose}\isanewline
\ \ \isakeyword{shows}\ {\isachardoublequoteopen}pcp{\isacharunderscore}lim\ C\ S\ {\isasymin}\ C{\isachardoublequoteclose}\ \isanewline
%
\isadelimproof
%
\endisadelimproof
%
\isatagproof
\isacommand{proof}\isamarkupfalse%
\ {\isacharminus}\isanewline
\ \ \isacommand{have}\isamarkupfalse%
\ {\isachardoublequoteopen}{\isasymforall}S{\isachardot}\ S\ {\isasymin}\ C\ {\isasymlongleftrightarrow}\ {\isacharparenleft}{\isasymforall}S{\isacharprime}\ {\isasymsubseteq}\ S{\isachardot}\ finite\ S{\isacharprime}\ {\isasymlongrightarrow}\ S{\isacharprime}\ {\isasymin}\ C{\isacharparenright}{\isachardoublequoteclose}\isanewline
\ \ \ \ \isacommand{using}\isamarkupfalse%
\ assms{\isacharparenleft}{\isadigit{4}}{\isacharparenright}\ \isacommand{unfolding}\isamarkupfalse%
\ finite{\isacharunderscore}character{\isacharunderscore}def\ \isacommand{by}\isamarkupfalse%
\ this\isanewline
\ \ \isacommand{then}\isamarkupfalse%
\ \isacommand{have}\isamarkupfalse%
\ FC{\isadigit{1}}{\isacharcolon}{\isachardoublequoteopen}pcp{\isacharunderscore}lim\ C\ S\ {\isasymin}\ C\ {\isasymlongleftrightarrow}\ {\isacharparenleft}{\isasymforall}S{\isacharprime}\ {\isasymsubseteq}\ {\isacharparenleft}pcp{\isacharunderscore}lim\ C\ S{\isacharparenright}{\isachardot}\ finite\ S{\isacharprime}\ {\isasymlongrightarrow}\ S{\isacharprime}\ {\isasymin}\ C{\isacharparenright}{\isachardoublequoteclose}\isanewline
\ \ \ \ \isacommand{by}\isamarkupfalse%
\ {\isacharparenleft}rule\ allE{\isacharparenright}\isanewline
\ \ \isacommand{have}\isamarkupfalse%
\ SC{\isacharcolon}{\isachardoublequoteopen}{\isasymforall}S\ {\isasymin}\ C{\isachardot}\ {\isasymforall}S{\isacharprime}{\isasymsubseteq}S{\isachardot}\ S{\isacharprime}\ {\isasymin}\ C{\isachardoublequoteclose}\isanewline
\ \ \ \ \isacommand{using}\isamarkupfalse%
\ assms{\isacharparenleft}{\isadigit{3}}{\isacharparenright}\ \isacommand{unfolding}\isamarkupfalse%
\ subset{\isacharunderscore}closed{\isacharunderscore}def\ \isacommand{by}\isamarkupfalse%
\ this\isanewline
\ \ \isacommand{have}\isamarkupfalse%
\ FC{\isadigit{2}}{\isacharcolon}{\isachardoublequoteopen}{\isasymforall}S{\isacharprime}\ {\isasymsubseteq}\ pcp{\isacharunderscore}lim\ C\ S{\isachardot}\ finite\ S{\isacharprime}\ {\isasymlongrightarrow}\ S{\isacharprime}\ {\isasymin}\ C{\isachardoublequoteclose}\isanewline
\ \ \isacommand{proof}\isamarkupfalse%
\ {\isacharparenleft}rule\ sallI{\isacharparenright}\isanewline
\ \ \ \ \isacommand{fix}\isamarkupfalse%
\ S{\isacharprime}\ {\isacharcolon}{\isacharcolon}\ {\isachardoublequoteopen}{\isacharprime}a\ formula\ set{\isachardoublequoteclose}\isanewline
\ \ \ \ \isacommand{assume}\isamarkupfalse%
\ {\isachardoublequoteopen}S{\isacharprime}\ {\isasymsubseteq}\ pcp{\isacharunderscore}lim\ C\ S{\isachardoublequoteclose}\isanewline
\ \ \ \ \isacommand{show}\isamarkupfalse%
\ {\isachardoublequoteopen}finite\ S{\isacharprime}\ {\isasymlongrightarrow}\ S{\isacharprime}\ {\isasymin}\ C{\isachardoublequoteclose}\isanewline
\ \ \ \ \isacommand{proof}\isamarkupfalse%
\ {\isacharparenleft}rule\ impI{\isacharparenright}\isanewline
\ \ \ \ \ \ \isacommand{assume}\isamarkupfalse%
\ {\isachardoublequoteopen}finite\ S{\isacharprime}{\isachardoublequoteclose}\isanewline
\ \ \ \ \ \ \isacommand{then}\isamarkupfalse%
\ \isacommand{have}\isamarkupfalse%
\ EX{\isacharcolon}{\isachardoublequoteopen}{\isasymexists}k{\isachardot}\ S{\isacharprime}\ {\isasymsubseteq}\ pcp{\isacharunderscore}seq\ C\ S\ k{\isachardoublequoteclose}\ \isanewline
\ \ \ \ \ \ \ \ \isacommand{using}\isamarkupfalse%
\ {\isacartoucheopen}S{\isacharprime}\ {\isasymsubseteq}\ pcp{\isacharunderscore}lim\ C\ S{\isacartoucheclose}\ \isacommand{by}\isamarkupfalse%
\ {\isacharparenleft}rule\ finite{\isacharunderscore}pcp{\isacharunderscore}lim{\isacharunderscore}EX{\isacharparenright}\isanewline
\ \ \ \ \ \ \isacommand{obtain}\isamarkupfalse%
\ k\ \isakeyword{where}\ {\isachardoublequoteopen}S{\isacharprime}\ {\isasymsubseteq}\ pcp{\isacharunderscore}seq\ C\ S\ k{\isachardoublequoteclose}\isanewline
\ \ \ \ \ \ \ \ \isacommand{using}\isamarkupfalse%
\ EX\ \isacommand{by}\isamarkupfalse%
\ {\isacharparenleft}rule\ exE{\isacharparenright}\isanewline
\ \ \ \ \ \ \isacommand{have}\isamarkupfalse%
\ {\isachardoublequoteopen}pcp{\isacharunderscore}seq\ C\ S\ k\ {\isasymin}\ C{\isachardoublequoteclose}\isanewline
\ \ \ \ \ \ \ \ \isacommand{using}\isamarkupfalse%
\ assms{\isacharparenleft}{\isadigit{1}}{\isacharparenright}\ assms{\isacharparenleft}{\isadigit{2}}{\isacharparenright}\ \isacommand{by}\isamarkupfalse%
\ {\isacharparenleft}rule\ pcp{\isacharunderscore}seq{\isacharunderscore}in{\isacharparenright}\isanewline
\ \ \ \ \ \ \isacommand{have}\isamarkupfalse%
\ {\isachardoublequoteopen}{\isasymforall}S{\isacharprime}\ {\isasymsubseteq}\ {\isacharparenleft}pcp{\isacharunderscore}seq\ C\ S\ k{\isacharparenright}{\isachardot}\ S{\isacharprime}\ {\isasymin}\ C{\isachardoublequoteclose}\isanewline
\ \ \ \ \ \ \ \ \isacommand{using}\isamarkupfalse%
\ SC\ {\isacartoucheopen}pcp{\isacharunderscore}seq\ C\ S\ k\ {\isasymin}\ C{\isacartoucheclose}\ \isacommand{by}\isamarkupfalse%
\ {\isacharparenleft}rule\ bspec{\isacharparenright}\isanewline
\ \ \ \ \ \ \isacommand{thus}\isamarkupfalse%
\ {\isachardoublequoteopen}S{\isacharprime}\ {\isasymin}\ C{\isachardoublequoteclose}\isanewline
\ \ \ \ \ \ \ \ \isacommand{using}\isamarkupfalse%
\ {\isacartoucheopen}S{\isacharprime}\ {\isasymsubseteq}\ pcp{\isacharunderscore}seq\ C\ S\ k{\isacartoucheclose}\ \isacommand{by}\isamarkupfalse%
\ {\isacharparenleft}rule\ sspec{\isacharparenright}\isanewline
\ \ \ \ \isacommand{qed}\isamarkupfalse%
\isanewline
\ \ \isacommand{qed}\isamarkupfalse%
\isanewline
\ \ \isacommand{show}\isamarkupfalse%
\ {\isachardoublequoteopen}pcp{\isacharunderscore}lim\ C\ S\ {\isasymin}\ C{\isachardoublequoteclose}\ \isanewline
\ \ \ \ \isacommand{using}\isamarkupfalse%
\ FC{\isadigit{1}}\ FC{\isadigit{2}}\ \isacommand{by}\isamarkupfalse%
\ {\isacharparenleft}rule\ forw{\isacharunderscore}subst{\isacharparenright}\isanewline
\isacommand{qed}\isamarkupfalse%
%
\endisatagproof
{\isafoldproof}%
%
\isadelimproof
%
\endisadelimproof
%
\begin{isamarkuptext}%
Por otra parte, podemos dar una prueba automática del resultado.%
\end{isamarkuptext}\isamarkuptrue%
\isacommand{lemma}\isamarkupfalse%
\ pcp{\isacharunderscore}lim{\isacharunderscore}in{\isacharcolon}\isanewline
\ \ \isakeyword{assumes}\ c{\isacharcolon}\ {\isachardoublequoteopen}pcp\ C{\isachardoublequoteclose}\isanewline
\ \ \isakeyword{and}\ el{\isacharcolon}\ {\isachardoublequoteopen}S\ {\isasymin}\ C{\isachardoublequoteclose}\isanewline
\ \ \isakeyword{and}\ sc{\isacharcolon}\ {\isachardoublequoteopen}subset{\isacharunderscore}closed\ C{\isachardoublequoteclose}\isanewline
\ \ \isakeyword{and}\ fc{\isacharcolon}\ {\isachardoublequoteopen}finite{\isacharunderscore}character\ C{\isachardoublequoteclose}\isanewline
\ \ \isakeyword{shows}\ {\isachardoublequoteopen}pcp{\isacharunderscore}lim\ C\ S\ {\isasymin}\ C{\isachardoublequoteclose}\ {\isacharparenleft}\isakeyword{is}\ {\isachardoublequoteopen}{\isacharquery}cl\ {\isasymin}\ C{\isachardoublequoteclose}{\isacharparenright}\isanewline
%
\isadelimproof
%
\endisadelimproof
%
\isatagproof
\isacommand{proof}\isamarkupfalse%
\ {\isacharminus}\isanewline
\ \ \isacommand{from}\isamarkupfalse%
\ pcp{\isacharunderscore}seq{\isacharunderscore}in{\isacharbrackleft}OF\ c\ el{\isacharcomma}\ THEN\ allI{\isacharbrackright}\ \isacommand{have}\isamarkupfalse%
\ {\isachardoublequoteopen}{\isasymforall}n{\isachardot}\ pcp{\isacharunderscore}seq\ C\ S\ n\ {\isasymin}\ C{\isachardoublequoteclose}\ \isacommand{{\isachardot}}\isamarkupfalse%
\isanewline
\ \ \isacommand{hence}\isamarkupfalse%
\ {\isachardoublequoteopen}{\isasymforall}m{\isachardot}\ {\isasymUnion}{\isacharbraceleft}pcp{\isacharunderscore}seq\ C\ S\ n{\isacharbar}n{\isachardot}\ n\ {\isasymle}\ m{\isacharbraceright}\ {\isasymin}\ C{\isachardoublequoteclose}\ \isacommand{unfolding}\isamarkupfalse%
\ pcp{\isacharunderscore}seq{\isacharunderscore}UN\ \isacommand{{\isachardot}}\isamarkupfalse%
\isanewline
\ \ \isacommand{have}\isamarkupfalse%
\ {\isachardoublequoteopen}{\isasymforall}S{\isacharprime}\ {\isasymsubseteq}\ {\isacharquery}cl{\isachardot}\ finite\ S{\isacharprime}\ {\isasymlongrightarrow}\ S{\isacharprime}\ {\isasymin}\ C{\isachardoublequoteclose}\isanewline
\ \ \isacommand{proof}\isamarkupfalse%
\ safe\isanewline
\ \ \ \ \isacommand{fix}\isamarkupfalse%
\ S{\isacharprime}\ {\isacharcolon}{\isacharcolon}\ {\isachardoublequoteopen}{\isacharprime}a\ formula\ set{\isachardoublequoteclose}\isanewline
\ \ \ \ \isacommand{have}\isamarkupfalse%
\ {\isachardoublequoteopen}pcp{\isacharunderscore}seq\ C\ S\ {\isacharparenleft}Suc\ {\isacharparenleft}Max\ {\isacharparenleft}to{\isacharunderscore}nat\ {\isacharbackquote}\ S{\isacharprime}{\isacharparenright}{\isacharparenright}{\isacharparenright}\ {\isasymsubseteq}\ pcp{\isacharunderscore}lim\ C\ S{\isachardoublequoteclose}\ \isanewline
\ \ \ \ \ \ \isacommand{using}\isamarkupfalse%
\ pcp{\isacharunderscore}seq{\isacharunderscore}sub\ \isacommand{by}\isamarkupfalse%
\ blast\isanewline
\ \ \ \ \isacommand{assume}\isamarkupfalse%
\ {\isacartoucheopen}finite\ S{\isacharprime}{\isacartoucheclose}\ {\isacartoucheopen}S{\isacharprime}\ {\isasymsubseteq}\ pcp{\isacharunderscore}lim\ C\ S{\isacartoucheclose}\isanewline
\ \ \ \ \isacommand{hence}\isamarkupfalse%
\ {\isachardoublequoteopen}{\isasymexists}k{\isachardot}\ S{\isacharprime}\ {\isasymsubseteq}\ pcp{\isacharunderscore}seq\ C\ S\ k{\isachardoublequoteclose}\ \isanewline
\ \ \ \ \isacommand{proof}\isamarkupfalse%
{\isacharparenleft}induction\ S{\isacharprime}\ rule{\isacharcolon}\ finite{\isacharunderscore}induct{\isacharparenright}\ \isanewline
\ \ \ \ \ \ \isacommand{case}\isamarkupfalse%
\ {\isacharparenleft}insert\ x\ S{\isacharprime}{\isacharparenright}\isanewline
\ \ \ \ \ \ \isacommand{hence}\isamarkupfalse%
\ {\isachardoublequoteopen}{\isasymexists}k{\isachardot}\ S{\isacharprime}\ {\isasymsubseteq}\ pcp{\isacharunderscore}seq\ C\ S\ k{\isachardoublequoteclose}\ \isacommand{by}\isamarkupfalse%
\ fast\isanewline
\ \ \ \ \ \ \isacommand{then}\isamarkupfalse%
\ \isacommand{guess}\isamarkupfalse%
\ k{\isadigit{1}}\ \isacommand{{\isachardot}{\isachardot}}\isamarkupfalse%
\isanewline
\ \ \ \ \ \ \isacommand{moreover}\isamarkupfalse%
\ \isacommand{obtain}\isamarkupfalse%
\ k{\isadigit{2}}\ \isakeyword{where}\ {\isachardoublequoteopen}x\ {\isasymin}\ pcp{\isacharunderscore}seq\ C\ S\ k{\isadigit{2}}{\isachardoublequoteclose}\isanewline
\ \ \ \ \ \ \ \ \isacommand{by}\isamarkupfalse%
\ {\isacharparenleft}meson\ pcp{\isacharunderscore}lim{\isacharunderscore}inserted{\isacharunderscore}at{\isacharunderscore}ex\ insert{\isachardot}prems\ insert{\isacharunderscore}subset{\isacharparenright}\isanewline
\ \ \ \ \ \ \isacommand{ultimately}\isamarkupfalse%
\ \isacommand{have}\isamarkupfalse%
\ {\isachardoublequoteopen}insert\ x\ S{\isacharprime}\ {\isasymsubseteq}\ pcp{\isacharunderscore}seq\ C\ S\ {\isacharparenleft}max\ k{\isadigit{1}}\ k{\isadigit{2}}{\isacharparenright}{\isachardoublequoteclose}\isanewline
\ \ \ \ \ \ \ \ \isacommand{by}\isamarkupfalse%
\ {\isacharparenleft}meson\ pcp{\isacharunderscore}seq{\isacharunderscore}mono\ dual{\isacharunderscore}order{\isachardot}trans\ insert{\isacharunderscore}subset\ max{\isachardot}bounded{\isacharunderscore}iff\ order{\isacharunderscore}refl\ subsetCE{\isacharparenright}\isanewline
\ \ \ \ \ \ \isacommand{thus}\isamarkupfalse%
\ {\isacharquery}case\ \isacommand{by}\isamarkupfalse%
\ blast\isanewline
\ \ \ \ \isacommand{qed}\isamarkupfalse%
\ simp\isanewline
\ \ \ \ \isacommand{with}\isamarkupfalse%
\ pcp{\isacharunderscore}seq{\isacharunderscore}in{\isacharbrackleft}OF\ c\ el{\isacharbrackright}\ sc\isanewline
\ \ \ \ \isacommand{show}\isamarkupfalse%
\ {\isachardoublequoteopen}S{\isacharprime}\ {\isasymin}\ C{\isachardoublequoteclose}\ \isacommand{unfolding}\isamarkupfalse%
\ subset{\isacharunderscore}closed{\isacharunderscore}def\ \isacommand{by}\isamarkupfalse%
\ blast\isanewline
\ \ \isacommand{qed}\isamarkupfalse%
\isanewline
\ \ \isacommand{thus}\isamarkupfalse%
\ {\isachardoublequoteopen}{\isacharquery}cl\ {\isasymin}\ C{\isachardoublequoteclose}\ \isacommand{using}\isamarkupfalse%
\ fc\ \isacommand{unfolding}\isamarkupfalse%
\ finite{\isacharunderscore}character{\isacharunderscore}def\ \isacommand{by}\isamarkupfalse%
\ blast\isanewline
\isacommand{qed}\isamarkupfalse%
%
\endisatagproof
{\isafoldproof}%
%
\isadelimproof
%
\endisadelimproof
%
\begin{isamarkuptext}%
Probemos que, además, el límite de las sucesión definida en \isa{{\isadigit{4}}{\isachardot}{\isadigit{1}}{\isachardot}{\isadigit{1}}} se trata de un elemento 
  maximal de la colección que lo define si esta verifica la propiedad de consistencia proposicional
  y es cerrada bajo subconjuntos.

  \begin{lema}
    Sea \isa{C} una colección de conjuntos que verifica la propiedad de consistencia proposicional y
    es cerrada bajo subconjuntos, \isa{S} un conjunto y \isa{{\isacharbraceleft}S\isactrlsub n{\isacharbraceright}} la sucesión de conjuntos de \isa{C} a partir 
    de \isa{S} según la definición \isa{{\isadigit{4}}{\isachardot}{\isadigit{1}}{\isachardot}{\isadigit{1}}}. Entonces, el límite de la sucesión \isa{{\isacharbraceleft}S\isactrlsub n{\isacharbraceright}} es un elemento 
    maximal de \isa{C}.
  \end{lema}

  \begin{demostracion}
    Por definición de elemento maximal, basta probar que para cualquier conjunto \isa{K\ {\isasymin}\ C} que
    contenga al límite de la sucesión se tiene que \isa{K} y el límite coinciden.

    La demostración se realizará por reducción al absurdo. Consideremos un conjunto \isa{K\ {\isasymin}\ C} que 
    contenga estrictamente al límite de la sucesión \isa{{\isacharbraceleft}S\isactrlsub n{\isacharbraceright}}. De este modo, existe una fórmula \isa{F} tal 
    que \isa{F\ {\isasymin}\ K} y \isa{F} no está en el límite. Supongamos que \isa{F} es la \isa{n}-ésima fórmula según la 
    enumeración de la definición \isa{{\isadigit{4}}{\isachardot}{\isadigit{1}}{\isachardot}{\isadigit{1}}} utilizada para construir la sucesión. 

    Por un lado, hemos probado que todo elemento de la sucesión está contenido en el límite, luego 
    en particular obtenemos que \isa{S\isactrlsub n\isactrlsub {\isacharplus}\isactrlsub {\isadigit{1}}} está contenido en el límite. De este modo, como \isa{F} no 
    pertenece al límite, es claro que \isa{F\ {\isasymnotin}\ S\isactrlsub n\isactrlsub {\isacharplus}\isactrlsub {\isadigit{1}}}. Además, \isa{{\isacharbraceleft}F{\isacharbraceright}\ {\isasymunion}\ S\isactrlsub n\ {\isasymnotin}\ C} ya que, en caso contrario, 
    por la definición \isa{{\isadigit{4}}{\isachardot}{\isadigit{1}}{\isachardot}{\isadigit{1}}} de la sucesión obtendríamos que\\ \isa{S\isactrlsub n\isactrlsub {\isacharplus}\isactrlsub {\isadigit{1}}\ {\isacharequal}\ {\isacharbraceleft}F{\isacharbraceright}\ {\isasymunion}\ S\isactrlsub n}, lo que contradice 
    que \isa{F\ {\isasymnotin}\ S\isactrlsub n\isactrlsub {\isacharplus}\isactrlsub {\isadigit{1}}}. 

    Por otro lado, como \isa{S\isactrlsub n} también está contenida en el límite que, a su vez, está contenido en 
    \isa{K}, se obtiene por transitividad que \isa{S\isactrlsub n\ {\isasymsubseteq}\ K}. Además, como \isa{F\ {\isasymin}\ K}, se verifica que 
    \isa{{\isacharbraceleft}F{\isacharbraceright}\ {\isasymunion}\ S\isactrlsub n\ {\isasymsubseteq}\ K}. Como \isa{C} es una colección cerrada bajo subconjuntos por hipótesis y \isa{K\ {\isasymin}\ C}, 
    por definición se tiene que \isa{{\isacharbraceleft}F{\isacharbraceright}\ {\isasymunion}\ S\isactrlsub n\ {\isasymin}\ C}, llegando así a una contradicción con lo demostrado 
    anteriormente.
  \end{demostracion}

  Su formalización y prueba detallada en Isabelle/HOL se muestran a continuación.%
\end{isamarkuptext}\isamarkuptrue%
\isacommand{lemma}\isamarkupfalse%
\isanewline
\ \ \isakeyword{assumes}\ {\isachardoublequoteopen}pcp\ C{\isachardoublequoteclose}\isanewline
\ \ \ \ \ \ \ \ \ \ {\isachardoublequoteopen}subset{\isacharunderscore}closed\ C{\isachardoublequoteclose}\isanewline
\ \ \ \ \ \ \ \ \ \ {\isachardoublequoteopen}K\ {\isasymin}\ C{\isachardoublequoteclose}\isanewline
\ \ \ \ \ \ \ \ \ \ {\isachardoublequoteopen}pcp{\isacharunderscore}lim\ C\ S\ {\isasymsubseteq}\ K{\isachardoublequoteclose}\isanewline
\ \ \isakeyword{shows}\ {\isachardoublequoteopen}pcp{\isacharunderscore}lim\ C\ S\ {\isacharequal}\ K{\isachardoublequoteclose}\isanewline
%
\isadelimproof
%
\endisadelimproof
%
\isatagproof
\isacommand{proof}\isamarkupfalse%
\ {\isacharparenleft}rule\ ccontr{\isacharparenright}\isanewline
\ \ \isacommand{assume}\isamarkupfalse%
\ H{\isacharcolon}{\isachardoublequoteopen}{\isasymnot}{\isacharparenleft}pcp{\isacharunderscore}lim\ C\ S\ {\isacharequal}\ K{\isacharparenright}{\isachardoublequoteclose}\isanewline
\ \ \isacommand{have}\isamarkupfalse%
\ CE{\isacharcolon}{\isachardoublequoteopen}pcp{\isacharunderscore}lim\ C\ S\ {\isasymsubseteq}\ K\ {\isasymand}\ pcp{\isacharunderscore}lim\ C\ S\ {\isasymnoteq}\ K{\isachardoublequoteclose}\isanewline
\ \ \ \ \isacommand{using}\isamarkupfalse%
\ assms{\isacharparenleft}{\isadigit{4}}{\isacharparenright}\ H\ \isacommand{by}\isamarkupfalse%
\ {\isacharparenleft}rule\ conjI{\isacharparenright}\isanewline
\ \ \isacommand{have}\isamarkupfalse%
\ {\isachardoublequoteopen}pcp{\isacharunderscore}lim\ C\ S\ {\isasymsubseteq}\ K\ {\isasymand}\ pcp{\isacharunderscore}lim\ C\ S\ {\isasymnoteq}\ K\ {\isasymlongleftrightarrow}\ pcp{\isacharunderscore}lim\ C\ S\ {\isasymsubset}\ K{\isachardoublequoteclose}\isanewline
\ \ \ \ \isacommand{by}\isamarkupfalse%
\ {\isacharparenleft}simp\ only{\isacharcolon}\ psubset{\isacharunderscore}eq{\isacharparenright}\isanewline
\ \ \isacommand{then}\isamarkupfalse%
\ \isacommand{have}\isamarkupfalse%
\ {\isachardoublequoteopen}pcp{\isacharunderscore}lim\ C\ S\ {\isasymsubset}\ K{\isachardoublequoteclose}\ \isanewline
\ \ \ \ \isacommand{using}\isamarkupfalse%
\ CE\ \isacommand{by}\isamarkupfalse%
\ {\isacharparenleft}rule\ iffD{\isadigit{1}}{\isacharparenright}\isanewline
\ \ \isacommand{then}\isamarkupfalse%
\ \isacommand{have}\isamarkupfalse%
\ {\isachardoublequoteopen}{\isasymexists}F{\isachardot}\ F\ {\isasymin}\ {\isacharparenleft}K\ {\isacharminus}\ {\isacharparenleft}pcp{\isacharunderscore}lim\ C\ S{\isacharparenright}{\isacharparenright}{\isachardoublequoteclose}\isanewline
\ \ \ \ \isacommand{by}\isamarkupfalse%
\ {\isacharparenleft}simp\ only{\isacharcolon}\ psubset{\isacharunderscore}imp{\isacharunderscore}ex{\isacharunderscore}mem{\isacharparenright}\ \isanewline
\ \ \isacommand{then}\isamarkupfalse%
\ \isacommand{have}\isamarkupfalse%
\ E{\isacharcolon}{\isachardoublequoteopen}{\isasymexists}F{\isachardot}\ F\ {\isasymin}\ K\ {\isasymand}\ F\ {\isasymnotin}\ {\isacharparenleft}pcp{\isacharunderscore}lim\ C\ S{\isacharparenright}{\isachardoublequoteclose}\isanewline
\ \ \ \ \isacommand{by}\isamarkupfalse%
\ {\isacharparenleft}simp\ only{\isacharcolon}\ Diff{\isacharunderscore}iff{\isacharparenright}\isanewline
\ \ \isacommand{obtain}\isamarkupfalse%
\ F\ \isakeyword{where}\ F{\isacharcolon}{\isachardoublequoteopen}F\ {\isasymin}\ K\ {\isasymand}\ F\ {\isasymnotin}\ pcp{\isacharunderscore}lim\ C\ S{\isachardoublequoteclose}\ \isanewline
\ \ \ \ \isacommand{using}\isamarkupfalse%
\ E\ \isacommand{by}\isamarkupfalse%
\ {\isacharparenleft}rule\ exE{\isacharparenright}\isanewline
\ \ \isacommand{have}\isamarkupfalse%
\ {\isachardoublequoteopen}F\ {\isasymin}\ K{\isachardoublequoteclose}\ \isanewline
\ \ \ \ \isacommand{using}\isamarkupfalse%
\ F\ \isacommand{by}\isamarkupfalse%
\ {\isacharparenleft}rule\ conjunct{\isadigit{1}}{\isacharparenright}\isanewline
\ \ \isacommand{have}\isamarkupfalse%
\ {\isachardoublequoteopen}F\ {\isasymnotin}\ pcp{\isacharunderscore}lim\ C\ S{\isachardoublequoteclose}\isanewline
\ \ \ \ \isacommand{using}\isamarkupfalse%
\ F\ \isacommand{by}\isamarkupfalse%
\ {\isacharparenleft}rule\ conjunct{\isadigit{2}}{\isacharparenright}\isanewline
\ \ \isacommand{have}\isamarkupfalse%
\ {\isachardoublequoteopen}pcp{\isacharunderscore}seq\ C\ S\ {\isacharparenleft}Suc\ {\isacharparenleft}to{\isacharunderscore}nat\ F{\isacharparenright}{\isacharparenright}\ {\isasymsubseteq}\ pcp{\isacharunderscore}lim\ C\ S{\isachardoublequoteclose}\isanewline
\ \ \ \ \isacommand{by}\isamarkupfalse%
\ {\isacharparenleft}rule\ pcp{\isacharunderscore}seq{\isacharunderscore}sub{\isacharparenright}\isanewline
\ \ \isacommand{then}\isamarkupfalse%
\ \isacommand{have}\isamarkupfalse%
\ {\isachardoublequoteopen}F\ {\isasymin}\ pcp{\isacharunderscore}seq\ C\ S\ {\isacharparenleft}Suc\ {\isacharparenleft}to{\isacharunderscore}nat\ F{\isacharparenright}{\isacharparenright}\ {\isasymlongrightarrow}\ F\ {\isasymin}\ pcp{\isacharunderscore}lim\ C\ S{\isachardoublequoteclose}\isanewline
\ \ \ \ \isacommand{by}\isamarkupfalse%
\ {\isacharparenleft}rule\ in{\isacharunderscore}mono{\isacharparenright}\isanewline
\ \ \isacommand{then}\isamarkupfalse%
\ \isacommand{have}\isamarkupfalse%
\ {\isadigit{1}}{\isacharcolon}{\isachardoublequoteopen}F\ {\isasymnotin}\ pcp{\isacharunderscore}seq\ C\ S\ {\isacharparenleft}Suc\ {\isacharparenleft}to{\isacharunderscore}nat\ F{\isacharparenright}{\isacharparenright}{\isachardoublequoteclose}\isanewline
\ \ \ \ \isacommand{using}\isamarkupfalse%
\ {\isacartoucheopen}F\ {\isasymnotin}\ pcp{\isacharunderscore}lim\ C\ S{\isacartoucheclose}\ \isacommand{by}\isamarkupfalse%
\ {\isacharparenleft}rule\ mt{\isacharparenright}\isanewline
\ \ \isacommand{have}\isamarkupfalse%
\ {\isadigit{2}}{\isacharcolon}\ {\isachardoublequoteopen}insert\ F\ {\isacharparenleft}pcp{\isacharunderscore}seq\ C\ S\ {\isacharparenleft}to{\isacharunderscore}nat\ F{\isacharparenright}{\isacharparenright}\ {\isasymnotin}\ C{\isachardoublequoteclose}\ \isanewline
\ \ \isacommand{proof}\isamarkupfalse%
\ {\isacharparenleft}rule\ ccontr{\isacharparenright}\isanewline
\ \ \ \ \isacommand{assume}\isamarkupfalse%
\ {\isachardoublequoteopen}{\isasymnot}{\isacharparenleft}insert\ F\ {\isacharparenleft}pcp{\isacharunderscore}seq\ C\ S\ {\isacharparenleft}to{\isacharunderscore}nat\ F{\isacharparenright}{\isacharparenright}\ {\isasymnotin}\ C{\isacharparenright}{\isachardoublequoteclose}\isanewline
\ \ \ \ \isacommand{then}\isamarkupfalse%
\ \isacommand{have}\isamarkupfalse%
\ {\isachardoublequoteopen}insert\ F\ {\isacharparenleft}pcp{\isacharunderscore}seq\ C\ S\ {\isacharparenleft}to{\isacharunderscore}nat\ F{\isacharparenright}{\isacharparenright}\ {\isasymin}\ C{\isachardoublequoteclose}\isanewline
\ \ \ \ \ \ \isacommand{by}\isamarkupfalse%
\ {\isacharparenleft}rule\ notnotD{\isacharparenright}\isanewline
\ \ \ \ \isacommand{then}\isamarkupfalse%
\ \isacommand{have}\isamarkupfalse%
\ C{\isacharcolon}{\isachardoublequoteopen}insert\ {\isacharparenleft}from{\isacharunderscore}nat\ {\isacharparenleft}to{\isacharunderscore}nat\ F{\isacharparenright}{\isacharparenright}\ {\isacharparenleft}pcp{\isacharunderscore}seq\ C\ S\ {\isacharparenleft}to{\isacharunderscore}nat\ F{\isacharparenright}{\isacharparenright}\ {\isasymin}\ C{\isachardoublequoteclose}\isanewline
\ \ \ \ \ \ \isacommand{by}\isamarkupfalse%
\ {\isacharparenleft}simp\ only{\isacharcolon}\ from{\isacharunderscore}nat{\isacharunderscore}to{\isacharunderscore}nat{\isacharparenright}\isanewline
\ \ \ \ \isacommand{have}\isamarkupfalse%
\ {\isachardoublequoteopen}pcp{\isacharunderscore}seq\ C\ S\ {\isacharparenleft}Suc\ {\isacharparenleft}to{\isacharunderscore}nat\ F{\isacharparenright}{\isacharparenright}\ {\isacharequal}\ {\isacharparenleft}let\ Sn\ {\isacharequal}\ pcp{\isacharunderscore}seq\ C\ S\ {\isacharparenleft}to{\isacharunderscore}nat\ F{\isacharparenright}{\isacharsemicolon}\ \isanewline
\ \ \ \ \ \ \ \ \ \ Sn{\isadigit{1}}\ {\isacharequal}\ insert\ {\isacharparenleft}from{\isacharunderscore}nat\ {\isacharparenleft}to{\isacharunderscore}nat\ F{\isacharparenright}{\isacharparenright}\ Sn\ in\ if\ Sn{\isadigit{1}}\ {\isasymin}\ C\ then\ Sn{\isadigit{1}}\ else\ Sn{\isacharparenright}{\isachardoublequoteclose}\ \isanewline
\ \ \ \ \ \ \isacommand{by}\isamarkupfalse%
\ {\isacharparenleft}simp\ only{\isacharcolon}\ pcp{\isacharunderscore}seq{\isachardot}simps{\isacharparenleft}{\isadigit{2}}{\isacharparenright}{\isacharparenright}\isanewline
\ \ \ \ \isacommand{then}\isamarkupfalse%
\ \isacommand{have}\isamarkupfalse%
\ SucDef{\isacharcolon}{\isachardoublequoteopen}pcp{\isacharunderscore}seq\ C\ S\ {\isacharparenleft}Suc\ {\isacharparenleft}to{\isacharunderscore}nat\ F{\isacharparenright}{\isacharparenright}\ {\isacharequal}\ {\isacharparenleft}if\ insert\ {\isacharparenleft}from{\isacharunderscore}nat\ {\isacharparenleft}to{\isacharunderscore}nat\ F{\isacharparenright}{\isacharparenright}\ {\isacharparenleft}pcp{\isacharunderscore}seq\ C\ S\ {\isacharparenleft}to{\isacharunderscore}nat\ F{\isacharparenright}{\isacharparenright}\ {\isasymin}\ C\ \isanewline
\ \ \ \ \ \ \ \ \ \ then\ insert\ {\isacharparenleft}from{\isacharunderscore}nat\ {\isacharparenleft}to{\isacharunderscore}nat\ F{\isacharparenright}{\isacharparenright}\ {\isacharparenleft}pcp{\isacharunderscore}seq\ C\ S\ {\isacharparenleft}to{\isacharunderscore}nat\ F{\isacharparenright}{\isacharparenright}\ else\ pcp{\isacharunderscore}seq\ C\ S\ {\isacharparenleft}to{\isacharunderscore}nat\ F{\isacharparenright}{\isacharparenright}{\isachardoublequoteclose}\ \isanewline
\ \ \ \ \ \ \isacommand{by}\isamarkupfalse%
\ {\isacharparenleft}simp\ only{\isacharcolon}\ Let{\isacharunderscore}def{\isacharparenright}\isanewline
\ \ \ \ \isacommand{then}\isamarkupfalse%
\ \isacommand{have}\isamarkupfalse%
\ {\isachardoublequoteopen}pcp{\isacharunderscore}seq\ C\ S\ {\isacharparenleft}Suc\ {\isacharparenleft}to{\isacharunderscore}nat\ F{\isacharparenright}{\isacharparenright}\ {\isacharequal}\ insert\ {\isacharparenleft}from{\isacharunderscore}nat\ {\isacharparenleft}to{\isacharunderscore}nat\ F{\isacharparenright}{\isacharparenright}\ {\isacharparenleft}pcp{\isacharunderscore}seq\ C\ S\ {\isacharparenleft}to{\isacharunderscore}nat\ F{\isacharparenright}{\isacharparenright}{\isachardoublequoteclose}\ \isanewline
\ \ \ \ \ \ \isacommand{using}\isamarkupfalse%
\ C\ \isacommand{by}\isamarkupfalse%
\ {\isacharparenleft}simp\ only{\isacharcolon}\ if{\isacharunderscore}True{\isacharparenright}\isanewline
\ \ \ \ \isacommand{then}\isamarkupfalse%
\ \isacommand{have}\isamarkupfalse%
\ {\isachardoublequoteopen}pcp{\isacharunderscore}seq\ C\ S\ {\isacharparenleft}Suc\ {\isacharparenleft}to{\isacharunderscore}nat\ F{\isacharparenright}{\isacharparenright}\ {\isacharequal}\ insert\ F\ {\isacharparenleft}pcp{\isacharunderscore}seq\ C\ S\ {\isacharparenleft}to{\isacharunderscore}nat\ F{\isacharparenright}{\isacharparenright}{\isachardoublequoteclose}\isanewline
\ \ \ \ \ \ \isacommand{by}\isamarkupfalse%
\ {\isacharparenleft}simp\ only{\isacharcolon}\ from{\isacharunderscore}nat{\isacharunderscore}to{\isacharunderscore}nat{\isacharparenright}\isanewline
\ \ \ \ \isacommand{then}\isamarkupfalse%
\ \isacommand{have}\isamarkupfalse%
\ {\isachardoublequoteopen}F\ {\isasymin}\ pcp{\isacharunderscore}seq\ C\ S\ {\isacharparenleft}Suc\ {\isacharparenleft}to{\isacharunderscore}nat\ F{\isacharparenright}{\isacharparenright}{\isachardoublequoteclose}\isanewline
\ \ \ \ \ \ \isacommand{by}\isamarkupfalse%
\ {\isacharparenleft}simp\ only{\isacharcolon}\ insertI{\isadigit{1}}{\isacharparenright}\isanewline
\ \ \ \ \isacommand{show}\isamarkupfalse%
\ {\isachardoublequoteopen}False{\isachardoublequoteclose}\isanewline
\ \ \ \ \ \ \isacommand{using}\isamarkupfalse%
\ {\isacartoucheopen}F\ {\isasymnotin}\ pcp{\isacharunderscore}seq\ C\ S\ {\isacharparenleft}Suc\ {\isacharparenleft}to{\isacharunderscore}nat\ F{\isacharparenright}{\isacharparenright}{\isacartoucheclose}\ {\isacartoucheopen}F\ {\isasymin}\ pcp{\isacharunderscore}seq\ C\ S\ {\isacharparenleft}Suc\ {\isacharparenleft}to{\isacharunderscore}nat\ F{\isacharparenright}{\isacharparenright}{\isacartoucheclose}\ \isacommand{by}\isamarkupfalse%
\ {\isacharparenleft}rule\ notE{\isacharparenright}\isanewline
\ \ \isacommand{qed}\isamarkupfalse%
\isanewline
\ \ \isacommand{have}\isamarkupfalse%
\ {\isachardoublequoteopen}pcp{\isacharunderscore}seq\ C\ S\ {\isacharparenleft}to{\isacharunderscore}nat\ F{\isacharparenright}\ {\isasymsubseteq}\ pcp{\isacharunderscore}lim\ C\ S{\isachardoublequoteclose}\isanewline
\ \ \ \ \isacommand{by}\isamarkupfalse%
\ {\isacharparenleft}rule\ pcp{\isacharunderscore}seq{\isacharunderscore}sub{\isacharparenright}\isanewline
\ \ \isacommand{then}\isamarkupfalse%
\ \isacommand{have}\isamarkupfalse%
\ {\isachardoublequoteopen}pcp{\isacharunderscore}seq\ C\ S\ {\isacharparenleft}to{\isacharunderscore}nat\ F{\isacharparenright}\ {\isasymsubseteq}\ K{\isachardoublequoteclose}\isanewline
\ \ \ \ \isacommand{using}\isamarkupfalse%
\ assms{\isacharparenleft}{\isadigit{4}}{\isacharparenright}\ \isacommand{by}\isamarkupfalse%
\ {\isacharparenleft}rule\ subset{\isacharunderscore}trans{\isacharparenright}\isanewline
\ \ \isacommand{then}\isamarkupfalse%
\ \isacommand{have}\isamarkupfalse%
\ {\isachardoublequoteopen}insert\ F\ {\isacharparenleft}pcp{\isacharunderscore}seq\ C\ S\ {\isacharparenleft}to{\isacharunderscore}nat\ F{\isacharparenright}{\isacharparenright}\ {\isasymsubseteq}\ K{\isachardoublequoteclose}\ \isanewline
\ \ \ \ \isacommand{using}\isamarkupfalse%
\ {\isacartoucheopen}F\ {\isasymin}\ K{\isacartoucheclose}\ \isacommand{by}\isamarkupfalse%
\ {\isacharparenleft}simp\ only{\isacharcolon}\ insert{\isacharunderscore}subset{\isacharparenright}\isanewline
\ \ \isacommand{have}\isamarkupfalse%
\ {\isachardoublequoteopen}{\isasymforall}S\ {\isasymin}\ C{\isachardot}\ {\isasymforall}s{\isasymsubseteq}S{\isachardot}\ s\ {\isasymin}\ C{\isachardoublequoteclose}\isanewline
\ \ \ \ \isacommand{using}\isamarkupfalse%
\ assms{\isacharparenleft}{\isadigit{2}}{\isacharparenright}\ \isacommand{by}\isamarkupfalse%
\ {\isacharparenleft}simp\ only{\isacharcolon}\ subset{\isacharunderscore}closed{\isacharunderscore}def{\isacharparenright}\isanewline
\ \ \isacommand{then}\isamarkupfalse%
\ \isacommand{have}\isamarkupfalse%
\ {\isachardoublequoteopen}{\isasymforall}s\ {\isasymsubseteq}\ K{\isachardot}\ s\ {\isasymin}\ C{\isachardoublequoteclose}\isanewline
\ \ \ \ \isacommand{using}\isamarkupfalse%
\ assms{\isacharparenleft}{\isadigit{3}}{\isacharparenright}\ \isacommand{by}\isamarkupfalse%
\ {\isacharparenleft}rule\ bspec{\isacharparenright}\isanewline
\ \ \isacommand{then}\isamarkupfalse%
\ \isacommand{have}\isamarkupfalse%
\ {\isadigit{3}}{\isacharcolon}{\isachardoublequoteopen}insert\ F\ {\isacharparenleft}pcp{\isacharunderscore}seq\ C\ S\ {\isacharparenleft}to{\isacharunderscore}nat\ F{\isacharparenright}{\isacharparenright}\ {\isasymin}\ C{\isachardoublequoteclose}\ \isanewline
\ \ \ \ \isacommand{using}\isamarkupfalse%
\ {\isacartoucheopen}insert\ F\ {\isacharparenleft}pcp{\isacharunderscore}seq\ C\ S\ {\isacharparenleft}to{\isacharunderscore}nat\ F{\isacharparenright}{\isacharparenright}\ {\isasymsubseteq}\ K{\isacartoucheclose}\ \isacommand{by}\isamarkupfalse%
\ {\isacharparenleft}rule\ sspec{\isacharparenright}\isanewline
\ \ \isacommand{show}\isamarkupfalse%
\ {\isachardoublequoteopen}False{\isachardoublequoteclose}\isanewline
\ \ \ \ \isacommand{using}\isamarkupfalse%
\ {\isadigit{2}}\ {\isadigit{3}}\ \isacommand{by}\isamarkupfalse%
\ {\isacharparenleft}rule\ notE{\isacharparenright}\isanewline
\isacommand{qed}\isamarkupfalse%
%
\endisatagproof
{\isafoldproof}%
%
\isadelimproof
%
\endisadelimproof
%
\begin{isamarkuptext}%
Análogamente a resultados anteriores, veamos su prueba automática.%
\end{isamarkuptext}\isamarkuptrue%
\isacommand{lemma}\isamarkupfalse%
\ cl{\isacharunderscore}max{\isacharcolon}\isanewline
\ \ \isakeyword{assumes}\ c{\isacharcolon}\ {\isachardoublequoteopen}pcp\ C{\isachardoublequoteclose}\isanewline
\ \ \isakeyword{assumes}\ sc{\isacharcolon}\ {\isachardoublequoteopen}subset{\isacharunderscore}closed\ C{\isachardoublequoteclose}\isanewline
\ \ \isakeyword{assumes}\ el{\isacharcolon}\ {\isachardoublequoteopen}K\ {\isasymin}\ C{\isachardoublequoteclose}\isanewline
\ \ \isakeyword{assumes}\ su{\isacharcolon}\ {\isachardoublequoteopen}pcp{\isacharunderscore}lim\ C\ S\ {\isasymsubseteq}\ K{\isachardoublequoteclose}\isanewline
\ \ \isakeyword{shows}\ {\isachardoublequoteopen}pcp{\isacharunderscore}lim\ C\ S\ {\isacharequal}\ K{\isachardoublequoteclose}\ {\isacharparenleft}\isakeyword{is}\ {\isacharquery}e{\isacharparenright}\isanewline
%
\isadelimproof
%
\endisadelimproof
%
\isatagproof
\isacommand{proof}\isamarkupfalse%
\ {\isacharparenleft}rule\ ccontr{\isacharparenright}\isanewline
\ \ \isacommand{assume}\isamarkupfalse%
\ {\isacartoucheopen}{\isasymnot}{\isacharquery}e{\isacartoucheclose}\isanewline
\ \ \isacommand{with}\isamarkupfalse%
\ su\ \isacommand{have}\isamarkupfalse%
\ {\isachardoublequoteopen}pcp{\isacharunderscore}lim\ C\ S\ {\isasymsubset}\ K{\isachardoublequoteclose}\ \isacommand{by}\isamarkupfalse%
\ simp\isanewline
\ \ \isacommand{then}\isamarkupfalse%
\ \isacommand{obtain}\isamarkupfalse%
\ F\ \isakeyword{where}\ e{\isacharcolon}\ {\isachardoublequoteopen}F\ {\isasymin}\ K{\isachardoublequoteclose}\ \isakeyword{and}\ ne{\isacharcolon}\ {\isachardoublequoteopen}F\ {\isasymnotin}\ pcp{\isacharunderscore}lim\ C\ S{\isachardoublequoteclose}\ \isacommand{by}\isamarkupfalse%
\ blast\isanewline
\ \ \isacommand{from}\isamarkupfalse%
\ ne\ \isacommand{have}\isamarkupfalse%
\ {\isachardoublequoteopen}F\ {\isasymnotin}\ pcp{\isacharunderscore}seq\ C\ S\ {\isacharparenleft}Suc\ {\isacharparenleft}to{\isacharunderscore}nat\ F{\isacharparenright}{\isacharparenright}{\isachardoublequoteclose}\ \isacommand{using}\isamarkupfalse%
\ pcp{\isacharunderscore}seq{\isacharunderscore}sub\ \isacommand{by}\isamarkupfalse%
\ fast\isanewline
\ \ \isacommand{hence}\isamarkupfalse%
\ {\isadigit{1}}{\isacharcolon}\ {\isachardoublequoteopen}insert\ F\ {\isacharparenleft}pcp{\isacharunderscore}seq\ C\ S\ {\isacharparenleft}to{\isacharunderscore}nat\ F{\isacharparenright}{\isacharparenright}\ {\isasymnotin}\ C{\isachardoublequoteclose}\ \isacommand{by}\isamarkupfalse%
\ {\isacharparenleft}simp\ add{\isacharcolon}\ Let{\isacharunderscore}def\ split{\isacharcolon}\ if{\isacharunderscore}splits{\isacharparenright}\isanewline
\ \ \isacommand{have}\isamarkupfalse%
\ {\isachardoublequoteopen}insert\ F\ {\isacharparenleft}pcp{\isacharunderscore}seq\ C\ S\ {\isacharparenleft}to{\isacharunderscore}nat\ F{\isacharparenright}{\isacharparenright}\ {\isasymsubseteq}\ K{\isachardoublequoteclose}\ \isacommand{using}\isamarkupfalse%
\ pcp{\isacharunderscore}seq{\isacharunderscore}sub\ e\ su\ \isacommand{by}\isamarkupfalse%
\ blast\isanewline
\ \ \isacommand{hence}\isamarkupfalse%
\ {\isachardoublequoteopen}insert\ F\ {\isacharparenleft}pcp{\isacharunderscore}seq\ C\ S\ {\isacharparenleft}to{\isacharunderscore}nat\ F{\isacharparenright}{\isacharparenright}\ {\isasymin}\ C{\isachardoublequoteclose}\ \isacommand{using}\isamarkupfalse%
\ sc\ \isanewline
\ \ \ \ \isacommand{unfolding}\isamarkupfalse%
\ subset{\isacharunderscore}closed{\isacharunderscore}def\ \isacommand{using}\isamarkupfalse%
\ el\ \isacommand{by}\isamarkupfalse%
\ blast\isanewline
\ \ \isacommand{with}\isamarkupfalse%
\ {\isadigit{1}}\ \isacommand{show}\isamarkupfalse%
\ False\ \isacommand{{\isachardot}{\isachardot}}\isamarkupfalse%
\isanewline
\isacommand{qed}\isamarkupfalse%
%
\endisatagproof
{\isafoldproof}%
%
\isadelimproof
%
\endisadelimproof
%
\begin{isamarkuptext}%
A continuación mostremos un resultado sobre el límite de la sucesión de \isa{{\isadigit{4}}{\isachardot}{\isadigit{1}}{\isachardot}{\isadigit{1}}} que es 
  consecuencia de que dicho límite sea un elemento maximal de la colección que lo define si esta
  verifica la propiedad de consistencia proposicional y es cerrada bajo subconjuntos.
  
  \begin{corolario}
    Sea \isa{C} una colección de conjuntos que verifica la propiedad de consistencia proposicional y
    es cerrada bajo subconjuntos, \isa{S} un conjunto, \isa{{\isacharbraceleft}S\isactrlsub n{\isacharbraceright}} la sucesión de conjuntos de \isa{C} a partir 
    de \isa{S} según la definición \isa{{\isadigit{4}}{\isachardot}{\isadigit{1}}{\isachardot}{\isadigit{1}}} y \isa{F} una fórmula proposicional. Entonces, si\\
    $\{F\} \cup \bigcup_{n = 0}^{\infty} S_{n} \in C$, se verifica que 
    $F \in \bigcup_{n = 0}^{\infty} S_{n}$. 
  \end{corolario}

  \begin{demostracion}
    Como \isa{C} es una colección que verifica la propiedad de consistencia proposicional y es cerrada 
    bajo subconjuntos, se tiene que el límite $\bigcup_{n = 0}^{\infty} S_{n}$ es maximal en \isa{C}. Por 
    lo tanto, si suponemos que $\{F\} \cup \bigcup_{n = 0}^{\infty} S_{n} \in C$, como el límite 
    está contenido en dicho conjunto, se cumple que 
    $\{F\} \cup \bigcup_{n = 0}^{\infty} S_{n} = \bigcup_{n = 0}^{\infty} S_{n}$, luego \isa{F} 
    pertenece al límite, como queríamos demostrar.
  \end{demostracion}

  Veamos su formalización y prueba detallada en Isabelle/HOL.%
\end{isamarkuptext}\isamarkuptrue%
\isacommand{lemma}\isamarkupfalse%
\isanewline
\ \ \isakeyword{assumes}\ {\isachardoublequoteopen}pcp\ C{\isachardoublequoteclose}\isanewline
\ \ \isakeyword{assumes}\ {\isachardoublequoteopen}subset{\isacharunderscore}closed\ C{\isachardoublequoteclose}\isanewline
\ \ \isakeyword{shows}\ {\isachardoublequoteopen}insert\ F\ {\isacharparenleft}pcp{\isacharunderscore}lim\ C\ S{\isacharparenright}\ {\isasymin}\ C\ {\isasymLongrightarrow}\ F\ {\isasymin}\ pcp{\isacharunderscore}lim\ C\ S{\isachardoublequoteclose}\isanewline
%
\isadelimproof
%
\endisadelimproof
%
\isatagproof
\isacommand{proof}\isamarkupfalse%
\ {\isacharminus}\isanewline
\ \ \isacommand{assume}\isamarkupfalse%
\ {\isachardoublequoteopen}insert\ F\ {\isacharparenleft}pcp{\isacharunderscore}lim\ C\ S{\isacharparenright}\ {\isasymin}\ C{\isachardoublequoteclose}\isanewline
\ \ \isacommand{have}\isamarkupfalse%
\ {\isachardoublequoteopen}pcp{\isacharunderscore}lim\ C\ S\ {\isasymsubseteq}\ insert\ F\ {\isacharparenleft}pcp{\isacharunderscore}lim\ C\ S{\isacharparenright}{\isachardoublequoteclose}\isanewline
\ \ \ \ \isacommand{by}\isamarkupfalse%
\ {\isacharparenleft}rule\ subset{\isacharunderscore}insertI{\isacharparenright}\ \isanewline
\ \ \isacommand{have}\isamarkupfalse%
\ {\isachardoublequoteopen}pcp{\isacharunderscore}lim\ C\ S\ {\isacharequal}\ insert\ F\ {\isacharparenleft}pcp{\isacharunderscore}lim\ C\ S{\isacharparenright}{\isachardoublequoteclose}\isanewline
\ \ \ \ \isacommand{using}\isamarkupfalse%
\ assms{\isacharparenleft}{\isadigit{1}}{\isacharparenright}\ assms{\isacharparenleft}{\isadigit{2}}{\isacharparenright}\ {\isacartoucheopen}insert\ F\ {\isacharparenleft}pcp{\isacharunderscore}lim\ C\ S{\isacharparenright}\ {\isasymin}\ C{\isacartoucheclose}\ {\isacartoucheopen}pcp{\isacharunderscore}lim\ C\ S\ {\isasymsubseteq}\ insert\ F\ {\isacharparenleft}pcp{\isacharunderscore}lim\ C\ S{\isacharparenright}{\isacartoucheclose}\ \isacommand{by}\isamarkupfalse%
\ {\isacharparenleft}rule\ cl{\isacharunderscore}max{\isacharparenright}\isanewline
\ \ \isacommand{then}\isamarkupfalse%
\ \isacommand{have}\isamarkupfalse%
\ {\isachardoublequoteopen}insert\ F\ {\isacharparenleft}pcp{\isacharunderscore}lim\ C\ S{\isacharparenright}\ {\isasymsubseteq}\ pcp{\isacharunderscore}lim\ C\ S{\isachardoublequoteclose}\isanewline
\ \ \ \ \isacommand{by}\isamarkupfalse%
\ {\isacharparenleft}rule\ equalityD{\isadigit{2}}{\isacharparenright}\isanewline
\ \ \isacommand{then}\isamarkupfalse%
\ \isacommand{have}\isamarkupfalse%
\ {\isachardoublequoteopen}F\ {\isasymin}\ pcp{\isacharunderscore}lim\ C\ S\ {\isasymand}\ pcp{\isacharunderscore}lim\ C\ S\ {\isasymsubseteq}\ pcp{\isacharunderscore}lim\ C\ S{\isachardoublequoteclose}\isanewline
\ \ \ \ \isacommand{by}\isamarkupfalse%
\ {\isacharparenleft}simp\ only{\isacharcolon}\ insert{\isacharunderscore}subset{\isacharparenright}\isanewline
\ \ \isacommand{thus}\isamarkupfalse%
\ {\isachardoublequoteopen}F\ {\isasymin}\ pcp{\isacharunderscore}lim\ C\ S{\isachardoublequoteclose}\isanewline
\ \ \ \ \isacommand{by}\isamarkupfalse%
\ {\isacharparenleft}rule\ conjunct{\isadigit{1}}{\isacharparenright}\isanewline
\isacommand{qed}\isamarkupfalse%
%
\endisatagproof
{\isafoldproof}%
%
\isadelimproof
%
\endisadelimproof
%
\begin{isamarkuptext}%
Mostremos su demostración automática.%
\end{isamarkuptext}\isamarkuptrue%
\isacommand{lemma}\isamarkupfalse%
\ cl{\isacharunderscore}max{\isacharprime}{\isacharcolon}\isanewline
\ \ \isakeyword{assumes}\ c{\isacharcolon}\ {\isachardoublequoteopen}pcp\ C{\isachardoublequoteclose}\isanewline
\ \ \isakeyword{assumes}\ sc{\isacharcolon}\ {\isachardoublequoteopen}subset{\isacharunderscore}closed\ C{\isachardoublequoteclose}\isanewline
\ \ \isakeyword{shows}\ {\isachardoublequoteopen}insert\ F\ {\isacharparenleft}pcp{\isacharunderscore}lim\ C\ S{\isacharparenright}\ {\isasymin}\ C\ {\isasymLongrightarrow}\ F\ {\isasymin}\ pcp{\isacharunderscore}lim\ C\ S{\isachardoublequoteclose}\isanewline
%
\isadelimproof
\ \ %
\endisadelimproof
%
\isatagproof
\isacommand{using}\isamarkupfalse%
\ cl{\isacharunderscore}max{\isacharbrackleft}OF\ assms{\isacharbrackright}\ \isacommand{by}\isamarkupfalse%
\ blast{\isacharplus}%
\endisatagproof
{\isafoldproof}%
%
\isadelimproof
%
\endisadelimproof
%
\begin{isamarkuptext}%
El siguiente resultado prueba que el límite de la sucesión definida en \isa{{\isadigit{4}}{\isachardot}{\isadigit{1}}{\isachardot}{\isadigit{1}}} es un conjunto
  de Hintikka si la colección que lo define verifica la propiedad de consistencia proposicional, es
  es cerrada bajo subconjuntos y es de carácter finito. Como consecuencia del \isa{teorema\ de\ Hintikka},
  se trata en particular de un conjunto satisfacible. 

  \begin{lema}
    Sea \isa{C} una colección de conjuntos que verifica la propiedad de consistencia proposicional, es
    es cerrada bajo subconjuntos y es de carácter finito. Sea \isa{S\ {\isasymin}\ C} y \isa{{\isacharbraceleft}S\isactrlsub n{\isacharbraceright}} la sucesión de
    conjuntos de \isa{C} a partir de \isa{S} según la definición \isa{{\isadigit{4}}{\isachardot}{\isadigit{1}}{\isachardot}{\isadigit{1}}}. Entonces, el límite de la sucesión
    \isa{{\isacharbraceleft}S\isactrlsub n{\isacharbraceright}} es un conjunto de Hintikka.
  \end{lema}

  \begin{demostracion}
    Para facilitar la lectura, vamos a notar por \isa{L\isactrlsub S\isactrlsub C} al límite de la sucesión \isa{{\isacharbraceleft}S\isactrlsub n{\isacharbraceright}} descrita 
    en el enunciado.

    Por resultados anteriores, como \isa{C} verifica la propiedad de consistencia proposicional, es
    es cerrada bajo subconjuntos y es de carácter finito, se tiene que \isa{L\isactrlsub S\isactrlsub C\ {\isasymin}\ C}. En particular, por 
    verificar la propiedad de consistencia proposicional, por el lema de\\ caracterización de dicha
    propiedad mediante notación uniforme, se cumplen las siguientes condiciones para \isa{L\isactrlsub S\isactrlsub C}:

    \begin{itemize}
      \item \isa{{\isasymbottom}\ {\isasymnotin}\ L\isactrlsub S\isactrlsub C}.
      \item Dada \isa{p} una fórmula atómica cualquiera, no se tiene 
      simultáneamente que\\ \isa{p\ {\isasymin}\ L\isactrlsub S\isactrlsub C} y \isa{{\isasymnot}\ p\ {\isasymin}\ L\isactrlsub S\isactrlsub C}.
      \item Para toda fórmula de tipo \isa{{\isasymalpha}} con componentes \isa{{\isasymalpha}\isactrlsub {\isadigit{1}}} y \isa{{\isasymalpha}\isactrlsub {\isadigit{2}}} tal que \isa{{\isasymalpha}}
      pertenece a \isa{L\isactrlsub S\isactrlsub C}, se tiene que \isa{{\isacharbraceleft}{\isasymalpha}\isactrlsub {\isadigit{1}}{\isacharcomma}{\isasymalpha}\isactrlsub {\isadigit{2}}{\isacharbraceright}\ {\isasymunion}\ L\isactrlsub S\isactrlsub C} pertenece a \isa{C}.
      \item Para toda fórmula de tipo \isa{{\isasymbeta}} con componentes \isa{{\isasymbeta}\isactrlsub {\isadigit{1}}} y \isa{{\isasymbeta}\isactrlsub {\isadigit{2}}} tal que \isa{{\isasymbeta}}
      pertenece a \isa{L\isactrlsub S\isactrlsub C}, se tiene que o bien \isa{{\isacharbraceleft}{\isasymbeta}\isactrlsub {\isadigit{1}}{\isacharbraceright}\ {\isasymunion}\ L\isactrlsub S\isactrlsub C} pertenece a \isa{C} o 
      bien \isa{{\isacharbraceleft}{\isasymbeta}\isactrlsub {\isadigit{2}}{\isacharbraceright}\ {\isasymunion}\ L\isactrlsub S\isactrlsub C} pertenece a \isa{C}.
    \end{itemize}

    Veamos que \isa{L\isactrlsub S\isactrlsub C} es un conjunto de Hintikka probando que cumple las condiciones del
    lema de caracterización de los conjuntos de Hintikka mediante notación uniforme, es decir,
    probaremos que \isa{L\isactrlsub S\isactrlsub C} verifica:

    \begin{itemize}
      \item \isa{{\isasymbottom}\ {\isasymnotin}\ L\isactrlsub S\isactrlsub C}.
      \item Dada \isa{p} una fórmula atómica cualquiera, no se tiene 
      simultáneamente que\\ \isa{p\ {\isasymin}\ L\isactrlsub S\isactrlsub C} y \isa{{\isasymnot}\ p\ {\isasymin}\ L\isactrlsub S\isactrlsub C}.
      \item Para toda fórmula de tipo \isa{{\isasymalpha}} con componentes \isa{{\isasymalpha}\isactrlsub {\isadigit{1}}} y \isa{{\isasymalpha}\isactrlsub {\isadigit{2}}} se verifica 
      que si la fórmula pertenece a \isa{L\isactrlsub S\isactrlsub C}, entonces \isa{{\isasymalpha}\isactrlsub {\isadigit{1}}} y \isa{{\isasymalpha}\isactrlsub {\isadigit{2}}} también.
      \item Para toda fórmula de tipo \isa{{\isasymbeta}} con componentes \isa{{\isasymbeta}\isactrlsub {\isadigit{1}}} y \isa{{\isasymbeta}\isactrlsub {\isadigit{2}}} se verifica 
      que si la fórmula pertenece a \isa{L\isactrlsub S\isactrlsub C}, entonces o bien \isa{{\isasymbeta}\isactrlsub {\isadigit{1}}} pertenece
      a \isa{L\isactrlsub S\isactrlsub C} o bien \isa{{\isasymbeta}\isactrlsub {\isadigit{2}}} pertenece a \isa{L\isactrlsub S\isactrlsub C}.
    \end{itemize} 

    Observemos que las dos primeras condiciones coinciden con las obtenidas anteriormente para \isa{L\isactrlsub S\isactrlsub C} 
    por el lema de caracterización de la propiedad de consistencia proposicional mediante notación
    uniforme. Veamos que, en efecto, se cumplen el resto de condiciones.

    En primer lugar, probemos que para una fórmula \isa{F} de tipo \isa{{\isasymalpha}} y componentes \isa{{\isasymalpha}\isactrlsub {\isadigit{1}}} y \isa{{\isasymalpha}\isactrlsub {\isadigit{2}}} tal que 
    \isa{F\ {\isasymin}\ L\isactrlsub S\isactrlsub C} se verifica que tanto \isa{{\isasymalpha}\isactrlsub {\isadigit{1}}} como \isa{{\isasymalpha}\isactrlsub {\isadigit{2}}} pertenecen a \isa{L\isactrlsub S\isactrlsub C}. Por la tercera condición 
    obtenida anteriormente para \isa{L\isactrlsub S\isactrlsub C} por el lema de caracterización de la propiedad de consistencia 
    proposicional mediante notación uniforme, se cumple que\\ \isa{{\isacharbraceleft}{\isasymalpha}\isactrlsub {\isadigit{1}}{\isacharcomma}{\isasymalpha}\isactrlsub {\isadigit{2}}{\isacharbraceright}\ {\isasymunion}\ L\isactrlsub S\isactrlsub C\ {\isasymin}\ C}. Observemos que
    se verifica \isa{L\isactrlsub S\isactrlsub C\ {\isasymsubseteq}\ {\isacharbraceleft}{\isasymalpha}\isactrlsub {\isadigit{1}}{\isacharcomma}{\isasymalpha}\isactrlsub {\isadigit{2}}{\isacharbraceright}\ {\isasymunion}\ L\isactrlsub S\isactrlsub C}. De este modo, como \isa{C} es una colección con la propiedad de 
    consistencia proposicional y cerrada bajo subconjuntos, por el lema \isa{{\isadigit{4}}{\isachardot}{\isadigit{2}}{\isachardot}{\isadigit{2}}} se tiene que 
    los conjuntos \isa{L\isactrlsub S\isactrlsub C} y \isa{{\isacharbraceleft}{\isasymalpha}\isactrlsub {\isadigit{1}}{\isacharcomma}{\isasymalpha}\isactrlsub {\isadigit{2}}{\isacharbraceright}\ {\isasymunion}\ L\isactrlsub S\isactrlsub C} coinciden. Por tanto, es claro que \isa{{\isasymalpha}\isactrlsub {\isadigit{1}}\ {\isasymin}\ L\isactrlsub S\isactrlsub C} y \isa{{\isasymalpha}\isactrlsub {\isadigit{2}}\ {\isasymin}\ L\isactrlsub S\isactrlsub C}, 
    como queríamos demostrar.

    Por último, demostremos que para una fórmula \isa{F} de tipo \isa{{\isasymbeta}} y componentes \isa{{\isasymbeta}\isactrlsub {\isadigit{1}}} y \isa{{\isasymbeta}\isactrlsub {\isadigit{2}}} tal que
    \isa{F\ {\isasymin}\ L\isactrlsub S\isactrlsub C} se verifica que o bien \isa{{\isasymbeta}\isactrlsub {\isadigit{1}}\ {\isasymin}\ L\isactrlsub S\isactrlsub C} o bien \isa{{\isasymbeta}\isactrlsub {\isadigit{2}}\ {\isasymin}\ L\isactrlsub S\isactrlsub C}. Por la cuarta condición obtenida 
    anteriormente para \isa{L\isactrlsub S\isactrlsub C} por el lema de caracterización de la propiedad de consistencia 
    proposicional mediante notación uniforme, se cumple que o bien\\ \isa{{\isacharbraceleft}{\isasymbeta}\isactrlsub {\isadigit{1}}{\isacharbraceright}\ {\isasymunion}\ L\isactrlsub S\isactrlsub C\ {\isasymin}\ C} o bien 
    \isa{{\isacharbraceleft}{\isasymbeta}\isactrlsub {\isadigit{2}}{\isacharbraceright}\ {\isasymunion}\ L\isactrlsub S\isactrlsub C\ {\isasymin}\ C}. De este modo, si suponemos que \isa{{\isacharbraceleft}{\isasymbeta}\isactrlsub {\isadigit{1}}{\isacharbraceright}\ {\isasymunion}\ L\isactrlsub S\isactrlsub C\ {\isasymin}\ C}, como \isa{C} tiene la propiedad de 
    consistencia proposicional y es cerrada bajo subconjuntos, por el corolario \isa{{\isadigit{4}}{\isachardot}{\isadigit{2}}{\isachardot}{\isadigit{3}}} se tiene 
    que \isa{{\isasymbeta}\isactrlsub {\isadigit{1}}\ {\isasymin}\ L\isactrlsub S\isactrlsub C}. Por tanto, se cumple que o bien \isa{{\isasymbeta}\isactrlsub {\isadigit{1}}\ {\isasymin}\ L\isactrlsub S\isactrlsub C} o bien \isa{{\isasymbeta}\isactrlsub {\isadigit{2}}\ {\isasymin}\ L\isactrlsub S\isactrlsub C}. Si suponemos que 
    \isa{{\isacharbraceleft}{\isasymbeta}\isactrlsub {\isadigit{2}}{\isacharbraceright}\ {\isasymunion}\ L\isactrlsub S\isactrlsub C\ {\isasymin}\ C}, se observa fácilmente que llegamos a la misma conclusión de manera análoga. 
    Por lo tanto, queda probado el resultado.
  \end{demostracion}

  Veamos su formalización y prueba detallada en Isabelle.%
\end{isamarkuptext}\isamarkuptrue%
\isacommand{lemma}\isamarkupfalse%
\isanewline
\ \ \isakeyword{assumes}\ {\isachardoublequoteopen}pcp\ C{\isachardoublequoteclose}\isanewline
\ \ \isakeyword{assumes}\ {\isachardoublequoteopen}subset{\isacharunderscore}closed\ C{\isachardoublequoteclose}\isanewline
\ \ \isakeyword{assumes}\ {\isachardoublequoteopen}finite{\isacharunderscore}character\ C{\isachardoublequoteclose}\isanewline
\ \ \isakeyword{assumes}\ {\isachardoublequoteopen}S\ {\isasymin}\ C{\isachardoublequoteclose}\isanewline
\ \ \isakeyword{shows}\ {\isachardoublequoteopen}Hintikka\ {\isacharparenleft}pcp{\isacharunderscore}lim\ C\ S{\isacharparenright}{\isachardoublequoteclose}\isanewline
%
\isadelimproof
%
\endisadelimproof
%
\isatagproof
\isacommand{proof}\isamarkupfalse%
\ {\isacharparenleft}rule\ Hintikka{\isacharunderscore}alt{\isadigit{2}}{\isacharparenright}\isanewline
\ \ \isacommand{let}\isamarkupfalse%
\ {\isacharquery}cl\ {\isacharequal}\ {\isachardoublequoteopen}pcp{\isacharunderscore}lim\ C\ S{\isachardoublequoteclose}\isanewline
\ \ \isacommand{have}\isamarkupfalse%
\ {\isachardoublequoteopen}{\isacharquery}cl\ {\isasymin}\ C{\isachardoublequoteclose}\isanewline
\ \ \ \ \isacommand{using}\isamarkupfalse%
\ assms{\isacharparenleft}{\isadigit{1}}{\isacharparenright}\ assms{\isacharparenleft}{\isadigit{4}}{\isacharparenright}\ assms{\isacharparenleft}{\isadigit{2}}{\isacharparenright}\ assms{\isacharparenleft}{\isadigit{3}}{\isacharparenright}\ \isacommand{by}\isamarkupfalse%
\ {\isacharparenleft}rule\ pcp{\isacharunderscore}lim{\isacharunderscore}in{\isacharparenright}\isanewline
\ \ \isacommand{have}\isamarkupfalse%
\ {\isachardoublequoteopen}{\isacharparenleft}{\isasymforall}S\ {\isasymin}\ C{\isachardot}\isanewline
\ \ {\isasymbottom}\ {\isasymnotin}\ S\isanewline
{\isasymand}\ {\isacharparenleft}{\isasymforall}k{\isachardot}\ Atom\ k\ {\isasymin}\ S\ {\isasymlongrightarrow}\ \isactrlbold {\isasymnot}\ {\isacharparenleft}Atom\ k{\isacharparenright}\ {\isasymin}\ S\ {\isasymlongrightarrow}\ False{\isacharparenright}\isanewline
{\isasymand}\ {\isacharparenleft}{\isasymforall}F\ G\ H{\isachardot}\ Con\ F\ G\ H\ {\isasymlongrightarrow}\ F\ {\isasymin}\ S\ {\isasymlongrightarrow}\ {\isacharbraceleft}G{\isacharcomma}H{\isacharbraceright}\ {\isasymunion}\ S\ {\isasymin}\ C{\isacharparenright}\isanewline
{\isasymand}\ {\isacharparenleft}{\isasymforall}F\ G\ H{\isachardot}\ Dis\ F\ G\ H\ {\isasymlongrightarrow}\ F\ {\isasymin}\ S\ {\isasymlongrightarrow}\ {\isacharbraceleft}G{\isacharbraceright}\ {\isasymunion}\ S\ {\isasymin}\ C\ {\isasymor}\ {\isacharbraceleft}H{\isacharbraceright}\ {\isasymunion}\ S\ {\isasymin}\ C{\isacharparenright}{\isacharparenright}{\isachardoublequoteclose}\isanewline
\ \ \ \ \isacommand{using}\isamarkupfalse%
\ assms{\isacharparenleft}{\isadigit{1}}{\isacharparenright}\ \isacommand{by}\isamarkupfalse%
\ {\isacharparenleft}rule\ pcp{\isacharunderscore}alt{\isadigit{1}}{\isacharparenright}\isanewline
\ \ \isacommand{then}\isamarkupfalse%
\ \isacommand{have}\isamarkupfalse%
\ d{\isacharcolon}{\isachardoublequoteopen}{\isasymbottom}\ {\isasymnotin}\ {\isacharquery}cl\isanewline
{\isasymand}\ {\isacharparenleft}{\isasymforall}k{\isachardot}\ Atom\ k\ {\isasymin}\ {\isacharquery}cl\ {\isasymlongrightarrow}\ \isactrlbold {\isasymnot}\ {\isacharparenleft}Atom\ k{\isacharparenright}\ {\isasymin}\ {\isacharquery}cl\ {\isasymlongrightarrow}\ False{\isacharparenright}\isanewline
{\isasymand}\ {\isacharparenleft}{\isasymforall}F\ G\ H{\isachardot}\ Con\ F\ G\ H\ {\isasymlongrightarrow}\ F\ {\isasymin}\ {\isacharquery}cl\ {\isasymlongrightarrow}\ {\isacharbraceleft}G{\isacharcomma}H{\isacharbraceright}\ {\isasymunion}\ {\isacharquery}cl\ {\isasymin}\ C{\isacharparenright}\isanewline
{\isasymand}\ {\isacharparenleft}{\isasymforall}F\ G\ H{\isachardot}\ Dis\ F\ G\ H\ {\isasymlongrightarrow}\ F\ {\isasymin}\ {\isacharquery}cl\ {\isasymlongrightarrow}\ {\isacharbraceleft}G{\isacharbraceright}\ {\isasymunion}\ {\isacharquery}cl\ {\isasymin}\ C\ {\isasymor}\ {\isacharbraceleft}H{\isacharbraceright}\ {\isasymunion}\ {\isacharquery}cl\ {\isasymin}\ C{\isacharparenright}{\isachardoublequoteclose}\isanewline
\ \ \ \ \isacommand{using}\isamarkupfalse%
\ {\isacartoucheopen}{\isacharquery}cl\ {\isasymin}\ C{\isacartoucheclose}\ \isacommand{by}\isamarkupfalse%
\ {\isacharparenleft}rule\ bspec{\isacharparenright}\isanewline
\ \ \isacommand{then}\isamarkupfalse%
\ \isacommand{have}\isamarkupfalse%
\ H{\isadigit{1}}{\isacharcolon}{\isachardoublequoteopen}{\isasymbottom}\ {\isasymnotin}\ {\isacharquery}cl{\isachardoublequoteclose}\isanewline
\ \ \ \ \isacommand{by}\isamarkupfalse%
\ {\isacharparenleft}rule\ conjunct{\isadigit{1}}{\isacharparenright}\isanewline
\ \ \isacommand{have}\isamarkupfalse%
\ H{\isadigit{2}}{\isacharcolon}{\isachardoublequoteopen}{\isasymforall}k{\isachardot}\ Atom\ k\ {\isasymin}\ {\isacharquery}cl\ {\isasymlongrightarrow}\ \isactrlbold {\isasymnot}\ {\isacharparenleft}Atom\ k{\isacharparenright}\ {\isasymin}\ {\isacharquery}cl\ {\isasymlongrightarrow}\ False{\isachardoublequoteclose}\isanewline
\ \ \ \ \isacommand{using}\isamarkupfalse%
\ d\ \isacommand{by}\isamarkupfalse%
\ {\isacharparenleft}iprover\ elim{\isacharcolon}\ conjunct{\isadigit{2}}\ conjunct{\isadigit{1}}{\isacharparenright}\isanewline
\ \ \isacommand{have}\isamarkupfalse%
\ Con{\isacharcolon}{\isachardoublequoteopen}{\isasymforall}F\ G\ H{\isachardot}\ Con\ F\ G\ H\ {\isasymlongrightarrow}\ F\ {\isasymin}\ {\isacharquery}cl\ {\isasymlongrightarrow}\ {\isacharbraceleft}G{\isacharcomma}H{\isacharbraceright}\ {\isasymunion}\ {\isacharquery}cl\ {\isasymin}\ C{\isachardoublequoteclose}\isanewline
\ \ \ \ \isacommand{using}\isamarkupfalse%
\ d\ \isacommand{by}\isamarkupfalse%
\ {\isacharparenleft}iprover\ elim{\isacharcolon}\ conjunct{\isadigit{2}}\ conjunct{\isadigit{1}}{\isacharparenright}\isanewline
\ \ \isacommand{have}\isamarkupfalse%
\ H{\isadigit{3}}{\isacharcolon}{\isachardoublequoteopen}{\isasymforall}F\ G\ H{\isachardot}\ Con\ F\ G\ H\ {\isasymlongrightarrow}\ F\ {\isasymin}\ {\isacharquery}cl\ {\isasymlongrightarrow}\ G\ {\isasymin}\ {\isacharquery}cl\ {\isasymand}\ H\ {\isasymin}\ {\isacharquery}cl{\isachardoublequoteclose}\isanewline
\ \ \isacommand{proof}\isamarkupfalse%
\ {\isacharparenleft}rule\ allI{\isacharparenright}{\isacharplus}\isanewline
\ \ \ \ \isacommand{fix}\isamarkupfalse%
\ F\ G\ H\isanewline
\ \ \ \ \isacommand{show}\isamarkupfalse%
\ {\isachardoublequoteopen}Con\ F\ G\ H\ {\isasymlongrightarrow}\ F\ {\isasymin}\ {\isacharquery}cl\ {\isasymlongrightarrow}\ G\ {\isasymin}\ {\isacharquery}cl\ {\isasymand}\ H\ {\isasymin}\ {\isacharquery}cl{\isachardoublequoteclose}\isanewline
\ \ \ \ \isacommand{proof}\isamarkupfalse%
\ {\isacharparenleft}rule\ impI{\isacharparenright}{\isacharplus}\isanewline
\ \ \ \ \ \ \isacommand{assume}\isamarkupfalse%
\ {\isachardoublequoteopen}Con\ F\ G\ H{\isachardoublequoteclose}\isanewline
\ \ \ \ \ \ \isacommand{assume}\isamarkupfalse%
\ {\isachardoublequoteopen}F\ {\isasymin}\ {\isacharquery}cl{\isachardoublequoteclose}\isanewline
\ \ \ \ \ \ \isacommand{have}\isamarkupfalse%
\ {\isachardoublequoteopen}Con\ F\ G\ H\ {\isasymlongrightarrow}\ F\ {\isasymin}\ {\isacharquery}cl\ {\isasymlongrightarrow}\ {\isacharbraceleft}G{\isacharcomma}H{\isacharbraceright}\ {\isasymunion}\ {\isacharquery}cl\ {\isasymin}\ C{\isachardoublequoteclose}\isanewline
\ \ \ \ \ \ \ \ \isacommand{using}\isamarkupfalse%
\ Con\ \isacommand{by}\isamarkupfalse%
\ {\isacharparenleft}iprover\ elim{\isacharcolon}\ allE{\isacharparenright}\isanewline
\ \ \ \ \ \ \isacommand{then}\isamarkupfalse%
\ \isacommand{have}\isamarkupfalse%
\ {\isachardoublequoteopen}F\ {\isasymin}\ {\isacharquery}cl\ {\isasymlongrightarrow}\ {\isacharbraceleft}G{\isacharcomma}H{\isacharbraceright}\ {\isasymunion}\ {\isacharquery}cl\ {\isasymin}\ C{\isachardoublequoteclose}\isanewline
\ \ \ \ \ \ \ \ \isacommand{using}\isamarkupfalse%
\ {\isacartoucheopen}Con\ F\ G\ H{\isacartoucheclose}\ \isacommand{by}\isamarkupfalse%
\ {\isacharparenleft}rule\ mp{\isacharparenright}\isanewline
\ \ \ \ \ \ \isacommand{then}\isamarkupfalse%
\ \isacommand{have}\isamarkupfalse%
\ {\isachardoublequoteopen}{\isacharbraceleft}G{\isacharcomma}H{\isacharbraceright}\ {\isasymunion}\ {\isacharquery}cl\ {\isasymin}\ C{\isachardoublequoteclose}\isanewline
\ \ \ \ \ \ \ \ \isacommand{using}\isamarkupfalse%
\ {\isacartoucheopen}F\ {\isasymin}\ {\isacharquery}cl{\isacartoucheclose}\ \isacommand{by}\isamarkupfalse%
\ {\isacharparenleft}rule\ mp{\isacharparenright}\isanewline
\ \ \ \ \ \ \isacommand{have}\isamarkupfalse%
\ {\isachardoublequoteopen}{\isacharparenleft}insert\ G\ {\isacharparenleft}insert\ H\ {\isacharquery}cl{\isacharparenright}{\isacharparenright}\ {\isacharequal}\ {\isacharbraceleft}G{\isacharcomma}H{\isacharbraceright}\ {\isasymunion}\ {\isacharquery}cl{\isachardoublequoteclose}\isanewline
\ \ \ \ \ \ \ \ \isacommand{by}\isamarkupfalse%
\ {\isacharparenleft}rule\ insertSetElem{\isacharparenright}\isanewline
\ \ \ \ \ \ \isacommand{then}\isamarkupfalse%
\ \isacommand{have}\isamarkupfalse%
\ {\isachardoublequoteopen}{\isacharparenleft}insert\ G\ {\isacharparenleft}insert\ H\ {\isacharquery}cl{\isacharparenright}{\isacharparenright}\ {\isasymin}\ C{\isachardoublequoteclose}\isanewline
\ \ \ \ \ \ \ \ \isacommand{using}\isamarkupfalse%
\ {\isacartoucheopen}{\isacharbraceleft}G{\isacharcomma}H{\isacharbraceright}\ {\isasymunion}\ {\isacharquery}cl\ {\isasymin}\ C{\isacartoucheclose}\ \isacommand{by}\isamarkupfalse%
\ {\isacharparenleft}simp\ only{\isacharcolon}\ {\isacartoucheopen}{\isacharparenleft}insert\ G\ {\isacharparenleft}insert\ H\ {\isacharquery}cl{\isacharparenright}{\isacharparenright}\ {\isacharequal}\ {\isacharbraceleft}G{\isacharcomma}H{\isacharbraceright}\ {\isasymunion}\ {\isacharquery}cl{\isacartoucheclose}{\isacharparenright}\isanewline
\ \ \ \ \ \ \isacommand{have}\isamarkupfalse%
\ {\isachardoublequoteopen}{\isacharquery}cl\ {\isasymsubseteq}\ insert\ H\ {\isacharquery}cl{\isachardoublequoteclose}\isanewline
\ \ \ \ \ \ \ \ \isacommand{by}\isamarkupfalse%
\ {\isacharparenleft}rule\ subset{\isacharunderscore}insertI{\isacharparenright}\isanewline
\ \ \ \ \ \ \isacommand{then}\isamarkupfalse%
\ \isacommand{have}\isamarkupfalse%
\ {\isachardoublequoteopen}{\isacharquery}cl\ {\isasymsubseteq}\ insert\ G\ {\isacharparenleft}insert\ H\ {\isacharquery}cl{\isacharparenright}{\isachardoublequoteclose}\isanewline
\ \ \ \ \ \ \ \ \isacommand{by}\isamarkupfalse%
\ {\isacharparenleft}rule\ subset{\isacharunderscore}insertI{\isadigit{2}}{\isacharparenright}\isanewline
\ \ \ \ \ \ \isacommand{have}\isamarkupfalse%
\ {\isachardoublequoteopen}{\isacharquery}cl\ {\isacharequal}\ insert\ G\ {\isacharparenleft}insert\ H\ {\isacharquery}cl{\isacharparenright}{\isachardoublequoteclose}\ \isanewline
\ \ \ \ \ \ \ \ \isacommand{using}\isamarkupfalse%
\ assms{\isacharparenleft}{\isadigit{1}}{\isacharparenright}\ assms{\isacharparenleft}{\isadigit{2}}{\isacharparenright}\ {\isacartoucheopen}insert\ G\ {\isacharparenleft}insert\ H\ {\isacharquery}cl{\isacharparenright}\ {\isasymin}\ C{\isacartoucheclose}\ {\isacartoucheopen}{\isacharquery}cl\ {\isasymsubseteq}\ insert\ G\ {\isacharparenleft}insert\ H\ {\isacharquery}cl{\isacharparenright}{\isacartoucheclose}\ \isacommand{by}\isamarkupfalse%
\ {\isacharparenleft}rule\ cl{\isacharunderscore}max{\isacharparenright}\isanewline
\ \ \ \ \ \ \isacommand{then}\isamarkupfalse%
\ \isacommand{have}\isamarkupfalse%
\ {\isachardoublequoteopen}insert\ G\ {\isacharparenleft}insert\ H\ {\isacharquery}cl{\isacharparenright}\ {\isasymsubseteq}\ {\isacharquery}cl{\isachardoublequoteclose}\isanewline
\ \ \ \ \ \ \ \ \isacommand{by}\isamarkupfalse%
\ {\isacharparenleft}simp\ only{\isacharcolon}\ equalityD{\isadigit{2}}{\isacharparenright}\isanewline
\ \ \ \ \ \ \isacommand{then}\isamarkupfalse%
\ \isacommand{have}\isamarkupfalse%
\ {\isachardoublequoteopen}G\ {\isasymin}\ {\isacharquery}cl\ {\isasymand}\ insert\ H\ {\isacharquery}cl\ {\isasymsubseteq}\ {\isacharquery}cl{\isachardoublequoteclose}\isanewline
\ \ \ \ \ \ \ \ \isacommand{by}\isamarkupfalse%
\ {\isacharparenleft}simp\ only{\isacharcolon}\ insert{\isacharunderscore}subset{\isacharparenright}\isanewline
\ \ \ \ \ \ \isacommand{then}\isamarkupfalse%
\ \isacommand{have}\isamarkupfalse%
\ {\isachardoublequoteopen}G\ {\isasymin}\ {\isacharquery}cl{\isachardoublequoteclose}\isanewline
\ \ \ \ \ \ \ \ \isacommand{by}\isamarkupfalse%
\ {\isacharparenleft}rule\ conjunct{\isadigit{1}}{\isacharparenright}\isanewline
\ \ \ \ \ \ \isacommand{have}\isamarkupfalse%
\ {\isachardoublequoteopen}insert\ H\ {\isacharquery}cl\ {\isasymsubseteq}\ {\isacharquery}cl{\isachardoublequoteclose}\isanewline
\ \ \ \ \ \ \ \ \isacommand{using}\isamarkupfalse%
\ {\isacartoucheopen}G\ {\isasymin}\ {\isacharquery}cl\ {\isasymand}\ insert\ H\ {\isacharquery}cl\ {\isasymsubseteq}\ {\isacharquery}cl{\isacartoucheclose}\ \isacommand{by}\isamarkupfalse%
\ {\isacharparenleft}rule\ conjunct{\isadigit{2}}{\isacharparenright}\isanewline
\ \ \ \ \ \ \isacommand{then}\isamarkupfalse%
\ \isacommand{have}\isamarkupfalse%
\ {\isachardoublequoteopen}H\ {\isasymin}\ {\isacharquery}cl\ {\isasymand}\ {\isacharquery}cl\ {\isasymsubseteq}\ {\isacharquery}cl{\isachardoublequoteclose}\isanewline
\ \ \ \ \ \ \ \ \isacommand{by}\isamarkupfalse%
\ {\isacharparenleft}simp\ only{\isacharcolon}\ insert{\isacharunderscore}subset{\isacharparenright}\isanewline
\ \ \ \ \ \ \isacommand{then}\isamarkupfalse%
\ \isacommand{have}\isamarkupfalse%
\ {\isachardoublequoteopen}H\ {\isasymin}\ {\isacharquery}cl{\isachardoublequoteclose}\isanewline
\ \ \ \ \ \ \ \ \isacommand{by}\isamarkupfalse%
\ {\isacharparenleft}rule\ conjunct{\isadigit{1}}{\isacharparenright}\isanewline
\ \ \ \ \ \ \isacommand{show}\isamarkupfalse%
\ {\isachardoublequoteopen}G\ {\isasymin}\ {\isacharquery}cl\ {\isasymand}\ H\ {\isasymin}\ {\isacharquery}cl{\isachardoublequoteclose}\isanewline
\ \ \ \ \ \ \ \ \isacommand{using}\isamarkupfalse%
\ {\isacartoucheopen}G\ {\isasymin}\ {\isacharquery}cl{\isacartoucheclose}\ {\isacartoucheopen}H\ {\isasymin}\ {\isacharquery}cl{\isacartoucheclose}\ \isacommand{by}\isamarkupfalse%
\ {\isacharparenleft}rule\ conjI{\isacharparenright}\isanewline
\ \ \ \ \isacommand{qed}\isamarkupfalse%
\isanewline
\ \ \isacommand{qed}\isamarkupfalse%
\isanewline
\ \ \isacommand{have}\isamarkupfalse%
\ Dis{\isacharcolon}{\isachardoublequoteopen}{\isasymforall}F\ G\ H{\isachardot}\ Dis\ F\ G\ H\ {\isasymlongrightarrow}\ F\ {\isasymin}\ {\isacharquery}cl\ {\isasymlongrightarrow}\ {\isacharbraceleft}G{\isacharbraceright}\ {\isasymunion}\ {\isacharquery}cl\ {\isasymin}\ C\ {\isasymor}\ {\isacharbraceleft}H{\isacharbraceright}\ {\isasymunion}\ {\isacharquery}cl\ {\isasymin}\ C{\isachardoublequoteclose}\isanewline
\ \ \ \ \isacommand{using}\isamarkupfalse%
\ d\ \isacommand{by}\isamarkupfalse%
\ {\isacharparenleft}iprover\ elim{\isacharcolon}\ conjunct{\isadigit{2}}\ conjunct{\isadigit{1}}{\isacharparenright}\isanewline
\ \ \isacommand{have}\isamarkupfalse%
\ H{\isadigit{4}}{\isacharcolon}{\isachardoublequoteopen}{\isasymforall}F\ G\ H{\isachardot}\ Dis\ F\ G\ H\ {\isasymlongrightarrow}\ F\ {\isasymin}\ {\isacharquery}cl\ {\isasymlongrightarrow}\ G\ {\isasymin}\ {\isacharquery}cl\ {\isasymor}\ H\ {\isasymin}\ {\isacharquery}cl{\isachardoublequoteclose}\isanewline
\ \ \isacommand{proof}\isamarkupfalse%
\ {\isacharparenleft}rule\ allI{\isacharparenright}{\isacharplus}\isanewline
\ \ \ \ \isacommand{fix}\isamarkupfalse%
\ F\ G\ H\isanewline
\ \ \ \ \isacommand{show}\isamarkupfalse%
\ {\isachardoublequoteopen}Dis\ F\ G\ H\ {\isasymlongrightarrow}\ F\ {\isasymin}\ {\isacharquery}cl\ {\isasymlongrightarrow}\ G\ {\isasymin}\ {\isacharquery}cl\ {\isasymor}\ H\ {\isasymin}\ {\isacharquery}cl{\isachardoublequoteclose}\isanewline
\ \ \ \ \isacommand{proof}\isamarkupfalse%
\ {\isacharparenleft}rule\ impI{\isacharparenright}{\isacharplus}\isanewline
\ \ \ \ \ \ \isacommand{assume}\isamarkupfalse%
\ {\isachardoublequoteopen}Dis\ F\ G\ H{\isachardoublequoteclose}\isanewline
\ \ \ \ \ \ \isacommand{assume}\isamarkupfalse%
\ {\isachardoublequoteopen}F\ {\isasymin}\ {\isacharquery}cl{\isachardoublequoteclose}\isanewline
\ \ \ \ \ \ \isacommand{have}\isamarkupfalse%
\ {\isachardoublequoteopen}Dis\ F\ G\ H\ {\isasymlongrightarrow}\ F\ {\isasymin}\ {\isacharquery}cl\ {\isasymlongrightarrow}\ {\isacharbraceleft}G{\isacharbraceright}\ {\isasymunion}\ {\isacharquery}cl\ {\isasymin}\ C\ {\isasymor}\ {\isacharbraceleft}H{\isacharbraceright}\ {\isasymunion}\ {\isacharquery}cl\ {\isasymin}\ C{\isachardoublequoteclose}\isanewline
\ \ \ \ \ \ \ \ \isacommand{using}\isamarkupfalse%
\ Dis\ \isacommand{by}\isamarkupfalse%
\ {\isacharparenleft}iprover\ elim{\isacharcolon}\ allE{\isacharparenright}\isanewline
\ \ \ \ \ \ \isacommand{then}\isamarkupfalse%
\ \isacommand{have}\isamarkupfalse%
\ {\isachardoublequoteopen}F\ {\isasymin}\ {\isacharquery}cl\ {\isasymlongrightarrow}\ {\isacharbraceleft}G{\isacharbraceright}\ {\isasymunion}\ {\isacharquery}cl\ {\isasymin}\ C\ {\isasymor}\ {\isacharbraceleft}H{\isacharbraceright}\ {\isasymunion}\ {\isacharquery}cl\ {\isasymin}\ C{\isachardoublequoteclose}\isanewline
\ \ \ \ \ \ \ \ \isacommand{using}\isamarkupfalse%
\ {\isacartoucheopen}Dis\ F\ G\ H{\isacartoucheclose}\ \isacommand{by}\isamarkupfalse%
\ {\isacharparenleft}rule\ mp{\isacharparenright}\isanewline
\ \ \ \ \ \ \isacommand{then}\isamarkupfalse%
\ \isacommand{have}\isamarkupfalse%
\ {\isachardoublequoteopen}{\isacharbraceleft}G{\isacharbraceright}\ {\isasymunion}\ {\isacharquery}cl\ {\isasymin}\ C\ {\isasymor}\ {\isacharbraceleft}H{\isacharbraceright}\ {\isasymunion}\ {\isacharquery}cl\ {\isasymin}\ C{\isachardoublequoteclose}\isanewline
\ \ \ \ \ \ \ \ \isacommand{using}\isamarkupfalse%
\ {\isacartoucheopen}F\ {\isasymin}\ {\isacharquery}cl{\isacartoucheclose}\ \isacommand{by}\isamarkupfalse%
\ {\isacharparenleft}rule\ mp{\isacharparenright}\isanewline
\ \ \ \ \ \ \isacommand{thus}\isamarkupfalse%
\ {\isachardoublequoteopen}G\ {\isasymin}\ {\isacharquery}cl\ {\isasymor}\ H\ {\isasymin}\ {\isacharquery}cl{\isachardoublequoteclose}\isanewline
\ \ \ \ \ \ \isacommand{proof}\isamarkupfalse%
\ {\isacharparenleft}rule\ disjE{\isacharparenright}\isanewline
\ \ \ \ \ \ \ \ \isacommand{assume}\isamarkupfalse%
\ {\isachardoublequoteopen}{\isacharbraceleft}G{\isacharbraceright}\ {\isasymunion}\ {\isacharquery}cl\ {\isasymin}\ C{\isachardoublequoteclose}\isanewline
\ \ \ \ \ \ \ \ \isacommand{have}\isamarkupfalse%
\ {\isachardoublequoteopen}insert\ G\ {\isacharquery}cl\ {\isacharequal}\ {\isacharbraceleft}G{\isacharbraceright}\ {\isasymunion}\ {\isacharquery}cl{\isachardoublequoteclose}\isanewline
\ \ \ \ \ \ \ \ \ \ \isacommand{by}\isamarkupfalse%
\ {\isacharparenleft}rule\ insert{\isacharunderscore}is{\isacharunderscore}Un{\isacharparenright}\isanewline
\ \ \ \ \ \ \ \ \isacommand{have}\isamarkupfalse%
\ {\isachardoublequoteopen}insert\ G\ {\isacharquery}cl\ {\isasymin}\ C{\isachardoublequoteclose}\isanewline
\ \ \ \ \ \ \ \ \ \ \isacommand{using}\isamarkupfalse%
\ {\isacartoucheopen}{\isacharbraceleft}G{\isacharbraceright}\ {\isasymunion}\ {\isacharquery}cl\ {\isasymin}\ C{\isacartoucheclose}\ \isacommand{by}\isamarkupfalse%
\ {\isacharparenleft}simp\ only{\isacharcolon}\ {\isacartoucheopen}insert\ G\ {\isacharquery}cl\ {\isacharequal}\ {\isacharbraceleft}G{\isacharbraceright}\ {\isasymunion}\ {\isacharquery}cl{\isacartoucheclose}{\isacharparenright}\isanewline
\ \ \ \ \ \ \ \ \isacommand{have}\isamarkupfalse%
\ {\isachardoublequoteopen}insert\ G\ {\isacharquery}cl\ {\isasymin}\ C\ {\isasymLongrightarrow}\ G\ {\isasymin}\ {\isacharquery}cl{\isachardoublequoteclose}\isanewline
\ \ \ \ \ \ \ \ \ \ \isacommand{using}\isamarkupfalse%
\ assms{\isacharparenleft}{\isadigit{1}}{\isacharparenright}\ assms{\isacharparenleft}{\isadigit{2}}{\isacharparenright}\ \isacommand{by}\isamarkupfalse%
\ {\isacharparenleft}rule\ cl{\isacharunderscore}max{\isacharprime}{\isacharparenright}\isanewline
\ \ \ \ \ \ \ \ \isacommand{then}\isamarkupfalse%
\ \isacommand{have}\isamarkupfalse%
\ {\isachardoublequoteopen}G\ {\isasymin}\ {\isacharquery}cl{\isachardoublequoteclose}\isanewline
\ \ \ \ \ \ \ \ \ \ \isacommand{by}\isamarkupfalse%
\ {\isacharparenleft}simp\ only{\isacharcolon}\ {\isacartoucheopen}insert\ G\ {\isacharquery}cl\ {\isasymin}\ C{\isacartoucheclose}{\isacharparenright}\isanewline
\ \ \ \ \ \ \ \ \isacommand{thus}\isamarkupfalse%
\ {\isachardoublequoteopen}G\ {\isasymin}\ {\isacharquery}cl\ {\isasymor}\ H\ {\isasymin}\ {\isacharquery}cl{\isachardoublequoteclose}\isanewline
\ \ \ \ \ \ \ \ \ \ \isacommand{by}\isamarkupfalse%
\ {\isacharparenleft}rule\ disjI{\isadigit{1}}{\isacharparenright}\isanewline
\ \ \ \ \ \ \isacommand{next}\isamarkupfalse%
\isanewline
\ \ \ \ \ \ \ \ \isacommand{assume}\isamarkupfalse%
\ {\isachardoublequoteopen}{\isacharbraceleft}H{\isacharbraceright}\ {\isasymunion}\ {\isacharquery}cl\ {\isasymin}\ C{\isachardoublequoteclose}\isanewline
\ \ \ \ \ \ \ \ \isacommand{have}\isamarkupfalse%
\ {\isachardoublequoteopen}insert\ H\ {\isacharquery}cl\ {\isacharequal}\ {\isacharbraceleft}H{\isacharbraceright}\ {\isasymunion}\ {\isacharquery}cl{\isachardoublequoteclose}\isanewline
\ \ \ \ \ \ \ \ \ \ \isacommand{by}\isamarkupfalse%
\ {\isacharparenleft}rule\ insert{\isacharunderscore}is{\isacharunderscore}Un{\isacharparenright}\isanewline
\ \ \ \ \ \ \ \ \isacommand{have}\isamarkupfalse%
\ {\isachardoublequoteopen}insert\ H\ {\isacharquery}cl\ {\isasymin}\ C{\isachardoublequoteclose}\isanewline
\ \ \ \ \ \ \ \ \ \ \isacommand{using}\isamarkupfalse%
\ {\isacartoucheopen}{\isacharbraceleft}H{\isacharbraceright}\ {\isasymunion}\ {\isacharquery}cl\ {\isasymin}\ C{\isacartoucheclose}\ \isacommand{by}\isamarkupfalse%
\ {\isacharparenleft}simp\ only{\isacharcolon}\ {\isacartoucheopen}insert\ H\ {\isacharquery}cl\ {\isacharequal}\ {\isacharbraceleft}H{\isacharbraceright}\ {\isasymunion}\ {\isacharquery}cl{\isacartoucheclose}{\isacharparenright}\isanewline
\ \ \ \ \ \ \ \ \isacommand{have}\isamarkupfalse%
\ {\isachardoublequoteopen}insert\ H\ {\isacharquery}cl\ {\isasymin}\ C\ {\isasymLongrightarrow}\ H\ {\isasymin}\ {\isacharquery}cl{\isachardoublequoteclose}\isanewline
\ \ \ \ \ \ \ \ \ \ \isacommand{using}\isamarkupfalse%
\ assms{\isacharparenleft}{\isadigit{1}}{\isacharparenright}\ assms{\isacharparenleft}{\isadigit{2}}{\isacharparenright}\ \isacommand{by}\isamarkupfalse%
\ {\isacharparenleft}rule\ cl{\isacharunderscore}max{\isacharprime}{\isacharparenright}\isanewline
\ \ \ \ \ \ \ \ \isacommand{then}\isamarkupfalse%
\ \isacommand{have}\isamarkupfalse%
\ {\isachardoublequoteopen}H\ {\isasymin}\ {\isacharquery}cl{\isachardoublequoteclose}\isanewline
\ \ \ \ \ \ \ \ \ \ \isacommand{by}\isamarkupfalse%
\ {\isacharparenleft}simp\ only{\isacharcolon}\ {\isacartoucheopen}insert\ H\ {\isacharquery}cl\ {\isasymin}\ C{\isacartoucheclose}{\isacharparenright}\isanewline
\ \ \ \ \ \ \ \ \isacommand{thus}\isamarkupfalse%
\ {\isachardoublequoteopen}G\ {\isasymin}\ {\isacharquery}cl\ {\isasymor}\ H\ {\isasymin}\ {\isacharquery}cl{\isachardoublequoteclose}\isanewline
\ \ \ \ \ \ \ \ \ \ \isacommand{by}\isamarkupfalse%
\ {\isacharparenleft}rule\ disjI{\isadigit{2}}{\isacharparenright}\isanewline
\ \ \ \ \ \ \isacommand{qed}\isamarkupfalse%
\isanewline
\ \ \ \ \isacommand{qed}\isamarkupfalse%
\isanewline
\ \ \isacommand{qed}\isamarkupfalse%
\isanewline
\ \ \isacommand{show}\isamarkupfalse%
\ {\isachardoublequoteopen}{\isasymbottom}\ {\isasymnotin}\ {\isacharquery}cl\ {\isasymand}\isanewline
\ \ \ \ {\isacharparenleft}{\isasymforall}k{\isachardot}\ Atom\ k\ {\isasymin}\ {\isacharquery}cl\ {\isasymlongrightarrow}\ \isactrlbold {\isasymnot}\ {\isacharparenleft}Atom\ k{\isacharparenright}\ {\isasymin}\ {\isacharquery}cl\ {\isasymlongrightarrow}\ False{\isacharparenright}\ {\isasymand}\isanewline
\ \ \ \ {\isacharparenleft}{\isasymforall}F\ G\ H{\isachardot}\ Con\ F\ G\ H\ {\isasymlongrightarrow}\ F\ {\isasymin}\ {\isacharquery}cl\ {\isasymlongrightarrow}\ G\ {\isasymin}\ {\isacharquery}cl\ {\isasymand}\ H\ {\isasymin}\ {\isacharquery}cl{\isacharparenright}\ {\isasymand}\isanewline
\ \ \ \ {\isacharparenleft}{\isasymforall}F\ G\ H{\isachardot}\ Dis\ F\ G\ H\ {\isasymlongrightarrow}\ F\ {\isasymin}\ {\isacharquery}cl\ {\isasymlongrightarrow}\ G\ {\isasymin}\ {\isacharquery}cl\ {\isasymor}\ H\ {\isasymin}\ {\isacharquery}cl{\isacharparenright}{\isachardoublequoteclose}\isanewline
\ \ \ \ \isacommand{using}\isamarkupfalse%
\ H{\isadigit{1}}\ H{\isadigit{2}}\ H{\isadigit{3}}\ H{\isadigit{4}}\ \isacommand{by}\isamarkupfalse%
\ {\isacharparenleft}iprover\ intro{\isacharcolon}\ conjI{\isacharparenright}\isanewline
\isacommand{qed}\isamarkupfalse%
%
\endisatagproof
{\isafoldproof}%
%
\isadelimproof
%
\endisadelimproof
%
\begin{isamarkuptext}%
Del mismo modo, podemos probar el resultado de manera automática como sigue.%
\end{isamarkuptext}\isamarkuptrue%
\isacommand{lemma}\isamarkupfalse%
\ pcp{\isacharunderscore}lim{\isacharunderscore}Hintikka{\isacharcolon}\isanewline
\ \ \isakeyword{assumes}\ c{\isacharcolon}\ {\isachardoublequoteopen}pcp\ C{\isachardoublequoteclose}\isanewline
\ \ \isakeyword{assumes}\ sc{\isacharcolon}\ {\isachardoublequoteopen}subset{\isacharunderscore}closed\ C{\isachardoublequoteclose}\isanewline
\ \ \isakeyword{assumes}\ fc{\isacharcolon}\ {\isachardoublequoteopen}finite{\isacharunderscore}character\ C{\isachardoublequoteclose}\isanewline
\ \ \isakeyword{assumes}\ el{\isacharcolon}\ {\isachardoublequoteopen}S\ {\isasymin}\ C{\isachardoublequoteclose}\isanewline
\ \ \isakeyword{shows}\ {\isachardoublequoteopen}Hintikka\ {\isacharparenleft}pcp{\isacharunderscore}lim\ C\ S{\isacharparenright}{\isachardoublequoteclose}\isanewline
%
\isadelimproof
%
\endisadelimproof
%
\isatagproof
\isacommand{proof}\isamarkupfalse%
\ {\isacharminus}\isanewline
\ \ \isacommand{let}\isamarkupfalse%
\ {\isacharquery}cl\ {\isacharequal}\ {\isachardoublequoteopen}pcp{\isacharunderscore}lim\ C\ S{\isachardoublequoteclose}\isanewline
\ \ \isacommand{have}\isamarkupfalse%
\ {\isachardoublequoteopen}{\isacharquery}cl\ {\isasymin}\ C{\isachardoublequoteclose}\ \isacommand{using}\isamarkupfalse%
\ pcp{\isacharunderscore}lim{\isacharunderscore}in{\isacharbrackleft}OF\ c\ el\ sc\ fc{\isacharbrackright}\ \isacommand{{\isachardot}}\isamarkupfalse%
\isanewline
\ \ \isacommand{from}\isamarkupfalse%
\ c{\isacharbrackleft}unfolded\ pcp{\isacharunderscore}alt{\isacharcomma}\ THEN\ bspec{\isacharcomma}\ OF\ this{\isacharbrackright}\isanewline
\ \ \isacommand{have}\isamarkupfalse%
\ d{\isacharcolon}\ {\isachardoublequoteopen}{\isasymbottom}\ {\isasymnotin}\ {\isacharquery}cl{\isachardoublequoteclose}\isanewline
\ \ \ \ {\isachardoublequoteopen}Atom\ k\ {\isasymin}\ {\isacharquery}cl\ {\isasymLongrightarrow}\ \isactrlbold {\isasymnot}\ {\isacharparenleft}Atom\ k{\isacharparenright}\ {\isasymin}\ {\isacharquery}cl\ {\isasymLongrightarrow}\ False{\isachardoublequoteclose}\isanewline
\ \ \ \ {\isachardoublequoteopen}Con\ F\ G\ H\ {\isasymLongrightarrow}\ F\ {\isasymin}\ {\isacharquery}cl\ {\isasymLongrightarrow}\ insert\ G\ {\isacharparenleft}insert\ H\ {\isacharquery}cl{\isacharparenright}\ {\isasymin}\ C{\isachardoublequoteclose}\isanewline
\ \ \ \ {\isachardoublequoteopen}Dis\ F\ G\ H\ {\isasymLongrightarrow}\ F\ {\isasymin}\ {\isacharquery}cl\ {\isasymLongrightarrow}\ insert\ G\ {\isacharquery}cl\ {\isasymin}\ C\ {\isasymor}\ insert\ H\ {\isacharquery}cl\ {\isasymin}\ C{\isachardoublequoteclose}\isanewline
\ \ \ \ \isakeyword{for}\ k\ F\ G\ H\ \isacommand{by}\isamarkupfalse%
\ force{\isacharplus}\isanewline
\ \ \isacommand{with}\isamarkupfalse%
\ d{\isacharparenleft}{\isadigit{1}}{\isacharcomma}{\isadigit{2}}{\isacharparenright}\ \isacommand{show}\isamarkupfalse%
\ {\isachardoublequoteopen}Hintikka\ {\isacharquery}cl{\isachardoublequoteclose}\ \isacommand{unfolding}\isamarkupfalse%
\ Hintikka{\isacharunderscore}alt\ \isanewline
\ \ \ \ \isacommand{using}\isamarkupfalse%
\ c\ cl{\isacharunderscore}max\ cl{\isacharunderscore}max{\isacharprime}\ d{\isacharparenleft}{\isadigit{4}}{\isacharparenright}\ sc\ \isacommand{by}\isamarkupfalse%
\ blast\isanewline
\isacommand{qed}\isamarkupfalse%
%
\endisatagproof
{\isafoldproof}%
%
\isadelimproof
%
\endisadelimproof
%
\begin{isamarkuptext}%
Finalmente, vamos a demostrar el \isa{teorema\ de\ existencia\ de\ modelo}. Para ello precisaremos de
  un resultado que indica que la satisfacibilidad de conjuntos de fórmulas se hereda por la 
  contención.

  \begin{lema}
    Todo subconjunto de un conjunto de fórmulas satisfacible es satisfacible.
  \end{lema}

  \begin{demostracion}
    Sea \isa{B} un conjunto de fórmulas satisfacible y \isa{A\ {\isasymsubseteq}\ B}. Veamos que \isa{A} es satisfacible.
    Por definición, como \isa{B} es satisfacible, existe una interpretación \isa{{\isasymA}} que es modelo de cada 
    fórmula de \isa{B}. Como \isa{A\ {\isasymsubseteq}\ B}, en particular \isa{{\isasymA}} es modelo de toda fórmula de \isa{A}. Por tanto, 
    \isa{A} es satisfacible, ya que existe una interpretación que es modelo de todas sus fórmulas.
  \end{demostracion}

  Su prueba detallada en Isabelle/HOL es la siguiente.%
\end{isamarkuptext}\isamarkuptrue%
\isacommand{lemma}\isamarkupfalse%
\ sat{\isacharunderscore}mono{\isacharcolon}\isanewline
\ \ \isakeyword{assumes}\ {\isachardoublequoteopen}A\ {\isasymsubseteq}\ B{\isachardoublequoteclose}\isanewline
\ \ \ \ \ \ \ \ \ \ {\isachardoublequoteopen}sat\ B{\isachardoublequoteclose}\isanewline
\ \ \ \ \ \ \ \ \isakeyword{shows}\ {\isachardoublequoteopen}sat\ A{\isachardoublequoteclose}\isanewline
%
\isadelimproof
\ \ %
\endisadelimproof
%
\isatagproof
\isacommand{unfolding}\isamarkupfalse%
\ sat{\isacharunderscore}def\isanewline
\isacommand{proof}\isamarkupfalse%
\ {\isacharminus}\isanewline
\ \isacommand{have}\isamarkupfalse%
\ satB{\isacharcolon}{\isachardoublequoteopen}{\isasymexists}{\isasymA}{\isachardot}\ {\isasymforall}F\ {\isasymin}\ B{\isachardot}\ {\isasymA}\ {\isasymTurnstile}\ F{\isachardoublequoteclose}\isanewline
\ \ \ \isacommand{using}\isamarkupfalse%
\ assms{\isacharparenleft}{\isadigit{2}}{\isacharparenright}\ \isacommand{by}\isamarkupfalse%
\ {\isacharparenleft}simp\ only{\isacharcolon}\ sat{\isacharunderscore}def{\isacharparenright}\isanewline
\ \isacommand{obtain}\isamarkupfalse%
\ {\isasymA}\ \isakeyword{where}\ {\isachardoublequoteopen}{\isasymforall}F\ {\isasymin}\ B{\isachardot}\ {\isasymA}\ {\isasymTurnstile}\ F{\isachardoublequoteclose}\isanewline
\ \ \ \ \isacommand{using}\isamarkupfalse%
\ satB\ \isacommand{by}\isamarkupfalse%
\ {\isacharparenleft}rule\ exE{\isacharparenright}\isanewline
\ \isacommand{have}\isamarkupfalse%
\ {\isachardoublequoteopen}{\isasymforall}F\ {\isasymin}\ A{\isachardot}\ {\isasymA}\ {\isasymTurnstile}\ F{\isachardoublequoteclose}\isanewline
\ \ \isacommand{proof}\isamarkupfalse%
\ {\isacharparenleft}rule\ ballI{\isacharparenright}\isanewline
\ \ \ \ \isacommand{fix}\isamarkupfalse%
\ F\isanewline
\ \ \ \ \isacommand{assume}\isamarkupfalse%
\ {\isachardoublequoteopen}F\ {\isasymin}\ A{\isachardoublequoteclose}\isanewline
\ \ \ \ \isacommand{have}\isamarkupfalse%
\ {\isachardoublequoteopen}F\ {\isasymin}\ A\ {\isasymlongrightarrow}\ F\ {\isasymin}\ B{\isachardoublequoteclose}\isanewline
\ \ \ \ \ \ \isacommand{using}\isamarkupfalse%
\ assms{\isacharparenleft}{\isadigit{1}}{\isacharparenright}\ \isacommand{by}\isamarkupfalse%
\ {\isacharparenleft}rule\ in{\isacharunderscore}mono{\isacharparenright}\isanewline
\ \ \ \ \isacommand{then}\isamarkupfalse%
\ \isacommand{have}\isamarkupfalse%
\ {\isachardoublequoteopen}F\ {\isasymin}\ B{\isachardoublequoteclose}\isanewline
\ \ \ \ \ \ \isacommand{using}\isamarkupfalse%
\ {\isacartoucheopen}F\ {\isasymin}\ A{\isacartoucheclose}\ \isacommand{by}\isamarkupfalse%
\ {\isacharparenleft}rule\ mp{\isacharparenright}\isanewline
\ \ \ \ \isacommand{show}\isamarkupfalse%
\ {\isachardoublequoteopen}{\isasymA}\ {\isasymTurnstile}\ F{\isachardoublequoteclose}\isanewline
\ \ \ \ \ \ \isacommand{using}\isamarkupfalse%
\ {\isacartoucheopen}{\isasymforall}F\ {\isasymin}\ B{\isachardot}\ {\isasymA}\ {\isasymTurnstile}\ F{\isacartoucheclose}\ {\isacartoucheopen}F\ {\isasymin}\ B{\isacartoucheclose}\ \isacommand{by}\isamarkupfalse%
\ {\isacharparenleft}rule\ bspec{\isacharparenright}\isanewline
\ \ \isacommand{qed}\isamarkupfalse%
\isanewline
\ \ \isacommand{thus}\isamarkupfalse%
\ {\isachardoublequoteopen}{\isasymexists}{\isasymA}{\isachardot}\ {\isasymforall}F\ {\isasymin}\ A{\isachardot}\ {\isasymA}\ {\isasymTurnstile}\ F{\isachardoublequoteclose}\isanewline
\ \ \ \ \isacommand{by}\isamarkupfalse%
\ {\isacharparenleft}simp\ only{\isacharcolon}\ exI{\isacharparenright}\isanewline
\isacommand{qed}\isamarkupfalse%
%
\endisatagproof
{\isafoldproof}%
%
\isadelimproof
%
\endisadelimproof
%
\begin{isamarkuptext}%
De este modo, procedamos finalmente con la demostración del teorema.

  \begin{teorema}[Teorema de Existencia de Modelo]
    Todo conjunto de fórmulas perteneciente a una colección que verifique la propiedad de
    consistencia proposicional es satisfacible. 
  \end{teorema}

  \begin{demostracion}
    Sea \isa{C} una colección de conjuntos de fórmulas proposicionales que verifique la propiedad de 
    consistencia proposicional y \isa{S\ {\isasymin}\ C}. Vamos a probar que \isa{S} es satisfacible.

    En primer lugar, como \isa{C} verifica la propiedad de consistencia proposicional, por el lema 
    \isa{{\isadigit{3}}{\isachardot}{\isadigit{0}}{\isachardot}{\isadigit{3}}} podemos extenderla a una colección \isa{C{\isacharprime}} que también verifique la propiedad y
    sea cerrada bajo subconjuntos. A su vez, por el lema \isa{{\isadigit{3}}{\isachardot}{\isadigit{0}}{\isachardot}{\isadigit{5}}}, como la extensión 
    \isa{C{\isacharprime}} es una colección con la propiedad de consistencia proposicional y cerrada bajo 
    subconjuntos, podemos extenderla a otra colección \isa{C{\isacharprime}{\isacharprime}} que también verifica la propiedad de
    consistencia proposicional y sea de carácter finito. De este modo, por la transitividad de la 
    contención, es claro que \isa{C{\isacharprime}{\isacharprime}} es una extensión de \isa{C}, luego \isa{S\ {\isasymin}\ C{\isacharprime}{\isacharprime}} por hipótesis. 
    Por otro lado, por el lema \isa{{\isadigit{3}}{\isachardot}{\isadigit{0}}{\isachardot}{\isadigit{4}}}, como \isa{C{\isacharprime}{\isacharprime}} es de carácter finito, se tiene que es 
    cerrada bajo subconjuntos. 

    En suma, \isa{C{\isacharprime}{\isacharprime}} es una extensión de \isa{C} que verifica la propiedad de consistencia proposicional, 
    es cerrada bajo subconjuntos y es de carácter finito. Luego, por el lema \isa{{\isadigit{4}}{\isachardot}{\isadigit{2}}{\isachardot}{\isadigit{4}}}, el límite de 
    la sucesión \isa{{\isacharbraceleft}S\isactrlsub n{\isacharbraceright}} de conjuntos de \isa{C{\isacharprime}{\isacharprime}} a partir de \isa{S} según la definición \isa{{\isadigit{4}}{\isachardot}{\isadigit{1}}{\isachardot}{\isadigit{1}}} es un 
    conjunto de Hintikka. Por tanto, por el \isa{teorema\ de\ Hintikka}, se trata de un conjunto 
    satisfacible. 

    Finalmente, puesto que para todo \isa{n\ {\isasymin}\ {\isasymnat}} se tiene que \isa{S\isactrlsub n} está contenido en el límite, en 
    particular el conjunto \isa{S\isactrlsub {\isadigit{0}}} está contenido en él. Por definición de la sucesión, dicho conjunto 
    coincide con \isa{S}. Por tanto, como \isa{S} está contenido en el límite que es un conjunto 
    satisfacible, queda demostrada la satisfacibilidad de \isa{S}.
  \end{demostracion}

  \comentario{Tal vez sería buena idea hacer un grafo similar al de ex3.}

  Mostremos su formalización y demostración detallada en Isabelle.%
\end{isamarkuptext}\isamarkuptrue%
\isacommand{theorem}\isamarkupfalse%
\isanewline
\ \ \isakeyword{fixes}\ S\ {\isacharcolon}{\isacharcolon}\ {\isachardoublequoteopen}{\isacharprime}a\ {\isacharcolon}{\isacharcolon}\ countable\ formula\ set{\isachardoublequoteclose}\isanewline
\ \ \isakeyword{assumes}\ {\isachardoublequoteopen}pcp\ C{\isachardoublequoteclose}\isanewline
\ \ \isakeyword{assumes}\ {\isachardoublequoteopen}S\ {\isasymin}\ C{\isachardoublequoteclose}\isanewline
\ \ \isakeyword{shows}\ {\isachardoublequoteopen}sat\ S{\isachardoublequoteclose}\isanewline
%
\isadelimproof
%
\endisadelimproof
%
\isatagproof
\isacommand{proof}\isamarkupfalse%
\ {\isacharminus}\isanewline
\ \ \isacommand{have}\isamarkupfalse%
\ {\isachardoublequoteopen}pcp\ C\ {\isasymLongrightarrow}\ {\isasymexists}C{\isacharprime}{\isachardot}\ C\ {\isasymsubseteq}\ C{\isacharprime}\ {\isasymand}\ pcp\ C{\isacharprime}\ {\isasymand}\ subset{\isacharunderscore}closed\ C{\isacharprime}{\isachardoublequoteclose}\isanewline
\ \ \ \ \isacommand{by}\isamarkupfalse%
\ {\isacharparenleft}rule\ ex{\isadigit{1}}{\isacharparenright}\isanewline
\ \ \isacommand{then}\isamarkupfalse%
\ \isacommand{have}\isamarkupfalse%
\ E{\isadigit{1}}{\isacharcolon}{\isachardoublequoteopen}{\isasymexists}C{\isacharprime}{\isachardot}\ C\ {\isasymsubseteq}\ C{\isacharprime}\ {\isasymand}\ pcp\ C{\isacharprime}\ {\isasymand}\ subset{\isacharunderscore}closed\ C{\isacharprime}{\isachardoublequoteclose}\isanewline
\ \ \ \ \isacommand{by}\isamarkupfalse%
\ {\isacharparenleft}simp\ only{\isacharcolon}\ assms{\isacharparenleft}{\isadigit{1}}{\isacharparenright}{\isacharparenright}\isanewline
\ \ \isacommand{obtain}\isamarkupfalse%
\ Ce{\isacharprime}\ \isakeyword{where}\ H{\isadigit{1}}{\isacharcolon}{\isachardoublequoteopen}C\ {\isasymsubseteq}\ Ce{\isacharprime}\ {\isasymand}\ pcp\ Ce{\isacharprime}\ {\isasymand}\ subset{\isacharunderscore}closed\ Ce{\isacharprime}{\isachardoublequoteclose}\isanewline
\ \ \ \ \isacommand{using}\isamarkupfalse%
\ E{\isadigit{1}}\ \isacommand{by}\isamarkupfalse%
\ {\isacharparenleft}rule\ exE{\isacharparenright}\isanewline
\ \ \isacommand{have}\isamarkupfalse%
\ {\isachardoublequoteopen}C\ {\isasymsubseteq}\ Ce{\isacharprime}{\isachardoublequoteclose}\isanewline
\ \ \ \ \isacommand{using}\isamarkupfalse%
\ H{\isadigit{1}}\ \isacommand{by}\isamarkupfalse%
\ {\isacharparenleft}rule\ conjunct{\isadigit{1}}{\isacharparenright}\isanewline
\ \ \isacommand{have}\isamarkupfalse%
\ {\isachardoublequoteopen}pcp\ Ce{\isacharprime}{\isachardoublequoteclose}\isanewline
\ \ \ \ \isacommand{using}\isamarkupfalse%
\ H{\isadigit{1}}\ \isacommand{by}\isamarkupfalse%
\ {\isacharparenleft}iprover\ elim{\isacharcolon}\ conjunct{\isadigit{2}}\ conjunct{\isadigit{1}}{\isacharparenright}\isanewline
\ \ \isacommand{have}\isamarkupfalse%
\ {\isachardoublequoteopen}subset{\isacharunderscore}closed\ Ce{\isacharprime}{\isachardoublequoteclose}\isanewline
\ \ \ \ \isacommand{using}\isamarkupfalse%
\ H{\isadigit{1}}\ \isacommand{by}\isamarkupfalse%
\ {\isacharparenleft}iprover\ elim{\isacharcolon}\ conjunct{\isadigit{2}}\ conjunct{\isadigit{1}}{\isacharparenright}\isanewline
\ \ \isacommand{have}\isamarkupfalse%
\ E{\isadigit{2}}{\isacharcolon}{\isachardoublequoteopen}{\isasymexists}Ce{\isachardot}\ Ce{\isacharprime}\ {\isasymsubseteq}\ Ce\ {\isasymand}\ pcp\ Ce\ {\isasymand}\ finite{\isacharunderscore}character\ Ce{\isachardoublequoteclose}\isanewline
\ \ \ \ \isacommand{using}\isamarkupfalse%
\ {\isacartoucheopen}pcp\ Ce{\isacharprime}{\isacartoucheclose}\ {\isacartoucheopen}subset{\isacharunderscore}closed\ Ce{\isacharprime}{\isacartoucheclose}\ \isacommand{by}\isamarkupfalse%
\ {\isacharparenleft}rule\ ex{\isadigit{3}}{\isacharparenright}\isanewline
\ \ \isacommand{obtain}\isamarkupfalse%
\ Ce\ \isakeyword{where}\ H{\isadigit{2}}{\isacharcolon}{\isachardoublequoteopen}Ce{\isacharprime}\ {\isasymsubseteq}\ Ce\ {\isasymand}\ pcp\ Ce\ {\isasymand}\ finite{\isacharunderscore}character\ Ce{\isachardoublequoteclose}\isanewline
\ \ \ \ \isacommand{using}\isamarkupfalse%
\ E{\isadigit{2}}\ \isacommand{by}\isamarkupfalse%
\ {\isacharparenleft}rule\ exE{\isacharparenright}\isanewline
\ \ \isacommand{have}\isamarkupfalse%
\ {\isachardoublequoteopen}Ce{\isacharprime}\ {\isasymsubseteq}\ Ce{\isachardoublequoteclose}\isanewline
\ \ \ \ \isacommand{using}\isamarkupfalse%
\ H{\isadigit{2}}\ \isacommand{by}\isamarkupfalse%
\ {\isacharparenleft}rule\ conjunct{\isadigit{1}}{\isacharparenright}\isanewline
\ \ \isacommand{then}\isamarkupfalse%
\ \isacommand{have}\isamarkupfalse%
\ Subset{\isacharcolon}{\isachardoublequoteopen}C\ {\isasymsubseteq}\ Ce{\isachardoublequoteclose}\isanewline
\ \ \ \ \isacommand{using}\isamarkupfalse%
\ {\isacartoucheopen}C\ {\isasymsubseteq}\ Ce{\isacharprime}{\isacartoucheclose}\ \isacommand{by}\isamarkupfalse%
\ {\isacharparenleft}simp\ only{\isacharcolon}\ subset{\isacharunderscore}trans{\isacharparenright}\isanewline
\ \ \isacommand{have}\isamarkupfalse%
\ Pcp{\isacharcolon}{\isachardoublequoteopen}pcp\ Ce{\isachardoublequoteclose}\isanewline
\ \ \ \ \isacommand{using}\isamarkupfalse%
\ H{\isadigit{2}}\ \isacommand{by}\isamarkupfalse%
\ {\isacharparenleft}iprover\ elim{\isacharcolon}\ conjunct{\isadigit{2}}\ conjunct{\isadigit{1}}{\isacharparenright}\isanewline
\ \ \isacommand{have}\isamarkupfalse%
\ FC{\isacharcolon}{\isachardoublequoteopen}finite{\isacharunderscore}character\ Ce{\isachardoublequoteclose}\isanewline
\ \ \ \ \isacommand{using}\isamarkupfalse%
\ H{\isadigit{2}}\ \isacommand{by}\isamarkupfalse%
\ {\isacharparenleft}iprover\ elim{\isacharcolon}\ conjunct{\isadigit{2}}\ conjunct{\isadigit{1}}{\isacharparenright}\isanewline
\ \ \isacommand{then}\isamarkupfalse%
\ \isacommand{have}\isamarkupfalse%
\ SC{\isacharcolon}{\isachardoublequoteopen}subset{\isacharunderscore}closed\ Ce{\isachardoublequoteclose}\isanewline
\ \ \ \ \isacommand{by}\isamarkupfalse%
\ {\isacharparenleft}rule\ ex{\isadigit{2}}{\isacharparenright}\isanewline
\ \ \isacommand{have}\isamarkupfalse%
\ {\isachardoublequoteopen}S\ {\isasymin}\ C\ {\isasymlongrightarrow}\ S\ {\isasymin}\ Ce{\isachardoublequoteclose}\isanewline
\ \ \ \ \isacommand{using}\isamarkupfalse%
\ {\isacartoucheopen}C\ {\isasymsubseteq}\ Ce{\isacartoucheclose}\ \isacommand{by}\isamarkupfalse%
\ {\isacharparenleft}rule\ in{\isacharunderscore}mono{\isacharparenright}\isanewline
\ \ \isacommand{then}\isamarkupfalse%
\ \isacommand{have}\isamarkupfalse%
\ {\isachardoublequoteopen}S\ {\isasymin}\ Ce{\isachardoublequoteclose}\ \isanewline
\ \ \ \ \isacommand{using}\isamarkupfalse%
\ assms{\isacharparenleft}{\isadigit{2}}{\isacharparenright}\ \isacommand{by}\isamarkupfalse%
\ {\isacharparenleft}rule\ mp{\isacharparenright}\isanewline
\ \ \isacommand{have}\isamarkupfalse%
\ {\isachardoublequoteopen}Hintikka\ {\isacharparenleft}pcp{\isacharunderscore}lim\ Ce\ S{\isacharparenright}{\isachardoublequoteclose}\isanewline
\ \ \ \ \isacommand{using}\isamarkupfalse%
\ Pcp\ SC\ FC\ {\isacartoucheopen}S\ {\isasymin}\ Ce{\isacartoucheclose}\ \isacommand{by}\isamarkupfalse%
\ {\isacharparenleft}rule\ pcp{\isacharunderscore}lim{\isacharunderscore}Hintikka{\isacharparenright}\isanewline
\ \ \isacommand{then}\isamarkupfalse%
\ \isacommand{have}\isamarkupfalse%
\ {\isachardoublequoteopen}sat\ {\isacharparenleft}pcp{\isacharunderscore}lim\ Ce\ S{\isacharparenright}{\isachardoublequoteclose}\isanewline
\ \ \ \ \isacommand{by}\isamarkupfalse%
\ {\isacharparenleft}rule\ Hintikkaslemma{\isacharparenright}\isanewline
\ \ \isacommand{have}\isamarkupfalse%
\ {\isachardoublequoteopen}pcp{\isacharunderscore}seq\ Ce\ S\ {\isadigit{0}}\ {\isacharequal}\ S{\isachardoublequoteclose}\isanewline
\ \ \ \ \isacommand{by}\isamarkupfalse%
\ {\isacharparenleft}simp\ only{\isacharcolon}\ pcp{\isacharunderscore}seq{\isachardot}simps{\isacharparenleft}{\isadigit{1}}{\isacharparenright}{\isacharparenright}\isanewline
\ \ \isacommand{have}\isamarkupfalse%
\ {\isachardoublequoteopen}pcp{\isacharunderscore}seq\ Ce\ S\ {\isadigit{0}}\ {\isasymsubseteq}\ pcp{\isacharunderscore}lim\ Ce\ S{\isachardoublequoteclose}\isanewline
\ \ \ \ \isacommand{by}\isamarkupfalse%
\ {\isacharparenleft}rule\ pcp{\isacharunderscore}seq{\isacharunderscore}sub{\isacharparenright}\isanewline
\ \ \isacommand{then}\isamarkupfalse%
\ \isacommand{have}\isamarkupfalse%
\ {\isachardoublequoteopen}S\ {\isasymsubseteq}\ pcp{\isacharunderscore}lim\ Ce\ S{\isachardoublequoteclose}\isanewline
\ \ \ \ \isacommand{by}\isamarkupfalse%
\ {\isacharparenleft}simp\ only{\isacharcolon}\ {\isacartoucheopen}pcp{\isacharunderscore}seq\ Ce\ S\ {\isadigit{0}}\ {\isacharequal}\ S{\isacartoucheclose}{\isacharparenright}\isanewline
\ \ \isacommand{thus}\isamarkupfalse%
\ {\isachardoublequoteopen}sat\ S{\isachardoublequoteclose}\isanewline
\ \ \ \ \isacommand{using}\isamarkupfalse%
\ {\isacartoucheopen}sat\ {\isacharparenleft}pcp{\isacharunderscore}lim\ Ce\ S{\isacharparenright}{\isacartoucheclose}\ \isacommand{by}\isamarkupfalse%
\ {\isacharparenleft}rule\ sat{\isacharunderscore}mono{\isacharparenright}\isanewline
\isacommand{qed}\isamarkupfalse%
%
\endisatagproof
{\isafoldproof}%
%
\isadelimproof
%
\endisadelimproof
%
\begin{isamarkuptext}%
Finalmente, demostremos el teorema de manera automática.%
\end{isamarkuptext}\isamarkuptrue%
\isacommand{theorem}\isamarkupfalse%
\ pcp{\isacharunderscore}sat{\isacharcolon}\isanewline
\ \ \isakeyword{fixes}\ S\ {\isacharcolon}{\isacharcolon}\ {\isachardoublequoteopen}{\isacharprime}a\ {\isacharcolon}{\isacharcolon}\ countable\ formula\ set{\isachardoublequoteclose}\isanewline
\ \ \isakeyword{assumes}\ c{\isacharcolon}\ {\isachardoublequoteopen}pcp\ C{\isachardoublequoteclose}\isanewline
\ \ \isakeyword{assumes}\ el{\isacharcolon}\ {\isachardoublequoteopen}S\ {\isasymin}\ C{\isachardoublequoteclose}\isanewline
\ \ \isakeyword{shows}\ {\isachardoublequoteopen}sat\ S{\isachardoublequoteclose}\isanewline
%
\isadelimproof
%
\endisadelimproof
%
\isatagproof
\isacommand{proof}\isamarkupfalse%
\ {\isacharminus}\isanewline
\ \ \isacommand{from}\isamarkupfalse%
\ c\ \isacommand{obtain}\isamarkupfalse%
\ Ce\ \isakeyword{where}\ \isanewline
\ \ \ \ \ \ {\isachardoublequoteopen}C\ {\isasymsubseteq}\ Ce{\isachardoublequoteclose}\ {\isachardoublequoteopen}pcp\ Ce{\isachardoublequoteclose}\ {\isachardoublequoteopen}subset{\isacharunderscore}closed\ Ce{\isachardoublequoteclose}\ {\isachardoublequoteopen}finite{\isacharunderscore}character\ Ce{\isachardoublequoteclose}\ \isanewline
\ \ \ \ \ \ \isacommand{using}\isamarkupfalse%
\ ex{\isadigit{1}}{\isacharbrackleft}\isakeyword{where}\ {\isacharprime}a{\isacharequal}{\isacharprime}a{\isacharbrackright}\ ex{\isadigit{2}}{\isacharbrackleft}\isakeyword{where}\ {\isacharprime}a{\isacharequal}{\isacharprime}a{\isacharbrackright}\ ex{\isadigit{3}}{\isacharbrackleft}\isakeyword{where}\ {\isacharprime}a{\isacharequal}{\isacharprime}a{\isacharbrackright}\isanewline
\ \ \ \ \isacommand{by}\isamarkupfalse%
\ {\isacharparenleft}meson\ dual{\isacharunderscore}order{\isachardot}trans\ ex{\isadigit{2}}{\isacharparenright}\isanewline
\ \ \isacommand{have}\isamarkupfalse%
\ {\isachardoublequoteopen}S\ {\isasymin}\ Ce{\isachardoublequoteclose}\ \isacommand{using}\isamarkupfalse%
\ {\isacartoucheopen}C\ {\isasymsubseteq}\ Ce{\isacartoucheclose}\ el\ \isacommand{{\isachardot}{\isachardot}}\isamarkupfalse%
\isanewline
\ \ \isacommand{with}\isamarkupfalse%
\ pcp{\isacharunderscore}lim{\isacharunderscore}Hintikka\ {\isacartoucheopen}pcp\ Ce{\isacartoucheclose}\ {\isacartoucheopen}subset{\isacharunderscore}closed\ Ce{\isacartoucheclose}\ {\isacartoucheopen}finite{\isacharunderscore}character\ Ce{\isacartoucheclose}\isanewline
\ \ \isacommand{have}\isamarkupfalse%
\ \ {\isachardoublequoteopen}Hintikka\ {\isacharparenleft}pcp{\isacharunderscore}lim\ Ce\ S{\isacharparenright}{\isachardoublequoteclose}\ \isacommand{{\isachardot}}\isamarkupfalse%
\isanewline
\ \ \isacommand{with}\isamarkupfalse%
\ Hintikkaslemma\ \isacommand{have}\isamarkupfalse%
\ {\isachardoublequoteopen}sat\ {\isacharparenleft}pcp{\isacharunderscore}lim\ Ce\ S{\isacharparenright}{\isachardoublequoteclose}\ \isacommand{{\isachardot}}\isamarkupfalse%
\isanewline
\ \ \isacommand{moreover}\isamarkupfalse%
\ \isacommand{have}\isamarkupfalse%
\ {\isachardoublequoteopen}S\ {\isasymsubseteq}\ pcp{\isacharunderscore}lim\ Ce\ S{\isachardoublequoteclose}\ \isanewline
\ \ \ \ \isacommand{using}\isamarkupfalse%
\ pcp{\isacharunderscore}seq{\isachardot}simps{\isacharparenleft}{\isadigit{1}}{\isacharparenright}\ pcp{\isacharunderscore}seq{\isacharunderscore}sub\ \isacommand{by}\isamarkupfalse%
\ fast\isanewline
\ \ \isacommand{ultimately}\isamarkupfalse%
\ \isacommand{show}\isamarkupfalse%
\ {\isacharquery}thesis\ \isacommand{unfolding}\isamarkupfalse%
\ sat{\isacharunderscore}def\ \isacommand{by}\isamarkupfalse%
\ fast\isanewline
\isacommand{qed}\isamarkupfalse%
%
\endisatagproof
{\isafoldproof}%
%
\isadelimproof
%
\endisadelimproof
%
\isadelimdocument
%
\endisadelimdocument
%
\isatagdocument
%
\isamarkupsection{Teorema de Compacidad%
}
\isamarkuptrue%
%
\endisatagdocument
{\isafolddocument}%
%
\isadelimdocument
%
\endisadelimdocument
%
\begin{isamarkuptext}%
En esta sección vamos demostrar el \isa{Teorema\ de\ Compacidad} para la lógica proposicional
  como consecuencia del \isa{Teorema\ de\ Existencia\ de\ Modelo}.

  \begin{teorema}[Teorema de Compacidad]
    Todo conjunto de fórmulas finitamente satisfacible es satisfacible.
  \end{teorema}

  Para su demostración consideraremos la colección formada por los conjuntos de fórmulas finitamente 
  satisfacibles. Probaremos que dicha colección verifica la propiedad de consistencia proposicional
  y, por el \isa{Teorema\ de\ Existencia\ de\ Modelo}, todo conjunto perteneciente a ella será
  satisfacible.

  Mostremos, previamente, dos resultados sobre subconjuntos finitos que emplearemos en la 
  demostración del teorema.

  \begin{lema}
    Sea un conjunto de la forma \isa{{\isacharbraceleft}a{\isacharbraceright}\ {\isasymunion}\ B} y \isa{S} un subconjunto finito suyo. Entonces,
    existe un subconjunto finito \isa{S{\isacharprime}} de \isa{B} tal que o bien \isa{S\ {\isacharequal}\ {\isacharbraceleft}a{\isacharbraceright}\ {\isasymunion}\ S{\isacharprime}} o bien \isa{S\ {\isacharequal}\ S{\isacharprime}}.
  \end{lema}

  \begin{demostracion}
    La prueba del resultado se realiza por inducción en la estructura recursiva de los conjuntos 
    finitos.

    En primer lugar, probemos el caso base correspondiente al conjunto vacío. Si consideramos 
    \isa{S\ {\isacharequal}\ {\isacharbraceleft}{\isacharbraceright}}, tomando también \isa{S{\isacharprime}\ {\isacharequal}\ {\isacharbraceleft}{\isacharbraceright}} es claro que verifica que es subconjunto finito de \isa{B}
    y o bien \isa{S\ {\isacharequal}\ {\isacharbraceleft}a{\isacharbraceright}\ {\isasymunion}\ S{\isacharprime}} o bien \isa{S\ {\isacharequal}\ S{\isacharprime}}.

    Veamos el caso de inducción. Sea \isa{S} un conjunto finito verificando la hipótesis de inducción:
    
    \isa{{\isacharparenleft}HI{\isacharparenright}{\isacharcolon}\ Si\ S\ {\isasymsubseteq}\ {\isacharbraceleft}a{\isacharbraceright}\ {\isasymunion}\ B{\isacharcomma}\ entonces\ existe\ un\ subconjunto\ finito\ S{\isacharprime}\ de\ B\ tal\ que\ o\ bien}

    \hspace{1cm}\isa{S\ {\isacharequal}\ {\isacharbraceleft}a{\isacharbraceright}\ {\isasymunion}\ S{\isacharprime}\ o\ bien\ S\ {\isacharequal}\ S{\isacharprime}{\isachardot}}

    Sea un elemento cualquiera \isa{x\ {\isasymnotin}\ S}. Vamos a probar que se verifica el resultado para el conjunto
    \isa{{\isacharbraceleft}x{\isacharbraceright}\ {\isasymunion}\ S}. Es decir, si \isa{{\isacharbraceleft}x{\isacharbraceright}\ {\isasymunion}\ S\ {\isasymsubseteq}\ {\isacharbraceleft}a{\isacharbraceright}\ {\isasymunion}\ B}, vamos a encontrar un subconjunto finito \isa{S{\isacharprime}{\isacharprime}} de 
    \isa{B} tal que o bien \isa{{\isacharbraceleft}x{\isacharbraceright}\ {\isasymunion}\ S\ {\isacharequal}\ {\isacharbraceleft}a{\isacharbraceright}\ {\isasymunion}\ S{\isacharprime}{\isacharprime}} o bien \isa{{\isacharbraceleft}x{\isacharbraceright}\ {\isasymunion}\ S\ {\isacharequal}\ S{\isacharprime}{\isacharprime}}.

    Supongamos, pues, que \isa{{\isacharbraceleft}x{\isacharbraceright}\ {\isasymunion}\ S\ {\isasymsubseteq}\ {\isacharbraceleft}a{\isacharbraceright}\ {\isasymunion}\ B}. En este caso, es claro que se verifica que
    \isa{x\ {\isasymin}\ {\isacharbraceleft}a{\isacharbraceright}\ {\isasymunion}\ B} y \isa{S\ {\isasymsubseteq}\ {\isacharbraceleft}a{\isacharbraceright}\ {\isasymunion}\ B}. Por lo último, aplicando hipótesis de inducción, podemos hallar 
    un subconjunto finito \isa{S{\isacharprime}} de \isa{B} tal que o bien \isa{S\ {\isacharequal}\ {\isacharbraceleft}a{\isacharbraceright}\ {\isasymunion}\ S{\isacharprime}} o bien \isa{S\ {\isacharequal}\ S{\isacharprime}}. Por otro lado,
    como \isa{x\ {\isasymin}\ {\isacharbraceleft}a{\isacharbraceright}\ {\isasymunion}\ B}, deducimos que o bien \isa{x\ {\isacharequal}\ a} o bien \isa{x\ {\isasymin}\ B}. En efecto, probemos que se 
    verifica el resultado para ambos casos de la última disyunción.

    En primer lugar, supongamos que \isa{x\ {\isacharequal}\ a}. En este caso, veremos que el conjunto \isa{S{\isacharprime}{\isacharprime}} que 
    verifica el resultado es el propio \isa{S{\isacharprime}}. Se observa fácilmente ya que, si \isa{S\ {\isacharequal}\ {\isacharbraceleft}a{\isacharbraceright}\ {\isasymunion}\ S{\isacharprime}}, como 
    \isa{x\ {\isacharequal}\ a} se obtiene que \isa{{\isacharbraceleft}x{\isacharbraceright}\ {\isasymunion}\ S\ {\isacharequal}\ {\isacharbraceleft}a{\isacharbraceright}\ {\isasymunion}\ S{\isacharprime}}, de modo que \isa{S{\isacharprime}{\isacharprime}\ {\isacharequal}\ S{\isacharprime}} es un subconjunto finito de 
    \isa{B} tal que o bien \isa{{\isacharbraceleft}x{\isacharbraceright}\ {\isasymunion}\ S\ {\isacharequal}\ {\isacharbraceleft}a{\isacharbraceright}\ {\isasymunion}\ S{\isacharprime}{\isacharprime}} o bien \isa{{\isacharbraceleft}x{\isacharbraceright}\ {\isasymunion}\ S\ {\isacharequal}\ S{\isacharprime}{\isacharprime}}. Por otro lado, suponiendo que 
    \isa{S\ {\isacharequal}\ S{\isacharprime}}, se deduce análogamente que \isa{{\isacharbraceleft}x{\isacharbraceright}\ {\isasymunion}\ S\ {\isacharequal}\ {\isacharbraceleft}a{\isacharbraceright}\ {\isasymunion}\ S{\isacharprime}} pues se tiene que \isa{x\ {\isacharequal}\ a}, llegando
    a la misma conclusión.

    Por otra parte, supongamos que \isa{x\ {\isasymin}\ B}. En este caso, el conjunto \isa{S{\isacharprime}{\isacharprime}} que verifica el
    resultado es el conjunto \isa{{\isacharbraceleft}x{\isacharbraceright}\ {\isasymunion}\ S{\isacharprime}}. Observemos que se trata de un subconjunto finito de \isa{B} ya 
    que \isa{S{\isacharprime}} es un subconjunto finito de \isa{B} y \isa{x\ {\isasymin}\ B}. Además, en efecto si \isa{S\ {\isacharequal}\ {\isacharbraceleft}a{\isacharbraceright}\ {\isasymunion}\ S{\isacharprime}}, se 
    deduce que \isa{{\isacharbraceleft}x{\isacharbraceright}\ {\isasymunion}\ S\ {\isacharequal}\ {\isacharbraceleft}x{\isacharcomma}a{\isacharbraceright}\ {\isasymunion}\ S{\isacharprime}\ {\isacharequal}\ {\isacharbraceleft}a{\isacharbraceright}\ {\isasymunion}\ S{\isacharprime}{\isacharprime}}, luego cumple que o bien\\ \isa{{\isacharbraceleft}x{\isacharbraceright}\ {\isasymunion}\ S\ {\isacharequal}\ {\isacharbraceleft}a{\isacharbraceright}\ {\isasymunion}\ S{\isacharprime}{\isacharprime}} o 
    bien \isa{{\isacharbraceleft}x{\isacharbraceright}\ {\isasymunion}\ S\ {\isacharequal}\ S{\isacharprime}{\isacharprime}}. Por otro lado, en el caso en que \isa{S\ {\isacharequal}\ S{\isacharprime}}, es claro que \isa{{\isacharbraceleft}x{\isacharbraceright}\ {\isasymunion}\ S\ {\isacharequal}\ S{\isacharprime}{\isacharprime}} 
    por la elección de \isa{S{\isacharprime}{\isacharprime}}, llegando la misma conclusión.
  \end{demostracion}

  Procedamos con la prueba detallada y formalización en Isabelle. Para ello, hemos utilizado el
  siguiente lema auxiliar.%
\end{isamarkuptext}\isamarkuptrue%
\isacommand{lemma}\isamarkupfalse%
\ subexI\ {\isacharbrackleft}intro{\isacharbrackright}{\isacharcolon}\ {\isachardoublequoteopen}P\ A\ {\isasymLongrightarrow}\ A\ {\isasymsubseteq}\ B\ {\isasymLongrightarrow}\ {\isasymexists}A{\isasymsubseteq}B{\isachardot}\ P\ A{\isachardoublequoteclose}\isanewline
%
\isadelimproof
\ \ %
\endisadelimproof
%
\isatagproof
\isacommand{by}\isamarkupfalse%
\ blast%
\endisatagproof
{\isafoldproof}%
%
\isadelimproof
%
\endisadelimproof
%
\begin{isamarkuptext}%
De este modo, probemos detalladamente el resultado.%
\end{isamarkuptext}\isamarkuptrue%
\isacommand{lemma}\isamarkupfalse%
\ finite{\isacharunderscore}subset{\isacharunderscore}insert{\isadigit{1}}{\isacharcolon}\isanewline
\ \ {\isachardoublequoteopen}{\isasymlbrakk}finite\ S{\isacharsemicolon}\ S\ {\isasymsubseteq}\ {\isacharbraceleft}a{\isacharbraceright}\ {\isasymunion}\ B\ {\isasymrbrakk}\ {\isasymLongrightarrow}\isanewline
\ \ \ \ \ {\isasymexists}S{\isacharprime}\ {\isasymsubseteq}\ B{\isachardot}\ finite\ S{\isacharprime}\ {\isasymand}\ {\isacharparenleft}S\ {\isacharequal}\ {\isacharbraceleft}a{\isacharbraceright}\ {\isasymunion}\ S{\isacharprime}\ {\isasymor}\ S\ {\isacharequal}\ S{\isacharprime}{\isacharparenright}{\isachardoublequoteclose}\isanewline
%
\isadelimproof
%
\endisadelimproof
%
\isatagproof
\isacommand{proof}\isamarkupfalse%
\ {\isacharparenleft}induct\ rule{\isacharcolon}\ finite{\isacharunderscore}induct{\isacharparenright}\isanewline
\ \ \isacommand{case}\isamarkupfalse%
\ empty\isanewline
\ \ \isacommand{have}\isamarkupfalse%
\ {\isachardoublequoteopen}{\isacharbraceleft}{\isacharbraceright}\ {\isasymsubseteq}\ B{\isachardoublequoteclose}\isanewline
\ \ \ \ \isacommand{by}\isamarkupfalse%
\ {\isacharparenleft}rule\ empty{\isacharunderscore}subsetI{\isacharparenright}\isanewline
\ \ \isacommand{have}\isamarkupfalse%
\ {\isadigit{1}}{\isacharcolon}{\isachardoublequoteopen}finite\ {\isacharbraceleft}{\isacharbraceright}{\isachardoublequoteclose}\isanewline
\ \ \ \ \isacommand{by}\isamarkupfalse%
\ {\isacharparenleft}simp\ only{\isacharcolon}\ finite{\isachardot}emptyI{\isacharparenright}\isanewline
\ \ \isacommand{have}\isamarkupfalse%
\ {\isachardoublequoteopen}{\isacharbraceleft}{\isacharbraceright}\ {\isacharequal}\ {\isacharbraceleft}{\isacharbraceright}{\isachardoublequoteclose}\isanewline
\ \ \ \ \isacommand{by}\isamarkupfalse%
\ {\isacharparenleft}simp\ only{\isacharcolon}\ simp{\isacharunderscore}thms{\isacharparenleft}{\isadigit{6}}{\isacharparenright}{\isacharparenright}\isanewline
\ \ \isacommand{then}\isamarkupfalse%
\ \isacommand{have}\isamarkupfalse%
\ {\isadigit{2}}{\isacharcolon}{\isachardoublequoteopen}{\isacharbraceleft}{\isacharbraceright}\ {\isacharequal}\ {\isacharbraceleft}a{\isacharbraceright}\ {\isasymunion}\ {\isacharbraceleft}{\isacharbraceright}\ {\isasymor}\ {\isacharbraceleft}{\isacharbraceright}\ {\isacharequal}\ {\isacharbraceleft}{\isacharbraceright}{\isachardoublequoteclose}\isanewline
\ \ \ \ \isacommand{by}\isamarkupfalse%
\ {\isacharparenleft}rule\ disjI{\isadigit{2}}{\isacharparenright}\isanewline
\ \ \isacommand{have}\isamarkupfalse%
\ {\isachardoublequoteopen}finite\ {\isacharbraceleft}{\isacharbraceright}\ {\isasymand}\ {\isacharparenleft}{\isacharbraceleft}{\isacharbraceright}\ {\isacharequal}\ {\isacharbraceleft}a{\isacharbraceright}\ {\isasymunion}\ {\isacharbraceleft}{\isacharbraceright}\ {\isasymor}\ {\isacharbraceleft}{\isacharbraceright}\ {\isacharequal}\ {\isacharbraceleft}{\isacharbraceright}{\isacharparenright}{\isachardoublequoteclose}\isanewline
\ \ \ \ \isacommand{using}\isamarkupfalse%
\ {\isadigit{1}}\ {\isadigit{2}}\ \isacommand{by}\isamarkupfalse%
\ {\isacharparenleft}rule\ conjI{\isacharparenright}\isanewline
\ \ \isacommand{thus}\isamarkupfalse%
\ {\isachardoublequoteopen}{\isasymexists}S{\isacharprime}\ {\isasymsubseteq}\ B{\isachardot}\ finite\ S{\isacharprime}\ {\isasymand}\ {\isacharparenleft}{\isacharbraceleft}{\isacharbraceright}\ {\isacharequal}\ {\isacharbraceleft}a{\isacharbraceright}\ {\isasymunion}\ S{\isacharprime}\ {\isasymor}\ {\isacharbraceleft}{\isacharbraceright}\ {\isacharequal}\ S{\isacharprime}{\isacharparenright}{\isachardoublequoteclose}\isanewline
\ \ \ \ \isacommand{using}\isamarkupfalse%
\ {\isacartoucheopen}{\isacharbraceleft}{\isacharbraceright}\ {\isasymsubseteq}\ B{\isacartoucheclose}\ \isacommand{by}\isamarkupfalse%
\ {\isacharparenleft}rule\ subexI{\isacharparenright}\isanewline
\isacommand{next}\isamarkupfalse%
\isanewline
\ \ \isacommand{case}\isamarkupfalse%
\ {\isacharparenleft}insert\ x\ S{\isacharparenright}\isanewline
\ \ \isacommand{assume}\isamarkupfalse%
\ {\isachardoublequoteopen}finite\ S{\isachardoublequoteclose}\isanewline
\ \ \isacommand{assume}\isamarkupfalse%
\ {\isachardoublequoteopen}x\ {\isasymnotin}\ S{\isachardoublequoteclose}\isanewline
\ \ \isacommand{assume}\isamarkupfalse%
\ HI{\isacharcolon}{\isachardoublequoteopen}S\ {\isasymsubseteq}\ {\isacharbraceleft}a{\isacharbraceright}\ {\isasymunion}\ B\ {\isasymLongrightarrow}\ {\isasymexists}S{\isacharprime}{\isasymsubseteq}B{\isachardot}\ finite\ S{\isacharprime}\ {\isasymand}\ {\isacharparenleft}S\ {\isacharequal}\ {\isacharbraceleft}a{\isacharbraceright}\ {\isasymunion}\ S{\isacharprime}\ {\isasymor}\ S\ {\isacharequal}\ S{\isacharprime}{\isacharparenright}{\isachardoublequoteclose}\isanewline
\ \ \isacommand{show}\isamarkupfalse%
\ {\isachardoublequoteopen}insert\ x\ S\ {\isasymsubseteq}\ {\isacharbraceleft}a{\isacharbraceright}\ {\isasymunion}\ B\ {\isasymLongrightarrow}\ {\isasymexists}S{\isacharprime}{\isacharprime}{\isasymsubseteq}B{\isachardot}\ finite\ S{\isacharprime}{\isacharprime}\ {\isasymand}\ {\isacharparenleft}insert\ x\ S\ {\isacharequal}\ {\isacharbraceleft}a{\isacharbraceright}\ {\isasymunion}\ S{\isacharprime}{\isacharprime}\ {\isasymor}\ insert\ x\ S\ {\isacharequal}\ S{\isacharprime}{\isacharprime}{\isacharparenright}{\isachardoublequoteclose}\isanewline
\ \ \isacommand{proof}\isamarkupfalse%
\ {\isacharminus}\isanewline
\ \ \ \ \isacommand{assume}\isamarkupfalse%
\ {\isachardoublequoteopen}insert\ x\ S\ {\isasymsubseteq}\ {\isacharbraceleft}a{\isacharbraceright}\ {\isasymunion}\ B{\isachardoublequoteclose}\ \isanewline
\ \ \ \ \isacommand{then}\isamarkupfalse%
\ \isacommand{have}\isamarkupfalse%
\ C{\isacharcolon}{\isachardoublequoteopen}x\ {\isasymin}\ {\isacharbraceleft}a{\isacharbraceright}\ {\isasymunion}\ B\ {\isasymand}\ S\ {\isasymsubseteq}\ {\isacharbraceleft}a{\isacharbraceright}\ {\isasymunion}\ B{\isachardoublequoteclose}\isanewline
\ \ \ \ \ \ \isacommand{by}\isamarkupfalse%
\ {\isacharparenleft}simp\ only{\isacharcolon}\ insert{\isacharunderscore}subset{\isacharparenright}\isanewline
\ \ \ \ \isacommand{then}\isamarkupfalse%
\ \isacommand{have}\isamarkupfalse%
\ {\isachardoublequoteopen}S\ {\isasymsubseteq}\ {\isacharbraceleft}a{\isacharbraceright}\ {\isasymunion}\ B{\isachardoublequoteclose}\isanewline
\ \ \ \ \ \ \isacommand{by}\isamarkupfalse%
\ {\isacharparenleft}rule\ conjunct{\isadigit{2}}{\isacharparenright}\isanewline
\ \ \ \ \isacommand{have}\isamarkupfalse%
\ Ex{\isadigit{1}}{\isacharcolon}{\isachardoublequoteopen}{\isasymexists}S{\isacharprime}{\isasymsubseteq}B{\isachardot}\ finite\ S{\isacharprime}\ {\isasymand}\ {\isacharparenleft}S\ {\isacharequal}\ {\isacharbraceleft}a{\isacharbraceright}\ {\isasymunion}\ S{\isacharprime}\ {\isasymor}\ S\ {\isacharequal}\ S{\isacharprime}{\isacharparenright}{\isachardoublequoteclose}\isanewline
\ \ \ \ \ \ \isacommand{using}\isamarkupfalse%
\ {\isacartoucheopen}S\ {\isasymsubseteq}\ {\isacharbraceleft}a{\isacharbraceright}\ {\isasymunion}\ B{\isacartoucheclose}\ \isacommand{by}\isamarkupfalse%
\ {\isacharparenleft}rule\ HI{\isacharparenright}\isanewline
\ \ \ \ \isacommand{obtain}\isamarkupfalse%
\ S{\isacharprime}\ \isakeyword{where}\ {\isachardoublequoteopen}S{\isacharprime}\ {\isasymsubseteq}\ B{\isachardoublequoteclose}\ \isakeyword{and}\ C{\isadigit{1}}{\isacharcolon}{\isachardoublequoteopen}finite\ S{\isacharprime}\ {\isasymand}\ {\isacharparenleft}S\ {\isacharequal}\ {\isacharbraceleft}a{\isacharbraceright}\ {\isasymunion}\ S{\isacharprime}\ {\isasymor}\ S\ {\isacharequal}\ S{\isacharprime}{\isacharparenright}{\isachardoublequoteclose}\isanewline
\ \ \ \ \ \ \isacommand{using}\isamarkupfalse%
\ Ex{\isadigit{1}}\ \isacommand{by}\isamarkupfalse%
\ {\isacharparenleft}rule\ subexE{\isacharparenright}\isanewline
\ \ \ \ \isacommand{have}\isamarkupfalse%
\ {\isachardoublequoteopen}finite\ S{\isacharprime}{\isachardoublequoteclose}\isanewline
\ \ \ \ \ \ \isacommand{using}\isamarkupfalse%
\ C{\isadigit{1}}\ \isacommand{by}\isamarkupfalse%
\ {\isacharparenleft}rule\ conjunct{\isadigit{1}}{\isacharparenright}\isanewline
\ \ \ \ \isacommand{then}\isamarkupfalse%
\ \isacommand{have}\isamarkupfalse%
\ {\isachardoublequoteopen}finite\ {\isacharparenleft}insert\ x\ S{\isacharprime}{\isacharparenright}{\isachardoublequoteclose}\isanewline
\ \ \ \ \ \ \isacommand{by}\isamarkupfalse%
\ {\isacharparenleft}simp\ only{\isacharcolon}\ finite{\isachardot}insertI{\isacharparenright}\isanewline
\ \ \ \ \isacommand{have}\isamarkupfalse%
\ {\isachardoublequoteopen}x\ {\isasymin}\ {\isacharbraceleft}a{\isacharbraceright}\ {\isasymunion}\ B{\isachardoublequoteclose}\isanewline
\ \ \ \ \ \ \isacommand{using}\isamarkupfalse%
\ C\ \isacommand{by}\isamarkupfalse%
\ {\isacharparenleft}rule\ conjunct{\isadigit{1}}{\isacharparenright}\isanewline
\ \ \ \ \isacommand{then}\isamarkupfalse%
\ \isacommand{have}\isamarkupfalse%
\ {\isachardoublequoteopen}x\ {\isasymin}\ {\isacharbraceleft}a{\isacharbraceright}\ {\isasymor}\ x\ {\isasymin}\ B{\isachardoublequoteclose}\isanewline
\ \ \ \ \ \ \isacommand{by}\isamarkupfalse%
\ {\isacharparenleft}simp\ only{\isacharcolon}\ Un{\isacharunderscore}iff{\isacharparenright}\isanewline
\ \ \ \ \isacommand{then}\isamarkupfalse%
\ \isacommand{have}\isamarkupfalse%
\ {\isachardoublequoteopen}x\ {\isacharequal}\ a\ {\isasymor}\ x\ {\isasymin}\ B{\isachardoublequoteclose}\isanewline
\ \ \ \ \ \ \isacommand{by}\isamarkupfalse%
\ {\isacharparenleft}simp\ only{\isacharcolon}\ singleton{\isacharunderscore}iff{\isacharparenright}\isanewline
\ \ \ \ \isacommand{thus}\isamarkupfalse%
\ {\isachardoublequoteopen}{\isasymexists}S{\isacharprime}{\isacharprime}{\isasymsubseteq}B{\isachardot}\ finite\ S{\isacharprime}{\isacharprime}\ {\isasymand}\ {\isacharparenleft}insert\ x\ S\ {\isacharequal}\ {\isacharbraceleft}a{\isacharbraceright}\ {\isasymunion}\ S{\isacharprime}{\isacharprime}\ {\isasymor}\ insert\ x\ S\ {\isacharequal}\ S{\isacharprime}{\isacharprime}{\isacharparenright}{\isachardoublequoteclose}\isanewline
\ \ \ \ \isacommand{proof}\isamarkupfalse%
\ {\isacharparenleft}rule\ disjE{\isacharparenright}\isanewline
\ \ \ \ \ \ \isacommand{assume}\isamarkupfalse%
\ {\isachardoublequoteopen}x\ {\isacharequal}\ a{\isachardoublequoteclose}\isanewline
\ \ \ \ \ \ \isacommand{have}\isamarkupfalse%
\ {\isachardoublequoteopen}S\ {\isacharequal}\ {\isacharbraceleft}a{\isacharbraceright}\ {\isasymunion}\ S{\isacharprime}\ {\isasymor}\ S\ {\isacharequal}\ S{\isacharprime}{\isachardoublequoteclose}\isanewline
\ \ \ \ \ \ \ \ \isacommand{using}\isamarkupfalse%
\ C{\isadigit{1}}\ \isacommand{by}\isamarkupfalse%
\ {\isacharparenleft}rule\ conjunct{\isadigit{2}}{\isacharparenright}\isanewline
\ \ \ \ \ \ \isacommand{thus}\isamarkupfalse%
\ {\isacharquery}thesis\isanewline
\ \ \ \ \ \ \isacommand{proof}\isamarkupfalse%
\ {\isacharparenleft}rule\ disjE{\isacharparenright}\isanewline
\ \ \ \ \ \ \ \ \isacommand{assume}\isamarkupfalse%
\ {\isachardoublequoteopen}S\ {\isacharequal}\ {\isacharbraceleft}a{\isacharbraceright}\ {\isasymunion}\ S{\isacharprime}{\isachardoublequoteclose}\isanewline
\ \ \ \ \ \ \ \ \isacommand{have}\isamarkupfalse%
\ {\isachardoublequoteopen}x\ {\isasymin}\ {\isacharbraceleft}a{\isacharbraceright}{\isachardoublequoteclose}\isanewline
\ \ \ \ \ \ \ \ \ \ \isacommand{using}\isamarkupfalse%
\ {\isacartoucheopen}x\ {\isacharequal}\ a{\isacartoucheclose}\ \isacommand{by}\isamarkupfalse%
\ {\isacharparenleft}simp\ only{\isacharcolon}\ singleton{\isacharunderscore}iff{\isacharparenright}\isanewline
\ \ \ \ \ \ \ \ \isacommand{then}\isamarkupfalse%
\ \isacommand{have}\isamarkupfalse%
\ {\isachardoublequoteopen}x\ {\isasymin}\ {\isacharbraceleft}a{\isacharbraceright}\ {\isasymunion}\ S{\isacharprime}{\isachardoublequoteclose}\ \isanewline
\ \ \ \ \ \ \ \ \ \ \isacommand{by}\isamarkupfalse%
\ {\isacharparenleft}simp\ only{\isacharcolon}\ UnI{\isadigit{1}}{\isacharparenright}\isanewline
\ \ \ \ \ \ \ \ \isacommand{then}\isamarkupfalse%
\ \isacommand{have}\isamarkupfalse%
\ {\isachardoublequoteopen}insert\ x\ {\isacharparenleft}{\isacharbraceleft}a{\isacharbraceright}\ {\isasymunion}\ S{\isacharprime}{\isacharparenright}\ {\isacharequal}\ {\isacharbraceleft}a{\isacharbraceright}\ {\isasymunion}\ S{\isacharprime}{\isachardoublequoteclose}\isanewline
\ \ \ \ \ \ \ \ \ \ \isacommand{by}\isamarkupfalse%
\ {\isacharparenleft}rule\ insert{\isacharunderscore}absorb{\isacharparenright}\isanewline
\ \ \ \ \ \ \ \ \isacommand{have}\isamarkupfalse%
\ {\isachardoublequoteopen}insert\ x\ S\ {\isacharequal}\ insert\ x\ {\isacharparenleft}{\isacharbraceleft}a{\isacharbraceright}\ {\isasymunion}\ S{\isacharprime}{\isacharparenright}{\isachardoublequoteclose}\isanewline
\ \ \ \ \ \ \ \ \ \ \isacommand{by}\isamarkupfalse%
\ {\isacharparenleft}simp\ only{\isacharcolon}\ {\isacartoucheopen}S\ {\isacharequal}\ {\isacharbraceleft}a{\isacharbraceright}\ {\isasymunion}\ S{\isacharprime}{\isacartoucheclose}{\isacharparenright}\isanewline
\ \ \ \ \ \ \ \ \isacommand{then}\isamarkupfalse%
\ \isacommand{have}\isamarkupfalse%
\ {\isachardoublequoteopen}insert\ x\ S\ {\isacharequal}\ {\isacharbraceleft}a{\isacharbraceright}\ {\isasymunion}\ S{\isacharprime}{\isachardoublequoteclose}\isanewline
\ \ \ \ \ \ \ \ \ \ \isacommand{by}\isamarkupfalse%
\ {\isacharparenleft}simp\ only{\isacharcolon}\ {\isacartoucheopen}insert\ x\ {\isacharparenleft}{\isacharbraceleft}a{\isacharbraceright}\ {\isasymunion}\ S{\isacharprime}{\isacharparenright}\ {\isacharequal}\ {\isacharbraceleft}a{\isacharbraceright}\ {\isasymunion}\ S{\isacharprime}{\isacartoucheclose}{\isacharparenright}\isanewline
\ \ \ \ \ \ \ \ \isacommand{then}\isamarkupfalse%
\ \isacommand{have}\isamarkupfalse%
\ {\isadigit{1}}{\isacharcolon}{\isachardoublequoteopen}insert\ x\ S\ {\isacharequal}\ {\isacharbraceleft}a{\isacharbraceright}\ {\isasymunion}\ S{\isacharprime}\ {\isasymor}\ insert\ x\ S\ {\isacharequal}\ S{\isacharprime}{\isachardoublequoteclose}\isanewline
\ \ \ \ \ \ \ \ \ \ \isacommand{by}\isamarkupfalse%
\ {\isacharparenleft}rule\ disjI{\isadigit{1}}{\isacharparenright}\isanewline
\ \ \ \ \ \ \ \ \isacommand{have}\isamarkupfalse%
\ {\isachardoublequoteopen}finite\ S{\isacharprime}\ {\isasymand}\ {\isacharparenleft}insert\ x\ S\ {\isacharequal}\ {\isacharbraceleft}a{\isacharbraceright}\ {\isasymunion}\ S{\isacharprime}\ {\isasymor}\ insert\ x\ S\ {\isacharequal}\ S{\isacharprime}{\isacharparenright}{\isachardoublequoteclose}\isanewline
\ \ \ \ \ \ \ \ \ \ \isacommand{using}\isamarkupfalse%
\ {\isacartoucheopen}finite\ S{\isacharprime}{\isacartoucheclose}\ {\isadigit{1}}\ \isacommand{by}\isamarkupfalse%
\ {\isacharparenleft}rule\ conjI{\isacharparenright}\isanewline
\ \ \ \ \ \ \ \ \isacommand{thus}\isamarkupfalse%
\ {\isacharquery}thesis\isanewline
\ \ \ \ \ \ \ \ \ \ \isacommand{using}\isamarkupfalse%
\ {\isacartoucheopen}S{\isacharprime}\ {\isasymsubseteq}\ B{\isacartoucheclose}\ \isacommand{by}\isamarkupfalse%
\ {\isacharparenleft}rule\ subexI{\isacharparenright}\isanewline
\ \ \ \ \ \ \isacommand{next}\isamarkupfalse%
\isanewline
\ \ \ \ \ \ \ \ \isacommand{assume}\isamarkupfalse%
\ {\isachardoublequoteopen}S\ {\isacharequal}\ S{\isacharprime}{\isachardoublequoteclose}\isanewline
\ \ \ \ \ \ \ \ \isacommand{have}\isamarkupfalse%
\ {\isachardoublequoteopen}insert\ x\ S\ {\isacharequal}\ {\isacharbraceleft}x{\isacharbraceright}\ {\isasymunion}\ S{\isachardoublequoteclose}\isanewline
\ \ \ \ \ \ \ \ \ \ \isacommand{by}\isamarkupfalse%
\ {\isacharparenleft}rule\ insert{\isacharunderscore}is{\isacharunderscore}Un{\isacharparenright}\isanewline
\ \ \ \ \ \ \ \ \isacommand{then}\isamarkupfalse%
\ \isacommand{have}\isamarkupfalse%
\ {\isachardoublequoteopen}insert\ x\ S\ {\isacharequal}\ {\isacharbraceleft}a{\isacharbraceright}\ {\isasymunion}\ S{\isacharprime}{\isachardoublequoteclose}\isanewline
\ \ \ \ \ \ \ \ \ \ \isacommand{by}\isamarkupfalse%
\ {\isacharparenleft}simp\ only{\isacharcolon}\ {\isacartoucheopen}x\ {\isacharequal}\ a{\isacartoucheclose}\ {\isacartoucheopen}S\ {\isacharequal}\ S{\isacharprime}{\isacartoucheclose}{\isacharparenright}\isanewline
\ \ \ \ \ \ \ \ \isacommand{then}\isamarkupfalse%
\ \isacommand{have}\isamarkupfalse%
\ {\isadigit{1}}{\isacharcolon}{\isachardoublequoteopen}insert\ x\ S\ {\isacharequal}\ {\isacharbraceleft}a{\isacharbraceright}\ {\isasymunion}\ S{\isacharprime}\ {\isasymor}\ insert\ x\ S\ {\isacharequal}\ S{\isacharprime}{\isachardoublequoteclose}\isanewline
\ \ \ \ \ \ \ \ \ \ \isacommand{by}\isamarkupfalse%
\ {\isacharparenleft}rule\ disjI{\isadigit{1}}{\isacharparenright}\isanewline
\ \ \ \ \ \ \ \ \isacommand{have}\isamarkupfalse%
\ {\isachardoublequoteopen}finite\ S{\isacharprime}\ {\isasymand}\ {\isacharparenleft}insert\ x\ S\ {\isacharequal}\ {\isacharbraceleft}a{\isacharbraceright}\ {\isasymunion}\ S{\isacharprime}\ {\isasymor}\ insert\ x\ S\ {\isacharequal}\ S{\isacharprime}{\isacharparenright}{\isachardoublequoteclose}\isanewline
\ \ \ \ \ \ \ \ \ \ \isacommand{using}\isamarkupfalse%
\ {\isacartoucheopen}finite\ S{\isacharprime}{\isacartoucheclose}\ {\isadigit{1}}\ \isacommand{by}\isamarkupfalse%
\ {\isacharparenleft}rule\ conjI{\isacharparenright}\isanewline
\ \ \ \ \ \ \ \ \isacommand{thus}\isamarkupfalse%
\ {\isacharquery}thesis\isanewline
\ \ \ \ \ \ \ \ \ \ \isacommand{using}\isamarkupfalse%
\ {\isacartoucheopen}S{\isacharprime}\ {\isasymsubseteq}\ B{\isacartoucheclose}\ \isacommand{by}\isamarkupfalse%
\ {\isacharparenleft}rule\ subexI{\isacharparenright}\isanewline
\ \ \ \ \ \ \isacommand{qed}\isamarkupfalse%
\isanewline
\ \ \ \ \isacommand{next}\isamarkupfalse%
\isanewline
\ \ \ \ \ \ \isacommand{assume}\isamarkupfalse%
\ {\isachardoublequoteopen}x\ {\isasymin}\ B{\isachardoublequoteclose}\isanewline
\ \ \ \ \ \ \isacommand{have}\isamarkupfalse%
\ {\isachardoublequoteopen}x\ {\isasymin}\ B\ {\isasymand}\ S{\isacharprime}\ {\isasymsubseteq}\ B{\isachardoublequoteclose}\isanewline
\ \ \ \ \ \ \ \ \isacommand{using}\isamarkupfalse%
\ {\isacartoucheopen}x\ {\isasymin}\ B{\isacartoucheclose}\ {\isacartoucheopen}S{\isacharprime}\ {\isasymsubseteq}\ B{\isacartoucheclose}\ \isacommand{by}\isamarkupfalse%
\ {\isacharparenleft}rule\ conjI{\isacharparenright}\isanewline
\ \ \ \ \ \ \isacommand{then}\isamarkupfalse%
\ \isacommand{have}\isamarkupfalse%
\ {\isachardoublequoteopen}insert\ x\ S{\isacharprime}\ {\isasymsubseteq}\ B{\isachardoublequoteclose}\isanewline
\ \ \ \ \ \ \ \ \isacommand{by}\isamarkupfalse%
\ {\isacharparenleft}simp\ only{\isacharcolon}\ insert{\isacharunderscore}subset{\isacharparenright}\isanewline
\ \ \ \ \ \ \isacommand{have}\isamarkupfalse%
\ {\isachardoublequoteopen}finite\ {\isacharparenleft}insert\ x\ S{\isacharprime}{\isacharparenright}{\isachardoublequoteclose}\isanewline
\ \ \ \ \ \ \ \ \isacommand{using}\isamarkupfalse%
\ {\isacartoucheopen}finite\ S{\isacharprime}{\isacartoucheclose}\ \isacommand{by}\isamarkupfalse%
\ {\isacharparenleft}simp\ only{\isacharcolon}\ finite{\isachardot}insertI{\isacharparenright}\isanewline
\ \ \ \ \ \ \isacommand{have}\isamarkupfalse%
\ {\isachardoublequoteopen}S\ {\isacharequal}\ {\isacharbraceleft}a{\isacharbraceright}\ {\isasymunion}\ S{\isacharprime}\ {\isasymor}\ S\ {\isacharequal}\ S{\isacharprime}{\isachardoublequoteclose}\isanewline
\ \ \ \ \ \ \ \ \isacommand{using}\isamarkupfalse%
\ C{\isadigit{1}}\ \isacommand{by}\isamarkupfalse%
\ {\isacharparenleft}rule\ conjunct{\isadigit{2}}{\isacharparenright}\isanewline
\ \ \ \ \ \ \isacommand{thus}\isamarkupfalse%
\ {\isacharquery}thesis\isanewline
\ \ \ \ \ \ \isacommand{proof}\isamarkupfalse%
\ {\isacharparenleft}rule\ disjE{\isacharparenright}\isanewline
\ \ \ \ \ \ \ \ \isacommand{assume}\isamarkupfalse%
\ {\isachardoublequoteopen}S\ {\isacharequal}\ {\isacharbraceleft}a{\isacharbraceright}\ {\isasymunion}\ S{\isacharprime}{\isachardoublequoteclose}\isanewline
\ \ \ \ \ \ \ \ \isacommand{have}\isamarkupfalse%
\ {\isachardoublequoteopen}insert\ x\ S\ {\isacharequal}\ insert\ x\ {\isacharparenleft}{\isacharbraceleft}a{\isacharbraceright}\ {\isasymunion}\ S{\isacharprime}{\isacharparenright}{\isachardoublequoteclose}\isanewline
\ \ \ \ \ \ \ \ \ \ \isacommand{by}\isamarkupfalse%
\ {\isacharparenleft}simp\ only{\isacharcolon}\ {\isacartoucheopen}S\ {\isacharequal}\ {\isacharbraceleft}a{\isacharbraceright}\ {\isasymunion}\ S{\isacharprime}{\isacartoucheclose}{\isacharparenright}\isanewline
\ \ \ \ \ \ \ \ \isacommand{then}\isamarkupfalse%
\ \isacommand{have}\isamarkupfalse%
\ {\isachardoublequoteopen}insert\ x\ S\ {\isacharequal}\ {\isacharbraceleft}a{\isacharbraceright}\ {\isasymunion}\ {\isacharparenleft}insert\ x\ S{\isacharprime}{\isacharparenright}{\isachardoublequoteclose}\isanewline
\ \ \ \ \ \ \ \ \ \ \isacommand{by}\isamarkupfalse%
\ blast\isanewline
\ \ \ \ \ \ \ \ \isacommand{then}\isamarkupfalse%
\ \isacommand{have}\isamarkupfalse%
\ {\isadigit{1}}{\isacharcolon}{\isachardoublequoteopen}insert\ x\ S\ {\isacharequal}\ {\isacharbraceleft}a{\isacharbraceright}\ {\isasymunion}\ {\isacharparenleft}insert\ x\ S{\isacharprime}{\isacharparenright}\ {\isasymor}\ insert\ x\ S\ {\isacharequal}\ insert\ x\ S{\isacharprime}{\isachardoublequoteclose}\isanewline
\ \ \ \ \ \ \ \ \ \ \isacommand{by}\isamarkupfalse%
\ {\isacharparenleft}rule\ disjI{\isadigit{1}}{\isacharparenright}\isanewline
\ \ \ \ \ \ \ \ \isacommand{have}\isamarkupfalse%
\ {\isachardoublequoteopen}finite\ {\isacharparenleft}insert\ x\ S{\isacharprime}{\isacharparenright}\ {\isasymand}\ {\isacharparenleft}insert\ x\ S\ {\isacharequal}\ {\isacharbraceleft}a{\isacharbraceright}\ {\isasymunion}\ {\isacharparenleft}insert\ x\ S{\isacharprime}{\isacharparenright}\ {\isasymor}\ insert\ x\ S\ {\isacharequal}\ insert\ x\ S{\isacharprime}{\isacharparenright}{\isachardoublequoteclose}\isanewline
\ \ \ \ \ \ \ \ \ \ \isacommand{using}\isamarkupfalse%
\ {\isacartoucheopen}finite\ {\isacharparenleft}insert\ x\ S{\isacharprime}{\isacharparenright}{\isacartoucheclose}\ {\isadigit{1}}\ \isacommand{by}\isamarkupfalse%
\ {\isacharparenleft}rule\ conjI{\isacharparenright}\isanewline
\ \ \ \ \ \ \ \ \isacommand{thus}\isamarkupfalse%
\ {\isacharquery}thesis\isanewline
\ \ \ \ \ \ \ \ \ \ \isacommand{using}\isamarkupfalse%
\ {\isacartoucheopen}insert\ x\ S{\isacharprime}\ {\isasymsubseteq}\ B{\isacartoucheclose}\ \isacommand{by}\isamarkupfalse%
\ {\isacharparenleft}rule\ subexI{\isacharparenright}\isanewline
\ \ \ \ \ \ \isacommand{next}\isamarkupfalse%
\isanewline
\ \ \ \ \ \ \ \ \isacommand{assume}\isamarkupfalse%
\ {\isachardoublequoteopen}S\ {\isacharequal}\ S{\isacharprime}{\isachardoublequoteclose}\isanewline
\ \ \ \ \ \ \ \ \isacommand{have}\isamarkupfalse%
\ {\isachardoublequoteopen}insert\ x\ S\ {\isacharequal}\ insert\ x\ S{\isacharprime}{\isachardoublequoteclose}\isanewline
\ \ \ \ \ \ \ \ \ \ \isacommand{by}\isamarkupfalse%
\ {\isacharparenleft}simp\ only{\isacharcolon}\ {\isacartoucheopen}S\ {\isacharequal}\ S{\isacharprime}{\isacartoucheclose}{\isacharparenright}\isanewline
\ \ \ \ \ \ \ \ \isacommand{then}\isamarkupfalse%
\ \isacommand{have}\isamarkupfalse%
\ {\isadigit{1}}{\isacharcolon}{\isachardoublequoteopen}insert\ x\ S\ {\isacharequal}\ {\isacharbraceleft}a{\isacharbraceright}\ {\isasymunion}\ {\isacharparenleft}insert\ x\ S{\isacharprime}{\isacharparenright}\ {\isasymor}\ insert\ x\ S\ {\isacharequal}\ insert\ x\ S{\isacharprime}{\isachardoublequoteclose}\isanewline
\ \ \ \ \ \ \ \ \ \ \isacommand{by}\isamarkupfalse%
\ {\isacharparenleft}rule\ disjI{\isadigit{2}}{\isacharparenright}\isanewline
\ \ \ \ \ \ \ \ \isacommand{have}\isamarkupfalse%
\ {\isachardoublequoteopen}finite\ {\isacharparenleft}insert\ x\ S{\isacharprime}{\isacharparenright}\ {\isasymand}\ {\isacharparenleft}insert\ x\ S\ {\isacharequal}\ {\isacharbraceleft}a{\isacharbraceright}\ {\isasymunion}\ {\isacharparenleft}insert\ x\ S{\isacharprime}{\isacharparenright}\ {\isasymor}\ insert\ x\ S\ {\isacharequal}\ insert\ x\ S{\isacharprime}{\isacharparenright}{\isachardoublequoteclose}\isanewline
\ \ \ \ \ \ \ \ \ \ \isacommand{using}\isamarkupfalse%
\ {\isacartoucheopen}finite\ {\isacharparenleft}insert\ x\ S{\isacharprime}{\isacharparenright}{\isacartoucheclose}\ {\isadigit{1}}\ \isacommand{by}\isamarkupfalse%
\ {\isacharparenleft}rule\ conjI{\isacharparenright}\isanewline
\ \ \ \ \ \ \ \ \isacommand{thus}\isamarkupfalse%
\ {\isacharquery}thesis\isanewline
\ \ \ \ \ \ \ \ \ \ \isacommand{using}\isamarkupfalse%
\ {\isacartoucheopen}insert\ x\ S{\isacharprime}\ {\isasymsubseteq}\ B{\isacartoucheclose}\ \isacommand{by}\isamarkupfalse%
\ {\isacharparenleft}rule\ subexI{\isacharparenright}\isanewline
\ \ \ \ \ \ \isacommand{qed}\isamarkupfalse%
\isanewline
\ \ \ \ \isacommand{qed}\isamarkupfalse%
\isanewline
\ \ \isacommand{qed}\isamarkupfalse%
\isanewline
\isacommand{qed}\isamarkupfalse%
%
\endisatagproof
{\isafoldproof}%
%
\isadelimproof
%
\endisadelimproof
%
\begin{isamarkuptext}%
El segundo resultado sobre subconjuntos finitos es consecuencia del anterior.

\begin{lema}
  Sea un conjunto de la forma \isa{{\isacharbraceleft}a{\isacharcomma}b{\isacharbraceright}\ {\isasymunion}\ B} y \isa{S} un subconjunto finito suyo. Entonces, existe un
  subconjunto finito \isa{S{\isacharprime}} de \isa{B} tal que se cumple \isa{S\ {\isacharequal}\ {\isacharbraceleft}a{\isacharcomma}b{\isacharbraceright}\ {\isasymunion}\ S{\isacharprime}}, \isa{S\ {\isacharequal}\ {\isacharbraceleft}a{\isacharbraceright}\ {\isasymunion}\ S{\isacharprime}},\\ \isa{S\ {\isacharequal}\ {\isacharbraceleft}b{\isacharbraceright}\ {\isasymunion}\ S{\isacharprime}} 
  o \isa{S\ {\isacharequal}\ S{\isacharprime}}.
\end{lema}

\begin{demostracion}
  En particular, \isa{S} es un subconjunto finito de \isa{{\isacharbraceleft}a{\isacharbraceright}\ {\isasymunion}\ {\isacharparenleft}{\isacharbraceleft}b{\isacharbraceright}\ {\isasymunion}\ B{\isacharparenright}} luego, aplicando el lema \isa{{\isadigit{4}}{\isachardot}{\isadigit{3}}{\isachardot}{\isadigit{2}}}
  anterior, podemos hallar un subconjunto finito \isa{S\isactrlsub {\isadigit{1}}} de \isa{{\isacharbraceleft}b{\isacharbraceright}\ {\isasymunion}\ B} tal que o bien \isa{S\ {\isacharequal}\ {\isacharbraceleft}a{\isacharbraceright}\ {\isasymunion}\ S\isactrlsub {\isadigit{1}}} o 
  bien \isa{S\ {\isacharequal}\ S\isactrlsub {\isadigit{1}}}. A su vez, podemos aplicar dicho resultado para el subconjunto finito \isa{S\isactrlsub {\isadigit{1}}} de 
  \isa{{\isacharbraceleft}b{\isacharbraceright}\ {\isasymunion}\ B}, obteniendo un subconjunto finito \isa{S\isactrlsub {\isadigit{2}}} de \isa{B} tal que o bien \isa{S\isactrlsub {\isadigit{1}}\ {\isacharequal}\ {\isacharbraceleft}b{\isacharbraceright}\ {\isasymunion}\ S\isactrlsub {\isadigit{2}}} o bien 
  \isa{S\isactrlsub {\isadigit{1}}\ {\isacharequal}\ S\isactrlsub {\isadigit{2}}}. Veamos que el lema se verifica en ambas opciones posibles de \isa{S\isactrlsub {\isadigit{1}}} para el conjunto 
  \isa{S{\isacharprime}\ {\isacharequal}\ S\isactrlsub {\isadigit{2}}}.

  En primer lugar, supongamos que \isa{S\isactrlsub {\isadigit{1}}\ {\isacharequal}\ {\isacharbraceleft}b{\isacharbraceright}\ {\isasymunion}\ S\isactrlsub {\isadigit{2}}}. De este modo, se verifica el resultado tanto para
  \isa{S\ {\isacharequal}\ {\isacharbraceleft}a{\isacharbraceright}\ {\isasymunion}\ S\isactrlsub {\isadigit{1}}} como para \isa{S\ {\isacharequal}\ S\isactrlsub {\isadigit{1}}}. En efecto, en la primera opción, por elección de \isa{S\isactrlsub {\isadigit{1}}}, es claro
  que \isa{S\ {\isacharequal}\ {\isacharbraceleft}a{\isacharcomma}b{\isacharbraceright}\ {\isasymunion}\ S\isactrlsub {\isadigit{2}}}. Finalmente, para \isa{S\ {\isacharequal}\ S\isactrlsub {\isadigit{1}}}, obtenemos que \isa{S\ {\isacharequal}\ {\isacharbraceleft}b{\isacharbraceright}\ {\isasymunion}\ S\isactrlsub {\isadigit{2}}}, lo que prueba
  igualmente el lema para \isa{S{\isacharprime}\ {\isacharequal}\ S\isactrlsub {\isadigit{2}}}.

  Por último, supongamos que \isa{S\isactrlsub {\isadigit{1}}\ {\isacharequal}\ S\isactrlsub {\isadigit{2}}}. Análogamente, el resultado es inmediato pues si 
  \isa{S\ {\isacharequal}\ {\isacharbraceleft}a{\isacharbraceright}\ {\isasymunion}\ S\isactrlsub {\isadigit{1}}} obtenemos que \isa{S\ {\isacharequal}\ {\isacharbraceleft}a{\isacharbraceright}\ {\isasymunion}\ S\isactrlsub {\isadigit{2}}}, y si suponemos \isa{S\ {\isacharequal}\ S\isactrlsub {\isadigit{1}}} obtenemos\\ \isa{S\ {\isacharequal}\ S\isactrlsub {\isadigit{2}}}, probando
  así el lema.
\end{demostracion}

  Su formalización y prueba detallada en Isabelle/HOL son las siguientes.%
\end{isamarkuptext}\isamarkuptrue%
\isacommand{lemma}\isamarkupfalse%
\ finite{\isacharunderscore}subset{\isacharunderscore}insert{\isadigit{2}}{\isacharcolon}\isanewline
\ \ \isakeyword{assumes}\ {\isachardoublequoteopen}finite\ S{\isachardoublequoteclose}\isanewline
\ \ \ \ \ \ \ \ \ \ {\isachardoublequoteopen}S\ {\isasymsubseteq}\ {\isacharbraceleft}a{\isacharcomma}b{\isacharbraceright}\ {\isasymunion}\ B{\isachardoublequoteclose}\isanewline
\ \ \ \ \ \ \ \ \isakeyword{shows}\ {\isachardoublequoteopen}{\isasymexists}S{\isacharprime}\ {\isasymsubseteq}\ B{\isachardot}\ finite\ S{\isacharprime}\ {\isasymand}\ {\isacharparenleft}S\ {\isacharequal}\ {\isacharbraceleft}a{\isacharcomma}b{\isacharbraceright}\ {\isasymunion}\ S{\isacharprime}\ {\isasymor}\ S\ {\isacharequal}\ {\isacharbraceleft}a{\isacharbraceright}\ {\isasymunion}\ S{\isacharprime}\ {\isasymor}\ S\ {\isacharequal}\ {\isacharbraceleft}b{\isacharbraceright}\ {\isasymunion}\ S{\isacharprime}\ {\isasymor}\ S\ {\isacharequal}\ S{\isacharprime}{\isacharparenright}{\isachardoublequoteclose}\isanewline
%
\isadelimproof
%
\endisadelimproof
%
\isatagproof
\isacommand{proof}\isamarkupfalse%
\ {\isacharminus}\isanewline
\ \ \isacommand{have}\isamarkupfalse%
\ {\isachardoublequoteopen}S\ {\isasymsubseteq}\ {\isacharbraceleft}a{\isacharbraceright}\ {\isasymunion}\ {\isacharparenleft}{\isacharbraceleft}b{\isacharbraceright}\ {\isasymunion}\ B{\isacharparenright}{\isachardoublequoteclose}\isanewline
\ \ \ \ \isacommand{using}\isamarkupfalse%
\ assms{\isacharparenleft}{\isadigit{2}}{\isacharparenright}\ \isacommand{by}\isamarkupfalse%
\ blast\isanewline
\ \ \isacommand{then}\isamarkupfalse%
\ \isacommand{have}\isamarkupfalse%
\ Ex{\isadigit{1}}{\isacharcolon}{\isachardoublequoteopen}{\isasymexists}S{\isadigit{1}}\ {\isasymsubseteq}\ {\isacharparenleft}{\isacharbraceleft}b{\isacharbraceright}\ {\isasymunion}\ B{\isacharparenright}{\isachardot}\ finite\ S{\isadigit{1}}\ {\isasymand}\ {\isacharparenleft}S\ {\isacharequal}\ {\isacharbraceleft}a{\isacharbraceright}\ {\isasymunion}\ S{\isadigit{1}}\ {\isasymor}\ S\ {\isacharequal}\ S{\isadigit{1}}{\isacharparenright}{\isachardoublequoteclose}\isanewline
\ \ \ \ \isacommand{using}\isamarkupfalse%
\ assms{\isacharparenleft}{\isadigit{1}}{\isacharparenright}\ \isacommand{by}\isamarkupfalse%
\ {\isacharparenleft}simp\ only{\isacharcolon}\ finite{\isacharunderscore}subset{\isacharunderscore}insert{\isadigit{1}}{\isacharparenright}\isanewline
\ \ \isacommand{obtain}\isamarkupfalse%
\ S{\isadigit{1}}\ \isakeyword{where}\ {\isachardoublequoteopen}S{\isadigit{1}}\ {\isasymsubseteq}\ {\isacharbraceleft}b{\isacharbraceright}\ {\isasymunion}\ B{\isachardoublequoteclose}\ \isakeyword{and}\ {\isadigit{1}}{\isacharcolon}{\isachardoublequoteopen}finite\ S{\isadigit{1}}\ {\isasymand}\ {\isacharparenleft}S\ {\isacharequal}\ {\isacharbraceleft}a{\isacharbraceright}\ {\isasymunion}\ S{\isadigit{1}}\ {\isasymor}\ S\ {\isacharequal}\ S{\isadigit{1}}{\isacharparenright}{\isachardoublequoteclose}\isanewline
\ \ \ \ \isacommand{using}\isamarkupfalse%
\ Ex{\isadigit{1}}\ \isacommand{by}\isamarkupfalse%
\ {\isacharparenleft}rule\ subexE{\isacharparenright}\isanewline
\ \ \isacommand{have}\isamarkupfalse%
\ {\isachardoublequoteopen}finite\ S{\isadigit{1}}{\isachardoublequoteclose}\isanewline
\ \ \ \ \isacommand{using}\isamarkupfalse%
\ {\isadigit{1}}\ \isacommand{by}\isamarkupfalse%
\ {\isacharparenleft}rule\ conjunct{\isadigit{1}}{\isacharparenright}\isanewline
\ \ \isacommand{have}\isamarkupfalse%
\ Ex{\isadigit{2}}{\isacharcolon}{\isachardoublequoteopen}{\isasymexists}S{\isadigit{2}}\ {\isasymsubseteq}\ B{\isachardot}\ finite\ S{\isadigit{2}}\ {\isasymand}\ {\isacharparenleft}S{\isadigit{1}}\ {\isacharequal}\ {\isacharbraceleft}b{\isacharbraceright}\ {\isasymunion}\ S{\isadigit{2}}\ {\isasymor}\ S{\isadigit{1}}\ {\isacharequal}\ S{\isadigit{2}}{\isacharparenright}{\isachardoublequoteclose}\isanewline
\ \ \ \ \isacommand{using}\isamarkupfalse%
\ {\isacartoucheopen}finite\ S{\isadigit{1}}{\isacartoucheclose}\ {\isacartoucheopen}S{\isadigit{1}}\ {\isasymsubseteq}\ {\isacharbraceleft}b{\isacharbraceright}\ {\isasymunion}\ B{\isacartoucheclose}\ \isacommand{by}\isamarkupfalse%
\ {\isacharparenleft}rule\ finite{\isacharunderscore}subset{\isacharunderscore}insert{\isadigit{1}}{\isacharparenright}\isanewline
\ \ \isacommand{obtain}\isamarkupfalse%
\ S{\isadigit{2}}\ \isakeyword{where}\ {\isachardoublequoteopen}S{\isadigit{2}}\ {\isasymsubseteq}\ B{\isachardoublequoteclose}\ \isakeyword{and}\ {\isadigit{2}}{\isacharcolon}{\isachardoublequoteopen}finite\ S{\isadigit{2}}\ {\isasymand}\ {\isacharparenleft}S{\isadigit{1}}\ {\isacharequal}\ {\isacharbraceleft}b{\isacharbraceright}\ {\isasymunion}\ S{\isadigit{2}}\ {\isasymor}\ S{\isadigit{1}}\ {\isacharequal}\ S{\isadigit{2}}{\isacharparenright}{\isachardoublequoteclose}\isanewline
\ \ \ \ \isacommand{using}\isamarkupfalse%
\ Ex{\isadigit{2}}\ \isacommand{by}\isamarkupfalse%
\ {\isacharparenleft}rule\ subexE{\isacharparenright}\isanewline
\ \ \isacommand{have}\isamarkupfalse%
\ {\isachardoublequoteopen}finite\ S{\isadigit{2}}{\isachardoublequoteclose}\isanewline
\ \ \ \ \isacommand{using}\isamarkupfalse%
\ {\isadigit{2}}\ \isacommand{by}\isamarkupfalse%
\ {\isacharparenleft}rule\ conjunct{\isadigit{1}}{\isacharparenright}\isanewline
\ \ \isacommand{have}\isamarkupfalse%
\ {\isachardoublequoteopen}S{\isadigit{1}}\ {\isacharequal}\ {\isacharbraceleft}b{\isacharbraceright}\ {\isasymunion}\ S{\isadigit{2}}\ {\isasymor}\ S{\isadigit{1}}\ {\isacharequal}\ S{\isadigit{2}}{\isachardoublequoteclose}\isanewline
\ \ \ \ \isacommand{using}\isamarkupfalse%
\ {\isadigit{2}}\ \isacommand{by}\isamarkupfalse%
\ {\isacharparenleft}rule\ conjunct{\isadigit{2}}{\isacharparenright}\isanewline
\ \ \isacommand{thus}\isamarkupfalse%
\ {\isacharquery}thesis\isanewline
\ \ \isacommand{proof}\isamarkupfalse%
\ {\isacharparenleft}rule\ disjE{\isacharparenright}\isanewline
\ \ \ \ \isacommand{assume}\isamarkupfalse%
\ {\isachardoublequoteopen}S{\isadigit{1}}\ {\isacharequal}\ {\isacharbraceleft}b{\isacharbraceright}\ {\isasymunion}\ S{\isadigit{2}}{\isachardoublequoteclose}\isanewline
\ \ \ \ \isacommand{have}\isamarkupfalse%
\ {\isachardoublequoteopen}S\ {\isacharequal}\ {\isacharbraceleft}a{\isacharbraceright}\ {\isasymunion}\ S{\isadigit{1}}\ {\isasymor}\ S\ {\isacharequal}\ S{\isadigit{1}}{\isachardoublequoteclose}\isanewline
\ \ \ \ \ \ \isacommand{using}\isamarkupfalse%
\ {\isadigit{1}}\ \isacommand{by}\isamarkupfalse%
\ {\isacharparenleft}rule\ conjunct{\isadigit{2}}{\isacharparenright}\isanewline
\ \ \ \ \isacommand{thus}\isamarkupfalse%
\ {\isacharquery}thesis\isanewline
\ \ \ \ \isacommand{proof}\isamarkupfalse%
\ {\isacharparenleft}rule\ disjE{\isacharparenright}\isanewline
\ \ \ \ \ \ \isacommand{assume}\isamarkupfalse%
\ {\isachardoublequoteopen}S\ {\isacharequal}\ {\isacharbraceleft}a{\isacharbraceright}\ {\isasymunion}\ S{\isadigit{1}}{\isachardoublequoteclose}\isanewline
\ \ \ \ \ \ \isacommand{then}\isamarkupfalse%
\ \isacommand{have}\isamarkupfalse%
\ {\isachardoublequoteopen}S\ {\isacharequal}\ {\isacharbraceleft}a{\isacharbraceright}\ {\isasymunion}\ {\isacharbraceleft}b{\isacharbraceright}\ {\isasymunion}\ S{\isadigit{2}}{\isachardoublequoteclose}\isanewline
\ \ \ \ \ \ \ \ \isacommand{by}\isamarkupfalse%
\ {\isacharparenleft}simp\ add{\isacharcolon}\ {\isacartoucheopen}S{\isadigit{1}}\ {\isacharequal}\ {\isacharbraceleft}b{\isacharbraceright}\ {\isasymunion}\ S{\isadigit{2}}{\isacartoucheclose}{\isacharparenright}\isanewline
\ \ \ \ \ \ \isacommand{then}\isamarkupfalse%
\ \isacommand{have}\isamarkupfalse%
\ {\isachardoublequoteopen}S\ {\isacharequal}\ {\isacharbraceleft}a{\isacharcomma}b{\isacharbraceright}\ {\isasymunion}\ S{\isadigit{2}}{\isachardoublequoteclose}\isanewline
\ \ \ \ \ \ \ \ \isacommand{by}\isamarkupfalse%
\ blast\isanewline
\ \ \ \ \ \ \isacommand{then}\isamarkupfalse%
\ \isacommand{have}\isamarkupfalse%
\ {\isachardoublequoteopen}S\ {\isacharequal}\ {\isacharbraceleft}a{\isacharcomma}b{\isacharbraceright}\ {\isasymunion}\ S{\isadigit{2}}\ {\isasymor}\ S\ {\isacharequal}\ {\isacharbraceleft}a{\isacharbraceright}\ {\isasymunion}\ S{\isadigit{2}}\ {\isasymor}\ S\ {\isacharequal}\ {\isacharbraceleft}b{\isacharbraceright}\ {\isasymunion}\ S{\isadigit{2}}\ {\isasymor}\ S\ {\isacharequal}\ S{\isadigit{2}}{\isachardoublequoteclose}\isanewline
\ \ \ \ \ \ \ \ \isacommand{by}\isamarkupfalse%
\ {\isacharparenleft}iprover\ intro{\isacharcolon}\ disjI{\isadigit{1}}{\isacharparenright}\isanewline
\ \ \ \ \ \ \isacommand{then}\isamarkupfalse%
\ \isacommand{have}\isamarkupfalse%
\ {\isachardoublequoteopen}finite\ S{\isadigit{2}}\ {\isasymand}\ {\isacharparenleft}S\ {\isacharequal}\ {\isacharbraceleft}a{\isacharcomma}b{\isacharbraceright}\ {\isasymunion}\ S{\isadigit{2}}\ {\isasymor}\ S\ {\isacharequal}\ {\isacharbraceleft}a{\isacharbraceright}\ {\isasymunion}\ S{\isadigit{2}}\ {\isasymor}\ S\ {\isacharequal}\ {\isacharbraceleft}b{\isacharbraceright}\ {\isasymunion}\ S{\isadigit{2}}\ {\isasymor}\ S\ {\isacharequal}\ S{\isadigit{2}}{\isacharparenright}{\isachardoublequoteclose}\isanewline
\ \ \ \ \ \ \ \ \isacommand{using}\isamarkupfalse%
\ {\isacartoucheopen}finite\ S{\isadigit{2}}{\isacartoucheclose}\ \isacommand{by}\isamarkupfalse%
\ {\isacharparenleft}iprover\ intro{\isacharcolon}\ conjI{\isacharparenright}\isanewline
\ \ \ \ \ \ \isacommand{thus}\isamarkupfalse%
\ {\isacharquery}thesis\isanewline
\ \ \ \ \ \ \ \ \isacommand{using}\isamarkupfalse%
\ {\isacartoucheopen}S{\isadigit{2}}\ {\isasymsubseteq}\ B{\isacartoucheclose}\ \isacommand{by}\isamarkupfalse%
\ {\isacharparenleft}rule\ subexI{\isacharparenright}\isanewline
\ \ \ \ \isacommand{next}\isamarkupfalse%
\isanewline
\ \ \ \ \ \ \isacommand{assume}\isamarkupfalse%
\ {\isachardoublequoteopen}S\ {\isacharequal}\ S{\isadigit{1}}{\isachardoublequoteclose}\isanewline
\ \ \ \ \ \ \isacommand{then}\isamarkupfalse%
\ \isacommand{have}\isamarkupfalse%
\ {\isachardoublequoteopen}S\ {\isacharequal}\ {\isacharbraceleft}b{\isacharbraceright}\ {\isasymunion}\ S{\isadigit{2}}{\isachardoublequoteclose}\isanewline
\ \ \ \ \ \ \ \ \isacommand{by}\isamarkupfalse%
\ {\isacharparenleft}simp\ add{\isacharcolon}\ {\isacartoucheopen}S{\isadigit{1}}\ {\isacharequal}\ {\isacharbraceleft}b{\isacharbraceright}\ {\isasymunion}\ S{\isadigit{2}}{\isacartoucheclose}{\isacharparenright}\isanewline
\ \ \ \ \ \ \isacommand{then}\isamarkupfalse%
\ \isacommand{have}\isamarkupfalse%
\ {\isachardoublequoteopen}S\ {\isacharequal}\ {\isacharbraceleft}a{\isacharcomma}b{\isacharbraceright}\ {\isasymunion}\ S{\isadigit{2}}\ {\isasymor}\ S\ {\isacharequal}\ {\isacharbraceleft}a{\isacharbraceright}\ {\isasymunion}\ S{\isadigit{2}}\ {\isasymor}\ S\ {\isacharequal}\ {\isacharbraceleft}b{\isacharbraceright}\ {\isasymunion}\ S{\isadigit{2}}\ {\isasymor}\ S\ {\isacharequal}\ S{\isadigit{2}}{\isachardoublequoteclose}\isanewline
\ \ \ \ \ \ \ \ \isacommand{by}\isamarkupfalse%
\ {\isacharparenleft}iprover\ intro{\isacharcolon}\ disjI{\isadigit{1}}{\isacharparenright}\isanewline
\ \ \ \ \ \ \isacommand{then}\isamarkupfalse%
\ \isacommand{have}\isamarkupfalse%
\ {\isachardoublequoteopen}finite\ S{\isadigit{2}}\ {\isasymand}\ {\isacharparenleft}S\ {\isacharequal}\ {\isacharbraceleft}a{\isacharcomma}b{\isacharbraceright}\ {\isasymunion}\ S{\isadigit{2}}\ {\isasymor}\ S\ {\isacharequal}\ {\isacharbraceleft}a{\isacharbraceright}\ {\isasymunion}\ S{\isadigit{2}}\ {\isasymor}\ S\ {\isacharequal}\ {\isacharbraceleft}b{\isacharbraceright}\ {\isasymunion}\ S{\isadigit{2}}\ {\isasymor}\ S\ {\isacharequal}\ S{\isadigit{2}}{\isacharparenright}{\isachardoublequoteclose}\isanewline
\ \ \ \ \ \ \ \ \isacommand{using}\isamarkupfalse%
\ {\isacartoucheopen}finite\ S{\isadigit{2}}{\isacartoucheclose}\ \isacommand{by}\isamarkupfalse%
\ {\isacharparenleft}iprover\ intro{\isacharcolon}\ conjI{\isacharparenright}\isanewline
\ \ \ \ \ \ \isacommand{thus}\isamarkupfalse%
\ {\isacharquery}thesis\isanewline
\ \ \ \ \ \ \ \ \isacommand{using}\isamarkupfalse%
\ {\isacartoucheopen}S{\isadigit{2}}\ {\isasymsubseteq}\ B{\isacartoucheclose}\ \isacommand{by}\isamarkupfalse%
\ {\isacharparenleft}rule\ subexI{\isacharparenright}\isanewline
\ \ \ \ \isacommand{qed}\isamarkupfalse%
\isanewline
\ \ \isacommand{next}\isamarkupfalse%
\isanewline
\ \ \ \ \isacommand{assume}\isamarkupfalse%
\ {\isachardoublequoteopen}S{\isadigit{1}}\ {\isacharequal}\ S{\isadigit{2}}{\isachardoublequoteclose}\isanewline
\ \ \ \ \isacommand{have}\isamarkupfalse%
\ {\isachardoublequoteopen}S\ {\isacharequal}\ {\isacharbraceleft}a{\isacharbraceright}\ {\isasymunion}\ S{\isadigit{1}}\ {\isasymor}\ S\ {\isacharequal}\ S{\isadigit{1}}{\isachardoublequoteclose}\isanewline
\ \ \ \ \ \ \isacommand{using}\isamarkupfalse%
\ {\isadigit{1}}\ \isacommand{by}\isamarkupfalse%
\ {\isacharparenleft}rule\ conjunct{\isadigit{2}}{\isacharparenright}\isanewline
\ \ \ \ \isacommand{thus}\isamarkupfalse%
\ {\isacharquery}thesis\isanewline
\ \ \ \ \isacommand{proof}\isamarkupfalse%
\ {\isacharparenleft}rule\ disjE{\isacharparenright}\isanewline
\ \ \ \ \ \ \isacommand{assume}\isamarkupfalse%
\ {\isachardoublequoteopen}S\ {\isacharequal}\ {\isacharbraceleft}a{\isacharbraceright}\ {\isasymunion}\ S{\isadigit{1}}{\isachardoublequoteclose}\isanewline
\ \ \ \ \ \ \isacommand{then}\isamarkupfalse%
\ \isacommand{have}\isamarkupfalse%
\ {\isachardoublequoteopen}S\ {\isacharequal}\ {\isacharbraceleft}a{\isacharbraceright}\ {\isasymunion}\ S{\isadigit{2}}{\isachardoublequoteclose}\isanewline
\ \ \ \ \ \ \ \ \isacommand{by}\isamarkupfalse%
\ {\isacharparenleft}simp\ only{\isacharcolon}\ {\isacartoucheopen}S{\isadigit{1}}\ {\isacharequal}\ S{\isadigit{2}}{\isacartoucheclose}{\isacharparenright}\isanewline
\ \ \ \ \ \ \isacommand{then}\isamarkupfalse%
\ \isacommand{have}\isamarkupfalse%
\ {\isachardoublequoteopen}S\ {\isacharequal}\ {\isacharbraceleft}a{\isacharcomma}b{\isacharbraceright}\ {\isasymunion}\ S{\isadigit{2}}\ {\isasymor}\ S\ {\isacharequal}\ {\isacharbraceleft}a{\isacharbraceright}\ {\isasymunion}\ S{\isadigit{2}}\ {\isasymor}\ S\ {\isacharequal}\ {\isacharbraceleft}b{\isacharbraceright}\ {\isasymunion}\ S{\isadigit{2}}\ {\isasymor}\ S\ {\isacharequal}\ S{\isadigit{2}}{\isachardoublequoteclose}\isanewline
\ \ \ \ \ \ \ \ \isacommand{by}\isamarkupfalse%
\ {\isacharparenleft}iprover\ intro{\isacharcolon}\ disjI{\isadigit{1}}{\isacharparenright}\isanewline
\ \ \ \ \ \ \isacommand{then}\isamarkupfalse%
\ \isacommand{have}\isamarkupfalse%
\ {\isachardoublequoteopen}finite\ S{\isadigit{2}}\ {\isasymand}\ {\isacharparenleft}S\ {\isacharequal}\ {\isacharbraceleft}a{\isacharcomma}b{\isacharbraceright}\ {\isasymunion}\ S{\isadigit{2}}\ {\isasymor}\ S\ {\isacharequal}\ {\isacharbraceleft}a{\isacharbraceright}\ {\isasymunion}\ S{\isadigit{2}}\ {\isasymor}\ S\ {\isacharequal}\ {\isacharbraceleft}b{\isacharbraceright}\ {\isasymunion}\ S{\isadigit{2}}\ {\isasymor}\ S\ {\isacharequal}\ S{\isadigit{2}}{\isacharparenright}{\isachardoublequoteclose}\isanewline
\ \ \ \ \ \ \ \ \isacommand{using}\isamarkupfalse%
\ {\isacartoucheopen}finite\ S{\isadigit{2}}{\isacartoucheclose}\ \isacommand{by}\isamarkupfalse%
\ {\isacharparenleft}iprover\ intro{\isacharcolon}\ conjI{\isacharparenright}\isanewline
\ \ \ \ \ \ \isacommand{thus}\isamarkupfalse%
\ {\isacharquery}thesis\isanewline
\ \ \ \ \ \ \ \ \isacommand{using}\isamarkupfalse%
\ {\isacartoucheopen}S{\isadigit{2}}\ {\isasymsubseteq}\ B{\isacartoucheclose}\ \isacommand{by}\isamarkupfalse%
\ {\isacharparenleft}rule\ subexI{\isacharparenright}\isanewline
\ \ \ \ \isacommand{next}\isamarkupfalse%
\isanewline
\ \ \ \ \ \ \isacommand{assume}\isamarkupfalse%
\ {\isachardoublequoteopen}S\ {\isacharequal}\ S{\isadigit{1}}{\isachardoublequoteclose}\isanewline
\ \ \ \ \ \ \isacommand{then}\isamarkupfalse%
\ \isacommand{have}\isamarkupfalse%
\ {\isachardoublequoteopen}S\ {\isacharequal}\ S{\isadigit{2}}{\isachardoublequoteclose}\isanewline
\ \ \ \ \ \ \ \ \isacommand{by}\isamarkupfalse%
\ {\isacharparenleft}simp\ only{\isacharcolon}\ {\isacartoucheopen}S{\isadigit{1}}\ {\isacharequal}\ S{\isadigit{2}}{\isacartoucheclose}{\isacharparenright}\isanewline
\ \ \ \ \ \ \isacommand{then}\isamarkupfalse%
\ \isacommand{have}\isamarkupfalse%
\ {\isachardoublequoteopen}S\ {\isacharequal}\ {\isacharbraceleft}a{\isacharcomma}b{\isacharbraceright}\ {\isasymunion}\ S{\isadigit{2}}\ {\isasymor}\ S\ {\isacharequal}\ {\isacharbraceleft}a{\isacharbraceright}\ {\isasymunion}\ S{\isadigit{2}}\ {\isasymor}\ S\ {\isacharequal}\ {\isacharbraceleft}b{\isacharbraceright}\ {\isasymunion}\ S{\isadigit{2}}\ {\isasymor}\ S\ {\isacharequal}\ S{\isadigit{2}}{\isachardoublequoteclose}\isanewline
\ \ \ \ \ \ \ \ \isacommand{by}\isamarkupfalse%
\ {\isacharparenleft}iprover\ intro{\isacharcolon}\ disjI{\isadigit{1}}{\isacharparenright}\isanewline
\ \ \ \ \ \ \isacommand{then}\isamarkupfalse%
\ \isacommand{have}\isamarkupfalse%
\ {\isachardoublequoteopen}finite\ S{\isadigit{2}}\ {\isasymand}\ {\isacharparenleft}S\ {\isacharequal}\ {\isacharbraceleft}a{\isacharcomma}b{\isacharbraceright}\ {\isasymunion}\ S{\isadigit{2}}\ {\isasymor}\ S\ {\isacharequal}\ {\isacharbraceleft}a{\isacharbraceright}\ {\isasymunion}\ S{\isadigit{2}}\ {\isasymor}\ S\ {\isacharequal}\ {\isacharbraceleft}b{\isacharbraceright}\ {\isasymunion}\ S{\isadigit{2}}\ {\isasymor}\ S\ {\isacharequal}\ S{\isadigit{2}}{\isacharparenright}{\isachardoublequoteclose}\isanewline
\ \ \ \ \ \ \ \ \isacommand{using}\isamarkupfalse%
\ {\isacartoucheopen}finite\ S{\isadigit{2}}{\isacartoucheclose}\ \isacommand{by}\isamarkupfalse%
\ {\isacharparenleft}iprover\ intro{\isacharcolon}\ conjI{\isacharparenright}\isanewline
\ \ \ \ \ \ \isacommand{thus}\isamarkupfalse%
\ {\isacharquery}thesis\isanewline
\ \ \ \ \ \ \ \ \isacommand{using}\isamarkupfalse%
\ {\isacartoucheopen}S{\isadigit{2}}\ {\isasymsubseteq}\ B{\isacartoucheclose}\ \isacommand{by}\isamarkupfalse%
\ {\isacharparenleft}rule\ subexI{\isacharparenright}\isanewline
\ \ \ \ \isacommand{qed}\isamarkupfalse%
\isanewline
\ \ \isacommand{qed}\isamarkupfalse%
\isanewline
\isacommand{qed}\isamarkupfalse%
%
\endisatagproof
{\isafoldproof}%
%
\isadelimproof
%
\endisadelimproof
%
\begin{isamarkuptext}%
Una vez introducidos los resultados anteriores, procedamos con la prueba del \isa{Teorema\ de\ Compacidad}.

  \begin{demostracion}
    Consideremos la colección \isa{C} formada por los conjuntos de fórmulas finitamente satisfacibles.
    Recordemos que un conjunto de fórmulas es finitamente satisfacible si todo subconjunto finito 
    suyo es satisfacible. Vamos a probar que dicho conjunto verifica la propiedad de consistencia 
    proposicional y, por el \isa{Teorema\ de\ Existencia\ de\ Modelo}, quedará probado que todo conjunto de 
    \isa{C} es satisfacible, lo que demuestra el teorema.

    Para probar que \isa{C} verifica la propiedad de consistencia proposicional, por el lema \isa{{\isadigit{2}}{\isachardot}{\isadigit{0}}{\isachardot}{\isadigit{2}}} de 
    caracterización mediante notación uniforme, basta demostrar que se verifican las siguientes 
    condiciones para todo conjunto \isa{W\ {\isasymin}\ C}:
    \begin{itemize}
     \item \isa{{\isasymbottom}\ {\isasymnotin}\ W}.
     \item Dada \isa{p} una fórmula atómica cualquiera, no se tiene 
      simultáneamente que\\ \isa{p\ {\isasymin}\ W} y \isa{{\isasymnot}\ p\ {\isasymin}\ W}.
     \item Para toda fórmula de tipo \isa{{\isasymalpha}} con componentes \isa{{\isasymalpha}\isactrlsub {\isadigit{1}}} y \isa{{\isasymalpha}\isactrlsub {\isadigit{2}}} tal que \isa{{\isasymalpha}}
      pertenece a \isa{W}, se tiene que \isa{{\isacharbraceleft}{\isasymalpha}\isactrlsub {\isadigit{1}}{\isacharcomma}{\isasymalpha}\isactrlsub {\isadigit{2}}{\isacharbraceright}\ {\isasymunion}\ W} pertenece a \isa{C}.
     \item Para toda fórmula de tipo \isa{{\isasymbeta}} con componentes \isa{{\isasymbeta}\isactrlsub {\isadigit{1}}} y \isa{{\isasymbeta}\isactrlsub {\isadigit{2}}} tal que \isa{{\isasymbeta}}
      pertenece a \isa{W}, se tiene que o bien \isa{{\isacharbraceleft}{\isasymbeta}\isactrlsub {\isadigit{1}}{\isacharbraceright}\ {\isasymunion}\ W} pertenece a \isa{C} o 
      bien \isa{{\isacharbraceleft}{\isasymbeta}\isactrlsub {\isadigit{2}}{\isacharbraceright}\ {\isasymunion}\ W} pertenece a \isa{C}.
    \end{itemize}

    De este modo, consideremos un conjunto cualquiera \isa{W\ {\isasymin}\ C} y procedamos a probar cada una de las
    condiciones anteriores.

    La primera condición se demuestra por reducción al absurdo. En efecto, si suponemos que 
    \isa{{\isasymbottom}\ {\isasymin}\ W}, es claro que \isa{{\isacharbraceleft}{\isasymbottom}{\isacharbraceright}} es un subconjunto finito de \isa{W}. Como \isa{W} es un conjunto
    finitamente satisfacible por pertenecer a \isa{C}, se tiene por lo anterior que \isa{{\isacharbraceleft}{\isasymbottom}{\isacharbraceright}} es 
    satisfacible. De este modo, llegamos a una contradicción pues, por definición, no existe ninguna 
    interpretación que sea modelo de \isa{{\isasymbottom}}.

    A continuación probaremos que, si \isa{W\ {\isasymin}\ C}, entonces dada \isa{p} una fórmula atómica cualquiera, no 
    se tiene simultáneamente que \isa{p\ {\isasymin}\ W} y \isa{{\isasymnot}\ p\ {\isasymin}\ W}. Veamos dicho resultado por reducción al 
    absurdo, suponiendo que tanto \isa{p} como \isa{{\isasymnot}\ p} están en \isa{W}. En este caso, \isa{{\isacharbraceleft}p{\isacharcomma}{\isasymnot}\ p{\isacharbraceright}} sería un
    subconjunto finito de \isa{W} y, por ser \isa{W} finitamente satisfacible ya que \isa{W\ {\isasymin}\ C}, obtendríamos 
    que \isa{{\isacharbraceleft}p{\isacharcomma}{\isasymnot}\ p{\isacharbraceright}} es satisfacible. Sin embargo, esto no es cierto ya que, en ese caso, existiría
    una interpretación que sería modelo tanto de \isa{p} como de \isa{{\isasymnot}\ p}, llegando así a una 
    contradicción.
  \end{demostracion}%
\end{isamarkuptext}\isamarkuptrue%
\isacommand{definition}\isamarkupfalse%
\ colecComp\ {\isacharcolon}{\isacharcolon}\ {\isachardoublequoteopen}{\isacharparenleft}{\isacharprime}a\ formula\ set{\isacharparenright}\ set{\isachardoublequoteclose}\isanewline
\ \ \isakeyword{where}\ colecComp{\isacharcolon}\ {\isachardoublequoteopen}colecComp\ {\isacharequal}\ {\isacharbraceleft}W{\isachardot}\ fin{\isacharunderscore}sat\ W{\isacharbraceright}{\isachardoublequoteclose}\isanewline
\isanewline
\isacommand{lemma}\isamarkupfalse%
\ set{\isacharunderscore}in{\isacharunderscore}colecComp{\isacharcolon}\ \isanewline
\ \ \isakeyword{assumes}\ {\isachardoublequoteopen}fin{\isacharunderscore}sat\ S{\isachardoublequoteclose}\isanewline
\ \ \isakeyword{shows}\ {\isachardoublequoteopen}S\ {\isasymin}\ colecComp{\isachardoublequoteclose}\isanewline
%
\isadelimproof
\ \ %
\endisadelimproof
%
\isatagproof
\isacommand{unfolding}\isamarkupfalse%
\ colecComp\ \isacommand{using}\isamarkupfalse%
\ assms\ \isacommand{unfolding}\isamarkupfalse%
\ fin{\isacharunderscore}sat{\isacharunderscore}def\ \isacommand{by}\isamarkupfalse%
\ {\isacharparenleft}rule\ CollectI{\isacharparenright}%
\endisatagproof
{\isafoldproof}%
%
\isadelimproof
\isanewline
%
\endisadelimproof
\isanewline
\isacommand{lemma}\isamarkupfalse%
\ colecComp{\isacharunderscore}subset{\isacharunderscore}finite{\isacharcolon}\ \isanewline
\ \ \isakeyword{assumes}\ {\isachardoublequoteopen}W\ {\isasymin}\ colecComp{\isachardoublequoteclose}\isanewline
\ \ \ \ \ \ \ \ \ \ {\isachardoublequoteopen}Wo\ {\isasymsubseteq}\ W{\isachardoublequoteclose}\isanewline
\ \ \ \ \ \ \ \ \ \ {\isachardoublequoteopen}finite\ Wo{\isachardoublequoteclose}\isanewline
\ \ \isakeyword{shows}\ {\isachardoublequoteopen}sat\ Wo{\isachardoublequoteclose}\ \isanewline
%
\isadelimproof
%
\endisadelimproof
%
\isatagproof
\isacommand{proof}\isamarkupfalse%
\ {\isacharminus}\isanewline
\ \ \isacommand{have}\isamarkupfalse%
\ {\isachardoublequoteopen}{\isasymforall}Wo\ {\isasymsubseteq}\ W{\isachardot}\ finite\ Wo\ {\isasymlongrightarrow}\ sat\ Wo{\isachardoublequoteclose}\isanewline
\ \ \ \ \isacommand{using}\isamarkupfalse%
\ assms{\isacharparenleft}{\isadigit{1}}{\isacharparenright}\ \isacommand{unfolding}\isamarkupfalse%
\ colecComp\ fin{\isacharunderscore}sat{\isacharunderscore}def\ \isacommand{by}\isamarkupfalse%
\ {\isacharparenleft}rule\ CollectD{\isacharparenright}\isanewline
\ \ \isacommand{then}\isamarkupfalse%
\ \isacommand{have}\isamarkupfalse%
\ {\isachardoublequoteopen}finite\ Wo\ {\isasymlongrightarrow}\ sat\ Wo{\isachardoublequoteclose}\isanewline
\ \ \ \ \isacommand{using}\isamarkupfalse%
\ {\isacartoucheopen}Wo\ {\isasymsubseteq}\ W{\isacartoucheclose}\ \isacommand{by}\isamarkupfalse%
\ {\isacharparenleft}rule\ sspec{\isacharparenright}\isanewline
\ \ \isacommand{thus}\isamarkupfalse%
\ {\isachardoublequoteopen}sat\ Wo{\isachardoublequoteclose}\isanewline
\ \ \ \ \isacommand{using}\isamarkupfalse%
\ {\isacartoucheopen}finite\ Wo{\isacartoucheclose}\ \isacommand{by}\isamarkupfalse%
\ {\isacharparenleft}rule\ mp{\isacharparenright}\isanewline
\isacommand{qed}\isamarkupfalse%
%
\endisatagproof
{\isafoldproof}%
%
\isadelimproof
\isanewline
%
\endisadelimproof
\isanewline
\isacommand{lemma}\isamarkupfalse%
\ not{\isacharunderscore}sat{\isacharunderscore}bot{\isacharcolon}\ {\isachardoublequoteopen}{\isasymnot}\ sat\ {\isacharbraceleft}{\isasymbottom}{\isacharbraceright}{\isachardoublequoteclose}\isanewline
%
\isadelimproof
%
\endisadelimproof
%
\isatagproof
\isacommand{proof}\isamarkupfalse%
\ {\isacharparenleft}rule\ ccontr{\isacharparenright}\isanewline
\ \ \isacommand{assume}\isamarkupfalse%
\ {\isachardoublequoteopen}{\isasymnot}{\isacharparenleft}{\isasymnot}sat{\isacharbraceleft}{\isasymbottom}\ {\isacharcolon}{\isacharcolon}\ {\isacharprime}a\ formula{\isacharbraceright}{\isacharparenright}{\isachardoublequoteclose}\isanewline
\ \ \isacommand{then}\isamarkupfalse%
\ \isacommand{have}\isamarkupfalse%
\ {\isachardoublequoteopen}sat\ {\isacharbraceleft}{\isasymbottom}\ {\isacharcolon}{\isacharcolon}\ {\isacharprime}a\ formula{\isacharbraceright}{\isachardoublequoteclose}\isanewline
\ \ \ \ \isacommand{by}\isamarkupfalse%
\ {\isacharparenleft}rule\ notnotD{\isacharparenright}\isanewline
\ \ \isacommand{then}\isamarkupfalse%
\ \isacommand{have}\isamarkupfalse%
\ Ex{\isacharcolon}{\isachardoublequoteopen}{\isasymexists}{\isasymA}{\isachardot}\ {\isasymforall}F\ {\isasymin}\ {\isacharbraceleft}{\isasymbottom}\ {\isacharcolon}{\isacharcolon}\ {\isacharprime}a\ formula{\isacharbraceright}{\isachardot}\ {\isasymA}\ {\isasymTurnstile}\ F{\isachardoublequoteclose}\isanewline
\ \ \ \ \isacommand{by}\isamarkupfalse%
\ {\isacharparenleft}simp\ only{\isacharcolon}\ sat{\isacharunderscore}def{\isacharparenright}\isanewline
\ \ \isacommand{obtain}\isamarkupfalse%
\ {\isasymA}\ \isakeyword{where}\ {\isadigit{1}}{\isacharcolon}{\isachardoublequoteopen}{\isasymforall}F\ {\isasymin}\ {\isacharbraceleft}{\isasymbottom}\ {\isacharcolon}{\isacharcolon}\ {\isacharprime}a\ formula{\isacharbraceright}{\isachardot}\ {\isasymA}\ {\isasymTurnstile}\ F{\isachardoublequoteclose}\isanewline
\ \ \ \ \isacommand{using}\isamarkupfalse%
\ Ex\ \isacommand{by}\isamarkupfalse%
\ {\isacharparenleft}rule\ exE{\isacharparenright}\isanewline
\ \ \isacommand{have}\isamarkupfalse%
\ {\isadigit{2}}{\isacharcolon}{\isachardoublequoteopen}{\isasymbottom}\ {\isasymin}\ {\isacharbraceleft}{\isasymbottom}{\isacharcolon}{\isacharcolon}\ {\isacharprime}a\ formula{\isacharbraceright}{\isachardoublequoteclose}\isanewline
\ \ \ \ \isacommand{by}\isamarkupfalse%
\ {\isacharparenleft}simp\ only{\isacharcolon}\ singletonI{\isacharparenright}\isanewline
\ \ \isacommand{have}\isamarkupfalse%
\ {\isachardoublequoteopen}{\isasymA}\ {\isasymTurnstile}\ {\isasymbottom}{\isachardoublequoteclose}\isanewline
\ \ \ \ \isacommand{using}\isamarkupfalse%
\ {\isadigit{1}}\ {\isadigit{2}}\ \isacommand{by}\isamarkupfalse%
\ {\isacharparenleft}rule\ bspec{\isacharparenright}\isanewline
\ \ \isacommand{thus}\isamarkupfalse%
\ {\isachardoublequoteopen}False{\isachardoublequoteclose}\isanewline
\ \ \ \ \isacommand{by}\isamarkupfalse%
\ {\isacharparenleft}simp\ only{\isacharcolon}\ formula{\isacharunderscore}semantics{\isachardot}simps{\isacharparenleft}{\isadigit{2}}{\isacharparenright}{\isacharparenright}\isanewline
\isacommand{qed}\isamarkupfalse%
%
\endisatagproof
{\isafoldproof}%
%
\isadelimproof
\isanewline
%
\endisadelimproof
\isanewline
\isacommand{lemma}\isamarkupfalse%
\ pcp{\isacharunderscore}colecComp{\isacharunderscore}bot{\isacharcolon}\isanewline
\ \ \isakeyword{assumes}\ {\isachardoublequoteopen}W\ {\isasymin}\ colecComp{\isachardoublequoteclose}\isanewline
\ \ \isakeyword{shows}\ {\isachardoublequoteopen}{\isasymbottom}\ {\isasymnotin}\ W{\isachardoublequoteclose}\isanewline
%
\isadelimproof
%
\endisadelimproof
%
\isatagproof
\isacommand{proof}\isamarkupfalse%
\ {\isacharparenleft}rule\ ccontr{\isacharparenright}\isanewline
\ \ \isacommand{assume}\isamarkupfalse%
\ {\isachardoublequoteopen}{\isasymnot}{\isacharparenleft}{\isasymbottom}\ {\isasymnotin}\ W{\isacharparenright}{\isachardoublequoteclose}\isanewline
\ \ \isacommand{then}\isamarkupfalse%
\ \isacommand{have}\isamarkupfalse%
\ {\isachardoublequoteopen}{\isasymbottom}\ {\isasymin}\ W{\isachardoublequoteclose}\isanewline
\ \ \ \ \isacommand{by}\isamarkupfalse%
\ {\isacharparenleft}rule\ notnotD{\isacharparenright}\isanewline
\ \ \isacommand{have}\isamarkupfalse%
\ {\isachardoublequoteopen}{\isacharbraceleft}{\isacharbraceright}\ {\isasymsubseteq}\ W{\isachardoublequoteclose}\ \isanewline
\ \ \ \ \isacommand{by}\isamarkupfalse%
\ {\isacharparenleft}simp\ only{\isacharcolon}\ empty{\isacharunderscore}subsetI{\isacharparenright}\ \isanewline
\ \ \isacommand{have}\isamarkupfalse%
\ {\isachardoublequoteopen}{\isasymbottom}\ {\isasymin}\ W\ {\isasymand}\ {\isacharbraceleft}{\isacharbraceright}\ {\isasymsubseteq}\ W{\isachardoublequoteclose}\isanewline
\ \ \ \ \isacommand{using}\isamarkupfalse%
\ {\isacartoucheopen}{\isasymbottom}\ {\isasymin}\ W{\isacartoucheclose}\ {\isacartoucheopen}{\isacharbraceleft}{\isacharbraceright}\ {\isasymsubseteq}\ W{\isacartoucheclose}\ \isacommand{by}\isamarkupfalse%
\ {\isacharparenleft}rule\ conjI{\isacharparenright}\isanewline
\ \ \isacommand{then}\isamarkupfalse%
\ \isacommand{have}\isamarkupfalse%
\ {\isachardoublequoteopen}{\isacharbraceleft}{\isasymbottom}{\isacharbraceright}\ {\isasymsubseteq}\ W{\isachardoublequoteclose}\isanewline
\ \ \ \ \isacommand{by}\isamarkupfalse%
\ {\isacharparenleft}simp\ only{\isacharcolon}\ insert{\isacharunderscore}subset{\isacharparenright}\isanewline
\ \ \isacommand{have}\isamarkupfalse%
\ {\isachardoublequoteopen}finite\ {\isacharbraceleft}{\isasymbottom}{\isacharbraceright}{\isachardoublequoteclose}\ \isanewline
\ \ \ \ \isacommand{by}\isamarkupfalse%
\ {\isacharparenleft}simp\ only{\isacharcolon}\ finite{\isachardot}emptyI\ finite{\isacharunderscore}insert{\isacharparenright}\isanewline
\ \ \isacommand{have}\isamarkupfalse%
\ {\isachardoublequoteopen}sat\ {\isacharbraceleft}{\isasymbottom}\ {\isacharcolon}{\isacharcolon}\ {\isacharprime}a\ formula{\isacharbraceright}{\isachardoublequoteclose}\ \isanewline
\ \ \ \ \isacommand{using}\isamarkupfalse%
\ assms\ {\isacartoucheopen}{\isacharbraceleft}{\isasymbottom}{\isacharbraceright}\ {\isasymsubseteq}\ W{\isacartoucheclose}\ {\isacartoucheopen}finite\ {\isacharbraceleft}{\isasymbottom}{\isacharbraceright}{\isacartoucheclose}\ \isacommand{by}\isamarkupfalse%
\ {\isacharparenleft}rule\ colecComp{\isacharunderscore}subset{\isacharunderscore}finite{\isacharparenright}\isanewline
\ \ \isacommand{have}\isamarkupfalse%
\ {\isachardoublequoteopen}{\isasymnot}\ sat\ {\isacharbraceleft}{\isasymbottom}\ {\isacharcolon}{\isacharcolon}\ {\isacharprime}a\ formula{\isacharbraceright}{\isachardoublequoteclose}\ \isanewline
\ \ \ \ \isacommand{by}\isamarkupfalse%
\ {\isacharparenleft}rule\ not{\isacharunderscore}sat{\isacharunderscore}bot{\isacharparenright}\isanewline
\ \ \isacommand{then}\isamarkupfalse%
\ \isacommand{show}\isamarkupfalse%
\ False\ \isanewline
\ \ \ \ \isacommand{using}\isamarkupfalse%
\ {\isacartoucheopen}sat\ {\isacharbraceleft}{\isasymbottom}\ {\isacharcolon}{\isacharcolon}\ {\isacharprime}a\ formula{\isacharbraceright}{\isacartoucheclose}\ \isacommand{by}\isamarkupfalse%
\ {\isacharparenleft}rule\ notE{\isacharparenright}\isanewline
\isacommand{qed}\isamarkupfalse%
%
\endisatagproof
{\isafoldproof}%
%
\isadelimproof
\isanewline
%
\endisadelimproof
\isanewline
\isacommand{lemma}\isamarkupfalse%
\ not{\isacharunderscore}sat{\isacharunderscore}atoms{\isacharcolon}\ {\isachardoublequoteopen}{\isasymnot}\ sat{\isacharparenleft}{\isacharbraceleft}Atom\ k{\isacharcomma}\ \isactrlbold {\isasymnot}\ {\isacharparenleft}Atom\ k{\isacharparenright}{\isacharbraceright}{\isacharparenright}{\isachardoublequoteclose}\isanewline
%
\isadelimproof
%
\endisadelimproof
%
\isatagproof
\isacommand{proof}\isamarkupfalse%
\ {\isacharparenleft}rule\ ccontr{\isacharparenright}\isanewline
\ \ \isacommand{assume}\isamarkupfalse%
\ {\isachardoublequoteopen}{\isasymnot}\ {\isasymnot}\ sat{\isacharparenleft}{\isacharbraceleft}Atom\ k{\isacharcomma}\ \isactrlbold {\isasymnot}\ {\isacharparenleft}Atom\ k{\isacharparenright}{\isacharbraceright}{\isacharparenright}{\isachardoublequoteclose}\isanewline
\ \ \isacommand{then}\isamarkupfalse%
\ \isacommand{have}\isamarkupfalse%
\ {\isachardoublequoteopen}sat{\isacharparenleft}{\isacharbraceleft}Atom\ k{\isacharcomma}\ \isactrlbold {\isasymnot}\ {\isacharparenleft}Atom\ k{\isacharparenright}{\isacharbraceright}{\isacharparenright}{\isachardoublequoteclose}\isanewline
\ \ \ \ \isacommand{by}\isamarkupfalse%
\ {\isacharparenleft}rule\ notnotD{\isacharparenright}\isanewline
\ \ \isacommand{then}\isamarkupfalse%
\ \isacommand{have}\isamarkupfalse%
\ Sat{\isacharcolon}{\isachardoublequoteopen}{\isasymexists}{\isasymA}{\isachardot}\ {\isasymforall}F\ {\isasymin}\ {\isacharbraceleft}Atom\ k{\isacharcomma}\ \isactrlbold {\isasymnot}{\isacharparenleft}Atom\ k{\isacharparenright}{\isacharbraceright}{\isachardot}\ {\isasymA}\ {\isasymTurnstile}\ F{\isachardoublequoteclose}\isanewline
\ \ \ \ \isacommand{by}\isamarkupfalse%
\ {\isacharparenleft}simp\ only{\isacharcolon}\ sat{\isacharunderscore}def{\isacharparenright}\isanewline
\ \ \isacommand{obtain}\isamarkupfalse%
\ {\isasymA}\ \isakeyword{where}\ H{\isacharcolon}{\isachardoublequoteopen}{\isasymforall}F\ {\isasymin}\ {\isacharbraceleft}Atom\ k{\isacharcomma}\ \isactrlbold {\isasymnot}{\isacharparenleft}Atom\ k{\isacharparenright}{\isacharbraceright}{\isachardot}\ {\isasymA}\ {\isasymTurnstile}\ F{\isachardoublequoteclose}\isanewline
\ \ \ \ \isacommand{using}\isamarkupfalse%
\ Sat\ \isacommand{by}\isamarkupfalse%
\ {\isacharparenleft}rule\ exE{\isacharparenright}\isanewline
\ \ \isacommand{have}\isamarkupfalse%
\ {\isachardoublequoteopen}Atom\ k\ {\isasymin}\ {\isacharbraceleft}Atom\ k{\isacharcomma}\ \isactrlbold {\isasymnot}{\isacharparenleft}Atom\ k{\isacharparenright}{\isacharbraceright}{\isachardoublequoteclose}\isanewline
\ \ \ \ \isacommand{by}\isamarkupfalse%
\ simp\isanewline
\ \ \isacommand{have}\isamarkupfalse%
\ {\isachardoublequoteopen}{\isasymA}\ {\isasymTurnstile}\ Atom\ k{\isachardoublequoteclose}\isanewline
\ \ \ \ \isacommand{using}\isamarkupfalse%
\ H\ {\isacartoucheopen}Atom\ k\ {\isasymin}\ {\isacharbraceleft}Atom\ k{\isacharcomma}\ \isactrlbold {\isasymnot}{\isacharparenleft}Atom\ k{\isacharparenright}{\isacharbraceright}{\isacartoucheclose}\ \isacommand{by}\isamarkupfalse%
\ {\isacharparenleft}rule\ bspec{\isacharparenright}\isanewline
\ \ \isacommand{have}\isamarkupfalse%
\ {\isachardoublequoteopen}\isactrlbold {\isasymnot}{\isacharparenleft}Atom\ k{\isacharparenright}\ {\isasymin}\ {\isacharbraceleft}Atom\ k{\isacharcomma}\ \isactrlbold {\isasymnot}{\isacharparenleft}Atom\ k{\isacharparenright}{\isacharbraceright}{\isachardoublequoteclose}\isanewline
\ \ \ \ \isacommand{by}\isamarkupfalse%
\ simp\isanewline
\ \ \isacommand{have}\isamarkupfalse%
\ {\isachardoublequoteopen}{\isasymA}\ {\isasymTurnstile}\ \isactrlbold {\isasymnot}{\isacharparenleft}Atom\ k{\isacharparenright}{\isachardoublequoteclose}\isanewline
\ \ \ \ \isacommand{using}\isamarkupfalse%
\ H\ {\isacartoucheopen}\isactrlbold {\isasymnot}{\isacharparenleft}Atom\ k{\isacharparenright}\ {\isasymin}\ {\isacharbraceleft}Atom\ k{\isacharcomma}\ \isactrlbold {\isasymnot}{\isacharparenleft}Atom\ k{\isacharparenright}{\isacharbraceright}{\isacartoucheclose}\ \isacommand{by}\isamarkupfalse%
\ {\isacharparenleft}rule\ bspec{\isacharparenright}\isanewline
\ \ \isacommand{then}\isamarkupfalse%
\ \isacommand{have}\isamarkupfalse%
\ {\isachardoublequoteopen}{\isasymnot}\ {\isasymA}\ {\isasymTurnstile}\ Atom\ k{\isachardoublequoteclose}\ \isanewline
\ \ \ \ \isacommand{by}\isamarkupfalse%
\ {\isacharparenleft}simp\ only{\isacharcolon}\ simp{\isacharunderscore}thms{\isacharparenleft}{\isadigit{8}}{\isacharparenright}\ formula{\isacharunderscore}semantics{\isachardot}simps{\isacharparenleft}{\isadigit{3}}{\isacharparenright}{\isacharparenright}\isanewline
\ \ \isacommand{thus}\isamarkupfalse%
\ {\isachardoublequoteopen}False{\isachardoublequoteclose}\isanewline
\ \ \ \ \isacommand{using}\isamarkupfalse%
\ {\isacartoucheopen}{\isasymA}\ {\isasymTurnstile}\ Atom\ k{\isacartoucheclose}\ \isacommand{by}\isamarkupfalse%
\ {\isacharparenleft}rule\ notE{\isacharparenright}\isanewline
\isacommand{qed}\isamarkupfalse%
%
\endisatagproof
{\isafoldproof}%
%
\isadelimproof
\isanewline
%
\endisadelimproof
\isanewline
\isacommand{lemma}\isamarkupfalse%
\ pcp{\isacharunderscore}colecComp{\isacharunderscore}atoms{\isacharcolon}\isanewline
\ \ \isakeyword{assumes}\ {\isachardoublequoteopen}W\ {\isasymin}\ colecComp{\isachardoublequoteclose}\isanewline
\ \ \isakeyword{shows}\ {\isachardoublequoteopen}{\isasymforall}k{\isachardot}\ Atom\ k\ {\isasymin}\ W\ {\isasymlongrightarrow}\ \isactrlbold {\isasymnot}\ {\isacharparenleft}Atom\ k{\isacharparenright}\ {\isasymin}\ W\ {\isasymlongrightarrow}\ False{\isachardoublequoteclose}\isanewline
%
\isadelimproof
%
\endisadelimproof
%
\isatagproof
\isacommand{proof}\isamarkupfalse%
\ {\isacharparenleft}rule\ allI{\isacharparenright}\isanewline
\ \ \isacommand{fix}\isamarkupfalse%
\ k\isanewline
\ \ \isacommand{show}\isamarkupfalse%
\ {\isachardoublequoteopen}Atom\ k\ {\isasymin}\ W\ {\isasymlongrightarrow}\ \isactrlbold {\isasymnot}\ {\isacharparenleft}Atom\ k{\isacharparenright}\ {\isasymin}\ W\ {\isasymlongrightarrow}\ False{\isachardoublequoteclose}\isanewline
\ \ \isacommand{proof}\isamarkupfalse%
\ {\isacharparenleft}rule\ impI{\isacharparenright}{\isacharplus}\isanewline
\ \ \ \ \isacommand{assume}\isamarkupfalse%
\ {\isadigit{1}}{\isacharcolon}{\isachardoublequoteopen}Atom\ k\ {\isasymin}\ W{\isachardoublequoteclose}\isanewline
\ \ \ \ \isacommand{assume}\isamarkupfalse%
\ {\isadigit{2}}{\isacharcolon}{\isachardoublequoteopen}\isactrlbold {\isasymnot}\ {\isacharparenleft}Atom\ k{\isacharparenright}\ {\isasymin}\ W{\isachardoublequoteclose}\isanewline
\ \ \ \ \isacommand{have}\isamarkupfalse%
\ {\isachardoublequoteopen}{\isacharbraceleft}{\isacharbraceright}\ {\isasymsubseteq}\ W{\isachardoublequoteclose}\isanewline
\ \ \ \ \ \ \isacommand{by}\isamarkupfalse%
\ {\isacharparenleft}simp\ only{\isacharcolon}\ empty{\isacharunderscore}subsetI{\isacharparenright}\ \isanewline
\ \ \ \ \isacommand{have}\isamarkupfalse%
\ {\isachardoublequoteopen}Atom\ k\ {\isasymin}\ W\ {\isasymand}\ {\isacharbraceleft}{\isacharbraceright}\ {\isasymsubseteq}\ W{\isachardoublequoteclose}\isanewline
\ \ \ \ \ \ \isacommand{using}\isamarkupfalse%
\ {\isadigit{1}}\ {\isacartoucheopen}{\isacharbraceleft}{\isacharbraceright}\ {\isasymsubseteq}\ W{\isacartoucheclose}\ \isacommand{by}\isamarkupfalse%
\ {\isacharparenleft}rule\ conjI{\isacharparenright}\isanewline
\ \ \ \ \isacommand{then}\isamarkupfalse%
\ \isacommand{have}\isamarkupfalse%
\ {\isachardoublequoteopen}{\isacharbraceleft}Atom\ k{\isacharbraceright}\ {\isasymsubseteq}\ W{\isachardoublequoteclose}\isanewline
\ \ \ \ \ \ \isacommand{by}\isamarkupfalse%
\ {\isacharparenleft}simp\ only{\isacharcolon}\ insert{\isacharunderscore}subset{\isacharparenright}\isanewline
\ \ \ \ \isacommand{have}\isamarkupfalse%
\ {\isachardoublequoteopen}\isactrlbold {\isasymnot}\ {\isacharparenleft}Atom\ k{\isacharparenright}\ {\isasymin}\ W\ {\isasymand}\ {\isacharbraceleft}{\isacharbraceright}\ {\isasymsubseteq}\ W{\isachardoublequoteclose}\isanewline
\ \ \ \ \ \ \isacommand{using}\isamarkupfalse%
\ {\isadigit{2}}\ {\isacartoucheopen}{\isacharbraceleft}{\isacharbraceright}\ {\isasymsubseteq}\ W{\isacartoucheclose}\ \isacommand{by}\isamarkupfalse%
\ {\isacharparenleft}rule\ conjI{\isacharparenright}\isanewline
\ \ \ \ \isacommand{then}\isamarkupfalse%
\ \isacommand{have}\isamarkupfalse%
\ {\isachardoublequoteopen}{\isacharbraceleft}\isactrlbold {\isasymnot}{\isacharparenleft}Atom\ k{\isacharparenright}{\isacharbraceright}\ {\isasymsubseteq}\ W{\isachardoublequoteclose}\isanewline
\ \ \ \ \ \ \isacommand{by}\isamarkupfalse%
\ {\isacharparenleft}simp\ only{\isacharcolon}\ insert{\isacharunderscore}subset{\isacharparenright}\isanewline
\ \ \ \ \isacommand{have}\isamarkupfalse%
\ {\isachardoublequoteopen}{\isacharbraceleft}Atom\ k{\isacharbraceright}\ {\isasymunion}\ {\isacharbraceleft}\isactrlbold {\isasymnot}{\isacharparenleft}Atom\ k{\isacharparenright}{\isacharbraceright}\ {\isasymsubseteq}\ W{\isachardoublequoteclose}\isanewline
\ \ \ \ \ \ \isacommand{using}\isamarkupfalse%
\ {\isacartoucheopen}{\isacharbraceleft}Atom\ k{\isacharbraceright}\ {\isasymsubseteq}\ W{\isacartoucheclose}\ {\isacartoucheopen}{\isacharbraceleft}\isactrlbold {\isasymnot}{\isacharparenleft}Atom\ k{\isacharparenright}{\isacharbraceright}\ {\isasymsubseteq}\ W{\isacartoucheclose}\ \isacommand{by}\isamarkupfalse%
\ {\isacharparenleft}simp\ only{\isacharcolon}\ Un{\isacharunderscore}least{\isacharparenright}\isanewline
\ \ \ \ \isacommand{then}\isamarkupfalse%
\ \isacommand{have}\isamarkupfalse%
\ {\isachardoublequoteopen}{\isacharbraceleft}Atom\ k{\isacharcomma}\ \isactrlbold {\isasymnot}{\isacharparenleft}Atom\ k{\isacharparenright}{\isacharbraceright}\ {\isasymsubseteq}\ W{\isachardoublequoteclose}\isanewline
\ \ \ \ \ \ \isacommand{by}\isamarkupfalse%
\ simp\ \isanewline
\ \ \ \ \isacommand{have}\isamarkupfalse%
\ {\isachardoublequoteopen}finite\ {\isacharbraceleft}Atom\ k{\isacharcomma}\ \isactrlbold {\isasymnot}{\isacharparenleft}Atom\ k{\isacharparenright}{\isacharbraceright}{\isachardoublequoteclose}\isanewline
\ \ \ \ \ \ \isacommand{by}\isamarkupfalse%
\ blast\isanewline
\ \ \ \ \isacommand{have}\isamarkupfalse%
\ {\isachardoublequoteopen}sat\ {\isacharparenleft}{\isacharbraceleft}Atom\ k{\isacharcomma}\ \isactrlbold {\isasymnot}{\isacharparenleft}Atom\ k{\isacharparenright}{\isacharbraceright}{\isacharparenright}{\isachardoublequoteclose}\isanewline
\ \ \ \ \ \ \isacommand{using}\isamarkupfalse%
\ assms\ {\isacartoucheopen}{\isacharbraceleft}Atom\ k{\isacharcomma}\ \isactrlbold {\isasymnot}{\isacharparenleft}Atom\ k{\isacharparenright}{\isacharbraceright}\ {\isasymsubseteq}\ W{\isacartoucheclose}\ {\isacartoucheopen}finite\ {\isacharbraceleft}Atom\ k{\isacharcomma}\ \isactrlbold {\isasymnot}{\isacharparenleft}Atom\ k{\isacharparenright}{\isacharbraceright}{\isacartoucheclose}\ \isacommand{by}\isamarkupfalse%
\ {\isacharparenleft}rule\ colecComp{\isacharunderscore}subset{\isacharunderscore}finite{\isacharparenright}\isanewline
\ \ \ \ \isacommand{have}\isamarkupfalse%
\ {\isachardoublequoteopen}{\isasymnot}\ sat\ {\isacharparenleft}{\isacharbraceleft}Atom\ k{\isacharcomma}\ \isactrlbold {\isasymnot}{\isacharparenleft}Atom\ k{\isacharparenright}{\isacharbraceright}{\isacharparenright}{\isachardoublequoteclose}\isanewline
\ \ \ \ \ \ \isacommand{by}\isamarkupfalse%
\ {\isacharparenleft}rule\ not{\isacharunderscore}sat{\isacharunderscore}atoms{\isacharparenright}\isanewline
\ \ \ \ \isacommand{thus}\isamarkupfalse%
\ {\isachardoublequoteopen}False{\isachardoublequoteclose}\isanewline
\ \ \ \ \ \ \isacommand{using}\isamarkupfalse%
\ {\isacartoucheopen}sat\ {\isacharparenleft}{\isacharbraceleft}Atom\ k{\isacharcomma}\ \isactrlbold {\isasymnot}{\isacharparenleft}Atom\ k{\isacharparenright}{\isacharbraceright}{\isacharparenright}{\isacartoucheclose}\ \isacommand{by}\isamarkupfalse%
\ {\isacharparenleft}rule\ notE{\isacharparenright}\isanewline
\ \ \isacommand{qed}\isamarkupfalse%
\isanewline
\isacommand{qed}\isamarkupfalse%
%
\endisatagproof
{\isafoldproof}%
%
\isadelimproof
\isanewline
%
\endisadelimproof
\isanewline
\isacommand{lemma}\isamarkupfalse%
\ pcp{\isacharunderscore}colecComp{\isacharunderscore}elem{\isacharunderscore}sat{\isacharcolon}\isanewline
\ \ \isakeyword{assumes}\ {\isachardoublequoteopen}W\ {\isasymin}\ colecComp{\isachardoublequoteclose}\isanewline
\ \ \ \ \ \ \ \ \ \ {\isachardoublequoteopen}F\ {\isasymin}\ W{\isachardoublequoteclose}\isanewline
\ \ \ \ \ \ \ \ \ \ {\isachardoublequoteopen}finite\ Wo{\isachardoublequoteclose}\isanewline
\ \ \ \ \ \ \ \ \ \ {\isachardoublequoteopen}Wo\ {\isasymsubseteq}\ W{\isachardoublequoteclose}\isanewline
\ \ \ \ \ \ \ \ \isakeyword{shows}\ {\isachardoublequoteopen}sat\ {\isacharparenleft}{\isacharbraceleft}F{\isacharbraceright}\ {\isasymunion}\ Wo{\isacharparenright}{\isachardoublequoteclose}\isanewline
%
\isadelimproof
%
\endisadelimproof
%
\isatagproof
\isacommand{proof}\isamarkupfalse%
\ {\isacharminus}\isanewline
\ \ \isacommand{have}\isamarkupfalse%
\ {\isadigit{1}}{\isacharcolon}{\isachardoublequoteopen}insert\ F\ Wo\ {\isacharequal}\ {\isacharbraceleft}F{\isacharbraceright}\ {\isasymunion}\ Wo{\isachardoublequoteclose}\isanewline
\ \ \ \ \isacommand{by}\isamarkupfalse%
\ {\isacharparenleft}rule\ insert{\isacharunderscore}is{\isacharunderscore}Un{\isacharparenright}\isanewline
\ \ \isacommand{have}\isamarkupfalse%
\ {\isachardoublequoteopen}finite\ {\isacharparenleft}insert\ F\ Wo{\isacharparenright}{\isachardoublequoteclose}\isanewline
\ \ \ \ \isacommand{using}\isamarkupfalse%
\ assms{\isacharparenleft}{\isadigit{3}}{\isacharparenright}\ \isacommand{by}\isamarkupfalse%
\ {\isacharparenleft}simp\ only{\isacharcolon}\ finite{\isacharunderscore}insert{\isacharparenright}\isanewline
\ \ \isacommand{then}\isamarkupfalse%
\ \isacommand{have}\isamarkupfalse%
\ {\isachardoublequoteopen}finite\ {\isacharparenleft}{\isacharbraceleft}F{\isacharbraceright}\ {\isasymunion}\ Wo{\isacharparenright}{\isachardoublequoteclose}\isanewline
\ \ \ \ \isacommand{by}\isamarkupfalse%
\ {\isacharparenleft}simp\ only{\isacharcolon}\ {\isadigit{1}}{\isacharparenright}\ \isanewline
\ \ \isacommand{have}\isamarkupfalse%
\ {\isachardoublequoteopen}F\ {\isasymin}\ W\ {\isasymand}\ Wo\ {\isasymsubseteq}\ W{\isachardoublequoteclose}\isanewline
\ \ \ \ \isacommand{using}\isamarkupfalse%
\ assms{\isacharparenleft}{\isadigit{2}}{\isacharparenright}\ assms{\isacharparenleft}{\isadigit{4}}{\isacharparenright}\ \isacommand{by}\isamarkupfalse%
\ {\isacharparenleft}rule\ conjI{\isacharparenright}\isanewline
\ \ \isacommand{then}\isamarkupfalse%
\ \isacommand{have}\isamarkupfalse%
\ {\isachardoublequoteopen}insert\ F\ Wo\ {\isasymsubseteq}\ W{\isachardoublequoteclose}\isanewline
\ \ \ \ \isacommand{by}\isamarkupfalse%
\ {\isacharparenleft}simp\ only{\isacharcolon}\ insert{\isacharunderscore}subset{\isacharparenright}\isanewline
\ \ \isacommand{then}\isamarkupfalse%
\ \isacommand{have}\isamarkupfalse%
\ {\isachardoublequoteopen}{\isacharbraceleft}F{\isacharbraceright}\ {\isasymunion}\ Wo\ {\isasymsubseteq}\ W{\isachardoublequoteclose}\isanewline
\ \ \ \ \isacommand{by}\isamarkupfalse%
\ {\isacharparenleft}simp\ only{\isacharcolon}\ {\isadigit{1}}{\isacharparenright}\isanewline
\ \ \isacommand{show}\isamarkupfalse%
\ {\isachardoublequoteopen}sat\ {\isacharparenleft}{\isacharbraceleft}F{\isacharbraceright}\ {\isasymunion}\ Wo{\isacharparenright}{\isachardoublequoteclose}\isanewline
\ \ \ \ \isacommand{using}\isamarkupfalse%
\ assms{\isacharparenleft}{\isadigit{1}}{\isacharparenright}\ {\isacartoucheopen}{\isacharbraceleft}F{\isacharbraceright}\ {\isasymunion}\ Wo\ {\isasymsubseteq}\ W{\isacartoucheclose}\ {\isacartoucheopen}finite\ {\isacharparenleft}{\isacharbraceleft}F{\isacharbraceright}\ {\isasymunion}\ Wo{\isacharparenright}{\isacartoucheclose}\ \isacommand{by}\isamarkupfalse%
\ {\isacharparenleft}rule\ colecComp{\isacharunderscore}subset{\isacharunderscore}finite{\isacharparenright}\isanewline
\isacommand{qed}\isamarkupfalse%
%
\endisatagproof
{\isafoldproof}%
%
\isadelimproof
\isanewline
%
\endisadelimproof
\isanewline
\isacommand{lemma}\isamarkupfalse%
\ ball{\isacharunderscore}Un{\isacharcolon}\ \isanewline
\ \ \isakeyword{assumes}\ {\isachardoublequoteopen}{\isasymforall}x\ {\isasymin}\ A{\isachardot}\ P\ x{\isachardoublequoteclose}\isanewline
\ \ \ \ \ \ \ \ \ \ {\isachardoublequoteopen}{\isasymforall}x\ {\isasymin}\ B{\isachardot}\ P\ x{\isachardoublequoteclose}\isanewline
\ \ \ \ \ \ \ \ \isakeyword{shows}\ {\isachardoublequoteopen}{\isasymforall}x\ {\isasymin}\ {\isacharparenleft}A\ {\isasymunion}\ B{\isacharparenright}{\isachardot}\ P\ x{\isachardoublequoteclose}\ \isanewline
%
\isadelimproof
\ \ %
\endisadelimproof
%
\isatagproof
\isacommand{using}\isamarkupfalse%
\ assms\ \isacommand{by}\isamarkupfalse%
\ blast%
\endisatagproof
{\isafoldproof}%
%
\isadelimproof
\isanewline
%
\endisadelimproof
\isanewline
\isacommand{lemma}\isamarkupfalse%
\ pcp{\isacharunderscore}colecComp{\isacharunderscore}CON{\isacharunderscore}sat{\isadigit{1}}{\isacharcolon}\isanewline
\ \ \isakeyword{assumes}\ {\isachardoublequoteopen}W\ {\isasymin}\ colecComp{\isachardoublequoteclose}\isanewline
\ \ \ \ \ \ \ \ \ \ {\isachardoublequoteopen}F\ {\isacharequal}\ G\ \isactrlbold {\isasymand}\ H{\isachardoublequoteclose}\isanewline
\ \ \ \ \ \ \ \ \ \ {\isachardoublequoteopen}F\ {\isasymin}\ W{\isachardoublequoteclose}\isanewline
\ \ \ \ \ \ \ \ \ \ {\isachardoublequoteopen}finite\ Wo{\isachardoublequoteclose}\isanewline
\ \ \ \ \ \ \ \ \ \ {\isachardoublequoteopen}Wo\ {\isasymsubseteq}\ W{\isachardoublequoteclose}\isanewline
\ \ \ \ \ \ \ \ \isakeyword{shows}\ {\isachardoublequoteopen}sat\ {\isacharparenleft}{\isacharbraceleft}G{\isacharcomma}H{\isacharcomma}F{\isacharbraceright}\ {\isasymunion}\ Wo{\isacharparenright}{\isachardoublequoteclose}\isanewline
%
\isadelimproof
%
\endisadelimproof
%
\isatagproof
\isacommand{proof}\isamarkupfalse%
\ {\isacharminus}\isanewline
\ \ \isacommand{have}\isamarkupfalse%
\ {\isachardoublequoteopen}sat\ {\isacharparenleft}{\isacharbraceleft}F{\isacharbraceright}\ {\isasymunion}\ Wo{\isacharparenright}{\isachardoublequoteclose}\isanewline
\ \ \ \ \isacommand{using}\isamarkupfalse%
\ assms{\isacharparenleft}{\isadigit{1}}{\isacharcomma}{\isadigit{3}}{\isacharcomma}{\isadigit{4}}{\isacharcomma}{\isadigit{5}}{\isacharparenright}\ \isacommand{by}\isamarkupfalse%
\ {\isacharparenleft}rule\ pcp{\isacharunderscore}colecComp{\isacharunderscore}elem{\isacharunderscore}sat{\isacharparenright}\isanewline
\ \ \isacommand{have}\isamarkupfalse%
\ {\isachardoublequoteopen}F\ {\isasymin}\ {\isacharbraceleft}F{\isacharbraceright}\ {\isasymunion}\ Wo{\isachardoublequoteclose}\isanewline
\ \ \ \ \isacommand{by}\isamarkupfalse%
\ {\isacharparenleft}simp\ add{\isacharcolon}\ insertI{\isadigit{1}}{\isacharparenright}\isanewline
\ \ \isacommand{have}\isamarkupfalse%
\ Ex{\isadigit{1}}{\isacharcolon}{\isachardoublequoteopen}{\isasymexists}{\isasymA}{\isachardot}\ {\isasymforall}F\ {\isasymin}\ {\isacharparenleft}{\isacharbraceleft}F{\isacharbraceright}\ {\isasymunion}\ Wo{\isacharparenright}{\isachardot}\ {\isasymA}\ {\isasymTurnstile}\ F{\isachardoublequoteclose}\isanewline
\ \ \ \ \isacommand{using}\isamarkupfalse%
\ {\isacartoucheopen}sat\ {\isacharparenleft}{\isacharbraceleft}F{\isacharbraceright}\ {\isasymunion}\ Wo{\isacharparenright}{\isacartoucheclose}\ \isacommand{by}\isamarkupfalse%
\ {\isacharparenleft}simp\ only{\isacharcolon}\ sat{\isacharunderscore}def{\isacharparenright}\isanewline
\ \ \isacommand{obtain}\isamarkupfalse%
\ {\isasymA}\ \isakeyword{where}\ Forall{\isadigit{1}}{\isacharcolon}{\isachardoublequoteopen}{\isasymforall}F\ {\isasymin}\ {\isacharparenleft}{\isacharbraceleft}F{\isacharbraceright}\ {\isasymunion}\ Wo{\isacharparenright}{\isachardot}\ {\isasymA}\ {\isasymTurnstile}\ F{\isachardoublequoteclose}\isanewline
\ \ \ \ \isacommand{using}\isamarkupfalse%
\ Ex{\isadigit{1}}\ \isacommand{by}\isamarkupfalse%
\ {\isacharparenleft}rule\ exE{\isacharparenright}\isanewline
\ \ \isacommand{have}\isamarkupfalse%
\ {\isachardoublequoteopen}{\isasymA}\ {\isasymTurnstile}\ F{\isachardoublequoteclose}\isanewline
\ \ \ \ \isacommand{using}\isamarkupfalse%
\ Forall{\isadigit{1}}\ {\isacartoucheopen}F\ {\isasymin}\ {\isacharbraceleft}F{\isacharbraceright}\ {\isasymunion}\ Wo{\isacartoucheclose}\ \isacommand{by}\isamarkupfalse%
\ {\isacharparenleft}rule\ bspec{\isacharparenright}\isanewline
\ \ \isacommand{then}\isamarkupfalse%
\ \isacommand{have}\isamarkupfalse%
\ {\isachardoublequoteopen}{\isasymA}\ {\isasymTurnstile}\ {\isacharparenleft}G\ \isactrlbold {\isasymand}\ H{\isacharparenright}{\isachardoublequoteclose}\isanewline
\ \ \ \ \isacommand{using}\isamarkupfalse%
\ assms{\isacharparenleft}{\isadigit{2}}{\isacharparenright}\ \isacommand{by}\isamarkupfalse%
\ {\isacharparenleft}simp\ only{\isacharcolon}\ {\isacartoucheopen}{\isasymA}\ {\isasymTurnstile}\ F{\isacartoucheclose}{\isacharparenright}\isanewline
\ \ \isacommand{then}\isamarkupfalse%
\ \isacommand{have}\isamarkupfalse%
\ {\isachardoublequoteopen}{\isasymA}\ {\isasymTurnstile}\ G\ {\isasymand}\ {\isasymA}\ {\isasymTurnstile}\ H{\isachardoublequoteclose}\isanewline
\ \ \ \ \isacommand{by}\isamarkupfalse%
\ {\isacharparenleft}simp\ only{\isacharcolon}\ formula{\isacharunderscore}semantics{\isachardot}simps{\isacharparenleft}{\isadigit{4}}{\isacharparenright}{\isacharparenright}\isanewline
\ \ \isacommand{then}\isamarkupfalse%
\ \isacommand{have}\isamarkupfalse%
\ {\isachardoublequoteopen}{\isasymA}\ {\isasymTurnstile}\ G{\isachardoublequoteclose}\isanewline
\ \ \ \ \isacommand{by}\isamarkupfalse%
\ {\isacharparenleft}rule\ conjunct{\isadigit{1}}{\isacharparenright}\isanewline
\ \ \isacommand{then}\isamarkupfalse%
\ \isacommand{have}\isamarkupfalse%
\ {\isadigit{1}}{\isacharcolon}{\isachardoublequoteopen}{\isasymforall}F\ {\isasymin}\ {\isacharbraceleft}G{\isacharbraceright}{\isachardot}\ {\isasymA}\ {\isasymTurnstile}\ F{\isachardoublequoteclose}\isanewline
\ \ \ \ \isacommand{by}\isamarkupfalse%
\ simp\isanewline
\ \ \isacommand{have}\isamarkupfalse%
\ {\isachardoublequoteopen}{\isasymA}\ {\isasymTurnstile}\ H{\isachardoublequoteclose}\isanewline
\ \ \ \ \isacommand{using}\isamarkupfalse%
\ {\isacartoucheopen}{\isasymA}\ {\isasymTurnstile}\ G\ {\isasymand}\ {\isasymA}\ {\isasymTurnstile}\ H{\isacartoucheclose}\ \isacommand{by}\isamarkupfalse%
\ {\isacharparenleft}rule\ conjunct{\isadigit{2}}{\isacharparenright}\isanewline
\ \ \isacommand{then}\isamarkupfalse%
\ \isacommand{have}\isamarkupfalse%
\ {\isadigit{2}}{\isacharcolon}{\isachardoublequoteopen}{\isasymforall}F\ {\isasymin}\ {\isacharbraceleft}H{\isacharbraceright}{\isachardot}\ {\isasymA}\ {\isasymTurnstile}\ F{\isachardoublequoteclose}\isanewline
\ \ \ \ \isacommand{by}\isamarkupfalse%
\ simp\isanewline
\ \ \isacommand{have}\isamarkupfalse%
\ {\isachardoublequoteopen}{\isasymforall}F\ {\isasymin}\ {\isacharparenleft}{\isacharbraceleft}G{\isacharbraceright}\ {\isasymunion}\ {\isacharbraceleft}H{\isacharbraceright}{\isacharparenright}\ {\isasymunion}\ {\isacharparenleft}{\isacharbraceleft}F{\isacharbraceright}\ {\isasymunion}\ Wo{\isacharparenright}{\isachardot}\ {\isasymA}\ {\isasymTurnstile}\ F{\isachardoublequoteclose}\isanewline
\ \ \ \ \isacommand{using}\isamarkupfalse%
\ Forall{\isadigit{1}}\ {\isadigit{1}}\ {\isadigit{2}}\ \isacommand{by}\isamarkupfalse%
\ {\isacharparenleft}iprover\ intro{\isacharcolon}\ ball{\isacharunderscore}Un{\isacharparenright}\isanewline
\ \ \isacommand{then}\isamarkupfalse%
\ \isacommand{have}\isamarkupfalse%
\ {\isachardoublequoteopen}{\isasymforall}F\ {\isasymin}\ {\isacharparenleft}{\isacharbraceleft}G{\isacharcomma}H{\isacharcomma}F{\isacharbraceright}\ {\isasymunion}\ Wo{\isacharparenright}{\isachardot}\ {\isasymA}\ {\isasymTurnstile}\ F{\isachardoublequoteclose}\isanewline
\ \ \ \ \isacommand{by}\isamarkupfalse%
\ simp\isanewline
\ \ \isacommand{then}\isamarkupfalse%
\ \isacommand{have}\isamarkupfalse%
\ {\isachardoublequoteopen}{\isasymexists}{\isasymA}{\isachardot}\ {\isasymforall}F\ {\isasymin}\ {\isacharparenleft}{\isacharbraceleft}G{\isacharcomma}H{\isacharcomma}F{\isacharbraceright}\ {\isasymunion}\ Wo{\isacharparenright}{\isachardot}\ {\isasymA}\ {\isasymTurnstile}\ F{\isachardoublequoteclose}\isanewline
\ \ \ \ \isacommand{by}\isamarkupfalse%
\ {\isacharparenleft}iprover\ intro{\isacharcolon}\ exI{\isacharparenright}\isanewline
\ \ \isacommand{thus}\isamarkupfalse%
\ {\isachardoublequoteopen}sat\ {\isacharparenleft}{\isacharbraceleft}G{\isacharcomma}H{\isacharcomma}F{\isacharbraceright}\ {\isasymunion}\ Wo{\isacharparenright}{\isachardoublequoteclose}\isanewline
\ \ \ \ \isacommand{by}\isamarkupfalse%
\ {\isacharparenleft}simp\ only{\isacharcolon}\ sat{\isacharunderscore}def{\isacharparenright}\isanewline
\isacommand{qed}\isamarkupfalse%
%
\endisatagproof
{\isafoldproof}%
%
\isadelimproof
\isanewline
%
\endisadelimproof
\isanewline
\isacommand{lemma}\isamarkupfalse%
\ pcp{\isacharunderscore}colecComp{\isacharunderscore}CON{\isacharunderscore}sat{\isadigit{2}}{\isacharcolon}\isanewline
\ \ \isakeyword{assumes}\ {\isachardoublequoteopen}W\ {\isasymin}\ colecComp{\isachardoublequoteclose}\isanewline
\ \ \ \ \ \ \ \ \ \ {\isachardoublequoteopen}F\ {\isacharequal}\ \isactrlbold {\isasymnot}{\isacharparenleft}G\ \isactrlbold {\isasymor}\ H{\isacharparenright}{\isachardoublequoteclose}\isanewline
\ \ \ \ \ \ \ \ \ \ {\isachardoublequoteopen}F\ {\isasymin}\ W{\isachardoublequoteclose}\isanewline
\ \ \ \ \ \ \ \ \ \ {\isachardoublequoteopen}finite\ Wo{\isachardoublequoteclose}\isanewline
\ \ \ \ \ \ \ \ \ \ {\isachardoublequoteopen}Wo\ {\isasymsubseteq}\ W{\isachardoublequoteclose}\isanewline
\ \ \ \ \ \ \ \ \isakeyword{shows}\ {\isachardoublequoteopen}sat\ {\isacharparenleft}{\isacharbraceleft}\isactrlbold {\isasymnot}\ G{\isacharcomma}\isactrlbold {\isasymnot}\ H{\isacharcomma}F{\isacharbraceright}\ {\isasymunion}\ Wo{\isacharparenright}{\isachardoublequoteclose}\isanewline
%
\isadelimproof
%
\endisadelimproof
%
\isatagproof
\isacommand{proof}\isamarkupfalse%
\ {\isacharminus}\isanewline
\ \ \isacommand{have}\isamarkupfalse%
\ {\isachardoublequoteopen}sat\ {\isacharparenleft}{\isacharbraceleft}F{\isacharbraceright}\ {\isasymunion}\ Wo{\isacharparenright}{\isachardoublequoteclose}\isanewline
\ \ \ \ \isacommand{using}\isamarkupfalse%
\ assms{\isacharparenleft}{\isadigit{1}}{\isacharcomma}{\isadigit{3}}{\isacharcomma}{\isadigit{4}}{\isacharcomma}{\isadigit{5}}{\isacharparenright}\ \isacommand{by}\isamarkupfalse%
\ {\isacharparenleft}rule\ pcp{\isacharunderscore}colecComp{\isacharunderscore}elem{\isacharunderscore}sat{\isacharparenright}\isanewline
\ \ \isacommand{have}\isamarkupfalse%
\ {\isachardoublequoteopen}F\ {\isasymin}\ {\isacharbraceleft}F{\isacharbraceright}\ {\isasymunion}\ Wo{\isachardoublequoteclose}\isanewline
\ \ \ \ \isacommand{by}\isamarkupfalse%
\ {\isacharparenleft}simp\ add{\isacharcolon}\ insertI{\isadigit{1}}{\isacharparenright}\isanewline
\ \ \isacommand{have}\isamarkupfalse%
\ Ex{\isadigit{1}}{\isacharcolon}{\isachardoublequoteopen}{\isasymexists}{\isasymA}{\isachardot}\ {\isasymforall}F\ {\isasymin}\ {\isacharparenleft}{\isacharbraceleft}F{\isacharbraceright}\ {\isasymunion}\ Wo{\isacharparenright}{\isachardot}\ {\isasymA}\ {\isasymTurnstile}\ F{\isachardoublequoteclose}\isanewline
\ \ \ \ \isacommand{using}\isamarkupfalse%
\ {\isacartoucheopen}sat\ {\isacharparenleft}{\isacharbraceleft}F{\isacharbraceright}\ {\isasymunion}\ Wo{\isacharparenright}{\isacartoucheclose}\ \isacommand{by}\isamarkupfalse%
\ {\isacharparenleft}simp\ only{\isacharcolon}\ sat{\isacharunderscore}def{\isacharparenright}\isanewline
\ \ \isacommand{obtain}\isamarkupfalse%
\ {\isasymA}\ \isakeyword{where}\ Forall{\isadigit{1}}{\isacharcolon}{\isachardoublequoteopen}{\isasymforall}F\ {\isasymin}\ {\isacharparenleft}{\isacharbraceleft}F{\isacharbraceright}\ {\isasymunion}\ Wo{\isacharparenright}{\isachardot}\ {\isasymA}\ {\isasymTurnstile}\ F{\isachardoublequoteclose}\isanewline
\ \ \ \ \isacommand{using}\isamarkupfalse%
\ Ex{\isadigit{1}}\ \isacommand{by}\isamarkupfalse%
\ {\isacharparenleft}rule\ exE{\isacharparenright}\isanewline
\ \ \isacommand{have}\isamarkupfalse%
\ {\isachardoublequoteopen}{\isasymA}\ {\isasymTurnstile}\ F{\isachardoublequoteclose}\isanewline
\ \ \ \ \isacommand{using}\isamarkupfalse%
\ Forall{\isadigit{1}}\ {\isacartoucheopen}F\ {\isasymin}\ {\isacharbraceleft}F{\isacharbraceright}\ {\isasymunion}\ Wo{\isacartoucheclose}\ \isacommand{by}\isamarkupfalse%
\ {\isacharparenleft}rule\ bspec{\isacharparenright}\isanewline
\ \ \isacommand{then}\isamarkupfalse%
\ \isacommand{have}\isamarkupfalse%
\ {\isachardoublequoteopen}{\isasymA}\ {\isasymTurnstile}\ \isactrlbold {\isasymnot}{\isacharparenleft}G\ \isactrlbold {\isasymor}\ H{\isacharparenright}{\isachardoublequoteclose}\isanewline
\ \ \ \ \isacommand{using}\isamarkupfalse%
\ assms{\isacharparenleft}{\isadigit{2}}{\isacharparenright}\ \isacommand{by}\isamarkupfalse%
\ {\isacharparenleft}simp\ only{\isacharcolon}\ {\isacartoucheopen}{\isasymA}\ {\isasymTurnstile}\ F{\isacartoucheclose}{\isacharparenright}\isanewline
\ \ \isacommand{then}\isamarkupfalse%
\ \isacommand{have}\isamarkupfalse%
\ {\isachardoublequoteopen}{\isasymnot}{\isacharparenleft}{\isasymA}\ {\isasymTurnstile}\ {\isacharparenleft}G\ \isactrlbold {\isasymor}\ H{\isacharparenright}{\isacharparenright}{\isachardoublequoteclose}\isanewline
\ \ \ \ \isacommand{by}\isamarkupfalse%
\ {\isacharparenleft}simp\ only{\isacharcolon}\ formula{\isacharunderscore}semantics{\isachardot}simps{\isacharparenleft}{\isadigit{3}}{\isacharparenright}\ simp{\isacharunderscore}thms{\isacharparenleft}{\isadigit{8}}{\isacharparenright}{\isacharparenright}\isanewline
\ \ \isacommand{then}\isamarkupfalse%
\ \isacommand{have}\isamarkupfalse%
\ {\isachardoublequoteopen}{\isasymnot}{\isacharparenleft}{\isasymA}\ {\isasymTurnstile}\ G\ {\isasymor}\ {\isasymA}\ {\isasymTurnstile}\ H{\isacharparenright}{\isachardoublequoteclose}\isanewline
\ \ \ \ \isacommand{by}\isamarkupfalse%
\ {\isacharparenleft}simp\ only{\isacharcolon}\ formula{\isacharunderscore}semantics{\isachardot}simps{\isacharparenleft}{\isadigit{5}}{\isacharparenright}\ simp{\isacharunderscore}thms{\isacharparenleft}{\isadigit{8}}{\isacharparenright}{\isacharparenright}\isanewline
\ \ \isacommand{then}\isamarkupfalse%
\ \isacommand{have}\isamarkupfalse%
\ {\isachardoublequoteopen}{\isasymnot}\ {\isasymA}\ {\isasymTurnstile}\ G\ {\isasymand}\ {\isasymnot}\ {\isasymA}\ {\isasymTurnstile}\ H{\isachardoublequoteclose}\ \isanewline
\ \ \ \ \isacommand{by}\isamarkupfalse%
\ {\isacharparenleft}simp\ only{\isacharcolon}\ de{\isacharunderscore}Morgan{\isacharunderscore}disj\ simp{\isacharunderscore}thms{\isacharparenleft}{\isadigit{8}}{\isacharparenright}{\isacharparenright}\isanewline
\ \ \isacommand{then}\isamarkupfalse%
\ \isacommand{have}\isamarkupfalse%
\ {\isachardoublequoteopen}{\isasymA}\ {\isasymTurnstile}\ \isactrlbold {\isasymnot}\ G\ {\isasymand}\ {\isasymA}\ {\isasymTurnstile}\ \isactrlbold {\isasymnot}\ H{\isachardoublequoteclose}\isanewline
\ \ \ \ \isacommand{by}\isamarkupfalse%
\ {\isacharparenleft}simp\ only{\isacharcolon}\ formula{\isacharunderscore}semantics{\isachardot}simps{\isacharparenleft}{\isadigit{3}}{\isacharparenright}\ simp{\isacharunderscore}thms{\isacharparenleft}{\isadigit{8}}{\isacharparenright}{\isacharparenright}\ \isanewline
\ \ \isacommand{then}\isamarkupfalse%
\ \isacommand{have}\isamarkupfalse%
\ {\isachardoublequoteopen}{\isasymA}\ {\isasymTurnstile}\ \isactrlbold {\isasymnot}\ G{\isachardoublequoteclose}\isanewline
\ \ \ \ \isacommand{by}\isamarkupfalse%
\ {\isacharparenleft}rule\ conjunct{\isadigit{1}}{\isacharparenright}\isanewline
\ \ \isacommand{then}\isamarkupfalse%
\ \isacommand{have}\isamarkupfalse%
\ {\isadigit{1}}{\isacharcolon}{\isachardoublequoteopen}{\isasymforall}F\ {\isasymin}\ {\isacharbraceleft}\isactrlbold {\isasymnot}\ G{\isacharbraceright}{\isachardot}\ {\isasymA}\ {\isasymTurnstile}\ F{\isachardoublequoteclose}\isanewline
\ \ \ \ \isacommand{by}\isamarkupfalse%
\ simp\isanewline
\ \ \isacommand{have}\isamarkupfalse%
\ {\isachardoublequoteopen}{\isasymA}\ {\isasymTurnstile}\ \isactrlbold {\isasymnot}\ H{\isachardoublequoteclose}\isanewline
\ \ \ \ \isacommand{using}\isamarkupfalse%
\ {\isacartoucheopen}{\isasymA}\ {\isasymTurnstile}\ \isactrlbold {\isasymnot}\ G\ {\isasymand}\ {\isasymA}\ {\isasymTurnstile}\ \isactrlbold {\isasymnot}\ H{\isacartoucheclose}\ \isacommand{by}\isamarkupfalse%
\ {\isacharparenleft}rule\ conjunct{\isadigit{2}}{\isacharparenright}\isanewline
\ \ \isacommand{then}\isamarkupfalse%
\ \isacommand{have}\isamarkupfalse%
\ {\isadigit{2}}{\isacharcolon}{\isachardoublequoteopen}{\isasymforall}F\ {\isasymin}\ {\isacharbraceleft}\isactrlbold {\isasymnot}\ H{\isacharbraceright}{\isachardot}\ {\isasymA}\ {\isasymTurnstile}\ F{\isachardoublequoteclose}\isanewline
\ \ \ \ \isacommand{by}\isamarkupfalse%
\ simp\isanewline
\ \ \isacommand{have}\isamarkupfalse%
\ {\isachardoublequoteopen}{\isasymforall}F\ {\isasymin}\ {\isacharparenleft}{\isacharbraceleft}\isactrlbold {\isasymnot}\ G{\isacharbraceright}\ {\isasymunion}\ {\isacharbraceleft}\isactrlbold {\isasymnot}\ H{\isacharbraceright}{\isacharparenright}\ {\isasymunion}\ {\isacharparenleft}{\isacharbraceleft}F{\isacharbraceright}\ {\isasymunion}\ Wo{\isacharparenright}{\isachardot}\ {\isasymA}\ {\isasymTurnstile}\ F{\isachardoublequoteclose}\isanewline
\ \ \ \ \isacommand{using}\isamarkupfalse%
\ Forall{\isadigit{1}}\ {\isadigit{1}}\ {\isadigit{2}}\ \isacommand{by}\isamarkupfalse%
\ {\isacharparenleft}iprover\ intro{\isacharcolon}\ ball{\isacharunderscore}Un{\isacharparenright}\isanewline
\ \ \isacommand{then}\isamarkupfalse%
\ \isacommand{have}\isamarkupfalse%
\ {\isachardoublequoteopen}{\isasymforall}F\ {\isasymin}\ {\isacharparenleft}{\isacharbraceleft}\isactrlbold {\isasymnot}\ G{\isacharcomma}\isactrlbold {\isasymnot}\ H{\isacharcomma}\ F{\isacharbraceright}\ {\isasymunion}\ Wo{\isacharparenright}{\isachardot}\ {\isasymA}\ {\isasymTurnstile}\ F{\isachardoublequoteclose}\isanewline
\ \ \ \ \isacommand{by}\isamarkupfalse%
\ simp\isanewline
\ \ \isacommand{then}\isamarkupfalse%
\ \isacommand{have}\isamarkupfalse%
\ {\isachardoublequoteopen}{\isasymexists}{\isasymA}{\isachardot}\ {\isasymforall}F\ {\isasymin}\ {\isacharparenleft}{\isacharbraceleft}\isactrlbold {\isasymnot}\ G{\isacharcomma}\isactrlbold {\isasymnot}\ H{\isacharcomma}F{\isacharbraceright}\ {\isasymunion}\ Wo{\isacharparenright}{\isachardot}\ {\isasymA}\ {\isasymTurnstile}\ F{\isachardoublequoteclose}\isanewline
\ \ \ \ \isacommand{by}\isamarkupfalse%
\ {\isacharparenleft}iprover\ intro{\isacharcolon}\ exI{\isacharparenright}\isanewline
\ \ \isacommand{thus}\isamarkupfalse%
\ {\isachardoublequoteopen}sat\ {\isacharparenleft}{\isacharbraceleft}\isactrlbold {\isasymnot}\ G{\isacharcomma}\isactrlbold {\isasymnot}\ H{\isacharcomma}F{\isacharbraceright}\ {\isasymunion}\ Wo{\isacharparenright}{\isachardoublequoteclose}\isanewline
\ \ \ \ \isacommand{by}\isamarkupfalse%
\ {\isacharparenleft}simp\ only{\isacharcolon}\ sat{\isacharunderscore}def{\isacharparenright}\isanewline
\isacommand{qed}\isamarkupfalse%
%
\endisatagproof
{\isafoldproof}%
%
\isadelimproof
\isanewline
%
\endisadelimproof
\isanewline
\isacommand{lemma}\isamarkupfalse%
\ pcp{\isacharunderscore}colecComp{\isacharunderscore}CON{\isacharunderscore}sat{\isadigit{3}}{\isacharcolon}\isanewline
\ \ \isakeyword{assumes}\ {\isachardoublequoteopen}W\ {\isasymin}\ colecComp{\isachardoublequoteclose}\isanewline
\ \ \ \ \ \ \ \ \ \ {\isachardoublequoteopen}F\ {\isacharequal}\ \isactrlbold {\isasymnot}\ {\isacharparenleft}G\ \isactrlbold {\isasymrightarrow}\ H{\isacharparenright}{\isachardoublequoteclose}\isanewline
\ \ \ \ \ \ \ \ \ \ {\isachardoublequoteopen}F\ {\isasymin}\ W{\isachardoublequoteclose}\isanewline
\ \ \ \ \ \ \ \ \ \ {\isachardoublequoteopen}finite\ Wo{\isachardoublequoteclose}\isanewline
\ \ \ \ \ \ \ \ \ \ {\isachardoublequoteopen}Wo\ {\isasymsubseteq}\ W{\isachardoublequoteclose}\isanewline
\ \ \ \ \ \ \ \ \isakeyword{shows}\ {\isachardoublequoteopen}sat\ {\isacharparenleft}{\isacharbraceleft}G{\isacharcomma}\isactrlbold {\isasymnot}\ H{\isacharcomma}F{\isacharbraceright}\ {\isasymunion}\ Wo{\isacharparenright}{\isachardoublequoteclose}\isanewline
%
\isadelimproof
%
\endisadelimproof
%
\isatagproof
\isacommand{proof}\isamarkupfalse%
\ {\isacharminus}\isanewline
\ \ \isacommand{have}\isamarkupfalse%
\ {\isachardoublequoteopen}sat\ {\isacharparenleft}{\isacharbraceleft}F{\isacharbraceright}\ {\isasymunion}\ Wo{\isacharparenright}{\isachardoublequoteclose}\isanewline
\ \ \ \ \isacommand{using}\isamarkupfalse%
\ assms{\isacharparenleft}{\isadigit{1}}{\isacharcomma}{\isadigit{3}}{\isacharcomma}{\isadigit{4}}{\isacharcomma}{\isadigit{5}}{\isacharparenright}\ \isacommand{by}\isamarkupfalse%
\ {\isacharparenleft}rule\ pcp{\isacharunderscore}colecComp{\isacharunderscore}elem{\isacharunderscore}sat{\isacharparenright}\isanewline
\ \ \isacommand{have}\isamarkupfalse%
\ {\isachardoublequoteopen}F\ {\isasymin}\ {\isacharbraceleft}F{\isacharbraceright}\ {\isasymunion}\ Wo{\isachardoublequoteclose}\isanewline
\ \ \ \ \isacommand{by}\isamarkupfalse%
\ {\isacharparenleft}simp\ add{\isacharcolon}\ insertI{\isadigit{1}}{\isacharparenright}\isanewline
\ \ \isacommand{have}\isamarkupfalse%
\ Ex{\isadigit{1}}{\isacharcolon}{\isachardoublequoteopen}{\isasymexists}{\isasymA}{\isachardot}\ {\isasymforall}F\ {\isasymin}\ {\isacharparenleft}{\isacharbraceleft}F{\isacharbraceright}\ {\isasymunion}\ Wo{\isacharparenright}{\isachardot}\ {\isasymA}\ {\isasymTurnstile}\ F{\isachardoublequoteclose}\isanewline
\ \ \ \ \isacommand{using}\isamarkupfalse%
\ {\isacartoucheopen}sat\ {\isacharparenleft}{\isacharbraceleft}F{\isacharbraceright}\ {\isasymunion}\ Wo{\isacharparenright}{\isacartoucheclose}\ \isacommand{by}\isamarkupfalse%
\ {\isacharparenleft}simp\ only{\isacharcolon}\ sat{\isacharunderscore}def{\isacharparenright}\isanewline
\ \ \isacommand{obtain}\isamarkupfalse%
\ {\isasymA}\ \isakeyword{where}\ Forall{\isadigit{1}}{\isacharcolon}{\isachardoublequoteopen}{\isasymforall}F\ {\isasymin}\ {\isacharparenleft}{\isacharbraceleft}F{\isacharbraceright}\ {\isasymunion}\ Wo{\isacharparenright}{\isachardot}\ {\isasymA}\ {\isasymTurnstile}\ F{\isachardoublequoteclose}\isanewline
\ \ \ \ \isacommand{using}\isamarkupfalse%
\ Ex{\isadigit{1}}\ \isacommand{by}\isamarkupfalse%
\ {\isacharparenleft}rule\ exE{\isacharparenright}\isanewline
\ \ \isacommand{have}\isamarkupfalse%
\ {\isachardoublequoteopen}{\isasymA}\ {\isasymTurnstile}\ F{\isachardoublequoteclose}\isanewline
\ \ \ \ \isacommand{using}\isamarkupfalse%
\ Forall{\isadigit{1}}\ {\isacartoucheopen}F\ {\isasymin}\ {\isacharbraceleft}F{\isacharbraceright}\ {\isasymunion}\ Wo{\isacartoucheclose}\ \isacommand{by}\isamarkupfalse%
\ {\isacharparenleft}rule\ bspec{\isacharparenright}\isanewline
\ \ \isacommand{then}\isamarkupfalse%
\ \isacommand{have}\isamarkupfalse%
\ {\isachardoublequoteopen}{\isasymA}\ {\isasymTurnstile}\ \isactrlbold {\isasymnot}{\isacharparenleft}G\ \isactrlbold {\isasymrightarrow}\ H{\isacharparenright}{\isachardoublequoteclose}\isanewline
\ \ \ \ \isacommand{using}\isamarkupfalse%
\ assms{\isacharparenleft}{\isadigit{2}}{\isacharparenright}\ \isacommand{by}\isamarkupfalse%
\ {\isacharparenleft}simp\ only{\isacharcolon}\ {\isacartoucheopen}{\isasymA}\ {\isasymTurnstile}\ F{\isacartoucheclose}{\isacharparenright}\isanewline
\ \ \isacommand{then}\isamarkupfalse%
\ \isacommand{have}\isamarkupfalse%
\ {\isachardoublequoteopen}{\isasymnot}{\isacharparenleft}{\isasymA}\ {\isasymTurnstile}\ {\isacharparenleft}G\ \isactrlbold {\isasymrightarrow}\ H{\isacharparenright}{\isacharparenright}{\isachardoublequoteclose}\isanewline
\ \ \ \ \isacommand{by}\isamarkupfalse%
\ {\isacharparenleft}simp\ only{\isacharcolon}\ formula{\isacharunderscore}semantics{\isachardot}simps{\isacharparenleft}{\isadigit{3}}{\isacharparenright}\ simp{\isacharunderscore}thms{\isacharparenleft}{\isadigit{8}}{\isacharparenright}{\isacharparenright}\isanewline
\ \ \isacommand{then}\isamarkupfalse%
\ \isacommand{have}\isamarkupfalse%
\ {\isachardoublequoteopen}{\isasymnot}{\isacharparenleft}{\isasymA}\ {\isasymTurnstile}\ G\ {\isasymlongrightarrow}\ {\isasymA}\ {\isasymTurnstile}\ H{\isacharparenright}{\isachardoublequoteclose}\isanewline
\ \ \ \ \isacommand{by}\isamarkupfalse%
\ {\isacharparenleft}simp\ only{\isacharcolon}\ formula{\isacharunderscore}semantics{\isachardot}simps{\isacharparenleft}{\isadigit{6}}{\isacharparenright}\ simp{\isacharunderscore}thms{\isacharparenleft}{\isadigit{8}}{\isacharparenright}{\isacharparenright}\isanewline
\ \ \isacommand{then}\isamarkupfalse%
\ \isacommand{have}\isamarkupfalse%
\ {\isachardoublequoteopen}{\isasymA}\ {\isasymTurnstile}\ G\ {\isasymand}\ {\isasymnot}\ {\isasymA}\ {\isasymTurnstile}\ H{\isachardoublequoteclose}\isanewline
\ \ \ \ \isacommand{by}\isamarkupfalse%
\ {\isacharparenleft}simp\ only{\isacharcolon}\ not{\isacharunderscore}imp\ simp{\isacharunderscore}thms{\isacharparenleft}{\isadigit{8}}{\isacharparenright}{\isacharparenright}\isanewline
\ \ \isacommand{then}\isamarkupfalse%
\ \isacommand{have}\isamarkupfalse%
\ {\isachardoublequoteopen}{\isasymA}\ {\isasymTurnstile}\ G\ {\isasymand}\ {\isasymA}\ {\isasymTurnstile}\ \isactrlbold {\isasymnot}\ H{\isachardoublequoteclose}\isanewline
\ \ \ \ \isacommand{by}\isamarkupfalse%
\ {\isacharparenleft}simp\ only{\isacharcolon}\ formula{\isacharunderscore}semantics{\isachardot}simps{\isacharparenleft}{\isadigit{3}}{\isacharparenright}\ simp{\isacharunderscore}thms{\isacharparenleft}{\isadigit{8}}{\isacharparenright}{\isacharparenright}\ \isanewline
\ \ \isacommand{then}\isamarkupfalse%
\ \isacommand{have}\isamarkupfalse%
\ {\isachardoublequoteopen}{\isasymA}\ {\isasymTurnstile}\ G{\isachardoublequoteclose}\isanewline
\ \ \ \ \isacommand{by}\isamarkupfalse%
\ {\isacharparenleft}rule\ conjunct{\isadigit{1}}{\isacharparenright}\isanewline
\ \ \isacommand{then}\isamarkupfalse%
\ \isacommand{have}\isamarkupfalse%
\ {\isadigit{1}}{\isacharcolon}{\isachardoublequoteopen}{\isasymforall}F\ {\isasymin}\ {\isacharbraceleft}G{\isacharbraceright}{\isachardot}\ {\isasymA}\ {\isasymTurnstile}\ F{\isachardoublequoteclose}\isanewline
\ \ \ \ \isacommand{by}\isamarkupfalse%
\ simp\isanewline
\ \ \isacommand{have}\isamarkupfalse%
\ {\isachardoublequoteopen}{\isasymA}\ {\isasymTurnstile}\ \isactrlbold {\isasymnot}\ H{\isachardoublequoteclose}\isanewline
\ \ \ \ \isacommand{using}\isamarkupfalse%
\ {\isacartoucheopen}{\isasymA}\ {\isasymTurnstile}\ G\ {\isasymand}\ {\isasymA}\ {\isasymTurnstile}\ \isactrlbold {\isasymnot}\ H{\isacartoucheclose}\ \isacommand{by}\isamarkupfalse%
\ {\isacharparenleft}rule\ conjunct{\isadigit{2}}{\isacharparenright}\isanewline
\ \ \isacommand{then}\isamarkupfalse%
\ \isacommand{have}\isamarkupfalse%
\ {\isadigit{2}}{\isacharcolon}{\isachardoublequoteopen}{\isasymforall}F\ {\isasymin}\ {\isacharbraceleft}\isactrlbold {\isasymnot}\ H{\isacharbraceright}{\isachardot}\ {\isasymA}\ {\isasymTurnstile}\ F{\isachardoublequoteclose}\isanewline
\ \ \ \ \isacommand{by}\isamarkupfalse%
\ simp\isanewline
\ \ \isacommand{have}\isamarkupfalse%
\ {\isachardoublequoteopen}{\isasymforall}F\ {\isasymin}\ {\isacharparenleft}{\isacharbraceleft}G{\isacharbraceright}\ {\isasymunion}\ {\isacharbraceleft}\isactrlbold {\isasymnot}\ H{\isacharbraceright}{\isacharparenright}\ {\isasymunion}\ {\isacharparenleft}{\isacharbraceleft}F{\isacharbraceright}\ {\isasymunion}\ Wo{\isacharparenright}{\isachardot}\ {\isasymA}\ {\isasymTurnstile}\ F{\isachardoublequoteclose}\isanewline
\ \ \ \ \isacommand{using}\isamarkupfalse%
\ Forall{\isadigit{1}}\ {\isadigit{1}}\ {\isadigit{2}}\ \isacommand{by}\isamarkupfalse%
\ {\isacharparenleft}iprover\ intro{\isacharcolon}\ ball{\isacharunderscore}Un{\isacharparenright}\isanewline
\ \ \isacommand{then}\isamarkupfalse%
\ \isacommand{have}\isamarkupfalse%
\ {\isachardoublequoteopen}{\isasymforall}F\ {\isasymin}\ {\isacharbraceleft}G{\isacharcomma}\isactrlbold {\isasymnot}\ H{\isacharcomma}F{\isacharbraceright}\ {\isasymunion}\ Wo{\isachardot}\ {\isasymA}\ {\isasymTurnstile}\ F{\isachardoublequoteclose}\isanewline
\ \ \ \ \isacommand{by}\isamarkupfalse%
\ simp\isanewline
\ \ \isacommand{then}\isamarkupfalse%
\ \isacommand{have}\isamarkupfalse%
\ {\isachardoublequoteopen}{\isasymexists}{\isasymA}{\isachardot}\ {\isasymforall}F\ {\isasymin}\ {\isacharparenleft}{\isacharbraceleft}G{\isacharcomma}\isactrlbold {\isasymnot}\ H{\isacharcomma}F{\isacharbraceright}\ {\isasymunion}\ Wo{\isacharparenright}{\isachardot}\ {\isasymA}\ {\isasymTurnstile}\ F{\isachardoublequoteclose}\isanewline
\ \ \ \ \isacommand{by}\isamarkupfalse%
\ {\isacharparenleft}iprover\ intro{\isacharcolon}\ exI{\isacharparenright}\isanewline
\ \ \isacommand{thus}\isamarkupfalse%
\ {\isachardoublequoteopen}sat\ {\isacharparenleft}{\isacharbraceleft}G{\isacharcomma}\isactrlbold {\isasymnot}\ H{\isacharcomma}F{\isacharbraceright}\ {\isasymunion}\ Wo{\isacharparenright}{\isachardoublequoteclose}\isanewline
\ \ \ \ \isacommand{by}\isamarkupfalse%
\ {\isacharparenleft}simp\ only{\isacharcolon}\ sat{\isacharunderscore}def{\isacharparenright}\isanewline
\isacommand{qed}\isamarkupfalse%
%
\endisatagproof
{\isafoldproof}%
%
\isadelimproof
\isanewline
%
\endisadelimproof
\isanewline
\isacommand{lemma}\isamarkupfalse%
\ pcp{\isacharunderscore}colecComp{\isacharunderscore}CON{\isacharunderscore}sat{\isadigit{4}}{\isacharcolon}\isanewline
\ \ \isakeyword{assumes}\ {\isachardoublequoteopen}W\ {\isasymin}\ colecComp{\isachardoublequoteclose}\isanewline
\ \ \ \ \ \ \ \ \ \ {\isachardoublequoteopen}F\ {\isacharequal}\ \isactrlbold {\isasymnot}\ {\isacharparenleft}\isactrlbold {\isasymnot}\ G{\isacharparenright}{\isachardoublequoteclose}\isanewline
\ \ \ \ \ \ \ \ \ \ {\isachardoublequoteopen}H\ {\isacharequal}\ G{\isachardoublequoteclose}\isanewline
\ \ \ \ \ \ \ \ \ \ {\isachardoublequoteopen}F\ {\isasymin}\ W{\isachardoublequoteclose}\isanewline
\ \ \ \ \ \ \ \ \ \ {\isachardoublequoteopen}finite\ Wo{\isachardoublequoteclose}\isanewline
\ \ \ \ \ \ \ \ \ \ {\isachardoublequoteopen}Wo\ {\isasymsubseteq}\ W{\isachardoublequoteclose}\isanewline
\ \ \ \ \ \ \ \ \isakeyword{shows}\ {\isachardoublequoteopen}sat\ {\isacharparenleft}{\isacharbraceleft}G{\isacharcomma}H{\isacharcomma}F{\isacharbraceright}\ {\isasymunion}\ Wo{\isacharparenright}{\isachardoublequoteclose}\isanewline
%
\isadelimproof
%
\endisadelimproof
%
\isatagproof
\isacommand{proof}\isamarkupfalse%
\ {\isacharminus}\isanewline
\ \ \isacommand{have}\isamarkupfalse%
\ {\isachardoublequoteopen}sat\ {\isacharparenleft}{\isacharbraceleft}F{\isacharbraceright}\ {\isasymunion}\ Wo{\isacharparenright}{\isachardoublequoteclose}\isanewline
\ \ \ \ \isacommand{using}\isamarkupfalse%
\ assms{\isacharparenleft}{\isadigit{1}}{\isacharcomma}{\isadigit{4}}{\isacharcomma}{\isadigit{5}}{\isacharcomma}{\isadigit{6}}{\isacharparenright}\ \isacommand{by}\isamarkupfalse%
\ {\isacharparenleft}rule\ pcp{\isacharunderscore}colecComp{\isacharunderscore}elem{\isacharunderscore}sat{\isacharparenright}\isanewline
\ \ \isacommand{have}\isamarkupfalse%
\ {\isachardoublequoteopen}F\ {\isasymin}\ {\isacharbraceleft}F{\isacharbraceright}\ {\isasymunion}\ Wo{\isachardoublequoteclose}\isanewline
\ \ \ \ \isacommand{by}\isamarkupfalse%
\ {\isacharparenleft}simp\ add{\isacharcolon}\ insertI{\isadigit{1}}{\isacharparenright}\isanewline
\ \ \isacommand{have}\isamarkupfalse%
\ Ex{\isadigit{1}}{\isacharcolon}{\isachardoublequoteopen}{\isasymexists}{\isasymA}{\isachardot}\ {\isasymforall}F\ {\isasymin}\ {\isacharparenleft}{\isacharbraceleft}F{\isacharbraceright}\ {\isasymunion}\ Wo{\isacharparenright}{\isachardot}\ {\isasymA}\ {\isasymTurnstile}\ F{\isachardoublequoteclose}\isanewline
\ \ \ \ \isacommand{using}\isamarkupfalse%
\ {\isacartoucheopen}sat\ {\isacharparenleft}{\isacharbraceleft}F{\isacharbraceright}\ {\isasymunion}\ Wo{\isacharparenright}{\isacartoucheclose}\ \isacommand{by}\isamarkupfalse%
\ {\isacharparenleft}simp\ only{\isacharcolon}\ sat{\isacharunderscore}def{\isacharparenright}\isanewline
\ \ \isacommand{obtain}\isamarkupfalse%
\ {\isasymA}\ \isakeyword{where}\ Forall{\isadigit{1}}{\isacharcolon}{\isachardoublequoteopen}{\isasymforall}F\ {\isasymin}\ {\isacharparenleft}{\isacharbraceleft}F{\isacharbraceright}\ {\isasymunion}\ Wo{\isacharparenright}{\isachardot}\ {\isasymA}\ {\isasymTurnstile}\ F{\isachardoublequoteclose}\isanewline
\ \ \ \ \isacommand{using}\isamarkupfalse%
\ Ex{\isadigit{1}}\ \isacommand{by}\isamarkupfalse%
\ {\isacharparenleft}rule\ exE{\isacharparenright}\isanewline
\ \ \isacommand{have}\isamarkupfalse%
\ {\isachardoublequoteopen}{\isasymA}\ {\isasymTurnstile}\ F{\isachardoublequoteclose}\isanewline
\ \ \ \ \isacommand{using}\isamarkupfalse%
\ Forall{\isadigit{1}}\ {\isacartoucheopen}F\ {\isasymin}\ {\isacharbraceleft}F{\isacharbraceright}\ {\isasymunion}\ Wo{\isacartoucheclose}\ \isacommand{by}\isamarkupfalse%
\ {\isacharparenleft}rule\ bspec{\isacharparenright}\isanewline
\ \ \isacommand{then}\isamarkupfalse%
\ \isacommand{have}\isamarkupfalse%
\ {\isachardoublequoteopen}{\isasymA}\ {\isasymTurnstile}\ \isactrlbold {\isasymnot}{\isacharparenleft}\isactrlbold {\isasymnot}\ G{\isacharparenright}{\isachardoublequoteclose}\isanewline
\ \ \ \ \isacommand{using}\isamarkupfalse%
\ assms{\isacharparenleft}{\isadigit{2}}{\isacharparenright}\ \isacommand{by}\isamarkupfalse%
\ {\isacharparenleft}simp\ only{\isacharcolon}\ {\isacartoucheopen}{\isasymA}\ {\isasymTurnstile}\ F{\isacartoucheclose}{\isacharparenright}\isanewline
\ \ \isacommand{then}\isamarkupfalse%
\ \isacommand{have}\isamarkupfalse%
\ {\isachardoublequoteopen}{\isasymnot}\ {\isasymA}\ {\isasymTurnstile}\ \isactrlbold {\isasymnot}\ G{\isachardoublequoteclose}\isanewline
\ \ \ \ \isacommand{by}\isamarkupfalse%
\ {\isacharparenleft}simp\ only{\isacharcolon}\ formula{\isacharunderscore}semantics{\isachardot}simps{\isacharparenleft}{\isadigit{3}}{\isacharparenright}\ simp{\isacharunderscore}thms{\isacharparenleft}{\isadigit{8}}{\isacharparenright}{\isacharparenright}\isanewline
\ \ \isacommand{then}\isamarkupfalse%
\ \isacommand{have}\isamarkupfalse%
\ {\isachardoublequoteopen}{\isasymnot}\ {\isasymnot}{\isasymA}\ {\isasymTurnstile}\ G{\isachardoublequoteclose}\isanewline
\ \ \ \ \isacommand{by}\isamarkupfalse%
\ {\isacharparenleft}simp\ only{\isacharcolon}\ formula{\isacharunderscore}semantics{\isachardot}simps{\isacharparenleft}{\isadigit{3}}{\isacharparenright}\ simp{\isacharunderscore}thms{\isacharparenleft}{\isadigit{8}}{\isacharparenright}{\isacharparenright}\isanewline
\ \ \isacommand{then}\isamarkupfalse%
\ \isacommand{have}\isamarkupfalse%
\ {\isachardoublequoteopen}{\isasymA}\ {\isasymTurnstile}\ G{\isachardoublequoteclose}\isanewline
\ \ \ \ \isacommand{by}\isamarkupfalse%
\ {\isacharparenleft}rule\ notnotD{\isacharparenright}\isanewline
\ \ \isacommand{then}\isamarkupfalse%
\ \isacommand{have}\isamarkupfalse%
\ {\isadigit{1}}{\isacharcolon}{\isachardoublequoteopen}{\isasymforall}F\ {\isasymin}\ {\isacharbraceleft}G{\isacharbraceright}{\isachardot}\ {\isasymA}\ {\isasymTurnstile}\ F{\isachardoublequoteclose}\isanewline
\ \ \ \ \isacommand{by}\isamarkupfalse%
\ simp\isanewline
\ \ \isacommand{have}\isamarkupfalse%
\ {\isachardoublequoteopen}{\isasymA}\ {\isasymTurnstile}\ H{\isachardoublequoteclose}\isanewline
\ \ \ \ \isacommand{using}\isamarkupfalse%
\ {\isacartoucheopen}{\isasymA}\ {\isasymTurnstile}\ G{\isacartoucheclose}\ \isacommand{by}\isamarkupfalse%
\ {\isacharparenleft}simp\ only{\isacharcolon}\ {\isacartoucheopen}H\ {\isacharequal}\ G{\isacartoucheclose}{\isacharparenright}\isanewline
\ \ \isacommand{then}\isamarkupfalse%
\ \isacommand{have}\isamarkupfalse%
\ {\isadigit{2}}{\isacharcolon}{\isachardoublequoteopen}{\isasymforall}F\ {\isasymin}\ {\isacharbraceleft}H{\isacharbraceright}{\isachardot}\ {\isasymA}\ {\isasymTurnstile}\ F{\isachardoublequoteclose}\isanewline
\ \ \ \ \isacommand{by}\isamarkupfalse%
\ simp\isanewline
\ \ \isacommand{have}\isamarkupfalse%
\ {\isachardoublequoteopen}{\isasymforall}F\ {\isasymin}\ {\isacharparenleft}{\isacharbraceleft}G{\isacharbraceright}\ {\isasymunion}\ {\isacharbraceleft}H{\isacharbraceright}{\isacharparenright}\ {\isasymunion}\ {\isacharparenleft}{\isacharbraceleft}F{\isacharbraceright}\ {\isasymunion}\ Wo{\isacharparenright}{\isachardot}\ {\isasymA}\ {\isasymTurnstile}\ F{\isachardoublequoteclose}\isanewline
\ \ \ \ \isacommand{using}\isamarkupfalse%
\ Forall{\isadigit{1}}\ {\isadigit{1}}\ {\isadigit{2}}\ \isacommand{by}\isamarkupfalse%
\ {\isacharparenleft}iprover\ intro{\isacharcolon}\ ball{\isacharunderscore}Un{\isacharparenright}\isanewline
\ \ \isacommand{then}\isamarkupfalse%
\ \isacommand{have}\isamarkupfalse%
\ {\isachardoublequoteopen}{\isasymforall}F\ {\isasymin}\ {\isacharbraceleft}G{\isacharcomma}H{\isacharcomma}F{\isacharbraceright}\ {\isasymunion}\ Wo{\isachardot}\ {\isasymA}\ {\isasymTurnstile}\ F{\isachardoublequoteclose}\isanewline
\ \ \ \ \isacommand{by}\isamarkupfalse%
\ simp\isanewline
\ \ \isacommand{then}\isamarkupfalse%
\ \isacommand{have}\isamarkupfalse%
\ {\isachardoublequoteopen}{\isasymexists}{\isasymA}{\isachardot}\ {\isasymforall}F\ {\isasymin}\ {\isacharparenleft}{\isacharbraceleft}G{\isacharcomma}H{\isacharcomma}F{\isacharbraceright}\ {\isasymunion}\ Wo{\isacharparenright}{\isachardot}\ {\isasymA}\ {\isasymTurnstile}\ F{\isachardoublequoteclose}\isanewline
\ \ \ \ \isacommand{by}\isamarkupfalse%
\ {\isacharparenleft}iprover\ intro{\isacharcolon}\ exI{\isacharparenright}\isanewline
\ \ \isacommand{thus}\isamarkupfalse%
\ {\isachardoublequoteopen}sat\ {\isacharparenleft}{\isacharbraceleft}G{\isacharcomma}H{\isacharcomma}F{\isacharbraceright}\ {\isasymunion}\ Wo{\isacharparenright}{\isachardoublequoteclose}\isanewline
\ \ \ \ \isacommand{by}\isamarkupfalse%
\ {\isacharparenleft}simp\ only{\isacharcolon}\ sat{\isacharunderscore}def{\isacharparenright}\isanewline
\isacommand{qed}\isamarkupfalse%
%
\endisatagproof
{\isafoldproof}%
%
\isadelimproof
\isanewline
%
\endisadelimproof
\isanewline
\isacommand{lemma}\isamarkupfalse%
\ pcp{\isacharunderscore}colecComp{\isacharunderscore}CON{\isacharunderscore}sat{\isacharcolon}\isanewline
\ \ \isakeyword{assumes}\ {\isachardoublequoteopen}W\ {\isasymin}\ colecComp{\isachardoublequoteclose}\isanewline
\ \ \ \ \ \ \ \ \ \ {\isachardoublequoteopen}Con\ F\ G\ H{\isachardoublequoteclose}\isanewline
\ \ \ \ \ \ \ \ \ \ {\isachardoublequoteopen}F\ {\isasymin}\ W{\isachardoublequoteclose}\isanewline
\ \ \ \ \ \ \ \ \ \ {\isachardoublequoteopen}finite\ Wo{\isachardoublequoteclose}\isanewline
\ \ \ \ \ \ \ \ \ \ {\isachardoublequoteopen}Wo\ {\isasymsubseteq}\ W{\isachardoublequoteclose}\isanewline
\ \ \ \ \ \ \ \ \isakeyword{shows}\ {\isachardoublequoteopen}sat\ {\isacharparenleft}{\isacharbraceleft}G{\isacharcomma}H{\isacharbraceright}\ {\isasymunion}\ Wo{\isacharparenright}{\isachardoublequoteclose}\isanewline
%
\isadelimproof
%
\endisadelimproof
%
\isatagproof
\isacommand{proof}\isamarkupfalse%
\ {\isacharminus}\isanewline
\ \ \isacommand{have}\isamarkupfalse%
\ {\isachardoublequoteopen}{\isacharbraceleft}G{\isacharcomma}H{\isacharbraceright}\ {\isasymunion}\ Wo\ {\isasymsubseteq}\ {\isacharbraceleft}G{\isacharcomma}H{\isacharcomma}F{\isacharbraceright}\ {\isasymunion}\ Wo{\isachardoublequoteclose}\isanewline
\ \ \ \ \isacommand{by}\isamarkupfalse%
\ blast\isanewline
\ \ \isacommand{have}\isamarkupfalse%
\ {\isachardoublequoteopen}F\ {\isacharequal}\ G\ \isactrlbold {\isasymand}\ H\ {\isasymor}\ \isanewline
\ \ \ \ {\isacharparenleft}{\isasymexists}F{\isadigit{1}}\ G{\isadigit{1}}{\isachardot}\ F\ {\isacharequal}\ \isactrlbold {\isasymnot}\ {\isacharparenleft}F{\isadigit{1}}\ \isactrlbold {\isasymor}\ G{\isadigit{1}}{\isacharparenright}\ {\isasymand}\ G\ {\isacharequal}\ \isactrlbold {\isasymnot}\ F{\isadigit{1}}\ {\isasymand}\ H\ {\isacharequal}\ \isactrlbold {\isasymnot}\ G{\isadigit{1}}{\isacharparenright}\ {\isasymor}\ \isanewline
\ \ \ \ {\isacharparenleft}{\isasymexists}H{\isadigit{1}}{\isachardot}\ F\ {\isacharequal}\ \isactrlbold {\isasymnot}\ {\isacharparenleft}G\ \isactrlbold {\isasymrightarrow}\ H{\isadigit{1}}{\isacharparenright}\ {\isasymand}\ H\ {\isacharequal}\ \isactrlbold {\isasymnot}\ H{\isadigit{1}}{\isacharparenright}\ {\isasymor}\ \isanewline
\ \ \ \ F\ {\isacharequal}\ \isactrlbold {\isasymnot}\ {\isacharparenleft}\isactrlbold {\isasymnot}\ G{\isacharparenright}\ {\isasymand}\ H\ {\isacharequal}\ G{\isachardoublequoteclose}\isanewline
\ \ \ \ \isacommand{using}\isamarkupfalse%
\ assms{\isacharparenleft}{\isadigit{2}}{\isacharparenright}\ \isacommand{by}\isamarkupfalse%
\ {\isacharparenleft}simp\ only{\isacharcolon}\ con{\isacharunderscore}dis{\isacharunderscore}simps{\isacharparenleft}{\isadigit{1}}{\isacharparenright}{\isacharparenright}\isanewline
\ \ \isacommand{then}\isamarkupfalse%
\ \isacommand{have}\isamarkupfalse%
\ {\isachardoublequoteopen}sat\ {\isacharparenleft}{\isacharbraceleft}G{\isacharcomma}H{\isacharcomma}F{\isacharbraceright}\ {\isasymunion}\ Wo{\isacharparenright}{\isachardoublequoteclose}\isanewline
\ \ \isacommand{proof}\isamarkupfalse%
\ {\isacharparenleft}rule\ disjE{\isacharparenright}\isanewline
\ \ \ \ \isacommand{assume}\isamarkupfalse%
\ {\isachardoublequoteopen}F\ {\isacharequal}\ G\ \isactrlbold {\isasymand}\ H{\isachardoublequoteclose}\isanewline
\ \ \ \ \isacommand{show}\isamarkupfalse%
\ {\isachardoublequoteopen}sat\ {\isacharparenleft}{\isacharbraceleft}G{\isacharcomma}H{\isacharcomma}F{\isacharbraceright}\ {\isasymunion}\ Wo{\isacharparenright}{\isachardoublequoteclose}\isanewline
\ \ \ \ \ \ \isacommand{using}\isamarkupfalse%
\ assms{\isacharparenleft}{\isadigit{1}}{\isacharparenright}\ {\isacartoucheopen}F\ {\isacharequal}\ G\ \isactrlbold {\isasymand}\ H{\isacartoucheclose}\ assms{\isacharparenleft}{\isadigit{3}}{\isacharcomma}{\isadigit{4}}{\isacharcomma}{\isadigit{5}}{\isacharparenright}\ \isacommand{by}\isamarkupfalse%
\ {\isacharparenleft}rule\ pcp{\isacharunderscore}colecComp{\isacharunderscore}CON{\isacharunderscore}sat{\isadigit{1}}{\isacharparenright}\isanewline
\ \ \isacommand{next}\isamarkupfalse%
\isanewline
\ \ \ \ \isacommand{assume}\isamarkupfalse%
\ {\isachardoublequoteopen}{\isacharparenleft}{\isasymexists}F{\isadigit{1}}\ G{\isadigit{1}}{\isachardot}\ F\ {\isacharequal}\ \isactrlbold {\isasymnot}\ {\isacharparenleft}F{\isadigit{1}}\ \isactrlbold {\isasymor}\ G{\isadigit{1}}{\isacharparenright}\ {\isasymand}\ G\ {\isacharequal}\ \isactrlbold {\isasymnot}\ F{\isadigit{1}}\ {\isasymand}\ H\ {\isacharequal}\ \isactrlbold {\isasymnot}\ G{\isadigit{1}}{\isacharparenright}\ {\isasymor}\ \isanewline
\ \ \ \ {\isacharparenleft}{\isasymexists}H{\isadigit{1}}{\isachardot}\ F\ {\isacharequal}\ \isactrlbold {\isasymnot}\ {\isacharparenleft}G\ \isactrlbold {\isasymrightarrow}\ H{\isadigit{1}}{\isacharparenright}\ {\isasymand}\ H\ {\isacharequal}\ \isactrlbold {\isasymnot}\ H{\isadigit{1}}{\isacharparenright}\ {\isasymor}\ \isanewline
\ \ \ \ F\ {\isacharequal}\ \isactrlbold {\isasymnot}\ {\isacharparenleft}\isactrlbold {\isasymnot}\ G{\isacharparenright}\ {\isasymand}\ H\ {\isacharequal}\ G{\isachardoublequoteclose}\isanewline
\ \ \ \ \isacommand{thus}\isamarkupfalse%
\ {\isachardoublequoteopen}sat\ {\isacharparenleft}{\isacharbraceleft}G{\isacharcomma}H{\isacharcomma}F{\isacharbraceright}\ {\isasymunion}\ Wo{\isacharparenright}{\isachardoublequoteclose}\isanewline
\ \ \ \ \isacommand{proof}\isamarkupfalse%
\ {\isacharparenleft}rule\ disjE{\isacharparenright}\isanewline
\ \ \ \ \ \ \isacommand{assume}\isamarkupfalse%
\ Ex{\isadigit{2}}{\isacharcolon}{\isachardoublequoteopen}{\isasymexists}F{\isadigit{1}}\ G{\isadigit{1}}{\isachardot}\ F\ {\isacharequal}\ \isactrlbold {\isasymnot}\ {\isacharparenleft}F{\isadigit{1}}\ \isactrlbold {\isasymor}\ G{\isadigit{1}}{\isacharparenright}\ {\isasymand}\ G\ {\isacharequal}\ \isactrlbold {\isasymnot}\ F{\isadigit{1}}\ {\isasymand}\ H\ {\isacharequal}\ \isactrlbold {\isasymnot}\ G{\isadigit{1}}{\isachardoublequoteclose}\ \isanewline
\ \ \ \ \ \ \isacommand{obtain}\isamarkupfalse%
\ F{\isadigit{1}}\ G{\isadigit{1}}\ \isakeyword{where}\ {\isadigit{2}}{\isacharcolon}{\isachardoublequoteopen}F\ {\isacharequal}\ \isactrlbold {\isasymnot}{\isacharparenleft}F{\isadigit{1}}\ \isactrlbold {\isasymor}\ G{\isadigit{1}}{\isacharparenright}\ {\isasymand}\ G\ {\isacharequal}\ \isactrlbold {\isasymnot}\ F{\isadigit{1}}\ {\isasymand}\ H\ {\isacharequal}\ \isactrlbold {\isasymnot}\ G{\isadigit{1}}{\isachardoublequoteclose}\isanewline
\ \ \ \ \ \ \ \ \isacommand{using}\isamarkupfalse%
\ Ex{\isadigit{2}}\ \isacommand{by}\isamarkupfalse%
\ {\isacharparenleft}iprover\ elim{\isacharcolon}\ exE{\isacharparenright}\isanewline
\ \ \ \ \ \ \isacommand{have}\isamarkupfalse%
\ {\isachardoublequoteopen}G\ {\isacharequal}\ \isactrlbold {\isasymnot}\ F{\isadigit{1}}{\isachardoublequoteclose}\isanewline
\ \ \ \ \ \ \ \ \isacommand{using}\isamarkupfalse%
\ {\isadigit{2}}\ \isacommand{by}\isamarkupfalse%
\ {\isacharparenleft}iprover\ elim{\isacharcolon}\ conjunct{\isadigit{1}}{\isacharparenright}\isanewline
\ \ \ \ \ \ \isacommand{have}\isamarkupfalse%
\ {\isachardoublequoteopen}H\ {\isacharequal}\ \isactrlbold {\isasymnot}\ G{\isadigit{1}}{\isachardoublequoteclose}\isanewline
\ \ \ \ \ \ \ \ \isacommand{using}\isamarkupfalse%
\ {\isadigit{2}}\ \isacommand{by}\isamarkupfalse%
\ {\isacharparenleft}iprover\ elim{\isacharcolon}\ conjunct{\isadigit{2}}{\isacharparenright}\isanewline
\ \ \ \ \ \ \isacommand{have}\isamarkupfalse%
\ {\isachardoublequoteopen}F\ {\isacharequal}\ \isactrlbold {\isasymnot}{\isacharparenleft}F{\isadigit{1}}\ \isactrlbold {\isasymor}\ G{\isadigit{1}}{\isacharparenright}{\isachardoublequoteclose}\isanewline
\ \ \ \ \ \ \ \ \isacommand{using}\isamarkupfalse%
\ {\isadigit{2}}\ \isacommand{by}\isamarkupfalse%
\ {\isacharparenleft}rule\ conjunct{\isadigit{1}}{\isacharparenright}\isanewline
\ \ \ \ \ \ \isacommand{have}\isamarkupfalse%
\ {\isachardoublequoteopen}sat\ {\isacharparenleft}{\isacharbraceleft}\isactrlbold {\isasymnot}\ F{\isadigit{1}}{\isacharcomma}\ \isactrlbold {\isasymnot}\ G{\isadigit{1}}{\isacharcomma}\ F{\isacharbraceright}\ {\isasymunion}\ Wo{\isacharparenright}{\isachardoublequoteclose}\isanewline
\ \ \ \ \ \ \ \ \isacommand{using}\isamarkupfalse%
\ assms{\isacharparenleft}{\isadigit{1}}{\isacharparenright}\ {\isacartoucheopen}F\ {\isacharequal}\ \isactrlbold {\isasymnot}{\isacharparenleft}F{\isadigit{1}}\ \isactrlbold {\isasymor}\ G{\isadigit{1}}{\isacharparenright}{\isacartoucheclose}\ assms{\isacharparenleft}{\isadigit{3}}{\isacharcomma}{\isadigit{4}}{\isacharcomma}{\isadigit{5}}{\isacharparenright}\ \isacommand{by}\isamarkupfalse%
\ {\isacharparenleft}rule\ pcp{\isacharunderscore}colecComp{\isacharunderscore}CON{\isacharunderscore}sat{\isadigit{2}}{\isacharparenright}\isanewline
\ \ \ \ \ \ \isacommand{thus}\isamarkupfalse%
\ {\isachardoublequoteopen}sat\ {\isacharparenleft}{\isacharbraceleft}G{\isacharcomma}H{\isacharcomma}F{\isacharbraceright}\ {\isasymunion}\ Wo{\isacharparenright}{\isachardoublequoteclose}\isanewline
\ \ \ \ \ \ \ \ \isacommand{by}\isamarkupfalse%
\ {\isacharparenleft}simp\ only{\isacharcolon}\ {\isacartoucheopen}G\ {\isacharequal}\ \isactrlbold {\isasymnot}\ F{\isadigit{1}}{\isacartoucheclose}\ {\isacartoucheopen}H\ {\isacharequal}\ \isactrlbold {\isasymnot}\ G{\isadigit{1}}{\isacartoucheclose}{\isacharparenright}\isanewline
\ \ \ \ \isacommand{next}\isamarkupfalse%
\isanewline
\ \ \ \ \ \ \isacommand{assume}\isamarkupfalse%
\ {\isachardoublequoteopen}{\isacharparenleft}{\isasymexists}H{\isadigit{1}}{\isachardot}\ F\ {\isacharequal}\ \isactrlbold {\isasymnot}\ {\isacharparenleft}G\ \isactrlbold {\isasymrightarrow}\ H{\isadigit{1}}{\isacharparenright}\ {\isasymand}\ H\ {\isacharequal}\ \isactrlbold {\isasymnot}\ H{\isadigit{1}}{\isacharparenright}\ {\isasymor}\ \isanewline
\ \ \ \ \ \ \ \ \ \ \ \ \ \ F\ {\isacharequal}\ \isactrlbold {\isasymnot}\ {\isacharparenleft}\isactrlbold {\isasymnot}\ G{\isacharparenright}\ {\isasymand}\ H\ {\isacharequal}\ G{\isachardoublequoteclose}\isanewline
\ \ \ \ \ \ \isacommand{thus}\isamarkupfalse%
\ {\isachardoublequoteopen}sat\ {\isacharparenleft}{\isacharbraceleft}G{\isacharcomma}H{\isacharcomma}F{\isacharbraceright}\ {\isasymunion}\ Wo{\isacharparenright}{\isachardoublequoteclose}\isanewline
\ \ \ \ \ \ \isacommand{proof}\isamarkupfalse%
\ {\isacharparenleft}rule\ disjE{\isacharparenright}\isanewline
\ \ \ \ \ \ \ \ \isacommand{assume}\isamarkupfalse%
\ Ex{\isadigit{3}}{\isacharcolon}{\isachardoublequoteopen}{\isasymexists}H{\isadigit{1}}{\isachardot}\ F\ {\isacharequal}\ \isactrlbold {\isasymnot}\ {\isacharparenleft}G\ \isactrlbold {\isasymrightarrow}\ H{\isadigit{1}}{\isacharparenright}\ {\isasymand}\ H\ {\isacharequal}\ \isactrlbold {\isasymnot}\ H{\isadigit{1}}{\isachardoublequoteclose}\isanewline
\ \ \ \ \ \ \ \ \isacommand{obtain}\isamarkupfalse%
\ H{\isadigit{1}}\ \isakeyword{where}\ {\isadigit{3}}{\isacharcolon}{\isachardoublequoteopen}F\ {\isacharequal}\ \isactrlbold {\isasymnot}{\isacharparenleft}G\ \isactrlbold {\isasymrightarrow}\ H{\isadigit{1}}{\isacharparenright}\ {\isasymand}\ H\ {\isacharequal}\ \isactrlbold {\isasymnot}\ H{\isadigit{1}}{\isachardoublequoteclose}\isanewline
\ \ \ \ \ \ \ \ \ \ \isacommand{using}\isamarkupfalse%
\ Ex{\isadigit{3}}\ \isacommand{by}\isamarkupfalse%
\ {\isacharparenleft}rule\ exE{\isacharparenright}\isanewline
\ \ \ \ \ \ \ \ \isacommand{have}\isamarkupfalse%
\ {\isachardoublequoteopen}H\ {\isacharequal}\ \isactrlbold {\isasymnot}\ H{\isadigit{1}}{\isachardoublequoteclose}\isanewline
\ \ \ \ \ \ \ \ \ \ \isacommand{using}\isamarkupfalse%
\ {\isadigit{3}}\ \isacommand{by}\isamarkupfalse%
\ {\isacharparenleft}rule\ conjunct{\isadigit{2}}{\isacharparenright}\isanewline
\ \ \ \ \ \ \ \ \isacommand{have}\isamarkupfalse%
\ {\isachardoublequoteopen}F\ {\isacharequal}\ \isactrlbold {\isasymnot}{\isacharparenleft}G\ \isactrlbold {\isasymrightarrow}\ H{\isadigit{1}}{\isacharparenright}{\isachardoublequoteclose}\isanewline
\ \ \ \ \ \ \ \ \ \ \isacommand{using}\isamarkupfalse%
\ {\isadigit{3}}\ \isacommand{by}\isamarkupfalse%
\ {\isacharparenleft}rule\ conjunct{\isadigit{1}}{\isacharparenright}\isanewline
\ \ \ \ \ \ \ \ \isacommand{have}\isamarkupfalse%
\ {\isachardoublequoteopen}sat\ {\isacharparenleft}{\isacharbraceleft}G{\isacharcomma}\ \isactrlbold {\isasymnot}\ H{\isadigit{1}}{\isacharcomma}\ F{\isacharbraceright}\ {\isasymunion}\ Wo{\isacharparenright}{\isachardoublequoteclose}\isanewline
\ \ \ \ \ \ \ \ \ \ \isacommand{using}\isamarkupfalse%
\ assms{\isacharparenleft}{\isadigit{1}}{\isacharparenright}\ {\isacartoucheopen}F\ {\isacharequal}\ \isactrlbold {\isasymnot}{\isacharparenleft}G\ \isactrlbold {\isasymrightarrow}\ H{\isadigit{1}}{\isacharparenright}{\isacartoucheclose}\ assms{\isacharparenleft}{\isadigit{3}}{\isacharcomma}{\isadigit{4}}{\isacharcomma}{\isadigit{5}}{\isacharparenright}\ \isacommand{by}\isamarkupfalse%
\ {\isacharparenleft}rule\ pcp{\isacharunderscore}colecComp{\isacharunderscore}CON{\isacharunderscore}sat{\isadigit{3}}{\isacharparenright}\isanewline
\ \ \ \ \ \ \ \ \isacommand{thus}\isamarkupfalse%
\ {\isachardoublequoteopen}sat\ {\isacharparenleft}{\isacharbraceleft}G{\isacharcomma}H{\isacharcomma}F{\isacharbraceright}\ {\isasymunion}\ Wo{\isacharparenright}{\isachardoublequoteclose}\isanewline
\ \ \ \ \ \ \ \ \ \ \isacommand{by}\isamarkupfalse%
\ {\isacharparenleft}simp\ only{\isacharcolon}\ {\isacartoucheopen}H\ {\isacharequal}\ \isactrlbold {\isasymnot}\ H{\isadigit{1}}{\isacartoucheclose}{\isacharparenright}\isanewline
\ \ \ \ \ \ \isacommand{next}\isamarkupfalse%
\isanewline
\ \ \ \ \ \ \ \ \isacommand{assume}\isamarkupfalse%
\ {\isachardoublequoteopen}F\ {\isacharequal}\ \isactrlbold {\isasymnot}\ {\isacharparenleft}\isactrlbold {\isasymnot}\ G{\isacharparenright}\ {\isasymand}\ H\ {\isacharequal}\ G{\isachardoublequoteclose}\isanewline
\ \ \ \ \ \ \ \ \isacommand{then}\isamarkupfalse%
\ \isacommand{have}\isamarkupfalse%
\ {\isachardoublequoteopen}H\ {\isacharequal}\ G{\isachardoublequoteclose}\isanewline
\ \ \ \ \ \ \ \ \ \ \isacommand{by}\isamarkupfalse%
\ {\isacharparenleft}rule\ conjunct{\isadigit{2}}{\isacharparenright}\isanewline
\ \ \ \ \ \ \ \ \isacommand{have}\isamarkupfalse%
\ {\isachardoublequoteopen}F\ {\isacharequal}\ \isactrlbold {\isasymnot}\ {\isacharparenleft}\isactrlbold {\isasymnot}\ G{\isacharparenright}{\isachardoublequoteclose}\isanewline
\ \ \ \ \ \ \ \ \ \ \isacommand{using}\isamarkupfalse%
\ {\isacartoucheopen}F\ {\isacharequal}\ \isactrlbold {\isasymnot}\ {\isacharparenleft}\isactrlbold {\isasymnot}\ G{\isacharparenright}\ {\isasymand}\ H\ {\isacharequal}\ G{\isacartoucheclose}\ \isacommand{by}\isamarkupfalse%
\ {\isacharparenleft}rule\ conjunct{\isadigit{1}}{\isacharparenright}\isanewline
\ \ \ \ \ \ \ \ \isacommand{show}\isamarkupfalse%
\ {\isachardoublequoteopen}sat\ {\isacharparenleft}{\isacharbraceleft}G{\isacharcomma}\ H{\isacharcomma}\ F{\isacharbraceright}\ {\isasymunion}\ Wo{\isacharparenright}{\isachardoublequoteclose}\isanewline
\ \ \ \ \ \ \ \ \ \ \isacommand{using}\isamarkupfalse%
\ assms{\isacharparenleft}{\isadigit{1}}{\isacharparenright}\ {\isacartoucheopen}F\ {\isacharequal}\ \isactrlbold {\isasymnot}{\isacharparenleft}\isactrlbold {\isasymnot}\ G{\isacharparenright}{\isacartoucheclose}\ {\isacartoucheopen}H\ {\isacharequal}\ G{\isacartoucheclose}\ assms{\isacharparenleft}{\isadigit{3}}{\isacharcomma}{\isadigit{4}}{\isacharcomma}{\isadigit{5}}{\isacharparenright}\ \isacommand{by}\isamarkupfalse%
\ {\isacharparenleft}rule\ pcp{\isacharunderscore}colecComp{\isacharunderscore}CON{\isacharunderscore}sat{\isadigit{4}}{\isacharparenright}\isanewline
\ \ \ \ \ \ \isacommand{qed}\isamarkupfalse%
\isanewline
\ \ \ \ \isacommand{qed}\isamarkupfalse%
\isanewline
\ \ \isacommand{qed}\isamarkupfalse%
\isanewline
\ \ \isacommand{thus}\isamarkupfalse%
\ {\isachardoublequoteopen}sat\ {\isacharparenleft}{\isacharbraceleft}G{\isacharcomma}H{\isacharbraceright}\ {\isasymunion}\ Wo{\isacharparenright}{\isachardoublequoteclose}\isanewline
\ \ \ \ \isacommand{using}\isamarkupfalse%
\ {\isacartoucheopen}{\isacharbraceleft}G{\isacharcomma}H{\isacharbraceright}\ {\isasymunion}\ Wo\ {\isasymsubseteq}\ {\isacharbraceleft}G{\isacharcomma}H{\isacharcomma}F{\isacharbraceright}\ {\isasymunion}\ Wo{\isacartoucheclose}\ \isacommand{by}\isamarkupfalse%
\ {\isacharparenleft}simp\ only{\isacharcolon}\ sat{\isacharunderscore}mono{\isacharparenright}\isanewline
\isacommand{qed}\isamarkupfalse%
%
\endisatagproof
{\isafoldproof}%
%
\isadelimproof
\isanewline
%
\endisadelimproof
\ \ \ \ \ \ \isanewline
\isacommand{lemma}\isamarkupfalse%
\ pcp{\isacharunderscore}colecComp{\isacharunderscore}CON{\isacharcolon}\isanewline
\ \ \isakeyword{assumes}\ {\isachardoublequoteopen}W\ {\isasymin}\ colecComp{\isachardoublequoteclose}\isanewline
\ \ \isakeyword{shows}\ {\isachardoublequoteopen}{\isasymforall}F\ G\ H{\isachardot}\ Con\ F\ G\ H\ {\isasymlongrightarrow}\ F\ {\isasymin}\ W\ {\isasymlongrightarrow}\ {\isacharbraceleft}G{\isacharcomma}H{\isacharbraceright}\ {\isasymunion}\ W\ {\isasymin}\ colecComp{\isachardoublequoteclose}\isanewline
%
\isadelimproof
%
\endisadelimproof
%
\isatagproof
\isacommand{proof}\isamarkupfalse%
\ {\isacharparenleft}rule\ allI{\isacharparenright}{\isacharplus}\isanewline
\ \ \isacommand{fix}\isamarkupfalse%
\ F\ G\ H\isanewline
\ \ \isacommand{show}\isamarkupfalse%
\ {\isachardoublequoteopen}Con\ F\ G\ H\ {\isasymlongrightarrow}\ F\ {\isasymin}\ W\ {\isasymlongrightarrow}\ {\isacharbraceleft}G{\isacharcomma}H{\isacharbraceright}\ {\isasymunion}\ W\ {\isasymin}\ colecComp{\isachardoublequoteclose}\isanewline
\ \ \isacommand{proof}\isamarkupfalse%
\ {\isacharparenleft}rule\ impI{\isacharparenright}{\isacharplus}\isanewline
\ \ \ \ \isacommand{assume}\isamarkupfalse%
\ {\isachardoublequoteopen}Con\ F\ G\ H{\isachardoublequoteclose}\isanewline
\ \ \ \ \isacommand{assume}\isamarkupfalse%
\ {\isachardoublequoteopen}F\ {\isasymin}\ W{\isachardoublequoteclose}\isanewline
\ \ \ \ \isacommand{show}\isamarkupfalse%
\ {\isachardoublequoteopen}{\isacharbraceleft}G{\isacharcomma}H{\isacharbraceright}\ {\isasymunion}\ W\ {\isasymin}\ colecComp{\isachardoublequoteclose}\isanewline
\ \ \ \ \ \ \isacommand{unfolding}\isamarkupfalse%
\ colecComp\ fin{\isacharunderscore}sat{\isacharunderscore}def\isanewline
\ \ \ \ \isacommand{proof}\isamarkupfalse%
\ {\isacharparenleft}rule\ CollectI{\isacharparenright}\isanewline
\ \ \ \ \ \ \isacommand{show}\isamarkupfalse%
\ {\isachardoublequoteopen}{\isasymforall}S{\isacharprime}\ {\isasymsubseteq}\ {\isacharbraceleft}G{\isacharcomma}H{\isacharbraceright}\ {\isasymunion}\ W{\isachardot}\ finite\ S{\isacharprime}\ {\isasymlongrightarrow}\ sat\ S{\isacharprime}{\isachardoublequoteclose}\isanewline
\ \ \ \ \ \ \isacommand{proof}\isamarkupfalse%
\ {\isacharparenleft}rule\ sallI{\isacharparenright}\isanewline
\ \ \ \ \ \ \ \ \isacommand{fix}\isamarkupfalse%
\ S{\isacharprime}\isanewline
\ \ \ \ \ \ \ \ \isacommand{assume}\isamarkupfalse%
\ {\isachardoublequoteopen}S{\isacharprime}\ {\isasymsubseteq}\ {\isacharbraceleft}G{\isacharcomma}H{\isacharbraceright}\ {\isasymunion}\ W{\isachardoublequoteclose}\isanewline
\ \ \ \ \ \ \ \ \isacommand{then}\isamarkupfalse%
\ \isacommand{have}\isamarkupfalse%
\ {\isachardoublequoteopen}S{\isacharprime}\ {\isasymsubseteq}\ {\isacharbraceleft}G{\isacharbraceright}\ {\isasymunion}\ {\isacharparenleft}{\isacharbraceleft}H{\isacharbraceright}\ {\isasymunion}\ W{\isacharparenright}{\isachardoublequoteclose}\isanewline
\ \ \ \ \ \ \ \ \ \ \isacommand{by}\isamarkupfalse%
\ blast\ \isanewline
\ \ \ \ \ \ \ \ \isacommand{show}\isamarkupfalse%
\ {\isachardoublequoteopen}finite\ S{\isacharprime}\ {\isasymlongrightarrow}\ sat\ S{\isacharprime}{\isachardoublequoteclose}\isanewline
\ \ \ \ \ \ \ \ \isacommand{proof}\isamarkupfalse%
\ {\isacharparenleft}rule\ impI{\isacharparenright}\isanewline
\ \ \ \ \ \ \ \ \ \ \isacommand{assume}\isamarkupfalse%
\ {\isachardoublequoteopen}finite\ S{\isacharprime}{\isachardoublequoteclose}\ \isanewline
\ \ \ \ \ \ \ \ \ \ \isacommand{have}\isamarkupfalse%
\ Ex{\isacharcolon}{\isachardoublequoteopen}{\isasymexists}Wo\ {\isasymsubseteq}\ W{\isachardot}\ finite\ Wo\ {\isasymand}\ {\isacharparenleft}S{\isacharprime}\ {\isacharequal}\ {\isacharbraceleft}G{\isacharcomma}H{\isacharbraceright}\ {\isasymunion}\ Wo\ {\isasymor}\ S{\isacharprime}\ {\isacharequal}\ {\isacharbraceleft}G{\isacharbraceright}\ {\isasymunion}\ Wo\ {\isasymor}\ S{\isacharprime}\ {\isacharequal}\ {\isacharbraceleft}H{\isacharbraceright}\ {\isasymunion}\ Wo\ {\isasymor}\ S{\isacharprime}\ {\isacharequal}\ Wo{\isacharparenright}{\isachardoublequoteclose}\isanewline
\ \ \ \ \ \ \ \ \ \ \ \ \isacommand{using}\isamarkupfalse%
\ {\isacartoucheopen}finite\ S{\isacharprime}{\isacartoucheclose}\ {\isacartoucheopen}S{\isacharprime}\ {\isasymsubseteq}\ {\isacharbraceleft}G{\isacharcomma}H{\isacharbraceright}\ {\isasymunion}\ W{\isacartoucheclose}\ \isacommand{by}\isamarkupfalse%
\ {\isacharparenleft}rule\ finite{\isacharunderscore}subset{\isacharunderscore}insert{\isadigit{2}}{\isacharparenright}\isanewline
\ \ \ \ \ \ \ \ \ \ \isacommand{obtain}\isamarkupfalse%
\ Wo{\isacharprime}\ \isakeyword{where}\ {\isachardoublequoteopen}Wo{\isacharprime}\ {\isasymsubseteq}\ W{\isachardoublequoteclose}\ \isakeyword{and}\ {\isadigit{1}}{\isacharcolon}{\isachardoublequoteopen}finite\ Wo{\isacharprime}\ {\isasymand}\ {\isacharparenleft}S{\isacharprime}\ {\isacharequal}\ {\isacharbraceleft}G{\isacharcomma}H{\isacharbraceright}\ {\isasymunion}\ Wo{\isacharprime}\ {\isasymor}\ S{\isacharprime}\ {\isacharequal}\ {\isacharbraceleft}G{\isacharbraceright}\ {\isasymunion}\ Wo{\isacharprime}\ {\isasymor}\ S{\isacharprime}\ {\isacharequal}\ {\isacharbraceleft}H{\isacharbraceright}\ {\isasymunion}\ Wo{\isacharprime}\ {\isasymor}\ S{\isacharprime}\ {\isacharequal}\ Wo{\isacharprime}{\isacharparenright}{\isachardoublequoteclose}\isanewline
\ \ \ \ \ \ \ \ \ \ \ \ \isacommand{using}\isamarkupfalse%
\ Ex\ \isacommand{by}\isamarkupfalse%
\ {\isacharparenleft}rule\ subexE{\isacharparenright}\isanewline
\ \ \ \ \ \ \ \ \ \ \isacommand{have}\isamarkupfalse%
\ {\isachardoublequoteopen}finite\ Wo{\isacharprime}{\isachardoublequoteclose}\isanewline
\ \ \ \ \ \ \ \ \ \ \ \ \isacommand{using}\isamarkupfalse%
\ {\isadigit{1}}\ \isacommand{by}\isamarkupfalse%
\ {\isacharparenleft}rule\ conjunct{\isadigit{1}}{\isacharparenright}\isanewline
\ \ \ \ \ \ \ \ \ \ \ \ \isacommand{have}\isamarkupfalse%
\ {\isachardoublequoteopen}sat\ {\isacharparenleft}{\isacharbraceleft}G{\isacharcomma}H{\isacharbraceright}\ {\isasymunion}\ Wo{\isacharprime}{\isacharparenright}{\isachardoublequoteclose}\ \isanewline
\ \ \ \ \ \ \ \ \ \ \ \ \ \ \isacommand{using}\isamarkupfalse%
\ {\isacartoucheopen}W\ {\isasymin}\ colecComp{\isacartoucheclose}\ {\isacartoucheopen}Con\ F\ G\ H{\isacartoucheclose}\ {\isacartoucheopen}F\ {\isasymin}\ W{\isacartoucheclose}\ {\isacartoucheopen}finite\ Wo{\isacharprime}{\isacartoucheclose}\ {\isacartoucheopen}Wo{\isacharprime}\ {\isasymsubseteq}\ W{\isacartoucheclose}\ \isacommand{by}\isamarkupfalse%
\ {\isacharparenleft}rule\ pcp{\isacharunderscore}colecComp{\isacharunderscore}CON{\isacharunderscore}sat{\isacharparenright}\isanewline
\ \ \ \ \ \ \ \ \ \ \isacommand{have}\isamarkupfalse%
\ {\isachardoublequoteopen}S{\isacharprime}\ {\isacharequal}\ {\isacharbraceleft}G{\isacharcomma}H{\isacharbraceright}\ {\isasymunion}\ Wo{\isacharprime}\ {\isasymor}\ S{\isacharprime}\ {\isacharequal}\ {\isacharbraceleft}G{\isacharbraceright}\ {\isasymunion}\ Wo{\isacharprime}\ {\isasymor}\ S{\isacharprime}\ {\isacharequal}\ {\isacharbraceleft}H{\isacharbraceright}\ {\isasymunion}\ Wo{\isacharprime}\ {\isasymor}\ S{\isacharprime}\ {\isacharequal}\ Wo{\isacharprime}{\isachardoublequoteclose}\isanewline
\ \ \ \ \ \ \ \ \ \ \ \ \isacommand{using}\isamarkupfalse%
\ {\isadigit{1}}\ \isacommand{by}\isamarkupfalse%
\ {\isacharparenleft}rule\ conjunct{\isadigit{2}}{\isacharparenright}\isanewline
\ \ \ \ \ \ \ \ \ \ \isacommand{thus}\isamarkupfalse%
\ {\isachardoublequoteopen}sat\ S{\isacharprime}{\isachardoublequoteclose}\isanewline
\ \ \ \ \ \ \ \ \ \ \isacommand{proof}\isamarkupfalse%
\ {\isacharparenleft}rule\ disjE{\isacharparenright}\isanewline
\ \ \ \ \ \ \ \ \ \ \ \ \isacommand{assume}\isamarkupfalse%
\ {\isachardoublequoteopen}S{\isacharprime}\ {\isacharequal}\ {\isacharbraceleft}G{\isacharcomma}H{\isacharbraceright}\ {\isasymunion}\ Wo{\isacharprime}{\isachardoublequoteclose}\isanewline
\ \ \ \ \ \ \ \ \ \ \ \ \isacommand{show}\isamarkupfalse%
\ {\isachardoublequoteopen}sat\ S{\isacharprime}{\isachardoublequoteclose}\isanewline
\ \ \ \ \ \ \ \ \ \ \ \ \ \ \isacommand{using}\isamarkupfalse%
\ {\isacartoucheopen}sat{\isacharparenleft}{\isacharbraceleft}G{\isacharcomma}H{\isacharbraceright}\ {\isasymunion}\ Wo{\isacharprime}{\isacharparenright}{\isacartoucheclose}\ \isacommand{by}\isamarkupfalse%
\ {\isacharparenleft}simp\ only{\isacharcolon}\ {\isacartoucheopen}S{\isacharprime}\ {\isacharequal}\ {\isacharbraceleft}G{\isacharcomma}H{\isacharbraceright}\ {\isasymunion}\ Wo{\isacharprime}{\isacartoucheclose}{\isacharparenright}\isanewline
\ \ \ \ \ \ \ \ \ \ \isacommand{next}\isamarkupfalse%
\isanewline
\ \ \ \ \ \ \ \ \ \ \ \ \isacommand{assume}\isamarkupfalse%
\ {\isachardoublequoteopen}S{\isacharprime}\ {\isacharequal}\ {\isacharbraceleft}G{\isacharbraceright}\ {\isasymunion}\ Wo{\isacharprime}\ {\isasymor}\ S{\isacharprime}\ {\isacharequal}\ {\isacharbraceleft}H{\isacharbraceright}\ {\isasymunion}\ Wo{\isacharprime}\ {\isasymor}\ S{\isacharprime}\ {\isacharequal}\ Wo{\isacharprime}{\isachardoublequoteclose}\isanewline
\ \ \ \ \ \ \ \ \ \ \ \ \isacommand{thus}\isamarkupfalse%
\ {\isachardoublequoteopen}sat\ S{\isacharprime}{\isachardoublequoteclose}\isanewline
\ \ \ \ \ \ \ \ \ \ \ \ \isacommand{proof}\isamarkupfalse%
\ {\isacharparenleft}rule\ disjE{\isacharparenright}\isanewline
\ \ \ \ \ \ \ \ \ \ \ \ \ \ \isacommand{assume}\isamarkupfalse%
\ {\isachardoublequoteopen}S{\isacharprime}\ {\isacharequal}\ {\isacharbraceleft}G{\isacharbraceright}\ {\isasymunion}\ Wo{\isacharprime}{\isachardoublequoteclose}\isanewline
\ \ \ \ \ \ \ \ \ \ \ \ \ \ \isacommand{then}\isamarkupfalse%
\ \isacommand{have}\isamarkupfalse%
\ {\isachardoublequoteopen}S{\isacharprime}\ {\isasymsubseteq}\ {\isacharbraceleft}G{\isacharcomma}H{\isacharbraceright}\ {\isasymunion}\ Wo{\isacharprime}{\isachardoublequoteclose}\isanewline
\ \ \ \ \ \ \ \ \ \ \ \ \ \ \ \ \isacommand{by}\isamarkupfalse%
\ blast\ \isanewline
\ \ \ \ \ \ \ \ \ \ \ \ \ \ \isacommand{thus}\isamarkupfalse%
\ {\isachardoublequoteopen}sat\ S{\isacharprime}{\isachardoublequoteclose}\isanewline
\ \ \ \ \ \ \ \ \ \ \ \ \ \ \ \ \isacommand{using}\isamarkupfalse%
\ {\isacartoucheopen}sat{\isacharparenleft}{\isacharbraceleft}G{\isacharcomma}H{\isacharbraceright}\ {\isasymunion}\ Wo{\isacharprime}{\isacharparenright}{\isacartoucheclose}\ \isacommand{by}\isamarkupfalse%
\ {\isacharparenleft}rule\ sat{\isacharunderscore}mono{\isacharparenright}\isanewline
\ \ \ \ \ \ \ \ \ \ \ \ \isacommand{next}\isamarkupfalse%
\isanewline
\ \ \ \ \ \ \ \ \ \ \ \ \ \ \isacommand{assume}\isamarkupfalse%
\ {\isachardoublequoteopen}S{\isacharprime}\ {\isacharequal}\ {\isacharbraceleft}H{\isacharbraceright}\ {\isasymunion}\ Wo{\isacharprime}\ {\isasymor}\ S{\isacharprime}\ {\isacharequal}\ Wo{\isacharprime}{\isachardoublequoteclose}\isanewline
\ \ \ \ \ \ \ \ \ \ \ \ \ \ \isacommand{thus}\isamarkupfalse%
\ {\isachardoublequoteopen}sat\ S{\isacharprime}{\isachardoublequoteclose}\isanewline
\ \ \ \ \ \ \ \ \ \ \ \ \ \ \isacommand{proof}\isamarkupfalse%
\ {\isacharparenleft}rule\ disjE{\isacharparenright}\isanewline
\ \ \ \ \ \ \ \ \ \ \ \ \ \ \ \ \isacommand{assume}\isamarkupfalse%
\ {\isachardoublequoteopen}S{\isacharprime}\ {\isacharequal}\ {\isacharbraceleft}H{\isacharbraceright}\ {\isasymunion}\ Wo{\isacharprime}{\isachardoublequoteclose}\isanewline
\ \ \ \ \ \ \ \ \ \ \ \ \ \ \ \ \isacommand{then}\isamarkupfalse%
\ \isacommand{have}\isamarkupfalse%
\ {\isachardoublequoteopen}S{\isacharprime}\ {\isasymsubseteq}\ {\isacharbraceleft}G{\isacharcomma}H{\isacharbraceright}\ {\isasymunion}\ Wo{\isacharprime}{\isachardoublequoteclose}\isanewline
\ \ \ \ \ \ \ \ \ \ \ \ \ \ \ \ \ \ \isacommand{by}\isamarkupfalse%
\ blast\ \isanewline
\ \ \ \ \ \ \ \ \ \ \ \ \ \ \ \ \isacommand{thus}\isamarkupfalse%
\ {\isachardoublequoteopen}sat\ S{\isacharprime}{\isachardoublequoteclose}\isanewline
\ \ \ \ \ \ \ \ \ \ \ \ \ \ \ \ \ \ \isacommand{using}\isamarkupfalse%
\ {\isacartoucheopen}sat{\isacharparenleft}{\isacharbraceleft}G{\isacharcomma}H{\isacharbraceright}\ {\isasymunion}\ Wo{\isacharprime}{\isacharparenright}{\isacartoucheclose}\ \isacommand{by}\isamarkupfalse%
\ {\isacharparenleft}rule\ sat{\isacharunderscore}mono{\isacharparenright}\isanewline
\ \ \ \ \ \ \ \ \ \ \ \ \ \ \isacommand{next}\isamarkupfalse%
\isanewline
\ \ \ \ \ \ \ \ \ \ \ \ \ \ \ \ \isacommand{assume}\isamarkupfalse%
\ {\isachardoublequoteopen}S{\isacharprime}\ {\isacharequal}\ Wo{\isacharprime}{\isachardoublequoteclose}\isanewline
\ \ \ \ \ \ \ \ \ \ \ \ \ \ \ \ \isacommand{then}\isamarkupfalse%
\ \isacommand{have}\isamarkupfalse%
\ {\isachardoublequoteopen}S{\isacharprime}\ {\isasymsubseteq}\ {\isacharbraceleft}G{\isacharcomma}H{\isacharbraceright}\ {\isasymunion}\ Wo{\isacharprime}{\isachardoublequoteclose}\isanewline
\ \ \ \ \ \ \ \ \ \ \ \ \ \ \ \ \ \ \isacommand{by}\isamarkupfalse%
\ {\isacharparenleft}simp\ only{\isacharcolon}\ Un{\isacharunderscore}upper{\isadigit{2}}{\isacharparenright}\isanewline
\ \ \ \ \ \ \ \ \ \ \ \ \ \ \ \ \isacommand{thus}\isamarkupfalse%
\ {\isachardoublequoteopen}sat\ S{\isacharprime}{\isachardoublequoteclose}\isanewline
\ \ \ \ \ \ \ \ \ \ \ \ \ \ \ \ \ \ \isacommand{using}\isamarkupfalse%
\ {\isacartoucheopen}sat{\isacharparenleft}{\isacharbraceleft}G{\isacharcomma}H{\isacharbraceright}\ {\isasymunion}\ Wo{\isacharprime}{\isacharparenright}{\isacartoucheclose}\ \isacommand{by}\isamarkupfalse%
\ {\isacharparenleft}rule\ sat{\isacharunderscore}mono{\isacharparenright}\isanewline
\ \ \ \ \ \ \ \ \ \ \ \ \ \ \isacommand{qed}\isamarkupfalse%
\isanewline
\ \ \ \ \ \ \ \ \ \ \ \ \isacommand{qed}\isamarkupfalse%
\isanewline
\ \ \ \ \ \ \ \ \ \ \isacommand{qed}\isamarkupfalse%
\isanewline
\ \ \ \ \ \ \ \ \isacommand{qed}\isamarkupfalse%
\isanewline
\ \ \ \ \ \ \isacommand{qed}\isamarkupfalse%
\isanewline
\ \ \ \ \isacommand{qed}\isamarkupfalse%
\isanewline
\ \ \isacommand{qed}\isamarkupfalse%
\isanewline
\isacommand{qed}\isamarkupfalse%
%
\endisatagproof
{\isafoldproof}%
%
\isadelimproof
\isanewline
%
\endisadelimproof
\isanewline
\isacommand{lemma}\isamarkupfalse%
\ pcp{\isacharunderscore}colecComp{\isacharunderscore}DIS{\isacharunderscore}sat{\isadigit{1}}{\isacharcolon}\isanewline
\ \ \isakeyword{assumes}\ {\isachardoublequoteopen}W\ {\isasymin}\ colecComp{\isachardoublequoteclose}\isanewline
\ \ \ \ \ \ \ \ \ \ {\isachardoublequoteopen}F\ {\isacharequal}\ G\ \isactrlbold {\isasymor}\ H{\isachardoublequoteclose}\isanewline
\ \ \ \ \ \ \ \ \ \ {\isachardoublequoteopen}F\ {\isasymin}\ W{\isachardoublequoteclose}\isanewline
\ \ \ \ \ \ \ \ \ \ {\isachardoublequoteopen}finite\ Wo{\isachardoublequoteclose}\isanewline
\ \ \ \ \ \ \ \ \ \ {\isachardoublequoteopen}Wo\ {\isasymsubseteq}\ W{\isachardoublequoteclose}\isanewline
\ \ \ \ \ \ \ \ \isakeyword{shows}\ {\isachardoublequoteopen}sat\ {\isacharparenleft}{\isacharbraceleft}G{\isacharcomma}F{\isacharbraceright}\ {\isasymunion}\ Wo{\isacharparenright}\ {\isasymor}\ sat\ {\isacharparenleft}{\isacharbraceleft}H{\isacharcomma}F{\isacharbraceright}\ {\isasymunion}\ Wo{\isacharparenright}{\isachardoublequoteclose}\isanewline
%
\isadelimproof
%
\endisadelimproof
%
\isatagproof
\isacommand{proof}\isamarkupfalse%
\ {\isacharminus}\isanewline
\ \ \isacommand{have}\isamarkupfalse%
\ {\isachardoublequoteopen}sat\ {\isacharparenleft}{\isacharbraceleft}F{\isacharbraceright}\ {\isasymunion}\ Wo{\isacharparenright}{\isachardoublequoteclose}\isanewline
\ \ \ \ \isacommand{using}\isamarkupfalse%
\ assms{\isacharparenleft}{\isadigit{1}}{\isacharcomma}{\isadigit{3}}{\isacharcomma}{\isadigit{4}}{\isacharcomma}{\isadigit{5}}{\isacharparenright}\ \isacommand{by}\isamarkupfalse%
\ {\isacharparenleft}rule\ pcp{\isacharunderscore}colecComp{\isacharunderscore}elem{\isacharunderscore}sat{\isacharparenright}\isanewline
\ \ \isacommand{have}\isamarkupfalse%
\ {\isachardoublequoteopen}F\ {\isasymin}\ {\isacharbraceleft}F{\isacharbraceright}\ {\isasymunion}\ Wo{\isachardoublequoteclose}\isanewline
\ \ \ \ \isacommand{by}\isamarkupfalse%
\ simp\ \isanewline
\ \ \isacommand{have}\isamarkupfalse%
\ Ex{\isadigit{1}}{\isacharcolon}{\isachardoublequoteopen}{\isasymexists}{\isasymA}{\isachardot}\ {\isasymforall}F\ {\isasymin}\ {\isacharparenleft}{\isacharbraceleft}F{\isacharbraceright}\ {\isasymunion}\ Wo{\isacharparenright}{\isachardot}\ {\isasymA}\ {\isasymTurnstile}\ F{\isachardoublequoteclose}\isanewline
\ \ \ \ \isacommand{using}\isamarkupfalse%
\ {\isacartoucheopen}sat\ {\isacharparenleft}{\isacharbraceleft}F{\isacharbraceright}\ {\isasymunion}\ Wo{\isacharparenright}{\isacartoucheclose}\ \isacommand{by}\isamarkupfalse%
\ {\isacharparenleft}simp\ only{\isacharcolon}\ sat{\isacharunderscore}def{\isacharparenright}\isanewline
\ \ \isacommand{obtain}\isamarkupfalse%
\ {\isasymA}\ \isakeyword{where}\ Forall{\isadigit{1}}{\isacharcolon}{\isachardoublequoteopen}{\isasymforall}F\ {\isasymin}\ {\isacharparenleft}{\isacharbraceleft}F{\isacharbraceright}\ {\isasymunion}\ Wo{\isacharparenright}{\isachardot}\ {\isasymA}\ {\isasymTurnstile}\ F{\isachardoublequoteclose}\isanewline
\ \ \ \ \isacommand{using}\isamarkupfalse%
\ Ex{\isadigit{1}}\ \isacommand{by}\isamarkupfalse%
\ {\isacharparenleft}rule\ exE{\isacharparenright}\isanewline
\ \ \isacommand{have}\isamarkupfalse%
\ {\isachardoublequoteopen}{\isasymA}\ {\isasymTurnstile}\ F{\isachardoublequoteclose}\isanewline
\ \ \ \ \isacommand{using}\isamarkupfalse%
\ Forall{\isadigit{1}}\ {\isacartoucheopen}F\ {\isasymin}\ {\isacharbraceleft}F{\isacharbraceright}\ {\isasymunion}\ Wo{\isacartoucheclose}\ \isacommand{by}\isamarkupfalse%
\ {\isacharparenleft}rule\ bspec{\isacharparenright}\isanewline
\ \ \isacommand{then}\isamarkupfalse%
\ \isacommand{have}\isamarkupfalse%
\ {\isachardoublequoteopen}{\isasymA}\ {\isasymTurnstile}\ {\isacharparenleft}G\ \isactrlbold {\isasymor}\ H{\isacharparenright}{\isachardoublequoteclose}\isanewline
\ \ \ \ \isacommand{using}\isamarkupfalse%
\ assms{\isacharparenleft}{\isadigit{2}}{\isacharparenright}\ \isacommand{by}\isamarkupfalse%
\ {\isacharparenleft}simp\ only{\isacharcolon}\ {\isacartoucheopen}{\isasymA}\ {\isasymTurnstile}\ F{\isacartoucheclose}{\isacharparenright}\isanewline
\ \ \isacommand{then}\isamarkupfalse%
\ \isacommand{have}\isamarkupfalse%
\ {\isachardoublequoteopen}{\isasymA}\ {\isasymTurnstile}\ G\ {\isasymor}\ {\isasymA}\ {\isasymTurnstile}\ H{\isachardoublequoteclose}\isanewline
\ \ \ \ \isacommand{by}\isamarkupfalse%
\ {\isacharparenleft}simp\ only{\isacharcolon}\ formula{\isacharunderscore}semantics{\isachardot}simps{\isacharparenleft}{\isadigit{5}}{\isacharparenright}{\isacharparenright}\isanewline
\ \ \isacommand{thus}\isamarkupfalse%
\ {\isacharquery}thesis\isanewline
\ \ \isacommand{proof}\isamarkupfalse%
\ {\isacharparenleft}rule\ disjE{\isacharparenright}\isanewline
\ \ \ \ \isacommand{assume}\isamarkupfalse%
\ {\isachardoublequoteopen}{\isasymA}\ {\isasymTurnstile}\ G{\isachardoublequoteclose}\isanewline
\ \ \ \ \isacommand{then}\isamarkupfalse%
\ \isacommand{have}\isamarkupfalse%
\ {\isachardoublequoteopen}{\isasymforall}F\ {\isasymin}\ {\isacharbraceleft}G{\isacharbraceright}{\isachardot}\ {\isasymA}\ {\isasymTurnstile}\ F{\isachardoublequoteclose}\isanewline
\ \ \ \ \ \ \isacommand{by}\isamarkupfalse%
\ simp\isanewline
\ \ \ \ \isacommand{then}\isamarkupfalse%
\ \isacommand{have}\isamarkupfalse%
\ {\isachardoublequoteopen}{\isasymforall}F\ {\isasymin}\ {\isacharparenleft}{\isacharbraceleft}G{\isacharbraceright}\ {\isasymunion}\ {\isacharparenleft}{\isacharbraceleft}F{\isacharbraceright}\ {\isasymunion}\ Wo{\isacharparenright}{\isacharparenright}{\isachardot}\ {\isasymA}\ {\isasymTurnstile}\ F{\isachardoublequoteclose}\isanewline
\ \ \ \ \ \ \isacommand{using}\isamarkupfalse%
\ Forall{\isadigit{1}}\ \isacommand{by}\isamarkupfalse%
\ {\isacharparenleft}rule\ ball{\isacharunderscore}Un{\isacharparenright}\isanewline
\ \ \ \ \isacommand{then}\isamarkupfalse%
\ \isacommand{have}\isamarkupfalse%
\ {\isachardoublequoteopen}{\isasymforall}F\ {\isasymin}\ {\isacharbraceleft}G{\isacharcomma}F{\isacharbraceright}\ {\isasymunion}\ Wo{\isachardot}\ {\isasymA}\ {\isasymTurnstile}\ F{\isachardoublequoteclose}\isanewline
\ \ \ \ \ \ \isacommand{by}\isamarkupfalse%
\ simp\ \isanewline
\ \ \ \ \isacommand{then}\isamarkupfalse%
\ \isacommand{have}\isamarkupfalse%
\ {\isachardoublequoteopen}{\isasymexists}{\isasymA}{\isachardot}\ {\isasymforall}F\ {\isasymin}\ {\isacharparenleft}{\isacharbraceleft}G{\isacharcomma}F{\isacharbraceright}\ {\isasymunion}\ Wo{\isacharparenright}{\isachardot}\ {\isasymA}\ {\isasymTurnstile}\ F{\isachardoublequoteclose}\isanewline
\ \ \ \ \ \ \isacommand{by}\isamarkupfalse%
\ {\isacharparenleft}iprover\ intro{\isacharcolon}\ exI{\isacharparenright}\isanewline
\ \ \ \ \isacommand{then}\isamarkupfalse%
\ \isacommand{have}\isamarkupfalse%
\ {\isachardoublequoteopen}sat\ {\isacharparenleft}{\isacharbraceleft}G{\isacharcomma}F{\isacharbraceright}\ {\isasymunion}\ Wo{\isacharparenright}{\isachardoublequoteclose}\isanewline
\ \ \ \ \ \ \isacommand{by}\isamarkupfalse%
\ {\isacharparenleft}simp\ only{\isacharcolon}\ sat{\isacharunderscore}def{\isacharparenright}\isanewline
\ \ \ \ \isacommand{thus}\isamarkupfalse%
\ {\isacharquery}thesis\isanewline
\ \ \ \ \ \ \isacommand{by}\isamarkupfalse%
\ {\isacharparenleft}rule\ disjI{\isadigit{1}}{\isacharparenright}\isanewline
\ \ \isacommand{next}\isamarkupfalse%
\isanewline
\ \ \ \ \isacommand{assume}\isamarkupfalse%
\ {\isachardoublequoteopen}{\isasymA}\ {\isasymTurnstile}\ H{\isachardoublequoteclose}\isanewline
\ \ \ \ \isacommand{then}\isamarkupfalse%
\ \isacommand{have}\isamarkupfalse%
\ {\isachardoublequoteopen}{\isasymforall}F\ {\isasymin}\ {\isacharbraceleft}H{\isacharbraceright}{\isachardot}\ {\isasymA}\ {\isasymTurnstile}\ F{\isachardoublequoteclose}\isanewline
\ \ \ \ \ \ \isacommand{by}\isamarkupfalse%
\ simp\isanewline
\ \ \ \ \isacommand{then}\isamarkupfalse%
\ \isacommand{have}\isamarkupfalse%
\ {\isachardoublequoteopen}{\isasymforall}F\ {\isasymin}\ {\isacharparenleft}{\isacharbraceleft}H{\isacharbraceright}\ {\isasymunion}\ {\isacharparenleft}{\isacharbraceleft}F{\isacharbraceright}\ {\isasymunion}\ Wo{\isacharparenright}{\isacharparenright}{\isachardot}\ {\isasymA}\ {\isasymTurnstile}\ F{\isachardoublequoteclose}\isanewline
\ \ \ \ \ \ \isacommand{using}\isamarkupfalse%
\ Forall{\isadigit{1}}\ \isacommand{by}\isamarkupfalse%
\ {\isacharparenleft}rule\ ball{\isacharunderscore}Un{\isacharparenright}\isanewline
\ \ \ \ \isacommand{then}\isamarkupfalse%
\ \isacommand{have}\isamarkupfalse%
\ {\isachardoublequoteopen}{\isasymforall}F\ {\isasymin}\ {\isacharbraceleft}H{\isacharcomma}F{\isacharbraceright}\ {\isasymunion}\ Wo{\isachardot}\ {\isasymA}\ {\isasymTurnstile}\ F{\isachardoublequoteclose}\isanewline
\ \ \ \ \ \ \isacommand{by}\isamarkupfalse%
\ simp\isanewline
\ \ \ \ \isacommand{then}\isamarkupfalse%
\ \isacommand{have}\isamarkupfalse%
\ {\isachardoublequoteopen}{\isasymexists}{\isasymA}{\isachardot}\ {\isasymforall}F\ {\isasymin}\ {\isacharparenleft}{\isacharbraceleft}H{\isacharcomma}F{\isacharbraceright}\ {\isasymunion}\ Wo{\isacharparenright}{\isachardot}\ {\isasymA}\ {\isasymTurnstile}\ F{\isachardoublequoteclose}\isanewline
\ \ \ \ \ \ \isacommand{by}\isamarkupfalse%
\ {\isacharparenleft}iprover\ intro{\isacharcolon}\ exI{\isacharparenright}\isanewline
\ \ \ \ \isacommand{then}\isamarkupfalse%
\ \isacommand{have}\isamarkupfalse%
\ {\isachardoublequoteopen}sat\ {\isacharparenleft}{\isacharbraceleft}H{\isacharcomma}F{\isacharbraceright}\ {\isasymunion}\ Wo{\isacharparenright}{\isachardoublequoteclose}\isanewline
\ \ \ \ \ \ \isacommand{by}\isamarkupfalse%
\ {\isacharparenleft}simp\ only{\isacharcolon}\ sat{\isacharunderscore}def{\isacharparenright}\isanewline
\ \ \ \ \isacommand{thus}\isamarkupfalse%
\ {\isacharquery}thesis\isanewline
\ \ \ \ \ \ \isacommand{by}\isamarkupfalse%
\ {\isacharparenleft}rule\ disjI{\isadigit{2}}{\isacharparenright}\isanewline
\ \ \isacommand{qed}\isamarkupfalse%
\isanewline
\isacommand{qed}\isamarkupfalse%
%
\endisatagproof
{\isafoldproof}%
%
\isadelimproof
\isanewline
%
\endisadelimproof
\isanewline
\isacommand{lemma}\isamarkupfalse%
\ pcp{\isacharunderscore}colecComp{\isacharunderscore}DIS{\isacharunderscore}sat{\isadigit{2}}{\isacharcolon}\isanewline
\ \ \isakeyword{assumes}\ {\isachardoublequoteopen}W\ {\isasymin}\ colecComp{\isachardoublequoteclose}\isanewline
\ \ \ \ \ \ \ \ \ \ {\isachardoublequoteopen}F\ {\isacharequal}\ G\ \isactrlbold {\isasymrightarrow}\ H{\isachardoublequoteclose}\isanewline
\ \ \ \ \ \ \ \ \ \ {\isachardoublequoteopen}F\ {\isasymin}\ W{\isachardoublequoteclose}\isanewline
\ \ \ \ \ \ \ \ \ \ {\isachardoublequoteopen}finite\ Wo{\isachardoublequoteclose}\isanewline
\ \ \ \ \ \ \ \ \ \ {\isachardoublequoteopen}Wo\ {\isasymsubseteq}\ W{\isachardoublequoteclose}\isanewline
\ \ \ \ \ \ \ \ \isakeyword{shows}\ {\isachardoublequoteopen}sat\ {\isacharparenleft}{\isacharbraceleft}\isactrlbold {\isasymnot}\ G{\isacharcomma}F{\isacharbraceright}\ {\isasymunion}\ Wo{\isacharparenright}\ {\isasymor}\ sat\ {\isacharparenleft}{\isacharbraceleft}H{\isacharcomma}F{\isacharbraceright}\ {\isasymunion}\ Wo{\isacharparenright}{\isachardoublequoteclose}\isanewline
%
\isadelimproof
%
\endisadelimproof
%
\isatagproof
\isacommand{proof}\isamarkupfalse%
\ {\isacharminus}\isanewline
\ \ \isacommand{have}\isamarkupfalse%
\ {\isachardoublequoteopen}sat\ {\isacharparenleft}{\isacharbraceleft}F{\isacharbraceright}\ {\isasymunion}\ Wo{\isacharparenright}{\isachardoublequoteclose}\isanewline
\ \ \ \ \isacommand{using}\isamarkupfalse%
\ assms{\isacharparenleft}{\isadigit{1}}{\isacharcomma}{\isadigit{3}}{\isacharcomma}{\isadigit{4}}{\isacharcomma}{\isadigit{5}}{\isacharparenright}\ \isacommand{by}\isamarkupfalse%
\ {\isacharparenleft}rule\ pcp{\isacharunderscore}colecComp{\isacharunderscore}elem{\isacharunderscore}sat{\isacharparenright}\isanewline
\ \ \isacommand{have}\isamarkupfalse%
\ {\isachardoublequoteopen}F\ {\isasymin}\ {\isacharbraceleft}F{\isacharbraceright}\ {\isasymunion}\ Wo{\isachardoublequoteclose}\isanewline
\ \ \ \ \isacommand{by}\isamarkupfalse%
\ simp\isanewline
\ \ \isacommand{have}\isamarkupfalse%
\ Ex{\isadigit{1}}{\isacharcolon}{\isachardoublequoteopen}{\isasymexists}{\isasymA}{\isachardot}\ {\isasymforall}F\ {\isasymin}\ {\isacharparenleft}{\isacharbraceleft}F{\isacharbraceright}\ {\isasymunion}\ Wo{\isacharparenright}{\isachardot}\ {\isasymA}\ {\isasymTurnstile}\ F{\isachardoublequoteclose}\isanewline
\ \ \ \ \isacommand{using}\isamarkupfalse%
\ {\isacartoucheopen}sat\ {\isacharparenleft}{\isacharbraceleft}F{\isacharbraceright}\ {\isasymunion}\ Wo{\isacharparenright}{\isacartoucheclose}\ \isacommand{by}\isamarkupfalse%
\ {\isacharparenleft}simp\ only{\isacharcolon}\ sat{\isacharunderscore}def{\isacharparenright}\isanewline
\ \ \isacommand{obtain}\isamarkupfalse%
\ {\isasymA}\ \isakeyword{where}\ Forall{\isadigit{1}}{\isacharcolon}{\isachardoublequoteopen}{\isasymforall}F\ {\isasymin}\ {\isacharparenleft}{\isacharbraceleft}F{\isacharbraceright}\ {\isasymunion}\ Wo{\isacharparenright}{\isachardot}\ {\isasymA}\ {\isasymTurnstile}\ F{\isachardoublequoteclose}\isanewline
\ \ \ \ \isacommand{using}\isamarkupfalse%
\ Ex{\isadigit{1}}\ \isacommand{by}\isamarkupfalse%
\ {\isacharparenleft}rule\ exE{\isacharparenright}\isanewline
\ \ \isacommand{have}\isamarkupfalse%
\ {\isachardoublequoteopen}{\isasymA}\ {\isasymTurnstile}\ F{\isachardoublequoteclose}\isanewline
\ \ \ \ \isacommand{using}\isamarkupfalse%
\ Forall{\isadigit{1}}\ {\isacartoucheopen}F\ {\isasymin}\ {\isacharbraceleft}F{\isacharbraceright}\ {\isasymunion}\ Wo{\isacartoucheclose}\ \isacommand{by}\isamarkupfalse%
\ {\isacharparenleft}rule\ bspec{\isacharparenright}\isanewline
\ \ \isacommand{then}\isamarkupfalse%
\ \isacommand{have}\isamarkupfalse%
\ {\isachardoublequoteopen}{\isasymA}\ {\isasymTurnstile}\ {\isacharparenleft}G\ \isactrlbold {\isasymrightarrow}\ H{\isacharparenright}{\isachardoublequoteclose}\isanewline
\ \ \ \ \isacommand{using}\isamarkupfalse%
\ assms{\isacharparenleft}{\isadigit{2}}{\isacharparenright}\ \isacommand{by}\isamarkupfalse%
\ {\isacharparenleft}simp\ only{\isacharcolon}\ {\isacartoucheopen}{\isasymA}\ {\isasymTurnstile}\ F{\isacartoucheclose}{\isacharparenright}\isanewline
\ \ \isacommand{then}\isamarkupfalse%
\ \isacommand{have}\isamarkupfalse%
\ {\isachardoublequoteopen}{\isasymA}\ {\isasymTurnstile}\ G\ {\isasymlongrightarrow}\ {\isasymA}\ {\isasymTurnstile}\ H{\isachardoublequoteclose}\isanewline
\ \ \ \ \isacommand{by}\isamarkupfalse%
\ {\isacharparenleft}simp\ only{\isacharcolon}\ formula{\isacharunderscore}semantics{\isachardot}simps{\isacharparenleft}{\isadigit{6}}{\isacharparenright}{\isacharparenright}\isanewline
\ \ \isacommand{then}\isamarkupfalse%
\ \isacommand{have}\isamarkupfalse%
\ {\isachardoublequoteopen}{\isacharparenleft}{\isasymnot}{\isacharparenleft}{\isasymnot}\ {\isasymA}\ {\isasymTurnstile}\ G{\isacharparenright}{\isacharparenright}\ {\isasymlongrightarrow}\ {\isasymA}\ {\isasymTurnstile}\ H{\isachardoublequoteclose}\isanewline
\ \ \ \ \isacommand{by}\isamarkupfalse%
\ {\isacharparenleft}simp\ only{\isacharcolon}\ not{\isacharunderscore}not{\isacharparenright}\isanewline
\ \ \isacommand{then}\isamarkupfalse%
\ \isacommand{have}\isamarkupfalse%
\ {\isachardoublequoteopen}{\isacharparenleft}{\isasymnot}\ {\isasymA}\ {\isasymTurnstile}\ G{\isacharparenright}\ {\isasymor}\ {\isasymA}\ {\isasymTurnstile}\ H{\isachardoublequoteclose}\isanewline
\ \ \ \ \isacommand{by}\isamarkupfalse%
\ {\isacharparenleft}simp\ only{\isacharcolon}\ disj{\isacharunderscore}imp{\isacharparenright}\isanewline
\ \ \isacommand{thus}\isamarkupfalse%
\ {\isacharquery}thesis\isanewline
\ \ \isacommand{proof}\isamarkupfalse%
\ {\isacharparenleft}rule\ disjE{\isacharparenright}\isanewline
\ \ \ \ \isacommand{assume}\isamarkupfalse%
\ {\isachardoublequoteopen}{\isasymnot}\ {\isasymA}\ {\isasymTurnstile}\ G{\isachardoublequoteclose}\isanewline
\ \ \ \ \isacommand{then}\isamarkupfalse%
\ \isacommand{have}\isamarkupfalse%
\ {\isachardoublequoteopen}{\isasymA}\ {\isasymTurnstile}\ {\isacharparenleft}\isactrlbold {\isasymnot}\ G{\isacharparenright}{\isachardoublequoteclose}\isanewline
\ \ \ \ \ \ \isacommand{by}\isamarkupfalse%
\ {\isacharparenleft}simp\ only{\isacharcolon}\ formula{\isacharunderscore}semantics{\isachardot}simps{\isacharparenleft}{\isadigit{3}}{\isacharparenright}\ simp{\isacharunderscore}thms{\isacharparenleft}{\isadigit{8}}{\isacharparenright}{\isacharparenright}\isanewline
\ \ \ \ \isacommand{then}\isamarkupfalse%
\ \isacommand{have}\isamarkupfalse%
\ {\isachardoublequoteopen}{\isasymforall}F\ {\isasymin}\ {\isacharbraceleft}\isactrlbold {\isasymnot}\ G{\isacharbraceright}{\isachardot}\ {\isasymA}\ {\isasymTurnstile}\ F{\isachardoublequoteclose}\isanewline
\ \ \ \ \ \ \isacommand{by}\isamarkupfalse%
\ simp\isanewline
\ \ \ \ \isacommand{then}\isamarkupfalse%
\ \isacommand{have}\isamarkupfalse%
\ {\isachardoublequoteopen}{\isasymforall}F\ {\isasymin}\ {\isacharparenleft}{\isacharbraceleft}\isactrlbold {\isasymnot}\ G{\isacharbraceright}\ {\isasymunion}\ {\isacharparenleft}{\isacharbraceleft}F{\isacharbraceright}\ {\isasymunion}\ Wo{\isacharparenright}{\isacharparenright}{\isachardot}\ {\isasymA}\ {\isasymTurnstile}\ F{\isachardoublequoteclose}\isanewline
\ \ \ \ \ \ \isacommand{using}\isamarkupfalse%
\ Forall{\isadigit{1}}\ \isacommand{by}\isamarkupfalse%
\ {\isacharparenleft}rule\ ball{\isacharunderscore}Un{\isacharparenright}\isanewline
\ \ \ \ \isacommand{then}\isamarkupfalse%
\ \isacommand{have}\isamarkupfalse%
\ {\isachardoublequoteopen}{\isasymforall}F\ {\isasymin}\ {\isacharbraceleft}\isactrlbold {\isasymnot}\ G{\isacharcomma}F{\isacharbraceright}\ {\isasymunion}\ Wo{\isachardot}\ {\isasymA}\ {\isasymTurnstile}\ F{\isachardoublequoteclose}\isanewline
\ \ \ \ \ \ \isacommand{by}\isamarkupfalse%
\ simp\isanewline
\ \ \ \ \isacommand{then}\isamarkupfalse%
\ \isacommand{have}\isamarkupfalse%
\ {\isachardoublequoteopen}{\isasymexists}{\isasymA}{\isachardot}\ {\isasymforall}F\ {\isasymin}\ {\isacharparenleft}{\isacharbraceleft}\isactrlbold {\isasymnot}\ G{\isacharcomma}F{\isacharbraceright}\ {\isasymunion}\ Wo{\isacharparenright}{\isachardot}\ {\isasymA}\ {\isasymTurnstile}\ F{\isachardoublequoteclose}\isanewline
\ \ \ \ \ \ \isacommand{by}\isamarkupfalse%
\ {\isacharparenleft}iprover\ intro{\isacharcolon}\ exI{\isacharparenright}\isanewline
\ \ \ \ \isacommand{then}\isamarkupfalse%
\ \isacommand{have}\isamarkupfalse%
\ {\isachardoublequoteopen}sat\ {\isacharparenleft}{\isacharbraceleft}\isactrlbold {\isasymnot}\ G{\isacharcomma}F{\isacharbraceright}\ {\isasymunion}\ Wo{\isacharparenright}{\isachardoublequoteclose}\isanewline
\ \ \ \ \ \ \isacommand{by}\isamarkupfalse%
\ {\isacharparenleft}simp\ only{\isacharcolon}\ sat{\isacharunderscore}def{\isacharparenright}\isanewline
\ \ \ \ \isacommand{thus}\isamarkupfalse%
\ {\isacharquery}thesis\isanewline
\ \ \ \ \ \ \isacommand{by}\isamarkupfalse%
\ {\isacharparenleft}rule\ disjI{\isadigit{1}}{\isacharparenright}\isanewline
\ \ \isacommand{next}\isamarkupfalse%
\isanewline
\ \ \ \ \isacommand{assume}\isamarkupfalse%
\ {\isachardoublequoteopen}{\isasymA}\ {\isasymTurnstile}\ H{\isachardoublequoteclose}\isanewline
\ \ \ \ \isacommand{then}\isamarkupfalse%
\ \isacommand{have}\isamarkupfalse%
\ {\isachardoublequoteopen}{\isasymforall}F\ {\isasymin}\ {\isacharbraceleft}H{\isacharbraceright}{\isachardot}\ {\isasymA}\ {\isasymTurnstile}\ F{\isachardoublequoteclose}\isanewline
\ \ \ \ \ \ \isacommand{by}\isamarkupfalse%
\ simp\isanewline
\ \ \ \ \isacommand{then}\isamarkupfalse%
\ \isacommand{have}\isamarkupfalse%
\ {\isachardoublequoteopen}{\isasymforall}F\ {\isasymin}\ {\isacharparenleft}{\isacharbraceleft}H{\isacharbraceright}\ {\isasymunion}\ {\isacharparenleft}{\isacharbraceleft}F{\isacharbraceright}\ {\isasymunion}\ Wo{\isacharparenright}{\isacharparenright}{\isachardot}\ {\isasymA}\ {\isasymTurnstile}\ F{\isachardoublequoteclose}\isanewline
\ \ \ \ \ \ \isacommand{using}\isamarkupfalse%
\ Forall{\isadigit{1}}\ \isacommand{by}\isamarkupfalse%
\ {\isacharparenleft}rule\ ball{\isacharunderscore}Un{\isacharparenright}\isanewline
\ \ \ \ \isacommand{then}\isamarkupfalse%
\ \isacommand{have}\isamarkupfalse%
\ {\isachardoublequoteopen}{\isasymforall}F\ {\isasymin}\ {\isacharbraceleft}H{\isacharcomma}F{\isacharbraceright}\ {\isasymunion}\ Wo{\isachardot}\ {\isasymA}\ {\isasymTurnstile}\ F{\isachardoublequoteclose}\isanewline
\ \ \ \ \ \ \isacommand{by}\isamarkupfalse%
\ simp\isanewline
\ \ \ \ \isacommand{then}\isamarkupfalse%
\ \isacommand{have}\isamarkupfalse%
\ {\isachardoublequoteopen}{\isasymexists}{\isasymA}{\isachardot}\ {\isasymforall}F\ {\isasymin}\ {\isacharparenleft}{\isacharbraceleft}H{\isacharcomma}F{\isacharbraceright}\ {\isasymunion}\ Wo{\isacharparenright}{\isachardot}\ {\isasymA}\ {\isasymTurnstile}\ F{\isachardoublequoteclose}\isanewline
\ \ \ \ \ \ \isacommand{by}\isamarkupfalse%
\ {\isacharparenleft}iprover\ intro{\isacharcolon}\ exI{\isacharparenright}\isanewline
\ \ \ \ \isacommand{then}\isamarkupfalse%
\ \isacommand{have}\isamarkupfalse%
\ {\isachardoublequoteopen}sat\ {\isacharparenleft}{\isacharbraceleft}H{\isacharcomma}F{\isacharbraceright}\ {\isasymunion}\ Wo{\isacharparenright}{\isachardoublequoteclose}\isanewline
\ \ \ \ \ \ \isacommand{by}\isamarkupfalse%
\ {\isacharparenleft}simp\ only{\isacharcolon}\ sat{\isacharunderscore}def{\isacharparenright}\isanewline
\ \ \ \ \isacommand{thus}\isamarkupfalse%
\ {\isacharquery}thesis\isanewline
\ \ \ \ \ \ \isacommand{by}\isamarkupfalse%
\ {\isacharparenleft}rule\ disjI{\isadigit{2}}{\isacharparenright}\isanewline
\ \ \isacommand{qed}\isamarkupfalse%
\isanewline
\isacommand{qed}\isamarkupfalse%
%
\endisatagproof
{\isafoldproof}%
%
\isadelimproof
\isanewline
%
\endisadelimproof
\isanewline
\isacommand{lemma}\isamarkupfalse%
\ pcp{\isacharunderscore}colecComp{\isacharunderscore}DIS{\isacharunderscore}sat{\isadigit{3}}{\isacharcolon}\isanewline
\ \ \isakeyword{assumes}\ {\isachardoublequoteopen}W\ {\isasymin}\ colecComp{\isachardoublequoteclose}\isanewline
\ \ \ \ \ \ \ \ \ \ {\isachardoublequoteopen}F\ {\isacharequal}\ \isactrlbold {\isasymnot}\ {\isacharparenleft}G\ \isactrlbold {\isasymand}\ H{\isacharparenright}{\isachardoublequoteclose}\isanewline
\ \ \ \ \ \ \ \ \ \ {\isachardoublequoteopen}F\ {\isasymin}\ W{\isachardoublequoteclose}\isanewline
\ \ \ \ \ \ \ \ \ \ {\isachardoublequoteopen}finite\ Wo{\isachardoublequoteclose}\isanewline
\ \ \ \ \ \ \ \ \ \ {\isachardoublequoteopen}Wo\ {\isasymsubseteq}\ W{\isachardoublequoteclose}\isanewline
\ \ \ \ \ \ \ \ \isakeyword{shows}\ {\isachardoublequoteopen}sat\ {\isacharparenleft}{\isacharbraceleft}\isactrlbold {\isasymnot}\ G{\isacharcomma}F{\isacharbraceright}\ {\isasymunion}\ Wo{\isacharparenright}\ {\isasymor}\ sat\ {\isacharparenleft}{\isacharbraceleft}\isactrlbold {\isasymnot}\ H{\isacharcomma}F{\isacharbraceright}\ {\isasymunion}\ Wo{\isacharparenright}{\isachardoublequoteclose}\isanewline
%
\isadelimproof
%
\endisadelimproof
%
\isatagproof
\isacommand{proof}\isamarkupfalse%
\ {\isacharminus}\isanewline
\ \ \isacommand{have}\isamarkupfalse%
\ {\isachardoublequoteopen}sat\ {\isacharparenleft}{\isacharbraceleft}F{\isacharbraceright}\ {\isasymunion}\ Wo{\isacharparenright}{\isachardoublequoteclose}\isanewline
\ \ \ \ \isacommand{using}\isamarkupfalse%
\ assms{\isacharparenleft}{\isadigit{1}}{\isacharcomma}{\isadigit{3}}{\isacharcomma}{\isadigit{4}}{\isacharcomma}{\isadigit{5}}{\isacharparenright}\ \isacommand{by}\isamarkupfalse%
\ {\isacharparenleft}rule\ pcp{\isacharunderscore}colecComp{\isacharunderscore}elem{\isacharunderscore}sat{\isacharparenright}\isanewline
\ \ \isacommand{have}\isamarkupfalse%
\ {\isachardoublequoteopen}F\ {\isasymin}\ {\isacharbraceleft}F{\isacharbraceright}\ {\isasymunion}\ Wo{\isachardoublequoteclose}\isanewline
\ \ \ \ \isacommand{by}\isamarkupfalse%
\ simp\isanewline
\ \ \isacommand{have}\isamarkupfalse%
\ Ex{\isadigit{1}}{\isacharcolon}{\isachardoublequoteopen}{\isasymexists}{\isasymA}{\isachardot}\ {\isasymforall}F\ {\isasymin}\ {\isacharparenleft}{\isacharbraceleft}F{\isacharbraceright}\ {\isasymunion}\ Wo{\isacharparenright}{\isachardot}\ {\isasymA}\ {\isasymTurnstile}\ F{\isachardoublequoteclose}\isanewline
\ \ \ \ \isacommand{using}\isamarkupfalse%
\ {\isacartoucheopen}sat\ {\isacharparenleft}{\isacharbraceleft}F{\isacharbraceright}\ {\isasymunion}\ Wo{\isacharparenright}{\isacartoucheclose}\ \isacommand{by}\isamarkupfalse%
\ {\isacharparenleft}simp\ only{\isacharcolon}\ sat{\isacharunderscore}def{\isacharparenright}\isanewline
\ \ \isacommand{obtain}\isamarkupfalse%
\ {\isasymA}\ \isakeyword{where}\ Forall{\isadigit{1}}{\isacharcolon}{\isachardoublequoteopen}{\isasymforall}F\ {\isasymin}\ {\isacharparenleft}{\isacharbraceleft}F{\isacharbraceright}\ {\isasymunion}\ Wo{\isacharparenright}{\isachardot}\ {\isasymA}\ {\isasymTurnstile}\ F{\isachardoublequoteclose}\isanewline
\ \ \ \ \isacommand{using}\isamarkupfalse%
\ Ex{\isadigit{1}}\ \isacommand{by}\isamarkupfalse%
\ {\isacharparenleft}rule\ exE{\isacharparenright}\isanewline
\ \ \isacommand{have}\isamarkupfalse%
\ {\isachardoublequoteopen}{\isasymA}\ {\isasymTurnstile}\ F{\isachardoublequoteclose}\isanewline
\ \ \ \ \isacommand{using}\isamarkupfalse%
\ Forall{\isadigit{1}}\ {\isacartoucheopen}F\ {\isasymin}\ {\isacharbraceleft}F{\isacharbraceright}\ {\isasymunion}\ Wo{\isacartoucheclose}\ \isacommand{by}\isamarkupfalse%
\ {\isacharparenleft}rule\ bspec{\isacharparenright}\isanewline
\ \ \isacommand{then}\isamarkupfalse%
\ \isacommand{have}\isamarkupfalse%
\ {\isachardoublequoteopen}{\isasymA}\ {\isasymTurnstile}\ \isactrlbold {\isasymnot}\ {\isacharparenleft}G\ \isactrlbold {\isasymand}\ H{\isacharparenright}{\isachardoublequoteclose}\isanewline
\ \ \ \ \isacommand{using}\isamarkupfalse%
\ assms{\isacharparenleft}{\isadigit{2}}{\isacharparenright}\ \isacommand{by}\isamarkupfalse%
\ {\isacharparenleft}simp\ only{\isacharcolon}\ {\isacartoucheopen}{\isasymA}\ {\isasymTurnstile}\ F{\isacartoucheclose}{\isacharparenright}\isanewline
\ \ \isacommand{then}\isamarkupfalse%
\ \isacommand{have}\isamarkupfalse%
\ {\isachardoublequoteopen}{\isasymnot}\ {\isacharparenleft}{\isasymA}\ {\isasymTurnstile}\ {\isacharparenleft}G\ \isactrlbold {\isasymand}\ H{\isacharparenright}{\isacharparenright}{\isachardoublequoteclose}\isanewline
\ \ \ \ \isacommand{by}\isamarkupfalse%
\ {\isacharparenleft}simp\ only{\isacharcolon}\ formula{\isacharunderscore}semantics{\isachardot}simps{\isacharparenleft}{\isadigit{3}}{\isacharparenright}\ simp{\isacharunderscore}thms{\isacharparenleft}{\isadigit{8}}{\isacharparenright}{\isacharparenright}\isanewline
\ \ \isacommand{then}\isamarkupfalse%
\ \isacommand{have}\isamarkupfalse%
\ {\isachardoublequoteopen}{\isasymnot}{\isacharparenleft}{\isasymA}\ {\isasymTurnstile}\ G\ {\isasymand}\ {\isasymA}\ {\isasymTurnstile}\ H{\isacharparenright}{\isachardoublequoteclose}\isanewline
\ \ \ \ \isacommand{by}\isamarkupfalse%
\ {\isacharparenleft}simp\ only{\isacharcolon}\ formula{\isacharunderscore}semantics{\isachardot}simps{\isacharparenleft}{\isadigit{4}}{\isacharparenright}\ simp{\isacharunderscore}thms{\isacharparenleft}{\isadigit{8}}{\isacharparenright}{\isacharparenright}\isanewline
\ \ \isacommand{then}\isamarkupfalse%
\ \isacommand{have}\isamarkupfalse%
\ {\isachardoublequoteopen}{\isasymnot}\ {\isacharparenleft}{\isasymA}\ {\isasymTurnstile}\ G{\isacharparenright}\ {\isasymor}\ {\isasymnot}\ {\isacharparenleft}{\isasymA}\ {\isasymTurnstile}\ H{\isacharparenright}{\isachardoublequoteclose}\isanewline
\ \ \ \ \isacommand{by}\isamarkupfalse%
\ {\isacharparenleft}simp\ only{\isacharcolon}\ de{\isacharunderscore}Morgan{\isacharunderscore}conj{\isacharparenright}\isanewline
\ \ \isacommand{thus}\isamarkupfalse%
\ {\isacharquery}thesis\isanewline
\ \ \isacommand{proof}\isamarkupfalse%
\ {\isacharparenleft}rule\ disjE{\isacharparenright}\isanewline
\ \ \ \ \isacommand{assume}\isamarkupfalse%
\ {\isachardoublequoteopen}{\isasymnot}\ {\isacharparenleft}{\isasymA}\ {\isasymTurnstile}\ G{\isacharparenright}{\isachardoublequoteclose}\isanewline
\ \ \ \ \isacommand{then}\isamarkupfalse%
\ \isacommand{have}\isamarkupfalse%
\ {\isachardoublequoteopen}{\isasymA}\ {\isasymTurnstile}\ \isactrlbold {\isasymnot}\ G{\isachardoublequoteclose}\isanewline
\ \ \ \ \ \ \isacommand{by}\isamarkupfalse%
\ {\isacharparenleft}simp\ only{\isacharcolon}\ formula{\isacharunderscore}semantics{\isachardot}simps{\isacharparenleft}{\isadigit{3}}{\isacharparenright}\ simp{\isacharunderscore}thms{\isacharparenleft}{\isadigit{8}}{\isacharparenright}{\isacharparenright}\isanewline
\ \ \ \ \isacommand{then}\isamarkupfalse%
\ \isacommand{have}\isamarkupfalse%
\ {\isachardoublequoteopen}{\isasymforall}F\ {\isasymin}\ {\isacharbraceleft}\isactrlbold {\isasymnot}\ G{\isacharbraceright}{\isachardot}\ {\isasymA}\ {\isasymTurnstile}\ F{\isachardoublequoteclose}\isanewline
\ \ \ \ \ \ \isacommand{by}\isamarkupfalse%
\ simp\isanewline
\ \ \ \ \isacommand{then}\isamarkupfalse%
\ \isacommand{have}\isamarkupfalse%
\ {\isachardoublequoteopen}{\isasymforall}F\ {\isasymin}\ {\isacharparenleft}{\isacharbraceleft}\isactrlbold {\isasymnot}\ G{\isacharbraceright}\ {\isasymunion}\ {\isacharparenleft}{\isacharbraceleft}F{\isacharbraceright}\ {\isasymunion}\ Wo{\isacharparenright}{\isacharparenright}{\isachardot}\ {\isasymA}\ {\isasymTurnstile}\ F{\isachardoublequoteclose}\isanewline
\ \ \ \ \ \ \isacommand{using}\isamarkupfalse%
\ Forall{\isadigit{1}}\ \isacommand{by}\isamarkupfalse%
\ {\isacharparenleft}rule\ ball{\isacharunderscore}Un{\isacharparenright}\isanewline
\ \ \ \ \isacommand{then}\isamarkupfalse%
\ \isacommand{have}\isamarkupfalse%
\ {\isachardoublequoteopen}{\isasymforall}F\ {\isasymin}\ {\isacharbraceleft}\isactrlbold {\isasymnot}\ G{\isacharcomma}F{\isacharbraceright}\ {\isasymunion}\ Wo{\isachardot}\ {\isasymA}\ {\isasymTurnstile}\ F{\isachardoublequoteclose}\isanewline
\ \ \ \ \ \ \isacommand{by}\isamarkupfalse%
\ simp\isanewline
\ \ \ \ \isacommand{then}\isamarkupfalse%
\ \isacommand{have}\isamarkupfalse%
\ {\isachardoublequoteopen}{\isasymexists}{\isasymA}{\isachardot}\ {\isasymforall}F\ {\isasymin}\ {\isacharparenleft}{\isacharbraceleft}\isactrlbold {\isasymnot}\ G{\isacharcomma}F{\isacharbraceright}\ {\isasymunion}\ Wo{\isacharparenright}{\isachardot}\ {\isasymA}\ {\isasymTurnstile}\ F{\isachardoublequoteclose}\isanewline
\ \ \ \ \ \ \isacommand{by}\isamarkupfalse%
\ {\isacharparenleft}iprover\ intro{\isacharcolon}\ exI{\isacharparenright}\isanewline
\ \ \ \ \isacommand{then}\isamarkupfalse%
\ \isacommand{have}\isamarkupfalse%
\ {\isachardoublequoteopen}sat\ {\isacharparenleft}{\isacharbraceleft}\isactrlbold {\isasymnot}\ G{\isacharcomma}F{\isacharbraceright}\ {\isasymunion}\ Wo{\isacharparenright}{\isachardoublequoteclose}\isanewline
\ \ \ \ \ \ \isacommand{by}\isamarkupfalse%
\ {\isacharparenleft}simp\ only{\isacharcolon}\ sat{\isacharunderscore}def{\isacharparenright}\isanewline
\ \ \ \ \isacommand{thus}\isamarkupfalse%
\ {\isacharquery}thesis\isanewline
\ \ \ \ \ \ \isacommand{by}\isamarkupfalse%
\ {\isacharparenleft}rule\ disjI{\isadigit{1}}{\isacharparenright}\isanewline
\ \ \isacommand{next}\isamarkupfalse%
\isanewline
\ \ \ \ \isacommand{assume}\isamarkupfalse%
\ {\isachardoublequoteopen}{\isasymnot}\ {\isacharparenleft}{\isasymA}\ {\isasymTurnstile}\ H{\isacharparenright}{\isachardoublequoteclose}\isanewline
\ \ \ \ \isacommand{then}\isamarkupfalse%
\ \isacommand{have}\isamarkupfalse%
\ {\isachardoublequoteopen}{\isasymA}\ {\isasymTurnstile}\ \isactrlbold {\isasymnot}\ H{\isachardoublequoteclose}\isanewline
\ \ \ \ \ \ \isacommand{by}\isamarkupfalse%
\ {\isacharparenleft}simp\ only{\isacharcolon}\ formula{\isacharunderscore}semantics{\isachardot}simps{\isacharparenleft}{\isadigit{3}}{\isacharparenright}\ simp{\isacharunderscore}thms{\isacharparenleft}{\isadigit{8}}{\isacharparenright}{\isacharparenright}\isanewline
\ \ \ \ \isacommand{then}\isamarkupfalse%
\ \isacommand{have}\isamarkupfalse%
\ {\isachardoublequoteopen}{\isasymforall}F\ {\isasymin}\ {\isacharbraceleft}\isactrlbold {\isasymnot}\ H{\isacharbraceright}{\isachardot}\ {\isasymA}\ {\isasymTurnstile}\ F{\isachardoublequoteclose}\isanewline
\ \ \ \ \ \ \isacommand{by}\isamarkupfalse%
\ simp\isanewline
\ \ \ \ \isacommand{then}\isamarkupfalse%
\ \isacommand{have}\isamarkupfalse%
\ {\isachardoublequoteopen}{\isasymforall}F\ {\isasymin}\ {\isacharparenleft}{\isacharbraceleft}\isactrlbold {\isasymnot}\ H{\isacharbraceright}\ {\isasymunion}\ {\isacharparenleft}{\isacharbraceleft}F{\isacharbraceright}\ {\isasymunion}\ Wo{\isacharparenright}{\isacharparenright}{\isachardot}\ {\isasymA}\ {\isasymTurnstile}\ F{\isachardoublequoteclose}\isanewline
\ \ \ \ \ \ \isacommand{using}\isamarkupfalse%
\ Forall{\isadigit{1}}\ \isacommand{by}\isamarkupfalse%
\ {\isacharparenleft}rule\ ball{\isacharunderscore}Un{\isacharparenright}\isanewline
\ \ \ \ \isacommand{then}\isamarkupfalse%
\ \isacommand{have}\isamarkupfalse%
\ {\isachardoublequoteopen}{\isasymforall}F\ {\isasymin}\ {\isacharbraceleft}\isactrlbold {\isasymnot}\ H{\isacharcomma}F{\isacharbraceright}\ {\isasymunion}\ Wo{\isachardot}\ {\isasymA}\ {\isasymTurnstile}\ F{\isachardoublequoteclose}\isanewline
\ \ \ \ \ \ \isacommand{by}\isamarkupfalse%
\ simp\isanewline
\ \ \ \ \isacommand{then}\isamarkupfalse%
\ \isacommand{have}\isamarkupfalse%
\ {\isachardoublequoteopen}{\isasymexists}{\isasymA}{\isachardot}\ {\isasymforall}F\ {\isasymin}\ {\isacharparenleft}{\isacharbraceleft}\isactrlbold {\isasymnot}\ H{\isacharcomma}F{\isacharbraceright}\ {\isasymunion}\ Wo{\isacharparenright}{\isachardot}\ {\isasymA}\ {\isasymTurnstile}\ F{\isachardoublequoteclose}\isanewline
\ \ \ \ \ \ \isacommand{by}\isamarkupfalse%
\ {\isacharparenleft}iprover\ intro{\isacharcolon}\ exI{\isacharparenright}\isanewline
\ \ \ \ \isacommand{then}\isamarkupfalse%
\ \isacommand{have}\isamarkupfalse%
\ {\isachardoublequoteopen}sat\ {\isacharparenleft}{\isacharbraceleft}\isactrlbold {\isasymnot}\ H{\isacharcomma}F{\isacharbraceright}\ {\isasymunion}\ Wo{\isacharparenright}{\isachardoublequoteclose}\isanewline
\ \ \ \ \ \ \isacommand{by}\isamarkupfalse%
\ {\isacharparenleft}simp\ only{\isacharcolon}\ sat{\isacharunderscore}def{\isacharparenright}\isanewline
\ \ \ \ \isacommand{thus}\isamarkupfalse%
\ {\isacharquery}thesis\isanewline
\ \ \ \ \ \ \isacommand{by}\isamarkupfalse%
\ {\isacharparenleft}rule\ disjI{\isadigit{2}}{\isacharparenright}\isanewline
\ \ \isacommand{qed}\isamarkupfalse%
\isanewline
\isacommand{qed}\isamarkupfalse%
%
\endisatagproof
{\isafoldproof}%
%
\isadelimproof
\isanewline
%
\endisadelimproof
\isanewline
\isacommand{lemma}\isamarkupfalse%
\ pcp{\isacharunderscore}colecComp{\isacharunderscore}DIS{\isacharunderscore}sat{\isadigit{4}}{\isacharcolon}\isanewline
\ \ \isakeyword{assumes}\ {\isachardoublequoteopen}W\ {\isasymin}\ colecComp{\isachardoublequoteclose}\isanewline
\ \ \ \ \ \ \ \ \ \ {\isachardoublequoteopen}F\ {\isacharequal}\ \isactrlbold {\isasymnot}\ {\isacharparenleft}\isactrlbold {\isasymnot}\ G{\isacharparenright}{\isachardoublequoteclose}\isanewline
\ \ \ \ \ \ \ \ \ \ {\isachardoublequoteopen}H\ {\isacharequal}\ G{\isachardoublequoteclose}\isanewline
\ \ \ \ \ \ \ \ \ \ {\isachardoublequoteopen}F\ {\isasymin}\ W{\isachardoublequoteclose}\isanewline
\ \ \ \ \ \ \ \ \ \ {\isachardoublequoteopen}finite\ Wo{\isachardoublequoteclose}\isanewline
\ \ \ \ \ \ \ \ \ \ {\isachardoublequoteopen}Wo\ {\isasymsubseteq}\ W{\isachardoublequoteclose}\isanewline
\ \ \ \ \ \ \ \ \isakeyword{shows}\ {\isachardoublequoteopen}sat\ {\isacharparenleft}{\isacharbraceleft}G{\isacharcomma}F{\isacharbraceright}\ {\isasymunion}\ Wo{\isacharparenright}\ {\isasymor}\ sat\ {\isacharparenleft}{\isacharbraceleft}H{\isacharcomma}F{\isacharbraceright}\ {\isasymunion}\ Wo{\isacharparenright}{\isachardoublequoteclose}\isanewline
%
\isadelimproof
%
\endisadelimproof
%
\isatagproof
\isacommand{proof}\isamarkupfalse%
\ {\isacharminus}\isanewline
\ \ \isacommand{have}\isamarkupfalse%
\ {\isachardoublequoteopen}sat\ {\isacharparenleft}{\isacharbraceleft}F{\isacharbraceright}\ {\isasymunion}\ Wo{\isacharparenright}{\isachardoublequoteclose}\isanewline
\ \ \ \ \isacommand{using}\isamarkupfalse%
\ assms{\isacharparenleft}{\isadigit{1}}{\isacharcomma}{\isadigit{4}}{\isacharcomma}{\isadigit{5}}{\isacharcomma}{\isadigit{6}}{\isacharparenright}\ \isacommand{by}\isamarkupfalse%
\ {\isacharparenleft}rule\ pcp{\isacharunderscore}colecComp{\isacharunderscore}elem{\isacharunderscore}sat{\isacharparenright}\isanewline
\ \ \isacommand{have}\isamarkupfalse%
\ {\isachardoublequoteopen}F\ {\isasymin}\ {\isacharbraceleft}F{\isacharbraceright}\ {\isasymunion}\ Wo{\isachardoublequoteclose}\isanewline
\ \ \ \ \isacommand{by}\isamarkupfalse%
\ simp\ \isanewline
\ \ \isacommand{have}\isamarkupfalse%
\ Ex{\isadigit{1}}{\isacharcolon}{\isachardoublequoteopen}{\isasymexists}{\isasymA}{\isachardot}\ {\isasymforall}F\ {\isasymin}\ {\isacharparenleft}{\isacharbraceleft}F{\isacharbraceright}\ {\isasymunion}\ Wo{\isacharparenright}{\isachardot}\ {\isasymA}\ {\isasymTurnstile}\ F{\isachardoublequoteclose}\isanewline
\ \ \ \ \isacommand{using}\isamarkupfalse%
\ {\isacartoucheopen}sat\ {\isacharparenleft}{\isacharbraceleft}F{\isacharbraceright}\ {\isasymunion}\ Wo{\isacharparenright}{\isacartoucheclose}\ \isacommand{by}\isamarkupfalse%
\ {\isacharparenleft}simp\ only{\isacharcolon}\ sat{\isacharunderscore}def{\isacharparenright}\isanewline
\ \ \isacommand{obtain}\isamarkupfalse%
\ {\isasymA}\ \isakeyword{where}\ Forall{\isadigit{1}}{\isacharcolon}{\isachardoublequoteopen}{\isasymforall}F\ {\isasymin}\ {\isacharparenleft}{\isacharbraceleft}F{\isacharbraceright}\ {\isasymunion}\ Wo{\isacharparenright}{\isachardot}\ {\isasymA}\ {\isasymTurnstile}\ F{\isachardoublequoteclose}\isanewline
\ \ \ \ \isacommand{using}\isamarkupfalse%
\ Ex{\isadigit{1}}\ \isacommand{by}\isamarkupfalse%
\ {\isacharparenleft}rule\ exE{\isacharparenright}\isanewline
\ \ \isacommand{have}\isamarkupfalse%
\ {\isachardoublequoteopen}{\isasymA}\ {\isasymTurnstile}\ F{\isachardoublequoteclose}\isanewline
\ \ \ \ \isacommand{using}\isamarkupfalse%
\ Forall{\isadigit{1}}\ {\isacartoucheopen}F\ {\isasymin}\ {\isacharbraceleft}F{\isacharbraceright}\ {\isasymunion}\ Wo{\isacartoucheclose}\ \isacommand{by}\isamarkupfalse%
\ {\isacharparenleft}rule\ bspec{\isacharparenright}\isanewline
\ \ \isacommand{then}\isamarkupfalse%
\ \isacommand{have}\isamarkupfalse%
\ {\isachardoublequoteopen}{\isasymA}\ {\isasymTurnstile}\ \isactrlbold {\isasymnot}{\isacharparenleft}\isactrlbold {\isasymnot}\ G{\isacharparenright}{\isachardoublequoteclose}\isanewline
\ \ \ \ \isacommand{using}\isamarkupfalse%
\ assms{\isacharparenleft}{\isadigit{2}}{\isacharparenright}\ \isacommand{by}\isamarkupfalse%
\ {\isacharparenleft}simp\ only{\isacharcolon}\ {\isacartoucheopen}{\isasymA}\ {\isasymTurnstile}\ F{\isacartoucheclose}{\isacharparenright}\isanewline
\ \ \isacommand{then}\isamarkupfalse%
\ \isacommand{have}\isamarkupfalse%
\ {\isachardoublequoteopen}{\isasymnot}\ {\isasymA}\ {\isasymTurnstile}\ \isactrlbold {\isasymnot}\ G{\isachardoublequoteclose}\isanewline
\ \ \ \ \isacommand{by}\isamarkupfalse%
\ {\isacharparenleft}simp\ only{\isacharcolon}\ formula{\isacharunderscore}semantics{\isachardot}simps{\isacharparenleft}{\isadigit{3}}{\isacharparenright}\ simp{\isacharunderscore}thms{\isacharparenleft}{\isadigit{8}}{\isacharparenright}{\isacharparenright}\isanewline
\ \ \isacommand{then}\isamarkupfalse%
\ \isacommand{have}\isamarkupfalse%
\ {\isachardoublequoteopen}{\isasymnot}\ {\isasymnot}{\isasymA}\ {\isasymTurnstile}\ G{\isachardoublequoteclose}\isanewline
\ \ \ \ \isacommand{by}\isamarkupfalse%
\ {\isacharparenleft}simp\ only{\isacharcolon}\ formula{\isacharunderscore}semantics{\isachardot}simps{\isacharparenleft}{\isadigit{3}}{\isacharparenright}\ simp{\isacharunderscore}thms{\isacharparenleft}{\isadigit{8}}{\isacharparenright}{\isacharparenright}\isanewline
\ \ \isacommand{then}\isamarkupfalse%
\ \isacommand{have}\isamarkupfalse%
\ {\isachardoublequoteopen}{\isasymA}\ {\isasymTurnstile}\ G{\isachardoublequoteclose}\isanewline
\ \ \ \ \isacommand{by}\isamarkupfalse%
\ {\isacharparenleft}rule\ notnotD{\isacharparenright}\isanewline
\ \ \isacommand{then}\isamarkupfalse%
\ \isacommand{have}\isamarkupfalse%
\ {\isachardoublequoteopen}{\isasymforall}F\ {\isasymin}\ {\isacharbraceleft}G{\isacharbraceright}{\isachardot}\ {\isasymA}\ {\isasymTurnstile}\ F{\isachardoublequoteclose}\isanewline
\ \ \ \ \isacommand{by}\isamarkupfalse%
\ simp\isanewline
\ \ \isacommand{then}\isamarkupfalse%
\ \isacommand{have}\isamarkupfalse%
\ {\isachardoublequoteopen}{\isasymforall}F\ {\isasymin}\ {\isacharparenleft}{\isacharbraceleft}G{\isacharbraceright}\ {\isasymunion}\ {\isacharparenleft}{\isacharbraceleft}F{\isacharbraceright}\ {\isasymunion}\ Wo{\isacharparenright}{\isacharparenright}{\isachardot}\ {\isasymA}\ {\isasymTurnstile}\ F{\isachardoublequoteclose}\isanewline
\ \ \ \ \isacommand{using}\isamarkupfalse%
\ Forall{\isadigit{1}}\ \isacommand{by}\isamarkupfalse%
\ {\isacharparenleft}rule\ ball{\isacharunderscore}Un{\isacharparenright}\isanewline
\ \ \isacommand{then}\isamarkupfalse%
\ \isacommand{have}\isamarkupfalse%
\ {\isachardoublequoteopen}{\isasymforall}F\ {\isasymin}\ {\isacharbraceleft}G{\isacharcomma}F{\isacharbraceright}\ {\isasymunion}\ Wo{\isachardot}\ {\isasymA}\ {\isasymTurnstile}\ F{\isachardoublequoteclose}\isanewline
\ \ \ \ \isacommand{by}\isamarkupfalse%
\ simp\isanewline
\ \ \isacommand{then}\isamarkupfalse%
\ \isacommand{have}\isamarkupfalse%
\ {\isachardoublequoteopen}{\isasymexists}{\isasymA}{\isachardot}\ {\isasymforall}F\ {\isasymin}\ {\isacharparenleft}{\isacharbraceleft}G{\isacharcomma}F{\isacharbraceright}\ {\isasymunion}\ Wo{\isacharparenright}{\isachardot}\ {\isasymA}\ {\isasymTurnstile}\ F{\isachardoublequoteclose}\isanewline
\ \ \ \ \isacommand{by}\isamarkupfalse%
\ {\isacharparenleft}iprover\ intro{\isacharcolon}\ exI{\isacharparenright}\isanewline
\ \ \isacommand{then}\isamarkupfalse%
\ \isacommand{have}\isamarkupfalse%
\ {\isachardoublequoteopen}sat\ {\isacharparenleft}{\isacharbraceleft}G{\isacharcomma}F{\isacharbraceright}\ {\isasymunion}\ Wo{\isacharparenright}{\isachardoublequoteclose}\isanewline
\ \ \ \ \isacommand{by}\isamarkupfalse%
\ {\isacharparenleft}simp\ only{\isacharcolon}\ sat{\isacharunderscore}def{\isacharparenright}\isanewline
\ \ \isacommand{thus}\isamarkupfalse%
\ {\isacharquery}thesis\isanewline
\ \ \ \ \isacommand{by}\isamarkupfalse%
\ {\isacharparenleft}rule\ disjI{\isadigit{1}}{\isacharparenright}\isanewline
\isacommand{qed}\isamarkupfalse%
%
\endisatagproof
{\isafoldproof}%
%
\isadelimproof
\isanewline
%
\endisadelimproof
\isanewline
\isacommand{lemma}\isamarkupfalse%
\ pcp{\isacharunderscore}colecComp{\isacharunderscore}DIS{\isacharunderscore}sat{\isacharcolon}\isanewline
\ \ \isakeyword{assumes}\ {\isachardoublequoteopen}W\ {\isasymin}\ colecComp{\isachardoublequoteclose}\isanewline
\ \ \ \ \ \ \ \ \ \ {\isachardoublequoteopen}Dis\ F\ G\ H{\isachardoublequoteclose}\isanewline
\ \ \ \ \ \ \ \ \ \ {\isachardoublequoteopen}F\ {\isasymin}\ W{\isachardoublequoteclose}\isanewline
\ \ \ \ \ \ \ \ \ \ {\isachardoublequoteopen}finite\ Wo{\isachardoublequoteclose}\isanewline
\ \ \ \ \ \ \ \ \ \ {\isachardoublequoteopen}Wo\ {\isasymsubseteq}\ W{\isachardoublequoteclose}\isanewline
\ \ \ \ \ \ \ \ \isakeyword{shows}\ {\isachardoublequoteopen}sat\ {\isacharparenleft}{\isacharbraceleft}G{\isacharbraceright}\ {\isasymunion}\ Wo{\isacharparenright}\ {\isasymor}\ sat\ {\isacharparenleft}{\isacharbraceleft}H{\isacharbraceright}\ {\isasymunion}\ Wo{\isacharparenright}{\isachardoublequoteclose}\isanewline
%
\isadelimproof
%
\endisadelimproof
%
\isatagproof
\isacommand{proof}\isamarkupfalse%
\ {\isacharminus}\isanewline
\ \ \isacommand{have}\isamarkupfalse%
\ {\isachardoublequoteopen}{\isacharparenleft}F\ {\isacharequal}\ G\ \isactrlbold {\isasymor}\ H\ {\isasymor}\ \isanewline
\ \ \ \ \ \ \ \ {\isacharparenleft}{\isasymexists}G{\isadigit{1}}\ H{\isadigit{1}}{\isachardot}\ F\ {\isacharequal}\ G{\isadigit{1}}\ \isactrlbold {\isasymrightarrow}\ H{\isadigit{1}}\ {\isasymand}\ G\ {\isacharequal}\ \isactrlbold {\isasymnot}\ G{\isadigit{1}}\ {\isasymand}\ H\ {\isacharequal}\ H{\isadigit{1}}{\isacharparenright}\ {\isasymor}\ \isanewline
\ \ \ \ \ \ \ \ {\isacharparenleft}{\isasymexists}G{\isadigit{1}}\ H{\isadigit{1}}{\isachardot}\ F\ {\isacharequal}\ \isactrlbold {\isasymnot}\ {\isacharparenleft}G{\isadigit{1}}\ \isactrlbold {\isasymand}\ H{\isadigit{1}}{\isacharparenright}\ {\isasymand}\ G\ {\isacharequal}\ \isactrlbold {\isasymnot}\ G{\isadigit{1}}\ {\isasymand}\ H\ {\isacharequal}\ \isactrlbold {\isasymnot}\ H{\isadigit{1}}{\isacharparenright}\ {\isasymor}\ \isanewline
\ \ \ \ \ \ \ \ F\ {\isacharequal}\ \isactrlbold {\isasymnot}\ {\isacharparenleft}\isactrlbold {\isasymnot}\ G{\isacharparenright}\ {\isasymand}\ H\ {\isacharequal}\ G{\isacharparenright}{\isachardoublequoteclose}\isanewline
\ \ \ \ \isacommand{using}\isamarkupfalse%
\ assms{\isacharparenleft}{\isadigit{2}}{\isacharparenright}\ \isacommand{by}\isamarkupfalse%
\ {\isacharparenleft}simp\ only{\isacharcolon}\ con{\isacharunderscore}dis{\isacharunderscore}simps{\isacharparenleft}{\isadigit{2}}{\isacharparenright}{\isacharparenright}\isanewline
\ \ \isacommand{thus}\isamarkupfalse%
\ {\isacharquery}thesis\isanewline
\ \ \isacommand{proof}\isamarkupfalse%
\ {\isacharparenleft}rule\ disjE{\isacharparenright}\isanewline
\ \ \ \ \isacommand{assume}\isamarkupfalse%
\ {\isachardoublequoteopen}F\ {\isacharequal}\ G\ \isactrlbold {\isasymor}\ H{\isachardoublequoteclose}\isanewline
\ \ \ \ \isacommand{have}\isamarkupfalse%
\ {\isachardoublequoteopen}sat\ {\isacharparenleft}{\isacharbraceleft}G{\isacharcomma}F{\isacharbraceright}\ {\isasymunion}\ Wo{\isacharparenright}\ {\isasymor}\ sat\ {\isacharparenleft}{\isacharbraceleft}H{\isacharcomma}F{\isacharbraceright}\ {\isasymunion}\ Wo{\isacharparenright}{\isachardoublequoteclose}\isanewline
\ \ \ \ \ \ \isacommand{using}\isamarkupfalse%
\ assms{\isacharparenleft}{\isadigit{1}}{\isacharparenright}\ {\isacartoucheopen}F\ {\isacharequal}\ G\ \isactrlbold {\isasymor}\ H{\isacartoucheclose}\ assms{\isacharparenleft}{\isadigit{3}}{\isacharcomma}{\isadigit{4}}{\isacharcomma}{\isadigit{5}}{\isacharparenright}\ \isacommand{by}\isamarkupfalse%
\ {\isacharparenleft}rule\ pcp{\isacharunderscore}colecComp{\isacharunderscore}DIS{\isacharunderscore}sat{\isadigit{1}}{\isacharparenright}\isanewline
\ \ \ \ \isacommand{thus}\isamarkupfalse%
\ {\isacharquery}thesis\isanewline
\ \ \ \ \isacommand{proof}\isamarkupfalse%
\ {\isacharparenleft}rule\ disjE{\isacharparenright}\isanewline
\ \ \ \ \ \ \isacommand{assume}\isamarkupfalse%
\ {\isachardoublequoteopen}sat\ {\isacharparenleft}{\isacharbraceleft}G{\isacharcomma}F{\isacharbraceright}\ {\isasymunion}\ Wo{\isacharparenright}{\isachardoublequoteclose}\isanewline
\ \ \ \ \ \ \isacommand{have}\isamarkupfalse%
\ {\isachardoublequoteopen}{\isacharbraceleft}G{\isacharbraceright}\ {\isasymunion}\ Wo\ {\isasymsubseteq}\ {\isacharbraceleft}G{\isacharcomma}F{\isacharbraceright}\ {\isasymunion}\ Wo{\isachardoublequoteclose}\isanewline
\ \ \ \ \ \ \ \ \isacommand{by}\isamarkupfalse%
\ blast\isanewline
\ \ \ \ \ \ \isacommand{then}\isamarkupfalse%
\ \isacommand{have}\isamarkupfalse%
\ {\isachardoublequoteopen}sat{\isacharparenleft}{\isacharbraceleft}G{\isacharbraceright}\ {\isasymunion}\ Wo{\isacharparenright}{\isachardoublequoteclose}\isanewline
\ \ \ \ \ \ \ \ \isacommand{using}\isamarkupfalse%
\ {\isacartoucheopen}sat{\isacharparenleft}{\isacharbraceleft}G{\isacharcomma}F{\isacharbraceright}\ {\isasymunion}\ Wo{\isacharparenright}{\isacartoucheclose}\ \isacommand{by}\isamarkupfalse%
\ {\isacharparenleft}rule\ sat{\isacharunderscore}mono{\isacharparenright}\isanewline
\ \ \ \ \ \ \isacommand{thus}\isamarkupfalse%
\ {\isacharquery}thesis\isanewline
\ \ \ \ \ \ \ \ \isacommand{by}\isamarkupfalse%
\ {\isacharparenleft}rule\ disjI{\isadigit{1}}{\isacharparenright}\isanewline
\ \ \ \ \isacommand{next}\isamarkupfalse%
\isanewline
\ \ \ \ \ \ \isacommand{assume}\isamarkupfalse%
\ {\isachardoublequoteopen}sat\ {\isacharparenleft}{\isacharbraceleft}H{\isacharcomma}F{\isacharbraceright}\ {\isasymunion}\ Wo{\isacharparenright}{\isachardoublequoteclose}\isanewline
\ \ \ \ \ \ \isacommand{have}\isamarkupfalse%
\ {\isachardoublequoteopen}{\isacharbraceleft}H{\isacharbraceright}\ {\isasymunion}\ Wo\ {\isasymsubseteq}\ {\isacharbraceleft}H{\isacharcomma}F{\isacharbraceright}\ {\isasymunion}\ Wo{\isachardoublequoteclose}\isanewline
\ \ \ \ \ \ \ \ \isacommand{by}\isamarkupfalse%
\ blast\isanewline
\ \ \ \ \ \ \isacommand{then}\isamarkupfalse%
\ \isacommand{have}\isamarkupfalse%
\ {\isachardoublequoteopen}sat{\isacharparenleft}{\isacharbraceleft}H{\isacharbraceright}\ {\isasymunion}\ Wo{\isacharparenright}{\isachardoublequoteclose}\isanewline
\ \ \ \ \ \ \ \ \isacommand{using}\isamarkupfalse%
\ {\isacartoucheopen}sat{\isacharparenleft}{\isacharbraceleft}H{\isacharcomma}F{\isacharbraceright}\ {\isasymunion}\ Wo{\isacharparenright}{\isacartoucheclose}\ \isacommand{by}\isamarkupfalse%
\ {\isacharparenleft}rule\ sat{\isacharunderscore}mono{\isacharparenright}\isanewline
\ \ \ \ \ \ \isacommand{thus}\isamarkupfalse%
\ {\isacharquery}thesis\isanewline
\ \ \ \ \ \ \ \ \isacommand{by}\isamarkupfalse%
\ {\isacharparenleft}rule\ disjI{\isadigit{2}}{\isacharparenright}\isanewline
\ \ \ \ \isacommand{qed}\isamarkupfalse%
\isanewline
\ \ \isacommand{next}\isamarkupfalse%
\isanewline
\ \ \ \ \isacommand{assume}\isamarkupfalse%
\ {\isachardoublequoteopen}{\isacharparenleft}{\isasymexists}G{\isadigit{1}}\ H{\isadigit{1}}{\isachardot}\ F\ {\isacharequal}\ G{\isadigit{1}}\ \isactrlbold {\isasymrightarrow}\ H{\isadigit{1}}\ {\isasymand}\ G\ {\isacharequal}\ \isactrlbold {\isasymnot}\ G{\isadigit{1}}\ {\isasymand}\ H\ {\isacharequal}\ H{\isadigit{1}}{\isacharparenright}\ {\isasymor}\ \isanewline
\ \ \ \ \ \ \ \ {\isacharparenleft}{\isasymexists}G{\isadigit{1}}\ H{\isadigit{1}}{\isachardot}\ F\ {\isacharequal}\ \isactrlbold {\isasymnot}\ {\isacharparenleft}G{\isadigit{1}}\ \isactrlbold {\isasymand}\ H{\isadigit{1}}{\isacharparenright}\ {\isasymand}\ G\ {\isacharequal}\ \isactrlbold {\isasymnot}\ G{\isadigit{1}}\ {\isasymand}\ H\ {\isacharequal}\ \isactrlbold {\isasymnot}\ H{\isadigit{1}}{\isacharparenright}\ {\isasymor}\ \isanewline
\ \ \ \ \ \ \ \ F\ {\isacharequal}\ \isactrlbold {\isasymnot}\ {\isacharparenleft}\isactrlbold {\isasymnot}\ G{\isacharparenright}\ {\isasymand}\ H\ {\isacharequal}\ G{\isachardoublequoteclose}\isanewline
\ \ \ \ \isacommand{thus}\isamarkupfalse%
\ {\isacharquery}thesis\isanewline
\ \ \ \ \isacommand{proof}\isamarkupfalse%
\ {\isacharparenleft}rule\ disjE{\isacharparenright}\isanewline
\ \ \ \ \ \ \isacommand{assume}\isamarkupfalse%
\ Ex{\isadigit{1}}{\isacharcolon}{\isachardoublequoteopen}{\isasymexists}G{\isadigit{1}}\ H{\isadigit{1}}{\isachardot}\ F\ {\isacharequal}\ G{\isadigit{1}}\ \isactrlbold {\isasymrightarrow}\ H{\isadigit{1}}\ {\isasymand}\ G\ {\isacharequal}\ \isactrlbold {\isasymnot}\ G{\isadigit{1}}\ {\isasymand}\ H\ {\isacharequal}\ H{\isadigit{1}}{\isachardoublequoteclose}\isanewline
\ \ \ \ \ \ \isacommand{obtain}\isamarkupfalse%
\ G{\isadigit{1}}\ H{\isadigit{1}}\ \isakeyword{where}\ C{\isadigit{1}}{\isacharcolon}{\isachardoublequoteopen}F\ {\isacharequal}\ G{\isadigit{1}}\ \isactrlbold {\isasymrightarrow}\ H{\isadigit{1}}\ {\isasymand}\ G\ {\isacharequal}\ \isactrlbold {\isasymnot}\ G{\isadigit{1}}\ {\isasymand}\ H\ {\isacharequal}\ H{\isadigit{1}}{\isachardoublequoteclose}\isanewline
\ \ \ \ \ \ \ \ \isacommand{using}\isamarkupfalse%
\ Ex{\isadigit{1}}\ \isacommand{by}\isamarkupfalse%
\ {\isacharparenleft}iprover\ elim{\isacharcolon}\ exE{\isacharparenright}\isanewline
\ \ \ \ \ \ \isacommand{have}\isamarkupfalse%
\ {\isachardoublequoteopen}F\ {\isacharequal}\ G{\isadigit{1}}\ \isactrlbold {\isasymrightarrow}\ H{\isadigit{1}}{\isachardoublequoteclose}\isanewline
\ \ \ \ \ \ \ \ \isacommand{using}\isamarkupfalse%
\ C{\isadigit{1}}\ \isacommand{by}\isamarkupfalse%
\ {\isacharparenleft}rule\ conjunct{\isadigit{1}}{\isacharparenright}\isanewline
\ \ \ \ \ \ \isacommand{have}\isamarkupfalse%
\ {\isachardoublequoteopen}G\ {\isacharequal}\ \isactrlbold {\isasymnot}\ G{\isadigit{1}}{\isachardoublequoteclose}\isanewline
\ \ \ \ \ \ \ \ \isacommand{using}\isamarkupfalse%
\ C{\isadigit{1}}\ \isacommand{by}\isamarkupfalse%
\ {\isacharparenleft}iprover\ elim{\isacharcolon}\ conjunct{\isadigit{1}}{\isacharparenright}\isanewline
\ \ \ \ \ \ \isacommand{have}\isamarkupfalse%
\ {\isachardoublequoteopen}H\ {\isacharequal}\ H{\isadigit{1}}{\isachardoublequoteclose}\isanewline
\ \ \ \ \ \ \ \ \isacommand{using}\isamarkupfalse%
\ C{\isadigit{1}}\ \isacommand{by}\isamarkupfalse%
\ {\isacharparenleft}iprover\ elim{\isacharcolon}\ conjunct{\isadigit{2}}{\isacharparenright}\isanewline
\ \ \ \ \ \ \isacommand{have}\isamarkupfalse%
\ {\isachardoublequoteopen}sat\ {\isacharparenleft}{\isacharbraceleft}\isactrlbold {\isasymnot}\ G{\isadigit{1}}{\isacharcomma}F{\isacharbraceright}\ {\isasymunion}\ Wo{\isacharparenright}\ {\isasymor}\ sat\ {\isacharparenleft}{\isacharbraceleft}H{\isadigit{1}}{\isacharcomma}F{\isacharbraceright}\ {\isasymunion}\ Wo{\isacharparenright}{\isachardoublequoteclose}\isanewline
\ \ \ \ \ \ \ \ \isacommand{using}\isamarkupfalse%
\ assms{\isacharparenleft}{\isadigit{1}}{\isacharparenright}\ {\isacartoucheopen}F\ {\isacharequal}\ G{\isadigit{1}}\ \isactrlbold {\isasymrightarrow}\ H{\isadigit{1}}{\isacartoucheclose}\ assms{\isacharparenleft}{\isadigit{3}}{\isacharcomma}{\isadigit{4}}{\isacharcomma}{\isadigit{5}}{\isacharparenright}\ \isacommand{by}\isamarkupfalse%
\ {\isacharparenleft}rule\ pcp{\isacharunderscore}colecComp{\isacharunderscore}DIS{\isacharunderscore}sat{\isadigit{2}}{\isacharparenright}\isanewline
\ \ \ \ \ \ \isacommand{thus}\isamarkupfalse%
\ {\isacharquery}thesis\isanewline
\ \ \ \ \ \ \isacommand{proof}\isamarkupfalse%
\ {\isacharparenleft}rule\ disjE{\isacharparenright}\isanewline
\ \ \ \ \ \ \ \ \isacommand{assume}\isamarkupfalse%
\ {\isachardoublequoteopen}sat\ {\isacharparenleft}{\isacharbraceleft}\isactrlbold {\isasymnot}\ G{\isadigit{1}}{\isacharcomma}F{\isacharbraceright}\ {\isasymunion}\ Wo{\isacharparenright}{\isachardoublequoteclose}\isanewline
\ \ \ \ \ \ \ \ \isacommand{have}\isamarkupfalse%
\ {\isachardoublequoteopen}{\isacharbraceleft}\isactrlbold {\isasymnot}\ G{\isadigit{1}}{\isacharbraceright}\ {\isasymunion}\ Wo\ {\isasymsubseteq}\ {\isacharbraceleft}\isactrlbold {\isasymnot}\ G{\isadigit{1}}{\isacharcomma}F{\isacharbraceright}\ {\isasymunion}\ Wo{\isachardoublequoteclose}\isanewline
\ \ \ \ \ \ \ \ \ \ \isacommand{by}\isamarkupfalse%
\ blast\isanewline
\ \ \ \ \ \ \ \ \isacommand{then}\isamarkupfalse%
\ \isacommand{have}\isamarkupfalse%
\ {\isachardoublequoteopen}sat{\isacharparenleft}{\isacharbraceleft}\isactrlbold {\isasymnot}\ G{\isadigit{1}}{\isacharbraceright}\ {\isasymunion}\ Wo{\isacharparenright}{\isachardoublequoteclose}\isanewline
\ \ \ \ \ \ \ \ \ \ \isacommand{using}\isamarkupfalse%
\ {\isacartoucheopen}sat{\isacharparenleft}{\isacharbraceleft}\isactrlbold {\isasymnot}\ G{\isadigit{1}}{\isacharcomma}F{\isacharbraceright}\ {\isasymunion}\ Wo{\isacharparenright}{\isacartoucheclose}\ \isacommand{by}\isamarkupfalse%
\ {\isacharparenleft}rule\ sat{\isacharunderscore}mono{\isacharparenright}\isanewline
\ \ \ \ \ \ \ \ \isacommand{then}\isamarkupfalse%
\ \isacommand{have}\isamarkupfalse%
\ {\isachardoublequoteopen}sat{\isacharparenleft}{\isacharbraceleft}G{\isacharbraceright}\ {\isasymunion}\ Wo{\isacharparenright}{\isachardoublequoteclose}\isanewline
\ \ \ \ \ \ \ \ \ \ \isacommand{by}\isamarkupfalse%
\ {\isacharparenleft}simp\ only{\isacharcolon}\ {\isacartoucheopen}G\ {\isacharequal}\ \isactrlbold {\isasymnot}\ G{\isadigit{1}}{\isacartoucheclose}{\isacharparenright}\isanewline
\ \ \ \ \ \ \ \ \isacommand{thus}\isamarkupfalse%
\ {\isacharquery}thesis\isanewline
\ \ \ \ \ \ \ \ \ \ \isacommand{by}\isamarkupfalse%
\ {\isacharparenleft}rule\ disjI{\isadigit{1}}{\isacharparenright}\isanewline
\ \ \ \ \isacommand{next}\isamarkupfalse%
\isanewline
\ \ \ \ \ \ \ \ \isacommand{assume}\isamarkupfalse%
\ {\isachardoublequoteopen}sat\ {\isacharparenleft}{\isacharbraceleft}H{\isadigit{1}}{\isacharcomma}F{\isacharbraceright}\ {\isasymunion}\ Wo{\isacharparenright}{\isachardoublequoteclose}\isanewline
\ \ \ \ \ \ \ \ \isacommand{have}\isamarkupfalse%
\ {\isachardoublequoteopen}{\isacharbraceleft}H{\isadigit{1}}{\isacharbraceright}\ {\isasymunion}\ Wo\ {\isasymsubseteq}\ {\isacharbraceleft}H{\isadigit{1}}{\isacharcomma}F{\isacharbraceright}\ {\isasymunion}\ Wo{\isachardoublequoteclose}\isanewline
\ \ \ \ \ \ \ \ \ \ \isacommand{by}\isamarkupfalse%
\ blast\isanewline
\ \ \ \ \ \ \ \ \isacommand{then}\isamarkupfalse%
\ \isacommand{have}\isamarkupfalse%
\ {\isachardoublequoteopen}sat{\isacharparenleft}{\isacharbraceleft}H{\isadigit{1}}{\isacharbraceright}\ {\isasymunion}\ Wo{\isacharparenright}{\isachardoublequoteclose}\isanewline
\ \ \ \ \ \ \ \ \ \ \isacommand{using}\isamarkupfalse%
\ {\isacartoucheopen}sat{\isacharparenleft}{\isacharbraceleft}H{\isadigit{1}}{\isacharcomma}F{\isacharbraceright}\ {\isasymunion}\ Wo{\isacharparenright}{\isacartoucheclose}\ \isacommand{by}\isamarkupfalse%
\ {\isacharparenleft}rule\ sat{\isacharunderscore}mono{\isacharparenright}\isanewline
\ \ \ \ \ \ \ \ \isacommand{then}\isamarkupfalse%
\ \isacommand{have}\isamarkupfalse%
\ {\isachardoublequoteopen}sat{\isacharparenleft}{\isacharbraceleft}H{\isacharbraceright}\ {\isasymunion}\ Wo{\isacharparenright}{\isachardoublequoteclose}\isanewline
\ \ \ \ \ \ \ \ \ \ \isacommand{by}\isamarkupfalse%
\ {\isacharparenleft}simp\ only{\isacharcolon}\ {\isacartoucheopen}H\ {\isacharequal}\ H{\isadigit{1}}{\isacartoucheclose}{\isacharparenright}\isanewline
\ \ \ \ \ \ \ \ \isacommand{thus}\isamarkupfalse%
\ {\isacharquery}thesis\isanewline
\ \ \ \ \ \ \ \ \ \ \isacommand{by}\isamarkupfalse%
\ {\isacharparenleft}rule\ disjI{\isadigit{2}}{\isacharparenright}\isanewline
\ \ \ \ \ \ \isacommand{qed}\isamarkupfalse%
\isanewline
\ \ \ \ \isacommand{next}\isamarkupfalse%
\isanewline
\ \ \ \ \ \ \isacommand{assume}\isamarkupfalse%
\ {\isachardoublequoteopen}{\isacharparenleft}{\isasymexists}G{\isadigit{1}}\ H{\isadigit{1}}{\isachardot}\ F\ {\isacharequal}\ \isactrlbold {\isasymnot}\ {\isacharparenleft}G{\isadigit{1}}\ \isactrlbold {\isasymand}\ H{\isadigit{1}}{\isacharparenright}\ {\isasymand}\ G\ {\isacharequal}\ \isactrlbold {\isasymnot}\ G{\isadigit{1}}\ {\isasymand}\ H\ {\isacharequal}\ \isactrlbold {\isasymnot}\ H{\isadigit{1}}{\isacharparenright}\ {\isasymor}\ \isanewline
\ \ \ \ \ \ \ \ F\ {\isacharequal}\ \isactrlbold {\isasymnot}\ {\isacharparenleft}\isactrlbold {\isasymnot}\ G{\isacharparenright}\ {\isasymand}\ H\ {\isacharequal}\ G{\isachardoublequoteclose}\isanewline
\ \ \ \ \ \ \isacommand{thus}\isamarkupfalse%
\ {\isacharquery}thesis\isanewline
\ \ \ \ \ \ \isacommand{proof}\isamarkupfalse%
\ {\isacharparenleft}rule\ disjE{\isacharparenright}\isanewline
\ \ \ \ \ \ \ \ \isacommand{assume}\isamarkupfalse%
\ Ex{\isadigit{2}}{\isacharcolon}{\isachardoublequoteopen}{\isasymexists}G{\isadigit{1}}\ H{\isadigit{1}}{\isachardot}\ F\ {\isacharequal}\ \isactrlbold {\isasymnot}\ {\isacharparenleft}G{\isadigit{1}}\ \isactrlbold {\isasymand}\ H{\isadigit{1}}{\isacharparenright}\ {\isasymand}\ G\ {\isacharequal}\ \isactrlbold {\isasymnot}\ G{\isadigit{1}}\ {\isasymand}\ H\ {\isacharequal}\ \isactrlbold {\isasymnot}\ H{\isadigit{1}}{\isachardoublequoteclose}\isanewline
\ \ \ \ \ \ \ \ \isacommand{obtain}\isamarkupfalse%
\ G{\isadigit{1}}\ H{\isadigit{1}}\ \isakeyword{where}\ C{\isadigit{2}}{\isacharcolon}{\isachardoublequoteopen}F\ {\isacharequal}\ \isactrlbold {\isasymnot}\ {\isacharparenleft}G{\isadigit{1}}\ \isactrlbold {\isasymand}\ H{\isadigit{1}}{\isacharparenright}\ {\isasymand}\ G\ {\isacharequal}\ \isactrlbold {\isasymnot}\ G{\isadigit{1}}\ {\isasymand}\ H\ {\isacharequal}\ \isactrlbold {\isasymnot}\ H{\isadigit{1}}{\isachardoublequoteclose}\isanewline
\ \ \ \ \ \ \ \ \ \ \isacommand{using}\isamarkupfalse%
\ Ex{\isadigit{2}}\ \isacommand{by}\isamarkupfalse%
\ {\isacharparenleft}iprover\ elim{\isacharcolon}\ exE{\isacharparenright}\isanewline
\ \ \ \ \ \ \ \ \isacommand{have}\isamarkupfalse%
\ {\isachardoublequoteopen}F\ {\isacharequal}\ \isactrlbold {\isasymnot}\ {\isacharparenleft}G{\isadigit{1}}\ \isactrlbold {\isasymand}\ H{\isadigit{1}}{\isacharparenright}{\isachardoublequoteclose}\isanewline
\ \ \ \ \ \ \ \ \ \ \isacommand{using}\isamarkupfalse%
\ C{\isadigit{2}}\ \isacommand{by}\isamarkupfalse%
\ {\isacharparenleft}rule\ conjunct{\isadigit{1}}{\isacharparenright}\isanewline
\ \ \ \ \ \ \ \ \isacommand{have}\isamarkupfalse%
\ {\isachardoublequoteopen}G\ {\isacharequal}\ \isactrlbold {\isasymnot}\ G{\isadigit{1}}{\isachardoublequoteclose}\isanewline
\ \ \ \ \ \ \ \ \ \ \isacommand{using}\isamarkupfalse%
\ C{\isadigit{2}}\ \isacommand{by}\isamarkupfalse%
\ {\isacharparenleft}iprover\ elim{\isacharcolon}\ conjunct{\isadigit{1}}{\isacharparenright}\isanewline
\ \ \ \ \ \ \ \ \isacommand{have}\isamarkupfalse%
\ {\isachardoublequoteopen}H\ {\isacharequal}\ \isactrlbold {\isasymnot}\ H{\isadigit{1}}{\isachardoublequoteclose}\isanewline
\ \ \ \ \ \ \ \ \ \ \isacommand{using}\isamarkupfalse%
\ C{\isadigit{2}}\ \isacommand{by}\isamarkupfalse%
\ {\isacharparenleft}iprover\ elim{\isacharcolon}\ conjunct{\isadigit{2}}{\isacharparenright}\isanewline
\ \ \ \ \ \ \ \ \isacommand{have}\isamarkupfalse%
\ {\isachardoublequoteopen}sat\ {\isacharparenleft}{\isacharbraceleft}\isactrlbold {\isasymnot}\ G{\isadigit{1}}{\isacharcomma}F{\isacharbraceright}\ {\isasymunion}\ Wo{\isacharparenright}\ {\isasymor}\ sat\ {\isacharparenleft}{\isacharbraceleft}\isactrlbold {\isasymnot}\ H{\isadigit{1}}{\isacharcomma}F{\isacharbraceright}\ {\isasymunion}\ Wo{\isacharparenright}{\isachardoublequoteclose}\isanewline
\ \ \ \ \ \ \ \ \ \ \isacommand{using}\isamarkupfalse%
\ assms{\isacharparenleft}{\isadigit{1}}{\isacharparenright}\ {\isacartoucheopen}F\ {\isacharequal}\ \isactrlbold {\isasymnot}\ {\isacharparenleft}G{\isadigit{1}}\ \isactrlbold {\isasymand}\ H{\isadigit{1}}{\isacharparenright}{\isacartoucheclose}\ assms{\isacharparenleft}{\isadigit{3}}{\isacharcomma}{\isadigit{4}}{\isacharcomma}{\isadigit{5}}{\isacharparenright}\ \isacommand{by}\isamarkupfalse%
\ {\isacharparenleft}rule\ pcp{\isacharunderscore}colecComp{\isacharunderscore}DIS{\isacharunderscore}sat{\isadigit{3}}{\isacharparenright}\isanewline
\ \ \ \ \ \ \ \ \isacommand{thus}\isamarkupfalse%
\ {\isacharquery}thesis\isanewline
\ \ \ \ \ \ \ \ \isacommand{proof}\isamarkupfalse%
\ {\isacharparenleft}rule\ disjE{\isacharparenright}\isanewline
\ \ \ \ \ \ \ \ \ \ \isacommand{assume}\isamarkupfalse%
\ {\isachardoublequoteopen}sat\ {\isacharparenleft}{\isacharbraceleft}\isactrlbold {\isasymnot}\ G{\isadigit{1}}{\isacharcomma}F{\isacharbraceright}\ {\isasymunion}\ Wo{\isacharparenright}{\isachardoublequoteclose}\isanewline
\ \ \ \ \ \ \ \ \ \ \isacommand{have}\isamarkupfalse%
\ {\isachardoublequoteopen}{\isacharbraceleft}\isactrlbold {\isasymnot}\ G{\isadigit{1}}{\isacharbraceright}\ {\isasymunion}\ Wo\ {\isasymsubseteq}\ {\isacharbraceleft}\isactrlbold {\isasymnot}\ G{\isadigit{1}}{\isacharcomma}F{\isacharbraceright}\ {\isasymunion}\ Wo{\isachardoublequoteclose}\isanewline
\ \ \ \ \ \ \ \ \ \ \ \ \isacommand{by}\isamarkupfalse%
\ blast\isanewline
\ \ \ \ \ \ \ \ \ \ \isacommand{then}\isamarkupfalse%
\ \isacommand{have}\isamarkupfalse%
\ {\isachardoublequoteopen}sat{\isacharparenleft}{\isacharbraceleft}\isactrlbold {\isasymnot}\ G{\isadigit{1}}{\isacharbraceright}\ {\isasymunion}\ Wo{\isacharparenright}{\isachardoublequoteclose}\isanewline
\ \ \ \ \ \ \ \ \ \ \ \ \isacommand{using}\isamarkupfalse%
\ {\isacartoucheopen}sat{\isacharparenleft}{\isacharbraceleft}\isactrlbold {\isasymnot}\ G{\isadigit{1}}{\isacharcomma}F{\isacharbraceright}\ {\isasymunion}\ Wo{\isacharparenright}{\isacartoucheclose}\ \isacommand{by}\isamarkupfalse%
\ {\isacharparenleft}rule\ sat{\isacharunderscore}mono{\isacharparenright}\isanewline
\ \ \ \ \ \ \ \ \ \ \isacommand{then}\isamarkupfalse%
\ \isacommand{have}\isamarkupfalse%
\ {\isachardoublequoteopen}sat{\isacharparenleft}{\isacharbraceleft}G{\isacharbraceright}\ {\isasymunion}\ Wo{\isacharparenright}{\isachardoublequoteclose}\isanewline
\ \ \ \ \ \ \ \ \ \ \ \ \isacommand{by}\isamarkupfalse%
\ {\isacharparenleft}simp\ only{\isacharcolon}\ {\isacartoucheopen}G\ {\isacharequal}\ \isactrlbold {\isasymnot}\ G{\isadigit{1}}{\isacartoucheclose}{\isacharparenright}\isanewline
\ \ \ \ \ \ \ \ \ \ \isacommand{thus}\isamarkupfalse%
\ {\isacharquery}thesis\isanewline
\ \ \ \ \ \ \ \ \ \ \ \ \isacommand{by}\isamarkupfalse%
\ {\isacharparenleft}rule\ disjI{\isadigit{1}}{\isacharparenright}\isanewline
\ \ \ \ \ \ \ \ \isacommand{next}\isamarkupfalse%
\isanewline
\ \ \ \ \ \ \ \ \ \ \isacommand{assume}\isamarkupfalse%
\ {\isachardoublequoteopen}sat\ {\isacharparenleft}{\isacharbraceleft}\isactrlbold {\isasymnot}\ H{\isadigit{1}}{\isacharcomma}F{\isacharbraceright}\ {\isasymunion}\ Wo{\isacharparenright}{\isachardoublequoteclose}\isanewline
\ \ \ \ \ \ \ \ \ \ \isacommand{have}\isamarkupfalse%
\ {\isachardoublequoteopen}{\isacharbraceleft}\isactrlbold {\isasymnot}\ H{\isadigit{1}}{\isacharbraceright}\ {\isasymunion}\ Wo\ {\isasymsubseteq}\ {\isacharbraceleft}\isactrlbold {\isasymnot}\ H{\isadigit{1}}{\isacharcomma}F{\isacharbraceright}\ {\isasymunion}\ Wo{\isachardoublequoteclose}\isanewline
\ \ \ \ \ \ \ \ \ \ \ \ \isacommand{by}\isamarkupfalse%
\ blast\isanewline
\ \ \ \ \ \ \ \ \ \ \isacommand{then}\isamarkupfalse%
\ \isacommand{have}\isamarkupfalse%
\ {\isachardoublequoteopen}sat{\isacharparenleft}{\isacharbraceleft}\isactrlbold {\isasymnot}\ H{\isadigit{1}}{\isacharbraceright}\ {\isasymunion}\ Wo{\isacharparenright}{\isachardoublequoteclose}\isanewline
\ \ \ \ \ \ \ \ \ \ \ \ \isacommand{using}\isamarkupfalse%
\ {\isacartoucheopen}sat{\isacharparenleft}{\isacharbraceleft}\isactrlbold {\isasymnot}\ H{\isadigit{1}}{\isacharcomma}F{\isacharbraceright}\ {\isasymunion}\ Wo{\isacharparenright}{\isacartoucheclose}\ \isacommand{by}\isamarkupfalse%
\ {\isacharparenleft}rule\ sat{\isacharunderscore}mono{\isacharparenright}\isanewline
\ \ \ \ \ \ \ \ \ \ \isacommand{then}\isamarkupfalse%
\ \isacommand{have}\isamarkupfalse%
\ {\isachardoublequoteopen}sat{\isacharparenleft}{\isacharbraceleft}H{\isacharbraceright}\ {\isasymunion}\ Wo{\isacharparenright}{\isachardoublequoteclose}\isanewline
\ \ \ \ \ \ \ \ \ \ \ \ \isacommand{by}\isamarkupfalse%
\ {\isacharparenleft}simp\ only{\isacharcolon}\ {\isacartoucheopen}H\ {\isacharequal}\ \isactrlbold {\isasymnot}\ H{\isadigit{1}}{\isacartoucheclose}{\isacharparenright}\isanewline
\ \ \ \ \ \ \ \ \ \ \isacommand{thus}\isamarkupfalse%
\ {\isacharquery}thesis\isanewline
\ \ \ \ \ \ \ \ \ \ \ \ \isacommand{by}\isamarkupfalse%
\ {\isacharparenleft}rule\ disjI{\isadigit{2}}{\isacharparenright}\isanewline
\ \ \ \ \ \ \ \ \isacommand{qed}\isamarkupfalse%
\isanewline
\ \ \ \ \ \ \isacommand{next}\isamarkupfalse%
\isanewline
\ \ \ \ \ \ \ \ \isacommand{assume}\isamarkupfalse%
\ {\isachardoublequoteopen}F\ {\isacharequal}\ \isactrlbold {\isasymnot}\ {\isacharparenleft}\isactrlbold {\isasymnot}\ G{\isacharparenright}\ {\isasymand}\ H\ {\isacharequal}\ G{\isachardoublequoteclose}\isanewline
\ \ \ \ \ \ \ \ \isacommand{then}\isamarkupfalse%
\ \isacommand{have}\isamarkupfalse%
\ {\isachardoublequoteopen}F\ {\isacharequal}\ \isactrlbold {\isasymnot}\ {\isacharparenleft}\isactrlbold {\isasymnot}\ G{\isacharparenright}{\isachardoublequoteclose}\isanewline
\ \ \ \ \ \ \ \ \ \ \isacommand{by}\isamarkupfalse%
\ {\isacharparenleft}rule\ conjunct{\isadigit{1}}{\isacharparenright}\isanewline
\ \ \ \ \ \ \ \ \isacommand{have}\isamarkupfalse%
\ {\isachardoublequoteopen}H\ {\isacharequal}\ G{\isachardoublequoteclose}\isanewline
\ \ \ \ \ \ \ \ \ \ \isacommand{using}\isamarkupfalse%
\ {\isacartoucheopen}F\ {\isacharequal}\ \isactrlbold {\isasymnot}\ {\isacharparenleft}\isactrlbold {\isasymnot}\ G{\isacharparenright}\ {\isasymand}\ H\ {\isacharequal}\ G{\isacartoucheclose}\ \isacommand{by}\isamarkupfalse%
\ {\isacharparenleft}rule\ conjunct{\isadigit{2}}{\isacharparenright}\isanewline
\ \ \ \ \ \ \ \ \isacommand{have}\isamarkupfalse%
\ {\isachardoublequoteopen}sat\ {\isacharparenleft}{\isacharbraceleft}G{\isacharcomma}F{\isacharbraceright}\ {\isasymunion}\ Wo{\isacharparenright}\ {\isasymor}\ sat\ {\isacharparenleft}{\isacharbraceleft}H{\isacharcomma}F{\isacharbraceright}\ {\isasymunion}\ Wo{\isacharparenright}{\isachardoublequoteclose}\isanewline
\ \ \ \ \ \ \ \ \ \ \isacommand{using}\isamarkupfalse%
\ assms{\isacharparenleft}{\isadigit{1}}{\isacharparenright}\ {\isacartoucheopen}F\ {\isacharequal}\ \isactrlbold {\isasymnot}\ {\isacharparenleft}\isactrlbold {\isasymnot}\ G{\isacharparenright}{\isacartoucheclose}\ {\isacartoucheopen}H\ {\isacharequal}\ G{\isacartoucheclose}\ assms{\isacharparenleft}{\isadigit{3}}{\isacharcomma}{\isadigit{4}}{\isacharcomma}{\isadigit{5}}{\isacharparenright}\ \isacommand{by}\isamarkupfalse%
\ {\isacharparenleft}rule\ pcp{\isacharunderscore}colecComp{\isacharunderscore}DIS{\isacharunderscore}sat{\isadigit{4}}{\isacharparenright}\isanewline
\ \ \ \ \ \ \ \ \isacommand{thus}\isamarkupfalse%
\ {\isacharquery}thesis\isanewline
\ \ \ \ \ \ \ \ \isacommand{proof}\isamarkupfalse%
\ {\isacharparenleft}rule\ disjE{\isacharparenright}\isanewline
\ \ \ \ \ \ \ \ \ \ \isacommand{assume}\isamarkupfalse%
\ {\isachardoublequoteopen}sat\ {\isacharparenleft}{\isacharbraceleft}G{\isacharcomma}F{\isacharbraceright}\ {\isasymunion}\ Wo{\isacharparenright}{\isachardoublequoteclose}\isanewline
\ \ \ \ \ \ \ \ \ \ \isacommand{have}\isamarkupfalse%
\ {\isachardoublequoteopen}{\isacharbraceleft}G{\isacharbraceright}\ {\isasymunion}\ Wo\ {\isasymsubseteq}\ {\isacharbraceleft}G{\isacharcomma}F{\isacharbraceright}\ {\isasymunion}\ Wo{\isachardoublequoteclose}\isanewline
\ \ \ \ \ \ \ \ \ \ \ \ \isacommand{by}\isamarkupfalse%
\ blast\isanewline
\ \ \ \ \ \ \ \ \ \ \isacommand{then}\isamarkupfalse%
\ \isacommand{have}\isamarkupfalse%
\ {\isachardoublequoteopen}sat{\isacharparenleft}{\isacharbraceleft}G{\isacharbraceright}\ {\isasymunion}\ Wo{\isacharparenright}{\isachardoublequoteclose}\isanewline
\ \ \ \ \ \ \ \ \ \ \ \ \isacommand{using}\isamarkupfalse%
\ {\isacartoucheopen}sat{\isacharparenleft}{\isacharbraceleft}G{\isacharcomma}F{\isacharbraceright}\ {\isasymunion}\ Wo{\isacharparenright}{\isacartoucheclose}\ \isacommand{by}\isamarkupfalse%
\ {\isacharparenleft}rule\ sat{\isacharunderscore}mono{\isacharparenright}\isanewline
\ \ \ \ \ \ \ \ \ \ \isacommand{thus}\isamarkupfalse%
\ {\isacharquery}thesis\isanewline
\ \ \ \ \ \ \ \ \ \ \ \ \isacommand{by}\isamarkupfalse%
\ {\isacharparenleft}rule\ disjI{\isadigit{1}}{\isacharparenright}\isanewline
\ \ \ \ \ \ \ \ \isacommand{next}\isamarkupfalse%
\isanewline
\ \ \ \ \ \ \ \ \ \ \isacommand{assume}\isamarkupfalse%
\ {\isachardoublequoteopen}sat\ {\isacharparenleft}{\isacharbraceleft}H{\isacharcomma}F{\isacharbraceright}\ {\isasymunion}\ Wo{\isacharparenright}{\isachardoublequoteclose}\isanewline
\ \ \ \ \ \ \ \ \ \ \isacommand{have}\isamarkupfalse%
\ {\isachardoublequoteopen}{\isacharbraceleft}H{\isacharbraceright}\ {\isasymunion}\ Wo\ {\isasymsubseteq}\ {\isacharbraceleft}H{\isacharcomma}F{\isacharbraceright}\ {\isasymunion}\ Wo{\isachardoublequoteclose}\isanewline
\ \ \ \ \ \ \ \ \ \ \ \ \isacommand{by}\isamarkupfalse%
\ blast\isanewline
\ \ \ \ \ \ \ \ \ \ \isacommand{then}\isamarkupfalse%
\ \isacommand{have}\isamarkupfalse%
\ {\isachardoublequoteopen}sat{\isacharparenleft}{\isacharbraceleft}H{\isacharbraceright}\ {\isasymunion}\ Wo{\isacharparenright}{\isachardoublequoteclose}\isanewline
\ \ \ \ \ \ \ \ \ \ \ \ \isacommand{using}\isamarkupfalse%
\ {\isacartoucheopen}sat{\isacharparenleft}{\isacharbraceleft}H{\isacharcomma}F{\isacharbraceright}\ {\isasymunion}\ Wo{\isacharparenright}{\isacartoucheclose}\ \isacommand{by}\isamarkupfalse%
\ {\isacharparenleft}rule\ sat{\isacharunderscore}mono{\isacharparenright}\isanewline
\ \ \ \ \ \ \ \ \ \ \isacommand{thus}\isamarkupfalse%
\ {\isacharquery}thesis\isanewline
\ \ \ \ \ \ \ \ \ \ \ \ \isacommand{by}\isamarkupfalse%
\ {\isacharparenleft}rule\ disjI{\isadigit{2}}{\isacharparenright}\isanewline
\ \ \ \ \ \ \ \ \isacommand{qed}\isamarkupfalse%
\isanewline
\ \ \ \ \ \ \isacommand{qed}\isamarkupfalse%
\isanewline
\ \ \ \ \isacommand{qed}\isamarkupfalse%
\isanewline
\ \ \isacommand{qed}\isamarkupfalse%
\isanewline
\isacommand{qed}\isamarkupfalse%
%
\endisatagproof
{\isafoldproof}%
%
\isadelimproof
\isanewline
%
\endisadelimproof
\isanewline
\isacommand{lemma}\isamarkupfalse%
\ sat{\isacharunderscore}subset{\isacharunderscore}ccontr{\isacharcolon}\isanewline
\ \ \isakeyword{assumes}\ {\isachardoublequoteopen}A\ {\isasymsubseteq}\ B{\isachardoublequoteclose}\isanewline
\ \ \ \ \ \ \ \ \ \ {\isachardoublequoteopen}{\isasymnot}\ sat\ A{\isachardoublequoteclose}\isanewline
\ \ \ \ \ \ \ \ \isakeyword{shows}\ {\isachardoublequoteopen}{\isasymnot}\ sat\ B{\isachardoublequoteclose}\isanewline
%
\isadelimproof
%
\endisadelimproof
%
\isatagproof
\isacommand{proof}\isamarkupfalse%
\ {\isacharminus}\isanewline
\ \ \isacommand{have}\isamarkupfalse%
\ {\isachardoublequoteopen}A\ {\isasymsubseteq}\ B\ {\isasymand}\ sat\ B\ {\isasymlongrightarrow}\ sat\ A{\isachardoublequoteclose}\isanewline
\ \ \ \ \isacommand{using}\isamarkupfalse%
\ sat{\isacharunderscore}mono\ \isacommand{by}\isamarkupfalse%
\ blast\isanewline
\ \ \isacommand{then}\isamarkupfalse%
\ \isacommand{have}\isamarkupfalse%
\ {\isachardoublequoteopen}{\isasymnot}{\isacharparenleft}A\ {\isasymsubseteq}\ B\ {\isasymand}\ sat\ B{\isacharparenright}{\isachardoublequoteclose}\isanewline
\ \ \ \ \isacommand{using}\isamarkupfalse%
\ assms{\isacharparenleft}{\isadigit{2}}{\isacharparenright}\ \isacommand{by}\isamarkupfalse%
\ {\isacharparenleft}rule\ mt{\isacharparenright}\isanewline
\ \ \isacommand{then}\isamarkupfalse%
\ \isacommand{have}\isamarkupfalse%
\ {\isachardoublequoteopen}{\isasymnot}{\isacharparenleft}A\ {\isasymsubseteq}\ B{\isacharparenright}\ {\isasymor}\ {\isasymnot}{\isacharparenleft}sat\ B{\isacharparenright}{\isachardoublequoteclose}\isanewline
\ \ \ \ \isacommand{by}\isamarkupfalse%
\ {\isacharparenleft}simp\ only{\isacharcolon}\ de{\isacharunderscore}Morgan{\isacharunderscore}conj{\isacharparenright}\isanewline
\ \ \isacommand{thus}\isamarkupfalse%
\ {\isacharquery}thesis\isanewline
\ \ \isacommand{proof}\isamarkupfalse%
\ {\isacharparenleft}rule\ disjE{\isacharparenright}\isanewline
\ \ \ \ \isacommand{assume}\isamarkupfalse%
\ {\isachardoublequoteopen}{\isasymnot}{\isacharparenleft}A\ {\isasymsubseteq}\ B{\isacharparenright}{\isachardoublequoteclose}\isanewline
\ \ \ \ \isacommand{thus}\isamarkupfalse%
\ {\isacharquery}thesis\isanewline
\ \ \ \ \ \ \isacommand{using}\isamarkupfalse%
\ assms{\isacharparenleft}{\isadigit{1}}{\isacharparenright}\ \isacommand{by}\isamarkupfalse%
\ {\isacharparenleft}rule\ notE{\isacharparenright}\isanewline
\ \ \isacommand{next}\isamarkupfalse%
\isanewline
\ \ \ \ \isacommand{assume}\isamarkupfalse%
\ {\isachardoublequoteopen}{\isasymnot}{\isacharparenleft}sat\ B{\isacharparenright}{\isachardoublequoteclose}\isanewline
\ \ \ \ \isacommand{thus}\isamarkupfalse%
\ {\isacharquery}thesis\isanewline
\ \ \ \ \ \ \isacommand{by}\isamarkupfalse%
\ this\isanewline
\ \ \isacommand{qed}\isamarkupfalse%
\isanewline
\isacommand{qed}\isamarkupfalse%
%
\endisatagproof
{\isafoldproof}%
%
\isadelimproof
\isanewline
%
\endisadelimproof
\isanewline
\isacommand{lemma}\isamarkupfalse%
\ not{\isacharunderscore}colecComp{\isacharcolon}\isanewline
\ \ \isakeyword{assumes}\ {\isachardoublequoteopen}W\ {\isasymin}\ colecComp{\isachardoublequoteclose}\isanewline
\ \ \ \ \ \ \ \ \ \ {\isachardoublequoteopen}{\isacharbraceleft}x{\isacharbraceright}\ {\isasymunion}\ W\ {\isasymnotin}\ colecComp{\isachardoublequoteclose}\isanewline
\ \ \ \ \ \ \ \ \isakeyword{shows}\ {\isachardoublequoteopen}{\isasymexists}Wo\ {\isasymsubseteq}\ W{\isachardot}\ finite\ Wo\ {\isasymand}\ {\isasymnot}{\isacharparenleft}sat\ {\isacharparenleft}{\isacharbraceleft}x{\isacharbraceright}\ {\isasymunion}\ Wo{\isacharparenright}{\isacharparenright}{\isachardoublequoteclose}\isanewline
%
\isadelimproof
%
\endisadelimproof
%
\isatagproof
\isacommand{proof}\isamarkupfalse%
\ {\isacharminus}\isanewline
\ \ \isacommand{have}\isamarkupfalse%
\ WCol{\isacharcolon}{\isachardoublequoteopen}{\isasymforall}S{\isacharprime}\ {\isasymsubseteq}\ W{\isachardot}\ finite\ S{\isacharprime}\ {\isasymlongrightarrow}\ sat\ S{\isacharprime}{\isachardoublequoteclose}\isanewline
\ \ \ \ \isacommand{using}\isamarkupfalse%
\ assms{\isacharparenleft}{\isadigit{1}}{\isacharparenright}\ \isacommand{unfolding}\isamarkupfalse%
\ colecComp\ fin{\isacharunderscore}sat{\isacharunderscore}def\ \isacommand{by}\isamarkupfalse%
\ {\isacharparenleft}rule\ CollectD{\isacharparenright}\ \isanewline
\ \ \isacommand{have}\isamarkupfalse%
\ {\isachardoublequoteopen}{\isasymnot}{\isacharparenleft}{\isasymforall}Wo\ {\isasymsubseteq}\ {\isacharbraceleft}x{\isacharbraceright}\ {\isasymunion}\ W{\isachardot}\ finite\ Wo\ {\isasymlongrightarrow}\ sat\ Wo{\isacharparenright}{\isachardoublequoteclose}\isanewline
\ \ \ \ \isacommand{using}\isamarkupfalse%
\ assms{\isacharparenleft}{\isadigit{2}}{\isacharparenright}\ \isacommand{unfolding}\isamarkupfalse%
\ colecComp\ fin{\isacharunderscore}sat{\isacharunderscore}def\ \isacommand{by}\isamarkupfalse%
\ {\isacharparenleft}simp\ only{\isacharcolon}\ mem{\isacharunderscore}Collect{\isacharunderscore}eq\ simp{\isacharunderscore}thms{\isacharparenleft}{\isadigit{8}}{\isacharparenright}{\isacharparenright}\isanewline
\ \ \isacommand{then}\isamarkupfalse%
\ \isacommand{have}\isamarkupfalse%
\ {\isachardoublequoteopen}{\isasymexists}Wo\ {\isasymsubseteq}\ {\isacharbraceleft}x{\isacharbraceright}\ {\isasymunion}\ W{\isachardot}\ {\isasymnot}{\isacharparenleft}finite\ Wo\ {\isasymlongrightarrow}\ sat\ Wo{\isacharparenright}{\isachardoublequoteclose}\isanewline
\ \ \ \ \isacommand{by}\isamarkupfalse%
\ {\isacharparenleft}rule\ sall{\isacharunderscore}simps{\isacharunderscore}not{\isacharunderscore}all{\isacharparenright}\isanewline
\ \ \isacommand{then}\isamarkupfalse%
\ \isacommand{have}\isamarkupfalse%
\ Ex{\isadigit{1}}{\isacharcolon}{\isachardoublequoteopen}{\isasymexists}Wo\ {\isasymsubseteq}\ {\isacharbraceleft}x{\isacharbraceright}\ {\isasymunion}\ W{\isachardot}\ finite\ Wo\ {\isasymand}\ {\isasymnot}{\isacharparenleft}sat\ Wo{\isacharparenright}{\isachardoublequoteclose}\isanewline
\ \ \ \ \isacommand{by}\isamarkupfalse%
\ {\isacharparenleft}simp\ only{\isacharcolon}\ not{\isacharunderscore}imp{\isacharparenright}\isanewline
\ \ \isacommand{obtain}\isamarkupfalse%
\ Wo{\isacharprime}\ \isakeyword{where}\ {\isachardoublequoteopen}Wo{\isacharprime}\ {\isasymsubseteq}\ {\isacharbraceleft}x{\isacharbraceright}\ {\isasymunion}\ W{\isachardoublequoteclose}\ \isakeyword{and}\ C{\isadigit{1}}{\isacharcolon}{\isachardoublequoteopen}finite\ Wo{\isacharprime}\ {\isasymand}\ {\isasymnot}{\isacharparenleft}sat\ Wo{\isacharprime}{\isacharparenright}{\isachardoublequoteclose}\isanewline
\ \ \ \ \isacommand{using}\isamarkupfalse%
\ Ex{\isadigit{1}}\ \isacommand{by}\isamarkupfalse%
\ {\isacharparenleft}rule\ subexE{\isacharparenright}\isanewline
\ \ \isacommand{have}\isamarkupfalse%
\ {\isachardoublequoteopen}finite\ Wo{\isacharprime}{\isachardoublequoteclose}\isanewline
\ \ \ \ \isacommand{using}\isamarkupfalse%
\ C{\isadigit{1}}\ \isacommand{by}\isamarkupfalse%
\ {\isacharparenleft}rule\ conjunct{\isadigit{1}}{\isacharparenright}\isanewline
\ \ \isacommand{have}\isamarkupfalse%
\ {\isachardoublequoteopen}{\isasymnot}{\isacharparenleft}sat\ Wo{\isacharprime}{\isacharparenright}{\isachardoublequoteclose}\isanewline
\ \ \ \ \isacommand{using}\isamarkupfalse%
\ C{\isadigit{1}}\ \isacommand{by}\isamarkupfalse%
\ {\isacharparenleft}rule\ conjunct{\isadigit{2}}{\isacharparenright}\isanewline
\ \ \isacommand{have}\isamarkupfalse%
\ Ex{\isadigit{2}}{\isacharcolon}{\isachardoublequoteopen}{\isasymexists}Wo\ {\isasymsubseteq}\ W{\isachardot}\ finite\ Wo\ {\isasymand}\ {\isacharparenleft}Wo{\isacharprime}\ {\isacharequal}\ {\isacharbraceleft}x{\isacharbraceright}\ {\isasymunion}\ Wo\ {\isasymor}\ Wo{\isacharprime}\ {\isacharequal}\ Wo{\isacharparenright}{\isachardoublequoteclose}\isanewline
\ \ \ \ \isacommand{using}\isamarkupfalse%
\ {\isacartoucheopen}finite\ Wo{\isacharprime}{\isacartoucheclose}\ {\isacartoucheopen}Wo{\isacharprime}\ {\isasymsubseteq}\ {\isacharbraceleft}x{\isacharbraceright}\ {\isasymunion}\ W{\isacartoucheclose}\ \isacommand{by}\isamarkupfalse%
\ {\isacharparenleft}rule\ finite{\isacharunderscore}subset{\isacharunderscore}insert{\isadigit{1}}{\isacharparenright}\isanewline
\ \ \isacommand{obtain}\isamarkupfalse%
\ Wo\ \isakeyword{where}\ {\isachardoublequoteopen}Wo\ {\isasymsubseteq}\ W{\isachardoublequoteclose}\ \isakeyword{and}\ C{\isadigit{2}}{\isacharcolon}{\isachardoublequoteopen}finite\ Wo\ {\isasymand}\ {\isacharparenleft}Wo{\isacharprime}\ {\isacharequal}\ {\isacharbraceleft}x{\isacharbraceright}\ {\isasymunion}\ Wo\ {\isasymor}\ Wo{\isacharprime}\ {\isacharequal}\ Wo{\isacharparenright}{\isachardoublequoteclose}\isanewline
\ \ \ \ \isacommand{using}\isamarkupfalse%
\ Ex{\isadigit{2}}\ \isacommand{by}\isamarkupfalse%
\ blast\isanewline
\ \ \isacommand{have}\isamarkupfalse%
\ {\isachardoublequoteopen}finite\ Wo{\isachardoublequoteclose}\isanewline
\ \ \ \ \isacommand{using}\isamarkupfalse%
\ C{\isadigit{2}}\ \isacommand{by}\isamarkupfalse%
\ {\isacharparenleft}rule\ conjunct{\isadigit{1}}{\isacharparenright}\isanewline
\ \ \isacommand{have}\isamarkupfalse%
\ {\isachardoublequoteopen}Wo{\isacharprime}\ {\isacharequal}\ {\isacharbraceleft}x{\isacharbraceright}\ {\isasymunion}\ Wo\ {\isasymor}\ Wo{\isacharprime}\ {\isacharequal}\ Wo{\isachardoublequoteclose}\isanewline
\ \ \ \ \isacommand{using}\isamarkupfalse%
\ C{\isadigit{2}}\ \isacommand{by}\isamarkupfalse%
\ {\isacharparenleft}rule\ conjunct{\isadigit{2}}{\isacharparenright}\isanewline
\ \ \isacommand{thus}\isamarkupfalse%
\ {\isacharquery}thesis\isanewline
\ \ \isacommand{proof}\isamarkupfalse%
\ {\isacharparenleft}rule\ disjE{\isacharparenright}\isanewline
\ \ \ \ \isacommand{assume}\isamarkupfalse%
\ {\isachardoublequoteopen}Wo{\isacharprime}\ {\isacharequal}\ {\isacharbraceleft}x{\isacharbraceright}\ {\isasymunion}\ Wo{\isachardoublequoteclose}\isanewline
\ \ \ \ \isacommand{then}\isamarkupfalse%
\ \isacommand{have}\isamarkupfalse%
\ {\isachardoublequoteopen}{\isasymnot}{\isacharparenleft}sat\ {\isacharparenleft}{\isacharbraceleft}x{\isacharbraceright}\ {\isasymunion}\ Wo{\isacharparenright}{\isacharparenright}{\isachardoublequoteclose}\ \isanewline
\ \ \ \ \ \ \isacommand{using}\isamarkupfalse%
\ {\isacartoucheopen}{\isasymnot}\ sat\ Wo{\isacharprime}{\isacartoucheclose}\ \isacommand{by}\isamarkupfalse%
\ {\isacharparenleft}simp\ only{\isacharcolon}\ {\isacartoucheopen}Wo{\isacharprime}\ {\isacharequal}\ {\isacharbraceleft}x{\isacharbraceright}\ {\isasymunion}\ Wo{\isacartoucheclose}\ simp{\isacharunderscore}thms{\isacharparenleft}{\isadigit{8}}{\isacharparenright}{\isacharparenright}\isanewline
\ \ \ \ \isacommand{have}\isamarkupfalse%
\ {\isachardoublequoteopen}finite\ Wo\ {\isasymand}\ {\isasymnot}{\isacharparenleft}sat\ {\isacharparenleft}{\isacharbraceleft}x{\isacharbraceright}\ {\isasymunion}\ Wo{\isacharparenright}{\isacharparenright}{\isachardoublequoteclose}\isanewline
\ \ \ \ \ \ \isacommand{using}\isamarkupfalse%
\ {\isacartoucheopen}finite\ Wo{\isacartoucheclose}\ {\isacartoucheopen}{\isasymnot}{\isacharparenleft}sat\ {\isacharparenleft}{\isacharbraceleft}x{\isacharbraceright}\ {\isasymunion}\ Wo{\isacharparenright}{\isacharparenright}{\isacartoucheclose}\ \isacommand{by}\isamarkupfalse%
\ {\isacharparenleft}rule\ conjI{\isacharparenright}\isanewline
\ \ \ \ \isacommand{thus}\isamarkupfalse%
\ {\isachardoublequoteopen}{\isasymexists}Wo\ {\isasymsubseteq}\ W{\isachardot}\ finite\ Wo\ {\isasymand}\ {\isasymnot}{\isacharparenleft}sat\ {\isacharparenleft}{\isacharbraceleft}x{\isacharbraceright}\ {\isasymunion}\ Wo{\isacharparenright}{\isacharparenright}{\isachardoublequoteclose}\isanewline
\ \ \ \ \ \ \isacommand{using}\isamarkupfalse%
\ {\isacartoucheopen}Wo\ {\isasymsubseteq}\ W{\isacartoucheclose}\ \isacommand{by}\isamarkupfalse%
\ {\isacharparenleft}rule\ subexI{\isacharparenright}\isanewline
\ \ \isacommand{next}\isamarkupfalse%
\isanewline
\ \ \ \ \isacommand{assume}\isamarkupfalse%
\ {\isachardoublequoteopen}Wo{\isacharprime}\ {\isacharequal}\ Wo{\isachardoublequoteclose}\isanewline
\ \ \ \ \isacommand{then}\isamarkupfalse%
\ \isacommand{have}\isamarkupfalse%
\ {\isachardoublequoteopen}Wo{\isacharprime}\ {\isasymsubseteq}\ W{\isachardoublequoteclose}\isanewline
\ \ \ \ \ \ \isacommand{using}\isamarkupfalse%
\ {\isacartoucheopen}Wo\ {\isasymsubseteq}\ W{\isacartoucheclose}\ \isacommand{by}\isamarkupfalse%
\ {\isacharparenleft}simp\ only{\isacharcolon}\ {\isacartoucheopen}Wo{\isacharprime}\ {\isacharequal}\ Wo{\isacartoucheclose}{\isacharparenright}\isanewline
\ \ \ \ \isacommand{have}\isamarkupfalse%
\ {\isachardoublequoteopen}finite\ Wo{\isacharprime}\ {\isasymlongrightarrow}\ sat\ Wo{\isacharprime}{\isachardoublequoteclose}\isanewline
\ \ \ \ \ \ \isacommand{using}\isamarkupfalse%
\ WCol\ {\isacartoucheopen}Wo{\isacharprime}\ {\isasymsubseteq}\ W{\isacartoucheclose}\ \isacommand{by}\isamarkupfalse%
\ {\isacharparenleft}rule\ sspec{\isacharparenright}\isanewline
\ \ \ \ \isacommand{then}\isamarkupfalse%
\ \isacommand{have}\isamarkupfalse%
\ {\isachardoublequoteopen}sat\ Wo{\isacharprime}{\isachardoublequoteclose}\isanewline
\ \ \ \ \ \ \isacommand{using}\isamarkupfalse%
\ {\isacartoucheopen}finite\ Wo{\isacharprime}{\isacartoucheclose}\ \isacommand{by}\isamarkupfalse%
\ {\isacharparenleft}rule\ mp{\isacharparenright}\isanewline
\ \ \ \ \isacommand{show}\isamarkupfalse%
\ {\isacharquery}thesis\isanewline
\ \ \ \ \ \ \isacommand{using}\isamarkupfalse%
\ {\isacartoucheopen}{\isasymnot}\ sat\ Wo{\isacharprime}{\isacartoucheclose}\ {\isacartoucheopen}sat\ Wo{\isacharprime}{\isacartoucheclose}\ \isacommand{by}\isamarkupfalse%
\ {\isacharparenleft}rule\ notE{\isacharparenright}\isanewline
\ \ \isacommand{qed}\isamarkupfalse%
\isanewline
\isacommand{qed}\isamarkupfalse%
%
\endisatagproof
{\isafoldproof}%
%
\isadelimproof
\isanewline
%
\endisadelimproof
\isanewline
\isacommand{lemma}\isamarkupfalse%
\ pcp{\isacharunderscore}colecComp{\isacharunderscore}DIS{\isacharcolon}\isanewline
\ \ \isakeyword{assumes}\ {\isachardoublequoteopen}W\ {\isasymin}\ colecComp{\isachardoublequoteclose}\isanewline
\ \ \isakeyword{shows}\ {\isachardoublequoteopen}{\isasymforall}F\ G\ H{\isachardot}\ Dis\ F\ G\ H\ {\isasymlongrightarrow}\ F\ {\isasymin}\ W\ {\isasymlongrightarrow}\ {\isacharbraceleft}G{\isacharbraceright}\ {\isasymunion}\ W\ {\isasymin}\ colecComp\ {\isasymor}\ {\isacharbraceleft}H{\isacharbraceright}\ {\isasymunion}\ W\ {\isasymin}\ colecComp{\isachardoublequoteclose}\isanewline
%
\isadelimproof
%
\endisadelimproof
%
\isatagproof
\isacommand{proof}\isamarkupfalse%
\ {\isacharparenleft}rule\ allI{\isacharparenright}{\isacharplus}\isanewline
\ \ \isacommand{fix}\isamarkupfalse%
\ F\ G\ H\isanewline
\ \ \isacommand{show}\isamarkupfalse%
\ {\isachardoublequoteopen}Dis\ F\ G\ H\ {\isasymlongrightarrow}\ F\ {\isasymin}\ W\ {\isasymlongrightarrow}\ {\isacharbraceleft}G{\isacharbraceright}\ {\isasymunion}\ W\ {\isasymin}\ colecComp\ {\isasymor}\ {\isacharbraceleft}H{\isacharbraceright}\ {\isasymunion}\ W\ {\isasymin}\ colecComp{\isachardoublequoteclose}\isanewline
\ \ \isacommand{proof}\isamarkupfalse%
\ {\isacharparenleft}rule\ impI{\isacharparenright}{\isacharplus}\isanewline
\ \ \ \ \isacommand{assume}\isamarkupfalse%
\ {\isachardoublequoteopen}Dis\ F\ G\ H{\isachardoublequoteclose}\isanewline
\ \ \ \ \isacommand{assume}\isamarkupfalse%
\ {\isachardoublequoteopen}F\ {\isasymin}\ W{\isachardoublequoteclose}\isanewline
\ \ \ \ \isacommand{show}\isamarkupfalse%
\ {\isachardoublequoteopen}{\isacharbraceleft}G{\isacharbraceright}\ {\isasymunion}\ W\ {\isasymin}\ colecComp\ {\isasymor}\ {\isacharbraceleft}H{\isacharbraceright}\ {\isasymunion}\ W\ {\isasymin}\ colecComp{\isachardoublequoteclose}\isanewline
\ \ \ \ \isacommand{proof}\isamarkupfalse%
\ {\isacharparenleft}rule\ ccontr{\isacharparenright}\isanewline
\ \ \ \ \ \ \isacommand{assume}\isamarkupfalse%
\ {\isachardoublequoteopen}{\isasymnot}{\isacharparenleft}{\isacharbraceleft}G{\isacharbraceright}\ {\isasymunion}\ W\ {\isasymin}\ colecComp\ {\isasymor}\ {\isacharbraceleft}H{\isacharbraceright}\ {\isasymunion}\ W\ {\isasymin}\ colecComp{\isacharparenright}{\isachardoublequoteclose}\isanewline
\ \ \ \ \ \ \isacommand{then}\isamarkupfalse%
\ \isacommand{have}\isamarkupfalse%
\ C{\isacharcolon}{\isachardoublequoteopen}{\isacharbraceleft}G{\isacharbraceright}\ {\isasymunion}\ W\ {\isasymnotin}\ colecComp\ {\isasymand}\ {\isacharbraceleft}H{\isacharbraceright}\ {\isasymunion}\ W\ {\isasymnotin}\ colecComp{\isachardoublequoteclose}\isanewline
\ \ \ \ \ \ \ \ \isacommand{by}\isamarkupfalse%
\ {\isacharparenleft}simp\ only{\isacharcolon}\ de{\isacharunderscore}Morgan{\isacharunderscore}disj\ simp{\isacharunderscore}thms{\isacharparenleft}{\isadigit{8}}{\isacharparenright}{\isacharparenright}\isanewline
\ \ \ \ \ \ \isacommand{then}\isamarkupfalse%
\ \isacommand{have}\isamarkupfalse%
\ {\isachardoublequoteopen}{\isacharbraceleft}G{\isacharbraceright}\ {\isasymunion}\ W\ {\isasymnotin}\ colecComp{\isachardoublequoteclose}\isanewline
\ \ \ \ \ \ \ \ \isacommand{by}\isamarkupfalse%
\ {\isacharparenleft}rule\ conjunct{\isadigit{1}}{\isacharparenright}\isanewline
\ \ \ \ \ \ \isacommand{have}\isamarkupfalse%
\ Ex{\isadigit{1}}{\isacharcolon}{\isachardoublequoteopen}{\isasymexists}Wo\ {\isasymsubseteq}\ W{\isachardot}\ finite\ Wo\ {\isasymand}\ {\isasymnot}{\isacharparenleft}sat\ {\isacharparenleft}{\isacharbraceleft}G{\isacharbraceright}\ {\isasymunion}\ Wo{\isacharparenright}{\isacharparenright}{\isachardoublequoteclose}\isanewline
\ \ \ \ \ \ \ \ \isacommand{using}\isamarkupfalse%
\ assms\ {\isacartoucheopen}{\isacharbraceleft}G{\isacharbraceright}\ {\isasymunion}\ W\ {\isasymnotin}\ colecComp{\isacartoucheclose}\ \isacommand{by}\isamarkupfalse%
\ {\isacharparenleft}rule\ not{\isacharunderscore}colecComp{\isacharparenright}\isanewline
\ \ \ \ \ \ \isacommand{obtain}\isamarkupfalse%
\ W{\isadigit{1}}\ \isakeyword{where}\ {\isachardoublequoteopen}W{\isadigit{1}}\ {\isasymsubseteq}\ W{\isachardoublequoteclose}\ \isakeyword{and}\ C{\isadigit{1}}{\isacharcolon}{\isachardoublequoteopen}finite\ W{\isadigit{1}}\ {\isasymand}\ {\isasymnot}{\isacharparenleft}sat\ {\isacharparenleft}{\isacharbraceleft}G{\isacharbraceright}\ {\isasymunion}\ W{\isadigit{1}}{\isacharparenright}{\isacharparenright}{\isachardoublequoteclose}\isanewline
\ \ \ \ \ \ \ \ \isacommand{using}\isamarkupfalse%
\ Ex{\isadigit{1}}\ \isacommand{by}\isamarkupfalse%
\ {\isacharparenleft}rule\ subexE{\isacharparenright}\isanewline
\ \ \ \ \ \ \isacommand{have}\isamarkupfalse%
\ {\isachardoublequoteopen}finite\ W{\isadigit{1}}{\isachardoublequoteclose}\isanewline
\ \ \ \ \ \ \ \ \isacommand{using}\isamarkupfalse%
\ C{\isadigit{1}}\ \isacommand{by}\isamarkupfalse%
\ {\isacharparenleft}rule\ conjunct{\isadigit{1}}{\isacharparenright}\isanewline
\ \ \ \ \ \ \isacommand{have}\isamarkupfalse%
\ {\isachardoublequoteopen}{\isasymnot}{\isacharparenleft}sat\ {\isacharparenleft}{\isacharbraceleft}G{\isacharbraceright}\ {\isasymunion}\ W{\isadigit{1}}{\isacharparenright}{\isacharparenright}{\isachardoublequoteclose}\isanewline
\ \ \ \ \ \ \ \ \isacommand{using}\isamarkupfalse%
\ C{\isadigit{1}}\ \isacommand{by}\isamarkupfalse%
\ {\isacharparenleft}rule\ conjunct{\isadigit{2}}{\isacharparenright}\isanewline
\ \ \ \ \ \ \isacommand{have}\isamarkupfalse%
\ {\isachardoublequoteopen}{\isacharbraceleft}H{\isacharbraceright}\ {\isasymunion}\ W\ {\isasymnotin}\ colecComp{\isachardoublequoteclose}\isanewline
\ \ \ \ \ \ \ \ \isacommand{using}\isamarkupfalse%
\ C\ \isacommand{by}\isamarkupfalse%
\ {\isacharparenleft}rule\ conjunct{\isadigit{2}}{\isacharparenright}\ \isanewline
\ \ \ \ \ \ \isacommand{have}\isamarkupfalse%
\ Ex{\isadigit{2}}{\isacharcolon}{\isachardoublequoteopen}{\isasymexists}Wo\ {\isasymsubseteq}\ W{\isachardot}\ finite\ Wo\ {\isasymand}\ {\isasymnot}{\isacharparenleft}sat\ {\isacharparenleft}{\isacharbraceleft}H{\isacharbraceright}\ {\isasymunion}\ Wo{\isacharparenright}{\isacharparenright}{\isachardoublequoteclose}\isanewline
\ \ \ \ \ \ \ \ \isacommand{using}\isamarkupfalse%
\ assms\ {\isacartoucheopen}{\isacharbraceleft}H{\isacharbraceright}\ {\isasymunion}\ W\ {\isasymnotin}\ colecComp{\isacartoucheclose}\ \isacommand{by}\isamarkupfalse%
\ {\isacharparenleft}rule\ not{\isacharunderscore}colecComp{\isacharparenright}\isanewline
\ \ \ \ \ \ \isacommand{obtain}\isamarkupfalse%
\ W{\isadigit{2}}\ \isakeyword{where}\ {\isachardoublequoteopen}W{\isadigit{2}}\ {\isasymsubseteq}\ W{\isachardoublequoteclose}\ \isakeyword{and}\ C{\isadigit{2}}{\isacharcolon}{\isachardoublequoteopen}finite\ W{\isadigit{2}}\ {\isasymand}\ {\isasymnot}{\isacharparenleft}sat\ {\isacharparenleft}{\isacharbraceleft}H{\isacharbraceright}\ {\isasymunion}\ W{\isadigit{2}}{\isacharparenright}{\isacharparenright}{\isachardoublequoteclose}\isanewline
\ \ \ \ \ \ \ \ \isacommand{using}\isamarkupfalse%
\ Ex{\isadigit{2}}\ \isacommand{by}\isamarkupfalse%
\ {\isacharparenleft}rule\ subexE{\isacharparenright}\isanewline
\ \ \ \ \ \ \isacommand{have}\isamarkupfalse%
\ {\isachardoublequoteopen}finite\ W{\isadigit{2}}{\isachardoublequoteclose}\isanewline
\ \ \ \ \ \ \ \ \isacommand{using}\isamarkupfalse%
\ C{\isadigit{2}}\ \isacommand{by}\isamarkupfalse%
\ {\isacharparenleft}rule\ conjunct{\isadigit{1}}{\isacharparenright}\isanewline
\ \ \ \ \ \ \isacommand{have}\isamarkupfalse%
\ {\isachardoublequoteopen}{\isasymnot}{\isacharparenleft}sat\ {\isacharparenleft}{\isacharbraceleft}H{\isacharbraceright}\ {\isasymunion}\ W{\isadigit{2}}{\isacharparenright}{\isacharparenright}{\isachardoublequoteclose}\isanewline
\ \ \ \ \ \ \ \ \isacommand{using}\isamarkupfalse%
\ C{\isadigit{2}}\ \isacommand{by}\isamarkupfalse%
\ {\isacharparenleft}rule\ conjunct{\isadigit{2}}{\isacharparenright}\isanewline
\ \ \ \ \ \ \isacommand{let}\isamarkupfalse%
\ {\isacharquery}Wo\ {\isacharequal}\ {\isachardoublequoteopen}W{\isadigit{1}}\ {\isasymunion}\ W{\isadigit{2}}{\isachardoublequoteclose}\isanewline
\ \ \ \ \ \ \isacommand{have}\isamarkupfalse%
\ {\isachardoublequoteopen}{\isacharquery}Wo\ {\isasymsubseteq}\ W{\isachardoublequoteclose}\isanewline
\ \ \ \ \ \ \ \ \isacommand{using}\isamarkupfalse%
\ {\isacartoucheopen}W{\isadigit{1}}\ {\isasymsubseteq}\ W{\isacartoucheclose}\ {\isacartoucheopen}W{\isadigit{2}}\ {\isasymsubseteq}\ W{\isacartoucheclose}\ \isacommand{by}\isamarkupfalse%
\ {\isacharparenleft}simp\ only{\isacharcolon}\ Un{\isacharunderscore}least{\isacharparenright}\isanewline
\ \ \ \ \ \ \isacommand{have}\isamarkupfalse%
\ {\isachardoublequoteopen}finite\ {\isacharquery}Wo{\isachardoublequoteclose}\isanewline
\ \ \ \ \ \ \ \ \isacommand{using}\isamarkupfalse%
\ {\isacartoucheopen}finite\ W{\isadigit{1}}{\isacartoucheclose}\ {\isacartoucheopen}finite\ W{\isadigit{2}}{\isacartoucheclose}\ \isacommand{by}\isamarkupfalse%
\ {\isacharparenleft}simp\ only{\isacharcolon}\ finite{\isacharunderscore}Un{\isacharparenright}\isanewline
\ \ \ \ \ \ \isacommand{have}\isamarkupfalse%
\ {\isachardoublequoteopen}{\isacharbraceleft}G{\isacharbraceright}\ {\isasymunion}\ W{\isadigit{1}}\ {\isasymsubseteq}\ {\isacharparenleft}{\isacharbraceleft}G{\isacharbraceright}\ {\isasymunion}\ W{\isadigit{1}}{\isacharparenright}\ {\isasymunion}\ W{\isadigit{2}}{\isachardoublequoteclose}\isanewline
\ \ \ \ \ \ \ \ \isacommand{by}\isamarkupfalse%
\ {\isacharparenleft}simp\ only{\isacharcolon}\ Un{\isacharunderscore}upper{\isadigit{1}}{\isacharparenright}\isanewline
\ \ \ \ \ \ \isacommand{then}\isamarkupfalse%
\ \isacommand{have}\isamarkupfalse%
\ {\isachardoublequoteopen}{\isacharbraceleft}G{\isacharbraceright}\ {\isasymunion}\ W{\isadigit{1}}\ {\isasymsubseteq}\ {\isacharbraceleft}G{\isacharbraceright}\ {\isasymunion}\ {\isacharquery}Wo{\isachardoublequoteclose}\isanewline
\ \ \ \ \ \ \ \ \isacommand{by}\isamarkupfalse%
\ {\isacharparenleft}simp\ only{\isacharcolon}\ Un{\isacharunderscore}assoc{\isacharparenright}\isanewline
\ \ \ \ \ \ \isacommand{have}\isamarkupfalse%
\ {\isachardoublequoteopen}{\isasymnot}\ sat\ {\isacharparenleft}{\isacharbraceleft}G{\isacharbraceright}\ {\isasymunion}\ {\isacharquery}Wo{\isacharparenright}{\isachardoublequoteclose}\isanewline
\ \ \ \ \ \ \ \ \isacommand{using}\isamarkupfalse%
\ {\isacartoucheopen}{\isacharbraceleft}G{\isacharbraceright}\ {\isasymunion}\ W{\isadigit{1}}\ {\isasymsubseteq}\ {\isacharbraceleft}G{\isacharbraceright}\ {\isasymunion}\ {\isacharquery}Wo{\isacartoucheclose}\ {\isacartoucheopen}{\isasymnot}\ sat\ {\isacharparenleft}{\isacharbraceleft}G{\isacharbraceright}\ {\isasymunion}\ W{\isadigit{1}}{\isacharparenright}{\isacartoucheclose}\ \isacommand{by}\isamarkupfalse%
\ {\isacharparenleft}rule\ sat{\isacharunderscore}subset{\isacharunderscore}ccontr{\isacharparenright}\isanewline
\ \ \ \ \ \ \isacommand{have}\isamarkupfalse%
\ {\isachardoublequoteopen}{\isacharbraceleft}H{\isacharbraceright}\ {\isasymunion}\ W{\isadigit{2}}\ {\isasymsubseteq}\ {\isacharparenleft}{\isacharbraceleft}H{\isacharbraceright}\ {\isasymunion}\ W{\isadigit{2}}{\isacharparenright}\ {\isasymunion}\ W{\isadigit{1}}{\isachardoublequoteclose}\isanewline
\ \ \ \ \ \ \ \ \isacommand{by}\isamarkupfalse%
\ {\isacharparenleft}simp\ only{\isacharcolon}\ Un{\isacharunderscore}upper{\isadigit{1}}{\isacharparenright}\isanewline
\ \ \ \ \ \ \isacommand{then}\isamarkupfalse%
\ \isacommand{have}\isamarkupfalse%
\ {\isachardoublequoteopen}{\isacharbraceleft}H{\isacharbraceright}\ {\isasymunion}\ W{\isadigit{2}}\ {\isasymsubseteq}\ {\isacharbraceleft}H{\isacharbraceright}\ {\isasymunion}\ {\isacharparenleft}W{\isadigit{2}}\ {\isasymunion}\ W{\isadigit{1}}{\isacharparenright}{\isachardoublequoteclose}\isanewline
\ \ \ \ \ \ \ \ \isacommand{by}\isamarkupfalse%
\ {\isacharparenleft}simp\ only{\isacharcolon}\ Un{\isacharunderscore}assoc{\isacharparenright}\ \isanewline
\ \ \ \ \ \ \isacommand{then}\isamarkupfalse%
\ \isacommand{have}\isamarkupfalse%
\ {\isachardoublequoteopen}{\isacharbraceleft}H{\isacharbraceright}\ {\isasymunion}\ W{\isadigit{2}}\ {\isasymsubseteq}\ {\isacharbraceleft}H{\isacharbraceright}\ {\isasymunion}\ {\isacharquery}Wo{\isachardoublequoteclose}\isanewline
\ \ \ \ \ \ \ \ \isacommand{by}\isamarkupfalse%
\ {\isacharparenleft}simp\ only{\isacharcolon}\ Un{\isacharunderscore}commute{\isacharparenright}\isanewline
\ \ \ \ \ \ \isacommand{have}\isamarkupfalse%
\ {\isachardoublequoteopen}{\isasymnot}\ sat\ {\isacharparenleft}{\isacharbraceleft}H{\isacharbraceright}\ {\isasymunion}\ {\isacharquery}Wo{\isacharparenright}{\isachardoublequoteclose}\isanewline
\ \ \ \ \ \ \ \ \isacommand{using}\isamarkupfalse%
\ {\isacartoucheopen}{\isacharbraceleft}H{\isacharbraceright}\ {\isasymunion}\ W{\isadigit{2}}\ {\isasymsubseteq}\ {\isacharbraceleft}H{\isacharbraceright}\ {\isasymunion}\ {\isacharquery}Wo{\isacartoucheclose}\ {\isacartoucheopen}{\isasymnot}\ sat\ {\isacharparenleft}{\isacharbraceleft}H{\isacharbraceright}\ {\isasymunion}\ W{\isadigit{2}}{\isacharparenright}{\isacartoucheclose}\ \isacommand{by}\isamarkupfalse%
\ {\isacharparenleft}rule\ sat{\isacharunderscore}subset{\isacharunderscore}ccontr{\isacharparenright}\isanewline
\ \ \ \ \ \ \isacommand{have}\isamarkupfalse%
\ {\isachardoublequoteopen}{\isasymnot}\ sat\ {\isacharparenleft}{\isacharbraceleft}G{\isacharbraceright}\ {\isasymunion}\ {\isacharquery}Wo{\isacharparenright}\ {\isasymand}\ {\isasymnot}\ sat\ {\isacharparenleft}{\isacharbraceleft}H{\isacharbraceright}\ {\isasymunion}\ {\isacharquery}Wo{\isacharparenright}{\isachardoublequoteclose}\isanewline
\ \ \ \ \ \ \ \ \isacommand{using}\isamarkupfalse%
\ {\isacartoucheopen}{\isasymnot}\ sat\ {\isacharparenleft}{\isacharbraceleft}G{\isacharbraceright}\ {\isasymunion}\ {\isacharquery}Wo{\isacharparenright}{\isacartoucheclose}\ {\isacartoucheopen}{\isasymnot}\ sat\ {\isacharparenleft}{\isacharbraceleft}H{\isacharbraceright}\ {\isasymunion}\ {\isacharquery}Wo{\isacharparenright}{\isacartoucheclose}\ \isacommand{by}\isamarkupfalse%
\ {\isacharparenleft}rule\ conjI{\isacharparenright}\isanewline
\ \ \ \ \ \ \isacommand{have}\isamarkupfalse%
\ {\isachardoublequoteopen}sat\ {\isacharparenleft}{\isacharbraceleft}G{\isacharbraceright}\ {\isasymunion}\ {\isacharquery}Wo{\isacharparenright}\ {\isasymor}\ sat\ {\isacharparenleft}{\isacharbraceleft}H{\isacharbraceright}\ {\isasymunion}\ {\isacharquery}Wo{\isacharparenright}{\isachardoublequoteclose}\isanewline
\ \ \ \ \ \ \ \ \isacommand{using}\isamarkupfalse%
\ assms{\isacharparenleft}{\isadigit{1}}{\isacharparenright}\ {\isacartoucheopen}Dis\ F\ G\ H{\isacartoucheclose}\ {\isacartoucheopen}F\ {\isasymin}\ W{\isacartoucheclose}\ {\isacartoucheopen}finite\ {\isacharquery}Wo{\isacartoucheclose}\ {\isacartoucheopen}{\isacharquery}Wo\ {\isasymsubseteq}\ W{\isacartoucheclose}\ \isacommand{by}\isamarkupfalse%
\ {\isacharparenleft}rule\ pcp{\isacharunderscore}colecComp{\isacharunderscore}DIS{\isacharunderscore}sat{\isacharparenright}\isanewline
\ \ \ \ \ \ \isacommand{then}\isamarkupfalse%
\ \isacommand{have}\isamarkupfalse%
\ {\isachardoublequoteopen}{\isasymnot}{\isasymnot}{\isacharparenleft}sat\ {\isacharparenleft}{\isacharbraceleft}G{\isacharbraceright}\ {\isasymunion}\ {\isacharquery}Wo{\isacharparenright}\ {\isasymor}\ sat\ {\isacharparenleft}{\isacharbraceleft}H{\isacharbraceright}\ {\isasymunion}\ {\isacharquery}Wo{\isacharparenright}{\isacharparenright}{\isachardoublequoteclose}\isanewline
\ \ \ \ \ \ \ \ \isacommand{by}\isamarkupfalse%
\ {\isacharparenleft}simp\ only{\isacharcolon}\ not{\isacharunderscore}not{\isacharparenright}\isanewline
\ \ \ \ \ \ \isacommand{then}\isamarkupfalse%
\ \isacommand{have}\isamarkupfalse%
\ {\isachardoublequoteopen}{\isasymnot}{\isacharparenleft}{\isasymnot}{\isacharparenleft}sat\ {\isacharparenleft}{\isacharbraceleft}G{\isacharbraceright}\ {\isasymunion}\ {\isacharquery}Wo{\isacharparenright}{\isacharparenright}\ {\isasymand}\ {\isasymnot}{\isacharparenleft}sat\ {\isacharparenleft}{\isacharbraceleft}H{\isacharbraceright}\ {\isasymunion}\ {\isacharquery}Wo{\isacharparenright}{\isacharparenright}{\isacharparenright}{\isachardoublequoteclose}\isanewline
\ \ \ \ \ \ \ \ \isacommand{by}\isamarkupfalse%
\ {\isacharparenleft}simp\ only{\isacharcolon}\ de{\isacharunderscore}Morgan{\isacharunderscore}disj\ simp{\isacharunderscore}thms{\isacharparenleft}{\isadigit{8}}{\isacharparenright}{\isacharparenright}\isanewline
\ \ \ \ \ \ \isacommand{thus}\isamarkupfalse%
\ {\isachardoublequoteopen}False{\isachardoublequoteclose}\isanewline
\ \ \ \ \ \ \ \ \isacommand{using}\isamarkupfalse%
\ {\isacartoucheopen}{\isasymnot}{\isacharparenleft}sat\ {\isacharparenleft}{\isacharbraceleft}G{\isacharbraceright}\ {\isasymunion}\ {\isacharquery}Wo{\isacharparenright}{\isacharparenright}\ {\isasymand}\ {\isasymnot}{\isacharparenleft}sat\ {\isacharparenleft}{\isacharbraceleft}H{\isacharbraceright}\ {\isasymunion}\ {\isacharquery}Wo{\isacharparenright}{\isacharparenright}{\isacartoucheclose}\ \isacommand{by}\isamarkupfalse%
\ {\isacharparenleft}rule\ notE{\isacharparenright}\isanewline
\ \ \ \ \isacommand{qed}\isamarkupfalse%
\isanewline
\ \ \isacommand{qed}\isamarkupfalse%
\isanewline
\isacommand{qed}\isamarkupfalse%
%
\endisatagproof
{\isafoldproof}%
%
\isadelimproof
\isanewline
%
\endisadelimproof
\isanewline
\isacommand{lemma}\isamarkupfalse%
\ pcp{\isacharunderscore}colecComp{\isacharcolon}\ {\isachardoublequoteopen}pcp\ colecComp{\isachardoublequoteclose}\isanewline
%
\isadelimproof
%
\endisadelimproof
%
\isatagproof
\isacommand{proof}\isamarkupfalse%
\ {\isacharparenleft}rule\ pcp{\isacharunderscore}alt{\isadigit{2}}{\isacharparenright}\isanewline
\ \ \isacommand{show}\isamarkupfalse%
\ {\isachardoublequoteopen}{\isasymforall}W\ {\isasymin}\ colecComp{\isachardot}\ {\isasymbottom}\ {\isasymnotin}\ W\isanewline
\ \ \ \ \ \ \ \ {\isasymand}\ {\isacharparenleft}{\isasymforall}k{\isachardot}\ Atom\ k\ {\isasymin}\ W\ {\isasymlongrightarrow}\ \isactrlbold {\isasymnot}\ {\isacharparenleft}Atom\ k{\isacharparenright}\ {\isasymin}\ W\ {\isasymlongrightarrow}\ False{\isacharparenright}\isanewline
\ \ \ \ \ \ \ \ {\isasymand}\ {\isacharparenleft}{\isasymforall}F\ G\ H{\isachardot}\ Con\ F\ G\ H\ {\isasymlongrightarrow}\ F\ {\isasymin}\ W\ {\isasymlongrightarrow}\ {\isacharbraceleft}G{\isacharcomma}H{\isacharbraceright}\ {\isasymunion}\ W\ {\isasymin}\ colecComp{\isacharparenright}\isanewline
\ \ \ \ \ \ \ \ {\isasymand}\ {\isacharparenleft}{\isasymforall}F\ G\ H{\isachardot}\ Dis\ F\ G\ H\ {\isasymlongrightarrow}\ F\ {\isasymin}\ W\ {\isasymlongrightarrow}\ {\isacharbraceleft}G{\isacharbraceright}\ {\isasymunion}\ W\ {\isasymin}\ colecComp\ {\isasymor}\ {\isacharbraceleft}H{\isacharbraceright}\ {\isasymunion}\ W\ {\isasymin}\ colecComp{\isacharparenright}{\isachardoublequoteclose}\isanewline
\ \ \isacommand{proof}\isamarkupfalse%
\ {\isacharparenleft}rule\ ballI{\isacharparenright}\isanewline
\ \ \ \ \isacommand{fix}\isamarkupfalse%
\ W\isanewline
\ \ \ \ \isacommand{assume}\isamarkupfalse%
\ H{\isacharcolon}{\isachardoublequoteopen}W\ {\isasymin}\ colecComp{\isachardoublequoteclose}\isanewline
\ \ \ \ \isacommand{have}\isamarkupfalse%
\ C{\isadigit{1}}{\isacharcolon}{\isachardoublequoteopen}{\isasymbottom}\ {\isasymnotin}\ W{\isachardoublequoteclose}\isanewline
\ \ \ \ \ \ \isacommand{using}\isamarkupfalse%
\ H\ \isacommand{by}\isamarkupfalse%
\ {\isacharparenleft}rule\ pcp{\isacharunderscore}colecComp{\isacharunderscore}bot{\isacharparenright}\isanewline
\ \ \ \ \isacommand{have}\isamarkupfalse%
\ C{\isadigit{2}}{\isacharcolon}{\isachardoublequoteopen}{\isasymforall}k{\isachardot}\ Atom\ k\ {\isasymin}\ W\ {\isasymlongrightarrow}\ \isactrlbold {\isasymnot}\ {\isacharparenleft}Atom\ k{\isacharparenright}\ {\isasymin}\ W\ {\isasymlongrightarrow}\ False{\isachardoublequoteclose}\isanewline
\ \ \ \ \ \ \isacommand{using}\isamarkupfalse%
\ H\ \isacommand{by}\isamarkupfalse%
\ {\isacharparenleft}rule\ pcp{\isacharunderscore}colecComp{\isacharunderscore}atoms{\isacharparenright}\isanewline
\ \ \ \ \isacommand{have}\isamarkupfalse%
\ C{\isadigit{3}}{\isacharcolon}{\isachardoublequoteopen}{\isasymforall}F\ G\ H{\isachardot}\ Con\ F\ G\ H\ {\isasymlongrightarrow}\ F\ {\isasymin}\ W\ {\isasymlongrightarrow}\ {\isacharbraceleft}G{\isacharcomma}H{\isacharbraceright}\ {\isasymunion}\ W\ {\isasymin}\ colecComp{\isachardoublequoteclose}\isanewline
\ \ \ \ \ \ \isacommand{using}\isamarkupfalse%
\ H\ \isacommand{by}\isamarkupfalse%
\ {\isacharparenleft}rule\ pcp{\isacharunderscore}colecComp{\isacharunderscore}CON{\isacharparenright}\isanewline
\ \ \ \ \isacommand{have}\isamarkupfalse%
\ C{\isadigit{4}}{\isacharcolon}{\isachardoublequoteopen}{\isasymforall}F\ G\ H{\isachardot}\ Dis\ F\ G\ H\ {\isasymlongrightarrow}\ F\ {\isasymin}\ W\ {\isasymlongrightarrow}\ {\isacharbraceleft}G{\isacharbraceright}\ {\isasymunion}\ W\ {\isasymin}\ colecComp\ {\isasymor}\ {\isacharbraceleft}H{\isacharbraceright}\ {\isasymunion}\ W\ {\isasymin}\ colecComp{\isachardoublequoteclose}\isanewline
\ \ \ \ \ \ \isacommand{using}\isamarkupfalse%
\ H\ \isacommand{by}\isamarkupfalse%
\ {\isacharparenleft}rule\ pcp{\isacharunderscore}colecComp{\isacharunderscore}DIS{\isacharparenright}\isanewline
\ \ \ \ \isacommand{show}\isamarkupfalse%
\ {\isachardoublequoteopen}{\isasymbottom}\ {\isasymnotin}\ W\isanewline
\ \ \ \ \ \ \ \ \ \ {\isasymand}\ {\isacharparenleft}{\isasymforall}k{\isachardot}\ Atom\ k\ {\isasymin}\ W\ {\isasymlongrightarrow}\ \isactrlbold {\isasymnot}\ {\isacharparenleft}Atom\ k{\isacharparenright}\ {\isasymin}\ W\ {\isasymlongrightarrow}\ False{\isacharparenright}\isanewline
\ \ \ \ \ \ \ \ \ \ {\isasymand}\ {\isacharparenleft}{\isasymforall}F\ G\ H{\isachardot}\ Con\ F\ G\ H\ {\isasymlongrightarrow}\ F\ {\isasymin}\ W\ {\isasymlongrightarrow}\ {\isacharbraceleft}G{\isacharcomma}H{\isacharbraceright}\ {\isasymunion}\ W\ {\isasymin}\ colecComp{\isacharparenright}\isanewline
\ \ \ \ \ \ \ \ \ \ {\isasymand}\ {\isacharparenleft}{\isasymforall}F\ G\ H{\isachardot}\ Dis\ F\ G\ H\ {\isasymlongrightarrow}\ F\ {\isasymin}\ W\ {\isasymlongrightarrow}\ {\isacharbraceleft}G{\isacharbraceright}\ {\isasymunion}\ W\ {\isasymin}\ colecComp\ {\isasymor}\ {\isacharbraceleft}H{\isacharbraceright}\ {\isasymunion}\ W\ {\isasymin}\ colecComp{\isacharparenright}{\isachardoublequoteclose}\isanewline
\ \ \ \ \ \ \isacommand{using}\isamarkupfalse%
\ C{\isadigit{1}}\ C{\isadigit{2}}\ C{\isadigit{3}}\ C{\isadigit{4}}\ \isacommand{by}\isamarkupfalse%
\ {\isacharparenleft}iprover\ intro{\isacharcolon}\ conjI{\isacharparenright}\isanewline
\ \ \isacommand{qed}\isamarkupfalse%
\isanewline
\isacommand{qed}\isamarkupfalse%
%
\endisatagproof
{\isafoldproof}%
%
\isadelimproof
\isanewline
%
\endisadelimproof
\isanewline
\isacommand{lemma}\isamarkupfalse%
\ prop{\isacharunderscore}Compactness{\isacharcolon}\isanewline
\ \ \isakeyword{fixes}\ S\ {\isacharcolon}{\isacharcolon}\ {\isachardoublequoteopen}{\isacharprime}a\ {\isacharcolon}{\isacharcolon}\ countable\ formula\ set{\isachardoublequoteclose}\isanewline
\ \ \isakeyword{assumes}\ {\isachardoublequoteopen}fin{\isacharunderscore}sat\ S{\isachardoublequoteclose}\isanewline
\ \ \isakeyword{shows}\ {\isachardoublequoteopen}sat\ S{\isachardoublequoteclose}\isanewline
%
\isadelimproof
%
\endisadelimproof
%
\isatagproof
\isacommand{proof}\isamarkupfalse%
\ {\isacharminus}\isanewline
\ \ \isacommand{have}\isamarkupfalse%
\ {\isachardoublequoteopen}pcp\ colecComp{\isachardoublequoteclose}\isanewline
\ \ \ \ \isacommand{by}\isamarkupfalse%
\ {\isacharparenleft}rule\ pcp{\isacharunderscore}colecComp{\isacharparenright}\isanewline
\ \ \isacommand{have}\isamarkupfalse%
\ {\isachardoublequoteopen}S\ {\isasymin}\ colecComp{\isachardoublequoteclose}\isanewline
\ \ \ \ \isacommand{using}\isamarkupfalse%
\ assms\ \isacommand{by}\isamarkupfalse%
\ {\isacharparenleft}rule\ set{\isacharunderscore}in{\isacharunderscore}colecComp{\isacharparenright}\isanewline
\ \ \isacommand{thus}\isamarkupfalse%
\ {\isachardoublequoteopen}sat\ S{\isachardoublequoteclose}\ \isanewline
\ \ \ \ \isacommand{using}\isamarkupfalse%
\ {\isacartoucheopen}pcp\ colecComp{\isacartoucheclose}\ pcp{\isacharunderscore}sat\ \isacommand{by}\isamarkupfalse%
\ blast\ \isanewline
\isacommand{qed}\isamarkupfalse%
\isanewline
%
\endisatagproof
{\isafoldproof}%
%
\isadelimproof
%
\endisadelimproof
%
\isadelimtheory
%
\endisadelimtheory
%
\isatagtheory
%
\endisatagtheory
{\isafoldtheory}%
%
\isadelimtheory
%
\endisadelimtheory
%
\end{isabellebody}%
\endinput
%:%file=~/TFM/TFM/TeoremaEx.thy%:%
%:%19=13%:%
%:%23=15%:%
%:%32=17%:%
%:%44=19%:%
%:%45=20%:%
%:%46=21%:%
%:%47=22%:%
%:%48=23%:%
%:%49=24%:%
%:%50=25%:%
%:%51=26%:%
%:%52=27%:%
%:%53=28%:%
%:%54=29%:%
%:%55=30%:%
%:%56=31%:%
%:%57=32%:%
%:%58=33%:%
%:%59=34%:%
%:%59=35%:%
%:%60=36%:%
%:%61=37%:%
%:%62=38%:%
%:%63=39%:%
%:%64=40%:%
%:%65=41%:%
%:%66=42%:%
%:%67=43%:%
%:%68=44%:%
%:%69=45%:%
%:%70=46%:%
%:%71=47%:%
%:%72=48%:%
%:%73=49%:%
%:%74=50%:%
%:%75=51%:%
%:%76=52%:%
%:%77=53%:%
%:%78=54%:%
%:%79=55%:%
%:%80=56%:%
%:%82=58%:%
%:%83=58%:%
%:%84=59%:%
%:%85=60%:%
%:%88=63%:%
%:%89=64%:%
%:%90=65%:%
%:%91=66%:%
%:%92=67%:%
%:%93=68%:%
%:%94=69%:%
%:%95=70%:%
%:%96=71%:%
%:%97=72%:%
%:%98=73%:%
%:%99=74%:%
%:%100=75%:%
%:%101=76%:%
%:%102=77%:%
%:%103=78%:%
%:%104=79%:%
%:%105=80%:%
%:%106=81%:%
%:%107=82%:%
%:%108=83%:%
%:%109=84%:%
%:%110=85%:%
%:%112=87%:%
%:%113=87%:%
%:%114=88%:%
%:%115=89%:%
%:%116=90%:%
%:%123=91%:%
%:%124=91%:%
%:%125=92%:%
%:%126=92%:%
%:%127=93%:%
%:%128=93%:%
%:%129=94%:%
%:%130=94%:%
%:%131=95%:%
%:%132=95%:%
%:%133=96%:%
%:%134=96%:%
%:%135=97%:%
%:%136=97%:%
%:%137=98%:%
%:%138=99%:%
%:%139=99%:%
%:%140=100%:%
%:%141=100%:%
%:%142=100%:%
%:%143=101%:%
%:%144=102%:%
%:%145=102%:%
%:%146=103%:%
%:%147=103%:%
%:%148=104%:%
%:%149=104%:%
%:%150=105%:%
%:%151=105%:%
%:%152=106%:%
%:%153=106%:%
%:%154=107%:%
%:%155=107%:%
%:%156=107%:%
%:%157=108%:%
%:%158=108%:%
%:%159=109%:%
%:%160=109%:%
%:%161=110%:%
%:%162=110%:%
%:%163=111%:%
%:%164=111%:%
%:%165=112%:%
%:%166=112%:%
%:%167=113%:%
%:%168=113%:%
%:%169=113%:%
%:%170=114%:%
%:%171=114%:%
%:%172=115%:%
%:%173=115%:%
%:%174=116%:%
%:%175=116%:%
%:%176=117%:%
%:%186=119%:%
%:%188=121%:%
%:%189=121%:%
%:%196=122%:%
%:%197=122%:%
%:%198=123%:%
%:%199=123%:%
%:%200=124%:%
%:%201=124%:%
%:%202=124%:%
%:%203=125%:%
%:%204=125%:%
%:%205=125%:%
%:%206=126%:%
%:%207=126%:%
%:%216=128%:%
%:%217=129%:%
%:%218=130%:%
%:%219=131%:%
%:%220=132%:%
%:%221=133%:%
%:%222=134%:%
%:%223=135%:%
%:%225=137%:%
%:%226=137%:%
%:%229=138%:%
%:%233=138%:%
%:%243=140%:%
%:%244=141%:%
%:%245=142%:%
%:%246=143%:%
%:%247=144%:%
%:%248=145%:%
%:%249=146%:%
%:%250=147%:%
%:%251=148%:%
%:%252=149%:%
%:%253=150%:%
%:%254=151%:%
%:%255=152%:%
%:%257=154%:%
%:%258=154%:%
%:%265=155%:%
%:%266=155%:%
%:%267=156%:%
%:%268=156%:%
%:%269=157%:%
%:%270=158%:%
%:%271=158%:%
%:%272=159%:%
%:%273=159%:%
%:%274=159%:%
%:%275=160%:%
%:%276=161%:%
%:%277=161%:%
%:%278=162%:%
%:%279=162%:%
%:%280=163%:%
%:%281=163%:%
%:%282=164%:%
%:%283=164%:%
%:%284=165%:%
%:%285=165%:%
%:%286=166%:%
%:%287=166%:%
%:%288=166%:%
%:%289=167%:%
%:%290=167%:%
%:%291=168%:%
%:%292=168%:%
%:%293=169%:%
%:%294=169%:%
%:%295=170%:%
%:%296=170%:%
%:%297=171%:%
%:%298=171%:%
%:%299=172%:%
%:%300=172%:%
%:%301=172%:%
%:%302=173%:%
%:%303=173%:%
%:%304=174%:%
%:%305=174%:%
%:%306=175%:%
%:%307=175%:%
%:%308=176%:%
%:%318=178%:%
%:%320=180%:%
%:%321=180%:%
%:%324=181%:%
%:%328=181%:%
%:%329=181%:%
%:%338=183%:%
%:%339=184%:%
%:%341=186%:%
%:%342=186%:%
%:%343=187%:%
%:%344=188%:%
%:%347=189%:%
%:%351=189%:%
%:%352=189%:%
%:%353=189%:%
%:%362=191%:%
%:%363=192%:%
%:%364=193%:%
%:%365=194%:%
%:%366=195%:%
%:%367=196%:%
%:%368=197%:%
%:%369=198%:%
%:%370=199%:%
%:%371=200%:%
%:%372=201%:%
%:%373=202%:%
%:%374=203%:%
%:%375=204%:%
%:%376=205%:%
%:%377=206%:%
%:%378=207%:%
%:%379=208%:%
%:%380=209%:%
%:%381=210%:%
%:%382=211%:%
%:%383=212%:%
%:%384=213%:%
%:%385=214%:%
%:%386=215%:%
%:%387=216%:%
%:%388=217%:%
%:%389=218%:%
%:%390=219%:%
%:%391=220%:%
%:%392=221%:%
%:%393=222%:%
%:%394=223%:%
%:%395=224%:%
%:%396=225%:%
%:%397=226%:%
%:%398=227%:%
%:%399=228%:%
%:%400=229%:%
%:%401=230%:%
%:%402=231%:%
%:%403=232%:%
%:%404=233%:%
%:%406=235%:%
%:%407=235%:%
%:%414=236%:%
%:%415=236%:%
%:%416=237%:%
%:%417=237%:%
%:%418=238%:%
%:%419=238%:%
%:%420=239%:%
%:%421=239%:%
%:%422=239%:%
%:%423=240%:%
%:%424=240%:%
%:%425=241%:%
%:%426=241%:%
%:%427=241%:%
%:%428=242%:%
%:%429=242%:%
%:%430=243%:%
%:%431=243%:%
%:%432=243%:%
%:%433=244%:%
%:%434=244%:%
%:%435=245%:%
%:%436=245%:%
%:%437=245%:%
%:%438=246%:%
%:%439=246%:%
%:%440=247%:%
%:%441=247%:%
%:%442=248%:%
%:%443=248%:%
%:%444=249%:%
%:%445=249%:%
%:%446=250%:%
%:%447=250%:%
%:%448=251%:%
%:%449=251%:%
%:%450=252%:%
%:%451=252%:%
%:%452=252%:%
%:%453=253%:%
%:%454=253%:%
%:%455=254%:%
%:%456=254%:%
%:%457=255%:%
%:%458=255%:%
%:%459=256%:%
%:%460=256%:%
%:%461=256%:%
%:%462=257%:%
%:%463=257%:%
%:%464=258%:%
%:%465=258%:%
%:%466=258%:%
%:%467=259%:%
%:%468=259%:%
%:%469=260%:%
%:%470=260%:%
%:%471=260%:%
%:%472=261%:%
%:%473=261%:%
%:%474=262%:%
%:%475=262%:%
%:%476=262%:%
%:%477=263%:%
%:%478=263%:%
%:%479=264%:%
%:%480=264%:%
%:%481=264%:%
%:%482=265%:%
%:%483=265%:%
%:%484=266%:%
%:%485=266%:%
%:%486=266%:%
%:%487=267%:%
%:%488=267%:%
%:%489=267%:%
%:%490=268%:%
%:%491=268%:%
%:%492=268%:%
%:%493=269%:%
%:%494=269%:%
%:%495=270%:%
%:%505=272%:%
%:%507=274%:%
%:%508=274%:%
%:%515=275%:%
%:%516=275%:%
%:%517=276%:%
%:%518=276%:%
%:%519=277%:%
%:%520=277%:%
%:%521=278%:%
%:%522=278%:%
%:%523=278%:%
%:%524=279%:%
%:%525=279%:%
%:%526=280%:%
%:%527=280%:%
%:%528=281%:%
%:%529=281%:%
%:%530=282%:%
%:%531=282%:%
%:%532=283%:%
%:%533=283%:%
%:%534=283%:%
%:%535=283%:%
%:%536=284%:%
%:%537=284%:%
%:%546=286%:%
%:%547=287%:%
%:%548=288%:%
%:%549=289%:%
%:%550=290%:%
%:%551=291%:%
%:%552=292%:%
%:%553=293%:%
%:%554=294%:%
%:%556=296%:%
%:%557=296%:%
%:%559=298%:%
%:%560=299%:%
%:%561=300%:%
%:%562=301%:%
%:%563=302%:%
%:%564=303%:%
%:%565=304%:%
%:%566=305%:%
%:%567=306%:%
%:%568=307%:%
%:%569=308%:%
%:%570=309%:%
%:%571=310%:%
%:%572=311%:%
%:%573=312%:%
%:%574=313%:%
%:%576=315%:%
%:%577=315%:%
%:%580=316%:%
%:%584=316%:%
%:%585=316%:%
%:%586=317%:%
%:%587=317%:%
%:%588=318%:%
%:%589=318%:%
%:%590=319%:%
%:%591=319%:%
%:%592=320%:%
%:%593=320%:%
%:%594=320%:%
%:%595=321%:%
%:%596=321%:%
%:%597=322%:%
%:%598=322%:%
%:%599=322%:%
%:%600=323%:%
%:%601=323%:%
%:%602=324%:%
%:%603=324%:%
%:%604=325%:%
%:%605=325%:%
%:%606=326%:%
%:%616=328%:%
%:%618=330%:%
%:%619=330%:%
%:%622=331%:%
%:%626=331%:%
%:%627=331%:%
%:%628=331%:%
%:%637=333%:%
%:%638=334%:%
%:%639=335%:%
%:%640=336%:%
%:%641=337%:%
%:%642=338%:%
%:%643=339%:%
%:%644=340%:%
%:%645=341%:%
%:%646=342%:%
%:%647=343%:%
%:%648=344%:%
%:%649=345%:%
%:%650=346%:%
%:%651=347%:%
%:%652=348%:%
%:%654=350%:%
%:%655=350%:%
%:%656=351%:%
%:%657=352%:%
%:%664=353%:%
%:%665=353%:%
%:%666=354%:%
%:%667=354%:%
%:%668=355%:%
%:%669=355%:%
%:%670=355%:%
%:%671=356%:%
%:%672=356%:%
%:%673=356%:%
%:%674=357%:%
%:%675=357%:%
%:%676=358%:%
%:%677=358%:%
%:%678=358%:%
%:%679=359%:%
%:%680=359%:%
%:%681=360%:%
%:%682=360%:%
%:%683=361%:%
%:%684=361%:%
%:%685=362%:%
%:%695=364%:%
%:%697=366%:%
%:%698=366%:%
%:%699=367%:%
%:%702=368%:%
%:%706=368%:%
%:%707=368%:%
%:%708=368%:%
%:%717=370%:%
%:%718=371%:%
%:%719=372%:%
%:%720=373%:%
%:%721=374%:%
%:%722=375%:%
%:%723=376%:%
%:%724=377%:%
%:%725=378%:%
%:%726=379%:%
%:%727=380%:%
%:%728=381%:%
%:%729=382%:%
%:%730=383%:%
%:%731=384%:%
%:%732=385%:%
%:%733=386%:%
%:%734=387%:%
%:%735=388%:%
%:%736=389%:%
%:%737=390%:%
%:%738=391%:%
%:%739=392%:%
%:%740=393%:%
%:%741=394%:%
%:%742=395%:%
%:%743=396%:%
%:%744=397%:%
%:%745=398%:%
%:%746=399%:%
%:%747=400%:%
%:%748=401%:%
%:%749=402%:%
%:%750=403%:%
%:%751=404%:%
%:%753=406%:%
%:%754=406%:%
%:%755=407%:%
%:%756=408%:%
%:%757=409%:%
%:%760=410%:%
%:%764=410%:%
%:%765=410%:%
%:%766=411%:%
%:%767=411%:%
%:%768=412%:%
%:%769=412%:%
%:%770=413%:%
%:%771=413%:%
%:%772=414%:%
%:%773=414%:%
%:%774=415%:%
%:%775=415%:%
%:%776=416%:%
%:%777=416%:%
%:%778=417%:%
%:%779=417%:%
%:%780=417%:%
%:%781=418%:%
%:%782=418%:%
%:%783=419%:%
%:%784=419%:%
%:%785=419%:%
%:%786=420%:%
%:%787=420%:%
%:%788=421%:%
%:%789=421%:%
%:%790=422%:%
%:%791=422%:%
%:%792=423%:%
%:%793=423%:%
%:%794=423%:%
%:%795=424%:%
%:%796=424%:%
%:%797=425%:%
%:%798=425%:%
%:%799=425%:%
%:%800=426%:%
%:%801=426%:%
%:%802=427%:%
%:%803=427%:%
%:%804=427%:%
%:%805=428%:%
%:%806=428%:%
%:%807=429%:%
%:%808=429%:%
%:%809=429%:%
%:%810=430%:%
%:%811=430%:%
%:%812=431%:%
%:%813=431%:%
%:%814=432%:%
%:%815=432%:%
%:%816=432%:%
%:%817=433%:%
%:%818=433%:%
%:%819=434%:%
%:%820=434%:%
%:%821=434%:%
%:%822=435%:%
%:%823=435%:%
%:%824=435%:%
%:%825=436%:%
%:%826=436%:%
%:%827=437%:%
%:%828=437%:%
%:%829=438%:%
%:%830=438%:%
%:%831=438%:%
%:%832=439%:%
%:%833=439%:%
%:%834=440%:%
%:%835=440%:%
%:%836=441%:%
%:%837=441%:%
%:%838=441%:%
%:%839=442%:%
%:%840=442%:%
%:%841=443%:%
%:%842=443%:%
%:%843=444%:%
%:%844=444%:%
%:%845=445%:%
%:%846=445%:%
%:%847=445%:%
%:%848=446%:%
%:%849=446%:%
%:%850=447%:%
%:%851=447%:%
%:%852=448%:%
%:%853=448%:%
%:%854=448%:%
%:%855=449%:%
%:%856=449%:%
%:%857=450%:%
%:%858=450%:%
%:%859=450%:%
%:%860=451%:%
%:%861=451%:%
%:%862=451%:%
%:%863=452%:%
%:%864=452%:%
%:%865=452%:%
%:%866=453%:%
%:%867=453%:%
%:%868=454%:%
%:%869=454%:%
%:%870=455%:%
%:%880=457%:%
%:%882=459%:%
%:%883=459%:%
%:%884=460%:%
%:%885=461%:%
%:%886=462%:%
%:%889=463%:%
%:%893=463%:%
%:%894=463%:%
%:%895=464%:%
%:%896=464%:%
%:%897=465%:%
%:%898=465%:%
%:%899=466%:%
%:%900=466%:%
%:%901=466%:%
%:%902=467%:%
%:%903=467%:%
%:%904=467%:%
%:%905=467%:%
%:%906=468%:%
%:%907=468%:%
%:%908=468%:%
%:%909=469%:%
%:%910=469%:%
%:%911=470%:%
%:%912=470%:%
%:%913=470%:%
%:%914=471%:%
%:%915=471%:%
%:%916=472%:%
%:%917=472%:%
%:%918=472%:%
%:%919=473%:%
%:%920=473%:%
%:%934=475%:%
%:%946=477%:%
%:%947=478%:%
%:%948=479%:%
%:%949=480%:%
%:%950=481%:%
%:%951=482%:%
%:%952=483%:%
%:%953=484%:%
%:%954=485%:%
%:%954=486%:%
%:%955=487%:%
%:%956=488%:%
%:%957=489%:%
%:%961=491%:%
%:%962=492%:%
%:%963=493%:%
%:%964=494%:%
%:%965=495%:%
%:%966=496%:%
%:%967=497%:%
%:%968=498%:%
%:%969=499%:%
%:%970=500%:%
%:%971=501%:%
%:%972=502%:%
%:%973=503%:%
%:%974=504%:%
%:%975=505%:%
%:%976=506%:%
%:%977=507%:%
%:%978=508%:%
%:%979=509%:%
%:%980=510%:%
%:%981=511%:%
%:%982=512%:%
%:%983=513%:%
%:%984=514%:%
%:%985=515%:%
%:%986=516%:%
%:%987=517%:%
%:%989=519%:%
%:%990=519%:%
%:%991=520%:%
%:%992=521%:%
%:%993=522%:%
%:%994=523%:%
%:%995=524%:%
%:%1002=525%:%
%:%1003=525%:%
%:%1004=526%:%
%:%1005=526%:%
%:%1006=527%:%
%:%1007=527%:%
%:%1008=527%:%
%:%1009=527%:%
%:%1010=528%:%
%:%1011=528%:%
%:%1012=528%:%
%:%1013=529%:%
%:%1014=529%:%
%:%1015=530%:%
%:%1016=530%:%
%:%1017=531%:%
%:%1018=531%:%
%:%1019=531%:%
%:%1020=531%:%
%:%1021=532%:%
%:%1022=532%:%
%:%1023=533%:%
%:%1024=533%:%
%:%1025=534%:%
%:%1026=534%:%
%:%1027=535%:%
%:%1028=535%:%
%:%1029=536%:%
%:%1030=536%:%
%:%1031=537%:%
%:%1032=537%:%
%:%1033=538%:%
%:%1034=538%:%
%:%1035=539%:%
%:%1036=539%:%
%:%1037=539%:%
%:%1038=540%:%
%:%1039=540%:%
%:%1040=540%:%
%:%1041=541%:%
%:%1042=541%:%
%:%1043=542%:%
%:%1044=542%:%
%:%1045=542%:%
%:%1046=543%:%
%:%1047=543%:%
%:%1048=544%:%
%:%1049=544%:%
%:%1050=544%:%
%:%1051=545%:%
%:%1052=545%:%
%:%1053=546%:%
%:%1054=546%:%
%:%1055=546%:%
%:%1056=547%:%
%:%1057=547%:%
%:%1058=548%:%
%:%1059=548%:%
%:%1060=548%:%
%:%1061=549%:%
%:%1062=549%:%
%:%1063=550%:%
%:%1064=550%:%
%:%1065=551%:%
%:%1066=551%:%
%:%1067=552%:%
%:%1068=552%:%
%:%1069=552%:%
%:%1070=553%:%
%:%1080=555%:%
%:%1082=557%:%
%:%1083=557%:%
%:%1084=558%:%
%:%1085=559%:%
%:%1086=560%:%
%:%1087=561%:%
%:%1088=562%:%
%:%1095=563%:%
%:%1096=563%:%
%:%1097=564%:%
%:%1098=564%:%
%:%1099=564%:%
%:%1100=564%:%
%:%1101=565%:%
%:%1102=565%:%
%:%1103=565%:%
%:%1104=565%:%
%:%1105=566%:%
%:%1106=566%:%
%:%1107=567%:%
%:%1108=567%:%
%:%1109=568%:%
%:%1110=568%:%
%:%1111=569%:%
%:%1112=569%:%
%:%1113=570%:%
%:%1114=570%:%
%:%1115=570%:%
%:%1116=571%:%
%:%1117=571%:%
%:%1118=572%:%
%:%1119=572%:%
%:%1120=573%:%
%:%1121=573%:%
%:%1122=574%:%
%:%1123=574%:%
%:%1124=575%:%
%:%1125=575%:%
%:%1126=575%:%
%:%1127=576%:%
%:%1128=576%:%
%:%1129=576%:%
%:%1130=576%:%
%:%1131=577%:%
%:%1132=577%:%
%:%1133=577%:%
%:%1134=578%:%
%:%1135=578%:%
%:%1136=579%:%
%:%1137=579%:%
%:%1138=579%:%
%:%1139=580%:%
%:%1140=580%:%
%:%1141=581%:%
%:%1142=581%:%
%:%1143=581%:%
%:%1144=582%:%
%:%1145=582%:%
%:%1146=583%:%
%:%1147=583%:%
%:%1148=584%:%
%:%1149=584%:%
%:%1150=584%:%
%:%1151=584%:%
%:%1152=585%:%
%:%1153=585%:%
%:%1154=586%:%
%:%1155=586%:%
%:%1156=586%:%
%:%1157=586%:%
%:%1158=586%:%
%:%1159=587%:%
%:%1169=589%:%
%:%1170=590%:%
%:%1171=591%:%
%:%1172=592%:%
%:%1173=593%:%
%:%1174=594%:%
%:%1175=595%:%
%:%1176=596%:%
%:%1177=597%:%
%:%1178=598%:%
%:%1179=599%:%
%:%1180=600%:%
%:%1181=601%:%
%:%1182=602%:%
%:%1183=603%:%
%:%1184=604%:%
%:%1185=605%:%
%:%1186=606%:%
%:%1187=607%:%
%:%1188=608%:%
%:%1189=609%:%
%:%1190=610%:%
%:%1191=611%:%
%:%1192=612%:%
%:%1193=613%:%
%:%1194=614%:%
%:%1195=615%:%
%:%1196=616%:%
%:%1197=617%:%
%:%1198=618%:%
%:%1199=619%:%
%:%1200=620%:%
%:%1201=621%:%
%:%1202=622%:%
%:%1204=624%:%
%:%1205=624%:%
%:%1206=625%:%
%:%1207=626%:%
%:%1208=627%:%
%:%1209=628%:%
%:%1210=629%:%
%:%1217=630%:%
%:%1218=630%:%
%:%1219=631%:%
%:%1220=631%:%
%:%1221=632%:%
%:%1222=632%:%
%:%1223=633%:%
%:%1224=633%:%
%:%1225=633%:%
%:%1226=634%:%
%:%1227=634%:%
%:%1228=635%:%
%:%1229=635%:%
%:%1230=636%:%
%:%1231=636%:%
%:%1232=636%:%
%:%1233=637%:%
%:%1234=637%:%
%:%1235=637%:%
%:%1236=638%:%
%:%1237=638%:%
%:%1238=638%:%
%:%1239=639%:%
%:%1240=639%:%
%:%1241=640%:%
%:%1242=640%:%
%:%1243=640%:%
%:%1244=641%:%
%:%1245=641%:%
%:%1246=642%:%
%:%1247=642%:%
%:%1248=643%:%
%:%1249=643%:%
%:%1250=643%:%
%:%1251=644%:%
%:%1252=644%:%
%:%1253=645%:%
%:%1254=645%:%
%:%1255=645%:%
%:%1256=646%:%
%:%1257=646%:%
%:%1258=647%:%
%:%1259=647%:%
%:%1260=647%:%
%:%1261=648%:%
%:%1262=648%:%
%:%1263=649%:%
%:%1264=649%:%
%:%1265=650%:%
%:%1266=650%:%
%:%1267=650%:%
%:%1268=651%:%
%:%1269=651%:%
%:%1270=652%:%
%:%1271=652%:%
%:%1272=652%:%
%:%1273=653%:%
%:%1274=653%:%
%:%1275=653%:%
%:%1276=654%:%
%:%1277=654%:%
%:%1278=655%:%
%:%1279=655%:%
%:%1280=656%:%
%:%1281=656%:%
%:%1282=657%:%
%:%1283=657%:%
%:%1284=657%:%
%:%1285=658%:%
%:%1286=658%:%
%:%1287=659%:%
%:%1288=659%:%
%:%1289=659%:%
%:%1290=660%:%
%:%1291=660%:%
%:%1292=661%:%
%:%1293=661%:%
%:%1294=662%:%
%:%1295=663%:%
%:%1296=663%:%
%:%1297=664%:%
%:%1298=664%:%
%:%1299=664%:%
%:%1300=665%:%
%:%1301=666%:%
%:%1302=666%:%
%:%1303=667%:%
%:%1304=667%:%
%:%1305=667%:%
%:%1306=668%:%
%:%1307=668%:%
%:%1308=668%:%
%:%1309=669%:%
%:%1310=669%:%
%:%1311=669%:%
%:%1312=670%:%
%:%1313=670%:%
%:%1314=671%:%
%:%1315=671%:%
%:%1316=671%:%
%:%1317=672%:%
%:%1318=672%:%
%:%1319=673%:%
%:%1320=673%:%
%:%1321=674%:%
%:%1322=674%:%
%:%1323=674%:%
%:%1324=675%:%
%:%1325=675%:%
%:%1326=676%:%
%:%1327=676%:%
%:%1328=677%:%
%:%1329=677%:%
%:%1330=678%:%
%:%1331=678%:%
%:%1332=678%:%
%:%1333=679%:%
%:%1334=679%:%
%:%1335=679%:%
%:%1336=680%:%
%:%1337=680%:%
%:%1338=680%:%
%:%1339=681%:%
%:%1340=681%:%
%:%1341=681%:%
%:%1342=682%:%
%:%1343=682%:%
%:%1344=683%:%
%:%1345=683%:%
%:%1346=683%:%
%:%1347=684%:%
%:%1348=684%:%
%:%1349=684%:%
%:%1350=685%:%
%:%1351=685%:%
%:%1352=685%:%
%:%1353=686%:%
%:%1354=686%:%
%:%1355=686%:%
%:%1356=687%:%
%:%1357=687%:%
%:%1358=687%:%
%:%1359=688%:%
%:%1360=688%:%
%:%1361=689%:%
%:%1362=689%:%
%:%1363=689%:%
%:%1364=690%:%
%:%1374=692%:%
%:%1376=694%:%
%:%1377=694%:%
%:%1378=695%:%
%:%1379=696%:%
%:%1380=697%:%
%:%1381=698%:%
%:%1382=699%:%
%:%1389=700%:%
%:%1390=700%:%
%:%1391=701%:%
%:%1392=701%:%
%:%1393=702%:%
%:%1394=702%:%
%:%1395=702%:%
%:%1396=702%:%
%:%1397=703%:%
%:%1398=703%:%
%:%1399=703%:%
%:%1400=703%:%
%:%1401=704%:%
%:%1402=704%:%
%:%1403=704%:%
%:%1404=704%:%
%:%1405=704%:%
%:%1406=705%:%
%:%1407=705%:%
%:%1408=705%:%
%:%1409=706%:%
%:%1410=706%:%
%:%1411=706%:%
%:%1412=706%:%
%:%1413=707%:%
%:%1414=707%:%
%:%1415=707%:%
%:%1416=708%:%
%:%1417=708%:%
%:%1418=708%:%
%:%1419=708%:%
%:%1420=709%:%
%:%1421=709%:%
%:%1422=709%:%
%:%1423=709%:%
%:%1424=710%:%
%:%1434=712%:%
%:%1435=713%:%
%:%1436=714%:%
%:%1437=715%:%
%:%1438=716%:%
%:%1439=717%:%
%:%1440=718%:%
%:%1441=719%:%
%:%1442=720%:%
%:%1443=721%:%
%:%1444=722%:%
%:%1445=723%:%
%:%1446=724%:%
%:%1447=725%:%
%:%1448=726%:%
%:%1449=727%:%
%:%1450=728%:%
%:%1451=729%:%
%:%1452=730%:%
%:%1453=731%:%
%:%1454=732%:%
%:%1455=733%:%
%:%1457=735%:%
%:%1458=735%:%
%:%1459=736%:%
%:%1460=737%:%
%:%1461=738%:%
%:%1468=739%:%
%:%1469=739%:%
%:%1470=740%:%
%:%1471=740%:%
%:%1472=741%:%
%:%1473=741%:%
%:%1474=742%:%
%:%1475=742%:%
%:%1476=743%:%
%:%1477=743%:%
%:%1478=744%:%
%:%1479=744%:%
%:%1480=744%:%
%:%1481=745%:%
%:%1482=745%:%
%:%1483=745%:%
%:%1484=746%:%
%:%1485=746%:%
%:%1486=747%:%
%:%1487=747%:%
%:%1488=747%:%
%:%1489=748%:%
%:%1490=748%:%
%:%1491=749%:%
%:%1492=749%:%
%:%1493=750%:%
%:%1494=750%:%
%:%1495=751%:%
%:%1505=753%:%
%:%1507=755%:%
%:%1508=755%:%
%:%1509=756%:%
%:%1510=757%:%
%:%1511=758%:%
%:%1514=759%:%
%:%1518=759%:%
%:%1519=759%:%
%:%1520=759%:%
%:%1529=761%:%
%:%1530=762%:%
%:%1531=763%:%
%:%1532=764%:%
%:%1533=765%:%
%:%1534=766%:%
%:%1535=767%:%
%:%1536=768%:%
%:%1537=769%:%
%:%1538=770%:%
%:%1539=771%:%
%:%1540=772%:%
%:%1541=773%:%
%:%1542=774%:%
%:%1543=775%:%
%:%1544=776%:%
%:%1545=777%:%
%:%1546=778%:%
%:%1547=779%:%
%:%1548=780%:%
%:%1549=781%:%
%:%1550=782%:%
%:%1551=783%:%
%:%1552=784%:%
%:%1553=785%:%
%:%1554=786%:%
%:%1555=787%:%
%:%1556=788%:%
%:%1557=789%:%
%:%1558=790%:%
%:%1559=791%:%
%:%1560=792%:%
%:%1561=793%:%
%:%1562=794%:%
%:%1563=795%:%
%:%1564=796%:%
%:%1565=797%:%
%:%1566=798%:%
%:%1567=799%:%
%:%1568=800%:%
%:%1569=801%:%
%:%1570=802%:%
%:%1571=803%:%
%:%1572=804%:%
%:%1573=805%:%
%:%1574=806%:%
%:%1575=807%:%
%:%1576=808%:%
%:%1577=809%:%
%:%1578=810%:%
%:%1579=811%:%
%:%1580=812%:%
%:%1581=813%:%
%:%1582=814%:%
%:%1583=815%:%
%:%1584=816%:%
%:%1585=817%:%
%:%1586=818%:%
%:%1587=819%:%
%:%1588=820%:%
%:%1589=821%:%
%:%1590=822%:%
%:%1591=823%:%
%:%1592=824%:%
%:%1593=825%:%
%:%1594=826%:%
%:%1595=827%:%
%:%1596=828%:%
%:%1597=829%:%
%:%1598=830%:%
%:%1599=831%:%
%:%1600=832%:%
%:%1602=834%:%
%:%1603=834%:%
%:%1604=835%:%
%:%1605=836%:%
%:%1606=837%:%
%:%1607=838%:%
%:%1608=839%:%
%:%1615=840%:%
%:%1616=840%:%
%:%1617=841%:%
%:%1618=841%:%
%:%1619=842%:%
%:%1620=842%:%
%:%1621=843%:%
%:%1622=843%:%
%:%1623=843%:%
%:%1624=844%:%
%:%1625=844%:%
%:%1629=848%:%
%:%1630=849%:%
%:%1631=849%:%
%:%1632=849%:%
%:%1633=850%:%
%:%1634=850%:%
%:%1635=850%:%
%:%1638=853%:%
%:%1639=854%:%
%:%1640=854%:%
%:%1641=854%:%
%:%1642=855%:%
%:%1643=855%:%
%:%1644=855%:%
%:%1645=856%:%
%:%1646=856%:%
%:%1647=857%:%
%:%1648=857%:%
%:%1649=858%:%
%:%1650=858%:%
%:%1651=858%:%
%:%1652=859%:%
%:%1653=859%:%
%:%1654=860%:%
%:%1655=860%:%
%:%1656=860%:%
%:%1657=861%:%
%:%1658=861%:%
%:%1659=862%:%
%:%1660=862%:%
%:%1661=863%:%
%:%1662=863%:%
%:%1663=864%:%
%:%1664=864%:%
%:%1665=865%:%
%:%1666=865%:%
%:%1667=866%:%
%:%1668=866%:%
%:%1669=867%:%
%:%1670=867%:%
%:%1671=868%:%
%:%1672=868%:%
%:%1673=869%:%
%:%1674=869%:%
%:%1675=869%:%
%:%1676=870%:%
%:%1677=870%:%
%:%1678=870%:%
%:%1679=871%:%
%:%1680=871%:%
%:%1681=871%:%
%:%1682=872%:%
%:%1683=872%:%
%:%1684=872%:%
%:%1685=873%:%
%:%1686=873%:%
%:%1687=873%:%
%:%1688=874%:%
%:%1689=874%:%
%:%1690=875%:%
%:%1691=875%:%
%:%1692=876%:%
%:%1693=876%:%
%:%1694=876%:%
%:%1695=877%:%
%:%1696=877%:%
%:%1697=877%:%
%:%1698=878%:%
%:%1699=878%:%
%:%1700=879%:%
%:%1701=879%:%
%:%1702=880%:%
%:%1703=880%:%
%:%1704=880%:%
%:%1705=881%:%
%:%1706=881%:%
%:%1707=882%:%
%:%1708=882%:%
%:%1709=883%:%
%:%1710=883%:%
%:%1711=883%:%
%:%1712=884%:%
%:%1713=884%:%
%:%1714=884%:%
%:%1715=885%:%
%:%1716=885%:%
%:%1717=886%:%
%:%1718=886%:%
%:%1719=886%:%
%:%1720=887%:%
%:%1721=887%:%
%:%1722=888%:%
%:%1723=888%:%
%:%1724=888%:%
%:%1725=889%:%
%:%1726=889%:%
%:%1727=890%:%
%:%1728=890%:%
%:%1729=891%:%
%:%1730=891%:%
%:%1731=891%:%
%:%1732=892%:%
%:%1733=892%:%
%:%1734=892%:%
%:%1735=893%:%
%:%1736=893%:%
%:%1737=894%:%
%:%1738=894%:%
%:%1739=894%:%
%:%1740=895%:%
%:%1741=895%:%
%:%1742=896%:%
%:%1743=896%:%
%:%1744=897%:%
%:%1745=897%:%
%:%1746=897%:%
%:%1747=898%:%
%:%1748=898%:%
%:%1749=899%:%
%:%1750=899%:%
%:%1751=900%:%
%:%1752=900%:%
%:%1753=901%:%
%:%1754=901%:%
%:%1755=901%:%
%:%1756=902%:%
%:%1757=902%:%
%:%1758=903%:%
%:%1759=903%:%
%:%1760=904%:%
%:%1761=904%:%
%:%1762=905%:%
%:%1763=905%:%
%:%1764=906%:%
%:%1765=906%:%
%:%1766=907%:%
%:%1767=907%:%
%:%1768=908%:%
%:%1769=908%:%
%:%1770=909%:%
%:%1771=909%:%
%:%1772=910%:%
%:%1773=910%:%
%:%1774=910%:%
%:%1775=911%:%
%:%1776=911%:%
%:%1777=911%:%
%:%1778=912%:%
%:%1779=912%:%
%:%1780=912%:%
%:%1781=913%:%
%:%1782=913%:%
%:%1783=913%:%
%:%1784=914%:%
%:%1785=914%:%
%:%1786=914%:%
%:%1787=915%:%
%:%1788=915%:%
%:%1789=916%:%
%:%1790=916%:%
%:%1791=917%:%
%:%1792=917%:%
%:%1793=918%:%
%:%1794=918%:%
%:%1795=919%:%
%:%1796=919%:%
%:%1797=920%:%
%:%1798=920%:%
%:%1799=921%:%
%:%1800=921%:%
%:%1801=921%:%
%:%1802=922%:%
%:%1803=922%:%
%:%1804=923%:%
%:%1805=923%:%
%:%1806=923%:%
%:%1807=924%:%
%:%1808=924%:%
%:%1809=924%:%
%:%1810=925%:%
%:%1811=925%:%
%:%1812=926%:%
%:%1813=926%:%
%:%1814=927%:%
%:%1815=927%:%
%:%1816=928%:%
%:%1817=928%:%
%:%1818=929%:%
%:%1819=929%:%
%:%1820=930%:%
%:%1821=930%:%
%:%1822=931%:%
%:%1823=931%:%
%:%1824=932%:%
%:%1825=932%:%
%:%1826=933%:%
%:%1827=933%:%
%:%1828=933%:%
%:%1829=934%:%
%:%1830=934%:%
%:%1831=935%:%
%:%1832=935%:%
%:%1833=935%:%
%:%1834=936%:%
%:%1835=936%:%
%:%1836=936%:%
%:%1837=937%:%
%:%1838=937%:%
%:%1839=938%:%
%:%1840=938%:%
%:%1841=939%:%
%:%1842=939%:%
%:%1843=940%:%
%:%1844=940%:%
%:%1845=941%:%
%:%1846=941%:%
%:%1847=942%:%
%:%1848=942%:%
%:%1849=943%:%
%:%1850=943%:%
%:%1853=946%:%
%:%1854=947%:%
%:%1855=947%:%
%:%1856=947%:%
%:%1857=948%:%
%:%1867=950%:%
%:%1869=952%:%
%:%1870=952%:%
%:%1871=953%:%
%:%1872=954%:%
%:%1873=955%:%
%:%1874=956%:%
%:%1875=957%:%
%:%1882=958%:%
%:%1883=958%:%
%:%1884=959%:%
%:%1885=959%:%
%:%1886=960%:%
%:%1887=960%:%
%:%1888=960%:%
%:%1889=960%:%
%:%1890=961%:%
%:%1891=961%:%
%:%1892=962%:%
%:%1893=962%:%
%:%1894=963%:%
%:%1895=964%:%
%:%1896=965%:%
%:%1897=966%:%
%:%1898=966%:%
%:%1899=967%:%
%:%1900=967%:%
%:%1901=967%:%
%:%1902=967%:%
%:%1903=968%:%
%:%1904=968%:%
%:%1905=968%:%
%:%1906=969%:%
%:%1916=971%:%
%:%1917=972%:%
%:%1918=973%:%
%:%1919=974%:%
%:%1920=975%:%
%:%1921=976%:%
%:%1922=977%:%
%:%1923=978%:%
%:%1924=979%:%
%:%1925=980%:%
%:%1926=981%:%
%:%1927=982%:%
%:%1928=983%:%
%:%1929=984%:%
%:%1930=985%:%
%:%1931=986%:%
%:%1933=988%:%
%:%1934=988%:%
%:%1935=989%:%
%:%1936=990%:%
%:%1937=991%:%
%:%1940=992%:%
%:%1944=992%:%
%:%1945=992%:%
%:%1946=993%:%
%:%1947=993%:%
%:%1948=994%:%
%:%1949=994%:%
%:%1950=995%:%
%:%1951=995%:%
%:%1952=995%:%
%:%1953=996%:%
%:%1954=996%:%
%:%1955=997%:%
%:%1956=997%:%
%:%1957=997%:%
%:%1958=998%:%
%:%1959=998%:%
%:%1960=999%:%
%:%1961=999%:%
%:%1962=1000%:%
%:%1963=1000%:%
%:%1964=1001%:%
%:%1965=1001%:%
%:%1966=1002%:%
%:%1967=1002%:%
%:%1968=1003%:%
%:%1969=1003%:%
%:%1970=1003%:%
%:%1971=1004%:%
%:%1972=1004%:%
%:%1973=1004%:%
%:%1974=1005%:%
%:%1975=1005%:%
%:%1976=1005%:%
%:%1977=1006%:%
%:%1978=1006%:%
%:%1979=1007%:%
%:%1980=1007%:%
%:%1981=1007%:%
%:%1982=1008%:%
%:%1983=1008%:%
%:%1984=1009%:%
%:%1985=1009%:%
%:%1986=1010%:%
%:%1987=1010%:%
%:%1988=1011%:%
%:%1998=1013%:%
%:%1999=1014%:%
%:%2000=1015%:%
%:%2001=1016%:%
%:%2002=1017%:%
%:%2003=1018%:%
%:%2004=1019%:%
%:%2005=1020%:%
%:%2006=1021%:%
%:%2007=1022%:%
%:%2008=1023%:%
%:%2009=1024%:%
%:%2010=1025%:%
%:%2011=1026%:%
%:%2012=1027%:%
%:%2013=1028%:%
%:%2014=1029%:%
%:%2015=1030%:%
%:%2016=1031%:%
%:%2017=1032%:%
%:%2018=1033%:%
%:%2019=1034%:%
%:%2020=1035%:%
%:%2021=1036%:%
%:%2022=1037%:%
%:%2023=1038%:%
%:%2024=1039%:%
%:%2025=1040%:%
%:%2026=1041%:%
%:%2027=1042%:%
%:%2028=1043%:%
%:%2029=1044%:%
%:%2030=1045%:%
%:%2031=1046%:%
%:%2032=1047%:%
%:%2033=1048%:%
%:%2035=1050%:%
%:%2036=1050%:%
%:%2037=1051%:%
%:%2038=1052%:%
%:%2039=1053%:%
%:%2040=1054%:%
%:%2047=1055%:%
%:%2048=1055%:%
%:%2049=1056%:%
%:%2050=1056%:%
%:%2051=1057%:%
%:%2052=1057%:%
%:%2053=1058%:%
%:%2054=1058%:%
%:%2055=1058%:%
%:%2056=1059%:%
%:%2057=1059%:%
%:%2058=1060%:%
%:%2059=1060%:%
%:%2060=1061%:%
%:%2061=1061%:%
%:%2062=1061%:%
%:%2063=1062%:%
%:%2064=1062%:%
%:%2065=1063%:%
%:%2066=1063%:%
%:%2067=1063%:%
%:%2068=1064%:%
%:%2069=1064%:%
%:%2070=1065%:%
%:%2071=1065%:%
%:%2072=1065%:%
%:%2073=1066%:%
%:%2074=1066%:%
%:%2075=1067%:%
%:%2076=1067%:%
%:%2077=1067%:%
%:%2078=1068%:%
%:%2079=1068%:%
%:%2080=1069%:%
%:%2081=1069%:%
%:%2082=1069%:%
%:%2083=1070%:%
%:%2084=1070%:%
%:%2085=1071%:%
%:%2086=1071%:%
%:%2087=1071%:%
%:%2088=1072%:%
%:%2089=1072%:%
%:%2090=1073%:%
%:%2091=1073%:%
%:%2092=1073%:%
%:%2093=1074%:%
%:%2094=1074%:%
%:%2095=1074%:%
%:%2096=1075%:%
%:%2097=1075%:%
%:%2098=1075%:%
%:%2099=1076%:%
%:%2100=1076%:%
%:%2101=1077%:%
%:%2102=1077%:%
%:%2103=1077%:%
%:%2104=1078%:%
%:%2105=1078%:%
%:%2106=1079%:%
%:%2107=1079%:%
%:%2108=1079%:%
%:%2109=1080%:%
%:%2110=1080%:%
%:%2111=1080%:%
%:%2112=1081%:%
%:%2113=1081%:%
%:%2114=1082%:%
%:%2115=1082%:%
%:%2116=1083%:%
%:%2117=1083%:%
%:%2118=1083%:%
%:%2119=1084%:%
%:%2120=1084%:%
%:%2121=1084%:%
%:%2122=1085%:%
%:%2123=1085%:%
%:%2124=1085%:%
%:%2125=1086%:%
%:%2126=1086%:%
%:%2127=1087%:%
%:%2128=1087%:%
%:%2129=1087%:%
%:%2130=1088%:%
%:%2131=1088%:%
%:%2132=1088%:%
%:%2133=1089%:%
%:%2134=1089%:%
%:%2135=1090%:%
%:%2136=1090%:%
%:%2137=1091%:%
%:%2138=1091%:%
%:%2139=1092%:%
%:%2140=1092%:%
%:%2141=1093%:%
%:%2142=1093%:%
%:%2143=1094%:%
%:%2144=1094%:%
%:%2145=1094%:%
%:%2146=1095%:%
%:%2147=1095%:%
%:%2148=1096%:%
%:%2149=1096%:%
%:%2150=1097%:%
%:%2151=1097%:%
%:%2152=1097%:%
%:%2153=1098%:%
%:%2163=1100%:%
%:%2165=1102%:%
%:%2166=1102%:%
%:%2167=1103%:%
%:%2168=1104%:%
%:%2169=1105%:%
%:%2170=1106%:%
%:%2177=1107%:%
%:%2178=1107%:%
%:%2179=1108%:%
%:%2180=1108%:%
%:%2181=1108%:%
%:%2182=1109%:%
%:%2183=1110%:%
%:%2184=1110%:%
%:%2185=1111%:%
%:%2186=1111%:%
%:%2187=1112%:%
%:%2188=1112%:%
%:%2189=1112%:%
%:%2190=1112%:%
%:%2191=1113%:%
%:%2192=1113%:%
%:%2193=1114%:%
%:%2194=1114%:%
%:%2195=1114%:%
%:%2196=1115%:%
%:%2197=1115%:%
%:%2198=1115%:%
%:%2199=1115%:%
%:%2200=1116%:%
%:%2201=1116%:%
%:%2202=1116%:%
%:%2203=1117%:%
%:%2204=1117%:%
%:%2205=1117%:%
%:%2206=1118%:%
%:%2207=1118%:%
%:%2208=1118%:%
%:%2209=1118%:%
%:%2210=1118%:%
%:%2211=1119%:%
%:%2226=1121%:%
%:%2238=1123%:%
%:%2239=1124%:%
%:%2240=1125%:%
%:%2241=1126%:%
%:%2242=1127%:%
%:%2243=1128%:%
%:%2244=1129%:%
%:%2245=1130%:%
%:%2246=1131%:%
%:%2247=1132%:%
%:%2248=1133%:%
%:%2249=1134%:%
%:%2250=1135%:%
%:%2251=1136%:%
%:%2252=1137%:%
%:%2253=1138%:%
%:%2254=1139%:%
%:%2255=1140%:%
%:%2256=1141%:%
%:%2257=1142%:%
%:%2258=1143%:%
%:%2259=1144%:%
%:%2260=1145%:%
%:%2261=1146%:%
%:%2262=1147%:%
%:%2263=1148%:%
%:%2264=1149%:%
%:%2265=1150%:%
%:%2266=1151%:%
%:%2267=1152%:%
%:%2268=1153%:%
%:%2269=1154%:%
%:%2270=1155%:%
%:%2271=1156%:%
%:%2272=1157%:%
%:%2273=1158%:%
%:%2274=1159%:%
%:%2275=1160%:%
%:%2276=1161%:%
%:%2277=1162%:%
%:%2278=1163%:%
%:%2279=1164%:%
%:%2280=1165%:%
%:%2281=1166%:%
%:%2282=1167%:%
%:%2283=1168%:%
%:%2284=1169%:%
%:%2285=1170%:%
%:%2286=1171%:%
%:%2287=1172%:%
%:%2288=1173%:%
%:%2289=1174%:%
%:%2290=1175%:%
%:%2291=1176%:%
%:%2292=1177%:%
%:%2293=1178%:%
%:%2294=1179%:%
%:%2295=1180%:%
%:%2296=1181%:%
%:%2297=1182%:%
%:%2298=1183%:%
%:%2300=1185%:%
%:%2301=1185%:%
%:%2304=1186%:%
%:%2308=1186%:%
%:%2309=1186%:%
%:%2318=1188%:%
%:%2320=1190%:%
%:%2321=1190%:%
%:%2322=1191%:%
%:%2323=1192%:%
%:%2330=1193%:%
%:%2331=1193%:%
%:%2332=1194%:%
%:%2333=1194%:%
%:%2334=1195%:%
%:%2335=1195%:%
%:%2336=1196%:%
%:%2337=1196%:%
%:%2338=1197%:%
%:%2339=1197%:%
%:%2340=1198%:%
%:%2341=1198%:%
%:%2342=1199%:%
%:%2343=1199%:%
%:%2344=1200%:%
%:%2345=1200%:%
%:%2346=1201%:%
%:%2347=1201%:%
%:%2348=1201%:%
%:%2349=1202%:%
%:%2350=1202%:%
%:%2351=1203%:%
%:%2352=1203%:%
%:%2353=1204%:%
%:%2354=1204%:%
%:%2355=1204%:%
%:%2356=1205%:%
%:%2357=1205%:%
%:%2358=1206%:%
%:%2359=1206%:%
%:%2360=1206%:%
%:%2361=1207%:%
%:%2362=1207%:%
%:%2363=1208%:%
%:%2364=1208%:%
%:%2365=1209%:%
%:%2366=1209%:%
%:%2367=1210%:%
%:%2368=1210%:%
%:%2369=1211%:%
%:%2370=1211%:%
%:%2371=1212%:%
%:%2372=1212%:%
%:%2373=1213%:%
%:%2374=1213%:%
%:%2375=1214%:%
%:%2376=1214%:%
%:%2377=1215%:%
%:%2378=1215%:%
%:%2379=1215%:%
%:%2380=1216%:%
%:%2381=1216%:%
%:%2382=1217%:%
%:%2383=1217%:%
%:%2384=1217%:%
%:%2385=1218%:%
%:%2386=1218%:%
%:%2387=1219%:%
%:%2388=1219%:%
%:%2389=1220%:%
%:%2390=1220%:%
%:%2391=1220%:%
%:%2392=1221%:%
%:%2393=1221%:%
%:%2394=1222%:%
%:%2395=1222%:%
%:%2396=1222%:%
%:%2397=1223%:%
%:%2398=1223%:%
%:%2399=1224%:%
%:%2400=1224%:%
%:%2401=1224%:%
%:%2402=1225%:%
%:%2403=1225%:%
%:%2404=1225%:%
%:%2405=1226%:%
%:%2406=1226%:%
%:%2407=1227%:%
%:%2408=1227%:%
%:%2409=1228%:%
%:%2410=1228%:%
%:%2411=1228%:%
%:%2412=1229%:%
%:%2413=1229%:%
%:%2414=1229%:%
%:%2415=1230%:%
%:%2416=1230%:%
%:%2417=1231%:%
%:%2418=1231%:%
%:%2419=1231%:%
%:%2420=1232%:%
%:%2421=1232%:%
%:%2422=1233%:%
%:%2423=1233%:%
%:%2424=1234%:%
%:%2425=1234%:%
%:%2426=1235%:%
%:%2427=1235%:%
%:%2428=1236%:%
%:%2429=1236%:%
%:%2430=1237%:%
%:%2431=1237%:%
%:%2432=1237%:%
%:%2433=1238%:%
%:%2434=1238%:%
%:%2435=1239%:%
%:%2436=1239%:%
%:%2437=1240%:%
%:%2438=1240%:%
%:%2439=1241%:%
%:%2440=1241%:%
%:%2441=1242%:%
%:%2442=1242%:%
%:%2443=1242%:%
%:%2444=1243%:%
%:%2445=1243%:%
%:%2446=1243%:%
%:%2447=1244%:%
%:%2448=1244%:%
%:%2449=1245%:%
%:%2450=1245%:%
%:%2451=1245%:%
%:%2452=1246%:%
%:%2453=1246%:%
%:%2454=1247%:%
%:%2455=1247%:%
%:%2456=1248%:%
%:%2457=1248%:%
%:%2458=1249%:%
%:%2459=1249%:%
%:%2460=1249%:%
%:%2461=1250%:%
%:%2462=1250%:%
%:%2463=1251%:%
%:%2464=1251%:%
%:%2465=1251%:%
%:%2466=1252%:%
%:%2467=1252%:%
%:%2468=1253%:%
%:%2469=1253%:%
%:%2470=1254%:%
%:%2471=1254%:%
%:%2472=1254%:%
%:%2473=1255%:%
%:%2474=1255%:%
%:%2475=1256%:%
%:%2476=1256%:%
%:%2477=1256%:%
%:%2478=1257%:%
%:%2479=1257%:%
%:%2480=1258%:%
%:%2481=1258%:%
%:%2482=1259%:%
%:%2483=1259%:%
%:%2484=1260%:%
%:%2485=1260%:%
%:%2486=1261%:%
%:%2487=1261%:%
%:%2488=1261%:%
%:%2489=1262%:%
%:%2490=1262%:%
%:%2491=1263%:%
%:%2492=1263%:%
%:%2493=1263%:%
%:%2494=1264%:%
%:%2495=1264%:%
%:%2496=1265%:%
%:%2497=1265%:%
%:%2498=1266%:%
%:%2499=1266%:%
%:%2500=1266%:%
%:%2501=1267%:%
%:%2502=1267%:%
%:%2503=1268%:%
%:%2504=1268%:%
%:%2505=1268%:%
%:%2506=1269%:%
%:%2507=1269%:%
%:%2508=1270%:%
%:%2509=1270%:%
%:%2510=1271%:%
%:%2511=1271%:%
%:%2512=1272%:%
%:%2513=1272%:%
%:%2514=1273%:%
%:%2515=1273%:%
%:%2516=1273%:%
%:%2517=1274%:%
%:%2518=1274%:%
%:%2519=1274%:%
%:%2520=1275%:%
%:%2521=1275%:%
%:%2522=1276%:%
%:%2523=1276%:%
%:%2524=1277%:%
%:%2525=1277%:%
%:%2526=1277%:%
%:%2527=1278%:%
%:%2528=1278%:%
%:%2529=1279%:%
%:%2530=1279%:%
%:%2531=1279%:%
%:%2532=1280%:%
%:%2533=1280%:%
%:%2534=1281%:%
%:%2535=1281%:%
%:%2536=1282%:%
%:%2537=1282%:%
%:%2538=1283%:%
%:%2539=1283%:%
%:%2540=1284%:%
%:%2541=1284%:%
%:%2542=1285%:%
%:%2543=1285%:%
%:%2544=1285%:%
%:%2545=1286%:%
%:%2546=1286%:%
%:%2547=1287%:%
%:%2548=1287%:%
%:%2549=1287%:%
%:%2550=1288%:%
%:%2551=1288%:%
%:%2552=1289%:%
%:%2553=1289%:%
%:%2554=1290%:%
%:%2555=1290%:%
%:%2556=1290%:%
%:%2557=1291%:%
%:%2558=1291%:%
%:%2559=1292%:%
%:%2560=1292%:%
%:%2561=1292%:%
%:%2562=1293%:%
%:%2563=1293%:%
%:%2564=1294%:%
%:%2565=1294%:%
%:%2566=1295%:%
%:%2567=1295%:%
%:%2568=1296%:%
%:%2569=1296%:%
%:%2570=1297%:%
%:%2571=1297%:%
%:%2572=1297%:%
%:%2573=1298%:%
%:%2574=1298%:%
%:%2575=1299%:%
%:%2576=1299%:%
%:%2577=1300%:%
%:%2578=1300%:%
%:%2579=1300%:%
%:%2580=1301%:%
%:%2581=1301%:%
%:%2582=1302%:%
%:%2583=1302%:%
%:%2584=1302%:%
%:%2585=1303%:%
%:%2586=1303%:%
%:%2587=1304%:%
%:%2588=1304%:%
%:%2589=1305%:%
%:%2590=1305%:%
%:%2591=1306%:%
%:%2601=1308%:%
%:%2602=1309%:%
%:%2603=1310%:%
%:%2604=1311%:%
%:%2605=1312%:%
%:%2606=1313%:%
%:%2607=1314%:%
%:%2608=1315%:%
%:%2609=1316%:%
%:%2610=1317%:%
%:%2611=1318%:%
%:%2612=1319%:%
%:%2613=1320%:%
%:%2614=1321%:%
%:%2615=1322%:%
%:%2616=1323%:%
%:%2617=1324%:%
%:%2618=1325%:%
%:%2619=1326%:%
%:%2620=1327%:%
%:%2621=1328%:%
%:%2622=1329%:%
%:%2623=1330%:%
%:%2624=1331%:%
%:%2625=1332%:%
%:%2626=1333%:%
%:%2627=1334%:%
%:%2629=1336%:%
%:%2630=1336%:%
%:%2631=1337%:%
%:%2632=1338%:%
%:%2633=1339%:%
%:%2640=1340%:%
%:%2641=1340%:%
%:%2642=1341%:%
%:%2643=1341%:%
%:%2644=1342%:%
%:%2645=1342%:%
%:%2646=1342%:%
%:%2647=1343%:%
%:%2648=1343%:%
%:%2649=1343%:%
%:%2650=1344%:%
%:%2651=1344%:%
%:%2652=1344%:%
%:%2653=1345%:%
%:%2654=1345%:%
%:%2655=1346%:%
%:%2656=1346%:%
%:%2657=1346%:%
%:%2658=1347%:%
%:%2659=1347%:%
%:%2660=1348%:%
%:%2661=1348%:%
%:%2662=1348%:%
%:%2663=1349%:%
%:%2664=1349%:%
%:%2665=1350%:%
%:%2666=1350%:%
%:%2667=1350%:%
%:%2668=1351%:%
%:%2669=1351%:%
%:%2670=1352%:%
%:%2671=1352%:%
%:%2672=1352%:%
%:%2673=1353%:%
%:%2674=1353%:%
%:%2675=1354%:%
%:%2676=1354%:%
%:%2677=1354%:%
%:%2678=1355%:%
%:%2679=1355%:%
%:%2680=1356%:%
%:%2681=1356%:%
%:%2682=1356%:%
%:%2683=1357%:%
%:%2684=1357%:%
%:%2685=1358%:%
%:%2686=1358%:%
%:%2687=1359%:%
%:%2688=1359%:%
%:%2689=1360%:%
%:%2690=1360%:%
%:%2691=1361%:%
%:%2692=1361%:%
%:%2693=1361%:%
%:%2694=1362%:%
%:%2695=1362%:%
%:%2696=1363%:%
%:%2697=1363%:%
%:%2698=1364%:%
%:%2699=1364%:%
%:%2700=1365%:%
%:%2701=1365%:%
%:%2702=1365%:%
%:%2703=1366%:%
%:%2704=1366%:%
%:%2705=1367%:%
%:%2706=1367%:%
%:%2707=1367%:%
%:%2708=1368%:%
%:%2709=1368%:%
%:%2710=1369%:%
%:%2711=1369%:%
%:%2712=1369%:%
%:%2713=1370%:%
%:%2714=1370%:%
%:%2715=1371%:%
%:%2716=1371%:%
%:%2717=1371%:%
%:%2718=1372%:%
%:%2719=1372%:%
%:%2720=1372%:%
%:%2721=1373%:%
%:%2722=1373%:%
%:%2723=1374%:%
%:%2724=1374%:%
%:%2725=1374%:%
%:%2726=1375%:%
%:%2727=1375%:%
%:%2728=1376%:%
%:%2729=1376%:%
%:%2730=1377%:%
%:%2731=1377%:%
%:%2732=1377%:%
%:%2733=1378%:%
%:%2734=1378%:%
%:%2735=1379%:%
%:%2736=1379%:%
%:%2737=1379%:%
%:%2738=1380%:%
%:%2739=1380%:%
%:%2740=1381%:%
%:%2741=1381%:%
%:%2742=1381%:%
%:%2743=1382%:%
%:%2744=1382%:%
%:%2745=1382%:%
%:%2746=1383%:%
%:%2747=1383%:%
%:%2748=1384%:%
%:%2749=1384%:%
%:%2750=1384%:%
%:%2751=1385%:%
%:%2752=1385%:%
%:%2753=1386%:%
%:%2754=1386%:%
%:%2755=1387%:%
%:%2756=1387%:%
%:%2757=1388%:%
%:%2758=1388%:%
%:%2759=1389%:%
%:%2760=1389%:%
%:%2761=1389%:%
%:%2762=1390%:%
%:%2763=1390%:%
%:%2764=1391%:%
%:%2765=1391%:%
%:%2766=1392%:%
%:%2767=1392%:%
%:%2768=1393%:%
%:%2769=1393%:%
%:%2770=1393%:%
%:%2771=1394%:%
%:%2772=1394%:%
%:%2773=1395%:%
%:%2774=1395%:%
%:%2775=1395%:%
%:%2776=1396%:%
%:%2777=1396%:%
%:%2778=1397%:%
%:%2779=1397%:%
%:%2780=1397%:%
%:%2781=1398%:%
%:%2782=1398%:%
%:%2783=1398%:%
%:%2784=1399%:%
%:%2785=1399%:%
%:%2786=1400%:%
%:%2787=1400%:%
%:%2788=1400%:%
%:%2789=1401%:%
%:%2790=1401%:%
%:%2791=1402%:%
%:%2792=1402%:%
%:%2793=1403%:%
%:%2794=1403%:%
%:%2795=1403%:%
%:%2796=1404%:%
%:%2797=1404%:%
%:%2798=1405%:%
%:%2799=1405%:%
%:%2800=1405%:%
%:%2801=1406%:%
%:%2802=1406%:%
%:%2803=1407%:%
%:%2804=1407%:%
%:%2805=1407%:%
%:%2806=1408%:%
%:%2807=1408%:%
%:%2808=1408%:%
%:%2809=1409%:%
%:%2810=1409%:%
%:%2811=1410%:%
%:%2812=1410%:%
%:%2813=1410%:%
%:%2814=1411%:%
%:%2815=1411%:%
%:%2816=1412%:%
%:%2817=1412%:%
%:%2818=1413%:%
%:%2828=1415%:%
%:%2828=1416%:%
%:%2829=1417%:%
%:%2830=1418%:%
%:%2831=1419%:%
%:%2832=1420%:%
%:%2833=1421%:%
%:%2834=1422%:%
%:%2835=1423%:%
%:%2836=1424%:%
%:%2837=1425%:%
%:%2838=1426%:%
%:%2839=1427%:%
%:%2840=1428%:%
%:%2841=1429%:%
%:%2842=1430%:%
%:%2843=1431%:%
%:%2844=1432%:%
%:%2845=1433%:%
%:%2846=1434%:%
%:%2847=1435%:%
%:%2848=1436%:%
%:%2849=1437%:%
%:%2850=1438%:%
%:%2851=1439%:%
%:%2852=1440%:%
%:%2853=1441%:%
%:%2854=1442%:%
%:%2855=1443%:%
%:%2856=1444%:%
%:%2857=1445%:%
%:%2858=1446%:%
%:%2859=1447%:%
%:%2860=1448%:%
%:%2861=1449%:%
%:%2862=1450%:%
%:%2863=1451%:%
%:%2864=1452%:%
%:%2865=1453%:%
%:%2866=1454%:%
%:%2867=1455%:%
%:%2869=1457%:%
%:%2870=1457%:%
%:%2871=1458%:%
%:%2872=1459%:%
%:%2873=1460%:%
%:%2874=1460%:%
%:%2875=1461%:%
%:%2876=1462%:%
%:%2879=1463%:%
%:%2883=1463%:%
%:%2884=1463%:%
%:%2885=1463%:%
%:%2886=1463%:%
%:%2887=1463%:%
%:%2892=1463%:%
%:%2895=1464%:%
%:%2896=1465%:%
%:%2897=1465%:%
%:%2898=1466%:%
%:%2899=1467%:%
%:%2900=1468%:%
%:%2901=1469%:%
%:%2908=1470%:%
%:%2909=1470%:%
%:%2910=1471%:%
%:%2911=1471%:%
%:%2912=1472%:%
%:%2913=1472%:%
%:%2914=1472%:%
%:%2915=1472%:%
%:%2916=1473%:%
%:%2917=1473%:%
%:%2918=1473%:%
%:%2919=1474%:%
%:%2920=1474%:%
%:%2921=1474%:%
%:%2922=1475%:%
%:%2923=1475%:%
%:%2924=1476%:%
%:%2925=1476%:%
%:%2926=1476%:%
%:%2927=1477%:%
%:%2933=1477%:%
%:%2936=1478%:%
%:%2937=1479%:%
%:%2938=1479%:%
%:%2945=1480%:%
%:%2946=1480%:%
%:%2947=1481%:%
%:%2948=1481%:%
%:%2949=1482%:%
%:%2950=1482%:%
%:%2951=1482%:%
%:%2952=1483%:%
%:%2953=1483%:%
%:%2954=1484%:%
%:%2955=1484%:%
%:%2956=1484%:%
%:%2957=1485%:%
%:%2958=1485%:%
%:%2959=1486%:%
%:%2960=1486%:%
%:%2961=1487%:%
%:%2962=1487%:%
%:%2963=1487%:%
%:%2964=1488%:%
%:%2965=1488%:%
%:%2966=1489%:%
%:%2967=1489%:%
%:%2968=1490%:%
%:%2969=1490%:%
%:%2970=1491%:%
%:%2971=1491%:%
%:%2972=1491%:%
%:%2973=1492%:%
%:%2974=1492%:%
%:%2975=1493%:%
%:%2976=1493%:%
%:%2977=1494%:%
%:%2983=1494%:%
%:%2986=1495%:%
%:%2987=1496%:%
%:%2988=1496%:%
%:%2989=1497%:%
%:%2990=1498%:%
%:%2997=1499%:%
%:%2998=1499%:%
%:%2999=1500%:%
%:%3000=1500%:%
%:%3001=1501%:%
%:%3002=1501%:%
%:%3003=1501%:%
%:%3004=1502%:%
%:%3005=1502%:%
%:%3006=1503%:%
%:%3007=1503%:%
%:%3008=1504%:%
%:%3009=1504%:%
%:%3010=1505%:%
%:%3011=1505%:%
%:%3012=1506%:%
%:%3013=1506%:%
%:%3014=1506%:%
%:%3015=1507%:%
%:%3016=1507%:%
%:%3017=1507%:%
%:%3018=1508%:%
%:%3019=1508%:%
%:%3020=1509%:%
%:%3021=1509%:%
%:%3022=1510%:%
%:%3023=1510%:%
%:%3024=1511%:%
%:%3025=1511%:%
%:%3026=1512%:%
%:%3027=1512%:%
%:%3028=1512%:%
%:%3029=1513%:%
%:%3030=1513%:%
%:%3031=1514%:%
%:%3032=1514%:%
%:%3033=1515%:%
%:%3034=1515%:%
%:%3035=1515%:%
%:%3036=1516%:%
%:%3037=1516%:%
%:%3038=1516%:%
%:%3039=1517%:%
%:%3045=1517%:%
%:%3048=1518%:%
%:%3049=1519%:%
%:%3050=1519%:%
%:%3057=1520%:%
%:%3058=1520%:%
%:%3059=1521%:%
%:%3060=1521%:%
%:%3061=1522%:%
%:%3062=1522%:%
%:%3063=1522%:%
%:%3064=1523%:%
%:%3065=1523%:%
%:%3066=1524%:%
%:%3067=1524%:%
%:%3068=1524%:%
%:%3069=1525%:%
%:%3070=1525%:%
%:%3071=1526%:%
%:%3072=1526%:%
%:%3073=1527%:%
%:%3074=1527%:%
%:%3075=1527%:%
%:%3076=1528%:%
%:%3077=1528%:%
%:%3078=1529%:%
%:%3079=1529%:%
%:%3080=1530%:%
%:%3081=1530%:%
%:%3082=1531%:%
%:%3083=1531%:%
%:%3084=1531%:%
%:%3085=1532%:%
%:%3086=1532%:%
%:%3087=1533%:%
%:%3088=1533%:%
%:%3089=1534%:%
%:%3090=1534%:%
%:%3091=1535%:%
%:%3092=1535%:%
%:%3093=1535%:%
%:%3094=1536%:%
%:%3095=1536%:%
%:%3096=1536%:%
%:%3097=1537%:%
%:%3098=1537%:%
%:%3099=1538%:%
%:%3100=1538%:%
%:%3101=1539%:%
%:%3102=1539%:%
%:%3103=1539%:%
%:%3104=1540%:%
%:%3110=1540%:%
%:%3113=1541%:%
%:%3114=1542%:%
%:%3115=1542%:%
%:%3116=1543%:%
%:%3117=1544%:%
%:%3124=1545%:%
%:%3125=1545%:%
%:%3126=1546%:%
%:%3127=1546%:%
%:%3128=1547%:%
%:%3129=1547%:%
%:%3130=1548%:%
%:%3131=1548%:%
%:%3132=1549%:%
%:%3133=1549%:%
%:%3134=1550%:%
%:%3135=1550%:%
%:%3136=1551%:%
%:%3137=1551%:%
%:%3138=1552%:%
%:%3139=1552%:%
%:%3140=1553%:%
%:%3141=1553%:%
%:%3142=1554%:%
%:%3143=1554%:%
%:%3144=1554%:%
%:%3145=1555%:%
%:%3146=1555%:%
%:%3147=1555%:%
%:%3148=1556%:%
%:%3149=1556%:%
%:%3150=1557%:%
%:%3151=1557%:%
%:%3152=1558%:%
%:%3153=1558%:%
%:%3154=1558%:%
%:%3155=1559%:%
%:%3156=1559%:%
%:%3157=1559%:%
%:%3158=1560%:%
%:%3159=1560%:%
%:%3160=1561%:%
%:%3161=1561%:%
%:%3162=1562%:%
%:%3163=1562%:%
%:%3164=1562%:%
%:%3165=1563%:%
%:%3166=1563%:%
%:%3167=1563%:%
%:%3168=1564%:%
%:%3169=1564%:%
%:%3170=1565%:%
%:%3171=1565%:%
%:%3172=1566%:%
%:%3173=1566%:%
%:%3174=1567%:%
%:%3175=1567%:%
%:%3176=1568%:%
%:%3177=1568%:%
%:%3178=1568%:%
%:%3179=1569%:%
%:%3180=1569%:%
%:%3181=1570%:%
%:%3182=1570%:%
%:%3183=1571%:%
%:%3184=1571%:%
%:%3185=1572%:%
%:%3186=1572%:%
%:%3187=1572%:%
%:%3188=1573%:%
%:%3189=1573%:%
%:%3190=1574%:%
%:%3196=1574%:%
%:%3199=1575%:%
%:%3200=1576%:%
%:%3201=1576%:%
%:%3202=1577%:%
%:%3203=1578%:%
%:%3204=1579%:%
%:%3205=1580%:%
%:%3206=1581%:%
%:%3213=1582%:%
%:%3214=1582%:%
%:%3215=1583%:%
%:%3216=1583%:%
%:%3217=1584%:%
%:%3218=1584%:%
%:%3219=1585%:%
%:%3220=1585%:%
%:%3221=1586%:%
%:%3222=1586%:%
%:%3223=1586%:%
%:%3224=1587%:%
%:%3225=1587%:%
%:%3226=1587%:%
%:%3227=1588%:%
%:%3228=1588%:%
%:%3229=1589%:%
%:%3230=1589%:%
%:%3231=1590%:%
%:%3232=1590%:%
%:%3233=1590%:%
%:%3234=1591%:%
%:%3235=1591%:%
%:%3236=1591%:%
%:%3237=1592%:%
%:%3238=1592%:%
%:%3239=1593%:%
%:%3240=1593%:%
%:%3241=1593%:%
%:%3242=1594%:%
%:%3243=1594%:%
%:%3244=1595%:%
%:%3245=1595%:%
%:%3246=1596%:%
%:%3247=1596%:%
%:%3248=1596%:%
%:%3249=1597%:%
%:%3255=1597%:%
%:%3258=1598%:%
%:%3259=1599%:%
%:%3260=1599%:%
%:%3261=1600%:%
%:%3262=1601%:%
%:%3263=1602%:%
%:%3266=1603%:%
%:%3270=1603%:%
%:%3271=1603%:%
%:%3272=1603%:%
%:%3277=1603%:%
%:%3280=1604%:%
%:%3281=1605%:%
%:%3282=1605%:%
%:%3283=1606%:%
%:%3284=1607%:%
%:%3285=1608%:%
%:%3286=1609%:%
%:%3287=1610%:%
%:%3288=1611%:%
%:%3295=1612%:%
%:%3296=1612%:%
%:%3297=1613%:%
%:%3298=1613%:%
%:%3299=1614%:%
%:%3300=1614%:%
%:%3301=1614%:%
%:%3302=1615%:%
%:%3303=1615%:%
%:%3304=1616%:%
%:%3305=1616%:%
%:%3306=1617%:%
%:%3307=1617%:%
%:%3308=1618%:%
%:%3309=1618%:%
%:%3310=1618%:%
%:%3311=1619%:%
%:%3312=1619%:%
%:%3313=1620%:%
%:%3314=1620%:%
%:%3315=1620%:%
%:%3316=1621%:%
%:%3317=1621%:%
%:%3318=1622%:%
%:%3319=1622%:%
%:%3320=1622%:%
%:%3321=1623%:%
%:%3322=1623%:%
%:%3323=1623%:%
%:%3324=1624%:%
%:%3325=1624%:%
%:%3326=1624%:%
%:%3327=1625%:%
%:%3328=1625%:%
%:%3329=1625%:%
%:%3330=1626%:%
%:%3331=1626%:%
%:%3332=1627%:%
%:%3333=1627%:%
%:%3334=1627%:%
%:%3335=1628%:%
%:%3336=1628%:%
%:%3337=1629%:%
%:%3338=1629%:%
%:%3339=1629%:%
%:%3340=1630%:%
%:%3341=1630%:%
%:%3342=1631%:%
%:%3343=1631%:%
%:%3344=1632%:%
%:%3345=1632%:%
%:%3346=1632%:%
%:%3347=1633%:%
%:%3348=1633%:%
%:%3349=1633%:%
%:%3350=1634%:%
%:%3351=1634%:%
%:%3352=1635%:%
%:%3353=1635%:%
%:%3354=1636%:%
%:%3355=1636%:%
%:%3356=1636%:%
%:%3357=1637%:%
%:%3358=1637%:%
%:%3359=1637%:%
%:%3360=1638%:%
%:%3361=1638%:%
%:%3362=1639%:%
%:%3363=1639%:%
%:%3364=1639%:%
%:%3365=1640%:%
%:%3366=1640%:%
%:%3367=1641%:%
%:%3368=1641%:%
%:%3369=1642%:%
%:%3370=1642%:%
%:%3371=1643%:%
%:%3377=1643%:%
%:%3380=1644%:%
%:%3381=1645%:%
%:%3382=1645%:%
%:%3383=1646%:%
%:%3384=1647%:%
%:%3385=1648%:%
%:%3386=1649%:%
%:%3387=1650%:%
%:%3388=1651%:%
%:%3395=1652%:%
%:%3396=1652%:%
%:%3397=1653%:%
%:%3398=1653%:%
%:%3399=1654%:%
%:%3400=1654%:%
%:%3401=1654%:%
%:%3402=1655%:%
%:%3403=1655%:%
%:%3404=1656%:%
%:%3405=1656%:%
%:%3406=1657%:%
%:%3407=1657%:%
%:%3408=1658%:%
%:%3409=1658%:%
%:%3410=1658%:%
%:%3411=1659%:%
%:%3412=1659%:%
%:%3413=1660%:%
%:%3414=1660%:%
%:%3415=1660%:%
%:%3416=1661%:%
%:%3417=1661%:%
%:%3418=1662%:%
%:%3419=1662%:%
%:%3420=1662%:%
%:%3421=1663%:%
%:%3422=1663%:%
%:%3423=1663%:%
%:%3424=1664%:%
%:%3425=1664%:%
%:%3426=1664%:%
%:%3427=1665%:%
%:%3428=1665%:%
%:%3429=1665%:%
%:%3430=1666%:%
%:%3431=1666%:%
%:%3432=1667%:%
%:%3433=1667%:%
%:%3434=1667%:%
%:%3435=1668%:%
%:%3436=1668%:%
%:%3437=1669%:%
%:%3438=1669%:%
%:%3439=1669%:%
%:%3440=1670%:%
%:%3441=1670%:%
%:%3442=1671%:%
%:%3443=1671%:%
%:%3444=1671%:%
%:%3445=1672%:%
%:%3446=1672%:%
%:%3447=1673%:%
%:%3448=1673%:%
%:%3449=1673%:%
%:%3450=1674%:%
%:%3451=1674%:%
%:%3452=1675%:%
%:%3453=1675%:%
%:%3454=1675%:%
%:%3455=1676%:%
%:%3456=1676%:%
%:%3457=1677%:%
%:%3458=1677%:%
%:%3459=1678%:%
%:%3460=1678%:%
%:%3461=1678%:%
%:%3462=1679%:%
%:%3463=1679%:%
%:%3464=1679%:%
%:%3465=1680%:%
%:%3466=1680%:%
%:%3467=1681%:%
%:%3468=1681%:%
%:%3469=1682%:%
%:%3470=1682%:%
%:%3471=1682%:%
%:%3472=1683%:%
%:%3473=1683%:%
%:%3474=1683%:%
%:%3475=1684%:%
%:%3476=1684%:%
%:%3477=1685%:%
%:%3478=1685%:%
%:%3479=1685%:%
%:%3480=1686%:%
%:%3481=1686%:%
%:%3482=1687%:%
%:%3483=1687%:%
%:%3484=1688%:%
%:%3485=1688%:%
%:%3486=1689%:%
%:%3492=1689%:%
%:%3495=1690%:%
%:%3496=1691%:%
%:%3497=1691%:%
%:%3498=1692%:%
%:%3499=1693%:%
%:%3500=1694%:%
%:%3501=1695%:%
%:%3502=1696%:%
%:%3503=1697%:%
%:%3510=1698%:%
%:%3511=1698%:%
%:%3512=1699%:%
%:%3513=1699%:%
%:%3514=1700%:%
%:%3515=1700%:%
%:%3516=1700%:%
%:%3517=1701%:%
%:%3518=1701%:%
%:%3519=1702%:%
%:%3520=1702%:%
%:%3521=1703%:%
%:%3522=1703%:%
%:%3523=1704%:%
%:%3524=1704%:%
%:%3525=1704%:%
%:%3526=1705%:%
%:%3527=1705%:%
%:%3528=1706%:%
%:%3529=1706%:%
%:%3530=1706%:%
%:%3531=1707%:%
%:%3532=1707%:%
%:%3533=1708%:%
%:%3534=1708%:%
%:%3535=1708%:%
%:%3536=1709%:%
%:%3537=1709%:%
%:%3538=1709%:%
%:%3539=1710%:%
%:%3540=1710%:%
%:%3541=1710%:%
%:%3542=1711%:%
%:%3543=1711%:%
%:%3544=1711%:%
%:%3545=1712%:%
%:%3546=1712%:%
%:%3547=1713%:%
%:%3548=1713%:%
%:%3549=1713%:%
%:%3550=1714%:%
%:%3551=1714%:%
%:%3552=1715%:%
%:%3553=1715%:%
%:%3554=1715%:%
%:%3555=1716%:%
%:%3556=1716%:%
%:%3557=1717%:%
%:%3558=1717%:%
%:%3559=1717%:%
%:%3560=1718%:%
%:%3561=1718%:%
%:%3562=1719%:%
%:%3563=1719%:%
%:%3564=1719%:%
%:%3565=1720%:%
%:%3566=1720%:%
%:%3567=1721%:%
%:%3568=1721%:%
%:%3569=1721%:%
%:%3570=1722%:%
%:%3571=1722%:%
%:%3572=1723%:%
%:%3573=1723%:%
%:%3574=1724%:%
%:%3575=1724%:%
%:%3576=1724%:%
%:%3577=1725%:%
%:%3578=1725%:%
%:%3579=1725%:%
%:%3580=1726%:%
%:%3581=1726%:%
%:%3582=1727%:%
%:%3583=1727%:%
%:%3584=1728%:%
%:%3585=1728%:%
%:%3586=1728%:%
%:%3587=1729%:%
%:%3588=1729%:%
%:%3589=1729%:%
%:%3590=1730%:%
%:%3591=1730%:%
%:%3592=1731%:%
%:%3593=1731%:%
%:%3594=1731%:%
%:%3595=1732%:%
%:%3596=1732%:%
%:%3597=1733%:%
%:%3598=1733%:%
%:%3599=1734%:%
%:%3600=1734%:%
%:%3601=1735%:%
%:%3607=1735%:%
%:%3610=1736%:%
%:%3611=1737%:%
%:%3612=1737%:%
%:%3613=1738%:%
%:%3614=1739%:%
%:%3615=1740%:%
%:%3616=1741%:%
%:%3617=1742%:%
%:%3618=1743%:%
%:%3619=1744%:%
%:%3626=1745%:%
%:%3627=1745%:%
%:%3628=1746%:%
%:%3629=1746%:%
%:%3630=1747%:%
%:%3631=1747%:%
%:%3632=1747%:%
%:%3633=1748%:%
%:%3634=1748%:%
%:%3635=1749%:%
%:%3636=1749%:%
%:%3637=1750%:%
%:%3638=1750%:%
%:%3639=1751%:%
%:%3640=1751%:%
%:%3641=1751%:%
%:%3642=1752%:%
%:%3643=1752%:%
%:%3644=1753%:%
%:%3645=1753%:%
%:%3646=1753%:%
%:%3647=1754%:%
%:%3648=1754%:%
%:%3649=1755%:%
%:%3650=1755%:%
%:%3651=1755%:%
%:%3652=1756%:%
%:%3653=1756%:%
%:%3654=1756%:%
%:%3655=1757%:%
%:%3656=1757%:%
%:%3657=1757%:%
%:%3658=1758%:%
%:%3659=1758%:%
%:%3660=1758%:%
%:%3661=1759%:%
%:%3662=1759%:%
%:%3663=1760%:%
%:%3664=1760%:%
%:%3665=1760%:%
%:%3666=1761%:%
%:%3667=1761%:%
%:%3668=1762%:%
%:%3669=1762%:%
%:%3670=1762%:%
%:%3671=1763%:%
%:%3672=1763%:%
%:%3673=1764%:%
%:%3674=1764%:%
%:%3675=1764%:%
%:%3676=1765%:%
%:%3677=1765%:%
%:%3678=1766%:%
%:%3679=1766%:%
%:%3680=1767%:%
%:%3681=1767%:%
%:%3682=1767%:%
%:%3683=1768%:%
%:%3684=1768%:%
%:%3685=1768%:%
%:%3686=1769%:%
%:%3687=1769%:%
%:%3688=1770%:%
%:%3689=1770%:%
%:%3690=1771%:%
%:%3691=1771%:%
%:%3692=1771%:%
%:%3693=1772%:%
%:%3694=1772%:%
%:%3695=1772%:%
%:%3696=1773%:%
%:%3697=1773%:%
%:%3698=1774%:%
%:%3699=1774%:%
%:%3700=1774%:%
%:%3701=1775%:%
%:%3702=1775%:%
%:%3703=1776%:%
%:%3704=1776%:%
%:%3705=1777%:%
%:%3706=1777%:%
%:%3707=1778%:%
%:%3713=1778%:%
%:%3716=1779%:%
%:%3717=1780%:%
%:%3718=1780%:%
%:%3719=1781%:%
%:%3720=1782%:%
%:%3721=1783%:%
%:%3722=1784%:%
%:%3723=1785%:%
%:%3724=1786%:%
%:%3731=1787%:%
%:%3732=1787%:%
%:%3733=1788%:%
%:%3734=1788%:%
%:%3735=1789%:%
%:%3736=1789%:%
%:%3737=1790%:%
%:%3738=1790%:%
%:%3741=1793%:%
%:%3742=1794%:%
%:%3743=1794%:%
%:%3744=1794%:%
%:%3745=1795%:%
%:%3746=1795%:%
%:%3747=1795%:%
%:%3748=1796%:%
%:%3749=1796%:%
%:%3750=1797%:%
%:%3751=1797%:%
%:%3752=1798%:%
%:%3753=1798%:%
%:%3754=1799%:%
%:%3755=1799%:%
%:%3756=1799%:%
%:%3757=1800%:%
%:%3758=1800%:%
%:%3759=1801%:%
%:%3760=1801%:%
%:%3762=1803%:%
%:%3763=1804%:%
%:%3764=1804%:%
%:%3765=1805%:%
%:%3766=1805%:%
%:%3767=1806%:%
%:%3768=1806%:%
%:%3769=1807%:%
%:%3770=1807%:%
%:%3771=1808%:%
%:%3772=1808%:%
%:%3773=1808%:%
%:%3774=1809%:%
%:%3775=1809%:%
%:%3776=1810%:%
%:%3777=1810%:%
%:%3778=1810%:%
%:%3779=1811%:%
%:%3780=1811%:%
%:%3781=1812%:%
%:%3782=1812%:%
%:%3783=1812%:%
%:%3784=1813%:%
%:%3785=1813%:%
%:%3786=1814%:%
%:%3787=1814%:%
%:%3788=1814%:%
%:%3789=1815%:%
%:%3790=1815%:%
%:%3791=1816%:%
%:%3792=1816%:%
%:%3793=1816%:%
%:%3794=1817%:%
%:%3795=1817%:%
%:%3796=1818%:%
%:%3797=1818%:%
%:%3798=1819%:%
%:%3799=1819%:%
%:%3800=1820%:%
%:%3801=1820%:%
%:%3802=1821%:%
%:%3803=1822%:%
%:%3804=1822%:%
%:%3805=1823%:%
%:%3806=1823%:%
%:%3807=1824%:%
%:%3808=1824%:%
%:%3809=1825%:%
%:%3810=1825%:%
%:%3811=1826%:%
%:%3812=1826%:%
%:%3813=1826%:%
%:%3814=1827%:%
%:%3815=1827%:%
%:%3816=1828%:%
%:%3817=1828%:%
%:%3818=1828%:%
%:%3819=1829%:%
%:%3820=1829%:%
%:%3821=1830%:%
%:%3822=1830%:%
%:%3823=1830%:%
%:%3824=1831%:%
%:%3825=1831%:%
%:%3826=1832%:%
%:%3827=1832%:%
%:%3828=1832%:%
%:%3829=1833%:%
%:%3830=1833%:%
%:%3831=1834%:%
%:%3832=1834%:%
%:%3833=1835%:%
%:%3834=1835%:%
%:%3835=1836%:%
%:%3836=1836%:%
%:%3837=1837%:%
%:%3838=1837%:%
%:%3839=1837%:%
%:%3840=1838%:%
%:%3841=1838%:%
%:%3842=1839%:%
%:%3843=1839%:%
%:%3844=1840%:%
%:%3845=1840%:%
%:%3846=1840%:%
%:%3847=1841%:%
%:%3848=1841%:%
%:%3849=1842%:%
%:%3850=1842%:%
%:%3851=1842%:%
%:%3852=1843%:%
%:%3853=1843%:%
%:%3854=1844%:%
%:%3855=1844%:%
%:%3856=1845%:%
%:%3857=1845%:%
%:%3858=1846%:%
%:%3859=1846%:%
%:%3860=1847%:%
%:%3861=1847%:%
%:%3862=1847%:%
%:%3863=1848%:%
%:%3869=1848%:%
%:%3872=1849%:%
%:%3873=1850%:%
%:%3874=1850%:%
%:%3875=1851%:%
%:%3876=1852%:%
%:%3883=1853%:%
%:%3884=1853%:%
%:%3885=1854%:%
%:%3886=1854%:%
%:%3887=1855%:%
%:%3888=1855%:%
%:%3889=1856%:%
%:%3890=1856%:%
%:%3891=1857%:%
%:%3892=1857%:%
%:%3893=1858%:%
%:%3894=1858%:%
%:%3895=1859%:%
%:%3896=1859%:%
%:%3897=1860%:%
%:%3898=1860%:%
%:%3899=1861%:%
%:%3900=1861%:%
%:%3901=1862%:%
%:%3902=1862%:%
%:%3903=1863%:%
%:%3904=1863%:%
%:%3905=1864%:%
%:%3906=1864%:%
%:%3907=1865%:%
%:%3908=1865%:%
%:%3909=1866%:%
%:%3910=1866%:%
%:%3911=1866%:%
%:%3912=1867%:%
%:%3913=1867%:%
%:%3914=1868%:%
%:%3915=1868%:%
%:%3916=1869%:%
%:%3917=1869%:%
%:%3918=1870%:%
%:%3919=1870%:%
%:%3920=1871%:%
%:%3921=1871%:%
%:%3922=1872%:%
%:%3923=1872%:%
%:%3924=1872%:%
%:%3925=1873%:%
%:%3926=1873%:%
%:%3927=1874%:%
%:%3928=1874%:%
%:%3929=1874%:%
%:%3930=1875%:%
%:%3931=1875%:%
%:%3932=1876%:%
%:%3933=1876%:%
%:%3934=1876%:%
%:%3935=1877%:%
%:%3936=1877%:%
%:%3937=1878%:%
%:%3938=1878%:%
%:%3939=1878%:%
%:%3940=1879%:%
%:%3941=1879%:%
%:%3942=1880%:%
%:%3943=1880%:%
%:%3944=1880%:%
%:%3945=1881%:%
%:%3946=1881%:%
%:%3947=1882%:%
%:%3948=1882%:%
%:%3949=1883%:%
%:%3950=1883%:%
%:%3951=1884%:%
%:%3952=1884%:%
%:%3953=1885%:%
%:%3954=1885%:%
%:%3955=1885%:%
%:%3956=1886%:%
%:%3957=1886%:%
%:%3958=1887%:%
%:%3959=1887%:%
%:%3960=1888%:%
%:%3961=1888%:%
%:%3962=1889%:%
%:%3963=1889%:%
%:%3964=1890%:%
%:%3965=1890%:%
%:%3966=1891%:%
%:%3967=1891%:%
%:%3968=1891%:%
%:%3969=1892%:%
%:%3970=1892%:%
%:%3971=1893%:%
%:%3972=1893%:%
%:%3973=1894%:%
%:%3974=1894%:%
%:%3975=1894%:%
%:%3976=1895%:%
%:%3977=1895%:%
%:%3978=1896%:%
%:%3979=1896%:%
%:%3980=1897%:%
%:%3981=1897%:%
%:%3982=1898%:%
%:%3983=1898%:%
%:%3984=1899%:%
%:%3985=1899%:%
%:%3986=1900%:%
%:%3987=1900%:%
%:%3988=1900%:%
%:%3989=1901%:%
%:%3990=1901%:%
%:%3991=1902%:%
%:%3992=1902%:%
%:%3993=1903%:%
%:%3994=1903%:%
%:%3995=1903%:%
%:%3996=1904%:%
%:%3997=1904%:%
%:%3998=1905%:%
%:%3999=1905%:%
%:%4000=1906%:%
%:%4001=1906%:%
%:%4002=1906%:%
%:%4003=1907%:%
%:%4004=1907%:%
%:%4005=1908%:%
%:%4006=1908%:%
%:%4007=1909%:%
%:%4008=1909%:%
%:%4009=1909%:%
%:%4010=1910%:%
%:%4011=1910%:%
%:%4012=1911%:%
%:%4013=1911%:%
%:%4014=1912%:%
%:%4015=1912%:%
%:%4016=1913%:%
%:%4017=1913%:%
%:%4018=1914%:%
%:%4019=1914%:%
%:%4020=1915%:%
%:%4021=1915%:%
%:%4022=1916%:%
%:%4023=1916%:%
%:%4024=1917%:%
%:%4030=1917%:%
%:%4033=1918%:%
%:%4034=1919%:%
%:%4035=1919%:%
%:%4036=1920%:%
%:%4037=1921%:%
%:%4038=1922%:%
%:%4039=1923%:%
%:%4040=1924%:%
%:%4041=1925%:%
%:%4048=1926%:%
%:%4049=1926%:%
%:%4050=1927%:%
%:%4051=1927%:%
%:%4052=1928%:%
%:%4053=1928%:%
%:%4054=1928%:%
%:%4055=1929%:%
%:%4056=1929%:%
%:%4057=1930%:%
%:%4058=1930%:%
%:%4059=1931%:%
%:%4060=1931%:%
%:%4061=1932%:%
%:%4062=1932%:%
%:%4063=1932%:%
%:%4064=1933%:%
%:%4065=1933%:%
%:%4066=1934%:%
%:%4067=1934%:%
%:%4068=1934%:%
%:%4069=1935%:%
%:%4070=1935%:%
%:%4071=1936%:%
%:%4072=1936%:%
%:%4073=1936%:%
%:%4074=1937%:%
%:%4075=1937%:%
%:%4076=1937%:%
%:%4077=1938%:%
%:%4078=1938%:%
%:%4079=1938%:%
%:%4080=1939%:%
%:%4081=1939%:%
%:%4082=1939%:%
%:%4083=1940%:%
%:%4084=1940%:%
%:%4085=1941%:%
%:%4086=1941%:%
%:%4087=1942%:%
%:%4088=1942%:%
%:%4089=1943%:%
%:%4090=1943%:%
%:%4091=1944%:%
%:%4092=1944%:%
%:%4093=1944%:%
%:%4094=1945%:%
%:%4095=1945%:%
%:%4096=1946%:%
%:%4097=1946%:%
%:%4098=1946%:%
%:%4099=1947%:%
%:%4100=1947%:%
%:%4101=1947%:%
%:%4102=1948%:%
%:%4103=1948%:%
%:%4104=1948%:%
%:%4105=1949%:%
%:%4106=1949%:%
%:%4107=1950%:%
%:%4108=1950%:%
%:%4109=1950%:%
%:%4110=1951%:%
%:%4111=1951%:%
%:%4112=1952%:%
%:%4113=1952%:%
%:%4114=1952%:%
%:%4115=1953%:%
%:%4116=1953%:%
%:%4117=1954%:%
%:%4118=1954%:%
%:%4119=1955%:%
%:%4120=1955%:%
%:%4121=1956%:%
%:%4122=1956%:%
%:%4123=1957%:%
%:%4124=1957%:%
%:%4125=1958%:%
%:%4126=1958%:%
%:%4127=1958%:%
%:%4128=1959%:%
%:%4129=1959%:%
%:%4130=1960%:%
%:%4131=1960%:%
%:%4132=1960%:%
%:%4133=1961%:%
%:%4134=1961%:%
%:%4135=1961%:%
%:%4136=1962%:%
%:%4137=1962%:%
%:%4138=1962%:%
%:%4139=1963%:%
%:%4140=1963%:%
%:%4141=1964%:%
%:%4142=1964%:%
%:%4143=1964%:%
%:%4144=1965%:%
%:%4145=1965%:%
%:%4146=1966%:%
%:%4147=1966%:%
%:%4148=1966%:%
%:%4149=1967%:%
%:%4150=1967%:%
%:%4151=1968%:%
%:%4152=1968%:%
%:%4153=1969%:%
%:%4154=1969%:%
%:%4155=1970%:%
%:%4156=1970%:%
%:%4157=1971%:%
%:%4163=1971%:%
%:%4166=1972%:%
%:%4167=1973%:%
%:%4168=1973%:%
%:%4169=1974%:%
%:%4170=1975%:%
%:%4171=1976%:%
%:%4172=1977%:%
%:%4173=1978%:%
%:%4174=1979%:%
%:%4181=1980%:%
%:%4182=1980%:%
%:%4183=1981%:%
%:%4184=1981%:%
%:%4185=1982%:%
%:%4186=1982%:%
%:%4187=1982%:%
%:%4188=1983%:%
%:%4189=1983%:%
%:%4190=1984%:%
%:%4191=1984%:%
%:%4192=1985%:%
%:%4193=1985%:%
%:%4194=1986%:%
%:%4195=1986%:%
%:%4196=1986%:%
%:%4197=1987%:%
%:%4198=1987%:%
%:%4199=1988%:%
%:%4200=1988%:%
%:%4201=1988%:%
%:%4202=1989%:%
%:%4203=1989%:%
%:%4204=1990%:%
%:%4205=1990%:%
%:%4206=1990%:%
%:%4207=1991%:%
%:%4208=1991%:%
%:%4209=1991%:%
%:%4210=1992%:%
%:%4211=1992%:%
%:%4212=1992%:%
%:%4213=1993%:%
%:%4214=1993%:%
%:%4215=1993%:%
%:%4216=1994%:%
%:%4217=1994%:%
%:%4218=1995%:%
%:%4219=1995%:%
%:%4220=1995%:%
%:%4221=1996%:%
%:%4222=1996%:%
%:%4223=1997%:%
%:%4224=1997%:%
%:%4225=1997%:%
%:%4226=1998%:%
%:%4227=1998%:%
%:%4228=1999%:%
%:%4229=1999%:%
%:%4230=2000%:%
%:%4231=2000%:%
%:%4232=2001%:%
%:%4233=2001%:%
%:%4234=2002%:%
%:%4235=2002%:%
%:%4236=2002%:%
%:%4237=2003%:%
%:%4238=2003%:%
%:%4239=2004%:%
%:%4240=2004%:%
%:%4241=2004%:%
%:%4242=2005%:%
%:%4243=2005%:%
%:%4244=2006%:%
%:%4245=2006%:%
%:%4246=2006%:%
%:%4247=2007%:%
%:%4248=2007%:%
%:%4249=2007%:%
%:%4250=2008%:%
%:%4251=2008%:%
%:%4252=2008%:%
%:%4253=2009%:%
%:%4254=2009%:%
%:%4255=2010%:%
%:%4256=2010%:%
%:%4257=2010%:%
%:%4258=2011%:%
%:%4259=2011%:%
%:%4260=2012%:%
%:%4261=2012%:%
%:%4262=2012%:%
%:%4263=2013%:%
%:%4264=2013%:%
%:%4265=2014%:%
%:%4266=2014%:%
%:%4267=2015%:%
%:%4268=2015%:%
%:%4269=2016%:%
%:%4270=2016%:%
%:%4271=2017%:%
%:%4272=2017%:%
%:%4273=2018%:%
%:%4274=2018%:%
%:%4275=2018%:%
%:%4276=2019%:%
%:%4277=2019%:%
%:%4278=2020%:%
%:%4279=2020%:%
%:%4280=2020%:%
%:%4281=2021%:%
%:%4282=2021%:%
%:%4283=2021%:%
%:%4284=2022%:%
%:%4285=2022%:%
%:%4286=2022%:%
%:%4287=2023%:%
%:%4288=2023%:%
%:%4289=2024%:%
%:%4290=2024%:%
%:%4291=2024%:%
%:%4292=2025%:%
%:%4293=2025%:%
%:%4294=2026%:%
%:%4295=2026%:%
%:%4296=2026%:%
%:%4297=2027%:%
%:%4298=2027%:%
%:%4299=2028%:%
%:%4300=2028%:%
%:%4301=2029%:%
%:%4302=2029%:%
%:%4303=2030%:%
%:%4304=2030%:%
%:%4305=2031%:%
%:%4311=2031%:%
%:%4314=2032%:%
%:%4315=2033%:%
%:%4316=2033%:%
%:%4317=2034%:%
%:%4318=2035%:%
%:%4319=2036%:%
%:%4320=2037%:%
%:%4321=2038%:%
%:%4322=2039%:%
%:%4329=2040%:%
%:%4330=2040%:%
%:%4331=2041%:%
%:%4332=2041%:%
%:%4333=2042%:%
%:%4334=2042%:%
%:%4335=2042%:%
%:%4336=2043%:%
%:%4337=2043%:%
%:%4338=2044%:%
%:%4339=2044%:%
%:%4340=2045%:%
%:%4341=2045%:%
%:%4342=2046%:%
%:%4343=2046%:%
%:%4344=2046%:%
%:%4345=2047%:%
%:%4346=2047%:%
%:%4347=2048%:%
%:%4348=2048%:%
%:%4349=2048%:%
%:%4350=2049%:%
%:%4351=2049%:%
%:%4352=2050%:%
%:%4353=2050%:%
%:%4354=2050%:%
%:%4355=2051%:%
%:%4356=2051%:%
%:%4357=2051%:%
%:%4358=2052%:%
%:%4359=2052%:%
%:%4360=2052%:%
%:%4361=2053%:%
%:%4362=2053%:%
%:%4363=2053%:%
%:%4364=2054%:%
%:%4365=2054%:%
%:%4366=2055%:%
%:%4367=2055%:%
%:%4368=2055%:%
%:%4369=2056%:%
%:%4370=2056%:%
%:%4371=2057%:%
%:%4372=2057%:%
%:%4373=2057%:%
%:%4374=2058%:%
%:%4375=2058%:%
%:%4376=2059%:%
%:%4377=2059%:%
%:%4378=2060%:%
%:%4379=2060%:%
%:%4380=2061%:%
%:%4381=2061%:%
%:%4382=2062%:%
%:%4383=2062%:%
%:%4384=2062%:%
%:%4385=2063%:%
%:%4386=2063%:%
%:%4387=2064%:%
%:%4388=2064%:%
%:%4389=2064%:%
%:%4390=2065%:%
%:%4391=2065%:%
%:%4392=2066%:%
%:%4393=2066%:%
%:%4394=2066%:%
%:%4395=2067%:%
%:%4396=2067%:%
%:%4397=2067%:%
%:%4398=2068%:%
%:%4399=2068%:%
%:%4400=2068%:%
%:%4401=2069%:%
%:%4402=2069%:%
%:%4403=2070%:%
%:%4404=2070%:%
%:%4405=2070%:%
%:%4406=2071%:%
%:%4407=2071%:%
%:%4408=2072%:%
%:%4409=2072%:%
%:%4410=2072%:%
%:%4411=2073%:%
%:%4412=2073%:%
%:%4413=2074%:%
%:%4414=2074%:%
%:%4415=2075%:%
%:%4416=2075%:%
%:%4417=2076%:%
%:%4418=2076%:%
%:%4419=2077%:%
%:%4420=2077%:%
%:%4421=2078%:%
%:%4422=2078%:%
%:%4423=2078%:%
%:%4424=2079%:%
%:%4425=2079%:%
%:%4426=2080%:%
%:%4427=2080%:%
%:%4428=2080%:%
%:%4429=2081%:%
%:%4430=2081%:%
%:%4431=2082%:%
%:%4432=2082%:%
%:%4433=2082%:%
%:%4434=2083%:%
%:%4435=2083%:%
%:%4436=2083%:%
%:%4437=2084%:%
%:%4438=2084%:%
%:%4439=2084%:%
%:%4440=2085%:%
%:%4441=2085%:%
%:%4442=2086%:%
%:%4443=2086%:%
%:%4444=2086%:%
%:%4445=2087%:%
%:%4446=2087%:%
%:%4447=2088%:%
%:%4448=2088%:%
%:%4449=2088%:%
%:%4450=2089%:%
%:%4451=2089%:%
%:%4452=2090%:%
%:%4453=2090%:%
%:%4454=2091%:%
%:%4455=2091%:%
%:%4456=2092%:%
%:%4457=2092%:%
%:%4458=2093%:%
%:%4464=2093%:%
%:%4467=2094%:%
%:%4468=2095%:%
%:%4469=2095%:%
%:%4470=2096%:%
%:%4471=2097%:%
%:%4472=2098%:%
%:%4473=2099%:%
%:%4474=2100%:%
%:%4475=2101%:%
%:%4476=2102%:%
%:%4483=2103%:%
%:%4484=2103%:%
%:%4485=2104%:%
%:%4486=2104%:%
%:%4487=2105%:%
%:%4488=2105%:%
%:%4489=2105%:%
%:%4490=2106%:%
%:%4491=2106%:%
%:%4492=2107%:%
%:%4493=2107%:%
%:%4494=2108%:%
%:%4495=2108%:%
%:%4496=2109%:%
%:%4497=2109%:%
%:%4498=2109%:%
%:%4499=2110%:%
%:%4500=2110%:%
%:%4501=2111%:%
%:%4502=2111%:%
%:%4503=2111%:%
%:%4504=2112%:%
%:%4505=2112%:%
%:%4506=2113%:%
%:%4507=2113%:%
%:%4508=2113%:%
%:%4509=2114%:%
%:%4510=2114%:%
%:%4511=2114%:%
%:%4512=2115%:%
%:%4513=2115%:%
%:%4514=2115%:%
%:%4515=2116%:%
%:%4516=2116%:%
%:%4517=2116%:%
%:%4518=2117%:%
%:%4519=2117%:%
%:%4520=2118%:%
%:%4521=2118%:%
%:%4522=2118%:%
%:%4523=2119%:%
%:%4524=2119%:%
%:%4525=2120%:%
%:%4526=2120%:%
%:%4527=2120%:%
%:%4528=2121%:%
%:%4529=2121%:%
%:%4530=2122%:%
%:%4531=2122%:%
%:%4532=2122%:%
%:%4533=2123%:%
%:%4534=2123%:%
%:%4535=2124%:%
%:%4536=2124%:%
%:%4537=2124%:%
%:%4538=2125%:%
%:%4539=2125%:%
%:%4540=2125%:%
%:%4541=2126%:%
%:%4542=2126%:%
%:%4543=2126%:%
%:%4544=2127%:%
%:%4545=2127%:%
%:%4546=2128%:%
%:%4547=2128%:%
%:%4548=2128%:%
%:%4549=2129%:%
%:%4550=2129%:%
%:%4551=2130%:%
%:%4552=2130%:%
%:%4553=2130%:%
%:%4554=2131%:%
%:%4555=2131%:%
%:%4556=2132%:%
%:%4557=2132%:%
%:%4558=2133%:%
%:%4559=2133%:%
%:%4560=2134%:%
%:%4566=2134%:%
%:%4569=2135%:%
%:%4570=2136%:%
%:%4571=2136%:%
%:%4572=2137%:%
%:%4573=2138%:%
%:%4574=2139%:%
%:%4575=2140%:%
%:%4576=2141%:%
%:%4577=2142%:%
%:%4584=2143%:%
%:%4585=2143%:%
%:%4586=2144%:%
%:%4587=2144%:%
%:%4590=2147%:%
%:%4591=2148%:%
%:%4592=2148%:%
%:%4593=2148%:%
%:%4594=2149%:%
%:%4595=2149%:%
%:%4596=2150%:%
%:%4597=2150%:%
%:%4598=2151%:%
%:%4599=2151%:%
%:%4600=2152%:%
%:%4601=2152%:%
%:%4602=2153%:%
%:%4603=2153%:%
%:%4604=2153%:%
%:%4605=2154%:%
%:%4606=2154%:%
%:%4607=2155%:%
%:%4608=2155%:%
%:%4609=2156%:%
%:%4610=2156%:%
%:%4611=2157%:%
%:%4612=2157%:%
%:%4613=2158%:%
%:%4614=2158%:%
%:%4615=2159%:%
%:%4616=2159%:%
%:%4617=2159%:%
%:%4618=2160%:%
%:%4619=2160%:%
%:%4620=2160%:%
%:%4621=2161%:%
%:%4622=2161%:%
%:%4623=2162%:%
%:%4624=2162%:%
%:%4625=2163%:%
%:%4626=2163%:%
%:%4627=2164%:%
%:%4628=2164%:%
%:%4629=2165%:%
%:%4630=2165%:%
%:%4631=2166%:%
%:%4632=2166%:%
%:%4633=2167%:%
%:%4634=2167%:%
%:%4635=2167%:%
%:%4636=2168%:%
%:%4637=2168%:%
%:%4638=2168%:%
%:%4639=2169%:%
%:%4640=2169%:%
%:%4641=2170%:%
%:%4642=2170%:%
%:%4643=2171%:%
%:%4644=2171%:%
%:%4645=2172%:%
%:%4646=2172%:%
%:%4647=2173%:%
%:%4648=2173%:%
%:%4650=2175%:%
%:%4651=2176%:%
%:%4652=2176%:%
%:%4653=2177%:%
%:%4654=2177%:%
%:%4655=2178%:%
%:%4656=2178%:%
%:%4657=2179%:%
%:%4658=2179%:%
%:%4659=2180%:%
%:%4660=2180%:%
%:%4661=2180%:%
%:%4662=2181%:%
%:%4663=2181%:%
%:%4664=2182%:%
%:%4665=2182%:%
%:%4666=2182%:%
%:%4667=2183%:%
%:%4668=2183%:%
%:%4669=2184%:%
%:%4670=2184%:%
%:%4671=2184%:%
%:%4672=2185%:%
%:%4673=2185%:%
%:%4674=2186%:%
%:%4675=2186%:%
%:%4676=2186%:%
%:%4677=2187%:%
%:%4678=2187%:%
%:%4679=2188%:%
%:%4680=2188%:%
%:%4681=2188%:%
%:%4682=2189%:%
%:%4683=2189%:%
%:%4684=2190%:%
%:%4685=2190%:%
%:%4686=2191%:%
%:%4687=2191%:%
%:%4688=2192%:%
%:%4689=2192%:%
%:%4690=2193%:%
%:%4691=2193%:%
%:%4692=2194%:%
%:%4693=2194%:%
%:%4694=2194%:%
%:%4695=2195%:%
%:%4696=2195%:%
%:%4697=2195%:%
%:%4698=2196%:%
%:%4699=2196%:%
%:%4700=2196%:%
%:%4701=2197%:%
%:%4702=2197%:%
%:%4703=2198%:%
%:%4704=2198%:%
%:%4705=2199%:%
%:%4706=2199%:%
%:%4707=2200%:%
%:%4708=2200%:%
%:%4709=2201%:%
%:%4710=2201%:%
%:%4711=2202%:%
%:%4712=2202%:%
%:%4713=2203%:%
%:%4714=2203%:%
%:%4715=2204%:%
%:%4716=2204%:%
%:%4717=2204%:%
%:%4718=2205%:%
%:%4719=2205%:%
%:%4720=2205%:%
%:%4721=2206%:%
%:%4722=2206%:%
%:%4723=2206%:%
%:%4724=2207%:%
%:%4725=2207%:%
%:%4726=2208%:%
%:%4727=2208%:%
%:%4728=2209%:%
%:%4729=2209%:%
%:%4730=2210%:%
%:%4731=2210%:%
%:%4732=2211%:%
%:%4733=2211%:%
%:%4734=2212%:%
%:%4735=2212%:%
%:%4736=2213%:%
%:%4737=2214%:%
%:%4738=2214%:%
%:%4739=2215%:%
%:%4740=2215%:%
%:%4741=2216%:%
%:%4742=2216%:%
%:%4743=2217%:%
%:%4744=2217%:%
%:%4745=2218%:%
%:%4746=2218%:%
%:%4747=2218%:%
%:%4748=2219%:%
%:%4749=2219%:%
%:%4750=2220%:%
%:%4751=2220%:%
%:%4752=2220%:%
%:%4753=2221%:%
%:%4754=2221%:%
%:%4755=2222%:%
%:%4756=2222%:%
%:%4757=2222%:%
%:%4758=2223%:%
%:%4759=2223%:%
%:%4760=2224%:%
%:%4761=2224%:%
%:%4762=2224%:%
%:%4763=2225%:%
%:%4764=2225%:%
%:%4765=2226%:%
%:%4766=2226%:%
%:%4767=2226%:%
%:%4768=2227%:%
%:%4769=2227%:%
%:%4770=2228%:%
%:%4771=2228%:%
%:%4772=2229%:%
%:%4773=2229%:%
%:%4774=2230%:%
%:%4775=2230%:%
%:%4776=2231%:%
%:%4777=2231%:%
%:%4778=2232%:%
%:%4779=2232%:%
%:%4780=2232%:%
%:%4781=2233%:%
%:%4782=2233%:%
%:%4783=2233%:%
%:%4784=2234%:%
%:%4785=2234%:%
%:%4786=2234%:%
%:%4787=2235%:%
%:%4788=2235%:%
%:%4789=2236%:%
%:%4790=2236%:%
%:%4791=2237%:%
%:%4792=2237%:%
%:%4793=2238%:%
%:%4794=2238%:%
%:%4795=2239%:%
%:%4796=2239%:%
%:%4797=2240%:%
%:%4798=2240%:%
%:%4799=2241%:%
%:%4800=2241%:%
%:%4801=2242%:%
%:%4802=2242%:%
%:%4803=2242%:%
%:%4804=2243%:%
%:%4805=2243%:%
%:%4806=2243%:%
%:%4807=2244%:%
%:%4808=2244%:%
%:%4809=2244%:%
%:%4810=2245%:%
%:%4811=2245%:%
%:%4812=2246%:%
%:%4813=2246%:%
%:%4814=2247%:%
%:%4815=2247%:%
%:%4816=2248%:%
%:%4817=2248%:%
%:%4818=2249%:%
%:%4819=2249%:%
%:%4820=2250%:%
%:%4821=2250%:%
%:%4822=2251%:%
%:%4823=2251%:%
%:%4824=2251%:%
%:%4825=2252%:%
%:%4826=2252%:%
%:%4827=2253%:%
%:%4828=2253%:%
%:%4829=2254%:%
%:%4830=2254%:%
%:%4831=2254%:%
%:%4832=2255%:%
%:%4833=2255%:%
%:%4834=2256%:%
%:%4835=2256%:%
%:%4836=2256%:%
%:%4837=2257%:%
%:%4838=2257%:%
%:%4839=2258%:%
%:%4840=2258%:%
%:%4841=2259%:%
%:%4842=2259%:%
%:%4843=2260%:%
%:%4844=2260%:%
%:%4845=2261%:%
%:%4846=2261%:%
%:%4847=2262%:%
%:%4848=2262%:%
%:%4849=2262%:%
%:%4850=2263%:%
%:%4851=2263%:%
%:%4852=2263%:%
%:%4853=2264%:%
%:%4854=2264%:%
%:%4855=2265%:%
%:%4856=2265%:%
%:%4857=2266%:%
%:%4858=2266%:%
%:%4859=2267%:%
%:%4860=2267%:%
%:%4861=2268%:%
%:%4862=2268%:%
%:%4863=2269%:%
%:%4864=2269%:%
%:%4865=2270%:%
%:%4866=2270%:%
%:%4867=2270%:%
%:%4868=2271%:%
%:%4869=2271%:%
%:%4870=2271%:%
%:%4871=2272%:%
%:%4872=2272%:%
%:%4873=2273%:%
%:%4874=2273%:%
%:%4875=2274%:%
%:%4876=2274%:%
%:%4877=2275%:%
%:%4878=2275%:%
%:%4879=2276%:%
%:%4880=2276%:%
%:%4881=2277%:%
%:%4882=2277%:%
%:%4883=2278%:%
%:%4889=2278%:%
%:%4892=2279%:%
%:%4893=2280%:%
%:%4894=2280%:%
%:%4895=2281%:%
%:%4896=2282%:%
%:%4897=2283%:%
%:%4904=2284%:%
%:%4905=2284%:%
%:%4906=2285%:%
%:%4907=2285%:%
%:%4908=2286%:%
%:%4909=2286%:%
%:%4910=2286%:%
%:%4911=2287%:%
%:%4912=2287%:%
%:%4913=2287%:%
%:%4914=2288%:%
%:%4915=2288%:%
%:%4916=2288%:%
%:%4917=2289%:%
%:%4918=2289%:%
%:%4919=2289%:%
%:%4920=2290%:%
%:%4921=2290%:%
%:%4922=2291%:%
%:%4923=2291%:%
%:%4924=2292%:%
%:%4925=2292%:%
%:%4926=2293%:%
%:%4927=2293%:%
%:%4928=2294%:%
%:%4929=2294%:%
%:%4930=2295%:%
%:%4931=2295%:%
%:%4932=2295%:%
%:%4933=2296%:%
%:%4934=2296%:%
%:%4935=2297%:%
%:%4936=2297%:%
%:%4937=2298%:%
%:%4938=2298%:%
%:%4939=2299%:%
%:%4940=2299%:%
%:%4941=2300%:%
%:%4942=2300%:%
%:%4943=2301%:%
%:%4949=2301%:%
%:%4952=2302%:%
%:%4953=2303%:%
%:%4954=2303%:%
%:%4955=2304%:%
%:%4956=2305%:%
%:%4957=2306%:%
%:%4964=2307%:%
%:%4965=2307%:%
%:%4966=2308%:%
%:%4967=2308%:%
%:%4968=2309%:%
%:%4969=2309%:%
%:%4970=2309%:%
%:%4971=2309%:%
%:%4972=2310%:%
%:%4973=2310%:%
%:%4974=2311%:%
%:%4975=2311%:%
%:%4976=2311%:%
%:%4977=2311%:%
%:%4978=2312%:%
%:%4979=2312%:%
%:%4980=2312%:%
%:%4981=2313%:%
%:%4982=2313%:%
%:%4983=2314%:%
%:%4984=2314%:%
%:%4985=2314%:%
%:%4986=2315%:%
%:%4987=2315%:%
%:%4988=2316%:%
%:%4989=2316%:%
%:%4990=2317%:%
%:%4991=2317%:%
%:%4992=2317%:%
%:%4993=2318%:%
%:%4994=2318%:%
%:%4995=2319%:%
%:%4996=2319%:%
%:%4997=2319%:%
%:%4998=2320%:%
%:%4999=2320%:%
%:%5000=2321%:%
%:%5001=2321%:%
%:%5002=2321%:%
%:%5003=2322%:%
%:%5004=2322%:%
%:%5005=2323%:%
%:%5006=2323%:%
%:%5007=2323%:%
%:%5008=2324%:%
%:%5009=2324%:%
%:%5010=2325%:%
%:%5011=2325%:%
%:%5012=2325%:%
%:%5013=2326%:%
%:%5014=2326%:%
%:%5015=2327%:%
%:%5016=2327%:%
%:%5017=2327%:%
%:%5018=2328%:%
%:%5019=2328%:%
%:%5020=2329%:%
%:%5021=2329%:%
%:%5022=2329%:%
%:%5023=2330%:%
%:%5024=2330%:%
%:%5025=2331%:%
%:%5026=2331%:%
%:%5027=2332%:%
%:%5028=2332%:%
%:%5029=2333%:%
%:%5030=2333%:%
%:%5031=2333%:%
%:%5032=2334%:%
%:%5033=2334%:%
%:%5034=2334%:%
%:%5035=2335%:%
%:%5036=2335%:%
%:%5037=2336%:%
%:%5038=2336%:%
%:%5039=2336%:%
%:%5040=2337%:%
%:%5041=2337%:%
%:%5042=2338%:%
%:%5043=2338%:%
%:%5044=2338%:%
%:%5045=2339%:%
%:%5046=2339%:%
%:%5047=2340%:%
%:%5048=2340%:%
%:%5049=2341%:%
%:%5050=2341%:%
%:%5051=2341%:%
%:%5052=2342%:%
%:%5053=2342%:%
%:%5054=2342%:%
%:%5055=2343%:%
%:%5056=2343%:%
%:%5057=2344%:%
%:%5058=2344%:%
%:%5059=2344%:%
%:%5060=2345%:%
%:%5061=2345%:%
%:%5062=2345%:%
%:%5063=2346%:%
%:%5064=2346%:%
%:%5065=2346%:%
%:%5066=2347%:%
%:%5067=2347%:%
%:%5068=2348%:%
%:%5069=2348%:%
%:%5070=2348%:%
%:%5071=2349%:%
%:%5072=2349%:%
%:%5073=2350%:%
%:%5079=2350%:%
%:%5082=2351%:%
%:%5083=2352%:%
%:%5084=2352%:%
%:%5085=2353%:%
%:%5086=2354%:%
%:%5093=2355%:%
%:%5094=2355%:%
%:%5095=2356%:%
%:%5096=2356%:%
%:%5097=2357%:%
%:%5098=2357%:%
%:%5099=2358%:%
%:%5100=2358%:%
%:%5101=2359%:%
%:%5102=2359%:%
%:%5103=2360%:%
%:%5104=2360%:%
%:%5105=2361%:%
%:%5106=2361%:%
%:%5107=2362%:%
%:%5108=2362%:%
%:%5109=2363%:%
%:%5110=2363%:%
%:%5111=2364%:%
%:%5112=2364%:%
%:%5113=2364%:%
%:%5114=2365%:%
%:%5115=2365%:%
%:%5116=2366%:%
%:%5117=2366%:%
%:%5118=2366%:%
%:%5119=2367%:%
%:%5120=2367%:%
%:%5121=2368%:%
%:%5122=2368%:%
%:%5123=2369%:%
%:%5124=2369%:%
%:%5125=2369%:%
%:%5126=2370%:%
%:%5127=2370%:%
%:%5128=2371%:%
%:%5129=2371%:%
%:%5130=2371%:%
%:%5131=2372%:%
%:%5132=2372%:%
%:%5133=2373%:%
%:%5134=2373%:%
%:%5135=2373%:%
%:%5136=2374%:%
%:%5137=2374%:%
%:%5138=2375%:%
%:%5139=2375%:%
%:%5140=2375%:%
%:%5141=2376%:%
%:%5142=2376%:%
%:%5143=2377%:%
%:%5144=2377%:%
%:%5145=2377%:%
%:%5146=2378%:%
%:%5147=2378%:%
%:%5148=2379%:%
%:%5149=2379%:%
%:%5150=2379%:%
%:%5151=2380%:%
%:%5152=2380%:%
%:%5153=2381%:%
%:%5154=2381%:%
%:%5155=2381%:%
%:%5156=2382%:%
%:%5157=2382%:%
%:%5158=2383%:%
%:%5159=2383%:%
%:%5160=2383%:%
%:%5161=2384%:%
%:%5162=2384%:%
%:%5163=2385%:%
%:%5164=2385%:%
%:%5165=2385%:%
%:%5166=2386%:%
%:%5167=2386%:%
%:%5168=2387%:%
%:%5169=2387%:%
%:%5170=2388%:%
%:%5171=2388%:%
%:%5172=2388%:%
%:%5173=2389%:%
%:%5174=2389%:%
%:%5175=2390%:%
%:%5176=2390%:%
%:%5177=2390%:%
%:%5178=2391%:%
%:%5179=2391%:%
%:%5180=2392%:%
%:%5181=2392%:%
%:%5182=2393%:%
%:%5183=2393%:%
%:%5184=2393%:%
%:%5185=2394%:%
%:%5186=2394%:%
%:%5187=2395%:%
%:%5188=2395%:%
%:%5189=2396%:%
%:%5190=2396%:%
%:%5191=2396%:%
%:%5192=2397%:%
%:%5193=2397%:%
%:%5194=2398%:%
%:%5195=2398%:%
%:%5196=2399%:%
%:%5197=2399%:%
%:%5198=2399%:%
%:%5199=2400%:%
%:%5200=2400%:%
%:%5201=2401%:%
%:%5202=2401%:%
%:%5203=2401%:%
%:%5204=2402%:%
%:%5205=2402%:%
%:%5206=2403%:%
%:%5207=2403%:%
%:%5208=2404%:%
%:%5209=2404%:%
%:%5210=2404%:%
%:%5211=2405%:%
%:%5212=2405%:%
%:%5213=2406%:%
%:%5214=2406%:%
%:%5215=2406%:%
%:%5216=2407%:%
%:%5217=2407%:%
%:%5218=2408%:%
%:%5219=2408%:%
%:%5220=2408%:%
%:%5221=2409%:%
%:%5222=2409%:%
%:%5223=2409%:%
%:%5224=2410%:%
%:%5225=2410%:%
%:%5226=2411%:%
%:%5227=2411%:%
%:%5228=2411%:%
%:%5229=2412%:%
%:%5230=2412%:%
%:%5231=2413%:%
%:%5232=2413%:%
%:%5233=2414%:%
%:%5234=2414%:%
%:%5235=2414%:%
%:%5236=2415%:%
%:%5237=2415%:%
%:%5238=2416%:%
%:%5239=2416%:%
%:%5240=2417%:%
%:%5246=2417%:%
%:%5249=2418%:%
%:%5250=2419%:%
%:%5251=2419%:%
%:%5258=2420%:%
%:%5259=2420%:%
%:%5260=2421%:%
%:%5261=2421%:%
%:%5264=2424%:%
%:%5265=2425%:%
%:%5266=2425%:%
%:%5267=2426%:%
%:%5268=2426%:%
%:%5269=2427%:%
%:%5270=2427%:%
%:%5271=2428%:%
%:%5272=2428%:%
%:%5273=2429%:%
%:%5274=2429%:%
%:%5275=2429%:%
%:%5276=2430%:%
%:%5277=2430%:%
%:%5278=2431%:%
%:%5279=2431%:%
%:%5280=2431%:%
%:%5281=2432%:%
%:%5282=2432%:%
%:%5283=2433%:%
%:%5284=2433%:%
%:%5285=2433%:%
%:%5286=2434%:%
%:%5287=2434%:%
%:%5288=2435%:%
%:%5289=2435%:%
%:%5290=2435%:%
%:%5291=2436%:%
%:%5292=2436%:%
%:%5295=2439%:%
%:%5296=2440%:%
%:%5297=2440%:%
%:%5298=2440%:%
%:%5299=2441%:%
%:%5300=2441%:%
%:%5301=2442%:%
%:%5307=2442%:%
%:%5310=2443%:%
%:%5311=2444%:%
%:%5312=2444%:%
%:%5313=2445%:%
%:%5314=2446%:%
%:%5315=2447%:%
%:%5322=2448%:%
%:%5323=2448%:%
%:%5324=2449%:%
%:%5325=2449%:%
%:%5326=2450%:%
%:%5327=2450%:%
%:%5328=2451%:%
%:%5329=2451%:%
%:%5330=2452%:%
%:%5331=2452%:%
%:%5332=2452%:%
%:%5333=2453%:%
%:%5334=2453%:%
%:%5335=2454%:%
%:%5336=2454%:%
%:%5337=2454%:%
%:%5338=2455%:%
%:%5339=2455%:%

%
\begin{isabellebody}%
\setisabellecontext{Glosario}%
%
\isadelimtheory
%
\endisadelimtheory
%
\isatagtheory
%
\endisatagtheory
{\isafoldtheory}%
%
\isadelimtheory
%
\endisadelimtheory
%
\begin{isamarkuptext}%
En este glosario se recoge la lista de los lemas y reglas usadas
  indicando la página del \href{https://bit.ly/2KvG2ej}{libro de HOL} 
  donde se encuentran.%
\end{isamarkuptext}\isamarkuptrue%
%
\isadelimdocument
%
\endisadelimdocument
%
\isatagdocument
%
\isamarkupsection{La base de lógica de primer orden (2)%
}
\isamarkuptrue%
%
\endisatagdocument
{\isafolddocument}%
%
\isadelimdocument
%
\endisadelimdocument
%
\begin{isamarkuptext}%
En Isabelle corresponde a la teoría 
  \href{https://bit.ly/3bGva9s}{HOL.thy}%
\end{isamarkuptext}\isamarkuptrue%
%
\isadelimdocument
%
\endisadelimdocument
%
\isatagdocument
%
\isamarkupsubsection{Lógica primitiva (2.1)%
}
\isamarkuptrue%
%
\isamarkupsubsubsection{Conectivas y cuantificadores definidos (2.1.2)%
}
\isamarkuptrue%
%
\endisatagdocument
{\isafolddocument}%
%
\isadelimdocument
%
\endisadelimdocument
%
\begin{isamarkuptext}%
\begin{itemize}
    \item (p.34) \isa{{\isasymnot}\ P\ {\isasymequiv}\ P\ {\isasymlongrightarrow}\ False} 
      \hfill (\isa{not{\isacharunderscore}def})
  \end{itemize}%
\end{isamarkuptext}\isamarkuptrue%
%
\isadelimdocument
%
\endisadelimdocument
%
\isatagdocument
%
\isamarkupsubsubsection{Axiomas y definiciones básicas (2.1.4)%
}
\isamarkuptrue%
%
\endisatagdocument
{\isafolddocument}%
%
\isadelimdocument
%
\endisadelimdocument
%
\begin{isamarkuptext}%
\begin{itemize}
    \item (p.36) \isa{\mbox{}\inferrule{\mbox{\mbox{}\inferrule{\mbox{P}}{\mbox{Q}}}}{\mbox{P\ {\isasymlongrightarrow}\ Q}}} 
      \hfill (\isa{impI})
    \item (p.36) \isa{\mbox{}\inferrule{\mbox{{\isacharparenleft}P\ {\isasymlongrightarrow}\ Q{\isacharparenright}\ {\isasymand}\ P}}{\mbox{Q}}} 
      \hfill (\isa{mp})
  \end{itemize}%
\end{isamarkuptext}\isamarkuptrue%
%
\isadelimdocument
%
\endisadelimdocument
%
\isatagdocument
%
\isamarkupsubsection{Reglas fundamentales (2.2)%
}
\isamarkuptrue%
%
\isamarkupsubsubsection{Reglas de congruencia para aplicaciones (2.2.2)%
}
\isamarkuptrue%
%
\endisatagdocument
{\isafolddocument}%
%
\isadelimdocument
%
\endisadelimdocument
%
\begin{isamarkuptext}%
\begin{itemize}
    \item (p.37) \isa{\mbox{}\inferrule{\mbox{x\ {\isacharequal}\ y}}{\mbox{f\ x\ {\isacharequal}\ f\ y}}} 
      \hfill (\isa{arg{\isacharunderscore}cong})
    \item (p.37) \isa{\mbox{}\inferrule{\mbox{a\ {\isacharequal}\ b\ {\isasymand}\ c\ {\isacharequal}\ d}}{\mbox{f\ a\ c\ {\isacharequal}\ f\ b\ d}}} 
      \hfill (\isa{arg{\isacharunderscore}cong{\isadigit{2}}})
  \end{itemize}%
\end{isamarkuptext}\isamarkuptrue%
%
\isadelimdocument
%
\endisadelimdocument
%
\isatagdocument
%
\isamarkupsubsubsection{Igualdad de booleanos - \isa{iff} (2.2.3)%
}
\isamarkuptrue%
%
\endisatagdocument
{\isafolddocument}%
%
\isadelimdocument
%
\endisadelimdocument
%
\begin{isamarkuptext}%
\begin{itemize}
    \item (p.38) \isa{\mbox{}\inferrule{\mbox{Q\ {\isacharequal}\ P\ {\isasymand}\ Q}}{\mbox{P}}} 
      \hfill (\isa{iffD{\isadigit{1}}})
  \end{itemize}%
\end{isamarkuptext}\isamarkuptrue%
%
\isadelimdocument
%
\endisadelimdocument
%
\isatagdocument
%
\isamarkupsubsubsection{Cuantificador universal I (2.2.5)%
}
\isamarkuptrue%
%
\endisatagdocument
{\isafolddocument}%
%
\isadelimdocument
%
\endisadelimdocument
%
\begin{isamarkuptext}%
\begin{itemize}
    \item (p.38) \isa{\mbox{}\inferrule{\mbox{{\isasymforall}x{\isachardot}\ P\ x}\\\ \mbox{\mbox{}\inferrule{\mbox{P\ x}}{\mbox{R}}}}{\mbox{R}}} 
      \hfill (\isa{allE})
  \end{itemize}%
\end{isamarkuptext}\isamarkuptrue%
%
\isadelimdocument
%
\endisadelimdocument
%
\isatagdocument
%
\isamarkupsubsubsection{Negación (2.2.7)%
}
\isamarkuptrue%
%
\endisatagdocument
{\isafolddocument}%
%
\isadelimdocument
%
\endisadelimdocument
%
\begin{isamarkuptext}%
\begin{itemize}
    \item (p.39) \isa{\mbox{}\inferrule{\mbox{\mbox{}\inferrule{\mbox{P}}{\mbox{False}}}}{\mbox{{\isasymnot}\ P}}} 
      \hfill (\isa{notI})
    \item (p.39) \isa{\mbox{}\inferrule{\mbox{{\isasymnot}\ P\ {\isasymand}\ P}}{\mbox{R}}} 
      \hfill (\isa{notE})
  \end{itemize}%
\end{isamarkuptext}\isamarkuptrue%
%
\isadelimdocument
%
\endisadelimdocument
%
\isatagdocument
%
\isamarkupsubsubsection{Implicación (2.2.8)%
}
\isamarkuptrue%
%
\endisatagdocument
{\isafolddocument}%
%
\isadelimdocument
%
\endisadelimdocument
%
\begin{isamarkuptext}%
\begin{itemize}
    \item (p.40) \isa{\mbox{}\inferrule{\mbox{Q}\\\ \mbox{\mbox{}\inferrule{\mbox{P}}{\mbox{{\isasymnot}\ Q}}}}{\mbox{{\isasymnot}\ P}}} 
      \hfill (\isa{contrapos{\isacharunderscore}pn})
  \end{itemize}%
\end{isamarkuptext}\isamarkuptrue%
%
\isadelimdocument
%
\endisadelimdocument
%
\isatagdocument
%
\isamarkupsubsubsection{Disyunción I (2.2.9)%
}
\isamarkuptrue%
%
\endisatagdocument
{\isafolddocument}%
%
\isadelimdocument
%
\endisadelimdocument
%
\begin{isamarkuptext}%
\begin{itemize}
    \item (p.40) \isa{\mbox{}\inferrule{\mbox{P\ {\isasymor}\ Q}\\\ \mbox{\mbox{}\inferrule{\mbox{P}}{\mbox{R}}}\\\ \mbox{\mbox{}\inferrule{\mbox{Q}}{\mbox{R}}}}{\mbox{R}}} 
      \hfill (\isa{disjE})
  \end{itemize}%
\end{isamarkuptext}\isamarkuptrue%
%
\isadelimdocument
%
\endisadelimdocument
%
\isatagdocument
%
\isamarkupsubsubsection{Derivación de \isa{iffI} (2.2.10)%
}
\isamarkuptrue%
%
\endisatagdocument
{\isafolddocument}%
%
\isadelimdocument
%
\endisadelimdocument
%
\begin{isamarkuptext}%
\begin{itemize}
    \item (p.40) \isa{\mbox{}\inferrule{\mbox{\mbox{}\inferrule{\mbox{P}}{\mbox{Q}}}\\\ \mbox{\mbox{}\inferrule{\mbox{Q}}{\mbox{P}}}}{\mbox{P\ {\isacharequal}\ Q}}} 
      \hfill (\isa{iffI})
  \end{itemize}%
\end{isamarkuptext}\isamarkuptrue%
%
\isadelimdocument
%
\endisadelimdocument
%
\isatagdocument
%
\isamarkupsubsubsection{Cuantificador universal II (2.2.12)%
}
\isamarkuptrue%
%
\endisatagdocument
{\isafolddocument}%
%
\isadelimdocument
%
\endisadelimdocument
%
\begin{isamarkuptext}%
\begin{itemize}
    \item (p.41) \isa{\mbox{}\inferrule{\mbox{{\isasymAnd}x{\isachardot}\ P\ x}}{\mbox{{\isasymforall}x{\isachardot}\ P\ x}}} 
      \hfill (\isa{allI})
  \end{itemize}%
\end{isamarkuptext}\isamarkuptrue%
%
\isadelimdocument
%
\endisadelimdocument
%
\isatagdocument
%
\isamarkupsubsubsection{Cuantificador existencia (2.2.13)%
}
\isamarkuptrue%
%
\endisatagdocument
{\isafolddocument}%
%
\isadelimdocument
%
\endisadelimdocument
%
\begin{isamarkuptext}%
\begin{itemize}
    \item (p.41) \isa{\mbox{}\inferrule{\mbox{P\ x}}{\mbox{{\isasymexists}x{\isachardot}\ P\ x}}} 
      \hfill (\isa{exI})
    \item (p.41) \isa{\mbox{}\inferrule{\mbox{{\isasymexists}x{\isachardot}\ P\ x}\\\ \mbox{{\isasymAnd}x{\isachardot}\ \mbox{}\inferrule{\mbox{P\ x}}{\mbox{Q}}}}{\mbox{Q}}} 
      \hfill (\isa{exE})
  \end{itemize}%
\end{isamarkuptext}\isamarkuptrue%
%
\isadelimdocument
%
\endisadelimdocument
%
\isatagdocument
%
\isamarkupsubsubsection{Conjunción (2.2.14)%
}
\isamarkuptrue%
%
\endisatagdocument
{\isafolddocument}%
%
\isadelimdocument
%
\endisadelimdocument
%
\begin{isamarkuptext}%
\begin{itemize}
    \item (p.41) \isa{\mbox{}\inferrule{\mbox{P\ {\isasymand}\ Q}}{\mbox{P\ {\isasymand}\ Q}}} 
      \hfill (\isa{conjI})
    \item (p.41) \isa{\mbox{}\inferrule{\mbox{P\ {\isasymand}\ Q}}{\mbox{P}}} 
      \hfill (\isa{conjunct{\isadigit{1}}})
    \item (p.41) \isa{\mbox{}\inferrule{\mbox{P\ {\isasymand}\ Q}}{\mbox{Q}}} 
      \hfill (\isa{conjunct{\isadigit{2}}})
  \end{itemize}%
\end{isamarkuptext}\isamarkuptrue%
%
\isadelimdocument
%
\endisadelimdocument
%
\isatagdocument
%
\isamarkupsubsubsection{Disyunción II (2.2.15)%
}
\isamarkuptrue%
%
\endisatagdocument
{\isafolddocument}%
%
\isadelimdocument
%
\endisadelimdocument
%
\begin{isamarkuptext}%
\begin{itemize}
    \item (p.42) \isa{\mbox{}\inferrule{\mbox{P}}{\mbox{P\ {\isasymor}\ Q}}} 
      \hfill (\isa{disjI{\isadigit{1}}})
    \item (p.42) \isa{\mbox{}\inferrule{\mbox{Q}}{\mbox{P\ {\isasymor}\ Q}}} 
      \hfill (\isa{disjI{\isadigit{2}}})
  \end{itemize}%
\end{isamarkuptext}\isamarkuptrue%
%
\isadelimdocument
%
\endisadelimdocument
%
\isatagdocument
%
\isamarkupsubsubsection{Atomización de conectivas de nivel intermedio (2.2.20)%
}
\isamarkuptrue%
%
\endisatagdocument
{\isafolddocument}%
%
\isadelimdocument
%
\endisadelimdocument
%
\begin{isamarkuptext}%
\begin{itemize}
    \item (p.46) \isa{{\isacharparenleft}x\ {\isasymequiv}\ y{\isacharparenright}\ {\isasymequiv}\ x\ {\isacharequal}\ y} 
      \hfill (\isa{atomize{\isacharunderscore}eq})
  \end{itemize}%
\end{isamarkuptext}\isamarkuptrue%
%
\isadelimdocument
%
\endisadelimdocument
%
\isatagdocument
%
\isamarkupsubsection{Configuración del paquete (2.3)%
}
\isamarkuptrue%
%
\isamarkupsubsubsection{Simplificadores (2.3.4)%
}
\isamarkuptrue%
%
\endisatagdocument
{\isafolddocument}%
%
\isadelimdocument
%
\endisadelimdocument
%
\begin{isamarkuptext}%
\begin{itemize}
    \item (p.50) \isa{{\isacharparenleft}{\isasymnot}\ False{\isacharparenright}\ {\isacharequal}\ True} 
      \hfill (\isa{not{\isacharunderscore}False{\isacharunderscore}eq{\isacharunderscore}True})
    \item (p.53) \isa{{\isacharparenleft}{\isasymnexists}x{\isachardot}\ P\ x{\isacharparenright}\ {\isacharequal}\ {\isacharparenleft}{\isasymforall}x{\isachardot}\ {\isasymnot}\ P\ x{\isacharparenright}} 
      \hfill (\isa{not{\isacharunderscore}ex})
  \end{itemize}%
\end{isamarkuptext}\isamarkuptrue%
%
\isadelimdocument
%
\endisadelimdocument
%
\isatagdocument
%
\isamarkupsection{Grupos, también combinados con órdenes (5)%
}
\isamarkuptrue%
%
\endisatagdocument
{\isafolddocument}%
%
\isadelimdocument
%
\endisadelimdocument
%
\begin{isamarkuptext}%
Los siguientes resultados pertenecen a la teoría de 
  grupos \href{https://bit.ly/3fwjIPe}{Groups.thy}.%
\end{isamarkuptext}\isamarkuptrue%
%
\isadelimdocument
%
\endisadelimdocument
%
\isatagdocument
%
\isamarkupsubsection{Estructuras abstractas%
}
\isamarkuptrue%
%
\endisatagdocument
{\isafolddocument}%
%
\isadelimdocument
%
\endisadelimdocument
%
\begin{isamarkuptext}%
\begin{itemize}
    \item (p.109) \isa{sup\ bot\ a\ {\isacharequal}\ a} 
      \hfill (\isa{sup{\isacharunderscore}bot{\isachardot}left{\isacharunderscore}neutral})
  \end{itemize}%
\end{isamarkuptext}\isamarkuptrue%
%
\isadelimdocument
%
\endisadelimdocument
%
\isatagdocument
%
\isamarkupsection{Retículos abstractos (6)%
}
\isamarkuptrue%
%
\endisatagdocument
{\isafolddocument}%
%
\isadelimdocument
%
\endisadelimdocument
%
\begin{isamarkuptext}%
Los resultados expuestos a continuación pertenecen a la teoría de 
  retículos \href{https://bit.ly/2N4lbjn}{Lattices.thy}.%
\end{isamarkuptext}\isamarkuptrue%
%
\begin{isamarkuptext}%
\begin{itemize}
    \item (p.139) \isa{{\isacharparenleft}sup\ b\ c\ {\isasymle}\ a{\isacharparenright}\ {\isacharequal}\ {\isacharparenleft}b\ {\isasymle}\ a\ {\isasymand}\ c\ {\isasymle}\ a{\isacharparenright}} 
      \hfill (\isa{sup{\isachardot}bounded{\isacharunderscore}iff})
  \end{itemize}%
\end{isamarkuptext}\isamarkuptrue%
%
\isadelimdocument
%
\endisadelimdocument
%
\isatagdocument
%
\isamarkupsection{Teoría de conjuntos para lógica de orden superior (7)%
}
\isamarkuptrue%
%
\endisatagdocument
{\isafolddocument}%
%
\isadelimdocument
%
\endisadelimdocument
%
\begin{isamarkuptext}%
Los siguientes resultados corresponden a la teoría de conjuntos 
  \href{https://bit.ly/3ePFv4B}{Set.thy}.%
\end{isamarkuptext}\isamarkuptrue%
%
\isadelimdocument
%
\endisadelimdocument
%
\isatagdocument
%
\isamarkupsubsection{Subconjuntos y cuantificadores acotados (7.2)%
}
\isamarkuptrue%
%
\endisatagdocument
{\isafolddocument}%
%
\isadelimdocument
%
\endisadelimdocument
%
\begin{isamarkuptext}%
\begin{itemize}
    \item (p.163) \isa{\mbox{}\inferrule{\mbox{{\isasymAnd}x{\isachardot}\ \mbox{}\inferrule{\mbox{x\ {\isasymin}\ A}}{\mbox{P\ x}}}}{\mbox{{\isasymforall}x{\isasymin}A{\isachardot}\ P\ x}}} 
      \hfill (\isa{ballI})
    \item (p.163) \isa{\mbox{}\inferrule{\mbox{{\isacharparenleft}{\isasymforall}x{\isasymin}A{\isachardot}\ P\ x{\isacharparenright}\ {\isasymand}\ x\ {\isasymin}\ A}}{\mbox{P\ x}}} 
      \hfill (\isa{bspec})
  \end{itemize}%
\end{isamarkuptext}\isamarkuptrue%
%
\isadelimdocument
%
\endisadelimdocument
%
\isatagdocument
%
\isamarkupsubsection{Operaciones básicas (7.3)%
}
\isamarkuptrue%
%
\isamarkupsubsubsection{Subconjuntos (7.3.1)%
}
\isamarkuptrue%
%
\endisatagdocument
{\isafolddocument}%
%
\isadelimdocument
%
\endisadelimdocument
%
\begin{isamarkuptext}%
\begin{itemize}
    \item (p.165) \isa{\mbox{}\inferrule{\mbox{c\ {\isasymin}\ A\ {\isasymand}\ A\ {\isasymsubseteq}\ B}}{\mbox{c\ {\isasymin}\ B}}} 
      \hfill (\isa{rev{\isacharunderscore}subsetD})
    \item (p.166) \isa{A\ {\isasymsubseteq}\ A} 
      \hfill (\isa{subset{\isacharunderscore}refl})
    \item (p.166) \isa{\mbox{}\inferrule{\mbox{A\ {\isasymsubseteq}\ B\ {\isasymand}\ B\ {\isasymsubseteq}\ C}}{\mbox{A\ {\isasymsubseteq}\ C}}} 
      \hfill (\isa{subset{\isacharunderscore}trans})
  \end{itemize}%
\end{isamarkuptext}\isamarkuptrue%
%
\isadelimdocument
%
\endisadelimdocument
%
\isatagdocument
%
\isamarkupsubsubsection{El conjunto vacío (7.3.3)%
}
\isamarkuptrue%
%
\endisatagdocument
{\isafolddocument}%
%
\isadelimdocument
%
\endisadelimdocument
%
\begin{isamarkuptext}%
\begin{itemize}
    \item (p.167) \isa{{\isasymemptyset}\ {\isasymsubseteq}\ A} 
      \hfill (\isa{empty{\isacharunderscore}subsetI})
    \item (p.167) \isa{Ball\ {\isasymemptyset}\ P\ {\isacharequal}\ True} 
      \hfill (\isa{ball{\isacharunderscore}empty})
    \item (p.167) \isa{Bex\ {\isasymemptyset}\ P\ {\isacharequal}\ False} 
      \hfill (\isa{bex{\isacharunderscore}empty})
  \end{itemize}%
\end{isamarkuptext}\isamarkuptrue%
%
\isadelimdocument
%
\endisadelimdocument
%
\isatagdocument
%
\isamarkupsubsubsection{Unión binaria (7.3.8)%
}
\isamarkuptrue%
%
\endisatagdocument
{\isafolddocument}%
%
\isadelimdocument
%
\endisadelimdocument
%
\begin{isamarkuptext}%
\begin{itemize}
    \item (p.169) \isa{{\isacharparenleft}c\ {\isasymin}\ A\ {\isasymunion}\ B{\isacharparenright}\ {\isacharequal}\ {\isacharparenleft}c\ {\isasymin}\ A\ {\isasymor}\ c\ {\isasymin}\ B{\isacharparenright}} 
      \hfill (\isa{Un{\isacharunderscore}iff})
    \item (p.169) \isa{\mbox{}\inferrule{\mbox{c\ {\isasymin}\ A}}{\mbox{c\ {\isasymin}\ A\ {\isasymunion}\ B}}} 
      \hfill (\isa{UnI{\isadigit{1}}})
    \item (p.170) \isa{\mbox{}\inferrule{\mbox{c\ {\isasymin}\ B}}{\mbox{c\ {\isasymin}\ A\ {\isasymunion}\ B}}} 
      \hfill (\isa{UnI{\isadigit{2}}})
  \end{itemize}%
\end{isamarkuptext}\isamarkuptrue%
%
\isadelimdocument
%
\endisadelimdocument
%
\isatagdocument
%
\isamarkupsubsubsection{Aumentar un conjunto - insertar (7.3.10)%
}
\isamarkuptrue%
%
\endisatagdocument
{\isafolddocument}%
%
\isadelimdocument
%
\endisadelimdocument
%
\begin{isamarkuptext}%
\begin{itemize}
    \item (p.171) \isa{List{\isachardot}insert\ x\ xs\ {\isacharequal}\ {\isacharbraceleft}x{\isacharbraceright}\ {\isasymunion}\ xs} 
      \hfill (\isa{set{\isacharunderscore}insert})
  \end{itemize}%
\end{isamarkuptext}\isamarkuptrue%
%
\isadelimdocument
%
\endisadelimdocument
%
\isatagdocument
%
\isamarkupsubsubsection{Conjuntos unitarios, insertar (7.3.11)%
}
\isamarkuptrue%
%
\endisatagdocument
{\isafolddocument}%
%
\isadelimdocument
%
\endisadelimdocument
%
\begin{isamarkuptext}%
\begin{itemize}
    \item (p.172) \isa{a\ {\isasymin}\ {\isacharbraceleft}a{\isacharbraceright}} 
      \hfill (\isa{singletonI})
    \item (p.172) \isa{\mbox{}\inferrule{\mbox{b\ {\isasymin}\ {\isacharbraceleft}a{\isacharbraceright}}}{\mbox{b\ {\isacharequal}\ a}}} 
      \hfill (\isa{singletonD})
    \item (p.172) \isa{{\isacharparenleft}b\ {\isasymin}\ {\isacharbraceleft}a{\isacharbraceright}{\isacharparenright}\ {\isacharequal}\ {\isacharparenleft}b\ {\isacharequal}\ a{\isacharparenright}} 
      \hfill (\isa{singleton{\isacharunderscore}iff})
  \end{itemize}%
\end{isamarkuptext}\isamarkuptrue%
%
\isadelimdocument
%
\endisadelimdocument
%
\isatagdocument
%
\isamarkupsubsubsection{Imagen de un conjunto por una función (7.3.12)%
}
\isamarkuptrue%
%
\endisatagdocument
{\isafolddocument}%
%
\isadelimdocument
%
\endisadelimdocument
%
\begin{isamarkuptext}%
\begin{itemize}
    \item (p.173) \isa{f\ {\isacharbackquote}\ A\ {\isacharequal}\ {\isacharbraceleft}y\ {\isacharbar}\ {\isasymexists}x{\isasymin}A{\isachardot}\ y\ {\isacharequal}\ f\ x{\isacharbraceright}} 
      \hfill (\isa{image{\isacharunderscore}def})
    \item (p.173) \isa{f\ {\isacharbackquote}\ {\isacharparenleft}A\ {\isasymunion}\ B{\isacharparenright}\ {\isacharequal}\ f\ {\isacharbackquote}\ A\ {\isasymunion}\ f\ {\isacharbackquote}\ B} 
      \hfill (\isa{image{\isacharunderscore}Un})
    \item (p.174) \isa{f\ {\isacharbackquote}\ {\isasymemptyset}\ {\isacharequal}\ {\isasymemptyset}} 
      \hfill (\isa{image{\isacharunderscore}empty})
    \item (p.174) \isa{f\ {\isacharbackquote}\ {\isacharparenleft}{\isacharbraceleft}a{\isacharbraceright}\ {\isasymunion}\ B{\isacharparenright}\ {\isacharequal}\ {\isacharbraceleft}f\ a{\isacharbraceright}\ {\isasymunion}\ f\ {\isacharbackquote}\ B} 
      \hfill (\isa{image{\isacharunderscore}insert})
  \end{itemize}%
\end{isamarkuptext}\isamarkuptrue%
%
\isadelimdocument
%
\endisadelimdocument
%
\isatagdocument
%
\isamarkupsubsection{Más operaciones y lemas (7.4)%
}
\isamarkuptrue%
%
\isamarkupsubsubsection{Reglas derivadas sobre subconjuntos (7.4.2)%
}
\isamarkuptrue%
%
\endisatagdocument
{\isafolddocument}%
%
\isadelimdocument
%
\endisadelimdocument
%
\begin{isamarkuptext}%
\begin{itemize}
    \item (p.177) \isa{A\ {\isasymsubseteq}\ A\ {\isasymunion}\ B} 
      \hfill (\isa{Un{\isacharunderscore}upper{\isadigit{1}}})
    \item (p.177) \isa{B\ {\isasymsubseteq}\ A\ {\isasymunion}\ B} 
      \hfill (\isa{Un{\isacharunderscore}upper{\isadigit{2}}})
  \end{itemize}%
\end{isamarkuptext}\isamarkuptrue%
%
\isadelimdocument
%
\endisadelimdocument
%
\isatagdocument
%
\isamarkupsubsubsection{Igualdades sobre la union, intersección, inclusion, 
  etc. (7.4.3)%
}
\isamarkuptrue%
%
\endisatagdocument
{\isafolddocument}%
%
\isadelimdocument
%
\endisadelimdocument
%
\begin{isamarkuptext}%
\begin{itemize}
    \item (p.179) \isa{{\isacharbraceleft}a{\isacharbraceright}\ {\isasymunion}\ A\ {\isacharequal}\ {\isacharbraceleft}a{\isacharbraceright}\ {\isasymunion}\ A} 
      \hfill (\isa{insert{\isacharunderscore}is{\isacharunderscore}Un})
    \item (p.181) \isa{A\ {\isasymunion}\ A\ {\isacharequal}\ A} 
      \hfill (\isa{Un{\isacharunderscore}absorb})
    \item (p.181) \isa{A\ {\isasymunion}\ {\isasymemptyset}\ {\isacharequal}\ A} 
      \hfill (\isa{Un{\isacharunderscore}empty{\isacharunderscore}right})
    \item (p.182) \isa{{\isacharbraceleft}a{\isacharbraceright}\ {\isasymunion}\ B\ {\isasymunion}\ C\ {\isacharequal}\ {\isacharbraceleft}a{\isacharbraceright}\ {\isasymunion}\ {\isacharparenleft}B\ {\isasymunion}\ C{\isacharparenright}} 
      \hfill (\isa{Un{\isacharunderscore}insert{\isacharunderscore}left})
    \item (p.187) \isa{{\isacharparenleft}{\isasymforall}x{\isasymin}A{\isachardot}\ P\ x\ {\isasymor}\ Q{\isacharparenright}\ {\isacharequal}\ {\isacharparenleft}{\isacharparenleft}{\isasymforall}x{\isasymin}A{\isachardot}\ P\ x{\isacharparenright}\ {\isasymor}\ Q{\isacharparenright}\isasep\isanewline%
{\isacharparenleft}{\isasymforall}x{\isasymin}A{\isachardot}\ P\ {\isasymor}\ Q\ x{\isacharparenright}\ {\isacharequal}\ {\isacharparenleft}P\ {\isasymor}\ {\isacharparenleft}{\isasymforall}x{\isasymin}A{\isachardot}\ Q\ x{\isacharparenright}{\isacharparenright}\isasep\isanewline%
{\isacharparenleft}{\isasymforall}x{\isasymin}A{\isachardot}\ P\ {\isasymlongrightarrow}\ Q\ x{\isacharparenright}\ {\isacharequal}\ {\isacharparenleft}P\ {\isasymlongrightarrow}\ {\isacharparenleft}{\isasymforall}x{\isasymin}A{\isachardot}\ Q\ x{\isacharparenright}{\isacharparenright}\isasep\isanewline%
{\isacharparenleft}{\isasymforall}x{\isasymin}A{\isachardot}\ P\ x\ {\isasymlongrightarrow}\ Q{\isacharparenright}\ {\isacharequal}\ {\isacharparenleft}{\isacharparenleft}{\isasymexists}x{\isasymin}A{\isachardot}\ P\ x{\isacharparenright}\ {\isasymlongrightarrow}\ Q{\isacharparenright}\isasep\isanewline%
{\isacharparenleft}{\isasymforall}x{\isasymin}{\isasymemptyset}{\isachardot}\ P\ x{\isacharparenright}\ {\isacharequal}\ True\isasep\isanewline%
{\isacharparenleft}{\isasymforall}x{\isasymin}UNIV{\isachardot}\ P\ x{\isacharparenright}\ {\isacharequal}\ {\isacharparenleft}{\isasymforall}x{\isachardot}\ P\ x{\isacharparenright}\isasep\isanewline%
{\isacharparenleft}{\isasymforall}x{\isasymin}{\isacharbraceleft}a{\isacharbraceright}\ {\isasymunion}\ B{\isachardot}\ P\ x{\isacharparenright}\ {\isacharequal}\ {\isacharparenleft}P\ a\ {\isasymand}\ {\isacharparenleft}{\isasymforall}x{\isasymin}B{\isachardot}\ P\ x{\isacharparenright}{\isacharparenright}\isasep\isanewline%
{\isacharparenleft}{\isasymforall}x{\isasymin}Collect\ Q{\isachardot}\ P\ x{\isacharparenright}\ {\isacharequal}\ {\isacharparenleft}{\isasymforall}x{\isachardot}\ Q\ x\ {\isasymlongrightarrow}\ P\ x{\isacharparenright}\isasep\isanewline%
{\isacharparenleft}{\isasymforall}x{\isasymin}f\ {\isacharbackquote}\ A{\isachardot}\ P\ x{\isacharparenright}\ {\isacharequal}\ {\isacharparenleft}{\isasymforall}x{\isasymin}A{\isachardot}\ P\ {\isacharparenleft}f\ x{\isacharparenright}{\isacharparenright}\isasep\isanewline%
{\isacharparenleft}{\isasymnot}\ {\isacharparenleft}{\isasymforall}x{\isasymin}A{\isachardot}\ P\ x{\isacharparenright}{\isacharparenright}\ {\isacharequal}\ {\isacharparenleft}{\isasymexists}x{\isasymin}A{\isachardot}\ {\isasymnot}\ P\ x{\isacharparenright}} 
      \hfill (\isa{ball{\isacharunderscore}simps})
    \item (p.187) \isa{{\isacharparenleft}{\isasymexists}x{\isasymin}A{\isachardot}\ P\ x\ {\isasymand}\ Q{\isacharparenright}\ {\isacharequal}\ {\isacharparenleft}{\isacharparenleft}{\isasymexists}x{\isasymin}A{\isachardot}\ P\ x{\isacharparenright}\ {\isasymand}\ Q{\isacharparenright}\isasep\isanewline%
{\isacharparenleft}{\isasymexists}x{\isasymin}A{\isachardot}\ P\ {\isasymand}\ Q\ x{\isacharparenright}\ {\isacharequal}\ {\isacharparenleft}P\ {\isasymand}\ {\isacharparenleft}{\isasymexists}x{\isasymin}A{\isachardot}\ Q\ x{\isacharparenright}{\isacharparenright}\isasep\isanewline%
{\isacharparenleft}{\isasymexists}x{\isasymin}{\isasymemptyset}{\isachardot}\ P\ x{\isacharparenright}\ {\isacharequal}\ False\isasep\isanewline%
{\isacharparenleft}{\isasymexists}x{\isasymin}UNIV{\isachardot}\ P\ x{\isacharparenright}\ {\isacharequal}\ {\isacharparenleft}{\isasymexists}x{\isachardot}\ P\ x{\isacharparenright}\isasep\isanewline%
{\isacharparenleft}{\isasymexists}x{\isasymin}{\isacharbraceleft}a{\isacharbraceright}\ {\isasymunion}\ B{\isachardot}\ P\ x{\isacharparenright}\ {\isacharequal}\ {\isacharparenleft}P\ a\ {\isasymor}\ {\isacharparenleft}{\isasymexists}x{\isasymin}B{\isachardot}\ P\ x{\isacharparenright}{\isacharparenright}\isasep\isanewline%
{\isacharparenleft}{\isasymexists}x{\isasymin}Collect\ Q{\isachardot}\ P\ x{\isacharparenright}\ {\isacharequal}\ {\isacharparenleft}{\isasymexists}x{\isachardot}\ Q\ x\ {\isasymand}\ P\ x{\isacharparenright}\isasep\isanewline%
{\isacharparenleft}{\isasymexists}x{\isasymin}f\ {\isacharbackquote}\ A{\isachardot}\ P\ x{\isacharparenright}\ {\isacharequal}\ {\isacharparenleft}{\isasymexists}x{\isasymin}A{\isachardot}\ P\ {\isacharparenleft}f\ x{\isacharparenright}{\isacharparenright}\isasep\isanewline%
{\isacharparenleft}{\isasymnot}\ {\isacharparenleft}{\isasymexists}x{\isasymin}A{\isachardot}\ P\ x{\isacharparenright}{\isacharparenright}\ {\isacharequal}\ {\isacharparenleft}{\isasymforall}x{\isasymin}A{\isachardot}\ {\isasymnot}\ P\ x{\isacharparenright}} 
      \hfill (\isa{bex{\isacharunderscore}simps})
  \end{itemize}%
\end{isamarkuptext}\isamarkuptrue%
%
\isadelimdocument
%
\endisadelimdocument
%
\isatagdocument
%
\isamarkupsubsubsection{Monotonía de varias operaciones (7.4.4)%
}
\isamarkuptrue%
%
\endisatagdocument
{\isafolddocument}%
%
\isadelimdocument
%
\endisadelimdocument
%
\begin{isamarkuptext}%
\begin{itemize}
    \item (p.188) \isa{\mbox{}\inferrule{\mbox{A\ {\isasymsubseteq}\ C\ {\isasymand}\ B\ {\isasymsubseteq}\ D}}{\mbox{A\ {\isasymunion}\ B\ {\isasymsubseteq}\ C\ {\isasymunion}\ D}}} 
      \hfill (\isa{Un{\isacharunderscore}mono})
    \item (p.188) \isa{P\ {\isasymlongrightarrow}\ P} 
      \hfill (\isa{imp{\isacharunderscore}refl})
    \item (p.188) \isa{\mbox{}\inferrule{\mbox{Q\ {\isasymlongrightarrow}\ P}}{\mbox{{\isasymnot}\ P\ {\isasymlongrightarrow}\ {\isasymnot}\ Q}}} 
      \hfill (\isa{not{\isacharunderscore}mono})
  \end{itemize}%
\end{isamarkuptext}\isamarkuptrue%
%
\isadelimdocument
%
\endisadelimdocument
%
\isatagdocument
%
\isamarkupsection{Nociones sobre funciones (9)%
}
\isamarkuptrue%
%
\endisatagdocument
{\isafolddocument}%
%
\isadelimdocument
%
\endisadelimdocument
%
\begin{isamarkuptext}%
En Isabelle, la teoría de funciones se corresponde con 
  \href{https://bit.ly/2VBe1Im}{Fun.thy}.%
\end{isamarkuptext}\isamarkuptrue%
%
\isadelimdocument
%
\endisadelimdocument
%
\isatagdocument
%
\isamarkupsubsection{Actualización de funciones (9.6)%
}
\isamarkuptrue%
%
\endisatagdocument
{\isafolddocument}%
%
\isadelimdocument
%
\endisadelimdocument
%
\begin{isamarkuptext}%
\begin{itemize}
    \item (p.212) \isa{f{\isacharparenleft}a\ {\isacharcolon}{\isacharequal}\ b{\isacharparenright}\ {\isacharequal}\ {\isacharparenleft}{\isasymlambda}x{\isachardot}\ \textsf{if}\ x\ {\isacharequal}\ a\ \textsf{then}\ b\ \textsf{else}\ f\ x{\isacharparenright}} 
      \hfill (\isa{fun{\isacharunderscore}upd{\isacharunderscore}def})
    \item (p.213) \isa{\mbox{}\inferrule{\mbox{z\ {\isasymnoteq}\ x}}{\mbox{{\isacharparenleft}f{\isacharparenleft}x\ {\isacharcolon}{\isacharequal}\ y{\isacharparenright}{\isacharparenright}\ z\ {\isacharequal}\ f\ z}}} 
      \hfill (\isa{fun{\isacharunderscore}upd{\isacharunderscore}other})
  \end{itemize}%
\end{isamarkuptext}\isamarkuptrue%
%
\isadelimdocument
%
\endisadelimdocument
%
\isatagdocument
%
\isamarkupsection{Retículos completos (10)%
}
\isamarkuptrue%
%
\endisatagdocument
{\isafolddocument}%
%
\isadelimdocument
%
\endisadelimdocument
%
\begin{isamarkuptext}%
En Isabelle corresponde a la teoría 
  \href{https://bit.ly/2Y5wxKA}{Complete-Lattices.thy}.%
\end{isamarkuptext}\isamarkuptrue%
%
\isadelimdocument
%
\endisadelimdocument
%
\isatagdocument
%
\isamarkupsubsection{Retículos completos en conjuntos (10.6)%
}
\isamarkuptrue%
%
\isamarkupsubsubsection{Unión (10.6.3)%
}
\isamarkuptrue%
%
\endisatagdocument
{\isafolddocument}%
%
\isadelimdocument
%
\endisadelimdocument
%
\begin{isamarkuptext}%
\begin{itemize}
    \item (p.238) \isa{{\isasymUnion}\ {\isasymemptyset}\ {\isacharequal}\ {\isasymemptyset}} 
      \hfill (\isa{Union{\isacharunderscore}empty})
  \end{itemize}%
\end{isamarkuptext}\isamarkuptrue%
%
\isadelimdocument
%
\endisadelimdocument
%
\isatagdocument
%
\isamarkupsection{Conjuntos finitos (18)%
}
\isamarkuptrue%
%
\endisatagdocument
{\isafolddocument}%
%
\isadelimdocument
%
\endisadelimdocument
%
\begin{isamarkuptext}%
A continuación se muestran resultados relativos a la teoría 
  \href{https://bit.ly/3bEIScG}{Finite-Set.thy}.%
\end{isamarkuptext}\isamarkuptrue%
%
\isadelimdocument
%
\endisadelimdocument
%
\isatagdocument
%
\isamarkupsubsection{Predicado de conjuntos finitos (18.1)%
}
\isamarkuptrue%
%
\endisatagdocument
{\isafolddocument}%
%
\isadelimdocument
%
\endisadelimdocument
%
\begin{isamarkuptext}%
\begin{itemize}
    \item (p.419) \isa{finite\ A} 
      \hfill (\isa{finite})
  \end{itemize}%
\end{isamarkuptext}\isamarkuptrue%
%
\isadelimdocument
%
\endisadelimdocument
%
\isatagdocument
%
\isamarkupsubsection{Finitud y operaciones de conjuntos comunes (18.2)%
}
\isamarkuptrue%
%
\endisatagdocument
{\isafolddocument}%
%
\isadelimdocument
%
\endisadelimdocument
%
\begin{isamarkuptext}%
\begin{itemize}
    \item (p.422) \isa{\mbox{}\inferrule{\mbox{finite\ F\ {\isasymand}\ finite\ G}}{\mbox{finite\ {\isacharparenleft}F\ {\isasymunion}\ G{\isacharparenright}}}} 
      \hfill (\isa{finite{\isacharunderscore}UnI})
    \item (p.423) \isa{finite\ {\isacharparenleft}{\isacharbraceleft}a{\isacharbraceright}\ {\isasymunion}\ A{\isacharparenright}\ {\isacharequal}\ finite\ A} 
      \hfill (\isa{finite{\isacharunderscore}insert})
  \end{itemize}%
\end{isamarkuptext}\isamarkuptrue%
%
\isadelimdocument
%
\endisadelimdocument
%
\isatagdocument
%
\isamarkupsection{Composición de functores naturales acotados (33)%
}
\isamarkuptrue%
%
\endisatagdocument
{\isafolddocument}%
%
\isadelimdocument
%
\endisadelimdocument
%
\begin{isamarkuptext}%
En esta sección se muestran resultados pertenecientes a la
  teoría de composición de functores naturales acotados de Isabelle 
  \href{https://bit.ly/2zGl9v6}{BNFComposition.thy}.%
\end{isamarkuptext}\isamarkuptrue%
%
\begin{isamarkuptext}%
\begin{itemize}
    \item (p.718) \isa{{\isasymUnion}\ {\isacharparenleft}f\ {\isacharbackquote}\ {\isacharparenleft}{\isacharbraceleft}a{\isacharbraceright}\ {\isasymunion}\ B{\isacharparenright}{\isacharparenright}\ {\isacharequal}\ f\ a\ {\isasymunion}\ {\isasymUnion}\ {\isacharparenleft}f\ {\isacharbackquote}\ B{\isacharparenright}} 
      \hfill (\isa{Union{\isacharunderscore}image{\isacharunderscore}insert})
  \end{itemize}%
\end{isamarkuptext}\isamarkuptrue%
%
\isadelimdocument
%
\endisadelimdocument
%
\isatagdocument
%
\isamarkupsection{El tipo de datos de la listas finitas (66)%
}
\isamarkuptrue%
%
\endisatagdocument
{\isafolddocument}%
%
\isadelimdocument
%
\endisadelimdocument
%
\begin{isamarkuptext}%
En esta sección se muestran resultados sobre listas finitas 
  dentro de la teoría de listas de Isabelle 
  \href{https://bit.ly/3bCNxvX}{List.thy}.%
\end{isamarkuptext}\isamarkuptrue%
%
\begin{isamarkuptext}%
\begin{itemize}
    \item (p.1169) \isa{{\isacharbrackleft}{\isacharbrackright}\ {\isacharequal}\ {\isasymemptyset}\isasep\isanewline%
x{\isadigit{2}}{\isadigit{1}}\ {\isasymcdot}\ x{\isadigit{2}}{\isadigit{2}}\ {\isacharequal}\ {\isacharbraceleft}x{\isadigit{2}}{\isadigit{1}}{\isacharbraceright}\ {\isasymunion}\ x{\isadigit{2}}{\isadigit{2}}} 
      \hfill (\isa{list{\isachardot}set})
  \end{itemize}%
\end{isamarkuptext}\isamarkuptrue%
%
\isadelimdocument
%
\endisadelimdocument
%
\isatagdocument
%
\isamarkupsubsection{Funciones básicas de procesamiento de listas (66.1)%
}
\isamarkuptrue%
%
\isamarkupsubsubsection{Función \isa{set}%
}
\isamarkuptrue%
%
\endisatagdocument
{\isafolddocument}%
%
\isadelimdocument
%
\endisadelimdocument
%
\begin{isamarkuptext}%
\begin{itemize}
    \item (p.1195) \isa{xs\ \isacharat\ ys\ {\isacharequal}\ xs\ {\isasymunion}\ ys} 
      \hfill (\isa{set{\isacharunderscore}append})
  \end{itemize}%
\end{isamarkuptext}\isamarkuptrue%
%
\isadelimtheory
%
\endisadelimtheory
%
\isatagtheory
%
\endisatagtheory
{\isafoldtheory}%
%
\isadelimtheory
%
\endisadelimtheory
%
\end{isabellebody}%
\endinput
%:%file=~/Desktop/TFM/TFM1/Glosario.thy%:%
%:%19=13%:%
%:%20=14%:%
%:%21=15%:%
%:%30=17%:%
%:%42=19%:%
%:%43=20%:%
%:%52=22%:%
%:%56=24%:%
%:%68=27%:%
%:%69=28%:%
%:%70=29%:%
%:%71=30%:%
%:%80=32%:%
%:%92=35%:%
%:%93=36%:%
%:%94=37%:%
%:%95=38%:%
%:%96=39%:%
%:%97=40%:%
%:%106=42%:%
%:%110=44%:%
%:%122=47%:%
%:%123=48%:%
%:%124=49%:%
%:%125=50%:%
%:%126=51%:%
%:%127=52%:%
%:%136=54%:%
%:%148=57%:%
%:%149=58%:%
%:%150=59%:%
%:%151=60%:%
%:%160=62%:%
%:%172=65%:%
%:%173=66%:%
%:%174=67%:%
%:%175=68%:%
%:%184=70%:%
%:%196=73%:%
%:%197=74%:%
%:%198=75%:%
%:%199=76%:%
%:%200=77%:%
%:%201=78%:%
%:%210=80%:%
%:%222=83%:%
%:%223=84%:%
%:%224=85%:%
%:%225=86%:%
%:%234=88%:%
%:%246=91%:%
%:%247=92%:%
%:%248=93%:%
%:%249=94%:%
%:%258=96%:%
%:%270=99%:%
%:%271=100%:%
%:%272=101%:%
%:%273=102%:%
%:%282=104%:%
%:%294=107%:%
%:%295=108%:%
%:%296=109%:%
%:%297=110%:%
%:%306=112%:%
%:%318=115%:%
%:%319=116%:%
%:%320=117%:%
%:%321=118%:%
%:%322=119%:%
%:%323=120%:%
%:%332=122%:%
%:%344=125%:%
%:%345=126%:%
%:%346=127%:%
%:%347=128%:%
%:%348=129%:%
%:%349=130%:%
%:%350=131%:%
%:%351=132%:%
%:%360=134%:%
%:%372=137%:%
%:%373=138%:%
%:%374=139%:%
%:%375=140%:%
%:%376=141%:%
%:%377=142%:%
%:%386=144%:%
%:%398=147%:%
%:%399=148%:%
%:%400=149%:%
%:%401=150%:%
%:%410=152%:%
%:%414=154%:%
%:%426=157%:%
%:%427=158%:%
%:%428=159%:%
%:%429=160%:%
%:%430=161%:%
%:%431=162%:%
%:%440=164%:%
%:%452=166%:%
%:%453=167%:%
%:%462=169%:%
%:%474=172%:%
%:%475=173%:%
%:%476=174%:%
%:%477=175%:%
%:%486=177%:%
%:%498=179%:%
%:%499=180%:%
%:%503=183%:%
%:%504=184%:%
%:%505=185%:%
%:%506=186%:%
%:%515=188%:%
%:%527=190%:%
%:%528=191%:%
%:%537=193%:%
%:%549=196%:%
%:%550=197%:%
%:%551=198%:%
%:%552=199%:%
%:%553=200%:%
%:%554=201%:%
%:%563=203%:%
%:%567=205%:%
%:%579=208%:%
%:%580=209%:%
%:%581=210%:%
%:%582=211%:%
%:%583=212%:%
%:%584=213%:%
%:%585=214%:%
%:%586=215%:%
%:%595=217%:%
%:%607=220%:%
%:%608=221%:%
%:%609=222%:%
%:%610=223%:%
%:%611=224%:%
%:%612=225%:%
%:%613=226%:%
%:%614=227%:%
%:%623=229%:%
%:%635=232%:%
%:%636=233%:%
%:%637=234%:%
%:%638=235%:%
%:%639=236%:%
%:%640=237%:%
%:%641=238%:%
%:%642=239%:%
%:%651=241%:%
%:%663=244%:%
%:%664=245%:%
%:%665=246%:%
%:%666=247%:%
%:%675=249%:%
%:%687=252%:%
%:%688=253%:%
%:%689=254%:%
%:%690=255%:%
%:%691=256%:%
%:%692=257%:%
%:%693=258%:%
%:%694=259%:%
%:%703=262%:%
%:%715=265%:%
%:%716=266%:%
%:%717=267%:%
%:%718=268%:%
%:%719=269%:%
%:%720=270%:%
%:%721=271%:%
%:%722=272%:%
%:%723=273%:%
%:%724=274%:%
%:%733=277%:%
%:%737=279%:%
%:%749=282%:%
%:%750=283%:%
%:%751=284%:%
%:%752=285%:%
%:%753=286%:%
%:%754=287%:%
%:%763=289%:%
%:%764=290%:%
%:%776=293%:%
%:%777=294%:%
%:%778=295%:%
%:%779=296%:%
%:%780=297%:%
%:%781=298%:%
%:%782=299%:%
%:%783=300%:%
%:%784=301%:%
%:%785=302%:%
%:%794=302%:%
%:%795=303%:%
%:%796=304%:%
%:%803=304%:%
%:%804=305%:%
%:%805=306%:%
%:%814=308%:%
%:%826=311%:%
%:%827=312%:%
%:%828=313%:%
%:%829=314%:%
%:%830=315%:%
%:%831=316%:%
%:%832=317%:%
%:%833=318%:%
%:%842=321%:%
%:%854=323%:%
%:%855=324%:%
%:%864=326%:%
%:%876=329%:%
%:%877=330%:%
%:%878=331%:%
%:%879=332%:%
%:%880=333%:%
%:%881=334%:%
%:%890=336%:%
%:%902=338%:%
%:%903=339%:%
%:%912=341%:%
%:%916=343%:%
%:%928=346%:%
%:%929=347%:%
%:%930=348%:%
%:%931=349%:%
%:%940=351%:%
%:%952=353%:%
%:%953=354%:%
%:%962=356%:%
%:%974=359%:%
%:%975=360%:%
%:%976=361%:%
%:%977=362%:%
%:%986=364%:%
%:%998=367%:%
%:%999=368%:%
%:%1000=369%:%
%:%1001=370%:%
%:%1002=371%:%
%:%1003=372%:%
%:%1012=374%:%
%:%1024=376%:%
%:%1025=377%:%
%:%1026=378%:%
%:%1030=381%:%
%:%1031=382%:%
%:%1032=383%:%
%:%1033=384%:%
%:%1042=386%:%
%:%1054=388%:%
%:%1055=389%:%
%:%1056=390%:%
%:%1060=393%:%
%:%1061=394%:%
%:%1062=394%:%
%:%1063=395%:%
%:%1064=396%:%
%:%1073=398%:%
%:%1077=400%:%
%:%1089=403%:%
%:%1090=404%:%
%:%1091=405%:%
%:%1092=406%:%


\chapter*{Sumario}
\addcontentsline{toc}{chapter}{Sumario}
%
\begin{isabellebody}%
\setisabellecontext{Resumen}%
%
\isadelimtheory
%
\endisadelimtheory
%
\isatagtheory
%
\endisatagtheory
{\isafoldtheory}%
%
\isadelimtheory
%
\endisadelimtheory
%
\begin{isamarkuptext}%
El objetivo de la Lógica es la formalización del conocimiento y su
razonamiento. Este trabajo constituye una continuación del Trabajo de Fin de
Grado \isa{Elementos\ de\ lógica\ formalizados\ en\ Isabelle{\isacharslash}HOL} \isa{{\isacharbrackleft}{\isadigit{4}}{\isacharbrackright}}, donde se estudiaron
la sintaxis, semántica y \isa{Lema\ de\ Hintikka} de la Lógica Proposicional
desde la perspectiva teórica de \isa{First−Order\ Logic\ and\ Automated\ Theorem\ Proving} 
\isa{{\isacharbrackleft}{\isadigit{5}}{\isacharbrackright}} de Melvin Fitting. Manteniendo dicha perspectiva, nos centraremos en la 
demostración del \isa{Teorema\ de\ Existencia\ de\ Modelo}, concluyendo con 
el \isa{Teorema\ de\ Compacidad} como consecuencia del mismo. Siguiendo la 
inspiración de \isa{Propositional\ Proof\ Systems} \isa{{\isacharbrackleft}{\isadigit{1}}{\isadigit{0}}{\isacharbrackright}} por Julius Michaelis y Tobias 
Nipkow, los resultados expuestos serán formalizados mediante Isabelle: un demostrador 
interactivo que incluye herramientas de razonamiento automático para guiar al usuario 
en el proceso de formalización, verificación y automatización de resultados. 
Concretamente, Isabelle/HOL es una especialización de Isabelle para la lógica de orden 
superior. Las demostraciones de los resultados en Isabelle/HOL se elaborarán siguiendo 
dos tácticas distintas a lo largo del trabajo. En primer lugar, cada lema será probado 
de manera detallada prescindiendo de toda herramienta de razonamiento automático, como 
resultado de una búsqueda inversa en cada paso de la prueba. En contraposición, 
elaboraremos una demostración automática alternativa de cada resultado que utilice todas 
las herramientas de razonamiento automático que proporciona el demostrador. De este modo, 
se evidenciará la capacidad de razonamiento automático de Isabelle.%
\end{isamarkuptext}\isamarkuptrue%
%
\begin{isamarkuptext}%
Logic’s purpose is about knowledge’s formalisation and its 
reasoning. This project is a continuation of \isa{Elementos\ de\ lógica\ formalizados\ en\ Isabelle{\isacharslash}HOL} \isa{{\isacharbrackleft}{\isadigit{4}}{\isacharbrackright}} in which we studied Syntax, Semantics and
a propositional version of Hintikka's lemma from the theoretical perspective 
of \isa{First−Order\ Logic\ and\ Automated\ Theorem\ Proving} \isa{{\isacharbrackleft}{\isadigit{5}}{\isacharbrackright}} by Melvin Fitting. 
Following the same perspective, we will focus on the demonstration of 
\isa{Propositional\ Model\ Existence} theorem, concluding with the \isa{Propositional\ Compactness} theorem as a consecuence. Inspired by \isa{Propositional\ Proof\ Systems} \isa{{\isacharbrackleft}{\isadigit{1}}{\isadigit{0}}{\isacharbrackright}} by 
Julius Michaelis and Tobias Nipkow, these results will be formalised using 
Isabelle: a proof assistant including automatic reasoning tools to guide the 
user on formalising, verifying and automating results. In particular, 
Isabelle/HOL is the specialization of Isabelle for High-Order Logic. The 
processing of the results formalised in Isabelle/HOL follows two directions. 
In the first place, each lemma will be proved on detail without any automation, 
as the result of an inverse research on every step of the demonstration until 
it is only completed with deductions based on elementary rules and definitions. 
Conversely, we will alternatively prove the results using all the 
automatic reasoning tools that are provide by the proof assistant. In 
this way, Isabelle’s power of automatic reasoning will be shown as the 
contrast between these two different proving tactics.%
\end{isamarkuptext}\isamarkuptrue%
%
\isadelimtheory
%
\endisadelimtheory
%
\isatagtheory
%
\endisatagtheory
{\isafoldtheory}%
%
\isadelimtheory
%
\endisadelimtheory
%
\end{isabellebody}%
\endinput
%:%file=~/TFM/TFM/Resumen.thy%:%
%:%19=8%:%
%:%20=9%:%
%:%21=10%:%
%:%22=11%:%
%:%23=12%:%
%:%24=13%:%
%:%25=14%:%
%:%26=15%:%
%:%27=16%:%
%:%28=17%:%
%:%29=18%:%
%:%30=19%:%
%:%31=20%:%
%:%32=21%:%
%:%33=22%:%
%:%34=23%:%
%:%35=24%:%
%:%36=25%:%
%:%37=26%:%
%:%38=27%:%
%:%42=31%:%
%:%43=32%:%
%:%43=33%:%
%:%44=34%:%
%:%45=35%:%
%:%46=36%:%
%:%47=37%:%
%:%47=38%:%
%:%48=39%:%
%:%49=40%:%
%:%50=41%:%
%:%51=42%:%
%:%52=43%:%
%:%53=44%:%
%:%54=45%:%
%:%55=46%:%
%:%56=47%:%
%:%57=48%:%
%:%58=49%:%
%:%59=50%:%
\chapter*{Introducción}
\addcontentsline{toc}{chapter}{Introducción}
%
\begin{isabellebody}%
\setisabellecontext{Introduccion}%
%
\isadelimtheory
%
\endisadelimtheory
%
\isatagtheory
%
\endisatagtheory
{\isafoldtheory}%
%
\isadelimtheory
%
\endisadelimtheory
%
\begin{isamarkuptext}%
El objetivo de la Lógica es la formalización del conocimiento y el 
  razonamiento sobre el mismo. Tiene su origen en la Antigua Grecia con 
  Aristóteles y su investigación acerca de los principios del 
  razonamiento válido o correcto, recogidos fundamentalmente en su obra 
  \isa{Organon}. De este modo, dio lugar a la lógica silogística, que 
  consistía en la deducción de conclusiones a partir de dos premisas 
  iniciales. 

	Posteriormente, los estoicos (400-200 a.C) comenzaron a cuestionarse 
  temas relacionados con la semántica, como la naturaleza de la verdad. 
  Formularon la \isa{paradoja\ del\ mentiroso}, que plantea una incongruencia 
  acerca de la veracidad del siguiente predicado.

  \isa{Esta\ oración\ es\ falsa{\isachardot}}

  Sin embargo, no fue hasta el siglo XVII que el matemático y filósofo 
  Gottfried Wilhelm Leibniz (1646 – 1716) instaura un programa lógico 
  que propone la búsqueda de un sistema simbólico del lenguaje natural 
  junto con la matematización del concepto de validez. Estas ideas 
  fueron la principal motivación del desarrollo de la lógica moderna del 
  siglo XIX de la mano de matemáticos y filósofos como Bernard Bolzano 
  (1781 – 1848), George Boole (1815 – 1864), Charles Saunders Pierce 
  (1839 – 1914) y Gottlob Frege (1848 – 1925). Fue este último quien 
  introdujo el primer tratamiento sistemático de la lógica 
  proposicional. Frege basó su tesis en el desarrollo de una sintaxis 
  completa que combina el razonamiento de deducción de la silogística 
  aristotélica con la noción estoica de conectivas para relacionar 
  ideas. Paralelamente desarrolló una semántica asociada a dicha 
  sintaxis que permitiese verificar la validez de los procesos 
  deductivos. La lógica proposicional de Frege formó parte de la 
  escuela denominada logicismo. Su objetivo consistía en investigar 
  los fundamentos de las matemáticas con el fin de formalizarlos 
  lógicamente, para así realizar deducciones y razonamientos 
  válidos.

	En las últimas décadas, el desarrollo de la computación y la 
  inteligencia artificial ha permitido la formalización de las 
  matemáticas y la lógica mediante el lenguaje computacional. 
  Concretamente, el razonamiento automático es un área que investiga los
  distintos aspectos del razonamiento con el fin de crear programas y
  algoritmos para razonar de manera prácticamente automática.
  Se fundamenta en el programa lógico desarrollado por Leibniz,
  estructurado en base a dos principios: la formalización rigurosa
  de resultados y el desarrollo de algoritmos que permitan manipular
  y razonar a partir de dichas formalizaciones. Entre las principales 
  aplicaciones de este áres se encuentra la verificación y síntesis 
  automáticas de programas. De este modo, podemos validar distintos 
  razonamientos, así como crear herramientas de razonamiento automático 
  que permitan el desarrollo de nuevos resultados.

  En este contexto nace Isabelle en 1986, desarrollada por Larry Paulson 
  de la Universidad de Cambridge y Tobias Nipkow del Techniche 
  Universität München. Isabelle es un demostrador interactivo que,
  desde el razonamiento automático, facilita la formalización lógica de 
  resultados y proporciona herramientas para realizar deducciones. En 
  particular, Isabelle/HOL es la especialización de Isabelle para la 
  lógica de orden superior. Junto con Coq, ACL2 y PVS, entre
  otros, constituye uno de los demostradores interactivos más 
  influyentes.

  Como demostrador interactivo, Isabelle permite automatizar 
  razonamientos guiados por el usuario, verificando cada paso de una 
  deducción de manera precisa. Además, incorpora herramientas de 
  razonamiento automático para mejorar la productividad del proceso de 
  demostración. Para ello, cuenta con una extensa librería de resultados 
  lógicos y matemáticos que han sido formalizados y continúan en 
  desarrollo por parte de proyectos como \isa{The\ Alexandria\ Project{\isacharcolon}\ Large{\isacharminus}Scale\ Formal\ Proof\ for\ the\ Working\ Mathematician}. Este proyecto 
  comienza en 2017, dirigido por Lawrence Paulson desde la Universidad 
  de Cambridge. Tiene como finalidad la formalización de distintas 
  teorías para ampliar la librería de Isabelle, junto con la creación de 
  herramientas interactivas que asistan a los matemáticos en el proceso 
  de formalización, demostración y búsqueda de nuevos resultados. 

  Este trabajo constituye una continuación del Trabajo de Fin de Grado
  \isa{Elementos\ de\ lógica\ formalizados\ en\ Isabelle{\isacharslash}HOL} \isa{{\isacharbrackleft}{\isadigit{4}}{\isacharbrackright}} dedicado al estudio
  y formalización de la sintaxis, semántica y lema de Hintikka de la lógica
  proposicional. Inspirado en la primera sección de la publicación 
  \isa{Propositional\ Proof\ Systems} \isa{{\isacharbrackleft}{\isadigit{1}}{\isadigit{0}}{\isacharbrackright}} de Julius Michaelis y Tobias Nipkow, y
  basando su contenido teórico en el libro \isa{First{\isacharminus}Order\ Logic\ and\ Automated\ Theorem\ Proving} \isa{{\isacharbrackleft}{\isadigit{5}}{\isacharbrackright}} de Melvin Fitting, este trabajo tiene como objetivo la
  demostración y formalización del \isa{Teorema\ de\ Existencia\ de\ Modelo}, concluyendo
  con el \isa{Teorema\ de\ Compacidad} como consecuencia del mismo. Para ello, consta
  de cuatro capítulos. En el primero se introduce la notación uniforme para 
  fórmulas proposicionales, en el segundo capítulo se define la propiedad de 
  consistencia proposicional para colecciones de conjuntos de fórmulas, en el 
  tercero se definen las colecciones cerradas bajo subconjuntos y de carácter 
  finito y en el último capítulo se demuestra finalmente \isa{Teorema\ de\ Existencia\ de\ Modelo}, concluyendo con la prueba del \isa{Teorema\ de\ Compacidad}.

  El primer capítulo introduce la notación uniforme para las fórmulas
  proposicionales con el fin de reducir el número de casos a considerar sobre 
  la estructura de las fórmulas al clasificarlas en dos categorías. Para 
  justificar la clasificación, se introduce la definición de fórmulas 
  semánticamente equivalentes como aquellas que tienen el mismo 
  valor para toda interpretación. De este modo, las fórmulas proposicionales
  pueden ser de dos tipos: las de tipo conjuntivo (las fórmulas \isa{{\isasymalpha}}) y las 
  de tipo disyuntivo (las fórmulas \isa{{\isasymbeta}}). Cada fórmula de tipo \isa{{\isasymalpha}}, o \isa{{\isasymbeta}} 
  respectivamente, tiene asociada sus dos componentes \isa{{\isasymalpha}\isactrlsub {\isadigit{1}}} y \isa{{\isasymalpha}\isactrlsub {\isadigit{2}}}, o \isa{{\isasymbeta}\isactrlsub {\isadigit{1}}} y 
  \isa{{\isasymbeta}\isactrlsub {\isadigit{2}}} respectivamente. Intuitivamente, una fórmula es de tipo \isa{{\isasymalpha}} con 
  componentes \isa{{\isasymalpha}\isactrlsub {\isadigit{1}}} y \isa{{\isasymalpha}\isactrlsub {\isadigit{2}}} si es semánticamente equivalente a la fórmula 
  \isa{{\isasymalpha}\isactrlsub {\isadigit{1}}\ {\isasymand}\ {\isasymalpha}\isactrlsub {\isadigit{2}}}, y una fórmula será de tipo \isa{{\isasymbeta}} con componentes \isa{{\isasymbeta}\isactrlsub {\isadigit{1}}} y \isa{{\isasymbeta}\isactrlsub {\isadigit{2}}} si es 
  semánticamente equivalente a la fórmula\\ \isa{{\isasymbeta}\isactrlsub {\isadigit{1}}\ {\isasymor}\ {\isasymbeta}\isactrlsub {\isadigit{2}}}. En Isabelle, se formalizarán
  sintácticamente los conjuntos de fórmulas de tipo \isa{{\isasymalpha}} y de tipo \isa{{\isasymbeta}} como
  predicados inductivos, simplificando la intuición original al prescindir 
  de la noción de equivalencia semántica que permite clasificar la totalidad 
  de las fórmulas proposicionales. Finalmente, el capítulo concluye con un lema
  de caracterización de los conjuntos de Hintikka mediante la notación uniforme.

  En el siguiente capítulo se define la propiedad de consistencia proposicional
  para colecciones de conjuntos de fórmulas. Al final del capítulo se introduce 
  un lema que caracteriza dicha propiedad mediante notación uniforme, lo que 
  facilita las demostraciones de los resultados posteriores.

  En el tercer capítulo se definen las colecciones de conjuntos de fórmulas
  cerradas bajo subconjuntos y de carácter finito. Una colección de conjuntos 
  es cerrada bajo subconjuntos si todo subconjunto de cada conjunto de la 
  colección pertenece a la colección, mientras que una colección es de 
  carácter finito si para todo conjunto son equivalentes que el conjunto
  pertenezca a la colección y que todo subconjunto finito del mismo pertenezca
  a la colección. Posteriormente se muestran tres resultados sobre las mismas. 
  El primero de ellos permite extender una colección con la propiedad de 
  consistencia proposicional a otra que también la verifique y sea cerrada bajo 
  subconjuntos. Seguidamente se probará que toda colección de carácter finito
  es cerrada bajo subconjuntos. En último lugar, se demuestra que una colección
  cerrada bajo subconjuntos que verifique la propiedad de consistencia 
  proposicional se puede extender a otra que también verifique dicha propiedad 
  y sea de carácter finito. 

  Finalmente, en el cuarto capítulo se demuestra el \isa{Teorema\ de\ Existencia\ de\ Modelo} y el \isa{Teorema\ de\ Compacidad} como consecuencia de este. El capítulo 
  está dividido en tres apartados: el primero donde se definen ciertas 
  sucesiones de conjuntos de fórmulas a partir de una colección y un conjunto, 
  el segundo apartado dedicado a la demostración del \isa{Teorema\ de\ Existencia\ de\ Modelo} y el tercer apartado donde se demuestra el \isa{Teorema\ de\ Compacidad}. 

  En la primera parte del cuarto capítulo se definen ciertas sucesiones 
  monótonas \isa{{\isacharbraceleft}S\isactrlsub n{\isacharbraceright}} de conjuntos de fórmulas a partir de una colección \isa{C} y un 
  conjunto \isa{S\ {\isasymin}\ C}. De este modo, se prueban distintas propiedades sobre dichas 
  sucesiones de conjuntos que permiten la prueba del \isa{Teorema\ de\ Existencia\ de\ Modelo}. En primer lugar, se prueba que si \isa{C} verifica la propiedad de 
  consistencia proposicional, entonces todo elemento de la suceción \isa{{\isacharbraceleft}S\isactrlsub n{\isacharbraceright}} 
  pertenece a la colección. Igualmente, se dará un resultado que permite 
  caracterizar conjuntos de \isa{{\isacharbraceleft}S\isactrlsub n{\isacharbraceright}} en función de los anteriores. Por otro lado, 
  definiremos el límite de dichas sucesiones, probando que todo conjunto de 
  \isa{{\isacharbraceleft}S\isactrlsub n{\isacharbraceright}} está contenido en el límite. Además, se demostrará que si una fórmula 
  pertenece al límite, entonces pertenece a algún conjunto de la sucesión \isa{{\isacharbraceleft}S\isactrlsub n{\isacharbraceright}}. 
  Finalmente, se mostrará un resultado sobre conjuntos finitos contenidos en el 
  límite.

  En la segunda parte del capítulo se demuestra el \isa{Teorema\ de\ Existencia\ de\ Modelo}, que prueba que todo conjunto de fórmulas \isa{S} perteneciente a una 
  colección \isa{C} que verifique la propiedad de consistencia proposicional es 
  satisfacible. Para ello, extenderemos la colección \isa{C} a otra \isa{C{\isacharprime}} que 
  tenga la propiedad de consistencia proposicional, sea cerrada bajo 
  subconjuntos y sea de carácter finito, de modo que se introducirán
  distintos resultados sobre colecciones con las características anteriores. 
  En primer lugar, probaremos que el límite de la sucesión \isa{{\isacharbraceleft}S\isactrlsub n{\isacharbraceright}} definida a 
  partir de \isa{C{\isacharprime}} y \isa{S} pertenece a \isa{C{\isacharprime}}. En particular se probará que dicho 
  límite es un elemento maximal de la colección que lo define si esta es 
  cerrada bajo subconjuntos y verifica la propiedad de consistencia 
  proposicional. Por otra parte se demostrará que el límite de \isa{{\isacharbraceleft}S\isactrlsub n{\isacharbraceright}} es un 
  conjunto de Hintikka si está definido a partir de una colección \isa{C{\isacharprime}} con 
  las propiedades descritas luego, por el \isa{Teorema\ de\ Hintikka}, es
  satisfacible. Finalmente, como \isa{S\ {\isasymin}\ C} pertenece también a la extensión \isa{C{\isacharprime}}, 
  se verifica que está contenido en el límite de la sucesión \isa{{\isacharbraceleft}S\isactrlsub n{\isacharbraceright}} definida a 
  partir de \isa{C{\isacharprime}} y\\ \isa{S\ {\isasymin}\ C{\isacharprime}}. Por tanto, quedará demostrada la satisfacibilidad 
  del conjunto \isa{S} al heredarla por contención del límite.

  Por último, el apartado final del capítulo se dedica a la demostración del 
  \isa{Teorema\ de\ Compacidad} que prueba que todo conjunto de fórmulas finitamente 
  satisfacible es satisfacible. Para su demostración consideraremos la 
  colección formada por los conjuntos de fórmulas finitamente satisfacibles. 
  Probaremos que dicha colección verifica la propiedad de consistencia 
  proposicional y, por el \isa{Teorema\ de\ Existencia\ de\ Modelo}, todo conjunto 
  perteneciente a ella será satisfacible. 

  En lo referente a las demostraciones asistidas por Isabelle/HOL de
  los resultados formalizados a lo largo de las secciones, se elaborarán 
  dos tipos de pruebas correspondientes a dos tácticas distintas. En 
  primer lugar, se probará cada resultado siguiendo un esquema de 
  demostración detallado. En él utilizaremos únicamente y de manera 
  precisa las reglas de simplificación y definiciones incluidas en la 
  librería de Isabelle, prescindiendo de las herramientas de 
  razonamiento automático del demostrador. Para ello, se realiza una 
  búsqueda inversa en cada paso de la demostración automática hasta 
  llegar a un desarrollo de la prueba basado en deducciones a partir de
  resultados elementales que la completen de manera rigurosa. En 
  contraposición, se evidenciará la capacidad de razonamiento 
  automático de Isabelle/HOL mediante la realización de una prueba 
  alternativa siguiendo un esquema de demostración automático. Para 
  ello se utilizarán las herramientas de razonamiento que han sido 
  elaboradas en Isabelle/HOL con el objetivo de realizar deducciones de 
  la manera más eficiente.

  \isa{Este\ trabajo\ está\ disponible\ en\ la\ plataforma} GitHub \isa{mediante\ el\ siguiente\ enlace{\isacharcolon}} 

\href{https://github.com/sofsanfer/TFM}
                  {\url{https://github.com/sofsanfer/TFM}}%
\end{isamarkuptext}\isamarkuptrue%
%
\isadelimtheory
%
\endisadelimtheory
%
\isatagtheory
%
\endisatagtheory
{\isafoldtheory}%
%
\isadelimtheory
%
\endisadelimtheory
%
\end{isabellebody}%
\endinput
%:%file=~/TFM/TFM/Introduccion.thy%:%
%:%19=14%:%
%:%20=15%:%
%:%21=16%:%
%:%22=17%:%
%:%23=18%:%
%:%24=19%:%
%:%25=20%:%
%:%26=21%:%
%:%27=22%:%
%:%28=23%:%
%:%29=24%:%
%:%30=25%:%
%:%31=26%:%
%:%32=27%:%
%:%33=28%:%
%:%34=29%:%
%:%35=30%:%
%:%36=31%:%
%:%37=32%:%
%:%38=33%:%
%:%39=34%:%
%:%40=35%:%
%:%41=36%:%
%:%42=37%:%
%:%43=38%:%
%:%44=39%:%
%:%45=40%:%
%:%46=41%:%
%:%47=42%:%
%:%48=43%:%
%:%49=44%:%
%:%50=45%:%
%:%51=46%:%
%:%52=47%:%
%:%53=48%:%
%:%54=49%:%
%:%55=50%:%
%:%56=51%:%
%:%57=52%:%
%:%58=53%:%
%:%59=54%:%
%:%60=55%:%
%:%61=56%:%
%:%62=57%:%
%:%63=58%:%
%:%64=59%:%
%:%65=60%:%
%:%66=61%:%
%:%67=62%:%
%:%68=63%:%
%:%69=64%:%
%:%70=65%:%
%:%71=66%:%
%:%72=67%:%
%:%73=68%:%
%:%74=69%:%
%:%75=70%:%
%:%76=71%:%
%:%77=72%:%
%:%78=73%:%
%:%79=74%:%
%:%80=75%:%
%:%81=76%:%
%:%82=77%:%
%:%83=78%:%
%:%84=79%:%
%:%85=80%:%
%:%85=81%:%
%:%86=82%:%
%:%87=83%:%
%:%88=84%:%
%:%89=85%:%
%:%90=86%:%
%:%91=87%:%
%:%92=88%:%
%:%93=89%:%
%:%94=90%:%
%:%95=91%:%
%:%96=92%:%
%:%97=93%:%
%:%97=94%:%
%:%98=95%:%
%:%99=96%:%
%:%100=97%:%
%:%101=98%:%
%:%102=99%:%
%:%103=100%:%
%:%104=101%:%
%:%104=102%:%
%:%105=103%:%
%:%106=104%:%
%:%107=105%:%
%:%108=106%:%
%:%109=107%:%
%:%110=108%:%
%:%111=109%:%
%:%112=110%:%
%:%113=111%:%
%:%114=112%:%
%:%115=113%:%
%:%116=114%:%
%:%117=115%:%
%:%118=116%:%
%:%119=117%:%
%:%120=118%:%
%:%121=119%:%
%:%122=120%:%
%:%123=121%:%
%:%124=122%:%
%:%125=123%:%
%:%126=124%:%
%:%127=125%:%
%:%128=126%:%
%:%129=127%:%
%:%130=128%:%
%:%131=129%:%
%:%132=130%:%
%:%133=131%:%
%:%134=132%:%
%:%135=133%:%
%:%136=134%:%
%:%137=135%:%
%:%138=136%:%
%:%139=137%:%
%:%140=138%:%
%:%141=139%:%
%:%142=140%:%
%:%143=141%:%
%:%144=142%:%
%:%145=143%:%
%:%145=144%:%
%:%146=145%:%
%:%147=146%:%
%:%148=147%:%
%:%148=148%:%
%:%149=149%:%
%:%150=150%:%
%:%151=151%:%
%:%152=152%:%
%:%153=153%:%
%:%153=154%:%
%:%154=155%:%
%:%155=156%:%
%:%156=157%:%
%:%157=158%:%
%:%158=159%:%
%:%159=160%:%
%:%160=161%:%
%:%161=162%:%
%:%162=163%:%
%:%163=164%:%
%:%163=165%:%
%:%164=166%:%
%:%165=167%:%
%:%166=168%:%
%:%167=169%:%
%:%168=170%:%
%:%169=171%:%
%:%170=172%:%
%:%171=173%:%
%:%172=174%:%
%:%173=175%:%
%:%174=176%:%
%:%175=177%:%
%:%176=178%:%
%:%177=179%:%
%:%178=180%:%
%:%179=181%:%
%:%180=182%:%
%:%181=183%:%
%:%182=184%:%
%:%183=185%:%
%:%184=186%:%
%:%185=187%:%
%:%186=188%:%
%:%187=189%:%
%:%188=190%:%
%:%189=191%:%
%:%190=192%:%
%:%191=193%:%
%:%192=194%:%
%:%193=195%:%
%:%194=196%:%
%:%195=197%:%
%:%196=198%:%
%:%197=199%:%
%:%198=200%:%
%:%199=201%:%
%:%200=202%:%
%:%201=203%:%
%:%202=204%:%
%:%203=205%:%
%:%204=206%:%
%:%205=207%:%
%:%206=208%:%
%:%207=209%:%
%:%207=210%:%
%:%208=211%:%
%:%209=212%:%
%:%210=213%:%

\chapter{Notación uniforme}
%
\begin{isabellebody}%
\setisabellecontext{Notunif}%
%
\isadelimtheory
%
\endisadelimtheory
%
\isatagtheory
%
\endisatagtheory
{\isafoldtheory}%
%
\isadelimtheory
%
\endisadelimtheory
%
\begin{isamarkuptext}%
\comentario{Localización de sello.png.}
\comentario{Cambiar los directores}
\comentario{Introducción. Mirar fitting p. 53 y 54}%
\end{isamarkuptext}\isamarkuptrue%
%
\begin{isamarkuptext}%
En esta sección introduciremos la notación uniforme inicialmente 
  desarrollada por \isa{R{\isachardot}\ M{\isachardot}\ Smullyan} (añadir referencia bibliográfica). La finalidad
  de dicha notación es reducir el número de casos a considerar sobre la estructura de 
  las fórmulas al clasificar éstas en dos categorías, facilitando las demostraciones
  y métodos empleados en adelante.

  \comentario{Añadir referencia bibliográfica.}

  De este modo, las fórmulas proposicionales pueden ser de dos tipos: aquellas que 
  de tipo conjuntivo (las fórmulas \isa{{\isasymalpha}}) y las de tipo disyuntivo (las fórmulas \isa{{\isasymbeta}}). 
  Cada fórmula de tipo \isa{{\isasymalpha}}, o \isa{{\isasymbeta}} respectivamente, tiene asociada sus  
  dos componentes \isa{{\isasymalpha}\isactrlsub {\isadigit{1}}} y \isa{{\isasymalpha}\isactrlsub {\isadigit{2}}}, o \isa{{\isasymbeta}\isactrlsub {\isadigit{1}}} y \isa{{\isasymbeta}\isactrlsub {\isadigit{2}}} respectivamente. Para justificar dicha clasificación,
  introduzcamos inicialmente la definición de fórmulas semánticamente equivalentes.

  \begin{definicion}
    Dos fórmulas son \isa{semánticamente\ equivalentes} si tienen el mismo valor para toda 
    interpretación.
  \end{definicion}

  En Isabelle podemos formalizar la definición de la siguiente manera.%
\end{isamarkuptext}\isamarkuptrue%
\isacommand{definition}\isamarkupfalse%
\ {\isachardoublequoteopen}semanticEq\ F\ G\ {\isasymequiv}\ {\isasymforall}{\isasymA}{\isachardot}\ {\isacharparenleft}{\isasymA}\ {\isasymTurnstile}\ F{\isacharparenright}\ {\isasymlongleftrightarrow}\ {\isacharparenleft}{\isasymA}\ {\isasymTurnstile}\ G{\isacharparenright}{\isachardoublequoteclose}%
\begin{isamarkuptext}%
Según la definición del valor de verdad de una fórmula proposicional en una 
  interpretación dada, podemos ver los siguientes ejemplos de fórmulas semánticamente equivalentes.%
\end{isamarkuptext}\isamarkuptrue%
\isacommand{lemma}\isamarkupfalse%
\ {\isachardoublequoteopen}semanticEq\ {\isacharparenleft}Atom\ p{\isacharparenright}\ {\isacharparenleft}{\isacharparenleft}Atom\ p{\isacharparenright}\ \isactrlbold {\isasymor}\ {\isacharparenleft}Atom\ p{\isacharparenright}{\isacharparenright}{\isachardoublequoteclose}\ \isanewline
%
\isadelimproof
\ \ %
\endisadelimproof
%
\isatagproof
\isacommand{by}\isamarkupfalse%
\ {\isacharparenleft}simp\ add{\isacharcolon}\ semanticEq{\isacharunderscore}def{\isacharparenright}%
\endisatagproof
{\isafoldproof}%
%
\isadelimproof
\isanewline
%
\endisadelimproof
\isanewline
\isacommand{lemma}\isamarkupfalse%
\ {\isachardoublequoteopen}semanticEq\ {\isacharparenleft}Atom\ p{\isacharparenright}\ {\isacharparenleft}{\isacharparenleft}Atom\ p{\isacharparenright}\ \isactrlbold {\isasymand}\ {\isacharparenleft}Atom\ p{\isacharparenright}{\isacharparenright}{\isachardoublequoteclose}\ \isanewline
%
\isadelimproof
\ \ %
\endisadelimproof
%
\isatagproof
\isacommand{by}\isamarkupfalse%
\ {\isacharparenleft}simp\ add{\isacharcolon}\ semanticEq{\isacharunderscore}def{\isacharparenright}%
\endisatagproof
{\isafoldproof}%
%
\isadelimproof
\isanewline
%
\endisadelimproof
\isanewline
\isacommand{lemma}\isamarkupfalse%
\ {\isachardoublequoteopen}semanticEq\ {\isasymbottom}\ {\isacharparenleft}{\isasymbottom}\ \isactrlbold {\isasymand}\ {\isasymbottom}{\isacharparenright}{\isachardoublequoteclose}\ \isanewline
%
\isadelimproof
\ \ %
\endisadelimproof
%
\isatagproof
\isacommand{by}\isamarkupfalse%
\ {\isacharparenleft}simp\ add{\isacharcolon}\ semanticEq{\isacharunderscore}def{\isacharparenright}%
\endisatagproof
{\isafoldproof}%
%
\isadelimproof
\isanewline
%
\endisadelimproof
\isanewline
\isacommand{lemma}\isamarkupfalse%
\ {\isachardoublequoteopen}semanticEq\ {\isasymbottom}\ {\isacharparenleft}{\isasymbottom}\ \isactrlbold {\isasymor}\ {\isasymbottom}{\isacharparenright}{\isachardoublequoteclose}\ \isanewline
%
\isadelimproof
\ \ %
\endisadelimproof
%
\isatagproof
\isacommand{by}\isamarkupfalse%
\ {\isacharparenleft}simp\ add{\isacharcolon}\ semanticEq{\isacharunderscore}def{\isacharparenright}%
\endisatagproof
{\isafoldproof}%
%
\isadelimproof
\isanewline
%
\endisadelimproof
\isanewline
\isacommand{lemma}\isamarkupfalse%
\ {\isachardoublequoteopen}semanticEq\ {\isasymbottom}\ {\isacharparenleft}\isactrlbold {\isasymnot}\ {\isasymtop}{\isacharparenright}{\isachardoublequoteclose}\isanewline
%
\isadelimproof
\ \ %
\endisadelimproof
%
\isatagproof
\isacommand{by}\isamarkupfalse%
\ {\isacharparenleft}simp\ add{\isacharcolon}\ semanticEq{\isacharunderscore}def\ top{\isacharunderscore}semantics{\isacharparenright}%
\endisatagproof
{\isafoldproof}%
%
\isadelimproof
\isanewline
%
\endisadelimproof
\isanewline
\isacommand{lemma}\isamarkupfalse%
\ {\isachardoublequoteopen}semanticEq\ F\ {\isacharparenleft}\isactrlbold {\isasymnot}{\isacharparenleft}\isactrlbold {\isasymnot}\ F{\isacharparenright}{\isacharparenright}{\isachardoublequoteclose}\isanewline
%
\isadelimproof
\ \ %
\endisadelimproof
%
\isatagproof
\isacommand{by}\isamarkupfalse%
\ {\isacharparenleft}simp\ add{\isacharcolon}\ semanticEq{\isacharunderscore}def{\isacharparenright}%
\endisatagproof
{\isafoldproof}%
%
\isadelimproof
\isanewline
%
\endisadelimproof
\isanewline
\isacommand{lemma}\isamarkupfalse%
\ {\isachardoublequoteopen}semanticEq\ {\isacharparenleft}\isactrlbold {\isasymnot}{\isacharparenleft}\isactrlbold {\isasymnot}\ F{\isacharparenright}{\isacharparenright}\ {\isacharparenleft}F\ \isactrlbold {\isasymor}\ F{\isacharparenright}{\isachardoublequoteclose}\isanewline
%
\isadelimproof
\ \ %
\endisadelimproof
%
\isatagproof
\isacommand{by}\isamarkupfalse%
\ {\isacharparenleft}simp\ add{\isacharcolon}\ semanticEq{\isacharunderscore}def{\isacharparenright}%
\endisatagproof
{\isafoldproof}%
%
\isadelimproof
\isanewline
%
\endisadelimproof
\isanewline
\isacommand{lemma}\isamarkupfalse%
\ {\isachardoublequoteopen}semanticEq\ {\isacharparenleft}\isactrlbold {\isasymnot}{\isacharparenleft}\isactrlbold {\isasymnot}\ F{\isacharparenright}{\isacharparenright}\ {\isacharparenleft}F\ \isactrlbold {\isasymand}\ F{\isacharparenright}{\isachardoublequoteclose}\isanewline
%
\isadelimproof
\ \ %
\endisadelimproof
%
\isatagproof
\isacommand{by}\isamarkupfalse%
\ {\isacharparenleft}simp\ add{\isacharcolon}\ semanticEq{\isacharunderscore}def{\isacharparenright}%
\endisatagproof
{\isafoldproof}%
%
\isadelimproof
\isanewline
%
\endisadelimproof
\isanewline
\isacommand{lemma}\isamarkupfalse%
\ {\isachardoublequoteopen}semanticEq\ {\isacharparenleft}\isactrlbold {\isasymnot}\ F\ \isactrlbold {\isasymand}\ \isactrlbold {\isasymnot}\ G{\isacharparenright}\ {\isacharparenleft}\isactrlbold {\isasymnot}{\isacharparenleft}F\ \isactrlbold {\isasymor}\ G{\isacharparenright}{\isacharparenright}{\isachardoublequoteclose}\isanewline
%
\isadelimproof
\ \ %
\endisadelimproof
%
\isatagproof
\isacommand{by}\isamarkupfalse%
\ {\isacharparenleft}simp\ add{\isacharcolon}\ semanticEq{\isacharunderscore}def{\isacharparenright}%
\endisatagproof
{\isafoldproof}%
%
\isadelimproof
\isanewline
%
\endisadelimproof
\isanewline
\isacommand{lemma}\isamarkupfalse%
\ {\isachardoublequoteopen}semanticEq\ {\isacharparenleft}F\ \isactrlbold {\isasymrightarrow}\ G{\isacharparenright}\ {\isacharparenleft}\isactrlbold {\isasymnot}\ F\ \isactrlbold {\isasymor}\ G{\isacharparenright}{\isachardoublequoteclose}\isanewline
%
\isadelimproof
\ \ %
\endisadelimproof
%
\isatagproof
\isacommand{by}\isamarkupfalse%
\ {\isacharparenleft}simp\ add{\isacharcolon}\ semanticEq{\isacharunderscore}def{\isacharparenright}%
\endisatagproof
{\isafoldproof}%
%
\isadelimproof
%
\endisadelimproof
%
\begin{isamarkuptext}%
En contraposición, también podemos dar ejemplos de fórmulas que no son semánticamente 
  equivalentes.%
\end{isamarkuptext}\isamarkuptrue%
\isacommand{lemma}\isamarkupfalse%
\ {\isachardoublequoteopen}{\isasymnot}\ semanticEq\ {\isacharparenleft}Atom\ p{\isacharparenright}\ {\isacharparenleft}\isactrlbold {\isasymnot}{\isacharparenleft}Atom\ p{\isacharparenright}{\isacharparenright}{\isachardoublequoteclose}\isanewline
%
\isadelimproof
\ \ %
\endisadelimproof
%
\isatagproof
\isacommand{by}\isamarkupfalse%
\ {\isacharparenleft}simp\ add{\isacharcolon}\ semanticEq{\isacharunderscore}def{\isacharparenright}%
\endisatagproof
{\isafoldproof}%
%
\isadelimproof
\isanewline
%
\endisadelimproof
\isanewline
\isacommand{lemma}\isamarkupfalse%
\ {\isachardoublequoteopen}{\isasymnot}\ semanticEq\ {\isasymbottom}\ {\isasymtop}{\isachardoublequoteclose}\isanewline
%
\isadelimproof
\ \ %
\endisadelimproof
%
\isatagproof
\isacommand{by}\isamarkupfalse%
\ {\isacharparenleft}simp\ add{\isacharcolon}\ semanticEq{\isacharunderscore}def\ top{\isacharunderscore}semantics{\isacharparenright}%
\endisatagproof
{\isafoldproof}%
%
\isadelimproof
%
\endisadelimproof
%
\begin{isamarkuptext}%
Por tanto, diremos intuitivamente que una fórmula es de tipo \isa{{\isasymalpha}} con componentes \isa{{\isasymalpha}\isactrlsub {\isadigit{1}}} y \isa{{\isasymalpha}\isactrlsub {\isadigit{2}}}
  si es semánticamente equivalente a la fórmula \isa{{\isasymalpha}\isactrlsub {\isadigit{1}}\ {\isasymand}\ {\isasymalpha}\isactrlsub {\isadigit{2}}}. Del mismo modo, una fórmula será de tipo
  \isa{{\isasymbeta}} con componentes \isa{{\isasymbeta}\isactrlsub {\isadigit{1}}} y \isa{{\isasymbeta}\isactrlsub {\isadigit{2}}} si es semánticamente equivalente a la fórmula \isa{{\isasymbeta}\isactrlsub {\isadigit{1}}\ {\isasymor}\ {\isasymbeta}\isactrlsub {\isadigit{2}}}.

  \begin{definicion}
    Las fórmulas de tipo \isa{{\isasymalpha}} (\isa{fórmulas\ conjuntivas}) y sus correspondientes componentes
    \isa{{\isasymalpha}\isactrlsub {\isadigit{1}}} y \isa{{\isasymalpha}\isactrlsub {\isadigit{2}}} se definen como sigue: dadas \isa{F} y \isa{G} fórmulas cualesquiera,
    \begin{enumerate}
      \item \isa{F\ {\isasymand}\ G} es una fórmula de tipo \isa{{\isasymalpha}} cuyas componentes son \isa{F} y \isa{G}.
      \item \isa{{\isasymnot}{\isacharparenleft}F\ {\isasymor}\ G{\isacharparenright}} es una fórmula de tipo \isa{{\isasymalpha}} cuyas componentes son \isa{{\isasymnot}\ F} y \isa{{\isasymnot}\ G}.
      \item \isa{{\isasymnot}{\isacharparenleft}F\ {\isasymlongrightarrow}\ G{\isacharparenright}} es una fórmula de tipo \isa{{\isasymalpha}} cuyas componentes son \isa{F} y \isa{{\isasymnot}\ G}.
    \end{enumerate} 
  \end{definicion}

  De este modo, de los ejemplos anteriores podemos deducir que las fórmulas atómicas son de tipo \isa{{\isasymalpha}}
  y sus componentes \isa{{\isasymalpha}\isactrlsub {\isadigit{1}}} y \isa{{\isasymalpha}\isactrlsub {\isadigit{2}}} son la propia fórmula. Del mismo modo, la constante \isa{{\isasymbottom}} también es 
  una fórmula conjuntiva cuyas componentes son ella misma. Por último, podemos observar que dada
  una fórmula cualquiera \isa{F}, su doble negación \isa{{\isasymnot}{\isacharparenleft}{\isasymnot}\ F{\isacharparenright}} es una fórmula de tipo \isa{{\isasymalpha}} y componentes
  \isa{F} y \isa{F}.

  Formalizaremos en Isabelle el conjunto de fórmulas \isa{{\isasymalpha}} como un predicato inductivo. De este modo,
  las reglas anteriores que construyen el conjunto de fórmulas de tipo \isa{{\isasymalpha}} se formalizan en Isabelle 
  como reglas de introducción. Además, añadiremos explícitamente una cuarta regla que introduce la 
  doble negación de una fórmula como fórmula de tipo \isa{{\isasymalpha}}. De este modo, facilitaremos la prueba de 
  resultados posteriores relacionados con la definición de conjunto de Hintikka, que constituyen una
  base para la demostración del \isa{teorema\ de\ existencia\ de\ modelo}.%
\end{isamarkuptext}\isamarkuptrue%
\isacommand{inductive}\isamarkupfalse%
\ Con\ {\isacharcolon}{\isacharcolon}\ {\isachardoublequoteopen}{\isacharprime}a\ formula\ {\isacharequal}{\isachargreater}\ {\isacharprime}a\ formula\ {\isacharequal}{\isachargreater}\ {\isacharprime}a\ formula\ {\isacharequal}{\isachargreater}\ bool{\isachardoublequoteclose}\ \isakeyword{where}\isanewline
{\isachardoublequoteopen}Con\ {\isacharparenleft}And\ F\ G{\isacharparenright}\ F\ G{\isachardoublequoteclose}\ {\isacharbar}\isanewline
{\isachardoublequoteopen}Con\ {\isacharparenleft}Not\ {\isacharparenleft}Or\ F\ G{\isacharparenright}{\isacharparenright}\ {\isacharparenleft}Not\ F{\isacharparenright}\ {\isacharparenleft}Not\ G{\isacharparenright}{\isachardoublequoteclose}\ {\isacharbar}\isanewline
{\isachardoublequoteopen}Con\ {\isacharparenleft}Not\ {\isacharparenleft}Imp\ F\ G{\isacharparenright}{\isacharparenright}\ F\ {\isacharparenleft}Not\ G{\isacharparenright}{\isachardoublequoteclose}\ {\isacharbar}\isanewline
{\isachardoublequoteopen}Con\ {\isacharparenleft}Not\ {\isacharparenleft}Not\ F{\isacharparenright}{\isacharparenright}\ F\ F{\isachardoublequoteclose}%
\begin{isamarkuptext}%
Las reglas de introducción que proporciona la definición anterior son
  las siguientes.

  \begin{itemize}
    \item[] \isa{Con\ {\isacharparenleft}F\ \isactrlbold {\isasymand}\ G{\isacharparenright}\ F\ G\isasep\isanewline%
Con\ {\isacharparenleft}\isactrlbold {\isasymnot}\ {\isacharparenleft}F\ \isactrlbold {\isasymor}\ G{\isacharparenright}{\isacharparenright}\ {\isacharparenleft}\isactrlbold {\isasymnot}\ F{\isacharparenright}\ {\isacharparenleft}\isactrlbold {\isasymnot}\ G{\isacharparenright}\isasep\isanewline%
Con\ {\isacharparenleft}\isactrlbold {\isasymnot}\ {\isacharparenleft}F\ \isactrlbold {\isasymrightarrow}\ G{\isacharparenright}{\isacharparenright}\ F\ {\isacharparenleft}\isactrlbold {\isasymnot}\ G{\isacharparenright}\isasep\isanewline%
Con\ {\isacharparenleft}\isactrlbold {\isasymnot}\ {\isacharparenleft}\isactrlbold {\isasymnot}\ F{\isacharparenright}{\isacharparenright}\ F\ F} 
      \hfill (\isa{Con{\isachardot}intros})
  \end{itemize}
  
  Por otro lado, definamos las fórmulas disyuntivas.

  \begin{definicion}
    Las fórmulas de tipo \isa{{\isasymbeta}} (\isa{fórmulas\ disyuntivas}) y sus correspondientes componentes
    \isa{{\isasymbeta}\isactrlsub {\isadigit{1}}} y \isa{{\isasymbeta}\isactrlsub {\isadigit{2}}} se definen como sigue: dadas \isa{F} y \isa{G} fórmulas cualesquiera,
    \begin{enumerate}
      \item \isa{F\ {\isasymor}\ G} es una fórmula de tipo \isa{{\isasymbeta}} cuyas componentes son \isa{F} y \isa{G}.
      \item \isa{F\ {\isasymlongrightarrow}\ G} es una fórmula de tipo \isa{{\isasymbeta}} cuyas componentes son \isa{{\isasymnot}\ F} y \isa{G}.
      \item \isa{{\isasymnot}{\isacharparenleft}F\ {\isasymand}\ G{\isacharparenright}} es una fórmula de tipo \isa{{\isasymbeta}} cuyas componentes son \isa{{\isasymnot}\ F} y \isa{{\isasymnot}\ G}.
    \end{enumerate} 
  \end{definicion}

  De los ejemplos dados anteriormente, podemos deducir análogamente que las fórmulas atómicas, la
  constante \isa{{\isasymbottom}} y la doble negación sob también fórmulas disyuntivas con las mismas componentes que
  las dadas para el tipo conjuntivo.

  Del mismo modo, su formalización se realiza como un predicado inductivo, de manera que las reglas 
  que definen el conjunto de fórmulas de tipo \isa{{\isasymbeta}} se formalizan en Isabelle como reglas de 
  introducción. Análogamente, introduciremos de manera explícita una regla que señala que la doble 
  negación de una fórmula es una fórmula de tipo disyuntivo.%
\end{isamarkuptext}\isamarkuptrue%
\isacommand{inductive}\isamarkupfalse%
\ Dis\ {\isacharcolon}{\isacharcolon}\ {\isachardoublequoteopen}{\isacharprime}a\ formula\ {\isacharequal}{\isachargreater}\ {\isacharprime}a\ formula\ {\isacharequal}{\isachargreater}\ {\isacharprime}a\ formula\ {\isacharequal}{\isachargreater}\ bool{\isachardoublequoteclose}\ \isakeyword{where}\isanewline
{\isachardoublequoteopen}Dis\ {\isacharparenleft}Or\ F\ G{\isacharparenright}\ F\ G{\isachardoublequoteclose}\ {\isacharbar}\isanewline
{\isachardoublequoteopen}Dis\ {\isacharparenleft}Imp\ F\ G{\isacharparenright}\ {\isacharparenleft}Not\ F{\isacharparenright}\ G{\isachardoublequoteclose}\ {\isacharbar}\isanewline
{\isachardoublequoteopen}Dis\ {\isacharparenleft}Not\ {\isacharparenleft}And\ F\ G{\isacharparenright}{\isacharparenright}\ {\isacharparenleft}Not\ F{\isacharparenright}\ {\isacharparenleft}Not\ G{\isacharparenright}{\isachardoublequoteclose}\ {\isacharbar}\isanewline
{\isachardoublequoteopen}Dis\ {\isacharparenleft}Not\ {\isacharparenleft}Not\ F{\isacharparenright}{\isacharparenright}\ F\ F{\isachardoublequoteclose}%
\begin{isamarkuptext}%
Las reglas de introducción que proporciona esta formalización se muestran a continuación.

  \begin{itemize}
    \item[] \isa{Dis\ {\isacharparenleft}F\ \isactrlbold {\isasymor}\ G{\isacharparenright}\ F\ G\isasep\isanewline%
Dis\ {\isacharparenleft}F\ \isactrlbold {\isasymrightarrow}\ G{\isacharparenright}\ {\isacharparenleft}\isactrlbold {\isasymnot}\ F{\isacharparenright}\ G\isasep\isanewline%
Dis\ {\isacharparenleft}\isactrlbold {\isasymnot}\ {\isacharparenleft}F\ \isactrlbold {\isasymand}\ G{\isacharparenright}{\isacharparenright}\ {\isacharparenleft}\isactrlbold {\isasymnot}\ F{\isacharparenright}\ {\isacharparenleft}\isactrlbold {\isasymnot}\ G{\isacharparenright}\isasep\isanewline%
Dis\ {\isacharparenleft}\isactrlbold {\isasymnot}\ {\isacharparenleft}\isactrlbold {\isasymnot}\ F{\isacharparenright}{\isacharparenright}\ F\ F} 
      \hfill (\isa{Dis{\isachardot}intros})
  \end{itemize}

  Cabe observar que las formalizaciones de la definiciones de fórmulas de tipo \isa{{\isasymalpha}} y \isa{{\isasymbeta}} son 
  definiciones sintácticas, pues construyen los correspondientes conjuntos de fórmulas a partir de 
  una reglas sintácticas concretas. Se trata de una simplificación de la intuición original de la 
  clasificación de las fórmulas mediante notación uniforme, ya que se prescinde de la noción de 
  equivalencia semántica que permite clasificar la totalidad de las fórmulas proposicionales. 

  Veamos la clasificación de casos concretos de fórmulas. Por ejemplo, según hemos definido la 
  fórmula \isa{{\isasymtop}}, es sencillo comprobar que se trata de una fórmula disyuntiva.%
\end{isamarkuptext}\isamarkuptrue%
\isacommand{lemma}\isamarkupfalse%
\ {\isachardoublequoteopen}Dis\ {\isasymtop}\ {\isacharparenleft}\isactrlbold {\isasymnot}\ {\isasymbottom}{\isacharparenright}\ {\isasymbottom}{\isachardoublequoteclose}\ \isanewline
%
\isadelimproof
\ \ %
\endisadelimproof
%
\isatagproof
\isacommand{unfolding}\isamarkupfalse%
\ Top{\isacharunderscore}def\ \isacommand{by}\isamarkupfalse%
\ {\isacharparenleft}simp\ only{\isacharcolon}\ Dis{\isachardot}intros{\isacharparenleft}{\isadigit{2}}{\isacharparenright}{\isacharparenright}%
\endisatagproof
{\isafoldproof}%
%
\isadelimproof
%
\endisadelimproof
%
\begin{isamarkuptext}%
Por otro lado, se observa a partir de las correspondientes definiciones que la conjunción
  generalizada de una lista de fórmulas es una fórmula de tipo \isa{{\isasymalpha}} y la disyunción generalizada de
  una lista de fórmulas es una fórmula de tipo \isa{{\isasymbeta}}.%
\end{isamarkuptext}\isamarkuptrue%
\isacommand{lemma}\isamarkupfalse%
\ {\isachardoublequoteopen}Con\ {\isacharparenleft}\isactrlbold {\isasymAnd}{\isacharparenleft}F{\isacharhash}Fs{\isacharparenright}{\isacharparenright}\ F\ {\isacharparenleft}\isactrlbold {\isasymAnd}Fs{\isacharparenright}{\isachardoublequoteclose}\isanewline
%
\isadelimproof
\ \ %
\endisadelimproof
%
\isatagproof
\isacommand{by}\isamarkupfalse%
\ {\isacharparenleft}simp\ only{\isacharcolon}\ BigAnd{\isachardot}simps\ Con{\isachardot}intros{\isacharparenleft}{\isadigit{1}}{\isacharparenright}{\isacharparenright}%
\endisatagproof
{\isafoldproof}%
%
\isadelimproof
\isanewline
%
\endisadelimproof
\isanewline
\isacommand{lemma}\isamarkupfalse%
\ {\isachardoublequoteopen}Dis\ {\isacharparenleft}\isactrlbold {\isasymOr}{\isacharparenleft}F{\isacharhash}Fs{\isacharparenright}{\isacharparenright}\ F\ {\isacharparenleft}\isactrlbold {\isasymOr}Fs{\isacharparenright}{\isachardoublequoteclose}\isanewline
%
\isadelimproof
\ \ %
\endisadelimproof
%
\isatagproof
\isacommand{by}\isamarkupfalse%
\ {\isacharparenleft}simp\ only{\isacharcolon}\ BigOr{\isachardot}simps\ Dis{\isachardot}intros{\isacharparenleft}{\isadigit{1}}{\isacharparenright}{\isacharparenright}%
\endisatagproof
{\isafoldproof}%
%
\isadelimproof
%
\endisadelimproof
%
\begin{isamarkuptext}%
Finalmente, de las reglas que definen las fórmulas conjuntivas y disyuntivas se deduce que
  la doble negación de una fórmula es una fórmula perteneciente a ambos tipos.%
\end{isamarkuptext}\isamarkuptrue%
\isacommand{lemma}\isamarkupfalse%
\ notDisCon{\isacharcolon}\ {\isachardoublequoteopen}Con\ {\isacharparenleft}Not\ {\isacharparenleft}Not\ F{\isacharparenright}{\isacharparenright}\ F\ F{\isachardoublequoteclose}\ {\isachardoublequoteopen}Dis\ {\isacharparenleft}Not\ {\isacharparenleft}Not\ F{\isacharparenright}{\isacharparenright}\ F\ F{\isachardoublequoteclose}\ \isanewline
%
\isadelimproof
\ \ %
\endisadelimproof
%
\isatagproof
\isacommand{by}\isamarkupfalse%
\ {\isacharparenleft}simp\ only{\isacharcolon}\ Con{\isachardot}intros{\isacharparenleft}{\isadigit{4}}{\isacharparenright}\ Dis{\isachardot}intros{\isacharparenleft}{\isadigit{4}}{\isacharparenright}{\isacharparenright}{\isacharplus}%
\endisatagproof
{\isafoldproof}%
%
\isadelimproof
%
\endisadelimproof
%
\begin{isamarkuptext}%
A continuación vamos a introducir el siguiente lema que caracteriza las fórmulas de tipo \isa{{\isasymalpha}} 
  y \isa{{\isasymbeta}}, facilitando el uso de la notación uniforme en Isabelle.%
\end{isamarkuptext}\isamarkuptrue%
\isacommand{lemma}\isamarkupfalse%
\ con{\isacharunderscore}dis{\isacharunderscore}simps{\isacharcolon}\isanewline
\ \ {\isachardoublequoteopen}Con\ a{\isadigit{1}}\ a{\isadigit{2}}\ a{\isadigit{3}}\ {\isacharequal}\ {\isacharparenleft}a{\isadigit{1}}\ {\isacharequal}\ a{\isadigit{2}}\ \isactrlbold {\isasymand}\ a{\isadigit{3}}\ {\isasymor}\ \isanewline
\ \ \ \ {\isacharparenleft}{\isasymexists}F\ G{\isachardot}\ a{\isadigit{1}}\ {\isacharequal}\ \isactrlbold {\isasymnot}\ {\isacharparenleft}F\ \isactrlbold {\isasymor}\ G{\isacharparenright}\ {\isasymand}\ a{\isadigit{2}}\ {\isacharequal}\ \isactrlbold {\isasymnot}\ F\ {\isasymand}\ a{\isadigit{3}}\ {\isacharequal}\ \isactrlbold {\isasymnot}\ G{\isacharparenright}\ {\isasymor}\ \isanewline
\ \ \ \ {\isacharparenleft}{\isasymexists}G{\isachardot}\ a{\isadigit{1}}\ {\isacharequal}\ \isactrlbold {\isasymnot}\ {\isacharparenleft}a{\isadigit{2}}\ \isactrlbold {\isasymrightarrow}\ G{\isacharparenright}\ {\isasymand}\ a{\isadigit{3}}\ {\isacharequal}\ \isactrlbold {\isasymnot}\ G{\isacharparenright}\ {\isasymor}\ \isanewline
\ \ \ \ a{\isadigit{1}}\ {\isacharequal}\ \isactrlbold {\isasymnot}\ {\isacharparenleft}\isactrlbold {\isasymnot}\ a{\isadigit{2}}{\isacharparenright}\ {\isasymand}\ a{\isadigit{3}}\ {\isacharequal}\ a{\isadigit{2}}{\isacharparenright}{\isachardoublequoteclose}\isanewline
\ \ {\isachardoublequoteopen}Dis\ a{\isadigit{1}}\ a{\isadigit{2}}\ a{\isadigit{3}}\ {\isacharequal}\ {\isacharparenleft}a{\isadigit{1}}\ {\isacharequal}\ a{\isadigit{2}}\ \isactrlbold {\isasymor}\ a{\isadigit{3}}\ {\isasymor}\ \isanewline
\ \ \ \ {\isacharparenleft}{\isasymexists}F\ G{\isachardot}\ a{\isadigit{1}}\ {\isacharequal}\ F\ \isactrlbold {\isasymrightarrow}\ G\ {\isasymand}\ a{\isadigit{2}}\ {\isacharequal}\ \isactrlbold {\isasymnot}\ F\ {\isasymand}\ a{\isadigit{3}}\ {\isacharequal}\ G{\isacharparenright}\ {\isasymor}\ \isanewline
\ \ \ \ {\isacharparenleft}{\isasymexists}F\ G{\isachardot}\ a{\isadigit{1}}\ {\isacharequal}\ \isactrlbold {\isasymnot}\ {\isacharparenleft}F\ \isactrlbold {\isasymand}\ G{\isacharparenright}\ {\isasymand}\ a{\isadigit{2}}\ {\isacharequal}\ \isactrlbold {\isasymnot}\ F\ {\isasymand}\ a{\isadigit{3}}\ {\isacharequal}\ \isactrlbold {\isasymnot}\ G{\isacharparenright}\ {\isasymor}\ \isanewline
\ \ \ \ a{\isadigit{1}}\ {\isacharequal}\ \isactrlbold {\isasymnot}\ {\isacharparenleft}\isactrlbold {\isasymnot}\ a{\isadigit{2}}{\isacharparenright}\ {\isasymand}\ a{\isadigit{3}}\ {\isacharequal}\ a{\isadigit{2}}{\isacharparenright}{\isachardoublequoteclose}\ \isanewline
%
\isadelimproof
\ \ %
\endisadelimproof
%
\isatagproof
\isacommand{by}\isamarkupfalse%
\ {\isacharparenleft}simp{\isacharunderscore}all\ add{\isacharcolon}\ Con{\isachardot}simps\ Dis{\isachardot}simps{\isacharparenright}%
\endisatagproof
{\isafoldproof}%
%
\isadelimproof
%
\endisadelimproof
%
\begin{isamarkuptext}%
Por último, introduzcamos un resultado que permite caracterizar los conjuntos de Hintikka 
  empleando la notación uniforme.

  \begin{lema}[Caracterización de los conjuntos de Hintikka mediante la notación uniforme]
    Dado un conjunto de fórmulas proposicionales \isa{S}, son equivalentes:
    \begin{enumerate}
      \item \isa{S} es un conjunto de Hintikka.
      \item Se verifican las condiciones siguientes:
      \begin{itemize}
        \item \isa{{\isasymbottom}} no pertenece a \isa{S}.
        \item Dada \isa{p} una fórmula atómica cualquiera, no se tiene 
        simultáneamente que\\ \isa{p\ {\isasymin}\ S} y \isa{{\isasymnot}\ p\ {\isasymin}\ S}.
        \item Para toda fórmula de tipo \isa{{\isasymalpha}} con componentes \isa{{\isasymalpha}\isactrlsub {\isadigit{1}}} y \isa{{\isasymalpha}\isactrlsub {\isadigit{2}}} se verifica 
        que si la fórmula pertenece a \isa{S}, entonces \isa{{\isasymalpha}\isactrlsub {\isadigit{1}}} y \isa{{\isasymalpha}\isactrlsub {\isadigit{2}}} también.
        \item Para toda fórmula de tipo \isa{{\isasymbeta}} con componentes \isa{{\isasymbeta}\isactrlsub {\isadigit{1}}} y \isa{{\isasymbeta}\isactrlsub {\isadigit{2}}} se verifica 
        que si la fórmula pertenece a \isa{S}, entonces o bien \isa{{\isasymbeta}\isactrlsub {\isadigit{1}}} pertenece
        a \isa{S} o bien \isa{{\isasymbeta}\isactrlsub {\isadigit{2}}} pertenece a \isa{S}.
      \end{itemize} 
    \end{enumerate}
  \end{lema}

  En Isabelle/HOL se formaliza del siguiente modo.%
\end{isamarkuptext}\isamarkuptrue%
\isacommand{lemma}\isamarkupfalse%
\ {\isachardoublequoteopen}Hintikka\ S\ {\isacharequal}\ {\isacharparenleft}{\isasymbottom}\ {\isasymnotin}\ S\isanewline
{\isasymand}\ {\isacharparenleft}{\isasymforall}k{\isachardot}\ Atom\ k\ {\isasymin}\ S\ {\isasymlongrightarrow}\ \isactrlbold {\isasymnot}\ {\isacharparenleft}Atom\ k{\isacharparenright}\ {\isasymin}\ S\ {\isasymlongrightarrow}\ False{\isacharparenright}\isanewline
{\isasymand}\ {\isacharparenleft}{\isasymforall}F\ G\ H{\isachardot}\ Con\ F\ G\ H\ {\isasymlongrightarrow}\ F\ {\isasymin}\ S\ {\isasymlongrightarrow}\ G\ {\isasymin}\ S\ {\isasymand}\ H\ {\isasymin}\ S{\isacharparenright}\isanewline
{\isasymand}\ {\isacharparenleft}{\isasymforall}F\ G\ H{\isachardot}\ Dis\ F\ G\ H\ {\isasymlongrightarrow}\ F\ {\isasymin}\ S\ {\isasymlongrightarrow}\ G\ {\isasymin}\ S\ {\isasymor}\ H\ {\isasymin}\ S{\isacharparenright}{\isacharparenright}{\isachardoublequoteclose}\ \isanewline
%
\isadelimproof
\ \ %
\endisadelimproof
%
\isatagproof
\isacommand{oops}\isamarkupfalse%
%
\endisatagproof
{\isafoldproof}%
%
\isadelimproof
%
\endisadelimproof
%
\begin{isamarkuptext}%
Procedamos a la demostración del resultado.

\begin{demostracion}
  Para probar la equivalencia, veamos cada una de las implicaciones por separado.

\textbf{\isa{{\isadigit{1}}{\isacharparenright}\ {\isasymLongrightarrow}\ {\isadigit{2}}{\isacharparenright}}}

  Supongamos que \isa{S} es un conjunto de Hintikka. Vamos a probar que, en efecto, se 
  verifican las condiciones del enunciado del lema.

  Por definición de conjunto de Hintikka, \isa{S} verifica las siguientes condiciones:
  \begin{enumerate}
    \item \isa{{\isasymbottom}\ {\isasymnotin}\ S}.
    \item Dada \isa{p} una fórmula atómica cualquiera, no se tiene 
      simultáneamente que\\ \isa{p\ {\isasymin}\ S} y \isa{{\isasymnot}\ p\ {\isasymin}\ S}.
    \item Si \isa{G\ {\isasymand}\ H\ {\isasymin}\ S}, entonces \isa{G\ {\isasymin}\ S} y \isa{H\ {\isasymin}\ S}.
    \item Si \isa{G\ {\isasymor}\ H\ {\isasymin}\ S}, entonces \isa{G\ {\isasymin}\ S} o \isa{H\ {\isasymin}\ S}.
    \item Si \isa{G\ {\isasymrightarrow}\ H\ {\isasymin}\ S}, entonces \isa{{\isasymnot}\ G\ {\isasymin}\ S} o \isa{H\ {\isasymin}\ S}.
    \item Si \isa{{\isasymnot}{\isacharparenleft}{\isasymnot}\ G{\isacharparenright}\ {\isasymin}\ S}, entonces \isa{G\ {\isasymin}\ S}.
    \item Si \isa{{\isasymnot}{\isacharparenleft}G\ {\isasymand}\ H{\isacharparenright}\ {\isasymin}\ S}, entonces \isa{{\isasymnot}\ G\ {\isasymin}\ S} o \isa{{\isasymnot}\ H\ {\isasymin}\ S}.
    \item Si \isa{{\isasymnot}{\isacharparenleft}G\ {\isasymor}\ H{\isacharparenright}\ {\isasymin}\ S}, entonces \isa{{\isasymnot}\ G\ {\isasymin}\ S} y \isa{{\isasymnot}\ H\ {\isasymin}\ S}. 
    \item Si \isa{{\isasymnot}{\isacharparenleft}G\ {\isasymrightarrow}\ H{\isacharparenright}\ {\isasymin}\ S}, entonces \isa{G\ {\isasymin}\ S} y \isa{{\isasymnot}\ H\ {\isasymin}\ S}. 
  \end{enumerate}  
  De este modo, el conjunto \isa{S} cumple la primera y la segunda condición del
  enunciado del lema, que se corresponden con las dos primeras condiciones
  de la definición de conjunto de Hintikka. Veamos que, además, verifica las
  dos últimas condiciones del resultado.

  En primer lugar, probemos que para toda fórmula de tipo \isa{{\isasymalpha}} con 
  componentes \isa{{\isasymalpha}\isactrlsub {\isadigit{1}}} y \isa{{\isasymalpha}\isactrlsub {\isadigit{2}}} se verifica que si la fórmula pertenece al conjunto 
  \isa{S}, entonces \isa{{\isasymalpha}\isactrlsub {\isadigit{1}}} y \isa{{\isasymalpha}\isactrlsub {\isadigit{2}}} también. Para ello, supongamos que una fórmula 
  cualquiera de tipo \isa{{\isasymalpha}} pertence a \isa{S}. Por definición de este tipo de
  fórmulas, tenemos que \isa{{\isasymalpha}} puede ser de la forma \isa{G\ {\isasymand}\ H}, \isa{{\isasymnot}{\isacharparenleft}{\isasymnot}\ G{\isacharparenright}},\\ \isa{{\isasymnot}{\isacharparenleft}G\ {\isasymor}\ H{\isacharparenright}} 
  o \isa{{\isasymnot}{\isacharparenleft}G\ {\isasymlongrightarrow}\ H{\isacharparenright}} para fórmulas \isa{G} y \isa{H} cualesquiera. Probemos que, para cada
  tipo de fórmula \isa{{\isasymalpha}} perteneciente a \isa{S}, sus componentes \isa{{\isasymalpha}\isactrlsub {\isadigit{1}}} y \isa{{\isasymalpha}\isactrlsub {\isadigit{2}}} están en
  \isa{S}.

  \isa{{\isasymsqdot}\ Fórmula\ del\ tipo\ G\ {\isasymand}\ H}: Sus componentes conjuntivas son \isa{G} y \isa{H}. 
  Por la tercera condición de la definición de conjunto de Hintikka, obtenemos 
  que si \isa{G\ {\isasymand}\ H} pertenece a \isa{S}, entonces \isa{G} y \isa{H} están ambas en el conjunto,
  lo que prueba este caso.
    
  \isa{{\isasymsqdot}\ Fórmula\ del\ tipo\ {\isasymnot}{\isacharparenleft}{\isasymnot}\ G{\isacharparenright}}: Sus componentes conjuntivas son ambas \isa{G}.
  Por la sexta condición de la definición de conjunto de Hintikka, obtenemos que
  si \isa{{\isasymnot}{\isacharparenleft}{\isasymnot}\ G{\isacharparenright}} pertenece a \isa{S}, entonces \isa{G} pertenece al conjunto, lo que prueba
  este caso.

  \isa{{\isasymsqdot}\ Fórmula\ del\ tipo\ {\isasymnot}{\isacharparenleft}G\ {\isasymor}\ H{\isacharparenright}}: Sus componentes conjuntivas son \isa{{\isasymnot}\ G} y \isa{{\isasymnot}\ H}. 
  Por la octava condición de la definición de conjunto de Hintikka, obtenemos 
  que si \isa{{\isasymnot}{\isacharparenleft}G\ {\isasymor}\ H{\isacharparenright}} pertenece a \isa{S}, entonces \isa{{\isasymnot}\ G} y \isa{{\isasymnot}\ H} están ambas en el conjunto,
  lo que prueba este caso.

  \isa{{\isasymsqdot}\ Fórmula\ del\ tipo\ {\isasymnot}{\isacharparenleft}G\ {\isasymlongrightarrow}\ H{\isacharparenright}}: Sus componentes conjuntivas son \isa{G} y \isa{{\isasymnot}\ H}. 
  Por la novena condición de la definición de conjunto de Hintikka, obtenemos 
  que si\\ \isa{{\isasymnot}{\isacharparenleft}G\ {\isasymlongrightarrow}\ H{\isacharparenright}} pertenece a \isa{S}, entonces \isa{G} y \isa{{\isasymnot}\ H} están ambas en el conjunto,
  lo que prueba este caso.

  Finalmente, probemos que para toda fórmula de tipo \isa{{\isasymbeta}} con componentes \isa{{\isasymbeta}\isactrlsub {\isadigit{1}}} y 
  \isa{{\isasymbeta}\isactrlsub {\isadigit{2}}} se verifica que si la fórmula pertenece al conjunto \isa{S}, entonces o bien \isa{{\isasymbeta}\isactrlsub {\isadigit{1}}} 
  pertenece al conjunto o bien \isa{{\isasymbeta}\isactrlsub {\isadigit{2}}} pertenece a conjunto. Para ello, supongamos que 
  una fórmula cualquiera de tipo \isa{{\isasymbeta}} pertence a \isa{S}. Por definición de este tipo de
  fórmulas, tenemos que \isa{{\isasymbeta}} puede ser de la forma \isa{G\ {\isasymor}\ H}, \isa{G\ {\isasymlongrightarrow}\ H}, \isa{{\isasymnot}{\isacharparenleft}{\isasymnot}\ G{\isacharparenright}} 
  o \isa{{\isasymnot}{\isacharparenleft}G\ {\isasymand}\ H{\isacharparenright}} para fórmulas \isa{G} y \isa{H} cualesquiera. Probemos que, para cada
  tipo de fórmula \isa{{\isasymbeta}} perteneciente a \isa{S}, o bien su componente \isa{{\isasymbeta}\isactrlsub {\isadigit{1}}} pertenece a \isa{S} 
  o bien su componente \isa{{\isasymbeta}\isactrlsub {\isadigit{2}}} pertenece a \isa{S}.

  \isa{{\isasymsqdot}\ Fórmula\ del\ tipo\ G\ {\isasymor}\ H}: Sus componentes disyuntivas son \isa{G} y \isa{H}. 
    Por la cuarta condición de la definición de conjunto de Hintikka, obtenemos 
    que si \isa{G\ {\isasymor}\ H} pertenece a \isa{S}, entonces o bien \isa{G} está en \isa{S} o bien \isa{H} está
    en \isa{S}, lo que prueba este caso.

  \isa{{\isasymsqdot}\ Fórmula\ del\ tipo\ G\ {\isasymlongrightarrow}\ H}: Sus componentes disyuntivas son \isa{{\isasymnot}\ G} y \isa{H}.
    Por la quinta condición de la definición de conjunto de Hintikka, obtenemos que
    si \isa{G\ {\isasymlongrightarrow}\ H} pertenece a \isa{S}, entonces o bien \isa{{\isasymnot}\ G} pertenece al conjunto o bien
    \isa{H} pertenece al conjunto, lo que prueba este caso.

  \isa{{\isasymsqdot}\ Fórmula\ del\ tipo\ {\isasymnot}{\isacharparenleft}{\isasymnot}\ G{\isacharparenright}}: Sus componentes conjuntivas son ambas \isa{G}.
    Por la sexta condición de la definición de conjunto de Hintikka, obtenemos 
    que si \isa{{\isasymnot}{\isacharparenleft}{\isasymnot}\ G{\isacharparenright}} pertenece a \isa{S}, entonces \isa{G} pertenece al conjunto. De este modo,
    por la regla de introducción a la disyunción, se prueba que o bien una de las 
    componentes está en el conjunto o bien lo está la otra pues, en este caso,
    coinciden.

  \isa{{\isasymsqdot}\ Fórmula\ del\ tipo\ {\isasymnot}{\isacharparenleft}G\ {\isasymand}\ H{\isacharparenright}}: Sus componentes conjuntivas son \isa{{\isasymnot}\ G} y \isa{{\isasymnot}\ H}. 
    Por la séptima condición de la definición de conjunto de Hintikka, obtenemos 
    que si\\ \isa{{\isasymnot}{\isacharparenleft}G\ {\isasymand}\ H{\isacharparenright}} pertenece a \isa{S}, entonces o bien \isa{{\isasymnot}\ G} pertenece al conjunto
    o bien \isa{{\isasymnot}\ H} pertenece al conjunto, lo que prueba este caso.

\textbf{\isa{{\isadigit{2}}{\isacharparenright}\ {\isasymLongrightarrow}\ {\isadigit{1}}{\isacharparenright}}}

  Supongamos que se verifican las condiciones del enunciado del lema:

  \begin{itemize}
    \item \isa{{\isasymbottom}} no pertenece a \isa{S}.
    \item Dada \isa{p} una fórmula atómica cualquiera, no se tiene 
    simultáneamente que\\ \isa{p\ {\isasymin}\ S} y \isa{{\isasymnot}\ p\ {\isasymin}\ S}.
    \item Para toda fórmula de tipo \isa{{\isasymalpha}} con componentes \isa{{\isasymalpha}\isactrlsub {\isadigit{1}}} y \isa{{\isasymalpha}\isactrlsub {\isadigit{2}}} se verifica 
    que si la fórmula pertenece a \isa{S}, entonces \isa{{\isasymalpha}\isactrlsub {\isadigit{1}}} y \isa{{\isasymalpha}\isactrlsub {\isadigit{2}}} también.
    \item Para toda fórmula de tipo \isa{{\isasymbeta}} con componentes \isa{{\isasymbeta}\isactrlsub {\isadigit{1}}} y \isa{{\isasymbeta}\isactrlsub {\isadigit{2}}} se verifica 
    que si la fórmula pertenece a \isa{S}, entonces o bien \isa{{\isasymbeta}\isactrlsub {\isadigit{1}}} pertenece
    a \isa{S} o bien \isa{{\isasymbeta}\isactrlsub {\isadigit{2}}} pertenece a \isa{S}.
  \end{itemize}  

  Vamos a probar que \isa{S} es un conjunto de Hintikka.

  Por la definición de conjunto de Hintikka, es suficiente probar las siguientes
  condiciones:

  \begin{enumerate}
    \item \isa{{\isasymbottom}\ {\isasymnotin}\ S}.
    \item Dada \isa{p} una fórmula atómica cualquiera, no se tiene 
      simultáneamente que\\ \isa{p\ {\isasymin}\ S} y \isa{{\isasymnot}\ p\ {\isasymin}\ S}.
    \item Si \isa{G\ {\isasymand}\ H\ {\isasymin}\ S}, entonces \isa{G\ {\isasymin}\ S} y \isa{H\ {\isasymin}\ S}.
    \item Si \isa{G\ {\isasymor}\ H\ {\isasymin}\ S}, entonces \isa{G\ {\isasymin}\ S} o \isa{H\ {\isasymin}\ S}.
    \item Si \isa{G\ {\isasymrightarrow}\ H\ {\isasymin}\ S}, entonces \isa{{\isasymnot}\ G\ {\isasymin}\ S} o \isa{H\ {\isasymin}\ S}.
    \item Si \isa{{\isasymnot}{\isacharparenleft}{\isasymnot}\ G{\isacharparenright}\ {\isasymin}\ S}, entonces \isa{G\ {\isasymin}\ S}.
    \item Si \isa{{\isasymnot}{\isacharparenleft}G\ {\isasymand}\ H{\isacharparenright}\ {\isasymin}\ S}, entonces \isa{{\isasymnot}\ G\ {\isasymin}\ S} o \isa{{\isasymnot}\ H\ {\isasymin}\ S}.
    \item Si \isa{{\isasymnot}{\isacharparenleft}G\ {\isasymor}\ H{\isacharparenright}\ {\isasymin}\ S}, entonces \isa{{\isasymnot}\ G\ {\isasymin}\ S} y \isa{{\isasymnot}\ H\ {\isasymin}\ S}. 
    \item Si \isa{{\isasymnot}{\isacharparenleft}G\ {\isasymrightarrow}\ H{\isacharparenright}\ {\isasymin}\ S}, entonces \isa{G\ {\isasymin}\ S} y \isa{{\isasymnot}\ H\ {\isasymin}\ S}. 
  \end{enumerate} 

  En primer lugar se observa que, por hipótesis, se verifican las dos primeras
  condiciones de la definición de conjunto de Hintikka. Veamos que, en efecto, se
  cumplen las demás.

  \begin{enumerate}
    \item[\isa{{\isadigit{3}}{\isacharparenright}}] Supongamos que \isa{G\ {\isasymand}\ H} está en \isa{S} para fórmulas \isa{G} y \isa{H} cualesquiera.
    Por definición, \isa{G\ {\isasymand}\ H} es una fórmula de tipo \isa{{\isasymalpha}} con componentes \isa{G} y \isa{H}. 
    Por lo tanto, por hipótesis se cumple que \isa{G} y \isa{H} están en \isa{S}.
    \item[\isa{{\isadigit{4}}{\isacharparenright}}] Supongamos que \isa{G\ {\isasymor}\ H} está en \isa{S} para fórmulas \isa{G} y \isa{H} cualesquiera.
    Por definición, \isa{G\ {\isasymor}\ H} es una fórmula de tipo \isa{{\isasymbeta}} con componentes \isa{G} y \isa{H}. 
    Por lo tanto, por hipótesis se cumple que o bien \isa{G} está en \isa{S} o bien \isa{H} está 
    en \isa{S}.
    \item[\isa{{\isadigit{5}}{\isacharparenright}}] Supongamos que \isa{G\ {\isasymlongrightarrow}\ H} está en \isa{S} para fórmulas \isa{G} y \isa{H} cualesquiera.
    Por definición, \isa{G\ {\isasymlongrightarrow}\ H} es una fórmula de tipo \isa{{\isasymbeta}} con componentes \isa{{\isasymnot}\ G} y \isa{H}. 
    Por lo tanto, por hipótesis se cumple que o bien \isa{{\isasymnot}\ G} está en \isa{S} o bien \isa{H} está 
    en \isa{S}.
    \item[\isa{{\isadigit{6}}{\isacharparenright}}] Supongamos que \isa{{\isasymnot}{\isacharparenleft}{\isasymnot}\ G{\isacharparenright}} está en \isa{S} para una fórmula \isa{G} cualquiera.
    Por definición, \isa{{\isasymnot}{\isacharparenleft}{\isasymnot}\ G{\isacharparenright}} es una fórmula de tipo \isa{{\isasymalpha}} cuyas componentes son ambas \isa{G}. 
    Por lo tanto, por hipótesis se cumple que \isa{G} está en \isa{S}.
    \item[\isa{{\isadigit{7}}{\isacharparenright}}] Supongamos que \isa{{\isasymnot}{\isacharparenleft}G\ {\isasymand}\ H{\isacharparenright}} está en \isa{S} para fórmulas \isa{G} y \isa{H} cualesquiera.
    Por definición, \isa{{\isasymnot}{\isacharparenleft}G\ {\isasymand}\ H{\isacharparenright}} es una fórmula de tipo \isa{{\isasymbeta}} con componentes \isa{{\isasymnot}\ G} y \isa{{\isasymnot}\ H}. 
    Por lo tanto, por hipótesis se cumple que o bien \isa{{\isasymnot}\ G} está en \isa{S} o bien \isa{{\isasymnot}\ H} está 
    en \isa{S}.
    \item[\isa{{\isadigit{8}}{\isacharparenright}}] Supongamos que \isa{{\isasymnot}{\isacharparenleft}G\ {\isasymor}\ H{\isacharparenright}} está en \isa{S} para fórmulas \isa{G} y \isa{H} cualesquiera.
    Por definición, \isa{{\isasymnot}{\isacharparenleft}G\ {\isasymor}\ H{\isacharparenright}} es una fórmula de tipo \isa{{\isasymalpha}} con componentes \isa{{\isasymnot}\ G} y \isa{{\isasymnot}\ H}. 
    Por lo tanto, por hipótesis se cumple que \isa{{\isasymnot}\ G} y \isa{{\isasymnot}\ H} están en \isa{S}.
    \item[\isa{{\isadigit{9}}{\isacharparenright}}] Supongamos que \isa{{\isasymnot}{\isacharparenleft}G\ {\isasymlongrightarrow}\ H{\isacharparenright}} está en \isa{S} para fórmulas \isa{G} y \isa{H} cualesquiera. 
    Por definición, \isa{{\isasymnot}{\isacharparenleft}G\ {\isasymlongrightarrow}\ H{\isacharparenright}} es una fórmula de tipo \isa{{\isasymalpha}} con componentes \isa{G} y \isa{{\isasymnot}\ H}.
    Por lo tanto, por hipótesis se cumple que \isa{G} y \isa{{\isasymnot}\ H} están en \isa{S}.
  \end{enumerate}

  Por tanto, queda probado el resultado.
\end{demostracion}

  Para probar de manera detallada el lema en Isabelle vamos a demostrar
  cada una de las implicaciones de la equivalencia por separado. 

  La primera implicación del lema se basa en dos lemas auxiliares. El primero de ellos 
  prueba que la tercera, sexta, octava y novena condición de la definición de conjunto de 
  Hintikka son suficientes para probar que para toda fórmula de tipo \isa{{\isasymalpha}} con componentes 
  \isa{{\isasymalpha}\isactrlsub {\isadigit{1}}} y \isa{{\isasymalpha}\isactrlsub {\isadigit{2}}} se verifica que si la fórmula pertenece al conjunto \isa{S}, entonces \isa{{\isasymalpha}\isactrlsub {\isadigit{1}}} y 
  \isa{{\isasymalpha}\isactrlsub {\isadigit{2}}} también. Su demostración detallada en Isabelle se muestra a continuación.%
\end{isamarkuptext}\isamarkuptrue%
\isacommand{lemma}\isamarkupfalse%
\ Hintikka{\isacharunderscore}alt{\isadigit{1}}Con{\isacharcolon}\isanewline
\ \ \isakeyword{assumes}\ {\isachardoublequoteopen}{\isacharparenleft}{\isasymforall}G\ H{\isachardot}\ G\ \isactrlbold {\isasymand}\ H\ {\isasymin}\ S\ {\isasymlongrightarrow}\ G\ {\isasymin}\ S\ {\isasymand}\ H\ {\isasymin}\ S{\isacharparenright}\isanewline
\ \ {\isasymand}\ {\isacharparenleft}{\isasymforall}G{\isachardot}\ \isactrlbold {\isasymnot}\ {\isacharparenleft}\isactrlbold {\isasymnot}\ G{\isacharparenright}\ {\isasymin}\ S\ {\isasymlongrightarrow}\ G\ {\isasymin}\ S{\isacharparenright}\isanewline
\ \ {\isasymand}\ {\isacharparenleft}{\isasymforall}G\ H{\isachardot}\ \isactrlbold {\isasymnot}{\isacharparenleft}G\ \isactrlbold {\isasymor}\ H{\isacharparenright}\ {\isasymin}\ S\ {\isasymlongrightarrow}\ \isactrlbold {\isasymnot}\ G\ {\isasymin}\ S\ {\isasymand}\ \isactrlbold {\isasymnot}\ H\ {\isasymin}\ S{\isacharparenright}\isanewline
\ \ {\isasymand}\ {\isacharparenleft}{\isasymforall}G\ H{\isachardot}\ \isactrlbold {\isasymnot}{\isacharparenleft}G\ \isactrlbold {\isasymrightarrow}\ H{\isacharparenright}\ {\isasymin}\ S\ {\isasymlongrightarrow}\ G\ {\isasymin}\ S\ {\isasymand}\ \isactrlbold {\isasymnot}\ H\ {\isasymin}\ S{\isacharparenright}{\isachardoublequoteclose}\isanewline
\ \ \isakeyword{shows}\ {\isachardoublequoteopen}Con\ F\ G\ H\ {\isasymlongrightarrow}\ F\ {\isasymin}\ S\ {\isasymlongrightarrow}\ G\ {\isasymin}\ S\ {\isasymand}\ H\ {\isasymin}\ S{\isachardoublequoteclose}\isanewline
%
\isadelimproof
%
\endisadelimproof
%
\isatagproof
\isacommand{proof}\isamarkupfalse%
\ {\isacharparenleft}rule\ impI{\isacharparenright}\isanewline
\ \ \isacommand{assume}\isamarkupfalse%
\ {\isachardoublequoteopen}Con\ F\ G\ H{\isachardoublequoteclose}\isanewline
\ \ \isacommand{then}\isamarkupfalse%
\ \isacommand{have}\isamarkupfalse%
\ {\isachardoublequoteopen}F\ {\isacharequal}\ G\ \isactrlbold {\isasymand}\ H\ {\isasymor}\ \isanewline
\ \ \ \ {\isacharparenleft}{\isacharparenleft}{\isasymexists}G{\isadigit{1}}\ H{\isadigit{1}}{\isachardot}\ F\ {\isacharequal}\ \isactrlbold {\isasymnot}\ {\isacharparenleft}G{\isadigit{1}}\ \isactrlbold {\isasymor}\ H{\isadigit{1}}{\isacharparenright}\ {\isasymand}\ G\ {\isacharequal}\ \isactrlbold {\isasymnot}\ G{\isadigit{1}}\ {\isasymand}\ H\ {\isacharequal}\ \isactrlbold {\isasymnot}\ H{\isadigit{1}}{\isacharparenright}\ {\isasymor}\ \isanewline
\ \ \ \ {\isacharparenleft}{\isasymexists}H{\isadigit{2}}{\isachardot}\ F\ {\isacharequal}\ \isactrlbold {\isasymnot}\ {\isacharparenleft}G\ \isactrlbold {\isasymrightarrow}\ H{\isadigit{2}}{\isacharparenright}\ {\isasymand}\ H\ {\isacharequal}\ \isactrlbold {\isasymnot}\ H{\isadigit{2}}{\isacharparenright}\ {\isasymor}\ \isanewline
\ \ \ \ F\ {\isacharequal}\ \isactrlbold {\isasymnot}\ {\isacharparenleft}\isactrlbold {\isasymnot}\ G{\isacharparenright}\ {\isasymand}\ H\ {\isacharequal}\ G{\isacharparenright}{\isachardoublequoteclose}\isanewline
\ \ \ \ \isacommand{by}\isamarkupfalse%
\ {\isacharparenleft}simp\ only{\isacharcolon}\ con{\isacharunderscore}dis{\isacharunderscore}simps{\isacharparenleft}{\isadigit{1}}{\isacharparenright}{\isacharparenright}\isanewline
\ \ \isacommand{thus}\isamarkupfalse%
\ {\isachardoublequoteopen}F\ {\isasymin}\ S\ {\isasymlongrightarrow}\ G\ {\isasymin}\ S\ {\isasymand}\ H\ {\isasymin}\ S{\isachardoublequoteclose}\isanewline
\ \ \isacommand{proof}\isamarkupfalse%
\ {\isacharparenleft}rule\ disjE{\isacharparenright}\isanewline
\ \ \ \ \isacommand{assume}\isamarkupfalse%
\ {\isachardoublequoteopen}F\ {\isacharequal}\ G\ \isactrlbold {\isasymand}\ H{\isachardoublequoteclose}\isanewline
\ \ \ \ \isacommand{have}\isamarkupfalse%
\ {\isachardoublequoteopen}{\isasymforall}G\ H{\isachardot}\ G\ \isactrlbold {\isasymand}\ H\ {\isasymin}\ S\ {\isasymlongrightarrow}\ G\ {\isasymin}\ S\ {\isasymand}\ H\ {\isasymin}\ S{\isachardoublequoteclose}\isanewline
\ \ \ \ \ \ \isacommand{using}\isamarkupfalse%
\ assms\ \isacommand{by}\isamarkupfalse%
\ {\isacharparenleft}rule\ conjunct{\isadigit{1}}{\isacharparenright}\isanewline
\ \ \ \ \isacommand{thus}\isamarkupfalse%
\ {\isachardoublequoteopen}F\ {\isasymin}\ S\ {\isasymlongrightarrow}\ G\ {\isasymin}\ S\ {\isasymand}\ H\ {\isasymin}\ S{\isachardoublequoteclose}\isanewline
\ \ \ \ \ \ \isacommand{using}\isamarkupfalse%
\ {\isacartoucheopen}F\ {\isacharequal}\ G\ \isactrlbold {\isasymand}\ H{\isacartoucheclose}\ \isacommand{by}\isamarkupfalse%
\ {\isacharparenleft}iprover\ elim{\isacharcolon}\ allE{\isacharparenright}\isanewline
\ \ \isacommand{next}\isamarkupfalse%
\ \isanewline
\ \ \ \ \isacommand{assume}\isamarkupfalse%
\ {\isachardoublequoteopen}{\isacharparenleft}{\isasymexists}G{\isadigit{1}}\ H{\isadigit{1}}{\isachardot}\ F\ {\isacharequal}\ \isactrlbold {\isasymnot}\ {\isacharparenleft}G{\isadigit{1}}\ \isactrlbold {\isasymor}\ H{\isadigit{1}}{\isacharparenright}\ {\isasymand}\ G\ {\isacharequal}\ \isactrlbold {\isasymnot}\ G{\isadigit{1}}\ {\isasymand}\ H\ {\isacharequal}\ \isactrlbold {\isasymnot}\ H{\isadigit{1}}{\isacharparenright}\ {\isasymor}\ \isanewline
\ \ \ \ {\isacharparenleft}{\isacharparenleft}{\isasymexists}H{\isadigit{2}}{\isachardot}\ F\ {\isacharequal}\ \isactrlbold {\isasymnot}\ {\isacharparenleft}G\ \isactrlbold {\isasymrightarrow}\ H{\isadigit{2}}{\isacharparenright}\ {\isasymand}\ H\ {\isacharequal}\ \isactrlbold {\isasymnot}\ H{\isadigit{2}}{\isacharparenright}\ {\isasymor}\ \isanewline
\ \ \ \ F\ {\isacharequal}\ \isactrlbold {\isasymnot}\ {\isacharparenleft}\isactrlbold {\isasymnot}\ G{\isacharparenright}\ {\isasymand}\ H\ {\isacharequal}\ G{\isacharparenright}{\isachardoublequoteclose}\isanewline
\ \ \ \ \isacommand{thus}\isamarkupfalse%
\ {\isachardoublequoteopen}F\ {\isasymin}\ S\ {\isasymlongrightarrow}\ G\ {\isasymin}\ S\ {\isasymand}\ H\ {\isasymin}\ S{\isachardoublequoteclose}\ \isanewline
\ \ \ \ \isacommand{proof}\isamarkupfalse%
\ {\isacharparenleft}rule\ disjE{\isacharparenright}\isanewline
\ \ \ \ \ \ \isacommand{assume}\isamarkupfalse%
\ E{\isadigit{1}}{\isacharcolon}{\isachardoublequoteopen}{\isasymexists}G{\isadigit{1}}\ H{\isadigit{1}}{\isachardot}\ F\ {\isacharequal}\ \isactrlbold {\isasymnot}\ {\isacharparenleft}G{\isadigit{1}}\ \isactrlbold {\isasymor}\ H{\isadigit{1}}{\isacharparenright}\ {\isasymand}\ G\ {\isacharequal}\ \isactrlbold {\isasymnot}\ G{\isadigit{1}}\ {\isasymand}\ H\ {\isacharequal}\ \isactrlbold {\isasymnot}\ H{\isadigit{1}}{\isachardoublequoteclose}\isanewline
\ \ \ \ \ \ \isacommand{obtain}\isamarkupfalse%
\ G{\isadigit{1}}\ H{\isadigit{1}}\ \isakeyword{where}\ A{\isadigit{1}}{\isacharcolon}{\isachardoublequoteopen}F\ {\isacharequal}\ \isactrlbold {\isasymnot}\ {\isacharparenleft}G{\isadigit{1}}\ \isactrlbold {\isasymor}\ H{\isadigit{1}}{\isacharparenright}\ {\isasymand}\ G\ {\isacharequal}\ \isactrlbold {\isasymnot}\ G{\isadigit{1}}\ {\isasymand}\ H\ {\isacharequal}\ \isactrlbold {\isasymnot}\ H{\isadigit{1}}{\isachardoublequoteclose}\isanewline
\ \ \ \ \ \ \ \ \isacommand{using}\isamarkupfalse%
\ E{\isadigit{1}}\ \isacommand{by}\isamarkupfalse%
\ {\isacharparenleft}iprover\ elim{\isacharcolon}\ exE{\isacharparenright}\isanewline
\ \ \ \ \ \ \isacommand{then}\isamarkupfalse%
\ \isacommand{have}\isamarkupfalse%
\ {\isachardoublequoteopen}F\ {\isacharequal}\ \isactrlbold {\isasymnot}\ {\isacharparenleft}G{\isadigit{1}}\ \isactrlbold {\isasymor}\ H{\isadigit{1}}{\isacharparenright}{\isachardoublequoteclose}\isanewline
\ \ \ \ \ \ \ \ \isacommand{by}\isamarkupfalse%
\ {\isacharparenleft}rule\ conjunct{\isadigit{1}}{\isacharparenright}\isanewline
\ \ \ \ \ \ \isacommand{have}\isamarkupfalse%
\ {\isachardoublequoteopen}G\ {\isacharequal}\ \isactrlbold {\isasymnot}\ G{\isadigit{1}}{\isachardoublequoteclose}\isanewline
\ \ \ \ \ \ \ \ \isacommand{using}\isamarkupfalse%
\ A{\isadigit{1}}\ \isacommand{by}\isamarkupfalse%
\ {\isacharparenleft}iprover\ elim{\isacharcolon}\ conjunct{\isadigit{1}}{\isacharparenright}\isanewline
\ \ \ \ \ \ \isacommand{have}\isamarkupfalse%
\ {\isachardoublequoteopen}H\ {\isacharequal}\ \isactrlbold {\isasymnot}\ H{\isadigit{1}}{\isachardoublequoteclose}\isanewline
\ \ \ \ \ \ \ \ \isacommand{using}\isamarkupfalse%
\ A{\isadigit{1}}\ \isacommand{by}\isamarkupfalse%
\ {\isacharparenleft}iprover\ elim{\isacharcolon}\ conjunct{\isadigit{1}}{\isacharparenright}\isanewline
\ \ \ \ \ \ \isacommand{have}\isamarkupfalse%
\ {\isachardoublequoteopen}{\isasymforall}G\ H{\isachardot}\ \isactrlbold {\isasymnot}{\isacharparenleft}G\ \isactrlbold {\isasymor}\ H{\isacharparenright}\ {\isasymin}\ S\ {\isasymlongrightarrow}\ \isactrlbold {\isasymnot}\ G\ {\isasymin}\ S\ {\isasymand}\ \isactrlbold {\isasymnot}\ H\ {\isasymin}\ S{\isachardoublequoteclose}\isanewline
\ \ \ \ \ \ \ \ \isacommand{using}\isamarkupfalse%
\ assms\ \isacommand{by}\isamarkupfalse%
\ {\isacharparenleft}iprover\ elim{\isacharcolon}\ conjunct{\isadigit{2}}\ conjunct{\isadigit{1}}{\isacharparenright}\isanewline
\ \ \ \ \ \ \isacommand{thus}\isamarkupfalse%
\ {\isachardoublequoteopen}F\ {\isasymin}\ S\ {\isasymlongrightarrow}\ G\ {\isasymin}\ S\ {\isasymand}\ H\ {\isasymin}\ S{\isachardoublequoteclose}\isanewline
\ \ \ \ \ \ \ \ \isacommand{using}\isamarkupfalse%
\ {\isacartoucheopen}F\ {\isacharequal}\ \isactrlbold {\isasymnot}\ {\isacharparenleft}G{\isadigit{1}}\ \isactrlbold {\isasymor}\ H{\isadigit{1}}{\isacharparenright}{\isacartoucheclose}\ {\isacartoucheopen}G\ {\isacharequal}\ \isactrlbold {\isasymnot}\ G{\isadigit{1}}{\isacartoucheclose}\ {\isacartoucheopen}H\ {\isacharequal}\ \isactrlbold {\isasymnot}\ H{\isadigit{1}}{\isacartoucheclose}\ \isacommand{by}\isamarkupfalse%
\ {\isacharparenleft}iprover\ elim{\isacharcolon}\ allE{\isacharparenright}\isanewline
\ \ \ \ \isacommand{next}\isamarkupfalse%
\isanewline
\ \ \ \ \ \ \isacommand{assume}\isamarkupfalse%
\ {\isachardoublequoteopen}{\isacharparenleft}{\isasymexists}H{\isadigit{2}}{\isachardot}\ F\ {\isacharequal}\ \isactrlbold {\isasymnot}\ {\isacharparenleft}G\ \isactrlbold {\isasymrightarrow}\ H{\isadigit{2}}{\isacharparenright}\ {\isasymand}\ H\ {\isacharequal}\ \isactrlbold {\isasymnot}\ H{\isadigit{2}}{\isacharparenright}\ {\isasymor}\ \isanewline
\ \ \ \ \ \ F\ {\isacharequal}\ \isactrlbold {\isasymnot}\ {\isacharparenleft}\isactrlbold {\isasymnot}\ G{\isacharparenright}\ {\isasymand}\ H\ {\isacharequal}\ G{\isachardoublequoteclose}\isanewline
\ \ \ \ \ \ \isacommand{thus}\isamarkupfalse%
\ {\isachardoublequoteopen}F\ {\isasymin}\ S\ {\isasymlongrightarrow}\ G\ {\isasymin}\ S\ {\isasymand}\ H\ {\isasymin}\ S{\isachardoublequoteclose}\ \isanewline
\ \ \ \ \ \ \isacommand{proof}\isamarkupfalse%
\ {\isacharparenleft}rule\ disjE{\isacharparenright}\isanewline
\ \ \ \ \ \ \ \ \isacommand{assume}\isamarkupfalse%
\ E{\isadigit{2}}{\isacharcolon}{\isachardoublequoteopen}{\isasymexists}H{\isadigit{2}}{\isachardot}\ F\ {\isacharequal}\ \isactrlbold {\isasymnot}\ {\isacharparenleft}G\ \isactrlbold {\isasymrightarrow}\ H{\isadigit{2}}{\isacharparenright}\ {\isasymand}\ H\ {\isacharequal}\ \isactrlbold {\isasymnot}\ H{\isadigit{2}}{\isachardoublequoteclose}\isanewline
\ \ \ \ \ \ \ \ \isacommand{obtain}\isamarkupfalse%
\ H{\isadigit{2}}\ \isakeyword{where}\ A{\isadigit{2}}{\isacharcolon}{\isachardoublequoteopen}F\ {\isacharequal}\ \isactrlbold {\isasymnot}\ {\isacharparenleft}G\ \isactrlbold {\isasymrightarrow}\ H{\isadigit{2}}{\isacharparenright}\ {\isasymand}\ H\ {\isacharequal}\ \isactrlbold {\isasymnot}\ H{\isadigit{2}}{\isachardoublequoteclose}\isanewline
\ \ \ \ \ \ \ \ \ \ \isacommand{using}\isamarkupfalse%
\ E{\isadigit{2}}\ \isacommand{by}\isamarkupfalse%
\ {\isacharparenleft}rule\ exE{\isacharparenright}\isanewline
\ \ \ \ \ \ \ \ \isacommand{have}\isamarkupfalse%
\ {\isachardoublequoteopen}F\ {\isacharequal}\ \isactrlbold {\isasymnot}\ {\isacharparenleft}G\ \isactrlbold {\isasymrightarrow}\ H{\isadigit{2}}{\isacharparenright}{\isachardoublequoteclose}\isanewline
\ \ \ \ \ \ \ \ \ \ \isacommand{using}\isamarkupfalse%
\ A{\isadigit{2}}\ \isacommand{by}\isamarkupfalse%
\ {\isacharparenleft}rule\ conjunct{\isadigit{1}}{\isacharparenright}\isanewline
\ \ \ \ \ \ \ \ \isacommand{have}\isamarkupfalse%
\ {\isachardoublequoteopen}H\ {\isacharequal}\ \isactrlbold {\isasymnot}\ H{\isadigit{2}}{\isachardoublequoteclose}\isanewline
\ \ \ \ \ \ \ \ \ \ \isacommand{using}\isamarkupfalse%
\ A{\isadigit{2}}\ \isacommand{by}\isamarkupfalse%
\ {\isacharparenleft}rule\ conjunct{\isadigit{2}}{\isacharparenright}\isanewline
\ \ \ \ \ \ \ \ \isacommand{have}\isamarkupfalse%
\ {\isachardoublequoteopen}{\isasymforall}G\ H{\isachardot}\ \isactrlbold {\isasymnot}{\isacharparenleft}G\ \isactrlbold {\isasymrightarrow}\ H{\isacharparenright}\ {\isasymin}\ S\ {\isasymlongrightarrow}\ G\ {\isasymin}\ S\ {\isasymand}\ \isactrlbold {\isasymnot}\ H\ {\isasymin}\ S{\isachardoublequoteclose}\isanewline
\ \ \ \ \ \ \ \ \ \ \isacommand{using}\isamarkupfalse%
\ assms\ \isacommand{by}\isamarkupfalse%
\ {\isacharparenleft}iprover\ elim{\isacharcolon}\ conjunct{\isadigit{2}}\ conjunct{\isadigit{1}}{\isacharparenright}\isanewline
\ \ \ \ \ \ \ \ \isacommand{thus}\isamarkupfalse%
\ {\isachardoublequoteopen}F\ {\isasymin}\ S\ {\isasymlongrightarrow}\ G\ {\isasymin}\ S\ {\isasymand}\ H\ {\isasymin}\ S{\isachardoublequoteclose}\isanewline
\ \ \ \ \ \ \ \ \ \ \isacommand{using}\isamarkupfalse%
\ {\isacartoucheopen}F\ {\isacharequal}\ \isactrlbold {\isasymnot}\ {\isacharparenleft}G\ \isactrlbold {\isasymrightarrow}\ H{\isadigit{2}}{\isacharparenright}{\isacartoucheclose}\ {\isacartoucheopen}H\ {\isacharequal}\ \isactrlbold {\isasymnot}\ H{\isadigit{2}}{\isacartoucheclose}\ \isacommand{by}\isamarkupfalse%
\ {\isacharparenleft}iprover\ elim{\isacharcolon}\ allE{\isacharparenright}\isanewline
\ \ \ \ \ \ \isacommand{next}\isamarkupfalse%
\ \isanewline
\ \ \ \ \ \ \ \ \isacommand{assume}\isamarkupfalse%
\ {\isachardoublequoteopen}F\ {\isacharequal}\ \isactrlbold {\isasymnot}\ {\isacharparenleft}\isactrlbold {\isasymnot}\ G{\isacharparenright}\ {\isasymand}\ H\ {\isacharequal}\ G{\isachardoublequoteclose}\isanewline
\ \ \ \ \ \ \ \ \isacommand{then}\isamarkupfalse%
\ \isacommand{have}\isamarkupfalse%
\ {\isachardoublequoteopen}F\ {\isacharequal}\ \isactrlbold {\isasymnot}\ {\isacharparenleft}\isactrlbold {\isasymnot}\ G{\isacharparenright}{\isachardoublequoteclose}\isanewline
\ \ \ \ \ \ \ \ \ \ \isacommand{by}\isamarkupfalse%
\ {\isacharparenleft}rule\ conjunct{\isadigit{1}}{\isacharparenright}\isanewline
\ \ \ \ \ \ \ \ \isacommand{have}\isamarkupfalse%
\ {\isachardoublequoteopen}H\ {\isacharequal}\ G{\isachardoublequoteclose}\isanewline
\ \ \ \ \ \ \ \ \ \ \isacommand{using}\isamarkupfalse%
\ {\isacartoucheopen}F\ {\isacharequal}\ \isactrlbold {\isasymnot}\ {\isacharparenleft}\isactrlbold {\isasymnot}\ G{\isacharparenright}\ {\isasymand}\ H\ {\isacharequal}\ G{\isacartoucheclose}\ \isacommand{by}\isamarkupfalse%
\ {\isacharparenleft}rule\ conjunct{\isadigit{2}}{\isacharparenright}\isanewline
\ \ \ \ \ \ \ \ \isacommand{have}\isamarkupfalse%
\ {\isachardoublequoteopen}{\isasymforall}G{\isachardot}\ \isactrlbold {\isasymnot}\ {\isacharparenleft}\isactrlbold {\isasymnot}\ G{\isacharparenright}\ {\isasymin}\ S\ {\isasymlongrightarrow}\ G\ {\isasymin}\ S{\isachardoublequoteclose}\isanewline
\ \ \ \ \ \ \ \ \ \ \isacommand{using}\isamarkupfalse%
\ assms\ \isacommand{by}\isamarkupfalse%
\ {\isacharparenleft}iprover\ elim{\isacharcolon}\ conjunct{\isadigit{2}}\ conjunct{\isadigit{1}}{\isacharparenright}\isanewline
\ \ \ \ \ \ \ \ \isacommand{then}\isamarkupfalse%
\ \isacommand{have}\isamarkupfalse%
\ {\isachardoublequoteopen}\isactrlbold {\isasymnot}\ {\isacharparenleft}\isactrlbold {\isasymnot}\ G{\isacharparenright}\ {\isasymin}\ S\ {\isasymlongrightarrow}\ G\ {\isasymin}\ S{\isachardoublequoteclose}\isanewline
\ \ \ \ \ \ \ \ \ \ \isacommand{by}\isamarkupfalse%
\ {\isacharparenleft}rule\ allE{\isacharparenright}\isanewline
\ \ \ \ \ \ \ \ \isacommand{then}\isamarkupfalse%
\ \isacommand{have}\isamarkupfalse%
\ {\isachardoublequoteopen}F\ {\isasymin}\ S\ {\isasymlongrightarrow}\ G\ {\isasymin}\ S{\isachardoublequoteclose}\isanewline
\ \ \ \ \ \ \ \ \ \ \isacommand{by}\isamarkupfalse%
\ {\isacharparenleft}simp\ only{\isacharcolon}\ {\isacartoucheopen}F\ {\isacharequal}\ \isactrlbold {\isasymnot}\ {\isacharparenleft}\isactrlbold {\isasymnot}\ G{\isacharparenright}{\isacartoucheclose}{\isacharparenright}\ \isanewline
\ \ \ \ \ \ \ \ \isacommand{then}\isamarkupfalse%
\ \isacommand{have}\isamarkupfalse%
\ {\isachardoublequoteopen}F\ {\isasymin}\ S\ {\isasymlongrightarrow}\ G\ {\isasymin}\ S\ {\isasymand}\ G\ {\isasymin}\ S{\isachardoublequoteclose}\isanewline
\ \ \ \ \ \ \ \ \ \ \isacommand{by}\isamarkupfalse%
\ {\isacharparenleft}simp\ only{\isacharcolon}\ conj{\isacharunderscore}absorb{\isacharparenright}\isanewline
\ \ \ \ \ \ \ \ \isacommand{thus}\isamarkupfalse%
\ {\isachardoublequoteopen}F\ {\isasymin}\ S\ {\isasymlongrightarrow}\ G\ {\isasymin}\ S\ {\isasymand}\ H\ {\isasymin}\ S{\isachardoublequoteclose}\isanewline
\ \ \ \ \ \ \ \ \ \ \isacommand{by}\isamarkupfalse%
\ {\isacharparenleft}simp\ only{\isacharcolon}\ {\isacartoucheopen}H{\isacharequal}G{\isacartoucheclose}{\isacharparenright}\isanewline
\ \ \ \ \ \ \isacommand{qed}\isamarkupfalse%
\isanewline
\ \ \ \ \isacommand{qed}\isamarkupfalse%
\isanewline
\ \ \isacommand{qed}\isamarkupfalse%
\isanewline
\isacommand{qed}\isamarkupfalse%
%
\endisatagproof
{\isafoldproof}%
%
\isadelimproof
%
\endisadelimproof
%
\begin{isamarkuptext}%
Por otro lado, el segundo lema auxiliar prueba que la cuarta, quinta, sexta
  y séptima condición de la definición de conjunto de Hintikka son suficientes para
  probar que para toda fórmula de tipo \isa{{\isasymbeta}} con componentes \isa{{\isasymbeta}\isactrlsub {\isadigit{1}}} y \isa{{\isasymbeta}\isactrlsub {\isadigit{2}}} se verifica 
  que si la fórmula pertenece al conjunto \isa{S}, entonces o bien \isa{{\isasymbeta}\isactrlsub {\isadigit{1}}} pertenece al
  conjunto o bien \isa{{\isasymbeta}\isactrlsub {\isadigit{2}}} pertenece al conjunto. Veamos su prueba detallada en 
  Isabelle/HOL.%
\end{isamarkuptext}\isamarkuptrue%
\isacommand{lemma}\isamarkupfalse%
\ Hintikka{\isacharunderscore}alt{\isadigit{1}}Dis{\isacharcolon}\isanewline
\ \ \isakeyword{assumes}\ \ {\isachardoublequoteopen}{\isacharparenleft}{\isasymforall}\ G\ H{\isachardot}\ G\ \isactrlbold {\isasymor}\ H\ {\isasymin}\ S\ {\isasymlongrightarrow}\ G\ {\isasymin}\ S\ {\isasymor}\ H\ {\isasymin}\ S{\isacharparenright}\isanewline
\ \ {\isasymand}\ {\isacharparenleft}{\isasymforall}\ G\ H{\isachardot}\ G\ \isactrlbold {\isasymrightarrow}\ H\ {\isasymin}\ S\ {\isasymlongrightarrow}\ \isactrlbold {\isasymnot}\ G\ {\isasymin}\ S\ {\isasymor}\ H\ {\isasymin}\ S{\isacharparenright}\isanewline
\ \ {\isasymand}\ {\isacharparenleft}{\isasymforall}\ G{\isachardot}\ \isactrlbold {\isasymnot}\ {\isacharparenleft}\isactrlbold {\isasymnot}\ G{\isacharparenright}\ {\isasymin}\ S\ {\isasymlongrightarrow}\ G\ {\isasymin}\ S{\isacharparenright}\isanewline
\ \ {\isasymand}\ {\isacharparenleft}{\isasymforall}\ G\ H{\isachardot}\ \isactrlbold {\isasymnot}{\isacharparenleft}G\ \isactrlbold {\isasymand}\ H{\isacharparenright}\ {\isasymin}\ S\ {\isasymlongrightarrow}\ \isactrlbold {\isasymnot}\ G\ {\isasymin}\ S\ {\isasymor}\ \isactrlbold {\isasymnot}\ H\ {\isasymin}\ S{\isacharparenright}{\isachardoublequoteclose}\isanewline
\ \ \isakeyword{shows}\ {\isachardoublequoteopen}Dis\ F\ G\ H\ {\isasymlongrightarrow}\ F\ {\isasymin}\ S\ {\isasymlongrightarrow}\ G\ {\isasymin}\ S\ {\isasymor}\ H\ {\isasymin}\ S{\isachardoublequoteclose}\isanewline
%
\isadelimproof
%
\endisadelimproof
%
\isatagproof
\isacommand{proof}\isamarkupfalse%
\ {\isacharparenleft}rule\ impI{\isacharparenright}\isanewline
\ \ \isacommand{assume}\isamarkupfalse%
\ {\isachardoublequoteopen}Dis\ F\ G\ H{\isachardoublequoteclose}\isanewline
\ \ \isacommand{then}\isamarkupfalse%
\ \isacommand{have}\isamarkupfalse%
\ {\isachardoublequoteopen}F\ {\isacharequal}\ G\ \isactrlbold {\isasymor}\ H\ {\isasymor}\ \isanewline
\ \ \ \ {\isacharparenleft}{\isasymexists}G{\isadigit{1}}\ H{\isadigit{1}}{\isachardot}\ F\ {\isacharequal}\ G{\isadigit{1}}\ \isactrlbold {\isasymrightarrow}\ H{\isadigit{1}}\ {\isasymand}\ G\ {\isacharequal}\ \isactrlbold {\isasymnot}\ G{\isadigit{1}}\ {\isasymand}\ H\ {\isacharequal}\ H{\isadigit{1}}{\isacharparenright}\ {\isasymor}\ \isanewline
\ \ \ \ {\isacharparenleft}{\isasymexists}G{\isadigit{2}}\ H{\isadigit{2}}{\isachardot}\ F\ {\isacharequal}\ \isactrlbold {\isasymnot}\ {\isacharparenleft}G{\isadigit{2}}\ \isactrlbold {\isasymand}\ H{\isadigit{2}}{\isacharparenright}\ {\isasymand}\ G\ {\isacharequal}\ \isactrlbold {\isasymnot}\ G{\isadigit{2}}\ {\isasymand}\ H\ {\isacharequal}\ \isactrlbold {\isasymnot}\ H{\isadigit{2}}{\isacharparenright}\ {\isasymor}\ \isanewline
\ \ \ \ F\ {\isacharequal}\ \isactrlbold {\isasymnot}\ {\isacharparenleft}\isactrlbold {\isasymnot}\ G{\isacharparenright}\ {\isasymand}\ H\ {\isacharequal}\ G{\isachardoublequoteclose}\ \isanewline
\ \ \ \ \isacommand{by}\isamarkupfalse%
\ {\isacharparenleft}simp\ only{\isacharcolon}\ con{\isacharunderscore}dis{\isacharunderscore}simps{\isacharparenleft}{\isadigit{2}}{\isacharparenright}{\isacharparenright}\isanewline
\ \ \isacommand{thus}\isamarkupfalse%
\ {\isachardoublequoteopen}F\ {\isasymin}\ S\ {\isasymlongrightarrow}\ G\ {\isasymin}\ S\ {\isasymor}\ H\ {\isasymin}\ S{\isachardoublequoteclose}\ \isanewline
\ \ \isacommand{proof}\isamarkupfalse%
\ {\isacharparenleft}rule\ disjE{\isacharparenright}\isanewline
\ \ \ \ \isacommand{assume}\isamarkupfalse%
\ {\isachardoublequoteopen}F\ {\isacharequal}\ G\ \isactrlbold {\isasymor}\ H{\isachardoublequoteclose}\isanewline
\ \ \ \ \isacommand{have}\isamarkupfalse%
\ {\isachardoublequoteopen}{\isasymforall}G\ H{\isachardot}\ G\ \isactrlbold {\isasymor}\ H\ {\isasymin}\ S\ {\isasymlongrightarrow}\ G\ {\isasymin}\ S\ {\isasymor}\ H\ {\isasymin}\ S{\isachardoublequoteclose}\isanewline
\ \ \ \ \ \ \isacommand{using}\isamarkupfalse%
\ assms\ \isacommand{by}\isamarkupfalse%
\ {\isacharparenleft}rule\ conjunct{\isadigit{1}}{\isacharparenright}\isanewline
\ \ \ \ \isacommand{thus}\isamarkupfalse%
\ {\isachardoublequoteopen}F\ {\isasymin}\ S\ {\isasymlongrightarrow}\ G\ {\isasymin}\ S\ {\isasymor}\ H\ {\isasymin}\ S{\isachardoublequoteclose}\ \isanewline
\ \ \ \ \ \ \isacommand{using}\isamarkupfalse%
\ {\isacartoucheopen}F\ {\isacharequal}\ G\ \isactrlbold {\isasymor}\ H{\isacartoucheclose}\ \isacommand{by}\isamarkupfalse%
\ {\isacharparenleft}iprover\ elim{\isacharcolon}\ allE{\isacharparenright}\isanewline
\ \ \isacommand{next}\isamarkupfalse%
\isanewline
\ \ \ \ \isacommand{assume}\isamarkupfalse%
\ {\isachardoublequoteopen}{\isacharparenleft}{\isasymexists}G{\isadigit{1}}\ H{\isadigit{1}}{\isachardot}\ F\ {\isacharequal}\ G{\isadigit{1}}\ \isactrlbold {\isasymrightarrow}\ H{\isadigit{1}}\ {\isasymand}\ G\ {\isacharequal}\ \isactrlbold {\isasymnot}\ G{\isadigit{1}}\ {\isasymand}\ H\ {\isacharequal}\ H{\isadigit{1}}{\isacharparenright}\ {\isasymor}\ \isanewline
\ \ \ \ {\isacharparenleft}{\isasymexists}G{\isadigit{2}}\ H{\isadigit{2}}{\isachardot}\ F\ {\isacharequal}\ \isactrlbold {\isasymnot}\ {\isacharparenleft}G{\isadigit{2}}\ \isactrlbold {\isasymand}\ H{\isadigit{2}}{\isacharparenright}\ {\isasymand}\ G\ {\isacharequal}\ \isactrlbold {\isasymnot}\ G{\isadigit{2}}\ {\isasymand}\ H\ {\isacharequal}\ \isactrlbold {\isasymnot}\ H{\isadigit{2}}{\isacharparenright}\ {\isasymor}\ \isanewline
\ \ \ \ F\ {\isacharequal}\ \isactrlbold {\isasymnot}\ {\isacharparenleft}\isactrlbold {\isasymnot}\ G{\isacharparenright}\ {\isasymand}\ H\ {\isacharequal}\ G{\isachardoublequoteclose}\isanewline
\ \ \ \ \isacommand{thus}\isamarkupfalse%
\ {\isachardoublequoteopen}F\ {\isasymin}\ S\ {\isasymlongrightarrow}\ G\ {\isasymin}\ S\ {\isasymor}\ H\ {\isasymin}\ S{\isachardoublequoteclose}\isanewline
\ \ \ \ \isacommand{proof}\isamarkupfalse%
\ {\isacharparenleft}rule\ disjE{\isacharparenright}\isanewline
\ \ \ \ \ \ \isacommand{assume}\isamarkupfalse%
\ E{\isadigit{1}}{\isacharcolon}{\isachardoublequoteopen}{\isasymexists}G{\isadigit{1}}\ H{\isadigit{1}}{\isachardot}\ F\ {\isacharequal}\ G{\isadigit{1}}\ \isactrlbold {\isasymrightarrow}\ H{\isadigit{1}}\ {\isasymand}\ G\ {\isacharequal}\ \isactrlbold {\isasymnot}\ G{\isadigit{1}}\ {\isasymand}\ H\ {\isacharequal}\ H{\isadigit{1}}{\isachardoublequoteclose}\isanewline
\ \ \ \ \ \ \isacommand{obtain}\isamarkupfalse%
\ G{\isadigit{1}}\ H{\isadigit{1}}\ \isakeyword{where}\ A{\isadigit{1}}{\isacharcolon}{\isachardoublequoteopen}F\ {\isacharequal}\ G{\isadigit{1}}\ \isactrlbold {\isasymrightarrow}\ H{\isadigit{1}}\ {\isasymand}\ G\ {\isacharequal}\ \isactrlbold {\isasymnot}\ G{\isadigit{1}}\ {\isasymand}\ H\ {\isacharequal}\ H{\isadigit{1}}{\isachardoublequoteclose}\isanewline
\ \ \ \ \ \ \ \ \isacommand{using}\isamarkupfalse%
\ E{\isadigit{1}}\ \isacommand{by}\isamarkupfalse%
\ {\isacharparenleft}iprover\ elim{\isacharcolon}\ exE{\isacharparenright}\isanewline
\ \ \ \ \ \ \isacommand{have}\isamarkupfalse%
\ {\isachardoublequoteopen}F\ {\isacharequal}\ G{\isadigit{1}}\ \isactrlbold {\isasymrightarrow}\ H{\isadigit{1}}{\isachardoublequoteclose}\isanewline
\ \ \ \ \ \ \ \ \isacommand{using}\isamarkupfalse%
\ A{\isadigit{1}}\ \isacommand{by}\isamarkupfalse%
\ {\isacharparenleft}rule\ conjunct{\isadigit{1}}{\isacharparenright}\isanewline
\ \ \ \ \ \ \isacommand{have}\isamarkupfalse%
\ {\isachardoublequoteopen}G\ {\isacharequal}\ \isactrlbold {\isasymnot}\ G{\isadigit{1}}{\isachardoublequoteclose}\isanewline
\ \ \ \ \ \ \ \ \isacommand{using}\isamarkupfalse%
\ A{\isadigit{1}}\ \isacommand{by}\isamarkupfalse%
\ {\isacharparenleft}iprover\ elim{\isacharcolon}\ conjunct{\isadigit{1}}{\isacharparenright}\isanewline
\ \ \ \ \ \ \isacommand{have}\isamarkupfalse%
\ {\isachardoublequoteopen}H\ {\isacharequal}\ H{\isadigit{1}}{\isachardoublequoteclose}\isanewline
\ \ \ \ \ \ \ \ \isacommand{using}\isamarkupfalse%
\ A{\isadigit{1}}\ \isacommand{by}\isamarkupfalse%
\ {\isacharparenleft}iprover\ elim{\isacharcolon}\ conjunct{\isadigit{2}}\ conjunct{\isadigit{1}}{\isacharparenright}\isanewline
\ \ \ \ \ \ \isacommand{have}\isamarkupfalse%
\ {\isachardoublequoteopen}{\isasymforall}G\ H{\isachardot}\ G\ \isactrlbold {\isasymrightarrow}\ H\ {\isasymin}\ S\ {\isasymlongrightarrow}\ \isactrlbold {\isasymnot}\ G\ {\isasymin}\ S\ {\isasymor}\ H\ {\isasymin}\ S{\isachardoublequoteclose}\isanewline
\ \ \ \ \ \ \ \ \isacommand{using}\isamarkupfalse%
\ assms\ \isacommand{by}\isamarkupfalse%
\ {\isacharparenleft}iprover\ elim{\isacharcolon}\ conjunct{\isadigit{2}}\ conjunct{\isadigit{1}}{\isacharparenright}\isanewline
\ \ \ \ \ \ \isacommand{thus}\isamarkupfalse%
\ {\isachardoublequoteopen}F\ {\isasymin}\ S\ {\isasymlongrightarrow}\ G\ {\isasymin}\ S\ {\isasymor}\ H\ {\isasymin}\ S{\isachardoublequoteclose}\isanewline
\ \ \ \ \ \ \ \ \isacommand{using}\isamarkupfalse%
\ {\isacartoucheopen}F\ {\isacharequal}\ G{\isadigit{1}}\ \isactrlbold {\isasymrightarrow}\ H{\isadigit{1}}{\isacartoucheclose}\ {\isacartoucheopen}G\ {\isacharequal}\ \isactrlbold {\isasymnot}\ G{\isadigit{1}}{\isacartoucheclose}\ {\isacartoucheopen}H\ {\isacharequal}\ H{\isadigit{1}}{\isacartoucheclose}\ \isacommand{by}\isamarkupfalse%
\ {\isacharparenleft}iprover\ elim{\isacharcolon}\ allE{\isacharparenright}\isanewline
\ \ \ \ \isacommand{next}\isamarkupfalse%
\isanewline
\ \ \ \ \ \ \isacommand{assume}\isamarkupfalse%
\ {\isachardoublequoteopen}{\isacharparenleft}{\isasymexists}G{\isadigit{2}}\ H{\isadigit{2}}{\isachardot}\ F\ {\isacharequal}\ \isactrlbold {\isasymnot}\ {\isacharparenleft}G{\isadigit{2}}\ \isactrlbold {\isasymand}\ H{\isadigit{2}}{\isacharparenright}\ {\isasymand}\ G\ {\isacharequal}\ \isactrlbold {\isasymnot}\ G{\isadigit{2}}\ {\isasymand}\ H\ {\isacharequal}\ \isactrlbold {\isasymnot}\ H{\isadigit{2}}{\isacharparenright}\ {\isasymor}\ \isanewline
\ \ \ \ \ \ F\ {\isacharequal}\ \isactrlbold {\isasymnot}\ {\isacharparenleft}\isactrlbold {\isasymnot}\ G{\isacharparenright}\ {\isasymand}\ H\ {\isacharequal}\ G{\isachardoublequoteclose}\isanewline
\ \ \ \ \ \ \isacommand{thus}\isamarkupfalse%
\ {\isachardoublequoteopen}F\ {\isasymin}\ S\ {\isasymlongrightarrow}\ G\ {\isasymin}\ S\ {\isasymor}\ H\ {\isasymin}\ S{\isachardoublequoteclose}\isanewline
\ \ \ \ \ \ \isacommand{proof}\isamarkupfalse%
\ {\isacharparenleft}rule\ disjE{\isacharparenright}\isanewline
\ \ \ \ \ \ \ \ \isacommand{assume}\isamarkupfalse%
\ E{\isadigit{2}}{\isacharcolon}{\isachardoublequoteopen}{\isasymexists}G{\isadigit{2}}\ H{\isadigit{2}}{\isachardot}\ F\ {\isacharequal}\ \isactrlbold {\isasymnot}\ {\isacharparenleft}G{\isadigit{2}}\ \isactrlbold {\isasymand}\ H{\isadigit{2}}{\isacharparenright}\ {\isasymand}\ G\ {\isacharequal}\ \isactrlbold {\isasymnot}\ G{\isadigit{2}}\ {\isasymand}\ H\ {\isacharequal}\ \isactrlbold {\isasymnot}\ H{\isadigit{2}}{\isachardoublequoteclose}\isanewline
\ \ \ \ \ \ \ \ \isacommand{obtain}\isamarkupfalse%
\ G{\isadigit{2}}\ H{\isadigit{2}}\ \isakeyword{where}\ A{\isadigit{2}}{\isacharcolon}{\isachardoublequoteopen}F\ {\isacharequal}\ \isactrlbold {\isasymnot}\ {\isacharparenleft}G{\isadigit{2}}\ \isactrlbold {\isasymand}\ H{\isadigit{2}}{\isacharparenright}\ {\isasymand}\ G\ {\isacharequal}\ \isactrlbold {\isasymnot}\ G{\isadigit{2}}\ {\isasymand}\ H\ {\isacharequal}\ \isactrlbold {\isasymnot}\ H{\isadigit{2}}{\isachardoublequoteclose}\ \isanewline
\ \ \ \ \ \ \ \ \ \ \isacommand{using}\isamarkupfalse%
\ E{\isadigit{2}}\ \isacommand{by}\isamarkupfalse%
\ {\isacharparenleft}iprover\ elim{\isacharcolon}\ exE{\isacharparenright}\isanewline
\ \ \ \ \ \ \ \ \isacommand{have}\isamarkupfalse%
\ {\isachardoublequoteopen}F\ {\isacharequal}\ \isactrlbold {\isasymnot}\ {\isacharparenleft}G{\isadigit{2}}\ \isactrlbold {\isasymand}\ H{\isadigit{2}}{\isacharparenright}{\isachardoublequoteclose}\ \isanewline
\ \ \ \ \ \ \ \ \ \ \isacommand{using}\isamarkupfalse%
\ A{\isadigit{2}}\ \isacommand{by}\isamarkupfalse%
\ {\isacharparenleft}rule\ conjunct{\isadigit{1}}{\isacharparenright}\isanewline
\ \ \ \ \ \ \ \ \isacommand{have}\isamarkupfalse%
\ {\isachardoublequoteopen}G\ {\isacharequal}\ \isactrlbold {\isasymnot}\ G{\isadigit{2}}{\isachardoublequoteclose}\isanewline
\ \ \ \ \ \ \ \ \ \ \isacommand{using}\isamarkupfalse%
\ A{\isadigit{2}}\ \isacommand{by}\isamarkupfalse%
\ {\isacharparenleft}iprover\ elim{\isacharcolon}\ conjunct{\isadigit{2}}\ conjunct{\isadigit{1}}{\isacharparenright}\isanewline
\ \ \ \ \ \ \ \ \isacommand{have}\isamarkupfalse%
\ {\isachardoublequoteopen}H\ {\isacharequal}\ \isactrlbold {\isasymnot}\ H{\isadigit{2}}{\isachardoublequoteclose}\isanewline
\ \ \ \ \ \ \ \ \ \ \isacommand{using}\isamarkupfalse%
\ A{\isadigit{2}}\ \isacommand{by}\isamarkupfalse%
\ {\isacharparenleft}iprover\ elim{\isacharcolon}\ conjunct{\isadigit{1}}{\isacharparenright}\isanewline
\ \ \ \ \ \ \ \ \isacommand{have}\isamarkupfalse%
\ {\isachardoublequoteopen}{\isasymforall}\ G\ H{\isachardot}\ \isactrlbold {\isasymnot}{\isacharparenleft}G\ \isactrlbold {\isasymand}\ H{\isacharparenright}\ {\isasymin}\ S\ {\isasymlongrightarrow}\ \isactrlbold {\isasymnot}\ G\ {\isasymin}\ S\ {\isasymor}\ \isactrlbold {\isasymnot}\ H\ {\isasymin}\ S{\isachardoublequoteclose}\isanewline
\ \ \ \ \ \ \ \ \ \ \isacommand{using}\isamarkupfalse%
\ assms\ \isacommand{by}\isamarkupfalse%
\ {\isacharparenleft}iprover\ elim{\isacharcolon}\ conjunct{\isadigit{2}}\ conjunct{\isadigit{1}}{\isacharparenright}\isanewline
\ \ \ \ \ \ \ \ \isacommand{thus}\isamarkupfalse%
\ {\isachardoublequoteopen}F\ {\isasymin}\ S\ {\isasymlongrightarrow}\ G\ {\isasymin}\ S\ {\isasymor}\ H\ {\isasymin}\ S{\isachardoublequoteclose}\isanewline
\ \ \ \ \ \ \ \ \ \ \isacommand{using}\isamarkupfalse%
\ {\isacartoucheopen}F\ {\isacharequal}\ \isactrlbold {\isasymnot}{\isacharparenleft}G{\isadigit{2}}\ \isactrlbold {\isasymand}\ H{\isadigit{2}}{\isacharparenright}{\isacartoucheclose}\ {\isacartoucheopen}G\ {\isacharequal}\ \isactrlbold {\isasymnot}\ G{\isadigit{2}}{\isacartoucheclose}\ {\isacartoucheopen}H\ {\isacharequal}\ \isactrlbold {\isasymnot}\ H{\isadigit{2}}{\isacartoucheclose}\ \isacommand{by}\isamarkupfalse%
\ {\isacharparenleft}iprover\ elim{\isacharcolon}\ allE{\isacharparenright}\isanewline
\ \ \ \ \ \ \isacommand{next}\isamarkupfalse%
\isanewline
\ \ \ \ \ \ \ \ \isacommand{assume}\isamarkupfalse%
\ {\isachardoublequoteopen}F\ {\isacharequal}\ \isactrlbold {\isasymnot}\ {\isacharparenleft}\isactrlbold {\isasymnot}\ G{\isacharparenright}\ {\isasymand}\ H\ {\isacharequal}\ G{\isachardoublequoteclose}\isanewline
\ \ \ \ \ \ \ \ \isacommand{then}\isamarkupfalse%
\ \isacommand{have}\isamarkupfalse%
\ {\isachardoublequoteopen}F\ {\isacharequal}\ \isactrlbold {\isasymnot}\ {\isacharparenleft}\isactrlbold {\isasymnot}\ G{\isacharparenright}{\isachardoublequoteclose}\ \isanewline
\ \ \ \ \ \ \ \ \ \ \isacommand{by}\isamarkupfalse%
\ {\isacharparenleft}rule\ conjunct{\isadigit{1}}{\isacharparenright}\isanewline
\ \ \ \ \ \ \ \ \isacommand{have}\isamarkupfalse%
\ {\isachardoublequoteopen}H\ {\isacharequal}\ G{\isachardoublequoteclose}\isanewline
\ \ \ \ \ \ \ \ \ \ \isacommand{using}\isamarkupfalse%
\ {\isacartoucheopen}F\ {\isacharequal}\ \isactrlbold {\isasymnot}\ {\isacharparenleft}\isactrlbold {\isasymnot}\ G{\isacharparenright}\ {\isasymand}\ H\ {\isacharequal}\ G{\isacartoucheclose}\ \isacommand{by}\isamarkupfalse%
\ {\isacharparenleft}rule\ conjunct{\isadigit{2}}{\isacharparenright}\isanewline
\ \ \ \ \ \ \ \ \isacommand{have}\isamarkupfalse%
\ {\isachardoublequoteopen}{\isasymforall}\ G{\isachardot}\ \isactrlbold {\isasymnot}\ {\isacharparenleft}\isactrlbold {\isasymnot}\ G{\isacharparenright}\ {\isasymin}\ S\ {\isasymlongrightarrow}\ G\ {\isasymin}\ S{\isachardoublequoteclose}\isanewline
\ \ \ \ \ \ \ \ \ \ \isacommand{using}\isamarkupfalse%
\ assms\ \isacommand{by}\isamarkupfalse%
\ {\isacharparenleft}iprover\ elim{\isacharcolon}\ conjunct{\isadigit{2}}\ conjunct{\isadigit{1}}{\isacharparenright}\isanewline
\ \ \ \ \ \ \ \ \isacommand{then}\isamarkupfalse%
\ \isacommand{have}\isamarkupfalse%
\ {\isachardoublequoteopen}\isactrlbold {\isasymnot}\ {\isacharparenleft}\isactrlbold {\isasymnot}\ G{\isacharparenright}\ {\isasymin}\ S\ {\isasymlongrightarrow}\ G\ {\isasymin}\ S{\isachardoublequoteclose}\isanewline
\ \ \ \ \ \ \ \ \ \ \isacommand{by}\isamarkupfalse%
\ {\isacharparenleft}rule\ allE{\isacharparenright}\isanewline
\ \ \ \ \ \ \ \ \isacommand{then}\isamarkupfalse%
\ \isacommand{have}\isamarkupfalse%
\ {\isachardoublequoteopen}F\ {\isasymin}\ S\ {\isasymlongrightarrow}\ G\ {\isasymin}\ S{\isachardoublequoteclose}\isanewline
\ \ \ \ \ \ \ \ \ \ \isacommand{by}\isamarkupfalse%
\ {\isacharparenleft}simp\ only{\isacharcolon}\ {\isacartoucheopen}F\ {\isacharequal}\ \isactrlbold {\isasymnot}\ {\isacharparenleft}\isactrlbold {\isasymnot}\ G{\isacharparenright}{\isacartoucheclose}{\isacharparenright}\isanewline
\ \ \ \ \ \ \ \ \isacommand{then}\isamarkupfalse%
\ \isacommand{have}\isamarkupfalse%
\ {\isachardoublequoteopen}F\ {\isasymin}\ S\ {\isasymlongrightarrow}\ G\ {\isasymin}\ S\ {\isasymor}\ G\ {\isasymin}\ S{\isachardoublequoteclose}\isanewline
\ \ \ \ \ \ \ \ \ \ \isacommand{by}\isamarkupfalse%
\ {\isacharparenleft}simp\ only{\isacharcolon}\ disj{\isacharunderscore}absorb{\isacharparenright}\isanewline
\ \ \ \ \ \ \ \ \isacommand{thus}\isamarkupfalse%
\ {\isachardoublequoteopen}F\ {\isasymin}\ S\ {\isasymlongrightarrow}\ G\ {\isasymin}\ S\ {\isasymor}\ H\ {\isasymin}\ S{\isachardoublequoteclose}\isanewline
\ \ \ \ \ \ \ \ \isacommand{by}\isamarkupfalse%
\ {\isacharparenleft}simp\ only{\isacharcolon}\ {\isacartoucheopen}H\ {\isacharequal}\ G{\isacartoucheclose}{\isacharparenright}\isanewline
\ \ \ \ \ \ \isacommand{qed}\isamarkupfalse%
\isanewline
\ \ \ \ \isacommand{qed}\isamarkupfalse%
\isanewline
\ \ \isacommand{qed}\isamarkupfalse%
\isanewline
\isacommand{qed}\isamarkupfalse%
%
\endisatagproof
{\isafoldproof}%
%
\isadelimproof
%
\endisadelimproof
%
\begin{isamarkuptext}%
Finalmente, podemos demostrar detalladamente esta primera implicación de la
  equivalencia del lema en Isabelle.%
\end{isamarkuptext}\isamarkuptrue%
\isacommand{lemma}\isamarkupfalse%
\ Hintikka{\isacharunderscore}alt{\isadigit{1}}{\isacharcolon}\isanewline
\ \ \isakeyword{assumes}\ {\isachardoublequoteopen}Hintikka\ S{\isachardoublequoteclose}\isanewline
\ \ \isakeyword{shows}\ {\isachardoublequoteopen}{\isasymbottom}\ {\isasymnotin}\ S\isanewline
{\isasymand}\ {\isacharparenleft}{\isasymforall}k{\isachardot}\ Atom\ k\ {\isasymin}\ S\ {\isasymlongrightarrow}\ \isactrlbold {\isasymnot}\ {\isacharparenleft}Atom\ k{\isacharparenright}\ {\isasymin}\ S\ {\isasymlongrightarrow}\ False{\isacharparenright}\isanewline
{\isasymand}\ {\isacharparenleft}{\isasymforall}F\ G\ H{\isachardot}\ Con\ F\ G\ H\ {\isasymlongrightarrow}\ F\ {\isasymin}\ S\ {\isasymlongrightarrow}\ G\ {\isasymin}\ S\ {\isasymand}\ H\ {\isasymin}\ S{\isacharparenright}\isanewline
{\isasymand}\ {\isacharparenleft}{\isasymforall}F\ G\ H{\isachardot}\ Dis\ F\ G\ H\ {\isasymlongrightarrow}\ F\ {\isasymin}\ S\ {\isasymlongrightarrow}\ G\ {\isasymin}\ S\ {\isasymor}\ H\ {\isasymin}\ S{\isacharparenright}{\isachardoublequoteclose}\isanewline
%
\isadelimproof
%
\endisadelimproof
%
\isatagproof
\isacommand{proof}\isamarkupfalse%
\ {\isacharminus}\isanewline
\ \ \isacommand{have}\isamarkupfalse%
\ Hk{\isacharcolon}{\isachardoublequoteopen}{\isacharparenleft}{\isasymbottom}\ {\isasymnotin}\ S\isanewline
\ \ {\isasymand}\ {\isacharparenleft}{\isasymforall}k{\isachardot}\ Atom\ k\ {\isasymin}\ S\ {\isasymlongrightarrow}\ \isactrlbold {\isasymnot}\ {\isacharparenleft}Atom\ k{\isacharparenright}\ {\isasymin}\ S\ {\isasymlongrightarrow}\ False{\isacharparenright}\isanewline
\ \ {\isasymand}\ {\isacharparenleft}{\isasymforall}G\ H{\isachardot}\ G\ \isactrlbold {\isasymand}\ H\ {\isasymin}\ S\ {\isasymlongrightarrow}\ G\ {\isasymin}\ S\ {\isasymand}\ H\ {\isasymin}\ S{\isacharparenright}\isanewline
\ \ {\isasymand}\ {\isacharparenleft}{\isasymforall}G\ H{\isachardot}\ G\ \isactrlbold {\isasymor}\ H\ {\isasymin}\ S\ {\isasymlongrightarrow}\ G\ {\isasymin}\ S\ {\isasymor}\ H\ {\isasymin}\ S{\isacharparenright}\isanewline
\ \ {\isasymand}\ {\isacharparenleft}{\isasymforall}G\ H{\isachardot}\ G\ \isactrlbold {\isasymrightarrow}\ H\ {\isasymin}\ S\ {\isasymlongrightarrow}\ \isactrlbold {\isasymnot}G\ {\isasymin}\ S\ {\isasymor}\ H\ {\isasymin}\ S{\isacharparenright}\isanewline
\ \ {\isasymand}\ {\isacharparenleft}{\isasymforall}G{\isachardot}\ \isactrlbold {\isasymnot}\ {\isacharparenleft}\isactrlbold {\isasymnot}G{\isacharparenright}\ {\isasymin}\ S\ {\isasymlongrightarrow}\ G\ {\isasymin}\ S{\isacharparenright}\isanewline
\ \ {\isasymand}\ {\isacharparenleft}{\isasymforall}G\ H{\isachardot}\ \isactrlbold {\isasymnot}{\isacharparenleft}G\ \isactrlbold {\isasymand}\ H{\isacharparenright}\ {\isasymin}\ S\ {\isasymlongrightarrow}\ \isactrlbold {\isasymnot}\ G\ {\isasymin}\ S\ {\isasymor}\ \isactrlbold {\isasymnot}\ H\ {\isasymin}\ S{\isacharparenright}\isanewline
\ \ {\isasymand}\ {\isacharparenleft}{\isasymforall}G\ H{\isachardot}\ \isactrlbold {\isasymnot}{\isacharparenleft}G\ \isactrlbold {\isasymor}\ H{\isacharparenright}\ {\isasymin}\ S\ {\isasymlongrightarrow}\ \isactrlbold {\isasymnot}\ G\ {\isasymin}\ S\ {\isasymand}\ \isactrlbold {\isasymnot}\ H\ {\isasymin}\ S{\isacharparenright}\isanewline
\ \ {\isasymand}\ {\isacharparenleft}{\isasymforall}G\ H{\isachardot}\ \isactrlbold {\isasymnot}{\isacharparenleft}G\ \isactrlbold {\isasymrightarrow}\ H{\isacharparenright}\ {\isasymin}\ S\ {\isasymlongrightarrow}\ G\ {\isasymin}\ S\ {\isasymand}\ \isactrlbold {\isasymnot}\ H\ {\isasymin}\ S{\isacharparenright}{\isacharparenright}{\isachardoublequoteclose}\isanewline
\ \ \ \ \isacommand{using}\isamarkupfalse%
\ assms\ \isacommand{by}\isamarkupfalse%
\ {\isacharparenleft}rule\ auxEq{\isacharparenright}\isanewline
\ \ \isacommand{then}\isamarkupfalse%
\ \isacommand{have}\isamarkupfalse%
\ C{\isadigit{1}}{\isacharcolon}\ {\isachardoublequoteopen}{\isasymbottom}\ {\isasymnotin}\ S{\isachardoublequoteclose}\isanewline
\ \ \ \ \isacommand{by}\isamarkupfalse%
\ {\isacharparenleft}rule\ conjunct{\isadigit{1}}{\isacharparenright}\isanewline
\ \ \isacommand{have}\isamarkupfalse%
\ C{\isadigit{2}}{\isacharcolon}\ {\isachardoublequoteopen}{\isasymforall}k{\isachardot}\ Atom\ k\ {\isasymin}\ S\ {\isasymlongrightarrow}\ \isactrlbold {\isasymnot}\ {\isacharparenleft}Atom\ k{\isacharparenright}\ {\isasymin}\ S\ {\isasymlongrightarrow}\ False{\isachardoublequoteclose}\isanewline
\ \ \ \ \isacommand{using}\isamarkupfalse%
\ Hk\ \isacommand{by}\isamarkupfalse%
\ {\isacharparenleft}iprover\ elim{\isacharcolon}\ conjunct{\isadigit{2}}\ conjunct{\isadigit{1}}{\isacharparenright}\isanewline
\ \ \isacommand{have}\isamarkupfalse%
\ C{\isadigit{3}}{\isacharcolon}\ {\isachardoublequoteopen}{\isasymforall}F\ G\ H{\isachardot}\ Con\ F\ G\ H\ {\isasymlongrightarrow}\ F\ {\isasymin}\ S\ {\isasymlongrightarrow}\ G\ {\isasymin}\ S\ {\isasymand}\ H\ {\isasymin}\ S{\isachardoublequoteclose}\isanewline
\ \ \isacommand{proof}\isamarkupfalse%
\ {\isacharparenleft}rule\ allI{\isacharparenright}{\isacharplus}\isanewline
\ \ \ \ \isacommand{fix}\isamarkupfalse%
\ F\ G\ H\isanewline
\ \ \ \ \isacommand{have}\isamarkupfalse%
\ C{\isadigit{3}}{\isadigit{1}}{\isacharcolon}{\isachardoublequoteopen}{\isasymforall}G\ H{\isachardot}\ G\ \isactrlbold {\isasymand}\ H\ {\isasymin}\ S\ {\isasymlongrightarrow}\ G\ {\isasymin}\ S\ {\isasymand}\ H\ {\isasymin}\ S{\isachardoublequoteclose}\isanewline
\ \ \ \ \ \ \isacommand{using}\isamarkupfalse%
\ Hk\ \isacommand{by}\isamarkupfalse%
\ {\isacharparenleft}iprover\ elim{\isacharcolon}\ conjunct{\isadigit{2}}\ conjunct{\isadigit{1}}{\isacharparenright}\isanewline
\ \ \ \ \isacommand{have}\isamarkupfalse%
\ C{\isadigit{3}}{\isadigit{2}}{\isacharcolon}{\isachardoublequoteopen}{\isasymforall}G{\isachardot}\ \isactrlbold {\isasymnot}\ {\isacharparenleft}\isactrlbold {\isasymnot}\ G{\isacharparenright}\ {\isasymin}\ S\ {\isasymlongrightarrow}\ G\ {\isasymin}\ S{\isachardoublequoteclose}\isanewline
\ \ \ \ \ \ \isacommand{using}\isamarkupfalse%
\ Hk\ \isacommand{by}\isamarkupfalse%
\ {\isacharparenleft}iprover\ elim{\isacharcolon}\ conjunct{\isadigit{2}}\ conjunct{\isadigit{1}}{\isacharparenright}\isanewline
\ \ \ \ \isacommand{have}\isamarkupfalse%
\ C{\isadigit{3}}{\isadigit{3}}{\isacharcolon}{\isachardoublequoteopen}{\isasymforall}G\ H{\isachardot}\ \isactrlbold {\isasymnot}{\isacharparenleft}G\ \isactrlbold {\isasymor}\ H{\isacharparenright}\ {\isasymin}\ S\ {\isasymlongrightarrow}\ \isactrlbold {\isasymnot}\ G\ {\isasymin}\ S\ {\isasymand}\ \isactrlbold {\isasymnot}\ H\ {\isasymin}\ S{\isachardoublequoteclose}\isanewline
\ \ \ \ \ \ \isacommand{using}\isamarkupfalse%
\ Hk\ \isacommand{by}\isamarkupfalse%
\ {\isacharparenleft}iprover\ elim{\isacharcolon}\ conjunct{\isadigit{2}}\ conjunct{\isadigit{1}}{\isacharparenright}\isanewline
\ \ \ \ \isacommand{have}\isamarkupfalse%
\ C{\isadigit{3}}{\isadigit{4}}{\isacharcolon}{\isachardoublequoteopen}{\isasymforall}G\ H{\isachardot}\ \isactrlbold {\isasymnot}{\isacharparenleft}G\ \isactrlbold {\isasymrightarrow}\ H{\isacharparenright}\ {\isasymin}\ S\ {\isasymlongrightarrow}\ G\ {\isasymin}\ S\ {\isasymand}\ \isactrlbold {\isasymnot}\ H\ {\isasymin}\ S{\isachardoublequoteclose}\isanewline
\ \ \ \ \ \ \isacommand{using}\isamarkupfalse%
\ Hk\ \isacommand{by}\isamarkupfalse%
\ {\isacharparenleft}iprover\ elim{\isacharcolon}\ conjunct{\isadigit{2}}\ conjunct{\isadigit{1}}{\isacharparenright}\isanewline
\ \ \ \ \isacommand{have}\isamarkupfalse%
\ {\isachardoublequoteopen}{\isacharparenleft}{\isasymforall}G\ H{\isachardot}\ G\ \isactrlbold {\isasymand}\ H\ {\isasymin}\ S\ {\isasymlongrightarrow}\ G\ {\isasymin}\ S\ {\isasymand}\ H\ {\isasymin}\ S{\isacharparenright}\isanewline
\ \ \ \ \ \ \ \ \ \ {\isasymand}\ {\isacharparenleft}{\isasymforall}G{\isachardot}\ \isactrlbold {\isasymnot}\ {\isacharparenleft}\isactrlbold {\isasymnot}\ G{\isacharparenright}\ {\isasymin}\ S\ {\isasymlongrightarrow}\ G\ {\isasymin}\ S{\isacharparenright}\isanewline
\ \ \ \ \ \ \ \ \ \ {\isasymand}\ {\isacharparenleft}{\isasymforall}G\ H{\isachardot}\ \isactrlbold {\isasymnot}{\isacharparenleft}G\ \isactrlbold {\isasymor}\ H{\isacharparenright}\ {\isasymin}\ S\ {\isasymlongrightarrow}\ \isactrlbold {\isasymnot}\ G\ {\isasymin}\ S\ {\isasymand}\ \isactrlbold {\isasymnot}\ H\ {\isasymin}\ S{\isacharparenright}\isanewline
\ \ \ \ \ \ \ \ \ \ {\isasymand}\ {\isacharparenleft}{\isasymforall}G\ H{\isachardot}\ \isactrlbold {\isasymnot}{\isacharparenleft}G\ \isactrlbold {\isasymrightarrow}\ H{\isacharparenright}\ {\isasymin}\ S\ {\isasymlongrightarrow}\ G\ {\isasymin}\ S\ {\isasymand}\ \isactrlbold {\isasymnot}\ H\ {\isasymin}\ S{\isacharparenright}{\isachardoublequoteclose}\ \isanewline
\ \ \ \ \ \ \isacommand{using}\isamarkupfalse%
\ C{\isadigit{3}}{\isadigit{1}}\ C{\isadigit{3}}{\isadigit{2}}\ C{\isadigit{3}}{\isadigit{3}}\ C{\isadigit{3}}{\isadigit{4}}\ \isacommand{by}\isamarkupfalse%
\ {\isacharparenleft}iprover\ intro{\isacharcolon}\ conjI{\isacharparenright}\isanewline
\ \ \ \ \isacommand{thus}\isamarkupfalse%
\ {\isachardoublequoteopen}Con\ F\ G\ H\ {\isasymlongrightarrow}\ F\ {\isasymin}\ S\ {\isasymlongrightarrow}\ G\ {\isasymin}\ S\ {\isasymand}\ H\ {\isasymin}\ S{\isachardoublequoteclose}\isanewline
\ \ \ \ \ \ \isacommand{by}\isamarkupfalse%
\ {\isacharparenleft}rule\ Hintikka{\isacharunderscore}alt{\isadigit{1}}Con{\isacharparenright}\isanewline
\ \ \isacommand{qed}\isamarkupfalse%
\isanewline
\ \ \isacommand{have}\isamarkupfalse%
\ C{\isadigit{4}}{\isacharcolon}{\isachardoublequoteopen}{\isasymforall}F\ G\ H{\isachardot}\ Dis\ F\ G\ H\ {\isasymlongrightarrow}\ F\ {\isasymin}\ S\ {\isasymlongrightarrow}\ G\ {\isasymin}\ S\ {\isasymor}\ H\ {\isasymin}\ S{\isachardoublequoteclose}\isanewline
\ \ \isacommand{proof}\isamarkupfalse%
\ {\isacharparenleft}rule\ allI{\isacharparenright}{\isacharplus}\isanewline
\ \ \ \ \isacommand{fix}\isamarkupfalse%
\ F\ G\ H\isanewline
\ \ \ \ \isacommand{have}\isamarkupfalse%
\ C{\isadigit{4}}{\isadigit{1}}{\isacharcolon}{\isachardoublequoteopen}{\isasymforall}G\ H{\isachardot}\ G\ \isactrlbold {\isasymor}\ H\ {\isasymin}\ S\ {\isasymlongrightarrow}\ G\ {\isasymin}\ S\ {\isasymor}\ H\ {\isasymin}\ S{\isachardoublequoteclose}\isanewline
\ \ \ \ \ \ \isacommand{using}\isamarkupfalse%
\ Hk\ \isacommand{by}\isamarkupfalse%
\ {\isacharparenleft}iprover\ elim{\isacharcolon}\ conjunct{\isadigit{2}}\ conjunct{\isadigit{1}}{\isacharparenright}\isanewline
\ \ \ \ \isacommand{have}\isamarkupfalse%
\ C{\isadigit{4}}{\isadigit{2}}{\isacharcolon}{\isachardoublequoteopen}{\isasymforall}G\ H{\isachardot}\ G\ \isactrlbold {\isasymrightarrow}\ H\ {\isasymin}\ S\ {\isasymlongrightarrow}\ \isactrlbold {\isasymnot}\ G\ {\isasymin}\ S\ {\isasymor}\ H\ {\isasymin}\ S{\isachardoublequoteclose}\isanewline
\ \ \ \ \ \ \isacommand{using}\isamarkupfalse%
\ Hk\ \isacommand{by}\isamarkupfalse%
\ {\isacharparenleft}iprover\ elim{\isacharcolon}\ conjunct{\isadigit{2}}\ conjunct{\isadigit{1}}{\isacharparenright}\isanewline
\ \ \ \ \isacommand{have}\isamarkupfalse%
\ C{\isadigit{4}}{\isadigit{3}}{\isacharcolon}{\isachardoublequoteopen}{\isasymforall}G{\isachardot}\ \isactrlbold {\isasymnot}\ {\isacharparenleft}\isactrlbold {\isasymnot}\ G{\isacharparenright}\ {\isasymin}\ S\ {\isasymlongrightarrow}\ G\ {\isasymin}\ S{\isachardoublequoteclose}\isanewline
\ \ \ \ \ \ \isacommand{using}\isamarkupfalse%
\ Hk\ \isacommand{by}\isamarkupfalse%
\ {\isacharparenleft}iprover\ elim{\isacharcolon}\ conjunct{\isadigit{2}}\ conjunct{\isadigit{1}}{\isacharparenright}\isanewline
\ \ \ \ \isacommand{have}\isamarkupfalse%
\ C{\isadigit{4}}{\isadigit{4}}{\isacharcolon}{\isachardoublequoteopen}{\isasymforall}G\ H{\isachardot}\ \isactrlbold {\isasymnot}{\isacharparenleft}G\ \isactrlbold {\isasymand}\ H{\isacharparenright}\ {\isasymin}\ S\ {\isasymlongrightarrow}\ \isactrlbold {\isasymnot}\ G\ {\isasymin}\ S\ {\isasymor}\ \isactrlbold {\isasymnot}\ H\ {\isasymin}\ S{\isachardoublequoteclose}\isanewline
\ \ \ \ \ \ \isacommand{using}\isamarkupfalse%
\ Hk\ \isacommand{by}\isamarkupfalse%
\ {\isacharparenleft}iprover\ elim{\isacharcolon}\ conjunct{\isadigit{2}}\ conjunct{\isadigit{1}}{\isacharparenright}\isanewline
\ \ \ \ \isacommand{have}\isamarkupfalse%
\ {\isachardoublequoteopen}{\isacharparenleft}{\isasymforall}G\ H{\isachardot}\ G\ \isactrlbold {\isasymor}\ H\ {\isasymin}\ S\ {\isasymlongrightarrow}\ G\ {\isasymin}\ S\ {\isasymor}\ H\ {\isasymin}\ S{\isacharparenright}\isanewline
\ \ \ \ \ \ \ \ \ \ {\isasymand}\ {\isacharparenleft}{\isasymforall}G\ H{\isachardot}\ G\ \isactrlbold {\isasymrightarrow}\ H\ {\isasymin}\ S\ {\isasymlongrightarrow}\ \isactrlbold {\isasymnot}\ G\ {\isasymin}\ S\ {\isasymor}\ H\ {\isasymin}\ S{\isacharparenright}\isanewline
\ \ \ \ \ \ \ \ \ \ {\isasymand}\ {\isacharparenleft}{\isasymforall}G{\isachardot}\ \isactrlbold {\isasymnot}\ {\isacharparenleft}\isactrlbold {\isasymnot}\ G{\isacharparenright}\ {\isasymin}\ S\ {\isasymlongrightarrow}\ G\ {\isasymin}\ S{\isacharparenright}\isanewline
\ \ \ \ \ \ \ \ \ \ {\isasymand}\ {\isacharparenleft}{\isasymforall}G\ H{\isachardot}\ \isactrlbold {\isasymnot}{\isacharparenleft}G\ \isactrlbold {\isasymand}\ H{\isacharparenright}\ {\isasymin}\ S\ {\isasymlongrightarrow}\ \isactrlbold {\isasymnot}\ G\ {\isasymin}\ S\ {\isasymor}\ \isactrlbold {\isasymnot}\ H\ {\isasymin}\ S{\isacharparenright}{\isachardoublequoteclose}\isanewline
\ \ \ \ \ \ \isacommand{using}\isamarkupfalse%
\ C{\isadigit{4}}{\isadigit{1}}\ C{\isadigit{4}}{\isadigit{2}}\ C{\isadigit{4}}{\isadigit{3}}\ C{\isadigit{4}}{\isadigit{4}}\ \isacommand{by}\isamarkupfalse%
\ {\isacharparenleft}iprover\ intro{\isacharcolon}\ conjI{\isacharparenright}\isanewline
\ \ \ \ \isacommand{thus}\isamarkupfalse%
\ {\isachardoublequoteopen}Dis\ F\ G\ H\ {\isasymlongrightarrow}\ F\ {\isasymin}\ S\ {\isasymlongrightarrow}\ G\ {\isasymin}\ S\ {\isasymor}\ H\ {\isasymin}\ S{\isachardoublequoteclose}\isanewline
\ \ \ \ \ \ \isacommand{by}\isamarkupfalse%
\ {\isacharparenleft}rule\ Hintikka{\isacharunderscore}alt{\isadigit{1}}Dis{\isacharparenright}\isanewline
\ \ \isacommand{qed}\isamarkupfalse%
\isanewline
\ \ \isacommand{show}\isamarkupfalse%
\ {\isachardoublequoteopen}{\isasymbottom}\ {\isasymnotin}\ S\isanewline
\ \ {\isasymand}\ {\isacharparenleft}{\isasymforall}k{\isachardot}\ Atom\ k\ {\isasymin}\ S\ {\isasymlongrightarrow}\ \isactrlbold {\isasymnot}\ {\isacharparenleft}Atom\ k{\isacharparenright}\ {\isasymin}\ S\ {\isasymlongrightarrow}\ False{\isacharparenright}\isanewline
\ \ {\isasymand}\ {\isacharparenleft}{\isasymforall}F\ G\ H{\isachardot}\ Con\ F\ G\ H\ {\isasymlongrightarrow}\ F\ {\isasymin}\ S\ {\isasymlongrightarrow}\ G\ {\isasymin}\ S\ {\isasymand}\ H\ {\isasymin}\ S{\isacharparenright}\isanewline
\ \ {\isasymand}\ {\isacharparenleft}{\isasymforall}F\ G\ H{\isachardot}\ Dis\ F\ G\ H\ {\isasymlongrightarrow}\ F\ {\isasymin}\ S\ {\isasymlongrightarrow}\ G\ {\isasymin}\ S\ {\isasymor}\ H\ {\isasymin}\ S{\isacharparenright}{\isachardoublequoteclose}\isanewline
\ \ \ \ \isacommand{using}\isamarkupfalse%
\ C{\isadigit{1}}\ C{\isadigit{2}}\ C{\isadigit{3}}\ C{\isadigit{4}}\ \isacommand{by}\isamarkupfalse%
\ {\isacharparenleft}iprover\ intro{\isacharcolon}\ conjI{\isacharparenright}\isanewline
\isacommand{qed}\isamarkupfalse%
%
\endisatagproof
{\isafoldproof}%
%
\isadelimproof
%
\endisadelimproof
%
\begin{isamarkuptext}%
Por último, probamos la implicación recíproca de forma detallada en Isabelle mediante
  el siguiente lema.%
\end{isamarkuptext}\isamarkuptrue%
\isacommand{lemma}\isamarkupfalse%
\ Hintikka{\isacharunderscore}alt{\isadigit{2}}{\isacharcolon}\isanewline
\ \ \isakeyword{assumes}\ {\isachardoublequoteopen}{\isasymbottom}\ {\isasymnotin}\ S\isanewline
{\isasymand}\ {\isacharparenleft}{\isasymforall}k{\isachardot}\ Atom\ k\ {\isasymin}\ S\ {\isasymlongrightarrow}\ \isactrlbold {\isasymnot}\ {\isacharparenleft}Atom\ k{\isacharparenright}\ {\isasymin}\ S\ {\isasymlongrightarrow}\ False{\isacharparenright}\isanewline
{\isasymand}\ {\isacharparenleft}{\isasymforall}F\ G\ H{\isachardot}\ Con\ F\ G\ H\ {\isasymlongrightarrow}\ F\ {\isasymin}\ S\ {\isasymlongrightarrow}\ G\ {\isasymin}\ S\ {\isasymand}\ H\ {\isasymin}\ S{\isacharparenright}\ \isanewline
{\isasymand}\ {\isacharparenleft}{\isasymforall}F\ G\ H{\isachardot}\ Dis\ F\ G\ H\ {\isasymlongrightarrow}\ F\ {\isasymin}\ S\ {\isasymlongrightarrow}\ G\ {\isasymin}\ S\ {\isasymor}\ H\ {\isasymin}\ S{\isacharparenright}{\isachardoublequoteclose}\ \ \isanewline
\ \ \isakeyword{shows}\ {\isachardoublequoteopen}Hintikka\ S{\isachardoublequoteclose}\isanewline
%
\isadelimproof
%
\endisadelimproof
%
\isatagproof
\isacommand{proof}\isamarkupfalse%
\ {\isacharminus}\isanewline
\ \ \isacommand{have}\isamarkupfalse%
\ Con{\isacharcolon}{\isachardoublequoteopen}{\isasymforall}F\ G\ H{\isachardot}\ Con\ F\ G\ H\ {\isasymlongrightarrow}\ F\ {\isasymin}\ S\ {\isasymlongrightarrow}\ G\ {\isasymin}\ S\ {\isasymand}\ H\ {\isasymin}\ S{\isachardoublequoteclose}\isanewline
\ \ \ \ \isacommand{using}\isamarkupfalse%
\ assms\ \isacommand{by}\isamarkupfalse%
\ {\isacharparenleft}iprover\ elim{\isacharcolon}\ conjunct{\isadigit{2}}\ conjunct{\isadigit{1}}{\isacharparenright}\isanewline
\ \ \isacommand{have}\isamarkupfalse%
\ Dis{\isacharcolon}{\isachardoublequoteopen}{\isasymforall}F\ G\ H{\isachardot}\ Dis\ F\ G\ H\ {\isasymlongrightarrow}\ F\ {\isasymin}\ S\ {\isasymlongrightarrow}\ G\ {\isasymin}\ S\ {\isasymor}\ H\ {\isasymin}\ S{\isachardoublequoteclose}\isanewline
\ \ \ \ \isacommand{using}\isamarkupfalse%
\ assms\ \isacommand{by}\isamarkupfalse%
\ {\isacharparenleft}iprover\ elim{\isacharcolon}\ conjunct{\isadigit{2}}\ conjunct{\isadigit{1}}{\isacharparenright}\isanewline
\ \ \isacommand{have}\isamarkupfalse%
\ {\isachardoublequoteopen}{\isasymbottom}\ {\isasymnotin}\ S\isanewline
\ \ {\isasymand}\ {\isacharparenleft}{\isasymforall}k{\isachardot}\ Atom\ k\ {\isasymin}\ S\ {\isasymlongrightarrow}\ \isactrlbold {\isasymnot}\ {\isacharparenleft}Atom\ k{\isacharparenright}\ {\isasymin}\ S\ {\isasymlongrightarrow}\ False{\isacharparenright}\isanewline
\ \ {\isasymand}\ {\isacharparenleft}{\isasymforall}G\ H{\isachardot}\ G\ \isactrlbold {\isasymand}\ H\ {\isasymin}\ S\ {\isasymlongrightarrow}\ G\ {\isasymin}\ S\ {\isasymand}\ H\ {\isasymin}\ S{\isacharparenright}\isanewline
\ \ {\isasymand}\ {\isacharparenleft}{\isasymforall}G\ H{\isachardot}\ G\ \isactrlbold {\isasymor}\ H\ {\isasymin}\ S\ {\isasymlongrightarrow}\ G\ {\isasymin}\ S\ {\isasymor}\ H\ {\isasymin}\ S{\isacharparenright}\isanewline
\ \ {\isasymand}\ {\isacharparenleft}{\isasymforall}G\ H{\isachardot}\ G\ \isactrlbold {\isasymrightarrow}\ H\ {\isasymin}\ S\ {\isasymlongrightarrow}\ \isactrlbold {\isasymnot}G\ {\isasymin}\ S\ {\isasymor}\ H\ {\isasymin}\ S{\isacharparenright}\isanewline
\ \ {\isasymand}\ {\isacharparenleft}{\isasymforall}G{\isachardot}\ \isactrlbold {\isasymnot}\ {\isacharparenleft}\isactrlbold {\isasymnot}G{\isacharparenright}\ {\isasymin}\ S\ {\isasymlongrightarrow}\ G\ {\isasymin}\ S{\isacharparenright}\isanewline
\ \ {\isasymand}\ {\isacharparenleft}{\isasymforall}G\ H{\isachardot}\ \isactrlbold {\isasymnot}{\isacharparenleft}G\ \isactrlbold {\isasymand}\ H{\isacharparenright}\ {\isasymin}\ S\ {\isasymlongrightarrow}\ \isactrlbold {\isasymnot}\ G\ {\isasymin}\ S\ {\isasymor}\ \isactrlbold {\isasymnot}\ H\ {\isasymin}\ S{\isacharparenright}\isanewline
\ \ {\isasymand}\ {\isacharparenleft}{\isasymforall}G\ H{\isachardot}\ \isactrlbold {\isasymnot}{\isacharparenleft}G\ \isactrlbold {\isasymor}\ H{\isacharparenright}\ {\isasymin}\ S\ {\isasymlongrightarrow}\ \isactrlbold {\isasymnot}\ G\ {\isasymin}\ S\ {\isasymand}\ \isactrlbold {\isasymnot}\ H\ {\isasymin}\ S{\isacharparenright}\isanewline
\ \ {\isasymand}\ {\isacharparenleft}{\isasymforall}G\ H{\isachardot}\ \isactrlbold {\isasymnot}{\isacharparenleft}G\ \isactrlbold {\isasymrightarrow}\ H{\isacharparenright}\ {\isasymin}\ S\ {\isasymlongrightarrow}\ G\ {\isasymin}\ S\ {\isasymand}\ \isactrlbold {\isasymnot}\ H\ {\isasymin}\ S{\isacharparenright}{\isachardoublequoteclose}\isanewline
\ \ \isacommand{proof}\isamarkupfalse%
\ {\isacharminus}\isanewline
\ \ \ \ \isacommand{have}\isamarkupfalse%
\ C{\isadigit{1}}{\isacharcolon}{\isachardoublequoteopen}{\isasymbottom}\ {\isasymnotin}\ S{\isachardoublequoteclose}\isanewline
\ \ \ \ \ \ \isacommand{using}\isamarkupfalse%
\ assms\ \isacommand{by}\isamarkupfalse%
\ {\isacharparenleft}rule\ conjunct{\isadigit{1}}{\isacharparenright}\isanewline
\ \ \ \ \isacommand{have}\isamarkupfalse%
\ C{\isadigit{2}}{\isacharcolon}{\isachardoublequoteopen}{\isasymforall}k{\isachardot}\ Atom\ k\ {\isasymin}\ S\ {\isasymlongrightarrow}\ \isactrlbold {\isasymnot}\ {\isacharparenleft}Atom\ k{\isacharparenright}\ {\isasymin}\ S\ {\isasymlongrightarrow}\ False{\isachardoublequoteclose}\isanewline
\ \ \ \ \ \ \isacommand{using}\isamarkupfalse%
\ assms\ \isacommand{by}\isamarkupfalse%
\ {\isacharparenleft}iprover\ elim{\isacharcolon}\ conjunct{\isadigit{2}}\ conjunct{\isadigit{1}}{\isacharparenright}\isanewline
\ \ \ \ \isacommand{have}\isamarkupfalse%
\ C{\isadigit{3}}{\isacharcolon}{\isachardoublequoteopen}{\isasymforall}G\ H{\isachardot}\ G\ \isactrlbold {\isasymand}\ H\ {\isasymin}\ S\ {\isasymlongrightarrow}\ G\ {\isasymin}\ S\ {\isasymand}\ H\ {\isasymin}\ S{\isachardoublequoteclose}\isanewline
\ \ \ \ \isacommand{proof}\isamarkupfalse%
\ {\isacharparenleft}rule\ allI{\isacharparenright}{\isacharplus}\isanewline
\ \ \ \ \ \ \isacommand{fix}\isamarkupfalse%
\ G\ H\isanewline
\ \ \ \ \ \ \isacommand{show}\isamarkupfalse%
\ {\isachardoublequoteopen}G\ \isactrlbold {\isasymand}\ H\ {\isasymin}\ S\ {\isasymlongrightarrow}\ G\ {\isasymin}\ S\ {\isasymand}\ H\ {\isasymin}\ S{\isachardoublequoteclose}\isanewline
\ \ \ \ \ \ \isacommand{proof}\isamarkupfalse%
\ {\isacharparenleft}rule\ impI{\isacharparenright}\isanewline
\ \ \ \ \ \ \ \ \isacommand{assume}\isamarkupfalse%
\ {\isachardoublequoteopen}G\ \isactrlbold {\isasymand}\ H\ {\isasymin}\ S{\isachardoublequoteclose}\isanewline
\ \ \ \ \ \ \ \ \isacommand{have}\isamarkupfalse%
\ {\isachardoublequoteopen}Con\ {\isacharparenleft}G\ \isactrlbold {\isasymand}\ H{\isacharparenright}\ G\ H{\isachardoublequoteclose}\isanewline
\ \ \ \ \ \ \ \ \ \ \isacommand{by}\isamarkupfalse%
\ {\isacharparenleft}simp\ only{\isacharcolon}\ Con{\isachardot}intros{\isacharparenleft}{\isadigit{1}}{\isacharparenright}{\isacharparenright}\isanewline
\ \ \ \ \ \ \ \ \isacommand{have}\isamarkupfalse%
\ {\isachardoublequoteopen}Con\ {\isacharparenleft}G\ \isactrlbold {\isasymand}\ H{\isacharparenright}\ G\ H\ {\isasymlongrightarrow}\ G\ \isactrlbold {\isasymand}\ H\ {\isasymin}\ S\ {\isasymlongrightarrow}\ G\ {\isasymin}\ S\ {\isasymand}\ H\ {\isasymin}\ S{\isachardoublequoteclose}\isanewline
\ \ \ \ \ \ \ \ \ \ \isacommand{using}\isamarkupfalse%
\ Con\ \isacommand{by}\isamarkupfalse%
\ {\isacharparenleft}iprover\ elim{\isacharcolon}\ allE{\isacharparenright}\isanewline
\ \ \ \ \ \ \ \ \isacommand{then}\isamarkupfalse%
\ \isacommand{have}\isamarkupfalse%
\ {\isachardoublequoteopen}G\ \isactrlbold {\isasymand}\ H\ {\isasymin}\ S\ {\isasymlongrightarrow}\ G\ {\isasymin}\ S\ {\isasymand}\ H\ {\isasymin}\ S{\isachardoublequoteclose}\isanewline
\ \ \ \ \ \ \ \ \ \ \isacommand{using}\isamarkupfalse%
\ {\isacartoucheopen}Con\ {\isacharparenleft}G\ \isactrlbold {\isasymand}\ H{\isacharparenright}\ G\ H{\isacartoucheclose}\ \isacommand{by}\isamarkupfalse%
\ {\isacharparenleft}rule\ mp{\isacharparenright}\isanewline
\ \ \ \ \ \ \ \ \isacommand{thus}\isamarkupfalse%
\ {\isachardoublequoteopen}G\ {\isasymin}\ S\ {\isasymand}\ H\ {\isasymin}\ S{\isachardoublequoteclose}\isanewline
\ \ \ \ \ \ \ \ \ \ \isacommand{using}\isamarkupfalse%
\ {\isacartoucheopen}G\ \isactrlbold {\isasymand}\ H\ {\isasymin}\ S{\isacartoucheclose}\ \isacommand{by}\isamarkupfalse%
\ {\isacharparenleft}rule\ mp{\isacharparenright}\isanewline
\ \ \ \ \ \ \isacommand{qed}\isamarkupfalse%
\isanewline
\ \ \ \ \isacommand{qed}\isamarkupfalse%
\isanewline
\ \ \ \ \isacommand{have}\isamarkupfalse%
\ C{\isadigit{4}}{\isacharcolon}{\isachardoublequoteopen}{\isasymforall}G\ H{\isachardot}\ G\ \isactrlbold {\isasymor}\ H\ {\isasymin}\ S\ {\isasymlongrightarrow}\ G\ {\isasymin}\ S\ {\isasymor}\ H\ {\isasymin}\ S{\isachardoublequoteclose}\isanewline
\ \ \ \ \isacommand{proof}\isamarkupfalse%
\ {\isacharparenleft}rule\ allI{\isacharparenright}{\isacharplus}\isanewline
\ \ \ \ \ \ \isacommand{fix}\isamarkupfalse%
\ G\ H\isanewline
\ \ \ \ \ \ \isacommand{show}\isamarkupfalse%
\ {\isachardoublequoteopen}G\ \isactrlbold {\isasymor}\ H\ {\isasymin}\ S\ {\isasymlongrightarrow}\ G\ {\isasymin}\ S\ {\isasymor}\ H\ {\isasymin}\ S{\isachardoublequoteclose}\isanewline
\ \ \ \ \ \ \isacommand{proof}\isamarkupfalse%
\ {\isacharparenleft}rule\ impI{\isacharparenright}\isanewline
\ \ \ \ \ \ \ \ \isacommand{assume}\isamarkupfalse%
\ {\isachardoublequoteopen}G\ \isactrlbold {\isasymor}\ H\ {\isasymin}\ S{\isachardoublequoteclose}\isanewline
\ \ \ \ \ \ \ \ \isacommand{have}\isamarkupfalse%
\ {\isachardoublequoteopen}Dis\ {\isacharparenleft}G\ \isactrlbold {\isasymor}\ H{\isacharparenright}\ G\ H{\isachardoublequoteclose}\isanewline
\ \ \ \ \ \ \ \ \ \ \isacommand{by}\isamarkupfalse%
\ {\isacharparenleft}simp\ only{\isacharcolon}\ Dis{\isachardot}intros{\isacharparenleft}{\isadigit{1}}{\isacharparenright}{\isacharparenright}\isanewline
\ \ \ \ \ \ \ \ \isacommand{have}\isamarkupfalse%
\ {\isachardoublequoteopen}Dis\ {\isacharparenleft}G\ \isactrlbold {\isasymor}\ H{\isacharparenright}\ G\ H\ {\isasymlongrightarrow}\ G\ \isactrlbold {\isasymor}\ H\ {\isasymin}\ S\ {\isasymlongrightarrow}\ G\ {\isasymin}\ S\ {\isasymor}\ H\ {\isasymin}\ S{\isachardoublequoteclose}\isanewline
\ \ \ \ \ \ \ \ \ \ \isacommand{using}\isamarkupfalse%
\ Dis\ \isacommand{by}\isamarkupfalse%
\ {\isacharparenleft}iprover\ elim{\isacharcolon}\ allE{\isacharparenright}\isanewline
\ \ \ \ \ \ \ \ \isacommand{then}\isamarkupfalse%
\ \isacommand{have}\isamarkupfalse%
\ {\isachardoublequoteopen}G\ \isactrlbold {\isasymor}\ H\ {\isasymin}\ S\ {\isasymlongrightarrow}\ G\ {\isasymin}\ S\ {\isasymor}\ H\ {\isasymin}\ S{\isachardoublequoteclose}\isanewline
\ \ \ \ \ \ \ \ \ \ \isacommand{using}\isamarkupfalse%
\ {\isacartoucheopen}Dis\ {\isacharparenleft}G\ \isactrlbold {\isasymor}\ H{\isacharparenright}\ G\ H{\isacartoucheclose}\ \isacommand{by}\isamarkupfalse%
\ {\isacharparenleft}rule\ mp{\isacharparenright}\isanewline
\ \ \ \ \ \ \ \ \isacommand{thus}\isamarkupfalse%
\ {\isachardoublequoteopen}G\ {\isasymin}\ S\ {\isasymor}\ H\ {\isasymin}\ S{\isachardoublequoteclose}\isanewline
\ \ \ \ \ \ \ \ \ \ \isacommand{using}\isamarkupfalse%
\ {\isacartoucheopen}G\ \isactrlbold {\isasymor}\ H\ {\isasymin}\ S{\isacartoucheclose}\ \isacommand{by}\isamarkupfalse%
\ {\isacharparenleft}rule\ mp{\isacharparenright}\isanewline
\ \ \ \ \ \ \isacommand{qed}\isamarkupfalse%
\isanewline
\ \ \ \ \isacommand{qed}\isamarkupfalse%
\isanewline
\ \ \ \ \isacommand{have}\isamarkupfalse%
\ C{\isadigit{5}}{\isacharcolon}{\isachardoublequoteopen}{\isasymforall}G\ H{\isachardot}\ G\ \isactrlbold {\isasymrightarrow}\ H\ {\isasymin}\ S\ {\isasymlongrightarrow}\ \isactrlbold {\isasymnot}\ G\ {\isasymin}\ S\ {\isasymor}\ H\ {\isasymin}\ S{\isachardoublequoteclose}\isanewline
\ \ \ \ \isacommand{proof}\isamarkupfalse%
\ {\isacharparenleft}rule\ allI{\isacharparenright}{\isacharplus}\isanewline
\ \ \ \ \ \ \isacommand{fix}\isamarkupfalse%
\ G\ H\isanewline
\ \ \ \ \ \ \isacommand{show}\isamarkupfalse%
\ {\isachardoublequoteopen}G\ \isactrlbold {\isasymrightarrow}\ H\ {\isasymin}\ S\ {\isasymlongrightarrow}\ \isactrlbold {\isasymnot}\ G\ {\isasymin}\ S\ {\isasymor}\ H\ {\isasymin}\ S{\isachardoublequoteclose}\isanewline
\ \ \ \ \ \ \isacommand{proof}\isamarkupfalse%
\ {\isacharparenleft}rule\ impI{\isacharparenright}\isanewline
\ \ \ \ \ \ \ \ \isacommand{assume}\isamarkupfalse%
\ {\isachardoublequoteopen}G\ \isactrlbold {\isasymrightarrow}\ H\ {\isasymin}\ S{\isachardoublequoteclose}\ \isanewline
\ \ \ \ \ \ \ \ \isacommand{have}\isamarkupfalse%
\ {\isachardoublequoteopen}Dis\ {\isacharparenleft}G\ \isactrlbold {\isasymrightarrow}\ H{\isacharparenright}\ {\isacharparenleft}\isactrlbold {\isasymnot}\ G{\isacharparenright}\ H{\isachardoublequoteclose}\isanewline
\ \ \ \ \ \ \ \ \ \ \isacommand{by}\isamarkupfalse%
\ {\isacharparenleft}simp\ only{\isacharcolon}\ Dis{\isachardot}intros{\isacharparenleft}{\isadigit{2}}{\isacharparenright}{\isacharparenright}\isanewline
\ \ \ \ \ \ \ \ \isacommand{have}\isamarkupfalse%
\ {\isachardoublequoteopen}Dis\ {\isacharparenleft}G\ \isactrlbold {\isasymrightarrow}\ H{\isacharparenright}\ {\isacharparenleft}\isactrlbold {\isasymnot}\ G{\isacharparenright}\ H\ {\isasymlongrightarrow}\ G\ \isactrlbold {\isasymrightarrow}\ H\ {\isasymin}\ S\ {\isasymlongrightarrow}\ \isactrlbold {\isasymnot}\ G\ {\isasymin}\ S\ {\isasymor}\ H\ {\isasymin}\ S{\isachardoublequoteclose}\isanewline
\ \ \ \ \ \ \ \ \ \ \isacommand{using}\isamarkupfalse%
\ Dis\ \isacommand{by}\isamarkupfalse%
\ {\isacharparenleft}iprover\ elim{\isacharcolon}\ allE{\isacharparenright}\isanewline
\ \ \ \ \ \ \ \ \isacommand{then}\isamarkupfalse%
\ \isacommand{have}\isamarkupfalse%
\ {\isachardoublequoteopen}G\ \isactrlbold {\isasymrightarrow}\ H\ {\isasymin}\ S\ {\isasymlongrightarrow}\ \isactrlbold {\isasymnot}\ G\ {\isasymin}\ S\ {\isasymor}\ H\ {\isasymin}\ S{\isachardoublequoteclose}\ \isanewline
\ \ \ \ \ \ \ \ \ \ \isacommand{using}\isamarkupfalse%
\ {\isacartoucheopen}Dis\ {\isacharparenleft}G\ \isactrlbold {\isasymrightarrow}\ H{\isacharparenright}\ {\isacharparenleft}\isactrlbold {\isasymnot}\ G{\isacharparenright}\ H{\isacartoucheclose}\ \isacommand{by}\isamarkupfalse%
\ {\isacharparenleft}rule\ mp{\isacharparenright}\isanewline
\ \ \ \ \ \ \ \ \isacommand{thus}\isamarkupfalse%
\ {\isachardoublequoteopen}\isactrlbold {\isasymnot}\ G\ {\isasymin}\ S\ {\isasymor}\ H\ {\isasymin}\ S{\isachardoublequoteclose}\isanewline
\ \ \ \ \ \ \ \ \ \ \isacommand{using}\isamarkupfalse%
\ {\isacartoucheopen}G\ \isactrlbold {\isasymrightarrow}\ H\ {\isasymin}\ S{\isacartoucheclose}\ \isacommand{by}\isamarkupfalse%
\ {\isacharparenleft}rule\ mp{\isacharparenright}\isanewline
\ \ \ \ \ \ \isacommand{qed}\isamarkupfalse%
\isanewline
\ \ \ \ \isacommand{qed}\isamarkupfalse%
\isanewline
\ \ \ \ \isacommand{have}\isamarkupfalse%
\ C{\isadigit{6}}{\isacharcolon}{\isachardoublequoteopen}{\isasymforall}G{\isachardot}\ \isactrlbold {\isasymnot}{\isacharparenleft}\isactrlbold {\isasymnot}\ G{\isacharparenright}\ {\isasymin}\ S\ {\isasymlongrightarrow}\ G\ {\isasymin}\ S{\isachardoublequoteclose}\isanewline
\ \ \ \ \isacommand{proof}\isamarkupfalse%
\ {\isacharparenleft}rule\ allI{\isacharparenright}\isanewline
\ \ \ \ \ \ \isacommand{fix}\isamarkupfalse%
\ G\isanewline
\ \ \ \ \ \ \isacommand{show}\isamarkupfalse%
\ {\isachardoublequoteopen}\isactrlbold {\isasymnot}{\isacharparenleft}\isactrlbold {\isasymnot}\ G{\isacharparenright}\ {\isasymin}\ S\ {\isasymlongrightarrow}\ G\ {\isasymin}\ S{\isachardoublequoteclose}\isanewline
\ \ \ \ \ \ \isacommand{proof}\isamarkupfalse%
\ {\isacharparenleft}rule\ impI{\isacharparenright}\isanewline
\ \ \ \ \ \ \ \ \isacommand{assume}\isamarkupfalse%
\ {\isachardoublequoteopen}\isactrlbold {\isasymnot}\ {\isacharparenleft}\isactrlbold {\isasymnot}\ G{\isacharparenright}\ {\isasymin}\ S{\isachardoublequoteclose}\ \isanewline
\ \ \ \ \ \ \ \ \isacommand{have}\isamarkupfalse%
\ {\isachardoublequoteopen}Con\ {\isacharparenleft}\isactrlbold {\isasymnot}\ {\isacharparenleft}\isactrlbold {\isasymnot}\ G{\isacharparenright}{\isacharparenright}\ G\ G{\isachardoublequoteclose}\isanewline
\ \ \ \ \ \ \ \ \ \ \isacommand{by}\isamarkupfalse%
\ {\isacharparenleft}simp\ only{\isacharcolon}\ Con{\isachardot}intros{\isacharparenleft}{\isadigit{4}}{\isacharparenright}{\isacharparenright}\isanewline
\ \ \ \ \ \ \ \ \isacommand{have}\isamarkupfalse%
\ {\isachardoublequoteopen}Con\ {\isacharparenleft}\isactrlbold {\isasymnot}{\isacharparenleft}\isactrlbold {\isasymnot}\ G{\isacharparenright}{\isacharparenright}\ G\ G\ {\isasymlongrightarrow}\ {\isacharparenleft}\isactrlbold {\isasymnot}{\isacharparenleft}\isactrlbold {\isasymnot}\ G{\isacharparenright}{\isacharparenright}\ {\isasymin}\ S\ {\isasymlongrightarrow}\ G\ {\isasymin}\ S\ {\isasymand}\ G\ {\isasymin}\ S{\isachardoublequoteclose}\isanewline
\ \ \ \ \ \ \ \ \ \ \isacommand{using}\isamarkupfalse%
\ Con\ \isacommand{by}\isamarkupfalse%
\ {\isacharparenleft}iprover\ elim{\isacharcolon}\ allE{\isacharparenright}\isanewline
\ \ \ \ \ \ \ \ \isacommand{then}\isamarkupfalse%
\ \isacommand{have}\isamarkupfalse%
\ {\isachardoublequoteopen}{\isacharparenleft}\isactrlbold {\isasymnot}{\isacharparenleft}\isactrlbold {\isasymnot}\ G{\isacharparenright}{\isacharparenright}\ {\isasymin}\ S\ {\isasymlongrightarrow}\ G\ {\isasymin}\ S\ {\isasymand}\ G\ {\isasymin}\ S{\isachardoublequoteclose}\isanewline
\ \ \ \ \ \ \ \ \ \ \isacommand{using}\isamarkupfalse%
\ {\isacartoucheopen}Con\ {\isacharparenleft}\isactrlbold {\isasymnot}\ {\isacharparenleft}\isactrlbold {\isasymnot}\ G{\isacharparenright}{\isacharparenright}\ G\ G{\isacartoucheclose}\ \isacommand{by}\isamarkupfalse%
\ {\isacharparenleft}rule\ mp{\isacharparenright}\isanewline
\ \ \ \ \ \ \ \ \isacommand{then}\isamarkupfalse%
\ \isacommand{have}\isamarkupfalse%
\ {\isachardoublequoteopen}G\ {\isasymin}\ S\ {\isasymand}\ G\ {\isasymin}\ S{\isachardoublequoteclose}\isanewline
\ \ \ \ \ \ \ \ \ \ \isacommand{using}\isamarkupfalse%
\ {\isacartoucheopen}\isactrlbold {\isasymnot}\ {\isacharparenleft}\isactrlbold {\isasymnot}\ G{\isacharparenright}\ {\isasymin}\ S{\isacartoucheclose}\ \isacommand{by}\isamarkupfalse%
\ {\isacharparenleft}rule\ mp{\isacharparenright}\isanewline
\ \ \ \ \ \ \ \ \isacommand{thus}\isamarkupfalse%
\ {\isachardoublequoteopen}G\ {\isasymin}\ S{\isachardoublequoteclose}\isanewline
\ \ \ \ \ \ \ \ \ \ \isacommand{by}\isamarkupfalse%
\ {\isacharparenleft}simp\ only{\isacharcolon}\ conj{\isacharunderscore}absorb{\isacharparenright}\isanewline
\ \ \ \ \ \ \isacommand{qed}\isamarkupfalse%
\isanewline
\ \ \ \ \isacommand{qed}\isamarkupfalse%
\isanewline
\ \ \ \ \isacommand{have}\isamarkupfalse%
\ C{\isadigit{7}}{\isacharcolon}{\isachardoublequoteopen}{\isasymforall}G\ H{\isachardot}\ \isactrlbold {\isasymnot}{\isacharparenleft}G\ \isactrlbold {\isasymand}\ H{\isacharparenright}\ {\isasymin}\ S\ {\isasymlongrightarrow}\ \isactrlbold {\isasymnot}\ G\ {\isasymin}\ S\ {\isasymor}\ \isactrlbold {\isasymnot}\ H\ {\isasymin}\ S{\isachardoublequoteclose}\isanewline
\ \ \ \ \isacommand{proof}\isamarkupfalse%
\ {\isacharparenleft}rule\ allI{\isacharparenright}{\isacharplus}\isanewline
\ \ \ \ \ \ \isacommand{fix}\isamarkupfalse%
\ G\ H\isanewline
\ \ \ \ \ \ \isacommand{show}\isamarkupfalse%
\ {\isachardoublequoteopen}\isactrlbold {\isasymnot}{\isacharparenleft}G\ \isactrlbold {\isasymand}\ H{\isacharparenright}\ {\isasymin}\ S\ {\isasymlongrightarrow}\ \isactrlbold {\isasymnot}\ G\ {\isasymin}\ S\ {\isasymor}\ \isactrlbold {\isasymnot}\ H\ {\isasymin}\ S{\isachardoublequoteclose}\isanewline
\ \ \ \ \ \ \isacommand{proof}\isamarkupfalse%
\ {\isacharparenleft}rule\ impI{\isacharparenright}\isanewline
\ \ \ \ \ \ \ \ \isacommand{assume}\isamarkupfalse%
\ {\isachardoublequoteopen}\isactrlbold {\isasymnot}{\isacharparenleft}G\ \isactrlbold {\isasymand}\ H{\isacharparenright}\ {\isasymin}\ S{\isachardoublequoteclose}\isanewline
\ \ \ \ \ \ \ \ \isacommand{have}\isamarkupfalse%
\ {\isachardoublequoteopen}Dis\ {\isacharparenleft}\isactrlbold {\isasymnot}{\isacharparenleft}G\ \isactrlbold {\isasymand}\ H{\isacharparenright}{\isacharparenright}\ {\isacharparenleft}\isactrlbold {\isasymnot}\ G{\isacharparenright}\ {\isacharparenleft}\isactrlbold {\isasymnot}\ H{\isacharparenright}{\isachardoublequoteclose}\isanewline
\ \ \ \ \ \ \ \ \ \ \isacommand{by}\isamarkupfalse%
\ {\isacharparenleft}simp\ only{\isacharcolon}\ Dis{\isachardot}intros{\isacharparenleft}{\isadigit{3}}{\isacharparenright}{\isacharparenright}\isanewline
\ \ \ \ \ \ \ \ \isacommand{have}\isamarkupfalse%
\ {\isachardoublequoteopen}Dis\ {\isacharparenleft}\isactrlbold {\isasymnot}{\isacharparenleft}G\ \isactrlbold {\isasymand}\ H{\isacharparenright}{\isacharparenright}\ {\isacharparenleft}\isactrlbold {\isasymnot}\ G{\isacharparenright}\ {\isacharparenleft}\isactrlbold {\isasymnot}\ H{\isacharparenright}\ {\isasymlongrightarrow}\ \isactrlbold {\isasymnot}{\isacharparenleft}G\ \isactrlbold {\isasymand}\ H{\isacharparenright}\ {\isasymin}\ S\ {\isasymlongrightarrow}\ \isactrlbold {\isasymnot}\ G\ {\isasymin}\ S\ {\isasymor}\ \isactrlbold {\isasymnot}\ H\ {\isasymin}\ S{\isachardoublequoteclose}\isanewline
\ \ \ \ \ \ \ \ \ \ \isacommand{using}\isamarkupfalse%
\ Dis\ \isacommand{by}\isamarkupfalse%
\ {\isacharparenleft}iprover\ elim{\isacharcolon}\ allE{\isacharparenright}\isanewline
\ \ \ \ \ \ \ \ \isacommand{then}\isamarkupfalse%
\ \isacommand{have}\isamarkupfalse%
\ {\isachardoublequoteopen}\isactrlbold {\isasymnot}{\isacharparenleft}G\ \isactrlbold {\isasymand}\ H{\isacharparenright}\ {\isasymin}\ S\ {\isasymlongrightarrow}\ \isactrlbold {\isasymnot}\ G\ {\isasymin}\ S\ {\isasymor}\ \isactrlbold {\isasymnot}\ H\ {\isasymin}\ S{\isachardoublequoteclose}\isanewline
\ \ \ \ \ \ \ \ \ \ \isacommand{using}\isamarkupfalse%
\ {\isacartoucheopen}Dis\ {\isacharparenleft}\isactrlbold {\isasymnot}{\isacharparenleft}G\ \isactrlbold {\isasymand}\ H{\isacharparenright}{\isacharparenright}\ {\isacharparenleft}\isactrlbold {\isasymnot}\ G{\isacharparenright}\ {\isacharparenleft}\isactrlbold {\isasymnot}\ H{\isacharparenright}{\isacartoucheclose}\ \isacommand{by}\isamarkupfalse%
\ {\isacharparenleft}rule\ mp{\isacharparenright}\isanewline
\ \ \ \ \ \ \ \ \isacommand{thus}\isamarkupfalse%
\ {\isachardoublequoteopen}\isactrlbold {\isasymnot}\ G\ {\isasymin}\ S\ {\isasymor}\ \isactrlbold {\isasymnot}\ H\ {\isasymin}\ S{\isachardoublequoteclose}\isanewline
\ \ \ \ \ \ \ \ \ \ \isacommand{using}\isamarkupfalse%
\ {\isacartoucheopen}\isactrlbold {\isasymnot}{\isacharparenleft}G\ \isactrlbold {\isasymand}\ H{\isacharparenright}\ {\isasymin}\ S{\isacartoucheclose}\ \isacommand{by}\isamarkupfalse%
\ {\isacharparenleft}rule\ mp{\isacharparenright}\isanewline
\ \ \ \ \ \ \isacommand{qed}\isamarkupfalse%
\isanewline
\ \ \ \ \isacommand{qed}\isamarkupfalse%
\isanewline
\ \ \ \ \isacommand{have}\isamarkupfalse%
\ C{\isadigit{8}}{\isacharcolon}{\isachardoublequoteopen}{\isasymforall}G\ H{\isachardot}\ \isactrlbold {\isasymnot}{\isacharparenleft}G\ \isactrlbold {\isasymor}\ H{\isacharparenright}\ {\isasymin}\ S\ {\isasymlongrightarrow}\ \isactrlbold {\isasymnot}\ G\ {\isasymin}\ S\ {\isasymand}\ \isactrlbold {\isasymnot}\ H\ {\isasymin}\ S{\isachardoublequoteclose}\isanewline
\ \ \ \ \isacommand{proof}\isamarkupfalse%
\ {\isacharparenleft}rule\ allI{\isacharparenright}{\isacharplus}\isanewline
\ \ \ \ \ \ \isacommand{fix}\isamarkupfalse%
\ G\ H\isanewline
\ \ \ \ \ \ \isacommand{show}\isamarkupfalse%
\ {\isachardoublequoteopen}\isactrlbold {\isasymnot}{\isacharparenleft}G\ \isactrlbold {\isasymor}\ H{\isacharparenright}\ {\isasymin}\ S\ {\isasymlongrightarrow}\ \isactrlbold {\isasymnot}\ G\ {\isasymin}\ S\ {\isasymand}\ \isactrlbold {\isasymnot}\ H\ {\isasymin}\ S{\isachardoublequoteclose}\isanewline
\ \ \ \ \ \ \isacommand{proof}\isamarkupfalse%
\ {\isacharparenleft}rule\ impI{\isacharparenright}\isanewline
\ \ \ \ \ \ \ \ \isacommand{assume}\isamarkupfalse%
\ {\isachardoublequoteopen}\isactrlbold {\isasymnot}{\isacharparenleft}G\ \isactrlbold {\isasymor}\ H{\isacharparenright}\ {\isasymin}\ S{\isachardoublequoteclose}\isanewline
\ \ \ \ \ \ \ \ \isacommand{have}\isamarkupfalse%
\ {\isachardoublequoteopen}Con\ {\isacharparenleft}\isactrlbold {\isasymnot}{\isacharparenleft}G\ \isactrlbold {\isasymor}\ H{\isacharparenright}{\isacharparenright}\ {\isacharparenleft}\isactrlbold {\isasymnot}\ G{\isacharparenright}\ {\isacharparenleft}\isactrlbold {\isasymnot}\ H{\isacharparenright}{\isachardoublequoteclose}\isanewline
\ \ \ \ \ \ \ \ \ \ \isacommand{by}\isamarkupfalse%
\ {\isacharparenleft}simp\ only{\isacharcolon}\ Con{\isachardot}intros{\isacharparenleft}{\isadigit{2}}{\isacharparenright}{\isacharparenright}\isanewline
\ \ \ \ \ \ \ \ \isacommand{have}\isamarkupfalse%
\ {\isachardoublequoteopen}Con\ {\isacharparenleft}\isactrlbold {\isasymnot}{\isacharparenleft}G\ \isactrlbold {\isasymor}\ H{\isacharparenright}{\isacharparenright}\ {\isacharparenleft}\isactrlbold {\isasymnot}\ G{\isacharparenright}\ {\isacharparenleft}\isactrlbold {\isasymnot}\ H{\isacharparenright}\ {\isasymlongrightarrow}\ \isactrlbold {\isasymnot}{\isacharparenleft}G\ \isactrlbold {\isasymor}\ H{\isacharparenright}\ {\isasymin}\ S\ {\isasymlongrightarrow}\ \isactrlbold {\isasymnot}\ G\ {\isasymin}\ S\ {\isasymand}\ \isactrlbold {\isasymnot}\ H\ {\isasymin}\ S{\isachardoublequoteclose}\isanewline
\ \ \ \ \ \ \ \ \ \ \isacommand{using}\isamarkupfalse%
\ Con\ \isacommand{by}\isamarkupfalse%
\ {\isacharparenleft}iprover\ elim{\isacharcolon}\ allE{\isacharparenright}\isanewline
\ \ \ \ \ \ \ \ \isacommand{then}\isamarkupfalse%
\ \isacommand{have}\isamarkupfalse%
\ {\isachardoublequoteopen}\isactrlbold {\isasymnot}{\isacharparenleft}G\ \isactrlbold {\isasymor}\ H{\isacharparenright}\ {\isasymin}\ S\ {\isasymlongrightarrow}\ \isactrlbold {\isasymnot}\ G\ {\isasymin}\ S\ {\isasymand}\ \isactrlbold {\isasymnot}\ H\ {\isasymin}\ S{\isachardoublequoteclose}\isanewline
\ \ \ \ \ \ \ \ \ \ \isacommand{using}\isamarkupfalse%
\ {\isacartoucheopen}Con\ {\isacharparenleft}\isactrlbold {\isasymnot}{\isacharparenleft}G\ \isactrlbold {\isasymor}\ H{\isacharparenright}{\isacharparenright}\ {\isacharparenleft}\isactrlbold {\isasymnot}\ G{\isacharparenright}\ {\isacharparenleft}\isactrlbold {\isasymnot}\ H{\isacharparenright}{\isacartoucheclose}\ \isacommand{by}\isamarkupfalse%
\ {\isacharparenleft}rule\ mp{\isacharparenright}\isanewline
\ \ \ \ \ \ \ \ \isacommand{thus}\isamarkupfalse%
\ {\isachardoublequoteopen}\isactrlbold {\isasymnot}\ G\ {\isasymin}\ S\ {\isasymand}\ \isactrlbold {\isasymnot}\ H\ {\isasymin}\ S{\isachardoublequoteclose}\isanewline
\ \ \ \ \ \ \ \ \ \ \isacommand{using}\isamarkupfalse%
\ {\isacartoucheopen}\isactrlbold {\isasymnot}{\isacharparenleft}G\ \isactrlbold {\isasymor}\ H{\isacharparenright}\ {\isasymin}\ S{\isacartoucheclose}\ \isacommand{by}\isamarkupfalse%
\ {\isacharparenleft}rule\ mp{\isacharparenright}\isanewline
\ \ \ \ \ \ \isacommand{qed}\isamarkupfalse%
\isanewline
\ \ \ \ \isacommand{qed}\isamarkupfalse%
\isanewline
\ \ \ \ \isacommand{have}\isamarkupfalse%
\ C{\isadigit{9}}{\isacharcolon}{\isachardoublequoteopen}{\isasymforall}G\ H{\isachardot}\ \isactrlbold {\isasymnot}{\isacharparenleft}G\ \isactrlbold {\isasymrightarrow}\ H{\isacharparenright}\ {\isasymin}\ S\ {\isasymlongrightarrow}\ G\ {\isasymin}\ S\ {\isasymand}\ \isactrlbold {\isasymnot}\ H\ {\isasymin}\ S{\isachardoublequoteclose}\isanewline
\ \ \ \ \isacommand{proof}\isamarkupfalse%
\ {\isacharparenleft}rule\ allI{\isacharparenright}{\isacharplus}\isanewline
\ \ \ \ \ \ \isacommand{fix}\isamarkupfalse%
\ G\ H\isanewline
\ \ \ \ \ \ \isacommand{show}\isamarkupfalse%
\ {\isachardoublequoteopen}\isactrlbold {\isasymnot}{\isacharparenleft}G\ \isactrlbold {\isasymrightarrow}\ H{\isacharparenright}\ {\isasymin}\ S\ {\isasymlongrightarrow}\ G\ {\isasymin}\ S\ {\isasymand}\ \isactrlbold {\isasymnot}\ H\ {\isasymin}\ S{\isachardoublequoteclose}\isanewline
\ \ \ \ \ \ \isacommand{proof}\isamarkupfalse%
\ {\isacharparenleft}rule\ impI{\isacharparenright}\isanewline
\ \ \ \ \ \ \ \ \isacommand{assume}\isamarkupfalse%
\ {\isachardoublequoteopen}\isactrlbold {\isasymnot}{\isacharparenleft}G\ \isactrlbold {\isasymrightarrow}\ H{\isacharparenright}\ {\isasymin}\ S{\isachardoublequoteclose}\isanewline
\ \ \ \ \ \ \ \ \isacommand{have}\isamarkupfalse%
\ {\isachardoublequoteopen}Con\ {\isacharparenleft}\isactrlbold {\isasymnot}{\isacharparenleft}G\ \isactrlbold {\isasymrightarrow}\ H{\isacharparenright}{\isacharparenright}\ G\ {\isacharparenleft}\isactrlbold {\isasymnot}\ H{\isacharparenright}{\isachardoublequoteclose}\isanewline
\ \ \ \ \ \ \ \ \ \ \isacommand{by}\isamarkupfalse%
\ {\isacharparenleft}simp\ only{\isacharcolon}\ Con{\isachardot}intros{\isacharparenleft}{\isadigit{3}}{\isacharparenright}{\isacharparenright}\isanewline
\ \ \ \ \ \ \ \ \isacommand{have}\isamarkupfalse%
\ {\isachardoublequoteopen}Con\ {\isacharparenleft}\isactrlbold {\isasymnot}{\isacharparenleft}G\ \isactrlbold {\isasymrightarrow}\ H{\isacharparenright}{\isacharparenright}\ G\ {\isacharparenleft}\isactrlbold {\isasymnot}\ H{\isacharparenright}\ {\isasymlongrightarrow}\ \isactrlbold {\isasymnot}{\isacharparenleft}G\ \isactrlbold {\isasymrightarrow}\ H{\isacharparenright}\ {\isasymin}\ S\ {\isasymlongrightarrow}\ G\ {\isasymin}\ S\ {\isasymand}\ \isactrlbold {\isasymnot}\ H\ {\isasymin}\ S{\isachardoublequoteclose}\isanewline
\ \ \ \ \ \ \ \ \ \ \isacommand{using}\isamarkupfalse%
\ Con\ \isacommand{by}\isamarkupfalse%
\ {\isacharparenleft}iprover\ elim{\isacharcolon}\ allE{\isacharparenright}\isanewline
\ \ \ \ \ \ \ \ \isacommand{then}\isamarkupfalse%
\ \isacommand{have}\isamarkupfalse%
\ {\isachardoublequoteopen}\isactrlbold {\isasymnot}{\isacharparenleft}G\ \isactrlbold {\isasymrightarrow}\ H{\isacharparenright}\ {\isasymin}\ S\ {\isasymlongrightarrow}\ G\ {\isasymin}\ S\ {\isasymand}\ \isactrlbold {\isasymnot}\ H\ {\isasymin}\ S{\isachardoublequoteclose}\isanewline
\ \ \ \ \ \ \ \ \ \ \isacommand{using}\isamarkupfalse%
\ {\isacartoucheopen}Con\ {\isacharparenleft}\isactrlbold {\isasymnot}{\isacharparenleft}G\ \isactrlbold {\isasymrightarrow}\ H{\isacharparenright}{\isacharparenright}\ G\ {\isacharparenleft}\isactrlbold {\isasymnot}\ H{\isacharparenright}{\isacartoucheclose}\ \isacommand{by}\isamarkupfalse%
\ {\isacharparenleft}rule\ mp{\isacharparenright}\isanewline
\ \ \ \ \ \ \ \ \isacommand{thus}\isamarkupfalse%
\ {\isachardoublequoteopen}G\ {\isasymin}\ S\ {\isasymand}\ \isactrlbold {\isasymnot}\ H\ {\isasymin}\ S{\isachardoublequoteclose}\isanewline
\ \ \ \ \ \ \ \ \ \ \isacommand{using}\isamarkupfalse%
\ {\isacartoucheopen}\isactrlbold {\isasymnot}{\isacharparenleft}G\ \isactrlbold {\isasymrightarrow}\ H{\isacharparenright}\ {\isasymin}\ S{\isacartoucheclose}\ \isacommand{by}\isamarkupfalse%
\ {\isacharparenleft}rule\ mp{\isacharparenright}\isanewline
\ \ \ \ \ \ \isacommand{qed}\isamarkupfalse%
\isanewline
\ \ \ \ \isacommand{qed}\isamarkupfalse%
\isanewline
\ \ \ \ \isacommand{have}\isamarkupfalse%
\ A{\isacharcolon}{\isachardoublequoteopen}{\isasymbottom}\ {\isasymnotin}\ S\isanewline
\ \ \ \ {\isasymand}\ {\isacharparenleft}{\isasymforall}k{\isachardot}\ Atom\ k\ {\isasymin}\ S\ {\isasymlongrightarrow}\ \isactrlbold {\isasymnot}\ {\isacharparenleft}Atom\ k{\isacharparenright}\ {\isasymin}\ S\ {\isasymlongrightarrow}\ False{\isacharparenright}\isanewline
\ \ \ \ {\isasymand}\ {\isacharparenleft}{\isasymforall}G\ H{\isachardot}\ G\ \isactrlbold {\isasymand}\ H\ {\isasymin}\ S\ {\isasymlongrightarrow}\ G\ {\isasymin}\ S\ {\isasymand}\ H\ {\isasymin}\ S{\isacharparenright}\isanewline
\ \ \ \ {\isasymand}\ {\isacharparenleft}{\isasymforall}G\ H{\isachardot}\ G\ \isactrlbold {\isasymor}\ H\ {\isasymin}\ S\ {\isasymlongrightarrow}\ G\ {\isasymin}\ S\ {\isasymor}\ H\ {\isasymin}\ S{\isacharparenright}\isanewline
\ \ \ \ {\isasymand}\ {\isacharparenleft}{\isasymforall}G\ H{\isachardot}\ G\ \isactrlbold {\isasymrightarrow}\ H\ {\isasymin}\ S\ {\isasymlongrightarrow}\ \isactrlbold {\isasymnot}G\ {\isasymin}\ S\ {\isasymor}\ H\ {\isasymin}\ S{\isacharparenright}{\isachardoublequoteclose}\isanewline
\ \ \ \ \ \ \isacommand{using}\isamarkupfalse%
\ C{\isadigit{1}}\ C{\isadigit{2}}\ C{\isadigit{3}}\ C{\isadigit{4}}\ C{\isadigit{5}}\ \isacommand{by}\isamarkupfalse%
\ {\isacharparenleft}iprover\ intro{\isacharcolon}\ conjI{\isacharparenright}\isanewline
\ \ \ \ \isacommand{have}\isamarkupfalse%
\ B{\isacharcolon}{\isachardoublequoteopen}{\isacharparenleft}{\isasymforall}G{\isachardot}\ \isactrlbold {\isasymnot}\ {\isacharparenleft}\isactrlbold {\isasymnot}G{\isacharparenright}\ {\isasymin}\ S\ {\isasymlongrightarrow}\ G\ {\isasymin}\ S{\isacharparenright}\isanewline
\ \ \ \ {\isasymand}\ {\isacharparenleft}{\isasymforall}G\ H{\isachardot}\ \isactrlbold {\isasymnot}{\isacharparenleft}G\ \isactrlbold {\isasymand}\ H{\isacharparenright}\ {\isasymin}\ S\ {\isasymlongrightarrow}\ \isactrlbold {\isasymnot}\ G\ {\isasymin}\ S\ {\isasymor}\ \isactrlbold {\isasymnot}\ H\ {\isasymin}\ S{\isacharparenright}\isanewline
\ \ \ \ {\isasymand}\ {\isacharparenleft}{\isasymforall}G\ H{\isachardot}\ \isactrlbold {\isasymnot}{\isacharparenleft}G\ \isactrlbold {\isasymor}\ H{\isacharparenright}\ {\isasymin}\ S\ {\isasymlongrightarrow}\ \isactrlbold {\isasymnot}\ G\ {\isasymin}\ S\ {\isasymand}\ \isactrlbold {\isasymnot}\ H\ {\isasymin}\ S{\isacharparenright}\isanewline
\ \ \ \ {\isasymand}\ {\isacharparenleft}{\isasymforall}G\ H{\isachardot}\ \isactrlbold {\isasymnot}{\isacharparenleft}G\ \isactrlbold {\isasymrightarrow}\ H{\isacharparenright}\ {\isasymin}\ S\ {\isasymlongrightarrow}\ G\ {\isasymin}\ S\ {\isasymand}\ \isactrlbold {\isasymnot}\ H\ {\isasymin}\ S{\isacharparenright}{\isachardoublequoteclose}\isanewline
\ \ \ \ \ \ \isacommand{using}\isamarkupfalse%
\ C{\isadigit{6}}\ C{\isadigit{7}}\ C{\isadigit{8}}\ C{\isadigit{9}}\ \isacommand{by}\isamarkupfalse%
\ {\isacharparenleft}iprover\ intro{\isacharcolon}\ conjI{\isacharparenright}\isanewline
\ \ \ \ \isacommand{have}\isamarkupfalse%
\ {\isachardoublequoteopen}{\isacharparenleft}{\isasymbottom}\ {\isasymnotin}\ S\isanewline
\ \ \ \ {\isasymand}\ {\isacharparenleft}{\isasymforall}k{\isachardot}\ Atom\ k\ {\isasymin}\ S\ {\isasymlongrightarrow}\ \isactrlbold {\isasymnot}\ {\isacharparenleft}Atom\ k{\isacharparenright}\ {\isasymin}\ S\ {\isasymlongrightarrow}\ False{\isacharparenright}\isanewline
\ \ \ \ {\isasymand}\ {\isacharparenleft}{\isasymforall}G\ H{\isachardot}\ G\ \isactrlbold {\isasymand}\ H\ {\isasymin}\ S\ {\isasymlongrightarrow}\ G\ {\isasymin}\ S\ {\isasymand}\ H\ {\isasymin}\ S{\isacharparenright}\isanewline
\ \ \ \ {\isasymand}\ {\isacharparenleft}{\isasymforall}G\ H{\isachardot}\ G\ \isactrlbold {\isasymor}\ H\ {\isasymin}\ S\ {\isasymlongrightarrow}\ G\ {\isasymin}\ S\ {\isasymor}\ H\ {\isasymin}\ S{\isacharparenright}\isanewline
\ \ \ \ {\isasymand}\ {\isacharparenleft}{\isasymforall}G\ H{\isachardot}\ G\ \isactrlbold {\isasymrightarrow}\ H\ {\isasymin}\ S\ {\isasymlongrightarrow}\ \isactrlbold {\isasymnot}G\ {\isasymin}\ S\ {\isasymor}\ H\ {\isasymin}\ S{\isacharparenright}{\isacharparenright}\isanewline
\ \ \ \ {\isasymand}\ {\isacharparenleft}{\isacharparenleft}{\isasymforall}G{\isachardot}\ \isactrlbold {\isasymnot}\ {\isacharparenleft}\isactrlbold {\isasymnot}G{\isacharparenright}\ {\isasymin}\ S\ {\isasymlongrightarrow}\ G\ {\isasymin}\ S{\isacharparenright}\isanewline
\ \ \ \ {\isasymand}\ {\isacharparenleft}{\isasymforall}G\ H{\isachardot}\ \isactrlbold {\isasymnot}{\isacharparenleft}G\ \isactrlbold {\isasymand}\ H{\isacharparenright}\ {\isasymin}\ S\ {\isasymlongrightarrow}\ \isactrlbold {\isasymnot}\ G\ {\isasymin}\ S\ {\isasymor}\ \isactrlbold {\isasymnot}\ H\ {\isasymin}\ S{\isacharparenright}\isanewline
\ \ \ \ {\isasymand}\ {\isacharparenleft}{\isasymforall}G\ H{\isachardot}\ \isactrlbold {\isasymnot}{\isacharparenleft}G\ \isactrlbold {\isasymor}\ H{\isacharparenright}\ {\isasymin}\ S\ {\isasymlongrightarrow}\ \isactrlbold {\isasymnot}\ G\ {\isasymin}\ S\ {\isasymand}\ \isactrlbold {\isasymnot}\ H\ {\isasymin}\ S{\isacharparenright}\isanewline
\ \ \ \ {\isasymand}\ {\isacharparenleft}{\isasymforall}G\ H{\isachardot}\ \isactrlbold {\isasymnot}{\isacharparenleft}G\ \isactrlbold {\isasymrightarrow}\ H{\isacharparenright}\ {\isasymin}\ S\ {\isasymlongrightarrow}\ G\ {\isasymin}\ S\ {\isasymand}\ \isactrlbold {\isasymnot}\ H\ {\isasymin}\ S{\isacharparenright}{\isacharparenright}{\isachardoublequoteclose}\isanewline
\ \ \ \ \ \ \isacommand{using}\isamarkupfalse%
\ A\ B\ \isacommand{by}\isamarkupfalse%
\ {\isacharparenleft}rule\ conjI{\isacharparenright}\isanewline
\ \ \ \ \isacommand{thus}\isamarkupfalse%
\ {\isachardoublequoteopen}{\isasymbottom}\ {\isasymnotin}\ S\isanewline
\ \ \ \ {\isasymand}\ {\isacharparenleft}{\isasymforall}k{\isachardot}\ Atom\ k\ {\isasymin}\ S\ {\isasymlongrightarrow}\ \isactrlbold {\isasymnot}\ {\isacharparenleft}Atom\ k{\isacharparenright}\ {\isasymin}\ S\ {\isasymlongrightarrow}\ False{\isacharparenright}\isanewline
\ \ \ \ {\isasymand}\ {\isacharparenleft}{\isasymforall}G\ H{\isachardot}\ G\ \isactrlbold {\isasymand}\ H\ {\isasymin}\ S\ {\isasymlongrightarrow}\ G\ {\isasymin}\ S\ {\isasymand}\ H\ {\isasymin}\ S{\isacharparenright}\isanewline
\ \ \ \ {\isasymand}\ {\isacharparenleft}{\isasymforall}G\ H{\isachardot}\ G\ \isactrlbold {\isasymor}\ H\ {\isasymin}\ S\ {\isasymlongrightarrow}\ G\ {\isasymin}\ S\ {\isasymor}\ H\ {\isasymin}\ S{\isacharparenright}\isanewline
\ \ \ \ {\isasymand}\ {\isacharparenleft}{\isasymforall}G\ H{\isachardot}\ G\ \isactrlbold {\isasymrightarrow}\ H\ {\isasymin}\ S\ {\isasymlongrightarrow}\ \isactrlbold {\isasymnot}G\ {\isasymin}\ S\ {\isasymor}\ H\ {\isasymin}\ S{\isacharparenright}\isanewline
\ \ \ \ {\isasymand}\ {\isacharparenleft}{\isasymforall}G{\isachardot}\ \isactrlbold {\isasymnot}\ {\isacharparenleft}\isactrlbold {\isasymnot}G{\isacharparenright}\ {\isasymin}\ S\ {\isasymlongrightarrow}\ G\ {\isasymin}\ S{\isacharparenright}\isanewline
\ \ \ \ {\isasymand}\ {\isacharparenleft}{\isasymforall}G\ H{\isachardot}\ \isactrlbold {\isasymnot}{\isacharparenleft}G\ \isactrlbold {\isasymand}\ H{\isacharparenright}\ {\isasymin}\ S\ {\isasymlongrightarrow}\ \isactrlbold {\isasymnot}\ G\ {\isasymin}\ S\ {\isasymor}\ \isactrlbold {\isasymnot}\ H\ {\isasymin}\ S{\isacharparenright}\isanewline
\ \ \ \ {\isasymand}\ {\isacharparenleft}{\isasymforall}G\ H{\isachardot}\ \isactrlbold {\isasymnot}{\isacharparenleft}G\ \isactrlbold {\isasymor}\ H{\isacharparenright}\ {\isasymin}\ S\ {\isasymlongrightarrow}\ \isactrlbold {\isasymnot}\ G\ {\isasymin}\ S\ {\isasymand}\ \isactrlbold {\isasymnot}\ H\ {\isasymin}\ S{\isacharparenright}\isanewline
\ \ \ \ {\isasymand}\ {\isacharparenleft}{\isasymforall}G\ H{\isachardot}\ \isactrlbold {\isasymnot}{\isacharparenleft}G\ \isactrlbold {\isasymrightarrow}\ H{\isacharparenright}\ {\isasymin}\ S\ {\isasymlongrightarrow}\ G\ {\isasymin}\ S\ {\isasymand}\ \isactrlbold {\isasymnot}\ H\ {\isasymin}\ S{\isacharparenright}{\isachardoublequoteclose}\ \isanewline
\ \ \ \ \ \ \isacommand{by}\isamarkupfalse%
\ {\isacharparenleft}iprover\ intro{\isacharcolon}\ conj{\isacharunderscore}assoc{\isacharparenright}\isanewline
\ \ \isacommand{qed}\isamarkupfalse%
\isanewline
\ \ \isacommand{thus}\isamarkupfalse%
\ {\isachardoublequoteopen}Hintikka\ S{\isachardoublequoteclose}\isanewline
\ \ \ \ \isacommand{unfolding}\isamarkupfalse%
\ Hintikka{\isacharunderscore}def\ \isacommand{by}\isamarkupfalse%
\ this\isanewline
\isacommand{qed}\isamarkupfalse%
%
\endisatagproof
{\isafoldproof}%
%
\isadelimproof
%
\endisadelimproof
%
\begin{isamarkuptext}%
En conclusión, el lema completo se demuestra detalladamente en Isabelle/HOL como sigue.%
\end{isamarkuptext}\isamarkuptrue%
\isacommand{lemma}\isamarkupfalse%
\ {\isachardoublequoteopen}Hintikka\ S\ {\isacharequal}\ {\isacharparenleft}{\isasymbottom}\ {\isasymnotin}\ S\isanewline
{\isasymand}\ {\isacharparenleft}{\isasymforall}k{\isachardot}\ Atom\ k\ {\isasymin}\ S\ {\isasymlongrightarrow}\ \isactrlbold {\isasymnot}\ {\isacharparenleft}Atom\ k{\isacharparenright}\ {\isasymin}\ S\ {\isasymlongrightarrow}\ False{\isacharparenright}\isanewline
{\isasymand}\ {\isacharparenleft}{\isasymforall}F\ G\ H{\isachardot}\ Con\ F\ G\ H\ {\isasymlongrightarrow}\ F\ {\isasymin}\ S\ {\isasymlongrightarrow}\ G\ {\isasymin}\ S\ {\isasymand}\ H\ {\isasymin}\ S{\isacharparenright}\isanewline
{\isasymand}\ {\isacharparenleft}{\isasymforall}F\ G\ H{\isachardot}\ Dis\ F\ G\ H\ {\isasymlongrightarrow}\ F\ {\isasymin}\ S\ {\isasymlongrightarrow}\ G\ {\isasymin}\ S\ {\isasymor}\ H\ {\isasymin}\ S{\isacharparenright}{\isacharparenright}{\isachardoublequoteclose}\ \ \isanewline
%
\isadelimproof
%
\endisadelimproof
%
\isatagproof
\isacommand{proof}\isamarkupfalse%
\ {\isacharparenleft}rule\ iffI{\isacharparenright}\isanewline
\ \ \isacommand{assume}\isamarkupfalse%
\ {\isachardoublequoteopen}Hintikka\ S{\isachardoublequoteclose}\isanewline
\ \ \isacommand{thus}\isamarkupfalse%
\ {\isachardoublequoteopen}{\isacharparenleft}{\isasymbottom}\ {\isasymnotin}\ S\isanewline
\ \ {\isasymand}\ {\isacharparenleft}{\isasymforall}k{\isachardot}\ Atom\ k\ {\isasymin}\ S\ {\isasymlongrightarrow}\ \isactrlbold {\isasymnot}\ {\isacharparenleft}Atom\ k{\isacharparenright}\ {\isasymin}\ S\ {\isasymlongrightarrow}\ False{\isacharparenright}\isanewline
\ \ {\isasymand}\ {\isacharparenleft}{\isasymforall}F\ G\ H{\isachardot}\ Con\ F\ G\ H\ {\isasymlongrightarrow}\ F\ {\isasymin}\ S\ {\isasymlongrightarrow}\ G\ {\isasymin}\ S\ {\isasymand}\ H\ {\isasymin}\ S{\isacharparenright}\isanewline
\ \ {\isasymand}\ {\isacharparenleft}{\isasymforall}F\ G\ H{\isachardot}\ Dis\ F\ G\ H\ {\isasymlongrightarrow}\ F\ {\isasymin}\ S\ {\isasymlongrightarrow}\ G\ {\isasymin}\ S\ {\isasymor}\ H\ {\isasymin}\ S{\isacharparenright}{\isacharparenright}{\isachardoublequoteclose}\isanewline
\ \ \ \ \isacommand{by}\isamarkupfalse%
\ {\isacharparenleft}rule\ Hintikka{\isacharunderscore}alt{\isadigit{1}}{\isacharparenright}\isanewline
\isacommand{next}\isamarkupfalse%
\isanewline
\ \ \isacommand{assume}\isamarkupfalse%
\ {\isachardoublequoteopen}{\isacharparenleft}{\isasymbottom}\ {\isasymnotin}\ S\isanewline
\ \ {\isasymand}\ {\isacharparenleft}{\isasymforall}k{\isachardot}\ Atom\ k\ {\isasymin}\ S\ {\isasymlongrightarrow}\ \isactrlbold {\isasymnot}\ {\isacharparenleft}Atom\ k{\isacharparenright}\ {\isasymin}\ S\ {\isasymlongrightarrow}\ False{\isacharparenright}\isanewline
\ \ {\isasymand}\ {\isacharparenleft}{\isasymforall}F\ G\ H{\isachardot}\ Con\ F\ G\ H\ {\isasymlongrightarrow}\ F\ {\isasymin}\ S\ {\isasymlongrightarrow}\ G\ {\isasymin}\ S\ {\isasymand}\ H\ {\isasymin}\ S{\isacharparenright}\isanewline
\ \ {\isasymand}\ {\isacharparenleft}{\isasymforall}F\ G\ H{\isachardot}\ Dis\ F\ G\ H\ {\isasymlongrightarrow}\ F\ {\isasymin}\ S\ {\isasymlongrightarrow}\ G\ {\isasymin}\ S\ {\isasymor}\ H\ {\isasymin}\ S{\isacharparenright}{\isacharparenright}{\isachardoublequoteclose}\isanewline
\ \ \isacommand{thus}\isamarkupfalse%
\ {\isachardoublequoteopen}Hintikka\ S{\isachardoublequoteclose}\isanewline
\ \ \ \ \isacommand{by}\isamarkupfalse%
\ {\isacharparenleft}rule\ Hintikka{\isacharunderscore}alt{\isadigit{2}}{\isacharparenright}\isanewline
\isacommand{qed}\isamarkupfalse%
%
\endisatagproof
{\isafoldproof}%
%
\isadelimproof
%
\endisadelimproof
%
\begin{isamarkuptext}%
Por último, veamos su demostración automática.%
\end{isamarkuptext}\isamarkuptrue%
\isacommand{lemma}\isamarkupfalse%
\ Hintikka{\isacharunderscore}alt{\isacharcolon}\ {\isachardoublequoteopen}Hintikka\ S\ {\isacharequal}\ {\isacharparenleft}{\isasymbottom}\ {\isasymnotin}\ S\isanewline
{\isasymand}\ {\isacharparenleft}{\isasymforall}k{\isachardot}\ Atom\ k\ {\isasymin}\ S\ {\isasymlongrightarrow}\ \isactrlbold {\isasymnot}\ {\isacharparenleft}Atom\ k{\isacharparenright}\ {\isasymin}\ S\ {\isasymlongrightarrow}\ False{\isacharparenright}\isanewline
{\isasymand}\ {\isacharparenleft}{\isasymforall}F\ G\ H{\isachardot}\ Con\ F\ G\ H\ {\isasymlongrightarrow}\ F\ {\isasymin}\ S\ {\isasymlongrightarrow}\ G\ {\isasymin}\ S\ {\isasymand}\ H\ {\isasymin}\ S{\isacharparenright}\isanewline
{\isasymand}\ {\isacharparenleft}{\isasymforall}F\ G\ H{\isachardot}\ Dis\ F\ G\ H\ {\isasymlongrightarrow}\ F\ {\isasymin}\ S\ {\isasymlongrightarrow}\ G\ {\isasymin}\ S\ {\isasymor}\ H\ {\isasymin}\ S{\isacharparenright}{\isacharparenright}{\isachardoublequoteclose}\ \ \isanewline
%
\isadelimproof
\ \ %
\endisadelimproof
%
\isatagproof
\isacommand{apply}\isamarkupfalse%
{\isacharparenleft}simp\ add{\isacharcolon}\ Hintikka{\isacharunderscore}def\ con{\isacharunderscore}dis{\isacharunderscore}simps{\isacharparenright}\isanewline
\ \ \isacommand{apply}\isamarkupfalse%
{\isacharparenleft}rule\ iffI{\isacharparenright}\isanewline
\ \ \ \isacommand{subgoal}\isamarkupfalse%
\ \isacommand{by}\isamarkupfalse%
\ blast\isanewline
\ \ \isacommand{subgoal}\isamarkupfalse%
\ \isacommand{by}\isamarkupfalse%
\ safe\ metis{\isacharplus}\isanewline
\ \ \isacommand{done}\isamarkupfalse%
\isanewline
%
\endisatagproof
{\isafoldproof}%
%
\isadelimproof
%
\endisadelimproof
%
\isadelimtheory
%
\endisadelimtheory
%
\isatagtheory
%
\endisatagtheory
{\isafoldtheory}%
%
\isadelimtheory
%
\endisadelimtheory
%
\end{isabellebody}%
\endinput
%:%file=~/TFM/TFM/Notunif.thy%:%
%:%19=11%:%
%:%20=12%:%
%:%21=13%:%
%:%25=17%:%
%:%26=18%:%
%:%27=19%:%
%:%28=20%:%
%:%29=21%:%
%:%30=22%:%
%:%31=23%:%
%:%32=24%:%
%:%33=25%:%
%:%34=26%:%
%:%35=27%:%
%:%36=28%:%
%:%37=29%:%
%:%38=30%:%
%:%39=31%:%
%:%40=32%:%
%:%41=33%:%
%:%42=34%:%
%:%43=35%:%
%:%44=36%:%
%:%46=38%:%
%:%47=38%:%
%:%49=40%:%
%:%50=41%:%
%:%52=43%:%
%:%53=43%:%
%:%56=44%:%
%:%60=44%:%
%:%61=44%:%
%:%66=44%:%
%:%69=45%:%
%:%70=46%:%
%:%71=46%:%
%:%74=47%:%
%:%78=47%:%
%:%79=47%:%
%:%84=47%:%
%:%87=48%:%
%:%88=49%:%
%:%89=49%:%
%:%92=50%:%
%:%96=50%:%
%:%97=50%:%
%:%102=50%:%
%:%105=51%:%
%:%106=52%:%
%:%107=52%:%
%:%110=53%:%
%:%114=53%:%
%:%115=53%:%
%:%120=53%:%
%:%123=54%:%
%:%124=55%:%
%:%125=55%:%
%:%128=56%:%
%:%132=56%:%
%:%133=56%:%
%:%138=56%:%
%:%141=57%:%
%:%142=58%:%
%:%143=58%:%
%:%146=59%:%
%:%150=59%:%
%:%151=59%:%
%:%156=59%:%
%:%159=60%:%
%:%160=61%:%
%:%161=61%:%
%:%164=62%:%
%:%168=62%:%
%:%169=62%:%
%:%174=62%:%
%:%177=63%:%
%:%178=64%:%
%:%179=64%:%
%:%182=65%:%
%:%186=65%:%
%:%187=65%:%
%:%192=65%:%
%:%195=66%:%
%:%196=67%:%
%:%197=67%:%
%:%200=68%:%
%:%204=68%:%
%:%205=68%:%
%:%210=68%:%
%:%213=69%:%
%:%214=70%:%
%:%215=70%:%
%:%218=71%:%
%:%222=71%:%
%:%223=71%:%
%:%232=73%:%
%:%233=74%:%
%:%235=76%:%
%:%236=76%:%
%:%239=77%:%
%:%243=77%:%
%:%244=77%:%
%:%249=77%:%
%:%252=78%:%
%:%253=79%:%
%:%254=79%:%
%:%257=80%:%
%:%261=80%:%
%:%262=80%:%
%:%271=82%:%
%:%272=83%:%
%:%273=84%:%
%:%274=85%:%
%:%275=86%:%
%:%276=87%:%
%:%277=88%:%
%:%278=89%:%
%:%279=90%:%
%:%280=91%:%
%:%281=92%:%
%:%282=93%:%
%:%283=94%:%
%:%284=95%:%
%:%285=96%:%
%:%286=97%:%
%:%287=98%:%
%:%288=99%:%
%:%289=100%:%
%:%290=101%:%
%:%291=102%:%
%:%292=103%:%
%:%293=104%:%
%:%294=105%:%
%:%295=106%:%
%:%296=107%:%
%:%298=109%:%
%:%299=109%:%
%:%300=110%:%
%:%301=111%:%
%:%302=112%:%
%:%303=113%:%
%:%305=115%:%
%:%306=116%:%
%:%307=117%:%
%:%308=118%:%
%:%309=119%:%
%:%312=119%:%
%:%313=120%:%
%:%314=121%:%
%:%315=122%:%
%:%316=123%:%
%:%317=124%:%
%:%318=125%:%
%:%319=126%:%
%:%320=127%:%
%:%321=128%:%
%:%322=129%:%
%:%323=130%:%
%:%324=131%:%
%:%325=132%:%
%:%326=133%:%
%:%327=134%:%
%:%328=135%:%
%:%329=136%:%
%:%330=137%:%
%:%331=138%:%
%:%332=139%:%
%:%333=140%:%
%:%334=141%:%
%:%335=142%:%
%:%337=144%:%
%:%338=144%:%
%:%339=145%:%
%:%340=146%:%
%:%341=147%:%
%:%342=148%:%
%:%344=150%:%
%:%345=151%:%
%:%346=152%:%
%:%347=153%:%
%:%350=153%:%
%:%351=154%:%
%:%352=155%:%
%:%353=156%:%
%:%354=157%:%
%:%355=158%:%
%:%356=159%:%
%:%357=160%:%
%:%358=161%:%
%:%359=162%:%
%:%360=163%:%
%:%361=164%:%
%:%363=166%:%
%:%364=166%:%
%:%367=167%:%
%:%371=167%:%
%:%372=167%:%
%:%373=167%:%
%:%382=169%:%
%:%383=170%:%
%:%384=171%:%
%:%386=173%:%
%:%387=173%:%
%:%390=174%:%
%:%394=174%:%
%:%395=174%:%
%:%400=174%:%
%:%403=175%:%
%:%404=176%:%
%:%405=176%:%
%:%408=177%:%
%:%412=177%:%
%:%413=177%:%
%:%422=179%:%
%:%423=180%:%
%:%425=182%:%
%:%426=182%:%
%:%429=183%:%
%:%433=183%:%
%:%434=183%:%
%:%443=185%:%
%:%444=186%:%
%:%446=188%:%
%:%447=188%:%
%:%448=189%:%
%:%451=192%:%
%:%452=193%:%
%:%455=196%:%
%:%458=197%:%
%:%462=197%:%
%:%463=197%:%
%:%472=199%:%
%:%473=200%:%
%:%474=201%:%
%:%475=202%:%
%:%476=203%:%
%:%477=204%:%
%:%478=205%:%
%:%479=206%:%
%:%480=207%:%
%:%481=208%:%
%:%482=209%:%
%:%483=210%:%
%:%484=211%:%
%:%485=212%:%
%:%486=213%:%
%:%487=214%:%
%:%488=215%:%
%:%489=216%:%
%:%490=217%:%
%:%491=218%:%
%:%492=219%:%
%:%493=220%:%
%:%495=222%:%
%:%496=222%:%
%:%499=225%:%
%:%502=226%:%
%:%506=226%:%
%:%516=228%:%
%:%517=229%:%
%:%518=230%:%
%:%519=231%:%
%:%520=232%:%
%:%521=233%:%
%:%522=234%:%
%:%523=235%:%
%:%524=236%:%
%:%525=237%:%
%:%526=238%:%
%:%527=239%:%
%:%528=240%:%
%:%529=241%:%
%:%530=242%:%
%:%531=243%:%
%:%532=244%:%
%:%533=245%:%
%:%534=246%:%
%:%535=247%:%
%:%536=248%:%
%:%537=249%:%
%:%538=250%:%
%:%539=251%:%
%:%540=252%:%
%:%541=253%:%
%:%542=254%:%
%:%543=255%:%
%:%544=256%:%
%:%545=257%:%
%:%546=258%:%
%:%547=259%:%
%:%548=260%:%
%:%549=261%:%
%:%550=262%:%
%:%551=263%:%
%:%552=264%:%
%:%553=265%:%
%:%554=266%:%
%:%555=267%:%
%:%556=268%:%
%:%557=269%:%
%:%558=270%:%
%:%559=271%:%
%:%560=272%:%
%:%561=273%:%
%:%562=274%:%
%:%563=275%:%
%:%564=276%:%
%:%565=277%:%
%:%566=278%:%
%:%567=279%:%
%:%568=280%:%
%:%569=281%:%
%:%570=282%:%
%:%571=283%:%
%:%572=284%:%
%:%573=285%:%
%:%574=286%:%
%:%575=287%:%
%:%576=288%:%
%:%577=289%:%
%:%578=290%:%
%:%579=291%:%
%:%580=292%:%
%:%581=293%:%
%:%582=294%:%
%:%583=295%:%
%:%584=296%:%
%:%585=297%:%
%:%586=298%:%
%:%587=299%:%
%:%588=300%:%
%:%589=301%:%
%:%590=302%:%
%:%591=303%:%
%:%592=304%:%
%:%593=305%:%
%:%594=306%:%
%:%595=307%:%
%:%596=308%:%
%:%597=309%:%
%:%598=310%:%
%:%599=311%:%
%:%600=312%:%
%:%601=313%:%
%:%602=314%:%
%:%603=315%:%
%:%604=316%:%
%:%605=317%:%
%:%606=318%:%
%:%607=319%:%
%:%608=320%:%
%:%609=321%:%
%:%610=322%:%
%:%611=323%:%
%:%612=324%:%
%:%613=325%:%
%:%614=326%:%
%:%615=327%:%
%:%616=328%:%
%:%617=329%:%
%:%618=330%:%
%:%619=331%:%
%:%620=332%:%
%:%621=333%:%
%:%622=334%:%
%:%623=335%:%
%:%624=336%:%
%:%625=337%:%
%:%626=338%:%
%:%627=339%:%
%:%628=340%:%
%:%629=341%:%
%:%630=342%:%
%:%631=343%:%
%:%632=344%:%
%:%633=345%:%
%:%634=346%:%
%:%635=347%:%
%:%636=348%:%
%:%637=349%:%
%:%638=350%:%
%:%639=351%:%
%:%640=352%:%
%:%641=353%:%
%:%642=354%:%
%:%643=355%:%
%:%644=356%:%
%:%645=357%:%
%:%646=358%:%
%:%647=359%:%
%:%648=360%:%
%:%649=361%:%
%:%650=362%:%
%:%651=363%:%
%:%652=364%:%
%:%653=365%:%
%:%654=366%:%
%:%655=367%:%
%:%656=368%:%
%:%657=369%:%
%:%658=370%:%
%:%659=371%:%
%:%660=372%:%
%:%661=373%:%
%:%662=374%:%
%:%663=375%:%
%:%664=376%:%
%:%665=377%:%
%:%666=378%:%
%:%667=379%:%
%:%668=380%:%
%:%669=381%:%
%:%670=382%:%
%:%671=383%:%
%:%672=384%:%
%:%673=385%:%
%:%674=386%:%
%:%675=387%:%
%:%676=388%:%
%:%677=389%:%
%:%678=390%:%
%:%680=392%:%
%:%681=392%:%
%:%682=393%:%
%:%685=396%:%
%:%686=397%:%
%:%693=398%:%
%:%694=398%:%
%:%695=399%:%
%:%696=399%:%
%:%697=400%:%
%:%698=400%:%
%:%699=400%:%
%:%702=403%:%
%:%703=404%:%
%:%704=404%:%
%:%705=405%:%
%:%706=405%:%
%:%707=406%:%
%:%708=406%:%
%:%709=407%:%
%:%710=407%:%
%:%711=408%:%
%:%712=408%:%
%:%713=409%:%
%:%714=409%:%
%:%715=409%:%
%:%716=410%:%
%:%717=410%:%
%:%718=411%:%
%:%719=411%:%
%:%720=411%:%
%:%721=412%:%
%:%722=412%:%
%:%723=413%:%
%:%724=413%:%
%:%726=415%:%
%:%727=416%:%
%:%728=416%:%
%:%729=417%:%
%:%730=417%:%
%:%731=418%:%
%:%732=418%:%
%:%733=419%:%
%:%734=419%:%
%:%735=420%:%
%:%736=420%:%
%:%737=420%:%
%:%738=421%:%
%:%739=421%:%
%:%740=421%:%
%:%741=422%:%
%:%742=422%:%
%:%743=423%:%
%:%744=423%:%
%:%745=424%:%
%:%746=424%:%
%:%747=424%:%
%:%748=425%:%
%:%749=425%:%
%:%750=426%:%
%:%751=426%:%
%:%752=426%:%
%:%753=427%:%
%:%754=427%:%
%:%755=428%:%
%:%756=428%:%
%:%757=428%:%
%:%758=429%:%
%:%759=429%:%
%:%760=430%:%
%:%761=430%:%
%:%762=430%:%
%:%763=431%:%
%:%764=431%:%
%:%765=432%:%
%:%766=432%:%
%:%767=433%:%
%:%768=434%:%
%:%769=434%:%
%:%770=435%:%
%:%771=435%:%
%:%772=436%:%
%:%773=436%:%
%:%774=437%:%
%:%775=437%:%
%:%776=438%:%
%:%777=438%:%
%:%778=438%:%
%:%779=439%:%
%:%780=439%:%
%:%781=440%:%
%:%782=440%:%
%:%783=440%:%
%:%784=441%:%
%:%785=441%:%
%:%786=442%:%
%:%787=442%:%
%:%788=442%:%
%:%789=443%:%
%:%790=443%:%
%:%791=444%:%
%:%792=444%:%
%:%793=444%:%
%:%794=445%:%
%:%795=445%:%
%:%796=446%:%
%:%797=446%:%
%:%798=446%:%
%:%799=447%:%
%:%800=447%:%
%:%801=448%:%
%:%802=448%:%
%:%803=449%:%
%:%804=449%:%
%:%805=449%:%
%:%806=450%:%
%:%807=450%:%
%:%808=451%:%
%:%809=451%:%
%:%810=452%:%
%:%811=452%:%
%:%812=452%:%
%:%813=453%:%
%:%814=453%:%
%:%815=454%:%
%:%816=454%:%
%:%817=454%:%
%:%818=455%:%
%:%819=455%:%
%:%820=455%:%
%:%821=456%:%
%:%822=456%:%
%:%823=457%:%
%:%824=457%:%
%:%825=457%:%
%:%826=458%:%
%:%827=458%:%
%:%828=459%:%
%:%829=459%:%
%:%830=459%:%
%:%831=460%:%
%:%832=460%:%
%:%833=461%:%
%:%834=461%:%
%:%835=462%:%
%:%836=462%:%
%:%837=463%:%
%:%838=463%:%
%:%839=464%:%
%:%840=464%:%
%:%841=465%:%
%:%842=465%:%
%:%843=466%:%
%:%853=468%:%
%:%854=469%:%
%:%855=470%:%
%:%856=471%:%
%:%857=472%:%
%:%858=473%:%
%:%860=475%:%
%:%861=475%:%
%:%862=476%:%
%:%865=479%:%
%:%866=480%:%
%:%873=481%:%
%:%874=481%:%
%:%875=482%:%
%:%876=482%:%
%:%877=483%:%
%:%878=483%:%
%:%879=483%:%
%:%882=486%:%
%:%883=487%:%
%:%884=487%:%
%:%885=488%:%
%:%886=488%:%
%:%887=489%:%
%:%888=489%:%
%:%889=490%:%
%:%890=490%:%
%:%891=491%:%
%:%892=491%:%
%:%893=492%:%
%:%894=492%:%
%:%895=492%:%
%:%896=493%:%
%:%897=493%:%
%:%898=494%:%
%:%899=494%:%
%:%900=494%:%
%:%901=495%:%
%:%902=495%:%
%:%903=496%:%
%:%904=496%:%
%:%906=498%:%
%:%907=499%:%
%:%908=499%:%
%:%909=500%:%
%:%910=500%:%
%:%911=501%:%
%:%912=501%:%
%:%913=502%:%
%:%914=502%:%
%:%915=503%:%
%:%916=503%:%
%:%917=503%:%
%:%918=504%:%
%:%919=504%:%
%:%920=505%:%
%:%921=505%:%
%:%922=505%:%
%:%923=506%:%
%:%924=506%:%
%:%925=507%:%
%:%926=507%:%
%:%927=507%:%
%:%928=508%:%
%:%929=508%:%
%:%930=509%:%
%:%931=509%:%
%:%932=509%:%
%:%933=510%:%
%:%934=510%:%
%:%935=511%:%
%:%936=511%:%
%:%937=511%:%
%:%938=512%:%
%:%939=512%:%
%:%940=513%:%
%:%941=513%:%
%:%942=513%:%
%:%943=514%:%
%:%944=514%:%
%:%945=515%:%
%:%946=515%:%
%:%947=516%:%
%:%948=517%:%
%:%949=517%:%
%:%950=518%:%
%:%951=518%:%
%:%952=519%:%
%:%953=519%:%
%:%954=520%:%
%:%955=520%:%
%:%956=521%:%
%:%957=521%:%
%:%958=521%:%
%:%959=522%:%
%:%960=522%:%
%:%961=523%:%
%:%962=523%:%
%:%963=523%:%
%:%964=524%:%
%:%965=524%:%
%:%966=525%:%
%:%967=525%:%
%:%968=525%:%
%:%969=526%:%
%:%970=526%:%
%:%971=527%:%
%:%972=527%:%
%:%973=527%:%
%:%974=528%:%
%:%975=528%:%
%:%976=529%:%
%:%977=529%:%
%:%978=529%:%
%:%979=530%:%
%:%980=530%:%
%:%981=531%:%
%:%982=531%:%
%:%983=531%:%
%:%984=532%:%
%:%985=532%:%
%:%986=533%:%
%:%987=533%:%
%:%988=534%:%
%:%989=534%:%
%:%990=534%:%
%:%991=535%:%
%:%992=535%:%
%:%993=536%:%
%:%994=536%:%
%:%995=537%:%
%:%996=537%:%
%:%997=537%:%
%:%998=538%:%
%:%999=538%:%
%:%1000=539%:%
%:%1001=539%:%
%:%1002=539%:%
%:%1003=540%:%
%:%1004=540%:%
%:%1005=540%:%
%:%1006=541%:%
%:%1007=541%:%
%:%1008=542%:%
%:%1009=542%:%
%:%1010=542%:%
%:%1011=543%:%
%:%1012=543%:%
%:%1013=544%:%
%:%1014=544%:%
%:%1015=544%:%
%:%1016=545%:%
%:%1017=545%:%
%:%1018=546%:%
%:%1019=546%:%
%:%1020=547%:%
%:%1021=547%:%
%:%1022=548%:%
%:%1023=548%:%
%:%1024=549%:%
%:%1025=549%:%
%:%1026=550%:%
%:%1027=550%:%
%:%1028=551%:%
%:%1038=553%:%
%:%1039=554%:%
%:%1041=556%:%
%:%1042=556%:%
%:%1043=557%:%
%:%1044=558%:%
%:%1047=561%:%
%:%1054=562%:%
%:%1055=562%:%
%:%1056=563%:%
%:%1057=563%:%
%:%1065=571%:%
%:%1066=572%:%
%:%1067=572%:%
%:%1068=572%:%
%:%1069=573%:%
%:%1070=573%:%
%:%1071=573%:%
%:%1072=574%:%
%:%1073=574%:%
%:%1074=575%:%
%:%1075=575%:%
%:%1076=576%:%
%:%1077=576%:%
%:%1078=576%:%
%:%1079=577%:%
%:%1080=577%:%
%:%1081=578%:%
%:%1082=578%:%
%:%1083=579%:%
%:%1084=579%:%
%:%1085=580%:%
%:%1086=580%:%
%:%1087=581%:%
%:%1088=581%:%
%:%1089=581%:%
%:%1090=582%:%
%:%1091=582%:%
%:%1092=583%:%
%:%1093=583%:%
%:%1094=583%:%
%:%1095=584%:%
%:%1096=584%:%
%:%1097=585%:%
%:%1098=585%:%
%:%1099=585%:%
%:%1100=586%:%
%:%1101=586%:%
%:%1102=587%:%
%:%1103=587%:%
%:%1104=587%:%
%:%1105=588%:%
%:%1106=588%:%
%:%1109=591%:%
%:%1110=592%:%
%:%1111=592%:%
%:%1112=592%:%
%:%1113=593%:%
%:%1114=593%:%
%:%1115=594%:%
%:%1116=594%:%
%:%1117=595%:%
%:%1118=595%:%
%:%1119=596%:%
%:%1120=596%:%
%:%1121=597%:%
%:%1122=597%:%
%:%1123=598%:%
%:%1124=598%:%
%:%1125=599%:%
%:%1126=599%:%
%:%1127=600%:%
%:%1128=600%:%
%:%1129=600%:%
%:%1130=601%:%
%:%1131=601%:%
%:%1132=602%:%
%:%1133=602%:%
%:%1134=602%:%
%:%1135=603%:%
%:%1136=603%:%
%:%1137=604%:%
%:%1138=604%:%
%:%1139=604%:%
%:%1140=605%:%
%:%1141=605%:%
%:%1142=606%:%
%:%1143=606%:%
%:%1144=606%:%
%:%1145=607%:%
%:%1146=607%:%
%:%1149=610%:%
%:%1150=611%:%
%:%1151=611%:%
%:%1152=611%:%
%:%1153=612%:%
%:%1154=612%:%
%:%1155=613%:%
%:%1156=613%:%
%:%1157=614%:%
%:%1158=614%:%
%:%1159=615%:%
%:%1160=615%:%
%:%1163=618%:%
%:%1164=619%:%
%:%1165=619%:%
%:%1166=619%:%
%:%1167=620%:%
%:%1177=622%:%
%:%1178=623%:%
%:%1180=625%:%
%:%1181=625%:%
%:%1182=626%:%
%:%1185=629%:%
%:%1186=630%:%
%:%1193=631%:%
%:%1194=631%:%
%:%1195=632%:%
%:%1196=632%:%
%:%1197=633%:%
%:%1198=633%:%
%:%1199=633%:%
%:%1200=634%:%
%:%1201=634%:%
%:%1202=635%:%
%:%1203=635%:%
%:%1204=635%:%
%:%1205=636%:%
%:%1206=636%:%
%:%1214=644%:%
%:%1215=645%:%
%:%1216=645%:%
%:%1217=646%:%
%:%1218=646%:%
%:%1219=647%:%
%:%1220=647%:%
%:%1221=647%:%
%:%1222=648%:%
%:%1223=648%:%
%:%1224=649%:%
%:%1225=649%:%
%:%1226=649%:%
%:%1227=650%:%
%:%1228=650%:%
%:%1229=651%:%
%:%1230=651%:%
%:%1231=652%:%
%:%1232=652%:%
%:%1233=653%:%
%:%1234=653%:%
%:%1235=654%:%
%:%1236=654%:%
%:%1237=655%:%
%:%1238=655%:%
%:%1239=656%:%
%:%1240=656%:%
%:%1241=657%:%
%:%1242=657%:%
%:%1243=658%:%
%:%1244=658%:%
%:%1245=659%:%
%:%1246=659%:%
%:%1247=659%:%
%:%1248=660%:%
%:%1249=660%:%
%:%1250=660%:%
%:%1251=661%:%
%:%1252=661%:%
%:%1253=661%:%
%:%1254=662%:%
%:%1255=662%:%
%:%1256=663%:%
%:%1257=663%:%
%:%1258=663%:%
%:%1259=664%:%
%:%1260=664%:%
%:%1261=665%:%
%:%1262=665%:%
%:%1263=666%:%
%:%1264=666%:%
%:%1265=667%:%
%:%1266=667%:%
%:%1267=668%:%
%:%1268=668%:%
%:%1269=669%:%
%:%1270=669%:%
%:%1271=670%:%
%:%1272=670%:%
%:%1273=671%:%
%:%1274=671%:%
%:%1275=672%:%
%:%1276=672%:%
%:%1277=673%:%
%:%1278=673%:%
%:%1279=674%:%
%:%1280=674%:%
%:%1281=675%:%
%:%1282=675%:%
%:%1283=675%:%
%:%1284=676%:%
%:%1285=676%:%
%:%1286=676%:%
%:%1287=677%:%
%:%1288=677%:%
%:%1289=677%:%
%:%1290=678%:%
%:%1291=678%:%
%:%1292=679%:%
%:%1293=679%:%
%:%1294=679%:%
%:%1295=680%:%
%:%1296=680%:%
%:%1297=681%:%
%:%1298=681%:%
%:%1299=682%:%
%:%1300=682%:%
%:%1301=683%:%
%:%1302=683%:%
%:%1303=684%:%
%:%1304=684%:%
%:%1305=685%:%
%:%1306=685%:%
%:%1307=686%:%
%:%1308=686%:%
%:%1309=687%:%
%:%1310=687%:%
%:%1311=688%:%
%:%1312=688%:%
%:%1313=689%:%
%:%1314=689%:%
%:%1315=690%:%
%:%1316=690%:%
%:%1317=691%:%
%:%1318=691%:%
%:%1319=691%:%
%:%1320=692%:%
%:%1321=692%:%
%:%1322=692%:%
%:%1323=693%:%
%:%1324=693%:%
%:%1325=693%:%
%:%1326=694%:%
%:%1327=694%:%
%:%1328=695%:%
%:%1329=695%:%
%:%1330=695%:%
%:%1331=696%:%
%:%1332=696%:%
%:%1333=697%:%
%:%1334=697%:%
%:%1335=698%:%
%:%1336=698%:%
%:%1337=699%:%
%:%1338=699%:%
%:%1339=700%:%
%:%1340=700%:%
%:%1341=701%:%
%:%1342=701%:%
%:%1343=702%:%
%:%1344=702%:%
%:%1345=703%:%
%:%1346=703%:%
%:%1347=704%:%
%:%1348=704%:%
%:%1349=705%:%
%:%1350=705%:%
%:%1351=706%:%
%:%1352=706%:%
%:%1353=707%:%
%:%1354=707%:%
%:%1355=707%:%
%:%1356=708%:%
%:%1357=708%:%
%:%1358=708%:%
%:%1359=709%:%
%:%1360=709%:%
%:%1361=709%:%
%:%1362=710%:%
%:%1363=710%:%
%:%1364=710%:%
%:%1365=711%:%
%:%1366=711%:%
%:%1367=711%:%
%:%1368=712%:%
%:%1369=712%:%
%:%1370=713%:%
%:%1371=713%:%
%:%1372=714%:%
%:%1373=714%:%
%:%1374=715%:%
%:%1375=715%:%
%:%1376=716%:%
%:%1377=716%:%
%:%1378=717%:%
%:%1379=717%:%
%:%1380=718%:%
%:%1381=718%:%
%:%1382=719%:%
%:%1383=719%:%
%:%1384=720%:%
%:%1385=720%:%
%:%1386=721%:%
%:%1387=721%:%
%:%1388=722%:%
%:%1389=722%:%
%:%1390=723%:%
%:%1391=723%:%
%:%1392=724%:%
%:%1393=724%:%
%:%1394=725%:%
%:%1395=725%:%
%:%1396=725%:%
%:%1397=726%:%
%:%1398=726%:%
%:%1399=726%:%
%:%1400=727%:%
%:%1401=727%:%
%:%1402=727%:%
%:%1403=728%:%
%:%1404=728%:%
%:%1405=729%:%
%:%1406=729%:%
%:%1407=729%:%
%:%1408=730%:%
%:%1409=730%:%
%:%1410=731%:%
%:%1411=731%:%
%:%1412=732%:%
%:%1413=732%:%
%:%1414=733%:%
%:%1415=733%:%
%:%1416=734%:%
%:%1417=734%:%
%:%1418=735%:%
%:%1419=735%:%
%:%1420=736%:%
%:%1421=736%:%
%:%1422=737%:%
%:%1423=737%:%
%:%1424=738%:%
%:%1425=738%:%
%:%1426=739%:%
%:%1427=739%:%
%:%1428=740%:%
%:%1429=740%:%
%:%1430=741%:%
%:%1431=741%:%
%:%1432=741%:%
%:%1433=742%:%
%:%1434=742%:%
%:%1435=742%:%
%:%1436=743%:%
%:%1437=743%:%
%:%1438=743%:%
%:%1439=744%:%
%:%1440=744%:%
%:%1441=745%:%
%:%1442=745%:%
%:%1443=745%:%
%:%1444=746%:%
%:%1445=746%:%
%:%1446=747%:%
%:%1447=747%:%
%:%1448=748%:%
%:%1449=748%:%
%:%1450=749%:%
%:%1451=749%:%
%:%1452=750%:%
%:%1453=750%:%
%:%1454=751%:%
%:%1455=751%:%
%:%1456=752%:%
%:%1457=752%:%
%:%1458=753%:%
%:%1459=753%:%
%:%1460=754%:%
%:%1461=754%:%
%:%1462=755%:%
%:%1463=755%:%
%:%1464=756%:%
%:%1465=756%:%
%:%1466=757%:%
%:%1467=757%:%
%:%1468=757%:%
%:%1469=758%:%
%:%1470=758%:%
%:%1471=758%:%
%:%1472=759%:%
%:%1473=759%:%
%:%1474=759%:%
%:%1475=760%:%
%:%1476=760%:%
%:%1477=761%:%
%:%1478=761%:%
%:%1479=761%:%
%:%1480=762%:%
%:%1481=762%:%
%:%1482=763%:%
%:%1483=763%:%
%:%1484=764%:%
%:%1485=764%:%
%:%1489=768%:%
%:%1490=769%:%
%:%1491=769%:%
%:%1492=769%:%
%:%1493=770%:%
%:%1494=770%:%
%:%1497=773%:%
%:%1498=774%:%
%:%1499=774%:%
%:%1500=774%:%
%:%1501=775%:%
%:%1502=775%:%
%:%1510=783%:%
%:%1511=784%:%
%:%1512=784%:%
%:%1513=784%:%
%:%1514=785%:%
%:%1515=785%:%
%:%1523=793%:%
%:%1524=794%:%
%:%1525=794%:%
%:%1526=795%:%
%:%1527=795%:%
%:%1528=796%:%
%:%1529=796%:%
%:%1530=797%:%
%:%1531=797%:%
%:%1532=797%:%
%:%1533=798%:%
%:%1543=800%:%
%:%1545=802%:%
%:%1546=802%:%
%:%1549=805%:%
%:%1556=806%:%
%:%1557=806%:%
%:%1558=807%:%
%:%1559=807%:%
%:%1560=808%:%
%:%1561=808%:%
%:%1564=811%:%
%:%1565=812%:%
%:%1566=812%:%
%:%1567=813%:%
%:%1568=813%:%
%:%1569=814%:%
%:%1570=814%:%
%:%1573=817%:%
%:%1574=818%:%
%:%1575=818%:%
%:%1576=819%:%
%:%1577=819%:%
%:%1578=820%:%
%:%1588=822%:%
%:%1590=824%:%
%:%1591=824%:%
%:%1594=827%:%
%:%1597=828%:%
%:%1601=828%:%
%:%1602=828%:%
%:%1603=829%:%
%:%1604=829%:%
%:%1605=830%:%
%:%1606=830%:%
%:%1607=830%:%
%:%1608=831%:%
%:%1609=831%:%
%:%1610=831%:%
%:%1611=832%:%
%:%1612=832%:%

\chapter{Propiedad de consistencia proposicional}
%
\begin{isabellebody}%
\setisabellecontext{Pcp}%
%
\isadelimtheory
%
\endisadelimtheory
%
\isatagtheory
%
\endisatagtheory
{\isafoldtheory}%
%
\isadelimtheory
%
\endisadelimtheory
%
\begin{isamarkuptext}%
En este capítulo nos centraremos en demostrar el \isa{teorema\ de\ existencia\ de\ modelos}.
  Dicho teorema prueba la satisfacibilidad de un conjunto de fórmulas \isa{S} si este pertenece a una 
  colección de conjuntos \isa{C} que verifica la \isa{propiedad\ de\ consistencia\ proposicional}. Para su 
  prueba, definiremos las propiedades de \isa{carácter\ finito} y \isa{ser\ cerrada\ bajo\ subconjuntos} para
  colecciones de conjuntos de fórmulas. De este modo, mediante distintos resultados que relacionan
  estas propiedades con la \isa{propiedad\ de\ consistencia\ proposicional}, dada una colección \isa{C} 
  cualquiera en las condiciones anteriormente descritas, podemos encontrar una colección \isa{C{\isacharprime}} que la 
  contenga que verifique la \isa{propiedad\ de\ consistencia\ proposicional}, sea \isa{cerrada\ bajo\ subconjuntos} y de \isa{carácter\ finito}. Por otro lado, definiremos una sucesión de conjuntos de
  fórmulas a partir de la colección \isa{C{\isacharprime}} y el conjunto \isa{S}. Además, definiremos el límite de dicha
  sucesión que, en particular, contendrá al conjunto \isa{S}. Finalmente probaremos que dicho límite es 
  un conjunto satisfacible por el \isa{lema\ de\ Hintikka} y, por contención, quedará probada la 
  satisfacibilidad del conjunto \isa{S}.

  \comentario{Modificar dicho párrafo al final y ver cómo reubicar explicación completa del
  teorema de existencia de modelo.}%
\end{isamarkuptext}\isamarkuptrue%
%
\begin{isamarkuptext}%
En esta sección vamos a definir la \isa{propiedad\ de\ consistencia\ proposicional} para una 
  colección de conjuntos de fórmulas proposicionales. Finalmente mostraremos un lema
  que caracteriza dicha propiedad mediante la notación uniforme.%
\end{isamarkuptext}\isamarkuptrue%
%
\begin{isamarkuptext}%
\begin{definicion}
    Sea \isa{C} una colección de conjuntos de fórmulas proposicionales. Decimos que
    \isa{C} verifica la \isa{propiedad\ de\ consistencia\ proposicional} si, para todo
    conjunto \isa{S} perteneciente a la colección, se verifica:
    \begin{enumerate}
      \item \isa{{\isasymbottom}\ {\isasymnotin}\ S}.
      \item Dada \isa{p} una fórmula atómica cualquiera, no se tiene 
        simultáneamente que\\ \isa{p\ {\isasymin}\ S} y \isa{{\isasymnot}\ p\ {\isasymin}\ S}.
      \item Si \isa{F\ {\isasymand}\ G\ {\isasymin}\ S}, entonces el conjunto \isa{{\isacharbraceleft}F{\isacharcomma}G{\isacharbraceright}\ {\isasymunion}\ S} pertenece a \isa{C}.
      \item Si \isa{F\ {\isasymor}\ G\ {\isasymin}\ S}, entonces o bien el conjunto \isa{{\isacharbraceleft}F{\isacharbraceright}\ {\isasymunion}\ S} pertenece a \isa{C}, o bien el 
        conjunto \isa{{\isacharbraceleft}G{\isacharbraceright}\ {\isasymunion}\ S} pertenece a \isa{C}.
      \item Si \isa{F\ {\isasymrightarrow}\ G\ {\isasymin}\ S}, entonces o bien el conjunto \isa{{\isacharbraceleft}{\isasymnot}\ F{\isacharbraceright}\ {\isasymunion}\ S} pertenece a \isa{C}, o bien el 
        conjunto \isa{{\isacharbraceleft}G{\isacharbraceright}\ {\isasymunion}\ S} pertenece a \isa{C}.
      \item Si \isa{{\isasymnot}{\isacharparenleft}{\isasymnot}\ F{\isacharparenright}\ {\isasymin}\ S}, entonces el conjunto \isa{{\isacharbraceleft}F{\isacharbraceright}\ {\isasymunion}\ S} pertenece a \isa{C}.
      \item Si \isa{{\isasymnot}{\isacharparenleft}F\ {\isasymand}\ G{\isacharparenright}\ {\isasymin}\ S}, entonces o bien el conjunto \isa{{\isacharbraceleft}{\isasymnot}\ F{\isacharbraceright}\ {\isasymunion}\ S} pertenece a \isa{C}, o bien el 
        conjunto \isa{{\isacharbraceleft}{\isasymnot}\ G{\isacharbraceright}\ {\isasymunion}\ S} pertenece a \isa{C}.
      \item Si \isa{{\isasymnot}{\isacharparenleft}F\ {\isasymor}\ G{\isacharparenright}\ {\isasymin}\ S}, entonces el conjunto \isa{{\isacharbraceleft}{\isasymnot}\ F{\isacharcomma}\ {\isasymnot}\ G{\isacharbraceright}\ {\isasymunion}\ S} pertenece a \isa{C}.
      \item Si \isa{{\isasymnot}{\isacharparenleft}F\ {\isasymrightarrow}\ G{\isacharparenright}\ {\isasymin}\ S}, entonces el conjunto \isa{{\isacharbraceleft}F{\isacharcomma}\ {\isasymnot}\ G{\isacharbraceright}\ {\isasymunion}\ S} pertenece a \isa{C}.
    \end{enumerate}
  \end{definicion}

  Veamos, a continuación, su formalización en Isabelle mediante el tipo \isa{definition}.%
\end{isamarkuptext}\isamarkuptrue%
\isacommand{definition}\isamarkupfalse%
\ {\isachardoublequoteopen}pcp\ C\ {\isasymequiv}\ {\isacharparenleft}{\isasymforall}S\ {\isasymin}\ C{\isachardot}\isanewline
\ \ {\isasymbottom}\ {\isasymnotin}\ S\isanewline
{\isasymand}\ {\isacharparenleft}{\isasymforall}k{\isachardot}\ Atom\ k\ {\isasymin}\ S\ {\isasymlongrightarrow}\ \isactrlbold {\isasymnot}\ {\isacharparenleft}Atom\ k{\isacharparenright}\ {\isasymin}\ S\ {\isasymlongrightarrow}\ False{\isacharparenright}\isanewline
{\isasymand}\ {\isacharparenleft}{\isasymforall}F\ G{\isachardot}\ F\ \isactrlbold {\isasymand}\ G\ {\isasymin}\ S\ {\isasymlongrightarrow}\ {\isacharbraceleft}F{\isacharcomma}G{\isacharbraceright}\ {\isasymunion}\ S\ {\isasymin}\ C{\isacharparenright}\isanewline
{\isasymand}\ {\isacharparenleft}{\isasymforall}F\ G{\isachardot}\ F\ \isactrlbold {\isasymor}\ G\ {\isasymin}\ S\ {\isasymlongrightarrow}\ {\isacharbraceleft}F{\isacharbraceright}\ {\isasymunion}\ S\ {\isasymin}\ C\ {\isasymor}\ {\isacharbraceleft}G{\isacharbraceright}\ {\isasymunion}\ S\ {\isasymin}\ C{\isacharparenright}\isanewline
{\isasymand}\ {\isacharparenleft}{\isasymforall}F\ G{\isachardot}\ F\ \isactrlbold {\isasymrightarrow}\ G\ {\isasymin}\ S\ {\isasymlongrightarrow}\ {\isacharbraceleft}\isactrlbold {\isasymnot}F{\isacharbraceright}\ {\isasymunion}\ S\ {\isasymin}\ C\ {\isasymor}\ {\isacharbraceleft}G{\isacharbraceright}\ {\isasymunion}\ S\ {\isasymin}\ C{\isacharparenright}\isanewline
{\isasymand}\ {\isacharparenleft}{\isasymforall}F{\isachardot}\ \isactrlbold {\isasymnot}\ {\isacharparenleft}\isactrlbold {\isasymnot}F{\isacharparenright}\ {\isasymin}\ S\ {\isasymlongrightarrow}\ {\isacharbraceleft}F{\isacharbraceright}\ {\isasymunion}\ S\ {\isasymin}\ C{\isacharparenright}\isanewline
{\isasymand}\ {\isacharparenleft}{\isasymforall}F\ G{\isachardot}\ \isactrlbold {\isasymnot}{\isacharparenleft}F\ \isactrlbold {\isasymand}\ G{\isacharparenright}\ {\isasymin}\ S\ {\isasymlongrightarrow}\ {\isacharbraceleft}\isactrlbold {\isasymnot}\ F{\isacharbraceright}\ {\isasymunion}\ S\ {\isasymin}\ C\ {\isasymor}\ {\isacharbraceleft}\isactrlbold {\isasymnot}\ G{\isacharbraceright}\ {\isasymunion}\ S\ {\isasymin}\ C{\isacharparenright}\isanewline
{\isasymand}\ {\isacharparenleft}{\isasymforall}F\ G{\isachardot}\ \isactrlbold {\isasymnot}{\isacharparenleft}F\ \isactrlbold {\isasymor}\ G{\isacharparenright}\ {\isasymin}\ S\ {\isasymlongrightarrow}\ {\isacharbraceleft}\isactrlbold {\isasymnot}\ F{\isacharcomma}\ \isactrlbold {\isasymnot}\ G{\isacharbraceright}\ {\isasymunion}\ S\ {\isasymin}\ C{\isacharparenright}\isanewline
{\isasymand}\ {\isacharparenleft}{\isasymforall}F\ G{\isachardot}\ \isactrlbold {\isasymnot}{\isacharparenleft}F\ \isactrlbold {\isasymrightarrow}\ G{\isacharparenright}\ {\isasymin}\ S\ {\isasymlongrightarrow}\ {\isacharbraceleft}F{\isacharcomma}\isactrlbold {\isasymnot}\ G{\isacharbraceright}\ {\isasymunion}\ S\ {\isasymin}\ C{\isacharparenright}{\isacharparenright}{\isachardoublequoteclose}%
\begin{isamarkuptext}%
Observando la definición anterior, se prueba fácilmente que la colección trivial
  formada por el conjunto vacío de fórmulas verifica la propiedad de consistencia 
  proposicional.%
\end{isamarkuptext}\isamarkuptrue%
\isacommand{lemma}\isamarkupfalse%
\ {\isachardoublequoteopen}pcp\ {\isacharbraceleft}{\isacharbraceleft}{\isacharbraceright}{\isacharbraceright}{\isachardoublequoteclose}\isanewline
%
\isadelimproof
\ \ %
\endisadelimproof
%
\isatagproof
\isacommand{unfolding}\isamarkupfalse%
\ pcp{\isacharunderscore}def\ \isacommand{by}\isamarkupfalse%
\ simp%
\endisatagproof
{\isafoldproof}%
%
\isadelimproof
%
\endisadelimproof
%
\begin{isamarkuptext}%
Del mismo modo, aplicando la definición, se demuestra que los siguientes ejemplos
  de colecciones de conjuntos de fórmulas proposicionales verifican igualmente la 
  propiedad.%
\end{isamarkuptext}\isamarkuptrue%
\isacommand{lemma}\isamarkupfalse%
\ {\isachardoublequoteopen}pcp\ {\isacharbraceleft}{\isacharbraceleft}Atom\ {\isadigit{0}}{\isacharbraceright}{\isacharbraceright}{\isachardoublequoteclose}\isanewline
%
\isadelimproof
\ \ %
\endisadelimproof
%
\isatagproof
\isacommand{unfolding}\isamarkupfalse%
\ pcp{\isacharunderscore}def\ \isacommand{by}\isamarkupfalse%
\ simp%
\endisatagproof
{\isafoldproof}%
%
\isadelimproof
\isanewline
%
\endisadelimproof
\isanewline
\isacommand{lemma}\isamarkupfalse%
\ {\isachardoublequoteopen}pcp\ {\isacharbraceleft}{\isacharbraceleft}{\isacharparenleft}\isactrlbold {\isasymnot}\ {\isacharparenleft}Atom\ {\isadigit{1}}{\isacharparenright}{\isacharparenright}\ \isactrlbold {\isasymrightarrow}\ Atom\ {\isadigit{2}}{\isacharbraceright}{\isacharcomma}\isanewline
\ \ \ {\isacharbraceleft}{\isacharparenleft}{\isacharparenleft}\isactrlbold {\isasymnot}\ {\isacharparenleft}Atom\ {\isadigit{1}}{\isacharparenright}{\isacharparenright}\ \isactrlbold {\isasymrightarrow}\ Atom\ {\isadigit{2}}{\isacharparenright}{\isacharcomma}\ \isactrlbold {\isasymnot}{\isacharparenleft}\isactrlbold {\isasymnot}\ {\isacharparenleft}Atom\ {\isadigit{1}}{\isacharparenright}{\isacharparenright}{\isacharbraceright}{\isacharcomma}\isanewline
\ \ {\isacharbraceleft}{\isacharparenleft}{\isacharparenleft}\isactrlbold {\isasymnot}\ {\isacharparenleft}Atom\ {\isadigit{1}}{\isacharparenright}{\isacharparenright}\ \isactrlbold {\isasymrightarrow}\ Atom\ {\isadigit{2}}{\isacharparenright}{\isacharcomma}\ \isactrlbold {\isasymnot}{\isacharparenleft}\isactrlbold {\isasymnot}\ {\isacharparenleft}Atom\ {\isadigit{1}}{\isacharparenright}{\isacharparenright}{\isacharcomma}\ \ Atom\ {\isadigit{1}}{\isacharbraceright}{\isacharbraceright}{\isachardoublequoteclose}\ \isanewline
%
\isadelimproof
\ \ %
\endisadelimproof
%
\isatagproof
\isacommand{unfolding}\isamarkupfalse%
\ pcp{\isacharunderscore}def\ \isacommand{by}\isamarkupfalse%
\ auto%
\endisatagproof
{\isafoldproof}%
%
\isadelimproof
%
\endisadelimproof
%
\begin{isamarkuptext}%
Por último, en contraposición podemos ilustrar un caso de colección que no verifique la 
  propiedad con la siguiente colección obtenida al modificar el último ejemplo. De
  esta manera, aunque la colección verifique correctamente la quinta condición de la
  definición, no cumplirá la sexta.%
\end{isamarkuptext}\isamarkuptrue%
\isacommand{lemma}\isamarkupfalse%
\ {\isachardoublequoteopen}{\isasymnot}\ pcp\ {\isacharbraceleft}{\isacharbraceleft}{\isacharparenleft}\isactrlbold {\isasymnot}\ {\isacharparenleft}Atom\ {\isadigit{1}}{\isacharparenright}{\isacharparenright}\ \isactrlbold {\isasymrightarrow}\ Atom\ {\isadigit{2}}{\isacharbraceright}{\isacharcomma}\isanewline
\ \ \ {\isacharbraceleft}{\isacharparenleft}{\isacharparenleft}\isactrlbold {\isasymnot}\ {\isacharparenleft}Atom\ {\isadigit{1}}{\isacharparenright}{\isacharparenright}\ \isactrlbold {\isasymrightarrow}\ Atom\ {\isadigit{2}}{\isacharparenright}{\isacharcomma}\ \isactrlbold {\isasymnot}{\isacharparenleft}\isactrlbold {\isasymnot}\ {\isacharparenleft}Atom\ {\isadigit{1}}{\isacharparenright}{\isacharparenright}{\isacharbraceright}{\isacharbraceright}{\isachardoublequoteclose}\ \isanewline
%
\isadelimproof
\ \ %
\endisadelimproof
%
\isatagproof
\isacommand{unfolding}\isamarkupfalse%
\ pcp{\isacharunderscore}def\ \isacommand{by}\isamarkupfalse%
\ auto%
\endisatagproof
{\isafoldproof}%
%
\isadelimproof
%
\endisadelimproof
%
\begin{isamarkuptext}%
Por otra parte, veamos un resultado que permite la caracterización de la 
  propiedad de consistencia proposicional mediante la notación uniforme.

  \begin{lema}[Caracterización de \isa{P{\isachardot}C{\isachardot}P} mediante la notación uniforme]
    Dada una colección \isa{C} de conjuntos de fórmulas proposicionales, son equivalentes:
    \begin{enumerate}
      \item \isa{C} verifica la propiedad de consistencia proposicional.
      \item Para cualquier conjunto de fórmulas \isa{S} de la colección, se verifican las 
      condiciones:
      \begin{itemize}
        \item \isa{{\isasymbottom}} no pertenece a \isa{S}.
        \item Dada \isa{p} una fórmula atómica cualquiera, no se tiene 
        simultáneamente que\\ \isa{p\ {\isasymin}\ S} y \isa{{\isasymnot}\ p\ {\isasymin}\ S}.
        \item Para toda fórmula de tipo \isa{{\isasymalpha}} con componentes \isa{{\isasymalpha}\isactrlsub {\isadigit{1}}} y \isa{{\isasymalpha}\isactrlsub {\isadigit{2}}} tal que \isa{{\isasymalpha}}
        pertenece a \isa{S}, se tiene que \isa{{\isacharbraceleft}{\isasymalpha}\isactrlsub {\isadigit{1}}{\isacharcomma}{\isasymalpha}\isactrlsub {\isadigit{2}}{\isacharbraceright}\ {\isasymunion}\ S} pertenece a \isa{C}.
        \item Para toda fórmula de tipo \isa{{\isasymbeta}} con componentes \isa{{\isasymbeta}\isactrlsub {\isadigit{1}}} y \isa{{\isasymbeta}\isactrlsub {\isadigit{2}}} tal que \isa{{\isasymbeta}}
        pertenece a \isa{S}, se tiene que o bien \isa{{\isacharbraceleft}{\isasymbeta}\isactrlsub {\isadigit{1}}{\isacharbraceright}\ {\isasymunion}\ S} pertenece a \isa{C} o 
        bien \isa{{\isacharbraceleft}{\isasymbeta}\isactrlsub {\isadigit{2}}{\isacharbraceright}\ {\isasymunion}\ S} pertenece a \isa{C}.
      \end{itemize} 
    \end{enumerate}
  \end{lema}

  En Isabelle/HOL se formaliza el resultado como sigue.%
\end{isamarkuptext}\isamarkuptrue%
\isacommand{lemma}\isamarkupfalse%
\ {\isachardoublequoteopen}pcp\ C\ {\isacharequal}\ {\isacharparenleft}{\isasymforall}S\ {\isasymin}\ C{\isachardot}\ {\isasymbottom}\ {\isasymnotin}\ S\isanewline
{\isasymand}\ {\isacharparenleft}{\isasymforall}k{\isachardot}\ Atom\ k\ {\isasymin}\ S\ {\isasymlongrightarrow}\ \isactrlbold {\isasymnot}\ {\isacharparenleft}Atom\ k{\isacharparenright}\ {\isasymin}\ S\ {\isasymlongrightarrow}\ False{\isacharparenright}\isanewline
{\isasymand}\ {\isacharparenleft}{\isasymforall}F\ G\ H{\isachardot}\ Con\ F\ G\ H\ {\isasymlongrightarrow}\ F\ {\isasymin}\ S\ {\isasymlongrightarrow}\ {\isacharbraceleft}G{\isacharcomma}H{\isacharbraceright}\ {\isasymunion}\ S\ {\isasymin}\ C{\isacharparenright}\isanewline
{\isasymand}\ {\isacharparenleft}{\isasymforall}F\ G\ H{\isachardot}\ Dis\ F\ G\ H\ {\isasymlongrightarrow}\ F\ {\isasymin}\ S\ {\isasymlongrightarrow}\ {\isacharbraceleft}G{\isacharbraceright}\ {\isasymunion}\ S\ {\isasymin}\ C\ {\isasymor}\ {\isacharbraceleft}H{\isacharbraceright}\ {\isasymunion}\ S\ {\isasymin}\ C{\isacharparenright}{\isacharparenright}{\isachardoublequoteclose}\isanewline
%
\isadelimproof
\ \ %
\endisadelimproof
%
\isatagproof
\isacommand{oops}\isamarkupfalse%
%
\endisatagproof
{\isafoldproof}%
%
\isadelimproof
%
\endisadelimproof
%
\begin{isamarkuptext}%
En primer lugar, veamos la demostración del lema.

\begin{demostracion}
  Para probar la equivalencia, veamos cada una de las implicaciones por separado.

\textbf{\isa{{\isadigit{1}}{\isacharparenright}\ {\isasymLongrightarrow}\ {\isadigit{2}}{\isacharparenright}}}
  
  Supongamos que \isa{C} es una colección de conjuntos de fórmulas proposicionales que
  verifica la propiedad de consistencia proposicional. Vamos a probar que, en efecto,
  cumple las condiciones de \isa{{\isadigit{2}}{\isacharparenright}}. 

  Consideremos un conjunto de fórmulas \isa{S} perteneciente a la colección \isa{C}.
  Por hipótesis, de la definición de propiedad de consistencia proposicional obtenemos
  que \isa{S} verifica las siguientes condiciones:
 \begin{enumerate}
      \item \isa{{\isasymbottom}\ {\isasymnotin}\ S}.
      \item Dada \isa{p} una fórmula atómica cualquiera, no se tiene 
        simultáneamente que\\ \isa{p\ {\isasymin}\ S} y \isa{{\isasymnot}\ p\ {\isasymin}\ S}.
      \item Si \isa{G\ {\isasymand}\ H\ {\isasymin}\ S}, entonces el conjunto \isa{{\isacharbraceleft}G{\isacharcomma}H{\isacharbraceright}\ {\isasymunion}\ S} pertenece a \isa{C}.
      \item Si \isa{G\ {\isasymor}\ H\ {\isasymin}\ S}, entonces o bien el conjunto \isa{{\isacharbraceleft}G{\isacharbraceright}\ {\isasymunion}\ S} pertenece a \isa{C}, o bien el 
        conjunto \isa{{\isacharbraceleft}H{\isacharbraceright}\ {\isasymunion}\ S} pertenece a \isa{C}.
      \item Si \isa{G\ {\isasymrightarrow}\ H\ {\isasymin}\ S}, entonces o bien el conjunto \isa{{\isacharbraceleft}{\isasymnot}\ G{\isacharbraceright}\ {\isasymunion}\ S} pertenece a \isa{C}, o bien el 
        conjunto \isa{{\isacharbraceleft}H{\isacharbraceright}\ {\isasymunion}\ S} pertenece a \isa{C}.
      \item Si \isa{{\isasymnot}{\isacharparenleft}{\isasymnot}\ G{\isacharparenright}\ {\isasymin}\ S}, entonces el conjunto \isa{{\isacharbraceleft}G{\isacharbraceright}\ {\isasymunion}\ S} pertenece a \isa{C}.
      \item Si \isa{{\isasymnot}{\isacharparenleft}G\ {\isasymand}\ H{\isacharparenright}\ {\isasymin}\ S}, entonces o bien el conjunto \isa{{\isacharbraceleft}{\isasymnot}\ G{\isacharbraceright}\ {\isasymunion}\ S} pertenece a \isa{C}, o bien el 
        conjunto \isa{{\isacharbraceleft}{\isasymnot}\ H{\isacharbraceright}\ {\isasymunion}\ S} pertenece a \isa{C}.
      \item Si \isa{{\isasymnot}{\isacharparenleft}G\ {\isasymor}\ H{\isacharparenright}\ {\isasymin}\ S}, entonces el conjunto \isa{{\isacharbraceleft}{\isasymnot}\ G{\isacharcomma}\ {\isasymnot}\ H{\isacharbraceright}\ {\isasymunion}\ S} pertenece a \isa{C}.
      \item Si \isa{{\isasymnot}{\isacharparenleft}G\ {\isasymrightarrow}\ H{\isacharparenright}\ {\isasymin}\ S}, entonces el conjunto \isa{{\isacharbraceleft}G{\isacharcomma}\ {\isasymnot}\ H{\isacharbraceright}\ {\isasymunion}\ S} pertenece a \isa{C}.
 \end{enumerate}

  Las dos primeras condiciones se corresponden con los dos primeros resultados que queríamos
  demostrar. De este modo, falta probar:
  \begin{itemize}
     \item Para toda fórmula de tipo \isa{{\isasymalpha}} con componentes \isa{{\isasymalpha}\isactrlsub {\isadigit{1}}} y \isa{{\isasymalpha}\isactrlsub {\isadigit{2}}} tal que \isa{{\isasymalpha}}
     pertenece a \isa{S}, se tiene que \isa{{\isacharbraceleft}{\isasymalpha}\isactrlsub {\isadigit{1}}{\isacharcomma}{\isasymalpha}\isactrlsub {\isadigit{2}}{\isacharbraceright}\ {\isasymunion}\ S} pertenece a \isa{C}.
     \item Para toda fórmula de tipo \isa{{\isasymbeta}} con componentes \isa{{\isasymbeta}\isactrlsub {\isadigit{1}}} y \isa{{\isasymbeta}\isactrlsub {\isadigit{2}}} tal que \isa{{\isasymbeta}}
     pertenece a \isa{S}, se tiene que o bien \isa{{\isacharbraceleft}{\isasymbeta}\isactrlsub {\isadigit{1}}{\isacharbraceright}\ {\isasymunion}\ S} pertenece a \isa{C} o 
     bien \isa{{\isacharbraceleft}{\isasymbeta}\isactrlsub {\isadigit{2}}{\isacharbraceright}\ {\isasymunion}\ S} pertenece a \isa{C}.   
  \end{itemize} 

  En primer lugar, vamos a deducir el primer resultado correspondiente a las fórmulas
  de tipo \isa{{\isasymalpha}} de las condiciones tercera, sexta, octava y novena de la definición de 
  propiedad de consistencia proposicional. En efecto, consideremos una fórmula de tipo 
  \isa{{\isasymalpha}} cualquiera con componentes \isa{{\isasymalpha}\isactrlsub {\isadigit{1}}} y \isa{{\isasymalpha}\isactrlsub {\isadigit{2}}} tal que \isa{{\isasymalpha}} pertenece a \isa{S}. Sabemos que 
  la fórmula es de la forma \isa{G\ {\isasymand}\ H}, \isa{{\isasymnot}\ {\isacharparenleft}{\isasymnot}\ G{\isacharparenright}}, \isa{{\isasymnot}\ {\isacharparenleft}G\ {\isasymor}\ H{\isacharparenright}} o 
  \isa{{\isasymnot}{\isacharparenleft}G\ {\isasymlongrightarrow}\ H{\isacharparenright}} para ciertas fórmulas \isa{G} y \isa{H}. Vamos a probar que para cada caso se cumple que 
  \isa{{\isacharbraceleft}{\isasymalpha}\isactrlsub {\isadigit{1}}{\isacharcomma}\ {\isasymalpha}\isactrlsub {\isadigit{2}}{\isacharbraceright}\ {\isasymunion}\ S} pertenece a la colección:

  \isa{{\isasymsqdot}\ Fórmula\ de\ tipo\ G\ {\isasymand}\ H}: En este caso, sus componentes conjuntivas \isa{{\isasymalpha}\isactrlsub {\isadigit{1}}} y \isa{{\isasymalpha}\isactrlsub {\isadigit{2}}} son \isa{G} 
    y \isa{H} respectivamente. Luego tenemos que \isa{{\isacharbraceleft}{\isasymalpha}\isactrlsub {\isadigit{1}}{\isacharcomma}\ {\isasymalpha}\isactrlsub {\isadigit{2}}{\isacharbraceright}\ {\isasymunion}\ S}  pertenece a \isa{C} por
    la tercera condición de la definición de propiedad de consistencia
    proposicional.

  \isa{{\isasymsqdot}\ Fórmula\ de\ tipo\ {\isasymnot}\ {\isacharparenleft}{\isasymnot}\ G{\isacharparenright}}: En este caso, sus componentes conjuntivas \isa{{\isasymalpha}\isactrlsub {\isadigit{1}}} y \isa{{\isasymalpha}\isactrlsub {\isadigit{2}}} son 
    ambas \isa{G}. Como el conjunto \isa{{\isacharbraceleft}{\isasymalpha}\isactrlsub {\isadigit{1}}{\isacharbraceright}\ {\isasymunion}\ S} es equivalente a \isa{{\isacharbraceleft}{\isasymalpha}\isactrlsub {\isadigit{1}}{\isacharcomma}\ {\isasymalpha}\isactrlsub {\isadigit{2}}{\isacharbraceright}\ {\isasymunion}\ S} ya
    que \isa{{\isasymalpha}\isactrlsub {\isadigit{1}}} y \isa{{\isasymalpha}\isactrlsub {\isadigit{2}}} son iguales, tenemos que este último pertenece a \isa{C} por la sexta 
    condición de la definición de propiedad de consistencia proposicional.

  \isa{{\isasymsqdot}\ Fórmula\ de\ tipo\ {\isasymnot}{\isacharparenleft}G\ {\isasymor}\ H{\isacharparenright}}: En este caso, sus componentes conjuntivas \isa{{\isasymalpha}\isactrlsub {\isadigit{1}}} y \isa{{\isasymalpha}\isactrlsub {\isadigit{2}}} son\\ 
    \isa{{\isasymnot}\ G} y \isa{{\isasymnot}\ H} respectivamente. Luego tenemos que \isa{{\isacharbraceleft}{\isasymalpha}\isactrlsub {\isadigit{1}}{\isacharcomma}\ {\isasymalpha}\isactrlsub {\isadigit{2}}{\isacharbraceright}\ {\isasymunion}\ S}  pertenece a \isa{C} por
    la octava condición de la definición de propiedad de consistencia proposicional.

  \isa{{\isasymsqdot}\ Fórmula\ de\ tipo\ {\isasymnot}{\isacharparenleft}G\ {\isasymlongrightarrow}\ H{\isacharparenright}}: En este caso, sus componentes conjuntivas \isa{{\isasymalpha}\isactrlsub {\isadigit{1}}} y \isa{{\isasymalpha}\isactrlsub {\isadigit{2}}} son \isa{G} y 
    \isa{{\isasymnot}\ H} respectivamente. Luego tenemos que \isa{{\isacharbraceleft}{\isasymalpha}\isactrlsub {\isadigit{1}}{\isacharcomma}\ {\isasymalpha}\isactrlsub {\isadigit{2}}{\isacharbraceright}\ {\isasymunion}\ S}  pertenece a \isa{C} por la novena 
    condición de la definición de propiedad de consistencia proposicional.

  Finalmente, el resultado correspondiente a las fórmulas de tipo \isa{{\isasymbeta}} se obtiene de las 
  condiciones cuarta, quinta, sexta y séptima de la definición de propiedad de consistencia 
  proposicional. Para probarlo, consideremos una fórmula cualquiera de tipo \isa{{\isasymbeta}} perteneciente
  al conjunto \isa{S} y cuyas componentes disyuntivas son \isa{{\isasymbeta}\isactrlsub {\isadigit{1}}} y \isa{{\isasymbeta}\isactrlsub {\isadigit{2}}}. Por simplificación, sabemos 
  que dicha fórmula es de la forma \isa{G\ {\isasymor}\ H}, \isa{G\ {\isasymlongrightarrow}\ H}, \isa{{\isasymnot}\ {\isacharparenleft}{\isasymnot}\ G{\isacharparenright}} o \isa{{\isasymnot}{\isacharparenleft}G\ {\isasymand}\ H{\isacharparenright}} para ciertas 
  fórmulas \isa{G} y \isa{H}. Deduzcamos que, en efecto, tenemos que o bien \isa{{\isacharbraceleft}{\isasymbeta}\isactrlsub {\isadigit{1}}{\isacharbraceright}\ {\isasymunion}\ S} está en \isa{C} o bien 
  \isa{{\isacharbraceleft}{\isasymbeta}\isactrlsub {\isadigit{2}}{\isacharbraceright}\ {\isasymunion}\ S} está en \isa{C}.

  \isa{{\isasymsqdot}\ Fórmula\ de\ tipo\ G\ {\isasymor}\ H}: En este caso, sus componentes disyuntivas \isa{{\isasymbeta}\isactrlsub {\isadigit{1}}} y \isa{{\isasymbeta}\isactrlsub {\isadigit{2}}} son \isa{G} y 
    \isa{H} respectivamente. Luego tenemos que o bien \isa{{\isacharbraceleft}{\isasymbeta}\isactrlsub {\isadigit{1}}{\isacharbraceright}\ {\isasymunion}\ S}  pertenece a \isa{C} o bien\\
    \isa{{\isacharbraceleft}{\isasymbeta}\isactrlsub {\isadigit{2}}{\isacharbraceright}\ {\isasymunion}\ S} pertenece a \isa{C} por la cuarta condición de la definición de propiedad de 
    consistencia proposicional.

  \isa{{\isasymsqdot}\ Fórmula\ de\ tipo\ G\ {\isasymlongrightarrow}\ H}: En este caso, sus componentes disyuntivas \isa{{\isasymbeta}\isactrlsub {\isadigit{1}}} y \isa{{\isasymbeta}\isactrlsub {\isadigit{2}}} son\\ 
    \isa{{\isasymnot}\ G} y \isa{H} respectivamente. Luego tenemos que o bien \isa{{\isacharbraceleft}{\isasymbeta}\isactrlsub {\isadigit{1}}{\isacharbraceright}\ {\isasymunion}\ S}  pertenece a \isa{C} o 
    bien\\ \isa{{\isacharbraceleft}{\isasymbeta}\isactrlsub {\isadigit{2}}{\isacharbraceright}\ {\isasymunion}\ S} pertenece a \isa{C} por la quinta condición de la definición de propiedad 
    de consistencia proposicional.

  \isa{{\isasymsqdot}\ Fórmula\ de\ tipo\ {\isasymnot}{\isacharparenleft}{\isasymnot}\ G{\isacharparenright}}: En este caso, sus componentes disyuntivas \isa{{\isasymbeta}\isactrlsub {\isadigit{1}}} y \isa{{\isasymbeta}\isactrlsub {\isadigit{2}}} son ambas 
    \isa{G}. Luego tenemos que, en particular, el conjunto \isa{{\isacharbraceleft}{\isasymbeta}\isactrlsub {\isadigit{1}}{\isacharbraceright}\ {\isasymunion}\ S} pertenece a \isa{C} por la 
    sexta condición de la definición de propiedad de consistencia proposicional. Por tanto, se 
    verifica que o bien \isa{{\isacharbraceleft}{\isasymbeta}\isactrlsub {\isadigit{1}}{\isacharbraceright}\ {\isasymunion}\ S} está en \isa{C} o bien \isa{{\isacharbraceleft}{\isasymbeta}\isactrlsub {\isadigit{2}}{\isacharbraceright}\ {\isasymunion}\ S} está en \isa{C}.

  \isa{{\isasymsqdot}\ Fórmula\ de\ tipo\ {\isasymnot}{\isacharparenleft}G\ {\isasymand}\ H{\isacharparenright}}: En este caso, sus componentes disyuntivas \isa{{\isasymbeta}\isactrlsub {\isadigit{1}}} y \isa{{\isasymbeta}\isactrlsub {\isadigit{2}}} son \\ 
    \isa{{\isasymnot}\ G} y \isa{{\isasymnot}\ H} respectivamente. Luego tenemos que o bien \isa{{\isacharbraceleft}{\isasymbeta}\isactrlsub {\isadigit{1}}{\isacharbraceright}\ {\isasymunion}\ S} pertenece a \isa{C} o 
    bien \isa{{\isacharbraceleft}{\isasymbeta}\isactrlsub {\isadigit{2}}{\isacharbraceright}\ {\isasymunion}\ S} pertenece a \isa{C} por la séptima condición de la definición de propiedad 
    de consistencia proposicional.

  De este modo, queda probada la primera implicación de la equivalencia. Veamos la prueba de la 
  implicación contraria.

\textbf{\isa{{\isadigit{2}}{\isacharparenright}\ {\isasymLongrightarrow}\ {\isadigit{1}}{\isacharparenright}}}

  Supongamos que, dada una colección de conjuntos de fórmulas proposicionales \isa{C}, para cualquier
  conjunto \isa{S} de la colección se verifica:
  \begin{itemize}
    \item \isa{{\isasymbottom}} no pertenece a \isa{S}.
    \item Dada \isa{p} una fórmula atómica cualquiera, no se tiene 
    simultáneamente que\\ \isa{p\ {\isasymin}\ S} y \isa{{\isasymnot}\ p\ {\isasymin}\ S}.
    \item Para toda fórmula de tipo \isa{{\isasymalpha}} con componentes \isa{{\isasymalpha}\isactrlsub {\isadigit{1}}} y \isa{{\isasymalpha}\isactrlsub {\isadigit{2}}} tal que \isa{{\isasymalpha}}
    pertenece a \isa{S}, se tiene que \isa{{\isacharbraceleft}{\isasymalpha}\isactrlsub {\isadigit{1}}{\isacharcomma}{\isasymalpha}\isactrlsub {\isadigit{2}}{\isacharbraceright}\ {\isasymunion}\ S} pertenece a \isa{C}.
    \item Para toda fórmula de tipo \isa{{\isasymbeta}} con componentes \isa{{\isasymbeta}\isactrlsub {\isadigit{1}}} y \isa{{\isasymbeta}\isactrlsub {\isadigit{2}}} tal que \isa{{\isasymbeta}}
    pertenece a \isa{S}, se tiene que o bien \isa{{\isacharbraceleft}{\isasymbeta}\isactrlsub {\isadigit{1}}{\isacharbraceright}\ {\isasymunion}\ S} pertenece a \isa{C} o 
    bien \isa{{\isacharbraceleft}{\isasymbeta}\isactrlsub {\isadigit{2}}{\isacharbraceright}\ {\isasymunion}\ S} pertenece a \isa{C}.
  \end{itemize}

  Probemos que \isa{C} verifica la propiedad de consistencia proposicional. Por la definición
  de la propiedad basta probar que, dado un conjunto cualquiera \isa{S} perteneciente a \isa{C}, se
  verifican las siguientes condiciones:
  \begin{enumerate}
    \item \isa{{\isasymbottom}\ {\isasymnotin}\ S}.
    \item Dada \isa{p} una fórmula atómica cualquiera, no se tiene 
      simultáneamente que\\ \isa{p\ {\isasymin}\ S} y \isa{{\isasymnot}\ p\ {\isasymin}\ S}.
    \item Si \isa{G\ {\isasymand}\ H\ {\isasymin}\ S}, entonces el conjunto \isa{{\isacharbraceleft}G{\isacharcomma}H{\isacharbraceright}\ {\isasymunion}\ S} pertenece a \isa{C}.
    \item Si \isa{G\ {\isasymor}\ H\ {\isasymin}\ S}, entonces o bien el conjunto \isa{{\isacharbraceleft}G{\isacharbraceright}\ {\isasymunion}\ S} pertenece a \isa{C}, o bien el conjunto 
      \isa{{\isacharbraceleft}H{\isacharbraceright}\ {\isasymunion}\ S} pertenece a \isa{C}.
    \item Si \isa{G\ {\isasymrightarrow}\ H\ {\isasymin}\ S}, entonces o bien el conjunto \isa{{\isacharbraceleft}{\isasymnot}\ G{\isacharbraceright}\ {\isasymunion}\ S} pertenece a \isa{C}, o bien el 
      conjunto \isa{{\isacharbraceleft}H{\isacharbraceright}\ {\isasymunion}\ S} pertenece a \isa{C}.
    \item Si \isa{{\isasymnot}{\isacharparenleft}{\isasymnot}\ G{\isacharparenright}\ {\isasymin}\ S}, entonces el conjunto \isa{{\isacharbraceleft}G{\isacharbraceright}\ {\isasymunion}\ S} pertenece a \isa{C}.
    \item Si \isa{{\isasymnot}{\isacharparenleft}G\ {\isasymand}\ H{\isacharparenright}\ {\isasymin}\ S}, entonces o bien el conjunto \isa{{\isacharbraceleft}{\isasymnot}\ G{\isacharbraceright}\ {\isasymunion}\ S} pertenece a \isa{C}, o bien el 
      conjunto \isa{{\isacharbraceleft}{\isasymnot}\ H{\isacharbraceright}\ {\isasymunion}\ S} pertenece a \isa{C}.
    \item Si \isa{{\isasymnot}{\isacharparenleft}G\ {\isasymor}\ H{\isacharparenright}\ {\isasymin}\ S}, entonces el conjunto \isa{{\isacharbraceleft}{\isasymnot}\ G{\isacharcomma}\ {\isasymnot}\ H{\isacharbraceright}\ {\isasymunion}\ S} pertenece a \isa{C}.
    \item Si \isa{{\isasymnot}{\isacharparenleft}G\ {\isasymrightarrow}\ H{\isacharparenright}\ {\isasymin}\ S}, entonces el conjunto \isa{{\isacharbraceleft}G{\isacharcomma}\ {\isasymnot}\ H{\isacharbraceright}\ {\isasymunion}\ S} pertenece a \isa{C}.
  \end{enumerate}

  En primer lugar, se observa que por hipótesis se cumplen las dos primeras condiciones de
  la definición.

  Por otra parte, vamos a deducir las condiciones tercera, sexta, octava y novena de la
  definición de la propiedad de consistencia proposicional a partir de la hipótesis sobre las 
  fórmulas de tipo \isa{{\isasymalpha}}.
  \begin{enumerate}
    \item[\isa{{\isadigit{3}}{\isacharparenright}}:] Supongamos que la fórmula \isa{G\ {\isasymand}\ H} pertenece a \isa{S} para fórmulas \isa{G} y \isa{H}
    cualesquiera. Observemos que se trata de una fórmula de tipo \isa{{\isasymalpha}} de componentes conjuntivas
    \isa{G} y \isa{H}. Luego, por hipótesis, tenemos que \isa{{\isacharbraceleft}G{\isacharcomma}\ H{\isacharbraceright}\ {\isasymunion}\ S} pertenece a \isa{C}.
    \item[\isa{{\isadigit{6}}{\isacharparenright}}:] Supongamos que la fórmula \isa{{\isasymnot}{\isacharparenleft}{\isasymnot}\ G{\isacharparenright}} pertenece a \isa{S} para la fórmula \isa{G} 
    cualquiera. Observemos que se trata de una fórmula de tipo \isa{{\isasymalpha}} cuyas componentes conjuntivas 
    son ambas la fórmula \isa{G}. Por hipótesis, tenemos que el conjunto \isa{{\isacharbraceleft}G{\isacharcomma}G{\isacharbraceright}\ {\isasymunion}\ S} pertence a \isa{C}
    y, puesto que dicho conjunto es equivalente a \isa{{\isacharbraceleft}G{\isacharbraceright}\ {\isasymunion}\ S}, tenemos el resultado.
    \item[\isa{{\isadigit{8}}{\isacharparenright}}:] Supongamos que la fórmula \isa{{\isasymnot}{\isacharparenleft}G\ {\isasymor}\ H{\isacharparenright}} pertenece a \isa{S} para fórmulas \isa{G} y \isa{H}
    cualesquiera. Observemos que se trata de una fórmula de tipo \isa{{\isasymalpha}} de componentes conjuntivas
    \isa{{\isasymnot}\ G} y \isa{{\isasymnot}\ H}. Luego, por hipótesis, tenemos que \isa{{\isacharbraceleft}{\isasymnot}\ G{\isacharcomma}\ {\isasymnot}\ H{\isacharbraceright}\ {\isasymunion}\ S} pertenece a \isa{C}.
    \item[\isa{{\isadigit{9}}{\isacharparenright}}:] Supongamos que la fórmula \isa{{\isasymnot}{\isacharparenleft}G\ {\isasymlongrightarrow}\ H{\isacharparenright}} pertenece a \isa{S} para fórmulas \isa{G} y \isa{H}
    cualesquiera. Observemos que se trata de una fórmula de tipo \isa{{\isasymalpha}} de componentes conjuntivas
    \isa{G} y \isa{{\isasymnot}\ H}. Luego, por hipótesis, tenemos que \isa{{\isacharbraceleft}G{\isacharcomma}\ {\isasymnot}\ H{\isacharbraceright}\ {\isasymunion}\ S} pertenece a \isa{C}.
  \end{enumerate} 

  Finalmente, deduzcamos el resto de condiciones de la definición de propiedad de consistencia
  proposicional a partir de la hipótesis referente a las fórmulas de tipo \isa{{\isasymbeta}}.
  \begin{enumerate}
    \item[\isa{{\isadigit{4}}{\isacharparenright}}:] Supongamos que la fórmula \isa{G\ {\isasymor}\ H} pertenece a \isa{S} para fórmulas \isa{G} y \isa{H}
    cualesquiera. Observemos que se trata de una fórmula de tipo \isa{{\isasymbeta}} de componentes disyuntivas
    \isa{G} y \isa{H}. Luego, por hipótesis, tenemos que o bien \isa{{\isacharbraceleft}G{\isacharbraceright}\ {\isasymunion}\ S} pertenece a \isa{C} o bien\\
    \isa{{\isacharbraceleft}H{\isacharbraceright}\ {\isasymunion}\ S} pertenece a \isa{C}.
    \item[\isa{{\isadigit{5}}{\isacharparenright}}:] Supongamos que la fórmula \isa{G\ {\isasymlongrightarrow}\ H} pertenece a \isa{S} para fórmulas \isa{G} y \isa{H}
    cualesquiera. Observemos que se trata de una fórmula de tipo \isa{{\isasymbeta}} de componentes disyuntivas
    \isa{{\isasymnot}\ G} y \isa{H}. Luego, por hipótesis, tenemos que o bien \isa{{\isacharbraceleft}{\isasymnot}\ G{\isacharbraceright}\ {\isasymunion}\ S} pertenece a \isa{C} o
    bien \isa{{\isacharbraceleft}H{\isacharbraceright}\ {\isasymunion}\ S} pertenece a \isa{C}.
    \item[\isa{{\isadigit{7}}{\isacharparenright}}:] Supongamos que la fórmula \isa{{\isasymnot}{\isacharparenleft}G\ {\isasymand}\ H{\isacharparenright}} pertenece a \isa{S} para fórmulas \isa{G} y \isa{H}
    cualesquiera. Observemos que se trata de una fórmula de tipo \isa{{\isasymbeta}} de componentes disyuntivas
    \isa{{\isasymnot}\ G} y \isa{{\isasymnot}\ H}. Luego, por hipótesis, tenemos que o bien \isa{{\isacharbraceleft}{\isasymnot}\ G{\isacharbraceright}\ {\isasymunion}\ S} pertenece a \isa{C} o
    bien \isa{{\isacharbraceleft}{\isasymnot}\ H{\isacharbraceright}\ {\isasymunion}\ S} pertenece \isa{C}.
  \end{enumerate} 

  De este modo, hemos probado a partir de la hipótesis todas las condiciones que garantizan que la
  colección \isa{C} cumple la propiedad de consistencia proposicional. Por lo tanto, queda demostrado el
  resultado.
\end{demostracion}

  Para probar este resultado de manera detallada en Isabelle vamos a demostrar cada una de las 
  implicaciones de la equivalencia por separado. La primera implicación del lema se basa en dos 
  lemas auxiliares. El primero de ellos deduce la condición de \isa{{\isadigit{2}}{\isacharparenright}} sobre fórmulas de tipo \isa{{\isasymalpha}} a 
  partir de las condiciones tercera, sexta, octava y novena de la definición de propiedad de 
  consistencia proposicional. Su demostración detallada en Isabelle se muestra a continuación.%
\end{isamarkuptext}\isamarkuptrue%
\isacommand{lemma}\isamarkupfalse%
\ pcp{\isacharunderscore}alt{\isadigit{1}}Con{\isacharcolon}\isanewline
\ \ \isakeyword{assumes}\ {\isachardoublequoteopen}{\isacharparenleft}{\isasymforall}G\ H{\isachardot}\ G\ \isactrlbold {\isasymand}\ H\ {\isasymin}\ S\ {\isasymlongrightarrow}\ {\isacharbraceleft}G{\isacharcomma}H{\isacharbraceright}\ {\isasymunion}\ S\ {\isasymin}\ C{\isacharparenright}\isanewline
\ \ {\isasymand}\ {\isacharparenleft}{\isasymforall}G{\isachardot}\ \isactrlbold {\isasymnot}\ {\isacharparenleft}\isactrlbold {\isasymnot}G{\isacharparenright}\ {\isasymin}\ S\ {\isasymlongrightarrow}\ {\isacharbraceleft}G{\isacharbraceright}\ {\isasymunion}\ S\ {\isasymin}\ C{\isacharparenright}\isanewline
\ \ {\isasymand}\ {\isacharparenleft}{\isasymforall}G\ H{\isachardot}\ \isactrlbold {\isasymnot}{\isacharparenleft}G\ \isactrlbold {\isasymor}\ H{\isacharparenright}\ {\isasymin}\ S\ {\isasymlongrightarrow}\ {\isacharbraceleft}\isactrlbold {\isasymnot}\ G{\isacharcomma}\ \isactrlbold {\isasymnot}\ H{\isacharbraceright}\ {\isasymunion}\ S\ {\isasymin}\ C{\isacharparenright}\isanewline
\ \ {\isasymand}\ {\isacharparenleft}{\isasymforall}G\ H{\isachardot}\ \isactrlbold {\isasymnot}{\isacharparenleft}G\ \isactrlbold {\isasymrightarrow}\ H{\isacharparenright}\ {\isasymin}\ S\ {\isasymlongrightarrow}\ {\isacharbraceleft}G{\isacharcomma}\isactrlbold {\isasymnot}\ H{\isacharbraceright}\ {\isasymunion}\ S\ {\isasymin}\ C{\isacharparenright}{\isachardoublequoteclose}\isanewline
\ \ \isakeyword{shows}\ {\isachardoublequoteopen}{\isasymforall}F\ G\ H{\isachardot}\ Con\ F\ G\ H\ {\isasymlongrightarrow}\ F\ {\isasymin}\ S\ {\isasymlongrightarrow}\ {\isacharbraceleft}G{\isacharcomma}H{\isacharbraceright}\ {\isasymunion}\ S\ {\isasymin}\ C{\isachardoublequoteclose}\isanewline
%
\isadelimproof
%
\endisadelimproof
%
\isatagproof
\isacommand{proof}\isamarkupfalse%
\ {\isacharminus}\isanewline
\ \ \isacommand{have}\isamarkupfalse%
\ C{\isadigit{1}}{\isacharcolon}{\isachardoublequoteopen}{\isasymforall}G\ H{\isachardot}\ G\ \isactrlbold {\isasymand}\ H\ {\isasymin}\ S\ {\isasymlongrightarrow}\ {\isacharbraceleft}G{\isacharcomma}H{\isacharbraceright}\ {\isasymunion}\ S\ {\isasymin}\ C{\isachardoublequoteclose}\isanewline
\ \ \ \ \isacommand{using}\isamarkupfalse%
\ assms\ \isacommand{by}\isamarkupfalse%
\ {\isacharparenleft}rule\ conjunct{\isadigit{1}}{\isacharparenright}\isanewline
\ \ \isacommand{have}\isamarkupfalse%
\ C{\isadigit{2}}{\isacharcolon}{\isachardoublequoteopen}{\isasymforall}G{\isachardot}\ \isactrlbold {\isasymnot}\ {\isacharparenleft}\isactrlbold {\isasymnot}G{\isacharparenright}\ {\isasymin}\ S\ {\isasymlongrightarrow}\ {\isacharbraceleft}G{\isacharbraceright}\ {\isasymunion}\ S\ {\isasymin}\ C{\isachardoublequoteclose}\isanewline
\ \ \ \ \isacommand{using}\isamarkupfalse%
\ assms\ \isacommand{by}\isamarkupfalse%
\ {\isacharparenleft}iprover\ elim{\isacharcolon}\ conjunct{\isadigit{2}}\ conjunct{\isadigit{1}}{\isacharparenright}\isanewline
\ \ \isacommand{have}\isamarkupfalse%
\ C{\isadigit{3}}{\isacharcolon}{\isachardoublequoteopen}{\isasymforall}G\ H{\isachardot}\ \isactrlbold {\isasymnot}{\isacharparenleft}G\ \isactrlbold {\isasymor}\ H{\isacharparenright}\ {\isasymin}\ S\ {\isasymlongrightarrow}\ {\isacharbraceleft}\isactrlbold {\isasymnot}\ G{\isacharcomma}\ \isactrlbold {\isasymnot}\ H{\isacharbraceright}\ {\isasymunion}\ S\ {\isasymin}\ C{\isachardoublequoteclose}\isanewline
\ \ \ \ \isacommand{using}\isamarkupfalse%
\ assms\ \isacommand{by}\isamarkupfalse%
\ {\isacharparenleft}iprover\ elim{\isacharcolon}\ conjunct{\isadigit{2}}\ conjunct{\isadigit{1}}{\isacharparenright}\isanewline
\ \ \isacommand{have}\isamarkupfalse%
\ C{\isadigit{4}}{\isacharcolon}{\isachardoublequoteopen}{\isasymforall}G\ H{\isachardot}\ \isactrlbold {\isasymnot}{\isacharparenleft}G\ \isactrlbold {\isasymrightarrow}\ H{\isacharparenright}\ {\isasymin}\ S\ {\isasymlongrightarrow}\ {\isacharbraceleft}G{\isacharcomma}\isactrlbold {\isasymnot}\ H{\isacharbraceright}\ {\isasymunion}\ S\ {\isasymin}\ C{\isachardoublequoteclose}\isanewline
\ \ \ \ \isacommand{using}\isamarkupfalse%
\ assms\ \isacommand{by}\isamarkupfalse%
\ {\isacharparenleft}iprover\ elim{\isacharcolon}\ conjunct{\isadigit{2}}{\isacharparenright}\ \isanewline
\ \ \isacommand{show}\isamarkupfalse%
\ {\isachardoublequoteopen}{\isasymforall}F\ G\ H{\isachardot}\ Con\ F\ G\ H\ {\isasymlongrightarrow}\ F\ {\isasymin}\ S\ {\isasymlongrightarrow}\ {\isacharbraceleft}G{\isacharcomma}H{\isacharbraceright}\ {\isasymunion}\ S\ {\isasymin}\ C{\isachardoublequoteclose}\isanewline
\ \ \isacommand{proof}\isamarkupfalse%
\ {\isacharparenleft}rule\ allI{\isacharparenright}{\isacharplus}\isanewline
\ \ \ \ \isacommand{fix}\isamarkupfalse%
\ F\ G\ H\isanewline
\ \ \ \ \isacommand{show}\isamarkupfalse%
\ {\isachardoublequoteopen}Con\ F\ G\ H\ {\isasymlongrightarrow}\ F\ {\isasymin}\ S\ {\isasymlongrightarrow}\ {\isacharbraceleft}G{\isacharcomma}H{\isacharbraceright}\ {\isasymunion}\ S\ {\isasymin}\ C{\isachardoublequoteclose}\isanewline
\ \ \ \ \isacommand{proof}\isamarkupfalse%
\ {\isacharparenleft}rule\ impI{\isacharparenright}\isanewline
\ \ \ \ \ \ \isacommand{assume}\isamarkupfalse%
\ {\isachardoublequoteopen}Con\ F\ G\ H{\isachardoublequoteclose}\isanewline
\ \ \ \ \ \ \isacommand{then}\isamarkupfalse%
\ \isacommand{have}\isamarkupfalse%
\ {\isachardoublequoteopen}F\ {\isacharequal}\ G\ \isactrlbold {\isasymand}\ H\ {\isasymor}\ \isanewline
\ \ \ \ \ \ \ \ \ \ \ \ \ \ \ \ {\isacharparenleft}{\isacharparenleft}{\isasymexists}G{\isadigit{1}}\ H{\isadigit{1}}{\isachardot}\ F\ {\isacharequal}\ \isactrlbold {\isasymnot}\ {\isacharparenleft}G{\isadigit{1}}\ \isactrlbold {\isasymor}\ H{\isadigit{1}}{\isacharparenright}\ {\isasymand}\ G\ {\isacharequal}\ \isactrlbold {\isasymnot}\ G{\isadigit{1}}\ {\isasymand}\ H\ {\isacharequal}\ \isactrlbold {\isasymnot}\ H{\isadigit{1}}{\isacharparenright}\ {\isasymor}\ \isanewline
\ \ \ \ \ \ \ \ \ \ \ \ \ \ \ \ {\isacharparenleft}{\isasymexists}H{\isadigit{2}}{\isachardot}\ F\ {\isacharequal}\ \isactrlbold {\isasymnot}\ {\isacharparenleft}G\ \isactrlbold {\isasymrightarrow}\ H{\isadigit{2}}{\isacharparenright}\ {\isasymand}\ H\ {\isacharequal}\ \isactrlbold {\isasymnot}\ H{\isadigit{2}}{\isacharparenright}\ {\isasymor}\ \isanewline
\ \ \ \ \ \ \ \ \ \ \ \ \ \ \ \ F\ {\isacharequal}\ \isactrlbold {\isasymnot}\ {\isacharparenleft}\isactrlbold {\isasymnot}\ G{\isacharparenright}\ {\isasymand}\ H\ {\isacharequal}\ G{\isacharparenright}{\isachardoublequoteclose}\isanewline
\ \ \ \ \ \ \ \ \isacommand{by}\isamarkupfalse%
\ {\isacharparenleft}simp\ only{\isacharcolon}\ con{\isacharunderscore}dis{\isacharunderscore}simps{\isacharparenleft}{\isadigit{1}}{\isacharparenright}{\isacharparenright}\isanewline
\ \ \ \ \ \ \isacommand{thus}\isamarkupfalse%
\ {\isachardoublequoteopen}F\ {\isasymin}\ S\ {\isasymlongrightarrow}\ {\isacharbraceleft}G{\isacharcomma}H{\isacharbraceright}\ {\isasymunion}\ S\ {\isasymin}\ C{\isachardoublequoteclose}\isanewline
\ \ \ \ \ \ \isacommand{proof}\isamarkupfalse%
\ {\isacharparenleft}rule\ disjE{\isacharparenright}\isanewline
\ \ \ \ \ \ \ \ \isacommand{assume}\isamarkupfalse%
\ {\isachardoublequoteopen}F\ {\isacharequal}\ G\ \isactrlbold {\isasymand}\ H{\isachardoublequoteclose}\isanewline
\ \ \ \ \ \ \ \ \isacommand{show}\isamarkupfalse%
\ {\isachardoublequoteopen}F\ {\isasymin}\ S\ {\isasymlongrightarrow}\ {\isacharbraceleft}G{\isacharcomma}H{\isacharbraceright}\ {\isasymunion}\ S\ {\isasymin}\ C{\isachardoublequoteclose}\isanewline
\ \ \ \ \ \ \ \ \ \ \isacommand{using}\isamarkupfalse%
\ C{\isadigit{1}}\ {\isacartoucheopen}F\ {\isacharequal}\ G\ \isactrlbold {\isasymand}\ H{\isacartoucheclose}\ \isacommand{by}\isamarkupfalse%
\ {\isacharparenleft}iprover\ elim{\isacharcolon}\ allE{\isacharparenright}\isanewline
\ \ \ \ \ \ \isacommand{next}\isamarkupfalse%
\isanewline
\ \ \ \ \ \ \ \ \isacommand{assume}\isamarkupfalse%
\ {\isachardoublequoteopen}{\isacharparenleft}{\isasymexists}G{\isadigit{1}}\ H{\isadigit{1}}{\isachardot}\ F\ {\isacharequal}\ \isactrlbold {\isasymnot}\ {\isacharparenleft}G{\isadigit{1}}\ \isactrlbold {\isasymor}\ H{\isadigit{1}}{\isacharparenright}\ {\isasymand}\ G\ {\isacharequal}\ \isactrlbold {\isasymnot}\ G{\isadigit{1}}\ {\isasymand}\ H\ {\isacharequal}\ \isactrlbold {\isasymnot}\ H{\isadigit{1}}{\isacharparenright}\ {\isasymor}\ \isanewline
\ \ \ \ \ \ \ \ \ \ \ \ \ \ \ \ {\isacharparenleft}{\isasymexists}H{\isadigit{2}}{\isachardot}\ F\ {\isacharequal}\ \isactrlbold {\isasymnot}\ {\isacharparenleft}G\ \isactrlbold {\isasymrightarrow}\ H{\isadigit{2}}{\isacharparenright}\ {\isasymand}\ H\ {\isacharequal}\ \isactrlbold {\isasymnot}\ H{\isadigit{2}}{\isacharparenright}\ {\isasymor}\ \isanewline
\ \ \ \ \ \ \ \ \ \ \ \ \ \ \ \ F\ {\isacharequal}\ \isactrlbold {\isasymnot}\ {\isacharparenleft}\isactrlbold {\isasymnot}\ G{\isacharparenright}\ {\isasymand}\ H\ {\isacharequal}\ G{\isachardoublequoteclose}\isanewline
\ \ \ \ \ \ \ \ \isacommand{thus}\isamarkupfalse%
\ {\isachardoublequoteopen}F\ {\isasymin}\ S\ {\isasymlongrightarrow}\ {\isacharbraceleft}G{\isacharcomma}H{\isacharbraceright}\ {\isasymunion}\ S\ {\isasymin}\ C{\isachardoublequoteclose}\isanewline
\ \ \ \ \ \ \ \ \isacommand{proof}\isamarkupfalse%
\ {\isacharparenleft}rule\ disjE{\isacharparenright}\isanewline
\ \ \ \ \ \ \ \ \ \ \isacommand{assume}\isamarkupfalse%
\ E{\isadigit{1}}{\isacharcolon}{\isachardoublequoteopen}{\isasymexists}G{\isadigit{1}}\ H{\isadigit{1}}{\isachardot}\ F\ {\isacharequal}\ \isactrlbold {\isasymnot}\ {\isacharparenleft}G{\isadigit{1}}\ \isactrlbold {\isasymor}\ H{\isadigit{1}}{\isacharparenright}\ {\isasymand}\ G\ {\isacharequal}\ \isactrlbold {\isasymnot}\ G{\isadigit{1}}\ {\isasymand}\ H\ {\isacharequal}\ \isactrlbold {\isasymnot}\ H{\isadigit{1}}{\isachardoublequoteclose}\isanewline
\ \ \ \ \ \ \ \ \ \ \isacommand{obtain}\isamarkupfalse%
\ G{\isadigit{1}}\ H{\isadigit{1}}\ \isakeyword{where}\ A{\isadigit{1}}{\isacharcolon}{\isachardoublequoteopen}F\ {\isacharequal}\ \isactrlbold {\isasymnot}\ {\isacharparenleft}G{\isadigit{1}}\ \isactrlbold {\isasymor}\ H{\isadigit{1}}{\isacharparenright}\ {\isasymand}\ G\ {\isacharequal}\ \isactrlbold {\isasymnot}\ G{\isadigit{1}}\ {\isasymand}\ H\ {\isacharequal}\ \isactrlbold {\isasymnot}\ H{\isadigit{1}}{\isachardoublequoteclose}\isanewline
\ \ \ \ \ \ \ \ \ \ \ \ \isacommand{using}\isamarkupfalse%
\ E{\isadigit{1}}\ \isacommand{by}\isamarkupfalse%
\ {\isacharparenleft}iprover\ elim{\isacharcolon}\ exE{\isacharparenright}\isanewline
\ \ \ \ \ \ \ \ \ \ \isacommand{have}\isamarkupfalse%
\ {\isachardoublequoteopen}F\ {\isacharequal}\ \isactrlbold {\isasymnot}\ {\isacharparenleft}G{\isadigit{1}}\ \isactrlbold {\isasymor}\ H{\isadigit{1}}{\isacharparenright}{\isachardoublequoteclose}\isanewline
\ \ \ \ \ \ \ \ \ \ \ \ \isacommand{using}\isamarkupfalse%
\ A{\isadigit{1}}\ \isacommand{by}\isamarkupfalse%
\ {\isacharparenleft}rule\ conjunct{\isadigit{1}}{\isacharparenright}\isanewline
\ \ \ \ \ \ \ \ \ \ \isacommand{have}\isamarkupfalse%
\ {\isachardoublequoteopen}G\ {\isacharequal}\ \isactrlbold {\isasymnot}\ G{\isadigit{1}}{\isachardoublequoteclose}\isanewline
\ \ \ \ \ \ \ \ \ \ \ \ \isacommand{using}\isamarkupfalse%
\ A{\isadigit{1}}\ \isacommand{by}\isamarkupfalse%
\ {\isacharparenleft}iprover\ elim{\isacharcolon}\ conjunct{\isadigit{2}}\ conjunct{\isadigit{1}}{\isacharparenright}\isanewline
\ \ \ \ \ \ \ \ \ \ \isacommand{have}\isamarkupfalse%
\ {\isachardoublequoteopen}H\ {\isacharequal}\ \isactrlbold {\isasymnot}\ H{\isadigit{1}}{\isachardoublequoteclose}\isanewline
\ \ \ \ \ \ \ \ \ \ \ \ \isacommand{using}\isamarkupfalse%
\ A{\isadigit{1}}\ \isacommand{by}\isamarkupfalse%
\ {\isacharparenleft}iprover\ elim{\isacharcolon}\ conjunct{\isadigit{2}}{\isacharparenright}\isanewline
\ \ \ \ \ \ \ \ \ \ \isacommand{show}\isamarkupfalse%
\ {\isachardoublequoteopen}F\ {\isasymin}\ S\ {\isasymlongrightarrow}\ {\isacharbraceleft}G{\isacharcomma}H{\isacharbraceright}\ {\isasymunion}\ S\ {\isasymin}\ C{\isachardoublequoteclose}\isanewline
\ \ \ \ \ \ \ \ \ \ \ \ \isacommand{using}\isamarkupfalse%
\ C{\isadigit{3}}\ {\isacartoucheopen}F\ {\isacharequal}\ \isactrlbold {\isasymnot}\ {\isacharparenleft}G{\isadigit{1}}\ \isactrlbold {\isasymor}\ H{\isadigit{1}}{\isacharparenright}{\isacartoucheclose}\ {\isacartoucheopen}G\ {\isacharequal}\ \isactrlbold {\isasymnot}\ G{\isadigit{1}}{\isacartoucheclose}\ {\isacartoucheopen}H\ {\isacharequal}\ \isactrlbold {\isasymnot}\ H{\isadigit{1}}{\isacartoucheclose}\ \isacommand{by}\isamarkupfalse%
\ {\isacharparenleft}iprover\ elim{\isacharcolon}\ allE{\isacharparenright}\isanewline
\ \ \ \ \ \ \ \ \isacommand{next}\isamarkupfalse%
\isanewline
\ \ \ \ \ \ \ \ \ \ \isacommand{assume}\isamarkupfalse%
\ {\isachardoublequoteopen}{\isacharparenleft}{\isasymexists}H{\isadigit{2}}{\isachardot}\ F\ {\isacharequal}\ \isactrlbold {\isasymnot}\ {\isacharparenleft}G\ \isactrlbold {\isasymrightarrow}\ H{\isadigit{2}}{\isacharparenright}\ {\isasymand}\ H\ {\isacharequal}\ \isactrlbold {\isasymnot}\ H{\isadigit{2}}{\isacharparenright}\ {\isasymor}\ \isanewline
\ \ \ \ \ \ \ \ \ \ \ \ \ \ \ \ \ \ \ F\ {\isacharequal}\ \isactrlbold {\isasymnot}\ {\isacharparenleft}\isactrlbold {\isasymnot}\ G{\isacharparenright}\ {\isasymand}\ H\ {\isacharequal}\ G{\isachardoublequoteclose}\ \isanewline
\ \ \ \ \ \ \ \ \ \ \isacommand{thus}\isamarkupfalse%
\ {\isachardoublequoteopen}F\ {\isasymin}\ S\ {\isasymlongrightarrow}\ {\isacharbraceleft}G{\isacharcomma}H{\isacharbraceright}\ {\isasymunion}\ S\ {\isasymin}\ C{\isachardoublequoteclose}\isanewline
\ \ \ \ \ \ \ \ \ \ \isacommand{proof}\isamarkupfalse%
\ {\isacharparenleft}rule\ disjE{\isacharparenright}\isanewline
\ \ \ \ \ \ \ \ \ \ \ \ \isacommand{assume}\isamarkupfalse%
\ E{\isadigit{2}}{\isacharcolon}{\isachardoublequoteopen}{\isasymexists}H{\isadigit{2}}{\isachardot}\ F\ {\isacharequal}\ \isactrlbold {\isasymnot}\ {\isacharparenleft}G\ \isactrlbold {\isasymrightarrow}\ H{\isadigit{2}}{\isacharparenright}\ {\isasymand}\ H\ {\isacharequal}\ \isactrlbold {\isasymnot}\ H{\isadigit{2}}{\isachardoublequoteclose}\isanewline
\ \ \ \ \ \ \ \ \ \ \ \ \isacommand{obtain}\isamarkupfalse%
\ H{\isadigit{2}}\ \isakeyword{where}\ A{\isadigit{2}}{\isacharcolon}{\isachardoublequoteopen}F\ {\isacharequal}\ \isactrlbold {\isasymnot}\ {\isacharparenleft}G\ \isactrlbold {\isasymrightarrow}\ H{\isadigit{2}}{\isacharparenright}\ {\isasymand}\ H\ {\isacharequal}\ \isactrlbold {\isasymnot}\ H{\isadigit{2}}{\isachardoublequoteclose}\isanewline
\ \ \ \ \ \ \ \ \ \ \ \ \ \ \isacommand{using}\isamarkupfalse%
\ E{\isadigit{2}}\ \isacommand{by}\isamarkupfalse%
\ {\isacharparenleft}rule\ exE{\isacharparenright}\isanewline
\ \ \ \ \ \ \ \ \ \ \ \ \isacommand{have}\isamarkupfalse%
\ {\isachardoublequoteopen}F\ {\isacharequal}\ \isactrlbold {\isasymnot}\ {\isacharparenleft}G\ \isactrlbold {\isasymrightarrow}\ H{\isadigit{2}}{\isacharparenright}{\isachardoublequoteclose}\isanewline
\ \ \ \ \ \ \ \ \ \ \ \ \ \ \isacommand{using}\isamarkupfalse%
\ A{\isadigit{2}}\ \isacommand{by}\isamarkupfalse%
\ {\isacharparenleft}rule\ conjunct{\isadigit{1}}{\isacharparenright}\isanewline
\ \ \ \ \ \ \ \ \ \ \ \ \isacommand{have}\isamarkupfalse%
\ {\isachardoublequoteopen}H\ {\isacharequal}\ \isactrlbold {\isasymnot}\ H{\isadigit{2}}{\isachardoublequoteclose}\isanewline
\ \ \ \ \ \ \ \ \ \ \ \ \ \ \isacommand{using}\isamarkupfalse%
\ A{\isadigit{2}}\ \isacommand{by}\isamarkupfalse%
\ {\isacharparenleft}rule\ conjunct{\isadigit{2}}{\isacharparenright}\isanewline
\ \ \ \ \ \ \ \ \ \ \ \ \isacommand{show}\isamarkupfalse%
\ {\isachardoublequoteopen}F\ {\isasymin}\ S\ {\isasymlongrightarrow}\ {\isacharbraceleft}G{\isacharcomma}H{\isacharbraceright}\ {\isasymunion}\ S\ {\isasymin}\ C{\isachardoublequoteclose}\isanewline
\ \ \ \ \ \ \ \ \ \ \ \ \ \ \isacommand{using}\isamarkupfalse%
\ C{\isadigit{4}}\ {\isacartoucheopen}F\ {\isacharequal}\ \isactrlbold {\isasymnot}\ {\isacharparenleft}G\ \isactrlbold {\isasymrightarrow}\ H{\isadigit{2}}{\isacharparenright}{\isacartoucheclose}\ {\isacartoucheopen}H\ {\isacharequal}\ \isactrlbold {\isasymnot}\ H{\isadigit{2}}{\isacartoucheclose}\ \isacommand{by}\isamarkupfalse%
\ {\isacharparenleft}iprover\ elim{\isacharcolon}\ allE{\isacharparenright}\isanewline
\ \ \ \ \ \ \ \ \ \ \isacommand{next}\isamarkupfalse%
\isanewline
\ \ \ \ \ \ \ \ \ \ \ \ \isacommand{assume}\isamarkupfalse%
\ A{\isadigit{3}}{\isacharcolon}{\isachardoublequoteopen}F\ {\isacharequal}\ \isactrlbold {\isasymnot}{\isacharparenleft}\isactrlbold {\isasymnot}\ G{\isacharparenright}\ {\isasymand}\ H\ {\isacharequal}\ G{\isachardoublequoteclose}\isanewline
\ \ \ \ \ \ \ \ \ \ \ \ \isacommand{then}\isamarkupfalse%
\ \isacommand{have}\isamarkupfalse%
\ {\isachardoublequoteopen}F\ {\isacharequal}\ \isactrlbold {\isasymnot}{\isacharparenleft}\isactrlbold {\isasymnot}\ G{\isacharparenright}{\isachardoublequoteclose}\isanewline
\ \ \ \ \ \ \ \ \ \ \ \ \ \ \isacommand{by}\isamarkupfalse%
\ {\isacharparenleft}rule\ conjunct{\isadigit{1}}{\isacharparenright}\isanewline
\ \ \ \ \ \ \ \ \ \ \ \ \isacommand{have}\isamarkupfalse%
\ {\isachardoublequoteopen}H\ {\isacharequal}\ G{\isachardoublequoteclose}\isanewline
\ \ \ \ \ \ \ \ \ \ \ \ \ \ \isacommand{using}\isamarkupfalse%
\ A{\isadigit{3}}\ \isacommand{by}\isamarkupfalse%
\ {\isacharparenleft}rule\ conjunct{\isadigit{2}}{\isacharparenright}\isanewline
\ \ \ \ \ \ \ \ \ \ \ \ \isacommand{have}\isamarkupfalse%
\ {\isachardoublequoteopen}F\ {\isasymin}\ S\ {\isasymlongrightarrow}\ {\isacharbraceleft}G{\isacharbraceright}\ {\isasymunion}\ S\ {\isasymin}\ C{\isachardoublequoteclose}\isanewline
\ \ \ \ \ \ \ \ \ \ \ \ \ \ \isacommand{using}\isamarkupfalse%
\ C{\isadigit{2}}\ {\isacartoucheopen}F\ {\isacharequal}\ \isactrlbold {\isasymnot}{\isacharparenleft}\isactrlbold {\isasymnot}\ G{\isacharparenright}{\isacartoucheclose}\ \isacommand{by}\isamarkupfalse%
\ {\isacharparenleft}iprover\ elim{\isacharcolon}\ allE{\isacharparenright}\isanewline
\ \ \ \ \ \ \ \ \ \ \ \ \isacommand{then}\isamarkupfalse%
\ \isacommand{have}\isamarkupfalse%
\ {\isachardoublequoteopen}F\ {\isasymin}\ S\ {\isasymlongrightarrow}\ {\isacharbraceleft}G{\isacharcomma}G{\isacharbraceright}\ {\isasymunion}\ S\ {\isasymin}\ C{\isachardoublequoteclose}\isanewline
\ \ \ \ \ \ \ \ \ \ \ \ \ \ \isacommand{by}\isamarkupfalse%
\ {\isacharparenleft}simp\ only{\isacharcolon}\ insert{\isacharunderscore}absorb{\isadigit{2}}{\isacharparenright}\isanewline
\ \ \ \ \ \ \ \ \ \ \ \ \isacommand{thus}\isamarkupfalse%
\ {\isachardoublequoteopen}F\ {\isasymin}\ S\ {\isasymlongrightarrow}\ {\isacharbraceleft}G{\isacharcomma}H{\isacharbraceright}\ {\isasymunion}\ S\ {\isasymin}\ C{\isachardoublequoteclose}\ \isanewline
\ \ \ \ \ \ \ \ \ \ \ \ \ \ \isacommand{by}\isamarkupfalse%
\ {\isacharparenleft}simp\ only{\isacharcolon}\ {\isacartoucheopen}H\ {\isacharequal}\ G{\isacartoucheclose}{\isacharparenright}\isanewline
\ \ \ \ \ \ \ \ \ \ \isacommand{qed}\isamarkupfalse%
\isanewline
\ \ \ \ \ \ \ \ \isacommand{qed}\isamarkupfalse%
\isanewline
\ \ \ \ \ \ \isacommand{qed}\isamarkupfalse%
\isanewline
\ \ \ \ \isacommand{qed}\isamarkupfalse%
\isanewline
\ \ \isacommand{qed}\isamarkupfalse%
\isanewline
\isacommand{qed}\isamarkupfalse%
%
\endisatagproof
{\isafoldproof}%
%
\isadelimproof
%
\endisadelimproof
%
\begin{isamarkuptext}%
Finalmente, el siguiente lema auxiliar deduce la condición de \isa{{\isadigit{2}}{\isacharparenright}} sobre fórmulas de tipo \isa{{\isasymbeta}} 
  a partir de las condiciones cuarta, quinta, sexta y séptima de la definición de propiedad de 
  consistencia proposicional.%
\end{isamarkuptext}\isamarkuptrue%
\isacommand{lemma}\isamarkupfalse%
\ pcp{\isacharunderscore}alt{\isadigit{1}}Dis{\isacharcolon}\isanewline
\ \ \isakeyword{assumes}\ {\isachardoublequoteopen}{\isacharparenleft}{\isasymforall}G\ H{\isachardot}\ G\ \isactrlbold {\isasymor}\ H\ {\isasymin}\ S\ {\isasymlongrightarrow}\ {\isacharbraceleft}G{\isacharbraceright}\ {\isasymunion}\ S\ {\isasymin}\ C\ {\isasymor}\ {\isacharbraceleft}H{\isacharbraceright}\ {\isasymunion}\ S\ {\isasymin}\ C{\isacharparenright}\isanewline
\ \ {\isasymand}\ {\isacharparenleft}{\isasymforall}G\ H{\isachardot}\ G\ \isactrlbold {\isasymrightarrow}\ H\ {\isasymin}\ S\ {\isasymlongrightarrow}\ {\isacharbraceleft}\isactrlbold {\isasymnot}\ G{\isacharbraceright}\ {\isasymunion}\ S\ {\isasymin}\ C\ {\isasymor}\ {\isacharbraceleft}H{\isacharbraceright}\ {\isasymunion}\ S\ {\isasymin}\ C{\isacharparenright}\isanewline
\ \ {\isasymand}\ {\isacharparenleft}{\isasymforall}G{\isachardot}\ \isactrlbold {\isasymnot}\ {\isacharparenleft}\isactrlbold {\isasymnot}G{\isacharparenright}\ {\isasymin}\ S\ {\isasymlongrightarrow}\ {\isacharbraceleft}G{\isacharbraceright}\ {\isasymunion}\ S\ {\isasymin}\ C{\isacharparenright}\isanewline
\ \ {\isasymand}\ {\isacharparenleft}{\isasymforall}G\ H{\isachardot}\ \isactrlbold {\isasymnot}{\isacharparenleft}G\ \isactrlbold {\isasymand}\ H{\isacharparenright}\ {\isasymin}\ S\ {\isasymlongrightarrow}\ {\isacharbraceleft}\isactrlbold {\isasymnot}\ G{\isacharbraceright}\ {\isasymunion}\ S\ {\isasymin}\ C\ {\isasymor}\ {\isacharbraceleft}\isactrlbold {\isasymnot}\ H{\isacharbraceright}\ {\isasymunion}\ S\ {\isasymin}\ C{\isacharparenright}{\isachardoublequoteclose}\isanewline
\ \ \isakeyword{shows}\ {\isachardoublequoteopen}{\isasymforall}F\ G\ H{\isachardot}\ Dis\ F\ G\ H\ {\isasymlongrightarrow}\ F\ {\isasymin}\ S\ {\isasymlongrightarrow}\ {\isacharbraceleft}G{\isacharbraceright}\ {\isasymunion}\ S\ {\isasymin}\ C\ {\isasymor}\ {\isacharbraceleft}H{\isacharbraceright}\ {\isasymunion}\ S\ {\isasymin}\ C{\isachardoublequoteclose}\isanewline
%
\isadelimproof
%
\endisadelimproof
%
\isatagproof
\isacommand{proof}\isamarkupfalse%
\ {\isacharminus}\isanewline
\ \ \isacommand{have}\isamarkupfalse%
\ C{\isadigit{1}}{\isacharcolon}{\isachardoublequoteopen}{\isasymforall}G\ H{\isachardot}\ G\ \isactrlbold {\isasymor}\ H\ {\isasymin}\ S\ {\isasymlongrightarrow}\ {\isacharbraceleft}G{\isacharbraceright}\ {\isasymunion}\ S\ {\isasymin}\ C\ {\isasymor}\ {\isacharbraceleft}H{\isacharbraceright}\ {\isasymunion}\ S\ {\isasymin}\ C{\isachardoublequoteclose}\isanewline
\ \ \ \ \isacommand{using}\isamarkupfalse%
\ assms\ \isacommand{by}\isamarkupfalse%
\ {\isacharparenleft}rule\ conjunct{\isadigit{1}}{\isacharparenright}\isanewline
\ \ \isacommand{have}\isamarkupfalse%
\ C{\isadigit{2}}{\isacharcolon}{\isachardoublequoteopen}{\isasymforall}G\ H{\isachardot}\ G\ \isactrlbold {\isasymrightarrow}\ H\ {\isasymin}\ S\ {\isasymlongrightarrow}\ {\isacharbraceleft}\isactrlbold {\isasymnot}\ G{\isacharbraceright}\ {\isasymunion}\ S\ {\isasymin}\ C\ {\isasymor}\ {\isacharbraceleft}H{\isacharbraceright}\ {\isasymunion}\ S\ {\isasymin}\ C{\isachardoublequoteclose}\isanewline
\ \ \ \ \isacommand{using}\isamarkupfalse%
\ assms\ \isacommand{by}\isamarkupfalse%
\ {\isacharparenleft}iprover\ elim{\isacharcolon}\ conjunct{\isadigit{2}}\ conjunct{\isadigit{1}}{\isacharparenright}\isanewline
\ \ \isacommand{have}\isamarkupfalse%
\ C{\isadigit{3}}{\isacharcolon}{\isachardoublequoteopen}{\isasymforall}G{\isachardot}\ \isactrlbold {\isasymnot}\ {\isacharparenleft}\isactrlbold {\isasymnot}G{\isacharparenright}\ {\isasymin}\ S\ {\isasymlongrightarrow}\ {\isacharbraceleft}G{\isacharbraceright}\ {\isasymunion}\ S\ {\isasymin}\ C{\isachardoublequoteclose}\isanewline
\ \ \ \ \isacommand{using}\isamarkupfalse%
\ assms\ \isacommand{by}\isamarkupfalse%
\ {\isacharparenleft}iprover\ elim{\isacharcolon}\ conjunct{\isadigit{2}}\ conjunct{\isadigit{1}}{\isacharparenright}\isanewline
\ \ \isacommand{have}\isamarkupfalse%
\ C{\isadigit{4}}{\isacharcolon}{\isachardoublequoteopen}{\isasymforall}G\ H{\isachardot}\ \isactrlbold {\isasymnot}{\isacharparenleft}G\ \isactrlbold {\isasymand}\ H{\isacharparenright}\ {\isasymin}\ S\ {\isasymlongrightarrow}\ {\isacharbraceleft}\isactrlbold {\isasymnot}\ G{\isacharbraceright}\ {\isasymunion}\ S\ {\isasymin}\ C\ {\isasymor}\ {\isacharbraceleft}\isactrlbold {\isasymnot}\ H{\isacharbraceright}\ {\isasymunion}\ S\ {\isasymin}\ C{\isachardoublequoteclose}\isanewline
\ \ \ \ \isacommand{using}\isamarkupfalse%
\ assms\ \isacommand{by}\isamarkupfalse%
\ {\isacharparenleft}iprover\ elim{\isacharcolon}\ conjunct{\isadigit{2}}{\isacharparenright}\ \isanewline
\ \ \isacommand{show}\isamarkupfalse%
\ {\isachardoublequoteopen}{\isasymforall}F\ G\ H{\isachardot}\ Dis\ F\ G\ H\ {\isasymlongrightarrow}\ F\ {\isasymin}\ S\ {\isasymlongrightarrow}\ {\isacharbraceleft}G{\isacharbraceright}\ {\isasymunion}\ S\ {\isasymin}\ C\ {\isasymor}\ {\isacharbraceleft}H{\isacharbraceright}\ {\isasymunion}\ S\ {\isasymin}\ C{\isachardoublequoteclose}\isanewline
\ \ \isacommand{proof}\isamarkupfalse%
\ {\isacharparenleft}rule\ allI{\isacharparenright}{\isacharplus}\isanewline
\ \ \ \ \isacommand{fix}\isamarkupfalse%
\ F\ G\ H\isanewline
\ \ \ \ \isacommand{show}\isamarkupfalse%
\ {\isachardoublequoteopen}Dis\ F\ G\ H\ {\isasymlongrightarrow}\ F\ {\isasymin}\ S\ {\isasymlongrightarrow}\ {\isacharbraceleft}G{\isacharbraceright}\ {\isasymunion}\ S\ {\isasymin}\ C\ {\isasymor}\ {\isacharbraceleft}H{\isacharbraceright}\ {\isasymunion}\ S\ {\isasymin}\ C{\isachardoublequoteclose}\isanewline
\ \ \ \ \isacommand{proof}\isamarkupfalse%
\ {\isacharparenleft}rule\ impI{\isacharparenright}\isanewline
\ \ \ \ \ \ \isacommand{assume}\isamarkupfalse%
\ {\isachardoublequoteopen}Dis\ F\ G\ H{\isachardoublequoteclose}\isanewline
\ \ \ \ \ \ \isacommand{then}\isamarkupfalse%
\ \isacommand{have}\isamarkupfalse%
\ {\isachardoublequoteopen}F\ {\isacharequal}\ G\ \isactrlbold {\isasymor}\ H\ {\isasymor}\ \isanewline
\ \ \ \ \ \ \ \ \ \ \ \ \ \ \ \ {\isacharparenleft}{\isasymexists}G{\isadigit{1}}\ H{\isadigit{1}}{\isachardot}\ F\ {\isacharequal}\ G{\isadigit{1}}\ \isactrlbold {\isasymrightarrow}\ H{\isadigit{1}}\ {\isasymand}\ G\ {\isacharequal}\ \isactrlbold {\isasymnot}\ G{\isadigit{1}}\ {\isasymand}\ H\ {\isacharequal}\ H{\isadigit{1}}{\isacharparenright}\ {\isasymor}\ \isanewline
\ \ \ \ \ \ \ \ \ \ \ \ \ \ \ \ {\isacharparenleft}{\isasymexists}G{\isadigit{2}}\ H{\isadigit{2}}{\isachardot}\ F\ {\isacharequal}\ \isactrlbold {\isasymnot}\ {\isacharparenleft}G{\isadigit{2}}\ \isactrlbold {\isasymand}\ H{\isadigit{2}}{\isacharparenright}\ {\isasymand}\ G\ {\isacharequal}\ \isactrlbold {\isasymnot}\ G{\isadigit{2}}\ {\isasymand}\ H\ {\isacharequal}\ \isactrlbold {\isasymnot}\ H{\isadigit{2}}{\isacharparenright}\ {\isasymor}\ \isanewline
\ \ \ \ \ \ \ \ \ \ \ \ \ \ \ \ F\ {\isacharequal}\ \isactrlbold {\isasymnot}\ {\isacharparenleft}\isactrlbold {\isasymnot}\ G{\isacharparenright}\ {\isasymand}\ H\ {\isacharequal}\ G{\isachardoublequoteclose}\ \isanewline
\ \ \ \ \ \ \ \ \isacommand{by}\isamarkupfalse%
\ {\isacharparenleft}simp\ only{\isacharcolon}\ con{\isacharunderscore}dis{\isacharunderscore}simps{\isacharparenleft}{\isadigit{2}}{\isacharparenright}{\isacharparenright}\isanewline
\ \ \ \ \ \ \isacommand{thus}\isamarkupfalse%
\ {\isachardoublequoteopen}F\ {\isasymin}\ S\ {\isasymlongrightarrow}\ {\isacharbraceleft}G{\isacharbraceright}\ {\isasymunion}\ S\ {\isasymin}\ C\ {\isasymor}\ {\isacharbraceleft}H{\isacharbraceright}\ {\isasymunion}\ S\ {\isasymin}\ C{\isachardoublequoteclose}\isanewline
\ \ \ \ \ \ \isacommand{proof}\isamarkupfalse%
\ {\isacharparenleft}rule\ disjE{\isacharparenright}\isanewline
\ \ \ \ \ \ \ \ \isacommand{assume}\isamarkupfalse%
\ {\isachardoublequoteopen}F\ {\isacharequal}\ G\ \isactrlbold {\isasymor}\ H{\isachardoublequoteclose}\isanewline
\ \ \ \ \ \ \ \ \isacommand{show}\isamarkupfalse%
\ {\isachardoublequoteopen}F\ {\isasymin}\ S\ {\isasymlongrightarrow}\ {\isacharbraceleft}G{\isacharbraceright}\ {\isasymunion}\ S\ {\isasymin}\ C\ {\isasymor}\ {\isacharbraceleft}H{\isacharbraceright}\ {\isasymunion}\ S\ {\isasymin}\ C{\isachardoublequoteclose}\isanewline
\ \ \ \ \ \ \ \ \ \ \isacommand{using}\isamarkupfalse%
\ C{\isadigit{1}}\ {\isacartoucheopen}F\ {\isacharequal}\ G\ \isactrlbold {\isasymor}\ H{\isacartoucheclose}\ \isacommand{by}\isamarkupfalse%
\ {\isacharparenleft}iprover\ elim{\isacharcolon}\ allE{\isacharparenright}\isanewline
\ \ \ \ \ \ \isacommand{next}\isamarkupfalse%
\isanewline
\ \ \ \ \ \ \ \ \isacommand{assume}\isamarkupfalse%
\ {\isachardoublequoteopen}{\isacharparenleft}{\isasymexists}G{\isadigit{1}}\ H{\isadigit{1}}{\isachardot}\ F\ {\isacharequal}\ G{\isadigit{1}}\ \isactrlbold {\isasymrightarrow}\ H{\isadigit{1}}\ {\isasymand}\ G\ {\isacharequal}\ \isactrlbold {\isasymnot}\ G{\isadigit{1}}\ {\isasymand}\ H\ {\isacharequal}\ H{\isadigit{1}}{\isacharparenright}\ {\isasymor}\ \isanewline
\ \ \ \ \ \ \ \ \ \ \ \ \ \ {\isacharparenleft}{\isasymexists}G{\isadigit{2}}\ H{\isadigit{2}}{\isachardot}\ F\ {\isacharequal}\ \isactrlbold {\isasymnot}\ {\isacharparenleft}G{\isadigit{2}}\ \isactrlbold {\isasymand}\ H{\isadigit{2}}{\isacharparenright}\ {\isasymand}\ G\ {\isacharequal}\ \isactrlbold {\isasymnot}\ G{\isadigit{2}}\ {\isasymand}\ H\ {\isacharequal}\ \isactrlbold {\isasymnot}\ H{\isadigit{2}}{\isacharparenright}\ {\isasymor}\ \isanewline
\ \ \ \ \ \ \ \ \ \ \ \ \ \ F\ {\isacharequal}\ \isactrlbold {\isasymnot}\ {\isacharparenleft}\isactrlbold {\isasymnot}\ G{\isacharparenright}\ {\isasymand}\ H\ {\isacharequal}\ G{\isachardoublequoteclose}\isanewline
\ \ \ \ \ \ \ \ \isacommand{thus}\isamarkupfalse%
\ {\isachardoublequoteopen}F\ {\isasymin}\ S\ {\isasymlongrightarrow}\ {\isacharbraceleft}G{\isacharbraceright}\ {\isasymunion}\ S\ {\isasymin}\ C\ {\isasymor}\ {\isacharbraceleft}H{\isacharbraceright}\ {\isasymunion}\ S\ {\isasymin}\ C{\isachardoublequoteclose}\isanewline
\ \ \ \ \ \ \ \ \isacommand{proof}\isamarkupfalse%
\ {\isacharparenleft}rule\ disjE{\isacharparenright}\isanewline
\ \ \ \ \ \ \ \ \ \ \isacommand{assume}\isamarkupfalse%
\ E{\isadigit{1}}{\isacharcolon}{\isachardoublequoteopen}{\isasymexists}G{\isadigit{1}}\ H{\isadigit{1}}{\isachardot}\ F\ {\isacharequal}\ {\isacharparenleft}G{\isadigit{1}}\ \isactrlbold {\isasymrightarrow}\ H{\isadigit{1}}{\isacharparenright}\ {\isasymand}\ G\ {\isacharequal}\ \isactrlbold {\isasymnot}\ G{\isadigit{1}}\ {\isasymand}\ H\ {\isacharequal}\ H{\isadigit{1}}{\isachardoublequoteclose}\isanewline
\ \ \ \ \ \ \ \ \ \ \isacommand{obtain}\isamarkupfalse%
\ G{\isadigit{1}}\ H{\isadigit{1}}\ \isakeyword{where}\ A{\isadigit{1}}{\isacharcolon}{\isachardoublequoteopen}\ F\ {\isacharequal}\ {\isacharparenleft}G{\isadigit{1}}\ \isactrlbold {\isasymrightarrow}\ H{\isadigit{1}}{\isacharparenright}\ {\isasymand}\ G\ {\isacharequal}\ \isactrlbold {\isasymnot}\ G{\isadigit{1}}\ {\isasymand}\ H\ {\isacharequal}\ H{\isadigit{1}}{\isachardoublequoteclose}\isanewline
\ \ \ \ \ \ \ \ \ \ \ \ \isacommand{using}\isamarkupfalse%
\ E{\isadigit{1}}\ \isacommand{by}\isamarkupfalse%
\ {\isacharparenleft}iprover\ elim{\isacharcolon}\ exE{\isacharparenright}\isanewline
\ \ \ \ \ \ \ \ \ \ \isacommand{have}\isamarkupfalse%
\ {\isachardoublequoteopen}F\ {\isacharequal}\ {\isacharparenleft}G{\isadigit{1}}\ \isactrlbold {\isasymrightarrow}\ H{\isadigit{1}}{\isacharparenright}{\isachardoublequoteclose}\isanewline
\ \ \ \ \ \ \ \ \ \ \ \ \isacommand{using}\isamarkupfalse%
\ A{\isadigit{1}}\ \isacommand{by}\isamarkupfalse%
\ {\isacharparenleft}rule\ conjunct{\isadigit{1}}{\isacharparenright}\isanewline
\ \ \ \ \ \ \ \ \ \ \isacommand{have}\isamarkupfalse%
\ {\isachardoublequoteopen}G\ {\isacharequal}\ \isactrlbold {\isasymnot}\ G{\isadigit{1}}{\isachardoublequoteclose}\isanewline
\ \ \ \ \ \ \ \ \ \ \ \ \isacommand{using}\isamarkupfalse%
\ A{\isadigit{1}}\ \isacommand{by}\isamarkupfalse%
\ {\isacharparenleft}iprover\ elim{\isacharcolon}\ conjunct{\isadigit{2}}\ conjunct{\isadigit{1}}{\isacharparenright}\isanewline
\ \ \ \ \ \ \ \ \ \ \isacommand{have}\isamarkupfalse%
\ {\isachardoublequoteopen}H\ {\isacharequal}\ H{\isadigit{1}}{\isachardoublequoteclose}\isanewline
\ \ \ \ \ \ \ \ \ \ \ \ \isacommand{using}\isamarkupfalse%
\ A{\isadigit{1}}\ \isacommand{by}\isamarkupfalse%
\ {\isacharparenleft}iprover\ elim{\isacharcolon}\ conjunct{\isadigit{2}}{\isacharparenright}\isanewline
\ \ \ \ \ \ \ \ \ \ \isacommand{show}\isamarkupfalse%
\ {\isachardoublequoteopen}F\ {\isasymin}\ S\ {\isasymlongrightarrow}\ {\isacharbraceleft}G{\isacharbraceright}\ {\isasymunion}\ S\ {\isasymin}\ C\ {\isasymor}\ {\isacharbraceleft}H{\isacharbraceright}\ {\isasymunion}\ S\ {\isasymin}\ C{\isachardoublequoteclose}\isanewline
\ \ \ \ \ \ \ \ \ \ \ \ \isacommand{using}\isamarkupfalse%
\ C{\isadigit{2}}\ {\isacartoucheopen}F\ {\isacharequal}\ {\isacharparenleft}G{\isadigit{1}}\ \isactrlbold {\isasymrightarrow}\ H{\isadigit{1}}{\isacharparenright}{\isacartoucheclose}\ {\isacartoucheopen}G\ {\isacharequal}\ \isactrlbold {\isasymnot}\ G{\isadigit{1}}{\isacartoucheclose}\ {\isacartoucheopen}H\ {\isacharequal}\ H{\isadigit{1}}{\isacartoucheclose}\ \isacommand{by}\isamarkupfalse%
\ {\isacharparenleft}iprover\ elim{\isacharcolon}\ allE{\isacharparenright}\isanewline
\ \ \ \ \ \ \ \ \isacommand{next}\isamarkupfalse%
\isanewline
\ \ \ \ \ \ \ \ \ \ \isacommand{assume}\isamarkupfalse%
\ {\isachardoublequoteopen}{\isacharparenleft}{\isasymexists}G{\isadigit{2}}\ H{\isadigit{2}}{\isachardot}\ F\ {\isacharequal}\ \isactrlbold {\isasymnot}\ {\isacharparenleft}G{\isadigit{2}}\ \isactrlbold {\isasymand}\ H{\isadigit{2}}{\isacharparenright}\ {\isasymand}\ G\ {\isacharequal}\ \isactrlbold {\isasymnot}\ G{\isadigit{2}}\ {\isasymand}\ H\ {\isacharequal}\ \isactrlbold {\isasymnot}\ H{\isadigit{2}}{\isacharparenright}\ {\isasymor}\ \isanewline
\ \ \ \ \ \ \ \ \ \ \ \ \ \ \ \ \ \ F\ {\isacharequal}\ \isactrlbold {\isasymnot}\ {\isacharparenleft}\isactrlbold {\isasymnot}\ G{\isacharparenright}\ {\isasymand}\ H\ {\isacharequal}\ G{\isachardoublequoteclose}\ \isanewline
\ \ \ \ \ \ \ \ \ \ \isacommand{thus}\isamarkupfalse%
\ {\isachardoublequoteopen}F\ {\isasymin}\ S\ {\isasymlongrightarrow}\ {\isacharbraceleft}G{\isacharbraceright}\ {\isasymunion}\ S\ {\isasymin}\ C\ {\isasymor}\ {\isacharbraceleft}H{\isacharbraceright}\ {\isasymunion}\ S\ {\isasymin}\ C{\isachardoublequoteclose}\isanewline
\ \ \ \ \ \ \ \ \ \ \isacommand{proof}\isamarkupfalse%
\ {\isacharparenleft}rule\ disjE{\isacharparenright}\isanewline
\ \ \ \ \ \ \ \ \ \ \ \ \isacommand{assume}\isamarkupfalse%
\ E{\isadigit{2}}{\isacharcolon}{\isachardoublequoteopen}{\isasymexists}G{\isadigit{2}}\ H{\isadigit{2}}{\isachardot}\ F\ {\isacharequal}\ \isactrlbold {\isasymnot}\ {\isacharparenleft}G{\isadigit{2}}\ \isactrlbold {\isasymand}\ H{\isadigit{2}}{\isacharparenright}\ {\isasymand}\ G\ {\isacharequal}\ \isactrlbold {\isasymnot}\ G{\isadigit{2}}\ {\isasymand}\ H\ {\isacharequal}\ \isactrlbold {\isasymnot}\ H{\isadigit{2}}{\isachardoublequoteclose}\isanewline
\ \ \ \ \ \ \ \ \ \ \ \ \isacommand{obtain}\isamarkupfalse%
\ G{\isadigit{2}}\ H{\isadigit{2}}\ \isakeyword{where}\ A{\isadigit{2}}{\isacharcolon}{\isachardoublequoteopen}F\ {\isacharequal}\ \isactrlbold {\isasymnot}\ {\isacharparenleft}G{\isadigit{2}}\ \isactrlbold {\isasymand}\ H{\isadigit{2}}{\isacharparenright}\ {\isasymand}\ G\ {\isacharequal}\ \isactrlbold {\isasymnot}\ G{\isadigit{2}}\ {\isasymand}\ H\ {\isacharequal}\ \isactrlbold {\isasymnot}\ H{\isadigit{2}}{\isachardoublequoteclose}\isanewline
\ \ \ \ \ \ \ \ \ \ \ \ \ \ \isacommand{using}\isamarkupfalse%
\ E{\isadigit{2}}\ \isacommand{by}\isamarkupfalse%
\ {\isacharparenleft}iprover\ elim{\isacharcolon}\ exE{\isacharparenright}\isanewline
\ \ \ \ \ \ \ \ \ \ \ \ \isacommand{have}\isamarkupfalse%
\ {\isachardoublequoteopen}F\ {\isacharequal}\ \isactrlbold {\isasymnot}\ {\isacharparenleft}G{\isadigit{2}}\ \isactrlbold {\isasymand}\ H{\isadigit{2}}{\isacharparenright}{\isachardoublequoteclose}\isanewline
\ \ \ \ \ \ \ \ \ \ \ \ \ \ \isacommand{using}\isamarkupfalse%
\ A{\isadigit{2}}\ \isacommand{by}\isamarkupfalse%
\ {\isacharparenleft}rule\ conjunct{\isadigit{1}}{\isacharparenright}\isanewline
\ \ \ \ \ \ \ \ \ \ \ \ \isacommand{have}\isamarkupfalse%
\ {\isachardoublequoteopen}G\ {\isacharequal}\ \isactrlbold {\isasymnot}\ G{\isadigit{2}}{\isachardoublequoteclose}\isanewline
\ \ \ \ \ \ \ \ \ \ \ \ \ \ \isacommand{using}\isamarkupfalse%
\ A{\isadigit{2}}\ \isacommand{by}\isamarkupfalse%
\ {\isacharparenleft}iprover\ elim{\isacharcolon}\ conjunct{\isadigit{2}}\ conjunct{\isadigit{1}}{\isacharparenright}\isanewline
\ \ \ \ \ \ \ \ \ \ \ \ \isacommand{have}\isamarkupfalse%
\ {\isachardoublequoteopen}H\ {\isacharequal}\ \isactrlbold {\isasymnot}\ H{\isadigit{2}}{\isachardoublequoteclose}\isanewline
\ \ \ \ \ \ \ \ \ \ \ \ \ \ \isacommand{using}\isamarkupfalse%
\ A{\isadigit{2}}\ \isacommand{by}\isamarkupfalse%
\ {\isacharparenleft}iprover\ elim{\isacharcolon}\ conjunct{\isadigit{2}}{\isacharparenright}\isanewline
\ \ \ \ \ \ \ \ \ \ \ \ \isacommand{show}\isamarkupfalse%
\ {\isachardoublequoteopen}F\ {\isasymin}\ S\ {\isasymlongrightarrow}\ {\isacharbraceleft}G{\isacharbraceright}\ {\isasymunion}\ S\ {\isasymin}\ C\ {\isasymor}\ {\isacharbraceleft}H{\isacharbraceright}\ {\isasymunion}\ S\ {\isasymin}\ C{\isachardoublequoteclose}\isanewline
\ \ \ \ \ \ \ \ \ \ \ \ \ \ \isacommand{using}\isamarkupfalse%
\ C{\isadigit{4}}\ {\isacartoucheopen}F\ {\isacharequal}\ \isactrlbold {\isasymnot}\ {\isacharparenleft}G{\isadigit{2}}\ \isactrlbold {\isasymand}\ H{\isadigit{2}}{\isacharparenright}{\isacartoucheclose}\ {\isacartoucheopen}G\ {\isacharequal}\ \isactrlbold {\isasymnot}\ G{\isadigit{2}}{\isacartoucheclose}\ {\isacartoucheopen}H\ {\isacharequal}\ \isactrlbold {\isasymnot}\ H{\isadigit{2}}{\isacartoucheclose}\ \isacommand{by}\isamarkupfalse%
\ {\isacharparenleft}iprover\ elim{\isacharcolon}\ allE{\isacharparenright}\isanewline
\ \ \ \ \ \ \ \ \ \ \isacommand{next}\isamarkupfalse%
\isanewline
\ \ \ \ \ \ \ \ \ \ \ \ \isacommand{assume}\isamarkupfalse%
\ A{\isadigit{3}}{\isacharcolon}{\isachardoublequoteopen}F\ {\isacharequal}\ \isactrlbold {\isasymnot}{\isacharparenleft}\isactrlbold {\isasymnot}\ G{\isacharparenright}\ {\isasymand}\ H\ {\isacharequal}\ G{\isachardoublequoteclose}\isanewline
\ \ \ \ \ \ \ \ \ \ \ \ \isacommand{then}\isamarkupfalse%
\ \isacommand{have}\isamarkupfalse%
\ {\isachardoublequoteopen}F\ {\isacharequal}\ \isactrlbold {\isasymnot}{\isacharparenleft}\isactrlbold {\isasymnot}\ G{\isacharparenright}{\isachardoublequoteclose}\isanewline
\ \ \ \ \ \ \ \ \ \ \ \ \ \ \isacommand{by}\isamarkupfalse%
\ {\isacharparenleft}rule\ conjunct{\isadigit{1}}{\isacharparenright}\isanewline
\ \ \ \ \ \ \ \ \ \ \ \ \isacommand{have}\isamarkupfalse%
\ {\isachardoublequoteopen}H\ {\isacharequal}\ G{\isachardoublequoteclose}\isanewline
\ \ \ \ \ \ \ \ \ \ \ \ \ \ \isacommand{using}\isamarkupfalse%
\ A{\isadigit{3}}\ \isacommand{by}\isamarkupfalse%
\ {\isacharparenleft}rule\ conjunct{\isadigit{2}}{\isacharparenright}\isanewline
\ \ \ \ \ \ \ \ \ \ \ \ \isacommand{have}\isamarkupfalse%
\ {\isachardoublequoteopen}F\ {\isasymin}\ S\ {\isasymlongrightarrow}\ {\isacharbraceleft}G{\isacharbraceright}\ {\isasymunion}\ S\ {\isasymin}\ C{\isachardoublequoteclose}\isanewline
\ \ \ \ \ \ \ \ \ \ \ \ \ \ \isacommand{using}\isamarkupfalse%
\ C{\isadigit{3}}\ {\isacartoucheopen}F\ {\isacharequal}\ \isactrlbold {\isasymnot}{\isacharparenleft}\isactrlbold {\isasymnot}\ G{\isacharparenright}{\isacartoucheclose}\ \isacommand{by}\isamarkupfalse%
\ {\isacharparenleft}iprover\ elim{\isacharcolon}\ allE{\isacharparenright}\isanewline
\ \ \ \ \ \ \ \ \ \ \ \ \isacommand{then}\isamarkupfalse%
\ \isacommand{have}\isamarkupfalse%
\ {\isachardoublequoteopen}F\ {\isasymin}\ S\ {\isasymlongrightarrow}\ {\isacharbraceleft}G{\isacharbraceright}\ {\isasymunion}\ S\ {\isasymin}\ C\ {\isasymor}\ {\isacharbraceleft}G{\isacharbraceright}\ {\isasymunion}\ S\ {\isasymin}\ C{\isachardoublequoteclose}\isanewline
\ \ \ \ \ \ \ \ \ \ \ \ \ \ \isacommand{by}\isamarkupfalse%
\ {\isacharparenleft}simp\ only{\isacharcolon}\ disj{\isacharunderscore}absorb{\isacharparenright}\isanewline
\ \ \ \ \ \ \ \ \ \ \ \ \isacommand{thus}\isamarkupfalse%
\ {\isachardoublequoteopen}F\ {\isasymin}\ S\ {\isasymlongrightarrow}\ {\isacharbraceleft}G{\isacharbraceright}\ {\isasymunion}\ S\ {\isasymin}\ C\ {\isasymor}\ {\isacharbraceleft}H{\isacharbraceright}\ {\isasymunion}\ S\ {\isasymin}\ C{\isachardoublequoteclose}\isanewline
\ \ \ \ \ \ \ \ \ \ \ \ \ \ \isacommand{by}\isamarkupfalse%
\ {\isacharparenleft}simp\ only{\isacharcolon}\ {\isacartoucheopen}H\ {\isacharequal}\ G{\isacartoucheclose}{\isacharparenright}\isanewline
\ \ \ \ \ \ \ \ \ \ \isacommand{qed}\isamarkupfalse%
\isanewline
\ \ \ \ \ \ \ \ \isacommand{qed}\isamarkupfalse%
\isanewline
\ \ \ \ \ \ \isacommand{qed}\isamarkupfalse%
\isanewline
\ \ \ \ \isacommand{qed}\isamarkupfalse%
\isanewline
\ \ \isacommand{qed}\isamarkupfalse%
\isanewline
\isacommand{qed}\isamarkupfalse%
%
\endisatagproof
{\isafoldproof}%
%
\isadelimproof
%
\endisadelimproof
%
\begin{isamarkuptext}%
De esta manera, mediante los anteriores lemas auxiliares podemos probar la primera
  implicación detalladamente en Isabelle.%
\end{isamarkuptext}\isamarkuptrue%
\isacommand{lemma}\isamarkupfalse%
\ pcp{\isacharunderscore}alt{\isadigit{1}}{\isacharcolon}\ \isanewline
\ \ \isakeyword{assumes}\ {\isachardoublequoteopen}pcp\ C{\isachardoublequoteclose}\isanewline
\ \ \isakeyword{shows}\ {\isachardoublequoteopen}{\isasymforall}S\ {\isasymin}\ C{\isachardot}\ {\isasymbottom}\ {\isasymnotin}\ S\isanewline
\ \ {\isasymand}\ {\isacharparenleft}{\isasymforall}k{\isachardot}\ Atom\ k\ {\isasymin}\ S\ {\isasymlongrightarrow}\ \isactrlbold {\isasymnot}\ {\isacharparenleft}Atom\ k{\isacharparenright}\ {\isasymin}\ S\ {\isasymlongrightarrow}\ False{\isacharparenright}\isanewline
\ \ {\isasymand}\ {\isacharparenleft}{\isasymforall}F\ G\ H{\isachardot}\ Con\ F\ G\ H\ {\isasymlongrightarrow}\ F\ {\isasymin}\ S\ {\isasymlongrightarrow}\ {\isacharbraceleft}G{\isacharcomma}H{\isacharbraceright}\ {\isasymunion}\ S\ {\isasymin}\ C{\isacharparenright}\isanewline
\ \ {\isasymand}\ {\isacharparenleft}{\isasymforall}F\ G\ H{\isachardot}\ Dis\ F\ G\ H\ {\isasymlongrightarrow}\ F\ {\isasymin}\ S\ {\isasymlongrightarrow}\ {\isacharbraceleft}G{\isacharbraceright}\ {\isasymunion}\ S\ {\isasymin}\ C\ {\isasymor}\ {\isacharbraceleft}H{\isacharbraceright}\ {\isasymunion}\ S\ {\isasymin}\ C{\isacharparenright}{\isachardoublequoteclose}\isanewline
%
\isadelimproof
%
\endisadelimproof
%
\isatagproof
\isacommand{proof}\isamarkupfalse%
\ {\isacharparenleft}rule\ ballI{\isacharparenright}\isanewline
\ \ \isacommand{fix}\isamarkupfalse%
\ S\isanewline
\ \ \isacommand{assume}\isamarkupfalse%
\ {\isachardoublequoteopen}S\ {\isasymin}\ C{\isachardoublequoteclose}\isanewline
\ \ \isacommand{have}\isamarkupfalse%
\ {\isachardoublequoteopen}{\isacharparenleft}{\isasymforall}S\ {\isasymin}\ C{\isachardot}\isanewline
\ \ {\isasymbottom}\ {\isasymnotin}\ S\isanewline
\ \ {\isasymand}\ {\isacharparenleft}{\isasymforall}k{\isachardot}\ Atom\ k\ {\isasymin}\ S\ {\isasymlongrightarrow}\ \isactrlbold {\isasymnot}\ {\isacharparenleft}Atom\ k{\isacharparenright}\ {\isasymin}\ S\ {\isasymlongrightarrow}\ False{\isacharparenright}\isanewline
\ \ {\isasymand}\ {\isacharparenleft}{\isasymforall}G\ H{\isachardot}\ G\ \isactrlbold {\isasymand}\ H\ {\isasymin}\ S\ {\isasymlongrightarrow}\ {\isacharbraceleft}G{\isacharcomma}H{\isacharbraceright}\ {\isasymunion}\ S\ {\isasymin}\ C{\isacharparenright}\isanewline
\ \ {\isasymand}\ {\isacharparenleft}{\isasymforall}G\ H{\isachardot}\ G\ \isactrlbold {\isasymor}\ H\ {\isasymin}\ S\ {\isasymlongrightarrow}\ {\isacharbraceleft}G{\isacharbraceright}\ {\isasymunion}\ S\ {\isasymin}\ C\ {\isasymor}\ {\isacharbraceleft}H{\isacharbraceright}\ {\isasymunion}\ S\ {\isasymin}\ C{\isacharparenright}\isanewline
\ \ {\isasymand}\ {\isacharparenleft}{\isasymforall}G\ H{\isachardot}\ G\ \isactrlbold {\isasymrightarrow}\ H\ {\isasymin}\ S\ {\isasymlongrightarrow}\ {\isacharbraceleft}\isactrlbold {\isasymnot}G{\isacharbraceright}\ {\isasymunion}\ S\ {\isasymin}\ C\ {\isasymor}\ {\isacharbraceleft}H{\isacharbraceright}\ {\isasymunion}\ S\ {\isasymin}\ C{\isacharparenright}\isanewline
\ \ {\isasymand}\ {\isacharparenleft}{\isasymforall}G{\isachardot}\ \isactrlbold {\isasymnot}\ {\isacharparenleft}\isactrlbold {\isasymnot}G{\isacharparenright}\ {\isasymin}\ S\ {\isasymlongrightarrow}\ {\isacharbraceleft}G{\isacharbraceright}\ {\isasymunion}\ S\ {\isasymin}\ C{\isacharparenright}\isanewline
\ \ {\isasymand}\ {\isacharparenleft}{\isasymforall}G\ H{\isachardot}\ \isactrlbold {\isasymnot}{\isacharparenleft}G\ \isactrlbold {\isasymand}\ H{\isacharparenright}\ {\isasymin}\ S\ {\isasymlongrightarrow}\ {\isacharbraceleft}\isactrlbold {\isasymnot}\ G{\isacharbraceright}\ {\isasymunion}\ S\ {\isasymin}\ C\ {\isasymor}\ {\isacharbraceleft}\isactrlbold {\isasymnot}\ H{\isacharbraceright}\ {\isasymunion}\ S\ {\isasymin}\ C{\isacharparenright}\isanewline
\ \ {\isasymand}\ {\isacharparenleft}{\isasymforall}G\ H{\isachardot}\ \isactrlbold {\isasymnot}{\isacharparenleft}G\ \isactrlbold {\isasymor}\ H{\isacharparenright}\ {\isasymin}\ S\ {\isasymlongrightarrow}\ {\isacharbraceleft}\isactrlbold {\isasymnot}\ G{\isacharcomma}\ \isactrlbold {\isasymnot}\ H{\isacharbraceright}\ {\isasymunion}\ S\ {\isasymin}\ C{\isacharparenright}\isanewline
\ \ {\isasymand}\ {\isacharparenleft}{\isasymforall}G\ H{\isachardot}\ \isactrlbold {\isasymnot}{\isacharparenleft}G\ \isactrlbold {\isasymrightarrow}\ H{\isacharparenright}\ {\isasymin}\ S\ {\isasymlongrightarrow}\ {\isacharbraceleft}G{\isacharcomma}\isactrlbold {\isasymnot}\ H{\isacharbraceright}\ {\isasymunion}\ S\ {\isasymin}\ C{\isacharparenright}{\isacharparenright}{\isachardoublequoteclose}\isanewline
\ \ \ \ \isacommand{using}\isamarkupfalse%
\ assms\ \isacommand{by}\isamarkupfalse%
\ {\isacharparenleft}simp\ only{\isacharcolon}\ pcp{\isacharunderscore}def{\isacharparenright}\isanewline
\ \ \isacommand{then}\isamarkupfalse%
\ \isacommand{have}\isamarkupfalse%
\ pcpS{\isacharcolon}{\isachardoublequoteopen}{\isasymbottom}\ {\isasymnotin}\ S\isanewline
\ \ {\isasymand}\ {\isacharparenleft}{\isasymforall}k{\isachardot}\ Atom\ k\ {\isasymin}\ S\ {\isasymlongrightarrow}\ \isactrlbold {\isasymnot}\ {\isacharparenleft}Atom\ k{\isacharparenright}\ {\isasymin}\ S\ {\isasymlongrightarrow}\ False{\isacharparenright}\isanewline
\ \ {\isasymand}\ {\isacharparenleft}{\isasymforall}G\ H{\isachardot}\ G\ \isactrlbold {\isasymand}\ H\ {\isasymin}\ S\ {\isasymlongrightarrow}\ {\isacharbraceleft}G{\isacharcomma}H{\isacharbraceright}\ {\isasymunion}\ S\ {\isasymin}\ C{\isacharparenright}\isanewline
\ \ {\isasymand}\ {\isacharparenleft}{\isasymforall}G\ H{\isachardot}\ G\ \isactrlbold {\isasymor}\ H\ {\isasymin}\ S\ {\isasymlongrightarrow}\ {\isacharbraceleft}G{\isacharbraceright}\ {\isasymunion}\ S\ {\isasymin}\ C\ {\isasymor}\ {\isacharbraceleft}H{\isacharbraceright}\ {\isasymunion}\ S\ {\isasymin}\ C{\isacharparenright}\isanewline
\ \ {\isasymand}\ {\isacharparenleft}{\isasymforall}G\ H{\isachardot}\ G\ \isactrlbold {\isasymrightarrow}\ H\ {\isasymin}\ S\ {\isasymlongrightarrow}\ {\isacharbraceleft}\isactrlbold {\isasymnot}G{\isacharbraceright}\ {\isasymunion}\ S\ {\isasymin}\ C\ {\isasymor}\ {\isacharbraceleft}H{\isacharbraceright}\ {\isasymunion}\ S\ {\isasymin}\ C{\isacharparenright}\isanewline
\ \ {\isasymand}\ {\isacharparenleft}{\isasymforall}G{\isachardot}\ \isactrlbold {\isasymnot}\ {\isacharparenleft}\isactrlbold {\isasymnot}G{\isacharparenright}\ {\isasymin}\ S\ {\isasymlongrightarrow}\ {\isacharbraceleft}G{\isacharbraceright}\ {\isasymunion}\ S\ {\isasymin}\ C{\isacharparenright}\isanewline
\ \ {\isasymand}\ {\isacharparenleft}{\isasymforall}G\ H{\isachardot}\ \isactrlbold {\isasymnot}{\isacharparenleft}G\ \isactrlbold {\isasymand}\ H{\isacharparenright}\ {\isasymin}\ S\ {\isasymlongrightarrow}\ {\isacharbraceleft}\isactrlbold {\isasymnot}\ G{\isacharbraceright}\ {\isasymunion}\ S\ {\isasymin}\ C\ {\isasymor}\ {\isacharbraceleft}\isactrlbold {\isasymnot}\ H{\isacharbraceright}\ {\isasymunion}\ S\ {\isasymin}\ C{\isacharparenright}\isanewline
\ \ {\isasymand}\ {\isacharparenleft}{\isasymforall}G\ H{\isachardot}\ \isactrlbold {\isasymnot}{\isacharparenleft}G\ \isactrlbold {\isasymor}\ H{\isacharparenright}\ {\isasymin}\ S\ {\isasymlongrightarrow}\ {\isacharbraceleft}\isactrlbold {\isasymnot}\ G{\isacharcomma}\ \isactrlbold {\isasymnot}\ H{\isacharbraceright}\ {\isasymunion}\ S\ {\isasymin}\ C{\isacharparenright}\isanewline
\ \ {\isasymand}\ {\isacharparenleft}{\isasymforall}G\ H{\isachardot}\ \isactrlbold {\isasymnot}{\isacharparenleft}G\ \isactrlbold {\isasymrightarrow}\ H{\isacharparenright}\ {\isasymin}\ S\ {\isasymlongrightarrow}\ {\isacharbraceleft}G{\isacharcomma}\isactrlbold {\isasymnot}\ H{\isacharbraceright}\ {\isasymunion}\ S\ {\isasymin}\ C{\isacharparenright}{\isachardoublequoteclose}\isanewline
\ \ \ \ \isacommand{using}\isamarkupfalse%
\ {\isacartoucheopen}S\ {\isasymin}\ C{\isacartoucheclose}\ \isacommand{by}\isamarkupfalse%
\ {\isacharparenleft}rule\ bspec{\isacharparenright}\isanewline
\ \ \isacommand{then}\isamarkupfalse%
\ \isacommand{have}\isamarkupfalse%
\ C{\isadigit{1}}{\isacharcolon}{\isachardoublequoteopen}{\isasymbottom}\ {\isasymnotin}\ S{\isachardoublequoteclose}\isanewline
\ \ \ \ \isacommand{by}\isamarkupfalse%
\ {\isacharparenleft}rule\ conjunct{\isadigit{1}}{\isacharparenright}\isanewline
\ \ \isacommand{have}\isamarkupfalse%
\ C{\isadigit{2}}{\isacharcolon}{\isachardoublequoteopen}{\isasymforall}k{\isachardot}\ Atom\ k\ {\isasymin}\ S\ {\isasymlongrightarrow}\ \isactrlbold {\isasymnot}\ {\isacharparenleft}Atom\ k{\isacharparenright}\ {\isasymin}\ S\ {\isasymlongrightarrow}\ False{\isachardoublequoteclose}\isanewline
\ \ \ \ \isacommand{using}\isamarkupfalse%
\ pcpS\ \isacommand{by}\isamarkupfalse%
\ {\isacharparenleft}iprover\ elim{\isacharcolon}\ conjunct{\isadigit{2}}\ conjunct{\isadigit{1}}{\isacharparenright}\isanewline
\ \ \isacommand{have}\isamarkupfalse%
\ C{\isadigit{3}}{\isacharcolon}{\isachardoublequoteopen}{\isasymforall}G\ H{\isachardot}\ G\ \isactrlbold {\isasymand}\ H\ {\isasymin}\ S\ {\isasymlongrightarrow}\ {\isacharbraceleft}G{\isacharcomma}H{\isacharbraceright}\ {\isasymunion}\ S\ {\isasymin}\ C{\isachardoublequoteclose}\isanewline
\ \ \ \ \isacommand{using}\isamarkupfalse%
\ pcpS\ \isacommand{by}\isamarkupfalse%
\ {\isacharparenleft}iprover\ elim{\isacharcolon}\ conjunct{\isadigit{2}}\ conjunct{\isadigit{1}}{\isacharparenright}\isanewline
\ \ \isacommand{have}\isamarkupfalse%
\ C{\isadigit{4}}{\isacharcolon}{\isachardoublequoteopen}{\isasymforall}G\ H{\isachardot}\ G\ \isactrlbold {\isasymor}\ H\ {\isasymin}\ S\ {\isasymlongrightarrow}\ {\isacharbraceleft}G{\isacharbraceright}\ {\isasymunion}\ S\ {\isasymin}\ C\ {\isasymor}\ {\isacharbraceleft}H{\isacharbraceright}\ {\isasymunion}\ S\ {\isasymin}\ C{\isachardoublequoteclose}\isanewline
\ \ \ \ \isacommand{using}\isamarkupfalse%
\ pcpS\ \isacommand{by}\isamarkupfalse%
\ {\isacharparenleft}iprover\ elim{\isacharcolon}\ conjunct{\isadigit{2}}\ conjunct{\isadigit{1}}{\isacharparenright}\isanewline
\ \ \isacommand{have}\isamarkupfalse%
\ C{\isadigit{5}}{\isacharcolon}{\isachardoublequoteopen}{\isasymforall}G\ H{\isachardot}\ G\ \isactrlbold {\isasymrightarrow}\ H\ {\isasymin}\ S\ {\isasymlongrightarrow}\ {\isacharbraceleft}\isactrlbold {\isasymnot}G{\isacharbraceright}\ {\isasymunion}\ S\ {\isasymin}\ C\ {\isasymor}\ {\isacharbraceleft}H{\isacharbraceright}\ {\isasymunion}\ S\ {\isasymin}\ C{\isachardoublequoteclose}\isanewline
\ \ \ \ \isacommand{using}\isamarkupfalse%
\ pcpS\ \isacommand{by}\isamarkupfalse%
\ {\isacharparenleft}iprover\ elim{\isacharcolon}\ conjunct{\isadigit{2}}\ conjunct{\isadigit{1}}{\isacharparenright}\isanewline
\ \ \isacommand{have}\isamarkupfalse%
\ C{\isadigit{6}}{\isacharcolon}{\isachardoublequoteopen}{\isasymforall}G{\isachardot}\ \isactrlbold {\isasymnot}\ {\isacharparenleft}\isactrlbold {\isasymnot}G{\isacharparenright}\ {\isasymin}\ S\ {\isasymlongrightarrow}\ {\isacharbraceleft}G{\isacharbraceright}\ {\isasymunion}\ S\ {\isasymin}\ C{\isachardoublequoteclose}\isanewline
\ \ \ \ \isacommand{using}\isamarkupfalse%
\ pcpS\ \isacommand{by}\isamarkupfalse%
\ {\isacharparenleft}iprover\ elim{\isacharcolon}\ conjunct{\isadigit{2}}\ conjunct{\isadigit{1}}{\isacharparenright}\isanewline
\ \ \isacommand{have}\isamarkupfalse%
\ C{\isadigit{7}}{\isacharcolon}{\isachardoublequoteopen}{\isasymforall}G\ H{\isachardot}\ \isactrlbold {\isasymnot}{\isacharparenleft}G\ \isactrlbold {\isasymand}\ H{\isacharparenright}\ {\isasymin}\ S\ {\isasymlongrightarrow}\ {\isacharbraceleft}\isactrlbold {\isasymnot}\ G{\isacharbraceright}\ {\isasymunion}\ S\ {\isasymin}\ C\ {\isasymor}\ {\isacharbraceleft}\isactrlbold {\isasymnot}\ H{\isacharbraceright}\ {\isasymunion}\ S\ {\isasymin}\ C{\isachardoublequoteclose}\isanewline
\ \ \ \ \isacommand{using}\isamarkupfalse%
\ pcpS\ \isacommand{by}\isamarkupfalse%
\ {\isacharparenleft}iprover\ elim{\isacharcolon}\ conjunct{\isadigit{2}}\ conjunct{\isadigit{1}}{\isacharparenright}\isanewline
\ \ \isacommand{have}\isamarkupfalse%
\ C{\isadigit{8}}{\isacharcolon}{\isachardoublequoteopen}{\isasymforall}G\ H{\isachardot}\ \isactrlbold {\isasymnot}{\isacharparenleft}G\ \isactrlbold {\isasymor}\ H{\isacharparenright}\ {\isasymin}\ S\ {\isasymlongrightarrow}\ {\isacharbraceleft}\isactrlbold {\isasymnot}\ G{\isacharcomma}\ \isactrlbold {\isasymnot}\ H{\isacharbraceright}\ {\isasymunion}\ S\ {\isasymin}\ C{\isachardoublequoteclose}\isanewline
\ \ \ \ \isacommand{using}\isamarkupfalse%
\ pcpS\ \isacommand{by}\isamarkupfalse%
\ {\isacharparenleft}iprover\ elim{\isacharcolon}\ conjunct{\isadigit{2}}\ conjunct{\isadigit{1}}{\isacharparenright}\isanewline
\ \ \isacommand{have}\isamarkupfalse%
\ C{\isadigit{9}}{\isacharcolon}{\isachardoublequoteopen}{\isasymforall}G\ H{\isachardot}\ \isactrlbold {\isasymnot}{\isacharparenleft}G\ \isactrlbold {\isasymrightarrow}\ H{\isacharparenright}\ {\isasymin}\ S\ {\isasymlongrightarrow}\ {\isacharbraceleft}G{\isacharcomma}\isactrlbold {\isasymnot}\ H{\isacharbraceright}\ {\isasymunion}\ S\ {\isasymin}\ C{\isachardoublequoteclose}\isanewline
\ \ \ \ \isacommand{using}\isamarkupfalse%
\ pcpS\ \isacommand{by}\isamarkupfalse%
\ {\isacharparenleft}iprover\ elim{\isacharcolon}\ conjunct{\isadigit{2}}{\isacharparenright}\isanewline
\ \ \isacommand{have}\isamarkupfalse%
\ {\isachardoublequoteopen}{\isacharparenleft}{\isasymforall}G\ H{\isachardot}\ G\ \isactrlbold {\isasymand}\ H\ {\isasymin}\ S\ {\isasymlongrightarrow}\ {\isacharbraceleft}G{\isacharcomma}H{\isacharbraceright}\ {\isasymunion}\ S\ {\isasymin}\ C{\isacharparenright}\isanewline
\ \ {\isasymand}\ {\isacharparenleft}{\isasymforall}G{\isachardot}\ \isactrlbold {\isasymnot}\ {\isacharparenleft}\isactrlbold {\isasymnot}G{\isacharparenright}\ {\isasymin}\ S\ {\isasymlongrightarrow}\ {\isacharbraceleft}G{\isacharbraceright}\ {\isasymunion}\ S\ {\isasymin}\ C{\isacharparenright}\isanewline
\ \ {\isasymand}\ {\isacharparenleft}{\isasymforall}G\ H{\isachardot}\ \isactrlbold {\isasymnot}{\isacharparenleft}G\ \isactrlbold {\isasymor}\ H{\isacharparenright}\ {\isasymin}\ S\ {\isasymlongrightarrow}\ {\isacharbraceleft}\isactrlbold {\isasymnot}\ G{\isacharcomma}\ \isactrlbold {\isasymnot}\ H{\isacharbraceright}\ {\isasymunion}\ S\ {\isasymin}\ C{\isacharparenright}\isanewline
\ \ {\isasymand}\ {\isacharparenleft}{\isasymforall}G\ H{\isachardot}\ \isactrlbold {\isasymnot}{\isacharparenleft}G\ \isactrlbold {\isasymrightarrow}\ H{\isacharparenright}\ {\isasymin}\ S\ {\isasymlongrightarrow}\ {\isacharbraceleft}G{\isacharcomma}\isactrlbold {\isasymnot}\ H{\isacharbraceright}\ {\isasymunion}\ S\ {\isasymin}\ C{\isacharparenright}{\isachardoublequoteclose}\isanewline
\ \ \ \ \isacommand{using}\isamarkupfalse%
\ C{\isadigit{3}}\ C{\isadigit{6}}\ C{\isadigit{8}}\ C{\isadigit{9}}\ \isacommand{by}\isamarkupfalse%
\ {\isacharparenleft}iprover\ intro{\isacharcolon}\ conjI{\isacharparenright}\isanewline
\ \ \isacommand{then}\isamarkupfalse%
\ \isacommand{have}\isamarkupfalse%
\ Con{\isacharcolon}{\isachardoublequoteopen}{\isasymforall}F\ G\ H{\isachardot}\ Con\ F\ G\ H\ {\isasymlongrightarrow}\ F\ {\isasymin}\ S\ {\isasymlongrightarrow}\ {\isacharbraceleft}G{\isacharcomma}H{\isacharbraceright}\ {\isasymunion}\ S\ {\isasymin}\ C{\isachardoublequoteclose}\isanewline
\ \ \ \ \isacommand{by}\isamarkupfalse%
\ {\isacharparenleft}rule\ pcp{\isacharunderscore}alt{\isadigit{1}}Con{\isacharparenright}\isanewline
\ \ \isacommand{have}\isamarkupfalse%
\ {\isachardoublequoteopen}{\isacharparenleft}{\isasymforall}G\ H{\isachardot}\ G\ \isactrlbold {\isasymor}\ H\ {\isasymin}\ S\ {\isasymlongrightarrow}\ {\isacharbraceleft}G{\isacharbraceright}\ {\isasymunion}\ S\ {\isasymin}\ C\ {\isasymor}\ {\isacharbraceleft}H{\isacharbraceright}\ {\isasymunion}\ S\ {\isasymin}\ C{\isacharparenright}\isanewline
\ \ {\isasymand}\ {\isacharparenleft}{\isasymforall}G\ H{\isachardot}\ G\ \isactrlbold {\isasymrightarrow}\ H\ {\isasymin}\ S\ {\isasymlongrightarrow}\ {\isacharbraceleft}\isactrlbold {\isasymnot}\ G{\isacharbraceright}\ {\isasymunion}\ S\ {\isasymin}\ C\ {\isasymor}\ {\isacharbraceleft}H{\isacharbraceright}\ {\isasymunion}\ S\ {\isasymin}\ C{\isacharparenright}\isanewline
\ \ {\isasymand}\ {\isacharparenleft}{\isasymforall}G{\isachardot}\ \isactrlbold {\isasymnot}\ {\isacharparenleft}\isactrlbold {\isasymnot}G{\isacharparenright}\ {\isasymin}\ S\ {\isasymlongrightarrow}\ {\isacharbraceleft}G{\isacharbraceright}\ {\isasymunion}\ S\ {\isasymin}\ C{\isacharparenright}\isanewline
\ \ {\isasymand}\ {\isacharparenleft}{\isasymforall}G\ H{\isachardot}\ \isactrlbold {\isasymnot}{\isacharparenleft}G\ \isactrlbold {\isasymand}\ H{\isacharparenright}\ {\isasymin}\ S\ {\isasymlongrightarrow}\ {\isacharbraceleft}\isactrlbold {\isasymnot}\ G{\isacharbraceright}\ {\isasymunion}\ S\ {\isasymin}\ C\ {\isasymor}\ {\isacharbraceleft}\isactrlbold {\isasymnot}\ H{\isacharbraceright}\ {\isasymunion}\ S\ {\isasymin}\ C{\isacharparenright}{\isachardoublequoteclose}\isanewline
\ \ \ \ \isacommand{using}\isamarkupfalse%
\ C{\isadigit{4}}\ C{\isadigit{5}}\ C{\isadigit{6}}\ C{\isadigit{7}}\ \isacommand{by}\isamarkupfalse%
\ {\isacharparenleft}iprover\ intro{\isacharcolon}\ conjI{\isacharparenright}\isanewline
\ \ \isacommand{then}\isamarkupfalse%
\ \isacommand{have}\isamarkupfalse%
\ Dis{\isacharcolon}{\isachardoublequoteopen}{\isasymforall}F\ G\ H{\isachardot}\ Dis\ F\ G\ H\ {\isasymlongrightarrow}\ F\ {\isasymin}\ S\ {\isasymlongrightarrow}\ {\isacharbraceleft}G{\isacharbraceright}\ {\isasymunion}\ S\ {\isasymin}\ C\ {\isasymor}\ {\isacharbraceleft}H{\isacharbraceright}\ {\isasymunion}\ S\ {\isasymin}\ C{\isachardoublequoteclose}\isanewline
\ \ \ \ \isacommand{by}\isamarkupfalse%
\ {\isacharparenleft}rule\ pcp{\isacharunderscore}alt{\isadigit{1}}Dis{\isacharparenright}\isanewline
\ \ \isacommand{thus}\isamarkupfalse%
\ {\isachardoublequoteopen}{\isasymbottom}\ {\isasymnotin}\ S\isanewline
\ \ {\isasymand}\ {\isacharparenleft}{\isasymforall}k{\isachardot}\ Atom\ k\ {\isasymin}\ S\ {\isasymlongrightarrow}\ \isactrlbold {\isasymnot}\ {\isacharparenleft}Atom\ k{\isacharparenright}\ {\isasymin}\ S\ {\isasymlongrightarrow}\ False{\isacharparenright}\isanewline
\ \ {\isasymand}\ {\isacharparenleft}{\isasymforall}F\ G\ H{\isachardot}\ Con\ F\ G\ H\ {\isasymlongrightarrow}\ F\ {\isasymin}\ S\ {\isasymlongrightarrow}\ {\isacharbraceleft}G{\isacharcomma}H{\isacharbraceright}\ {\isasymunion}\ S\ {\isasymin}\ C{\isacharparenright}\isanewline
\ \ {\isasymand}\ {\isacharparenleft}{\isasymforall}F\ G\ H{\isachardot}\ Dis\ F\ G\ H\ {\isasymlongrightarrow}\ F\ {\isasymin}\ S\ {\isasymlongrightarrow}\ {\isacharbraceleft}G{\isacharbraceright}\ {\isasymunion}\ S\ {\isasymin}\ C\ {\isasymor}\ {\isacharbraceleft}H{\isacharbraceright}\ {\isasymunion}\ S\ {\isasymin}\ C{\isacharparenright}{\isachardoublequoteclose}\isanewline
\ \ \ \ \isacommand{using}\isamarkupfalse%
\ C{\isadigit{1}}\ C{\isadigit{2}}\ Con\ Dis\ \isacommand{by}\isamarkupfalse%
\ {\isacharparenleft}iprover\ intro{\isacharcolon}\ conjI{\isacharparenright}\isanewline
\isacommand{qed}\isamarkupfalse%
%
\endisatagproof
{\isafoldproof}%
%
\isadelimproof
%
\endisadelimproof
%
\begin{isamarkuptext}%
Por otro lado, veamos la demostración detallada de la implicación recíproca de la
  equivalencia. Para ello, utilizaremos distintos lemas auxiliares para deducir cada una de las 
  condiciones de la definición de propiedad de consistencia proposicional a partir de las
  hipótesis sobre las fórmulas de tipo \isa{{\isasymalpha}} y \isa{{\isasymbeta}}. En primer lugar, veamos los lemas que se deducen
  condiciones a partir de la hipótesis referente a las fórmulas de tipo \isa{{\isasymalpha}}.%
\end{isamarkuptext}\isamarkuptrue%
\isacommand{lemma}\isamarkupfalse%
\ pcp{\isacharunderscore}alt{\isadigit{2}}Con{\isadigit{1}}{\isacharcolon}\isanewline
\ \ \isakeyword{assumes}\ {\isachardoublequoteopen}{\isasymforall}F\ G\ H{\isachardot}\ Con\ F\ G\ H\ {\isasymlongrightarrow}\ F\ {\isasymin}\ S\ {\isasymlongrightarrow}\ {\isacharbraceleft}G{\isacharcomma}H{\isacharbraceright}\ {\isasymunion}\ S\ {\isasymin}\ C{\isachardoublequoteclose}\isanewline
\ \ \isakeyword{shows}\ {\isachardoublequoteopen}{\isasymforall}G\ H{\isachardot}\ G\ \isactrlbold {\isasymand}\ H\ {\isasymin}\ S\ {\isasymlongrightarrow}\ {\isacharbraceleft}G{\isacharcomma}H{\isacharbraceright}\ {\isasymunion}\ S\ {\isasymin}\ C{\isachardoublequoteclose}\isanewline
%
\isadelimproof
%
\endisadelimproof
%
\isatagproof
\isacommand{proof}\isamarkupfalse%
\ {\isacharparenleft}rule\ allI{\isacharparenright}{\isacharplus}\isanewline
\ \ \isacommand{fix}\isamarkupfalse%
\ G\ H\isanewline
\ \ \isacommand{show}\isamarkupfalse%
\ {\isachardoublequoteopen}G\ \isactrlbold {\isasymand}\ H\ {\isasymin}\ S\ {\isasymlongrightarrow}\ {\isacharbraceleft}G{\isacharcomma}H{\isacharbraceright}\ {\isasymunion}\ S\ {\isasymin}\ C{\isachardoublequoteclose}\isanewline
\ \ \isacommand{proof}\isamarkupfalse%
\ {\isacharparenleft}rule\ impI{\isacharparenright}\isanewline
\ \ \ \ \isacommand{assume}\isamarkupfalse%
\ {\isachardoublequoteopen}G\ \isactrlbold {\isasymand}\ H\ {\isasymin}\ S{\isachardoublequoteclose}\isanewline
\ \ \ \ \isacommand{then}\isamarkupfalse%
\ \isacommand{have}\isamarkupfalse%
\ {\isachardoublequoteopen}Con\ {\isacharparenleft}G\ \isactrlbold {\isasymand}\ H{\isacharparenright}\ G\ H{\isachardoublequoteclose}\isanewline
\ \ \ \ \ \ \isacommand{by}\isamarkupfalse%
\ {\isacharparenleft}simp\ only{\isacharcolon}\ Con{\isachardot}intros{\isacharparenleft}{\isadigit{1}}{\isacharparenright}{\isacharparenright}\isanewline
\ \ \ \ \isacommand{let}\isamarkupfalse%
\ {\isacharquery}F{\isacharequal}{\isachardoublequoteopen}G\ \isactrlbold {\isasymand}\ H{\isachardoublequoteclose}\isanewline
\ \ \ \ \isacommand{have}\isamarkupfalse%
\ {\isachardoublequoteopen}Con\ {\isacharquery}F\ G\ H\ {\isasymlongrightarrow}\ {\isacharquery}F\ {\isasymin}\ S\ {\isasymlongrightarrow}\ {\isacharbraceleft}G{\isacharcomma}H{\isacharbraceright}\ {\isasymunion}\ S\ {\isasymin}\ C{\isachardoublequoteclose}\isanewline
\ \ \ \ \ \ \isacommand{using}\isamarkupfalse%
\ assms\ \isacommand{by}\isamarkupfalse%
\ {\isacharparenleft}iprover\ elim{\isacharcolon}\ allE{\isacharparenright}\isanewline
\ \ \ \ \isacommand{then}\isamarkupfalse%
\ \isacommand{have}\isamarkupfalse%
\ {\isachardoublequoteopen}{\isacharquery}F\ {\isasymin}\ S\ {\isasymlongrightarrow}\ {\isacharbraceleft}G{\isacharcomma}H{\isacharbraceright}\ {\isasymunion}\ S\ {\isasymin}\ C{\isachardoublequoteclose}\isanewline
\ \ \ \ \ \ \isacommand{using}\isamarkupfalse%
\ {\isacartoucheopen}Con\ {\isacharparenleft}G\ \isactrlbold {\isasymand}\ H{\isacharparenright}\ G\ H{\isacartoucheclose}\ \isacommand{by}\isamarkupfalse%
\ {\isacharparenleft}rule\ mp{\isacharparenright}\isanewline
\ \ \ \ \isacommand{thus}\isamarkupfalse%
\ {\isachardoublequoteopen}{\isacharbraceleft}G{\isacharcomma}H{\isacharbraceright}\ {\isasymunion}\ S\ {\isasymin}\ C{\isachardoublequoteclose}\isanewline
\ \ \ \ \ \ \isacommand{using}\isamarkupfalse%
\ {\isacartoucheopen}{\isacharparenleft}G\ \isactrlbold {\isasymand}\ H{\isacharparenright}\ {\isasymin}\ S{\isacartoucheclose}\ \isacommand{by}\isamarkupfalse%
\ {\isacharparenleft}rule\ mp{\isacharparenright}\isanewline
\ \ \isacommand{qed}\isamarkupfalse%
\isanewline
\isacommand{qed}\isamarkupfalse%
%
\endisatagproof
{\isafoldproof}%
%
\isadelimproof
\isanewline
%
\endisadelimproof
\isanewline
\isacommand{lemma}\isamarkupfalse%
\ pcp{\isacharunderscore}alt{\isadigit{2}}Con{\isadigit{2}}{\isacharcolon}\isanewline
\ \ \isakeyword{assumes}\ {\isachardoublequoteopen}{\isasymforall}F\ G\ H{\isachardot}\ Con\ F\ G\ H\ {\isasymlongrightarrow}\ F\ {\isasymin}\ S\ {\isasymlongrightarrow}\ {\isacharbraceleft}G{\isacharcomma}H{\isacharbraceright}\ {\isasymunion}\ S\ {\isasymin}\ C{\isachardoublequoteclose}\isanewline
\ \ \isakeyword{shows}\ {\isachardoublequoteopen}{\isasymforall}G{\isachardot}\ \isactrlbold {\isasymnot}\ {\isacharparenleft}\isactrlbold {\isasymnot}G{\isacharparenright}\ {\isasymin}\ S\ {\isasymlongrightarrow}\ {\isacharbraceleft}G{\isacharbraceright}\ {\isasymunion}\ S\ {\isasymin}\ C{\isachardoublequoteclose}\isanewline
%
\isadelimproof
%
\endisadelimproof
%
\isatagproof
\isacommand{proof}\isamarkupfalse%
\ {\isacharparenleft}rule\ allI{\isacharparenright}\isanewline
\ \ \isacommand{fix}\isamarkupfalse%
\ G\ \isanewline
\ \ \isacommand{show}\isamarkupfalse%
\ {\isachardoublequoteopen}\isactrlbold {\isasymnot}\ {\isacharparenleft}\isactrlbold {\isasymnot}G{\isacharparenright}\ {\isasymin}\ S\ {\isasymlongrightarrow}\ {\isacharbraceleft}G{\isacharbraceright}\ {\isasymunion}\ S\ {\isasymin}\ C{\isachardoublequoteclose}\isanewline
\ \ \isacommand{proof}\isamarkupfalse%
\ {\isacharparenleft}rule\ impI{\isacharparenright}\isanewline
\ \ \ \ \isacommand{assume}\isamarkupfalse%
\ {\isachardoublequoteopen}\isactrlbold {\isasymnot}{\isacharparenleft}\isactrlbold {\isasymnot}G{\isacharparenright}\ {\isasymin}\ S{\isachardoublequoteclose}\isanewline
\ \ \ \ \isacommand{then}\isamarkupfalse%
\ \isacommand{have}\isamarkupfalse%
\ {\isachardoublequoteopen}Con\ {\isacharparenleft}\isactrlbold {\isasymnot}{\isacharparenleft}\isactrlbold {\isasymnot}G{\isacharparenright}{\isacharparenright}\ G\ G{\isachardoublequoteclose}\isanewline
\ \ \ \ \ \ \isacommand{by}\isamarkupfalse%
\ {\isacharparenleft}simp\ only{\isacharcolon}\ Con{\isachardot}intros{\isacharparenleft}{\isadigit{4}}{\isacharparenright}{\isacharparenright}\isanewline
\ \ \ \ \isacommand{let}\isamarkupfalse%
\ {\isacharquery}F{\isacharequal}{\isachardoublequoteopen}\isactrlbold {\isasymnot}{\isacharparenleft}\isactrlbold {\isasymnot}\ G{\isacharparenright}{\isachardoublequoteclose}\isanewline
\ \ \ \ \isacommand{have}\isamarkupfalse%
\ {\isachardoublequoteopen}{\isasymforall}G\ H{\isachardot}\ Con\ {\isacharquery}F\ G\ H\ {\isasymlongrightarrow}\ {\isacharquery}F\ {\isasymin}\ S\ {\isasymlongrightarrow}\ {\isacharbraceleft}G{\isacharcomma}H{\isacharbraceright}\ {\isasymunion}\ S\ {\isasymin}\ C{\isachardoublequoteclose}\isanewline
\ \ \ \ \ \ \isacommand{using}\isamarkupfalse%
\ assms\ \isacommand{by}\isamarkupfalse%
\ {\isacharparenleft}rule\ allE{\isacharparenright}\isanewline
\ \ \ \ \isacommand{then}\isamarkupfalse%
\ \isacommand{have}\isamarkupfalse%
\ {\isachardoublequoteopen}{\isasymforall}H{\isachardot}\ Con\ {\isacharquery}F\ G\ H\ {\isasymlongrightarrow}\ {\isacharquery}F\ {\isasymin}\ S\ {\isasymlongrightarrow}\ {\isacharbraceleft}G{\isacharcomma}H{\isacharbraceright}\ {\isasymunion}\ S\ {\isasymin}\ C{\isachardoublequoteclose}\isanewline
\ \ \ \ \ \ \isacommand{by}\isamarkupfalse%
\ {\isacharparenleft}rule\ allE{\isacharparenright}\isanewline
\ \ \ \ \isacommand{then}\isamarkupfalse%
\ \isacommand{have}\isamarkupfalse%
\ {\isachardoublequoteopen}Con\ {\isacharquery}F\ G\ G\ {\isasymlongrightarrow}\ {\isacharquery}F\ {\isasymin}\ S\ {\isasymlongrightarrow}\ {\isacharbraceleft}G{\isacharcomma}G{\isacharbraceright}\ {\isasymunion}\ S\ {\isasymin}\ C{\isachardoublequoteclose}\isanewline
\ \ \ \ \ \ \isacommand{by}\isamarkupfalse%
\ {\isacharparenleft}rule\ allE{\isacharparenright}\isanewline
\ \ \ \ \isacommand{then}\isamarkupfalse%
\ \isacommand{have}\isamarkupfalse%
\ {\isachardoublequoteopen}{\isacharquery}F\ {\isasymin}\ S\ {\isasymlongrightarrow}\ {\isacharbraceleft}G{\isacharcomma}G{\isacharbraceright}\ {\isasymunion}\ S\ {\isasymin}\ C{\isachardoublequoteclose}\isanewline
\ \ \ \ \ \ \isacommand{using}\isamarkupfalse%
\ {\isacartoucheopen}Con\ {\isacharparenleft}\isactrlbold {\isasymnot}{\isacharparenleft}\isactrlbold {\isasymnot}G{\isacharparenright}{\isacharparenright}\ G\ G{\isacartoucheclose}\ \isacommand{by}\isamarkupfalse%
\ {\isacharparenleft}rule\ mp{\isacharparenright}\isanewline
\ \ \ \ \isacommand{then}\isamarkupfalse%
\ \isacommand{have}\isamarkupfalse%
\ {\isachardoublequoteopen}{\isacharbraceleft}G{\isacharcomma}G{\isacharbraceright}\ {\isasymunion}\ S\ {\isasymin}\ C{\isachardoublequoteclose}\isanewline
\ \ \ \ \ \ \isacommand{using}\isamarkupfalse%
\ {\isacartoucheopen}{\isacharparenleft}\isactrlbold {\isasymnot}{\isacharparenleft}\isactrlbold {\isasymnot}G{\isacharparenright}{\isacharparenright}\ {\isasymin}\ S{\isacartoucheclose}\ \isacommand{by}\isamarkupfalse%
\ {\isacharparenleft}rule\ mp{\isacharparenright}\isanewline
\ \ \ \ \isacommand{thus}\isamarkupfalse%
\ {\isachardoublequoteopen}{\isacharbraceleft}G{\isacharbraceright}\ {\isasymunion}\ S\ {\isasymin}\ C{\isachardoublequoteclose}\isanewline
\ \ \ \ \ \ \isacommand{by}\isamarkupfalse%
\ {\isacharparenleft}simp\ only{\isacharcolon}\ insert{\isacharunderscore}absorb{\isadigit{2}}{\isacharparenright}\isanewline
\ \ \isacommand{qed}\isamarkupfalse%
\isanewline
\isacommand{qed}\isamarkupfalse%
%
\endisatagproof
{\isafoldproof}%
%
\isadelimproof
\isanewline
%
\endisadelimproof
\isanewline
\isacommand{lemma}\isamarkupfalse%
\ pcp{\isacharunderscore}alt{\isadigit{2}}Con{\isadigit{3}}{\isacharcolon}\isanewline
\ \ \isakeyword{assumes}\ {\isachardoublequoteopen}{\isasymforall}F\ G\ H{\isachardot}\ Con\ F\ G\ H\ {\isasymlongrightarrow}\ F\ {\isasymin}\ S\ {\isasymlongrightarrow}\ {\isacharbraceleft}G{\isacharcomma}H{\isacharbraceright}\ {\isasymunion}\ S\ {\isasymin}\ C{\isachardoublequoteclose}\isanewline
\ \ \isakeyword{shows}\ {\isachardoublequoteopen}{\isasymforall}G\ H{\isachardot}\ \isactrlbold {\isasymnot}{\isacharparenleft}G\ \isactrlbold {\isasymor}\ H{\isacharparenright}\ {\isasymin}\ S\ {\isasymlongrightarrow}\ {\isacharbraceleft}\isactrlbold {\isasymnot}\ G{\isacharcomma}\ \isactrlbold {\isasymnot}\ H{\isacharbraceright}\ {\isasymunion}\ S\ {\isasymin}\ C{\isachardoublequoteclose}\isanewline
%
\isadelimproof
%
\endisadelimproof
%
\isatagproof
\isacommand{proof}\isamarkupfalse%
\ {\isacharparenleft}rule\ allI{\isacharparenright}{\isacharplus}\isanewline
\ \ \isacommand{fix}\isamarkupfalse%
\ G\ H\isanewline
\ \ \isacommand{show}\isamarkupfalse%
\ {\isachardoublequoteopen}\isactrlbold {\isasymnot}{\isacharparenleft}G\ \isactrlbold {\isasymor}\ H{\isacharparenright}\ {\isasymin}\ S\ {\isasymlongrightarrow}\ {\isacharbraceleft}\isactrlbold {\isasymnot}\ G{\isacharcomma}\ \isactrlbold {\isasymnot}\ H{\isacharbraceright}\ {\isasymunion}\ S\ {\isasymin}\ C{\isachardoublequoteclose}\isanewline
\ \ \isacommand{proof}\isamarkupfalse%
\ {\isacharparenleft}rule\ impI{\isacharparenright}\isanewline
\ \ \ \ \isacommand{assume}\isamarkupfalse%
\ {\isachardoublequoteopen}\isactrlbold {\isasymnot}{\isacharparenleft}G\ \isactrlbold {\isasymor}\ H{\isacharparenright}\ {\isasymin}\ S{\isachardoublequoteclose}\isanewline
\ \ \ \ \isacommand{then}\isamarkupfalse%
\ \isacommand{have}\isamarkupfalse%
\ {\isachardoublequoteopen}Con\ {\isacharparenleft}\isactrlbold {\isasymnot}{\isacharparenleft}G\ \isactrlbold {\isasymor}\ H{\isacharparenright}{\isacharparenright}\ {\isacharparenleft}\isactrlbold {\isasymnot}G{\isacharparenright}\ {\isacharparenleft}\isactrlbold {\isasymnot}H{\isacharparenright}{\isachardoublequoteclose}\isanewline
\ \ \ \ \ \ \isacommand{by}\isamarkupfalse%
\ {\isacharparenleft}simp\ only{\isacharcolon}\ Con{\isachardot}intros{\isacharparenleft}{\isadigit{2}}{\isacharparenright}{\isacharparenright}\isanewline
\ \ \ \ \isacommand{let}\isamarkupfalse%
\ {\isacharquery}F\ {\isacharequal}\ {\isachardoublequoteopen}\isactrlbold {\isasymnot}{\isacharparenleft}G\ \isactrlbold {\isasymor}\ H{\isacharparenright}{\isachardoublequoteclose}\isanewline
\ \ \ \ \isacommand{have}\isamarkupfalse%
\ {\isachardoublequoteopen}Con\ {\isacharquery}F\ {\isacharparenleft}\isactrlbold {\isasymnot}G{\isacharparenright}\ {\isacharparenleft}\isactrlbold {\isasymnot}H{\isacharparenright}\ {\isasymlongrightarrow}\ {\isacharquery}F\ {\isasymin}\ S\ {\isasymlongrightarrow}\ {\isacharbraceleft}\isactrlbold {\isasymnot}G{\isacharcomma}\isactrlbold {\isasymnot}H{\isacharbraceright}\ {\isasymunion}\ S\ {\isasymin}\ C{\isachardoublequoteclose}\isanewline
\ \ \ \ \ \ \isacommand{using}\isamarkupfalse%
\ assms\ \isacommand{by}\isamarkupfalse%
\ {\isacharparenleft}iprover\ elim{\isacharcolon}\ allE{\isacharparenright}\isanewline
\ \ \ \ \isacommand{then}\isamarkupfalse%
\ \isacommand{have}\isamarkupfalse%
\ {\isachardoublequoteopen}{\isacharquery}F\ {\isasymin}\ S\ {\isasymlongrightarrow}\ {\isacharbraceleft}\isactrlbold {\isasymnot}G{\isacharcomma}\isactrlbold {\isasymnot}H{\isacharbraceright}\ {\isasymunion}\ S\ {\isasymin}\ C{\isachardoublequoteclose}\isanewline
\ \ \ \ \ \ \isacommand{using}\isamarkupfalse%
\ {\isacartoucheopen}Con\ {\isacharparenleft}\isactrlbold {\isasymnot}{\isacharparenleft}G\ \isactrlbold {\isasymor}\ H{\isacharparenright}{\isacharparenright}\ {\isacharparenleft}\isactrlbold {\isasymnot}G{\isacharparenright}\ {\isacharparenleft}\isactrlbold {\isasymnot}H{\isacharparenright}{\isacartoucheclose}\ \isacommand{by}\isamarkupfalse%
\ {\isacharparenleft}rule\ mp{\isacharparenright}\isanewline
\ \ \ \ \isacommand{thus}\isamarkupfalse%
\ {\isachardoublequoteopen}{\isacharbraceleft}\isactrlbold {\isasymnot}G{\isacharcomma}\isactrlbold {\isasymnot}H{\isacharbraceright}\ {\isasymunion}\ S\ {\isasymin}\ C{\isachardoublequoteclose}\isanewline
\ \ \ \ \ \ \isacommand{using}\isamarkupfalse%
\ {\isacartoucheopen}\isactrlbold {\isasymnot}{\isacharparenleft}G\ \isactrlbold {\isasymor}\ H{\isacharparenright}\ {\isasymin}\ S{\isacartoucheclose}\ \isacommand{by}\isamarkupfalse%
\ {\isacharparenleft}rule\ mp{\isacharparenright}\isanewline
\ \ \isacommand{qed}\isamarkupfalse%
\isanewline
\isacommand{qed}\isamarkupfalse%
%
\endisatagproof
{\isafoldproof}%
%
\isadelimproof
\isanewline
%
\endisadelimproof
\isanewline
\isacommand{lemma}\isamarkupfalse%
\ pcp{\isacharunderscore}alt{\isadigit{2}}Con{\isadigit{4}}{\isacharcolon}\isanewline
\ \ \isakeyword{assumes}\ {\isachardoublequoteopen}{\isasymforall}F\ G\ H{\isachardot}\ Con\ F\ G\ H\ {\isasymlongrightarrow}\ F\ {\isasymin}\ S\ {\isasymlongrightarrow}\ {\isacharbraceleft}G{\isacharcomma}H{\isacharbraceright}\ {\isasymunion}\ S\ {\isasymin}\ C{\isachardoublequoteclose}\isanewline
\ \ \isakeyword{shows}\ {\isachardoublequoteopen}{\isasymforall}G\ H{\isachardot}\ \isactrlbold {\isasymnot}{\isacharparenleft}G\ \isactrlbold {\isasymrightarrow}\ H{\isacharparenright}\ {\isasymin}\ S\ {\isasymlongrightarrow}\ {\isacharbraceleft}G{\isacharcomma}\isactrlbold {\isasymnot}\ H{\isacharbraceright}\ {\isasymunion}\ S\ {\isasymin}\ C{\isachardoublequoteclose}\isanewline
%
\isadelimproof
%
\endisadelimproof
%
\isatagproof
\isacommand{proof}\isamarkupfalse%
\ {\isacharparenleft}rule\ allI{\isacharparenright}{\isacharplus}\isanewline
\ \ \isacommand{fix}\isamarkupfalse%
\ G\ H\isanewline
\ \ \isacommand{show}\isamarkupfalse%
\ {\isachardoublequoteopen}\isactrlbold {\isasymnot}{\isacharparenleft}G\ \isactrlbold {\isasymrightarrow}\ H{\isacharparenright}\ {\isasymin}\ S\ {\isasymlongrightarrow}\ {\isacharbraceleft}G{\isacharcomma}\isactrlbold {\isasymnot}\ H{\isacharbraceright}\ {\isasymunion}\ S\ {\isasymin}\ C{\isachardoublequoteclose}\isanewline
\ \ \isacommand{proof}\isamarkupfalse%
\ {\isacharparenleft}rule\ impI{\isacharparenright}\isanewline
\ \ \ \ \isacommand{assume}\isamarkupfalse%
\ {\isachardoublequoteopen}\isactrlbold {\isasymnot}{\isacharparenleft}G\ \isactrlbold {\isasymrightarrow}\ H{\isacharparenright}\ {\isasymin}\ S{\isachardoublequoteclose}\isanewline
\ \ \ \ \isacommand{then}\isamarkupfalse%
\ \isacommand{have}\isamarkupfalse%
\ {\isachardoublequoteopen}Con\ {\isacharparenleft}\isactrlbold {\isasymnot}{\isacharparenleft}G\ \isactrlbold {\isasymrightarrow}\ H{\isacharparenright}{\isacharparenright}\ G\ {\isacharparenleft}\isactrlbold {\isasymnot}H{\isacharparenright}{\isachardoublequoteclose}\isanewline
\ \ \ \ \ \ \isacommand{by}\isamarkupfalse%
\ {\isacharparenleft}simp\ only{\isacharcolon}\ Con{\isachardot}intros{\isacharparenleft}{\isadigit{3}}{\isacharparenright}{\isacharparenright}\isanewline
\ \ \ \ \isacommand{let}\isamarkupfalse%
\ {\isacharquery}F\ {\isacharequal}\ {\isachardoublequoteopen}\isactrlbold {\isasymnot}{\isacharparenleft}G\ \isactrlbold {\isasymrightarrow}\ H{\isacharparenright}{\isachardoublequoteclose}\isanewline
\ \ \ \ \isacommand{have}\isamarkupfalse%
\ {\isachardoublequoteopen}Con\ {\isacharquery}F\ G\ {\isacharparenleft}\isactrlbold {\isasymnot}H{\isacharparenright}\ {\isasymlongrightarrow}\ {\isacharquery}F\ {\isasymin}\ S\ {\isasymlongrightarrow}\ {\isacharbraceleft}G{\isacharcomma}\isactrlbold {\isasymnot}H{\isacharbraceright}\ {\isasymunion}\ S\ {\isasymin}\ C{\isachardoublequoteclose}\isanewline
\ \ \ \ \ \ \isacommand{using}\isamarkupfalse%
\ assms\ \isacommand{by}\isamarkupfalse%
\ {\isacharparenleft}iprover\ elim{\isacharcolon}\ allE{\isacharparenright}\isanewline
\ \ \ \ \isacommand{then}\isamarkupfalse%
\ \isacommand{have}\isamarkupfalse%
\ {\isachardoublequoteopen}{\isacharquery}F\ {\isasymin}\ S\ {\isasymlongrightarrow}\ {\isacharbraceleft}G{\isacharcomma}\isactrlbold {\isasymnot}H{\isacharbraceright}\ {\isasymunion}\ S\ {\isasymin}\ C{\isachardoublequoteclose}\ \ \isanewline
\ \ \ \ \ \ \isacommand{using}\isamarkupfalse%
\ {\isacartoucheopen}Con\ {\isacharparenleft}\isactrlbold {\isasymnot}{\isacharparenleft}G\ \isactrlbold {\isasymrightarrow}\ H{\isacharparenright}{\isacharparenright}\ G\ {\isacharparenleft}\isactrlbold {\isasymnot}H{\isacharparenright}{\isacartoucheclose}\ \isacommand{by}\isamarkupfalse%
\ {\isacharparenleft}rule\ mp{\isacharparenright}\isanewline
\ \ \ \ \isacommand{thus}\isamarkupfalse%
\ {\isachardoublequoteopen}{\isacharbraceleft}G{\isacharcomma}\isactrlbold {\isasymnot}H{\isacharbraceright}\ {\isasymunion}\ S\ {\isasymin}\ C{\isachardoublequoteclose}\isanewline
\ \ \ \ \ \ \isacommand{using}\isamarkupfalse%
\ {\isacartoucheopen}\isactrlbold {\isasymnot}{\isacharparenleft}G\ \isactrlbold {\isasymrightarrow}\ H{\isacharparenright}\ {\isasymin}\ S{\isacartoucheclose}\ \isacommand{by}\isamarkupfalse%
\ {\isacharparenleft}rule\ mp{\isacharparenright}\isanewline
\ \ \isacommand{qed}\isamarkupfalse%
\isanewline
\isacommand{qed}\isamarkupfalse%
%
\endisatagproof
{\isafoldproof}%
%
\isadelimproof
%
\endisadelimproof
%
\begin{isamarkuptext}%
Por otro lado, los siguientes lemas auxiliares prueban el resto de condiciones de la
  definición de propiedad de consistencia proposicional a partir de la hipótesis referente a 
  fórmulas de tipo \isa{{\isasymbeta}}.%
\end{isamarkuptext}\isamarkuptrue%
\isacommand{lemma}\isamarkupfalse%
\ pcp{\isacharunderscore}alt{\isadigit{2}}Dis{\isadigit{1}}{\isacharcolon}\isanewline
\ \ \isakeyword{assumes}\ {\isachardoublequoteopen}{\isasymforall}F\ G\ H{\isachardot}\ Dis\ F\ G\ H\ {\isasymlongrightarrow}\ F\ {\isasymin}\ S\ {\isasymlongrightarrow}\ {\isacharbraceleft}G{\isacharbraceright}\ {\isasymunion}\ S\ {\isasymin}\ C\ {\isasymor}\ {\isacharbraceleft}H{\isacharbraceright}\ {\isasymunion}\ S\ {\isasymin}\ C{\isachardoublequoteclose}\isanewline
\ \ \isakeyword{shows}\ {\isachardoublequoteopen}{\isasymforall}G\ H{\isachardot}\ G\ \isactrlbold {\isasymor}\ H\ {\isasymin}\ S\ {\isasymlongrightarrow}\ {\isacharbraceleft}G{\isacharbraceright}\ {\isasymunion}\ S\ {\isasymin}\ C\ {\isasymor}\ {\isacharbraceleft}H{\isacharbraceright}\ {\isasymunion}\ S\ {\isasymin}\ C{\isachardoublequoteclose}\isanewline
%
\isadelimproof
%
\endisadelimproof
%
\isatagproof
\isacommand{proof}\isamarkupfalse%
\ {\isacharparenleft}rule\ allI{\isacharparenright}{\isacharplus}\isanewline
\ \ \isacommand{fix}\isamarkupfalse%
\ G\ H\isanewline
\ \ \isacommand{show}\isamarkupfalse%
\ {\isachardoublequoteopen}G\ \isactrlbold {\isasymor}\ H\ {\isasymin}\ S\ {\isasymlongrightarrow}\ {\isacharbraceleft}G{\isacharbraceright}\ {\isasymunion}\ S\ {\isasymin}\ C\ {\isasymor}\ {\isacharbraceleft}H{\isacharbraceright}\ {\isasymunion}\ S\ {\isasymin}\ C{\isachardoublequoteclose}\isanewline
\ \ \isacommand{proof}\isamarkupfalse%
\ {\isacharparenleft}rule\ impI{\isacharparenright}\isanewline
\ \ \ \ \isacommand{assume}\isamarkupfalse%
\ {\isachardoublequoteopen}G\ \isactrlbold {\isasymor}\ H\ {\isasymin}\ S{\isachardoublequoteclose}\isanewline
\ \ \ \ \isacommand{then}\isamarkupfalse%
\ \isacommand{have}\isamarkupfalse%
\ {\isachardoublequoteopen}Dis\ {\isacharparenleft}G\ \isactrlbold {\isasymor}\ H{\isacharparenright}\ G\ H{\isachardoublequoteclose}\isanewline
\ \ \ \ \ \ \isacommand{by}\isamarkupfalse%
\ {\isacharparenleft}simp\ only{\isacharcolon}\ Dis{\isachardot}intros{\isacharparenleft}{\isadigit{1}}{\isacharparenright}{\isacharparenright}\isanewline
\ \ \ \ \isacommand{let}\isamarkupfalse%
\ {\isacharquery}F{\isacharequal}{\isachardoublequoteopen}G\ \isactrlbold {\isasymor}\ H{\isachardoublequoteclose}\isanewline
\ \ \ \ \isacommand{have}\isamarkupfalse%
\ {\isachardoublequoteopen}Dis\ {\isacharquery}F\ G\ H\ {\isasymlongrightarrow}\ {\isacharquery}F\ {\isasymin}\ S\ {\isasymlongrightarrow}\ {\isacharbraceleft}G{\isacharbraceright}\ {\isasymunion}\ S\ {\isasymin}\ C\ {\isasymor}\ {\isacharbraceleft}H{\isacharbraceright}\ {\isasymunion}\ S\ {\isasymin}\ C{\isachardoublequoteclose}\isanewline
\ \ \ \ \ \ \isacommand{using}\isamarkupfalse%
\ assms\ \isacommand{by}\isamarkupfalse%
\ {\isacharparenleft}iprover\ elim{\isacharcolon}\ allE{\isacharparenright}\isanewline
\ \ \ \ \isacommand{then}\isamarkupfalse%
\ \isacommand{have}\isamarkupfalse%
\ {\isachardoublequoteopen}{\isacharquery}F\ {\isasymin}\ S\ {\isasymlongrightarrow}\ {\isacharbraceleft}G{\isacharbraceright}\ {\isasymunion}\ S\ {\isasymin}\ C\ {\isasymor}\ {\isacharbraceleft}H{\isacharbraceright}\ {\isasymunion}\ S\ {\isasymin}\ C{\isachardoublequoteclose}\isanewline
\ \ \ \ \ \ \isacommand{using}\isamarkupfalse%
\ {\isacartoucheopen}Dis\ {\isacharparenleft}G\ \isactrlbold {\isasymor}\ H{\isacharparenright}\ G\ H{\isacartoucheclose}\ \isacommand{by}\isamarkupfalse%
\ {\isacharparenleft}rule\ mp{\isacharparenright}\isanewline
\ \ \ \ \isacommand{thus}\isamarkupfalse%
\ {\isachardoublequoteopen}{\isacharbraceleft}G{\isacharbraceright}\ {\isasymunion}\ S\ {\isasymin}\ C\ {\isasymor}\ {\isacharbraceleft}H{\isacharbraceright}\ {\isasymunion}\ S\ {\isasymin}\ C{\isachardoublequoteclose}\isanewline
\ \ \ \ \ \ \isacommand{using}\isamarkupfalse%
\ {\isacartoucheopen}{\isacharparenleft}G\ \isactrlbold {\isasymor}\ H{\isacharparenright}\ {\isasymin}\ S{\isacartoucheclose}\ \isacommand{by}\isamarkupfalse%
\ {\isacharparenleft}rule\ mp{\isacharparenright}\isanewline
\ \ \isacommand{qed}\isamarkupfalse%
\isanewline
\isacommand{qed}\isamarkupfalse%
%
\endisatagproof
{\isafoldproof}%
%
\isadelimproof
\isanewline
%
\endisadelimproof
\isanewline
\isacommand{lemma}\isamarkupfalse%
\ pcp{\isacharunderscore}alt{\isadigit{2}}Dis{\isadigit{2}}{\isacharcolon}\isanewline
\ \ \isakeyword{assumes}\ {\isachardoublequoteopen}{\isasymforall}F\ G\ H{\isachardot}\ Dis\ F\ G\ H\ {\isasymlongrightarrow}\ F\ {\isasymin}\ S\ {\isasymlongrightarrow}\ {\isacharbraceleft}G{\isacharbraceright}\ {\isasymunion}\ S\ {\isasymin}\ C\ {\isasymor}\ {\isacharbraceleft}H{\isacharbraceright}\ {\isasymunion}\ S\ {\isasymin}\ C{\isachardoublequoteclose}\isanewline
\ \ \isakeyword{shows}\ {\isachardoublequoteopen}{\isasymforall}G\ H{\isachardot}\ G\ \isactrlbold {\isasymrightarrow}\ H\ {\isasymin}\ S\ {\isasymlongrightarrow}\ {\isacharbraceleft}\isactrlbold {\isasymnot}\ G{\isacharbraceright}\ {\isasymunion}\ S\ {\isasymin}\ C\ {\isasymor}\ {\isacharbraceleft}H{\isacharbraceright}\ {\isasymunion}\ S\ {\isasymin}\ C{\isachardoublequoteclose}\isanewline
%
\isadelimproof
%
\endisadelimproof
%
\isatagproof
\isacommand{proof}\isamarkupfalse%
\ {\isacharparenleft}rule\ allI{\isacharparenright}{\isacharplus}\isanewline
\ \ \isacommand{fix}\isamarkupfalse%
\ G\ H\isanewline
\ \ \isacommand{show}\isamarkupfalse%
\ {\isachardoublequoteopen}G\ \isactrlbold {\isasymrightarrow}\ H\ {\isasymin}\ S\ {\isasymlongrightarrow}\ {\isacharbraceleft}\isactrlbold {\isasymnot}\ G{\isacharbraceright}\ {\isasymunion}\ S\ {\isasymin}\ C\ {\isasymor}\ {\isacharbraceleft}H{\isacharbraceright}\ {\isasymunion}\ S\ {\isasymin}\ C{\isachardoublequoteclose}\isanewline
\ \ \isacommand{proof}\isamarkupfalse%
\ {\isacharparenleft}rule\ impI{\isacharparenright}\isanewline
\ \ \ \ \isacommand{assume}\isamarkupfalse%
\ {\isachardoublequoteopen}G\ \isactrlbold {\isasymrightarrow}\ H\ {\isasymin}\ S{\isachardoublequoteclose}\isanewline
\ \ \ \ \isacommand{then}\isamarkupfalse%
\ \isacommand{have}\isamarkupfalse%
\ {\isachardoublequoteopen}Dis\ {\isacharparenleft}G\ \isactrlbold {\isasymrightarrow}\ H{\isacharparenright}\ {\isacharparenleft}\isactrlbold {\isasymnot}G{\isacharparenright}\ H{\isachardoublequoteclose}\isanewline
\ \ \ \ \ \ \isacommand{by}\isamarkupfalse%
\ {\isacharparenleft}simp\ only{\isacharcolon}\ Dis{\isachardot}intros{\isacharparenleft}{\isadigit{2}}{\isacharparenright}{\isacharparenright}\isanewline
\ \ \ \ \isacommand{let}\isamarkupfalse%
\ {\isacharquery}F{\isacharequal}{\isachardoublequoteopen}G\ \isactrlbold {\isasymrightarrow}\ H{\isachardoublequoteclose}\ \isanewline
\ \ \ \ \isacommand{have}\isamarkupfalse%
\ {\isachardoublequoteopen}Dis\ {\isacharquery}F\ {\isacharparenleft}\isactrlbold {\isasymnot}G{\isacharparenright}\ H\ {\isasymlongrightarrow}\ {\isacharquery}F\ {\isasymin}\ S\ {\isasymlongrightarrow}\ {\isacharbraceleft}\isactrlbold {\isasymnot}G{\isacharbraceright}\ {\isasymunion}\ S\ {\isasymin}\ C\ {\isasymor}\ {\isacharbraceleft}H{\isacharbraceright}\ {\isasymunion}\ S\ {\isasymin}\ C{\isachardoublequoteclose}\isanewline
\ \ \ \ \ \ \isacommand{using}\isamarkupfalse%
\ assms\ \isacommand{by}\isamarkupfalse%
\ {\isacharparenleft}iprover\ elim{\isacharcolon}\ allE{\isacharparenright}\isanewline
\ \ \ \ \isacommand{then}\isamarkupfalse%
\ \isacommand{have}\isamarkupfalse%
\ {\isachardoublequoteopen}{\isacharquery}F\ {\isasymin}\ S\ {\isasymlongrightarrow}\ {\isacharbraceleft}\isactrlbold {\isasymnot}G{\isacharbraceright}\ {\isasymunion}\ S\ {\isasymin}\ C\ {\isasymor}\ {\isacharbraceleft}H{\isacharbraceright}\ {\isasymunion}\ S\ {\isasymin}\ C{\isachardoublequoteclose}\isanewline
\ \ \ \ \ \ \isacommand{using}\isamarkupfalse%
\ {\isacartoucheopen}Dis\ {\isacharparenleft}G\ \isactrlbold {\isasymrightarrow}\ H{\isacharparenright}\ {\isacharparenleft}\isactrlbold {\isasymnot}G{\isacharparenright}\ H{\isacartoucheclose}\ \isacommand{by}\isamarkupfalse%
\ {\isacharparenleft}rule\ mp{\isacharparenright}\isanewline
\ \ \ \ \isacommand{thus}\isamarkupfalse%
\ {\isachardoublequoteopen}{\isacharbraceleft}\isactrlbold {\isasymnot}G{\isacharbraceright}\ {\isasymunion}\ S\ {\isasymin}\ C\ {\isasymor}\ {\isacharbraceleft}H{\isacharbraceright}\ {\isasymunion}\ S\ {\isasymin}\ C{\isachardoublequoteclose}\isanewline
\ \ \ \ \ \ \isacommand{using}\isamarkupfalse%
\ {\isacartoucheopen}{\isacharparenleft}G\ \isactrlbold {\isasymrightarrow}\ H{\isacharparenright}\ {\isasymin}\ S{\isacartoucheclose}\ \isacommand{by}\isamarkupfalse%
\ {\isacharparenleft}rule\ mp{\isacharparenright}\isanewline
\ \ \isacommand{qed}\isamarkupfalse%
\isanewline
\isacommand{qed}\isamarkupfalse%
%
\endisatagproof
{\isafoldproof}%
%
\isadelimproof
\isanewline
%
\endisadelimproof
\isanewline
\isacommand{lemma}\isamarkupfalse%
\ pcp{\isacharunderscore}alt{\isadigit{2}}Dis{\isadigit{3}}{\isacharcolon}\isanewline
\ \ \isakeyword{assumes}\ {\isachardoublequoteopen}{\isasymforall}F\ G\ H{\isachardot}\ Dis\ F\ G\ H\ {\isasymlongrightarrow}\ F\ {\isasymin}\ S\ {\isasymlongrightarrow}\ {\isacharbraceleft}G{\isacharbraceright}\ {\isasymunion}\ S\ {\isasymin}\ C\ {\isasymor}\ {\isacharbraceleft}H{\isacharbraceright}\ {\isasymunion}\ S\ {\isasymin}\ C{\isachardoublequoteclose}\isanewline
\ \ \isakeyword{shows}\ {\isachardoublequoteopen}{\isasymforall}G\ H{\isachardot}\ \isactrlbold {\isasymnot}{\isacharparenleft}G\ \isactrlbold {\isasymand}\ H{\isacharparenright}\ {\isasymin}\ S\ {\isasymlongrightarrow}\ {\isacharbraceleft}\isactrlbold {\isasymnot}\ G{\isacharbraceright}\ {\isasymunion}\ S\ {\isasymin}\ C\ {\isasymor}\ {\isacharbraceleft}\isactrlbold {\isasymnot}\ H{\isacharbraceright}\ {\isasymunion}\ S\ {\isasymin}\ C{\isachardoublequoteclose}\isanewline
%
\isadelimproof
%
\endisadelimproof
%
\isatagproof
\isacommand{proof}\isamarkupfalse%
\ {\isacharparenleft}rule\ allI{\isacharparenright}{\isacharplus}\isanewline
\ \ \isacommand{fix}\isamarkupfalse%
\ G\ H\isanewline
\ \ \isacommand{show}\isamarkupfalse%
\ {\isachardoublequoteopen}\isactrlbold {\isasymnot}{\isacharparenleft}G\ \isactrlbold {\isasymand}\ H{\isacharparenright}\ {\isasymin}\ S\ {\isasymlongrightarrow}\ {\isacharbraceleft}\isactrlbold {\isasymnot}\ G{\isacharbraceright}\ {\isasymunion}\ S\ {\isasymin}\ C\ {\isasymor}\ {\isacharbraceleft}\isactrlbold {\isasymnot}\ H{\isacharbraceright}\ {\isasymunion}\ S\ {\isasymin}\ C{\isachardoublequoteclose}\isanewline
\ \ \isacommand{proof}\isamarkupfalse%
\ {\isacharparenleft}rule\ impI{\isacharparenright}\isanewline
\ \ \ \ \isacommand{assume}\isamarkupfalse%
\ {\isachardoublequoteopen}\isactrlbold {\isasymnot}{\isacharparenleft}G\ \isactrlbold {\isasymand}\ H{\isacharparenright}\ {\isasymin}\ S{\isachardoublequoteclose}\isanewline
\ \ \ \ \isacommand{then}\isamarkupfalse%
\ \isacommand{have}\isamarkupfalse%
\ {\isachardoublequoteopen}Dis\ {\isacharparenleft}\isactrlbold {\isasymnot}{\isacharparenleft}G\ \isactrlbold {\isasymand}\ H{\isacharparenright}{\isacharparenright}\ {\isacharparenleft}\isactrlbold {\isasymnot}G{\isacharparenright}\ {\isacharparenleft}\isactrlbold {\isasymnot}H{\isacharparenright}{\isachardoublequoteclose}\isanewline
\ \ \ \ \ \ \isacommand{by}\isamarkupfalse%
\ {\isacharparenleft}simp\ only{\isacharcolon}\ Dis{\isachardot}intros{\isacharparenleft}{\isadigit{3}}{\isacharparenright}{\isacharparenright}\isanewline
\ \ \ \ \isacommand{let}\isamarkupfalse%
\ {\isacharquery}F{\isacharequal}{\isachardoublequoteopen}\isactrlbold {\isasymnot}{\isacharparenleft}G\ \isactrlbold {\isasymand}\ H{\isacharparenright}{\isachardoublequoteclose}\isanewline
\ \ \ \ \isacommand{have}\isamarkupfalse%
\ {\isachardoublequoteopen}Dis\ {\isacharquery}F\ {\isacharparenleft}\isactrlbold {\isasymnot}G{\isacharparenright}\ {\isacharparenleft}\isactrlbold {\isasymnot}H{\isacharparenright}\ {\isasymlongrightarrow}\ {\isacharquery}F\ {\isasymin}\ S\ {\isasymlongrightarrow}\ {\isacharbraceleft}\isactrlbold {\isasymnot}G{\isacharbraceright}\ {\isasymunion}\ S\ {\isasymin}\ C\ {\isasymor}\ {\isacharbraceleft}\isactrlbold {\isasymnot}H{\isacharbraceright}\ {\isasymunion}\ S\ {\isasymin}\ C{\isachardoublequoteclose}\isanewline
\ \ \ \ \ \ \isacommand{using}\isamarkupfalse%
\ assms\ \isacommand{by}\isamarkupfalse%
\ {\isacharparenleft}iprover\ elim{\isacharcolon}\ allE{\isacharparenright}\isanewline
\ \ \ \ \isacommand{then}\isamarkupfalse%
\ \isacommand{have}\isamarkupfalse%
\ {\isachardoublequoteopen}{\isacharquery}F\ {\isasymin}\ S\ {\isasymlongrightarrow}\ {\isacharbraceleft}\isactrlbold {\isasymnot}G{\isacharbraceright}\ {\isasymunion}\ S\ {\isasymin}\ C\ {\isasymor}\ {\isacharbraceleft}\isactrlbold {\isasymnot}H{\isacharbraceright}\ {\isasymunion}\ S\ {\isasymin}\ C{\isachardoublequoteclose}\isanewline
\ \ \ \ \ \ \isacommand{using}\isamarkupfalse%
\ {\isacartoucheopen}Dis\ {\isacharparenleft}\isactrlbold {\isasymnot}{\isacharparenleft}G\ \isactrlbold {\isasymand}\ H{\isacharparenright}{\isacharparenright}\ {\isacharparenleft}\isactrlbold {\isasymnot}G{\isacharparenright}\ {\isacharparenleft}\isactrlbold {\isasymnot}H{\isacharparenright}{\isacartoucheclose}\ \isacommand{by}\isamarkupfalse%
\ {\isacharparenleft}rule\ mp{\isacharparenright}\isanewline
\ \ \ \ \isacommand{thus}\isamarkupfalse%
\ {\isachardoublequoteopen}{\isacharbraceleft}\isactrlbold {\isasymnot}G{\isacharbraceright}\ {\isasymunion}\ S\ {\isasymin}\ C\ {\isasymor}\ {\isacharbraceleft}\isactrlbold {\isasymnot}H{\isacharbraceright}\ {\isasymunion}\ S\ {\isasymin}\ C{\isachardoublequoteclose}\isanewline
\ \ \ \ \ \ \isacommand{using}\isamarkupfalse%
\ {\isacartoucheopen}\isactrlbold {\isasymnot}{\isacharparenleft}G\ \isactrlbold {\isasymand}\ H{\isacharparenright}\ {\isasymin}\ S{\isacartoucheclose}\ \isacommand{by}\isamarkupfalse%
\ {\isacharparenleft}rule\ mp{\isacharparenright}\isanewline
\ \ \isacommand{qed}\isamarkupfalse%
\isanewline
\isacommand{qed}\isamarkupfalse%
%
\endisatagproof
{\isafoldproof}%
%
\isadelimproof
%
\endisadelimproof
%
\begin{isamarkuptext}%
De este modo, procedemos a la demostración detallada de esta implicación en Isabelle.%
\end{isamarkuptext}\isamarkuptrue%
\isacommand{lemma}\isamarkupfalse%
\ pcp{\isacharunderscore}alt{\isadigit{2}}{\isacharcolon}\ \isanewline
\ \ \isakeyword{assumes}\ {\isachardoublequoteopen}{\isasymforall}S\ {\isasymin}\ C{\isachardot}\ {\isasymbottom}\ {\isasymnotin}\ S\isanewline
{\isasymand}\ {\isacharparenleft}{\isasymforall}k{\isachardot}\ Atom\ k\ {\isasymin}\ S\ {\isasymlongrightarrow}\ \isactrlbold {\isasymnot}\ {\isacharparenleft}Atom\ k{\isacharparenright}\ {\isasymin}\ S\ {\isasymlongrightarrow}\ False{\isacharparenright}\isanewline
{\isasymand}\ {\isacharparenleft}{\isasymforall}F\ G\ H{\isachardot}\ Con\ F\ G\ H\ {\isasymlongrightarrow}\ F\ {\isasymin}\ S\ {\isasymlongrightarrow}\ {\isacharbraceleft}G{\isacharcomma}H{\isacharbraceright}\ {\isasymunion}\ S\ {\isasymin}\ C{\isacharparenright}\isanewline
{\isasymand}\ {\isacharparenleft}{\isasymforall}F\ G\ H{\isachardot}\ Dis\ F\ G\ H\ {\isasymlongrightarrow}\ F\ {\isasymin}\ S\ {\isasymlongrightarrow}\ {\isacharbraceleft}G{\isacharbraceright}\ {\isasymunion}\ S\ {\isasymin}\ C\ {\isasymor}\ {\isacharbraceleft}H{\isacharbraceright}\ {\isasymunion}\ S\ {\isasymin}\ C{\isacharparenright}{\isachardoublequoteclose}\isanewline
\ \ \isakeyword{shows}\ {\isachardoublequoteopen}pcp\ C{\isachardoublequoteclose}\isanewline
%
\isadelimproof
\ \ %
\endisadelimproof
%
\isatagproof
\isacommand{unfolding}\isamarkupfalse%
\ pcp{\isacharunderscore}def\isanewline
\isacommand{proof}\isamarkupfalse%
\ {\isacharparenleft}rule\ ballI{\isacharparenright}\isanewline
\ \ \isacommand{fix}\isamarkupfalse%
\ S\isanewline
\ \ \isacommand{assume}\isamarkupfalse%
\ {\isachardoublequoteopen}S\ {\isasymin}\ C{\isachardoublequoteclose}\isanewline
\ \ \isacommand{have}\isamarkupfalse%
\ H{\isacharcolon}{\isachardoublequoteopen}{\isasymbottom}\ {\isasymnotin}\ S\isanewline
\ \ \ \ {\isasymand}\ {\isacharparenleft}{\isasymforall}k{\isachardot}\ Atom\ k\ {\isasymin}\ S\ {\isasymlongrightarrow}\ \isactrlbold {\isasymnot}\ {\isacharparenleft}Atom\ k{\isacharparenright}\ {\isasymin}\ S\ {\isasymlongrightarrow}\ False{\isacharparenright}\isanewline
\ \ \ \ {\isasymand}\ {\isacharparenleft}{\isasymforall}F\ G\ H{\isachardot}\ Con\ F\ G\ H\ {\isasymlongrightarrow}\ F\ {\isasymin}\ S\ {\isasymlongrightarrow}\ {\isacharbraceleft}G{\isacharcomma}H{\isacharbraceright}\ {\isasymunion}\ S\ {\isasymin}\ C{\isacharparenright}\isanewline
\ \ \ \ {\isasymand}\ {\isacharparenleft}{\isasymforall}F\ G\ H{\isachardot}\ Dis\ F\ G\ H\ {\isasymlongrightarrow}\ F\ {\isasymin}\ S\ {\isasymlongrightarrow}\ {\isacharbraceleft}G{\isacharbraceright}\ {\isasymunion}\ S\ {\isasymin}\ C\ {\isasymor}\ {\isacharbraceleft}H{\isacharbraceright}\ {\isasymunion}\ S\ {\isasymin}\ C{\isacharparenright}{\isachardoublequoteclose}\isanewline
\ \ \ \ \isacommand{using}\isamarkupfalse%
\ assms\ {\isacartoucheopen}S\ {\isasymin}\ C{\isacartoucheclose}\ \isacommand{by}\isamarkupfalse%
\ {\isacharparenleft}rule\ bspec{\isacharparenright}\isanewline
\ \ \isacommand{then}\isamarkupfalse%
\ \isacommand{have}\isamarkupfalse%
\ Con{\isacharcolon}{\isachardoublequoteopen}{\isasymforall}F\ G\ H{\isachardot}\ Con\ F\ G\ H\ {\isasymlongrightarrow}\ F\ {\isasymin}\ S\ {\isasymlongrightarrow}\ {\isacharbraceleft}G{\isacharcomma}H{\isacharbraceright}\ {\isasymunion}\ S\ {\isasymin}\ C{\isachardoublequoteclose}\isanewline
\ \ \ \ \isacommand{by}\isamarkupfalse%
\ {\isacharparenleft}iprover\ elim{\isacharcolon}\ conjunct{\isadigit{1}}\ conjunct{\isadigit{2}}{\isacharparenright}\isanewline
\ \ \isacommand{have}\isamarkupfalse%
\ Dis{\isacharcolon}{\isachardoublequoteopen}{\isasymforall}F\ G\ H{\isachardot}\ Dis\ F\ G\ H\ {\isasymlongrightarrow}\ F\ {\isasymin}\ S\ {\isasymlongrightarrow}\ {\isacharbraceleft}G{\isacharbraceright}\ {\isasymunion}\ S\ {\isasymin}\ C\ {\isasymor}\ {\isacharbraceleft}H{\isacharbraceright}\ {\isasymunion}\ S\ {\isasymin}\ C{\isachardoublequoteclose}\isanewline
\ \ \ \ \isacommand{using}\isamarkupfalse%
\ H\ \isacommand{by}\isamarkupfalse%
\ {\isacharparenleft}iprover\ elim{\isacharcolon}\ conjunct{\isadigit{1}}\ conjunct{\isadigit{2}}{\isacharparenright}\isanewline
\ \ \isacommand{have}\isamarkupfalse%
\ {\isadigit{1}}{\isacharcolon}{\isachardoublequoteopen}{\isasymbottom}\ {\isasymnotin}\ S\isanewline
\ \ \ \ {\isasymand}\ {\isacharparenleft}{\isasymforall}k{\isachardot}\ Atom\ k\ {\isasymin}\ S\ {\isasymlongrightarrow}\ \isactrlbold {\isasymnot}\ {\isacharparenleft}Atom\ k{\isacharparenright}\ {\isasymin}\ S\ {\isasymlongrightarrow}\ False{\isacharparenright}{\isachardoublequoteclose}\isanewline
\ \ \ \ \isacommand{using}\isamarkupfalse%
\ H\ \isacommand{by}\isamarkupfalse%
\ {\isacharparenleft}iprover\ elim{\isacharcolon}\ conjunct{\isadigit{1}}{\isacharparenright}\isanewline
\ \ \isacommand{have}\isamarkupfalse%
\ {\isadigit{2}}{\isacharcolon}{\isachardoublequoteopen}{\isasymforall}G\ H{\isachardot}\ G\ \isactrlbold {\isasymand}\ H\ {\isasymin}\ S\ {\isasymlongrightarrow}\ {\isacharbraceleft}G{\isacharcomma}H{\isacharbraceright}\ {\isasymunion}\ S\ {\isasymin}\ C{\isachardoublequoteclose}\isanewline
\ \ \ \ \isacommand{using}\isamarkupfalse%
\ Con\ \isacommand{by}\isamarkupfalse%
\ {\isacharparenleft}rule\ pcp{\isacharunderscore}alt{\isadigit{2}}Con{\isadigit{1}}{\isacharparenright}\isanewline
\ \ \isacommand{have}\isamarkupfalse%
\ {\isadigit{3}}{\isacharcolon}{\isachardoublequoteopen}{\isasymforall}G\ H{\isachardot}\ G\ \isactrlbold {\isasymor}\ H\ {\isasymin}\ S\ {\isasymlongrightarrow}\ {\isacharbraceleft}G{\isacharbraceright}\ {\isasymunion}\ S\ {\isasymin}\ C\ {\isasymor}\ {\isacharbraceleft}H{\isacharbraceright}\ {\isasymunion}\ S\ {\isasymin}\ C{\isachardoublequoteclose}\isanewline
\ \ \ \ \isacommand{using}\isamarkupfalse%
\ Dis\ \isacommand{by}\isamarkupfalse%
\ {\isacharparenleft}rule\ pcp{\isacharunderscore}alt{\isadigit{2}}Dis{\isadigit{1}}{\isacharparenright}\isanewline
\ \ \isacommand{have}\isamarkupfalse%
\ {\isadigit{4}}{\isacharcolon}{\isachardoublequoteopen}{\isasymforall}G\ H{\isachardot}\ G\ \isactrlbold {\isasymrightarrow}\ H\ {\isasymin}\ S\ {\isasymlongrightarrow}\ {\isacharbraceleft}\isactrlbold {\isasymnot}G{\isacharbraceright}\ {\isasymunion}\ S\ {\isasymin}\ C\ {\isasymor}\ {\isacharbraceleft}H{\isacharbraceright}\ {\isasymunion}\ S\ {\isasymin}\ C{\isachardoublequoteclose}\isanewline
\ \ \ \ \isacommand{using}\isamarkupfalse%
\ Dis\ \isacommand{by}\isamarkupfalse%
\ {\isacharparenleft}rule\ pcp{\isacharunderscore}alt{\isadigit{2}}Dis{\isadigit{2}}{\isacharparenright}\isanewline
\ \ \isacommand{have}\isamarkupfalse%
\ {\isadigit{5}}{\isacharcolon}{\isachardoublequoteopen}{\isasymforall}G{\isachardot}\ \isactrlbold {\isasymnot}\ {\isacharparenleft}\isactrlbold {\isasymnot}G{\isacharparenright}\ {\isasymin}\ S\ {\isasymlongrightarrow}\ {\isacharbraceleft}G{\isacharbraceright}\ {\isasymunion}\ S\ {\isasymin}\ C{\isachardoublequoteclose}\isanewline
\ \ \ \ \isacommand{using}\isamarkupfalse%
\ Con\ \isacommand{by}\isamarkupfalse%
\ {\isacharparenleft}rule\ pcp{\isacharunderscore}alt{\isadigit{2}}Con{\isadigit{2}}{\isacharparenright}\isanewline
\ \ \isacommand{have}\isamarkupfalse%
\ {\isadigit{6}}{\isacharcolon}{\isachardoublequoteopen}{\isasymforall}G\ H{\isachardot}\ \isactrlbold {\isasymnot}{\isacharparenleft}G\ \isactrlbold {\isasymand}\ H{\isacharparenright}\ {\isasymin}\ S\ {\isasymlongrightarrow}\ {\isacharbraceleft}\isactrlbold {\isasymnot}\ G{\isacharbraceright}\ {\isasymunion}\ S\ {\isasymin}\ C\ {\isasymor}\ {\isacharbraceleft}\isactrlbold {\isasymnot}\ H{\isacharbraceright}\ {\isasymunion}\ S\ {\isasymin}\ C{\isachardoublequoteclose}\isanewline
\ \ \ \ \isacommand{using}\isamarkupfalse%
\ Dis\ \isacommand{by}\isamarkupfalse%
\ {\isacharparenleft}rule\ pcp{\isacharunderscore}alt{\isadigit{2}}Dis{\isadigit{3}}{\isacharparenright}\isanewline
\ \ \isacommand{have}\isamarkupfalse%
\ {\isadigit{7}}{\isacharcolon}{\isachardoublequoteopen}{\isasymforall}G\ H{\isachardot}\ \isactrlbold {\isasymnot}{\isacharparenleft}G\ \isactrlbold {\isasymor}\ H{\isacharparenright}\ {\isasymin}\ S\ {\isasymlongrightarrow}\ {\isacharbraceleft}\isactrlbold {\isasymnot}\ G{\isacharcomma}\ \isactrlbold {\isasymnot}\ H{\isacharbraceright}\ {\isasymunion}\ S\ {\isasymin}\ C{\isachardoublequoteclose}\isanewline
\ \ \ \ \isacommand{using}\isamarkupfalse%
\ Con\ \isacommand{by}\isamarkupfalse%
\ {\isacharparenleft}rule\ pcp{\isacharunderscore}alt{\isadigit{2}}Con{\isadigit{3}}{\isacharparenright}\isanewline
\ \ \isacommand{have}\isamarkupfalse%
\ {\isadigit{8}}{\isacharcolon}{\isachardoublequoteopen}{\isasymforall}G\ H{\isachardot}\ \isactrlbold {\isasymnot}{\isacharparenleft}G\ \isactrlbold {\isasymrightarrow}\ H{\isacharparenright}\ {\isasymin}\ S\ {\isasymlongrightarrow}\ {\isacharbraceleft}G{\isacharcomma}\isactrlbold {\isasymnot}\ H{\isacharbraceright}\ {\isasymunion}\ S\ {\isasymin}\ C{\isachardoublequoteclose}\isanewline
\ \ \ \ \isacommand{using}\isamarkupfalse%
\ Con\ \isacommand{by}\isamarkupfalse%
\ {\isacharparenleft}rule\ pcp{\isacharunderscore}alt{\isadigit{2}}Con{\isadigit{4}}{\isacharparenright}\isanewline
\ \ \isacommand{have}\isamarkupfalse%
\ A{\isacharcolon}{\isachardoublequoteopen}{\isasymbottom}\ {\isasymnotin}\ S\isanewline
\ \ \ \ {\isasymand}\ {\isacharparenleft}{\isasymforall}k{\isachardot}\ Atom\ k\ {\isasymin}\ S\ {\isasymlongrightarrow}\ \isactrlbold {\isasymnot}\ {\isacharparenleft}Atom\ k{\isacharparenright}\ {\isasymin}\ S\ {\isasymlongrightarrow}\ False{\isacharparenright}\isanewline
\ \ \ \ {\isasymand}\ {\isacharparenleft}{\isasymforall}G\ H{\isachardot}\ G\ \isactrlbold {\isasymand}\ H\ {\isasymin}\ S\ {\isasymlongrightarrow}\ {\isacharbraceleft}G{\isacharcomma}H{\isacharbraceright}\ {\isasymunion}\ S\ {\isasymin}\ C{\isacharparenright}\isanewline
\ \ \ \ {\isasymand}\ {\isacharparenleft}{\isasymforall}G\ H{\isachardot}\ G\ \isactrlbold {\isasymor}\ H\ {\isasymin}\ S\ {\isasymlongrightarrow}\ {\isacharbraceleft}G{\isacharbraceright}\ {\isasymunion}\ S\ {\isasymin}\ C\ {\isasymor}\ {\isacharbraceleft}H{\isacharbraceright}\ {\isasymunion}\ S\ {\isasymin}\ C{\isacharparenright}\isanewline
\ \ \ \ {\isasymand}\ {\isacharparenleft}{\isasymforall}G\ H{\isachardot}\ G\ \isactrlbold {\isasymrightarrow}\ H\ {\isasymin}\ S\ {\isasymlongrightarrow}\ {\isacharbraceleft}\isactrlbold {\isasymnot}G{\isacharbraceright}\ {\isasymunion}\ S\ {\isasymin}\ C\ {\isasymor}\ {\isacharbraceleft}H{\isacharbraceright}\ {\isasymunion}\ S\ {\isasymin}\ C{\isacharparenright}{\isachardoublequoteclose}\isanewline
\ \ \ \ \isacommand{using}\isamarkupfalse%
\ {\isadigit{1}}\ {\isadigit{2}}\ {\isadigit{3}}\ {\isadigit{4}}\ \isacommand{by}\isamarkupfalse%
\ {\isacharparenleft}iprover\ intro{\isacharcolon}\ conjI{\isacharparenright}\isanewline
\ \ \isacommand{have}\isamarkupfalse%
\ B{\isacharcolon}{\isachardoublequoteopen}{\isacharparenleft}{\isasymforall}G{\isachardot}\ \isactrlbold {\isasymnot}\ {\isacharparenleft}\isactrlbold {\isasymnot}G{\isacharparenright}\ {\isasymin}\ S\ {\isasymlongrightarrow}\ {\isacharbraceleft}G{\isacharbraceright}\ {\isasymunion}\ S\ {\isasymin}\ C{\isacharparenright}\isanewline
\ \ \ \ {\isasymand}\ {\isacharparenleft}{\isasymforall}G\ H{\isachardot}\ \isactrlbold {\isasymnot}{\isacharparenleft}G\ \isactrlbold {\isasymand}\ H{\isacharparenright}\ {\isasymin}\ S\ {\isasymlongrightarrow}\ {\isacharbraceleft}\isactrlbold {\isasymnot}\ G{\isacharbraceright}\ {\isasymunion}\ S\ {\isasymin}\ C\ {\isasymor}\ {\isacharbraceleft}\isactrlbold {\isasymnot}\ H{\isacharbraceright}\ {\isasymunion}\ S\ {\isasymin}\ C{\isacharparenright}\isanewline
\ \ \ \ {\isasymand}\ {\isacharparenleft}{\isasymforall}G\ H{\isachardot}\ \isactrlbold {\isasymnot}{\isacharparenleft}G\ \isactrlbold {\isasymor}\ H{\isacharparenright}\ {\isasymin}\ S\ {\isasymlongrightarrow}\ {\isacharbraceleft}\isactrlbold {\isasymnot}\ G{\isacharcomma}\ \isactrlbold {\isasymnot}\ H{\isacharbraceright}\ {\isasymunion}\ S\ {\isasymin}\ C{\isacharparenright}\isanewline
\ \ \ \ {\isasymand}\ {\isacharparenleft}{\isasymforall}G\ H{\isachardot}\ \isactrlbold {\isasymnot}{\isacharparenleft}G\ \isactrlbold {\isasymrightarrow}\ H{\isacharparenright}\ {\isasymin}\ S\ {\isasymlongrightarrow}\ {\isacharbraceleft}G{\isacharcomma}\isactrlbold {\isasymnot}\ H{\isacharbraceright}\ {\isasymunion}\ S\ {\isasymin}\ C{\isacharparenright}{\isachardoublequoteclose}\isanewline
\ \ \ \ \isacommand{using}\isamarkupfalse%
\ {\isadigit{5}}\ {\isadigit{6}}\ {\isadigit{7}}\ {\isadigit{8}}\ \isacommand{by}\isamarkupfalse%
\ {\isacharparenleft}iprover\ intro{\isacharcolon}\ conjI{\isacharparenright}\isanewline
\ \ \isacommand{have}\isamarkupfalse%
\ {\isachardoublequoteopen}{\isacharparenleft}{\isasymbottom}\ {\isasymnotin}\ S\isanewline
\ \ \ \ {\isasymand}\ {\isacharparenleft}{\isasymforall}k{\isachardot}\ Atom\ k\ {\isasymin}\ S\ {\isasymlongrightarrow}\ \isactrlbold {\isasymnot}\ {\isacharparenleft}Atom\ k{\isacharparenright}\ {\isasymin}\ S\ {\isasymlongrightarrow}\ False{\isacharparenright}\isanewline
\ \ \ \ {\isasymand}\ {\isacharparenleft}{\isasymforall}G\ H{\isachardot}\ G\ \isactrlbold {\isasymand}\ H\ {\isasymin}\ S\ {\isasymlongrightarrow}\ {\isacharbraceleft}G{\isacharcomma}H{\isacharbraceright}\ {\isasymunion}\ S\ {\isasymin}\ C{\isacharparenright}\isanewline
\ \ \ \ {\isasymand}\ {\isacharparenleft}{\isasymforall}G\ H{\isachardot}\ G\ \isactrlbold {\isasymor}\ H\ {\isasymin}\ S\ {\isasymlongrightarrow}\ {\isacharbraceleft}G{\isacharbraceright}\ {\isasymunion}\ S\ {\isasymin}\ C\ {\isasymor}\ {\isacharbraceleft}H{\isacharbraceright}\ {\isasymunion}\ S\ {\isasymin}\ C{\isacharparenright}\isanewline
\ \ \ \ {\isasymand}\ {\isacharparenleft}{\isasymforall}G\ H{\isachardot}\ G\ \isactrlbold {\isasymrightarrow}\ H\ {\isasymin}\ S\ {\isasymlongrightarrow}\ {\isacharbraceleft}\isactrlbold {\isasymnot}G{\isacharbraceright}\ {\isasymunion}\ S\ {\isasymin}\ C\ {\isasymor}\ {\isacharbraceleft}H{\isacharbraceright}\ {\isasymunion}\ S\ {\isasymin}\ C{\isacharparenright}{\isacharparenright}\isanewline
\ \ \ \ {\isasymand}\ {\isacharparenleft}{\isacharparenleft}{\isasymforall}G{\isachardot}\ \isactrlbold {\isasymnot}\ {\isacharparenleft}\isactrlbold {\isasymnot}G{\isacharparenright}\ {\isasymin}\ S\ {\isasymlongrightarrow}\ {\isacharbraceleft}G{\isacharbraceright}\ {\isasymunion}\ S\ {\isasymin}\ C{\isacharparenright}\isanewline
\ \ \ \ {\isasymand}\ {\isacharparenleft}{\isasymforall}G\ H{\isachardot}\ \isactrlbold {\isasymnot}{\isacharparenleft}G\ \isactrlbold {\isasymand}\ H{\isacharparenright}\ {\isasymin}\ S\ {\isasymlongrightarrow}\ {\isacharbraceleft}\isactrlbold {\isasymnot}\ G{\isacharbraceright}\ {\isasymunion}\ S\ {\isasymin}\ C\ {\isasymor}\ {\isacharbraceleft}\isactrlbold {\isasymnot}\ H{\isacharbraceright}\ {\isasymunion}\ S\ {\isasymin}\ C{\isacharparenright}\isanewline
\ \ \ \ {\isasymand}\ {\isacharparenleft}{\isasymforall}G\ H{\isachardot}\ \isactrlbold {\isasymnot}{\isacharparenleft}G\ \isactrlbold {\isasymor}\ H{\isacharparenright}\ {\isasymin}\ S\ {\isasymlongrightarrow}\ {\isacharbraceleft}\isactrlbold {\isasymnot}\ G{\isacharcomma}\ \isactrlbold {\isasymnot}\ H{\isacharbraceright}\ {\isasymunion}\ S\ {\isasymin}\ C{\isacharparenright}\isanewline
\ \ \ \ {\isasymand}\ {\isacharparenleft}{\isasymforall}G\ H{\isachardot}\ \isactrlbold {\isasymnot}{\isacharparenleft}G\ \isactrlbold {\isasymrightarrow}\ H{\isacharparenright}\ {\isasymin}\ S\ {\isasymlongrightarrow}\ {\isacharbraceleft}G{\isacharcomma}\isactrlbold {\isasymnot}\ H{\isacharbraceright}\ {\isasymunion}\ S\ {\isasymin}\ C{\isacharparenright}{\isacharparenright}{\isachardoublequoteclose}\isanewline
\ \ \ \ \isacommand{using}\isamarkupfalse%
\ A\ B\ \isacommand{by}\isamarkupfalse%
\ {\isacharparenleft}rule\ conjI{\isacharparenright}\isanewline
\ \ \isacommand{thus}\isamarkupfalse%
\ {\isachardoublequoteopen}{\isasymbottom}\ {\isasymnotin}\ S\isanewline
\ \ \ \ {\isasymand}\ {\isacharparenleft}{\isasymforall}k{\isachardot}\ Atom\ k\ {\isasymin}\ S\ {\isasymlongrightarrow}\ \isactrlbold {\isasymnot}\ {\isacharparenleft}Atom\ k{\isacharparenright}\ {\isasymin}\ S\ {\isasymlongrightarrow}\ False{\isacharparenright}\isanewline
\ \ \ \ {\isasymand}\ {\isacharparenleft}{\isasymforall}G\ H{\isachardot}\ G\ \isactrlbold {\isasymand}\ H\ {\isasymin}\ S\ {\isasymlongrightarrow}\ {\isacharbraceleft}G{\isacharcomma}H{\isacharbraceright}\ {\isasymunion}\ S\ {\isasymin}\ C{\isacharparenright}\isanewline
\ \ \ \ {\isasymand}\ {\isacharparenleft}{\isasymforall}G\ H{\isachardot}\ G\ \isactrlbold {\isasymor}\ H\ {\isasymin}\ S\ {\isasymlongrightarrow}\ {\isacharbraceleft}G{\isacharbraceright}\ {\isasymunion}\ S\ {\isasymin}\ C\ {\isasymor}\ {\isacharbraceleft}H{\isacharbraceright}\ {\isasymunion}\ S\ {\isasymin}\ C{\isacharparenright}\isanewline
\ \ \ \ {\isasymand}\ {\isacharparenleft}{\isasymforall}G\ H{\isachardot}\ G\ \isactrlbold {\isasymrightarrow}\ H\ {\isasymin}\ S\ {\isasymlongrightarrow}\ {\isacharbraceleft}\isactrlbold {\isasymnot}G{\isacharbraceright}\ {\isasymunion}\ S\ {\isasymin}\ C\ {\isasymor}\ {\isacharbraceleft}H{\isacharbraceright}\ {\isasymunion}\ S\ {\isasymin}\ C{\isacharparenright}\isanewline
\ \ \ \ {\isasymand}\ {\isacharparenleft}{\isasymforall}G{\isachardot}\ \isactrlbold {\isasymnot}\ {\isacharparenleft}\isactrlbold {\isasymnot}G{\isacharparenright}\ {\isasymin}\ S\ {\isasymlongrightarrow}\ {\isacharbraceleft}G{\isacharbraceright}\ {\isasymunion}\ S\ {\isasymin}\ C{\isacharparenright}\isanewline
\ \ \ \ {\isasymand}\ {\isacharparenleft}{\isasymforall}G\ H{\isachardot}\ \isactrlbold {\isasymnot}{\isacharparenleft}G\ \isactrlbold {\isasymand}\ H{\isacharparenright}\ {\isasymin}\ S\ {\isasymlongrightarrow}\ {\isacharbraceleft}\isactrlbold {\isasymnot}\ G{\isacharbraceright}\ {\isasymunion}\ S\ {\isasymin}\ C\ {\isasymor}\ {\isacharbraceleft}\isactrlbold {\isasymnot}\ H{\isacharbraceright}\ {\isasymunion}\ S\ {\isasymin}\ C{\isacharparenright}\isanewline
\ \ \ \ {\isasymand}\ {\isacharparenleft}{\isasymforall}G\ H{\isachardot}\ \isactrlbold {\isasymnot}{\isacharparenleft}G\ \isactrlbold {\isasymor}\ H{\isacharparenright}\ {\isasymin}\ S\ {\isasymlongrightarrow}\ {\isacharbraceleft}\isactrlbold {\isasymnot}\ G{\isacharcomma}\ \isactrlbold {\isasymnot}\ H{\isacharbraceright}\ {\isasymunion}\ S\ {\isasymin}\ C{\isacharparenright}\isanewline
\ \ \ \ {\isasymand}\ {\isacharparenleft}{\isasymforall}G\ H{\isachardot}\ \isactrlbold {\isasymnot}{\isacharparenleft}G\ \isactrlbold {\isasymrightarrow}\ H{\isacharparenright}\ {\isasymin}\ S\ {\isasymlongrightarrow}\ {\isacharbraceleft}G{\isacharcomma}\isactrlbold {\isasymnot}\ H{\isacharbraceright}\ {\isasymunion}\ S\ {\isasymin}\ C{\isacharparenright}{\isachardoublequoteclose}\isanewline
\ \ \ \ \isacommand{by}\isamarkupfalse%
\ {\isacharparenleft}iprover\ intro{\isacharcolon}\ conj{\isacharunderscore}assoc{\isacharparenright}\isanewline
\isacommand{qed}\isamarkupfalse%
%
\endisatagproof
{\isafoldproof}%
%
\isadelimproof
%
\endisadelimproof
%
\begin{isamarkuptext}%
Una vez probadas detalladamente en Isabelle cada una de las implicaciones de la
  equivalencia, podemos finalmente concluir con la demostración del lema completo.%
\end{isamarkuptext}\isamarkuptrue%
\isacommand{lemma}\isamarkupfalse%
\ {\isachardoublequoteopen}pcp\ C\ {\isacharequal}\ {\isacharparenleft}{\isasymforall}S\ {\isasymin}\ C{\isachardot}\ {\isasymbottom}\ {\isasymnotin}\ S\isanewline
{\isasymand}\ {\isacharparenleft}{\isasymforall}k{\isachardot}\ Atom\ k\ {\isasymin}\ S\ {\isasymlongrightarrow}\ \isactrlbold {\isasymnot}\ {\isacharparenleft}Atom\ k{\isacharparenright}\ {\isasymin}\ S\ {\isasymlongrightarrow}\ False{\isacharparenright}\isanewline
{\isasymand}\ {\isacharparenleft}{\isasymforall}F\ G\ H{\isachardot}\ Con\ F\ G\ H\ {\isasymlongrightarrow}\ F\ {\isasymin}\ S\ {\isasymlongrightarrow}\ {\isacharbraceleft}G{\isacharcomma}H{\isacharbraceright}\ {\isasymunion}\ S\ {\isasymin}\ C{\isacharparenright}\isanewline
{\isasymand}\ {\isacharparenleft}{\isasymforall}F\ G\ H{\isachardot}\ Dis\ F\ G\ H\ {\isasymlongrightarrow}\ F\ {\isasymin}\ S\ {\isasymlongrightarrow}\ {\isacharbraceleft}G{\isacharbraceright}\ {\isasymunion}\ S\ {\isasymin}\ C\ {\isasymor}\ {\isacharbraceleft}H{\isacharbraceright}\ {\isasymunion}\ S\ {\isasymin}\ C{\isacharparenright}{\isacharparenright}{\isachardoublequoteclose}\isanewline
%
\isadelimproof
%
\endisadelimproof
%
\isatagproof
\isacommand{proof}\isamarkupfalse%
\ {\isacharparenleft}rule\ iffI{\isacharparenright}\isanewline
\ \ \isacommand{assume}\isamarkupfalse%
\ {\isachardoublequoteopen}pcp\ C{\isachardoublequoteclose}\isanewline
\ \ \isacommand{thus}\isamarkupfalse%
\ {\isachardoublequoteopen}{\isasymforall}S\ {\isasymin}\ C{\isachardot}\ {\isasymbottom}\ {\isasymnotin}\ S\isanewline
{\isasymand}\ {\isacharparenleft}{\isasymforall}k{\isachardot}\ Atom\ k\ {\isasymin}\ S\ {\isasymlongrightarrow}\ \isactrlbold {\isasymnot}\ {\isacharparenleft}Atom\ k{\isacharparenright}\ {\isasymin}\ S\ {\isasymlongrightarrow}\ False{\isacharparenright}\isanewline
{\isasymand}\ {\isacharparenleft}{\isasymforall}F\ G\ H{\isachardot}\ Con\ F\ G\ H\ {\isasymlongrightarrow}\ F\ {\isasymin}\ S\ {\isasymlongrightarrow}\ {\isacharbraceleft}G{\isacharcomma}H{\isacharbraceright}\ {\isasymunion}\ S\ {\isasymin}\ C{\isacharparenright}\isanewline
{\isasymand}\ {\isacharparenleft}{\isasymforall}F\ G\ H{\isachardot}\ Dis\ F\ G\ H\ {\isasymlongrightarrow}\ F\ {\isasymin}\ S\ {\isasymlongrightarrow}\ {\isacharbraceleft}G{\isacharbraceright}\ {\isasymunion}\ S\ {\isasymin}\ C\ {\isasymor}\ {\isacharbraceleft}H{\isacharbraceright}\ {\isasymunion}\ S\ {\isasymin}\ C{\isacharparenright}{\isachardoublequoteclose}\isanewline
\ \ \ \ \isacommand{by}\isamarkupfalse%
\ {\isacharparenleft}rule\ pcp{\isacharunderscore}alt{\isadigit{1}}{\isacharparenright}\isanewline
\isacommand{next}\isamarkupfalse%
\isanewline
\ \ \isacommand{assume}\isamarkupfalse%
\ {\isachardoublequoteopen}{\isasymforall}S\ {\isasymin}\ C{\isachardot}\ {\isasymbottom}\ {\isasymnotin}\ S\isanewline
{\isasymand}\ {\isacharparenleft}{\isasymforall}k{\isachardot}\ Atom\ k\ {\isasymin}\ S\ {\isasymlongrightarrow}\ \isactrlbold {\isasymnot}\ {\isacharparenleft}Atom\ k{\isacharparenright}\ {\isasymin}\ S\ {\isasymlongrightarrow}\ False{\isacharparenright}\isanewline
{\isasymand}\ {\isacharparenleft}{\isasymforall}F\ G\ H{\isachardot}\ Con\ F\ G\ H\ {\isasymlongrightarrow}\ F\ {\isasymin}\ S\ {\isasymlongrightarrow}\ {\isacharbraceleft}G{\isacharcomma}H{\isacharbraceright}\ {\isasymunion}\ S\ {\isasymin}\ C{\isacharparenright}\isanewline
{\isasymand}\ {\isacharparenleft}{\isasymforall}F\ G\ H{\isachardot}\ Dis\ F\ G\ H\ {\isasymlongrightarrow}\ F\ {\isasymin}\ S\ {\isasymlongrightarrow}\ {\isacharbraceleft}G{\isacharbraceright}\ {\isasymunion}\ S\ {\isasymin}\ C\ {\isasymor}\ {\isacharbraceleft}H{\isacharbraceright}\ {\isasymunion}\ S\ {\isasymin}\ C{\isacharparenright}{\isachardoublequoteclose}\isanewline
\ \ \isacommand{thus}\isamarkupfalse%
\ {\isachardoublequoteopen}pcp\ C{\isachardoublequoteclose}\isanewline
\ \ \ \ \isacommand{by}\isamarkupfalse%
\ {\isacharparenleft}rule\ pcp{\isacharunderscore}alt{\isadigit{2}}{\isacharparenright}\isanewline
\isacommand{qed}\isamarkupfalse%
%
\endisatagproof
{\isafoldproof}%
%
\isadelimproof
%
\endisadelimproof
%
\begin{isamarkuptext}%
La demostración automática del resultado en Isabelle/HOL se muestra finalmente a 
  continuación.%
\end{isamarkuptext}\isamarkuptrue%
\isacommand{lemma}\isamarkupfalse%
\ pcp{\isacharunderscore}alt{\isacharcolon}\ {\isachardoublequoteopen}pcp\ C\ {\isacharequal}\ {\isacharparenleft}{\isasymforall}S\ {\isasymin}\ C{\isachardot}\isanewline
\ \ {\isasymbottom}\ {\isasymnotin}\ S\isanewline
{\isasymand}\ {\isacharparenleft}{\isasymforall}k{\isachardot}\ Atom\ k\ {\isasymin}\ S\ {\isasymlongrightarrow}\ \isactrlbold {\isasymnot}\ {\isacharparenleft}Atom\ k{\isacharparenright}\ {\isasymin}\ S\ {\isasymlongrightarrow}\ False{\isacharparenright}\isanewline
{\isasymand}\ {\isacharparenleft}{\isasymforall}F\ G\ H{\isachardot}\ Con\ F\ G\ H\ {\isasymlongrightarrow}\ F\ {\isasymin}\ S\ {\isasymlongrightarrow}\ {\isacharbraceleft}G{\isacharcomma}H{\isacharbraceright}\ {\isasymunion}\ S\ {\isasymin}\ C{\isacharparenright}\isanewline
{\isasymand}\ {\isacharparenleft}{\isasymforall}F\ G\ H{\isachardot}\ Dis\ F\ G\ H\ {\isasymlongrightarrow}\ F\ {\isasymin}\ S\ {\isasymlongrightarrow}\ {\isacharbraceleft}G{\isacharbraceright}\ {\isasymunion}\ S\ {\isasymin}\ C\ {\isasymor}\ {\isacharbraceleft}H{\isacharbraceright}\ {\isasymunion}\ S\ {\isasymin}\ C{\isacharparenright}{\isacharparenright}{\isachardoublequoteclose}\isanewline
%
\isadelimproof
\ \ %
\endisadelimproof
%
\isatagproof
\isacommand{apply}\isamarkupfalse%
{\isacharparenleft}simp\ add{\isacharcolon}\ pcp{\isacharunderscore}def\ con{\isacharunderscore}dis{\isacharunderscore}simps{\isacharparenright}\isanewline
\ \ \isacommand{apply}\isamarkupfalse%
{\isacharparenleft}rule\ iffI{\isacharsemicolon}\ unfold\ Ball{\isacharunderscore}def{\isacharsemicolon}\ elim\ all{\isacharunderscore}forward{\isacharparenright}\isanewline
\ \ \isacommand{by}\isamarkupfalse%
\ {\isacharparenleft}auto\ simp\ add{\isacharcolon}\ insert{\isacharunderscore}absorb\ split{\isacharcolon}\ formula{\isachardot}splits{\isacharparenright}\isanewline
%
\endisatagproof
{\isafoldproof}%
%
\isadelimproof
%
\endisadelimproof
%
\isadelimtheory
%
\endisadelimtheory
%
\isatagtheory
%
\endisatagtheory
{\isafoldtheory}%
%
\isadelimtheory
%
\endisadelimtheory
%
\end{isabellebody}%
\endinput
%:%file=~/TFM/TFM/Pcp.thy%:%
%:%19=11%:%
%:%20=12%:%
%:%21=13%:%
%:%22=14%:%
%:%23=15%:%
%:%24=16%:%
%:%25=17%:%
%:%26=18%:%
%:%26=19%:%
%:%27=20%:%
%:%28=21%:%
%:%29=22%:%
%:%30=23%:%
%:%31=24%:%
%:%32=25%:%
%:%33=26%:%
%:%37=29%:%
%:%38=30%:%
%:%39=31%:%
%:%43=34%:%
%:%44=35%:%
%:%45=36%:%
%:%46=37%:%
%:%47=38%:%
%:%48=39%:%
%:%49=40%:%
%:%50=41%:%
%:%51=42%:%
%:%52=43%:%
%:%53=44%:%
%:%54=45%:%
%:%55=46%:%
%:%56=47%:%
%:%57=48%:%
%:%58=49%:%
%:%59=50%:%
%:%60=51%:%
%:%61=52%:%
%:%62=53%:%
%:%63=54%:%
%:%64=55%:%
%:%66=57%:%
%:%67=57%:%
%:%78=68%:%
%:%79=69%:%
%:%80=70%:%
%:%82=72%:%
%:%83=72%:%
%:%86=73%:%
%:%90=73%:%
%:%91=73%:%
%:%92=73%:%
%:%101=75%:%
%:%102=76%:%
%:%103=77%:%
%:%105=79%:%
%:%106=79%:%
%:%109=80%:%
%:%113=80%:%
%:%114=80%:%
%:%115=80%:%
%:%120=80%:%
%:%123=81%:%
%:%124=82%:%
%:%125=82%:%
%:%127=84%:%
%:%130=85%:%
%:%134=85%:%
%:%135=85%:%
%:%136=85%:%
%:%145=87%:%
%:%146=88%:%
%:%147=89%:%
%:%148=90%:%
%:%150=92%:%
%:%151=92%:%
%:%152=93%:%
%:%155=94%:%
%:%159=94%:%
%:%160=94%:%
%:%161=94%:%
%:%170=96%:%
%:%171=97%:%
%:%172=98%:%
%:%173=99%:%
%:%174=100%:%
%:%175=101%:%
%:%176=102%:%
%:%177=103%:%
%:%178=104%:%
%:%179=105%:%
%:%180=106%:%
%:%181=107%:%
%:%182=108%:%
%:%183=109%:%
%:%184=110%:%
%:%185=111%:%
%:%186=112%:%
%:%187=113%:%
%:%188=114%:%
%:%189=115%:%
%:%190=116%:%
%:%191=117%:%
%:%192=118%:%
%:%194=120%:%
%:%195=120%:%
%:%198=123%:%
%:%201=124%:%
%:%205=124%:%
%:%215=126%:%
%:%216=127%:%
%:%217=128%:%
%:%218=129%:%
%:%219=130%:%
%:%220=131%:%
%:%221=132%:%
%:%222=133%:%
%:%223=134%:%
%:%224=135%:%
%:%225=136%:%
%:%226=137%:%
%:%227=138%:%
%:%228=139%:%
%:%229=140%:%
%:%230=141%:%
%:%231=142%:%
%:%232=143%:%
%:%233=144%:%
%:%234=145%:%
%:%235=146%:%
%:%236=147%:%
%:%237=148%:%
%:%238=149%:%
%:%239=150%:%
%:%240=151%:%
%:%241=152%:%
%:%242=153%:%
%:%243=154%:%
%:%244=155%:%
%:%245=156%:%
%:%246=157%:%
%:%247=158%:%
%:%248=159%:%
%:%249=160%:%
%:%250=161%:%
%:%251=162%:%
%:%252=163%:%
%:%253=164%:%
%:%254=165%:%
%:%255=166%:%
%:%256=167%:%
%:%257=168%:%
%:%258=169%:%
%:%259=170%:%
%:%260=171%:%
%:%261=172%:%
%:%262=173%:%
%:%263=174%:%
%:%264=175%:%
%:%265=176%:%
%:%266=177%:%
%:%267=178%:%
%:%268=179%:%
%:%269=180%:%
%:%270=181%:%
%:%271=182%:%
%:%272=183%:%
%:%273=184%:%
%:%274=185%:%
%:%275=186%:%
%:%276=187%:%
%:%277=188%:%
%:%278=189%:%
%:%279=190%:%
%:%280=191%:%
%:%281=192%:%
%:%282=193%:%
%:%283=194%:%
%:%284=195%:%
%:%285=196%:%
%:%286=197%:%
%:%287=198%:%
%:%288=199%:%
%:%289=200%:%
%:%290=201%:%
%:%291=202%:%
%:%292=203%:%
%:%293=204%:%
%:%294=205%:%
%:%295=206%:%
%:%296=207%:%
%:%297=208%:%
%:%298=209%:%
%:%299=210%:%
%:%300=211%:%
%:%301=212%:%
%:%302=213%:%
%:%303=214%:%
%:%304=215%:%
%:%305=216%:%
%:%306=217%:%
%:%307=218%:%
%:%308=219%:%
%:%309=220%:%
%:%310=221%:%
%:%311=222%:%
%:%312=223%:%
%:%313=224%:%
%:%314=225%:%
%:%315=226%:%
%:%316=227%:%
%:%317=228%:%
%:%318=229%:%
%:%319=230%:%
%:%320=231%:%
%:%321=232%:%
%:%322=233%:%
%:%323=234%:%
%:%324=235%:%
%:%325=236%:%
%:%326=237%:%
%:%327=238%:%
%:%328=239%:%
%:%329=240%:%
%:%330=241%:%
%:%331=242%:%
%:%332=243%:%
%:%333=244%:%
%:%334=245%:%
%:%335=246%:%
%:%336=247%:%
%:%337=248%:%
%:%338=249%:%
%:%339=250%:%
%:%340=251%:%
%:%341=252%:%
%:%342=253%:%
%:%343=254%:%
%:%344=255%:%
%:%345=256%:%
%:%346=257%:%
%:%347=258%:%
%:%348=259%:%
%:%349=260%:%
%:%350=261%:%
%:%351=262%:%
%:%352=263%:%
%:%353=264%:%
%:%354=265%:%
%:%355=266%:%
%:%356=267%:%
%:%357=268%:%
%:%358=269%:%
%:%359=270%:%
%:%360=271%:%
%:%361=272%:%
%:%362=273%:%
%:%363=274%:%
%:%364=275%:%
%:%365=276%:%
%:%366=277%:%
%:%367=278%:%
%:%368=279%:%
%:%369=280%:%
%:%370=281%:%
%:%371=282%:%
%:%372=283%:%
%:%373=284%:%
%:%374=285%:%
%:%375=286%:%
%:%376=287%:%
%:%377=288%:%
%:%378=289%:%
%:%379=290%:%
%:%380=291%:%
%:%381=292%:%
%:%382=293%:%
%:%383=294%:%
%:%384=295%:%
%:%385=296%:%
%:%386=297%:%
%:%387=298%:%
%:%388=299%:%
%:%389=300%:%
%:%390=301%:%
%:%391=302%:%
%:%392=303%:%
%:%393=304%:%
%:%394=305%:%
%:%396=307%:%
%:%397=307%:%
%:%398=308%:%
%:%401=311%:%
%:%402=312%:%
%:%409=313%:%
%:%410=313%:%
%:%411=314%:%
%:%412=314%:%
%:%413=315%:%
%:%414=315%:%
%:%415=315%:%
%:%416=316%:%
%:%417=316%:%
%:%418=317%:%
%:%419=317%:%
%:%420=317%:%
%:%421=318%:%
%:%422=318%:%
%:%423=319%:%
%:%424=319%:%
%:%425=319%:%
%:%426=320%:%
%:%427=320%:%
%:%428=321%:%
%:%429=321%:%
%:%430=321%:%
%:%431=322%:%
%:%432=322%:%
%:%433=323%:%
%:%434=323%:%
%:%435=324%:%
%:%436=324%:%
%:%437=325%:%
%:%438=325%:%
%:%439=326%:%
%:%440=326%:%
%:%441=327%:%
%:%442=327%:%
%:%443=328%:%
%:%444=328%:%
%:%445=328%:%
%:%448=331%:%
%:%449=332%:%
%:%450=332%:%
%:%451=333%:%
%:%452=333%:%
%:%453=334%:%
%:%454=334%:%
%:%455=335%:%
%:%456=335%:%
%:%457=336%:%
%:%458=336%:%
%:%459=337%:%
%:%460=337%:%
%:%461=337%:%
%:%462=338%:%
%:%463=338%:%
%:%464=339%:%
%:%465=339%:%
%:%467=341%:%
%:%468=342%:%
%:%469=342%:%
%:%470=343%:%
%:%471=343%:%
%:%472=344%:%
%:%473=344%:%
%:%474=345%:%
%:%475=345%:%
%:%476=346%:%
%:%477=346%:%
%:%478=346%:%
%:%479=347%:%
%:%480=347%:%
%:%481=348%:%
%:%482=348%:%
%:%483=348%:%
%:%484=349%:%
%:%485=349%:%
%:%486=350%:%
%:%487=350%:%
%:%488=350%:%
%:%489=351%:%
%:%490=351%:%
%:%491=352%:%
%:%492=352%:%
%:%493=352%:%
%:%494=353%:%
%:%495=353%:%
%:%496=354%:%
%:%497=354%:%
%:%498=354%:%
%:%499=355%:%
%:%500=355%:%
%:%501=356%:%
%:%502=356%:%
%:%503=357%:%
%:%504=358%:%
%:%505=358%:%
%:%506=359%:%
%:%507=359%:%
%:%508=360%:%
%:%509=360%:%
%:%510=361%:%
%:%511=361%:%
%:%512=362%:%
%:%513=362%:%
%:%514=362%:%
%:%515=363%:%
%:%516=363%:%
%:%517=364%:%
%:%518=364%:%
%:%519=364%:%
%:%520=365%:%
%:%521=365%:%
%:%522=366%:%
%:%523=366%:%
%:%524=366%:%
%:%525=367%:%
%:%526=367%:%
%:%527=368%:%
%:%528=368%:%
%:%529=368%:%
%:%530=369%:%
%:%531=369%:%
%:%532=370%:%
%:%533=370%:%
%:%534=371%:%
%:%535=371%:%
%:%536=371%:%
%:%537=372%:%
%:%538=372%:%
%:%539=373%:%
%:%540=373%:%
%:%541=374%:%
%:%542=374%:%
%:%543=374%:%
%:%544=375%:%
%:%545=375%:%
%:%546=376%:%
%:%547=376%:%
%:%548=376%:%
%:%549=377%:%
%:%550=377%:%
%:%551=377%:%
%:%552=378%:%
%:%553=378%:%
%:%554=379%:%
%:%555=379%:%
%:%556=380%:%
%:%557=380%:%
%:%558=381%:%
%:%559=381%:%
%:%560=382%:%
%:%561=382%:%
%:%562=383%:%
%:%563=383%:%
%:%564=384%:%
%:%565=384%:%
%:%566=385%:%
%:%567=385%:%
%:%568=386%:%
%:%578=388%:%
%:%579=389%:%
%:%580=390%:%
%:%582=392%:%
%:%583=392%:%
%:%584=393%:%
%:%587=396%:%
%:%588=397%:%
%:%595=398%:%
%:%596=398%:%
%:%597=399%:%
%:%598=399%:%
%:%599=400%:%
%:%600=400%:%
%:%601=400%:%
%:%602=401%:%
%:%603=401%:%
%:%604=402%:%
%:%605=402%:%
%:%606=402%:%
%:%607=403%:%
%:%608=403%:%
%:%609=404%:%
%:%610=404%:%
%:%611=404%:%
%:%612=405%:%
%:%613=405%:%
%:%614=406%:%
%:%615=406%:%
%:%616=406%:%
%:%617=407%:%
%:%618=407%:%
%:%619=408%:%
%:%620=408%:%
%:%621=409%:%
%:%622=409%:%
%:%623=410%:%
%:%624=410%:%
%:%625=411%:%
%:%626=411%:%
%:%627=412%:%
%:%628=412%:%
%:%629=413%:%
%:%630=413%:%
%:%631=413%:%
%:%634=416%:%
%:%635=417%:%
%:%636=417%:%
%:%637=418%:%
%:%638=418%:%
%:%639=419%:%
%:%640=419%:%
%:%641=420%:%
%:%642=420%:%
%:%643=421%:%
%:%644=421%:%
%:%645=422%:%
%:%646=422%:%
%:%647=422%:%
%:%648=423%:%
%:%649=423%:%
%:%650=424%:%
%:%651=424%:%
%:%653=426%:%
%:%654=427%:%
%:%655=427%:%
%:%656=428%:%
%:%657=428%:%
%:%658=429%:%
%:%659=429%:%
%:%660=430%:%
%:%661=430%:%
%:%662=431%:%
%:%663=431%:%
%:%664=431%:%
%:%665=432%:%
%:%666=432%:%
%:%667=433%:%
%:%668=433%:%
%:%669=433%:%
%:%670=434%:%
%:%671=434%:%
%:%672=435%:%
%:%673=435%:%
%:%674=435%:%
%:%675=436%:%
%:%676=436%:%
%:%677=437%:%
%:%678=437%:%
%:%679=437%:%
%:%680=438%:%
%:%681=438%:%
%:%682=439%:%
%:%683=439%:%
%:%684=439%:%
%:%685=440%:%
%:%686=440%:%
%:%687=441%:%
%:%688=441%:%
%:%689=442%:%
%:%690=443%:%
%:%691=443%:%
%:%692=444%:%
%:%693=444%:%
%:%694=445%:%
%:%695=445%:%
%:%696=446%:%
%:%697=446%:%
%:%698=447%:%
%:%699=447%:%
%:%700=447%:%
%:%701=448%:%
%:%702=448%:%
%:%703=449%:%
%:%704=449%:%
%:%705=449%:%
%:%706=450%:%
%:%707=450%:%
%:%708=451%:%
%:%709=451%:%
%:%710=451%:%
%:%711=452%:%
%:%712=452%:%
%:%713=453%:%
%:%714=453%:%
%:%715=453%:%
%:%716=454%:%
%:%717=454%:%
%:%718=455%:%
%:%719=455%:%
%:%720=455%:%
%:%721=456%:%
%:%722=456%:%
%:%723=457%:%
%:%724=457%:%
%:%725=458%:%
%:%726=458%:%
%:%727=458%:%
%:%728=459%:%
%:%729=459%:%
%:%730=460%:%
%:%731=460%:%
%:%732=461%:%
%:%733=461%:%
%:%734=461%:%
%:%735=462%:%
%:%736=462%:%
%:%737=463%:%
%:%738=463%:%
%:%739=463%:%
%:%740=464%:%
%:%741=464%:%
%:%742=464%:%
%:%743=465%:%
%:%744=465%:%
%:%745=466%:%
%:%746=466%:%
%:%747=467%:%
%:%748=467%:%
%:%749=468%:%
%:%750=468%:%
%:%751=469%:%
%:%752=469%:%
%:%753=470%:%
%:%754=470%:%
%:%755=471%:%
%:%756=471%:%
%:%757=472%:%
%:%758=472%:%
%:%759=473%:%
%:%769=475%:%
%:%770=476%:%
%:%772=478%:%
%:%773=478%:%
%:%774=479%:%
%:%775=480%:%
%:%778=483%:%
%:%785=484%:%
%:%786=484%:%
%:%787=485%:%
%:%788=485%:%
%:%789=486%:%
%:%790=486%:%
%:%791=487%:%
%:%792=487%:%
%:%801=496%:%
%:%802=497%:%
%:%803=497%:%
%:%804=497%:%
%:%805=498%:%
%:%806=498%:%
%:%807=498%:%
%:%815=506%:%
%:%816=507%:%
%:%817=507%:%
%:%818=507%:%
%:%819=508%:%
%:%820=508%:%
%:%821=508%:%
%:%822=509%:%
%:%823=509%:%
%:%824=510%:%
%:%825=510%:%
%:%826=511%:%
%:%827=511%:%
%:%828=511%:%
%:%829=512%:%
%:%830=512%:%
%:%831=513%:%
%:%832=513%:%
%:%833=513%:%
%:%834=514%:%
%:%835=514%:%
%:%836=515%:%
%:%837=515%:%
%:%838=515%:%
%:%839=516%:%
%:%840=516%:%
%:%841=517%:%
%:%842=517%:%
%:%843=517%:%
%:%844=518%:%
%:%845=518%:%
%:%846=519%:%
%:%847=519%:%
%:%848=519%:%
%:%849=520%:%
%:%850=520%:%
%:%851=521%:%
%:%852=521%:%
%:%853=521%:%
%:%854=522%:%
%:%855=522%:%
%:%856=523%:%
%:%857=523%:%
%:%858=523%:%
%:%859=524%:%
%:%860=524%:%
%:%861=525%:%
%:%862=525%:%
%:%863=525%:%
%:%864=526%:%
%:%865=526%:%
%:%868=529%:%
%:%869=530%:%
%:%870=530%:%
%:%871=530%:%
%:%872=531%:%
%:%873=531%:%
%:%874=531%:%
%:%875=532%:%
%:%876=532%:%
%:%877=533%:%
%:%878=533%:%
%:%881=536%:%
%:%882=537%:%
%:%883=537%:%
%:%884=537%:%
%:%885=538%:%
%:%886=538%:%
%:%887=538%:%
%:%888=539%:%
%:%889=539%:%
%:%890=540%:%
%:%891=540%:%
%:%894=543%:%
%:%895=544%:%
%:%896=544%:%
%:%897=544%:%
%:%898=545%:%
%:%908=547%:%
%:%909=548%:%
%:%910=549%:%
%:%911=550%:%
%:%912=551%:%
%:%914=553%:%
%:%915=553%:%
%:%916=554%:%
%:%917=555%:%
%:%924=556%:%
%:%925=556%:%
%:%926=557%:%
%:%927=557%:%
%:%928=558%:%
%:%929=558%:%
%:%930=559%:%
%:%931=559%:%
%:%932=560%:%
%:%933=560%:%
%:%934=561%:%
%:%935=561%:%
%:%936=561%:%
%:%937=562%:%
%:%938=562%:%
%:%939=563%:%
%:%940=563%:%
%:%941=564%:%
%:%942=564%:%
%:%943=565%:%
%:%944=565%:%
%:%945=565%:%
%:%946=566%:%
%:%947=566%:%
%:%948=566%:%
%:%949=567%:%
%:%950=567%:%
%:%951=567%:%
%:%952=568%:%
%:%953=568%:%
%:%954=569%:%
%:%955=569%:%
%:%956=569%:%
%:%957=570%:%
%:%958=570%:%
%:%959=571%:%
%:%965=571%:%
%:%968=572%:%
%:%969=573%:%
%:%970=573%:%
%:%971=574%:%
%:%972=575%:%
%:%979=576%:%
%:%980=576%:%
%:%981=577%:%
%:%982=577%:%
%:%983=578%:%
%:%984=578%:%
%:%985=579%:%
%:%986=579%:%
%:%987=580%:%
%:%988=580%:%
%:%989=581%:%
%:%990=581%:%
%:%991=581%:%
%:%992=582%:%
%:%993=582%:%
%:%994=583%:%
%:%995=583%:%
%:%996=584%:%
%:%997=584%:%
%:%998=585%:%
%:%999=585%:%
%:%1000=585%:%
%:%1001=586%:%
%:%1002=586%:%
%:%1003=586%:%
%:%1004=587%:%
%:%1005=587%:%
%:%1006=588%:%
%:%1007=588%:%
%:%1008=588%:%
%:%1009=589%:%
%:%1010=589%:%
%:%1011=590%:%
%:%1012=590%:%
%:%1013=590%:%
%:%1014=591%:%
%:%1015=591%:%
%:%1016=591%:%
%:%1017=592%:%
%:%1018=592%:%
%:%1019=592%:%
%:%1020=593%:%
%:%1021=593%:%
%:%1022=593%:%
%:%1023=594%:%
%:%1024=594%:%
%:%1025=595%:%
%:%1026=595%:%
%:%1027=596%:%
%:%1028=596%:%
%:%1029=597%:%
%:%1035=597%:%
%:%1038=598%:%
%:%1039=599%:%
%:%1040=599%:%
%:%1041=600%:%
%:%1042=601%:%
%:%1049=602%:%
%:%1050=602%:%
%:%1051=603%:%
%:%1052=603%:%
%:%1053=604%:%
%:%1054=604%:%
%:%1055=605%:%
%:%1056=605%:%
%:%1057=606%:%
%:%1058=606%:%
%:%1059=607%:%
%:%1060=607%:%
%:%1061=607%:%
%:%1062=608%:%
%:%1063=608%:%
%:%1064=609%:%
%:%1065=609%:%
%:%1066=610%:%
%:%1067=610%:%
%:%1068=611%:%
%:%1069=611%:%
%:%1070=611%:%
%:%1071=612%:%
%:%1072=612%:%
%:%1073=612%:%
%:%1074=613%:%
%:%1075=613%:%
%:%1076=613%:%
%:%1077=614%:%
%:%1078=614%:%
%:%1079=615%:%
%:%1080=615%:%
%:%1081=615%:%
%:%1082=616%:%
%:%1083=616%:%
%:%1084=617%:%
%:%1090=617%:%
%:%1093=618%:%
%:%1094=619%:%
%:%1095=619%:%
%:%1096=620%:%
%:%1097=621%:%
%:%1104=622%:%
%:%1105=622%:%
%:%1106=623%:%
%:%1107=623%:%
%:%1108=624%:%
%:%1109=624%:%
%:%1110=625%:%
%:%1111=625%:%
%:%1112=626%:%
%:%1113=626%:%
%:%1114=627%:%
%:%1115=627%:%
%:%1116=627%:%
%:%1117=628%:%
%:%1118=628%:%
%:%1119=629%:%
%:%1120=629%:%
%:%1121=630%:%
%:%1122=630%:%
%:%1123=631%:%
%:%1124=631%:%
%:%1125=631%:%
%:%1126=632%:%
%:%1127=632%:%
%:%1128=632%:%
%:%1129=633%:%
%:%1130=633%:%
%:%1131=633%:%
%:%1132=634%:%
%:%1133=634%:%
%:%1134=635%:%
%:%1135=635%:%
%:%1136=635%:%
%:%1137=636%:%
%:%1138=636%:%
%:%1139=637%:%
%:%1149=639%:%
%:%1150=640%:%
%:%1151=641%:%
%:%1153=643%:%
%:%1154=643%:%
%:%1155=644%:%
%:%1156=645%:%
%:%1163=646%:%
%:%1164=646%:%
%:%1165=647%:%
%:%1166=647%:%
%:%1167=648%:%
%:%1168=648%:%
%:%1169=649%:%
%:%1170=649%:%
%:%1171=650%:%
%:%1172=650%:%
%:%1173=651%:%
%:%1174=651%:%
%:%1175=651%:%
%:%1176=652%:%
%:%1177=652%:%
%:%1178=653%:%
%:%1179=653%:%
%:%1180=654%:%
%:%1181=654%:%
%:%1182=655%:%
%:%1183=655%:%
%:%1184=655%:%
%:%1185=656%:%
%:%1186=656%:%
%:%1187=656%:%
%:%1188=657%:%
%:%1189=657%:%
%:%1190=657%:%
%:%1191=658%:%
%:%1192=658%:%
%:%1193=659%:%
%:%1194=659%:%
%:%1195=659%:%
%:%1196=660%:%
%:%1197=660%:%
%:%1198=661%:%
%:%1204=661%:%
%:%1207=662%:%
%:%1208=663%:%
%:%1209=663%:%
%:%1210=664%:%
%:%1211=665%:%
%:%1218=666%:%
%:%1219=666%:%
%:%1220=667%:%
%:%1221=667%:%
%:%1222=668%:%
%:%1223=668%:%
%:%1224=669%:%
%:%1225=669%:%
%:%1226=670%:%
%:%1227=670%:%
%:%1228=671%:%
%:%1229=671%:%
%:%1230=671%:%
%:%1231=672%:%
%:%1232=672%:%
%:%1233=673%:%
%:%1234=673%:%
%:%1235=674%:%
%:%1236=674%:%
%:%1237=675%:%
%:%1238=675%:%
%:%1239=675%:%
%:%1240=676%:%
%:%1241=676%:%
%:%1242=676%:%
%:%1243=677%:%
%:%1244=677%:%
%:%1245=677%:%
%:%1246=678%:%
%:%1247=678%:%
%:%1248=679%:%
%:%1249=679%:%
%:%1250=679%:%
%:%1251=680%:%
%:%1252=680%:%
%:%1253=681%:%
%:%1259=681%:%
%:%1262=682%:%
%:%1263=683%:%
%:%1264=683%:%
%:%1265=684%:%
%:%1266=685%:%
%:%1273=686%:%
%:%1274=686%:%
%:%1275=687%:%
%:%1276=687%:%
%:%1277=688%:%
%:%1278=688%:%
%:%1279=689%:%
%:%1280=689%:%
%:%1281=690%:%
%:%1282=690%:%
%:%1283=691%:%
%:%1284=691%:%
%:%1285=691%:%
%:%1286=692%:%
%:%1287=692%:%
%:%1288=693%:%
%:%1289=693%:%
%:%1290=694%:%
%:%1291=694%:%
%:%1292=695%:%
%:%1293=695%:%
%:%1294=695%:%
%:%1295=696%:%
%:%1296=696%:%
%:%1297=696%:%
%:%1298=697%:%
%:%1299=697%:%
%:%1300=697%:%
%:%1301=698%:%
%:%1302=698%:%
%:%1303=699%:%
%:%1304=699%:%
%:%1305=699%:%
%:%1306=700%:%
%:%1307=700%:%
%:%1308=701%:%
%:%1318=703%:%
%:%1320=705%:%
%:%1321=705%:%
%:%1322=706%:%
%:%1325=709%:%
%:%1326=710%:%
%:%1329=711%:%
%:%1333=711%:%
%:%1334=711%:%
%:%1335=712%:%
%:%1336=712%:%
%:%1337=713%:%
%:%1338=713%:%
%:%1339=714%:%
%:%1340=714%:%
%:%1341=715%:%
%:%1342=715%:%
%:%1345=718%:%
%:%1346=719%:%
%:%1347=719%:%
%:%1348=719%:%
%:%1349=720%:%
%:%1350=720%:%
%:%1351=720%:%
%:%1352=721%:%
%:%1353=721%:%
%:%1354=722%:%
%:%1355=722%:%
%:%1356=723%:%
%:%1357=723%:%
%:%1358=723%:%
%:%1359=724%:%
%:%1360=724%:%
%:%1361=725%:%
%:%1362=726%:%
%:%1363=726%:%
%:%1364=726%:%
%:%1365=727%:%
%:%1366=727%:%
%:%1367=728%:%
%:%1368=728%:%
%:%1369=728%:%
%:%1370=729%:%
%:%1371=729%:%
%:%1372=730%:%
%:%1373=730%:%
%:%1374=730%:%
%:%1375=731%:%
%:%1376=731%:%
%:%1377=732%:%
%:%1378=732%:%
%:%1379=732%:%
%:%1380=733%:%
%:%1381=733%:%
%:%1382=734%:%
%:%1383=734%:%
%:%1384=734%:%
%:%1385=735%:%
%:%1386=735%:%
%:%1387=736%:%
%:%1388=736%:%
%:%1389=736%:%
%:%1390=737%:%
%:%1391=737%:%
%:%1392=738%:%
%:%1393=738%:%
%:%1394=738%:%
%:%1395=739%:%
%:%1396=739%:%
%:%1397=740%:%
%:%1398=740%:%
%:%1399=740%:%
%:%1400=741%:%
%:%1401=741%:%
%:%1405=745%:%
%:%1406=746%:%
%:%1407=746%:%
%:%1408=746%:%
%:%1409=747%:%
%:%1410=747%:%
%:%1413=750%:%
%:%1414=751%:%
%:%1415=751%:%
%:%1416=751%:%
%:%1417=752%:%
%:%1418=752%:%
%:%1426=760%:%
%:%1427=761%:%
%:%1428=761%:%
%:%1429=761%:%
%:%1430=762%:%
%:%1431=762%:%
%:%1439=770%:%
%:%1440=771%:%
%:%1441=771%:%
%:%1442=772%:%
%:%1452=774%:%
%:%1453=775%:%
%:%1455=777%:%
%:%1456=777%:%
%:%1459=780%:%
%:%1466=781%:%
%:%1467=781%:%
%:%1468=782%:%
%:%1469=782%:%
%:%1470=783%:%
%:%1471=783%:%
%:%1474=786%:%
%:%1475=787%:%
%:%1476=787%:%
%:%1477=788%:%
%:%1478=788%:%
%:%1479=789%:%
%:%1480=789%:%
%:%1483=792%:%
%:%1484=793%:%
%:%1485=793%:%
%:%1486=794%:%
%:%1487=794%:%
%:%1488=795%:%
%:%1498=797%:%
%:%1499=798%:%
%:%1501=800%:%
%:%1502=800%:%
%:%1506=804%:%
%:%1509=805%:%
%:%1513=805%:%
%:%1514=805%:%
%:%1515=806%:%
%:%1516=806%:%
%:%1517=807%:%
%:%1518=807%:%

\chapter{Colecciones de conjuntos cerradas bajo subconjuntos y de carácter finito}
%
\begin{isabellebody}%
\setisabellecontext{ColecScfc}%
%
\isadelimtheory
%
\endisadelimtheory
%
\isatagtheory
%
\endisatagtheory
{\isafoldtheory}%
%
\isadelimtheory
%
\endisadelimtheory
%
\begin{isamarkuptext}%
En este apartado definiremos colecciones de conjuntos \isa{cerradas\ bajo\ subconjuntos} y de 
  \isa{carácter\ finito}, junto con tres resultados sobre las mismas. El primero de ellos permite
  extender una colección que verifique la propiedad de consistencia proposicional a otra que 
  también la verifique y sea cerrada bajo subconjuntos. Posteriormente probaremos que toda colección
  de carácter finito es cerrada bajo subconjuntos. Finalmente, mostraremos un resultado que 
  permite extender una colección cerrada bajo subconjuntos que verifique la propiedad de
  consistencia proposicional a otra que también verifique dicha propiedad y sea de carácter 
  finito.

  \begin{definicion}
    Una colección de conjuntos es \isa{cerrada\ bajo\ subconjuntos} si todo subconjunto de cada conjunto 
    de la colección pertenece a la colección.
  \end{definicion}

  En Isabelle se formaliza de la siguiente manera.%
\end{isamarkuptext}\isamarkuptrue%
\isacommand{definition}\isamarkupfalse%
\ {\isachardoublequoteopen}subset{\isacharunderscore}closed\ C\ {\isasymequiv}\ {\isacharparenleft}{\isasymforall}S\ {\isasymin}\ C{\isachardot}\ {\isasymforall}S{\isacharprime}{\isasymsubseteq}S{\isachardot}\ S{\isacharprime}\ {\isasymin}\ C{\isacharparenright}{\isachardoublequoteclose}%
\begin{isamarkuptext}%
Mostremos algunos ejemplos para ilustrar la definición. Para ello, veamos si las colecciones
  de conjuntos de fórmulas proposicionales expuestas en los ejemplos anteriores son cerradas bajo 
  subconjuntos.%
\end{isamarkuptext}\isamarkuptrue%
\isacommand{lemma}\isamarkupfalse%
\ {\isachardoublequoteopen}subset{\isacharunderscore}closed\ {\isacharbraceleft}{\isacharbraceleft}{\isacharbraceright}{\isacharbraceright}{\isachardoublequoteclose}\isanewline
%
\isadelimproof
\ \ %
\endisadelimproof
%
\isatagproof
\isacommand{unfolding}\isamarkupfalse%
\ subset{\isacharunderscore}closed{\isacharunderscore}def\ \isacommand{by}\isamarkupfalse%
\ simp%
\endisatagproof
{\isafoldproof}%
%
\isadelimproof
\isanewline
%
\endisadelimproof
\isanewline
\isacommand{lemma}\isamarkupfalse%
\ {\isachardoublequoteopen}{\isasymnot}\ subset{\isacharunderscore}closed\ {\isacharbraceleft}{\isacharbraceleft}Atom\ {\isadigit{0}}{\isacharbraceright}{\isacharbraceright}{\isachardoublequoteclose}\isanewline
%
\isadelimproof
\ \ %
\endisadelimproof
%
\isatagproof
\isacommand{unfolding}\isamarkupfalse%
\ subset{\isacharunderscore}closed{\isacharunderscore}def\ \isacommand{by}\isamarkupfalse%
\ auto%
\endisatagproof
{\isafoldproof}%
%
\isadelimproof
%
\endisadelimproof
%
\begin{isamarkuptext}%
Observemos que, puesto que el conjunto vacío es subconjunto de todo conjunto, una
  condición necesaria para que una colección sea cerrada bajo subconjuntos es que contenga al
  conjunto vacío.%
\end{isamarkuptext}\isamarkuptrue%
\isacommand{lemma}\isamarkupfalse%
\ {\isachardoublequoteopen}subset{\isacharunderscore}closed\ {\isacharbraceleft}{\isacharbraceleft}Atom\ {\isadigit{0}}{\isacharbraceright}{\isacharcomma}{\isacharbraceleft}{\isacharbraceright}{\isacharbraceright}{\isachardoublequoteclose}\isanewline
%
\isadelimproof
\ \ %
\endisadelimproof
%
\isatagproof
\isacommand{unfolding}\isamarkupfalse%
\ subset{\isacharunderscore}closed{\isacharunderscore}def\ \isacommand{by}\isamarkupfalse%
\ auto%
\endisatagproof
{\isafoldproof}%
%
\isadelimproof
%
\endisadelimproof
%
\begin{isamarkuptext}%
De este modo, se deduce fácilmente que el resto de colecciones expuestas en los ejemplos
  anteriores no son cerradas bajo subconjuntos.%
\end{isamarkuptext}\isamarkuptrue%
\isacommand{lemma}\isamarkupfalse%
\ {\isachardoublequoteopen}{\isasymnot}\ subset{\isacharunderscore}closed\ {\isacharbraceleft}{\isacharbraceleft}{\isacharparenleft}\isactrlbold {\isasymnot}\ {\isacharparenleft}Atom\ {\isadigit{1}}{\isacharparenright}{\isacharparenright}\ \isactrlbold {\isasymrightarrow}\ Atom\ {\isadigit{2}}{\isacharbraceright}{\isacharcomma}\isanewline
\ \ \ {\isacharbraceleft}{\isacharparenleft}{\isacharparenleft}\isactrlbold {\isasymnot}\ {\isacharparenleft}Atom\ {\isadigit{1}}{\isacharparenright}{\isacharparenright}\ \isactrlbold {\isasymrightarrow}\ Atom\ {\isadigit{2}}{\isacharparenright}{\isacharcomma}\ \isactrlbold {\isasymnot}{\isacharparenleft}\isactrlbold {\isasymnot}\ {\isacharparenleft}Atom\ {\isadigit{1}}{\isacharparenright}{\isacharparenright}{\isacharbraceright}{\isacharcomma}\isanewline
\ \ {\isacharbraceleft}{\isacharparenleft}{\isacharparenleft}\isactrlbold {\isasymnot}\ {\isacharparenleft}Atom\ {\isadigit{1}}{\isacharparenright}{\isacharparenright}\ \isactrlbold {\isasymrightarrow}\ Atom\ {\isadigit{2}}{\isacharparenright}{\isacharcomma}\ \isactrlbold {\isasymnot}{\isacharparenleft}\isactrlbold {\isasymnot}\ {\isacharparenleft}Atom\ {\isadigit{1}}{\isacharparenright}{\isacharparenright}{\isacharcomma}\ \ Atom\ {\isadigit{1}}{\isacharbraceright}{\isacharbraceright}{\isachardoublequoteclose}\ \isanewline
%
\isadelimproof
\ \ %
\endisadelimproof
%
\isatagproof
\isacommand{unfolding}\isamarkupfalse%
\ subset{\isacharunderscore}closed{\isacharunderscore}def\ \isacommand{by}\isamarkupfalse%
\ auto%
\endisatagproof
{\isafoldproof}%
%
\isadelimproof
\isanewline
%
\endisadelimproof
\isanewline
\isacommand{lemma}\isamarkupfalse%
\ {\isachardoublequoteopen}{\isasymnot}\ subset{\isacharunderscore}closed\ {\isacharbraceleft}{\isacharbraceleft}{\isacharparenleft}\isactrlbold {\isasymnot}\ {\isacharparenleft}Atom\ {\isadigit{1}}{\isacharparenright}{\isacharparenright}\ \isactrlbold {\isasymrightarrow}\ Atom\ {\isadigit{2}}{\isacharbraceright}{\isacharcomma}\isanewline
\ \ \ {\isacharbraceleft}{\isacharparenleft}{\isacharparenleft}\isactrlbold {\isasymnot}\ {\isacharparenleft}Atom\ {\isadigit{1}}{\isacharparenright}{\isacharparenright}\ \isactrlbold {\isasymrightarrow}\ Atom\ {\isadigit{2}}{\isacharparenright}{\isacharcomma}\ \isactrlbold {\isasymnot}{\isacharparenleft}\isactrlbold {\isasymnot}\ {\isacharparenleft}Atom\ {\isadigit{1}}{\isacharparenright}{\isacharparenright}{\isacharbraceright}{\isacharbraceright}{\isachardoublequoteclose}\ \isanewline
%
\isadelimproof
\ \ %
\endisadelimproof
%
\isatagproof
\isacommand{unfolding}\isamarkupfalse%
\ subset{\isacharunderscore}closed{\isacharunderscore}def\ \isacommand{by}\isamarkupfalse%
\ auto%
\endisatagproof
{\isafoldproof}%
%
\isadelimproof
%
\endisadelimproof
%
\begin{isamarkuptext}%
Continuemos con la noción de propiedad de carácter finito.

\begin{definicion}
  Una colección de conjuntos tiene la \isa{propiedad\ de\ carácter\ finito} si para cualquier conjunto
  son equivalentes:
  \begin{enumerate}
    \item El conjunto pertenece a la colección.
    \item Todo subconjunto finito suyo pertenece a la colección.
  \end{enumerate}
\end{definicion}

  La formalización en Isabelle/HOL de dicha definición se muestra a continuación.%
\end{isamarkuptext}\isamarkuptrue%
\isacommand{definition}\isamarkupfalse%
\ {\isachardoublequoteopen}finite{\isacharunderscore}character\ C\ {\isasymequiv}\ \isanewline
\ \ \ \ \ \ \ \ \ \ \ \ {\isacharparenleft}{\isasymforall}S{\isachardot}\ S\ {\isasymin}\ C\ {\isasymlongleftrightarrow}\ {\isacharparenleft}{\isasymforall}S{\isacharprime}\ {\isasymsubseteq}\ S{\isachardot}\ finite\ S{\isacharprime}\ {\isasymlongrightarrow}\ S{\isacharprime}\ {\isasymin}\ C{\isacharparenright}{\isacharparenright}{\isachardoublequoteclose}%
\begin{isamarkuptext}%
Distingamos las colecciones de los ejemplos anteriores que tengan la propiedad de carácter 
  finito. Análogamente, puesto que el conjunto vacío es finito y subconjunto de cualquier conjunto, 
  se observa que una condición necesaria para que una colección tenga la propiedad de carácter 
  finito es que contenga al conjunto vacío.%
\end{isamarkuptext}\isamarkuptrue%
\isacommand{lemma}\isamarkupfalse%
\ {\isachardoublequoteopen}finite{\isacharunderscore}character\ {\isacharbraceleft}{\isacharbraceleft}{\isacharbraceright}{\isacharbraceright}{\isachardoublequoteclose}\isanewline
%
\isadelimproof
\ \ %
\endisadelimproof
%
\isatagproof
\isacommand{unfolding}\isamarkupfalse%
\ finite{\isacharunderscore}character{\isacharunderscore}def\ \isacommand{by}\isamarkupfalse%
\ auto%
\endisatagproof
{\isafoldproof}%
%
\isadelimproof
\isanewline
%
\endisadelimproof
\isanewline
\isacommand{lemma}\isamarkupfalse%
\ {\isachardoublequoteopen}{\isasymnot}\ finite{\isacharunderscore}character\ {\isacharbraceleft}{\isacharbraceleft}Atom\ {\isadigit{0}}{\isacharbraceright}{\isacharbraceright}{\isachardoublequoteclose}\isanewline
%
\isadelimproof
\ \ %
\endisadelimproof
%
\isatagproof
\isacommand{unfolding}\isamarkupfalse%
\ finite{\isacharunderscore}character{\isacharunderscore}def\ \isacommand{by}\isamarkupfalse%
\ auto%
\endisatagproof
{\isafoldproof}%
%
\isadelimproof
\isanewline
%
\endisadelimproof
\isanewline
\isacommand{lemma}\isamarkupfalse%
\ {\isachardoublequoteopen}finite{\isacharunderscore}character\ {\isacharbraceleft}{\isacharbraceleft}Atom\ {\isadigit{0}}{\isacharbraceright}{\isacharcomma}{\isacharbraceleft}{\isacharbraceright}{\isacharbraceright}{\isachardoublequoteclose}\isanewline
%
\isadelimproof
\ \ %
\endisadelimproof
%
\isatagproof
\isacommand{unfolding}\isamarkupfalse%
\ finite{\isacharunderscore}character{\isacharunderscore}def\ \isacommand{by}\isamarkupfalse%
\ auto%
\endisatagproof
{\isafoldproof}%
%
\isadelimproof
%
\endisadelimproof
%
\begin{isamarkuptext}%
Una vez introducidas las definiciones anteriores, veamos los resultados que las relacionan
  con la propiedad de consistencia proposicional. De este modo, combinándolos en la prueba del 
  \isa{teorema\ de\ existencia\ de\ modelo}, dada una colección \isa{C} cualquiera que verifique la propiedad 
  de consistencia proposicional, podemos extenderla a una colección \isa{C{\isacharprime}} que también la verifique y 
  además sea cerrada bajo subconjuntos y de carácter finito.

\comentario{Volver a revisar el párrafo anterior al final de la
redacción de la sección.}

  \begin{lema}
    Toda colección de conjuntos con la propiedad de consistencia proposicional se puede extender a
    una colección que también verifique la propiedad de consistencia proposicional y sea cerrada 
    bajo subconjuntos.
  \end{lema}

  En Isabelle se formaliza el resultado de la siguiente manera.%
\end{isamarkuptext}\isamarkuptrue%
\isacommand{lemma}\isamarkupfalse%
\ {\isachardoublequoteopen}pcp\ C\ {\isasymLongrightarrow}\ {\isasymexists}C{\isacharprime}{\isachardot}\ C\ {\isasymsubseteq}\ C{\isacharprime}\ {\isasymand}\ pcp\ C{\isacharprime}\ {\isasymand}\ subset{\isacharunderscore}closed\ C{\isacharprime}{\isachardoublequoteclose}\isanewline
%
\isadelimproof
\ \ %
\endisadelimproof
%
\isatagproof
\isacommand{oops}\isamarkupfalse%
%
\endisatagproof
{\isafoldproof}%
%
\isadelimproof
%
\endisadelimproof
%
\begin{isamarkuptext}%
Procedamos con su demostración.

\begin{demostracion}
  Dada una colección de conjuntos cualquiera \isa{C}, consideremos la colección formada por los 
  conjuntos tales que son subconjuntos de algún conjunto de \isa{C}. Notemos esta colección por 
  \isa{C{\isacharprime}\ {\isacharequal}\ {\isacharbraceleft}S{\isacharprime}{\isachardot}\ {\isasymexists}S{\isasymin}C{\isachardot}\ S{\isacharprime}\ {\isasymsubseteq}\ S{\isacharbraceright}}. Vamos a probar que, en efecto, \isa{C{\isacharprime}} verifica  las condiciones del lema.

  En primer lugar, veamos que \isa{C} está contenida en \isa{C{\isacharprime}}. Para ello, consideremos un conjunto
  cualquiera perteneciente a \isa{C}. Puesto que la propiedad de contención es reflexiva, dicho conjunto 
  es subconjunto de sí mismo. De este modo, por definición de \isa{C{\isacharprime}}, se verifica que el conjunto 
  pertenece a \isa{C{\isacharprime}}.

  Por otro lado, comprobemos que \isa{C{\isacharprime}} tiene la propiedad de consistencia proposicional.
  Por el lema de caracterización de la propiedad de consistencia proposicional mediante la
  notación uniforme basta probar que, para cualquier conjunto de fórmulas \isa{S} de \isa{C{\isacharprime}}, se 
  verifican las condiciones:
  \begin{itemize}
    \item \isa{{\isasymbottom}} no pertenece a \isa{S}.
    \item Dada \isa{p} una fórmula atómica cualquiera, no se tiene 
    simultáneamente que\\ \isa{p\ {\isasymin}\ S} y \isa{{\isasymnot}\ p\ {\isasymin}\ S}.
    \item Para toda fórmula de tipo \isa{{\isasymalpha}} con componentes \isa{{\isasymalpha}\isactrlsub {\isadigit{1}}} y \isa{{\isasymalpha}\isactrlsub {\isadigit{2}}} tal que \isa{{\isasymalpha}}
    pertenece a \isa{S}, se tiene que \isa{{\isacharbraceleft}{\isasymalpha}\isactrlsub {\isadigit{1}}{\isacharcomma}{\isasymalpha}\isactrlsub {\isadigit{2}}{\isacharbraceright}\ {\isasymunion}\ S} pertenece a \isa{C{\isacharprime}}.
    \item Para toda fórmula de tipo \isa{{\isasymbeta}} con componentes \isa{{\isasymbeta}\isactrlsub {\isadigit{1}}} y \isa{{\isasymbeta}\isactrlsub {\isadigit{2}}} tal que \isa{{\isasymbeta}}
    pertenece a \isa{S}, se tiene que o bien \isa{{\isacharbraceleft}{\isasymbeta}\isactrlsub {\isadigit{1}}{\isacharbraceright}\ {\isasymunion}\ S} pertenece a \isa{C{\isacharprime}} o 
    bien \isa{{\isacharbraceleft}{\isasymbeta}\isactrlsub {\isadigit{2}}{\isacharbraceright}\ {\isasymunion}\ S} pertenece a \isa{C{\isacharprime}}.
  \end{itemize} 

  De este modo, sea \isa{S} un conjunto de fórmulas cualquiera de la colección \isa{C{\isacharprime}}. Por definición de
  dicha colección, existe un conjunto \isa{S{\isacharprime}} pertenciente a \isa{C} tal que \isa{S} está contenido en \isa{S{\isacharprime}}.
  Como \isa{C} tiene la propiedad de consistencia proposicional por hipótesis, verifica las condiciones
  del lema de caracterización de la propiedad de consistencia proposicional mediante la notación 
  uniforme. En particular, puesto que \isa{S{\isacharprime}} pertenece a \isa{C}, se verifica: 
  \begin{itemize}
    \item \isa{{\isasymbottom}} no pertenece a \isa{S{\isacharprime}}.
    \item Dada \isa{p} una fórmula atómica cualquiera, no se tiene 
    simultáneamente que\\ \isa{p\ {\isasymin}\ S{\isacharprime}} y \isa{{\isasymnot}\ p\ {\isasymin}\ S{\isacharprime}}.
    \item Para toda fórmula de tipo \isa{{\isasymalpha}} con componentes \isa{{\isasymalpha}\isactrlsub {\isadigit{1}}} y \isa{{\isasymalpha}\isactrlsub {\isadigit{2}}} tal que \isa{{\isasymalpha}}
    pertenece a \isa{S{\isacharprime}}, se tiene que \isa{{\isacharbraceleft}{\isasymalpha}\isactrlsub {\isadigit{1}}{\isacharcomma}{\isasymalpha}\isactrlsub {\isadigit{2}}{\isacharbraceright}\ {\isasymunion}\ S{\isacharprime}} pertenece a \isa{C}.
    \item Para toda fórmula de tipo \isa{{\isasymbeta}} con componentes \isa{{\isasymbeta}\isactrlsub {\isadigit{1}}} y \isa{{\isasymbeta}\isactrlsub {\isadigit{2}}} tal que \isa{{\isasymbeta}}
    pertenece a \isa{S{\isacharprime}}, se tiene que o bien \isa{{\isacharbraceleft}{\isasymbeta}\isactrlsub {\isadigit{1}}{\isacharbraceright}\ {\isasymunion}\ S{\isacharprime}} pertenece a \isa{C} o 
    bien \isa{{\isacharbraceleft}{\isasymbeta}\isactrlsub {\isadigit{2}}{\isacharbraceright}\ {\isasymunion}\ S{\isacharprime}} pertenece a \isa{C}.
  \end{itemize} 

  Por tanto, como \isa{S} está contenida en \isa{S{\isacharprime}}, se verifica análogamente que \isa{{\isasymbottom}} no pertence a \isa{S}
  y que dada una fórmula atómica cualquiera \isa{p}, no se tiene simultáneamente que\\ \isa{p\ {\isasymin}\ S} y 
  \isa{{\isasymnot}\ p\ {\isasymin}\ S{\isachardot}} Veamos que se verifican el resto de condiciones del lema de caracterización:

  \isa{{\isasymsqdot}\ Condición\ para\ fórmulas\ de\ tipo\ {\isasymalpha}}: Sea una fórmula de tipo \isa{{\isasymalpha}} con componentes \isa{{\isasymalpha}\isactrlsub {\isadigit{1}}} y 
    \isa{{\isasymalpha}\isactrlsub {\isadigit{2}}} tal que \isa{{\isasymalpha}} pertenece a \isa{S}. Como \isa{S} está contenida en \isa{S{\isacharprime}}, tenemos que la fórmula 
    pertence también a \isa{S{\isacharprime}}. De este modo, se verifica que \isa{{\isacharbraceleft}{\isasymalpha}\isactrlsub {\isadigit{1}}{\isacharcomma}{\isasymalpha}\isactrlsub {\isadigit{2}}{\isacharbraceright}\ {\isasymunion}\ S{\isacharprime}} pertenece a la colección 
    \isa{C}. Por otro lado, como el conjunto \isa{S} está contenido en \isa{S{\isacharprime}}, se observa fácilmente que\\
    \isa{{\isacharbraceleft}{\isasymalpha}\isactrlsub {\isadigit{1}}{\isacharcomma}{\isasymalpha}\isactrlsub {\isadigit{2}}{\isacharbraceright}\ {\isasymunion}\ S} está contenido en \isa{{\isacharbraceleft}{\isasymalpha}\isactrlsub {\isadigit{1}}{\isacharcomma}{\isasymalpha}\isactrlsub {\isadigit{2}}{\isacharbraceright}\ {\isasymunion}\ S{\isacharprime}}. Por lo tanto, el conjunto \isa{{\isacharbraceleft}{\isasymalpha}\isactrlsub {\isadigit{1}}{\isacharcomma}{\isasymalpha}\isactrlsub {\isadigit{2}}{\isacharbraceright}\ {\isasymunion}\ S} está en 
    \isa{C{\isacharprime}} por definición de esta, ya que es subconjunto de \isa{{\isacharbraceleft}{\isasymalpha}\isactrlsub {\isadigit{1}}{\isacharcomma}{\isasymalpha}\isactrlsub {\isadigit{2}}{\isacharbraceright}\ {\isasymunion}\ S{\isacharprime}} que pertence a \isa{C}.

  \isa{{\isasymsqdot}\ Condición\ para\ fórmulas\ de\ tipo\ {\isasymbeta}}: Sea una fórmula de tipo \isa{{\isasymbeta}} con componentes \isa{{\isasymbeta}\isactrlsub {\isadigit{1}}} y
    \isa{{\isasymbeta}\isactrlsub {\isadigit{2}}} tal que la fórmula pertenece a \isa{S}. Como el conjunto \isa{S} está contenido en \isa{S{\isacharprime}}, tenemos 
    que la fórmula pertence, a su vez, a \isa{S{\isacharprime}}. De este modo, se verifica que o bien \isa{{\isacharbraceleft}{\isasymbeta}\isactrlsub {\isadigit{1}}{\isacharbraceright}\ {\isasymunion}\ S{\isacharprime}}
    pertenece a \isa{C} o bien \isa{{\isacharbraceleft}{\isasymbeta}\isactrlsub {\isadigit{2}}{\isacharbraceright}\ {\isasymunion}\ S{\isacharprime}} pertence a \isa{C}. Por eliminación de la disyunción anterior, 
    vamos a probar que o bien \isa{{\isacharbraceleft}{\isasymbeta}\isactrlsub {\isadigit{1}}{\isacharbraceright}\ {\isasymunion}\ S} pertenece a \isa{C{\isacharprime}} o bien \isa{{\isacharbraceleft}{\isasymbeta}\isactrlsub {\isadigit{2}}{\isacharbraceright}\ {\isasymunion}\ S} pertenece a \isa{C{\isacharprime}}.
    \begin{itemize}
      \item Supongamos, en primer lugar, que \isa{{\isacharbraceleft}{\isasymbeta}\isactrlsub {\isadigit{1}}{\isacharbraceright}\ {\isasymunion}\ S{\isacharprime}} pertenece a \isa{C}. Puesto que el conjunto \isa{S}
      está contenido en \isa{S{\isacharprime}}, se observa fácilmente que \isa{{\isacharbraceleft}{\isasymbeta}\isactrlsub {\isadigit{1}}{\isacharbraceright}\ {\isasymunion}\ S} está contenido en\\ \isa{{\isacharbraceleft}{\isasymbeta}\isactrlsub {\isadigit{1}}{\isacharbraceright}\ {\isasymunion}\ S{\isacharprime}}.
      Por definición de la colección \isa{C{\isacharprime}}, tenemos que \isa{{\isacharbraceleft}{\isasymbeta}\isactrlsub {\isadigit{1}}{\isacharbraceright}\ {\isasymunion}\ S} pertenece a \isa{C{\isacharprime}}, ya que es
      subconjunto de \isa{{\isacharbraceleft}{\isasymbeta}\isactrlsub {\isadigit{1}}{\isacharbraceright}\ {\isasymunion}\ S{\isacharprime}} que pertenece a \isa{C}. Por tanto, hemos probado que o bien \isa{{\isacharbraceleft}{\isasymbeta}\isactrlsub {\isadigit{1}}{\isacharbraceright}\ {\isasymunion}\ S} 
      pertenece a \isa{C{\isacharprime}} o bien \isa{{\isacharbraceleft}{\isasymbeta}\isactrlsub {\isadigit{2}}{\isacharbraceright}\ {\isasymunion}\ S} pertenece a \isa{C{\isacharprime}}.
      \item Supongamos, finalmente, que \isa{{\isacharbraceleft}{\isasymbeta}\isactrlsub {\isadigit{2}}{\isacharbraceright}\ {\isasymunion}\ S{\isacharprime}} pertenece a \isa{C}. Análogamente obtenemos que
      \isa{{\isacharbraceleft}{\isasymbeta}\isactrlsub {\isadigit{2}}{\isacharbraceright}\ {\isasymunion}\ S} está contenido en \isa{{\isacharbraceleft}{\isasymbeta}\isactrlsub {\isadigit{2}}{\isacharbraceright}\ {\isasymunion}\ S{\isacharprime}}, luego \isa{{\isacharbraceleft}{\isasymbeta}\isactrlsub {\isadigit{2}}{\isacharbraceright}\ {\isasymunion}\ S} pertenece a \isa{C{\isacharprime}} por definición.
      Por tanto, o bien \isa{{\isacharbraceleft}{\isasymbeta}\isactrlsub {\isadigit{1}}{\isacharbraceright}\ {\isasymunion}\ S} pertenece a \isa{C{\isacharprime}} o bien \isa{{\isacharbraceleft}{\isasymbeta}\isactrlsub {\isadigit{2}}{\isacharbraceright}\ {\isasymunion}\ S} pertenece a \isa{C{\isacharprime}}.
    \end{itemize}
    De esta manera, queda probado que dada una fórmula de tipo \isa{{\isasymbeta}} y componentes \isa{{\isasymbeta}\isactrlsub {\isadigit{1}}} y \isa{{\isasymbeta}\isactrlsub {\isadigit{2}}} tal que
    pertenezca al conjunto \isa{S}, se verifica que o bien \isa{{\isacharbraceleft}{\isasymbeta}\isactrlsub {\isadigit{1}}{\isacharbraceright}\ {\isasymunion}\ S} pertenece a \isa{C{\isacharprime}} o bien \isa{{\isacharbraceleft}{\isasymbeta}\isactrlsub {\isadigit{2}}{\isacharbraceright}\ {\isasymunion}\ S}
    pertenece a \isa{C{\isacharprime}}.

  En conclusión, por el lema de caracterización de la propiedad de consistencia proposicional
  mediante la notación uniforme, queda probado que \isa{C{\isacharprime}} tiene la propiedad de consistencia
  proposicional. 

  Finalmente probemos que, además, \isa{C{\isacharprime}} es cerrada bajo subconjuntos. Por definición de ser cerrado
  bajo subconjuntos, basta probar que dado un conjunto perteneciente a \isa{C{\isacharprime}} verifica que todo 
  subconjunto suyo pertenece a \isa{C{\isacharprime}}. Consideremos \isa{S} un conjunto cualquiera de \isa{C{\isacharprime}}. Por
  definición de \isa{C{\isacharprime}}, existe un conjunto \isa{S{\isacharprime}} perteneciente a la colección \isa{C} tal que \isa{S} es
  subconjunto de \isa{S{\isacharprime}}. Sea \isa{S{\isacharprime}{\isacharprime}} un subconjunto cualquiera de \isa{S}. Como \isa{S} es subconjunto de \isa{S{\isacharprime}},
  se tiene que \isa{S{\isacharprime}{\isacharprime}} es, a su vez, subconjunto de \isa{S{\isacharprime}}. De este modo, existe un conjunto 
  perteneciente a la colección \isa{C} del cual \isa{S{\isacharprime}{\isacharprime}} es subconjunto. Por tanto, por definición de \isa{C{\isacharprime}}, 
  \isa{S{\isacharprime}{\isacharprime}} pertenece a la colección \isa{C{\isacharprime}}, como quería demostrar.
\end{demostracion}

  Procedamos con las demostraciones del lema en Isabelle/HOL.

  En primer lugar, vamos a introducir dos lemas auxiliares que emplearemos a lo largo de
  esta sección. El primero se trata de un lema similar al lema \isa{ballI} definido en Isabelle pero 
  considerando la relación de contención en lugar de la de pertenencia.%
\end{isamarkuptext}\isamarkuptrue%
\isacommand{lemma}\isamarkupfalse%
\ sallI{\isacharcolon}\ {\isachardoublequoteopen}{\isacharparenleft}{\isasymAnd}S{\isachardot}\ S\ {\isasymsubseteq}\ A\ {\isasymLongrightarrow}\ P\ S{\isacharparenright}\ {\isasymLongrightarrow}\ {\isasymforall}S\ {\isasymsubseteq}\ A{\isachardot}\ P\ S{\isachardoublequoteclose}\isanewline
%
\isadelimproof
\ \ %
\endisadelimproof
%
\isatagproof
\isacommand{by}\isamarkupfalse%
\ simp%
\endisatagproof
{\isafoldproof}%
%
\isadelimproof
%
\endisadelimproof
%
\begin{isamarkuptext}%
Por último definimos el siguiente lema auxiliar similar al lema \isa{bspec} de Isabelle/HOL
  considerando, análogamente, la relación de contención en lugar de la de pertenencia.%
\end{isamarkuptext}\isamarkuptrue%
\isacommand{lemma}\isamarkupfalse%
\ sspec{\isacharcolon}\ {\isachardoublequoteopen}{\isasymforall}S\ {\isasymsubseteq}\ A{\isachardot}\ P\ S\ {\isasymLongrightarrow}\ S\ {\isasymsubseteq}\ A\ {\isasymLongrightarrow}\ P\ S{\isachardoublequoteclose}\isanewline
%
\isadelimproof
\ \ %
\endisadelimproof
%
\isatagproof
\isacommand{by}\isamarkupfalse%
\ simp%
\endisatagproof
{\isafoldproof}%
%
\isadelimproof
%
\endisadelimproof
%
\begin{isamarkuptext}%
Veamos la prueba detallada del lema en Isabelle/HOL. Esta se fundamenta en tres lemas
  auxiliares: el primero prueba que la colección \isa{C} está contenida en \isa{C{\isacharprime}}, el segundo que
  \isa{C{\isacharprime}} tiene la propiedad de consistencia proposicional y, finalmente, el tercer lema demuestra que
  \isa{C{\isacharprime}} es cerrada bajo subconjuntos. En primer lugar, dada una colección cualquiera \isa{C}, definiremos 
  en Isabelle su extensión \isa{C{\isacharprime}} como sigue.%
\end{isamarkuptext}\isamarkuptrue%
\isacommand{definition}\isamarkupfalse%
\ extensionSC\ {\isacharcolon}{\isacharcolon}\ {\isachardoublequoteopen}{\isacharparenleft}{\isacharparenleft}{\isacharprime}a\ formula{\isacharparenright}\ set{\isacharparenright}\ set\ {\isasymRightarrow}\ {\isacharparenleft}{\isacharparenleft}{\isacharprime}a\ formula{\isacharparenright}\ set{\isacharparenright}\ set{\isachardoublequoteclose}\isanewline
\ \ \isakeyword{where}\ extensionSC{\isacharcolon}\ {\isachardoublequoteopen}extensionSC\ C\ {\isacharequal}\ {\isacharbraceleft}s{\isachardot}\ {\isasymexists}S{\isasymin}C{\isachardot}\ s\ {\isasymsubseteq}\ S{\isacharbraceright}{\isachardoublequoteclose}%
\begin{isamarkuptext}%
Una vez formalizada la extensión en Isabelle, comencemos probando de manera detallada que toda
  colección está contenida en su extensión así definida.%
\end{isamarkuptext}\isamarkuptrue%
\isacommand{lemma}\isamarkupfalse%
\ ex{\isadigit{1}}{\isacharunderscore}subset{\isacharcolon}\ {\isachardoublequoteopen}C\ {\isasymsubseteq}\ {\isacharparenleft}extensionSC\ C{\isacharparenright}{\isachardoublequoteclose}\isanewline
%
\isadelimproof
%
\endisadelimproof
%
\isatagproof
\isacommand{proof}\isamarkupfalse%
\ {\isacharparenleft}rule\ subsetI{\isacharparenright}\isanewline
\ \ \isacommand{fix}\isamarkupfalse%
\ s\isanewline
\ \ \isacommand{assume}\isamarkupfalse%
\ {\isachardoublequoteopen}s\ {\isasymin}\ C{\isachardoublequoteclose}\isanewline
\ \ \isacommand{have}\isamarkupfalse%
\ {\isachardoublequoteopen}s\ {\isasymsubseteq}\ s{\isachardoublequoteclose}\isanewline
\ \ \ \ \isacommand{by}\isamarkupfalse%
\ {\isacharparenleft}rule\ subset{\isacharunderscore}refl{\isacharparenright}\isanewline
\ \ \isacommand{then}\isamarkupfalse%
\ \isacommand{have}\isamarkupfalse%
\ {\isachardoublequoteopen}{\isasymexists}S{\isasymin}C{\isachardot}\ s\ {\isasymsubseteq}\ S{\isachardoublequoteclose}\isanewline
\ \ \ \ \isacommand{using}\isamarkupfalse%
\ {\isacartoucheopen}s\ {\isasymin}\ C{\isacartoucheclose}\ \isacommand{by}\isamarkupfalse%
\ {\isacharparenleft}rule\ bexI{\isacharparenright}\isanewline
\ \ \isacommand{thus}\isamarkupfalse%
\ {\isachardoublequoteopen}s\ {\isasymin}\ {\isacharparenleft}extensionSC\ C{\isacharparenright}{\isachardoublequoteclose}\isanewline
\ \ \ \ \isacommand{by}\isamarkupfalse%
\ {\isacharparenleft}simp\ only{\isacharcolon}\ mem{\isacharunderscore}Collect{\isacharunderscore}eq\ extensionSC{\isacharparenright}\isanewline
\isacommand{qed}\isamarkupfalse%
%
\endisatagproof
{\isafoldproof}%
%
\isadelimproof
%
\endisadelimproof
%
\begin{isamarkuptext}%
Prosigamos con la prueba del lema auxiliar que demuestra que \isa{C{\isacharprime}} tiene la propiedad
  de consistencia proposicional. Para ello, emplearemos un lema auxiliar que amplia el lema de 
  Isabelle \isa{insert{\isacharunderscore}is{\isacharunderscore}Un} para la unión de dos elementos y un conjunto, como se muestra a 
  continuación.%
\end{isamarkuptext}\isamarkuptrue%
\isacommand{lemma}\isamarkupfalse%
\ insertSetElem{\isacharcolon}\ {\isachardoublequoteopen}insert\ a\ {\isacharparenleft}insert\ b\ C{\isacharparenright}\ {\isacharequal}\ {\isacharbraceleft}a{\isacharcomma}b{\isacharbraceright}\ {\isasymunion}\ C{\isachardoublequoteclose}\isanewline
%
\isadelimproof
\ \ %
\endisadelimproof
%
\isatagproof
\isacommand{by}\isamarkupfalse%
\ simp%
\endisatagproof
{\isafoldproof}%
%
\isadelimproof
%
\endisadelimproof
%
\begin{isamarkuptext}%
Una vez introducido dicho lema auxiliar, podemos dar la prueba detallada del lema que 
  demuestra que \isa{C{\isacharprime}} tiene la propiedad de consistencia proposicional.%
\end{isamarkuptext}\isamarkuptrue%
\isacommand{lemma}\isamarkupfalse%
\ ex{\isadigit{1}}{\isacharunderscore}pcp{\isacharcolon}\ \isanewline
\ \ \isakeyword{assumes}\ {\isachardoublequoteopen}pcp\ C{\isachardoublequoteclose}\isanewline
\ \ \isakeyword{shows}\ {\isachardoublequoteopen}pcp\ {\isacharparenleft}extensionSC\ C{\isacharparenright}{\isachardoublequoteclose}\isanewline
%
\isadelimproof
%
\endisadelimproof
%
\isatagproof
\isacommand{proof}\isamarkupfalse%
\ {\isacharminus}\isanewline
\ \ \isacommand{have}\isamarkupfalse%
\ C{\isadigit{1}}{\isacharcolon}\ {\isachardoublequoteopen}C\ {\isasymsubseteq}\ {\isacharparenleft}extensionSC\ C{\isacharparenright}{\isachardoublequoteclose}\isanewline
\ \ \ \ \isacommand{by}\isamarkupfalse%
\ {\isacharparenleft}rule\ ex{\isadigit{1}}{\isacharunderscore}subset{\isacharparenright}\isanewline
\ \ \isacommand{show}\isamarkupfalse%
\ {\isachardoublequoteopen}pcp\ {\isacharparenleft}extensionSC\ C{\isacharparenright}{\isachardoublequoteclose}\isanewline
\ \ \isacommand{proof}\isamarkupfalse%
\ {\isacharparenleft}rule\ pcp{\isacharunderscore}alt{\isadigit{2}}{\isacharparenright}\isanewline
\ \ \ \ \isacommand{show}\isamarkupfalse%
\ {\isachardoublequoteopen}{\isasymforall}S\ {\isasymin}\ {\isacharparenleft}extensionSC\ C{\isacharparenright}{\isachardot}\ {\isacharparenleft}{\isasymbottom}\ {\isasymnotin}\ S\isanewline
\ \ \ \ {\isasymand}\ {\isacharparenleft}{\isasymforall}k{\isachardot}\ Atom\ k\ {\isasymin}\ S\ {\isasymlongrightarrow}\ \isactrlbold {\isasymnot}\ {\isacharparenleft}Atom\ k{\isacharparenright}\ {\isasymin}\ S\ {\isasymlongrightarrow}\ False{\isacharparenright}\isanewline
\ \ \ \ {\isasymand}\ {\isacharparenleft}{\isasymforall}F\ G\ H{\isachardot}\ Con\ F\ G\ H\ {\isasymlongrightarrow}\ F\ {\isasymin}\ S\ {\isasymlongrightarrow}\ {\isacharbraceleft}G{\isacharcomma}H{\isacharbraceright}\ {\isasymunion}\ S\ {\isasymin}\ {\isacharparenleft}extensionSC\ C{\isacharparenright}{\isacharparenright}\isanewline
\ \ \ \ {\isasymand}\ {\isacharparenleft}{\isasymforall}F\ G\ H{\isachardot}\ Dis\ F\ G\ H\ {\isasymlongrightarrow}\ F\ {\isasymin}\ S\ {\isasymlongrightarrow}\ {\isacharbraceleft}G{\isacharbraceright}\ {\isasymunion}\ S\ {\isasymin}\ {\isacharparenleft}extensionSC\ C{\isacharparenright}\ {\isasymor}\ {\isacharbraceleft}H{\isacharbraceright}\ {\isasymunion}\ S\ {\isasymin}\ {\isacharparenleft}extensionSC\ C{\isacharparenright}{\isacharparenright}{\isacharparenright}{\isachardoublequoteclose}\isanewline
\ \ \ \ \isacommand{proof}\isamarkupfalse%
\ {\isacharparenleft}rule\ ballI{\isacharparenright}\isanewline
\ \ \ \ \ \ \isacommand{fix}\isamarkupfalse%
\ S{\isacharprime}\isanewline
\ \ \ \ \ \ \isacommand{assume}\isamarkupfalse%
\ {\isachardoublequoteopen}S{\isacharprime}\ {\isasymin}\ {\isacharparenleft}extensionSC\ C{\isacharparenright}{\isachardoublequoteclose}\isanewline
\ \ \ \ \ \ \isacommand{then}\isamarkupfalse%
\ \isacommand{have}\isamarkupfalse%
\ {\isadigit{1}}{\isacharcolon}{\isachardoublequoteopen}{\isasymexists}S\ {\isasymin}\ C{\isachardot}\ S{\isacharprime}\ {\isasymsubseteq}\ S{\isachardoublequoteclose}\isanewline
\ \ \ \ \ \ \ \ \isacommand{unfolding}\isamarkupfalse%
\ extensionSC\ \isacommand{by}\isamarkupfalse%
\ {\isacharparenleft}rule\ CollectD{\isacharparenright}\ \ \isanewline
\ \ \ \ \ \ \isacommand{obtain}\isamarkupfalse%
\ S\ \isakeyword{where}\ {\isachardoublequoteopen}S\ {\isasymin}\ C{\isachardoublequoteclose}\ {\isachardoublequoteopen}S{\isacharprime}\ {\isasymsubseteq}\ S{\isachardoublequoteclose}\isanewline
\ \ \ \ \ \ \ \ \isacommand{using}\isamarkupfalse%
\ {\isadigit{1}}\ \isacommand{by}\isamarkupfalse%
\ {\isacharparenleft}rule\ bexE{\isacharparenright}\isanewline
\ \ \ \ \ \ \isacommand{have}\isamarkupfalse%
\ {\isachardoublequoteopen}{\isasymforall}S\ {\isasymin}\ C{\isachardot}\isanewline
\ \ \ \ \ \ {\isasymbottom}\ {\isasymnotin}\ S\isanewline
\ \ \ \ \ \ {\isasymand}\ {\isacharparenleft}{\isasymforall}k{\isachardot}\ Atom\ k\ {\isasymin}\ S\ {\isasymlongrightarrow}\ \isactrlbold {\isasymnot}\ {\isacharparenleft}Atom\ k{\isacharparenright}\ {\isasymin}\ S\ {\isasymlongrightarrow}\ False{\isacharparenright}\isanewline
\ \ \ \ \ \ {\isasymand}\ {\isacharparenleft}{\isasymforall}F\ G\ H{\isachardot}\ Con\ F\ G\ H\ {\isasymlongrightarrow}\ F\ {\isasymin}\ S\ {\isasymlongrightarrow}\ {\isacharbraceleft}G{\isacharcomma}H{\isacharbraceright}\ {\isasymunion}\ S\ {\isasymin}\ C{\isacharparenright}\isanewline
\ \ \ \ \ \ {\isasymand}\ {\isacharparenleft}{\isasymforall}F\ G\ H{\isachardot}\ Dis\ F\ G\ H\ {\isasymlongrightarrow}\ F\ {\isasymin}\ S\ {\isasymlongrightarrow}\ {\isacharbraceleft}G{\isacharbraceright}\ {\isasymunion}\ S\ {\isasymin}\ C\ {\isasymor}\ {\isacharbraceleft}H{\isacharbraceright}\ {\isasymunion}\ S\ {\isasymin}\ C{\isacharparenright}{\isachardoublequoteclose}\isanewline
\ \ \ \ \ \ \ \ \isacommand{using}\isamarkupfalse%
\ assms\ \isacommand{by}\isamarkupfalse%
\ {\isacharparenleft}rule\ pcp{\isacharunderscore}alt{\isadigit{1}}{\isacharparenright}\isanewline
\ \ \ \ \ \ \isacommand{then}\isamarkupfalse%
\ \isacommand{have}\isamarkupfalse%
\ H{\isacharcolon}{\isachardoublequoteopen}{\isasymbottom}\ {\isasymnotin}\ S\isanewline
\ \ \ \ \ \ {\isasymand}\ {\isacharparenleft}{\isasymforall}k{\isachardot}\ Atom\ k\ {\isasymin}\ S\ {\isasymlongrightarrow}\ \isactrlbold {\isasymnot}\ {\isacharparenleft}Atom\ k{\isacharparenright}\ {\isasymin}\ S\ {\isasymlongrightarrow}\ False{\isacharparenright}\isanewline
\ \ \ \ \ \ {\isasymand}\ {\isacharparenleft}{\isasymforall}F\ G\ H{\isachardot}\ Con\ F\ G\ H\ {\isasymlongrightarrow}\ F\ {\isasymin}\ S\ {\isasymlongrightarrow}\ {\isacharbraceleft}G{\isacharcomma}H{\isacharbraceright}\ {\isasymunion}\ S\ {\isasymin}\ C{\isacharparenright}\isanewline
\ \ \ \ \ \ {\isasymand}\ {\isacharparenleft}{\isasymforall}F\ G\ H{\isachardot}\ Dis\ F\ G\ H\ {\isasymlongrightarrow}\ F\ {\isasymin}\ S\ {\isasymlongrightarrow}\ {\isacharbraceleft}G{\isacharbraceright}\ {\isasymunion}\ S\ {\isasymin}\ C\ {\isasymor}\ {\isacharbraceleft}H{\isacharbraceright}\ {\isasymunion}\ S\ {\isasymin}\ C{\isacharparenright}{\isachardoublequoteclose}\isanewline
\ \ \ \ \ \ \ \ \isacommand{using}\isamarkupfalse%
\ {\isacartoucheopen}S\ {\isasymin}\ C{\isacartoucheclose}\ \isacommand{by}\isamarkupfalse%
\ {\isacharparenleft}rule\ bspec{\isacharparenright}\isanewline
\ \ \ \ \ \ \isacommand{then}\isamarkupfalse%
\ \isacommand{have}\isamarkupfalse%
\ {\isachardoublequoteopen}{\isasymbottom}\ {\isasymnotin}\ S{\isachardoublequoteclose}\isanewline
\ \ \ \ \ \ \ \ \isacommand{by}\isamarkupfalse%
\ {\isacharparenleft}rule\ conjunct{\isadigit{1}}{\isacharparenright}\isanewline
\ \ \ \ \ \ \isacommand{have}\isamarkupfalse%
\ S{\isadigit{1}}{\isacharcolon}{\isachardoublequoteopen}{\isasymbottom}\ {\isasymnotin}\ S{\isacharprime}{\isachardoublequoteclose}\isanewline
\ \ \ \ \ \ \ \ \isacommand{using}\isamarkupfalse%
\ {\isacartoucheopen}S{\isacharprime}\ {\isasymsubseteq}\ S{\isacartoucheclose}\ {\isacartoucheopen}{\isasymbottom}\ {\isasymnotin}\ S{\isacartoucheclose}\ \isacommand{by}\isamarkupfalse%
\ {\isacharparenleft}rule\ contra{\isacharunderscore}subsetD{\isacharparenright}\isanewline
\ \ \ \ \ \ \isacommand{have}\isamarkupfalse%
\ Atom{\isacharcolon}{\isachardoublequoteopen}{\isasymforall}k{\isachardot}\ Atom\ k\ {\isasymin}\ S\ {\isasymlongrightarrow}\ \isactrlbold {\isasymnot}\ {\isacharparenleft}Atom\ k{\isacharparenright}\ {\isasymin}\ S\ {\isasymlongrightarrow}\ False{\isachardoublequoteclose}\isanewline
\ \ \ \ \ \ \ \ \isacommand{using}\isamarkupfalse%
\ H\ \isacommand{by}\isamarkupfalse%
\ {\isacharparenleft}iprover\ elim{\isacharcolon}\ conjunct{\isadigit{1}}\ conjunct{\isadigit{2}}{\isacharparenright}\isanewline
\ \ \ \ \ \ \isacommand{have}\isamarkupfalse%
\ S{\isadigit{2}}{\isacharcolon}{\isachardoublequoteopen}{\isasymforall}k{\isachardot}\ Atom\ k\ {\isasymin}\ S{\isacharprime}\ {\isasymlongrightarrow}\ \isactrlbold {\isasymnot}\ {\isacharparenleft}Atom\ k{\isacharparenright}\ {\isasymin}\ S{\isacharprime}\ {\isasymlongrightarrow}\ False{\isachardoublequoteclose}\isanewline
\ \ \ \ \ \ \isacommand{proof}\isamarkupfalse%
\ {\isacharparenleft}rule\ allI{\isacharparenright}\isanewline
\ \ \ \ \ \ \ \ \isacommand{fix}\isamarkupfalse%
\ k\isanewline
\ \ \ \ \ \ \ \ \isacommand{show}\isamarkupfalse%
\ {\isachardoublequoteopen}Atom\ k\ {\isasymin}\ S{\isacharprime}\ {\isasymlongrightarrow}\ \isactrlbold {\isasymnot}\ {\isacharparenleft}Atom\ k{\isacharparenright}\ {\isasymin}\ S{\isacharprime}\ {\isasymlongrightarrow}\ False{\isachardoublequoteclose}\isanewline
\ \ \ \ \ \ \ \ \isacommand{proof}\isamarkupfalse%
\ {\isacharparenleft}rule\ impI{\isacharparenright}{\isacharplus}\isanewline
\ \ \ \ \ \ \ \ \ \ \isacommand{assume}\isamarkupfalse%
\ {\isachardoublequoteopen}Atom\ k\ {\isasymin}\ S{\isacharprime}{\isachardoublequoteclose}\isanewline
\ \ \ \ \ \ \ \ \ \ \isacommand{assume}\isamarkupfalse%
\ {\isachardoublequoteopen}\isactrlbold {\isasymnot}\ {\isacharparenleft}Atom\ k{\isacharparenright}\ {\isasymin}\ S{\isacharprime}{\isachardoublequoteclose}\isanewline
\ \ \ \ \ \ \ \ \ \ \isacommand{have}\isamarkupfalse%
\ {\isachardoublequoteopen}Atom\ k\ {\isasymin}\ S{\isachardoublequoteclose}\ \isanewline
\ \ \ \ \ \ \ \ \ \ \ \ \isacommand{using}\isamarkupfalse%
\ {\isacartoucheopen}S{\isacharprime}\ {\isasymsubseteq}\ S{\isacartoucheclose}\ {\isacartoucheopen}Atom\ k\ {\isasymin}\ S{\isacharprime}{\isacartoucheclose}\ \isacommand{by}\isamarkupfalse%
\ {\isacharparenleft}rule\ set{\isacharunderscore}mp{\isacharparenright}\isanewline
\ \ \ \ \ \ \ \ \ \ \isacommand{have}\isamarkupfalse%
\ {\isachardoublequoteopen}\isactrlbold {\isasymnot}\ {\isacharparenleft}Atom\ k{\isacharparenright}\ {\isasymin}\ S{\isachardoublequoteclose}\isanewline
\ \ \ \ \ \ \ \ \ \ \ \ \isacommand{using}\isamarkupfalse%
\ {\isacartoucheopen}S{\isacharprime}\ {\isasymsubseteq}\ S{\isacartoucheclose}\ {\isacartoucheopen}\isactrlbold {\isasymnot}\ {\isacharparenleft}Atom\ k{\isacharparenright}\ {\isasymin}\ S{\isacharprime}{\isacartoucheclose}\ \isacommand{by}\isamarkupfalse%
\ {\isacharparenleft}rule\ set{\isacharunderscore}mp{\isacharparenright}\isanewline
\ \ \ \ \ \ \ \ \ \ \isacommand{have}\isamarkupfalse%
\ {\isachardoublequoteopen}Atom\ k\ {\isasymin}\ S\ {\isasymlongrightarrow}\ \isactrlbold {\isasymnot}\ {\isacharparenleft}Atom\ k{\isacharparenright}\ {\isasymin}\ S\ {\isasymlongrightarrow}\ False{\isachardoublequoteclose}\isanewline
\ \ \ \ \ \ \ \ \ \ \ \ \isacommand{using}\isamarkupfalse%
\ Atom\ \isacommand{by}\isamarkupfalse%
\ {\isacharparenleft}rule\ allE{\isacharparenright}\isanewline
\ \ \ \ \ \ \ \ \ \ \isacommand{then}\isamarkupfalse%
\ \isacommand{have}\isamarkupfalse%
\ {\isachardoublequoteopen}\isactrlbold {\isasymnot}\ {\isacharparenleft}Atom\ k{\isacharparenright}\ {\isasymin}\ S\ {\isasymlongrightarrow}\ False{\isachardoublequoteclose}\isanewline
\ \ \ \ \ \ \ \ \ \ \ \ \isacommand{using}\isamarkupfalse%
\ {\isacartoucheopen}Atom\ k\ {\isasymin}\ S{\isacartoucheclose}\ \isacommand{by}\isamarkupfalse%
\ {\isacharparenleft}rule\ mp{\isacharparenright}\isanewline
\ \ \ \ \ \ \ \ \ \ \isacommand{thus}\isamarkupfalse%
\ {\isachardoublequoteopen}False{\isachardoublequoteclose}\isanewline
\ \ \ \ \ \ \ \ \ \ \ \ \isacommand{using}\isamarkupfalse%
\ {\isacartoucheopen}\isactrlbold {\isasymnot}\ {\isacharparenleft}Atom\ k{\isacharparenright}\ {\isasymin}\ S{\isacartoucheclose}\ \isacommand{by}\isamarkupfalse%
\ {\isacharparenleft}rule\ mp{\isacharparenright}\isanewline
\ \ \ \ \ \ \ \ \isacommand{qed}\isamarkupfalse%
\isanewline
\ \ \ \ \ \ \isacommand{qed}\isamarkupfalse%
\isanewline
\ \ \ \ \ \ \isacommand{have}\isamarkupfalse%
\ Con{\isacharcolon}{\isachardoublequoteopen}{\isasymforall}F\ G\ H{\isachardot}\ Con\ F\ G\ H\ {\isasymlongrightarrow}\ F\ {\isasymin}\ S\ {\isasymlongrightarrow}\ {\isacharbraceleft}G{\isacharcomma}H{\isacharbraceright}\ {\isasymunion}\ S\ {\isasymin}\ C{\isachardoublequoteclose}\isanewline
\ \ \ \ \ \ \ \ \isacommand{using}\isamarkupfalse%
\ H\ \isacommand{by}\isamarkupfalse%
\ {\isacharparenleft}iprover\ elim{\isacharcolon}\ conjunct{\isadigit{1}}\ conjunct{\isadigit{2}}{\isacharparenright}\isanewline
\ \ \ \ \ \ \isacommand{have}\isamarkupfalse%
\ S{\isadigit{3}}{\isacharcolon}{\isachardoublequoteopen}{\isasymforall}F\ G\ H{\isachardot}\ Con\ F\ G\ H\ {\isasymlongrightarrow}\ F\ {\isasymin}\ S{\isacharprime}\ {\isasymlongrightarrow}\ {\isacharbraceleft}G{\isacharcomma}H{\isacharbraceright}\ {\isasymunion}\ S{\isacharprime}\ {\isasymin}\ {\isacharparenleft}extensionSC\ C{\isacharparenright}{\isachardoublequoteclose}\isanewline
\ \ \ \ \ \ \isacommand{proof}\isamarkupfalse%
\ {\isacharparenleft}rule\ allI{\isacharparenright}{\isacharplus}\isanewline
\ \ \ \ \ \ \ \ \isacommand{fix}\isamarkupfalse%
\ F\ G\ H\isanewline
\ \ \ \ \ \ \ \ \isacommand{show}\isamarkupfalse%
\ {\isachardoublequoteopen}Con\ F\ G\ H\ {\isasymlongrightarrow}\ F\ {\isasymin}\ S{\isacharprime}\ {\isasymlongrightarrow}\ {\isacharbraceleft}G{\isacharcomma}H{\isacharbraceright}\ {\isasymunion}\ S{\isacharprime}\ {\isasymin}\ {\isacharparenleft}extensionSC\ C{\isacharparenright}{\isachardoublequoteclose}\isanewline
\ \ \ \ \ \ \ \ \isacommand{proof}\isamarkupfalse%
\ {\isacharparenleft}rule\ impI{\isacharparenright}{\isacharplus}\isanewline
\ \ \ \ \ \ \ \ \ \ \isacommand{assume}\isamarkupfalse%
\ {\isachardoublequoteopen}Con\ F\ G\ H{\isachardoublequoteclose}\isanewline
\ \ \ \ \ \ \ \ \ \ \isacommand{assume}\isamarkupfalse%
\ {\isachardoublequoteopen}F\ {\isasymin}\ S{\isacharprime}{\isachardoublequoteclose}\isanewline
\ \ \ \ \ \ \ \ \ \ \isacommand{have}\isamarkupfalse%
\ {\isachardoublequoteopen}F\ {\isasymin}\ S{\isachardoublequoteclose}\isanewline
\ \ \ \ \ \ \ \ \ \ \ \ \isacommand{using}\isamarkupfalse%
\ {\isacartoucheopen}S{\isacharprime}\ {\isasymsubseteq}\ S{\isacartoucheclose}\ {\isacartoucheopen}F\ {\isasymin}\ S{\isacharprime}{\isacartoucheclose}\ \isacommand{by}\isamarkupfalse%
\ {\isacharparenleft}rule\ set{\isacharunderscore}mp{\isacharparenright}\isanewline
\ \ \ \ \ \ \ \ \ \ \isacommand{have}\isamarkupfalse%
\ {\isachardoublequoteopen}Con\ F\ G\ H\ {\isasymlongrightarrow}\ F\ {\isasymin}\ S\ {\isasymlongrightarrow}\ {\isacharbraceleft}G{\isacharcomma}H{\isacharbraceright}\ {\isasymunion}\ S\ {\isasymin}\ C{\isachardoublequoteclose}\isanewline
\ \ \ \ \ \ \ \ \ \ \ \ \isacommand{using}\isamarkupfalse%
\ Con\ \isacommand{by}\isamarkupfalse%
\ {\isacharparenleft}iprover\ elim{\isacharcolon}\ allE{\isacharparenright}\isanewline
\ \ \ \ \ \ \ \ \ \ \isacommand{then}\isamarkupfalse%
\ \isacommand{have}\isamarkupfalse%
\ {\isachardoublequoteopen}F\ {\isasymin}\ S\ {\isasymlongrightarrow}\ {\isacharbraceleft}G{\isacharcomma}H{\isacharbraceright}\ {\isasymunion}\ S\ {\isasymin}\ C{\isachardoublequoteclose}\isanewline
\ \ \ \ \ \ \ \ \ \ \ \ \isacommand{using}\isamarkupfalse%
\ {\isacartoucheopen}Con\ F\ G\ H{\isacartoucheclose}\ \isacommand{by}\isamarkupfalse%
\ {\isacharparenleft}rule\ mp{\isacharparenright}\isanewline
\ \ \ \ \ \ \ \ \ \ \isacommand{then}\isamarkupfalse%
\ \isacommand{have}\isamarkupfalse%
\ {\isachardoublequoteopen}{\isacharbraceleft}G{\isacharcomma}H{\isacharbraceright}\ {\isasymunion}\ S\ {\isasymin}\ C{\isachardoublequoteclose}\isanewline
\ \ \ \ \ \ \ \ \ \ \ \ \isacommand{using}\isamarkupfalse%
\ {\isacartoucheopen}F\ {\isasymin}\ S{\isacartoucheclose}\ \isacommand{by}\isamarkupfalse%
\ {\isacharparenleft}rule\ mp{\isacharparenright}\isanewline
\ \ \ \ \ \ \ \ \ \ \isacommand{have}\isamarkupfalse%
\ {\isachardoublequoteopen}S{\isacharprime}\ {\isasymsubseteq}\ insert\ H\ S{\isachardoublequoteclose}\isanewline
\ \ \ \ \ \ \ \ \ \ \ \ \isacommand{using}\isamarkupfalse%
\ {\isacartoucheopen}S{\isacharprime}\ {\isasymsubseteq}\ S{\isacartoucheclose}\ \isacommand{by}\isamarkupfalse%
\ {\isacharparenleft}rule\ subset{\isacharunderscore}insertI{\isadigit{2}}{\isacharparenright}\ \isanewline
\ \ \ \ \ \ \ \ \ \ \isacommand{then}\isamarkupfalse%
\ \isacommand{have}\isamarkupfalse%
\ {\isachardoublequoteopen}insert\ H\ S{\isacharprime}\ {\isasymsubseteq}\ insert\ H\ {\isacharparenleft}insert\ H\ S{\isacharparenright}{\isachardoublequoteclose}\isanewline
\ \ \ \ \ \ \ \ \ \ \ \ \isacommand{by}\isamarkupfalse%
\ {\isacharparenleft}simp\ only{\isacharcolon}\ insert{\isacharunderscore}mono{\isacharparenright}\isanewline
\ \ \ \ \ \ \ \ \ \ \isacommand{then}\isamarkupfalse%
\ \isacommand{have}\isamarkupfalse%
\ {\isachardoublequoteopen}insert\ H\ S{\isacharprime}\ {\isasymsubseteq}\ insert\ H\ S{\isachardoublequoteclose}\isanewline
\ \ \ \ \ \ \ \ \ \ \ \ \isacommand{by}\isamarkupfalse%
\ {\isacharparenleft}simp\ only{\isacharcolon}\ insert{\isacharunderscore}absorb{\isadigit{2}}{\isacharparenright}\isanewline
\ \ \ \ \ \ \ \ \ \ \isacommand{then}\isamarkupfalse%
\ \isacommand{have}\isamarkupfalse%
\ {\isachardoublequoteopen}insert\ G\ {\isacharparenleft}insert\ H\ S{\isacharprime}{\isacharparenright}\ {\isasymsubseteq}\ insert\ G\ {\isacharparenleft}insert\ H\ S{\isacharparenright}{\isachardoublequoteclose}\isanewline
\ \ \ \ \ \ \ \ \ \ \ \ \isacommand{by}\isamarkupfalse%
\ {\isacharparenleft}simp\ only{\isacharcolon}\ insert{\isacharunderscore}mono{\isacharparenright}\isanewline
\ \ \ \ \ \ \ \ \ \ \isacommand{have}\isamarkupfalse%
\ A{\isacharcolon}{\isachardoublequoteopen}insert\ G\ {\isacharparenleft}insert\ H\ S{\isacharprime}{\isacharparenright}\ {\isacharequal}\ {\isacharbraceleft}G{\isacharcomma}H{\isacharbraceright}\ {\isasymunion}\ S{\isacharprime}{\isachardoublequoteclose}\isanewline
\ \ \ \ \ \ \ \ \ \ \ \ \isacommand{by}\isamarkupfalse%
\ {\isacharparenleft}rule\ insertSetElem{\isacharparenright}\ \isanewline
\ \ \ \ \ \ \ \ \ \ \isacommand{have}\isamarkupfalse%
\ B{\isacharcolon}{\isachardoublequoteopen}insert\ G\ {\isacharparenleft}insert\ H\ S{\isacharparenright}\ {\isacharequal}\ {\isacharbraceleft}G{\isacharcomma}H{\isacharbraceright}\ {\isasymunion}\ S{\isachardoublequoteclose}\isanewline
\ \ \ \ \ \ \ \ \ \ \ \ \isacommand{by}\isamarkupfalse%
\ {\isacharparenleft}rule\ insertSetElem{\isacharparenright}\isanewline
\ \ \ \ \ \ \ \ \ \ \isacommand{have}\isamarkupfalse%
\ {\isachardoublequoteopen}{\isacharbraceleft}G{\isacharcomma}H{\isacharbraceright}\ {\isasymunion}\ S{\isacharprime}\ {\isasymsubseteq}\ {\isacharbraceleft}G{\isacharcomma}H{\isacharbraceright}\ {\isasymunion}\ S{\isachardoublequoteclose}\ \isanewline
\ \ \ \ \ \ \ \ \ \ \ \ \isacommand{using}\isamarkupfalse%
\ {\isacartoucheopen}insert\ G\ {\isacharparenleft}insert\ H\ S{\isacharprime}{\isacharparenright}\ {\isasymsubseteq}\ insert\ G\ {\isacharparenleft}insert\ H\ S{\isacharparenright}{\isacartoucheclose}\ \isacommand{by}\isamarkupfalse%
\ {\isacharparenleft}simp\ only{\isacharcolon}\ A\ B{\isacharparenright}\isanewline
\ \ \ \ \ \ \ \ \ \ \isacommand{then}\isamarkupfalse%
\ \isacommand{have}\isamarkupfalse%
\ {\isachardoublequoteopen}{\isasymexists}S\ {\isasymin}\ C{\isachardot}\ {\isacharbraceleft}G{\isacharcomma}H{\isacharbraceright}\ {\isasymunion}\ S{\isacharprime}\ {\isasymsubseteq}\ S{\isachardoublequoteclose}\isanewline
\ \ \ \ \ \ \ \ \ \ \ \ \isacommand{using}\isamarkupfalse%
\ {\isacartoucheopen}{\isacharbraceleft}G{\isacharcomma}H{\isacharbraceright}\ {\isasymunion}\ S\ {\isasymin}\ C{\isacartoucheclose}\ \isacommand{by}\isamarkupfalse%
\ {\isacharparenleft}rule\ bexI{\isacharparenright}\isanewline
\ \ \ \ \ \ \ \ \ \ \isacommand{thus}\isamarkupfalse%
\ {\isachardoublequoteopen}{\isacharbraceleft}G{\isacharcomma}H{\isacharbraceright}\ {\isasymunion}\ S{\isacharprime}\ {\isasymin}\ {\isacharparenleft}extensionSC\ C{\isacharparenright}{\isachardoublequoteclose}\ \isanewline
\ \ \ \ \ \ \ \ \ \ \ \ \isacommand{unfolding}\isamarkupfalse%
\ extensionSC\ \isacommand{by}\isamarkupfalse%
\ {\isacharparenleft}rule\ CollectI{\isacharparenright}\isanewline
\ \ \ \ \ \ \ \ \isacommand{qed}\isamarkupfalse%
\isanewline
\ \ \ \ \ \ \isacommand{qed}\isamarkupfalse%
\isanewline
\ \ \ \ \ \ \isacommand{have}\isamarkupfalse%
\ Dis{\isacharcolon}{\isachardoublequoteopen}{\isasymforall}F\ G\ H{\isachardot}\ Dis\ F\ G\ H\ {\isasymlongrightarrow}\ F\ {\isasymin}\ S\ {\isasymlongrightarrow}\ {\isacharbraceleft}G{\isacharbraceright}\ {\isasymunion}\ S\ {\isasymin}\ C\ {\isasymor}\ {\isacharbraceleft}H{\isacharbraceright}\ {\isasymunion}\ S\ {\isasymin}\ C{\isachardoublequoteclose}\isanewline
\ \ \ \ \ \ \ \ \isacommand{using}\isamarkupfalse%
\ H\ \isacommand{by}\isamarkupfalse%
\ {\isacharparenleft}iprover\ elim{\isacharcolon}\ conjunct{\isadigit{2}}{\isacharparenright}\isanewline
\ \ \ \ \ \ \isacommand{have}\isamarkupfalse%
\ S{\isadigit{4}}{\isacharcolon}{\isachardoublequoteopen}{\isasymforall}F\ G\ H{\isachardot}\ Dis\ F\ G\ H\ {\isasymlongrightarrow}\ F\ {\isasymin}\ S{\isacharprime}\ {\isasymlongrightarrow}\ {\isacharbraceleft}G{\isacharbraceright}\ {\isasymunion}\ S{\isacharprime}\ {\isasymin}\ {\isacharparenleft}extensionSC\ C{\isacharparenright}\ {\isasymor}\ {\isacharbraceleft}H{\isacharbraceright}\ {\isasymunion}\ S{\isacharprime}\ {\isasymin}\ {\isacharparenleft}extensionSC\ C{\isacharparenright}{\isachardoublequoteclose}\isanewline
\ \ \ \ \ \ \isacommand{proof}\isamarkupfalse%
\ {\isacharparenleft}rule\ allI{\isacharparenright}{\isacharplus}\isanewline
\ \ \ \ \ \ \ \ \isacommand{fix}\isamarkupfalse%
\ F\ G\ H\isanewline
\ \ \ \ \ \ \ \ \isacommand{show}\isamarkupfalse%
\ {\isachardoublequoteopen}Dis\ F\ G\ H\ {\isasymlongrightarrow}\ F\ {\isasymin}\ S{\isacharprime}\ {\isasymlongrightarrow}\ {\isacharbraceleft}G{\isacharbraceright}\ {\isasymunion}\ S{\isacharprime}\ {\isasymin}\ {\isacharparenleft}extensionSC\ C{\isacharparenright}\ {\isasymor}\ {\isacharbraceleft}H{\isacharbraceright}\ {\isasymunion}\ S{\isacharprime}\ {\isasymin}\ {\isacharparenleft}extensionSC\ C{\isacharparenright}{\isachardoublequoteclose}\isanewline
\ \ \ \ \ \ \ \ \isacommand{proof}\isamarkupfalse%
\ {\isacharparenleft}rule\ impI{\isacharparenright}{\isacharplus}\isanewline
\ \ \ \ \ \ \ \ \ \ \isacommand{assume}\isamarkupfalse%
\ {\isachardoublequoteopen}Dis\ F\ G\ H{\isachardoublequoteclose}\isanewline
\ \ \ \ \ \ \ \ \ \ \isacommand{assume}\isamarkupfalse%
\ {\isachardoublequoteopen}F\ {\isasymin}\ S{\isacharprime}{\isachardoublequoteclose}\isanewline
\ \ \ \ \ \ \ \ \ \ \isacommand{have}\isamarkupfalse%
\ {\isachardoublequoteopen}F\ {\isasymin}\ S{\isachardoublequoteclose}\isanewline
\ \ \ \ \ \ \ \ \ \ \ \ \isacommand{using}\isamarkupfalse%
\ {\isacartoucheopen}S{\isacharprime}\ {\isasymsubseteq}\ S{\isacartoucheclose}\ {\isacartoucheopen}F\ {\isasymin}\ S{\isacharprime}{\isacartoucheclose}\ \isacommand{by}\isamarkupfalse%
\ {\isacharparenleft}rule\ set{\isacharunderscore}mp{\isacharparenright}\isanewline
\ \ \ \ \ \ \ \ \ \ \isacommand{have}\isamarkupfalse%
\ {\isachardoublequoteopen}Dis\ F\ G\ H\ {\isasymlongrightarrow}\ F\ {\isasymin}\ S\ {\isasymlongrightarrow}\ {\isacharbraceleft}G{\isacharbraceright}\ {\isasymunion}\ S\ {\isasymin}\ C\ {\isasymor}\ {\isacharbraceleft}H{\isacharbraceright}\ {\isasymunion}\ S\ {\isasymin}\ C{\isachardoublequoteclose}\isanewline
\ \ \ \ \ \ \ \ \ \ \ \ \isacommand{using}\isamarkupfalse%
\ Dis\ \isacommand{by}\isamarkupfalse%
\ {\isacharparenleft}iprover\ elim{\isacharcolon}\ allE{\isacharparenright}\isanewline
\ \ \ \ \ \ \ \ \ \ \isacommand{then}\isamarkupfalse%
\ \isacommand{have}\isamarkupfalse%
\ {\isachardoublequoteopen}F\ {\isasymin}\ S\ {\isasymlongrightarrow}\ {\isacharbraceleft}G{\isacharbraceright}\ {\isasymunion}\ S\ {\isasymin}\ C\ {\isasymor}\ {\isacharbraceleft}H{\isacharbraceright}\ {\isasymunion}\ S\ {\isasymin}\ C{\isachardoublequoteclose}\isanewline
\ \ \ \ \ \ \ \ \ \ \ \ \isacommand{using}\isamarkupfalse%
\ {\isacartoucheopen}Dis\ F\ G\ H{\isacartoucheclose}\ \isacommand{by}\isamarkupfalse%
\ {\isacharparenleft}rule\ mp{\isacharparenright}\isanewline
\ \ \ \ \ \ \ \ \ \ \isacommand{then}\isamarkupfalse%
\ \isacommand{have}\isamarkupfalse%
\ {\isadigit{9}}{\isacharcolon}{\isachardoublequoteopen}{\isacharbraceleft}G{\isacharbraceright}\ {\isasymunion}\ S\ {\isasymin}\ C\ {\isasymor}\ {\isacharbraceleft}H{\isacharbraceright}\ {\isasymunion}\ S\ {\isasymin}\ C{\isachardoublequoteclose}\isanewline
\ \ \ \ \ \ \ \ \ \ \ \ \isacommand{using}\isamarkupfalse%
\ {\isacartoucheopen}F\ {\isasymin}\ S{\isacartoucheclose}\ \isacommand{by}\isamarkupfalse%
\ {\isacharparenleft}rule\ mp{\isacharparenright}\isanewline
\ \ \ \ \ \ \ \ \ \ \isacommand{show}\isamarkupfalse%
\ {\isachardoublequoteopen}{\isacharbraceleft}G{\isacharbraceright}\ {\isasymunion}\ S{\isacharprime}\ {\isasymin}\ {\isacharparenleft}extensionSC\ C{\isacharparenright}\ {\isasymor}\ {\isacharbraceleft}H{\isacharbraceright}\ {\isasymunion}\ S{\isacharprime}\ {\isasymin}\ {\isacharparenleft}extensionSC\ C{\isacharparenright}{\isachardoublequoteclose}\isanewline
\ \ \ \ \ \ \ \ \ \ \ \ \isacommand{using}\isamarkupfalse%
\ {\isadigit{9}}\isanewline
\ \ \ \ \ \ \ \ \ \ \isacommand{proof}\isamarkupfalse%
\ {\isacharparenleft}rule\ disjE{\isacharparenright}\isanewline
\ \ \ \ \ \ \ \ \ \ \ \ \isacommand{assume}\isamarkupfalse%
\ {\isachardoublequoteopen}{\isacharbraceleft}G{\isacharbraceright}\ {\isasymunion}\ S\ {\isasymin}\ C{\isachardoublequoteclose}\isanewline
\ \ \ \ \ \ \ \ \ \ \ \ \isacommand{have}\isamarkupfalse%
\ {\isachardoublequoteopen}insert\ G\ S{\isacharprime}\ {\isasymsubseteq}\ insert\ G\ S{\isachardoublequoteclose}\isanewline
\ \ \ \ \ \ \ \ \ \ \ \ \ \ \isacommand{using}\isamarkupfalse%
\ {\isacartoucheopen}S{\isacharprime}\ {\isasymsubseteq}\ S{\isacartoucheclose}\ \isacommand{by}\isamarkupfalse%
\ {\isacharparenleft}simp\ only{\isacharcolon}\ insert{\isacharunderscore}mono{\isacharparenright}\isanewline
\ \ \ \ \ \ \ \ \ \ \ \ \isacommand{have}\isamarkupfalse%
\ C{\isacharcolon}{\isachardoublequoteopen}insert\ G\ S{\isacharprime}\ {\isacharequal}\ {\isacharbraceleft}G{\isacharbraceright}\ {\isasymunion}\ S{\isacharprime}{\isachardoublequoteclose}\isanewline
\ \ \ \ \ \ \ \ \ \ \ \ \ \ \isacommand{by}\isamarkupfalse%
\ {\isacharparenleft}rule\ insert{\isacharunderscore}is{\isacharunderscore}Un{\isacharparenright}\isanewline
\ \ \ \ \ \ \ \ \ \ \ \ \isacommand{have}\isamarkupfalse%
\ D{\isacharcolon}{\isachardoublequoteopen}insert\ G\ S\ {\isacharequal}\ {\isacharbraceleft}G{\isacharbraceright}\ {\isasymunion}\ S{\isachardoublequoteclose}\isanewline
\ \ \ \ \ \ \ \ \ \ \ \ \ \ \isacommand{by}\isamarkupfalse%
\ {\isacharparenleft}rule\ insert{\isacharunderscore}is{\isacharunderscore}Un{\isacharparenright}\isanewline
\ \ \ \ \ \ \ \ \ \ \ \ \isacommand{have}\isamarkupfalse%
\ {\isachardoublequoteopen}{\isacharbraceleft}G{\isacharbraceright}\ {\isasymunion}\ S{\isacharprime}\ {\isasymsubseteq}\ {\isacharbraceleft}G{\isacharbraceright}\ {\isasymunion}\ S{\isachardoublequoteclose}\isanewline
\ \ \ \ \ \ \ \ \ \ \ \ \ \ \isacommand{using}\isamarkupfalse%
\ {\isacartoucheopen}insert\ G\ S{\isacharprime}\ {\isasymsubseteq}\ insert\ G\ S{\isacartoucheclose}\ \isacommand{by}\isamarkupfalse%
\ {\isacharparenleft}simp\ only{\isacharcolon}\ C\ D{\isacharparenright}\isanewline
\ \ \ \ \ \ \ \ \ \ \ \ \isacommand{then}\isamarkupfalse%
\ \isacommand{have}\isamarkupfalse%
\ {\isachardoublequoteopen}{\isasymexists}S\ {\isasymin}\ C{\isachardot}\ {\isacharbraceleft}G{\isacharbraceright}\ {\isasymunion}\ S{\isacharprime}\ {\isasymsubseteq}\ S{\isachardoublequoteclose}\isanewline
\ \ \ \ \ \ \ \ \ \ \ \ \ \ \isacommand{using}\isamarkupfalse%
\ {\isacartoucheopen}{\isacharbraceleft}G{\isacharbraceright}\ {\isasymunion}\ S\ {\isasymin}\ C{\isacartoucheclose}\ \isacommand{by}\isamarkupfalse%
\ {\isacharparenleft}rule\ bexI{\isacharparenright}\isanewline
\ \ \ \ \ \ \ \ \ \ \ \ \isacommand{then}\isamarkupfalse%
\ \isacommand{have}\isamarkupfalse%
\ {\isachardoublequoteopen}{\isacharbraceleft}G{\isacharbraceright}\ {\isasymunion}\ S{\isacharprime}\ {\isasymin}\ {\isacharparenleft}extensionSC\ C{\isacharparenright}{\isachardoublequoteclose}\isanewline
\ \ \ \ \ \ \ \ \ \ \ \ \ \ \isacommand{unfolding}\isamarkupfalse%
\ extensionSC\ \isacommand{by}\isamarkupfalse%
\ {\isacharparenleft}rule\ CollectI{\isacharparenright}\isanewline
\ \ \ \ \ \ \ \ \ \ \ \ \isacommand{thus}\isamarkupfalse%
\ {\isachardoublequoteopen}{\isacharbraceleft}G{\isacharbraceright}\ {\isasymunion}\ S{\isacharprime}\ {\isasymin}\ {\isacharparenleft}extensionSC\ C{\isacharparenright}\ {\isasymor}\ {\isacharbraceleft}H{\isacharbraceright}\ {\isasymunion}\ S{\isacharprime}\ {\isasymin}\ {\isacharparenleft}extensionSC\ C{\isacharparenright}{\isachardoublequoteclose}\isanewline
\ \ \ \ \ \ \ \ \ \ \ \ \ \ \isacommand{by}\isamarkupfalse%
\ {\isacharparenleft}rule\ disjI{\isadigit{1}}{\isacharparenright}\isanewline
\ \ \ \ \ \ \ \ \ \ \isacommand{next}\isamarkupfalse%
\isanewline
\ \ \ \ \ \ \ \ \ \ \ \ \isacommand{assume}\isamarkupfalse%
\ {\isachardoublequoteopen}{\isacharbraceleft}H{\isacharbraceright}\ {\isasymunion}\ S\ {\isasymin}\ C{\isachardoublequoteclose}\isanewline
\ \ \ \ \ \ \ \ \ \ \ \ \isacommand{have}\isamarkupfalse%
\ {\isachardoublequoteopen}insert\ H\ S{\isacharprime}\ {\isasymsubseteq}\ insert\ H\ S{\isachardoublequoteclose}\isanewline
\ \ \ \ \ \ \ \ \ \ \ \ \ \ \isacommand{using}\isamarkupfalse%
\ {\isacartoucheopen}S{\isacharprime}\ {\isasymsubseteq}\ S{\isacartoucheclose}\ \isacommand{by}\isamarkupfalse%
\ {\isacharparenleft}simp\ only{\isacharcolon}\ insert{\isacharunderscore}mono{\isacharparenright}\isanewline
\ \ \ \ \ \ \ \ \ \ \ \ \isacommand{have}\isamarkupfalse%
\ E{\isacharcolon}{\isachardoublequoteopen}insert\ H\ S{\isacharprime}\ {\isacharequal}\ {\isacharbraceleft}H{\isacharbraceright}\ {\isasymunion}\ S{\isacharprime}{\isachardoublequoteclose}\isanewline
\ \ \ \ \ \ \ \ \ \ \ \ \ \ \isacommand{by}\isamarkupfalse%
\ {\isacharparenleft}rule\ insert{\isacharunderscore}is{\isacharunderscore}Un{\isacharparenright}\isanewline
\ \ \ \ \ \ \ \ \ \ \ \ \isacommand{have}\isamarkupfalse%
\ F{\isacharcolon}{\isachardoublequoteopen}insert\ H\ S\ {\isacharequal}\ {\isacharbraceleft}H{\isacharbraceright}\ {\isasymunion}\ S{\isachardoublequoteclose}\isanewline
\ \ \ \ \ \ \ \ \ \ \ \ \ \ \isacommand{by}\isamarkupfalse%
\ {\isacharparenleft}rule\ insert{\isacharunderscore}is{\isacharunderscore}Un{\isacharparenright}\isanewline
\ \ \ \ \ \ \ \ \ \ \ \ \isacommand{then}\isamarkupfalse%
\ \isacommand{have}\isamarkupfalse%
\ {\isachardoublequoteopen}{\isacharbraceleft}H{\isacharbraceright}\ {\isasymunion}\ S{\isacharprime}\ {\isasymsubseteq}\ {\isacharbraceleft}H{\isacharbraceright}\ {\isasymunion}\ S{\isachardoublequoteclose}\isanewline
\ \ \ \ \ \ \ \ \ \ \ \ \ \ \isacommand{using}\isamarkupfalse%
\ {\isacartoucheopen}insert\ H\ S{\isacharprime}\ {\isasymsubseteq}\ insert\ H\ S{\isacartoucheclose}\ \isacommand{by}\isamarkupfalse%
\ {\isacharparenleft}simp\ only{\isacharcolon}\ E\ F{\isacharparenright}\isanewline
\ \ \ \ \ \ \ \ \ \ \ \ \isacommand{then}\isamarkupfalse%
\ \isacommand{have}\isamarkupfalse%
\ {\isachardoublequoteopen}{\isasymexists}S\ {\isasymin}\ C{\isachardot}\ {\isacharbraceleft}H{\isacharbraceright}\ {\isasymunion}\ S{\isacharprime}\ {\isasymsubseteq}\ S{\isachardoublequoteclose}\isanewline
\ \ \ \ \ \ \ \ \ \ \ \ \ \ \isacommand{using}\isamarkupfalse%
\ {\isacartoucheopen}{\isacharbraceleft}H{\isacharbraceright}\ {\isasymunion}\ S\ {\isasymin}\ C{\isacartoucheclose}\ \isacommand{by}\isamarkupfalse%
\ {\isacharparenleft}rule\ bexI{\isacharparenright}\isanewline
\ \ \ \ \ \ \ \ \ \ \ \ \isacommand{then}\isamarkupfalse%
\ \isacommand{have}\isamarkupfalse%
\ {\isachardoublequoteopen}{\isacharbraceleft}H{\isacharbraceright}\ {\isasymunion}\ S{\isacharprime}\ {\isasymin}\ {\isacharparenleft}extensionSC\ C{\isacharparenright}{\isachardoublequoteclose}\isanewline
\ \ \ \ \ \ \ \ \ \ \ \ \ \ \isacommand{unfolding}\isamarkupfalse%
\ extensionSC\ \isacommand{by}\isamarkupfalse%
\ {\isacharparenleft}rule\ CollectI{\isacharparenright}\isanewline
\ \ \ \ \ \ \ \ \ \ \ \ \isacommand{thus}\isamarkupfalse%
\ {\isachardoublequoteopen}{\isacharbraceleft}G{\isacharbraceright}\ {\isasymunion}\ S{\isacharprime}\ {\isasymin}\ {\isacharparenleft}extensionSC\ C{\isacharparenright}\ {\isasymor}\ {\isacharbraceleft}H{\isacharbraceright}\ {\isasymunion}\ S{\isacharprime}\ {\isasymin}\ {\isacharparenleft}extensionSC\ C{\isacharparenright}{\isachardoublequoteclose}\isanewline
\ \ \ \ \ \ \ \ \ \ \ \ \ \ \isacommand{by}\isamarkupfalse%
\ {\isacharparenleft}rule\ disjI{\isadigit{2}}{\isacharparenright}\isanewline
\ \ \ \ \ \ \ \ \ \ \isacommand{qed}\isamarkupfalse%
\isanewline
\ \ \ \ \ \ \ \ \isacommand{qed}\isamarkupfalse%
\isanewline
\ \ \ \ \ \ \isacommand{qed}\isamarkupfalse%
\isanewline
\ \ \ \ \ \ \isacommand{show}\isamarkupfalse%
\ {\isachardoublequoteopen}{\isasymbottom}\ {\isasymnotin}\ S{\isacharprime}\isanewline
\ \ \ \ {\isasymand}\ {\isacharparenleft}{\isasymforall}k{\isachardot}\ Atom\ k\ {\isasymin}\ S{\isacharprime}\ {\isasymlongrightarrow}\ \isactrlbold {\isasymnot}\ {\isacharparenleft}Atom\ k{\isacharparenright}\ {\isasymin}\ S{\isacharprime}\ {\isasymlongrightarrow}\ False{\isacharparenright}\isanewline
\ \ \ \ {\isasymand}\ {\isacharparenleft}{\isasymforall}F\ G\ H{\isachardot}\ Con\ F\ G\ H\ {\isasymlongrightarrow}\ F\ {\isasymin}\ S{\isacharprime}\ {\isasymlongrightarrow}\ {\isacharbraceleft}G{\isacharcomma}H{\isacharbraceright}\ {\isasymunion}\ S{\isacharprime}\ {\isasymin}\ {\isacharparenleft}extensionSC\ C{\isacharparenright}{\isacharparenright}\isanewline
\ \ \ \ {\isasymand}\ {\isacharparenleft}{\isasymforall}F\ G\ H{\isachardot}\ Dis\ F\ G\ H\ {\isasymlongrightarrow}\ F\ {\isasymin}\ S{\isacharprime}\ {\isasymlongrightarrow}\ {\isacharbraceleft}G{\isacharbraceright}\ {\isasymunion}\ S{\isacharprime}\ {\isasymin}\ {\isacharparenleft}extensionSC\ C{\isacharparenright}\ {\isasymor}\ {\isacharbraceleft}H{\isacharbraceright}\ {\isasymunion}\ S{\isacharprime}\ {\isasymin}\ {\isacharparenleft}extensionSC\ C{\isacharparenright}{\isacharparenright}{\isachardoublequoteclose}\isanewline
\ \ \ \ \ \ \ \ \isacommand{using}\isamarkupfalse%
\ S{\isadigit{1}}\ S{\isadigit{2}}\ S{\isadigit{3}}\ S{\isadigit{4}}\ \isacommand{by}\isamarkupfalse%
\ {\isacharparenleft}iprover\ intro{\isacharcolon}\ conjI{\isacharparenright}\isanewline
\ \ \ \ \isacommand{qed}\isamarkupfalse%
\isanewline
\ \ \isacommand{qed}\isamarkupfalse%
\isanewline
\isacommand{qed}\isamarkupfalse%
%
\endisatagproof
{\isafoldproof}%
%
\isadelimproof
%
\endisadelimproof
%
\begin{isamarkuptext}%
Finalmente, el siguiente lema auxiliar prueba que \isa{C{\isacharprime}} es cerrada bajo subconjuntos.%
\end{isamarkuptext}\isamarkuptrue%
\isacommand{lemma}\isamarkupfalse%
\ ex{\isadigit{1}}{\isacharunderscore}subset{\isacharunderscore}closed{\isacharcolon}\isanewline
\ \ \isakeyword{assumes}\ {\isachardoublequoteopen}pcp\ C{\isachardoublequoteclose}\isanewline
\ \ \isakeyword{shows}\ {\isachardoublequoteopen}subset{\isacharunderscore}closed\ {\isacharparenleft}extensionSC\ C{\isacharparenright}{\isachardoublequoteclose}\isanewline
%
\isadelimproof
\ \ %
\endisadelimproof
%
\isatagproof
\isacommand{unfolding}\isamarkupfalse%
\ subset{\isacharunderscore}closed{\isacharunderscore}def\isanewline
\isacommand{proof}\isamarkupfalse%
\ {\isacharparenleft}rule\ ballI{\isacharparenright}\isanewline
\ \ \isacommand{fix}\isamarkupfalse%
\ S{\isacharprime}\isanewline
\ \ \isacommand{assume}\isamarkupfalse%
\ {\isachardoublequoteopen}S{\isacharprime}\ {\isasymin}\ {\isacharparenleft}extensionSC\ C{\isacharparenright}{\isachardoublequoteclose}\isanewline
\ \ \isacommand{then}\isamarkupfalse%
\ \isacommand{have}\isamarkupfalse%
\ H{\isacharcolon}{\isachardoublequoteopen}{\isasymexists}S\ {\isasymin}\ C{\isachardot}\ S{\isacharprime}\ {\isasymsubseteq}\ S{\isachardoublequoteclose}\isanewline
\ \ \ \ \isacommand{unfolding}\isamarkupfalse%
\ extensionSC\ \isacommand{by}\isamarkupfalse%
\ {\isacharparenleft}rule\ CollectD{\isacharparenright}\isanewline
\ \ \isacommand{obtain}\isamarkupfalse%
\ S\ \isakeyword{where}\ {\isacartoucheopen}S\ {\isasymin}\ C{\isacartoucheclose}\ \isakeyword{and}\ {\isacartoucheopen}S{\isacharprime}\ {\isasymsubseteq}\ S{\isacartoucheclose}\ \isanewline
\ \ \ \ \isacommand{using}\isamarkupfalse%
\ H\ \isacommand{by}\isamarkupfalse%
\ {\isacharparenleft}rule\ bexE{\isacharparenright}\ \isanewline
\ \ \isacommand{show}\isamarkupfalse%
\ {\isachardoublequoteopen}{\isasymforall}S{\isacharprime}{\isacharprime}\ {\isasymsubseteq}\ S{\isacharprime}{\isachardot}\ S{\isacharprime}{\isacharprime}\ {\isasymin}\ {\isacharparenleft}extensionSC\ C{\isacharparenright}{\isachardoublequoteclose}\isanewline
\ \ \isacommand{proof}\isamarkupfalse%
\ {\isacharparenleft}rule\ sallI{\isacharparenright}\isanewline
\ \ \ \ \isacommand{fix}\isamarkupfalse%
\ S{\isacharprime}{\isacharprime}\isanewline
\ \ \ \ \isacommand{assume}\isamarkupfalse%
\ {\isachardoublequoteopen}S{\isacharprime}{\isacharprime}\ {\isasymsubseteq}\ S{\isacharprime}{\isachardoublequoteclose}\ \isanewline
\ \ \ \ \isacommand{then}\isamarkupfalse%
\ \isacommand{have}\isamarkupfalse%
\ {\isachardoublequoteopen}S{\isacharprime}{\isacharprime}\ {\isasymsubseteq}\ S{\isachardoublequoteclose}\isanewline
\ \ \ \ \ \ \isacommand{using}\isamarkupfalse%
\ {\isacartoucheopen}S{\isacharprime}\ {\isasymsubseteq}\ S{\isacartoucheclose}\ \isacommand{by}\isamarkupfalse%
\ {\isacharparenleft}rule\ subset{\isacharunderscore}trans{\isacharparenright}\isanewline
\ \ \ \ \isacommand{then}\isamarkupfalse%
\ \isacommand{have}\isamarkupfalse%
\ {\isachardoublequoteopen}{\isasymexists}S\ {\isasymin}\ C{\isachardot}\ S{\isacharprime}{\isacharprime}\ {\isasymsubseteq}\ S{\isachardoublequoteclose}\isanewline
\ \ \ \ \ \ \isacommand{using}\isamarkupfalse%
\ {\isacartoucheopen}S\ {\isasymin}\ C{\isacartoucheclose}\ \isacommand{by}\isamarkupfalse%
\ {\isacharparenleft}rule\ bexI{\isacharparenright}\isanewline
\ \ \ \ \isacommand{thus}\isamarkupfalse%
\ {\isachardoublequoteopen}S{\isacharprime}{\isacharprime}\ {\isasymin}\ {\isacharparenleft}extensionSC\ C{\isacharparenright}{\isachardoublequoteclose}\isanewline
\ \ \ \ \ \ \isacommand{unfolding}\isamarkupfalse%
\ extensionSC\ \isacommand{by}\isamarkupfalse%
\ {\isacharparenleft}rule\ CollectI{\isacharparenright}\isanewline
\ \ \isacommand{qed}\isamarkupfalse%
\isanewline
\isacommand{qed}\isamarkupfalse%
%
\endisatagproof
{\isafoldproof}%
%
\isadelimproof
%
\endisadelimproof
%
\begin{isamarkuptext}%
En conclusión, la prueba detallada del lema completo se muestra a continuación.%
\end{isamarkuptext}\isamarkuptrue%
\isacommand{lemma}\isamarkupfalse%
\ ex{\isadigit{1}}{\isacharcolon}\ \isanewline
\ \ \isakeyword{assumes}\ {\isachardoublequoteopen}pcp\ C{\isachardoublequoteclose}\isanewline
\ \ \isakeyword{shows}\ {\isachardoublequoteopen}{\isasymexists}C{\isacharprime}{\isachardot}\ C\ {\isasymsubseteq}\ C{\isacharprime}\ {\isasymand}\ pcp\ C{\isacharprime}\ {\isasymand}\ subset{\isacharunderscore}closed\ C{\isacharprime}{\isachardoublequoteclose}\isanewline
%
\isadelimproof
%
\endisadelimproof
%
\isatagproof
\isacommand{proof}\isamarkupfalse%
\ {\isacharminus}\isanewline
\ \ \isacommand{have}\isamarkupfalse%
\ C{\isadigit{1}}{\isacharcolon}{\isachardoublequoteopen}C\ {\isasymsubseteq}\ {\isacharparenleft}extensionSC\ C{\isacharparenright}{\isachardoublequoteclose}\isanewline
\ \ \ \ \isacommand{by}\isamarkupfalse%
\ {\isacharparenleft}rule\ ex{\isadigit{1}}{\isacharunderscore}subset{\isacharparenright}\isanewline
\ \ \isacommand{have}\isamarkupfalse%
\ C{\isadigit{2}}{\isacharcolon}{\isachardoublequoteopen}pcp\ {\isacharparenleft}extensionSC\ C{\isacharparenright}{\isachardoublequoteclose}\isanewline
\ \ \ \ \isacommand{using}\isamarkupfalse%
\ assms\ \isacommand{by}\isamarkupfalse%
\ {\isacharparenleft}rule\ ex{\isadigit{1}}{\isacharunderscore}pcp{\isacharparenright}\isanewline
\ \ \isacommand{have}\isamarkupfalse%
\ C{\isadigit{3}}{\isacharcolon}{\isachardoublequoteopen}subset{\isacharunderscore}closed\ {\isacharparenleft}extensionSC\ C{\isacharparenright}{\isachardoublequoteclose}\isanewline
\ \ \ \ \isacommand{using}\isamarkupfalse%
\ assms\ \isacommand{by}\isamarkupfalse%
\ {\isacharparenleft}rule\ ex{\isadigit{1}}{\isacharunderscore}subset{\isacharunderscore}closed{\isacharparenright}\isanewline
\ \ \isacommand{have}\isamarkupfalse%
\ {\isachardoublequoteopen}C\ {\isasymsubseteq}\ {\isacharparenleft}extensionSC\ C{\isacharparenright}\ {\isasymand}\ pcp\ {\isacharparenleft}extensionSC\ C{\isacharparenright}\ {\isasymand}\ subset{\isacharunderscore}closed\ {\isacharparenleft}extensionSC\ C{\isacharparenright}{\isachardoublequoteclose}\ \isanewline
\ \ \ \ \isacommand{using}\isamarkupfalse%
\ C{\isadigit{1}}\ C{\isadigit{2}}\ C{\isadigit{3}}\ \isacommand{by}\isamarkupfalse%
\ {\isacharparenleft}iprover\ intro{\isacharcolon}\ conjI{\isacharparenright}\isanewline
\ \ \isacommand{thus}\isamarkupfalse%
\ {\isacharquery}thesis\isanewline
\ \ \ \ \isacommand{by}\isamarkupfalse%
\ {\isacharparenleft}rule\ exI{\isacharparenright}\isanewline
\isacommand{qed}\isamarkupfalse%
%
\endisatagproof
{\isafoldproof}%
%
\isadelimproof
%
\endisadelimproof
%
\begin{isamarkuptext}%
Continuemos con el segundo resultado de este apartado.

  \begin{lema}
  Toda colección de conjuntos con la propiedad de carácter finito es cerrada bajo subconjuntos.
  \end{lema}

  En Isabelle, se formaliza como sigue.%
\end{isamarkuptext}\isamarkuptrue%
\isacommand{lemma}\isamarkupfalse%
\ \isanewline
\ \ \isakeyword{assumes}\ {\isachardoublequoteopen}finite{\isacharunderscore}character\ C{\isachardoublequoteclose}\isanewline
\ \ \isakeyword{shows}\ {\isachardoublequoteopen}subset{\isacharunderscore}closed\ C{\isachardoublequoteclose}\isanewline
%
\isadelimproof
\ \ %
\endisadelimproof
%
\isatagproof
\isacommand{oops}\isamarkupfalse%
%
\endisatagproof
{\isafoldproof}%
%
\isadelimproof
%
\endisadelimproof
%
\begin{isamarkuptext}%
Procedamos con la demostración del resultado.

  \begin{demostracion}
    Consideremos una colección de conjuntos \isa{C} con la propiedad de carácter finito. Probemos que, 
    en efecto, es cerrada bajo subconjuntos. Por definición de esta última propiedad, basta 
    demostrar que todo subconjunto de cada conjunto de \isa{C} pertenece también a \isa{C}.

    Para ello, tomemos un conjunto \isa{S} cualquiera perteneciente a \isa{C} y un subconjunto cualquiera 
    \isa{S{\isacharprime}} de \isa{S}. Probemos que \isa{S{\isacharprime}} está en \isa{C}. Por hipótesis, como \isa{C} tiene la propiedad de carácter 
    finito, verifica que, para cualquier conjunto \isa{A}, son equivalentes:
    \begin{enumerate}
      \item \isa{A} pertenece a \isa{C}.
      \item Todo subconjunto finito de \isa{A} pertenece a \isa{C}.
    \end{enumerate}

    Para probar que el subconjunto \isa{S{\isacharprime}} pertenece a \isa{C}, vamos a demostrar que todo subconjunto 
    finito de \isa{S{\isacharprime}} pertenece a \isa{C}.

    De este modo, consideremos un subconjunto cualquiera \isa{S{\isacharprime}{\isacharprime}} de \isa{S{\isacharprime}}. Como \isa{S{\isacharprime}} es subconjunto de \isa{S}, 
    por la transitividad de la relación de contención de conjuntos, se tiene que \isa{S{\isacharprime}{\isacharprime}} es subconjunto 
    de \isa{S}. Aplicando la definición de propiedad de carácter finito de \isa{C} para el conjunto \isa{S}, 
    como este pertenece a \isa{C}, verifica que todo subconjunto finito de \isa{S} pertenece a \isa{C}. En
    particular, como \isa{S{\isacharprime}{\isacharprime}} es subconjunto de \isa{S}, verifica que, si \isa{S{\isacharprime}{\isacharprime}} es finito, entonces \isa{S{\isacharprime}{\isacharprime}} 
    pertenece a \isa{C}. Por tanto, hemos probado que cualquier conjunto finito de \isa{S{\isacharprime}} pertenece a la
    colección. Finalmente por la propiedad de carácter finito de \isa{C}, se verifica que \isa{S{\isacharprime}} pertenece 
    a \isa{C}, como queríamos demostrar.
  \end{demostracion}

  Veamos, a continuación, la demostración detallada del resultado en Isabelle.%
\end{isamarkuptext}\isamarkuptrue%
\isacommand{lemma}\isamarkupfalse%
\isanewline
\ \ \isakeyword{assumes}\ {\isachardoublequoteopen}finite{\isacharunderscore}character\ C{\isachardoublequoteclose}\isanewline
\ \ \isakeyword{shows}\ {\isachardoublequoteopen}subset{\isacharunderscore}closed\ C{\isachardoublequoteclose}\isanewline
%
\isadelimproof
\ \ %
\endisadelimproof
%
\isatagproof
\isacommand{unfolding}\isamarkupfalse%
\ subset{\isacharunderscore}closed{\isacharunderscore}def\isanewline
\isacommand{proof}\isamarkupfalse%
\ {\isacharparenleft}intro\ ballI\ sallI{\isacharparenright}\isanewline
\ \ \isacommand{fix}\isamarkupfalse%
\ S{\isacharprime}\ S\isanewline
\ \ \isacommand{assume}\isamarkupfalse%
\ \ {\isacartoucheopen}S\ {\isasymin}\ C{\isacartoucheclose}\ \isakeyword{and}\ {\isacartoucheopen}S{\isacharprime}\ {\isasymsubseteq}\ S{\isacartoucheclose}\isanewline
\ \ \isacommand{have}\isamarkupfalse%
\ H{\isacharcolon}{\isachardoublequoteopen}{\isasymforall}A{\isachardot}\ A\ {\isasymin}\ C\ {\isasymlongleftrightarrow}\ {\isacharparenleft}{\isasymforall}A{\isacharprime}\ {\isasymsubseteq}\ A{\isachardot}\ finite\ A{\isacharprime}\ {\isasymlongrightarrow}\ A{\isacharprime}\ {\isasymin}\ C{\isacharparenright}{\isachardoublequoteclose}\isanewline
\ \ \ \ \isacommand{using}\isamarkupfalse%
\ assms\ \isacommand{unfolding}\isamarkupfalse%
\ finite{\isacharunderscore}character{\isacharunderscore}def\ \isacommand{by}\isamarkupfalse%
\ this\isanewline
\ \ \isacommand{have}\isamarkupfalse%
\ QPQ{\isacharcolon}{\isachardoublequoteopen}{\isasymforall}S{\isacharprime}{\isacharprime}\ {\isasymsubseteq}\ S{\isacharprime}{\isachardot}\ finite\ S{\isacharprime}{\isacharprime}\ {\isasymlongrightarrow}\ S{\isacharprime}{\isacharprime}\ {\isasymin}\ C{\isachardoublequoteclose}\isanewline
\ \ \isacommand{proof}\isamarkupfalse%
\ {\isacharparenleft}rule\ sallI{\isacharparenright}\isanewline
\ \ \ \ \isacommand{fix}\isamarkupfalse%
\ S{\isacharprime}{\isacharprime}\isanewline
\ \ \ \ \isacommand{assume}\isamarkupfalse%
\ {\isachardoublequoteopen}S{\isacharprime}{\isacharprime}\ {\isasymsubseteq}\ S{\isacharprime}{\isachardoublequoteclose}\isanewline
\ \ \ \ \isacommand{then}\isamarkupfalse%
\ \isacommand{have}\isamarkupfalse%
\ {\isachardoublequoteopen}S{\isacharprime}{\isacharprime}\ {\isasymsubseteq}\ S{\isachardoublequoteclose}\isanewline
\ \ \ \ \ \ \isacommand{using}\isamarkupfalse%
\ {\isacartoucheopen}S{\isacharprime}\ {\isasymsubseteq}\ S{\isacartoucheclose}\ \isacommand{by}\isamarkupfalse%
\ {\isacharparenleft}simp\ only{\isacharcolon}\ subset{\isacharunderscore}trans{\isacharparenright}\isanewline
\ \ \ \ \isacommand{have}\isamarkupfalse%
\ {\isadigit{1}}{\isacharcolon}{\isachardoublequoteopen}S\ {\isasymin}\ C\ {\isasymlongleftrightarrow}\ {\isacharparenleft}{\isasymforall}S{\isacharprime}\ {\isasymsubseteq}\ S{\isachardot}\ finite\ S{\isacharprime}\ {\isasymlongrightarrow}\ S{\isacharprime}\ {\isasymin}\ C{\isacharparenright}{\isachardoublequoteclose}\isanewline
\ \ \ \ \ \ \isacommand{using}\isamarkupfalse%
\ H\ \isacommand{by}\isamarkupfalse%
\ {\isacharparenleft}rule\ allE{\isacharparenright}\isanewline
\ \ \ \ \isacommand{have}\isamarkupfalse%
\ {\isachardoublequoteopen}{\isasymforall}S{\isacharprime}\ {\isasymsubseteq}\ S{\isachardot}\ finite\ S{\isacharprime}\ {\isasymlongrightarrow}\ S{\isacharprime}\ {\isasymin}\ C{\isachardoublequoteclose}\isanewline
\ \ \ \ \ \ \isacommand{using}\isamarkupfalse%
\ {\isacartoucheopen}S\ {\isasymin}\ C{\isacartoucheclose}\ {\isadigit{1}}\ \isacommand{by}\isamarkupfalse%
\ {\isacharparenleft}rule\ back{\isacharunderscore}subst{\isacharparenright}\isanewline
\ \ \ \ \isacommand{thus}\isamarkupfalse%
\ {\isachardoublequoteopen}finite\ S{\isacharprime}{\isacharprime}\ {\isasymlongrightarrow}\ S{\isacharprime}{\isacharprime}\ {\isasymin}\ C{\isachardoublequoteclose}\isanewline
\ \ \ \ \ \ \isacommand{using}\isamarkupfalse%
\ {\isacartoucheopen}S{\isacharprime}{\isacharprime}\ {\isasymsubseteq}\ S{\isacartoucheclose}\ \isacommand{by}\isamarkupfalse%
\ {\isacharparenleft}rule\ sspec{\isacharparenright}\isanewline
\ \ \isacommand{qed}\isamarkupfalse%
\isanewline
\ \ \isacommand{have}\isamarkupfalse%
\ {\isachardoublequoteopen}S{\isacharprime}\ {\isasymin}\ C\ {\isasymlongleftrightarrow}\ {\isacharparenleft}{\isasymforall}S{\isacharprime}{\isacharprime}\ {\isasymsubseteq}\ S{\isacharprime}{\isachardot}\ finite\ S{\isacharprime}{\isacharprime}\ {\isasymlongrightarrow}\ S{\isacharprime}{\isacharprime}\ {\isasymin}\ C{\isacharparenright}{\isachardoublequoteclose}\isanewline
\ \ \ \ \isacommand{using}\isamarkupfalse%
\ H\ \isacommand{by}\isamarkupfalse%
\ {\isacharparenleft}rule\ allE{\isacharparenright}\isanewline
\ \ \isacommand{thus}\isamarkupfalse%
\ {\isachardoublequoteopen}S{\isacharprime}\ {\isasymin}\ C{\isachardoublequoteclose}\isanewline
\ \ \ \ \isacommand{using}\isamarkupfalse%
\ QPQ\ \isacommand{by}\isamarkupfalse%
\ {\isacharparenleft}rule\ forw{\isacharunderscore}subst{\isacharparenright}\isanewline
\isacommand{qed}\isamarkupfalse%
%
\endisatagproof
{\isafoldproof}%
%
\isadelimproof
%
\endisadelimproof
%
\begin{isamarkuptext}%
Finalmente, su prueba automática en Isabelle/HOL es la siguiente.%
\end{isamarkuptext}\isamarkuptrue%
\isacommand{lemma}\isamarkupfalse%
\ ex{\isadigit{2}}{\isacharcolon}\ \isanewline
\ \ \isakeyword{assumes}\ fc{\isacharcolon}\ {\isachardoublequoteopen}finite{\isacharunderscore}character\ C{\isachardoublequoteclose}\isanewline
\ \ \isakeyword{shows}\ {\isachardoublequoteopen}subset{\isacharunderscore}closed\ C{\isachardoublequoteclose}\isanewline
%
\isadelimproof
\ \ %
\endisadelimproof
%
\isatagproof
\isacommand{unfolding}\isamarkupfalse%
\ subset{\isacharunderscore}closed{\isacharunderscore}def\isanewline
\isacommand{proof}\isamarkupfalse%
\ {\isacharparenleft}intro\ ballI\ sallI{\isacharparenright}\isanewline
\ \ \isacommand{fix}\isamarkupfalse%
\ S{\isacharprime}\ S\isanewline
\ \ \isacommand{assume}\isamarkupfalse%
\ e{\isacharcolon}\ {\isacartoucheopen}S\ {\isasymin}\ C{\isacartoucheclose}\ \isakeyword{and}\ s{\isacharcolon}\ {\isacartoucheopen}S{\isacharprime}\ {\isasymsubseteq}\ S{\isacartoucheclose}\isanewline
\ \ \isacommand{hence}\isamarkupfalse%
\ {\isacharasterisk}{\isacharcolon}\ {\isachardoublequoteopen}S{\isacharprime}{\isacharprime}\ {\isasymsubseteq}\ S{\isacharprime}\ {\isasymLongrightarrow}\ S{\isacharprime}{\isacharprime}\ {\isasymsubseteq}\ S{\isachardoublequoteclose}\ \isakeyword{for}\ S{\isacharprime}{\isacharprime}\ \isacommand{by}\isamarkupfalse%
\ simp\isanewline
\ \ \isacommand{from}\isamarkupfalse%
\ fc\ \isacommand{have}\isamarkupfalse%
\ {\isachardoublequoteopen}S{\isacharprime}{\isacharprime}\ {\isasymsubseteq}\ S\ {\isasymLongrightarrow}\ finite\ S{\isacharprime}{\isacharprime}\ {\isasymLongrightarrow}\ S{\isacharprime}{\isacharprime}\ {\isasymin}\ C{\isachardoublequoteclose}\ \isakeyword{for}\ S{\isacharprime}{\isacharprime}\ \isanewline
\ \ \ \ \isacommand{unfolding}\isamarkupfalse%
\ finite{\isacharunderscore}character{\isacharunderscore}def\ \isacommand{using}\isamarkupfalse%
\ e\ \isacommand{by}\isamarkupfalse%
\ blast\isanewline
\ \ \isacommand{hence}\isamarkupfalse%
\ {\isachardoublequoteopen}S{\isacharprime}{\isacharprime}\ {\isasymsubseteq}\ S{\isacharprime}\ {\isasymLongrightarrow}\ finite\ S{\isacharprime}{\isacharprime}\ {\isasymLongrightarrow}\ S{\isacharprime}{\isacharprime}\ {\isasymin}\ C{\isachardoublequoteclose}\ \isakeyword{for}\ S{\isacharprime}{\isacharprime}\ \isacommand{using}\isamarkupfalse%
\ {\isacharasterisk}\ \isacommand{by}\isamarkupfalse%
\ simp\isanewline
\ \ \isacommand{with}\isamarkupfalse%
\ fc\ \isacommand{show}\isamarkupfalse%
\ {\isacartoucheopen}S{\isacharprime}\ {\isasymin}\ C{\isacartoucheclose}\ \isacommand{unfolding}\isamarkupfalse%
\ finite{\isacharunderscore}character{\isacharunderscore}def\ \isacommand{by}\isamarkupfalse%
\ blast\isanewline
\isacommand{qed}\isamarkupfalse%
%
\endisatagproof
{\isafoldproof}%
%
\isadelimproof
%
\endisadelimproof
%
\begin{isamarkuptext}%
Introduzcamos el último resultado de la sección.

 \begin{lema}
    Toda colección de conjuntos con la propiedad de consistencia proposicional y cerrada bajo 
    subconjuntos se puede extender a una colección que también verifique la propiedad de 
    consistencia proposicional y sea de carácter finito.
 \end{lema}

 \begin{demostracion}
   Dada una colección de conjuntos \isa{C} en las condiciones del enunciado, vamos a considerar su 
   extensión \isa{C{\isacharprime}} definida como la unión de \isa{C} y la colección formada por aquellos conjuntos
   cuyos subconjuntos finitos pertenecen a \isa{C}. Es decir,\\ \isa{C{\isacharprime}\ {\isacharequal}\ C\ {\isasymunion}\ E} donde 
   \isa{E\ {\isacharequal}\ {\isacharbraceleft}S{\isachardot}\ {\isasymforall}S{\isacharprime}\ {\isasymsubseteq}\ S{\isachardot}\ finite\ S{\isacharprime}\ {\isasymlongrightarrow}\ S{\isacharprime}\ {\isasymin}\ C{\isacharbraceright}}. Es evidente que es extensión pues contiene 
   a la colección \isa{C}. Vamos a probar que, además es de carácter finito y verifica la 
   propiedad de consistencia proposicional.

   En primer lugar, demostremos que \isa{C{\isacharprime}} es de carácter finito. Por definición de dicha propiedad, 
   basta probar que, para cualquier conjunto, son equivalentes:
   \begin{enumerate}
    \item El conjunto pertenece \isa{C{\isacharprime}}.
    \item Todo subconjunto finito suyo pertenece a \isa{C{\isacharprime}}.
   \end{enumerate}

   Comencemos probando \isa{{\isadigit{1}}{\isacharparenright}\ {\isasymLongrightarrow}\ {\isadigit{2}}{\isacharparenright}}. Para ello, sea un conjunto \isa{S} de \isa{C{\isacharprime}} de modo que \isa{S{\isacharprime}} es un
   subconjunto finito suyo. Como \isa{S} pertenece a la extensión, por definición de la misma tenemos
   que o bien \isa{S} está en \isa{C} o bien \isa{S} está en \isa{E}. Vamos a probar que \isa{S{\isacharprime}} está en \isa{C{\isacharprime}} por
   eliminación de la disyunción anterior. En primer lugar, si suponemos que \isa{S} está en \isa{C}, como
   se trata de una colección cerrada bajo subconjuntos, tenemos que todo subconjunto de \isa{S} está en 
   \isa{C}. En particular, \isa{S{\isacharprime}} está en \isa{C} y, por definición de la extensión, se prueba
   que \isa{S{\isacharprime}} está en \isa{C{\isacharprime}}. Por otro lado, suponiendo que \isa{S} esté en \isa{E}, por definición de dicha 
   colección tenemos que todo subconjunto finito de \isa{S} está en \isa{C}. De este modo, por las hipótesis 
   se prueba que \isa{S{\isacharprime}} está en \isa{C} y, por tanto, pertenece a la extensión. 

   Por último, probemos la implicación \isa{{\isadigit{2}}{\isacharparenright}\ {\isasymLongrightarrow}\ {\isadigit{1}}{\isacharparenright}}. Sea un conjunto cualquiera \isa{S} tal que todo
   subconjunto finito suyo pertenece a \isa{C{\isacharprime}}. Vamos a probar que \isa{S} también pertenece a \isa{C{\isacharprime}}. En
   particular, probaremos que pertenece a \isa{E}. Luego basta probar que todo subconjunto finito de 
   \isa{S} pertenece a \isa{C}. Para ello, consideremos \isa{S{\isacharprime}} un subconjunto finito cualquiera de \isa{S}. Por
   hipótesis, tenemos que \isa{S{\isacharprime}} pertenece a \isa{C{\isacharprime}}. Por definición de la extensión, tenemos entonces
   que o bien \isa{S{\isacharprime}} está en \isa{C} (lo que daría por concluida la prueba) o bien \isa{S{\isacharprime}} está en \isa{E}. 
   De este modo, si suponemos que \isa{S{\isacharprime}} está en \isa{E}, por definición de dicha colección tenemos que
   todo subconjunto finito suyo está en \isa{C}. En particular, como todo conjunto es subconjunto de si
   mismo y como hemos supuesto que \isa{S{\isacharprime}} es finito, tenemos que \isa{S{\isacharprime}} está en \isa{C}, lo que prueba la
   implicación.

   Probemos, finalmente, que \isa{C{\isacharprime}} verifica la propiedad de consistencia proposicional. Para ello,
   vamos a considerar un conjunto cualquiera \isa{S} perteneciente a \isa{C{\isacharprime}} y probaremos que se verifican 
   las cuatro condiciones del lema de caracterización de la propiedad de consistencia proposicional
   mediante la notación uniforme. Como el conjunto \isa{S} pertenece a \isa{C{\isacharprime}}, se observa fácilmente por
   definición de la extensión que, o bien \isa{S} está en \isa{C} o bien \isa{S} está en \isa{E}. Veamos que, para 
   ambos casos, se verifican dichas condiciones.

   En primer lugar, supongamos que \isa{S} está en \isa{C}. Como \isa{C} verifica la propiedad de consistencia 
   proposicional por hipótesis, verifica el lema de caracterización en particular para el conjunto 
   \isa{S}. De este modo, se cumple:
   \begin{itemize}
     \item \isa{{\isasymbottom}} no pertenece a \isa{S}.
     \item Dada \isa{p} una fórmula atómica cualquiera, no se tiene 
      simultáneamente que\\ \isa{p\ {\isasymin}\ S} y \isa{{\isasymnot}\ p\ {\isasymin}\ S}.
     \item Para toda fórmula de tipo \isa{{\isasymalpha}} con componentes \isa{{\isasymalpha}\isactrlsub {\isadigit{1}}} y \isa{{\isasymalpha}\isactrlsub {\isadigit{2}}} tal que \isa{{\isasymalpha}}
      pertenece a \isa{S}, se tiene que \isa{{\isacharbraceleft}{\isasymalpha}\isactrlsub {\isadigit{1}}{\isacharcomma}{\isasymalpha}\isactrlsub {\isadigit{2}}{\isacharbraceright}\ {\isasymunion}\ S} pertenece a \isa{C}.
     \item Para toda fórmula de tipo \isa{{\isasymbeta}} con componentes \isa{{\isasymbeta}\isactrlsub {\isadigit{1}}} y \isa{{\isasymbeta}\isactrlsub {\isadigit{2}}} tal que \isa{{\isasymbeta}}
      pertenece a \isa{S}, se tiene que o bien \isa{{\isacharbraceleft}{\isasymbeta}\isactrlsub {\isadigit{1}}{\isacharbraceright}\ {\isasymunion}\ S} pertenece a \isa{C} o 
      bien \isa{{\isacharbraceleft}{\isasymbeta}\isactrlsub {\isadigit{2}}{\isacharbraceright}\ {\isasymunion}\ S} pertenece a \isa{C}.
   \end{itemize} 
  
  Por lo tanto, puesto que \isa{C} está contenida en la extensión \isa{C{\isacharprime}}, se verifican las cuatro
  condiciones del lema para \isa{C{\isacharprime}}.

  Supongamos ahora que \isa{S} está en \isa{E}. Probemos que, en efecto, verifica las condiciones del lema 
  de caracterización.

  En primer lugar vamos a demostrar que \isa{{\isasymbottom}\ {\isasymnotin}\ S} por reducción al absurdo. Si suponemos que \isa{{\isasymbottom}\ {\isasymin}\ S},
  se deduce que el conjunto \isa{{\isacharbraceleft}{\isasymbottom}{\isacharbraceright}} es un subconjunto finito de \isa{S}. Como \isa{S} está en \isa{E}, por
  definición tenemos que \isa{{\isacharbraceleft}{\isasymbottom}{\isacharbraceright}\ {\isasymin}\ C}. De este modo, aplicando el lema de\\ caracterización de la
  propiedad de consistencia proposicional para la colección \isa{C} y el conjunto \isa{{\isacharbraceleft}{\isasymbottom}{\isacharbraceright}}, por la primera
  condición obtenemos que \isa{{\isasymbottom}\ {\isasymnotin}\ {\isacharbraceleft}{\isasymbottom}{\isacharbraceright}}, llegando a una contradicción.

  Demostremos que se verifica la segunda condición del lema para las fórmulas atómicas. De este
  modo, vamos a probar que dada \isa{p} una fórmula atómica cualquiera, no se tiene simultáneamente que
  \isa{p\ {\isasymin}\ S} y \isa{{\isasymnot}\ p\ {\isasymin}\ S}. La prueba se realizará por reducción al absurdo, luego supongamos que para
  cierta fórmula atómica se verifica \isa{p\ {\isasymin}\ S} y\\ \isa{{\isasymnot}\ p\ {\isasymin}\ S}. Análogamente, se observa que el conjunto
  \isa{{\isacharbraceleft}p{\isacharcomma}\ {\isasymnot}\ p{\isacharbraceright}} es un subconjunto finito de \isa{S}, luego pertenece a \isa{C}. Aplicando el lema de
  caracterización de la propiedad de consistencia proposicional para la colección \isa{C} y el conjunto
  \isa{{\isacharbraceleft}p{\isacharcomma}\ {\isasymnot}\ p{\isacharbraceright}}, por la segunda condición obtenemos que no se tiene simultáneamente \isa{q\ {\isasymin}\ {\isacharbraceleft}p{\isacharcomma}\ {\isasymnot}\ p{\isacharbraceright}} y
  \isa{{\isasymnot}\ q\ {\isasymin}\ {\isacharbraceleft}p{\isacharcomma}\ {\isasymnot}\ p{\isacharbraceright}} para ninguna fórmula atómica \isa{q}, llegando así a una contradicción para la
  fórmula atómica \isa{p}.

  Por otro lado, vamos a probar que se verifica la tercera condición del lema de\\ caracterización
  sobre las fórmulas de tipo \isa{{\isasymalpha}}. Consideremos una fórmula cualquiera \isa{F} de tipo \isa{{\isasymalpha}} y componentes 
  \isa{{\isasymalpha}\isactrlsub {\isadigit{1}}} y \isa{{\isasymalpha}\isactrlsub {\isadigit{2}}}, y supongamos que \isa{F\ {\isasymin}\ S}. Demostraremos que\\ \isa{{\isacharbraceleft}{\isasymalpha}\isactrlsub {\isadigit{1}}{\isacharcomma}{\isasymalpha}\isactrlsub {\isadigit{2}}{\isacharbraceright}\ {\isasymunion}\ S\ {\isasymin}\ C{\isacharprime}}. 

  Para ello, probaremos inicialmente que todo subconjunto finito \isa{S{\isacharprime}} de \isa{S} tal que\\ \isa{F\ {\isasymin}\ S{\isacharprime}} 
  verifica \isa{{\isacharbraceleft}{\isasymalpha}\isactrlsub {\isadigit{1}}{\isacharcomma}{\isasymalpha}\isactrlsub {\isadigit{2}}{\isacharbraceright}\ {\isasymunion}\ S{\isacharprime}\ {\isasymin}\ C}. Consideremos \isa{S{\isacharprime}} subconjunto finito cualquiera de \isa{S} en las
  condiciones anteriores. Como \isa{S\ {\isasymin}\ E}, por definición tenemos que \isa{S{\isacharprime}\ {\isasymin}\ C}. Aplicando el lema de 
  caracterización de la propiedad de consistencia proposicional para la colección \isa{C} y el conjunto
  \isa{S{\isacharprime}}, por la tercera condición obtenemos que \isa{{\isacharbraceleft}{\isasymalpha}\isactrlsub {\isadigit{1}}{\isacharcomma}{\isasymalpha}\isactrlsub {\isadigit{2}}{\isacharbraceright}\ {\isasymunion}\ S{\isacharprime}\ {\isasymin}\ C} ya que hemos supuesto que 
  \isa{F\ {\isasymin}\ S{\isacharprime}}.

  Una vez probado el resultado anterior, demostremos que \isa{{\isacharbraceleft}{\isasymalpha}\isactrlsub {\isadigit{1}}{\isacharcomma}{\isasymalpha}\isactrlsub {\isadigit{2}}{\isacharbraceright}\ {\isasymunion}\ S\ {\isasymin}\ E} y, por definición de 
  \isa{C{\isacharprime}}, obtendremos \isa{{\isacharbraceleft}{\isasymalpha}\isactrlsub {\isadigit{1}}{\isacharcomma}{\isasymalpha}\isactrlsub {\isadigit{2}}{\isacharbraceright}\ {\isasymunion}\ S\ {\isasymin}\ C{\isacharprime}}. Además, por definición de \isa{E}, basta probar que todo 
  subconjunto finito de \isa{{\isacharbraceleft}{\isasymalpha}\isactrlsub {\isadigit{1}}{\isacharcomma}{\isasymalpha}\isactrlsub {\isadigit{2}}{\isacharbraceright}\ {\isasymunion}\ S} pertenece a \isa{C}. Consideremos \isa{S{\isacharprime}} un subconjunto finito 
  cualquiera de \isa{{\isacharbraceleft}{\isasymalpha}\isactrlsub {\isadigit{1}}{\isacharcomma}{\isasymalpha}\isactrlsub {\isadigit{2}}{\isacharbraceright}\ {\isasymunion}\ S}. Como \isa{F\ {\isasymin}\ S}, es sencillo comprobar que el conjunto 
  \isa{{\isacharbraceleft}F{\isacharbraceright}\ {\isasymunion}\ {\isacharparenleft}S{\isacharprime}\ {\isacharminus}\ {\isacharbraceleft}{\isasymalpha}\isactrlsub {\isadigit{1}}{\isacharcomma}{\isasymalpha}\isactrlsub {\isadigit{2}}{\isacharbraceright}{\isacharparenright}} es un subconjunto finito de \isa{S}. Por el resultado probado anteriormente, 
  tenemos que el conjunto \isa{{\isacharbraceleft}{\isasymalpha}\isactrlsub {\isadigit{1}}{\isacharcomma}{\isasymalpha}\isactrlsub {\isadigit{2}}{\isacharbraceright}\ {\isasymunion}\ {\isacharparenleft}{\isacharbraceleft}F{\isacharbraceright}\ {\isasymunion}\ {\isacharparenleft}S{\isacharprime}\ {\isacharminus}\ {\isacharbraceleft}{\isasymalpha}\isactrlsub {\isadigit{1}}{\isacharcomma}{\isasymalpha}\isactrlsub {\isadigit{2}}{\isacharbraceright}{\isacharparenright}{\isacharparenright}\ {\isacharequal}} \\ \isa{{\isacharequal}\ {\isacharbraceleft}F{\isacharcomma}{\isasymalpha}\isactrlsub {\isadigit{1}}{\isacharcomma}{\isasymalpha}\isactrlsub {\isadigit{2}}{\isacharbraceright}\ {\isasymunion}\ S{\isacharprime}} pertenece a \isa{C}. 
  Además, como \isa{C} es cerrada bajo subconjuntos, todo conjunto de \isa{C} verifica que cualquier 
  subconjunto suyo pertenece a la colección. Luego, como \isa{S{\isacharprime}} es un subconjunto de 
  \isa{{\isacharbraceleft}F{\isacharcomma}{\isasymalpha}\isactrlsub {\isadigit{1}}{\isacharcomma}{\isasymalpha}\isactrlsub {\isadigit{2}}{\isacharbraceright}\ {\isasymunion}\ S{\isacharprime}}, queda probado que \isa{S{\isacharprime}\ {\isasymin}\ C}.

  Finalmente, veamos que se verifica la última condición del lema de caracterización de la propiedad
  de consistencia proposicional referente a las fórmulas de tipo \isa{{\isasymbeta}}. Consideremos una fórmula 
  cualquiera \isa{F} de tipo \isa{{\isasymbeta}} con componentes \isa{{\isasymbeta}\isactrlsub {\isadigit{1}}} y \isa{{\isasymbeta}\isactrlsub {\isadigit{2}}} tal que \isa{F\ {\isasymin}\ S}. Vamos a probar que se
  tiene que o bien \isa{{\isacharbraceleft}{\isasymbeta}\isactrlsub {\isadigit{1}}{\isacharbraceright}\ {\isasymunion}\ S\ {\isasymin}\ E} o bien \isa{{\isacharbraceleft}{\isasymbeta}\isactrlsub {\isadigit{1}}{\isacharbraceright}\ {\isasymunion}\ S\ {\isasymin}\ E}. En tal caso, por definición de \isa{C{\isacharprime}} se
  cumple que o bien \isa{{\isacharbraceleft}{\isasymbeta}\isactrlsub {\isadigit{1}}{\isacharbraceright}\ {\isasymunion}\ S\ {\isasymin}\ C{\isacharprime}} o bien \isa{{\isacharbraceleft}{\isasymbeta}\isactrlsub {\isadigit{1}}{\isacharbraceright}\ {\isasymunion}\ S\ {\isasymin}\ C{\isacharprime}}. La prueba se realizará por reducción al
  absurdo. Para ello, probemos inicialmente dos resultados previos.

  \begin{description}
    \item[\isa{{\isasymone}{\isacharparenright}}] En las condiciones anteriores, si consideramos \isa{S\isactrlsub {\isadigit{1}}} y \isa{S\isactrlsub {\isadigit{2}}} subconjuntos finitos 
    cualesquiera de \isa{S} tales que \isa{F\ {\isasymin}\ S\isactrlsub {\isadigit{1}}} y \isa{F\ {\isasymin}\ S\isactrlsub {\isadigit{2}}}, entonces existe una fórmula \isa{I\ {\isasymin}\ {\isacharbraceleft}{\isasymbeta}\isactrlsub {\isadigit{1}}{\isacharcomma}{\isasymbeta}\isactrlsub {\isadigit{2}}{\isacharbraceright}} tal 
    que se verifica que tanto \isa{{\isacharbraceleft}I{\isacharbraceright}\ {\isasymunion}\ S\isactrlsub {\isadigit{1}}} como \isa{{\isacharbraceleft}I{\isacharbraceright}\ {\isasymunion}\ S\isactrlsub {\isadigit{2}}} están en \isa{C}.
  \end{description}
  
  Para probar \isa{{\isasymone}{\isacharparenright}}, consideremos el conjunto finito \isa{S\isactrlsub {\isadigit{1}}\ {\isasymunion}\ S\isactrlsub {\isadigit{2}}} que es subconjunto de \isa{S} por las 
  hipótesis. De este modo, como \isa{S\ {\isasymin}\ E}, tenemos que \isa{S\isactrlsub {\isadigit{1}}\ {\isasymunion}\ S\isactrlsub {\isadigit{2}}\ {\isasymin}\ C}. Aplicando el lema de 
  caracterización de la propiedad de consistencia proposicional para la colección \isa{C} y el conjunto 
  \isa{S\isactrlsub {\isadigit{1}}\ {\isasymunion}\ S\isactrlsub {\isadigit{2}}}, por la última condición sobre las fórmulas de tipo \isa{{\isasymbeta}}, como\\ \isa{F\ {\isasymin}\ S\isactrlsub {\isadigit{1}}\ {\isasymunion}\ S\isactrlsub {\isadigit{2}}} por las 
  hipótesis, se tiene que o bien \isa{{\isacharbraceleft}{\isasymbeta}\isactrlsub {\isadigit{1}}{\isacharbraceright}\ {\isasymunion}\ S\isactrlsub {\isadigit{1}}\ {\isasymunion}\ S\isactrlsub {\isadigit{2}}\ {\isasymin}\ C} o bien\\ \isa{{\isacharbraceleft}{\isasymbeta}\isactrlsub {\isadigit{2}}{\isacharbraceright}\ {\isasymunion}\ S\isactrlsub {\isadigit{1}}\ {\isasymunion}\ S\isactrlsub {\isadigit{2}}\ {\isasymin}\ C}. Por tanto, 
  existe una fórmula \isa{I\ {\isasymin}\ {\isacharbraceleft}{\isasymbeta}\isactrlsub {\isadigit{1}}{\isacharcomma}{\isasymbeta}\isactrlsub {\isadigit{2}}{\isacharbraceright}} tal que\\ \isa{{\isacharbraceleft}I{\isacharbraceright}\ {\isasymunion}\ S\isactrlsub {\isadigit{1}}\ {\isasymunion}\ S\isactrlsub {\isadigit{2}}\ {\isasymin}\ C}. Sea \isa{I} la fórmula que cumple lo 
  anterior. Como \isa{C} es cerrada bajo subconjuntos, los subconjuntos \isa{{\isacharbraceleft}I{\isacharbraceright}\ {\isasymunion}\ S\isactrlsub {\isadigit{1}}} y \isa{{\isacharbraceleft}I{\isacharbraceright}\ {\isasymunion}\ S\isactrlsub {\isadigit{2}}} de 
  \isa{{\isacharbraceleft}I{\isacharbraceright}\ {\isasymunion}\ S\isactrlsub {\isadigit{1}}\ {\isasymunion}\ S\isactrlsub {\isadigit{2}}} pertenecen también a \isa{C}. Por tanto, hemos probado que existe una fórmula 
  \isa{I\ {\isasymin}\ {\isacharbraceleft}{\isasymbeta}\isactrlsub {\isadigit{1}}{\isacharcomma}{\isasymbeta}\isactrlsub {\isadigit{2}}{\isacharbraceright}} tal que \isa{{\isacharbraceleft}I{\isacharbraceright}\ {\isasymunion}\ S\isactrlsub {\isadigit{1}}\ {\isasymin}\ C} y \isa{{\isacharbraceleft}I{\isacharbraceright}\ {\isasymunion}\ S\isactrlsub {\isadigit{2}}\ {\isasymin}\ C}.

  Por otra parte, veamos el segundo resultado. 

  \begin{description}
    \item[\isa{{\isasymtwo}{\isacharparenright}}] En las condiciones de \isa{{\isasymone}{\isacharparenright}} para conjuntos cualesquiera \isa{S\isactrlsub {\isadigit{1}}} y \isa{S\isactrlsub {\isadigit{2}}}, si además 
    suponemos que \isa{{\isacharbraceleft}{\isasymbeta}\isactrlsub {\isadigit{1}}{\isacharbraceright}\ {\isasymunion}\ S\isactrlsub {\isadigit{1}}\ {\isasymnotin}\ C} y \isa{{\isacharbraceleft}{\isasymbeta}\isactrlsub {\isadigit{2}}{\isacharbraceright}\ {\isasymunion}\ S\isactrlsub {\isadigit{2}}\ {\isasymnotin}\ C}, llegamos a una contradicción. 
  \end{description}

  Para probarlo, utilizaremos \isa{{\isasymone}{\isacharparenright}} para los conjuntos \isa{{\isacharbraceleft}F{\isacharbraceright}\ {\isasymunion}\ S\isactrlsub {\isadigit{1}}} y \isa{{\isacharbraceleft}F{\isacharbraceright}\ {\isasymunion}\ S\isactrlsub {\isadigit{2}}}. Como es evidente, 
  puesto que \isa{F\ {\isasymin}\ S}, se verifica que ambos conjuntos son subconjuntos de \isa{S}. Además, como \isa{S\isactrlsub {\isadigit{1}}} y 
  \isa{S\isactrlsub {\isadigit{2}}} son finitos, se tiene que \isa{{\isacharbraceleft}F{\isacharbraceright}\ {\isasymunion}\ S\isactrlsub {\isadigit{1}}} y \isa{{\isacharbraceleft}F{\isacharbraceright}\ {\isasymunion}\ S\isactrlsub {\isadigit{2}}} también lo son. Por último, es claro que 
  \isa{F} pertenece a ambos conjuntos. Por lo tanto, por \isa{{\isasymone}{\isacharparenright}} tenemos que existe una fórmula 
  \isa{I\ {\isasymin}\ {\isacharbraceleft}{\isasymbeta}\isactrlsub {\isadigit{1}}{\isacharcomma}{\isasymbeta}\isactrlsub {\isadigit{2}}{\isacharbraceright}} tal que \isa{{\isacharbraceleft}I{\isacharbraceright}\ {\isasymunion}\ {\isacharbraceleft}F{\isacharbraceright}\ {\isasymunion}\ S\isactrlsub {\isadigit{1}}\ {\isasymin}\ C} y \isa{{\isacharbraceleft}I{\isacharbraceright}\ {\isasymunion}\ {\isacharbraceleft}F{\isacharbraceright}\ {\isasymunion}\ S\isactrlsub {\isadigit{2}}\ {\isasymin}\ C}. Por otro lado, podemos probar 
  que \isa{{\isacharbraceleft}{\isasymbeta}\isactrlsub {\isadigit{1}}{\isacharbraceright}\ {\isasymunion}\ {\isacharbraceleft}F{\isacharbraceright}\ {\isasymunion}\ S\isactrlsub {\isadigit{1}}\ {\isasymnotin}\ C}. Esto se debe a que, en caso contrario, como \isa{C} es cerrado bajo 
  subconjuntos, tendríamos que el subconjunto\\ \isa{{\isacharbraceleft}{\isasymbeta}\isactrlsub {\isadigit{1}}{\isacharbraceright}\ {\isasymunion}\ S\isactrlsub {\isadigit{1}}} pertenecería a \isa{C}, lo que contradice las 
  hipótesis. Análogamente, obtenemos que \isa{{\isacharbraceleft}{\isasymbeta}\isactrlsub {\isadigit{2}}{\isacharbraceright}\ {\isasymunion}\ {\isacharbraceleft}F{\isacharbraceright}\ {\isasymunion}\ S\isactrlsub {\isadigit{2}}\ {\isasymnotin}\ C}. De este modo, obtenemos que para 
  toda fórmula \isa{I\ {\isasymin}\ {\isacharbraceleft}{\isasymbeta}\isactrlsub {\isadigit{1}}{\isacharcomma}{\isasymbeta}\isactrlsub {\isadigit{2}}{\isacharbraceright}} se cumple que o bien \isa{{\isacharbraceleft}I{\isacharbraceright}\ {\isasymunion}\ {\isacharbraceleft}F{\isacharbraceright}\ {\isasymunion}\ S\isactrlsub {\isadigit{1}}\ {\isasymnotin}\ C} o bien \isa{{\isacharbraceleft}I{\isacharbraceright}\ {\isasymunion}\ {\isacharbraceleft}F{\isacharbraceright}\ {\isasymunion}\ S\isactrlsub {\isadigit{2}}\ {\isasymnotin}\ C}. 
  Esto es equivalente a que no existe ninguna fórmula \isa{I\ {\isasymin}\ {\isacharbraceleft}{\isasymbeta}\isactrlsub {\isadigit{1}}{\isacharcomma}{\isasymbeta}\isactrlsub {\isadigit{2}}{\isacharbraceright}} tal que \isa{{\isacharbraceleft}I{\isacharbraceright}\ {\isasymunion}\ {\isacharbraceleft}F{\isacharbraceright}\ {\isasymunion}\ S\isactrlsub {\isadigit{1}}\ {\isasymin}\ C} y\\ 
  \isa{{\isacharbraceleft}I{\isacharbraceright}\ {\isasymunion}\ {\isacharbraceleft}F{\isacharbraceright}\ {\isasymunion}\ S\isactrlsub {\isadigit{2}}\ {\isasymin}\ C}, lo que contradice lo obtenido para los conjuntos \isa{{\isacharbraceleft}F{\isacharbraceright}\ {\isasymunion}\ S\isactrlsub {\isadigit{1}}} y\\ \isa{{\isacharbraceleft}F{\isacharbraceright}\ {\isasymunion}\ S\isactrlsub {\isadigit{2}}} 
  por \isa{{\isasymone}{\isacharparenright}}.

  Finalmente, con los resultados anteriores, podemos probar que o bien\\ \isa{{\isacharbraceleft}{\isasymbeta}\isactrlsub {\isadigit{1}}{\isacharbraceright}\ {\isasymunion}\ S\ {\isasymin}\ E} o bien 
  \isa{{\isacharbraceleft}{\isasymbeta}\isactrlsub {\isadigit{2}}{\isacharbraceright}\ {\isasymunion}\ S\ {\isasymin}\ E} por reducción al absurdo. Supongamos que\\ \isa{{\isacharbraceleft}{\isasymbeta}\isactrlsub {\isadigit{1}}{\isacharbraceright}\ {\isasymunion}\ S\ {\isasymnotin}\ E} y \isa{{\isacharbraceleft}{\isasymbeta}\isactrlsub {\isadigit{2}}{\isacharbraceright}\ {\isasymunion}\ S\ {\isasymnotin}\ E}. Por
  definición de \isa{E}, se verifica que existe algún subconjunto finito de \isa{{\isacharbraceleft}{\isasymbeta}\isactrlsub {\isadigit{1}}{\isacharbraceright}\ {\isasymunion}\ S} y existe algún 
  subconjunto finito de \isa{{\isacharbraceleft}{\isasymbeta}\isactrlsub {\isadigit{2}}{\isacharbraceright}\ {\isasymunion}\ S} tales que no pertenecen a \isa{C}. Notemos por \isa{S\isactrlsub {\isadigit{1}}} y \isa{S\isactrlsub {\isadigit{2}}} 
  respectivamente a los subconjuntos anteriores. Vamos a aplicar \isa{{\isasymtwo}{\isacharparenright}} para los conjuntos \isa{S\isactrlsub {\isadigit{1}}\ {\isacharminus}\ {\isacharbraceleft}{\isasymbeta}\isactrlsub {\isadigit{1}}{\isacharbraceright}} 
  y \isa{S\isactrlsub {\isadigit{2}}\ {\isacharminus}\ {\isacharbraceleft}{\isasymbeta}\isactrlsub {\isadigit{2}}{\isacharbraceright}} para llegar a la contradicción.

  Para ello, debemos probar que se verifican las hipótesis del resultado para los conjuntos
  señalados. Es claro que tanto \isa{S\isactrlsub {\isadigit{1}}\ {\isacharminus}\ {\isacharbraceleft}{\isasymbeta}\isactrlsub {\isadigit{1}}{\isacharbraceright}} como \isa{S\isactrlsub {\isadigit{2}}\ {\isacharminus}\ {\isacharbraceleft}{\isasymbeta}\isactrlsub {\isadigit{2}}{\isacharbraceright}} son subconjuntos de \isa{S}, ya que \isa{S\isactrlsub {\isadigit{1}}} y
  \isa{S\isactrlsub {\isadigit{2}}} son subconjuntos de \isa{{\isacharbraceleft}{\isasymbeta}\isactrlsub {\isadigit{1}}{\isacharbraceright}\ {\isasymunion}\ S} y \isa{{\isacharbraceleft}{\isasymbeta}\isactrlsub {\isadigit{2}}{\isacharbraceright}\ {\isasymunion}\ S} respectivamente. Además, como \isa{S\isactrlsub {\isadigit{1}}} y \isa{S\isactrlsub {\isadigit{2}}} son
  finitos, es evidente que \isa{S\isactrlsub {\isadigit{1}}\ {\isacharminus}\ {\isacharbraceleft}{\isasymbeta}\isactrlsub {\isadigit{1}}{\isacharbraceright}} y \isa{S\isactrlsub {\isadigit{2}}\ {\isacharminus}\ {\isacharbraceleft}{\isasymbeta}\isactrlsub {\isadigit{2}}{\isacharbraceright}} también lo son. Queda probar que los conjuntos 
  \isa{{\isacharbraceleft}{\isasymbeta}\isactrlsub {\isadigit{1}}{\isacharbraceright}\ {\isasymunion}\ {\isacharparenleft}S\isactrlsub {\isadigit{1}}\ {\isacharminus}\ {\isacharbraceleft}{\isasymbeta}\isactrlsub {\isadigit{1}}{\isacharbraceright}{\isacharparenright}\ {\isacharequal}\ {\isacharbraceleft}{\isasymbeta}\isactrlsub {\isadigit{1}}{\isacharbraceright}\ {\isasymunion}\ S\isactrlsub {\isadigit{1}}} y \isa{{\isacharbraceleft}{\isasymbeta}\isactrlsub {\isadigit{2}}{\isacharbraceright}\ {\isasymunion}\ {\isacharparenleft}S\isactrlsub {\isadigit{2}}\ {\isacharminus}\ {\isacharbraceleft}{\isasymbeta}\isactrlsub {\isadigit{2}}{\isacharbraceright}{\isacharparenright}\ {\isacharequal}\ {\isacharbraceleft}{\isasymbeta}\isactrlsub {\isadigit{2}}{\isacharbraceright}\ {\isasymunion}\ S\isactrlsub {\isadigit{2}}} no pertenecen a \isa{C}. Como ni 
  \isa{S\isactrlsub {\isadigit{1}}} ni \isa{S\isactrlsub {\isadigit{2}}} están en la colección \isa{C} cerrada bajo subconjuntos, se cumple que ninguno de ellos 
  son subconjuntos de \isa{S}. Sin embargo, se verifica que \isa{S\isactrlsub {\isadigit{1}}} es subconjunto de \isa{{\isacharbraceleft}{\isasymbeta}\isactrlsub {\isadigit{1}}{\isacharbraceright}\ {\isasymunion}\ S} y \isa{S\isactrlsub {\isadigit{2}}} es 
  subconjunto de \isa{{\isacharbraceleft}{\isasymbeta}\isactrlsub {\isadigit{2}}{\isacharbraceright}\ {\isasymunion}\ S}. Por tanto, se cumple que\\ \isa{{\isasymbeta}\isactrlsub {\isadigit{1}}\ {\isasymin}\ S\isactrlsub {\isadigit{1}}} y \isa{{\isasymbeta}\isactrlsub {\isadigit{2}}\ {\isasymin}\ S\isactrlsub {\isadigit{2}}}. Por lo tanto,
  tenemos finalmente que los conjuntos \isa{{\isacharbraceleft}{\isasymbeta}\isactrlsub {\isadigit{1}}{\isacharbraceright}\ {\isasymunion}\ S\isactrlsub {\isadigit{1}}\ {\isacharequal}\ S\isactrlsub {\isadigit{1}}} y\\ \isa{{\isacharbraceleft}{\isasymbeta}\isactrlsub {\isadigit{2}}{\isacharbraceright}\ {\isasymunion}\ S\isactrlsub {\isadigit{2}}\ {\isacharequal}\ S\isactrlsub {\isadigit{2}}} no pertenecen a \isa{C}.
  Finalmente, como se cumplen las condiciones del resultado \isa{{\isadigit{2}}{\isacharparenright}}, llegamos a una contradicción para 
  los conjuntos \isa{S\isactrlsub {\isadigit{1}}\ {\isacharminus}\ {\isacharbraceleft}{\isasymbeta}\isactrlsub {\isadigit{1}}{\isacharbraceright}} y \isa{S\isactrlsub {\isadigit{2}}\ {\isacharminus}\ {\isacharbraceleft}{\isasymbeta}\isactrlsub {\isadigit{2}}{\isacharbraceright}}, probando que o bien \isa{{\isacharbraceleft}{\isasymbeta}\isactrlsub {\isadigit{1}}{\isacharbraceright}\ {\isasymunion}\ S\ {\isasymin}\ E} o bien \isa{{\isacharbraceleft}{\isasymbeta}\isactrlsub {\isadigit{1}}{\isacharbraceright}\ {\isasymunion}\ S\ {\isasymin}\ E}. 
  Por lo tanto, obtenemos por definición de \isa{C{\isacharprime}} que o bien \isa{{\isacharbraceleft}{\isasymbeta}\isactrlsub {\isadigit{1}}{\isacharbraceright}\ {\isasymunion}\ S\ {\isasymin}\ C{\isacharprime}} o bien \isa{{\isacharbraceleft}{\isasymbeta}\isactrlsub {\isadigit{1}}{\isacharbraceright}\ {\isasymunion}\ S\ {\isasymin}\ C{\isacharprime}}.
 \end{demostracion}

  Finalmente, veamos la demostración detallada del lema en Isabelle. Debido a la cantidad de lemas
  auxiliares empleados en la prueba detallada, para facilitar la comprensión mostraremos a
  continuación un grafo que estructura las relaciones de necesidad de los lemas introducidos.
  
 \begin{tikzpicture}
  [
    grow                    = down,
    level 1/.style          = {sibling distance=7cm},
    level 2/.style          = {sibling distance=4cm},
    level 3/.style          = {sibling distance=5.7cm},
    level distance          = 1.5cm,
    edge from parent/.style = {draw},
    every node/.style       = {font=\tiny},
    sloped
  ]
  \node [root] {\isa{ex{\isadigit{3}}}\\ \isa{{\isacharparenleft}Lema\ {\isadigit{3}}{\isachardot}{\isadigit{0}}{\isachardot}{\isadigit{5}}{\isacharparenright}}}
    child { node [env] {\isa{ex{\isadigit{3}}{\isacharunderscore}finite{\isacharunderscore}character}\\ \isa{{\isacharparenleft}C{\isacharprime}\ tiene\ la\ propiedad\ de\ carácter\ finito{\isacharparenright}}}}
    child { node [env] {\isa{ex{\isadigit{3}}{\isacharunderscore}pcp}\\ \isa{{\isacharparenleft}C{\isacharprime}\ tiene\ la\ propiedad\ de\ consistencia\ proposicional{\isacharparenright}}}
      		child { node [env] {\isa{ex{\isadigit{3}}{\isacharunderscore}pcp{\isacharunderscore}SinC}\\ \isa{{\isacharparenleft}Caso\ del\ conjunto\ en\ C{\isacharparenright}}}}
      		child { node [env] {\isa{ex{\isadigit{3}}{\isacharunderscore}pcp{\isacharunderscore}SinE}\\ \isa{{\isacharparenleft}Caso\ del\ conjunto\ en\ E{\isacharparenright}}}
        				child { node [env] {\isa{ex{\isadigit{3}}{\isacharunderscore}pcp{\isacharunderscore}SinE{\isacharunderscore}CON}\\ \isa{{\isacharparenleft}Condición\ fórmulas\ de\ tipo\ {\isasymalpha}{\isacharparenright}}}}
        				child { node [env] {\isa{ex{\isadigit{3}}{\isacharunderscore}pcp{\isacharunderscore}SinE{\isacharunderscore}DIS}\\ \isa{{\isacharparenleft}Condición\ fórmulas\ de\ tipo\ {\isasymbeta}{\isacharparenright}}}
                      child { node [env] {\isa{ex{\isadigit{3}}{\isacharunderscore}pcp{\isacharunderscore}SinE{\isacharunderscore}DIS{\isacharunderscore}auxFalse}\\ \isa{{\isacharparenleft}Resultado\ {\isasymone}{\isacharparenright}}}
                            child { node [env] {\isa{ex{\isadigit{3}}{\isacharunderscore}pcp{\isacharunderscore}SinE{\isacharunderscore}DIS{\isacharunderscore}auxEx}\\ \isa{{\isacharparenleft}Resultado\ {\isasymtwo}{\isacharparenright}}}}}}}};
\end{tikzpicture}

  De este modo, la prueba del \isa{lema\ {\isadigit{1}}{\isachardot}{\isadigit{3}}{\isachardot}{\isadigit{5}}} se estructura fundamentalmente en dos lemas auxiliares. 
  El primero, formalizado como \isa{ex{\isadigit{3}}{\isacharunderscore}finite{\isacharunderscore}character} en Isabelle, prueba que la extensión 
  \isa{C{\isacharprime}\ {\isacharequal}\ C\ {\isasymunion}\ E}, donde \isa{E} es la colección formada por aquellos conjuntos cuyos subconjuntos finitos 
  pertenecen a \isa{C}, tiene la propiedad de carácter finito. El segundo, formalizado como \isa{ex{\isadigit{3}}{\isacharunderscore}pcp}, 
  demuestra que \isa{C{\isacharprime}} verifica la propiedad de consistencia proposicional demostrando que cumple las 
  condiciones suficientes de dicha propiedad por el lema de caracterización \isa{{\isadigit{1}}{\isachardot}{\isadigit{2}}{\isachardot}{\isadigit{5}}}. De este modo, 
  considerando un conjunto \isa{S\ {\isasymin}\ C{\isacharprime}}, \isa{ex{\isadigit{3}}{\isacharunderscore}pcp} precisa, a su vez, de dos lemas auxiliares que 
  prueben las condiciones suficientes de \isa{{\isadigit{1}}{\isachardot}{\isadigit{2}}{\isachardot}{\isadigit{5}}}: uno para el caso en que \isa{S\ {\isasymin}\ C} (\isa{ex{\isadigit{3}}{\isacharunderscore}pcp{\isacharunderscore}SinC}) y 
  otro para el caso en que \isa{S\ {\isasymin}\ E} (\isa{ex{\isadigit{3}}{\isacharunderscore}pcp{\isacharunderscore}SinE}). Por otro lado, para el último caso en que 
  \isa{S\ {\isasymin}\ E}, utilizaremos dos lemas auxiliares. El primero, formalizado como \isa{ex{\isadigit{3}}{\isacharunderscore}pcp{\isacharunderscore}SinE{\isacharunderscore}CON}, 
  prueba que para \isa{C} una colección con la propiedad de consistencia proposicional y cerrada bajo 
  subconjuntos, \isa{S\ {\isasymin}\ E} y sea \isa{F} una fórmula de tipo \isa{{\isasymalpha}} y componentes \isa{{\isasymalpha}\isactrlsub {\isadigit{1}}} y \isa{{\isasymalpha}\isactrlsub {\isadigit{2}}}, se tiene que\\ 
  \isa{{\isacharbraceleft}{\isasymalpha}\isactrlsub {\isadigit{1}}{\isacharcomma}{\isasymalpha}\isactrlsub {\isadigit{2}}{\isacharbraceright}\ {\isasymunion}\ S\ {\isasymin}\ C{\isacharprime}}. El segundo lema, formalizado como \isa{ex{\isadigit{3}}{\isacharunderscore}pcp{\isacharunderscore}SinE{\isacharunderscore}DIS}, prueba que para \isa{C} una 
  colección con la propiedad de consistencia proposicional y cerrada bajo subconjuntos, \isa{S\ {\isasymin}\ E} y 
  sea \isa{F} una fórmula de tipo \isa{{\isasymbeta}} y componentes \isa{{\isasymbeta}\isactrlsub {\isadigit{1}}} y \isa{{\isasymbeta}\isactrlsub {\isadigit{2}}}, se tiene que o bien \isa{{\isacharbraceleft}{\isasymbeta}\isactrlsub {\isadigit{1}}{\isacharbraceright}\ {\isasymunion}\ S\ {\isasymin}\ C{\isacharprime}} o 
  bien \isa{{\isacharbraceleft}{\isasymbeta}\isactrlsub {\isadigit{2}}{\isacharbraceright}\ {\isasymunion}\ S\ {\isasymin}\ C{\isacharprime}}. Por último, este segundo lema auxiliar se probará por reducción al absurdo, 
  precisando para ello de los siguientes resultados auxiliares:
  
  \begin{description}
    \item[\isa{Resultado\ {\isasymone}}] Formalizado como \isa{ex{\isadigit{3}}{\isacharunderscore}pcp{\isacharunderscore}SinE{\isacharunderscore}DIS{\isacharunderscore}auxEx}. Prueba que dada \isa{C} una 
    colección con la propiedad de consistencia proposicional y cerrada bajo subconjuntos,\\ \isa{S\ {\isasymin}\ E} y 
    sea \isa{F} es una fórmula de tipo \isa{{\isasymbeta}} de componentes \isa{{\isasymbeta}\isactrlsub {\isadigit{1}}} y \isa{{\isasymbeta}\isactrlsub {\isadigit{2}}}, si consideramos \isa{S\isactrlsub {\isadigit{1}}} y \isa{S\isactrlsub {\isadigit{2}}} 
    subconjuntos finitos cualesquiera de \isa{S} tales que \isa{F\ {\isasymin}\ S\isactrlsub {\isadigit{1}}} y \isa{F\ {\isasymin}\ S\isactrlsub {\isadigit{2}}}, entonces existe una 
    fórmula \isa{I\ {\isasymin}\ {\isacharbraceleft}{\isasymbeta}\isactrlsub {\isadigit{1}}{\isacharcomma}{\isasymbeta}\isactrlsub {\isadigit{2}}{\isacharbraceright}} tal que se verifica que tanto \isa{{\isacharbraceleft}I{\isacharbraceright}\ {\isasymunion}\ S\isactrlsub {\isadigit{1}}} como \isa{{\isacharbraceleft}I{\isacharbraceright}\ {\isasymunion}\ S\isactrlsub {\isadigit{2}}} están en \isa{C}. 
    \item[\isa{Resultado\ {\isasymtwo}}] Formalizado como \isa{ex{\isadigit{3}}{\isacharunderscore}pcp{\isacharunderscore}SinE{\isacharunderscore}DIS{\isacharunderscore}auxFalse}. Utiliza 
    \isa{ex{\isadigit{3}}{\isacharunderscore}pcp{\isacharunderscore}SinE{\isacharunderscore}DIS{\isacharunderscore}auxEx} como lema auxiliar. Prueba que, en las condiciones del \isa{Resultado\ {\isasymone}}, 
    si además suponemos que \isa{{\isacharbraceleft}{\isasymbeta}\isactrlsub {\isadigit{1}}{\isacharbraceright}\ {\isasymunion}\ S\isactrlsub {\isadigit{1}}\ {\isasymnotin}\ C} y \isa{{\isacharbraceleft}{\isasymbeta}\isactrlsub {\isadigit{2}}{\isacharbraceright}\ {\isasymunion}\ S\isactrlsub {\isadigit{2}}\ {\isasymnotin}\ C}, llegamos a una contradicción.
  \end{description} 

  Por otro lado, para facilitar la notación, dada una colección cualquiera \isa{C}, formalizamos las 
  colecciones \isa{E} y \isa{C{\isacharprime}} como \isa{extF\ C} y \isa{extensionFin\ C} respectivamente como se muestra a 
  continuación.%
\end{isamarkuptext}\isamarkuptrue%
\isacommand{definition}\isamarkupfalse%
\ extF\ {\isacharcolon}{\isacharcolon}\ {\isachardoublequoteopen}{\isacharparenleft}{\isacharparenleft}{\isacharprime}a\ formula{\isacharparenright}\ set{\isacharparenright}\ set\ {\isasymRightarrow}\ {\isacharparenleft}{\isacharparenleft}{\isacharprime}a\ formula{\isacharparenright}\ set{\isacharparenright}\ set{\isachardoublequoteclose}\isanewline
\ \ \isakeyword{where}\ extF{\isacharcolon}\ {\isachardoublequoteopen}extF\ C\ {\isacharequal}\ {\isacharbraceleft}S{\isachardot}\ {\isasymforall}S{\isacharprime}\ {\isasymsubseteq}\ S{\isachardot}\ finite\ S{\isacharprime}\ {\isasymlongrightarrow}\ S{\isacharprime}\ {\isasymin}\ C{\isacharbraceright}{\isachardoublequoteclose}\isanewline
\isanewline
\isacommand{definition}\isamarkupfalse%
\ extensionFin\ {\isacharcolon}{\isacharcolon}\ {\isachardoublequoteopen}{\isacharparenleft}{\isacharparenleft}{\isacharprime}a\ formula{\isacharparenright}\ set{\isacharparenright}\ set\ {\isasymRightarrow}\ {\isacharparenleft}{\isacharparenleft}{\isacharprime}a\ formula{\isacharparenright}\ set{\isacharparenright}\ set{\isachardoublequoteclose}\isanewline
\ \ \isakeyword{where}\ extensionFin{\isacharcolon}\ {\isachardoublequoteopen}extensionFin\ C\ {\isacharequal}\ C\ {\isasymunion}\ {\isacharparenleft}extF\ C{\isacharparenright}{\isachardoublequoteclose}%
\begin{isamarkuptext}%
Una vez hechas las aclaraciones anteriores, procedamos ordenadamente con la demostración 
  detallada de cada lema auxiliar que conforma la prueba del lema \isa{{\isadigit{1}}{\isachardot}{\isadigit{3}}{\isachardot}{\isadigit{5}}}. En primer lugar, probemos 
  detalladamente que la extensión \isa{C{\isacharprime}} tiene la propiedad de carácter finito.%
\end{isamarkuptext}\isamarkuptrue%
\isacommand{lemma}\isamarkupfalse%
\ ex{\isadigit{3}}{\isacharunderscore}finite{\isacharunderscore}character{\isacharcolon}\isanewline
\ \ \isakeyword{assumes}\ {\isachardoublequoteopen}subset{\isacharunderscore}closed\ C{\isachardoublequoteclose}\isanewline
\ \ \ \ \ \ \ \ \isakeyword{shows}\ {\isachardoublequoteopen}finite{\isacharunderscore}character\ {\isacharparenleft}extensionFin\ C{\isacharparenright}{\isachardoublequoteclose}\isanewline
%
\isadelimproof
%
\endisadelimproof
%
\isatagproof
\isacommand{proof}\isamarkupfalse%
\ {\isacharminus}\isanewline
\ \ \isacommand{show}\isamarkupfalse%
\ {\isachardoublequoteopen}finite{\isacharunderscore}character\ {\isacharparenleft}extensionFin\ C{\isacharparenright}{\isachardoublequoteclose}\isanewline
\ \ \ \ \isacommand{unfolding}\isamarkupfalse%
\ finite{\isacharunderscore}character{\isacharunderscore}def\isanewline
\ \ \isacommand{proof}\isamarkupfalse%
\ {\isacharparenleft}rule\ allI{\isacharparenright}\isanewline
\ \ \ \isacommand{fix}\isamarkupfalse%
\ S\isanewline
\ \ \ \isacommand{show}\isamarkupfalse%
\ {\isachardoublequoteopen}S\ {\isasymin}\ {\isacharparenleft}extensionFin\ C{\isacharparenright}\ {\isasymlongleftrightarrow}\ {\isacharparenleft}{\isasymforall}S{\isacharprime}\ {\isasymsubseteq}\ S{\isachardot}\ finite\ S{\isacharprime}\ {\isasymlongrightarrow}\ S{\isacharprime}\ {\isasymin}\ {\isacharparenleft}extensionFin\ C{\isacharparenright}{\isacharparenright}{\isachardoublequoteclose}\isanewline
\ \ \ \isacommand{proof}\isamarkupfalse%
\ {\isacharparenleft}rule\ iffI{\isacharparenright}\isanewline
\ \ \ \ \ \isacommand{assume}\isamarkupfalse%
\ {\isachardoublequoteopen}S\ {\isasymin}\ {\isacharparenleft}extensionFin\ C{\isacharparenright}{\isachardoublequoteclose}\isanewline
\ \ \ \ \ \isacommand{show}\isamarkupfalse%
\ {\isachardoublequoteopen}{\isasymforall}S{\isacharprime}\ {\isasymsubseteq}\ S{\isachardot}\ finite\ S{\isacharprime}\ {\isasymlongrightarrow}\ S{\isacharprime}\ {\isasymin}\ {\isacharparenleft}extensionFin\ C{\isacharparenright}{\isachardoublequoteclose}\isanewline
\ \ \ \ \ \isacommand{proof}\isamarkupfalse%
\ {\isacharparenleft}intro\ sallI\ impI{\isacharparenright}\isanewline
\ \ \ \ \ \ \ \isacommand{fix}\isamarkupfalse%
\ S{\isacharprime}\isanewline
\ \ \ \ \ \ \ \isacommand{assume}\isamarkupfalse%
\ {\isachardoublequoteopen}S{\isacharprime}\ {\isasymsubseteq}\ S{\isachardoublequoteclose}\isanewline
\ \ \ \ \ \ \ \isacommand{assume}\isamarkupfalse%
\ {\isachardoublequoteopen}finite\ S{\isacharprime}{\isachardoublequoteclose}\isanewline
\ \ \ \ \ \ \ \isacommand{have}\isamarkupfalse%
\ {\isachardoublequoteopen}S\ {\isasymin}\ C\ {\isasymor}\ S\ {\isasymin}\ {\isacharparenleft}extF\ C{\isacharparenright}{\isachardoublequoteclose}\isanewline
\ \ \ \ \ \ \ \ \ \isacommand{using}\isamarkupfalse%
\ {\isacartoucheopen}S\ {\isasymin}\ {\isacharparenleft}extensionFin\ C{\isacharparenright}{\isacartoucheclose}\ \isacommand{by}\isamarkupfalse%
\ {\isacharparenleft}simp\ only{\isacharcolon}\ extensionFin\ Un{\isacharunderscore}iff{\isacharparenright}\isanewline
\ \ \ \ \ \ \ \isacommand{thus}\isamarkupfalse%
\ {\isachardoublequoteopen}S{\isacharprime}\ {\isasymin}\ {\isacharparenleft}extensionFin\ C{\isacharparenright}{\isachardoublequoteclose}\isanewline
\ \ \ \ \ \ \ \isacommand{proof}\isamarkupfalse%
\ {\isacharparenleft}rule\ disjE{\isacharparenright}\isanewline
\ \ \ \ \ \ \ \ \ \isacommand{assume}\isamarkupfalse%
\ {\isachardoublequoteopen}S\ {\isasymin}\ C{\isachardoublequoteclose}\isanewline
\ \ \ \ \ \ \ \ \ \isacommand{have}\isamarkupfalse%
\ {\isachardoublequoteopen}{\isasymforall}S\ {\isasymin}\ C{\isachardot}\ {\isasymforall}S{\isacharprime}\ {\isasymsubseteq}\ S{\isachardot}\ S{\isacharprime}\ {\isasymin}\ C{\isachardoublequoteclose}\isanewline
\ \ \ \ \ \ \ \ \ \ \ \isacommand{using}\isamarkupfalse%
\ assms\ \isacommand{by}\isamarkupfalse%
\ {\isacharparenleft}simp\ only{\isacharcolon}\ subset{\isacharunderscore}closed{\isacharunderscore}def{\isacharparenright}\isanewline
\ \ \ \ \ \ \ \ \ \isacommand{then}\isamarkupfalse%
\ \isacommand{have}\isamarkupfalse%
\ {\isachardoublequoteopen}{\isasymforall}S{\isacharprime}\ {\isasymsubseteq}\ S{\isachardot}\ S{\isacharprime}\ {\isasymin}\ C{\isachardoublequoteclose}\isanewline
\ \ \ \ \ \ \ \ \ \ \ \isacommand{using}\isamarkupfalse%
\ {\isacartoucheopen}S\ {\isasymin}\ C{\isacartoucheclose}\ \isacommand{by}\isamarkupfalse%
\ {\isacharparenleft}rule\ bspec{\isacharparenright}\isanewline
\ \ \ \ \ \ \ \ \ \isacommand{then}\isamarkupfalse%
\ \isacommand{have}\isamarkupfalse%
\ {\isachardoublequoteopen}S{\isacharprime}\ {\isasymin}\ C{\isachardoublequoteclose}\isanewline
\ \ \ \ \ \ \ \ \ \ \ \isacommand{using}\isamarkupfalse%
\ {\isacartoucheopen}S{\isacharprime}\ {\isasymsubseteq}\ S{\isacartoucheclose}\ \isacommand{by}\isamarkupfalse%
\ {\isacharparenleft}rule\ sspec{\isacharparenright}\isanewline
\ \ \ \ \ \ \ \ \ \isacommand{thus}\isamarkupfalse%
\ {\isachardoublequoteopen}S{\isacharprime}\ {\isasymin}\ {\isacharparenleft}extensionFin\ C{\isacharparenright}{\isachardoublequoteclose}\ \isanewline
\ \ \ \ \ \ \ \ \ \ \ \isacommand{by}\isamarkupfalse%
\ {\isacharparenleft}simp\ only{\isacharcolon}\ extensionFin\ UnI{\isadigit{1}}{\isacharparenright}\isanewline
\ \ \ \ \ \ \ \isacommand{next}\isamarkupfalse%
\isanewline
\ \ \ \ \ \ \ \ \ \isacommand{assume}\isamarkupfalse%
\ {\isachardoublequoteopen}S\ {\isasymin}\ {\isacharparenleft}extF\ C{\isacharparenright}{\isachardoublequoteclose}\isanewline
\ \ \ \ \ \ \ \ \ \isacommand{then}\isamarkupfalse%
\ \isacommand{have}\isamarkupfalse%
\ {\isachardoublequoteopen}{\isasymforall}S{\isacharprime}\ {\isasymsubseteq}\ S{\isachardot}\ finite\ S{\isacharprime}\ {\isasymlongrightarrow}\ S{\isacharprime}\ {\isasymin}\ C{\isachardoublequoteclose}\isanewline
\ \ \ \ \ \ \ \ \ \ \ \isacommand{unfolding}\isamarkupfalse%
\ extF\ \isacommand{by}\isamarkupfalse%
\ {\isacharparenleft}rule\ CollectD{\isacharparenright}\isanewline
\ \ \ \ \ \ \ \ \ \isacommand{then}\isamarkupfalse%
\ \isacommand{have}\isamarkupfalse%
\ {\isachardoublequoteopen}finite\ S{\isacharprime}\ {\isasymlongrightarrow}\ S{\isacharprime}\ {\isasymin}\ C{\isachardoublequoteclose}\isanewline
\ \ \ \ \ \ \ \ \ \ \ \isacommand{using}\isamarkupfalse%
\ {\isacartoucheopen}S{\isacharprime}\ {\isasymsubseteq}\ S{\isacartoucheclose}\ \isacommand{by}\isamarkupfalse%
\ {\isacharparenleft}rule\ sspec{\isacharparenright}\isanewline
\ \ \ \ \ \ \ \ \ \isacommand{then}\isamarkupfalse%
\ \isacommand{have}\isamarkupfalse%
\ {\isachardoublequoteopen}S{\isacharprime}\ {\isasymin}\ C{\isachardoublequoteclose}\isanewline
\ \ \ \ \ \ \ \ \ \ \ \isacommand{using}\isamarkupfalse%
\ {\isacartoucheopen}finite\ S{\isacharprime}{\isacartoucheclose}\ \isacommand{by}\isamarkupfalse%
\ {\isacharparenleft}rule\ mp{\isacharparenright}\isanewline
\ \ \ \ \ \ \ \ \ \isacommand{thus}\isamarkupfalse%
\ {\isachardoublequoteopen}S{\isacharprime}\ {\isasymin}\ {\isacharparenleft}extensionFin\ C{\isacharparenright}{\isachardoublequoteclose}\isanewline
\ \ \ \ \ \ \ \ \ \ \ \isacommand{by}\isamarkupfalse%
\ {\isacharparenleft}simp\ only{\isacharcolon}\ extensionFin\ UnI{\isadigit{1}}{\isacharparenright}\isanewline
\ \ \ \ \ \ \ \isacommand{qed}\isamarkupfalse%
\isanewline
\ \ \ \ \ \isacommand{qed}\isamarkupfalse%
\isanewline
\ \ \ \isacommand{next}\isamarkupfalse%
\isanewline
\ \ \ \ \ \isacommand{assume}\isamarkupfalse%
\ {\isachardoublequoteopen}{\isasymforall}S{\isacharprime}\ {\isasymsubseteq}\ S{\isachardot}\ finite\ S{\isacharprime}\ {\isasymlongrightarrow}\ S{\isacharprime}\ {\isasymin}\ {\isacharparenleft}extensionFin\ C{\isacharparenright}{\isachardoublequoteclose}\isanewline
\ \ \ \ \ \isacommand{then}\isamarkupfalse%
\ \isacommand{have}\isamarkupfalse%
\ F{\isacharcolon}{\isachardoublequoteopen}{\isasymforall}S{\isacharprime}\ {\isasymsubseteq}\ S{\isachardot}\ finite\ S{\isacharprime}\ {\isasymlongrightarrow}\ S{\isacharprime}\ {\isasymin}\ C\ {\isasymor}\ S{\isacharprime}\ {\isasymin}\ {\isacharparenleft}extF\ C{\isacharparenright}{\isachardoublequoteclose}\isanewline
\ \ \ \ \ \ \ \isacommand{by}\isamarkupfalse%
\ {\isacharparenleft}simp\ only{\isacharcolon}\ extensionFin\ Un{\isacharunderscore}iff{\isacharparenright}\isanewline
\ \ \ \ \ \isacommand{have}\isamarkupfalse%
\ {\isachardoublequoteopen}{\isasymforall}S{\isacharprime}\ {\isasymsubseteq}\ S{\isachardot}\ finite\ S{\isacharprime}\ {\isasymlongrightarrow}\ S{\isacharprime}\ {\isasymin}\ C{\isachardoublequoteclose}\isanewline
\ \ \ \ \ \isacommand{proof}\isamarkupfalse%
\ {\isacharparenleft}rule\ sallI{\isacharparenright}\isanewline
\ \ \ \ \ \ \ \isacommand{fix}\isamarkupfalse%
\ S{\isacharprime}\isanewline
\ \ \ \ \ \ \ \isacommand{assume}\isamarkupfalse%
\ {\isachardoublequoteopen}S{\isacharprime}\ {\isasymsubseteq}\ S{\isachardoublequoteclose}\isanewline
\ \ \ \ \ \ \ \isacommand{show}\isamarkupfalse%
\ {\isachardoublequoteopen}finite\ S{\isacharprime}\ {\isasymlongrightarrow}\ S{\isacharprime}\ {\isasymin}\ C{\isachardoublequoteclose}\isanewline
\ \ \ \ \ \ \ \isacommand{proof}\isamarkupfalse%
\ {\isacharparenleft}rule\ impI{\isacharparenright}\isanewline
\ \ \ \ \ \ \ \ \ \isacommand{assume}\isamarkupfalse%
\ {\isachardoublequoteopen}finite\ S{\isacharprime}{\isachardoublequoteclose}\isanewline
\ \ \ \ \ \ \ \ \ \isacommand{have}\isamarkupfalse%
\ {\isachardoublequoteopen}finite\ S{\isacharprime}\ {\isasymlongrightarrow}\ S{\isacharprime}\ {\isasymin}\ C\ {\isasymor}\ S{\isacharprime}\ {\isasymin}\ {\isacharparenleft}extF\ C{\isacharparenright}{\isachardoublequoteclose}\ \isanewline
\ \ \ \ \ \ \ \ \ \ \ \isacommand{using}\isamarkupfalse%
\ F\ {\isacartoucheopen}S{\isacharprime}\ {\isasymsubseteq}\ S{\isacartoucheclose}\ \isacommand{by}\isamarkupfalse%
\ {\isacharparenleft}rule\ sspec{\isacharparenright}\isanewline
\ \ \ \ \ \ \ \ \ \isacommand{then}\isamarkupfalse%
\ \isacommand{have}\isamarkupfalse%
\ {\isachardoublequoteopen}S{\isacharprime}\ {\isasymin}\ C\ {\isasymor}\ S{\isacharprime}\ {\isasymin}\ {\isacharparenleft}extF\ C{\isacharparenright}{\isachardoublequoteclose}\isanewline
\ \ \ \ \ \ \ \ \ \ \ \isacommand{using}\isamarkupfalse%
\ {\isacartoucheopen}finite\ S{\isacharprime}{\isacartoucheclose}\ \isacommand{by}\isamarkupfalse%
\ {\isacharparenleft}rule\ mp{\isacharparenright}\isanewline
\ \ \ \ \ \ \ \ \ \isacommand{thus}\isamarkupfalse%
\ {\isachardoublequoteopen}S{\isacharprime}\ {\isasymin}\ C{\isachardoublequoteclose}\isanewline
\ \ \ \ \ \ \ \ \ \isacommand{proof}\isamarkupfalse%
\ {\isacharparenleft}rule\ disjE{\isacharparenright}\isanewline
\ \ \ \ \ \ \ \ \ \ \ \isacommand{assume}\isamarkupfalse%
\ {\isachardoublequoteopen}S{\isacharprime}\ {\isasymin}\ C{\isachardoublequoteclose}\isanewline
\ \ \ \ \ \ \ \ \ \ \ \isacommand{thus}\isamarkupfalse%
\ {\isachardoublequoteopen}S{\isacharprime}\ {\isasymin}\ C{\isachardoublequoteclose}\isanewline
\ \ \ \ \ \ \ \ \ \ \ \ \ \isacommand{by}\isamarkupfalse%
\ this\isanewline
\ \ \ \ \ \ \ \ \ \isacommand{next}\isamarkupfalse%
\isanewline
\ \ \ \ \ \ \ \ \ \ \ \isacommand{assume}\isamarkupfalse%
\ {\isachardoublequoteopen}S{\isacharprime}\ {\isasymin}\ {\isacharparenleft}extF\ C{\isacharparenright}{\isachardoublequoteclose}\isanewline
\ \ \ \ \ \ \ \ \ \ \ \isacommand{then}\isamarkupfalse%
\ \isacommand{have}\isamarkupfalse%
\ S{\isacharprime}{\isacharcolon}{\isachardoublequoteopen}{\isasymforall}S{\isacharprime}{\isacharprime}\ {\isasymsubseteq}\ S{\isacharprime}{\isachardot}\ finite\ S{\isacharprime}{\isacharprime}\ {\isasymlongrightarrow}\ S{\isacharprime}{\isacharprime}\ {\isasymin}\ C{\isachardoublequoteclose}\isanewline
\ \ \ \ \ \ \ \ \ \ \ \ \ \isacommand{unfolding}\isamarkupfalse%
\ extF\ \isacommand{by}\isamarkupfalse%
\ {\isacharparenleft}rule\ CollectD{\isacharparenright}\isanewline
\ \ \ \ \ \ \ \ \ \ \ \isacommand{have}\isamarkupfalse%
\ {\isachardoublequoteopen}S{\isacharprime}\ {\isasymsubseteq}\ S{\isacharprime}{\isachardoublequoteclose}\isanewline
\ \ \ \ \ \ \ \ \ \ \ \ \ \isacommand{by}\isamarkupfalse%
\ {\isacharparenleft}simp\ only{\isacharcolon}\ subset{\isacharunderscore}refl{\isacharparenright}\isanewline
\ \ \ \ \ \ \ \ \ \ \ \isacommand{have}\isamarkupfalse%
\ {\isachardoublequoteopen}finite\ S{\isacharprime}\ {\isasymlongrightarrow}\ S{\isacharprime}\ {\isasymin}\ C{\isachardoublequoteclose}\isanewline
\ \ \ \ \ \ \ \ \ \ \ \ \ \isacommand{using}\isamarkupfalse%
\ S{\isacharprime}\ {\isacartoucheopen}S{\isacharprime}\ {\isasymsubseteq}\ S{\isacharprime}{\isacartoucheclose}\ \isacommand{by}\isamarkupfalse%
\ {\isacharparenleft}rule\ sspec{\isacharparenright}\isanewline
\ \ \ \ \ \ \ \ \ \ \ \isacommand{thus}\isamarkupfalse%
\ {\isachardoublequoteopen}S{\isacharprime}\ {\isasymin}\ C{\isachardoublequoteclose}\isanewline
\ \ \ \ \ \ \ \ \ \ \ \ \ \isacommand{using}\isamarkupfalse%
\ {\isacartoucheopen}finite\ S{\isacharprime}{\isacartoucheclose}\ \isacommand{by}\isamarkupfalse%
\ {\isacharparenleft}rule\ mp{\isacharparenright}\isanewline
\ \ \ \ \ \ \ \ \ \isacommand{qed}\isamarkupfalse%
\isanewline
\ \ \ \ \ \ \ \isacommand{qed}\isamarkupfalse%
\isanewline
\ \ \ \ \ \isacommand{qed}\isamarkupfalse%
\isanewline
\ \ \ \ \ \isacommand{then}\isamarkupfalse%
\ \isacommand{have}\isamarkupfalse%
\ {\isachardoublequoteopen}S\ {\isasymin}\ {\isacharbraceleft}S{\isachardot}\ {\isasymforall}S{\isacharprime}\ {\isasymsubseteq}\ S{\isachardot}\ finite\ S{\isacharprime}\ {\isasymlongrightarrow}\ S{\isacharprime}\ {\isasymin}\ C{\isacharbraceright}{\isachardoublequoteclose}\isanewline
\ \ \ \ \ \ \ \isacommand{by}\isamarkupfalse%
\ {\isacharparenleft}rule\ CollectI{\isacharparenright}\isanewline
\ \ \ \ \ \isacommand{thus}\isamarkupfalse%
\ {\isachardoublequoteopen}S\ {\isasymin}\ {\isacharparenleft}extensionFin\ C{\isacharparenright}{\isachardoublequoteclose}\isanewline
\ \ \ \ \ \ \ \isacommand{by}\isamarkupfalse%
\ {\isacharparenleft}simp\ only{\isacharcolon}\ extF\ extensionFin\ UnI{\isadigit{2}}{\isacharparenright}\isanewline
\ \ \ \isacommand{qed}\isamarkupfalse%
\isanewline
\ \isacommand{qed}\isamarkupfalse%
\isanewline
\isacommand{qed}\isamarkupfalse%
%
\endisatagproof
{\isafoldproof}%
%
\isadelimproof
%
\endisadelimproof
%
\begin{isamarkuptext}%
Por otro lado, para probar que  \isa{C{\isacharprime}\ {\isacharequal}\ C\ {\isasymunion}\ E}  verifica la propiedad de consistencia 
  proposicional, consideraremos un conjunto \isa{S\ {\isasymin}\ C{\isacharprime}} y utilizaremos fundamentalmente dos lemas 
  auxiliares: uno para el caso en que \isa{S\ {\isasymin}\ C} y otro para el caso en que \isa{S\ {\isasymin}\ E}. 

  En primer lugar, vamos a probar el primer lema auxiliar para el caso en que\\ \isa{S\ {\isasymin}\ C}, formalizado
  como \isa{ex{\isadigit{3}}{\isacharunderscore}pcp{\isacharunderscore}SinC}. Dicho lema prueba que, si \isa{C} es una colección con la propiedad de 
  consistencia proposicional y cerrada bajo subconjuntos, y sea \isa{S\ {\isasymin}\ C}, se verifican
  las condiciones del lema de caracterización de la propiedad de consistencia proposicional para
  la extensión \isa{C{\isacharprime}}:
  \begin{itemize}
    \item \isa{{\isasymbottom}\ {\isasymnotin}\ S}.
    \item Dada \isa{p} una fórmula atómica cualquiera, no se tiene 
    simultáneamente que\\ \isa{p\ {\isasymin}\ S} y \isa{{\isasymnot}\ p\ {\isasymin}\ S}.
    \item Para toda fórmula de tipo \isa{{\isasymalpha}} con componentes \isa{{\isasymalpha}\isactrlsub {\isadigit{1}}} y \isa{{\isasymalpha}\isactrlsub {\isadigit{2}}} tal que \isa{{\isasymalpha}}
    pertenece a \isa{S}, se tiene que \isa{{\isacharbraceleft}{\isasymalpha}\isactrlsub {\isadigit{1}}{\isacharcomma}{\isasymalpha}\isactrlsub {\isadigit{2}}{\isacharbraceright}\ {\isasymunion}\ S} pertenece a \isa{C{\isacharprime}}.
    \item Para toda fórmula de tipo \isa{{\isasymbeta}} con componentes \isa{{\isasymbeta}\isactrlsub {\isadigit{1}}} y \isa{{\isasymbeta}\isactrlsub {\isadigit{2}}} tal que \isa{{\isasymbeta}}
    pertenece a \isa{S}, se tiene que o bien \isa{{\isacharbraceleft}{\isasymbeta}\isactrlsub {\isadigit{1}}{\isacharbraceright}\ {\isasymunion}\ S} pertenece a \isa{C{\isacharprime}} o 
    bien \isa{{\isacharbraceleft}{\isasymbeta}\isactrlsub {\isadigit{2}}{\isacharbraceright}\ {\isasymunion}\ S} pertenece a \isa{C{\isacharprime}}.
  \end{itemize}%
\end{isamarkuptext}\isamarkuptrue%
\isacommand{lemma}\isamarkupfalse%
\ ex{\isadigit{3}}{\isacharunderscore}pcp{\isacharunderscore}SinC{\isacharcolon}\isanewline
\ \ \isakeyword{assumes}\ {\isachardoublequoteopen}pcp\ C{\isachardoublequoteclose}\isanewline
\ \ \ \ \ \ \ \ \ \ {\isachardoublequoteopen}subset{\isacharunderscore}closed\ C{\isachardoublequoteclose}\isanewline
\ \ \ \ \ \ \ \ \ \ {\isachardoublequoteopen}S\ {\isasymin}\ C{\isachardoublequoteclose}\ \isanewline
\ \ \isakeyword{shows}\ {\isachardoublequoteopen}{\isasymbottom}\ {\isasymnotin}\ S\ {\isasymand}\isanewline
\ \ \ \ \ \ \ \ \ {\isacharparenleft}{\isasymforall}k{\isachardot}\ Atom\ k\ {\isasymin}\ S\ {\isasymlongrightarrow}\ \isactrlbold {\isasymnot}\ {\isacharparenleft}Atom\ k{\isacharparenright}\ {\isasymin}\ S\ {\isasymlongrightarrow}\ False{\isacharparenright}\ {\isasymand}\isanewline
\ \ \ \ \ \ \ \ \ {\isacharparenleft}{\isasymforall}F\ G\ H{\isachardot}\ Con\ F\ G\ H\ {\isasymlongrightarrow}\ F\ {\isasymin}\ S\ {\isasymlongrightarrow}\ {\isacharbraceleft}G{\isacharcomma}\ H{\isacharbraceright}\ {\isasymunion}\ S\ {\isasymin}\ {\isacharparenleft}extensionFin\ C{\isacharparenright}{\isacharparenright}\ {\isasymand}\isanewline
\ \ \ \ \ \ \ \ \ {\isacharparenleft}{\isasymforall}F\ G\ H{\isachardot}\ Dis\ F\ G\ H\ {\isasymlongrightarrow}\ F\ {\isasymin}\ S\ {\isasymlongrightarrow}\ {\isacharbraceleft}G{\isacharbraceright}\ {\isasymunion}\ S\ {\isasymin}{\isacharparenleft}extensionFin\ C{\isacharparenright}\ {\isasymor}\ {\isacharbraceleft}H{\isacharbraceright}\ {\isasymunion}\ S\ {\isasymin}\ {\isacharparenleft}extensionFin\ C{\isacharparenright}{\isacharparenright}{\isachardoublequoteclose}\isanewline
%
\isadelimproof
%
\endisadelimproof
%
\isatagproof
\isacommand{proof}\isamarkupfalse%
\ {\isacharminus}\isanewline
\ \ \isacommand{have}\isamarkupfalse%
\ PCP{\isacharcolon}{\isachardoublequoteopen}{\isasymforall}S\ {\isasymin}\ C{\isachardot}\isanewline
\ \ \ \ {\isasymbottom}\ {\isasymnotin}\ S\isanewline
\ \ \ \ {\isasymand}\ {\isacharparenleft}{\isasymforall}k{\isachardot}\ Atom\ k\ {\isasymin}\ S\ {\isasymlongrightarrow}\ \isactrlbold {\isasymnot}\ {\isacharparenleft}Atom\ k{\isacharparenright}\ {\isasymin}\ S\ {\isasymlongrightarrow}\ False{\isacharparenright}\isanewline
\ \ \ \ {\isasymand}\ {\isacharparenleft}{\isasymforall}F\ G\ H{\isachardot}\ Con\ F\ G\ H\ {\isasymlongrightarrow}\ F\ {\isasymin}\ S\ {\isasymlongrightarrow}\ {\isacharbraceleft}G{\isacharcomma}H{\isacharbraceright}\ {\isasymunion}\ S\ {\isasymin}\ C{\isacharparenright}\isanewline
\ \ \ \ {\isasymand}\ {\isacharparenleft}{\isasymforall}F\ G\ H{\isachardot}\ Dis\ F\ G\ H\ {\isasymlongrightarrow}\ F\ {\isasymin}\ S\ {\isasymlongrightarrow}\ {\isacharbraceleft}G{\isacharbraceright}\ {\isasymunion}\ S\ {\isasymin}\ C\ {\isasymor}\ {\isacharbraceleft}H{\isacharbraceright}\ {\isasymunion}\ S\ {\isasymin}\ C{\isacharparenright}{\isachardoublequoteclose}\isanewline
\ \ \ \ \isacommand{using}\isamarkupfalse%
\ assms{\isacharparenleft}{\isadigit{1}}{\isacharparenright}\ \isacommand{by}\isamarkupfalse%
\ {\isacharparenleft}rule\ pcp{\isacharunderscore}alt{\isadigit{1}}{\isacharparenright}\isanewline
\ \ \isacommand{have}\isamarkupfalse%
\ H{\isacharcolon}{\isachardoublequoteopen}{\isasymbottom}\ {\isasymnotin}\ S\isanewline
\ \ \ \ {\isasymand}\ {\isacharparenleft}{\isasymforall}k{\isachardot}\ Atom\ k\ {\isasymin}\ S\ {\isasymlongrightarrow}\ \isactrlbold {\isasymnot}\ {\isacharparenleft}Atom\ k{\isacharparenright}\ {\isasymin}\ S\ {\isasymlongrightarrow}\ False{\isacharparenright}\isanewline
\ \ \ \ {\isasymand}\ {\isacharparenleft}{\isasymforall}F\ G\ H{\isachardot}\ Con\ F\ G\ H\ {\isasymlongrightarrow}\ F\ {\isasymin}\ S\ {\isasymlongrightarrow}\ {\isacharbraceleft}G{\isacharcomma}H{\isacharbraceright}\ {\isasymunion}\ S\ {\isasymin}\ C{\isacharparenright}\isanewline
\ \ \ \ {\isasymand}\ {\isacharparenleft}{\isasymforall}F\ G\ H{\isachardot}\ Dis\ F\ G\ H\ {\isasymlongrightarrow}\ F\ {\isasymin}\ S\ {\isasymlongrightarrow}\ {\isacharbraceleft}G{\isacharbraceright}\ {\isasymunion}\ S\ {\isasymin}\ C\ {\isasymor}\ {\isacharbraceleft}H{\isacharbraceright}\ {\isasymunion}\ S\ {\isasymin}\ C{\isacharparenright}{\isachardoublequoteclose}\isanewline
\ \ \ \ \ \isacommand{using}\isamarkupfalse%
\ PCP\ {\isacartoucheopen}S\ {\isasymin}\ C{\isacartoucheclose}\ \isacommand{by}\isamarkupfalse%
\ {\isacharparenleft}rule\ bspec{\isacharparenright}\isanewline
\ \ \isacommand{then}\isamarkupfalse%
\ \isacommand{have}\isamarkupfalse%
\ A{\isadigit{1}}{\isacharcolon}{\isachardoublequoteopen}{\isasymbottom}\ {\isasymnotin}\ S{\isachardoublequoteclose}\isanewline
\ \ \ \ \isacommand{by}\isamarkupfalse%
\ {\isacharparenleft}rule\ conjunct{\isadigit{1}}{\isacharparenright}\isanewline
\ \ \isacommand{have}\isamarkupfalse%
\ A{\isadigit{2}}{\isacharcolon}{\isachardoublequoteopen}{\isasymforall}k{\isachardot}\ Atom\ k\ {\isasymin}\ S\ {\isasymlongrightarrow}\ \isactrlbold {\isasymnot}\ {\isacharparenleft}Atom\ k{\isacharparenright}\ {\isasymin}\ S\ {\isasymlongrightarrow}\ False{\isachardoublequoteclose}\isanewline
\ \ \ \ \isacommand{using}\isamarkupfalse%
\ H\ \isacommand{by}\isamarkupfalse%
\ {\isacharparenleft}iprover\ elim{\isacharcolon}\ conjunct{\isadigit{2}}\ conjunct{\isadigit{1}}{\isacharparenright}\isanewline
\ \ \isacommand{have}\isamarkupfalse%
\ S{\isadigit{3}}{\isacharcolon}{\isachardoublequoteopen}{\isasymforall}F\ G\ H{\isachardot}\ Con\ F\ G\ H\ {\isasymlongrightarrow}\ F\ {\isasymin}\ S\ {\isasymlongrightarrow}\ {\isacharbraceleft}G{\isacharcomma}H{\isacharbraceright}\ {\isasymunion}\ S\ {\isasymin}\ C{\isachardoublequoteclose}\isanewline
\ \ \ \ \isacommand{using}\isamarkupfalse%
\ H\ \isacommand{by}\isamarkupfalse%
\ {\isacharparenleft}iprover\ elim{\isacharcolon}\ conjunct{\isadigit{2}}\ conjunct{\isadigit{1}}{\isacharparenright}\isanewline
\ \ \isacommand{have}\isamarkupfalse%
\ A{\isadigit{3}}{\isacharcolon}{\isachardoublequoteopen}{\isasymforall}F\ G\ H{\isachardot}\ Con\ F\ G\ H\ {\isasymlongrightarrow}\ F\ {\isasymin}\ S\ {\isasymlongrightarrow}\ {\isacharbraceleft}G{\isacharcomma}\ H{\isacharbraceright}\ {\isasymunion}\ S\ {\isasymin}\ {\isacharparenleft}extensionFin\ C{\isacharparenright}{\isachardoublequoteclose}\isanewline
\ \ \isacommand{proof}\isamarkupfalse%
\ {\isacharparenleft}rule\ allI{\isacharparenright}{\isacharplus}\isanewline
\ \ \ \ \isacommand{fix}\isamarkupfalse%
\ F\ G\ H\isanewline
\ \ \ \ \isacommand{show}\isamarkupfalse%
\ {\isachardoublequoteopen}Con\ F\ G\ H\ {\isasymlongrightarrow}\ F\ {\isasymin}\ S\ {\isasymlongrightarrow}\ {\isacharbraceleft}G{\isacharcomma}\ H{\isacharbraceright}\ {\isasymunion}\ S\ {\isasymin}\ {\isacharparenleft}extensionFin\ C{\isacharparenright}{\isachardoublequoteclose}\isanewline
\ \ \ \ \isacommand{proof}\isamarkupfalse%
\ {\isacharparenleft}rule\ impI{\isacharparenright}{\isacharplus}\isanewline
\ \ \ \ \ \ \isacommand{assume}\isamarkupfalse%
\ {\isachardoublequoteopen}Con\ F\ G\ H{\isachardoublequoteclose}\isanewline
\ \ \ \ \ \ \isacommand{assume}\isamarkupfalse%
\ {\isachardoublequoteopen}F\ {\isasymin}\ S{\isachardoublequoteclose}\ \isanewline
\ \ \ \ \ \ \isacommand{have}\isamarkupfalse%
\ {\isachardoublequoteopen}Con\ F\ G\ H\ {\isasymlongrightarrow}\ F\ {\isasymin}\ S\ {\isasymlongrightarrow}\ {\isacharbraceleft}G{\isacharcomma}H{\isacharbraceright}\ {\isasymunion}\ S\ {\isasymin}\ C{\isachardoublequoteclose}\isanewline
\ \ \ \ \ \ \ \ \isacommand{using}\isamarkupfalse%
\ S{\isadigit{3}}\ \isacommand{by}\isamarkupfalse%
\ {\isacharparenleft}iprover\ elim{\isacharcolon}\ allE{\isacharparenright}\isanewline
\ \ \ \ \ \ \isacommand{then}\isamarkupfalse%
\ \isacommand{have}\isamarkupfalse%
\ {\isachardoublequoteopen}F\ {\isasymin}\ S\ {\isasymlongrightarrow}\ {\isacharbraceleft}G{\isacharcomma}H{\isacharbraceright}\ {\isasymunion}\ S\ {\isasymin}\ C{\isachardoublequoteclose}\isanewline
\ \ \ \ \ \ \ \ \isacommand{using}\isamarkupfalse%
\ {\isacartoucheopen}Con\ F\ G\ H{\isacartoucheclose}\ \isacommand{by}\isamarkupfalse%
\ {\isacharparenleft}rule\ mp{\isacharparenright}\isanewline
\ \ \ \ \ \ \isacommand{then}\isamarkupfalse%
\ \isacommand{have}\isamarkupfalse%
\ {\isachardoublequoteopen}{\isacharbraceleft}G{\isacharcomma}H{\isacharbraceright}\ {\isasymunion}\ S\ {\isasymin}\ C{\isachardoublequoteclose}\isanewline
\ \ \ \ \ \ \ \ \isacommand{using}\isamarkupfalse%
\ {\isacartoucheopen}F\ {\isasymin}\ S{\isacartoucheclose}\ \isacommand{by}\isamarkupfalse%
\ {\isacharparenleft}rule\ mp{\isacharparenright}\isanewline
\ \ \ \ \ \ \isacommand{thus}\isamarkupfalse%
\ {\isachardoublequoteopen}{\isacharbraceleft}G{\isacharcomma}H{\isacharbraceright}\ {\isasymunion}\ S\ {\isasymin}\ {\isacharparenleft}extensionFin\ C{\isacharparenright}{\isachardoublequoteclose}\isanewline
\ \ \ \ \ \ \ \ \isacommand{unfolding}\isamarkupfalse%
\ extensionFin\ \isacommand{by}\isamarkupfalse%
\ {\isacharparenleft}rule\ UnI{\isadigit{1}}{\isacharparenright}\isanewline
\ \ \ \ \isacommand{qed}\isamarkupfalse%
\isanewline
\ \ \isacommand{qed}\isamarkupfalse%
\isanewline
\ \ \isacommand{have}\isamarkupfalse%
\ S{\isadigit{4}}{\isacharcolon}{\isachardoublequoteopen}{\isasymforall}F\ G\ H{\isachardot}\ Dis\ F\ G\ H\ {\isasymlongrightarrow}\ F\ {\isasymin}\ S\ {\isasymlongrightarrow}\ {\isacharbraceleft}G{\isacharbraceright}\ {\isasymunion}\ S\ {\isasymin}\ C\ {\isasymor}\ {\isacharbraceleft}H{\isacharbraceright}\ {\isasymunion}\ S\ {\isasymin}\ C{\isachardoublequoteclose}\isanewline
\ \ \ \ \isacommand{using}\isamarkupfalse%
\ H\ \isacommand{by}\isamarkupfalse%
\ {\isacharparenleft}iprover\ elim{\isacharcolon}\ conjunct{\isadigit{2}}{\isacharparenright}\isanewline
\ \ \isacommand{have}\isamarkupfalse%
\ A{\isadigit{4}}{\isacharcolon}{\isachardoublequoteopen}{\isasymforall}F\ G\ H{\isachardot}\ Dis\ F\ G\ H\ {\isasymlongrightarrow}\ F\ {\isasymin}\ S\ {\isasymlongrightarrow}\ {\isacharbraceleft}G{\isacharbraceright}\ {\isasymunion}\ S\ {\isasymin}\ {\isacharparenleft}extensionFin\ C{\isacharparenright}\ {\isasymor}\ {\isacharbraceleft}H{\isacharbraceright}\ {\isasymunion}\ S\ {\isasymin}\ {\isacharparenleft}extensionFin\ C{\isacharparenright}{\isachardoublequoteclose}\isanewline
\ \ \isacommand{proof}\isamarkupfalse%
\ {\isacharparenleft}rule\ allI{\isacharparenright}{\isacharplus}\isanewline
\ \ \ \ \isacommand{fix}\isamarkupfalse%
\ F\ G\ H\isanewline
\ \ \ \ \isacommand{show}\isamarkupfalse%
\ {\isachardoublequoteopen}Dis\ F\ G\ H\ {\isasymlongrightarrow}\ F\ {\isasymin}\ S\ {\isasymlongrightarrow}\ {\isacharbraceleft}G{\isacharbraceright}\ {\isasymunion}\ S\ {\isasymin}\ {\isacharparenleft}extensionFin\ C{\isacharparenright}\ {\isasymor}\ {\isacharbraceleft}H{\isacharbraceright}\ {\isasymunion}\ S\ {\isasymin}\ {\isacharparenleft}extensionFin\ C{\isacharparenright}{\isachardoublequoteclose}\isanewline
\ \ \ \ \isacommand{proof}\isamarkupfalse%
\ {\isacharparenleft}rule\ impI{\isacharparenright}{\isacharplus}\isanewline
\ \ \ \ \ \ \isacommand{assume}\isamarkupfalse%
\ {\isachardoublequoteopen}Dis\ F\ G\ H{\isachardoublequoteclose}\isanewline
\ \ \ \ \ \ \isacommand{assume}\isamarkupfalse%
\ {\isachardoublequoteopen}F\ {\isasymin}\ S{\isachardoublequoteclose}\ \isanewline
\ \ \ \ \ \ \isacommand{have}\isamarkupfalse%
\ {\isachardoublequoteopen}Dis\ F\ G\ H\ {\isasymlongrightarrow}\ F\ {\isasymin}\ S\ {\isasymlongrightarrow}\ {\isacharbraceleft}G{\isacharbraceright}\ {\isasymunion}\ S\ {\isasymin}\ C\ {\isasymor}\ {\isacharbraceleft}H{\isacharbraceright}\ {\isasymunion}\ S\ {\isasymin}\ C{\isachardoublequoteclose}\isanewline
\ \ \ \ \ \ \ \ \isacommand{using}\isamarkupfalse%
\ S{\isadigit{4}}\ \isacommand{by}\isamarkupfalse%
\ {\isacharparenleft}iprover\ elim{\isacharcolon}\ allE{\isacharparenright}\isanewline
\ \ \ \ \ \ \isacommand{then}\isamarkupfalse%
\ \isacommand{have}\isamarkupfalse%
\ {\isachardoublequoteopen}F\ {\isasymin}\ S\ {\isasymlongrightarrow}\ {\isacharbraceleft}G{\isacharbraceright}\ {\isasymunion}\ S\ {\isasymin}\ C\ {\isasymor}\ {\isacharbraceleft}H{\isacharbraceright}\ {\isasymunion}\ S\ {\isasymin}\ C{\isachardoublequoteclose}\isanewline
\ \ \ \ \ \ \ \ \isacommand{using}\isamarkupfalse%
\ {\isacartoucheopen}Dis\ F\ G\ H{\isacartoucheclose}\ \isacommand{by}\isamarkupfalse%
\ {\isacharparenleft}rule\ mp{\isacharparenright}\isanewline
\ \ \ \ \ \ \isacommand{then}\isamarkupfalse%
\ \isacommand{have}\isamarkupfalse%
\ {\isachardoublequoteopen}{\isacharbraceleft}G{\isacharbraceright}\ {\isasymunion}\ S\ {\isasymin}\ C\ {\isasymor}\ {\isacharbraceleft}H{\isacharbraceright}\ {\isasymunion}\ S\ {\isasymin}\ C{\isachardoublequoteclose}\isanewline
\ \ \ \ \ \ \ \ \isacommand{using}\isamarkupfalse%
\ {\isacartoucheopen}F\ {\isasymin}\ S{\isacartoucheclose}\ \isacommand{by}\isamarkupfalse%
\ {\isacharparenleft}rule\ mp{\isacharparenright}\isanewline
\ \ \ \ \ \ \isacommand{thus}\isamarkupfalse%
\ {\isachardoublequoteopen}{\isacharbraceleft}G{\isacharbraceright}\ {\isasymunion}\ S\ {\isasymin}\ {\isacharparenleft}extensionFin\ C{\isacharparenright}\ {\isasymor}\ {\isacharbraceleft}H{\isacharbraceright}\ {\isasymunion}\ S\ {\isasymin}\ {\isacharparenleft}extensionFin\ C{\isacharparenright}{\isachardoublequoteclose}\isanewline
\ \ \ \ \ \ \isacommand{proof}\isamarkupfalse%
\ {\isacharparenleft}rule\ disjE{\isacharparenright}\isanewline
\ \ \ \ \ \ \ \ \isacommand{assume}\isamarkupfalse%
\ {\isachardoublequoteopen}{\isacharbraceleft}G{\isacharbraceright}\ {\isasymunion}\ S\ {\isasymin}\ C{\isachardoublequoteclose}\isanewline
\ \ \ \ \ \ \ \ \isacommand{then}\isamarkupfalse%
\ \isacommand{have}\isamarkupfalse%
\ {\isachardoublequoteopen}{\isacharbraceleft}G{\isacharbraceright}\ {\isasymunion}\ S\ {\isasymin}\ {\isacharparenleft}extensionFin\ C{\isacharparenright}{\isachardoublequoteclose}\isanewline
\ \ \ \ \ \ \ \ \ \ \isacommand{unfolding}\isamarkupfalse%
\ extensionFin\ \isacommand{by}\isamarkupfalse%
\ {\isacharparenleft}rule\ UnI{\isadigit{1}}{\isacharparenright}\isanewline
\ \ \ \ \ \ \ \ \isacommand{thus}\isamarkupfalse%
\ {\isachardoublequoteopen}{\isacharbraceleft}G{\isacharbraceright}\ {\isasymunion}\ S\ {\isasymin}\ {\isacharparenleft}extensionFin\ C{\isacharparenright}\ {\isasymor}\ {\isacharbraceleft}H{\isacharbraceright}\ {\isasymunion}\ S\ {\isasymin}\ {\isacharparenleft}extensionFin\ C{\isacharparenright}{\isachardoublequoteclose}\isanewline
\ \ \ \ \ \ \ \ \ \ \isacommand{by}\isamarkupfalse%
\ {\isacharparenleft}rule\ disjI{\isadigit{1}}{\isacharparenright}\isanewline
\ \ \ \ \ \ \isacommand{next}\isamarkupfalse%
\isanewline
\ \ \ \ \ \ \ \ \isacommand{assume}\isamarkupfalse%
\ {\isachardoublequoteopen}{\isacharbraceleft}H{\isacharbraceright}\ {\isasymunion}\ S\ {\isasymin}\ C{\isachardoublequoteclose}\isanewline
\ \ \ \ \ \ \ \ \isacommand{then}\isamarkupfalse%
\ \isacommand{have}\isamarkupfalse%
\ {\isachardoublequoteopen}{\isacharbraceleft}H{\isacharbraceright}\ {\isasymunion}\ S\ {\isasymin}\ {\isacharparenleft}extensionFin\ C{\isacharparenright}{\isachardoublequoteclose}\isanewline
\ \ \ \ \ \ \ \ \ \ \isacommand{unfolding}\isamarkupfalse%
\ extensionFin\ \isacommand{by}\isamarkupfalse%
\ {\isacharparenleft}rule\ UnI{\isadigit{1}}{\isacharparenright}\isanewline
\ \ \ \ \ \ \ \ \isacommand{thus}\isamarkupfalse%
\ {\isachardoublequoteopen}{\isacharbraceleft}G{\isacharbraceright}\ {\isasymunion}\ S\ {\isasymin}\ {\isacharparenleft}extensionFin\ C{\isacharparenright}\ {\isasymor}\ {\isacharbraceleft}H{\isacharbraceright}\ {\isasymunion}\ S\ {\isasymin}\ {\isacharparenleft}extensionFin\ C{\isacharparenright}{\isachardoublequoteclose}\isanewline
\ \ \ \ \ \ \ \ \ \ \isacommand{by}\isamarkupfalse%
\ {\isacharparenleft}rule\ disjI{\isadigit{2}}{\isacharparenright}\isanewline
\ \ \ \ \ \ \isacommand{qed}\isamarkupfalse%
\isanewline
\ \ \ \ \isacommand{qed}\isamarkupfalse%
\isanewline
\ \ \isacommand{qed}\isamarkupfalse%
\isanewline
\ \ \isacommand{show}\isamarkupfalse%
\ {\isachardoublequoteopen}{\isasymbottom}\ {\isasymnotin}\ S\ {\isasymand}\isanewline
\ \ \ \ \ \ \ \ {\isacharparenleft}{\isasymforall}k{\isachardot}\ Atom\ k\ {\isasymin}\ S\ {\isasymlongrightarrow}\ \isactrlbold {\isasymnot}\ {\isacharparenleft}Atom\ k{\isacharparenright}\ {\isasymin}\ S\ {\isasymlongrightarrow}\ False{\isacharparenright}\ {\isasymand}\isanewline
\ \ \ \ \ \ \ \ {\isacharparenleft}{\isasymforall}F\ G\ H{\isachardot}\ Con\ F\ G\ H\ {\isasymlongrightarrow}\ F\ {\isasymin}\ S\ {\isasymlongrightarrow}\ {\isacharbraceleft}G{\isacharcomma}\ H{\isacharbraceright}\ {\isasymunion}\ S\ {\isasymin}\ {\isacharparenleft}extensionFin\ C{\isacharparenright}{\isacharparenright}\ {\isasymand}\isanewline
\ \ \ \ \ \ \ \ {\isacharparenleft}{\isasymforall}F\ G\ H{\isachardot}\ Dis\ F\ G\ H\ {\isasymlongrightarrow}\ F\ {\isasymin}\ S\ {\isasymlongrightarrow}\ {\isacharbraceleft}G{\isacharbraceright}\ {\isasymunion}\ S\ {\isasymin}\ {\isacharparenleft}extensionFin\ C{\isacharparenright}\ {\isasymor}\ {\isacharbraceleft}H{\isacharbraceright}\ {\isasymunion}\ S\ {\isasymin}\ {\isacharparenleft}extensionFin\ C{\isacharparenright}{\isacharparenright}{\isachardoublequoteclose}\isanewline
\ \ \ \ \isacommand{using}\isamarkupfalse%
\ A{\isadigit{1}}\ A{\isadigit{2}}\ A{\isadigit{3}}\ A{\isadigit{4}}\ \isacommand{by}\isamarkupfalse%
\ {\isacharparenleft}iprover\ intro{\isacharcolon}\ conjI{\isacharparenright}\isanewline
\isacommand{qed}\isamarkupfalse%
%
\endisatagproof
{\isafoldproof}%
%
\isadelimproof
%
\endisadelimproof
%
\begin{isamarkuptext}%
Como hemos señalado con anterioridad, para probar el caso en que \isa{S\ {\isasymin}\ E}, donde \isa{E} es la 
  colección formada por aquellos conjuntos cuyos subconjuntos finitos pertenecen a \isa{C}, precisaremos 
  de distintos lemas auxiliares. El primero de ellos demuestra detalladamente que si \isa{C} es una
  colección con la propiedad de consistencia proposicional y cerrada bajo subconjuntos, \isa{S\ {\isasymin}\ E}
  y sea \isa{F} una fórmula de tipo \isa{{\isasymalpha}} con componentes \isa{{\isasymalpha}\isactrlsub {\isadigit{1}}} y \isa{{\isasymalpha}\isactrlsub {\isadigit{2}}}, se verifica que \isa{{\isacharbraceleft}{\isasymalpha}\isactrlsub {\isadigit{1}}{\isacharcomma}{\isasymalpha}\isactrlsub {\isadigit{2}}{\isacharbraceright}\ {\isasymunion}\ S} 
  pertenece a la extensión \isa{C{\isacharprime}\ {\isacharequal}\ C\ {\isasymunion}\ E}.%
\end{isamarkuptext}\isamarkuptrue%
\isacommand{lemma}\isamarkupfalse%
\ ex{\isadigit{3}}{\isacharunderscore}pcp{\isacharunderscore}SinE{\isacharunderscore}CON{\isacharcolon}\isanewline
\ \ \isakeyword{assumes}\ {\isachardoublequoteopen}pcp\ C{\isachardoublequoteclose}\isanewline
\ \ \ \ \ \ \ \ \ \ {\isachardoublequoteopen}subset{\isacharunderscore}closed\ C{\isachardoublequoteclose}\isanewline
\ \ \ \ \ \ \ \ \ \ {\isachardoublequoteopen}S\ {\isasymin}\ {\isacharparenleft}extF\ C{\isacharparenright}{\isachardoublequoteclose}\isanewline
\ \ \ \ \ \ \ \ \ \ {\isachardoublequoteopen}Con\ F\ G\ H{\isachardoublequoteclose}\isanewline
\ \ \ \ \ \ \ \ \ \ {\isachardoublequoteopen}F\ {\isasymin}\ S{\isachardoublequoteclose}\isanewline
\ \ \isakeyword{shows}\ {\isachardoublequoteopen}{\isacharbraceleft}G{\isacharcomma}H{\isacharbraceright}\ {\isasymunion}\ S\ {\isasymin}\ {\isacharparenleft}extensionFin\ C{\isacharparenright}{\isachardoublequoteclose}\ \isanewline
%
\isadelimproof
%
\endisadelimproof
%
\isatagproof
\isacommand{proof}\isamarkupfalse%
\ {\isacharminus}\isanewline
\ \ \isacommand{have}\isamarkupfalse%
\ PCP{\isacharcolon}{\isachardoublequoteopen}{\isasymforall}S\ {\isasymin}\ C{\isachardot}\isanewline
\ \ {\isasymbottom}\ {\isasymnotin}\ S\isanewline
{\isasymand}\ {\isacharparenleft}{\isasymforall}k{\isachardot}\ Atom\ k\ {\isasymin}\ S\ {\isasymlongrightarrow}\ \isactrlbold {\isasymnot}\ {\isacharparenleft}Atom\ k{\isacharparenright}\ {\isasymin}\ S\ {\isasymlongrightarrow}\ False{\isacharparenright}\isanewline
{\isasymand}\ {\isacharparenleft}{\isasymforall}F\ G\ H{\isachardot}\ Con\ F\ G\ H\ {\isasymlongrightarrow}\ F\ {\isasymin}\ S\ {\isasymlongrightarrow}\ {\isacharbraceleft}G{\isacharcomma}H{\isacharbraceright}\ {\isasymunion}\ S\ {\isasymin}\ C{\isacharparenright}\isanewline
{\isasymand}\ {\isacharparenleft}{\isasymforall}F\ G\ H{\isachardot}\ Dis\ F\ G\ H\ {\isasymlongrightarrow}\ F\ {\isasymin}\ S\ {\isasymlongrightarrow}\ {\isacharbraceleft}G{\isacharbraceright}\ {\isasymunion}\ S\ {\isasymin}\ C\ {\isasymor}\ {\isacharbraceleft}H{\isacharbraceright}\ {\isasymunion}\ S\ {\isasymin}\ C{\isacharparenright}{\isachardoublequoteclose}\isanewline
\ \ \ \ \isacommand{using}\isamarkupfalse%
\ assms{\isacharparenleft}{\isadigit{1}}{\isacharparenright}\ \isacommand{by}\isamarkupfalse%
\ {\isacharparenleft}rule\ pcp{\isacharunderscore}alt{\isadigit{1}}{\isacharparenright}\isanewline
\ \ \isacommand{have}\isamarkupfalse%
\ {\isadigit{1}}{\isacharcolon}{\isachardoublequoteopen}{\isasymforall}S{\isacharprime}\ {\isasymsubseteq}\ S{\isachardot}\ finite\ S{\isacharprime}\ {\isasymlongrightarrow}\ F\ {\isasymin}\ S{\isacharprime}\ {\isasymlongrightarrow}\ {\isacharbraceleft}G{\isacharcomma}H{\isacharbraceright}\ {\isasymunion}\ S{\isacharprime}\ {\isasymin}\ C{\isachardoublequoteclose}\isanewline
\ \ \isacommand{proof}\isamarkupfalse%
\ {\isacharparenleft}rule\ sallI{\isacharparenright}\isanewline
\ \ \ \ \isacommand{fix}\isamarkupfalse%
\ S{\isacharprime}\isanewline
\ \ \ \ \isacommand{assume}\isamarkupfalse%
\ {\isachardoublequoteopen}S{\isacharprime}\ {\isasymsubseteq}\ S{\isachardoublequoteclose}\isanewline
\ \ \ \ \isacommand{show}\isamarkupfalse%
\ {\isachardoublequoteopen}finite\ S{\isacharprime}\ {\isasymlongrightarrow}\ F\ {\isasymin}\ S{\isacharprime}\ {\isasymlongrightarrow}\ {\isacharbraceleft}G{\isacharcomma}H{\isacharbraceright}\ {\isasymunion}\ S{\isacharprime}\ {\isasymin}\ C{\isachardoublequoteclose}\isanewline
\ \ \ \ \isacommand{proof}\isamarkupfalse%
\ {\isacharparenleft}rule\ impI{\isacharparenright}{\isacharplus}\isanewline
\ \ \ \ \ \ \isacommand{assume}\isamarkupfalse%
\ {\isachardoublequoteopen}finite\ S{\isacharprime}{\isachardoublequoteclose}\isanewline
\ \ \ \ \ \ \isacommand{assume}\isamarkupfalse%
\ {\isachardoublequoteopen}F\ {\isasymin}\ S{\isacharprime}{\isachardoublequoteclose}\isanewline
\ \ \ \ \ \ \isacommand{have}\isamarkupfalse%
\ E{\isacharcolon}{\isachardoublequoteopen}{\isasymforall}S{\isacharprime}\ {\isasymsubseteq}\ S{\isachardot}\ finite\ S{\isacharprime}\ {\isasymlongrightarrow}\ S{\isacharprime}\ {\isasymin}\ C{\isachardoublequoteclose}\isanewline
\ \ \ \ \ \ \ \ \isacommand{using}\isamarkupfalse%
\ assms{\isacharparenleft}{\isadigit{3}}{\isacharparenright}\ \isacommand{unfolding}\isamarkupfalse%
\ extF\ \isacommand{by}\isamarkupfalse%
\ {\isacharparenleft}rule\ CollectD{\isacharparenright}\isanewline
\ \ \ \ \ \ \isacommand{then}\isamarkupfalse%
\ \isacommand{have}\isamarkupfalse%
\ {\isachardoublequoteopen}finite\ S{\isacharprime}\ {\isasymlongrightarrow}\ S{\isacharprime}\ {\isasymin}\ C{\isachardoublequoteclose}\isanewline
\ \ \ \ \ \ \ \ \isacommand{using}\isamarkupfalse%
\ {\isacartoucheopen}S{\isacharprime}\ {\isasymsubseteq}\ S{\isacartoucheclose}\ \isacommand{by}\isamarkupfalse%
\ {\isacharparenleft}rule\ sspec{\isacharparenright}\isanewline
\ \ \ \ \ \ \isacommand{then}\isamarkupfalse%
\ \isacommand{have}\isamarkupfalse%
\ {\isachardoublequoteopen}S{\isacharprime}\ {\isasymin}\ C{\isachardoublequoteclose}\isanewline
\ \ \ \ \ \ \ \ \isacommand{using}\isamarkupfalse%
\ {\isacartoucheopen}finite\ S{\isacharprime}{\isacartoucheclose}\ \isacommand{by}\isamarkupfalse%
\ {\isacharparenleft}rule\ mp{\isacharparenright}\isanewline
\ \ \ \ \ \ \isacommand{have}\isamarkupfalse%
\ {\isachardoublequoteopen}{\isasymbottom}\ {\isasymnotin}\ S{\isacharprime}\isanewline
\ \ \ \ \ \ \ \ \ \ \ \ {\isasymand}\ {\isacharparenleft}{\isasymforall}k{\isachardot}\ Atom\ k\ {\isasymin}\ S{\isacharprime}\ {\isasymlongrightarrow}\ \isactrlbold {\isasymnot}\ {\isacharparenleft}Atom\ k{\isacharparenright}\ {\isasymin}\ S{\isacharprime}\ {\isasymlongrightarrow}\ False{\isacharparenright}\isanewline
\ \ \ \ \ \ \ \ \ \ \ \ {\isasymand}\ {\isacharparenleft}{\isasymforall}F\ G\ H{\isachardot}\ Con\ F\ G\ H\ {\isasymlongrightarrow}\ F\ {\isasymin}\ S{\isacharprime}\ {\isasymlongrightarrow}\ {\isacharbraceleft}G{\isacharcomma}H{\isacharbraceright}\ {\isasymunion}\ S{\isacharprime}\ {\isasymin}\ C{\isacharparenright}\isanewline
\ \ \ \ \ \ \ \ \ \ \ \ {\isasymand}\ {\isacharparenleft}{\isasymforall}F\ G\ H{\isachardot}\ Dis\ F\ G\ H\ {\isasymlongrightarrow}\ F\ {\isasymin}\ S{\isacharprime}\ {\isasymlongrightarrow}\ {\isacharbraceleft}G{\isacharbraceright}\ {\isasymunion}\ S{\isacharprime}\ {\isasymin}\ C\ {\isasymor}\ {\isacharbraceleft}H{\isacharbraceright}\ {\isasymunion}\ S{\isacharprime}\ {\isasymin}\ C{\isacharparenright}{\isachardoublequoteclose}\isanewline
\ \ \ \ \ \ \ \ \isacommand{using}\isamarkupfalse%
\ PCP\ {\isacartoucheopen}S{\isacharprime}\ {\isasymin}\ C{\isacartoucheclose}\ \isacommand{by}\isamarkupfalse%
\ {\isacharparenleft}rule\ bspec{\isacharparenright}\isanewline
\ \ \ \ \ \ \isacommand{then}\isamarkupfalse%
\ \isacommand{have}\isamarkupfalse%
\ {\isachardoublequoteopen}{\isasymforall}F\ G\ H{\isachardot}\ Con\ F\ G\ H\ {\isasymlongrightarrow}\ F\ {\isasymin}\ S{\isacharprime}\ {\isasymlongrightarrow}\ {\isacharbraceleft}G{\isacharcomma}\ H{\isacharbraceright}\ {\isasymunion}\ S{\isacharprime}\ {\isasymin}\ C{\isachardoublequoteclose}\isanewline
\ \ \ \ \ \ \ \ \isacommand{by}\isamarkupfalse%
\ {\isacharparenleft}iprover\ elim{\isacharcolon}\ conjunct{\isadigit{2}}\ conjunct{\isadigit{1}}{\isacharparenright}\isanewline
\ \ \ \ \ \ \isacommand{then}\isamarkupfalse%
\ \isacommand{have}\isamarkupfalse%
\ {\isachardoublequoteopen}Con\ F\ G\ H\ {\isasymlongrightarrow}\ F\ {\isasymin}\ S{\isacharprime}\ {\isasymlongrightarrow}\ {\isacharbraceleft}G{\isacharcomma}\ H{\isacharbraceright}\ {\isasymunion}\ S{\isacharprime}\ {\isasymin}\ C{\isachardoublequoteclose}\isanewline
\ \ \ \ \ \ \ \ \isacommand{by}\isamarkupfalse%
\ {\isacharparenleft}iprover\ elim{\isacharcolon}\ allE{\isacharparenright}\isanewline
\ \ \ \ \ \ \isacommand{then}\isamarkupfalse%
\ \isacommand{have}\isamarkupfalse%
\ {\isachardoublequoteopen}F\ {\isasymin}\ S{\isacharprime}\ {\isasymlongrightarrow}\ {\isacharbraceleft}G{\isacharcomma}H{\isacharbraceright}\ {\isasymunion}\ S{\isacharprime}\ {\isasymin}\ C{\isachardoublequoteclose}\isanewline
\ \ \ \ \ \ \ \ \isacommand{using}\isamarkupfalse%
\ assms{\isacharparenleft}{\isadigit{4}}{\isacharparenright}\ \isacommand{by}\isamarkupfalse%
\ {\isacharparenleft}rule\ mp{\isacharparenright}\isanewline
\ \ \ \ \ \ \isacommand{thus}\isamarkupfalse%
\ {\isachardoublequoteopen}{\isacharbraceleft}G{\isacharcomma}\ H{\isacharbraceright}\ {\isasymunion}\ S{\isacharprime}\ {\isasymin}\ C{\isachardoublequoteclose}\isanewline
\ \ \ \ \ \ \ \ \isacommand{using}\isamarkupfalse%
\ {\isacartoucheopen}F\ {\isasymin}\ S{\isacharprime}{\isacartoucheclose}\ \isacommand{by}\isamarkupfalse%
\ {\isacharparenleft}rule\ mp{\isacharparenright}\isanewline
\ \ \ \ \isacommand{qed}\isamarkupfalse%
\isanewline
\ \ \isacommand{qed}\isamarkupfalse%
\isanewline
\ \ \isacommand{have}\isamarkupfalse%
\ {\isachardoublequoteopen}{\isacharbraceleft}G{\isacharcomma}H{\isacharbraceright}\ {\isasymunion}\ S\ {\isasymin}\ {\isacharparenleft}extF\ C{\isacharparenright}{\isachardoublequoteclose}\isanewline
\ \ \ \ \isacommand{unfolding}\isamarkupfalse%
\ mem{\isacharunderscore}Collect{\isacharunderscore}eq\ Un{\isacharunderscore}iff\ extF\isanewline
\ \ \isacommand{proof}\isamarkupfalse%
\ {\isacharparenleft}rule\ sallI{\isacharparenright}\isanewline
\ \ \ \ \isacommand{fix}\isamarkupfalse%
\ S{\isacharprime}\isanewline
\ \ \ \ \isacommand{assume}\isamarkupfalse%
\ H{\isacharcolon}{\isachardoublequoteopen}S{\isacharprime}\ {\isasymsubseteq}\ {\isacharbraceleft}G{\isacharcomma}H{\isacharbraceright}\ {\isasymunion}\ S{\isachardoublequoteclose}\isanewline
\ \ \ \ \isacommand{show}\isamarkupfalse%
\ {\isachardoublequoteopen}finite\ S{\isacharprime}\ {\isasymlongrightarrow}\ S{\isacharprime}\ {\isasymin}\ C{\isachardoublequoteclose}\isanewline
\ \ \ \ \isacommand{proof}\isamarkupfalse%
\ {\isacharparenleft}rule\ impI{\isacharparenright}\isanewline
\ \ \ \ \ \ \isacommand{assume}\isamarkupfalse%
\ {\isachardoublequoteopen}finite\ S{\isacharprime}{\isachardoublequoteclose}\isanewline
\ \ \ \ \ \ \isacommand{have}\isamarkupfalse%
\ {\isachardoublequoteopen}S{\isacharprime}\ {\isacharminus}\ {\isacharbraceleft}G{\isacharcomma}H{\isacharbraceright}\ {\isasymsubseteq}\ S{\isachardoublequoteclose}\isanewline
\ \ \ \ \ \ \ \ \isacommand{using}\isamarkupfalse%
\ H\ \isacommand{by}\isamarkupfalse%
\ {\isacharparenleft}simp\ only{\isacharcolon}\ Diff{\isacharunderscore}subset{\isacharunderscore}conv{\isacharparenright}\isanewline
\ \ \ \ \ \ \isacommand{have}\isamarkupfalse%
\ {\isachardoublequoteopen}F\ {\isasymin}\ S\ {\isasymand}\ {\isacharparenleft}S{\isacharprime}\ {\isacharminus}\ {\isacharbraceleft}G{\isacharcomma}H{\isacharbraceright}\ {\isasymsubseteq}\ S{\isacharparenright}{\isachardoublequoteclose}\isanewline
\ \ \ \ \ \ \ \ \isacommand{using}\isamarkupfalse%
\ assms{\isacharparenleft}{\isadigit{5}}{\isacharparenright}\ {\isacartoucheopen}S{\isacharprime}\ {\isacharminus}\ {\isacharbraceleft}G{\isacharcomma}H{\isacharbraceright}\ {\isasymsubseteq}\ S{\isacartoucheclose}\ \isacommand{by}\isamarkupfalse%
\ {\isacharparenleft}rule\ conjI{\isacharparenright}\isanewline
\ \ \ \ \ \ \isacommand{then}\isamarkupfalse%
\ \isacommand{have}\isamarkupfalse%
\ {\isachardoublequoteopen}insert\ F\ \ {\isacharparenleft}S{\isacharprime}\ {\isacharminus}\ {\isacharbraceleft}G{\isacharcomma}H{\isacharbraceright}{\isacharparenright}\ {\isasymsubseteq}\ S{\isachardoublequoteclose}\ \isanewline
\ \ \ \ \ \ \ \ \isacommand{by}\isamarkupfalse%
\ {\isacharparenleft}simp\ only{\isacharcolon}\ insert{\isacharunderscore}subset{\isacharparenright}\isanewline
\ \ \ \ \ \ \isacommand{have}\isamarkupfalse%
\ F{\isadigit{1}}{\isacharcolon}{\isachardoublequoteopen}finite\ {\isacharparenleft}insert\ F\ \ {\isacharparenleft}S{\isacharprime}\ {\isacharminus}\ {\isacharbraceleft}G{\isacharcomma}H{\isacharbraceright}{\isacharparenright}{\isacharparenright}\ {\isasymlongrightarrow}\ F\ {\isasymin}\ {\isacharparenleft}insert\ F\ \ {\isacharparenleft}S{\isacharprime}\ {\isacharminus}\ {\isacharbraceleft}G{\isacharcomma}H{\isacharbraceright}{\isacharparenright}{\isacharparenright}\ {\isasymlongrightarrow}\ {\isacharbraceleft}G{\isacharcomma}H{\isacharbraceright}\ {\isasymunion}\ {\isacharparenleft}insert\ F\ \ {\isacharparenleft}S{\isacharprime}\ {\isacharminus}\ {\isacharbraceleft}G{\isacharcomma}H{\isacharbraceright}{\isacharparenright}{\isacharparenright}\ {\isasymin}\ C{\isachardoublequoteclose}\isanewline
\ \ \ \ \ \ \ \ \isacommand{using}\isamarkupfalse%
\ {\isadigit{1}}\ {\isacartoucheopen}insert\ F\ \ {\isacharparenleft}S{\isacharprime}\ {\isacharminus}\ {\isacharbraceleft}G{\isacharcomma}H{\isacharbraceright}{\isacharparenright}\ {\isasymsubseteq}\ S{\isacartoucheclose}\ \isacommand{by}\isamarkupfalse%
\ {\isacharparenleft}rule\ sspec{\isacharparenright}\isanewline
\ \ \ \ \ \ \isacommand{have}\isamarkupfalse%
\ {\isachardoublequoteopen}finite\ {\isacharparenleft}S{\isacharprime}\ {\isacharminus}\ {\isacharbraceleft}G{\isacharcomma}H{\isacharbraceright}{\isacharparenright}{\isachardoublequoteclose}\isanewline
\ \ \ \ \ \ \ \ \isacommand{using}\isamarkupfalse%
\ {\isacartoucheopen}finite\ S{\isacharprime}{\isacartoucheclose}\ \isacommand{by}\isamarkupfalse%
\ {\isacharparenleft}rule\ finite{\isacharunderscore}Diff{\isacharparenright}\isanewline
\ \ \ \ \ \ \isacommand{then}\isamarkupfalse%
\ \isacommand{have}\isamarkupfalse%
\ {\isachardoublequoteopen}finite\ {\isacharparenleft}insert\ F\ {\isacharparenleft}S{\isacharprime}\ {\isacharminus}\ {\isacharbraceleft}G{\isacharcomma}H{\isacharbraceright}{\isacharparenright}{\isacharparenright}{\isachardoublequoteclose}\ \isanewline
\ \ \ \ \ \ \ \ \isacommand{by}\isamarkupfalse%
\ {\isacharparenleft}rule\ finite{\isachardot}insertI{\isacharparenright}\isanewline
\ \ \ \ \ \ \isacommand{have}\isamarkupfalse%
\ F{\isadigit{2}}{\isacharcolon}{\isachardoublequoteopen}F\ {\isasymin}\ {\isacharparenleft}insert\ F\ \ {\isacharparenleft}S{\isacharprime}\ {\isacharminus}\ {\isacharbraceleft}G{\isacharcomma}H{\isacharbraceright}{\isacharparenright}{\isacharparenright}\ {\isasymlongrightarrow}\ {\isacharbraceleft}G{\isacharcomma}H{\isacharbraceright}\ {\isasymunion}\ {\isacharparenleft}insert\ F\ \ {\isacharparenleft}S{\isacharprime}\ {\isacharminus}\ {\isacharbraceleft}G{\isacharcomma}H{\isacharbraceright}{\isacharparenright}{\isacharparenright}\ {\isasymin}\ C{\isachardoublequoteclose}\isanewline
\ \ \ \ \ \ \ \ \isacommand{using}\isamarkupfalse%
\ F{\isadigit{1}}\ {\isacartoucheopen}finite\ {\isacharparenleft}insert\ F\ {\isacharparenleft}S{\isacharprime}\ {\isacharminus}\ {\isacharbraceleft}G{\isacharcomma}H{\isacharbraceright}{\isacharparenright}{\isacharparenright}{\isacartoucheclose}\ \isacommand{by}\isamarkupfalse%
\ {\isacharparenleft}rule\ mp{\isacharparenright}\isanewline
\ \ \ \ \ \ \isacommand{have}\isamarkupfalse%
\ {\isachardoublequoteopen}F\ {\isasymin}\ {\isacharparenleft}insert\ F\ \ {\isacharparenleft}S{\isacharprime}\ {\isacharminus}\ {\isacharbraceleft}G{\isacharcomma}H{\isacharbraceright}{\isacharparenright}{\isacharparenright}{\isachardoublequoteclose}\isanewline
\ \ \ \ \ \ \ \ \isacommand{by}\isamarkupfalse%
\ {\isacharparenleft}simp\ only{\isacharcolon}\ insertI{\isadigit{1}}{\isacharparenright}\isanewline
\ \ \ \ \ \ \isacommand{have}\isamarkupfalse%
\ F{\isadigit{3}}{\isacharcolon}{\isachardoublequoteopen}{\isacharbraceleft}G{\isacharcomma}H{\isacharbraceright}\ {\isasymunion}\ {\isacharparenleft}insert\ F\ {\isacharparenleft}S{\isacharprime}\ {\isacharminus}\ {\isacharbraceleft}G{\isacharcomma}H{\isacharbraceright}{\isacharparenright}{\isacharparenright}\ {\isasymin}\ C{\isachardoublequoteclose}\isanewline
\ \ \ \ \ \ \ \ \isacommand{using}\isamarkupfalse%
\ F{\isadigit{2}}\ {\isacartoucheopen}F\ {\isasymin}\ insert\ F\ {\isacharparenleft}S{\isacharprime}\ {\isacharminus}\ {\isacharbraceleft}G{\isacharcomma}H{\isacharbraceright}{\isacharparenright}{\isacartoucheclose}\ \isacommand{by}\isamarkupfalse%
\ {\isacharparenleft}rule\ mp{\isacharparenright}\isanewline
\ \ \ \ \ \ \isacommand{have}\isamarkupfalse%
\ IU{\isadigit{1}}{\isacharcolon}{\isachardoublequoteopen}insert\ F\ {\isacharparenleft}S{\isacharprime}\ {\isacharminus}\ {\isacharbraceleft}G{\isacharcomma}H{\isacharbraceright}{\isacharparenright}\ {\isacharequal}\ {\isacharbraceleft}F{\isacharbraceright}\ {\isasymunion}\ {\isacharparenleft}S{\isacharprime}\ {\isacharminus}\ {\isacharbraceleft}G{\isacharcomma}H{\isacharbraceright}{\isacharparenright}{\isachardoublequoteclose}\isanewline
\ \ \ \ \ \ \ \ \isacommand{by}\isamarkupfalse%
\ {\isacharparenleft}rule\ insert{\isacharunderscore}is{\isacharunderscore}Un{\isacharparenright}\isanewline
\ \ \ \ \ \ \isacommand{have}\isamarkupfalse%
\ IU{\isadigit{2}}{\isacharcolon}{\isachardoublequoteopen}insert\ F\ {\isacharparenleft}{\isacharbraceleft}G{\isacharcomma}H{\isacharbraceright}\ {\isasymunion}\ S{\isacharprime}{\isacharparenright}\ {\isacharequal}\ {\isacharbraceleft}F{\isacharbraceright}\ {\isasymunion}\ {\isacharparenleft}{\isacharbraceleft}G{\isacharcomma}H{\isacharbraceright}\ {\isasymunion}\ S{\isacharprime}{\isacharparenright}{\isachardoublequoteclose}\isanewline
\ \ \ \ \ \ \ \ \isacommand{by}\isamarkupfalse%
\ {\isacharparenleft}rule\ insert{\isacharunderscore}is{\isacharunderscore}Un{\isacharparenright}\isanewline
\ \ \ \ \ \ \isacommand{have}\isamarkupfalse%
\ GH{\isacharcolon}{\isachardoublequoteopen}insert\ G\ {\isacharparenleft}insert\ H\ S{\isacharprime}{\isacharparenright}\ {\isacharequal}\ {\isacharbraceleft}G{\isacharcomma}H{\isacharbraceright}\ {\isasymunion}\ S{\isacharprime}{\isachardoublequoteclose}\isanewline
\ \ \ \ \ \ \ \ \isacommand{by}\isamarkupfalse%
\ {\isacharparenleft}rule\ insertSetElem{\isacharparenright}\isanewline
\ \ \ \ \ \ \isacommand{have}\isamarkupfalse%
\ {\isachardoublequoteopen}{\isacharbraceleft}G{\isacharcomma}H{\isacharbraceright}\ {\isasymunion}\ {\isacharparenleft}insert\ F\ {\isacharparenleft}S{\isacharprime}\ {\isacharminus}\ {\isacharbraceleft}G{\isacharcomma}H{\isacharbraceright}{\isacharparenright}{\isacharparenright}\ {\isacharequal}\ {\isacharbraceleft}G{\isacharcomma}H{\isacharbraceright}\ {\isasymunion}\ {\isacharparenleft}{\isacharbraceleft}F{\isacharbraceright}\ {\isasymunion}\ {\isacharparenleft}S{\isacharprime}\ {\isacharminus}\ {\isacharbraceleft}G{\isacharcomma}H{\isacharbraceright}{\isacharparenright}{\isacharparenright}{\isachardoublequoteclose}\isanewline
\ \ \ \ \ \ \ \ \isacommand{by}\isamarkupfalse%
\ {\isacharparenleft}simp\ only{\isacharcolon}\ IU{\isadigit{1}}{\isacharparenright}\isanewline
\ \ \ \ \ \ \isacommand{also}\isamarkupfalse%
\ \isacommand{have}\isamarkupfalse%
\ {\isachardoublequoteopen}{\isasymdots}\ {\isacharequal}\ {\isacharbraceleft}F{\isacharbraceright}\ {\isasymunion}\ {\isacharparenleft}{\isacharbraceleft}G{\isacharcomma}H{\isacharbraceright}\ {\isasymunion}\ {\isacharparenleft}S{\isacharprime}\ {\isacharminus}\ {\isacharbraceleft}G{\isacharcomma}H{\isacharbraceright}{\isacharparenright}{\isacharparenright}{\isachardoublequoteclose}\isanewline
\ \ \ \ \ \ \ \ \isacommand{by}\isamarkupfalse%
\ {\isacharparenleft}simp\ only{\isacharcolon}\ Un{\isacharunderscore}left{\isacharunderscore}commute{\isacharparenright}\isanewline
\ \ \ \ \ \ \isacommand{also}\isamarkupfalse%
\ \isacommand{have}\isamarkupfalse%
\ {\isachardoublequoteopen}{\isasymdots}\ {\isacharequal}\ {\isacharbraceleft}F{\isacharbraceright}\ {\isasymunion}\ {\isacharparenleft}{\isacharbraceleft}G{\isacharcomma}H{\isacharbraceright}\ {\isasymunion}\ S{\isacharprime}{\isacharparenright}{\isachardoublequoteclose}\isanewline
\ \ \ \ \ \ \ \ \isacommand{by}\isamarkupfalse%
\ {\isacharparenleft}simp\ only{\isacharcolon}\ Un{\isacharunderscore}Diff{\isacharunderscore}cancel{\isacharparenright}\isanewline
\ \ \ \ \ \ \isacommand{also}\isamarkupfalse%
\ \isacommand{have}\isamarkupfalse%
\ {\isachardoublequoteopen}{\isasymdots}\ {\isacharequal}\ insert\ F\ {\isacharparenleft}{\isacharbraceleft}G{\isacharcomma}H{\isacharbraceright}\ {\isasymunion}\ S{\isacharprime}{\isacharparenright}{\isachardoublequoteclose}\isanewline
\ \ \ \ \ \ \ \ \isacommand{by}\isamarkupfalse%
\ {\isacharparenleft}simp\ only{\isacharcolon}\ IU{\isadigit{2}}{\isacharparenright}\isanewline
\ \ \ \ \ \ \isacommand{also}\isamarkupfalse%
\ \isacommand{have}\isamarkupfalse%
\ {\isachardoublequoteopen}{\isasymdots}\ {\isacharequal}\ insert\ F\ {\isacharparenleft}insert\ G\ {\isacharparenleft}insert\ H\ S{\isacharprime}{\isacharparenright}{\isacharparenright}{\isachardoublequoteclose}\isanewline
\ \ \ \ \ \ \ \ \isacommand{by}\isamarkupfalse%
\ {\isacharparenleft}simp\ only{\isacharcolon}\ GH{\isacharparenright}\isanewline
\ \ \ \ \ \ \isacommand{finally}\isamarkupfalse%
\ \isacommand{have}\isamarkupfalse%
\ F{\isadigit{4}}{\isacharcolon}{\isachardoublequoteopen}{\isacharbraceleft}G{\isacharcomma}H{\isacharbraceright}\ {\isasymunion}\ {\isacharparenleft}insert\ F\ {\isacharparenleft}S{\isacharprime}\ {\isacharminus}\ {\isacharbraceleft}G{\isacharcomma}H{\isacharbraceright}{\isacharparenright}{\isacharparenright}\ {\isacharequal}\ insert\ F\ {\isacharparenleft}insert\ G\ {\isacharparenleft}insert\ H\ S{\isacharprime}{\isacharparenright}{\isacharparenright}{\isachardoublequoteclose}\isanewline
\ \ \ \ \ \ \ \ \isacommand{by}\isamarkupfalse%
\ this\isanewline
\ \ \ \ \ \ \isacommand{have}\isamarkupfalse%
\ C{\isadigit{1}}{\isacharcolon}{\isachardoublequoteopen}insert\ F\ {\isacharparenleft}insert\ G\ {\isacharparenleft}insert\ H\ S{\isacharprime}{\isacharparenright}{\isacharparenright}\ {\isasymin}\ C{\isachardoublequoteclose}\isanewline
\ \ \ \ \ \ \ \ \isacommand{using}\isamarkupfalse%
\ F{\isadigit{3}}\ \isacommand{by}\isamarkupfalse%
\ {\isacharparenleft}simp\ only{\isacharcolon}\ F{\isadigit{4}}{\isacharparenright}\isanewline
\ \ \ \ \ \ \isacommand{have}\isamarkupfalse%
\ {\isachardoublequoteopen}S{\isacharprime}\ {\isasymsubseteq}\ insert\ F\ S{\isacharprime}{\isachardoublequoteclose}\isanewline
\ \ \ \ \ \ \ \ \isacommand{by}\isamarkupfalse%
\ {\isacharparenleft}rule\ subset{\isacharunderscore}insertI{\isacharparenright}\isanewline
\ \ \ \ \ \ \isacommand{then}\isamarkupfalse%
\ \isacommand{have}\isamarkupfalse%
\ C{\isadigit{2}}{\isacharcolon}{\isachardoublequoteopen}S{\isacharprime}\ {\isasymsubseteq}\ insert\ F\ {\isacharparenleft}insert\ G\ {\isacharparenleft}insert\ H\ S{\isacharprime}{\isacharparenright}{\isacharparenright}{\isachardoublequoteclose}\isanewline
\ \ \ \ \ \ \ \ \isacommand{by}\isamarkupfalse%
\ {\isacharparenleft}simp\ only{\isacharcolon}\ subset{\isacharunderscore}insertI{\isadigit{2}}{\isacharparenright}\isanewline
\ \ \ \ \ \ \isacommand{let}\isamarkupfalse%
\ {\isacharquery}S{\isacharequal}{\isachardoublequoteopen}insert\ F\ {\isacharparenleft}insert\ G\ {\isacharparenleft}insert\ H\ S{\isacharprime}{\isacharparenright}{\isacharparenright}{\isachardoublequoteclose}\isanewline
\ \ \ \ \ \ \isacommand{have}\isamarkupfalse%
\ {\isachardoublequoteopen}{\isasymforall}S\ {\isasymin}\ C{\isachardot}\ {\isasymforall}S{\isacharprime}\ {\isasymsubseteq}\ S{\isachardot}\ S{\isacharprime}\ {\isasymin}\ C{\isachardoublequoteclose}\isanewline
\ \ \ \ \ \ \ \ \isacommand{using}\isamarkupfalse%
\ assms{\isacharparenleft}{\isadigit{2}}{\isacharparenright}\ \isacommand{by}\isamarkupfalse%
\ {\isacharparenleft}simp\ only{\isacharcolon}\ subset{\isacharunderscore}closed{\isacharunderscore}def{\isacharparenright}\isanewline
\ \ \ \ \ \ \isacommand{then}\isamarkupfalse%
\ \isacommand{have}\isamarkupfalse%
\ {\isachardoublequoteopen}{\isasymforall}S{\isacharprime}\ {\isasymsubseteq}\ {\isacharquery}S{\isachardot}\ S{\isacharprime}\ {\isasymin}\ C{\isachardoublequoteclose}\isanewline
\ \ \ \ \ \ \ \ \isacommand{using}\isamarkupfalse%
\ C{\isadigit{1}}\ \isacommand{by}\isamarkupfalse%
\ {\isacharparenleft}rule\ bspec{\isacharparenright}\isanewline
\ \ \ \ \ \ \isacommand{thus}\isamarkupfalse%
\ {\isachardoublequoteopen}S{\isacharprime}\ {\isasymin}\ C{\isachardoublequoteclose}\isanewline
\ \ \ \ \ \ \ \ \isacommand{using}\isamarkupfalse%
\ C{\isadigit{2}}\ \isacommand{by}\isamarkupfalse%
\ {\isacharparenleft}rule\ sspec{\isacharparenright}\isanewline
\ \ \ \ \isacommand{qed}\isamarkupfalse%
\isanewline
\ \ \isacommand{qed}\isamarkupfalse%
\isanewline
\ \ \isacommand{thus}\isamarkupfalse%
\ {\isachardoublequoteopen}{\isacharbraceleft}G{\isacharcomma}H{\isacharbraceright}\ {\isasymunion}\ S\ {\isasymin}\ {\isacharparenleft}extensionFin\ C{\isacharparenright}{\isachardoublequoteclose}\isanewline
\ \ \ \ \isacommand{unfolding}\isamarkupfalse%
\ extensionFin\ \isacommand{by}\isamarkupfalse%
\ {\isacharparenleft}rule\ UnI{\isadigit{2}}{\isacharparenright}\isanewline
\isacommand{qed}\isamarkupfalse%
%
\endisatagproof
{\isafoldproof}%
%
\isadelimproof
%
\endisadelimproof
%
\begin{isamarkuptext}%
Seguidamente, vamos a probar el lema auxiliar \isa{ex{\isadigit{3}}{\isacharunderscore}pcp{\isacharunderscore}SinE{\isacharunderscore}DIS}. Este demuestra que si \isa{C} es 
  una colección con la propiedad de consistencia proposicional y cerrada bajo subconjuntos, \isa{S\ {\isasymin}\ E}
  y sea \isa{F} una fórmula de tipo \isa{{\isasymbeta}} con componentes \isa{{\isasymbeta}\isactrlsub {\isadigit{1}}} y \isa{{\isasymbeta}\isactrlsub {\isadigit{2}}}, se verifica que o bien 
  \isa{{\isacharbraceleft}{\isasymbeta}\isactrlsub {\isadigit{1}}{\isacharbraceright}\ {\isasymunion}\ S\ {\isasymin}\ C{\isacharprime}} o bien \isa{{\isacharbraceleft}{\isasymbeta}\isactrlsub {\isadigit{2}}{\isacharbraceright}\ {\isasymunion}\ S\ {\isasymin}\ C{\isacharprime}}. Dicha prueba se realizará por reducción al absurdo. Para
  ello precisaremos de dos lemas previos que nos permitan llegar a una contradicción: 
  \isa{ex{\isadigit{3}}{\isacharunderscore}pcp{\isacharunderscore}SinE{\isacharunderscore}DIS{\isacharunderscore}auxEx} y \isa{ex{\isadigit{3}}{\isacharunderscore}pcp{\isacharunderscore}SinE{\isacharunderscore}DIS{\isacharunderscore}auxFalse}.

  En primer lugar, veamos la demostración del lema \isa{ex{\isadigit{3}}{\isacharunderscore}pcp{\isacharunderscore}SinE{\isacharunderscore}DIS{\isacharunderscore}auxEx}. Este prueba que dada 
  \isa{C} una colección con la propiedad de consistencia proposicional y cerrada bajo subconjuntos, 
  \isa{S\ {\isasymin}\ E} y sea \isa{F} es una fórmula de tipo \isa{{\isasymbeta}} de componentes \isa{{\isasymbeta}\isactrlsub {\isadigit{1}}} y \isa{{\isasymbeta}\isactrlsub {\isadigit{2}}}, si consideramos \isa{S\isactrlsub {\isadigit{1}}} y 
  \isa{S\isactrlsub {\isadigit{2}}} subconjuntos finitos cualesquiera de \isa{S} tales que \isa{F\ {\isasymin}\ S\isactrlsub {\isadigit{1}}} y\\ \isa{F\ {\isasymin}\ S\isactrlsub {\isadigit{2}}}, entonces existe una 
  fórmula \isa{I\ {\isasymin}\ {\isacharbraceleft}{\isasymbeta}\isactrlsub {\isadigit{1}}{\isacharcomma}{\isasymbeta}\isactrlsub {\isadigit{2}}{\isacharbraceright}} tal que se verifica que tanto \isa{{\isacharbraceleft}I{\isacharbraceright}\ {\isasymunion}\ S\isactrlsub {\isadigit{1}}} como \isa{{\isacharbraceleft}I{\isacharbraceright}\ {\isasymunion}\ S\isactrlsub {\isadigit{2}}} están en \isa{C}.%
\end{isamarkuptext}\isamarkuptrue%
\isacommand{lemma}\isamarkupfalse%
\ ex{\isadigit{3}}{\isacharunderscore}pcp{\isacharunderscore}SinE{\isacharunderscore}DIS{\isacharunderscore}auxEx{\isacharcolon}\isanewline
\ \ \isakeyword{assumes}\ {\isachardoublequoteopen}pcp\ C{\isachardoublequoteclose}\isanewline
\ \ \ \ \ \ \ \ \ \ {\isachardoublequoteopen}subset{\isacharunderscore}closed\ C{\isachardoublequoteclose}\isanewline
\ \ \ \ \ \ \ \ \ \ {\isachardoublequoteopen}S\ {\isasymin}\ {\isacharparenleft}extF\ C{\isacharparenright}{\isachardoublequoteclose}\isanewline
\ \ \ \ \ \ \ \ \ \ {\isachardoublequoteopen}Dis\ F\ G\ H{\isachardoublequoteclose}\isanewline
\ \ \ \ \ \ \ \ \ \ {\isachardoublequoteopen}S{\isadigit{1}}\ {\isasymsubseteq}\ S{\isachardoublequoteclose}\isanewline
\ \ \ \ \ \ \ \ \ \ {\isachardoublequoteopen}finite\ S{\isadigit{1}}{\isachardoublequoteclose}\isanewline
\ \ \ \ \ \ \ \ \ \ {\isachardoublequoteopen}F\ {\isasymin}\ S{\isadigit{1}}{\isachardoublequoteclose}\isanewline
\ \ \ \ \ \ \ \ \ \ {\isachardoublequoteopen}S{\isadigit{2}}\ {\isasymsubseteq}\ S{\isachardoublequoteclose}\isanewline
\ \ \ \ \ \ \ \ \ \ {\isachardoublequoteopen}finite\ S{\isadigit{2}}{\isachardoublequoteclose}\isanewline
\ \ \ \ \ \ \ \ \ \ {\isachardoublequoteopen}F\ {\isasymin}\ S{\isadigit{2}}{\isachardoublequoteclose}\isanewline
\ \ \isakeyword{shows}\ {\isachardoublequoteopen}{\isasymexists}I{\isasymin}{\isacharbraceleft}G{\isacharcomma}H{\isacharbraceright}{\isachardot}\ insert\ I\ S{\isadigit{1}}\ {\isasymin}\ C\ {\isasymand}\ insert\ I\ S{\isadigit{2}}\ {\isasymin}\ C{\isachardoublequoteclose}\ \isanewline
%
\isadelimproof
%
\endisadelimproof
%
\isatagproof
\isacommand{proof}\isamarkupfalse%
\ {\isacharminus}\isanewline
\ \ \isacommand{let}\isamarkupfalse%
\ {\isacharquery}S\ {\isacharequal}\ {\isachardoublequoteopen}S{\isadigit{1}}\ {\isasymunion}\ S{\isadigit{2}}{\isachardoublequoteclose}\isanewline
\ \ \isacommand{have}\isamarkupfalse%
\ {\isachardoublequoteopen}S{\isadigit{1}}\ {\isasymsubseteq}\ {\isacharquery}S{\isachardoublequoteclose}\isanewline
\ \ \ \ \isacommand{by}\isamarkupfalse%
\ {\isacharparenleft}simp\ only{\isacharcolon}\ Un{\isacharunderscore}upper{\isadigit{1}}{\isacharparenright}\isanewline
\ \ \isacommand{have}\isamarkupfalse%
\ {\isachardoublequoteopen}S{\isadigit{2}}\ {\isasymsubseteq}\ {\isacharquery}S{\isachardoublequoteclose}\isanewline
\ \ \ \ \isacommand{by}\isamarkupfalse%
\ {\isacharparenleft}simp\ only{\isacharcolon}\ Un{\isacharunderscore}upper{\isadigit{2}}{\isacharparenright}\isanewline
\ \ \isacommand{have}\isamarkupfalse%
\ {\isachardoublequoteopen}finite\ {\isacharquery}S{\isachardoublequoteclose}\isanewline
\ \ \ \ \isacommand{using}\isamarkupfalse%
\ assms{\isacharparenleft}{\isadigit{6}}{\isacharparenright}\ assms{\isacharparenleft}{\isadigit{9}}{\isacharparenright}\ \isacommand{by}\isamarkupfalse%
\ {\isacharparenleft}rule\ finite{\isacharunderscore}UnI{\isacharparenright}\isanewline
\ \ \isacommand{have}\isamarkupfalse%
\ {\isachardoublequoteopen}{\isacharquery}S\ {\isasymsubseteq}\ S{\isachardoublequoteclose}\ \isanewline
\ \ \ \ \isacommand{using}\isamarkupfalse%
\ assms{\isacharparenleft}{\isadigit{5}}{\isacharparenright}\ assms{\isacharparenleft}{\isadigit{8}}{\isacharparenright}\ \isacommand{by}\isamarkupfalse%
\ {\isacharparenleft}simp\ only{\isacharcolon}\ Un{\isacharunderscore}subset{\isacharunderscore}iff{\isacharparenright}\isanewline
\ \ \isacommand{have}\isamarkupfalse%
\ {\isachardoublequoteopen}{\isasymforall}S{\isacharprime}\ {\isasymsubseteq}\ S{\isachardot}\ finite\ S{\isacharprime}\ {\isasymlongrightarrow}\ S{\isacharprime}\ {\isasymin}\ C{\isachardoublequoteclose}\isanewline
\ \ \ \ \isacommand{using}\isamarkupfalse%
\ assms{\isacharparenleft}{\isadigit{3}}{\isacharparenright}\ \isacommand{unfolding}\isamarkupfalse%
\ extF\ \isacommand{by}\isamarkupfalse%
\ {\isacharparenleft}rule\ CollectD{\isacharparenright}\isanewline
\ \ \isacommand{then}\isamarkupfalse%
\ \isacommand{have}\isamarkupfalse%
\ {\isachardoublequoteopen}finite\ {\isacharquery}S\ {\isasymlongrightarrow}\ {\isacharquery}S\ {\isasymin}\ C{\isachardoublequoteclose}\isanewline
\ \ \ \ \isacommand{using}\isamarkupfalse%
\ {\isacartoucheopen}{\isacharquery}S\ {\isasymsubseteq}\ S{\isacartoucheclose}\ \isacommand{by}\isamarkupfalse%
\ {\isacharparenleft}rule\ sspec{\isacharparenright}\isanewline
\ \ \isacommand{then}\isamarkupfalse%
\ \isacommand{have}\isamarkupfalse%
\ {\isachardoublequoteopen}{\isacharquery}S\ {\isasymin}\ C{\isachardoublequoteclose}\ \isanewline
\ \ \ \ \isacommand{using}\isamarkupfalse%
\ {\isacartoucheopen}finite\ {\isacharquery}S{\isacartoucheclose}\ \isacommand{by}\isamarkupfalse%
\ {\isacharparenleft}rule\ mp{\isacharparenright}\isanewline
\ \ \isacommand{have}\isamarkupfalse%
\ {\isachardoublequoteopen}F\ {\isasymin}\ {\isacharquery}S{\isachardoublequoteclose}\ \isanewline
\ \ \ \ \isacommand{using}\isamarkupfalse%
\ assms{\isacharparenleft}{\isadigit{7}}{\isacharparenright}\ \isacommand{by}\isamarkupfalse%
\ {\isacharparenleft}rule\ UnI{\isadigit{1}}{\isacharparenright}\isanewline
\ \ \isacommand{have}\isamarkupfalse%
\ {\isachardoublequoteopen}{\isasymforall}S\ {\isasymin}\ C{\isachardot}\ {\isasymbottom}\ {\isasymnotin}\ S\isanewline
\ \ {\isasymand}\ {\isacharparenleft}{\isasymforall}k{\isachardot}\ Atom\ k\ {\isasymin}\ S\ {\isasymlongrightarrow}\ \isactrlbold {\isasymnot}\ {\isacharparenleft}Atom\ k{\isacharparenright}\ {\isasymin}\ S\ {\isasymlongrightarrow}\ False{\isacharparenright}\isanewline
\ \ {\isasymand}\ {\isacharparenleft}{\isasymforall}F\ G\ H{\isachardot}\ Con\ F\ G\ H\ {\isasymlongrightarrow}\ F\ {\isasymin}\ S\ {\isasymlongrightarrow}\ {\isacharbraceleft}G{\isacharcomma}H{\isacharbraceright}\ {\isasymunion}\ S\ {\isasymin}\ C{\isacharparenright}\isanewline
\ \ {\isasymand}\ {\isacharparenleft}{\isasymforall}F\ G\ H{\isachardot}\ Dis\ F\ G\ H\ {\isasymlongrightarrow}\ F\ {\isasymin}\ S\ {\isasymlongrightarrow}\ {\isacharbraceleft}G{\isacharbraceright}\ {\isasymunion}\ S\ {\isasymin}\ C\ {\isasymor}\ {\isacharbraceleft}H{\isacharbraceright}\ {\isasymunion}\ S\ {\isasymin}\ C{\isacharparenright}{\isachardoublequoteclose}\isanewline
\ \ \ \ \isacommand{using}\isamarkupfalse%
\ assms{\isacharparenleft}{\isadigit{1}}{\isacharparenright}\ \isacommand{by}\isamarkupfalse%
\ {\isacharparenleft}rule\ pcp{\isacharunderscore}alt{\isadigit{1}}{\isacharparenright}\isanewline
\ \ \isacommand{then}\isamarkupfalse%
\ \isacommand{have}\isamarkupfalse%
\ {\isachardoublequoteopen}{\isasymbottom}\ {\isasymnotin}\ {\isacharquery}S\isanewline
\ \ \ \ \ \ \ \ {\isasymand}\ {\isacharparenleft}{\isasymforall}k{\isachardot}\ Atom\ k\ {\isasymin}\ {\isacharquery}S\ {\isasymlongrightarrow}\ \isactrlbold {\isasymnot}\ {\isacharparenleft}Atom\ k{\isacharparenright}\ {\isasymin}\ {\isacharquery}S\ {\isasymlongrightarrow}\ False{\isacharparenright}\isanewline
\ \ \ \ \ \ \ \ {\isasymand}\ {\isacharparenleft}{\isasymforall}F\ G\ H{\isachardot}\ Con\ F\ G\ H\ {\isasymlongrightarrow}\ F\ {\isasymin}\ {\isacharquery}S\ {\isasymlongrightarrow}\ {\isacharbraceleft}G{\isacharcomma}H{\isacharbraceright}\ {\isasymunion}\ {\isacharquery}S\ {\isasymin}\ C{\isacharparenright}\isanewline
\ \ \ \ \ \ \ \ {\isasymand}\ {\isacharparenleft}{\isasymforall}F\ G\ H{\isachardot}\ Dis\ F\ G\ H\ {\isasymlongrightarrow}\ F\ {\isasymin}\ {\isacharquery}S\ {\isasymlongrightarrow}\ {\isacharbraceleft}G{\isacharbraceright}\ {\isasymunion}\ {\isacharquery}S\ {\isasymin}\ C\ {\isasymor}\ {\isacharbraceleft}H{\isacharbraceright}\ {\isasymunion}\ {\isacharquery}S\ {\isasymin}\ C{\isacharparenright}{\isachardoublequoteclose}\isanewline
\ \ \ \ \isacommand{using}\isamarkupfalse%
\ {\isacartoucheopen}{\isacharquery}S\ {\isasymin}\ C{\isacartoucheclose}\ \isacommand{by}\isamarkupfalse%
\ {\isacharparenleft}rule\ bspec{\isacharparenright}\isanewline
\ \ \isacommand{then}\isamarkupfalse%
\ \isacommand{have}\isamarkupfalse%
\ {\isachardoublequoteopen}{\isasymforall}F\ G\ H{\isachardot}\ Dis\ F\ G\ H\ {\isasymlongrightarrow}\ F\ {\isasymin}\ {\isacharquery}S\ {\isasymlongrightarrow}\ {\isacharbraceleft}G{\isacharbraceright}\ {\isasymunion}\ {\isacharquery}S\ {\isasymin}\ C\ {\isasymor}\ {\isacharbraceleft}H{\isacharbraceright}\ {\isasymunion}\ {\isacharquery}S\ {\isasymin}\ C{\isachardoublequoteclose}\isanewline
\ \ \ \ \isacommand{by}\isamarkupfalse%
\ {\isacharparenleft}iprover\ elim{\isacharcolon}\ conjunct{\isadigit{2}}{\isacharparenright}\isanewline
\ \ \isacommand{then}\isamarkupfalse%
\ \isacommand{have}\isamarkupfalse%
\ {\isachardoublequoteopen}Dis\ F\ G\ H\ {\isasymlongrightarrow}\ F\ {\isasymin}\ {\isacharquery}S\ {\isasymlongrightarrow}\ {\isacharbraceleft}G{\isacharbraceright}\ {\isasymunion}\ {\isacharquery}S\ {\isasymin}\ C\ {\isasymor}\ {\isacharbraceleft}H{\isacharbraceright}\ {\isasymunion}\ {\isacharquery}S\ {\isasymin}\ C{\isachardoublequoteclose}\isanewline
\ \ \ \ \isacommand{by}\isamarkupfalse%
\ {\isacharparenleft}iprover\ elim{\isacharcolon}\ allE{\isacharparenright}\isanewline
\ \ \isacommand{then}\isamarkupfalse%
\ \isacommand{have}\isamarkupfalse%
\ {\isachardoublequoteopen}F\ {\isasymin}\ {\isacharquery}S\ {\isasymlongrightarrow}\ {\isacharbraceleft}G{\isacharbraceright}\ {\isasymunion}\ {\isacharquery}S\ {\isasymin}\ C\ {\isasymor}\ {\isacharbraceleft}H{\isacharbraceright}\ {\isasymunion}\ {\isacharquery}S\ {\isasymin}\ C{\isachardoublequoteclose}\isanewline
\ \ \ \ \isacommand{using}\isamarkupfalse%
\ assms{\isacharparenleft}{\isadigit{4}}{\isacharparenright}\ \isacommand{by}\isamarkupfalse%
\ {\isacharparenleft}rule\ mp{\isacharparenright}\isanewline
\ \ \isacommand{then}\isamarkupfalse%
\ \isacommand{have}\isamarkupfalse%
\ insIsUn{\isacharcolon}{\isachardoublequoteopen}{\isacharbraceleft}G{\isacharbraceright}\ {\isasymunion}\ {\isacharquery}S\ {\isasymin}\ C\ {\isasymor}\ {\isacharbraceleft}H{\isacharbraceright}\ {\isasymunion}\ {\isacharquery}S\ {\isasymin}\ C{\isachardoublequoteclose}\isanewline
\ \ \ \ \isacommand{using}\isamarkupfalse%
\ {\isacartoucheopen}F\ {\isasymin}\ {\isacharquery}S{\isacartoucheclose}\ \isacommand{by}\isamarkupfalse%
\ {\isacharparenleft}rule\ mp{\isacharparenright}\isanewline
\ \ \isacommand{have}\isamarkupfalse%
\ insG{\isacharcolon}{\isachardoublequoteopen}insert\ G\ {\isacharquery}S\ {\isacharequal}\ {\isacharbraceleft}G{\isacharbraceright}\ {\isasymunion}\ {\isacharquery}S{\isachardoublequoteclose}\ \isanewline
\ \ \ \ \isacommand{by}\isamarkupfalse%
\ {\isacharparenleft}rule\ insert{\isacharunderscore}is{\isacharunderscore}Un{\isacharparenright}\isanewline
\ \ \isacommand{have}\isamarkupfalse%
\ insH{\isacharcolon}{\isachardoublequoteopen}insert\ H\ {\isacharquery}S\ {\isacharequal}\ {\isacharbraceleft}H{\isacharbraceright}\ {\isasymunion}\ {\isacharquery}S{\isachardoublequoteclose}\isanewline
\ \ \ \ \isacommand{by}\isamarkupfalse%
\ {\isacharparenleft}rule\ insert{\isacharunderscore}is{\isacharunderscore}Un{\isacharparenright}\isanewline
\ \ \isacommand{have}\isamarkupfalse%
\ {\isachardoublequoteopen}insert\ G\ {\isacharquery}S\ {\isasymin}\ C\ {\isasymor}\ insert\ H\ {\isacharquery}S\ {\isasymin}\ C{\isachardoublequoteclose}\isanewline
\ \ \ \ \isacommand{using}\isamarkupfalse%
\ insG\ insH\ \isacommand{by}\isamarkupfalse%
\ {\isacharparenleft}simp\ only{\isacharcolon}\ insIsUn{\isacharparenright}\isanewline
\ \ \isacommand{then}\isamarkupfalse%
\ \isacommand{have}\isamarkupfalse%
\ {\isachardoublequoteopen}{\isacharparenleft}insert\ G\ {\isacharquery}S\ {\isasymin}\ C\ {\isasymor}\ insert\ H\ {\isacharquery}S\ {\isasymin}\ C{\isacharparenright}\ {\isasymor}\ {\isacharparenleft}{\isasymexists}I\ {\isasymin}\ {\isacharbraceleft}{\isacharbraceright}{\isachardot}\ insert\ I\ {\isacharquery}S\ {\isasymin}\ C{\isacharparenright}{\isachardoublequoteclose}\isanewline
\ \ \ \ \isacommand{by}\isamarkupfalse%
\ {\isacharparenleft}simp\ only{\isacharcolon}\ disjI{\isadigit{1}}{\isacharparenright}\isanewline
\ \ \isacommand{then}\isamarkupfalse%
\ \isacommand{have}\isamarkupfalse%
\ {\isachardoublequoteopen}insert\ G\ {\isacharquery}S\ {\isasymin}\ C\ {\isasymor}\ {\isacharparenleft}insert\ H\ {\isacharquery}S\ {\isasymin}\ C\ {\isasymor}\ {\isacharparenleft}{\isasymexists}I\ {\isasymin}\ {\isacharbraceleft}{\isacharbraceright}{\isachardot}\ insert\ I\ {\isacharquery}S\ {\isasymin}\ C{\isacharparenright}{\isacharparenright}{\isachardoublequoteclose}\isanewline
\ \ \ \ \isacommand{by}\isamarkupfalse%
\ {\isacharparenleft}simp\ only{\isacharcolon}\ disj{\isacharunderscore}assoc{\isacharparenright}\isanewline
\ \ \isacommand{then}\isamarkupfalse%
\ \isacommand{have}\isamarkupfalse%
\ {\isachardoublequoteopen}insert\ G\ {\isacharquery}S\ {\isasymin}\ C\ {\isasymor}\ {\isacharparenleft}{\isasymexists}I\ {\isasymin}\ {\isacharbraceleft}H{\isacharbraceright}{\isachardot}\ insert\ I\ {\isacharquery}S\ {\isasymin}\ C{\isacharparenright}{\isachardoublequoteclose}\isanewline
\ \ \ \ \isacommand{by}\isamarkupfalse%
\ {\isacharparenleft}simp\ only{\isacharcolon}\ bex{\isacharunderscore}simps{\isacharparenleft}{\isadigit{5}}{\isacharparenright}{\isacharparenright}\isanewline
\ \ \isacommand{then}\isamarkupfalse%
\ \isacommand{have}\isamarkupfalse%
\ {\isadigit{1}}{\isacharcolon}{\isachardoublequoteopen}{\isasymexists}I\ {\isasymin}\ {\isacharbraceleft}G{\isacharcomma}H{\isacharbraceright}{\isachardot}\ insert\ I\ {\isacharquery}S\ {\isasymin}\ C{\isachardoublequoteclose}\ \isanewline
\ \ \ \ \isacommand{by}\isamarkupfalse%
\ {\isacharparenleft}simp\ only{\isacharcolon}\ bex{\isacharunderscore}simps{\isacharparenleft}{\isadigit{5}}{\isacharparenright}{\isacharparenright}\isanewline
\ \ \isacommand{obtain}\isamarkupfalse%
\ I\ \isakeyword{where}\ {\isachardoublequoteopen}I\ {\isasymin}\ {\isacharbraceleft}G{\isacharcomma}H{\isacharbraceright}{\isachardoublequoteclose}\ \isakeyword{and}\ {\isachardoublequoteopen}insert\ I\ {\isacharquery}S\ {\isasymin}\ C{\isachardoublequoteclose}\isanewline
\ \ \ \ \isacommand{using}\isamarkupfalse%
\ {\isadigit{1}}\ \isacommand{by}\isamarkupfalse%
\ {\isacharparenleft}rule\ bexE{\isacharparenright}\isanewline
\ \ \isacommand{have}\isamarkupfalse%
\ SC{\isacharcolon}{\isachardoublequoteopen}{\isasymforall}S\ {\isasymin}\ C{\isachardot}\ {\isasymforall}S{\isacharprime}{\isasymsubseteq}S{\isachardot}\ S{\isacharprime}\ {\isasymin}\ C{\isachardoublequoteclose}\isanewline
\ \ \ \ \isacommand{using}\isamarkupfalse%
\ assms{\isacharparenleft}{\isadigit{2}}{\isacharparenright}\ \isacommand{by}\isamarkupfalse%
\ {\isacharparenleft}simp\ only{\isacharcolon}\ subset{\isacharunderscore}closed{\isacharunderscore}def{\isacharparenright}\isanewline
\ \ \isacommand{then}\isamarkupfalse%
\ \isacommand{have}\isamarkupfalse%
\ {\isadigit{2}}{\isacharcolon}{\isachardoublequoteopen}{\isasymforall}S{\isacharprime}\ {\isasymsubseteq}\ {\isacharparenleft}insert\ I\ {\isacharquery}S{\isacharparenright}{\isachardot}\ S{\isacharprime}\ {\isasymin}\ C{\isachardoublequoteclose}\isanewline
\ \ \ \ \isacommand{using}\isamarkupfalse%
\ {\isacartoucheopen}insert\ I\ {\isacharquery}S\ {\isasymin}\ C{\isacartoucheclose}\ \isacommand{by}\isamarkupfalse%
\ {\isacharparenleft}rule\ bspec{\isacharparenright}\isanewline
\ \ \isacommand{have}\isamarkupfalse%
\ {\isachardoublequoteopen}insert\ I\ S{\isadigit{1}}\ {\isasymsubseteq}\ insert\ I\ {\isacharquery}S{\isachardoublequoteclose}\ \isanewline
\ \ \ \ \isacommand{using}\isamarkupfalse%
\ {\isacartoucheopen}S{\isadigit{1}}\ {\isasymsubseteq}\ {\isacharquery}S{\isacartoucheclose}\ \isacommand{by}\isamarkupfalse%
\ {\isacharparenleft}rule\ insert{\isacharunderscore}mono{\isacharparenright}\isanewline
\ \ \isacommand{have}\isamarkupfalse%
\ {\isachardoublequoteopen}insert\ I\ S{\isadigit{1}}\ {\isasymin}\ C{\isachardoublequoteclose}\isanewline
\ \ \ \ \isacommand{using}\isamarkupfalse%
\ {\isadigit{2}}\ {\isacartoucheopen}insert\ I\ S{\isadigit{1}}\ {\isasymsubseteq}\ insert\ I\ {\isacharquery}S{\isacartoucheclose}\ \isacommand{by}\isamarkupfalse%
\ {\isacharparenleft}rule\ sspec{\isacharparenright}\isanewline
\ \ \isacommand{have}\isamarkupfalse%
\ {\isachardoublequoteopen}insert\ I\ S{\isadigit{2}}\ {\isasymsubseteq}\ insert\ I\ {\isacharquery}S{\isachardoublequoteclose}\isanewline
\ \ \ \ \isacommand{using}\isamarkupfalse%
\ {\isacartoucheopen}S{\isadigit{2}}\ {\isasymsubseteq}\ {\isacharquery}S{\isacartoucheclose}\ \isacommand{by}\isamarkupfalse%
\ {\isacharparenleft}rule\ insert{\isacharunderscore}mono{\isacharparenright}\isanewline
\ \ \isacommand{have}\isamarkupfalse%
\ {\isachardoublequoteopen}insert\ I\ S{\isadigit{2}}\ {\isasymin}\ C{\isachardoublequoteclose}\isanewline
\ \ \ \ \isacommand{using}\isamarkupfalse%
\ {\isadigit{2}}\ {\isacartoucheopen}insert\ I\ S{\isadigit{2}}\ {\isasymsubseteq}\ insert\ I\ {\isacharquery}S{\isacartoucheclose}\ \isacommand{by}\isamarkupfalse%
\ {\isacharparenleft}rule\ sspec{\isacharparenright}\isanewline
\ \ \isacommand{have}\isamarkupfalse%
\ {\isachardoublequoteopen}insert\ I\ S{\isadigit{1}}\ {\isasymin}\ C\ {\isasymand}\ insert\ I\ S{\isadigit{2}}\ {\isasymin}\ C{\isachardoublequoteclose}\isanewline
\ \ \ \ \isacommand{using}\isamarkupfalse%
\ {\isacartoucheopen}insert\ I\ S{\isadigit{1}}\ {\isasymin}\ C{\isacartoucheclose}\ {\isacartoucheopen}insert\ I\ S{\isadigit{2}}\ {\isasymin}\ C{\isacartoucheclose}\ \isacommand{by}\isamarkupfalse%
\ {\isacharparenleft}rule\ conjI{\isacharparenright}\isanewline
\ \ \isacommand{thus}\isamarkupfalse%
\ {\isachardoublequoteopen}{\isasymexists}I{\isasymin}{\isacharbraceleft}G{\isacharcomma}H{\isacharbraceright}{\isachardot}\ insert\ I\ S{\isadigit{1}}\ {\isasymin}\ C\ {\isasymand}\ insert\ I\ S{\isadigit{2}}\ {\isasymin}\ C{\isachardoublequoteclose}\isanewline
\ \ \ \ \isacommand{using}\isamarkupfalse%
\ {\isacartoucheopen}I\ {\isasymin}\ {\isacharbraceleft}G{\isacharcomma}H{\isacharbraceright}{\isacartoucheclose}\ \isacommand{by}\isamarkupfalse%
\ {\isacharparenleft}rule\ bexI{\isacharparenright}\isanewline
\isacommand{qed}\isamarkupfalse%
%
\endisatagproof
{\isafoldproof}%
%
\isadelimproof
%
\endisadelimproof
%
\begin{isamarkuptext}%
Finalmente, el lema \isa{ex{\isadigit{3}}{\isacharunderscore}pcp{\isacharunderscore}SinE{\isacharunderscore}DIS{\isacharunderscore}auxFalse} prueba que dada una colección \isa{C} con la 
  propiedad de consistencia proposicional y cerrada bajo subconjuntos, \isa{S\ {\isasymin}\ E} y sea \isa{F} es una 
  fórmula de tipo \isa{{\isasymbeta}} de componentes \isa{{\isasymbeta}\isactrlsub {\isadigit{1}}} y \isa{{\isasymbeta}\isactrlsub {\isadigit{2}}}, si consideramos \isa{S\isactrlsub {\isadigit{1}}} y \isa{S\isactrlsub {\isadigit{2}}} subconjuntos finitos 
  cualesquiera de \isa{S} tales que \isa{F\ {\isasymin}\ S\isactrlsub {\isadigit{1}}}, \isa{F\ {\isasymin}\ S\isactrlsub {\isadigit{2}}}, \isa{{\isacharbraceleft}{\isasymbeta}\isactrlsub {\isadigit{1}}{\isacharbraceright}\ {\isasymunion}\ S\isactrlsub {\isadigit{1}}\ {\isasymnotin}\ C} y \isa{{\isacharbraceleft}{\isasymbeta}\isactrlsub {\isadigit{2}}{\isacharbraceright}\ {\isasymunion}\ S\isactrlsub {\isadigit{2}}\ {\isasymnotin}\ C}, llegamos a 
  una contradicción.%
\end{isamarkuptext}\isamarkuptrue%
\isacommand{lemma}\isamarkupfalse%
\ ex{\isadigit{3}}{\isacharunderscore}pcp{\isacharunderscore}SinE{\isacharunderscore}DIS{\isacharunderscore}auxFalse{\isacharcolon}\isanewline
\ \ \isakeyword{assumes}\ {\isachardoublequoteopen}pcp\ C{\isachardoublequoteclose}\ \isanewline
\ \ \ \ \ \ \ \ \ \ {\isachardoublequoteopen}subset{\isacharunderscore}closed\ C{\isachardoublequoteclose}\isanewline
\ \ \ \ \ \ \ \ \ \ {\isachardoublequoteopen}S\ {\isasymin}\ {\isacharparenleft}extF\ C{\isacharparenright}{\isachardoublequoteclose}\isanewline
\ \ \ \ \ \ \ \ \ \ {\isachardoublequoteopen}Dis\ F\ G\ H{\isachardoublequoteclose}\isanewline
\ \ \ \ \ \ \ \ \ \ {\isachardoublequoteopen}F\ {\isasymin}\ S{\isachardoublequoteclose}\isanewline
\ \ \ \ \ \ \ \ \ \ {\isachardoublequoteopen}S{\isadigit{1}}\ {\isasymsubseteq}\ S{\isachardoublequoteclose}\ \isanewline
\ \ \ \ \ \ \ \ \ \ {\isachardoublequoteopen}finite\ S{\isadigit{1}}{\isachardoublequoteclose}\ \isanewline
\ \ \ \ \ \ \ \ \ \ {\isachardoublequoteopen}insert\ G\ S{\isadigit{1}}\ {\isasymnotin}\ C{\isachardoublequoteclose}\ \isanewline
\ \ \ \ \ \ \ \ \ \ {\isachardoublequoteopen}S{\isadigit{2}}\ {\isasymsubseteq}\ S{\isachardoublequoteclose}\ \isanewline
\ \ \ \ \ \ \ \ \ \ {\isachardoublequoteopen}finite\ S{\isadigit{2}}{\isachardoublequoteclose}\ \isanewline
\ \ \ \ \ \ \ \ \ \ {\isachardoublequoteopen}insert\ H\ S{\isadigit{2}}\ {\isasymnotin}\ C{\isachardoublequoteclose}\isanewline
\ \ \ \ \ \ \ \ \isakeyword{shows}\ {\isachardoublequoteopen}False{\isachardoublequoteclose}\isanewline
%
\isadelimproof
%
\endisadelimproof
%
\isatagproof
\isacommand{proof}\isamarkupfalse%
\ {\isacharminus}\isanewline
\ \ \isacommand{let}\isamarkupfalse%
\ {\isacharquery}S{\isadigit{1}}{\isacharequal}{\isachardoublequoteopen}insert\ F\ S{\isadigit{1}}{\isachardoublequoteclose}\isanewline
\ \ \isacommand{let}\isamarkupfalse%
\ {\isacharquery}S{\isadigit{2}}{\isacharequal}{\isachardoublequoteopen}insert\ F\ S{\isadigit{2}}{\isachardoublequoteclose}\isanewline
\ \ \isacommand{have}\isamarkupfalse%
\ SC{\isacharcolon}{\isachardoublequoteopen}{\isasymforall}S\ {\isasymin}\ C{\isachardot}\ {\isasymforall}S{\isacharprime}{\isasymsubseteq}S{\isachardot}\ S{\isacharprime}\ {\isasymin}\ C{\isachardoublequoteclose}\isanewline
\ \ \ \ \isacommand{using}\isamarkupfalse%
\ assms{\isacharparenleft}{\isadigit{2}}{\isacharparenright}\ \isacommand{by}\isamarkupfalse%
\ {\isacharparenleft}simp\ only{\isacharcolon}\ subset{\isacharunderscore}closed{\isacharunderscore}def{\isacharparenright}\isanewline
\ \ \isacommand{have}\isamarkupfalse%
\ {\isadigit{1}}{\isacharcolon}{\isachardoublequoteopen}{\isacharquery}S{\isadigit{1}}\ {\isasymsubseteq}\ S{\isachardoublequoteclose}\isanewline
\ \ \ \ \isacommand{using}\isamarkupfalse%
\ {\isacartoucheopen}F\ {\isasymin}\ S{\isacartoucheclose}\ {\isacartoucheopen}S{\isadigit{1}}\ {\isasymsubseteq}\ S{\isacartoucheclose}\ \isacommand{by}\isamarkupfalse%
\ {\isacharparenleft}simp\ only{\isacharcolon}\ insert{\isacharunderscore}subset{\isacharparenright}\ \isanewline
\ \ \isacommand{have}\isamarkupfalse%
\ {\isadigit{2}}{\isacharcolon}{\isachardoublequoteopen}finite\ {\isacharquery}S{\isadigit{1}}{\isachardoublequoteclose}\isanewline
\ \ \ \ \isacommand{using}\isamarkupfalse%
\ {\isacartoucheopen}finite\ S{\isadigit{1}}{\isacartoucheclose}\ \isacommand{by}\isamarkupfalse%
\ {\isacharparenleft}simp\ only{\isacharcolon}\ finite{\isacharunderscore}insert{\isacharparenright}\ \isanewline
\ \ \isacommand{have}\isamarkupfalse%
\ {\isadigit{3}}{\isacharcolon}{\isachardoublequoteopen}F\ {\isasymin}\ {\isacharquery}S{\isadigit{1}}{\isachardoublequoteclose}\isanewline
\ \ \ \ \isacommand{by}\isamarkupfalse%
\ {\isacharparenleft}simp\ only{\isacharcolon}\ insertI{\isadigit{1}}{\isacharparenright}\ \isanewline
\ \ \isacommand{have}\isamarkupfalse%
\ {\isadigit{4}}{\isacharcolon}{\isachardoublequoteopen}insert\ G\ {\isacharquery}S{\isadigit{1}}\ {\isasymnotin}\ C{\isachardoublequoteclose}\ \isanewline
\ \ \isacommand{proof}\isamarkupfalse%
\ {\isacharparenleft}rule\ ccontr{\isacharparenright}\isanewline
\ \ \ \ \isacommand{assume}\isamarkupfalse%
\ {\isachardoublequoteopen}{\isasymnot}{\isacharparenleft}insert\ G\ {\isacharquery}S{\isadigit{1}}\ {\isasymnotin}\ C{\isacharparenright}{\isachardoublequoteclose}\isanewline
\ \ \ \ \isacommand{then}\isamarkupfalse%
\ \isacommand{have}\isamarkupfalse%
\ {\isachardoublequoteopen}insert\ G\ {\isacharquery}S{\isadigit{1}}\ {\isasymin}\ C{\isachardoublequoteclose}\isanewline
\ \ \ \ \ \ \isacommand{by}\isamarkupfalse%
\ {\isacharparenleft}rule\ notnotD{\isacharparenright}\isanewline
\ \ \ \ \isacommand{have}\isamarkupfalse%
\ SC{\isadigit{1}}{\isacharcolon}{\isachardoublequoteopen}{\isasymforall}S{\isacharprime}\ {\isasymsubseteq}\ {\isacharparenleft}insert\ G\ {\isacharquery}S{\isadigit{1}}{\isacharparenright}{\isachardot}\ S{\isacharprime}\ {\isasymin}\ C{\isachardoublequoteclose}\isanewline
\ \ \ \ \ \ \isacommand{using}\isamarkupfalse%
\ SC\ {\isacartoucheopen}insert\ G\ {\isacharquery}S{\isadigit{1}}\ {\isasymin}\ C{\isacartoucheclose}\ \isacommand{by}\isamarkupfalse%
\ {\isacharparenleft}rule\ bspec{\isacharparenright}\isanewline
\ \ \ \ \isacommand{have}\isamarkupfalse%
\ {\isachardoublequoteopen}insert\ G\ S{\isadigit{1}}\ {\isasymsubseteq}\ insert\ F\ {\isacharparenleft}insert\ G\ S{\isadigit{1}}{\isacharparenright}{\isachardoublequoteclose}\isanewline
\ \ \ \ \ \ \isacommand{by}\isamarkupfalse%
\ {\isacharparenleft}rule\ subset{\isacharunderscore}insertI{\isacharparenright}\isanewline
\ \ \ \ \isacommand{then}\isamarkupfalse%
\ \isacommand{have}\isamarkupfalse%
\ {\isachardoublequoteopen}insert\ G\ S{\isadigit{1}}\ {\isasymsubseteq}\ insert\ G\ {\isacharquery}S{\isadigit{1}}{\isachardoublequoteclose}\isanewline
\ \ \ \ \ \ \isacommand{by}\isamarkupfalse%
\ {\isacharparenleft}simp\ only{\isacharcolon}\ insert{\isacharunderscore}commute{\isacharparenright}\isanewline
\ \ \ \ \isacommand{have}\isamarkupfalse%
\ {\isachardoublequoteopen}insert\ G\ S{\isadigit{1}}\ {\isasymin}\ C{\isachardoublequoteclose}\isanewline
\ \ \ \ \ \ \isacommand{using}\isamarkupfalse%
\ SC{\isadigit{1}}\ {\isacartoucheopen}insert\ G\ S{\isadigit{1}}\ {\isasymsubseteq}\ insert\ G\ {\isacharquery}S{\isadigit{1}}{\isacartoucheclose}\ \isacommand{by}\isamarkupfalse%
\ {\isacharparenleft}rule\ sspec{\isacharparenright}\isanewline
\ \ \ \ \isacommand{show}\isamarkupfalse%
\ {\isachardoublequoteopen}False{\isachardoublequoteclose}\isanewline
\ \ \ \ \ \ \isacommand{using}\isamarkupfalse%
\ assms{\isacharparenleft}{\isadigit{8}}{\isacharparenright}\ {\isacartoucheopen}insert\ G\ S{\isadigit{1}}\ {\isasymin}\ C{\isacartoucheclose}\ \isacommand{by}\isamarkupfalse%
\ {\isacharparenleft}rule\ notE{\isacharparenright}\isanewline
\ \ \isacommand{qed}\isamarkupfalse%
\isanewline
\ \ \isacommand{have}\isamarkupfalse%
\ {\isadigit{5}}{\isacharcolon}{\isachardoublequoteopen}{\isacharquery}S{\isadigit{2}}\ {\isasymsubseteq}\ S{\isachardoublequoteclose}\isanewline
\ \ \ \ \isacommand{using}\isamarkupfalse%
\ {\isacartoucheopen}F\ {\isasymin}\ S{\isacartoucheclose}\ {\isacartoucheopen}S{\isadigit{2}}\ {\isasymsubseteq}\ S{\isacartoucheclose}\ \isacommand{by}\isamarkupfalse%
\ {\isacharparenleft}simp\ only{\isacharcolon}\ insert{\isacharunderscore}subset{\isacharparenright}\isanewline
\ \ \isacommand{have}\isamarkupfalse%
\ {\isadigit{6}}{\isacharcolon}{\isachardoublequoteopen}finite\ {\isacharquery}S{\isadigit{2}}{\isachardoublequoteclose}\isanewline
\ \ \ \ \isacommand{using}\isamarkupfalse%
\ {\isacartoucheopen}finite\ S{\isadigit{2}}{\isacartoucheclose}\ \isacommand{by}\isamarkupfalse%
\ {\isacharparenleft}simp\ only{\isacharcolon}\ finite{\isacharunderscore}insert{\isacharparenright}\isanewline
\ \ \isacommand{have}\isamarkupfalse%
\ {\isadigit{7}}{\isacharcolon}{\isachardoublequoteopen}F\ {\isasymin}\ {\isacharquery}S{\isadigit{2}}{\isachardoublequoteclose}\isanewline
\ \ \ \ \isacommand{by}\isamarkupfalse%
\ {\isacharparenleft}simp\ only{\isacharcolon}\ insertI{\isadigit{1}}{\isacharparenright}\isanewline
\ \ \isacommand{have}\isamarkupfalse%
\ {\isadigit{8}}{\isacharcolon}{\isachardoublequoteopen}insert\ H\ {\isacharquery}S{\isadigit{2}}\ {\isasymnotin}\ C{\isachardoublequoteclose}\ \isanewline
\ \ \isacommand{proof}\isamarkupfalse%
\ {\isacharparenleft}rule\ ccontr{\isacharparenright}\isanewline
\ \ \ \ \isacommand{assume}\isamarkupfalse%
\ {\isachardoublequoteopen}{\isasymnot}{\isacharparenleft}insert\ H\ {\isacharquery}S{\isadigit{2}}\ {\isasymnotin}\ C{\isacharparenright}{\isachardoublequoteclose}\isanewline
\ \ \ \ \isacommand{then}\isamarkupfalse%
\ \isacommand{have}\isamarkupfalse%
\ {\isachardoublequoteopen}insert\ H\ {\isacharquery}S{\isadigit{2}}\ {\isasymin}\ C{\isachardoublequoteclose}\isanewline
\ \ \ \ \ \ \isacommand{by}\isamarkupfalse%
\ {\isacharparenleft}rule\ notnotD{\isacharparenright}\isanewline
\ \ \ \ \isacommand{have}\isamarkupfalse%
\ SC{\isadigit{2}}{\isacharcolon}{\isachardoublequoteopen}{\isasymforall}S{\isacharprime}\ {\isasymsubseteq}\ {\isacharparenleft}insert\ H\ {\isacharquery}S{\isadigit{2}}{\isacharparenright}{\isachardot}\ S{\isacharprime}\ {\isasymin}\ C{\isachardoublequoteclose}\isanewline
\ \ \ \ \ \ \isacommand{using}\isamarkupfalse%
\ SC\ {\isacartoucheopen}insert\ H\ {\isacharquery}S{\isadigit{2}}\ {\isasymin}\ C{\isacartoucheclose}\ \isacommand{by}\isamarkupfalse%
\ {\isacharparenleft}rule\ bspec{\isacharparenright}\isanewline
\ \ \ \ \isacommand{have}\isamarkupfalse%
\ {\isachardoublequoteopen}insert\ H\ S{\isadigit{2}}\ {\isasymsubseteq}\ insert\ F\ {\isacharparenleft}insert\ H\ S{\isadigit{2}}{\isacharparenright}{\isachardoublequoteclose}\isanewline
\ \ \ \ \ \ \isacommand{by}\isamarkupfalse%
\ {\isacharparenleft}rule\ subset{\isacharunderscore}insertI{\isacharparenright}\isanewline
\ \ \ \ \isacommand{then}\isamarkupfalse%
\ \isacommand{have}\isamarkupfalse%
\ {\isachardoublequoteopen}insert\ H\ S{\isadigit{2}}\ {\isasymsubseteq}\ insert\ H\ {\isacharquery}S{\isadigit{2}}{\isachardoublequoteclose}\isanewline
\ \ \ \ \ \ \isacommand{by}\isamarkupfalse%
\ {\isacharparenleft}simp\ only{\isacharcolon}\ insert{\isacharunderscore}commute{\isacharparenright}\isanewline
\ \ \ \ \isacommand{have}\isamarkupfalse%
\ {\isachardoublequoteopen}insert\ H\ S{\isadigit{2}}\ {\isasymin}\ C{\isachardoublequoteclose}\isanewline
\ \ \ \ \ \ \isacommand{using}\isamarkupfalse%
\ SC{\isadigit{2}}\ {\isacartoucheopen}insert\ H\ S{\isadigit{2}}\ {\isasymsubseteq}\ insert\ H\ {\isacharquery}S{\isadigit{2}}{\isacartoucheclose}\ \isacommand{by}\isamarkupfalse%
\ {\isacharparenleft}rule\ sspec{\isacharparenright}\isanewline
\ \ \ \ \isacommand{show}\isamarkupfalse%
\ {\isachardoublequoteopen}False{\isachardoublequoteclose}\isanewline
\ \ \ \ \ \ \isacommand{using}\isamarkupfalse%
\ assms{\isacharparenleft}{\isadigit{1}}{\isadigit{1}}{\isacharparenright}\ {\isacartoucheopen}insert\ H\ S{\isadigit{2}}\ {\isasymin}\ C{\isacartoucheclose}\ \isacommand{by}\isamarkupfalse%
\ {\isacharparenleft}rule\ notE{\isacharparenright}\isanewline
\ \ \isacommand{qed}\isamarkupfalse%
\isanewline
\ \ \isacommand{have}\isamarkupfalse%
\ Ex{\isacharcolon}{\isachardoublequoteopen}{\isasymexists}I\ {\isasymin}\ {\isacharbraceleft}G{\isacharcomma}H{\isacharbraceright}{\isachardot}\ insert\ I\ {\isacharquery}S{\isadigit{1}}\ {\isasymin}\ C\ {\isasymand}\ insert\ I\ {\isacharquery}S{\isadigit{2}}\ {\isasymin}\ C{\isachardoublequoteclose}\isanewline
\ \ \ \ \isacommand{using}\isamarkupfalse%
\ assms{\isacharparenleft}{\isadigit{1}}{\isacharparenright}\ assms{\isacharparenleft}{\isadigit{2}}{\isacharparenright}\ assms{\isacharparenleft}{\isadigit{3}}{\isacharparenright}\ assms{\isacharparenleft}{\isadigit{4}}{\isacharparenright}\ {\isadigit{1}}\ {\isadigit{2}}\ {\isadigit{3}}\ {\isadigit{5}}\ {\isadigit{6}}\ {\isadigit{7}}\ \isacommand{by}\isamarkupfalse%
\ {\isacharparenleft}rule\ ex{\isadigit{3}}{\isacharunderscore}pcp{\isacharunderscore}SinE{\isacharunderscore}DIS{\isacharunderscore}auxEx{\isacharparenright}\isanewline
\ \ \isacommand{have}\isamarkupfalse%
\ {\isachardoublequoteopen}{\isasymforall}I\ {\isasymin}\ {\isacharbraceleft}G{\isacharcomma}H{\isacharbraceright}{\isachardot}\ insert\ I\ {\isacharquery}S{\isadigit{1}}\ {\isasymnotin}\ C\ {\isasymor}\ insert\ I\ {\isacharquery}S{\isadigit{2}}\ {\isasymnotin}\ C{\isachardoublequoteclose}\isanewline
\ \ \ \ \isacommand{using}\isamarkupfalse%
\ {\isadigit{4}}\ {\isadigit{8}}\ \isacommand{by}\isamarkupfalse%
\ simp\isanewline
\ \ \isacommand{then}\isamarkupfalse%
\ \isacommand{have}\isamarkupfalse%
\ {\isachardoublequoteopen}{\isasymforall}I\ {\isasymin}\ {\isacharbraceleft}G{\isacharcomma}H{\isacharbraceright}{\isachardot}\ {\isasymnot}{\isacharparenleft}insert\ I\ {\isacharquery}S{\isadigit{1}}\ {\isasymin}\ C\ {\isasymand}\ insert\ I\ {\isacharquery}S{\isadigit{2}}\ {\isasymin}\ C{\isacharparenright}{\isachardoublequoteclose}\isanewline
\ \ \ \ \isacommand{by}\isamarkupfalse%
\ {\isacharparenleft}simp\ only{\isacharcolon}\ de{\isacharunderscore}Morgan{\isacharunderscore}conj{\isacharparenright}\isanewline
\ \ \isacommand{then}\isamarkupfalse%
\ \isacommand{have}\isamarkupfalse%
\ {\isachardoublequoteopen}{\isasymnot}{\isacharparenleft}{\isasymexists}I\ {\isasymin}\ {\isacharbraceleft}G{\isacharcomma}H{\isacharbraceright}{\isachardot}\ insert\ I\ {\isacharquery}S{\isadigit{1}}\ {\isasymin}\ C\ {\isasymand}\ insert\ I\ {\isacharquery}S{\isadigit{2}}\ {\isasymin}\ C{\isacharparenright}{\isachardoublequoteclose}\isanewline
\ \ \ \ \isacommand{by}\isamarkupfalse%
\ {\isacharparenleft}simp\ only{\isacharcolon}\ bex{\isacharunderscore}simps{\isacharparenleft}{\isadigit{8}}{\isacharparenright}{\isacharparenright}\ \isanewline
\ \ \isacommand{thus}\isamarkupfalse%
\ {\isachardoublequoteopen}False{\isachardoublequoteclose}\isanewline
\ \ \ \ \isacommand{using}\isamarkupfalse%
\ Ex\ \isacommand{by}\isamarkupfalse%
\ {\isacharparenleft}rule\ notE{\isacharparenright}\isanewline
\isacommand{qed}\isamarkupfalse%
%
\endisatagproof
{\isafoldproof}%
%
\isadelimproof
%
\endisadelimproof
%
\begin{isamarkuptext}%
Una vez introducidos los lemas anteriores, podemos probar el lema \isa{ex{\isadigit{3}}{\isacharunderscore}pcp{\isacharunderscore}SinE{\isacharunderscore}DIS} que
  demuestra que si \isa{C} es una colección con la propiedad de consistencia proposicional y cerrada 
  bajo subconjuntos, \isa{S\ {\isasymin}\ E} y sea \isa{F} una fórmula de tipo \isa{{\isasymbeta}} con componentes \isa{{\isasymbeta}\isactrlsub {\isadigit{1}}} y \isa{{\isasymbeta}\isactrlsub {\isadigit{2}}}, se 
  verifica que o bien \isa{{\isacharbraceleft}{\isasymbeta}\isactrlsub {\isadigit{1}}{\isacharbraceright}\ {\isasymunion}\ S\ {\isasymin}\ C{\isacharprime}} o bien \isa{{\isacharbraceleft}{\isasymbeta}\isactrlsub {\isadigit{2}}{\isacharbraceright}\ {\isasymunion}\ S\ {\isasymin}\ C{\isacharprime}}. Además, para dicha prueba 
  necesitaremos los siguientes lemas auxiliares en Isabelle.%
\end{isamarkuptext}\isamarkuptrue%
\isacommand{lemma}\isamarkupfalse%
\ sall{\isacharunderscore}simps{\isacharunderscore}not{\isacharunderscore}all{\isacharcolon}\isanewline
\ \ \isakeyword{assumes}\ {\isachardoublequoteopen}{\isasymnot}{\isacharparenleft}{\isasymforall}x\ {\isasymsubseteq}\ A{\isachardot}\ P\ x{\isacharparenright}{\isachardoublequoteclose}\isanewline
\ \ \isakeyword{shows}\ {\isachardoublequoteopen}{\isasymexists}x\ {\isasymsubseteq}\ A{\isachardot}\ {\isacharparenleft}{\isasymnot}\ P\ x{\isacharparenright}{\isachardoublequoteclose}\isanewline
%
\isadelimproof
\ \ %
\endisadelimproof
%
\isatagproof
\isacommand{using}\isamarkupfalse%
\ assms\ \isacommand{by}\isamarkupfalse%
\ blast%
\endisatagproof
{\isafoldproof}%
%
\isadelimproof
\isanewline
%
\endisadelimproof
\isanewline
\isacommand{lemma}\isamarkupfalse%
\ subexE{\isacharcolon}\ {\isachardoublequoteopen}{\isasymexists}x{\isasymsubseteq}A{\isachardot}\ P\ x\ {\isasymLongrightarrow}\ {\isacharparenleft}{\isasymAnd}x{\isachardot}\ x{\isasymsubseteq}A\ {\isasymLongrightarrow}\ P\ x\ {\isasymLongrightarrow}\ Q{\isacharparenright}\ {\isasymLongrightarrow}\ Q{\isachardoublequoteclose}\isanewline
%
\isadelimproof
\ \ %
\endisadelimproof
%
\isatagproof
\isacommand{by}\isamarkupfalse%
\ blast%
\endisatagproof
{\isafoldproof}%
%
\isadelimproof
%
\endisadelimproof
%
\begin{isamarkuptext}%
De este modo, procedamos con la demostración detallada de \isa{ex{\isadigit{3}}{\isacharunderscore}pcp{\isacharunderscore}SinE{\isacharunderscore}DIS}.%
\end{isamarkuptext}\isamarkuptrue%
\isacommand{lemma}\isamarkupfalse%
\ ex{\isadigit{3}}{\isacharunderscore}pcp{\isacharunderscore}SinE{\isacharunderscore}DIS{\isacharcolon}\isanewline
\ \ \isakeyword{assumes}\ {\isachardoublequoteopen}pcp\ C{\isachardoublequoteclose}\isanewline
\ \ \ \ \ \ \ \ \ \ {\isachardoublequoteopen}subset{\isacharunderscore}closed\ C{\isachardoublequoteclose}\isanewline
\ \ \ \ \ \ \ \ \ \ {\isachardoublequoteopen}S\ {\isasymin}\ {\isacharparenleft}extF\ C{\isacharparenright}{\isachardoublequoteclose}\isanewline
\ \ \ \ \ \ \ \ \ \ {\isachardoublequoteopen}Dis\ F\ G\ H{\isachardoublequoteclose}\isanewline
\ \ \ \ \ \ \ \ \ \ {\isachardoublequoteopen}F\ {\isasymin}\ S{\isachardoublequoteclose}\isanewline
\ \ \isakeyword{shows}\ {\isachardoublequoteopen}{\isacharbraceleft}G{\isacharbraceright}\ {\isasymunion}\ S\ {\isasymin}\ {\isacharparenleft}extensionFin\ C{\isacharparenright}\ {\isasymor}\ {\isacharbraceleft}H{\isacharbraceright}\ {\isasymunion}\ S\ {\isasymin}\ {\isacharparenleft}extensionFin\ C{\isacharparenright}{\isachardoublequoteclose}\isanewline
%
\isadelimproof
%
\endisadelimproof
%
\isatagproof
\isacommand{proof}\isamarkupfalse%
\ {\isacharminus}\isanewline
\ \ \isacommand{have}\isamarkupfalse%
\ {\isachardoublequoteopen}{\isacharparenleft}extF\ C{\isacharparenright}\ {\isasymsubseteq}\ {\isacharparenleft}extensionFin\ C{\isacharparenright}{\isachardoublequoteclose}\ \isanewline
\ \ \ \ \isacommand{unfolding}\isamarkupfalse%
\ extensionFin\ \isacommand{by}\isamarkupfalse%
\ {\isacharparenleft}rule\ Un{\isacharunderscore}upper{\isadigit{2}}{\isacharparenright}\ \isanewline
\ \ \isacommand{have}\isamarkupfalse%
\ PCP{\isacharcolon}{\isachardoublequoteopen}{\isasymforall}S\ {\isasymin}\ C{\isachardot}\isanewline
\ \ \ \ \ \ \ \ \ \ \ \ {\isasymbottom}\ {\isasymnotin}\ S\isanewline
\ \ \ \ \ \ \ \ \ \ \ \ {\isasymand}\ {\isacharparenleft}{\isasymforall}k{\isachardot}\ Atom\ k\ {\isasymin}\ S\ {\isasymlongrightarrow}\ \isactrlbold {\isasymnot}\ {\isacharparenleft}Atom\ k{\isacharparenright}\ {\isasymin}\ S\ {\isasymlongrightarrow}\ False{\isacharparenright}\isanewline
\ \ \ \ \ \ \ \ \ \ \ \ {\isasymand}\ {\isacharparenleft}{\isasymforall}F\ G\ H{\isachardot}\ Con\ F\ G\ H\ {\isasymlongrightarrow}\ F\ {\isasymin}\ S\ {\isasymlongrightarrow}\ {\isacharbraceleft}G{\isacharcomma}H{\isacharbraceright}\ {\isasymunion}\ S\ {\isasymin}\ C{\isacharparenright}\isanewline
\ \ \ \ \ \ \ \ \ \ \ \ {\isasymand}\ {\isacharparenleft}{\isasymforall}F\ G\ H{\isachardot}\ Dis\ F\ G\ H\ {\isasymlongrightarrow}\ F\ {\isasymin}\ S\ {\isasymlongrightarrow}\ {\isacharbraceleft}G{\isacharbraceright}\ {\isasymunion}\ S\ {\isasymin}\ C\ {\isasymor}\ {\isacharbraceleft}H{\isacharbraceright}\ {\isasymunion}\ S\ {\isasymin}\ C{\isacharparenright}{\isachardoublequoteclose}\isanewline
\ \ \ \ \isacommand{using}\isamarkupfalse%
\ assms{\isacharparenleft}{\isadigit{1}}{\isacharparenright}\ \isacommand{by}\isamarkupfalse%
\ {\isacharparenleft}rule\ pcp{\isacharunderscore}alt{\isadigit{1}}{\isacharparenright}\isanewline
\ \ \isacommand{have}\isamarkupfalse%
\ E{\isacharcolon}{\isachardoublequoteopen}{\isasymforall}S{\isacharprime}\ {\isasymsubseteq}\ S{\isachardot}\ finite\ S{\isacharprime}\ {\isasymlongrightarrow}\ S{\isacharprime}\ {\isasymin}\ C{\isachardoublequoteclose}\isanewline
\ \ \ \ \isacommand{using}\isamarkupfalse%
\ assms{\isacharparenleft}{\isadigit{3}}{\isacharparenright}\ \isacommand{unfolding}\isamarkupfalse%
\ extF\ \isacommand{by}\isamarkupfalse%
\ {\isacharparenleft}rule\ CollectD{\isacharparenright}\isanewline
\ \ \isacommand{then}\isamarkupfalse%
\ \isacommand{have}\isamarkupfalse%
\ E{\isacharprime}{\isacharcolon}{\isachardoublequoteopen}{\isasymforall}S{\isacharprime}{\isachardot}\ S{\isacharprime}\ {\isasymsubseteq}\ S\ {\isasymlongrightarrow}\ finite\ S{\isacharprime}\ {\isasymlongrightarrow}\ S{\isacharprime}\ {\isasymin}\ C{\isachardoublequoteclose}\isanewline
\ \ \ \ \isacommand{by}\isamarkupfalse%
\ blast\isanewline
\ \ \isacommand{have}\isamarkupfalse%
\ SC{\isacharcolon}{\isachardoublequoteopen}{\isasymforall}S\ {\isasymin}\ C{\isachardot}\ {\isasymforall}S{\isacharprime}{\isasymsubseteq}S{\isachardot}\ S{\isacharprime}\ {\isasymin}\ C{\isachardoublequoteclose}\isanewline
\ \ \ \ \isacommand{using}\isamarkupfalse%
\ assms{\isacharparenleft}{\isadigit{2}}{\isacharparenright}\ \isacommand{by}\isamarkupfalse%
\ {\isacharparenleft}simp\ only{\isacharcolon}\ subset{\isacharunderscore}closed{\isacharunderscore}def{\isacharparenright}\isanewline
\ \ \isacommand{have}\isamarkupfalse%
\ {\isachardoublequoteopen}insert\ G\ S\ {\isasymin}\ {\isacharparenleft}extF\ C{\isacharparenright}\ {\isasymor}\ insert\ H\ S\ {\isasymin}\ {\isacharparenleft}extF\ C{\isacharparenright}{\isachardoublequoteclose}\ \isanewline
\ \ \isacommand{proof}\isamarkupfalse%
\ {\isacharparenleft}rule\ ccontr{\isacharparenright}\isanewline
\ \ \ \ \isacommand{assume}\isamarkupfalse%
\ {\isachardoublequoteopen}{\isasymnot}{\isacharparenleft}insert\ G\ S\ {\isasymin}\ {\isacharparenleft}extF\ C{\isacharparenright}\ {\isasymor}\ insert\ H\ S\ {\isasymin}\ {\isacharparenleft}extF\ C{\isacharparenright}{\isacharparenright}{\isachardoublequoteclose}\ \ \isanewline
\ \ \ \ \isacommand{then}\isamarkupfalse%
\ \isacommand{have}\isamarkupfalse%
\ Conj{\isacharcolon}{\isachardoublequoteopen}{\isasymnot}{\isacharparenleft}insert\ G\ S\ {\isasymin}\ {\isacharparenleft}extF\ C{\isacharparenright}{\isacharparenright}\ {\isasymand}\ {\isasymnot}{\isacharparenleft}insert\ H\ S\ {\isasymin}\ {\isacharparenleft}extF\ C{\isacharparenright}{\isacharparenright}{\isachardoublequoteclose}\ \isanewline
\ \ \ \ \ \ \isacommand{by}\isamarkupfalse%
\ {\isacharparenleft}simp\ only{\isacharcolon}\ simp{\isacharunderscore}thms{\isacharparenleft}{\isadigit{8}}{\isacharcomma}{\isadigit{2}}{\isadigit{5}}{\isacharparenright}\ de{\isacharunderscore}Morgan{\isacharunderscore}disj{\isacharparenright}\isanewline
\ \ \ \ \isacommand{then}\isamarkupfalse%
\ \isacommand{have}\isamarkupfalse%
\ {\isachardoublequoteopen}{\isasymnot}{\isacharparenleft}insert\ G\ S\ {\isasymin}\ {\isacharparenleft}extF\ C{\isacharparenright}{\isacharparenright}{\isachardoublequoteclose}\isanewline
\ \ \ \ \ \ \isacommand{by}\isamarkupfalse%
\ {\isacharparenleft}rule\ conjunct{\isadigit{1}}{\isacharparenright}\isanewline
\ \ \ \ \isacommand{then}\isamarkupfalse%
\ \isacommand{have}\isamarkupfalse%
\ {\isachardoublequoteopen}{\isasymnot}{\isacharparenleft}{\isasymforall}S{\isacharprime}\ {\isasymsubseteq}\ {\isacharparenleft}insert\ G\ S{\isacharparenright}{\isachardot}\ finite\ S{\isacharprime}\ {\isasymlongrightarrow}\ S{\isacharprime}\ {\isasymin}\ C{\isacharparenright}{\isachardoublequoteclose}\isanewline
\ \ \ \ \ \ \isacommand{unfolding}\isamarkupfalse%
\ extF\ \isacommand{by}\isamarkupfalse%
\ {\isacharparenleft}simp\ add{\isacharcolon}\ mem{\isacharunderscore}Collect{\isacharunderscore}eq{\isacharparenright}\isanewline
\ \ \ \ \isacommand{then}\isamarkupfalse%
\ \isacommand{have}\isamarkupfalse%
\ Ex{\isadigit{1}}{\isacharcolon}{\isachardoublequoteopen}{\isasymexists}S{\isacharprime}{\isasymsubseteq}\ {\isacharparenleft}insert\ G\ S{\isacharparenright}{\isachardot}\ {\isasymnot}{\isacharparenleft}finite\ S{\isacharprime}\ {\isasymlongrightarrow}\ S{\isacharprime}\ {\isasymin}\ C{\isacharparenright}{\isachardoublequoteclose}\isanewline
\ \ \ \ \ \ \isacommand{by}\isamarkupfalse%
\ {\isacharparenleft}rule\ sall{\isacharunderscore}simps{\isacharunderscore}not{\isacharunderscore}all{\isacharparenright}\isanewline
\ \ \ \ \isacommand{obtain}\isamarkupfalse%
\ S{\isadigit{1}}\ \isakeyword{where}\ {\isachardoublequoteopen}S{\isadigit{1}}\ {\isasymsubseteq}\ insert\ G\ S{\isachardoublequoteclose}\ \isakeyword{and}\ {\isachardoublequoteopen}{\isasymnot}{\isacharparenleft}finite\ S{\isadigit{1}}\ {\isasymlongrightarrow}\ S{\isadigit{1}}\ {\isasymin}\ C{\isacharparenright}{\isachardoublequoteclose}\isanewline
\ \ \ \ \ \ \isacommand{using}\isamarkupfalse%
\ Ex{\isadigit{1}}\ \isacommand{by}\isamarkupfalse%
\ {\isacharparenleft}rule\ subexE{\isacharparenright}\isanewline
\ \ \ \ \isacommand{have}\isamarkupfalse%
\ {\isachardoublequoteopen}finite\ S{\isadigit{1}}\ {\isasymand}\ S{\isadigit{1}}\ {\isasymnotin}\ C{\isachardoublequoteclose}\ \isanewline
\ \ \ \ \ \ \isacommand{using}\isamarkupfalse%
\ {\isacartoucheopen}{\isasymnot}{\isacharparenleft}finite\ S{\isadigit{1}}\ {\isasymlongrightarrow}\ S{\isadigit{1}}\ {\isasymin}\ C{\isacharparenright}{\isacartoucheclose}\ \isacommand{by}\isamarkupfalse%
\ {\isacharparenleft}simp\ only{\isacharcolon}\ simp{\isacharunderscore}thms{\isacharparenleft}{\isadigit{8}}{\isacharparenright}\ not{\isacharunderscore}imp{\isacharparenright}\isanewline
\ \ \ \ \isacommand{then}\isamarkupfalse%
\ \isacommand{have}\isamarkupfalse%
\ {\isachardoublequoteopen}finite\ S{\isadigit{1}}{\isachardoublequoteclose}\isanewline
\ \ \ \ \ \ \isacommand{by}\isamarkupfalse%
\ {\isacharparenleft}rule\ conjunct{\isadigit{1}}{\isacharparenright}\isanewline
\ \ \ \ \isacommand{have}\isamarkupfalse%
\ {\isachardoublequoteopen}S{\isadigit{1}}\ {\isasymnotin}\ C{\isachardoublequoteclose}\isanewline
\ \ \ \ \ \ \isacommand{using}\isamarkupfalse%
\ {\isacartoucheopen}finite\ S{\isadigit{1}}\ {\isasymand}\ S{\isadigit{1}}\ {\isasymnotin}\ C{\isacartoucheclose}\ \isacommand{by}\isamarkupfalse%
\ {\isacharparenleft}rule\ conjunct{\isadigit{2}}{\isacharparenright}\isanewline
\ \ \ \ \isacommand{then}\isamarkupfalse%
\ \isacommand{have}\isamarkupfalse%
\ {\isachardoublequoteopen}insert\ G\ S{\isadigit{1}}\ {\isasymnotin}\ C{\isachardoublequoteclose}\isanewline
\ \ \ \ \isacommand{proof}\isamarkupfalse%
\ {\isacharminus}\ \isanewline
\ \ \ \ \ \ \isacommand{have}\isamarkupfalse%
\ {\isachardoublequoteopen}S{\isadigit{1}}\ {\isasymsubseteq}\ S\ {\isasymlongrightarrow}\ finite\ S{\isadigit{1}}\ {\isasymlongrightarrow}\ S{\isadigit{1}}\ {\isasymin}\ C{\isachardoublequoteclose}\isanewline
\ \ \ \ \ \ \ \ \isacommand{using}\isamarkupfalse%
\ E{\isacharprime}\ \isacommand{by}\isamarkupfalse%
\ {\isacharparenleft}rule\ allE{\isacharparenright}\isanewline
\ \ \ \ \ \ \isacommand{then}\isamarkupfalse%
\ \isacommand{have}\isamarkupfalse%
\ {\isachardoublequoteopen}{\isasymnot}\ S{\isadigit{1}}\ {\isasymsubseteq}\ S{\isachardoublequoteclose}\isanewline
\ \ \ \ \ \ \ \ \isacommand{using}\isamarkupfalse%
\ {\isacartoucheopen}{\isasymnot}\ {\isacharparenleft}finite\ S{\isadigit{1}}\ {\isasymlongrightarrow}\ S{\isadigit{1}}\ {\isasymin}\ C{\isacharparenright}{\isacartoucheclose}\ \isacommand{by}\isamarkupfalse%
\ {\isacharparenleft}rule\ mt{\isacharparenright}\isanewline
\ \ \ \ \ \ \isacommand{then}\isamarkupfalse%
\ \isacommand{have}\isamarkupfalse%
\ {\isachardoublequoteopen}{\isacharparenleft}S{\isadigit{1}}\ {\isasymsubseteq}\ insert\ G\ S{\isacharparenright}\ {\isasymnoteq}\ {\isacharparenleft}S{\isadigit{1}}\ {\isasymsubseteq}\ S{\isacharparenright}{\isachardoublequoteclose}\isanewline
\ \ \ \ \ \ \ \ \isacommand{using}\isamarkupfalse%
\ {\isacartoucheopen}S{\isadigit{1}}\ {\isasymsubseteq}\ insert\ G\ S{\isacartoucheclose}\ \isacommand{by}\isamarkupfalse%
\ simp\isanewline
\ \ \ \ \ \ \isacommand{then}\isamarkupfalse%
\ \isacommand{have}\isamarkupfalse%
\ notSI{\isacharcolon}{\isachardoublequoteopen}{\isasymnot}{\isacharparenleft}S{\isadigit{1}}\ {\isasymsubseteq}\ insert\ G\ S\ {\isasymlongleftrightarrow}\ S{\isadigit{1}}\ {\isasymsubseteq}\ S{\isacharparenright}{\isachardoublequoteclose}\isanewline
\ \ \ \ \ \ \ \ \isacommand{by}\isamarkupfalse%
\ blast\isanewline
\ \ \ \ \ \ \isacommand{have}\isamarkupfalse%
\ subsetInsert{\isacharcolon}{\isachardoublequoteopen}G\ {\isasymnotin}\ S{\isadigit{1}}\ {\isasymLongrightarrow}\ S{\isadigit{1}}\ {\isasymsubseteq}\ insert\ G\ S\ {\isasymlongleftrightarrow}\ S{\isadigit{1}}\ {\isasymsubseteq}\ S{\isachardoublequoteclose}\isanewline
\ \ \ \ \ \ \ \ \isacommand{by}\isamarkupfalse%
\ {\isacharparenleft}rule\ subset{\isacharunderscore}insert{\isacharparenright}\isanewline
\ \ \ \ \ \ \isacommand{have}\isamarkupfalse%
\ {\isachardoublequoteopen}{\isasymnot}{\isacharparenleft}G\ {\isasymnotin}\ S{\isadigit{1}}{\isacharparenright}{\isachardoublequoteclose}\isanewline
\ \ \ \ \ \ \ \ \isacommand{using}\isamarkupfalse%
\ notSI\ subsetInsert\ \isacommand{by}\isamarkupfalse%
\ {\isacharparenleft}rule\ contrapos{\isacharunderscore}nn{\isacharparenright}\isanewline
\ \ \ \ \ \ \isacommand{then}\isamarkupfalse%
\ \isacommand{have}\isamarkupfalse%
\ {\isachardoublequoteopen}G\ {\isasymin}\ S{\isadigit{1}}{\isachardoublequoteclose}\isanewline
\ \ \ \ \ \ \ \ \isacommand{by}\isamarkupfalse%
\ {\isacharparenleft}rule\ notnotD{\isacharparenright}\isanewline
\ \ \ \ \ \ \isacommand{then}\isamarkupfalse%
\ \isacommand{have}\isamarkupfalse%
\ {\isachardoublequoteopen}insert\ G\ S{\isadigit{1}}\ {\isacharequal}\ S{\isadigit{1}}{\isachardoublequoteclose}\isanewline
\ \ \ \ \ \ \ \ \isacommand{by}\isamarkupfalse%
\ {\isacharparenleft}rule\ insert{\isacharunderscore}absorb{\isacharparenright}\isanewline
\ \ \ \ \ \ \isacommand{show}\isamarkupfalse%
\ {\isacharquery}thesis\isanewline
\ \ \ \ \ \ \ \ \isacommand{using}\isamarkupfalse%
\ {\isacartoucheopen}S{\isadigit{1}}\ {\isasymnotin}\ C{\isacartoucheclose}\ \isacommand{by}\isamarkupfalse%
\ {\isacharparenleft}simp\ only{\isacharcolon}\ simp{\isacharunderscore}thms{\isacharparenleft}{\isadigit{8}}{\isacharparenright}\ {\isacartoucheopen}insert\ G\ S{\isadigit{1}}\ {\isacharequal}\ S{\isadigit{1}}{\isacartoucheclose}{\isacharparenright}\isanewline
\ \ \ \ \isacommand{qed}\isamarkupfalse%
\ \isanewline
\ \ \ \ \isacommand{let}\isamarkupfalse%
\ {\isacharquery}S{\isadigit{1}}{\isacharequal}{\isachardoublequoteopen}S{\isadigit{1}}\ {\isacharminus}\ {\isacharbraceleft}G{\isacharbraceright}{\isachardoublequoteclose}\isanewline
\ \ \ \ \isacommand{have}\isamarkupfalse%
\ {\isachardoublequoteopen}insert\ G\ S\ {\isacharequal}\ {\isacharbraceleft}G{\isacharbraceright}\ {\isasymunion}\ S{\isachardoublequoteclose}\isanewline
\ \ \ \ \ \ \isacommand{by}\isamarkupfalse%
\ {\isacharparenleft}rule\ insert{\isacharunderscore}is{\isacharunderscore}Un{\isacharparenright}\isanewline
\ \ \ \ \isacommand{have}\isamarkupfalse%
\ {\isachardoublequoteopen}S{\isadigit{1}}\ {\isasymsubseteq}\ {\isacharbraceleft}G{\isacharbraceright}\ {\isasymunion}\ S{\isachardoublequoteclose}\isanewline
\ \ \ \ \ \ \isacommand{using}\isamarkupfalse%
\ {\isacartoucheopen}S{\isadigit{1}}\ {\isasymsubseteq}\ insert\ G\ S{\isacartoucheclose}\ \isacommand{by}\isamarkupfalse%
\ {\isacharparenleft}simp\ only{\isacharcolon}\ {\isacartoucheopen}insert\ G\ S\ {\isacharequal}\ {\isacharbraceleft}G{\isacharbraceright}\ {\isasymunion}\ S{\isacartoucheclose}{\isacharparenright}\isanewline
\ \ \ \ \isacommand{have}\isamarkupfalse%
\ {\isadigit{1}}{\isacharcolon}{\isachardoublequoteopen}{\isacharquery}S{\isadigit{1}}\ {\isasymsubseteq}\ S{\isachardoublequoteclose}\ \isanewline
\ \ \ \ \ \ \isacommand{using}\isamarkupfalse%
\ {\isacartoucheopen}S{\isadigit{1}}\ {\isasymsubseteq}\ {\isacharbraceleft}G{\isacharbraceright}\ {\isasymunion}\ S{\isacartoucheclose}\ \isacommand{by}\isamarkupfalse%
\ {\isacharparenleft}simp\ only{\isacharcolon}\ Diff{\isacharunderscore}subset{\isacharunderscore}conv{\isacharparenright}\isanewline
\ \ \ \ \isacommand{have}\isamarkupfalse%
\ {\isadigit{2}}{\isacharcolon}{\isachardoublequoteopen}finite\ {\isacharquery}S{\isadigit{1}}{\isachardoublequoteclose}\isanewline
\ \ \ \ \ \ \isacommand{using}\isamarkupfalse%
\ {\isacartoucheopen}finite\ S{\isadigit{1}}{\isacartoucheclose}\ \isacommand{by}\isamarkupfalse%
\ {\isacharparenleft}simp\ only{\isacharcolon}\ finite{\isacharunderscore}Diff{\isacharparenright}\isanewline
\ \ \ \ \isacommand{have}\isamarkupfalse%
\ {\isachardoublequoteopen}insert\ G\ {\isacharquery}S{\isadigit{1}}\ {\isacharequal}\ insert\ G\ S{\isadigit{1}}{\isachardoublequoteclose}\isanewline
\ \ \ \ \ \ \isacommand{by}\isamarkupfalse%
\ {\isacharparenleft}simp\ only{\isacharcolon}\ insert{\isacharunderscore}Diff{\isacharunderscore}single{\isacharparenright}\isanewline
\ \ \ \ \isacommand{then}\isamarkupfalse%
\ \isacommand{have}\isamarkupfalse%
\ {\isadigit{3}}{\isacharcolon}{\isachardoublequoteopen}insert\ G\ {\isacharquery}S{\isadigit{1}}\ {\isasymnotin}\ C{\isachardoublequoteclose}\isanewline
\ \ \ \ \ \ \isacommand{using}\isamarkupfalse%
\ {\isacartoucheopen}insert\ G\ S{\isadigit{1}}\ {\isasymnotin}\ C{\isacartoucheclose}\ \isacommand{by}\isamarkupfalse%
\ {\isacharparenleft}simp\ only{\isacharcolon}\ simp{\isacharunderscore}thms{\isacharparenleft}{\isadigit{6}}{\isacharcomma}{\isadigit{8}}{\isacharparenright}\ {\isacartoucheopen}insert\ G\ {\isacharquery}S{\isadigit{1}}\ {\isacharequal}\ insert\ G\ S{\isadigit{1}}{\isacartoucheclose}{\isacharparenright}\isanewline
\ \ \ \ \isacommand{have}\isamarkupfalse%
\ {\isachardoublequoteopen}insert\ H\ S\ {\isasymnotin}\ {\isacharparenleft}extF\ C{\isacharparenright}{\isachardoublequoteclose}\isanewline
\ \ \ \ \ \ \isacommand{using}\isamarkupfalse%
\ Conj\ \isacommand{by}\isamarkupfalse%
\ {\isacharparenleft}rule\ conjunct{\isadigit{2}}{\isacharparenright}\isanewline
\ \ \ \ \isacommand{then}\isamarkupfalse%
\ \isacommand{have}\isamarkupfalse%
\ {\isachardoublequoteopen}{\isasymnot}{\isacharparenleft}{\isasymforall}S{\isacharprime}\ {\isasymsubseteq}\ {\isacharparenleft}insert\ H\ S{\isacharparenright}{\isachardot}\ finite\ S{\isacharprime}\ {\isasymlongrightarrow}\ S{\isacharprime}\ {\isasymin}\ C{\isacharparenright}{\isachardoublequoteclose}\isanewline
\ \ \ \ \ \ \isacommand{unfolding}\isamarkupfalse%
\ extF\ \isacommand{by}\isamarkupfalse%
\ {\isacharparenleft}simp\ add{\isacharcolon}\ mem{\isacharunderscore}Collect{\isacharunderscore}eq{\isacharparenright}\isanewline
\ \ \ \ \isacommand{then}\isamarkupfalse%
\ \isacommand{have}\isamarkupfalse%
\ Ex{\isadigit{2}}{\isacharcolon}{\isachardoublequoteopen}{\isasymexists}S{\isacharprime}{\isasymsubseteq}\ {\isacharparenleft}insert\ H\ S{\isacharparenright}{\isachardot}\ {\isasymnot}{\isacharparenleft}finite\ S{\isacharprime}\ {\isasymlongrightarrow}\ S{\isacharprime}\ {\isasymin}\ C{\isacharparenright}{\isachardoublequoteclose}\isanewline
\ \ \ \ \ \ \isacommand{by}\isamarkupfalse%
\ {\isacharparenleft}rule\ sall{\isacharunderscore}simps{\isacharunderscore}not{\isacharunderscore}all{\isacharparenright}\isanewline
\ \ \ \ \isacommand{obtain}\isamarkupfalse%
\ S{\isadigit{2}}\ \isakeyword{where}\ {\isachardoublequoteopen}S{\isadigit{2}}\ {\isasymsubseteq}\ insert\ H\ S{\isachardoublequoteclose}\ \isakeyword{and}\ {\isachardoublequoteopen}{\isasymnot}{\isacharparenleft}finite\ S{\isadigit{2}}\ {\isasymlongrightarrow}\ S{\isadigit{2}}\ {\isasymin}\ C{\isacharparenright}{\isachardoublequoteclose}\isanewline
\ \ \ \ \ \ \isacommand{using}\isamarkupfalse%
\ Ex{\isadigit{2}}\ \isacommand{by}\isamarkupfalse%
\ {\isacharparenleft}rule\ subexE{\isacharparenright}\isanewline
\ \ \ \ \isacommand{have}\isamarkupfalse%
\ {\isachardoublequoteopen}finite\ S{\isadigit{2}}\ {\isasymand}\ S{\isadigit{2}}\ {\isasymnotin}\ C{\isachardoublequoteclose}\isanewline
\ \ \ \ \ \ \isacommand{using}\isamarkupfalse%
\ {\isacartoucheopen}{\isasymnot}{\isacharparenleft}finite\ S{\isadigit{2}}\ {\isasymlongrightarrow}\ S{\isadigit{2}}\ {\isasymin}\ C{\isacharparenright}{\isacartoucheclose}\ \isacommand{by}\isamarkupfalse%
\ {\isacharparenleft}simp\ only{\isacharcolon}\ simp{\isacharunderscore}thms{\isacharparenleft}{\isadigit{8}}{\isacharcomma}{\isadigit{2}}{\isadigit{5}}{\isacharparenright}\ not{\isacharunderscore}imp{\isacharparenright}\isanewline
\ \ \ \ \isacommand{then}\isamarkupfalse%
\ \isacommand{have}\isamarkupfalse%
\ {\isachardoublequoteopen}finite\ S{\isadigit{2}}{\isachardoublequoteclose}\isanewline
\ \ \ \ \ \ \isacommand{by}\isamarkupfalse%
\ {\isacharparenleft}rule\ conjunct{\isadigit{1}}{\isacharparenright}\isanewline
\ \ \ \ \isacommand{have}\isamarkupfalse%
\ {\isachardoublequoteopen}S{\isadigit{2}}\ {\isasymnotin}\ C{\isachardoublequoteclose}\isanewline
\ \ \ \ \ \ \isacommand{using}\isamarkupfalse%
\ {\isacartoucheopen}finite\ S{\isadigit{2}}\ {\isasymand}\ S{\isadigit{2}}\ {\isasymnotin}\ C{\isacartoucheclose}\ \isacommand{by}\isamarkupfalse%
\ {\isacharparenleft}rule\ conjunct{\isadigit{2}}{\isacharparenright}\isanewline
\ \ \ \ \isacommand{then}\isamarkupfalse%
\ \isacommand{have}\isamarkupfalse%
\ {\isachardoublequoteopen}insert\ H\ S{\isadigit{2}}\ {\isasymnotin}\ C{\isachardoublequoteclose}\isanewline
\ \ \ \ \isacommand{proof}\isamarkupfalse%
\ {\isacharminus}\isanewline
\ \ \ \ \ \ \isacommand{have}\isamarkupfalse%
\ {\isachardoublequoteopen}S{\isadigit{2}}\ {\isasymsubseteq}\ S\ {\isasymlongrightarrow}\ finite\ S{\isadigit{2}}\ {\isasymlongrightarrow}\ S{\isadigit{2}}\ {\isasymin}\ C{\isachardoublequoteclose}\isanewline
\ \ \ \ \ \ \ \ \isacommand{using}\isamarkupfalse%
\ E{\isacharprime}\ \isacommand{by}\isamarkupfalse%
\ {\isacharparenleft}rule\ allE{\isacharparenright}\isanewline
\ \ \ \ \ \ \isacommand{then}\isamarkupfalse%
\ \isacommand{have}\isamarkupfalse%
\ {\isachardoublequoteopen}{\isasymnot}\ S{\isadigit{2}}\ {\isasymsubseteq}\ S{\isachardoublequoteclose}\isanewline
\ \ \ \ \ \ \ \ \isacommand{using}\isamarkupfalse%
\ {\isacartoucheopen}{\isasymnot}\ {\isacharparenleft}finite\ S{\isadigit{2}}\ {\isasymlongrightarrow}\ S{\isadigit{2}}\ {\isasymin}\ C{\isacharparenright}{\isacartoucheclose}\ \isacommand{by}\isamarkupfalse%
\ {\isacharparenleft}rule\ mt{\isacharparenright}\isanewline
\ \ \ \ \ \ \isacommand{then}\isamarkupfalse%
\ \isacommand{have}\isamarkupfalse%
\ {\isachardoublequoteopen}{\isacharparenleft}S{\isadigit{2}}\ {\isasymsubseteq}\ insert\ H\ S{\isacharparenright}\ {\isasymnoteq}\ {\isacharparenleft}S{\isadigit{2}}\ {\isasymsubseteq}\ S{\isacharparenright}{\isachardoublequoteclose}\isanewline
\ \ \ \ \ \ \ \ \isacommand{using}\isamarkupfalse%
\ {\isacartoucheopen}S{\isadigit{2}}\ {\isasymsubseteq}\ insert\ H\ S{\isacartoucheclose}\ \isacommand{by}\isamarkupfalse%
\ simp\ \isanewline
\ \ \ \ \ \ \isacommand{then}\isamarkupfalse%
\ \isacommand{have}\isamarkupfalse%
\ notSI{\isacharcolon}{\isachardoublequoteopen}{\isasymnot}{\isacharparenleft}S{\isadigit{2}}\ {\isasymsubseteq}\ insert\ H\ S\ {\isasymlongleftrightarrow}\ S{\isadigit{2}}\ {\isasymsubseteq}\ S{\isacharparenright}{\isachardoublequoteclose}\isanewline
\ \ \ \ \ \ \ \ \isacommand{by}\isamarkupfalse%
\ blast\ \isanewline
\ \ \ \ \ \ \isacommand{have}\isamarkupfalse%
\ subsetInsert{\isacharcolon}{\isachardoublequoteopen}H\ {\isasymnotin}\ S{\isadigit{2}}\ {\isasymLongrightarrow}\ S{\isadigit{2}}\ {\isasymsubseteq}\ insert\ H\ S\ {\isasymlongleftrightarrow}\ S{\isadigit{2}}\ {\isasymsubseteq}\ S{\isachardoublequoteclose}\isanewline
\ \ \ \ \ \ \ \ \isacommand{by}\isamarkupfalse%
\ {\isacharparenleft}rule\ subset{\isacharunderscore}insert{\isacharparenright}\isanewline
\ \ \ \ \ \ \isacommand{have}\isamarkupfalse%
\ {\isachardoublequoteopen}{\isasymnot}{\isacharparenleft}H\ {\isasymnotin}\ S{\isadigit{2}}{\isacharparenright}{\isachardoublequoteclose}\isanewline
\ \ \ \ \ \ \ \ \isacommand{using}\isamarkupfalse%
\ notSI\ subsetInsert\ \isacommand{by}\isamarkupfalse%
\ {\isacharparenleft}rule\ contrapos{\isacharunderscore}nn{\isacharparenright}\isanewline
\ \ \ \ \ \ \isacommand{then}\isamarkupfalse%
\ \isacommand{have}\isamarkupfalse%
\ {\isachardoublequoteopen}H\ {\isasymin}\ S{\isadigit{2}}{\isachardoublequoteclose}\isanewline
\ \ \ \ \ \ \ \ \isacommand{by}\isamarkupfalse%
\ {\isacharparenleft}rule\ notnotD{\isacharparenright}\isanewline
\ \ \ \ \ \ \isacommand{then}\isamarkupfalse%
\ \isacommand{have}\isamarkupfalse%
\ {\isachardoublequoteopen}insert\ H\ S{\isadigit{2}}\ {\isacharequal}\ S{\isadigit{2}}{\isachardoublequoteclose}\isanewline
\ \ \ \ \ \ \ \ \isacommand{by}\isamarkupfalse%
\ {\isacharparenleft}rule\ insert{\isacharunderscore}absorb{\isacharparenright}\isanewline
\ \ \ \ \ \ \isacommand{show}\isamarkupfalse%
\ {\isacharquery}thesis\isanewline
\ \ \ \ \ \ \ \ \isacommand{using}\isamarkupfalse%
\ {\isacartoucheopen}S{\isadigit{2}}\ {\isasymnotin}\ C{\isacartoucheclose}\ \isacommand{by}\isamarkupfalse%
\ {\isacharparenleft}simp\ only{\isacharcolon}\ simp{\isacharunderscore}thms{\isacharparenleft}{\isadigit{8}}{\isacharparenright}\ {\isacartoucheopen}insert\ H\ S{\isadigit{2}}\ {\isacharequal}\ S{\isadigit{2}}{\isacartoucheclose}{\isacharparenright}\isanewline
\ \ \ \ \isacommand{qed}\isamarkupfalse%
\ \isanewline
\ \ \ \ \isacommand{let}\isamarkupfalse%
\ {\isacharquery}S{\isadigit{2}}{\isacharequal}{\isachardoublequoteopen}S{\isadigit{2}}\ {\isacharminus}\ {\isacharbraceleft}H{\isacharbraceright}{\isachardoublequoteclose}\isanewline
\ \ \ \ \isacommand{have}\isamarkupfalse%
\ {\isachardoublequoteopen}insert\ H\ S\ {\isacharequal}\ {\isacharbraceleft}H{\isacharbraceright}\ {\isasymunion}\ S{\isachardoublequoteclose}\isanewline
\ \ \ \ \ \ \isacommand{by}\isamarkupfalse%
\ {\isacharparenleft}rule\ insert{\isacharunderscore}is{\isacharunderscore}Un{\isacharparenright}\isanewline
\ \ \ \ \isacommand{have}\isamarkupfalse%
\ {\isachardoublequoteopen}S{\isadigit{2}}\ {\isasymsubseteq}\ {\isacharbraceleft}H{\isacharbraceright}\ {\isasymunion}\ S{\isachardoublequoteclose}\isanewline
\ \ \ \ \ \ \isacommand{using}\isamarkupfalse%
\ {\isacartoucheopen}S{\isadigit{2}}\ {\isasymsubseteq}\ insert\ H\ S{\isacartoucheclose}\ \isacommand{by}\isamarkupfalse%
\ {\isacharparenleft}simp\ only{\isacharcolon}\ {\isacartoucheopen}insert\ H\ S\ {\isacharequal}\ {\isacharbraceleft}H{\isacharbraceright}\ {\isasymunion}\ S{\isacartoucheclose}{\isacharparenright}\isanewline
\ \ \ \ \isacommand{have}\isamarkupfalse%
\ {\isadigit{4}}{\isacharcolon}{\isachardoublequoteopen}{\isacharquery}S{\isadigit{2}}\ {\isasymsubseteq}\ S{\isachardoublequoteclose}\ \isanewline
\ \ \ \ \ \ \isacommand{using}\isamarkupfalse%
\ {\isacartoucheopen}S{\isadigit{2}}\ {\isasymsubseteq}\ {\isacharbraceleft}H{\isacharbraceright}\ {\isasymunion}\ S{\isacartoucheclose}\ \isacommand{by}\isamarkupfalse%
\ {\isacharparenleft}simp\ only{\isacharcolon}\ Diff{\isacharunderscore}subset{\isacharunderscore}conv{\isacharparenright}\isanewline
\ \ \ \ \isacommand{have}\isamarkupfalse%
\ {\isadigit{5}}{\isacharcolon}{\isachardoublequoteopen}finite\ {\isacharquery}S{\isadigit{2}}{\isachardoublequoteclose}\ \isanewline
\ \ \ \ \ \ \isacommand{using}\isamarkupfalse%
\ {\isacartoucheopen}finite\ S{\isadigit{2}}{\isacartoucheclose}\ \isacommand{by}\isamarkupfalse%
\ {\isacharparenleft}simp\ only{\isacharcolon}\ finite{\isacharunderscore}Diff{\isacharparenright}\isanewline
\ \ \ \ \isacommand{have}\isamarkupfalse%
\ {\isachardoublequoteopen}insert\ H\ {\isacharquery}S{\isadigit{2}}\ {\isacharequal}\ insert\ H\ S{\isadigit{2}}{\isachardoublequoteclose}\isanewline
\ \ \ \ \ \ \isacommand{by}\isamarkupfalse%
\ {\isacharparenleft}simp\ only{\isacharcolon}\ insert{\isacharunderscore}Diff{\isacharunderscore}single{\isacharparenright}\isanewline
\ \ \ \ \isacommand{then}\isamarkupfalse%
\ \isacommand{have}\isamarkupfalse%
\ {\isadigit{6}}{\isacharcolon}{\isachardoublequoteopen}insert\ H\ {\isacharquery}S{\isadigit{2}}\ {\isasymnotin}\ C{\isachardoublequoteclose}\isanewline
\ \ \ \ \ \ \isacommand{using}\isamarkupfalse%
\ {\isacartoucheopen}insert\ H\ S{\isadigit{2}}\ {\isasymnotin}\ C{\isacartoucheclose}\ \isacommand{by}\isamarkupfalse%
\ {\isacharparenleft}simp\ only{\isacharcolon}\ simp{\isacharunderscore}thms{\isacharparenleft}{\isadigit{6}}{\isacharcomma}{\isadigit{8}}{\isacharparenright}\ {\isacartoucheopen}insert\ H\ {\isacharquery}S{\isadigit{2}}\ {\isacharequal}\ insert\ H\ S{\isadigit{2}}{\isacartoucheclose}{\isacharparenright}\isanewline
\ \ \ \ \isacommand{show}\isamarkupfalse%
\ {\isachardoublequoteopen}False{\isachardoublequoteclose}\isanewline
\ \ \ \ \ \ \isacommand{using}\isamarkupfalse%
\ assms{\isacharparenleft}{\isadigit{1}}{\isacharparenright}\ assms{\isacharparenleft}{\isadigit{2}}{\isacharparenright}\ assms{\isacharparenleft}{\isadigit{3}}{\isacharparenright}\ assms{\isacharparenleft}{\isadigit{4}}{\isacharparenright}\ assms{\isacharparenleft}{\isadigit{5}}{\isacharparenright}\ {\isadigit{1}}\ {\isadigit{2}}\ {\isadigit{3}}\ {\isadigit{4}}\ {\isadigit{5}}\ {\isadigit{6}}\ \isacommand{by}\isamarkupfalse%
\ {\isacharparenleft}rule\ ex{\isadigit{3}}{\isacharunderscore}pcp{\isacharunderscore}SinE{\isacharunderscore}DIS{\isacharunderscore}auxFalse{\isacharparenright}\isanewline
\ \ \isacommand{qed}\isamarkupfalse%
\isanewline
\ \ \isacommand{thus}\isamarkupfalse%
\ {\isacharquery}thesis\isanewline
\ \ \isacommand{proof}\isamarkupfalse%
\ {\isacharparenleft}rule\ disjE{\isacharparenright}\isanewline
\ \ \ \ \isacommand{assume}\isamarkupfalse%
\ {\isachardoublequoteopen}insert\ G\ S\ {\isasymin}\ {\isacharparenleft}extF\ C{\isacharparenright}{\isachardoublequoteclose}\isanewline
\ \ \ \ \isacommand{have}\isamarkupfalse%
\ insG{\isacharcolon}{\isachardoublequoteopen}insert\ G\ S\ {\isasymin}\ {\isacharparenleft}extensionFin\ C{\isacharparenright}{\isachardoublequoteclose}\isanewline
\ \ \ \ \ \ \isacommand{using}\isamarkupfalse%
\ {\isacartoucheopen}{\isacharparenleft}extF\ C{\isacharparenright}\ {\isasymsubseteq}\ {\isacharparenleft}extensionFin\ C{\isacharparenright}{\isacartoucheclose}\ {\isacartoucheopen}insert\ G\ S\ {\isasymin}\ {\isacharparenleft}extF\ C{\isacharparenright}{\isacartoucheclose}\ \isacommand{by}\isamarkupfalse%
\ {\isacharparenleft}simp\ only{\isacharcolon}\ in{\isacharunderscore}mono{\isacharparenright}\isanewline
\ \ \ \ \isacommand{have}\isamarkupfalse%
\ {\isachardoublequoteopen}insert\ G\ S\ {\isacharequal}\ {\isacharbraceleft}G{\isacharbraceright}\ {\isasymunion}\ S{\isachardoublequoteclose}\isanewline
\ \ \ \ \ \ \isacommand{by}\isamarkupfalse%
\ {\isacharparenleft}rule\ insert{\isacharunderscore}is{\isacharunderscore}Un{\isacharparenright}\isanewline
\ \ \ \ \isacommand{then}\isamarkupfalse%
\ \isacommand{have}\isamarkupfalse%
\ {\isachardoublequoteopen}{\isacharbraceleft}G{\isacharbraceright}\ {\isasymunion}\ S\ {\isasymin}\ {\isacharparenleft}extensionFin\ C{\isacharparenright}{\isachardoublequoteclose}\isanewline
\ \ \ \ \ \ \isacommand{using}\isamarkupfalse%
\ insG\ {\isacartoucheopen}insert\ G\ S\ {\isacharequal}\ {\isacharbraceleft}G{\isacharbraceright}\ {\isasymunion}\ S{\isacartoucheclose}\ \isacommand{by}\isamarkupfalse%
\ {\isacharparenleft}simp\ only{\isacharcolon}\ insG{\isacharparenright}\isanewline
\ \ \ \ \isacommand{thus}\isamarkupfalse%
\ {\isacharquery}thesis\isanewline
\ \ \ \ \ \ \isacommand{by}\isamarkupfalse%
\ {\isacharparenleft}rule\ disjI{\isadigit{1}}{\isacharparenright}\isanewline
\ \ \isacommand{next}\isamarkupfalse%
\isanewline
\ \ \ \ \isacommand{assume}\isamarkupfalse%
\ {\isachardoublequoteopen}insert\ H\ S\ {\isasymin}\ {\isacharparenleft}extF\ C{\isacharparenright}{\isachardoublequoteclose}\isanewline
\ \ \ \ \isacommand{have}\isamarkupfalse%
\ insH{\isacharcolon}{\isachardoublequoteopen}insert\ H\ S\ {\isasymin}\ {\isacharparenleft}extensionFin\ C{\isacharparenright}{\isachardoublequoteclose}\isanewline
\ \ \ \ \ \ \isacommand{using}\isamarkupfalse%
\ {\isacartoucheopen}{\isacharparenleft}extF\ C{\isacharparenright}\ {\isasymsubseteq}\ {\isacharparenleft}extensionFin\ C{\isacharparenright}{\isacartoucheclose}\ {\isacartoucheopen}insert\ H\ S\ {\isasymin}\ {\isacharparenleft}extF\ C{\isacharparenright}{\isacartoucheclose}\ \isacommand{by}\isamarkupfalse%
\ {\isacharparenleft}simp\ only{\isacharcolon}\ in{\isacharunderscore}mono{\isacharparenright}\isanewline
\ \ \ \ \isacommand{have}\isamarkupfalse%
\ {\isachardoublequoteopen}insert\ H\ S\ {\isacharequal}\ {\isacharbraceleft}H{\isacharbraceright}\ {\isasymunion}\ S{\isachardoublequoteclose}\isanewline
\ \ \ \ \ \ \isacommand{by}\isamarkupfalse%
\ {\isacharparenleft}rule\ insert{\isacharunderscore}is{\isacharunderscore}Un{\isacharparenright}\isanewline
\ \ \ \ \isacommand{then}\isamarkupfalse%
\ \isacommand{have}\isamarkupfalse%
\ {\isachardoublequoteopen}{\isacharbraceleft}H{\isacharbraceright}\ {\isasymunion}\ S\ {\isasymin}\ {\isacharparenleft}extensionFin\ C{\isacharparenright}{\isachardoublequoteclose}\isanewline
\ \ \ \ \ \ \isacommand{using}\isamarkupfalse%
\ insH\ {\isacartoucheopen}insert\ H\ S\ {\isacharequal}\ {\isacharbraceleft}H{\isacharbraceright}\ {\isasymunion}\ S{\isacartoucheclose}\ \isacommand{by}\isamarkupfalse%
\ {\isacharparenleft}simp\ only{\isacharcolon}\ insH{\isacharparenright}\isanewline
\ \ \ \ \isacommand{thus}\isamarkupfalse%
\ {\isacharquery}thesis\isanewline
\ \ \ \ \ \ \isacommand{by}\isamarkupfalse%
\ {\isacharparenleft}rule\ disjI{\isadigit{2}}{\isacharparenright}\isanewline
\ \ \isacommand{qed}\isamarkupfalse%
\isanewline
\isacommand{qed}\isamarkupfalse%
%
\endisatagproof
{\isafoldproof}%
%
\isadelimproof
%
\endisadelimproof
%
\begin{isamarkuptext}%
Probados los lemas \isa{ex{\isadigit{3}}{\isacharunderscore}pcp{\isacharunderscore}SinE{\isacharunderscore}CON} y \isa{ex{\isadigit{3}}{\isacharunderscore}pcp{\isacharunderscore}SinE{\isacharunderscore}DIS}, podemos demostrar que \isa{C{\isacharprime}\ {\isacharequal}\ C\ {\isasymunion}\ E} 
  verifica las condiciones del lema de caracterización de la propiedad de consistencia proposicional 
  para el caso en que \isa{S\ {\isasymin}\ E}, formalizado como \isa{ex{\isadigit{3}}{\isacharunderscore}pcp{\isacharunderscore}SinE}. Dicho lema prueba que, si \isa{C} es 
  una colección con la propiedad de consistencia proposicional y cerrada bajo subconjuntos, y sea 
  \isa{S\ {\isasymin}\ E}, se verifican las condiciones:
  \begin{itemize}
    \item \isa{{\isasymbottom}\ {\isasymnotin}\ S}.
    \item Dada \isa{p} una fórmula atómica cualquiera, no se tiene 
    simultáneamente que\\ \isa{p\ {\isasymin}\ S} y \isa{{\isasymnot}\ p\ {\isasymin}\ S}.
    \item Para toda fórmula de tipo \isa{{\isasymalpha}} con componentes \isa{{\isasymalpha}\isactrlsub {\isadigit{1}}} y \isa{{\isasymalpha}\isactrlsub {\isadigit{2}}} tal que \isa{{\isasymalpha}}
    pertenece a \isa{S}, se tiene que \isa{{\isacharbraceleft}{\isasymalpha}\isactrlsub {\isadigit{1}}{\isacharcomma}{\isasymalpha}\isactrlsub {\isadigit{2}}{\isacharbraceright}\ {\isasymunion}\ S} pertenece a \isa{C{\isacharprime}}.
    \item Para toda fórmula de tipo \isa{{\isasymbeta}} con componentes \isa{{\isasymbeta}\isactrlsub {\isadigit{1}}} y \isa{{\isasymbeta}\isactrlsub {\isadigit{2}}} tal que \isa{{\isasymbeta}}
    pertenece a \isa{S}, se tiene que o bien \isa{{\isacharbraceleft}{\isasymbeta}\isactrlsub {\isadigit{1}}{\isacharbraceright}\ {\isasymunion}\ S} pertenece a \isa{C{\isacharprime}} o 
    bien \isa{{\isacharbraceleft}{\isasymbeta}\isactrlsub {\isadigit{2}}{\isacharbraceright}\ {\isasymunion}\ S} pertenece a \isa{C{\isacharprime}}.
  \end{itemize}%
\end{isamarkuptext}\isamarkuptrue%
\isacommand{lemma}\isamarkupfalse%
\ ex{\isadigit{3}}{\isacharunderscore}pcp{\isacharunderscore}SinE{\isacharcolon}\isanewline
\ \ \isakeyword{assumes}\ {\isachardoublequoteopen}pcp\ C{\isachardoublequoteclose}\isanewline
\ \ \ \ \ \ \ \ \ \ {\isachardoublequoteopen}subset{\isacharunderscore}closed\ C{\isachardoublequoteclose}\isanewline
\ \ \ \ \ \ \ \ \ \ {\isachardoublequoteopen}S\ {\isasymin}\ {\isacharparenleft}extF\ C{\isacharparenright}{\isachardoublequoteclose}\ \isanewline
\ \ \isakeyword{shows}\ {\isachardoublequoteopen}{\isasymbottom}\ {\isasymnotin}\ S\ {\isasymand}\isanewline
\ \ \ \ \ \ \ \ \ {\isacharparenleft}{\isasymforall}k{\isachardot}\ Atom\ k\ {\isasymin}\ S\ {\isasymlongrightarrow}\ \isactrlbold {\isasymnot}\ {\isacharparenleft}Atom\ k{\isacharparenright}\ {\isasymin}\ S\ {\isasymlongrightarrow}\ False{\isacharparenright}\ {\isasymand}\isanewline
\ \ \ \ \ \ \ \ \ {\isacharparenleft}{\isasymforall}F\ G\ H{\isachardot}\ Con\ F\ G\ H\ {\isasymlongrightarrow}\ F\ {\isasymin}\ S\ {\isasymlongrightarrow}\ {\isacharbraceleft}G{\isacharcomma}\ H{\isacharbraceright}\ {\isasymunion}\ S\ {\isasymin}\ {\isacharparenleft}extensionFin\ C{\isacharparenright}{\isacharparenright}\ {\isasymand}\isanewline
\ \ \ \ \ \ \ \ \ {\isacharparenleft}{\isasymforall}F\ G\ H{\isachardot}\ Dis\ F\ G\ H\ {\isasymlongrightarrow}\ F\ {\isasymin}\ S\ {\isasymlongrightarrow}\ {\isacharbraceleft}G{\isacharbraceright}\ {\isasymunion}\ S\ {\isasymin}\ {\isacharparenleft}extensionFin\ C{\isacharparenright}\ {\isasymor}\ {\isacharbraceleft}H{\isacharbraceright}\ {\isasymunion}\ S\ {\isasymin}\ {\isacharparenleft}extensionFin\ C{\isacharparenright}{\isacharparenright}{\isachardoublequoteclose}\isanewline
%
\isadelimproof
%
\endisadelimproof
%
\isatagproof
\isacommand{proof}\isamarkupfalse%
\ {\isacharminus}\isanewline
\ \ \isacommand{have}\isamarkupfalse%
\ PCP{\isacharcolon}{\isachardoublequoteopen}{\isasymforall}S\ {\isasymin}\ C{\isachardot}\isanewline
\ \ \ \ \ \ \ \ \ {\isasymbottom}\ {\isasymnotin}\ S\ {\isasymand}\isanewline
\ \ \ \ \ \ \ \ \ {\isacharparenleft}{\isasymforall}k{\isachardot}\ Atom\ k\ {\isasymin}\ S\ {\isasymlongrightarrow}\ \isactrlbold {\isasymnot}\ {\isacharparenleft}Atom\ k{\isacharparenright}\ {\isasymin}\ S\ {\isasymlongrightarrow}\ False{\isacharparenright}\ {\isasymand}\isanewline
\ \ \ \ \ \ \ \ \ {\isacharparenleft}{\isasymforall}F\ G\ H{\isachardot}\ Con\ F\ G\ H\ {\isasymlongrightarrow}\ F\ {\isasymin}\ S\ {\isasymlongrightarrow}\ {\isacharbraceleft}G{\isacharcomma}\ H{\isacharbraceright}\ {\isasymunion}\ S\ {\isasymin}\ C{\isacharparenright}\ {\isasymand}\isanewline
\ \ \ \ \ \ \ \ \ {\isacharparenleft}{\isasymforall}F\ G\ H{\isachardot}\ Dis\ F\ G\ H\ {\isasymlongrightarrow}\ F\ {\isasymin}\ S\ {\isasymlongrightarrow}\ {\isacharbraceleft}G{\isacharbraceright}\ {\isasymunion}\ S\ {\isasymin}\ C\ {\isasymor}\ {\isacharbraceleft}H{\isacharbraceright}\ {\isasymunion}\ S\ {\isasymin}\ C{\isacharparenright}{\isachardoublequoteclose}\isanewline
\ \ \ \ \isacommand{using}\isamarkupfalse%
\ assms{\isacharparenleft}{\isadigit{1}}{\isacharparenright}\ \isacommand{by}\isamarkupfalse%
\ {\isacharparenleft}rule\ pcp{\isacharunderscore}alt{\isadigit{1}}{\isacharparenright}\isanewline
\ \ \isacommand{have}\isamarkupfalse%
\ E{\isacharcolon}{\isachardoublequoteopen}{\isasymforall}S{\isacharprime}\ {\isasymsubseteq}\ S{\isachardot}\ finite\ S{\isacharprime}\ {\isasymlongrightarrow}\ S{\isacharprime}\ {\isasymin}\ C{\isachardoublequoteclose}\isanewline
\ \ \ \ \isacommand{using}\isamarkupfalse%
\ assms{\isacharparenleft}{\isadigit{3}}{\isacharparenright}\ \isacommand{unfolding}\isamarkupfalse%
\ extF\ \isacommand{by}\isamarkupfalse%
\ {\isacharparenleft}rule\ CollectD{\isacharparenright}\isanewline
\ \ \isacommand{have}\isamarkupfalse%
\ {\isachardoublequoteopen}{\isacharbraceleft}{\isacharbraceright}\ {\isasymsubseteq}\ S{\isachardoublequoteclose}\isanewline
\ \ \ \ \isacommand{by}\isamarkupfalse%
\ {\isacharparenleft}rule\ empty{\isacharunderscore}subsetI{\isacharparenright}\isanewline
\ \ \isacommand{have}\isamarkupfalse%
\ C{\isadigit{1}}{\isacharcolon}{\isachardoublequoteopen}{\isasymbottom}\ {\isasymnotin}\ S{\isachardoublequoteclose}\isanewline
\ \ \isacommand{proof}\isamarkupfalse%
\ {\isacharparenleft}rule\ ccontr{\isacharparenright}\isanewline
\ \ \ \ \isacommand{assume}\isamarkupfalse%
\ {\isachardoublequoteopen}{\isasymnot}{\isacharparenleft}{\isasymbottom}\ {\isasymnotin}\ S{\isacharparenright}{\isachardoublequoteclose}\isanewline
\ \ \ \ \isacommand{then}\isamarkupfalse%
\ \isacommand{have}\isamarkupfalse%
\ {\isachardoublequoteopen}{\isasymbottom}\ {\isasymin}\ S{\isachardoublequoteclose}\isanewline
\ \ \ \ \ \ \isacommand{by}\isamarkupfalse%
\ {\isacharparenleft}rule\ notnotD{\isacharparenright}\isanewline
\ \ \ \ \isacommand{then}\isamarkupfalse%
\ \isacommand{have}\isamarkupfalse%
\ {\isachardoublequoteopen}{\isasymbottom}\ {\isasymin}\ S\ {\isasymand}\ {\isacharbraceleft}{\isacharbraceright}\ {\isasymsubseteq}\ S{\isachardoublequoteclose}\isanewline
\ \ \ \ \ \ \isacommand{using}\isamarkupfalse%
\ {\isacartoucheopen}{\isacharbraceleft}{\isacharbraceright}\ {\isasymsubseteq}\ S{\isacartoucheclose}\ \isacommand{by}\isamarkupfalse%
\ {\isacharparenleft}rule\ conjI{\isacharparenright}\isanewline
\ \ \ \ \isacommand{then}\isamarkupfalse%
\ \isacommand{have}\isamarkupfalse%
\ {\isachardoublequoteopen}insert\ {\isasymbottom}\ {\isacharbraceleft}{\isacharbraceright}\ {\isasymsubseteq}\ S{\isachardoublequoteclose}\ \isanewline
\ \ \ \ \ \ \isacommand{by}\isamarkupfalse%
\ {\isacharparenleft}simp\ only{\isacharcolon}\ insert{\isacharunderscore}subset{\isacharparenright}\isanewline
\ \ \ \ \isacommand{have}\isamarkupfalse%
\ {\isachardoublequoteopen}finite\ {\isacharbraceleft}{\isacharbraceright}{\isachardoublequoteclose}\isanewline
\ \ \ \ \ \ \isacommand{by}\isamarkupfalse%
\ {\isacharparenleft}rule\ finite{\isachardot}emptyI{\isacharparenright}\isanewline
\ \ \ \ \isacommand{then}\isamarkupfalse%
\ \isacommand{have}\isamarkupfalse%
\ {\isachardoublequoteopen}finite\ {\isacharparenleft}insert\ {\isasymbottom}\ {\isacharbraceleft}{\isacharbraceright}{\isacharparenright}{\isachardoublequoteclose}\isanewline
\ \ \ \ \ \ \isacommand{by}\isamarkupfalse%
\ {\isacharparenleft}rule\ finite{\isachardot}insertI{\isacharparenright}\isanewline
\ \ \ \ \isacommand{have}\isamarkupfalse%
\ {\isachardoublequoteopen}finite\ {\isacharparenleft}insert\ {\isasymbottom}\ {\isacharbraceleft}{\isacharbraceright}{\isacharparenright}\ {\isasymlongrightarrow}\ {\isacharparenleft}insert\ {\isasymbottom}\ {\isacharbraceleft}{\isacharbraceright}{\isacharparenright}\ {\isasymin}\ C{\isachardoublequoteclose}\isanewline
\ \ \ \ \ \ \isacommand{using}\isamarkupfalse%
\ E\ {\isacartoucheopen}{\isacharparenleft}insert\ {\isasymbottom}\ {\isacharbraceleft}{\isacharbraceright}{\isacharparenright}\ {\isasymsubseteq}\ S{\isacartoucheclose}\ \isacommand{by}\isamarkupfalse%
\ simp\ \isanewline
\ \ \ \ \isacommand{then}\isamarkupfalse%
\ \isacommand{have}\isamarkupfalse%
\ {\isachardoublequoteopen}{\isacharparenleft}insert\ {\isasymbottom}\ {\isacharbraceleft}{\isacharbraceright}{\isacharparenright}\ {\isasymin}\ C{\isachardoublequoteclose}\isanewline
\ \ \ \ \ \ \isacommand{using}\isamarkupfalse%
\ {\isacartoucheopen}finite\ {\isacharparenleft}insert\ {\isasymbottom}\ {\isacharbraceleft}{\isacharbraceright}{\isacharparenright}{\isacartoucheclose}\ \isacommand{by}\isamarkupfalse%
\ {\isacharparenleft}rule\ mp{\isacharparenright}\isanewline
\ \ \ \ \isacommand{have}\isamarkupfalse%
\ {\isachardoublequoteopen}{\isasymbottom}\ {\isasymnotin}\ {\isacharparenleft}insert\ {\isasymbottom}\ {\isacharbraceleft}{\isacharbraceright}{\isacharparenright}\ {\isasymand}\isanewline
\ \ \ \ \ \ \ \ \ {\isacharparenleft}{\isasymforall}k{\isachardot}\ Atom\ k\ {\isasymin}\ {\isacharparenleft}insert\ {\isasymbottom}\ {\isacharbraceleft}{\isacharbraceright}{\isacharparenright}\ {\isasymlongrightarrow}\ \isactrlbold {\isasymnot}\ {\isacharparenleft}Atom\ k{\isacharparenright}\ {\isasymin}\ {\isacharparenleft}insert\ {\isasymbottom}\ {\isacharbraceleft}{\isacharbraceright}{\isacharparenright}\ {\isasymlongrightarrow}\ False{\isacharparenright}\ {\isasymand}\isanewline
\ \ \ \ \ \ \ \ \ {\isacharparenleft}{\isasymforall}F\ G\ H{\isachardot}\ Con\ F\ G\ H\ {\isasymlongrightarrow}\ F\ {\isasymin}\ {\isacharparenleft}insert\ {\isasymbottom}\ {\isacharbraceleft}{\isacharbraceright}{\isacharparenright}\ {\isasymlongrightarrow}\ {\isacharbraceleft}G{\isacharcomma}\ H{\isacharbraceright}\ {\isasymunion}\ {\isacharparenleft}insert\ {\isasymbottom}\ {\isacharbraceleft}{\isacharbraceright}{\isacharparenright}\ {\isasymin}\ C{\isacharparenright}\ {\isasymand}\isanewline
\ \ \ \ \ \ \ \ \ {\isacharparenleft}{\isasymforall}F\ G\ H{\isachardot}\ Dis\ F\ G\ H\ {\isasymlongrightarrow}\ F\ {\isasymin}\ {\isacharparenleft}insert\ {\isasymbottom}\ {\isacharbraceleft}{\isacharbraceright}{\isacharparenright}\ {\isasymlongrightarrow}\ {\isacharbraceleft}G{\isacharbraceright}\ {\isasymunion}\ {\isacharparenleft}insert\ {\isasymbottom}\ {\isacharbraceleft}{\isacharbraceright}{\isacharparenright}\ {\isasymin}\ C\ {\isasymor}\ {\isacharbraceleft}H{\isacharbraceright}\ {\isasymunion}\ {\isacharparenleft}insert\ {\isasymbottom}\ {\isacharbraceleft}{\isacharbraceright}{\isacharparenright}\ {\isasymin}\ C{\isacharparenright}{\isachardoublequoteclose}\isanewline
\ \ \ \ \ \ \isacommand{using}\isamarkupfalse%
\ PCP\ {\isacartoucheopen}{\isacharparenleft}insert\ {\isasymbottom}\ {\isacharbraceleft}{\isacharbraceright}{\isacharparenright}\ {\isasymin}\ C{\isacartoucheclose}\ \isacommand{by}\isamarkupfalse%
\ blast\ \isanewline
\ \ \ \ \isacommand{then}\isamarkupfalse%
\ \isacommand{have}\isamarkupfalse%
\ {\isachardoublequoteopen}{\isasymbottom}\ {\isasymnotin}\ {\isacharparenleft}insert\ {\isasymbottom}\ {\isacharbraceleft}{\isacharbraceright}{\isacharparenright}{\isachardoublequoteclose}\isanewline
\ \ \ \ \ \ \isacommand{by}\isamarkupfalse%
\ {\isacharparenleft}rule\ conjunct{\isadigit{1}}{\isacharparenright}\isanewline
\ \ \ \ \isacommand{have}\isamarkupfalse%
\ {\isachardoublequoteopen}{\isasymbottom}\ {\isasymin}\ {\isacharparenleft}insert\ {\isasymbottom}\ {\isacharbraceleft}{\isacharbraceright}{\isacharparenright}{\isachardoublequoteclose}\isanewline
\ \ \ \ \ \ \isacommand{by}\isamarkupfalse%
\ {\isacharparenleft}rule\ insertI{\isadigit{1}}{\isacharparenright}\isanewline
\ \ \ \ \isacommand{show}\isamarkupfalse%
\ {\isachardoublequoteopen}False{\isachardoublequoteclose}\isanewline
\ \ \ \ \ \ \isacommand{using}\isamarkupfalse%
\ {\isacartoucheopen}{\isasymbottom}\ {\isasymnotin}\ {\isacharparenleft}insert\ {\isasymbottom}\ {\isacharbraceleft}{\isacharbraceright}{\isacharparenright}{\isacartoucheclose}\ {\isacartoucheopen}{\isasymbottom}\ {\isasymin}\ {\isacharparenleft}insert\ {\isasymbottom}\ {\isacharbraceleft}{\isacharbraceright}{\isacharparenright}{\isacartoucheclose}\ \isacommand{by}\isamarkupfalse%
\ {\isacharparenleft}rule\ notE{\isacharparenright}\isanewline
\ \ \isacommand{qed}\isamarkupfalse%
\isanewline
\ \ \isacommand{have}\isamarkupfalse%
\ C{\isadigit{2}}{\isacharcolon}{\isachardoublequoteopen}{\isasymforall}k{\isachardot}\ Atom\ k\ {\isasymin}\ S\ {\isasymlongrightarrow}\ \isactrlbold {\isasymnot}\ {\isacharparenleft}Atom\ k{\isacharparenright}\ {\isasymin}\ S\ {\isasymlongrightarrow}\ False{\isachardoublequoteclose}\isanewline
\ \ \isacommand{proof}\isamarkupfalse%
\ {\isacharparenleft}rule\ allI{\isacharparenright}\isanewline
\ \ \ \ \isacommand{fix}\isamarkupfalse%
\ k\isanewline
\ \ \ \ \isacommand{show}\isamarkupfalse%
\ {\isachardoublequoteopen}Atom\ k\ {\isasymin}\ S\ {\isasymlongrightarrow}\ \isactrlbold {\isasymnot}{\isacharparenleft}Atom\ k{\isacharparenright}\ {\isasymin}\ S\ {\isasymlongrightarrow}\ False{\isachardoublequoteclose}\isanewline
\ \ \ \ \isacommand{proof}\isamarkupfalse%
\ {\isacharparenleft}rule\ impI{\isacharparenright}{\isacharplus}\isanewline
\ \ \ \ \ \ \isacommand{assume}\isamarkupfalse%
\ {\isachardoublequoteopen}Atom\ k\ {\isasymin}\ S{\isachardoublequoteclose}\isanewline
\ \ \ \ \ \ \isacommand{assume}\isamarkupfalse%
\ {\isachardoublequoteopen}\isactrlbold {\isasymnot}{\isacharparenleft}Atom\ k{\isacharparenright}\ {\isasymin}\ S{\isachardoublequoteclose}\isanewline
\ \ \ \ \ \ \isacommand{let}\isamarkupfalse%
\ {\isacharquery}A{\isacharequal}{\isachardoublequoteopen}insert\ {\isacharparenleft}Atom\ k{\isacharparenright}\ {\isacharparenleft}insert\ {\isacharparenleft}\isactrlbold {\isasymnot}{\isacharparenleft}Atom\ k{\isacharparenright}{\isacharparenright}\ {\isacharbraceleft}{\isacharbraceright}{\isacharparenright}{\isachardoublequoteclose}\isanewline
\ \ \ \ \ \ \isacommand{have}\isamarkupfalse%
\ {\isachardoublequoteopen}Atom\ k\ {\isasymin}\ {\isacharquery}A{\isachardoublequoteclose}\isanewline
\ \ \ \ \ \ \ \ \isacommand{by}\isamarkupfalse%
\ {\isacharparenleft}simp\ only{\isacharcolon}\ insert{\isacharunderscore}iff\ simp{\isacharunderscore}thms{\isacharparenright}\ \isanewline
\ \ \ \ \ \ \isacommand{have}\isamarkupfalse%
\ {\isachardoublequoteopen}\isactrlbold {\isasymnot}{\isacharparenleft}Atom\ k{\isacharparenright}\ {\isasymin}\ {\isacharquery}A{\isachardoublequoteclose}\isanewline
\ \ \ \ \ \ \ \ \isacommand{by}\isamarkupfalse%
\ {\isacharparenleft}simp\ only{\isacharcolon}\ insert{\isacharunderscore}iff\ simp{\isacharunderscore}thms{\isacharparenright}\ \isanewline
\ \ \ \ \ \ \isacommand{have}\isamarkupfalse%
\ inSubset{\isacharcolon}{\isachardoublequoteopen}insert\ {\isacharparenleft}\isactrlbold {\isasymnot}{\isacharparenleft}Atom\ k{\isacharparenright}{\isacharparenright}\ {\isacharbraceleft}{\isacharbraceright}\ {\isasymsubseteq}\ S{\isachardoublequoteclose}\isanewline
\ \ \ \ \ \ \ \ \isacommand{using}\isamarkupfalse%
\ {\isacartoucheopen}\isactrlbold {\isasymnot}{\isacharparenleft}Atom\ k{\isacharparenright}\ {\isasymin}\ S{\isacartoucheclose}\ {\isacartoucheopen}{\isacharbraceleft}{\isacharbraceright}\ {\isasymsubseteq}\ S{\isacartoucheclose}\ \isacommand{by}\isamarkupfalse%
\ {\isacharparenleft}simp\ only{\isacharcolon}\ insert{\isacharunderscore}subset{\isacharparenright}\isanewline
\ \ \ \ \ \ \isacommand{have}\isamarkupfalse%
\ {\isachardoublequoteopen}{\isacharquery}A\ {\isasymsubseteq}\ S{\isachardoublequoteclose}\isanewline
\ \ \ \ \ \ \ \ \isacommand{using}\isamarkupfalse%
\ inSubset\ {\isacartoucheopen}Atom\ k\ {\isasymin}\ S{\isacartoucheclose}\ \isacommand{by}\isamarkupfalse%
\ {\isacharparenleft}simp\ only{\isacharcolon}\ insert{\isacharunderscore}subset{\isacharparenright}\isanewline
\ \ \ \ \ \ \isacommand{have}\isamarkupfalse%
\ {\isachardoublequoteopen}finite\ {\isacharbraceleft}{\isacharbraceright}{\isachardoublequoteclose}\isanewline
\ \ \ \ \ \ \ \ \isacommand{by}\isamarkupfalse%
\ {\isacharparenleft}simp\ only{\isacharcolon}\ finite{\isachardot}emptyI{\isacharparenright}\isanewline
\ \ \ \ \ \ \isacommand{then}\isamarkupfalse%
\ \isacommand{have}\isamarkupfalse%
\ {\isachardoublequoteopen}finite\ {\isacharparenleft}insert\ {\isacharparenleft}\isactrlbold {\isasymnot}{\isacharparenleft}Atom\ k{\isacharparenright}{\isacharparenright}\ {\isacharbraceleft}{\isacharbraceright}{\isacharparenright}{\isachardoublequoteclose}\isanewline
\ \ \ \ \ \ \ \ \isacommand{by}\isamarkupfalse%
\ {\isacharparenleft}rule\ finite{\isachardot}insertI{\isacharparenright}\isanewline
\ \ \ \ \ \ \isacommand{then}\isamarkupfalse%
\ \isacommand{have}\isamarkupfalse%
\ {\isachardoublequoteopen}finite\ {\isacharquery}A{\isachardoublequoteclose}\isanewline
\ \ \ \ \ \ \ \ \isacommand{by}\isamarkupfalse%
\ {\isacharparenleft}rule\ finite{\isachardot}insertI{\isacharparenright}\isanewline
\ \ \ \ \ \ \isacommand{have}\isamarkupfalse%
\ {\isachardoublequoteopen}finite\ {\isacharquery}A\ {\isasymlongrightarrow}\ {\isacharquery}A\ {\isasymin}\ C{\isachardoublequoteclose}\isanewline
\ \ \ \ \ \ \ \ \isacommand{using}\isamarkupfalse%
\ E\ {\isacartoucheopen}{\isacharquery}A\ {\isasymsubseteq}\ S{\isacartoucheclose}\ \isacommand{by}\isamarkupfalse%
\ {\isacharparenleft}rule\ sspec{\isacharparenright}\isanewline
\ \ \ \ \ \ \isacommand{then}\isamarkupfalse%
\ \isacommand{have}\isamarkupfalse%
\ {\isachardoublequoteopen}{\isacharquery}A\ {\isasymin}\ C{\isachardoublequoteclose}\isanewline
\ \ \ \ \ \ \ \ \isacommand{using}\isamarkupfalse%
\ {\isacartoucheopen}finite\ {\isacharquery}A{\isacartoucheclose}\ \isacommand{by}\isamarkupfalse%
\ {\isacharparenleft}rule\ mp{\isacharparenright}\isanewline
\ \ \ \ \ \ \isacommand{have}\isamarkupfalse%
\ {\isachardoublequoteopen}{\isasymbottom}\ {\isasymnotin}\ {\isacharquery}A\isanewline
\ \ \ \ \ \ \ \ \ \ \ \ {\isasymand}\ {\isacharparenleft}{\isasymforall}k{\isachardot}\ Atom\ k\ {\isasymin}\ {\isacharquery}A\ {\isasymlongrightarrow}\ \isactrlbold {\isasymnot}\ {\isacharparenleft}Atom\ k{\isacharparenright}\ {\isasymin}\ {\isacharquery}A\ {\isasymlongrightarrow}\ False{\isacharparenright}\isanewline
\ \ \ \ \ \ \ \ \ \ \ \ {\isasymand}\ {\isacharparenleft}{\isasymforall}F\ G\ H{\isachardot}\ Con\ F\ G\ H\ {\isasymlongrightarrow}\ F\ {\isasymin}\ {\isacharquery}A\ {\isasymlongrightarrow}\ {\isacharbraceleft}G{\isacharcomma}H{\isacharbraceright}\ {\isasymunion}\ {\isacharquery}A\ {\isasymin}\ C{\isacharparenright}\isanewline
\ \ \ \ \ \ \ \ \ \ \ \ {\isasymand}\ {\isacharparenleft}{\isasymforall}F\ G\ H{\isachardot}\ Dis\ F\ G\ H\ {\isasymlongrightarrow}\ F\ {\isasymin}\ {\isacharquery}A\ {\isasymlongrightarrow}\ {\isacharbraceleft}G{\isacharbraceright}\ {\isasymunion}\ {\isacharquery}A\ {\isasymin}\ C\ {\isasymor}\ {\isacharbraceleft}H{\isacharbraceright}\ {\isasymunion}\ {\isacharquery}A\ {\isasymin}\ C{\isacharparenright}{\isachardoublequoteclose}\isanewline
\ \ \ \ \ \ \ \ \isacommand{using}\isamarkupfalse%
\ PCP\ {\isacartoucheopen}{\isacharquery}A\ {\isasymin}\ C{\isacartoucheclose}\ \isacommand{by}\isamarkupfalse%
\ {\isacharparenleft}rule\ bspec{\isacharparenright}\isanewline
\ \ \ \ \ \ \isacommand{then}\isamarkupfalse%
\ \isacommand{have}\isamarkupfalse%
\ {\isachardoublequoteopen}{\isasymforall}k{\isachardot}\ Atom\ k\ {\isasymin}\ {\isacharquery}A\ {\isasymlongrightarrow}\ \isactrlbold {\isasymnot}\ {\isacharparenleft}Atom\ k{\isacharparenright}\ {\isasymin}\ {\isacharquery}A\ {\isasymlongrightarrow}\ False{\isachardoublequoteclose}\isanewline
\ \ \ \ \ \ \ \ \isacommand{by}\isamarkupfalse%
\ {\isacharparenleft}iprover\ elim{\isacharcolon}\ conjunct{\isadigit{2}}\ conjunct{\isadigit{1}}{\isacharparenright}\isanewline
\ \ \ \ \ \ \isacommand{then}\isamarkupfalse%
\ \isacommand{have}\isamarkupfalse%
\ {\isachardoublequoteopen}Atom\ k\ {\isasymin}\ {\isacharquery}A\ {\isasymlongrightarrow}\ \isactrlbold {\isasymnot}\ {\isacharparenleft}Atom\ k{\isacharparenright}\ {\isasymin}\ {\isacharquery}A\ {\isasymlongrightarrow}\ False{\isachardoublequoteclose}\isanewline
\ \ \ \ \ \ \ \ \isacommand{by}\isamarkupfalse%
\ {\isacharparenleft}iprover\ elim{\isacharcolon}\ allE{\isacharparenright}\isanewline
\ \ \ \ \ \ \isacommand{then}\isamarkupfalse%
\ \isacommand{have}\isamarkupfalse%
\ {\isachardoublequoteopen}\isactrlbold {\isasymnot}{\isacharparenleft}Atom\ k{\isacharparenright}\ {\isasymin}\ {\isacharquery}A\ {\isasymlongrightarrow}\ False{\isachardoublequoteclose}\isanewline
\ \ \ \ \ \ \ \ \isacommand{using}\isamarkupfalse%
\ {\isacartoucheopen}Atom\ k\ {\isasymin}\ {\isacharquery}A{\isacartoucheclose}\ \isacommand{by}\isamarkupfalse%
\ {\isacharparenleft}rule\ mp{\isacharparenright}\isanewline
\ \ \ \ \ \ \isacommand{thus}\isamarkupfalse%
\ {\isachardoublequoteopen}False{\isachardoublequoteclose}\isanewline
\ \ \ \ \ \ \ \ \isacommand{using}\isamarkupfalse%
\ {\isacartoucheopen}\isactrlbold {\isasymnot}{\isacharparenleft}Atom\ k{\isacharparenright}\ {\isasymin}\ {\isacharquery}A{\isacartoucheclose}\ \isacommand{by}\isamarkupfalse%
\ {\isacharparenleft}rule\ mp{\isacharparenright}\isanewline
\ \ \ \ \isacommand{qed}\isamarkupfalse%
\isanewline
\ \ \isacommand{qed}\isamarkupfalse%
\isanewline
\ \ \isacommand{have}\isamarkupfalse%
\ C{\isadigit{3}}{\isacharcolon}{\isachardoublequoteopen}{\isasymforall}F\ G\ H{\isachardot}\ Con\ F\ G\ H\ {\isasymlongrightarrow}\ F\ {\isasymin}\ S\ {\isasymlongrightarrow}\ {\isacharbraceleft}G{\isacharcomma}H{\isacharbraceright}\ {\isasymunion}\ S\ {\isasymin}\ {\isacharparenleft}extensionFin\ C{\isacharparenright}{\isachardoublequoteclose}\isanewline
\ \ \isacommand{proof}\isamarkupfalse%
\ {\isacharparenleft}rule\ allI{\isacharparenright}{\isacharplus}\isanewline
\ \ \ \ \isacommand{fix}\isamarkupfalse%
\ F\ G\ H\isanewline
\ \ \ \ \isacommand{show}\isamarkupfalse%
\ {\isachardoublequoteopen}Con\ F\ G\ H\ {\isasymlongrightarrow}\ F\ {\isasymin}\ S\ {\isasymlongrightarrow}\ {\isacharbraceleft}G{\isacharcomma}H{\isacharbraceright}\ {\isasymunion}\ S\ {\isasymin}\ {\isacharparenleft}extensionFin\ C{\isacharparenright}{\isachardoublequoteclose}\isanewline
\ \ \ \ \isacommand{proof}\isamarkupfalse%
\ {\isacharparenleft}rule\ impI{\isacharparenright}{\isacharplus}\isanewline
\ \ \ \ \ \ \isacommand{assume}\isamarkupfalse%
\ {\isachardoublequoteopen}Con\ F\ G\ H{\isachardoublequoteclose}\isanewline
\ \ \ \ \ \ \isacommand{assume}\isamarkupfalse%
\ {\isachardoublequoteopen}F\ {\isasymin}\ S{\isachardoublequoteclose}\ \isanewline
\ \ \ \ \ \ \isacommand{show}\isamarkupfalse%
\ {\isachardoublequoteopen}{\isacharbraceleft}G{\isacharcomma}H{\isacharbraceright}\ {\isasymunion}\ S\ {\isasymin}\ {\isacharparenleft}extensionFin\ C{\isacharparenright}{\isachardoublequoteclose}\ \isanewline
\ \ \ \ \ \ \ \ \isacommand{using}\isamarkupfalse%
\ assms{\isacharparenleft}{\isadigit{1}}{\isacharparenright}\ assms{\isacharparenleft}{\isadigit{2}}{\isacharparenright}\ assms{\isacharparenleft}{\isadigit{3}}{\isacharparenright}\ {\isacartoucheopen}Con\ F\ G\ H{\isacartoucheclose}\ {\isacartoucheopen}F\ {\isasymin}\ S{\isacartoucheclose}\ \isacommand{by}\isamarkupfalse%
\ {\isacharparenleft}simp\ only{\isacharcolon}\ ex{\isadigit{3}}{\isacharunderscore}pcp{\isacharunderscore}SinE{\isacharunderscore}CON{\isacharparenright}\isanewline
\ \ \ \ \isacommand{qed}\isamarkupfalse%
\isanewline
\ \ \isacommand{qed}\isamarkupfalse%
\isanewline
\ \ \isacommand{have}\isamarkupfalse%
\ C{\isadigit{4}}{\isacharcolon}{\isachardoublequoteopen}{\isasymforall}F\ G\ H{\isachardot}\ Dis\ F\ G\ H\ {\isasymlongrightarrow}\ F\ {\isasymin}\ S\ {\isasymlongrightarrow}\ {\isacharbraceleft}G{\isacharbraceright}\ {\isasymunion}\ S\ {\isasymin}\ {\isacharparenleft}extensionFin\ C{\isacharparenright}\ {\isasymor}\ {\isacharbraceleft}H{\isacharbraceright}\ {\isasymunion}\ S\ {\isasymin}\ {\isacharparenleft}extensionFin\ C{\isacharparenright}{\isachardoublequoteclose}\isanewline
\ \ \isacommand{proof}\isamarkupfalse%
\ {\isacharparenleft}rule\ allI{\isacharparenright}{\isacharplus}\isanewline
\ \ \ \ \isacommand{fix}\isamarkupfalse%
\ F\ G\ H\isanewline
\ \ \ \ \isacommand{show}\isamarkupfalse%
\ {\isachardoublequoteopen}Dis\ F\ G\ H\ {\isasymlongrightarrow}\ F\ {\isasymin}\ S\ {\isasymlongrightarrow}\ {\isacharbraceleft}G{\isacharbraceright}\ {\isasymunion}\ S\ {\isasymin}\ {\isacharparenleft}extensionFin\ C{\isacharparenright}\ {\isasymor}\ {\isacharbraceleft}H{\isacharbraceright}\ {\isasymunion}\ S\ {\isasymin}\ {\isacharparenleft}extensionFin\ C{\isacharparenright}{\isachardoublequoteclose}\isanewline
\ \ \ \ \isacommand{proof}\isamarkupfalse%
\ {\isacharparenleft}rule\ impI{\isacharparenright}{\isacharplus}\isanewline
\ \ \ \ \ \ \isacommand{assume}\isamarkupfalse%
\ {\isachardoublequoteopen}Dis\ F\ G\ H{\isachardoublequoteclose}\isanewline
\ \ \ \ \ \ \isacommand{assume}\isamarkupfalse%
\ {\isachardoublequoteopen}F\ {\isasymin}\ S{\isachardoublequoteclose}\ \isanewline
\ \ \ \ \ \ \isacommand{show}\isamarkupfalse%
\ {\isachardoublequoteopen}{\isacharbraceleft}G{\isacharbraceright}\ {\isasymunion}\ S\ {\isasymin}\ {\isacharparenleft}extensionFin\ C{\isacharparenright}\ {\isasymor}\ {\isacharbraceleft}H{\isacharbraceright}\ {\isasymunion}\ S\ {\isasymin}\ {\isacharparenleft}extensionFin\ C{\isacharparenright}{\isachardoublequoteclose}\isanewline
\ \ \ \ \ \ \ \ \isacommand{using}\isamarkupfalse%
\ assms{\isacharparenleft}{\isadigit{1}}{\isacharparenright}\ assms{\isacharparenleft}{\isadigit{2}}{\isacharparenright}\ assms{\isacharparenleft}{\isadigit{3}}{\isacharparenright}\ {\isacartoucheopen}Dis\ F\ G\ H{\isacartoucheclose}\ {\isacartoucheopen}F\ {\isasymin}\ S{\isacartoucheclose}\ \isacommand{by}\isamarkupfalse%
\ {\isacharparenleft}rule\ ex{\isadigit{3}}{\isacharunderscore}pcp{\isacharunderscore}SinE{\isacharunderscore}DIS{\isacharparenright}\isanewline
\ \ \ \ \isacommand{qed}\isamarkupfalse%
\isanewline
\ \ \isacommand{qed}\isamarkupfalse%
\isanewline
\ \ \isacommand{show}\isamarkupfalse%
\ {\isacharquery}thesis\isanewline
\ \ \ \ \isacommand{using}\isamarkupfalse%
\ C{\isadigit{1}}\ C{\isadigit{2}}\ C{\isadigit{3}}\ C{\isadigit{4}}\ \isacommand{by}\isamarkupfalse%
\ {\isacharparenleft}iprover\ intro{\isacharcolon}\ conjI{\isacharparenright}\isanewline
\isacommand{qed}\isamarkupfalse%
%
\endisatagproof
{\isafoldproof}%
%
\isadelimproof
%
\endisadelimproof
%
\begin{isamarkuptext}%
En conclusión, la prueba detallada completa en Isabelle que demuestra que la extensión \isa{C{\isacharprime}} 
  verifica la propiedad de consistencia proposicional dada una colección \isa{C} que también la
  verifique y sea cerrada bajo subconjuntos es la siguiente.%
\end{isamarkuptext}\isamarkuptrue%
\isacommand{lemma}\isamarkupfalse%
\ ex{\isadigit{3}}{\isacharunderscore}pcp{\isacharcolon}\isanewline
\ \ \isakeyword{assumes}\ {\isachardoublequoteopen}pcp\ C{\isachardoublequoteclose}\isanewline
\ \ \ \ \ \ \ \ \ \ {\isachardoublequoteopen}subset{\isacharunderscore}closed\ C{\isachardoublequoteclose}\isanewline
\ \ \ \ \ \ \ \ \isakeyword{shows}\ {\isachardoublequoteopen}pcp\ {\isacharparenleft}extensionFin\ C{\isacharparenright}{\isachardoublequoteclose}\isanewline
%
\isadelimproof
\ \ %
\endisadelimproof
%
\isatagproof
\isacommand{unfolding}\isamarkupfalse%
\ pcp{\isacharunderscore}alt\isanewline
\isacommand{proof}\isamarkupfalse%
\ {\isacharparenleft}rule\ ballI{\isacharparenright}\isanewline
\ \ \isacommand{have}\isamarkupfalse%
\ PCP{\isacharcolon}{\isachardoublequoteopen}{\isasymforall}S\ {\isasymin}\ C{\isachardot}\isanewline
\ \ \ \ {\isasymbottom}\ {\isasymnotin}\ S\isanewline
\ \ \ \ {\isasymand}\ {\isacharparenleft}{\isasymforall}k{\isachardot}\ Atom\ k\ {\isasymin}\ S\ {\isasymlongrightarrow}\ \isactrlbold {\isasymnot}\ {\isacharparenleft}Atom\ k{\isacharparenright}\ {\isasymin}\ S\ {\isasymlongrightarrow}\ False{\isacharparenright}\isanewline
\ \ \ \ {\isasymand}\ {\isacharparenleft}{\isasymforall}F\ G\ H{\isachardot}\ Con\ F\ G\ H\ {\isasymlongrightarrow}\ F\ {\isasymin}\ S\ {\isasymlongrightarrow}\ {\isacharbraceleft}G{\isacharcomma}H{\isacharbraceright}\ {\isasymunion}\ S\ {\isasymin}\ C{\isacharparenright}\isanewline
\ \ \ \ {\isasymand}\ {\isacharparenleft}{\isasymforall}F\ G\ H{\isachardot}\ Dis\ F\ G\ H\ {\isasymlongrightarrow}\ F\ {\isasymin}\ S\ {\isasymlongrightarrow}\ {\isacharbraceleft}G{\isacharbraceright}\ {\isasymunion}\ S\ {\isasymin}\ C\ {\isasymor}\ {\isacharbraceleft}H{\isacharbraceright}\ {\isasymunion}\ S\ {\isasymin}\ C{\isacharparenright}{\isachardoublequoteclose}\isanewline
\ \ \ \ \isacommand{using}\isamarkupfalse%
\ assms{\isacharparenleft}{\isadigit{1}}{\isacharparenright}\ \isacommand{by}\isamarkupfalse%
\ {\isacharparenleft}rule\ pcp{\isacharunderscore}alt{\isadigit{1}}{\isacharparenright}\isanewline
\ \ \isacommand{fix}\isamarkupfalse%
\ S\isanewline
\ \ \isacommand{assume}\isamarkupfalse%
\ {\isachardoublequoteopen}S\ {\isasymin}\ {\isacharparenleft}extensionFin\ C{\isacharparenright}{\isachardoublequoteclose}\isanewline
\ \ \isacommand{then}\isamarkupfalse%
\ \isacommand{have}\isamarkupfalse%
\ {\isachardoublequoteopen}S\ {\isasymin}\ C\ {\isasymor}\ S\ {\isasymin}\ {\isacharparenleft}extF\ C{\isacharparenright}{\isachardoublequoteclose}\isanewline
\ \ \ \ \isacommand{unfolding}\isamarkupfalse%
\ extensionFin\ \isacommand{by}\isamarkupfalse%
\ {\isacharparenleft}simp\ only{\isacharcolon}\ Un{\isacharunderscore}iff{\isacharparenright}\isanewline
\ \ \isacommand{thus}\isamarkupfalse%
\ {\isachardoublequoteopen}{\isasymbottom}\ {\isasymnotin}\ S\ {\isasymand}\isanewline
\ \ \ \ \ \ \ \ \ {\isacharparenleft}{\isasymforall}k{\isachardot}\ Atom\ k\ {\isasymin}\ S\ {\isasymlongrightarrow}\ \isactrlbold {\isasymnot}\ {\isacharparenleft}Atom\ k{\isacharparenright}\ {\isasymin}\ S\ {\isasymlongrightarrow}\ False{\isacharparenright}\ {\isasymand}\isanewline
\ \ \ \ \ \ \ \ \ {\isacharparenleft}{\isasymforall}F\ G\ H{\isachardot}\ Con\ F\ G\ H\ {\isasymlongrightarrow}\ F\ {\isasymin}\ S\ {\isasymlongrightarrow}\ {\isacharbraceleft}G{\isacharcomma}\ H{\isacharbraceright}\ {\isasymunion}\ S\ {\isasymin}\ {\isacharparenleft}extensionFin\ C{\isacharparenright}{\isacharparenright}\ {\isasymand}\isanewline
\ \ \ \ \ \ \ \ \ {\isacharparenleft}{\isasymforall}F\ G\ H{\isachardot}\ Dis\ F\ G\ H\ {\isasymlongrightarrow}\ F\ {\isasymin}\ S\ {\isasymlongrightarrow}\ {\isacharbraceleft}G{\isacharbraceright}\ {\isasymunion}\ S\ {\isasymin}\ {\isacharparenleft}extensionFin\ C{\isacharparenright}\ {\isasymor}\ {\isacharbraceleft}H{\isacharbraceright}\ {\isasymunion}\ S\ {\isasymin}\ {\isacharparenleft}extensionFin\ C{\isacharparenright}{\isacharparenright}{\isachardoublequoteclose}\isanewline
\ \ \isacommand{proof}\isamarkupfalse%
\ {\isacharparenleft}rule\ disjE{\isacharparenright}\isanewline
\ \ \ \ \isacommand{assume}\isamarkupfalse%
\ {\isachardoublequoteopen}S\ {\isasymin}\ C{\isachardoublequoteclose}\isanewline
\ \ \ \ \isacommand{show}\isamarkupfalse%
\ {\isacharquery}thesis\isanewline
\ \ \ \ \ \ \isacommand{using}\isamarkupfalse%
\ assms\ {\isacartoucheopen}S\ {\isasymin}\ C{\isacartoucheclose}\ \isacommand{by}\isamarkupfalse%
\ {\isacharparenleft}rule\ ex{\isadigit{3}}{\isacharunderscore}pcp{\isacharunderscore}SinC{\isacharparenright}\isanewline
\ \ \isacommand{next}\isamarkupfalse%
\isanewline
\ \ \ \ \isacommand{assume}\isamarkupfalse%
\ {\isachardoublequoteopen}S\ {\isasymin}\ {\isacharparenleft}extF\ C{\isacharparenright}{\isachardoublequoteclose}\isanewline
\ \ \ \ \isacommand{show}\isamarkupfalse%
\ {\isacharquery}thesis\isanewline
\ \ \ \ \ \ \isacommand{using}\isamarkupfalse%
\ assms\ {\isacartoucheopen}S\ {\isasymin}\ {\isacharparenleft}extF\ C{\isacharparenright}{\isacartoucheclose}\ \isacommand{by}\isamarkupfalse%
\ {\isacharparenleft}rule\ ex{\isadigit{3}}{\isacharunderscore}pcp{\isacharunderscore}SinE{\isacharparenright}\isanewline
\ \ \isacommand{qed}\isamarkupfalse%
\isanewline
\isacommand{qed}\isamarkupfalse%
%
\endisatagproof
{\isafoldproof}%
%
\isadelimproof
%
\endisadelimproof
%
\begin{isamarkuptext}%
Por último, podemos dar la prueba completa del lema \isa{{\isadigit{3}}{\isachardot}{\isadigit{0}}{\isachardot}{\isadigit{5}}} en Isabelle.%
\end{isamarkuptext}\isamarkuptrue%
\isacommand{lemma}\isamarkupfalse%
\ ex{\isadigit{3}}{\isacharcolon}\isanewline
\ \ \isakeyword{assumes}\ {\isachardoublequoteopen}pcp\ C{\isachardoublequoteclose}\isanewline
\ \ \ \ \ \ \ \ \ \ {\isachardoublequoteopen}subset{\isacharunderscore}closed\ C{\isachardoublequoteclose}\isanewline
\ \ \isakeyword{shows}\ {\isachardoublequoteopen}{\isasymexists}C{\isacharprime}{\isachardot}\ C\ {\isasymsubseteq}\ C{\isacharprime}\ {\isasymand}\ pcp\ C{\isacharprime}\ {\isasymand}\ finite{\isacharunderscore}character\ C{\isacharprime}{\isachardoublequoteclose}\isanewline
%
\isadelimproof
%
\endisadelimproof
%
\isatagproof
\isacommand{proof}\isamarkupfalse%
\ {\isacharminus}\isanewline
\ \ \isacommand{let}\isamarkupfalse%
\ {\isacharquery}C{\isacharprime}{\isacharequal}{\isachardoublequoteopen}extensionFin\ C{\isachardoublequoteclose}\isanewline
\ \ \isacommand{have}\isamarkupfalse%
\ C{\isadigit{1}}{\isacharcolon}{\isachardoublequoteopen}C\ {\isasymsubseteq}\ {\isacharquery}C{\isacharprime}{\isachardoublequoteclose}\isanewline
\ \ \ \ \isacommand{unfolding}\isamarkupfalse%
\ extensionFin\ \isacommand{by}\isamarkupfalse%
\ {\isacharparenleft}simp\ only{\isacharcolon}\ Un{\isacharunderscore}upper{\isadigit{1}}{\isacharparenright}\isanewline
\ \ \isacommand{have}\isamarkupfalse%
\ C{\isadigit{2}}{\isacharcolon}{\isachardoublequoteopen}finite{\isacharunderscore}character\ {\isacharparenleft}{\isacharquery}C{\isacharprime}{\isacharparenright}{\isachardoublequoteclose}\isanewline
\ \ \ \ \isacommand{using}\isamarkupfalse%
\ assms{\isacharparenleft}{\isadigit{2}}{\isacharparenright}\ \isacommand{by}\isamarkupfalse%
\ {\isacharparenleft}rule\ ex{\isadigit{3}}{\isacharunderscore}finite{\isacharunderscore}character{\isacharparenright}\isanewline
\ \ \isacommand{have}\isamarkupfalse%
\ C{\isadigit{3}}{\isacharcolon}{\isachardoublequoteopen}pcp\ {\isacharparenleft}{\isacharquery}C{\isacharprime}{\isacharparenright}{\isachardoublequoteclose}\isanewline
\ \ \ \ \isacommand{using}\isamarkupfalse%
\ assms\ \isacommand{by}\isamarkupfalse%
\ {\isacharparenleft}rule\ ex{\isadigit{3}}{\isacharunderscore}pcp{\isacharparenright}\isanewline
\ \ \isacommand{have}\isamarkupfalse%
\ {\isachardoublequoteopen}C\ {\isasymsubseteq}\ {\isacharquery}C{\isacharprime}\ {\isasymand}\ pcp\ {\isacharquery}C{\isacharprime}\ {\isasymand}\ finite{\isacharunderscore}character\ {\isacharquery}C{\isacharprime}{\isachardoublequoteclose}\isanewline
\ \ \ \ \isacommand{using}\isamarkupfalse%
\ C{\isadigit{1}}\ C{\isadigit{2}}\ C{\isadigit{3}}\ \isacommand{by}\isamarkupfalse%
\ {\isacharparenleft}iprover\ intro{\isacharcolon}\ conjI{\isacharparenright}\isanewline
\ \ \isacommand{thus}\isamarkupfalse%
\ {\isacharquery}thesis\isanewline
\ \ \ \ \isacommand{by}\isamarkupfalse%
\ {\isacharparenleft}rule\ exI{\isacharparenright}\isanewline
\isacommand{qed}\isamarkupfalse%
\isanewline
%
\endisatagproof
{\isafoldproof}%
%
\isadelimproof
%
\endisadelimproof
%
\isadelimtheory
%
\endisadelimtheory
%
\isatagtheory
%
\endisatagtheory
{\isafoldtheory}%
%
\isadelimtheory
%
\endisadelimtheory
%
\end{isabellebody}%
\endinput
%:%file=~/TFM/TFM/ColecScfc.thy%:%
%:%19=12%:%
%:%20=13%:%
%:%21=14%:%
%:%22=15%:%
%:%23=16%:%
%:%24=17%:%
%:%25=18%:%
%:%26=19%:%
%:%27=20%:%
%:%28=21%:%
%:%29=22%:%
%:%30=23%:%
%:%31=24%:%
%:%32=25%:%
%:%33=26%:%
%:%35=28%:%
%:%36=28%:%
%:%38=30%:%
%:%39=31%:%
%:%40=32%:%
%:%42=34%:%
%:%43=34%:%
%:%46=35%:%
%:%50=35%:%
%:%51=35%:%
%:%52=35%:%
%:%57=35%:%
%:%60=36%:%
%:%61=37%:%
%:%62=37%:%
%:%65=38%:%
%:%69=38%:%
%:%70=38%:%
%:%71=38%:%
%:%80=40%:%
%:%81=41%:%
%:%82=42%:%
%:%84=44%:%
%:%85=44%:%
%:%88=45%:%
%:%92=45%:%
%:%93=45%:%
%:%94=45%:%
%:%103=47%:%
%:%104=48%:%
%:%106=50%:%
%:%107=50%:%
%:%109=52%:%
%:%112=53%:%
%:%116=53%:%
%:%117=53%:%
%:%118=53%:%
%:%123=53%:%
%:%126=54%:%
%:%127=55%:%
%:%128=55%:%
%:%129=56%:%
%:%132=57%:%
%:%136=57%:%
%:%137=57%:%
%:%138=57%:%
%:%147=59%:%
%:%148=60%:%
%:%149=61%:%
%:%150=62%:%
%:%151=63%:%
%:%152=64%:%
%:%153=65%:%
%:%154=66%:%
%:%155=67%:%
%:%156=68%:%
%:%157=69%:%
%:%158=70%:%
%:%160=72%:%
%:%161=72%:%
%:%164=75%:%
%:%165=76%:%
%:%166=77%:%
%:%167=78%:%
%:%169=80%:%
%:%170=80%:%
%:%173=81%:%
%:%177=81%:%
%:%178=81%:%
%:%179=81%:%
%:%184=81%:%
%:%187=82%:%
%:%188=83%:%
%:%189=83%:%
%:%192=84%:%
%:%196=84%:%
%:%197=84%:%
%:%198=84%:%
%:%203=84%:%
%:%206=85%:%
%:%207=86%:%
%:%208=86%:%
%:%211=87%:%
%:%215=87%:%
%:%216=87%:%
%:%217=87%:%
%:%226=89%:%
%:%227=90%:%
%:%228=91%:%
%:%229=92%:%
%:%230=93%:%
%:%231=94%:%
%:%232=95%:%
%:%233=96%:%
%:%234=97%:%
%:%235=98%:%
%:%236=99%:%
%:%237=100%:%
%:%238=101%:%
%:%239=102%:%
%:%240=103%:%
%:%241=104%:%
%:%243=106%:%
%:%244=106%:%
%:%247=107%:%
%:%251=107%:%
%:%261=109%:%
%:%262=110%:%
%:%263=111%:%
%:%264=112%:%
%:%265=113%:%
%:%266=114%:%
%:%267=115%:%
%:%268=116%:%
%:%269=117%:%
%:%270=118%:%
%:%271=119%:%
%:%272=120%:%
%:%273=121%:%
%:%274=122%:%
%:%275=123%:%
%:%276=124%:%
%:%277=125%:%
%:%278=126%:%
%:%279=127%:%
%:%280=128%:%
%:%281=129%:%
%:%282=130%:%
%:%283=131%:%
%:%284=132%:%
%:%285=133%:%
%:%286=134%:%
%:%287=135%:%
%:%288=136%:%
%:%289=137%:%
%:%290=138%:%
%:%291=139%:%
%:%292=140%:%
%:%293=141%:%
%:%294=142%:%
%:%295=143%:%
%:%296=144%:%
%:%297=145%:%
%:%298=146%:%
%:%299=147%:%
%:%300=148%:%
%:%301=149%:%
%:%302=150%:%
%:%303=151%:%
%:%304=152%:%
%:%305=153%:%
%:%306=154%:%
%:%307=155%:%
%:%308=156%:%
%:%309=157%:%
%:%310=158%:%
%:%311=159%:%
%:%312=160%:%
%:%313=161%:%
%:%314=162%:%
%:%315=163%:%
%:%316=164%:%
%:%317=165%:%
%:%318=166%:%
%:%319=167%:%
%:%320=168%:%
%:%321=169%:%
%:%322=170%:%
%:%323=171%:%
%:%324=172%:%
%:%325=173%:%
%:%326=174%:%
%:%327=175%:%
%:%328=176%:%
%:%329=177%:%
%:%330=178%:%
%:%331=179%:%
%:%332=180%:%
%:%333=181%:%
%:%334=182%:%
%:%335=183%:%
%:%336=184%:%
%:%337=185%:%
%:%338=186%:%
%:%339=187%:%
%:%340=188%:%
%:%341=189%:%
%:%342=190%:%
%:%343=191%:%
%:%344=192%:%
%:%345=193%:%
%:%346=194%:%
%:%347=195%:%
%:%348=196%:%
%:%349=197%:%
%:%350=198%:%
%:%351=199%:%
%:%352=200%:%
%:%354=202%:%
%:%355=202%:%
%:%358=203%:%
%:%362=203%:%
%:%363=203%:%
%:%372=205%:%
%:%373=206%:%
%:%375=208%:%
%:%376=208%:%
%:%379=209%:%
%:%383=209%:%
%:%384=209%:%
%:%393=211%:%
%:%394=212%:%
%:%395=213%:%
%:%396=214%:%
%:%397=215%:%
%:%399=217%:%
%:%400=217%:%
%:%401=218%:%
%:%403=220%:%
%:%404=221%:%
%:%406=223%:%
%:%407=223%:%
%:%414=224%:%
%:%415=224%:%
%:%416=225%:%
%:%417=225%:%
%:%418=226%:%
%:%419=226%:%
%:%420=227%:%
%:%421=227%:%
%:%422=228%:%
%:%423=228%:%
%:%424=229%:%
%:%425=229%:%
%:%426=229%:%
%:%427=230%:%
%:%428=230%:%
%:%429=230%:%
%:%430=231%:%
%:%431=231%:%
%:%432=232%:%
%:%433=232%:%
%:%434=233%:%
%:%444=235%:%
%:%445=236%:%
%:%446=237%:%
%:%447=238%:%
%:%449=240%:%
%:%450=240%:%
%:%453=241%:%
%:%457=241%:%
%:%458=241%:%
%:%467=243%:%
%:%468=244%:%
%:%470=246%:%
%:%471=246%:%
%:%472=247%:%
%:%473=248%:%
%:%480=249%:%
%:%481=249%:%
%:%482=250%:%
%:%483=250%:%
%:%484=251%:%
%:%485=251%:%
%:%486=252%:%
%:%487=252%:%
%:%488=253%:%
%:%489=253%:%
%:%490=254%:%
%:%491=254%:%
%:%494=257%:%
%:%495=258%:%
%:%496=258%:%
%:%497=259%:%
%:%498=259%:%
%:%499=260%:%
%:%500=260%:%
%:%501=261%:%
%:%502=261%:%
%:%503=261%:%
%:%504=262%:%
%:%505=262%:%
%:%506=262%:%
%:%507=263%:%
%:%508=263%:%
%:%509=264%:%
%:%510=264%:%
%:%511=264%:%
%:%512=265%:%
%:%513=265%:%
%:%517=269%:%
%:%518=270%:%
%:%519=270%:%
%:%520=270%:%
%:%521=271%:%
%:%522=271%:%
%:%523=271%:%
%:%526=274%:%
%:%527=275%:%
%:%528=275%:%
%:%529=275%:%
%:%530=276%:%
%:%531=276%:%
%:%532=276%:%
%:%533=277%:%
%:%534=277%:%
%:%535=278%:%
%:%536=278%:%
%:%537=279%:%
%:%538=279%:%
%:%539=279%:%
%:%540=280%:%
%:%541=280%:%
%:%542=281%:%
%:%543=281%:%
%:%544=281%:%
%:%545=282%:%
%:%546=282%:%
%:%547=283%:%
%:%548=283%:%
%:%549=284%:%
%:%550=284%:%
%:%551=285%:%
%:%552=285%:%
%:%553=286%:%
%:%554=286%:%
%:%555=287%:%
%:%556=287%:%
%:%557=288%:%
%:%558=288%:%
%:%559=289%:%
%:%560=289%:%
%:%561=290%:%
%:%562=290%:%
%:%563=290%:%
%:%564=291%:%
%:%565=291%:%
%:%566=292%:%
%:%567=292%:%
%:%568=292%:%
%:%569=293%:%
%:%570=293%:%
%:%571=294%:%
%:%572=294%:%
%:%573=294%:%
%:%574=295%:%
%:%575=295%:%
%:%576=295%:%
%:%577=296%:%
%:%578=296%:%
%:%579=296%:%
%:%580=297%:%
%:%581=297%:%
%:%582=298%:%
%:%583=298%:%
%:%584=298%:%
%:%585=299%:%
%:%586=299%:%
%:%587=300%:%
%:%588=300%:%
%:%589=301%:%
%:%590=301%:%
%:%591=302%:%
%:%592=302%:%
%:%593=302%:%
%:%594=303%:%
%:%595=303%:%
%:%596=304%:%
%:%597=304%:%
%:%598=305%:%
%:%599=305%:%
%:%600=306%:%
%:%601=306%:%
%:%602=307%:%
%:%603=307%:%
%:%604=308%:%
%:%605=308%:%
%:%606=309%:%
%:%607=309%:%
%:%608=310%:%
%:%609=310%:%
%:%610=311%:%
%:%611=311%:%
%:%612=311%:%
%:%613=312%:%
%:%614=312%:%
%:%615=313%:%
%:%616=313%:%
%:%617=313%:%
%:%618=314%:%
%:%619=314%:%
%:%620=314%:%
%:%621=315%:%
%:%622=315%:%
%:%623=315%:%
%:%624=316%:%
%:%625=316%:%
%:%626=316%:%
%:%627=317%:%
%:%628=317%:%
%:%629=317%:%
%:%630=318%:%
%:%631=318%:%
%:%632=319%:%
%:%633=319%:%
%:%634=319%:%
%:%635=320%:%
%:%636=320%:%
%:%637=320%:%
%:%638=321%:%
%:%639=321%:%
%:%640=322%:%
%:%641=322%:%
%:%642=322%:%
%:%643=323%:%
%:%644=323%:%
%:%645=324%:%
%:%646=324%:%
%:%647=324%:%
%:%648=325%:%
%:%649=325%:%
%:%650=326%:%
%:%651=326%:%
%:%652=327%:%
%:%653=327%:%
%:%654=328%:%
%:%655=328%:%
%:%656=329%:%
%:%657=329%:%
%:%658=330%:%
%:%659=330%:%
%:%660=331%:%
%:%661=331%:%
%:%662=331%:%
%:%663=332%:%
%:%664=332%:%
%:%665=332%:%
%:%666=333%:%
%:%667=333%:%
%:%668=333%:%
%:%669=334%:%
%:%670=334%:%
%:%671=335%:%
%:%672=335%:%
%:%673=335%:%
%:%674=336%:%
%:%675=336%:%
%:%676=337%:%
%:%677=337%:%
%:%678=338%:%
%:%679=338%:%
%:%680=339%:%
%:%681=339%:%
%:%682=339%:%
%:%683=340%:%
%:%684=340%:%
%:%685=341%:%
%:%686=341%:%
%:%687=342%:%
%:%688=342%:%
%:%689=343%:%
%:%690=343%:%
%:%691=344%:%
%:%692=344%:%
%:%693=345%:%
%:%694=345%:%
%:%695=346%:%
%:%696=346%:%
%:%697=347%:%
%:%698=347%:%
%:%699=348%:%
%:%700=348%:%
%:%701=348%:%
%:%702=349%:%
%:%703=349%:%
%:%704=350%:%
%:%705=350%:%
%:%706=350%:%
%:%707=351%:%
%:%708=351%:%
%:%709=351%:%
%:%710=352%:%
%:%711=352%:%
%:%712=352%:%
%:%713=353%:%
%:%714=353%:%
%:%715=353%:%
%:%716=354%:%
%:%717=354%:%
%:%718=354%:%
%:%719=355%:%
%:%720=355%:%
%:%721=356%:%
%:%722=356%:%
%:%723=357%:%
%:%724=357%:%
%:%725=358%:%
%:%726=358%:%
%:%727=359%:%
%:%728=359%:%
%:%729=360%:%
%:%730=360%:%
%:%731=360%:%
%:%732=361%:%
%:%733=361%:%
%:%734=362%:%
%:%735=362%:%
%:%736=363%:%
%:%737=363%:%
%:%738=364%:%
%:%739=364%:%
%:%740=365%:%
%:%741=365%:%
%:%742=366%:%
%:%743=366%:%
%:%744=366%:%
%:%745=367%:%
%:%746=367%:%
%:%747=367%:%
%:%748=368%:%
%:%749=368%:%
%:%750=368%:%
%:%751=369%:%
%:%752=369%:%
%:%753=369%:%
%:%754=370%:%
%:%755=370%:%
%:%756=370%:%
%:%757=371%:%
%:%758=371%:%
%:%759=372%:%
%:%760=372%:%
%:%761=373%:%
%:%762=373%:%
%:%763=374%:%
%:%764=374%:%
%:%765=375%:%
%:%766=375%:%
%:%767=376%:%
%:%768=376%:%
%:%769=376%:%
%:%770=377%:%
%:%771=377%:%
%:%772=378%:%
%:%773=378%:%
%:%774=379%:%
%:%775=379%:%
%:%776=380%:%
%:%777=380%:%
%:%778=381%:%
%:%779=381%:%
%:%780=381%:%
%:%781=382%:%
%:%782=382%:%
%:%783=382%:%
%:%784=383%:%
%:%785=383%:%
%:%786=383%:%
%:%787=384%:%
%:%788=384%:%
%:%789=384%:%
%:%790=385%:%
%:%791=385%:%
%:%792=385%:%
%:%793=386%:%
%:%794=386%:%
%:%795=386%:%
%:%796=387%:%
%:%797=387%:%
%:%798=388%:%
%:%799=388%:%
%:%800=389%:%
%:%801=389%:%
%:%802=390%:%
%:%803=390%:%
%:%804=391%:%
%:%805=391%:%
%:%806=392%:%
%:%807=392%:%
%:%810=395%:%
%:%811=396%:%
%:%812=396%:%
%:%813=396%:%
%:%814=397%:%
%:%815=397%:%
%:%816=398%:%
%:%817=398%:%
%:%818=399%:%
%:%828=401%:%
%:%830=403%:%
%:%831=403%:%
%:%832=404%:%
%:%833=405%:%
%:%836=406%:%
%:%840=406%:%
%:%841=406%:%
%:%842=407%:%
%:%843=407%:%
%:%844=408%:%
%:%845=408%:%
%:%846=409%:%
%:%847=409%:%
%:%848=410%:%
%:%849=410%:%
%:%850=410%:%
%:%851=411%:%
%:%852=411%:%
%:%853=411%:%
%:%854=412%:%
%:%855=412%:%
%:%856=413%:%
%:%857=413%:%
%:%858=413%:%
%:%859=414%:%
%:%860=414%:%
%:%861=415%:%
%:%862=415%:%
%:%863=416%:%
%:%864=416%:%
%:%865=417%:%
%:%866=417%:%
%:%867=418%:%
%:%868=418%:%
%:%869=418%:%
%:%870=419%:%
%:%871=419%:%
%:%872=419%:%
%:%873=420%:%
%:%874=420%:%
%:%875=420%:%
%:%876=421%:%
%:%877=421%:%
%:%878=421%:%
%:%879=422%:%
%:%880=422%:%
%:%881=423%:%
%:%882=423%:%
%:%883=423%:%
%:%884=424%:%
%:%885=424%:%
%:%886=425%:%
%:%896=427%:%
%:%898=429%:%
%:%899=429%:%
%:%900=430%:%
%:%901=431%:%
%:%908=432%:%
%:%909=432%:%
%:%910=433%:%
%:%911=433%:%
%:%912=434%:%
%:%913=434%:%
%:%914=435%:%
%:%915=435%:%
%:%916=436%:%
%:%917=436%:%
%:%918=436%:%
%:%919=437%:%
%:%920=437%:%
%:%921=438%:%
%:%922=438%:%
%:%923=438%:%
%:%924=439%:%
%:%925=439%:%
%:%926=440%:%
%:%927=440%:%
%:%928=440%:%
%:%929=441%:%
%:%930=441%:%
%:%931=442%:%
%:%932=442%:%
%:%933=443%:%
%:%943=445%:%
%:%944=446%:%
%:%945=447%:%
%:%946=448%:%
%:%947=449%:%
%:%948=450%:%
%:%949=451%:%
%:%951=453%:%
%:%952=453%:%
%:%953=454%:%
%:%954=455%:%
%:%957=456%:%
%:%961=456%:%
%:%971=458%:%
%:%972=459%:%
%:%973=460%:%
%:%974=461%:%
%:%975=462%:%
%:%976=463%:%
%:%977=464%:%
%:%978=465%:%
%:%979=466%:%
%:%980=467%:%
%:%981=468%:%
%:%982=469%:%
%:%983=470%:%
%:%984=471%:%
%:%985=472%:%
%:%986=473%:%
%:%987=474%:%
%:%988=475%:%
%:%989=476%:%
%:%990=477%:%
%:%991=478%:%
%:%992=479%:%
%:%993=480%:%
%:%994=481%:%
%:%995=482%:%
%:%996=483%:%
%:%997=484%:%
%:%998=485%:%
%:%999=486%:%
%:%1001=488%:%
%:%1002=488%:%
%:%1003=489%:%
%:%1004=490%:%
%:%1007=491%:%
%:%1011=491%:%
%:%1012=491%:%
%:%1013=492%:%
%:%1014=492%:%
%:%1015=493%:%
%:%1016=493%:%
%:%1017=494%:%
%:%1018=494%:%
%:%1019=495%:%
%:%1020=495%:%
%:%1021=496%:%
%:%1022=496%:%
%:%1023=496%:%
%:%1024=496%:%
%:%1025=497%:%
%:%1026=497%:%
%:%1027=498%:%
%:%1028=498%:%
%:%1029=499%:%
%:%1030=499%:%
%:%1031=500%:%
%:%1032=500%:%
%:%1033=501%:%
%:%1034=501%:%
%:%1035=501%:%
%:%1036=502%:%
%:%1037=502%:%
%:%1038=502%:%
%:%1039=503%:%
%:%1040=503%:%
%:%1041=504%:%
%:%1042=504%:%
%:%1043=504%:%
%:%1044=505%:%
%:%1045=505%:%
%:%1046=506%:%
%:%1047=506%:%
%:%1048=506%:%
%:%1049=507%:%
%:%1050=507%:%
%:%1051=508%:%
%:%1052=508%:%
%:%1053=508%:%
%:%1054=509%:%
%:%1055=509%:%
%:%1056=510%:%
%:%1057=510%:%
%:%1058=511%:%
%:%1059=511%:%
%:%1060=511%:%
%:%1061=512%:%
%:%1062=512%:%
%:%1063=513%:%
%:%1064=513%:%
%:%1065=513%:%
%:%1066=514%:%
%:%1076=516%:%
%:%1078=518%:%
%:%1079=518%:%
%:%1080=519%:%
%:%1081=520%:%
%:%1084=521%:%
%:%1088=521%:%
%:%1089=521%:%
%:%1090=522%:%
%:%1091=522%:%
%:%1092=523%:%
%:%1093=523%:%
%:%1094=524%:%
%:%1095=524%:%
%:%1096=525%:%
%:%1097=525%:%
%:%1098=525%:%
%:%1099=526%:%
%:%1100=526%:%
%:%1101=526%:%
%:%1102=527%:%
%:%1103=527%:%
%:%1104=527%:%
%:%1105=527%:%
%:%1106=528%:%
%:%1107=528%:%
%:%1108=528%:%
%:%1109=528%:%
%:%1110=529%:%
%:%1111=529%:%
%:%1112=529%:%
%:%1113=529%:%
%:%1114=529%:%
%:%1115=530%:%
%:%1125=532%:%
%:%1126=533%:%
%:%1127=534%:%
%:%1128=535%:%
%:%1129=536%:%
%:%1130=537%:%
%:%1131=538%:%
%:%1132=539%:%
%:%1133=540%:%
%:%1134=541%:%
%:%1135=542%:%
%:%1136=543%:%
%:%1137=544%:%
%:%1138=545%:%
%:%1139=546%:%
%:%1140=547%:%
%:%1141=548%:%
%:%1142=549%:%
%:%1143=550%:%
%:%1144=551%:%
%:%1145=552%:%
%:%1146=553%:%
%:%1147=554%:%
%:%1148=555%:%
%:%1149=556%:%
%:%1150=557%:%
%:%1151=558%:%
%:%1152=559%:%
%:%1153=560%:%
%:%1154=561%:%
%:%1155=562%:%
%:%1156=563%:%
%:%1157=564%:%
%:%1158=565%:%
%:%1159=566%:%
%:%1160=567%:%
%:%1161=568%:%
%:%1162=569%:%
%:%1163=570%:%
%:%1164=571%:%
%:%1165=572%:%
%:%1166=573%:%
%:%1167=574%:%
%:%1168=575%:%
%:%1169=576%:%
%:%1170=577%:%
%:%1171=578%:%
%:%1172=579%:%
%:%1173=580%:%
%:%1174=581%:%
%:%1175=582%:%
%:%1176=583%:%
%:%1177=584%:%
%:%1178=585%:%
%:%1179=586%:%
%:%1180=587%:%
%:%1181=588%:%
%:%1182=589%:%
%:%1183=590%:%
%:%1184=591%:%
%:%1185=592%:%
%:%1186=593%:%
%:%1187=594%:%
%:%1188=595%:%
%:%1189=596%:%
%:%1190=597%:%
%:%1191=598%:%
%:%1192=599%:%
%:%1193=600%:%
%:%1194=601%:%
%:%1195=602%:%
%:%1196=603%:%
%:%1197=604%:%
%:%1198=605%:%
%:%1199=606%:%
%:%1200=607%:%
%:%1201=608%:%
%:%1202=609%:%
%:%1203=610%:%
%:%1204=611%:%
%:%1205=612%:%
%:%1206=613%:%
%:%1207=614%:%
%:%1208=615%:%
%:%1209=616%:%
%:%1210=617%:%
%:%1211=618%:%
%:%1212=619%:%
%:%1213=620%:%
%:%1214=621%:%
%:%1215=622%:%
%:%1216=623%:%
%:%1217=624%:%
%:%1218=625%:%
%:%1219=626%:%
%:%1220=627%:%
%:%1221=628%:%
%:%1222=629%:%
%:%1223=630%:%
%:%1224=631%:%
%:%1225=632%:%
%:%1226=633%:%
%:%1227=634%:%
%:%1228=635%:%
%:%1229=636%:%
%:%1230=637%:%
%:%1231=638%:%
%:%1232=639%:%
%:%1233=640%:%
%:%1234=641%:%
%:%1235=642%:%
%:%1236=643%:%
%:%1237=644%:%
%:%1238=645%:%
%:%1239=646%:%
%:%1240=647%:%
%:%1241=648%:%
%:%1242=649%:%
%:%1243=650%:%
%:%1244=651%:%
%:%1245=652%:%
%:%1246=653%:%
%:%1247=654%:%
%:%1248=655%:%
%:%1249=656%:%
%:%1250=657%:%
%:%1251=658%:%
%:%1252=659%:%
%:%1253=660%:%
%:%1254=661%:%
%:%1255=662%:%
%:%1256=663%:%
%:%1257=664%:%
%:%1258=665%:%
%:%1259=666%:%
%:%1260=667%:%
%:%1261=668%:%
%:%1262=669%:%
%:%1263=670%:%
%:%1264=671%:%
%:%1265=672%:%
%:%1266=673%:%
%:%1267=674%:%
%:%1268=675%:%
%:%1269=676%:%
%:%1270=677%:%
%:%1271=678%:%
%:%1272=679%:%
%:%1273=680%:%
%:%1274=681%:%
%:%1275=682%:%
%:%1276=683%:%
%:%1277=684%:%
%:%1278=685%:%
%:%1279=686%:%
%:%1280=687%:%
%:%1281=688%:%
%:%1282=689%:%
%:%1283=690%:%
%:%1284=691%:%
%:%1285=692%:%
%:%1286=693%:%
%:%1287=694%:%
%:%1288=695%:%
%:%1289=696%:%
%:%1290=697%:%
%:%1291=698%:%
%:%1292=699%:%
%:%1293=700%:%
%:%1294=701%:%
%:%1295=702%:%
%:%1296=703%:%
%:%1297=704%:%
%:%1298=705%:%
%:%1299=706%:%
%:%1300=707%:%
%:%1301=708%:%
%:%1302=709%:%
%:%1303=710%:%
%:%1304=711%:%
%:%1305=712%:%
%:%1306=713%:%
%:%1307=714%:%
%:%1308=715%:%
%:%1309=716%:%
%:%1310=717%:%
%:%1311=718%:%
%:%1312=719%:%
%:%1313=720%:%
%:%1314=721%:%
%:%1315=722%:%
%:%1316=723%:%
%:%1317=724%:%
%:%1318=725%:%
%:%1319=726%:%
%:%1320=727%:%
%:%1321=728%:%
%:%1322=729%:%
%:%1323=730%:%
%:%1324=731%:%
%:%1325=732%:%
%:%1326=733%:%
%:%1327=734%:%
%:%1328=735%:%
%:%1329=736%:%
%:%1330=737%:%
%:%1331=738%:%
%:%1332=739%:%
%:%1333=740%:%
%:%1334=741%:%
%:%1335=742%:%
%:%1336=743%:%
%:%1337=744%:%
%:%1338=745%:%
%:%1339=746%:%
%:%1340=747%:%
%:%1341=748%:%
%:%1342=749%:%
%:%1343=750%:%
%:%1344=751%:%
%:%1345=752%:%
%:%1346=753%:%
%:%1347=754%:%
%:%1348=755%:%
%:%1349=756%:%
%:%1350=757%:%
%:%1351=758%:%
%:%1352=759%:%
%:%1353=760%:%
%:%1354=761%:%
%:%1356=763%:%
%:%1357=763%:%
%:%1358=764%:%
%:%1359=765%:%
%:%1360=766%:%
%:%1361=766%:%
%:%1362=767%:%
%:%1364=769%:%
%:%1365=770%:%
%:%1366=771%:%
%:%1368=773%:%
%:%1369=773%:%
%:%1370=774%:%
%:%1371=775%:%
%:%1378=776%:%
%:%1379=776%:%
%:%1380=777%:%
%:%1381=777%:%
%:%1382=778%:%
%:%1383=778%:%
%:%1384=779%:%
%:%1385=779%:%
%:%1386=780%:%
%:%1387=780%:%
%:%1388=781%:%
%:%1389=781%:%
%:%1390=782%:%
%:%1391=782%:%
%:%1392=783%:%
%:%1393=783%:%
%:%1394=784%:%
%:%1395=784%:%
%:%1396=785%:%
%:%1397=785%:%
%:%1398=786%:%
%:%1399=786%:%
%:%1400=787%:%
%:%1401=787%:%
%:%1402=788%:%
%:%1403=788%:%
%:%1404=789%:%
%:%1405=789%:%
%:%1406=790%:%
%:%1407=790%:%
%:%1408=790%:%
%:%1409=791%:%
%:%1410=791%:%
%:%1411=792%:%
%:%1412=792%:%
%:%1413=793%:%
%:%1414=793%:%
%:%1415=794%:%
%:%1416=794%:%
%:%1417=795%:%
%:%1418=795%:%
%:%1419=795%:%
%:%1420=796%:%
%:%1421=796%:%
%:%1422=796%:%
%:%1423=797%:%
%:%1424=797%:%
%:%1425=797%:%
%:%1426=798%:%
%:%1427=798%:%
%:%1428=798%:%
%:%1429=799%:%
%:%1430=799%:%
%:%1431=799%:%
%:%1432=800%:%
%:%1433=800%:%
%:%1434=801%:%
%:%1435=801%:%
%:%1436=802%:%
%:%1437=802%:%
%:%1438=803%:%
%:%1439=803%:%
%:%1440=804%:%
%:%1441=804%:%
%:%1442=804%:%
%:%1443=805%:%
%:%1444=805%:%
%:%1445=805%:%
%:%1446=806%:%
%:%1447=806%:%
%:%1448=806%:%
%:%1449=807%:%
%:%1450=807%:%
%:%1451=807%:%
%:%1452=808%:%
%:%1453=808%:%
%:%1454=808%:%
%:%1455=809%:%
%:%1456=809%:%
%:%1457=809%:%
%:%1458=810%:%
%:%1459=810%:%
%:%1460=811%:%
%:%1461=811%:%
%:%1462=812%:%
%:%1463=812%:%
%:%1464=813%:%
%:%1465=813%:%
%:%1466=814%:%
%:%1467=814%:%
%:%1468=815%:%
%:%1469=815%:%
%:%1470=816%:%
%:%1471=816%:%
%:%1472=816%:%
%:%1473=817%:%
%:%1474=817%:%
%:%1475=818%:%
%:%1476=818%:%
%:%1477=819%:%
%:%1478=819%:%
%:%1479=820%:%
%:%1480=820%:%
%:%1481=821%:%
%:%1482=821%:%
%:%1483=822%:%
%:%1484=822%:%
%:%1485=823%:%
%:%1486=823%:%
%:%1487=824%:%
%:%1488=824%:%
%:%1489=825%:%
%:%1490=825%:%
%:%1491=826%:%
%:%1492=826%:%
%:%1493=826%:%
%:%1494=827%:%
%:%1495=827%:%
%:%1496=827%:%
%:%1497=828%:%
%:%1498=828%:%
%:%1499=828%:%
%:%1500=829%:%
%:%1501=829%:%
%:%1502=830%:%
%:%1503=830%:%
%:%1504=831%:%
%:%1505=831%:%
%:%1506=832%:%
%:%1507=832%:%
%:%1508=833%:%
%:%1509=833%:%
%:%1510=834%:%
%:%1511=834%:%
%:%1512=835%:%
%:%1513=835%:%
%:%1514=836%:%
%:%1515=836%:%
%:%1516=836%:%
%:%1517=837%:%
%:%1518=837%:%
%:%1519=837%:%
%:%1520=838%:%
%:%1521=838%:%
%:%1522=839%:%
%:%1523=839%:%
%:%1524=840%:%
%:%1525=840%:%
%:%1526=841%:%
%:%1527=841%:%
%:%1528=841%:%
%:%1529=842%:%
%:%1530=842%:%
%:%1531=843%:%
%:%1532=843%:%
%:%1533=843%:%
%:%1534=844%:%
%:%1535=844%:%
%:%1536=845%:%
%:%1537=845%:%
%:%1538=846%:%
%:%1539=846%:%
%:%1540=847%:%
%:%1541=847%:%
%:%1542=847%:%
%:%1543=848%:%
%:%1544=848%:%
%:%1545=849%:%
%:%1546=849%:%
%:%1547=850%:%
%:%1548=850%:%
%:%1549=851%:%
%:%1550=851%:%
%:%1551=852%:%
%:%1552=852%:%
%:%1553=853%:%
%:%1563=855%:%
%:%1564=856%:%
%:%1565=857%:%
%:%1566=858%:%
%:%1567=859%:%
%:%1568=860%:%
%:%1569=861%:%
%:%1570=862%:%
%:%1571=863%:%
%:%1572=864%:%
%:%1573=865%:%
%:%1574=866%:%
%:%1575=867%:%
%:%1576=868%:%
%:%1577=869%:%
%:%1578=870%:%
%:%1579=871%:%
%:%1580=872%:%
%:%1581=873%:%
%:%1583=875%:%
%:%1584=875%:%
%:%1585=876%:%
%:%1586=877%:%
%:%1587=878%:%
%:%1588=879%:%
%:%1591=882%:%
%:%1598=883%:%
%:%1599=883%:%
%:%1600=884%:%
%:%1601=884%:%
%:%1605=888%:%
%:%1606=889%:%
%:%1607=889%:%
%:%1608=889%:%
%:%1609=890%:%
%:%1610=890%:%
%:%1613=893%:%
%:%1614=894%:%
%:%1615=894%:%
%:%1616=894%:%
%:%1617=895%:%
%:%1618=895%:%
%:%1619=895%:%
%:%1620=896%:%
%:%1621=896%:%
%:%1622=897%:%
%:%1623=897%:%
%:%1624=898%:%
%:%1625=898%:%
%:%1626=898%:%
%:%1627=899%:%
%:%1628=899%:%
%:%1629=900%:%
%:%1630=900%:%
%:%1631=900%:%
%:%1632=901%:%
%:%1633=901%:%
%:%1634=902%:%
%:%1635=902%:%
%:%1636=903%:%
%:%1637=903%:%
%:%1638=904%:%
%:%1639=904%:%
%:%1640=905%:%
%:%1641=905%:%
%:%1642=906%:%
%:%1643=906%:%
%:%1644=907%:%
%:%1645=907%:%
%:%1646=908%:%
%:%1647=908%:%
%:%1648=909%:%
%:%1649=909%:%
%:%1650=909%:%
%:%1651=910%:%
%:%1652=910%:%
%:%1653=910%:%
%:%1654=911%:%
%:%1655=911%:%
%:%1656=911%:%
%:%1657=912%:%
%:%1658=912%:%
%:%1659=912%:%
%:%1660=913%:%
%:%1661=913%:%
%:%1662=913%:%
%:%1663=914%:%
%:%1664=914%:%
%:%1665=915%:%
%:%1666=915%:%
%:%1667=915%:%
%:%1668=916%:%
%:%1669=916%:%
%:%1670=917%:%
%:%1671=917%:%
%:%1672=918%:%
%:%1673=918%:%
%:%1674=919%:%
%:%1675=919%:%
%:%1676=919%:%
%:%1677=920%:%
%:%1678=920%:%
%:%1679=921%:%
%:%1680=921%:%
%:%1681=922%:%
%:%1682=922%:%
%:%1683=923%:%
%:%1684=923%:%
%:%1685=924%:%
%:%1686=924%:%
%:%1687=925%:%
%:%1688=925%:%
%:%1689=926%:%
%:%1690=926%:%
%:%1691=927%:%
%:%1692=927%:%
%:%1693=928%:%
%:%1694=928%:%
%:%1695=928%:%
%:%1696=929%:%
%:%1697=929%:%
%:%1698=929%:%
%:%1699=930%:%
%:%1700=930%:%
%:%1701=930%:%
%:%1702=931%:%
%:%1703=931%:%
%:%1704=931%:%
%:%1705=932%:%
%:%1706=932%:%
%:%1707=932%:%
%:%1708=933%:%
%:%1709=933%:%
%:%1710=934%:%
%:%1711=934%:%
%:%1712=935%:%
%:%1713=935%:%
%:%1714=936%:%
%:%1715=936%:%
%:%1716=936%:%
%:%1717=937%:%
%:%1718=937%:%
%:%1719=937%:%
%:%1720=938%:%
%:%1721=938%:%
%:%1722=939%:%
%:%1723=939%:%
%:%1724=940%:%
%:%1725=940%:%
%:%1726=941%:%
%:%1727=941%:%
%:%1728=942%:%
%:%1729=942%:%
%:%1730=942%:%
%:%1731=943%:%
%:%1732=943%:%
%:%1733=943%:%
%:%1734=944%:%
%:%1735=944%:%
%:%1736=945%:%
%:%1737=945%:%
%:%1738=946%:%
%:%1739=946%:%
%:%1740=947%:%
%:%1741=947%:%
%:%1742=948%:%
%:%1743=948%:%
%:%1744=949%:%
%:%1745=949%:%
%:%1748=952%:%
%:%1749=953%:%
%:%1750=953%:%
%:%1751=953%:%
%:%1752=954%:%
%:%1762=956%:%
%:%1763=957%:%
%:%1764=958%:%
%:%1765=959%:%
%:%1766=960%:%
%:%1767=961%:%
%:%1769=963%:%
%:%1770=963%:%
%:%1771=964%:%
%:%1772=965%:%
%:%1773=966%:%
%:%1774=967%:%
%:%1775=968%:%
%:%1776=969%:%
%:%1783=970%:%
%:%1784=970%:%
%:%1785=971%:%
%:%1786=971%:%
%:%1790=975%:%
%:%1791=976%:%
%:%1792=976%:%
%:%1793=976%:%
%:%1794=977%:%
%:%1795=977%:%
%:%1796=978%:%
%:%1797=978%:%
%:%1798=979%:%
%:%1799=979%:%
%:%1800=980%:%
%:%1801=980%:%
%:%1802=981%:%
%:%1803=981%:%
%:%1804=982%:%
%:%1805=982%:%
%:%1806=983%:%
%:%1807=983%:%
%:%1808=984%:%
%:%1809=984%:%
%:%1810=985%:%
%:%1811=985%:%
%:%1812=986%:%
%:%1813=986%:%
%:%1814=986%:%
%:%1815=986%:%
%:%1816=987%:%
%:%1817=987%:%
%:%1818=987%:%
%:%1819=988%:%
%:%1820=988%:%
%:%1821=988%:%
%:%1822=989%:%
%:%1823=989%:%
%:%1824=989%:%
%:%1825=990%:%
%:%1826=990%:%
%:%1827=990%:%
%:%1828=991%:%
%:%1829=991%:%
%:%1832=994%:%
%:%1833=995%:%
%:%1834=995%:%
%:%1835=995%:%
%:%1836=996%:%
%:%1837=996%:%
%:%1838=996%:%
%:%1839=997%:%
%:%1840=997%:%
%:%1841=998%:%
%:%1842=998%:%
%:%1843=998%:%
%:%1844=999%:%
%:%1845=999%:%
%:%1846=1000%:%
%:%1847=1000%:%
%:%1848=1000%:%
%:%1849=1001%:%
%:%1850=1001%:%
%:%1851=1001%:%
%:%1852=1002%:%
%:%1853=1002%:%
%:%1854=1003%:%
%:%1855=1003%:%
%:%1856=1003%:%
%:%1857=1004%:%
%:%1858=1004%:%
%:%1859=1005%:%
%:%1860=1005%:%
%:%1861=1006%:%
%:%1862=1006%:%
%:%1863=1007%:%
%:%1864=1007%:%
%:%1865=1008%:%
%:%1866=1008%:%
%:%1867=1009%:%
%:%1868=1009%:%
%:%1869=1010%:%
%:%1870=1010%:%
%:%1871=1011%:%
%:%1872=1011%:%
%:%1873=1012%:%
%:%1874=1012%:%
%:%1875=1013%:%
%:%1876=1013%:%
%:%1877=1014%:%
%:%1878=1014%:%
%:%1879=1015%:%
%:%1880=1015%:%
%:%1881=1015%:%
%:%1882=1016%:%
%:%1883=1016%:%
%:%1884=1017%:%
%:%1885=1017%:%
%:%1886=1017%:%
%:%1887=1018%:%
%:%1888=1018%:%
%:%1889=1018%:%
%:%1890=1019%:%
%:%1891=1019%:%
%:%1892=1020%:%
%:%1893=1020%:%
%:%1894=1021%:%
%:%1895=1021%:%
%:%1896=1021%:%
%:%1897=1022%:%
%:%1898=1022%:%
%:%1899=1023%:%
%:%1900=1023%:%
%:%1901=1023%:%
%:%1902=1024%:%
%:%1903=1024%:%
%:%1904=1024%:%
%:%1905=1025%:%
%:%1906=1025%:%
%:%1907=1026%:%
%:%1908=1026%:%
%:%1909=1027%:%
%:%1910=1027%:%
%:%1911=1027%:%
%:%1912=1028%:%
%:%1913=1028%:%
%:%1914=1029%:%
%:%1915=1029%:%
%:%1916=1030%:%
%:%1917=1030%:%
%:%1918=1031%:%
%:%1919=1031%:%
%:%1920=1031%:%
%:%1921=1032%:%
%:%1922=1032%:%
%:%1923=1033%:%
%:%1924=1033%:%
%:%1925=1034%:%
%:%1926=1034%:%
%:%1927=1035%:%
%:%1928=1035%:%
%:%1929=1036%:%
%:%1930=1036%:%
%:%1931=1037%:%
%:%1932=1037%:%
%:%1933=1038%:%
%:%1934=1038%:%
%:%1935=1039%:%
%:%1936=1039%:%
%:%1937=1040%:%
%:%1938=1040%:%
%:%1939=1040%:%
%:%1940=1041%:%
%:%1941=1041%:%
%:%1942=1042%:%
%:%1943=1042%:%
%:%1944=1042%:%
%:%1945=1043%:%
%:%1946=1043%:%
%:%1947=1044%:%
%:%1948=1044%:%
%:%1949=1044%:%
%:%1950=1045%:%
%:%1951=1045%:%
%:%1952=1046%:%
%:%1953=1046%:%
%:%1954=1046%:%
%:%1955=1047%:%
%:%1956=1047%:%
%:%1957=1048%:%
%:%1958=1048%:%
%:%1959=1048%:%
%:%1960=1049%:%
%:%1961=1049%:%
%:%1962=1050%:%
%:%1963=1050%:%
%:%1964=1051%:%
%:%1965=1051%:%
%:%1966=1051%:%
%:%1967=1052%:%
%:%1968=1052%:%
%:%1969=1053%:%
%:%1970=1053%:%
%:%1971=1054%:%
%:%1972=1054%:%
%:%1973=1054%:%
%:%1974=1055%:%
%:%1975=1055%:%
%:%1976=1056%:%
%:%1977=1056%:%
%:%1978=1057%:%
%:%1979=1057%:%
%:%1980=1058%:%
%:%1981=1058%:%
%:%1982=1058%:%
%:%1983=1059%:%
%:%1984=1059%:%
%:%1985=1059%:%
%:%1986=1060%:%
%:%1987=1060%:%
%:%1988=1060%:%
%:%1989=1061%:%
%:%1990=1061%:%
%:%1991=1062%:%
%:%1992=1062%:%
%:%1993=1062%:%
%:%1994=1063%:%
%:%1995=1063%:%
%:%1996=1064%:%
%:%1997=1064%:%
%:%1998=1065%:%
%:%1999=1065%:%
%:%2000=1066%:%
%:%2001=1066%:%
%:%2002=1066%:%
%:%2003=1067%:%
%:%2013=1069%:%
%:%2014=1070%:%
%:%2015=1071%:%
%:%2016=1072%:%
%:%2017=1073%:%
%:%2018=1074%:%
%:%2019=1075%:%
%:%2020=1076%:%
%:%2021=1077%:%
%:%2022=1078%:%
%:%2023=1079%:%
%:%2024=1080%:%
%:%2026=1082%:%
%:%2027=1082%:%
%:%2028=1083%:%
%:%2029=1084%:%
%:%2030=1085%:%
%:%2031=1086%:%
%:%2032=1087%:%
%:%2033=1088%:%
%:%2034=1089%:%
%:%2035=1090%:%
%:%2036=1091%:%
%:%2037=1092%:%
%:%2038=1093%:%
%:%2045=1094%:%
%:%2046=1094%:%
%:%2047=1095%:%
%:%2048=1095%:%
%:%2049=1096%:%
%:%2050=1096%:%
%:%2051=1097%:%
%:%2052=1097%:%
%:%2053=1098%:%
%:%2054=1098%:%
%:%2055=1099%:%
%:%2056=1099%:%
%:%2057=1100%:%
%:%2058=1100%:%
%:%2059=1101%:%
%:%2060=1101%:%
%:%2061=1101%:%
%:%2062=1102%:%
%:%2063=1102%:%
%:%2064=1103%:%
%:%2065=1103%:%
%:%2066=1103%:%
%:%2067=1104%:%
%:%2068=1104%:%
%:%2069=1105%:%
%:%2070=1105%:%
%:%2071=1105%:%
%:%2072=1105%:%
%:%2073=1106%:%
%:%2074=1106%:%
%:%2075=1106%:%
%:%2076=1107%:%
%:%2077=1107%:%
%:%2078=1107%:%
%:%2079=1108%:%
%:%2080=1108%:%
%:%2081=1108%:%
%:%2082=1109%:%
%:%2083=1109%:%
%:%2084=1109%:%
%:%2085=1110%:%
%:%2086=1110%:%
%:%2087=1111%:%
%:%2088=1111%:%
%:%2089=1111%:%
%:%2090=1112%:%
%:%2091=1112%:%
%:%2094=1115%:%
%:%2095=1116%:%
%:%2096=1116%:%
%:%2097=1116%:%
%:%2098=1117%:%
%:%2099=1117%:%
%:%2100=1117%:%
%:%2103=1120%:%
%:%2104=1121%:%
%:%2105=1121%:%
%:%2106=1121%:%
%:%2107=1122%:%
%:%2108=1122%:%
%:%2109=1122%:%
%:%2110=1123%:%
%:%2111=1123%:%
%:%2112=1124%:%
%:%2113=1124%:%
%:%2114=1124%:%
%:%2115=1125%:%
%:%2116=1125%:%
%:%2117=1126%:%
%:%2118=1126%:%
%:%2119=1126%:%
%:%2120=1127%:%
%:%2121=1127%:%
%:%2122=1127%:%
%:%2123=1128%:%
%:%2124=1128%:%
%:%2125=1128%:%
%:%2126=1129%:%
%:%2127=1129%:%
%:%2128=1129%:%
%:%2129=1130%:%
%:%2130=1130%:%
%:%2131=1131%:%
%:%2132=1131%:%
%:%2133=1132%:%
%:%2134=1132%:%
%:%2135=1133%:%
%:%2136=1133%:%
%:%2137=1134%:%
%:%2138=1134%:%
%:%2139=1135%:%
%:%2140=1135%:%
%:%2141=1135%:%
%:%2142=1136%:%
%:%2143=1136%:%
%:%2144=1136%:%
%:%2145=1137%:%
%:%2146=1137%:%
%:%2147=1138%:%
%:%2148=1138%:%
%:%2149=1138%:%
%:%2150=1139%:%
%:%2151=1139%:%
%:%2152=1140%:%
%:%2153=1140%:%
%:%2154=1140%:%
%:%2155=1141%:%
%:%2156=1141%:%
%:%2157=1142%:%
%:%2158=1142%:%
%:%2159=1142%:%
%:%2160=1143%:%
%:%2161=1143%:%
%:%2162=1144%:%
%:%2163=1144%:%
%:%2164=1145%:%
%:%2165=1145%:%
%:%2166=1145%:%
%:%2167=1146%:%
%:%2168=1146%:%
%:%2169=1147%:%
%:%2170=1147%:%
%:%2171=1147%:%
%:%2172=1148%:%
%:%2173=1148%:%
%:%2174=1148%:%
%:%2175=1149%:%
%:%2176=1149%:%
%:%2177=1149%:%
%:%2178=1150%:%
%:%2179=1150%:%
%:%2180=1151%:%
%:%2181=1151%:%
%:%2182=1151%:%
%:%2183=1152%:%
%:%2184=1152%:%
%:%2185=1153%:%
%:%2186=1153%:%
%:%2187=1153%:%
%:%2188=1154%:%
%:%2189=1154%:%
%:%2190=1155%:%
%:%2191=1155%:%
%:%2192=1155%:%
%:%2193=1156%:%
%:%2194=1156%:%
%:%2195=1157%:%
%:%2196=1157%:%
%:%2197=1157%:%
%:%2198=1158%:%
%:%2199=1158%:%
%:%2200=1159%:%
%:%2201=1159%:%
%:%2202=1159%:%
%:%2203=1160%:%
%:%2204=1160%:%
%:%2205=1161%:%
%:%2206=1161%:%
%:%2207=1161%:%
%:%2208=1162%:%
%:%2218=1164%:%
%:%2219=1165%:%
%:%2220=1166%:%
%:%2221=1167%:%
%:%2222=1168%:%
%:%2224=1170%:%
%:%2225=1170%:%
%:%2226=1171%:%
%:%2227=1172%:%
%:%2228=1173%:%
%:%2229=1174%:%
%:%2230=1175%:%
%:%2231=1176%:%
%:%2232=1177%:%
%:%2233=1178%:%
%:%2234=1179%:%
%:%2235=1180%:%
%:%2236=1181%:%
%:%2237=1182%:%
%:%2244=1183%:%
%:%2245=1183%:%
%:%2246=1184%:%
%:%2247=1184%:%
%:%2248=1185%:%
%:%2249=1185%:%
%:%2250=1186%:%
%:%2251=1186%:%
%:%2252=1187%:%
%:%2253=1187%:%
%:%2254=1187%:%
%:%2255=1188%:%
%:%2256=1188%:%
%:%2257=1189%:%
%:%2258=1189%:%
%:%2259=1189%:%
%:%2260=1190%:%
%:%2261=1190%:%
%:%2262=1191%:%
%:%2263=1191%:%
%:%2264=1191%:%
%:%2265=1192%:%
%:%2266=1192%:%
%:%2267=1193%:%
%:%2268=1193%:%
%:%2269=1194%:%
%:%2270=1194%:%
%:%2271=1195%:%
%:%2272=1195%:%
%:%2273=1196%:%
%:%2274=1196%:%
%:%2275=1197%:%
%:%2276=1197%:%
%:%2277=1197%:%
%:%2278=1198%:%
%:%2279=1198%:%
%:%2280=1199%:%
%:%2281=1199%:%
%:%2282=1200%:%
%:%2283=1200%:%
%:%2284=1200%:%
%:%2285=1201%:%
%:%2286=1201%:%
%:%2287=1202%:%
%:%2288=1202%:%
%:%2289=1203%:%
%:%2290=1203%:%
%:%2291=1203%:%
%:%2292=1204%:%
%:%2293=1204%:%
%:%2294=1205%:%
%:%2295=1205%:%
%:%2296=1206%:%
%:%2297=1206%:%
%:%2298=1206%:%
%:%2299=1207%:%
%:%2300=1207%:%
%:%2301=1208%:%
%:%2302=1208%:%
%:%2303=1208%:%
%:%2304=1209%:%
%:%2305=1209%:%
%:%2306=1210%:%
%:%2307=1210%:%
%:%2308=1211%:%
%:%2309=1211%:%
%:%2310=1211%:%
%:%2311=1212%:%
%:%2312=1212%:%
%:%2313=1213%:%
%:%2314=1213%:%
%:%2315=1213%:%
%:%2316=1214%:%
%:%2317=1214%:%
%:%2318=1215%:%
%:%2319=1215%:%
%:%2320=1216%:%
%:%2321=1216%:%
%:%2322=1217%:%
%:%2323=1217%:%
%:%2324=1218%:%
%:%2325=1218%:%
%:%2326=1219%:%
%:%2327=1219%:%
%:%2328=1219%:%
%:%2329=1220%:%
%:%2330=1220%:%
%:%2331=1221%:%
%:%2332=1221%:%
%:%2333=1222%:%
%:%2334=1222%:%
%:%2335=1222%:%
%:%2336=1223%:%
%:%2337=1223%:%
%:%2338=1224%:%
%:%2339=1224%:%
%:%2340=1225%:%
%:%2341=1225%:%
%:%2342=1225%:%
%:%2343=1226%:%
%:%2344=1226%:%
%:%2345=1227%:%
%:%2346=1227%:%
%:%2347=1228%:%
%:%2348=1228%:%
%:%2349=1228%:%
%:%2350=1229%:%
%:%2351=1229%:%
%:%2352=1230%:%
%:%2353=1230%:%
%:%2354=1230%:%
%:%2355=1231%:%
%:%2356=1231%:%
%:%2357=1232%:%
%:%2358=1232%:%
%:%2359=1233%:%
%:%2360=1233%:%
%:%2361=1233%:%
%:%2362=1234%:%
%:%2363=1234%:%
%:%2364=1235%:%
%:%2365=1235%:%
%:%2366=1235%:%
%:%2367=1236%:%
%:%2368=1236%:%
%:%2369=1236%:%
%:%2370=1237%:%
%:%2371=1237%:%
%:%2372=1238%:%
%:%2373=1238%:%
%:%2374=1238%:%
%:%2375=1239%:%
%:%2376=1239%:%
%:%2377=1240%:%
%:%2378=1240%:%
%:%2379=1241%:%
%:%2380=1241%:%
%:%2381=1241%:%
%:%2382=1242%:%
%:%2392=1244%:%
%:%2393=1245%:%
%:%2394=1246%:%
%:%2395=1247%:%
%:%2396=1248%:%
%:%2398=1250%:%
%:%2399=1250%:%
%:%2400=1251%:%
%:%2401=1252%:%
%:%2404=1253%:%
%:%2408=1253%:%
%:%2409=1253%:%
%:%2410=1253%:%
%:%2415=1253%:%
%:%2418=1254%:%
%:%2419=1255%:%
%:%2420=1255%:%
%:%2423=1256%:%
%:%2427=1256%:%
%:%2428=1256%:%
%:%2437=1258%:%
%:%2439=1260%:%
%:%2440=1260%:%
%:%2441=1261%:%
%:%2442=1262%:%
%:%2443=1263%:%
%:%2444=1264%:%
%:%2445=1265%:%
%:%2446=1266%:%
%:%2453=1267%:%
%:%2454=1267%:%
%:%2455=1268%:%
%:%2456=1268%:%
%:%2457=1269%:%
%:%2458=1269%:%
%:%2459=1269%:%
%:%2460=1270%:%
%:%2461=1270%:%
%:%2465=1274%:%
%:%2466=1275%:%
%:%2467=1275%:%
%:%2468=1275%:%
%:%2469=1276%:%
%:%2470=1276%:%
%:%2471=1277%:%
%:%2472=1277%:%
%:%2473=1277%:%
%:%2474=1277%:%
%:%2475=1278%:%
%:%2476=1278%:%
%:%2477=1278%:%
%:%2478=1279%:%
%:%2479=1279%:%
%:%2480=1280%:%
%:%2481=1280%:%
%:%2482=1281%:%
%:%2483=1281%:%
%:%2484=1281%:%
%:%2485=1282%:%
%:%2486=1282%:%
%:%2487=1283%:%
%:%2488=1283%:%
%:%2489=1284%:%
%:%2490=1284%:%
%:%2491=1285%:%
%:%2492=1285%:%
%:%2493=1285%:%
%:%2494=1286%:%
%:%2495=1286%:%
%:%2496=1287%:%
%:%2497=1287%:%
%:%2498=1287%:%
%:%2499=1288%:%
%:%2500=1288%:%
%:%2501=1289%:%
%:%2502=1289%:%
%:%2503=1289%:%
%:%2504=1290%:%
%:%2505=1290%:%
%:%2506=1290%:%
%:%2507=1291%:%
%:%2508=1291%:%
%:%2509=1291%:%
%:%2510=1292%:%
%:%2511=1292%:%
%:%2512=1293%:%
%:%2513=1293%:%
%:%2514=1294%:%
%:%2515=1294%:%
%:%2516=1294%:%
%:%2517=1295%:%
%:%2518=1295%:%
%:%2519=1296%:%
%:%2520=1296%:%
%:%2521=1296%:%
%:%2522=1297%:%
%:%2523=1297%:%
%:%2524=1297%:%
%:%2525=1298%:%
%:%2526=1298%:%
%:%2527=1299%:%
%:%2528=1299%:%
%:%2529=1300%:%
%:%2530=1300%:%
%:%2531=1300%:%
%:%2532=1301%:%
%:%2533=1301%:%
%:%2534=1301%:%
%:%2535=1302%:%
%:%2536=1302%:%
%:%2537=1303%:%
%:%2538=1303%:%
%:%2539=1304%:%
%:%2540=1304%:%
%:%2541=1304%:%
%:%2542=1305%:%
%:%2543=1305%:%
%:%2544=1305%:%
%:%2545=1306%:%
%:%2546=1306%:%
%:%2547=1306%:%
%:%2548=1307%:%
%:%2549=1307%:%
%:%2550=1307%:%
%:%2551=1308%:%
%:%2552=1308%:%
%:%2553=1308%:%
%:%2554=1309%:%
%:%2555=1309%:%
%:%2556=1309%:%
%:%2557=1310%:%
%:%2558=1310%:%
%:%2559=1311%:%
%:%2560=1311%:%
%:%2561=1312%:%
%:%2562=1312%:%
%:%2563=1313%:%
%:%2564=1313%:%
%:%2565=1314%:%
%:%2566=1314%:%
%:%2567=1314%:%
%:%2568=1315%:%
%:%2569=1315%:%
%:%2570=1315%:%
%:%2571=1316%:%
%:%2572=1316%:%
%:%2573=1317%:%
%:%2574=1317%:%
%:%2575=1317%:%
%:%2576=1318%:%
%:%2577=1318%:%
%:%2578=1319%:%
%:%2579=1319%:%
%:%2580=1320%:%
%:%2581=1320%:%
%:%2582=1320%:%
%:%2583=1321%:%
%:%2584=1321%:%
%:%2585=1322%:%
%:%2586=1322%:%
%:%2587=1323%:%
%:%2588=1323%:%
%:%2589=1324%:%
%:%2590=1324%:%
%:%2591=1325%:%
%:%2592=1325%:%
%:%2593=1326%:%
%:%2594=1326%:%
%:%2595=1326%:%
%:%2596=1327%:%
%:%2597=1327%:%
%:%2598=1328%:%
%:%2599=1328%:%
%:%2600=1328%:%
%:%2601=1329%:%
%:%2602=1329%:%
%:%2603=1330%:%
%:%2604=1330%:%
%:%2605=1330%:%
%:%2606=1331%:%
%:%2607=1331%:%
%:%2608=1332%:%
%:%2609=1332%:%
%:%2610=1333%:%
%:%2611=1333%:%
%:%2612=1333%:%
%:%2613=1334%:%
%:%2614=1334%:%
%:%2615=1334%:%
%:%2616=1335%:%
%:%2617=1335%:%
%:%2618=1336%:%
%:%2619=1336%:%
%:%2620=1336%:%
%:%2621=1337%:%
%:%2622=1337%:%
%:%2623=1337%:%
%:%2624=1338%:%
%:%2625=1338%:%
%:%2626=1338%:%
%:%2627=1339%:%
%:%2628=1339%:%
%:%2629=1339%:%
%:%2630=1340%:%
%:%2631=1340%:%
%:%2632=1341%:%
%:%2633=1341%:%
%:%2634=1342%:%
%:%2635=1342%:%
%:%2636=1342%:%
%:%2637=1343%:%
%:%2638=1343%:%
%:%2639=1344%:%
%:%2640=1344%:%
%:%2641=1344%:%
%:%2642=1345%:%
%:%2643=1345%:%
%:%2644=1345%:%
%:%2645=1346%:%
%:%2646=1346%:%
%:%2647=1347%:%
%:%2648=1347%:%
%:%2649=1348%:%
%:%2650=1348%:%
%:%2651=1348%:%
%:%2652=1349%:%
%:%2653=1349%:%
%:%2654=1349%:%
%:%2655=1350%:%
%:%2656=1350%:%
%:%2657=1351%:%
%:%2658=1351%:%
%:%2659=1352%:%
%:%2660=1352%:%
%:%2661=1352%:%
%:%2662=1353%:%
%:%2663=1353%:%
%:%2664=1353%:%
%:%2665=1354%:%
%:%2666=1354%:%
%:%2667=1354%:%
%:%2668=1355%:%
%:%2669=1355%:%
%:%2670=1355%:%
%:%2671=1356%:%
%:%2672=1356%:%
%:%2673=1356%:%
%:%2674=1357%:%
%:%2675=1357%:%
%:%2676=1357%:%
%:%2677=1358%:%
%:%2678=1358%:%
%:%2679=1359%:%
%:%2680=1359%:%
%:%2681=1360%:%
%:%2682=1360%:%
%:%2683=1361%:%
%:%2684=1361%:%
%:%2685=1362%:%
%:%2686=1362%:%
%:%2687=1362%:%
%:%2688=1363%:%
%:%2689=1363%:%
%:%2690=1363%:%
%:%2691=1364%:%
%:%2692=1364%:%
%:%2693=1365%:%
%:%2694=1365%:%
%:%2695=1365%:%
%:%2696=1366%:%
%:%2697=1366%:%
%:%2698=1367%:%
%:%2699=1367%:%
%:%2700=1368%:%
%:%2701=1368%:%
%:%2702=1368%:%
%:%2703=1369%:%
%:%2704=1369%:%
%:%2705=1370%:%
%:%2706=1370%:%
%:%2707=1371%:%
%:%2708=1371%:%
%:%2709=1372%:%
%:%2710=1372%:%
%:%2711=1373%:%
%:%2712=1373%:%
%:%2713=1374%:%
%:%2714=1374%:%
%:%2715=1374%:%
%:%2716=1375%:%
%:%2717=1375%:%
%:%2718=1376%:%
%:%2719=1376%:%
%:%2720=1376%:%
%:%2721=1377%:%
%:%2722=1377%:%
%:%2723=1378%:%
%:%2724=1378%:%
%:%2725=1378%:%
%:%2726=1379%:%
%:%2727=1379%:%
%:%2728=1380%:%
%:%2729=1380%:%
%:%2730=1381%:%
%:%2731=1381%:%
%:%2732=1381%:%
%:%2733=1382%:%
%:%2734=1382%:%
%:%2735=1382%:%
%:%2736=1383%:%
%:%2737=1383%:%
%:%2738=1384%:%
%:%2739=1384%:%
%:%2740=1384%:%
%:%2741=1385%:%
%:%2742=1385%:%
%:%2743=1386%:%
%:%2744=1386%:%
%:%2745=1387%:%
%:%2746=1387%:%
%:%2747=1388%:%
%:%2748=1388%:%
%:%2749=1389%:%
%:%2750=1389%:%
%:%2751=1390%:%
%:%2752=1390%:%
%:%2753=1390%:%
%:%2754=1391%:%
%:%2755=1391%:%
%:%2756=1392%:%
%:%2757=1392%:%
%:%2758=1393%:%
%:%2759=1393%:%
%:%2760=1393%:%
%:%2761=1394%:%
%:%2762=1394%:%
%:%2763=1394%:%
%:%2764=1395%:%
%:%2765=1395%:%
%:%2766=1396%:%
%:%2767=1396%:%
%:%2768=1397%:%
%:%2769=1397%:%
%:%2770=1398%:%
%:%2771=1398%:%
%:%2772=1399%:%
%:%2773=1399%:%
%:%2774=1400%:%
%:%2775=1400%:%
%:%2776=1400%:%
%:%2777=1401%:%
%:%2778=1401%:%
%:%2779=1402%:%
%:%2780=1402%:%
%:%2781=1403%:%
%:%2782=1403%:%
%:%2783=1403%:%
%:%2784=1404%:%
%:%2785=1404%:%
%:%2786=1404%:%
%:%2787=1405%:%
%:%2788=1405%:%
%:%2789=1406%:%
%:%2790=1406%:%
%:%2791=1407%:%
%:%2792=1407%:%
%:%2793=1408%:%
%:%2803=1410%:%
%:%2804=1411%:%
%:%2805=1412%:%
%:%2806=1413%:%
%:%2807=1414%:%
%:%2808=1415%:%
%:%2809=1416%:%
%:%2810=1417%:%
%:%2811=1418%:%
%:%2812=1419%:%
%:%2813=1420%:%
%:%2814=1421%:%
%:%2815=1422%:%
%:%2816=1423%:%
%:%2817=1424%:%
%:%2819=1426%:%
%:%2820=1426%:%
%:%2821=1427%:%
%:%2822=1428%:%
%:%2823=1429%:%
%:%2824=1430%:%
%:%2827=1433%:%
%:%2834=1434%:%
%:%2835=1434%:%
%:%2836=1435%:%
%:%2837=1435%:%
%:%2841=1439%:%
%:%2842=1440%:%
%:%2843=1440%:%
%:%2844=1440%:%
%:%2845=1441%:%
%:%2846=1441%:%
%:%2847=1442%:%
%:%2848=1442%:%
%:%2849=1442%:%
%:%2850=1442%:%
%:%2851=1443%:%
%:%2852=1443%:%
%:%2853=1444%:%
%:%2854=1444%:%
%:%2855=1445%:%
%:%2856=1445%:%
%:%2857=1446%:%
%:%2858=1446%:%
%:%2859=1447%:%
%:%2860=1447%:%
%:%2861=1448%:%
%:%2862=1448%:%
%:%2863=1448%:%
%:%2864=1449%:%
%:%2865=1449%:%
%:%2866=1450%:%
%:%2867=1450%:%
%:%2868=1450%:%
%:%2869=1451%:%
%:%2870=1451%:%
%:%2871=1451%:%
%:%2872=1452%:%
%:%2873=1452%:%
%:%2874=1452%:%
%:%2875=1453%:%
%:%2876=1453%:%
%:%2877=1454%:%
%:%2878=1454%:%
%:%2879=1455%:%
%:%2880=1455%:%
%:%2881=1456%:%
%:%2882=1456%:%
%:%2883=1456%:%
%:%2884=1457%:%
%:%2885=1457%:%
%:%2886=1458%:%
%:%2887=1458%:%
%:%2888=1459%:%
%:%2889=1459%:%
%:%2890=1459%:%
%:%2891=1460%:%
%:%2892=1460%:%
%:%2893=1460%:%
%:%2894=1461%:%
%:%2895=1461%:%
%:%2896=1461%:%
%:%2897=1462%:%
%:%2898=1462%:%
%:%2901=1465%:%
%:%2902=1466%:%
%:%2903=1466%:%
%:%2904=1466%:%
%:%2905=1467%:%
%:%2906=1467%:%
%:%2907=1467%:%
%:%2908=1468%:%
%:%2909=1468%:%
%:%2910=1469%:%
%:%2911=1469%:%
%:%2912=1470%:%
%:%2913=1470%:%
%:%2914=1471%:%
%:%2915=1471%:%
%:%2916=1472%:%
%:%2917=1472%:%
%:%2918=1472%:%
%:%2919=1473%:%
%:%2920=1473%:%
%:%2921=1474%:%
%:%2922=1474%:%
%:%2923=1475%:%
%:%2924=1475%:%
%:%2925=1476%:%
%:%2926=1476%:%
%:%2927=1477%:%
%:%2928=1477%:%
%:%2929=1478%:%
%:%2930=1478%:%
%:%2931=1479%:%
%:%2932=1479%:%
%:%2933=1480%:%
%:%2934=1480%:%
%:%2935=1481%:%
%:%2936=1481%:%
%:%2937=1482%:%
%:%2938=1482%:%
%:%2939=1483%:%
%:%2940=1483%:%
%:%2941=1484%:%
%:%2942=1484%:%
%:%2943=1485%:%
%:%2944=1485%:%
%:%2945=1486%:%
%:%2946=1486%:%
%:%2947=1487%:%
%:%2948=1487%:%
%:%2949=1487%:%
%:%2950=1488%:%
%:%2951=1488%:%
%:%2952=1489%:%
%:%2953=1489%:%
%:%2954=1489%:%
%:%2955=1490%:%
%:%2956=1490%:%
%:%2957=1491%:%
%:%2958=1491%:%
%:%2959=1492%:%
%:%2960=1492%:%
%:%2961=1492%:%
%:%2962=1493%:%
%:%2963=1493%:%
%:%2964=1494%:%
%:%2965=1494%:%
%:%2966=1494%:%
%:%2967=1495%:%
%:%2968=1495%:%
%:%2969=1496%:%
%:%2970=1496%:%
%:%2971=1497%:%
%:%2972=1497%:%
%:%2973=1497%:%
%:%2974=1498%:%
%:%2975=1498%:%
%:%2976=1498%:%
%:%2977=1499%:%
%:%2978=1499%:%
%:%2979=1499%:%
%:%2980=1500%:%
%:%2981=1500%:%
%:%2984=1503%:%
%:%2985=1504%:%
%:%2986=1504%:%
%:%2987=1504%:%
%:%2988=1505%:%
%:%2989=1505%:%
%:%2990=1505%:%
%:%2991=1506%:%
%:%2992=1506%:%
%:%2993=1507%:%
%:%2994=1507%:%
%:%2995=1507%:%
%:%2996=1508%:%
%:%2997=1508%:%
%:%2998=1509%:%
%:%2999=1509%:%
%:%3000=1509%:%
%:%3001=1510%:%
%:%3002=1510%:%
%:%3003=1510%:%
%:%3004=1511%:%
%:%3005=1511%:%
%:%3006=1512%:%
%:%3007=1512%:%
%:%3008=1512%:%
%:%3009=1513%:%
%:%3010=1513%:%
%:%3011=1514%:%
%:%3012=1514%:%
%:%3013=1515%:%
%:%3014=1515%:%
%:%3015=1516%:%
%:%3016=1516%:%
%:%3017=1517%:%
%:%3018=1517%:%
%:%3019=1518%:%
%:%3020=1518%:%
%:%3021=1519%:%
%:%3022=1519%:%
%:%3023=1520%:%
%:%3024=1520%:%
%:%3025=1521%:%
%:%3026=1521%:%
%:%3027=1522%:%
%:%3028=1522%:%
%:%3029=1523%:%
%:%3030=1523%:%
%:%3031=1523%:%
%:%3032=1524%:%
%:%3033=1524%:%
%:%3034=1525%:%
%:%3035=1525%:%
%:%3036=1526%:%
%:%3037=1526%:%
%:%3038=1527%:%
%:%3039=1527%:%
%:%3040=1528%:%
%:%3041=1528%:%
%:%3042=1529%:%
%:%3043=1529%:%
%:%3044=1530%:%
%:%3045=1530%:%
%:%3046=1531%:%
%:%3047=1531%:%
%:%3048=1532%:%
%:%3049=1532%:%
%:%3050=1533%:%
%:%3051=1533%:%
%:%3052=1534%:%
%:%3053=1534%:%
%:%3054=1534%:%
%:%3055=1535%:%
%:%3056=1535%:%
%:%3057=1536%:%
%:%3058=1536%:%
%:%3059=1537%:%
%:%3060=1537%:%
%:%3061=1538%:%
%:%3062=1538%:%
%:%3063=1538%:%
%:%3064=1539%:%
%:%3074=1541%:%
%:%3075=1542%:%
%:%3076=1543%:%
%:%3078=1545%:%
%:%3079=1545%:%
%:%3080=1546%:%
%:%3081=1547%:%
%:%3082=1548%:%
%:%3085=1549%:%
%:%3089=1549%:%
%:%3090=1549%:%
%:%3091=1550%:%
%:%3092=1550%:%
%:%3093=1551%:%
%:%3094=1551%:%
%:%3098=1555%:%
%:%3099=1556%:%
%:%3100=1556%:%
%:%3101=1556%:%
%:%3102=1557%:%
%:%3103=1557%:%
%:%3104=1558%:%
%:%3105=1558%:%
%:%3106=1559%:%
%:%3107=1559%:%
%:%3108=1559%:%
%:%3109=1560%:%
%:%3110=1560%:%
%:%3111=1560%:%
%:%3112=1561%:%
%:%3113=1561%:%
%:%3116=1564%:%
%:%3117=1565%:%
%:%3118=1565%:%
%:%3119=1566%:%
%:%3120=1566%:%
%:%3121=1567%:%
%:%3122=1567%:%
%:%3123=1568%:%
%:%3124=1568%:%
%:%3125=1568%:%
%:%3126=1569%:%
%:%3127=1569%:%
%:%3128=1570%:%
%:%3129=1570%:%
%:%3130=1571%:%
%:%3131=1571%:%
%:%3132=1572%:%
%:%3133=1572%:%
%:%3134=1572%:%
%:%3135=1573%:%
%:%3136=1573%:%
%:%3137=1574%:%
%:%3147=1576%:%
%:%3149=1578%:%
%:%3150=1578%:%
%:%3151=1579%:%
%:%3152=1580%:%
%:%3153=1581%:%
%:%3160=1582%:%
%:%3161=1582%:%
%:%3162=1583%:%
%:%3163=1583%:%
%:%3164=1584%:%
%:%3165=1584%:%
%:%3166=1585%:%
%:%3167=1585%:%
%:%3168=1585%:%
%:%3169=1586%:%
%:%3170=1586%:%
%:%3171=1587%:%
%:%3172=1587%:%
%:%3173=1587%:%
%:%3174=1588%:%
%:%3175=1588%:%
%:%3176=1589%:%
%:%3177=1589%:%
%:%3178=1589%:%
%:%3179=1590%:%
%:%3180=1590%:%
%:%3181=1591%:%
%:%3182=1591%:%
%:%3183=1591%:%
%:%3184=1592%:%
%:%3185=1592%:%
%:%3186=1593%:%
%:%3187=1593%:%
%:%3188=1594%:%
%:%3189=1594%:%

\chapter{Teorema de existencia de modelo}
%
\begin{isabellebody}%
\setisabellecontext{TeoremaEx}%
%
\isadelimtheory
%
\endisadelimtheory
%
\isatagtheory
%
\endisatagtheory
{\isafoldtheory}%
%
\isadelimtheory
%
\endisadelimtheory
%
\begin{isamarkuptext}%
\comentario{Añadir introducción.}%
\end{isamarkuptext}\isamarkuptrue%
%
\begin{isamarkuptext}%
\comentario{Cambiar referencias de los lemas tras el cambio de índice.}%
\end{isamarkuptext}\isamarkuptrue%
%
\isadelimdocument
%
\endisadelimdocument
%
\isatagdocument
%
\isamarkupsection{Sucesiones de conjuntos%
}
\isamarkuptrue%
%
\endisatagdocument
{\isafolddocument}%
%
\isadelimdocument
%
\endisadelimdocument
%
\begin{isamarkuptext}%
En este apartado daremos una introducción sobre sucesiones de conjuntos de fórmulas a 
  partir de una colección y un conjunto de la misma. De este modo, se mostrarán distintas 
  características sobre las sucesiones y se definirá su límite. En la siguiente sección 
  probaremos que dicho límite constituye un conjunto satisfacible por el lema de Hintikka.

\comentario{Revisar el párrafo anterior al final}

  Recordemos que el conjunto de las fórmulas proposicionales se define recursivamente a partir 
  de un alfabeto numerable de variables proposicionales. Por lo tanto, el conjunto de fórmulas 
  proposicionales es igualmente numerable, de modo que es posible enumerar sus elementos. Una vez 
  realizada esta observación, veamos la definición de sucesión de conjuntos de fórmulas 
  proposicionales a partir de una colección y un conjunto de la misma.

\begin{definicion}
  Sea \isa{C} una colección de conjuntos de fórmulas proposicionales, \isa{S\ {\isasymin}\ C} y \isa{F\isactrlsub {\isadigit{1}}{\isacharcomma}\ F\isactrlsub {\isadigit{2}}{\isacharcomma}\ F\isactrlsub {\isadigit{3}}\ {\isasymdots}} una 
  enumeración de las fórmulas proposicionales. Se define la \isa{sucesión\ de\ conjuntos\ de\ C\ a\ partir\ de\ S} como sigue:

  $S_{0} = S$

  $S_{n+1} = \left\{ \begin{array}{lcc} S_{n} \cup \{F_{n}\} &  si  & S_{n} \cup \{F_{n}\} \in C \\ \\ S_{n} & c.c \end{array} \right.$ 
\end{definicion}

  Para su formalización en Isabelle se ha introducido una instancia en la teoría de \isa{Sintaxis} que 
  indica explícitamente que el conjunto de las fórmulas proposicionales es numerable, probado
  mediante el método \isa{countable{\isacharunderscore}datatype} de Isabelle.

  \isa{instance\ formula\ {\isacharcolon}{\isacharcolon}\ {\isacharparenleft}countable{\isacharparenright}\ countable\ by\ countable{\isacharunderscore}datatype}

  De esta manera se genera en Isabelle una enumeración predeterminada de los elementos del conjunto,
  junto con herramientas para probar propiedades referentes a la numerabilidad. En particular, en la 
  formalización de la definición \isa{{\isadigit{4}}{\isachardot}{\isadigit{1}}{\isachardot}{\isadigit{1}}} se utilizará la función \isa{from{\isacharunderscore}nat} que, al aplicarla a un 
  número natural \isa{n}, nos devuelve la \isa{n}-ésima fórmula proposicional según una enumeración 
  predeterminada en Isabelle. 

  Puesto que la definición de las sucesiones en \isa{{\isadigit{4}}{\isachardot}{\isadigit{1}}{\isachardot}{\isadigit{1}}} se trata de una definición 
  recursiva sobre la estructura recursiva de los números naturales, se formalizará en Isabelle
  mediante el tipo de funciones primitivas recursivas de la siguiente manera.%
\end{isamarkuptext}\isamarkuptrue%
\isacommand{primrec}\isamarkupfalse%
\ pcp{\isacharunderscore}seq\ \isakeyword{where}\isanewline
{\isachardoublequoteopen}pcp{\isacharunderscore}seq\ C\ S\ {\isadigit{0}}\ {\isacharequal}\ S{\isachardoublequoteclose}\ {\isacharbar}\isanewline
{\isachardoublequoteopen}pcp{\isacharunderscore}seq\ C\ S\ {\isacharparenleft}Suc\ n{\isacharparenright}\ {\isacharequal}\ {\isacharparenleft}let\ Sn\ {\isacharequal}\ pcp{\isacharunderscore}seq\ C\ S\ n{\isacharsemicolon}\ Sn{\isadigit{1}}\ {\isacharequal}\ insert\ {\isacharparenleft}from{\isacharunderscore}nat\ n{\isacharparenright}\ Sn\ in\isanewline
\ \ \ \ \ \ \ \ \ \ \ \ \ \ \ \ \ \ \ \ \ \ \ \ if\ Sn{\isadigit{1}}\ {\isasymin}\ C\ then\ Sn{\isadigit{1}}\ else\ Sn{\isacharparenright}{\isachardoublequoteclose}%
\begin{isamarkuptext}%
Veamos el primer resultado sobre dichas sucesiones.

  \begin{lema}
    Sea \isa{C} una colección de conjuntos con la propiedad de consistencia proposicional,\\ \isa{S\ {\isasymin}\ C} y 
    \isa{{\isacharbraceleft}S\isactrlsub n{\isacharbraceright}} la sucesión de conjuntos de \isa{C} a partir de \isa{S} construida según la definición \isa{{\isadigit{4}}{\isachardot}{\isadigit{1}}{\isachardot}{\isadigit{1}}}. 
    Entonces, para todo \isa{n\ {\isasymin}\ {\isasymnat}} se verifica que \isa{S\isactrlsub n\ {\isasymin}\ C}.
  \end{lema}

  Procedamos con su demostración.

  \begin{demostracion}
    El resultado se prueba por inducción en los números naturales que conforman los subíndices de la 
    sucesión.

    En primer lugar, tenemos que \isa{S\isactrlsub {\isadigit{0}}\ {\isacharequal}\ S} pertenece a \isa{C} por hipótesis.

    Por otro lado, supongamos que \isa{S\isactrlsub n\ {\isasymin}\ C}. Probemos que \isa{S\isactrlsub n\isactrlsub {\isacharplus}\isactrlsub {\isadigit{1}}\ {\isasymin}\ C}. Si suponemos que \isa{S\isactrlsub n\ {\isasymunion}\ {\isacharbraceleft}F\isactrlsub n{\isacharbraceright}\ {\isasymin}\ C},
    por definición tenemos que \isa{S\isactrlsub n\isactrlsub {\isacharplus}\isactrlsub {\isadigit{1}}\ {\isacharequal}\ S\isactrlsub n\ {\isasymunion}\ {\isacharbraceleft}F\isactrlsub n{\isacharbraceright}}, luego pertenece a \isa{C}. En caso contrario, si
    suponemos que \isa{S\isactrlsub n\ {\isasymunion}\ {\isacharbraceleft}F\isactrlsub n{\isacharbraceright}\ {\isasymnotin}\ C}, por definición tenemos que \isa{S\isactrlsub n\isactrlsub {\isacharplus}\isactrlsub {\isadigit{1}}\ {\isacharequal}\ S\isactrlsub n}, que pertenece igualmente
    a \isa{C} por hipótesis de inducción. Por tanto, queda probado el resultado.
  \end{demostracion}

  La formalización y demostración detallada del lema en Isabelle son las siguientes.%
\end{isamarkuptext}\isamarkuptrue%
\isacommand{lemma}\isamarkupfalse%
\ \isanewline
\ \ \isakeyword{assumes}\ {\isachardoublequoteopen}pcp\ C{\isachardoublequoteclose}\ \isanewline
\ \ \ \ \ \ \ \ \ \ {\isachardoublequoteopen}S\ {\isasymin}\ C{\isachardoublequoteclose}\isanewline
\ \ \ \ \ \ \ \ \isakeyword{shows}\ {\isachardoublequoteopen}pcp{\isacharunderscore}seq\ C\ S\ n\ {\isasymin}\ C{\isachardoublequoteclose}\isanewline
%
\isadelimproof
%
\endisadelimproof
%
\isatagproof
\isacommand{proof}\isamarkupfalse%
\ {\isacharparenleft}induction\ n{\isacharparenright}\isanewline
\ \ \isacommand{show}\isamarkupfalse%
\ {\isachardoublequoteopen}pcp{\isacharunderscore}seq\ C\ S\ {\isadigit{0}}\ {\isasymin}\ C{\isachardoublequoteclose}\isanewline
\ \ \ \ \isacommand{by}\isamarkupfalse%
\ {\isacharparenleft}simp\ only{\isacharcolon}\ pcp{\isacharunderscore}seq{\isachardot}simps{\isacharparenleft}{\isadigit{1}}{\isacharparenright}\ {\isacartoucheopen}S\ {\isasymin}\ C{\isacartoucheclose}{\isacharparenright}\isanewline
\isacommand{next}\isamarkupfalse%
\isanewline
\ \ \isacommand{fix}\isamarkupfalse%
\ n\isanewline
\ \ \isacommand{assume}\isamarkupfalse%
\ HI{\isacharcolon}{\isachardoublequoteopen}pcp{\isacharunderscore}seq\ C\ S\ n\ {\isasymin}\ C{\isachardoublequoteclose}\isanewline
\ \ \isacommand{have}\isamarkupfalse%
\ {\isachardoublequoteopen}pcp{\isacharunderscore}seq\ C\ S\ {\isacharparenleft}Suc\ n{\isacharparenright}\ {\isacharequal}\ {\isacharparenleft}let\ Sn\ {\isacharequal}\ pcp{\isacharunderscore}seq\ C\ S\ n{\isacharsemicolon}\ Sn{\isadigit{1}}\ {\isacharequal}\ insert\ {\isacharparenleft}from{\isacharunderscore}nat\ n{\isacharparenright}\ Sn\ in\isanewline
\ \ \ \ \ \ \ \ \ \ \ \ \ \ \ \ \ \ \ \ \ \ \ \ if\ Sn{\isadigit{1}}\ {\isasymin}\ C\ then\ Sn{\isadigit{1}}\ else\ Sn{\isacharparenright}{\isachardoublequoteclose}\ \isanewline
\ \ \ \ \isacommand{by}\isamarkupfalse%
\ {\isacharparenleft}simp\ only{\isacharcolon}\ pcp{\isacharunderscore}seq{\isachardot}simps{\isacharparenleft}{\isadigit{2}}{\isacharparenright}{\isacharparenright}\isanewline
\ \ \isacommand{then}\isamarkupfalse%
\ \isacommand{have}\isamarkupfalse%
\ SucDef{\isacharcolon}{\isachardoublequoteopen}pcp{\isacharunderscore}seq\ C\ S\ {\isacharparenleft}Suc\ n{\isacharparenright}\ {\isacharequal}\ {\isacharparenleft}if\ insert\ {\isacharparenleft}from{\isacharunderscore}nat\ n{\isacharparenright}\ {\isacharparenleft}pcp{\isacharunderscore}seq\ C\ S\ n{\isacharparenright}\ {\isasymin}\ C\ then\ \isanewline
\ \ \ \ \ \ \ \ \ \ \ \ \ \ \ \ \ \ \ \ insert\ {\isacharparenleft}from{\isacharunderscore}nat\ n{\isacharparenright}\ {\isacharparenleft}pcp{\isacharunderscore}seq\ C\ S\ n{\isacharparenright}\ else\ pcp{\isacharunderscore}seq\ C\ S\ n{\isacharparenright}{\isachardoublequoteclose}\ \isanewline
\ \ \ \ \isacommand{by}\isamarkupfalse%
\ {\isacharparenleft}simp\ only{\isacharcolon}\ Let{\isacharunderscore}def{\isacharparenright}\isanewline
\ \ \isacommand{show}\isamarkupfalse%
\ {\isachardoublequoteopen}pcp{\isacharunderscore}seq\ C\ S\ {\isacharparenleft}Suc\ n{\isacharparenright}\ {\isasymin}\ C{\isachardoublequoteclose}\isanewline
\ \ \isacommand{proof}\isamarkupfalse%
\ {\isacharparenleft}cases{\isacharparenright}\isanewline
\ \ \ \ \isacommand{assume}\isamarkupfalse%
\ {\isadigit{1}}{\isacharcolon}{\isachardoublequoteopen}insert\ {\isacharparenleft}from{\isacharunderscore}nat\ n{\isacharparenright}\ {\isacharparenleft}pcp{\isacharunderscore}seq\ C\ S\ n{\isacharparenright}\ {\isasymin}\ C{\isachardoublequoteclose}\isanewline
\ \ \ \ \isacommand{have}\isamarkupfalse%
\ {\isachardoublequoteopen}pcp{\isacharunderscore}seq\ C\ S\ {\isacharparenleft}Suc\ n{\isacharparenright}\ {\isacharequal}\ insert\ {\isacharparenleft}from{\isacharunderscore}nat\ n{\isacharparenright}\ {\isacharparenleft}pcp{\isacharunderscore}seq\ C\ S\ n{\isacharparenright}{\isachardoublequoteclose}\isanewline
\ \ \ \ \ \ \isacommand{using}\isamarkupfalse%
\ SucDef\ {\isadigit{1}}\ \isacommand{by}\isamarkupfalse%
\ {\isacharparenleft}simp\ only{\isacharcolon}\ if{\isacharunderscore}True{\isacharparenright}\isanewline
\ \ \ \ \isacommand{thus}\isamarkupfalse%
\ {\isachardoublequoteopen}pcp{\isacharunderscore}seq\ C\ S\ {\isacharparenleft}Suc\ n{\isacharparenright}\ {\isasymin}\ C{\isachardoublequoteclose}\isanewline
\ \ \ \ \ \ \isacommand{by}\isamarkupfalse%
\ {\isacharparenleft}simp\ only{\isacharcolon}\ {\isadigit{1}}{\isacharparenright}\isanewline
\ \ \isacommand{next}\isamarkupfalse%
\isanewline
\ \ \ \ \isacommand{assume}\isamarkupfalse%
\ {\isadigit{2}}{\isacharcolon}{\isachardoublequoteopen}insert\ {\isacharparenleft}from{\isacharunderscore}nat\ n{\isacharparenright}\ {\isacharparenleft}pcp{\isacharunderscore}seq\ C\ S\ n{\isacharparenright}\ {\isasymnotin}\ C{\isachardoublequoteclose}\isanewline
\ \ \ \ \isacommand{have}\isamarkupfalse%
\ {\isachardoublequoteopen}pcp{\isacharunderscore}seq\ C\ S\ {\isacharparenleft}Suc\ n{\isacharparenright}\ {\isacharequal}\ pcp{\isacharunderscore}seq\ C\ S\ n{\isachardoublequoteclose}\isanewline
\ \ \ \ \ \ \isacommand{using}\isamarkupfalse%
\ SucDef\ {\isadigit{2}}\ \isacommand{by}\isamarkupfalse%
\ {\isacharparenleft}simp\ only{\isacharcolon}\ if{\isacharunderscore}False{\isacharparenright}\isanewline
\ \ \ \ \isacommand{thus}\isamarkupfalse%
\ {\isachardoublequoteopen}pcp{\isacharunderscore}seq\ C\ S\ {\isacharparenleft}Suc\ n{\isacharparenright}\ {\isasymin}\ C{\isachardoublequoteclose}\isanewline
\ \ \ \ \ \ \isacommand{by}\isamarkupfalse%
\ {\isacharparenleft}simp\ only{\isacharcolon}\ HI{\isacharparenright}\isanewline
\ \ \isacommand{qed}\isamarkupfalse%
\isanewline
\isacommand{qed}\isamarkupfalse%
%
\endisatagproof
{\isafoldproof}%
%
\isadelimproof
%
\endisadelimproof
%
\begin{isamarkuptext}%
Del mismo modo, podemos probar el lema de manera automática en Isabelle.%
\end{isamarkuptext}\isamarkuptrue%
\isacommand{lemma}\isamarkupfalse%
\ pcp{\isacharunderscore}seq{\isacharunderscore}in{\isacharcolon}\ {\isachardoublequoteopen}pcp\ C\ {\isasymLongrightarrow}\ S\ {\isasymin}\ C\ {\isasymLongrightarrow}\ pcp{\isacharunderscore}seq\ C\ S\ n\ {\isasymin}\ C{\isachardoublequoteclose}\isanewline
%
\isadelimproof
%
\endisadelimproof
%
\isatagproof
\isacommand{proof}\isamarkupfalse%
{\isacharparenleft}induction\ n{\isacharparenright}\isanewline
\ \ \isacommand{case}\isamarkupfalse%
\ {\isacharparenleft}Suc\ n{\isacharparenright}\ \ \isanewline
\ \ \isacommand{hence}\isamarkupfalse%
\ {\isachardoublequoteopen}pcp{\isacharunderscore}seq\ C\ S\ n\ {\isasymin}\ C{\isachardoublequoteclose}\ \isacommand{by}\isamarkupfalse%
\ simp\isanewline
\ \ \isacommand{thus}\isamarkupfalse%
\ {\isacharquery}case\ \isacommand{by}\isamarkupfalse%
\ {\isacharparenleft}simp\ add{\isacharcolon}\ Let{\isacharunderscore}def{\isacharparenright}\isanewline
\isacommand{qed}\isamarkupfalse%
\ simp%
\endisatagproof
{\isafoldproof}%
%
\isadelimproof
%
\endisadelimproof
%
\begin{isamarkuptext}%
Por otro lado, veamos la monotonía de dichas sucesiones.

  \begin{lema}
    Toda sucesión de conjuntos construida a partir de una colección y un conjunto según la
    definición \isa{{\isadigit{4}}{\isachardot}{\isadigit{1}}{\isachardot}{\isadigit{1}}} es monótona.
  \end{lema}

  En Isabelle, se formaliza de la siguiente forma.%
\end{isamarkuptext}\isamarkuptrue%
\isacommand{lemma}\isamarkupfalse%
\ {\isachardoublequoteopen}pcp{\isacharunderscore}seq\ C\ S\ n\ {\isasymsubseteq}\ pcp{\isacharunderscore}seq\ C\ S\ {\isacharparenleft}Suc\ n{\isacharparenright}{\isachardoublequoteclose}\isanewline
%
\isadelimproof
\ \ %
\endisadelimproof
%
\isatagproof
\isacommand{oops}\isamarkupfalse%
%
\endisatagproof
{\isafoldproof}%
%
\isadelimproof
%
\endisadelimproof
%
\begin{isamarkuptext}%
Procedamos con la demostración del lema.

  \begin{demostracion}
    Sea una colección de conjuntos \isa{C}, \isa{S\ {\isasymin}\ C} y \isa{{\isacharbraceleft}S\isactrlsub n{\isacharbraceright}} la sucesión de conjuntos de \isa{C} a partir de 
    \isa{S} según la definición \isa{{\isadigit{4}}{\isachardot}{\isadigit{1}}{\isachardot}{\isadigit{1}}}. Para probar que \isa{{\isacharbraceleft}S\isactrlsub n{\isacharbraceright}} es monótona, basta probar que \isa{S\isactrlsub n\ {\isasymsubseteq}\ S\isactrlsub n\isactrlsub {\isacharplus}\isactrlsub {\isadigit{1}}} 
    para todo \isa{n\ {\isasymin}\ {\isasymnat}}. En efecto, el resultado es inmediato al considerar dos casos para todo 
    \isa{n\ {\isasymin}\ {\isasymnat}}: \isa{S\isactrlsub n\ {\isasymunion}\ {\isacharbraceleft}F\isactrlsub n{\isacharbraceright}\ {\isasymin}\ C} o \isa{S\isactrlsub n\ {\isasymunion}\ {\isacharbraceleft}F\isactrlsub n{\isacharbraceright}\ {\isasymnotin}\ C}. Si suponemos que\\ \isa{S\isactrlsub n\ {\isasymunion}\ {\isacharbraceleft}F\isactrlsub n{\isacharbraceright}\ {\isasymin}\ C}, por definición 
    tenemos que \isa{S\isactrlsub n\isactrlsub {\isacharplus}\isactrlsub {\isadigit{1}}\ {\isacharequal}\ S\isactrlsub n\ {\isasymunion}\ {\isacharbraceleft}F\isactrlsub n{\isacharbraceright}}, luego es claro que\\ \isa{S\isactrlsub n\ {\isasymsubseteq}\ S\isactrlsub n\isactrlsub {\isacharplus}\isactrlsub {\isadigit{1}}}. En caso contrario, si 
    \isa{S\isactrlsub n\ {\isasymunion}\ {\isacharbraceleft}F\isactrlsub n{\isacharbraceright}\ {\isasymnotin}\ C}, por definición se tiene que \isa{S\isactrlsub n\isactrlsub {\isacharplus}\isactrlsub {\isadigit{1}}\ {\isacharequal}\ S\isactrlsub n}, obteniéndose igualmente el resultado
    por la propiedad reflexiva de la contención de conjuntos. 
  \end{demostracion}

  La prueba detallada en Isabelle se muestra a continuación.%
\end{isamarkuptext}\isamarkuptrue%
\isacommand{lemma}\isamarkupfalse%
\ {\isachardoublequoteopen}pcp{\isacharunderscore}seq\ C\ S\ n\ {\isasymsubseteq}\ pcp{\isacharunderscore}seq\ C\ S\ {\isacharparenleft}Suc\ n{\isacharparenright}{\isachardoublequoteclose}\isanewline
%
\isadelimproof
%
\endisadelimproof
%
\isatagproof
\isacommand{proof}\isamarkupfalse%
\ {\isacharminus}\isanewline
\ \ \isacommand{have}\isamarkupfalse%
\ {\isachardoublequoteopen}pcp{\isacharunderscore}seq\ C\ S\ {\isacharparenleft}Suc\ n{\isacharparenright}\ {\isacharequal}\ {\isacharparenleft}let\ Sn\ {\isacharequal}\ pcp{\isacharunderscore}seq\ C\ S\ n{\isacharsemicolon}\ Sn{\isadigit{1}}\ {\isacharequal}\ insert\ {\isacharparenleft}from{\isacharunderscore}nat\ n{\isacharparenright}\ Sn\ in\isanewline
\ \ \ \ \ \ \ \ \ \ \ \ \ \ \ \ \ \ \ \ \ \ \ \ if\ Sn{\isadigit{1}}\ {\isasymin}\ C\ then\ Sn{\isadigit{1}}\ else\ Sn{\isacharparenright}{\isachardoublequoteclose}\ \isanewline
\ \ \ \ \isacommand{by}\isamarkupfalse%
\ {\isacharparenleft}simp\ only{\isacharcolon}\ pcp{\isacharunderscore}seq{\isachardot}simps{\isacharparenleft}{\isadigit{2}}{\isacharparenright}{\isacharparenright}\isanewline
\ \ \isacommand{then}\isamarkupfalse%
\ \isacommand{have}\isamarkupfalse%
\ SucDef{\isacharcolon}{\isachardoublequoteopen}pcp{\isacharunderscore}seq\ C\ S\ {\isacharparenleft}Suc\ n{\isacharparenright}\ {\isacharequal}\ {\isacharparenleft}if\ insert\ {\isacharparenleft}from{\isacharunderscore}nat\ n{\isacharparenright}\ {\isacharparenleft}pcp{\isacharunderscore}seq\ C\ S\ n{\isacharparenright}\ {\isasymin}\ C\ then\ \isanewline
\ \ \ \ \ \ \ \ \ \ \ \ \ \ \ \ \ \ \ \ insert\ {\isacharparenleft}from{\isacharunderscore}nat\ n{\isacharparenright}\ {\isacharparenleft}pcp{\isacharunderscore}seq\ C\ S\ n{\isacharparenright}\ else\ pcp{\isacharunderscore}seq\ C\ S\ n{\isacharparenright}{\isachardoublequoteclose}\ \isanewline
\ \ \ \ \isacommand{by}\isamarkupfalse%
\ {\isacharparenleft}simp\ only{\isacharcolon}\ Let{\isacharunderscore}def{\isacharparenright}\isanewline
\ \ \isacommand{thus}\isamarkupfalse%
\ {\isachardoublequoteopen}pcp{\isacharunderscore}seq\ C\ S\ n\ {\isasymsubseteq}\ pcp{\isacharunderscore}seq\ C\ S\ {\isacharparenleft}Suc\ n{\isacharparenright}{\isachardoublequoteclose}\isanewline
\ \ \isacommand{proof}\isamarkupfalse%
\ {\isacharparenleft}cases{\isacharparenright}\isanewline
\ \ \ \ \isacommand{assume}\isamarkupfalse%
\ {\isadigit{1}}{\isacharcolon}{\isachardoublequoteopen}insert\ {\isacharparenleft}from{\isacharunderscore}nat\ n{\isacharparenright}\ {\isacharparenleft}pcp{\isacharunderscore}seq\ C\ S\ n{\isacharparenright}\ {\isasymin}\ C{\isachardoublequoteclose}\isanewline
\ \ \ \ \isacommand{have}\isamarkupfalse%
\ {\isachardoublequoteopen}pcp{\isacharunderscore}seq\ C\ S\ {\isacharparenleft}Suc\ n{\isacharparenright}\ {\isacharequal}\ insert\ {\isacharparenleft}from{\isacharunderscore}nat\ n{\isacharparenright}\ {\isacharparenleft}pcp{\isacharunderscore}seq\ C\ S\ n{\isacharparenright}{\isachardoublequoteclose}\isanewline
\ \ \ \ \ \ \isacommand{using}\isamarkupfalse%
\ SucDef\ {\isadigit{1}}\ \isacommand{by}\isamarkupfalse%
\ {\isacharparenleft}simp\ only{\isacharcolon}\ if{\isacharunderscore}True{\isacharparenright}\isanewline
\ \ \ \ \isacommand{thus}\isamarkupfalse%
\ {\isachardoublequoteopen}pcp{\isacharunderscore}seq\ C\ S\ n\ {\isasymsubseteq}\ pcp{\isacharunderscore}seq\ C\ S\ {\isacharparenleft}Suc\ n{\isacharparenright}{\isachardoublequoteclose}\isanewline
\ \ \ \ \ \ \isacommand{by}\isamarkupfalse%
\ {\isacharparenleft}simp\ only{\isacharcolon}\ subset{\isacharunderscore}insertI{\isacharparenright}\isanewline
\ \ \isacommand{next}\isamarkupfalse%
\isanewline
\ \ \ \ \isacommand{assume}\isamarkupfalse%
\ {\isadigit{2}}{\isacharcolon}{\isachardoublequoteopen}insert\ {\isacharparenleft}from{\isacharunderscore}nat\ n{\isacharparenright}\ {\isacharparenleft}pcp{\isacharunderscore}seq\ C\ S\ n{\isacharparenright}\ {\isasymnotin}\ C{\isachardoublequoteclose}\isanewline
\ \ \ \ \isacommand{have}\isamarkupfalse%
\ {\isachardoublequoteopen}pcp{\isacharunderscore}seq\ C\ S\ {\isacharparenleft}Suc\ n{\isacharparenright}\ {\isacharequal}\ pcp{\isacharunderscore}seq\ C\ S\ n{\isachardoublequoteclose}\isanewline
\ \ \ \ \ \ \isacommand{using}\isamarkupfalse%
\ SucDef\ {\isadigit{2}}\ \isacommand{by}\isamarkupfalse%
\ {\isacharparenleft}simp\ only{\isacharcolon}\ if{\isacharunderscore}False{\isacharparenright}\isanewline
\ \ \ \ \isacommand{thus}\isamarkupfalse%
\ {\isachardoublequoteopen}pcp{\isacharunderscore}seq\ C\ S\ n\ {\isasymsubseteq}\ pcp{\isacharunderscore}seq\ C\ S\ {\isacharparenleft}Suc\ n{\isacharparenright}{\isachardoublequoteclose}\isanewline
\ \ \ \ \ \ \isacommand{by}\isamarkupfalse%
\ {\isacharparenleft}simp\ only{\isacharcolon}\ subset{\isacharunderscore}refl{\isacharparenright}\isanewline
\ \ \isacommand{qed}\isamarkupfalse%
\isanewline
\isacommand{qed}\isamarkupfalse%
%
\endisatagproof
{\isafoldproof}%
%
\isadelimproof
%
\endisadelimproof
%
\begin{isamarkuptext}%
Del mismo modo, se puede probar automáticamente en Isabelle/HOL.%
\end{isamarkuptext}\isamarkuptrue%
\isacommand{lemma}\isamarkupfalse%
\ pcp{\isacharunderscore}seq{\isacharunderscore}monotonicity{\isacharcolon}{\isachardoublequoteopen}pcp{\isacharunderscore}seq\ C\ S\ n\ {\isasymsubseteq}\ pcp{\isacharunderscore}seq\ C\ S\ {\isacharparenleft}Suc\ n{\isacharparenright}{\isachardoublequoteclose}\isanewline
%
\isadelimproof
\ \ %
\endisadelimproof
%
\isatagproof
\isacommand{by}\isamarkupfalse%
\ {\isacharparenleft}smt\ eq{\isacharunderscore}iff\ pcp{\isacharunderscore}seq{\isachardot}simps{\isacharparenleft}{\isadigit{2}}{\isacharparenright}\ subset{\isacharunderscore}insertI{\isacharparenright}%
\endisatagproof
{\isafoldproof}%
%
\isadelimproof
%
\endisadelimproof
%
\begin{isamarkuptext}%
Por otra lado, para facilitar posteriores demostraciones en Isabelle/HOL, vamos a formalizar 
  el lema anterior empleando la siguiente definición generalizada de monotonía.%
\end{isamarkuptext}\isamarkuptrue%
\isacommand{lemma}\isamarkupfalse%
\ pcp{\isacharunderscore}seq{\isacharunderscore}mono{\isacharcolon}\isanewline
\ \ \isakeyword{assumes}\ {\isachardoublequoteopen}n\ {\isasymle}\ m{\isachardoublequoteclose}\ \isanewline
\ \ \isakeyword{shows}\ {\isachardoublequoteopen}pcp{\isacharunderscore}seq\ C\ S\ n\ {\isasymsubseteq}\ pcp{\isacharunderscore}seq\ C\ S\ m{\isachardoublequoteclose}\isanewline
%
\isadelimproof
\ \ %
\endisadelimproof
%
\isatagproof
\isacommand{using}\isamarkupfalse%
\ pcp{\isacharunderscore}seq{\isacharunderscore}monotonicity\ assms\ \isacommand{by}\isamarkupfalse%
\ {\isacharparenleft}rule\ lift{\isacharunderscore}Suc{\isacharunderscore}mono{\isacharunderscore}le{\isacharparenright}%
\endisatagproof
{\isafoldproof}%
%
\isadelimproof
%
\endisadelimproof
%
\begin{isamarkuptext}%
A continuación daremos un lema que permite caracterizar un elemento de la sucesión en función 
  de los anteriores.

\begin{lema}
  Sea \isa{C} una colección de conjuntos, \isa{S\ {\isasymin}\ C} y \isa{{\isacharbraceleft}S\isactrlsub n{\isacharbraceright}} la sucesión de conjuntos de \isa{C} a partir de 
  \isa{S} construida según la definición \isa{{\isadigit{4}}{\isachardot}{\isadigit{1}}{\isachardot}{\isadigit{1}}}. Entonces, para todos \isa{n}, \isa{m\ {\isasymin}\ {\isasymnat}} 
  se verifica $\bigcup_{n \leq m} S_{n} = S_{m}$
\end{lema}

\begin{demostracion}
  En las condiciones del enunciado, la prueba se realiza por inducción en \isa{m\ {\isasymin}\ {\isasymnat}}.

  En primer lugar, consideremos el caso base \isa{m\ {\isacharequal}\ {\isadigit{0}}}. El resultado se obtiene directamente:

  $\bigcup_{n \leq 0} S_{n} = \bigcup_{n = 0} S_{n} = S_{0} = S_{m}$

  Por otro lado, supongamos por hipótesis de inducción que $\bigcup_{n \leq m} S_{n} = S_{m}$.
  Veamos que se verifica $\bigcup_{n \leq m + 1} S_{n} = S_{m + 1}$. Observemos que si \isa{n\ {\isasymle}\ m\ {\isacharplus}\ {\isadigit{1}}},
  entonces se tiene que, o bien \isa{n\ {\isasymle}\ m}, o bien \isa{n\ {\isacharequal}\ m\ {\isacharplus}\ {\isadigit{1}}}. De este modo, aplicando la 
  hipótesis de inducción, deducimos lo siguiente.

  $\bigcup_{n \leq m + 1} S_{n} = \bigcup_{n \leq m} S_{n} \cup \bigcup_{n = m + 1} S_{n} = \bigcup_{n \leq m} S_{n} \cup S_{m + 1} = S_{m} \cup S_{m + 1}$

  Por la monotonía de la sucesión, se tiene que \isa{S\isactrlsub m\ {\isasymsubseteq}\ S\isactrlsub m\isactrlsub {\isacharplus}\isactrlsub {\isadigit{1}}}. Luego, se verifica:

  $\bigcup_{n \leq m + 1} S_{n} = S_{m} \cup S_{m + 1} = S_{m + 1}$

  Lo que prueba el resultado.
\end{demostracion}

  Procedamos a su formalización y demostración detallada. Para ello, emplearemos la unión 
  generalizada en Isabelle/HOL perteneciente a la teoría 
  \href{https://n9.cl/gtf5x}{Complete-Lattices.thy}, junto con distintas propiedades sobre la misma
  definidas en dicha teoría. El uso de teoría de retículos en este caso se debe a que, en Isabelle,
  los conjuntos se han formalizado como predicados según la teoría 
  \href{https://bit.ly/3ibCuje}{Set.thy}. De esta manera, un elemento pertenece a un conjunto si 
  verifica el predicado que lo caracteriza. Además, en dicha teoría se instancia que el tipo de los 
  conjuntos es un álgebra de \isa{Boole} acotada, es decir, es un retículo distributivo para las 
  operaciones unión e intersección que tiene un supremo y un ínfimo. En consecuencia, la unión 
  generalizada de conjuntos se formaliza en Isabelle como el supremo del retículo completo que 
  conforman.

  Veamos la prueba detallada del resultado en Isabelle/HOL.%
\end{isamarkuptext}\isamarkuptrue%
\isacommand{lemma}\isamarkupfalse%
\ {\isachardoublequoteopen}{\isasymUnion}{\isacharbraceleft}pcp{\isacharunderscore}seq\ C\ S\ n{\isacharbar}n{\isachardot}\ n\ {\isasymle}\ m{\isacharbraceright}\ {\isacharequal}\ pcp{\isacharunderscore}seq\ C\ S\ m{\isachardoublequoteclose}\isanewline
%
\isadelimproof
%
\endisadelimproof
%
\isatagproof
\isacommand{proof}\isamarkupfalse%
\ {\isacharparenleft}induct\ m{\isacharparenright}\isanewline
\ \ \isacommand{have}\isamarkupfalse%
\ \ {\isachardoublequoteopen}{\isasymUnion}{\isacharbraceleft}pcp{\isacharunderscore}seq\ C\ S\ n{\isacharbar}n{\isachardot}\ n\ {\isasymle}\ {\isadigit{0}}{\isacharbraceright}\ {\isacharequal}\ {\isasymUnion}{\isacharbraceleft}pcp{\isacharunderscore}seq\ C\ S\ n{\isacharbar}n{\isachardot}\ n\ {\isacharequal}\ {\isadigit{0}}{\isacharbraceright}{\isachardoublequoteclose}\isanewline
\ \ \ \ \isacommand{by}\isamarkupfalse%
\ {\isacharparenleft}simp\ only{\isacharcolon}\ le{\isacharunderscore}zero{\isacharunderscore}eq{\isacharparenright}\isanewline
\ \ \isacommand{also}\isamarkupfalse%
\ \isacommand{have}\isamarkupfalse%
\ {\isachardoublequoteopen}{\isasymdots}\ {\isacharequal}\ {\isasymUnion}{\isacharparenleft}{\isacharparenleft}pcp{\isacharunderscore}seq\ C\ S{\isacharparenright}{\isacharbackquote}{\isacharbraceleft}n{\isachardot}\ n\ {\isacharequal}\ {\isadigit{0}}{\isacharbraceright}{\isacharparenright}{\isachardoublequoteclose}\isanewline
\ \ \ \ \isacommand{by}\isamarkupfalse%
\ {\isacharparenleft}simp\ only{\isacharcolon}\ image{\isacharunderscore}Collect{\isacharparenright}\isanewline
\ \ \isacommand{also}\isamarkupfalse%
\ \isacommand{have}\isamarkupfalse%
\ {\isachardoublequoteopen}{\isasymdots}\ {\isacharequal}\ {\isasymUnion}{\isacharbraceleft}pcp{\isacharunderscore}seq\ C\ S\ {\isadigit{0}}{\isacharbraceright}{\isachardoublequoteclose}\isanewline
\ \ \ \ \isacommand{by}\isamarkupfalse%
\ {\isacharparenleft}simp\ only{\isacharcolon}\ singleton{\isacharunderscore}conv\ image{\isacharunderscore}insert\ image{\isacharunderscore}empty{\isacharparenright}\isanewline
\ \ \isacommand{also}\isamarkupfalse%
\ \isacommand{have}\isamarkupfalse%
\ {\isachardoublequoteopen}{\isasymdots}\ {\isacharequal}\ pcp{\isacharunderscore}seq\ C\ S\ {\isadigit{0}}{\isachardoublequoteclose}\ \isanewline
\ \ \ \ \isacommand{by}\isamarkupfalse%
\ \ {\isacharparenleft}simp\ only{\isacharcolon}cSup{\isacharunderscore}singleton{\isacharparenright}\isanewline
\ \ \isacommand{finally}\isamarkupfalse%
\ \isacommand{show}\isamarkupfalse%
\ {\isachardoublequoteopen}{\isasymUnion}{\isacharbraceleft}pcp{\isacharunderscore}seq\ C\ S\ n{\isacharbar}n{\isachardot}\ n\ {\isasymle}\ {\isadigit{0}}{\isacharbraceright}\ {\isacharequal}\ pcp{\isacharunderscore}seq\ C\ S\ {\isadigit{0}}{\isachardoublequoteclose}\ \isanewline
\ \ \ \ \isacommand{by}\isamarkupfalse%
\ this\isanewline
\isacommand{next}\isamarkupfalse%
\isanewline
\ \ \isacommand{fix}\isamarkupfalse%
\ m\isanewline
\ \ \isacommand{assume}\isamarkupfalse%
\ HI{\isacharcolon}{\isachardoublequoteopen}{\isasymUnion}{\isacharbraceleft}pcp{\isacharunderscore}seq\ C\ S\ n{\isacharbar}n{\isachardot}\ n\ {\isasymle}\ m{\isacharbraceright}\ {\isacharequal}\ pcp{\isacharunderscore}seq\ C\ S\ m{\isachardoublequoteclose}\isanewline
\ \ \isacommand{have}\isamarkupfalse%
\ {\isachardoublequoteopen}m\ {\isasymle}\ Suc\ m{\isachardoublequoteclose}\ \isanewline
\ \ \ \ \isacommand{by}\isamarkupfalse%
\ {\isacharparenleft}simp\ only{\isacharcolon}\ add{\isacharunderscore}{\isadigit{0}}{\isacharunderscore}right{\isacharparenright}\isanewline
\ \ \isacommand{then}\isamarkupfalse%
\ \isacommand{have}\isamarkupfalse%
\ Mon{\isacharcolon}{\isachardoublequoteopen}pcp{\isacharunderscore}seq\ C\ S\ m\ {\isasymsubseteq}\ pcp{\isacharunderscore}seq\ C\ S\ {\isacharparenleft}Suc\ m{\isacharparenright}{\isachardoublequoteclose}\isanewline
\ \ \ \ \isacommand{by}\isamarkupfalse%
\ {\isacharparenleft}rule\ pcp{\isacharunderscore}seq{\isacharunderscore}mono{\isacharparenright}\isanewline
\ \ \isacommand{have}\isamarkupfalse%
\ {\isachardoublequoteopen}{\isasymUnion}{\isacharbraceleft}pcp{\isacharunderscore}seq\ C\ S\ n\ {\isacharbar}\ n{\isachardot}\ n\ {\isasymle}\ Suc\ m{\isacharbraceright}\ {\isacharequal}\ {\isasymUnion}{\isacharparenleft}{\isacharparenleft}pcp{\isacharunderscore}seq\ C\ S{\isacharparenright}{\isacharbackquote}{\isacharparenleft}{\isacharbraceleft}n{\isachardot}\ n\ {\isasymle}\ Suc\ m{\isacharbraceright}{\isacharparenright}{\isacharparenright}{\isachardoublequoteclose}\isanewline
\ \ \ \ \isacommand{by}\isamarkupfalse%
\ {\isacharparenleft}simp\ only{\isacharcolon}\ image{\isacharunderscore}Collect{\isacharparenright}\isanewline
\ \ \isacommand{also}\isamarkupfalse%
\ \isacommand{have}\isamarkupfalse%
\ {\isachardoublequoteopen}{\isasymdots}\ {\isacharequal}\ {\isasymUnion}{\isacharparenleft}{\isacharparenleft}pcp{\isacharunderscore}seq\ C\ S{\isacharparenright}{\isacharbackquote}{\isacharparenleft}{\isacharbraceleft}Suc\ m{\isacharbraceright}\ {\isasymunion}\ {\isacharbraceleft}n{\isachardot}\ n\ {\isasymle}\ m{\isacharbraceright}{\isacharparenright}{\isacharparenright}{\isachardoublequoteclose}\isanewline
\ \ \ \ \isacommand{by}\isamarkupfalse%
\ {\isacharparenleft}simp\ only{\isacharcolon}\ le{\isacharunderscore}Suc{\isacharunderscore}eq\ Collect{\isacharunderscore}disj{\isacharunderscore}eq\ Un{\isacharunderscore}commute\ singleton{\isacharunderscore}conv{\isacharparenright}\isanewline
\ \ \isacommand{also}\isamarkupfalse%
\ \isacommand{have}\isamarkupfalse%
\ {\isachardoublequoteopen}{\isasymdots}\ {\isacharequal}\ {\isasymUnion}{\isacharparenleft}{\isacharbraceleft}pcp{\isacharunderscore}seq\ C\ S\ {\isacharparenleft}Suc\ m{\isacharparenright}{\isacharbraceright}\ {\isasymunion}\ {\isacharbraceleft}pcp{\isacharunderscore}seq\ C\ S\ n\ {\isacharbar}\ n{\isachardot}\ n\ {\isasymle}\ m{\isacharbraceright}{\isacharparenright}{\isachardoublequoteclose}\isanewline
\ \ \ \ \isacommand{by}\isamarkupfalse%
\ {\isacharparenleft}simp\ only{\isacharcolon}\ image{\isacharunderscore}Un\ image{\isacharunderscore}insert\ image{\isacharunderscore}empty\ image{\isacharunderscore}Collect{\isacharparenright}\isanewline
\ \ \isacommand{also}\isamarkupfalse%
\ \isacommand{have}\isamarkupfalse%
\ {\isachardoublequoteopen}{\isasymdots}\ {\isacharequal}\ {\isasymUnion}{\isacharbraceleft}pcp{\isacharunderscore}seq\ C\ S\ {\isacharparenleft}Suc\ m{\isacharparenright}{\isacharbraceright}\ {\isasymunion}\ {\isasymUnion}{\isacharbraceleft}pcp{\isacharunderscore}seq\ C\ S\ n\ {\isacharbar}\ n{\isachardot}\ n\ {\isasymle}\ m{\isacharbraceright}{\isachardoublequoteclose}\isanewline
\ \ \ \ \isacommand{by}\isamarkupfalse%
\ {\isacharparenleft}simp\ only{\isacharcolon}\ Union{\isacharunderscore}Un{\isacharunderscore}distrib{\isacharparenright}\isanewline
\ \ \isacommand{also}\isamarkupfalse%
\ \isacommand{have}\isamarkupfalse%
\ {\isachardoublequoteopen}{\isasymdots}\ {\isacharequal}\ {\isacharparenleft}pcp{\isacharunderscore}seq\ C\ S\ {\isacharparenleft}Suc\ m{\isacharparenright}{\isacharparenright}\ {\isasymunion}\ {\isasymUnion}{\isacharbraceleft}pcp{\isacharunderscore}seq\ C\ S\ n\ {\isacharbar}\ n{\isachardot}\ n\ {\isasymle}\ m{\isacharbraceright}{\isachardoublequoteclose}\isanewline
\ \ \ \ \isacommand{by}\isamarkupfalse%
\ {\isacharparenleft}simp\ only{\isacharcolon}\ cSup{\isacharunderscore}singleton{\isacharparenright}\isanewline
\ \ \isacommand{also}\isamarkupfalse%
\ \isacommand{have}\isamarkupfalse%
\ {\isachardoublequoteopen}{\isasymdots}\ {\isacharequal}\ {\isacharparenleft}pcp{\isacharunderscore}seq\ C\ S\ {\isacharparenleft}Suc\ m{\isacharparenright}{\isacharparenright}\ {\isasymunion}\ {\isacharparenleft}pcp{\isacharunderscore}seq\ C\ S\ m{\isacharparenright}{\isachardoublequoteclose}\isanewline
\ \ \ \ \isacommand{by}\isamarkupfalse%
\ {\isacharparenleft}simp\ only{\isacharcolon}\ HI{\isacharparenright}\isanewline
\ \ \isacommand{also}\isamarkupfalse%
\ \isacommand{have}\isamarkupfalse%
\ {\isachardoublequoteopen}{\isasymdots}\ {\isacharequal}\ pcp{\isacharunderscore}seq\ C\ S\ {\isacharparenleft}Suc\ m{\isacharparenright}{\isachardoublequoteclose}\isanewline
\ \ \ \ \isacommand{using}\isamarkupfalse%
\ Mon\ \isacommand{by}\isamarkupfalse%
\ {\isacharparenleft}simp\ only{\isacharcolon}\ Un{\isacharunderscore}absorb{\isadigit{2}}{\isacharparenright}\isanewline
\ \ \isacommand{finally}\isamarkupfalse%
\ \isacommand{show}\isamarkupfalse%
\ {\isachardoublequoteopen}{\isasymUnion}{\isacharbraceleft}pcp{\isacharunderscore}seq\ C\ S\ n{\isacharbar}n{\isachardot}\ n\ {\isasymle}\ {\isacharparenleft}Suc\ m{\isacharparenright}{\isacharbraceright}\ {\isacharequal}\ pcp{\isacharunderscore}seq\ C\ S\ {\isacharparenleft}Suc\ m{\isacharparenright}{\isachardoublequoteclose}\isanewline
\ \ \ \ \isacommand{by}\isamarkupfalse%
\ this\isanewline
\isacommand{qed}\isamarkupfalse%
%
\endisatagproof
{\isafoldproof}%
%
\isadelimproof
%
\endisadelimproof
%
\begin{isamarkuptext}%
Análogamente, podemos dar una prueba automática.%
\end{isamarkuptext}\isamarkuptrue%
\isacommand{lemma}\isamarkupfalse%
\ pcp{\isacharunderscore}seq{\isacharunderscore}UN{\isacharcolon}\ {\isachardoublequoteopen}{\isasymUnion}{\isacharbraceleft}pcp{\isacharunderscore}seq\ C\ S\ n{\isacharbar}n{\isachardot}\ n\ {\isasymle}\ m{\isacharbraceright}\ {\isacharequal}\ pcp{\isacharunderscore}seq\ C\ S\ m{\isachardoublequoteclose}\isanewline
%
\isadelimproof
%
\endisadelimproof
%
\isatagproof
\isacommand{proof}\isamarkupfalse%
{\isacharparenleft}induction\ m{\isacharparenright}\isanewline
\ \ \isacommand{case}\isamarkupfalse%
\ {\isacharparenleft}Suc\ m{\isacharparenright}\isanewline
\ \ \isacommand{have}\isamarkupfalse%
\ {\isachardoublequoteopen}{\isacharbraceleft}f\ n\ {\isacharbar}n{\isachardot}\ n\ {\isasymle}\ Suc\ m{\isacharbraceright}\ {\isacharequal}\ insert\ {\isacharparenleft}f\ {\isacharparenleft}Suc\ m{\isacharparenright}{\isacharparenright}\ {\isacharbraceleft}f\ n\ {\isacharbar}n{\isachardot}\ n\ {\isasymle}\ m{\isacharbraceright}{\isachardoublequoteclose}\ \isanewline
\ \ \ \ \isakeyword{for}\ f\ \isacommand{using}\isamarkupfalse%
\ le{\isacharunderscore}Suc{\isacharunderscore}eq\ \isacommand{by}\isamarkupfalse%
\ auto\isanewline
\ \ \isacommand{hence}\isamarkupfalse%
\ {\isachardoublequoteopen}{\isacharbraceleft}pcp{\isacharunderscore}seq\ C\ S\ n\ {\isacharbar}n{\isachardot}\ n\ {\isasymle}\ Suc\ m{\isacharbraceright}\ {\isacharequal}\ \isanewline
\ \ \ \ \ \ \ \ \ \ insert\ {\isacharparenleft}pcp{\isacharunderscore}seq\ C\ S\ {\isacharparenleft}Suc\ m{\isacharparenright}{\isacharparenright}\ {\isacharbraceleft}pcp{\isacharunderscore}seq\ C\ S\ n\ {\isacharbar}n{\isachardot}\ n\ {\isasymle}\ m{\isacharbraceright}{\isachardoublequoteclose}\ \isacommand{{\isachardot}}\isamarkupfalse%
\isanewline
\ \ \isacommand{hence}\isamarkupfalse%
\ {\isachardoublequoteopen}{\isasymUnion}{\isacharbraceleft}pcp{\isacharunderscore}seq\ C\ S\ n\ {\isacharbar}n{\isachardot}\ n\ {\isasymle}\ Suc\ m{\isacharbraceright}\ {\isacharequal}\ \isanewline
\ \ \ \ \ \ \ \ \ {\isasymUnion}{\isacharbraceleft}pcp{\isacharunderscore}seq\ C\ S\ n\ {\isacharbar}n{\isachardot}\ n\ {\isasymle}\ m{\isacharbraceright}\ {\isasymunion}\ pcp{\isacharunderscore}seq\ C\ S\ {\isacharparenleft}Suc\ m{\isacharparenright}{\isachardoublequoteclose}\ \isacommand{by}\isamarkupfalse%
\ auto\isanewline
\ \ \isacommand{thus}\isamarkupfalse%
\ {\isacharquery}case\ \isacommand{using}\isamarkupfalse%
\ Suc\ pcp{\isacharunderscore}seq{\isacharunderscore}mono\ \isacommand{by}\isamarkupfalse%
\ blast\isanewline
\isacommand{qed}\isamarkupfalse%
\ simp%
\endisatagproof
{\isafoldproof}%
%
\isadelimproof
%
\endisadelimproof
%
\begin{isamarkuptext}%
Finalmente, definamos el límite de las sucesiones presentadas en la definición \isa{{\isadigit{4}}{\isachardot}{\isadigit{1}}{\isachardot}{\isadigit{1}}}.

 \begin{definicion}
  Sea \isa{C} una colección, \isa{S\ {\isasymin}\ C} y \isa{{\isacharbraceleft}S\isactrlsub n{\isacharbraceright}} la sucesión de conjuntos de \isa{C} a partir de \isa{S} según la
  definición \isa{{\isadigit{4}}{\isachardot}{\isadigit{1}}{\isachardot}{\isadigit{1}}}. Se define el \isa{límite\ de\ la\ sucesión\ de\ conjuntos\ de\ C\ a\ partir\ de\ S} como 
  $\bigcup_{n = 0}^{\infty} S_{n}$
 \end{definicion}

  La definición del límite se formaliza utilizando la unión generalizada de Isabelle como sigue.%
\end{isamarkuptext}\isamarkuptrue%
\isacommand{definition}\isamarkupfalse%
\ {\isachardoublequoteopen}pcp{\isacharunderscore}lim\ C\ S\ {\isasymequiv}\ {\isasymUnion}{\isacharbraceleft}pcp{\isacharunderscore}seq\ C\ S\ n{\isacharbar}n{\isachardot}\ True{\isacharbraceright}{\isachardoublequoteclose}%
\begin{isamarkuptext}%
Veamos el primer resultado sobre el límite.

\begin{lema}
  Sea \isa{C} una colección de conjuntos, \isa{S\ {\isasymin}\ C} y \isa{{\isacharbraceleft}S\isactrlsub n{\isacharbraceright}} la sucesión de conjuntos de \isa{C} a partir de
  \isa{S} según la definición \isa{{\isadigit{4}}{\isachardot}{\isadigit{1}}{\isachardot}{\isadigit{1}}}. Entonces, para todo \isa{n\ {\isasymin}\ {\isasymnat}}, se verifica:

  $S_{n} \subseteq \bigcup_{n = 0}^{\infty} S_{n}$
\end{lema}

\begin{demostracion}
  El resultado se obtiene de manera inmediata ya que, para todo \isa{n\ {\isasymin}\ {\isasymnat}}, se verifica que 
  $S_{n} \in \{S_{n}\}_{n = 0}^{\infty}$. Por tanto, es claro que 
  $S_{n} \subseteq \bigcup_{n = 0}^{\infty} S_{n}$.
\end{demostracion}

  Su formalización y demostración detallada en Isabelle se muestran a continuación.%
\end{isamarkuptext}\isamarkuptrue%
\isacommand{lemma}\isamarkupfalse%
\ {\isachardoublequoteopen}pcp{\isacharunderscore}seq\ C\ S\ n\ {\isasymsubseteq}\ pcp{\isacharunderscore}lim\ C\ S{\isachardoublequoteclose}\isanewline
%
\isadelimproof
\ \ %
\endisadelimproof
%
\isatagproof
\isacommand{unfolding}\isamarkupfalse%
\ pcp{\isacharunderscore}lim{\isacharunderscore}def\isanewline
\isacommand{proof}\isamarkupfalse%
\ {\isacharminus}\isanewline
\ \ \isacommand{have}\isamarkupfalse%
\ {\isachardoublequoteopen}n\ {\isasymin}\ {\isacharbraceleft}n\ {\isacharbar}\ n{\isachardot}\ True{\isacharbraceright}{\isachardoublequoteclose}\ \isanewline
\ \ \ \ \isacommand{by}\isamarkupfalse%
\ {\isacharparenleft}simp\ only{\isacharcolon}\ simp{\isacharunderscore}thms{\isacharparenleft}{\isadigit{2}}{\isadigit{1}}{\isacharcomma}{\isadigit{3}}{\isadigit{8}}{\isacharparenright}\ Collect{\isacharunderscore}const\ if{\isacharunderscore}True\ UNIV{\isacharunderscore}I{\isacharparenright}\ \isanewline
\ \ \isacommand{then}\isamarkupfalse%
\ \isacommand{have}\isamarkupfalse%
\ {\isachardoublequoteopen}pcp{\isacharunderscore}seq\ C\ S\ n\ {\isasymin}\ {\isacharparenleft}pcp{\isacharunderscore}seq\ C\ S{\isacharparenright}{\isacharbackquote}{\isacharbraceleft}n\ {\isacharbar}\ n{\isachardot}\ True{\isacharbraceright}{\isachardoublequoteclose}\isanewline
\ \ \ \ \isacommand{by}\isamarkupfalse%
\ {\isacharparenleft}simp\ only{\isacharcolon}\ imageI{\isacharparenright}\isanewline
\ \ \isacommand{then}\isamarkupfalse%
\ \isacommand{have}\isamarkupfalse%
\ {\isachardoublequoteopen}pcp{\isacharunderscore}seq\ C\ S\ n\ {\isasymin}\ {\isacharbraceleft}pcp{\isacharunderscore}seq\ C\ S\ n\ {\isacharbar}\ n{\isachardot}\ True{\isacharbraceright}{\isachardoublequoteclose}\isanewline
\ \ \ \ \isacommand{by}\isamarkupfalse%
\ {\isacharparenleft}simp\ only{\isacharcolon}\ image{\isacharunderscore}Collect\ simp{\isacharunderscore}thms{\isacharparenleft}{\isadigit{4}}{\isadigit{0}}{\isacharparenright}{\isacharparenright}\isanewline
\ \ \isacommand{thus}\isamarkupfalse%
\ {\isachardoublequoteopen}pcp{\isacharunderscore}seq\ C\ S\ n\ {\isasymsubseteq}\ {\isasymUnion}{\isacharbraceleft}pcp{\isacharunderscore}seq\ C\ S\ n\ {\isacharbar}\ n{\isachardot}\ True{\isacharbraceright}{\isachardoublequoteclose}\isanewline
\ \ \ \ \isacommand{by}\isamarkupfalse%
\ {\isacharparenleft}simp\ only{\isacharcolon}\ Union{\isacharunderscore}upper{\isacharparenright}\isanewline
\isacommand{qed}\isamarkupfalse%
%
\endisatagproof
{\isafoldproof}%
%
\isadelimproof
%
\endisadelimproof
%
\begin{isamarkuptext}%
Podemos probarlo de manera automática como sigue.%
\end{isamarkuptext}\isamarkuptrue%
\isacommand{lemma}\isamarkupfalse%
\ pcp{\isacharunderscore}seq{\isacharunderscore}sub{\isacharcolon}\ {\isachardoublequoteopen}pcp{\isacharunderscore}seq\ C\ S\ n\ {\isasymsubseteq}\ pcp{\isacharunderscore}lim\ C\ S{\isachardoublequoteclose}\ \isanewline
%
\isadelimproof
\ \ %
\endisadelimproof
%
\isatagproof
\isacommand{unfolding}\isamarkupfalse%
\ pcp{\isacharunderscore}lim{\isacharunderscore}def\ \isacommand{by}\isamarkupfalse%
\ blast%
\endisatagproof
{\isafoldproof}%
%
\isadelimproof
%
\endisadelimproof
%
\begin{isamarkuptext}%
Mostremos otro resultado. 

  \begin{lema}
    Sea \isa{C} una colección de conjuntos de fórmulas proposicionales, \isa{S\ {\isasymin}\ C} y \isa{{\isacharbraceleft}S\isactrlsub n{\isacharbraceright}} la sucesión de 
    conjuntos de \isa{C} a partir de \isa{S} según la definición \isa{{\isadigit{4}}{\isachardot}{\isadigit{1}}{\isachardot}{\isadigit{1}}}. Si \isa{F} es una fórmula tal que
    $F \in \bigcup_{n = 0}^{\infty} S_{n}$, entonces existe un \isa{k\ {\isasymin}\ {\isasymnat}} tal que \isa{F\ {\isasymin}\ S\isactrlsub k}. 
  \end{lema}

  \begin{demostracion}
    La prueba es inmediata de la definición del límite de la sucesión de conjuntos \isa{{\isacharbraceleft}S\isactrlsub n{\isacharbraceright}}: si
    \isa{F} pertenece a la unión generalizada $\bigcup_{n = 0}^{\infty} S_{n}$, entonces existe algún
    conjunto \isa{S\isactrlsub k} tal que \isa{F\ {\isasymin}\ S\isactrlsub k}. Es decir, existe \isa{k\ {\isasymin}\ {\isasymnat}} tal que \isa{F\ {\isasymin}\ S\isactrlsub k}, como queríamos
    demostrar.
  \end{demostracion} 

  Su prueba detallada en Isabelle/HOL es la siguiente.%
\end{isamarkuptext}\isamarkuptrue%
\isacommand{lemma}\isamarkupfalse%
\ \isanewline
\ \ \isakeyword{assumes}\ {\isachardoublequoteopen}F\ {\isasymin}\ pcp{\isacharunderscore}lim\ C\ S{\isachardoublequoteclose}\isanewline
\ \ \isakeyword{shows}\ {\isachardoublequoteopen}{\isasymexists}k{\isachardot}\ F\ {\isasymin}\ pcp{\isacharunderscore}seq\ C\ S\ k{\isachardoublequoteclose}\ \isanewline
%
\isadelimproof
%
\endisadelimproof
%
\isatagproof
\isacommand{proof}\isamarkupfalse%
\ {\isacharminus}\isanewline
\ \ \isacommand{have}\isamarkupfalse%
\ {\isachardoublequoteopen}F\ {\isasymin}\ {\isasymUnion}{\isacharparenleft}{\isacharparenleft}pcp{\isacharunderscore}seq\ C\ S{\isacharparenright}\ {\isacharbackquote}\ {\isacharbraceleft}n\ {\isacharbar}\ n{\isachardot}\ True{\isacharbraceright}{\isacharparenright}{\isachardoublequoteclose}\isanewline
\ \ \ \ \isacommand{using}\isamarkupfalse%
\ assms\ \isacommand{by}\isamarkupfalse%
\ {\isacharparenleft}simp\ only{\isacharcolon}\ pcp{\isacharunderscore}lim{\isacharunderscore}def\ image{\isacharunderscore}Collect\ simp{\isacharunderscore}thms{\isacharparenleft}{\isadigit{4}}{\isadigit{0}}{\isacharparenright}{\isacharparenright}\isanewline
\ \ \isacommand{then}\isamarkupfalse%
\ \isacommand{have}\isamarkupfalse%
\ {\isachardoublequoteopen}{\isasymexists}k\ {\isasymin}\ {\isacharbraceleft}n{\isachardot}\ True{\isacharbraceright}{\isachardot}\ F\ {\isasymin}\ pcp{\isacharunderscore}seq\ C\ S\ k{\isachardoublequoteclose}\isanewline
\ \ \ \ \isacommand{by}\isamarkupfalse%
\ {\isacharparenleft}simp\ only{\isacharcolon}\ UN{\isacharunderscore}iff\ simp{\isacharunderscore}thms{\isacharparenleft}{\isadigit{4}}{\isadigit{0}}{\isacharparenright}{\isacharparenright}\isanewline
\ \ \isacommand{then}\isamarkupfalse%
\ \isacommand{have}\isamarkupfalse%
\ {\isachardoublequoteopen}{\isasymexists}k\ {\isasymin}\ UNIV{\isachardot}\ F\ {\isasymin}\ pcp{\isacharunderscore}seq\ C\ S\ k{\isachardoublequoteclose}\ \isanewline
\ \ \ \ \isacommand{by}\isamarkupfalse%
\ {\isacharparenleft}simp\ only{\isacharcolon}\ UNIV{\isacharunderscore}def{\isacharparenright}\isanewline
\ \ \isacommand{thus}\isamarkupfalse%
\ {\isachardoublequoteopen}{\isasymexists}k{\isachardot}\ F\ {\isasymin}\ pcp{\isacharunderscore}seq\ C\ S\ k{\isachardoublequoteclose}\ \isanewline
\ \ \ \ \isacommand{by}\isamarkupfalse%
\ {\isacharparenleft}simp\ only{\isacharcolon}\ bex{\isacharunderscore}UNIV{\isacharparenright}\isanewline
\isacommand{qed}\isamarkupfalse%
%
\endisatagproof
{\isafoldproof}%
%
\isadelimproof
%
\endisadelimproof
%
\begin{isamarkuptext}%
Mostremos, a continuación, la demostración automática del resultado.%
\end{isamarkuptext}\isamarkuptrue%
\isacommand{lemma}\isamarkupfalse%
\ pcp{\isacharunderscore}lim{\isacharunderscore}inserted{\isacharunderscore}at{\isacharunderscore}ex{\isacharcolon}\ \isanewline
\ \ \ \ {\isachardoublequoteopen}S{\isacharprime}\ {\isasymin}\ pcp{\isacharunderscore}lim\ C\ S\ {\isasymLongrightarrow}\ {\isasymexists}k{\isachardot}\ S{\isacharprime}\ {\isasymin}\ pcp{\isacharunderscore}seq\ C\ S\ k{\isachardoublequoteclose}\isanewline
%
\isadelimproof
\ \ %
\endisadelimproof
%
\isatagproof
\isacommand{unfolding}\isamarkupfalse%
\ pcp{\isacharunderscore}lim{\isacharunderscore}def\ \isacommand{by}\isamarkupfalse%
\ blast%
\endisatagproof
{\isafoldproof}%
%
\isadelimproof
%
\endisadelimproof
%
\begin{isamarkuptext}%
Por último, veamos la siguiente propiedad sobre conjuntos finitos contenidos en el límite de 
  las sucesiones definido en \isa{{\isadigit{4}}{\isachardot}{\isadigit{1}}{\isachardot}{\isadigit{5}}}.

\begin{lema}
  Sea \isa{C} una colección, \isa{S\ {\isasymin}\ C} y \isa{{\isacharbraceleft}S\isactrlsub n{\isacharbraceright}} la sucesión de conjuntos de \isa{C} a partir de \isa{S} según la
  definición \isa{{\isadigit{4}}{\isachardot}{\isadigit{1}}{\isachardot}{\isadigit{1}}}. Si \isa{S{\isacharprime}} es un conjunto finito tal que \isa{S{\isacharprime}\ {\isasymsubseteq}} $\bigcup_{n = 0}^{\infty} S_{n}$, 
  entonces existe un\\ \isa{k\ {\isasymin}\ {\isasymnat}} tal que \isa{S{\isacharprime}\ {\isasymsubseteq}\ S\isactrlsub k}.
\end{lema}

\begin{demostracion}
  La prueba del resultado se realiza por inducción sobre la estructura recursiva de los conjuntos 
  finitos.

  En primer lugar, probemos el caso base correspondiente al conjunto vacío. Supongamos que \isa{{\isacharbraceleft}{\isacharbraceright}} está 
  contenido en el límite de la sucesión de conjuntos de \isa{C} a partir de \isa{S}. Como \isa{{\isacharbraceleft}{\isacharbraceright}} es 
  subconjunto de todo conjunto, en particular lo es de \isa{S\ {\isacharequal}\ S\isactrlsub {\isadigit{0}}}, probando así el primer caso.

  Por otra parte, probemos el paso de inducción. Sea \isa{S{\isacharprime}} un conjunto finito verificando la 
  hipótesis de inducción: si \isa{S{\isacharprime}} está contenido en el límite de la sucesión de conjuntos de 
  \isa{C} a partir de \isa{S}, entonces también está contenido en algún \isa{S\isactrlsub k\isactrlsub {\isacharprime}} para cierto \isa{k{\isacharprime}\ {\isasymin}\ {\isasymnat}}. Sea 
  \isa{F} una fórmula tal que \isa{F\ {\isasymnotin}\ S{\isacharprime}}. Vamos a probar que si \isa{{\isacharbraceleft}F{\isacharbraceright}\ {\isasymunion}\ S{\isacharprime}} está contenido en el límite, 
  entonces está contenido en \isa{S\isactrlsub k} para algún \isa{k\ {\isasymin}\ {\isasymnat}}. 

  Como hemos supuesto que \isa{{\isacharbraceleft}F{\isacharbraceright}\ {\isasymunion}\ S{\isacharprime}} está contenido en el límite, entonces se verifica que \isa{F}
  pertenece al límite y \isa{S{\isacharprime}} está contenido en él. Por el lema \isa{{\isadigit{4}}{\isachardot}{\isadigit{1}}{\isachardot}{\isadigit{7}}}, como \isa{F} pertenece al 
  límite, deducimos que existe un \isa{k\ {\isasymin}\ {\isasymnat}} tal que \isa{F\ {\isasymin}\ S\isactrlsub k}. Por otro lado, como \isa{S{\isacharprime}} está contenido
  en el límite, por hipótesis de inducción existe algún \isa{k{\isacharprime}\ {\isasymin}\ {\isasymnat}} tal que \isa{S{\isacharprime}\ {\isasymsubseteq}\ S\isactrlsub k\isactrlsub {\isacharprime}}. El resultado 
  se obtiene considerando el máximo entre \isa{k} y \isa{k{\isacharprime}}, que notaremos por \isa{k{\isacharprime}{\isacharprime}}. En efecto, por la 
  monotonía de la sucesión, se verifica que tanto \isa{S\isactrlsub k} como \isa{S\isactrlsub k\isactrlsub {\isacharprime}} están contenidos en \isa{S\isactrlsub k\isactrlsub {\isacharprime}\isactrlsub {\isacharprime}}. De este 
  modo, como \isa{S{\isacharprime}\ {\isasymsubseteq}\ S\isactrlsub k\isactrlsub {\isacharprime}}, por la transitividad de la contención de conjuntos se tiene que 
  \isa{S{\isacharprime}\ {\isasymsubseteq}\ S\isactrlsub k\isactrlsub {\isacharprime}\isactrlsub {\isacharprime}}. Además, como \isa{F\ {\isasymin}\ S\isactrlsub k}, se tiene que \isa{F\ {\isasymin}\ S\isactrlsub k\isactrlsub {\isacharprime}\isactrlsub {\isacharprime}}. Por lo tanto, \isa{{\isacharbraceleft}F{\isacharbraceright}\ {\isasymunion}\ S{\isacharprime}\ {\isasymsubseteq}\ S\isactrlsub k\isactrlsub {\isacharprime}\isactrlsub {\isacharprime}}, como 
  queríamos demostrar. 
\end{demostracion}

  Procedamos con la demostración detallada en Isabelle.%
\end{isamarkuptext}\isamarkuptrue%
\isacommand{lemma}\isamarkupfalse%
\ \isanewline
\ \ \isakeyword{assumes}\ {\isachardoublequoteopen}finite\ S{\isacharprime}{\isachardoublequoteclose}\isanewline
\ \ \ \ \ \ \ \ \ \ {\isachardoublequoteopen}S{\isacharprime}\ {\isasymsubseteq}\ pcp{\isacharunderscore}lim\ C\ S{\isachardoublequoteclose}\isanewline
\ \ \ \ \ \ \ \ \isakeyword{shows}\ {\isachardoublequoteopen}{\isasymexists}k{\isachardot}\ S{\isacharprime}\ {\isasymsubseteq}\ pcp{\isacharunderscore}seq\ C\ S\ k{\isachardoublequoteclose}\isanewline
%
\isadelimproof
\ \ %
\endisadelimproof
%
\isatagproof
\isacommand{using}\isamarkupfalse%
\ assms\isanewline
\isacommand{proof}\isamarkupfalse%
\ {\isacharparenleft}induction\ S{\isacharprime}\ rule{\isacharcolon}\ finite{\isacharunderscore}induct{\isacharparenright}\isanewline
\ \ \isacommand{case}\isamarkupfalse%
\ empty\isanewline
\ \ \isacommand{have}\isamarkupfalse%
\ {\isachardoublequoteopen}pcp{\isacharunderscore}seq\ C\ S\ {\isadigit{0}}\ {\isacharequal}\ S{\isachardoublequoteclose}\isanewline
\ \ \ \ \isacommand{by}\isamarkupfalse%
\ {\isacharparenleft}simp\ only{\isacharcolon}\ pcp{\isacharunderscore}seq{\isachardot}simps{\isacharparenleft}{\isadigit{1}}{\isacharparenright}{\isacharparenright}\isanewline
\ \ \isacommand{have}\isamarkupfalse%
\ {\isachardoublequoteopen}{\isacharbraceleft}{\isacharbraceright}\ {\isasymsubseteq}\ S{\isachardoublequoteclose}\isanewline
\ \ \ \ \isacommand{by}\isamarkupfalse%
\ {\isacharparenleft}rule\ order{\isacharunderscore}bot{\isacharunderscore}class{\isachardot}bot{\isachardot}extremum{\isacharparenright}\isanewline
\ \ \isacommand{then}\isamarkupfalse%
\ \isacommand{have}\isamarkupfalse%
\ {\isachardoublequoteopen}{\isacharbraceleft}{\isacharbraceright}\ {\isasymsubseteq}\ pcp{\isacharunderscore}seq\ C\ S\ {\isadigit{0}}{\isachardoublequoteclose}\isanewline
\ \ \ \ \isacommand{by}\isamarkupfalse%
\ {\isacharparenleft}simp\ only{\isacharcolon}\ {\isacartoucheopen}pcp{\isacharunderscore}seq\ C\ S\ {\isadigit{0}}\ {\isacharequal}\ S{\isacartoucheclose}{\isacharparenright}\isanewline
\ \ \isacommand{then}\isamarkupfalse%
\ \isacommand{show}\isamarkupfalse%
\ {\isacharquery}case\ \isanewline
\ \ \ \ \isacommand{by}\isamarkupfalse%
\ {\isacharparenleft}rule\ exI{\isacharparenright}\isanewline
\isacommand{next}\isamarkupfalse%
\isanewline
\ \ \isacommand{case}\isamarkupfalse%
\ {\isacharparenleft}insert\ F\ S{\isacharprime}{\isacharparenright}\isanewline
\ \ \isacommand{then}\isamarkupfalse%
\ \isacommand{have}\isamarkupfalse%
\ {\isachardoublequoteopen}insert\ F\ S{\isacharprime}\ {\isasymsubseteq}\ pcp{\isacharunderscore}lim\ C\ S{\isachardoublequoteclose}\isanewline
\ \ \ \ \isacommand{by}\isamarkupfalse%
\ {\isacharparenleft}simp\ only{\isacharcolon}\ insert{\isachardot}prems{\isacharparenright}\isanewline
\ \ \isacommand{then}\isamarkupfalse%
\ \isacommand{have}\isamarkupfalse%
\ C{\isacharcolon}{\isachardoublequoteopen}F\ {\isasymin}\ {\isacharparenleft}pcp{\isacharunderscore}lim\ C\ S{\isacharparenright}\ {\isasymand}\ S{\isacharprime}\ {\isasymsubseteq}\ pcp{\isacharunderscore}lim\ C\ S{\isachardoublequoteclose}\isanewline
\ \ \ \ \isacommand{by}\isamarkupfalse%
\ {\isacharparenleft}simp\ only{\isacharcolon}\ insert{\isacharunderscore}subset{\isacharparenright}\ \isanewline
\ \ \isacommand{then}\isamarkupfalse%
\ \isacommand{have}\isamarkupfalse%
\ {\isachardoublequoteopen}S{\isacharprime}\ {\isasymsubseteq}\ pcp{\isacharunderscore}lim\ C\ S{\isachardoublequoteclose}\isanewline
\ \ \ \ \isacommand{by}\isamarkupfalse%
\ {\isacharparenleft}rule\ conjunct{\isadigit{2}}{\isacharparenright}\isanewline
\ \ \isacommand{then}\isamarkupfalse%
\ \isacommand{have}\isamarkupfalse%
\ EX{\isadigit{1}}{\isacharcolon}{\isachardoublequoteopen}{\isasymexists}k{\isachardot}\ S{\isacharprime}\ {\isasymsubseteq}\ pcp{\isacharunderscore}seq\ C\ S\ k{\isachardoublequoteclose}\isanewline
\ \ \ \ \isacommand{by}\isamarkupfalse%
\ {\isacharparenleft}simp\ only{\isacharcolon}\ insert{\isachardot}IH{\isacharparenright}\isanewline
\ \ \isacommand{obtain}\isamarkupfalse%
\ k{\isadigit{1}}\ \isakeyword{where}\ {\isachardoublequoteopen}S{\isacharprime}\ {\isasymsubseteq}\ pcp{\isacharunderscore}seq\ C\ S\ k{\isadigit{1}}{\isachardoublequoteclose}\isanewline
\ \ \ \ \isacommand{using}\isamarkupfalse%
\ EX{\isadigit{1}}\ \isacommand{by}\isamarkupfalse%
\ {\isacharparenleft}rule\ exE{\isacharparenright}\isanewline
\ \ \isacommand{have}\isamarkupfalse%
\ {\isachardoublequoteopen}F\ {\isasymin}\ pcp{\isacharunderscore}lim\ C\ S{\isachardoublequoteclose}\isanewline
\ \ \ \ \isacommand{using}\isamarkupfalse%
\ C\ \isacommand{by}\isamarkupfalse%
\ {\isacharparenleft}rule\ conjunct{\isadigit{1}}{\isacharparenright}\isanewline
\ \ \isacommand{then}\isamarkupfalse%
\ \isacommand{have}\isamarkupfalse%
\ EX{\isadigit{2}}{\isacharcolon}{\isachardoublequoteopen}{\isasymexists}k{\isachardot}\ F\ {\isasymin}\ pcp{\isacharunderscore}seq\ C\ S\ k{\isachardoublequoteclose}\isanewline
\ \ \ \ \isacommand{by}\isamarkupfalse%
\ {\isacharparenleft}rule\ pcp{\isacharunderscore}lim{\isacharunderscore}inserted{\isacharunderscore}at{\isacharunderscore}ex{\isacharparenright}\isanewline
\ \ \isacommand{obtain}\isamarkupfalse%
\ k{\isadigit{2}}\ \isakeyword{where}\ {\isachardoublequoteopen}F\ {\isasymin}\ pcp{\isacharunderscore}seq\ C\ S\ k{\isadigit{2}}{\isachardoublequoteclose}\ \isanewline
\ \ \ \ \isacommand{using}\isamarkupfalse%
\ EX{\isadigit{2}}\ \isacommand{by}\isamarkupfalse%
\ {\isacharparenleft}rule\ exE{\isacharparenright}\isanewline
\ \ \isacommand{have}\isamarkupfalse%
\ {\isachardoublequoteopen}k{\isadigit{1}}\ {\isasymle}\ max\ k{\isadigit{1}}\ k{\isadigit{2}}{\isachardoublequoteclose}\isanewline
\ \ \ \ \isacommand{by}\isamarkupfalse%
\ {\isacharparenleft}simp\ only{\isacharcolon}\ linorder{\isacharunderscore}class{\isachardot}max{\isachardot}cobounded{\isadigit{1}}{\isacharparenright}\isanewline
\ \ \isacommand{then}\isamarkupfalse%
\ \isacommand{have}\isamarkupfalse%
\ {\isachardoublequoteopen}pcp{\isacharunderscore}seq\ C\ S\ k{\isadigit{1}}\ {\isasymsubseteq}\ pcp{\isacharunderscore}seq\ C\ S\ {\isacharparenleft}max\ k{\isadigit{1}}\ k{\isadigit{2}}{\isacharparenright}{\isachardoublequoteclose}\isanewline
\ \ \ \ \isacommand{by}\isamarkupfalse%
\ {\isacharparenleft}rule\ pcp{\isacharunderscore}seq{\isacharunderscore}mono{\isacharparenright}\isanewline
\ \ \isacommand{have}\isamarkupfalse%
\ {\isachardoublequoteopen}k{\isadigit{2}}\ {\isasymle}\ max\ k{\isadigit{1}}\ k{\isadigit{2}}{\isachardoublequoteclose}\isanewline
\ \ \ \ \isacommand{by}\isamarkupfalse%
\ {\isacharparenleft}simp\ only{\isacharcolon}\ linorder{\isacharunderscore}class{\isachardot}max{\isachardot}cobounded{\isadigit{2}}{\isacharparenright}\isanewline
\ \ \isacommand{then}\isamarkupfalse%
\ \isacommand{have}\isamarkupfalse%
\ {\isachardoublequoteopen}pcp{\isacharunderscore}seq\ C\ S\ k{\isadigit{2}}\ {\isasymsubseteq}\ pcp{\isacharunderscore}seq\ C\ S\ {\isacharparenleft}max\ k{\isadigit{1}}\ k{\isadigit{2}}{\isacharparenright}{\isachardoublequoteclose}\isanewline
\ \ \ \ \isacommand{by}\isamarkupfalse%
\ {\isacharparenleft}rule\ pcp{\isacharunderscore}seq{\isacharunderscore}mono{\isacharparenright}\isanewline
\ \ \isacommand{have}\isamarkupfalse%
\ {\isachardoublequoteopen}S{\isacharprime}\ {\isasymsubseteq}\ pcp{\isacharunderscore}seq\ C\ S\ {\isacharparenleft}max\ k{\isadigit{1}}\ k{\isadigit{2}}{\isacharparenright}{\isachardoublequoteclose}\isanewline
\ \ \ \ \isacommand{using}\isamarkupfalse%
\ {\isacartoucheopen}S{\isacharprime}\ {\isasymsubseteq}\ pcp{\isacharunderscore}seq\ C\ S\ k{\isadigit{1}}{\isacartoucheclose}\ {\isacartoucheopen}pcp{\isacharunderscore}seq\ C\ S\ k{\isadigit{1}}\ {\isasymsubseteq}\ pcp{\isacharunderscore}seq\ C\ S\ {\isacharparenleft}max\ k{\isadigit{1}}\ k{\isadigit{2}}{\isacharparenright}{\isacartoucheclose}\ \isacommand{by}\isamarkupfalse%
\ {\isacharparenleft}rule\ subset{\isacharunderscore}trans{\isacharparenright}\isanewline
\ \ \isacommand{have}\isamarkupfalse%
\ {\isachardoublequoteopen}F\ {\isasymin}\ pcp{\isacharunderscore}seq\ C\ S\ {\isacharparenleft}max\ k{\isadigit{1}}\ k{\isadigit{2}}{\isacharparenright}{\isachardoublequoteclose}\isanewline
\ \ \ \ \isacommand{using}\isamarkupfalse%
\ {\isacartoucheopen}F\ {\isasymin}\ pcp{\isacharunderscore}seq\ C\ S\ k{\isadigit{2}}{\isacartoucheclose}\ {\isacartoucheopen}pcp{\isacharunderscore}seq\ C\ S\ k{\isadigit{2}}\ {\isasymsubseteq}\ pcp{\isacharunderscore}seq\ C\ S\ {\isacharparenleft}max\ k{\isadigit{1}}\ k{\isadigit{2}}{\isacharparenright}{\isacartoucheclose}\ \isacommand{by}\isamarkupfalse%
\ {\isacharparenleft}rule\ rev{\isacharunderscore}subsetD{\isacharparenright}\isanewline
\ \ \isacommand{then}\isamarkupfalse%
\ \isacommand{have}\isamarkupfalse%
\ {\isadigit{1}}{\isacharcolon}{\isachardoublequoteopen}insert\ F\ S{\isacharprime}\ {\isasymsubseteq}\ pcp{\isacharunderscore}seq\ C\ S\ {\isacharparenleft}max\ k{\isadigit{1}}\ k{\isadigit{2}}{\isacharparenright}{\isachardoublequoteclose}\isanewline
\ \ \ \ \isacommand{using}\isamarkupfalse%
\ {\isacartoucheopen}S{\isacharprime}\ {\isasymsubseteq}\ pcp{\isacharunderscore}seq\ C\ S\ {\isacharparenleft}max\ k{\isadigit{1}}\ k{\isadigit{2}}{\isacharparenright}{\isacartoucheclose}\ \isacommand{by}\isamarkupfalse%
\ {\isacharparenleft}simp\ only{\isacharcolon}\ insert{\isacharunderscore}subset{\isacharparenright}\isanewline
\ \ \isacommand{thus}\isamarkupfalse%
\ {\isacharquery}case\isanewline
\ \ \ \ \isacommand{by}\isamarkupfalse%
\ {\isacharparenleft}rule\ exI{\isacharparenright}\isanewline
\isacommand{qed}\isamarkupfalse%
%
\endisatagproof
{\isafoldproof}%
%
\isadelimproof
%
\endisadelimproof
%
\begin{isamarkuptext}%
Finalmente, su demostración automática en Isabelle/HOL es la siguiente.%
\end{isamarkuptext}\isamarkuptrue%
\isacommand{lemma}\isamarkupfalse%
\ finite{\isacharunderscore}pcp{\isacharunderscore}lim{\isacharunderscore}EX{\isacharcolon}\isanewline
\ \ \isakeyword{assumes}\ {\isachardoublequoteopen}finite\ S{\isacharprime}{\isachardoublequoteclose}\isanewline
\ \ \ \ \ \ \ \ \ \ {\isachardoublequoteopen}S{\isacharprime}\ {\isasymsubseteq}\ pcp{\isacharunderscore}lim\ C\ S{\isachardoublequoteclose}\isanewline
\ \ \ \ \ \ \ \ \isakeyword{shows}\ {\isachardoublequoteopen}{\isasymexists}k{\isachardot}\ S{\isacharprime}\ {\isasymsubseteq}\ pcp{\isacharunderscore}seq\ C\ S\ k{\isachardoublequoteclose}\isanewline
%
\isadelimproof
\ \ %
\endisadelimproof
%
\isatagproof
\isacommand{using}\isamarkupfalse%
\ assms\isanewline
\isacommand{proof}\isamarkupfalse%
{\isacharparenleft}induction\ S{\isacharprime}\ rule{\isacharcolon}\ finite{\isacharunderscore}induct{\isacharparenright}\ \isanewline
\ \ \isacommand{case}\isamarkupfalse%
\ {\isacharparenleft}insert\ F\ S{\isacharprime}{\isacharparenright}\isanewline
\ \ \isacommand{hence}\isamarkupfalse%
\ {\isachardoublequoteopen}{\isasymexists}k{\isachardot}\ S{\isacharprime}\ {\isasymsubseteq}\ pcp{\isacharunderscore}seq\ C\ S\ k{\isachardoublequoteclose}\ \isacommand{by}\isamarkupfalse%
\ fast\isanewline
\ \ \isacommand{then}\isamarkupfalse%
\ \isacommand{guess}\isamarkupfalse%
\ k{\isadigit{1}}\ \isacommand{{\isachardot}{\isachardot}}\isamarkupfalse%
\isanewline
\ \ \isacommand{moreover}\isamarkupfalse%
\ \isacommand{obtain}\isamarkupfalse%
\ k{\isadigit{2}}\ \isakeyword{where}\ {\isachardoublequoteopen}F\ {\isasymin}\ pcp{\isacharunderscore}seq\ C\ S\ k{\isadigit{2}}{\isachardoublequoteclose}\isanewline
\ \ \ \ \isacommand{by}\isamarkupfalse%
\ {\isacharparenleft}meson\ pcp{\isacharunderscore}lim{\isacharunderscore}inserted{\isacharunderscore}at{\isacharunderscore}ex\ insert{\isachardot}prems\ insert{\isacharunderscore}subset{\isacharparenright}\isanewline
\ \ \isacommand{ultimately}\isamarkupfalse%
\ \isacommand{have}\isamarkupfalse%
\ {\isachardoublequoteopen}insert\ F\ S{\isacharprime}\ {\isasymsubseteq}\ pcp{\isacharunderscore}seq\ C\ S\ {\isacharparenleft}max\ k{\isadigit{1}}\ k{\isadigit{2}}{\isacharparenright}{\isachardoublequoteclose}\isanewline
\ \ \ \ \isacommand{by}\isamarkupfalse%
\ {\isacharparenleft}meson\ pcp{\isacharunderscore}seq{\isacharunderscore}mono\ dual{\isacharunderscore}order{\isachardot}trans\ insert{\isacharunderscore}subset\ max{\isachardot}bounded{\isacharunderscore}iff\ order{\isacharunderscore}refl\ subsetCE{\isacharparenright}\isanewline
\ \ \isacommand{thus}\isamarkupfalse%
\ {\isacharquery}case\ \isacommand{by}\isamarkupfalse%
\ blast\isanewline
\isacommand{qed}\isamarkupfalse%
\ simp%
\endisatagproof
{\isafoldproof}%
%
\isadelimproof
%
\endisadelimproof
%
\isadelimdocument
%
\endisadelimdocument
%
\isatagdocument
%
\isamarkupsection{El Teorema de Existencia de Modelo%
}
\isamarkuptrue%
%
\endisatagdocument
{\isafolddocument}%
%
\isadelimdocument
%
\endisadelimdocument
%
\begin{isamarkuptext}%
En esta sección demostraremos finalmente el 
  \isa{teorema\ de\ existencia\ de\ modelo}, el cual prueba que todo conjunto de fórmulas perteneciente a 
  una colección que verifique la propiedad de consistencia proposicional es satisfacible. Para ello, 
  considerando una colección \isa{C} cualquiera y \isa{S\ {\isasymin}\ C}, empleando resultados anteriores extenderemos 
  la colección a una colección \isa{C{\isacharprime}{\isacharprime}} que tenga la propiedad de consistencia proposicional, sea
  cerrada bajo subconjuntos y sea de carácter finito. De este modo, en esta sección probaremos que el 
  límite de la sucesión formada a partir de una colección que tenga dichas condiciones y un conjunto
  cualquiera \isa{S} como se indica en la definición \isa{{\isadigit{1}}{\isachardot}{\isadigit{4}}{\isachardot}{\isadigit{1}}} pertenece a la colección. Es más, 
  demostraremos que dicho límite se trata de un conjunto de \isa{Hintikka} luego, por el \isa{teorema\ de\ Hintikka}, es satisfacible. Finalmente, como \isa{S} está contenido en el límite, quedará demostrada 
  la satisfacibilidad del conjunto \isa{S} al heredarla por contención.

  \comentario{Habrá que modificar el párrafo anterior al final.}%
\end{isamarkuptext}\isamarkuptrue%
%
\begin{isamarkuptext}%
En primer lugar, probemos que si \isa{C} es una colección que verifica la propiedad de 
  consistencia proposicional, es cerrada bajo subconjuntos y es de carácter finito, entonces el 
  límite de toda sucesión de conjuntos de \isa{C} según la definición \isa{{\isadigit{4}}{\isachardot}{\isadigit{1}}{\isachardot}{\isadigit{1}}} pertenece a \isa{C}.

  \begin{lema}
    Sea \isa{C} una colección de conjuntos que verifica la propiedad de consistencia proposicional, es 
    cerrada bajo subconjuntos y es de carácter finito. Sea \isa{S\ {\isasymin}\ C} y \isa{{\isacharbraceleft}S\isactrlsub n{\isacharbraceright}} la sucesión de conjuntos
    de \isa{C} a partir de \isa{S} según la definición \isa{{\isadigit{4}}{\isachardot}{\isadigit{1}}{\isachardot}{\isadigit{1}}}. Entonces, el límite de la sucesión está en
    \isa{C}.
  \end{lema}

  \begin{demostracion}
    Por definición, como \isa{C} es de carácter finito, para todo conjunto son equivalentes:
    \begin{enumerate}
      \item El conjunto pertenece a \isa{C}.
      \item Todo subconjunto finito suyo pertenece a \isa{C}.
    \end{enumerate}

    De este modo, para demostrar que el límite de la sucesión \isa{{\isacharbraceleft}S\isactrlsub n{\isacharbraceright}} pertenece a \isa{C}, basta probar
    que todo subconjunto finito suyo está en \isa{C}.

    Sea \isa{S{\isacharprime}} un subconjunto finito del límite de la sucesión. Por el lema \isa{{\isadigit{1}}{\isachardot}{\isadigit{4}}{\isachardot}{\isadigit{8}}}, existe un
    \isa{k\ {\isasymin}\ {\isasymnat}} tal que \isa{S{\isacharprime}\ {\isasymsubseteq}\ S\isactrlsub k}. Por tanto, como \isa{S\isactrlsub k\ {\isasymin}\ C} para todo \isa{k\ {\isasymin}\ {\isasymnat}} y \isa{C} es cerrada bajo
    subconjuntos, por definición se tiene que \isa{S{\isacharprime}\ {\isasymin}\ C}, como queríamos demostrar.
  \end{demostracion}

  En Isabelle se formaliza y demuestra detalladamente como sigue.%
\end{isamarkuptext}\isamarkuptrue%
\isacommand{lemma}\isamarkupfalse%
\isanewline
\ \ \isakeyword{assumes}\ {\isachardoublequoteopen}pcp\ C{\isachardoublequoteclose}\isanewline
\ \ \ \ \ \ \ \ \ \ {\isachardoublequoteopen}S\ {\isasymin}\ C{\isachardoublequoteclose}\isanewline
\ \ \ \ \ \ \ \ \ \ {\isachardoublequoteopen}subset{\isacharunderscore}closed\ C{\isachardoublequoteclose}\isanewline
\ \ \ \ \ \ \ \ \ \ {\isachardoublequoteopen}finite{\isacharunderscore}character\ C{\isachardoublequoteclose}\isanewline
\ \ \isakeyword{shows}\ {\isachardoublequoteopen}pcp{\isacharunderscore}lim\ C\ S\ {\isasymin}\ C{\isachardoublequoteclose}\ \isanewline
%
\isadelimproof
%
\endisadelimproof
%
\isatagproof
\isacommand{proof}\isamarkupfalse%
\ {\isacharminus}\isanewline
\ \ \isacommand{have}\isamarkupfalse%
\ {\isachardoublequoteopen}{\isasymforall}S{\isachardot}\ S\ {\isasymin}\ C\ {\isasymlongleftrightarrow}\ {\isacharparenleft}{\isasymforall}S{\isacharprime}\ {\isasymsubseteq}\ S{\isachardot}\ finite\ S{\isacharprime}\ {\isasymlongrightarrow}\ S{\isacharprime}\ {\isasymin}\ C{\isacharparenright}{\isachardoublequoteclose}\isanewline
\ \ \ \ \isacommand{using}\isamarkupfalse%
\ assms{\isacharparenleft}{\isadigit{4}}{\isacharparenright}\ \isacommand{unfolding}\isamarkupfalse%
\ finite{\isacharunderscore}character{\isacharunderscore}def\ \isacommand{by}\isamarkupfalse%
\ this\isanewline
\ \ \isacommand{then}\isamarkupfalse%
\ \isacommand{have}\isamarkupfalse%
\ FC{\isadigit{1}}{\isacharcolon}{\isachardoublequoteopen}pcp{\isacharunderscore}lim\ C\ S\ {\isasymin}\ C\ {\isasymlongleftrightarrow}\ {\isacharparenleft}{\isasymforall}S{\isacharprime}\ {\isasymsubseteq}\ {\isacharparenleft}pcp{\isacharunderscore}lim\ C\ S{\isacharparenright}{\isachardot}\ finite\ S{\isacharprime}\ {\isasymlongrightarrow}\ S{\isacharprime}\ {\isasymin}\ C{\isacharparenright}{\isachardoublequoteclose}\isanewline
\ \ \ \ \isacommand{by}\isamarkupfalse%
\ {\isacharparenleft}rule\ allE{\isacharparenright}\isanewline
\ \ \isacommand{have}\isamarkupfalse%
\ SC{\isacharcolon}{\isachardoublequoteopen}{\isasymforall}S\ {\isasymin}\ C{\isachardot}\ {\isasymforall}S{\isacharprime}{\isasymsubseteq}S{\isachardot}\ S{\isacharprime}\ {\isasymin}\ C{\isachardoublequoteclose}\isanewline
\ \ \ \ \isacommand{using}\isamarkupfalse%
\ assms{\isacharparenleft}{\isadigit{3}}{\isacharparenright}\ \isacommand{unfolding}\isamarkupfalse%
\ subset{\isacharunderscore}closed{\isacharunderscore}def\ \isacommand{by}\isamarkupfalse%
\ this\isanewline
\ \ \isacommand{have}\isamarkupfalse%
\ FC{\isadigit{2}}{\isacharcolon}{\isachardoublequoteopen}{\isasymforall}S{\isacharprime}\ {\isasymsubseteq}\ pcp{\isacharunderscore}lim\ C\ S{\isachardot}\ finite\ S{\isacharprime}\ {\isasymlongrightarrow}\ S{\isacharprime}\ {\isasymin}\ C{\isachardoublequoteclose}\isanewline
\ \ \isacommand{proof}\isamarkupfalse%
\ {\isacharparenleft}rule\ sallI{\isacharparenright}\isanewline
\ \ \ \ \isacommand{fix}\isamarkupfalse%
\ S{\isacharprime}\ {\isacharcolon}{\isacharcolon}\ {\isachardoublequoteopen}{\isacharprime}a\ formula\ set{\isachardoublequoteclose}\isanewline
\ \ \ \ \isacommand{assume}\isamarkupfalse%
\ {\isachardoublequoteopen}S{\isacharprime}\ {\isasymsubseteq}\ pcp{\isacharunderscore}lim\ C\ S{\isachardoublequoteclose}\isanewline
\ \ \ \ \isacommand{show}\isamarkupfalse%
\ {\isachardoublequoteopen}finite\ S{\isacharprime}\ {\isasymlongrightarrow}\ S{\isacharprime}\ {\isasymin}\ C{\isachardoublequoteclose}\isanewline
\ \ \ \ \isacommand{proof}\isamarkupfalse%
\ {\isacharparenleft}rule\ impI{\isacharparenright}\isanewline
\ \ \ \ \ \ \isacommand{assume}\isamarkupfalse%
\ {\isachardoublequoteopen}finite\ S{\isacharprime}{\isachardoublequoteclose}\isanewline
\ \ \ \ \ \ \isacommand{then}\isamarkupfalse%
\ \isacommand{have}\isamarkupfalse%
\ EX{\isacharcolon}{\isachardoublequoteopen}{\isasymexists}k{\isachardot}\ S{\isacharprime}\ {\isasymsubseteq}\ pcp{\isacharunderscore}seq\ C\ S\ k{\isachardoublequoteclose}\ \isanewline
\ \ \ \ \ \ \ \ \isacommand{using}\isamarkupfalse%
\ {\isacartoucheopen}S{\isacharprime}\ {\isasymsubseteq}\ pcp{\isacharunderscore}lim\ C\ S{\isacartoucheclose}\ \isacommand{by}\isamarkupfalse%
\ {\isacharparenleft}rule\ finite{\isacharunderscore}pcp{\isacharunderscore}lim{\isacharunderscore}EX{\isacharparenright}\isanewline
\ \ \ \ \ \ \isacommand{obtain}\isamarkupfalse%
\ k\ \isakeyword{where}\ {\isachardoublequoteopen}S{\isacharprime}\ {\isasymsubseteq}\ pcp{\isacharunderscore}seq\ C\ S\ k{\isachardoublequoteclose}\isanewline
\ \ \ \ \ \ \ \ \isacommand{using}\isamarkupfalse%
\ EX\ \isacommand{by}\isamarkupfalse%
\ {\isacharparenleft}rule\ exE{\isacharparenright}\isanewline
\ \ \ \ \ \ \isacommand{have}\isamarkupfalse%
\ {\isachardoublequoteopen}pcp{\isacharunderscore}seq\ C\ S\ k\ {\isasymin}\ C{\isachardoublequoteclose}\isanewline
\ \ \ \ \ \ \ \ \isacommand{using}\isamarkupfalse%
\ assms{\isacharparenleft}{\isadigit{1}}{\isacharparenright}\ assms{\isacharparenleft}{\isadigit{2}}{\isacharparenright}\ \isacommand{by}\isamarkupfalse%
\ {\isacharparenleft}rule\ pcp{\isacharunderscore}seq{\isacharunderscore}in{\isacharparenright}\isanewline
\ \ \ \ \ \ \isacommand{have}\isamarkupfalse%
\ {\isachardoublequoteopen}{\isasymforall}S{\isacharprime}\ {\isasymsubseteq}\ {\isacharparenleft}pcp{\isacharunderscore}seq\ C\ S\ k{\isacharparenright}{\isachardot}\ S{\isacharprime}\ {\isasymin}\ C{\isachardoublequoteclose}\isanewline
\ \ \ \ \ \ \ \ \isacommand{using}\isamarkupfalse%
\ SC\ {\isacartoucheopen}pcp{\isacharunderscore}seq\ C\ S\ k\ {\isasymin}\ C{\isacartoucheclose}\ \isacommand{by}\isamarkupfalse%
\ {\isacharparenleft}rule\ bspec{\isacharparenright}\isanewline
\ \ \ \ \ \ \isacommand{thus}\isamarkupfalse%
\ {\isachardoublequoteopen}S{\isacharprime}\ {\isasymin}\ C{\isachardoublequoteclose}\isanewline
\ \ \ \ \ \ \ \ \isacommand{using}\isamarkupfalse%
\ {\isacartoucheopen}S{\isacharprime}\ {\isasymsubseteq}\ pcp{\isacharunderscore}seq\ C\ S\ k{\isacartoucheclose}\ \isacommand{by}\isamarkupfalse%
\ {\isacharparenleft}rule\ sspec{\isacharparenright}\isanewline
\ \ \ \ \isacommand{qed}\isamarkupfalse%
\isanewline
\ \ \isacommand{qed}\isamarkupfalse%
\isanewline
\ \ \isacommand{show}\isamarkupfalse%
\ {\isachardoublequoteopen}pcp{\isacharunderscore}lim\ C\ S\ {\isasymin}\ C{\isachardoublequoteclose}\ \isanewline
\ \ \ \ \isacommand{using}\isamarkupfalse%
\ FC{\isadigit{1}}\ FC{\isadigit{2}}\ \isacommand{by}\isamarkupfalse%
\ {\isacharparenleft}rule\ forw{\isacharunderscore}subst{\isacharparenright}\isanewline
\isacommand{qed}\isamarkupfalse%
%
\endisatagproof
{\isafoldproof}%
%
\isadelimproof
%
\endisadelimproof
%
\begin{isamarkuptext}%
Por otra parte, podemos dar una prueba automática del resultado.%
\end{isamarkuptext}\isamarkuptrue%
\isacommand{lemma}\isamarkupfalse%
\ pcp{\isacharunderscore}lim{\isacharunderscore}in{\isacharcolon}\isanewline
\ \ \isakeyword{assumes}\ c{\isacharcolon}\ {\isachardoublequoteopen}pcp\ C{\isachardoublequoteclose}\isanewline
\ \ \isakeyword{and}\ el{\isacharcolon}\ {\isachardoublequoteopen}S\ {\isasymin}\ C{\isachardoublequoteclose}\isanewline
\ \ \isakeyword{and}\ sc{\isacharcolon}\ {\isachardoublequoteopen}subset{\isacharunderscore}closed\ C{\isachardoublequoteclose}\isanewline
\ \ \isakeyword{and}\ fc{\isacharcolon}\ {\isachardoublequoteopen}finite{\isacharunderscore}character\ C{\isachardoublequoteclose}\isanewline
\ \ \isakeyword{shows}\ {\isachardoublequoteopen}pcp{\isacharunderscore}lim\ C\ S\ {\isasymin}\ C{\isachardoublequoteclose}\ {\isacharparenleft}\isakeyword{is}\ {\isachardoublequoteopen}{\isacharquery}cl\ {\isasymin}\ C{\isachardoublequoteclose}{\isacharparenright}\isanewline
%
\isadelimproof
%
\endisadelimproof
%
\isatagproof
\isacommand{proof}\isamarkupfalse%
\ {\isacharminus}\isanewline
\ \ \isacommand{from}\isamarkupfalse%
\ pcp{\isacharunderscore}seq{\isacharunderscore}in{\isacharbrackleft}OF\ c\ el{\isacharcomma}\ THEN\ allI{\isacharbrackright}\ \isacommand{have}\isamarkupfalse%
\ {\isachardoublequoteopen}{\isasymforall}n{\isachardot}\ pcp{\isacharunderscore}seq\ C\ S\ n\ {\isasymin}\ C{\isachardoublequoteclose}\ \isacommand{{\isachardot}}\isamarkupfalse%
\isanewline
\ \ \isacommand{hence}\isamarkupfalse%
\ {\isachardoublequoteopen}{\isasymforall}m{\isachardot}\ {\isasymUnion}{\isacharbraceleft}pcp{\isacharunderscore}seq\ C\ S\ n{\isacharbar}n{\isachardot}\ n\ {\isasymle}\ m{\isacharbraceright}\ {\isasymin}\ C{\isachardoublequoteclose}\ \isacommand{unfolding}\isamarkupfalse%
\ pcp{\isacharunderscore}seq{\isacharunderscore}UN\ \isacommand{{\isachardot}}\isamarkupfalse%
\isanewline
\ \ \isacommand{have}\isamarkupfalse%
\ {\isachardoublequoteopen}{\isasymforall}S{\isacharprime}\ {\isasymsubseteq}\ {\isacharquery}cl{\isachardot}\ finite\ S{\isacharprime}\ {\isasymlongrightarrow}\ S{\isacharprime}\ {\isasymin}\ C{\isachardoublequoteclose}\isanewline
\ \ \isacommand{proof}\isamarkupfalse%
\ safe\isanewline
\ \ \ \ \isacommand{fix}\isamarkupfalse%
\ S{\isacharprime}\ {\isacharcolon}{\isacharcolon}\ {\isachardoublequoteopen}{\isacharprime}a\ formula\ set{\isachardoublequoteclose}\isanewline
\ \ \ \ \isacommand{have}\isamarkupfalse%
\ {\isachardoublequoteopen}pcp{\isacharunderscore}seq\ C\ S\ {\isacharparenleft}Suc\ {\isacharparenleft}Max\ {\isacharparenleft}to{\isacharunderscore}nat\ {\isacharbackquote}\ S{\isacharprime}{\isacharparenright}{\isacharparenright}{\isacharparenright}\ {\isasymsubseteq}\ pcp{\isacharunderscore}lim\ C\ S{\isachardoublequoteclose}\ \isanewline
\ \ \ \ \ \ \isacommand{using}\isamarkupfalse%
\ pcp{\isacharunderscore}seq{\isacharunderscore}sub\ \isacommand{by}\isamarkupfalse%
\ blast\isanewline
\ \ \ \ \isacommand{assume}\isamarkupfalse%
\ {\isacartoucheopen}finite\ S{\isacharprime}{\isacartoucheclose}\ {\isacartoucheopen}S{\isacharprime}\ {\isasymsubseteq}\ pcp{\isacharunderscore}lim\ C\ S{\isacartoucheclose}\isanewline
\ \ \ \ \isacommand{hence}\isamarkupfalse%
\ {\isachardoublequoteopen}{\isasymexists}k{\isachardot}\ S{\isacharprime}\ {\isasymsubseteq}\ pcp{\isacharunderscore}seq\ C\ S\ k{\isachardoublequoteclose}\ \isanewline
\ \ \ \ \isacommand{proof}\isamarkupfalse%
{\isacharparenleft}induction\ S{\isacharprime}\ rule{\isacharcolon}\ finite{\isacharunderscore}induct{\isacharparenright}\ \isanewline
\ \ \ \ \ \ \isacommand{case}\isamarkupfalse%
\ {\isacharparenleft}insert\ x\ S{\isacharprime}{\isacharparenright}\isanewline
\ \ \ \ \ \ \isacommand{hence}\isamarkupfalse%
\ {\isachardoublequoteopen}{\isasymexists}k{\isachardot}\ S{\isacharprime}\ {\isasymsubseteq}\ pcp{\isacharunderscore}seq\ C\ S\ k{\isachardoublequoteclose}\ \isacommand{by}\isamarkupfalse%
\ fast\isanewline
\ \ \ \ \ \ \isacommand{then}\isamarkupfalse%
\ \isacommand{guess}\isamarkupfalse%
\ k{\isadigit{1}}\ \isacommand{{\isachardot}{\isachardot}}\isamarkupfalse%
\isanewline
\ \ \ \ \ \ \isacommand{moreover}\isamarkupfalse%
\ \isacommand{obtain}\isamarkupfalse%
\ k{\isadigit{2}}\ \isakeyword{where}\ {\isachardoublequoteopen}x\ {\isasymin}\ pcp{\isacharunderscore}seq\ C\ S\ k{\isadigit{2}}{\isachardoublequoteclose}\isanewline
\ \ \ \ \ \ \ \ \isacommand{by}\isamarkupfalse%
\ {\isacharparenleft}meson\ pcp{\isacharunderscore}lim{\isacharunderscore}inserted{\isacharunderscore}at{\isacharunderscore}ex\ insert{\isachardot}prems\ insert{\isacharunderscore}subset{\isacharparenright}\isanewline
\ \ \ \ \ \ \isacommand{ultimately}\isamarkupfalse%
\ \isacommand{have}\isamarkupfalse%
\ {\isachardoublequoteopen}insert\ x\ S{\isacharprime}\ {\isasymsubseteq}\ pcp{\isacharunderscore}seq\ C\ S\ {\isacharparenleft}max\ k{\isadigit{1}}\ k{\isadigit{2}}{\isacharparenright}{\isachardoublequoteclose}\isanewline
\ \ \ \ \ \ \ \ \isacommand{by}\isamarkupfalse%
\ {\isacharparenleft}meson\ pcp{\isacharunderscore}seq{\isacharunderscore}mono\ dual{\isacharunderscore}order{\isachardot}trans\ insert{\isacharunderscore}subset\ max{\isachardot}bounded{\isacharunderscore}iff\ order{\isacharunderscore}refl\ subsetCE{\isacharparenright}\isanewline
\ \ \ \ \ \ \isacommand{thus}\isamarkupfalse%
\ {\isacharquery}case\ \isacommand{by}\isamarkupfalse%
\ blast\isanewline
\ \ \ \ \isacommand{qed}\isamarkupfalse%
\ simp\isanewline
\ \ \ \ \isacommand{with}\isamarkupfalse%
\ pcp{\isacharunderscore}seq{\isacharunderscore}in{\isacharbrackleft}OF\ c\ el{\isacharbrackright}\ sc\isanewline
\ \ \ \ \isacommand{show}\isamarkupfalse%
\ {\isachardoublequoteopen}S{\isacharprime}\ {\isasymin}\ C{\isachardoublequoteclose}\ \isacommand{unfolding}\isamarkupfalse%
\ subset{\isacharunderscore}closed{\isacharunderscore}def\ \isacommand{by}\isamarkupfalse%
\ blast\isanewline
\ \ \isacommand{qed}\isamarkupfalse%
\isanewline
\ \ \isacommand{thus}\isamarkupfalse%
\ {\isachardoublequoteopen}{\isacharquery}cl\ {\isasymin}\ C{\isachardoublequoteclose}\ \isacommand{using}\isamarkupfalse%
\ fc\ \isacommand{unfolding}\isamarkupfalse%
\ finite{\isacharunderscore}character{\isacharunderscore}def\ \isacommand{by}\isamarkupfalse%
\ blast\isanewline
\isacommand{qed}\isamarkupfalse%
%
\endisatagproof
{\isafoldproof}%
%
\isadelimproof
%
\endisadelimproof
%
\begin{isamarkuptext}%
Probemos que, además, el límite de las sucesión definida en \isa{{\isadigit{4}}{\isachardot}{\isadigit{1}}{\isachardot}{\isadigit{1}}} se trata de un elemento 
  maximal de la colección que lo define si esta verifica la propiedad de consistencia proposicional
  y es cerrada bajo subconjuntos.

  \begin{lema}
    Sea \isa{C} una colección de conjuntos que verifica la propiedad de consistencia proposicional y
    es cerrada bajo subconjuntos, \isa{S} un conjunto y \isa{{\isacharbraceleft}S\isactrlsub n{\isacharbraceright}} la sucesión de conjuntos de \isa{C} a partir 
    de \isa{S} según la definición \isa{{\isadigit{4}}{\isachardot}{\isadigit{1}}{\isachardot}{\isadigit{1}}}. Entonces, el límite de la sucesión \isa{{\isacharbraceleft}S\isactrlsub n{\isacharbraceright}} es un elemento 
    maximal de \isa{C}.
  \end{lema}

  \begin{demostracion}
    Por definición de elemento maximal, basta probar que para cualquier conjunto \isa{K\ {\isasymin}\ C} que
    contenga al límite de la sucesión se tiene que \isa{K} y el límite coinciden.

    La demostración se realizará por reducción al absurdo. Consideremos un conjunto \isa{K\ {\isasymin}\ C} que 
    contenga estrictamente al límite de la sucesión \isa{{\isacharbraceleft}S\isactrlsub n{\isacharbraceright}}. De este modo, existe una fórmula \isa{F} tal 
    que \isa{F\ {\isasymin}\ K} y \isa{F} no está en el límite. Supongamos que \isa{F} es la \isa{n}-ésima fórmula según la 
    enumeración de la definición \isa{{\isadigit{4}}{\isachardot}{\isadigit{1}}{\isachardot}{\isadigit{1}}} utilizada para construir la sucesión. 

    Por un lado, hemos probado que todo elemento de la sucesión está contenido en el límite, luego 
    en particular obtenemos que \isa{S\isactrlsub n\isactrlsub {\isacharplus}\isactrlsub {\isadigit{1}}} está contenido en el límite. De este modo, como \isa{F} no 
    pertenece al límite, es claro que \isa{F\ {\isasymnotin}\ S\isactrlsub n\isactrlsub {\isacharplus}\isactrlsub {\isadigit{1}}}. Además, \isa{{\isacharbraceleft}F{\isacharbraceright}\ {\isasymunion}\ S\isactrlsub n\ {\isasymnotin}\ C} ya que, en caso contrario, 
    por la definición \isa{{\isadigit{4}}{\isachardot}{\isadigit{1}}{\isachardot}{\isadigit{1}}} de la sucesión obtendríamos que\\ \isa{S\isactrlsub n\isactrlsub {\isacharplus}\isactrlsub {\isadigit{1}}\ {\isacharequal}\ {\isacharbraceleft}F{\isacharbraceright}\ {\isasymunion}\ S\isactrlsub n}, lo que contradice 
    que \isa{F\ {\isasymnotin}\ S\isactrlsub n\isactrlsub {\isacharplus}\isactrlsub {\isadigit{1}}}. 

    Por otro lado, como \isa{S\isactrlsub n} también está contenida en el límite que, a su vez, está contenido en 
    \isa{K}, se obtiene por transitividad que \isa{S\isactrlsub n\ {\isasymsubseteq}\ K}. Además, como \isa{F\ {\isasymin}\ K}, se verifica que 
    \isa{{\isacharbraceleft}F{\isacharbraceright}\ {\isasymunion}\ S\isactrlsub n\ {\isasymsubseteq}\ K}. Como \isa{C} es una colección cerrada bajo subconjuntos por hipótesis y \isa{K\ {\isasymin}\ C}, 
    por definición se tiene que \isa{{\isacharbraceleft}F{\isacharbraceright}\ {\isasymunion}\ S\isactrlsub n\ {\isasymin}\ C}, llegando así a una contradicción con lo demostrado 
    anteriormente.
  \end{demostracion}

  Su formalización y prueba detallada en Isabelle/HOL se muestran a continuación.%
\end{isamarkuptext}\isamarkuptrue%
\isacommand{lemma}\isamarkupfalse%
\isanewline
\ \ \isakeyword{assumes}\ {\isachardoublequoteopen}pcp\ C{\isachardoublequoteclose}\isanewline
\ \ \ \ \ \ \ \ \ \ {\isachardoublequoteopen}subset{\isacharunderscore}closed\ C{\isachardoublequoteclose}\isanewline
\ \ \ \ \ \ \ \ \ \ {\isachardoublequoteopen}K\ {\isasymin}\ C{\isachardoublequoteclose}\isanewline
\ \ \ \ \ \ \ \ \ \ {\isachardoublequoteopen}pcp{\isacharunderscore}lim\ C\ S\ {\isasymsubseteq}\ K{\isachardoublequoteclose}\isanewline
\ \ \isakeyword{shows}\ {\isachardoublequoteopen}pcp{\isacharunderscore}lim\ C\ S\ {\isacharequal}\ K{\isachardoublequoteclose}\isanewline
%
\isadelimproof
%
\endisadelimproof
%
\isatagproof
\isacommand{proof}\isamarkupfalse%
\ {\isacharparenleft}rule\ ccontr{\isacharparenright}\isanewline
\ \ \isacommand{assume}\isamarkupfalse%
\ H{\isacharcolon}{\isachardoublequoteopen}{\isasymnot}{\isacharparenleft}pcp{\isacharunderscore}lim\ C\ S\ {\isacharequal}\ K{\isacharparenright}{\isachardoublequoteclose}\isanewline
\ \ \isacommand{have}\isamarkupfalse%
\ CE{\isacharcolon}{\isachardoublequoteopen}pcp{\isacharunderscore}lim\ C\ S\ {\isasymsubseteq}\ K\ {\isasymand}\ pcp{\isacharunderscore}lim\ C\ S\ {\isasymnoteq}\ K{\isachardoublequoteclose}\isanewline
\ \ \ \ \isacommand{using}\isamarkupfalse%
\ assms{\isacharparenleft}{\isadigit{4}}{\isacharparenright}\ H\ \isacommand{by}\isamarkupfalse%
\ {\isacharparenleft}rule\ conjI{\isacharparenright}\isanewline
\ \ \isacommand{have}\isamarkupfalse%
\ {\isachardoublequoteopen}pcp{\isacharunderscore}lim\ C\ S\ {\isasymsubseteq}\ K\ {\isasymand}\ pcp{\isacharunderscore}lim\ C\ S\ {\isasymnoteq}\ K\ {\isasymlongleftrightarrow}\ pcp{\isacharunderscore}lim\ C\ S\ {\isasymsubset}\ K{\isachardoublequoteclose}\isanewline
\ \ \ \ \isacommand{by}\isamarkupfalse%
\ {\isacharparenleft}simp\ only{\isacharcolon}\ psubset{\isacharunderscore}eq{\isacharparenright}\isanewline
\ \ \isacommand{then}\isamarkupfalse%
\ \isacommand{have}\isamarkupfalse%
\ {\isachardoublequoteopen}pcp{\isacharunderscore}lim\ C\ S\ {\isasymsubset}\ K{\isachardoublequoteclose}\ \isanewline
\ \ \ \ \isacommand{using}\isamarkupfalse%
\ CE\ \isacommand{by}\isamarkupfalse%
\ {\isacharparenleft}rule\ iffD{\isadigit{1}}{\isacharparenright}\isanewline
\ \ \isacommand{then}\isamarkupfalse%
\ \isacommand{have}\isamarkupfalse%
\ {\isachardoublequoteopen}{\isasymexists}F{\isachardot}\ F\ {\isasymin}\ {\isacharparenleft}K\ {\isacharminus}\ {\isacharparenleft}pcp{\isacharunderscore}lim\ C\ S{\isacharparenright}{\isacharparenright}{\isachardoublequoteclose}\isanewline
\ \ \ \ \isacommand{by}\isamarkupfalse%
\ {\isacharparenleft}simp\ only{\isacharcolon}\ psubset{\isacharunderscore}imp{\isacharunderscore}ex{\isacharunderscore}mem{\isacharparenright}\ \isanewline
\ \ \isacommand{then}\isamarkupfalse%
\ \isacommand{have}\isamarkupfalse%
\ E{\isacharcolon}{\isachardoublequoteopen}{\isasymexists}F{\isachardot}\ F\ {\isasymin}\ K\ {\isasymand}\ F\ {\isasymnotin}\ {\isacharparenleft}pcp{\isacharunderscore}lim\ C\ S{\isacharparenright}{\isachardoublequoteclose}\isanewline
\ \ \ \ \isacommand{by}\isamarkupfalse%
\ {\isacharparenleft}simp\ only{\isacharcolon}\ Diff{\isacharunderscore}iff{\isacharparenright}\isanewline
\ \ \isacommand{obtain}\isamarkupfalse%
\ F\ \isakeyword{where}\ F{\isacharcolon}{\isachardoublequoteopen}F\ {\isasymin}\ K\ {\isasymand}\ F\ {\isasymnotin}\ pcp{\isacharunderscore}lim\ C\ S{\isachardoublequoteclose}\ \isanewline
\ \ \ \ \isacommand{using}\isamarkupfalse%
\ E\ \isacommand{by}\isamarkupfalse%
\ {\isacharparenleft}rule\ exE{\isacharparenright}\isanewline
\ \ \isacommand{have}\isamarkupfalse%
\ {\isachardoublequoteopen}F\ {\isasymin}\ K{\isachardoublequoteclose}\ \isanewline
\ \ \ \ \isacommand{using}\isamarkupfalse%
\ F\ \isacommand{by}\isamarkupfalse%
\ {\isacharparenleft}rule\ conjunct{\isadigit{1}}{\isacharparenright}\isanewline
\ \ \isacommand{have}\isamarkupfalse%
\ {\isachardoublequoteopen}F\ {\isasymnotin}\ pcp{\isacharunderscore}lim\ C\ S{\isachardoublequoteclose}\isanewline
\ \ \ \ \isacommand{using}\isamarkupfalse%
\ F\ \isacommand{by}\isamarkupfalse%
\ {\isacharparenleft}rule\ conjunct{\isadigit{2}}{\isacharparenright}\isanewline
\ \ \isacommand{have}\isamarkupfalse%
\ {\isachardoublequoteopen}pcp{\isacharunderscore}seq\ C\ S\ {\isacharparenleft}Suc\ {\isacharparenleft}to{\isacharunderscore}nat\ F{\isacharparenright}{\isacharparenright}\ {\isasymsubseteq}\ pcp{\isacharunderscore}lim\ C\ S{\isachardoublequoteclose}\isanewline
\ \ \ \ \isacommand{by}\isamarkupfalse%
\ {\isacharparenleft}rule\ pcp{\isacharunderscore}seq{\isacharunderscore}sub{\isacharparenright}\isanewline
\ \ \isacommand{then}\isamarkupfalse%
\ \isacommand{have}\isamarkupfalse%
\ {\isachardoublequoteopen}F\ {\isasymin}\ pcp{\isacharunderscore}seq\ C\ S\ {\isacharparenleft}Suc\ {\isacharparenleft}to{\isacharunderscore}nat\ F{\isacharparenright}{\isacharparenright}\ {\isasymlongrightarrow}\ F\ {\isasymin}\ pcp{\isacharunderscore}lim\ C\ S{\isachardoublequoteclose}\isanewline
\ \ \ \ \isacommand{by}\isamarkupfalse%
\ {\isacharparenleft}rule\ in{\isacharunderscore}mono{\isacharparenright}\isanewline
\ \ \isacommand{then}\isamarkupfalse%
\ \isacommand{have}\isamarkupfalse%
\ {\isadigit{1}}{\isacharcolon}{\isachardoublequoteopen}F\ {\isasymnotin}\ pcp{\isacharunderscore}seq\ C\ S\ {\isacharparenleft}Suc\ {\isacharparenleft}to{\isacharunderscore}nat\ F{\isacharparenright}{\isacharparenright}{\isachardoublequoteclose}\isanewline
\ \ \ \ \isacommand{using}\isamarkupfalse%
\ {\isacartoucheopen}F\ {\isasymnotin}\ pcp{\isacharunderscore}lim\ C\ S{\isacartoucheclose}\ \isacommand{by}\isamarkupfalse%
\ {\isacharparenleft}rule\ mt{\isacharparenright}\isanewline
\ \ \isacommand{have}\isamarkupfalse%
\ {\isadigit{2}}{\isacharcolon}\ {\isachardoublequoteopen}insert\ F\ {\isacharparenleft}pcp{\isacharunderscore}seq\ C\ S\ {\isacharparenleft}to{\isacharunderscore}nat\ F{\isacharparenright}{\isacharparenright}\ {\isasymnotin}\ C{\isachardoublequoteclose}\ \isanewline
\ \ \isacommand{proof}\isamarkupfalse%
\ {\isacharparenleft}rule\ ccontr{\isacharparenright}\isanewline
\ \ \ \ \isacommand{assume}\isamarkupfalse%
\ {\isachardoublequoteopen}{\isasymnot}{\isacharparenleft}insert\ F\ {\isacharparenleft}pcp{\isacharunderscore}seq\ C\ S\ {\isacharparenleft}to{\isacharunderscore}nat\ F{\isacharparenright}{\isacharparenright}\ {\isasymnotin}\ C{\isacharparenright}{\isachardoublequoteclose}\isanewline
\ \ \ \ \isacommand{then}\isamarkupfalse%
\ \isacommand{have}\isamarkupfalse%
\ {\isachardoublequoteopen}insert\ F\ {\isacharparenleft}pcp{\isacharunderscore}seq\ C\ S\ {\isacharparenleft}to{\isacharunderscore}nat\ F{\isacharparenright}{\isacharparenright}\ {\isasymin}\ C{\isachardoublequoteclose}\isanewline
\ \ \ \ \ \ \isacommand{by}\isamarkupfalse%
\ {\isacharparenleft}rule\ notnotD{\isacharparenright}\isanewline
\ \ \ \ \isacommand{then}\isamarkupfalse%
\ \isacommand{have}\isamarkupfalse%
\ C{\isacharcolon}{\isachardoublequoteopen}insert\ {\isacharparenleft}from{\isacharunderscore}nat\ {\isacharparenleft}to{\isacharunderscore}nat\ F{\isacharparenright}{\isacharparenright}\ {\isacharparenleft}pcp{\isacharunderscore}seq\ C\ S\ {\isacharparenleft}to{\isacharunderscore}nat\ F{\isacharparenright}{\isacharparenright}\ {\isasymin}\ C{\isachardoublequoteclose}\isanewline
\ \ \ \ \ \ \isacommand{by}\isamarkupfalse%
\ {\isacharparenleft}simp\ only{\isacharcolon}\ from{\isacharunderscore}nat{\isacharunderscore}to{\isacharunderscore}nat{\isacharparenright}\isanewline
\ \ \ \ \isacommand{have}\isamarkupfalse%
\ {\isachardoublequoteopen}pcp{\isacharunderscore}seq\ C\ S\ {\isacharparenleft}Suc\ {\isacharparenleft}to{\isacharunderscore}nat\ F{\isacharparenright}{\isacharparenright}\ {\isacharequal}\ {\isacharparenleft}let\ Sn\ {\isacharequal}\ pcp{\isacharunderscore}seq\ C\ S\ {\isacharparenleft}to{\isacharunderscore}nat\ F{\isacharparenright}{\isacharsemicolon}\ \isanewline
\ \ \ \ \ \ \ \ \ \ Sn{\isadigit{1}}\ {\isacharequal}\ insert\ {\isacharparenleft}from{\isacharunderscore}nat\ {\isacharparenleft}to{\isacharunderscore}nat\ F{\isacharparenright}{\isacharparenright}\ Sn\ in\ if\ Sn{\isadigit{1}}\ {\isasymin}\ C\ then\ Sn{\isadigit{1}}\ else\ Sn{\isacharparenright}{\isachardoublequoteclose}\ \isanewline
\ \ \ \ \ \ \isacommand{by}\isamarkupfalse%
\ {\isacharparenleft}simp\ only{\isacharcolon}\ pcp{\isacharunderscore}seq{\isachardot}simps{\isacharparenleft}{\isadigit{2}}{\isacharparenright}{\isacharparenright}\isanewline
\ \ \ \ \isacommand{then}\isamarkupfalse%
\ \isacommand{have}\isamarkupfalse%
\ SucDef{\isacharcolon}{\isachardoublequoteopen}pcp{\isacharunderscore}seq\ C\ S\ {\isacharparenleft}Suc\ {\isacharparenleft}to{\isacharunderscore}nat\ F{\isacharparenright}{\isacharparenright}\ {\isacharequal}\ {\isacharparenleft}if\ insert\ {\isacharparenleft}from{\isacharunderscore}nat\ {\isacharparenleft}to{\isacharunderscore}nat\ F{\isacharparenright}{\isacharparenright}\ {\isacharparenleft}pcp{\isacharunderscore}seq\ C\ S\ {\isacharparenleft}to{\isacharunderscore}nat\ F{\isacharparenright}{\isacharparenright}\ {\isasymin}\ C\ \isanewline
\ \ \ \ \ \ \ \ \ \ then\ insert\ {\isacharparenleft}from{\isacharunderscore}nat\ {\isacharparenleft}to{\isacharunderscore}nat\ F{\isacharparenright}{\isacharparenright}\ {\isacharparenleft}pcp{\isacharunderscore}seq\ C\ S\ {\isacharparenleft}to{\isacharunderscore}nat\ F{\isacharparenright}{\isacharparenright}\ else\ pcp{\isacharunderscore}seq\ C\ S\ {\isacharparenleft}to{\isacharunderscore}nat\ F{\isacharparenright}{\isacharparenright}{\isachardoublequoteclose}\ \isanewline
\ \ \ \ \ \ \isacommand{by}\isamarkupfalse%
\ {\isacharparenleft}simp\ only{\isacharcolon}\ Let{\isacharunderscore}def{\isacharparenright}\isanewline
\ \ \ \ \isacommand{then}\isamarkupfalse%
\ \isacommand{have}\isamarkupfalse%
\ {\isachardoublequoteopen}pcp{\isacharunderscore}seq\ C\ S\ {\isacharparenleft}Suc\ {\isacharparenleft}to{\isacharunderscore}nat\ F{\isacharparenright}{\isacharparenright}\ {\isacharequal}\ insert\ {\isacharparenleft}from{\isacharunderscore}nat\ {\isacharparenleft}to{\isacharunderscore}nat\ F{\isacharparenright}{\isacharparenright}\ {\isacharparenleft}pcp{\isacharunderscore}seq\ C\ S\ {\isacharparenleft}to{\isacharunderscore}nat\ F{\isacharparenright}{\isacharparenright}{\isachardoublequoteclose}\ \isanewline
\ \ \ \ \ \ \isacommand{using}\isamarkupfalse%
\ C\ \isacommand{by}\isamarkupfalse%
\ {\isacharparenleft}simp\ only{\isacharcolon}\ if{\isacharunderscore}True{\isacharparenright}\isanewline
\ \ \ \ \isacommand{then}\isamarkupfalse%
\ \isacommand{have}\isamarkupfalse%
\ {\isachardoublequoteopen}pcp{\isacharunderscore}seq\ C\ S\ {\isacharparenleft}Suc\ {\isacharparenleft}to{\isacharunderscore}nat\ F{\isacharparenright}{\isacharparenright}\ {\isacharequal}\ insert\ F\ {\isacharparenleft}pcp{\isacharunderscore}seq\ C\ S\ {\isacharparenleft}to{\isacharunderscore}nat\ F{\isacharparenright}{\isacharparenright}{\isachardoublequoteclose}\isanewline
\ \ \ \ \ \ \isacommand{by}\isamarkupfalse%
\ {\isacharparenleft}simp\ only{\isacharcolon}\ from{\isacharunderscore}nat{\isacharunderscore}to{\isacharunderscore}nat{\isacharparenright}\isanewline
\ \ \ \ \isacommand{then}\isamarkupfalse%
\ \isacommand{have}\isamarkupfalse%
\ {\isachardoublequoteopen}F\ {\isasymin}\ pcp{\isacharunderscore}seq\ C\ S\ {\isacharparenleft}Suc\ {\isacharparenleft}to{\isacharunderscore}nat\ F{\isacharparenright}{\isacharparenright}{\isachardoublequoteclose}\isanewline
\ \ \ \ \ \ \isacommand{by}\isamarkupfalse%
\ {\isacharparenleft}simp\ only{\isacharcolon}\ insertI{\isadigit{1}}{\isacharparenright}\isanewline
\ \ \ \ \isacommand{show}\isamarkupfalse%
\ {\isachardoublequoteopen}False{\isachardoublequoteclose}\isanewline
\ \ \ \ \ \ \isacommand{using}\isamarkupfalse%
\ {\isacartoucheopen}F\ {\isasymnotin}\ pcp{\isacharunderscore}seq\ C\ S\ {\isacharparenleft}Suc\ {\isacharparenleft}to{\isacharunderscore}nat\ F{\isacharparenright}{\isacharparenright}{\isacartoucheclose}\ {\isacartoucheopen}F\ {\isasymin}\ pcp{\isacharunderscore}seq\ C\ S\ {\isacharparenleft}Suc\ {\isacharparenleft}to{\isacharunderscore}nat\ F{\isacharparenright}{\isacharparenright}{\isacartoucheclose}\ \isacommand{by}\isamarkupfalse%
\ {\isacharparenleft}rule\ notE{\isacharparenright}\isanewline
\ \ \isacommand{qed}\isamarkupfalse%
\isanewline
\ \ \isacommand{have}\isamarkupfalse%
\ {\isachardoublequoteopen}pcp{\isacharunderscore}seq\ C\ S\ {\isacharparenleft}to{\isacharunderscore}nat\ F{\isacharparenright}\ {\isasymsubseteq}\ pcp{\isacharunderscore}lim\ C\ S{\isachardoublequoteclose}\isanewline
\ \ \ \ \isacommand{by}\isamarkupfalse%
\ {\isacharparenleft}rule\ pcp{\isacharunderscore}seq{\isacharunderscore}sub{\isacharparenright}\isanewline
\ \ \isacommand{then}\isamarkupfalse%
\ \isacommand{have}\isamarkupfalse%
\ {\isachardoublequoteopen}pcp{\isacharunderscore}seq\ C\ S\ {\isacharparenleft}to{\isacharunderscore}nat\ F{\isacharparenright}\ {\isasymsubseteq}\ K{\isachardoublequoteclose}\isanewline
\ \ \ \ \isacommand{using}\isamarkupfalse%
\ assms{\isacharparenleft}{\isadigit{4}}{\isacharparenright}\ \isacommand{by}\isamarkupfalse%
\ {\isacharparenleft}rule\ subset{\isacharunderscore}trans{\isacharparenright}\isanewline
\ \ \isacommand{then}\isamarkupfalse%
\ \isacommand{have}\isamarkupfalse%
\ {\isachardoublequoteopen}insert\ F\ {\isacharparenleft}pcp{\isacharunderscore}seq\ C\ S\ {\isacharparenleft}to{\isacharunderscore}nat\ F{\isacharparenright}{\isacharparenright}\ {\isasymsubseteq}\ K{\isachardoublequoteclose}\ \isanewline
\ \ \ \ \isacommand{using}\isamarkupfalse%
\ {\isacartoucheopen}F\ {\isasymin}\ K{\isacartoucheclose}\ \isacommand{by}\isamarkupfalse%
\ {\isacharparenleft}simp\ only{\isacharcolon}\ insert{\isacharunderscore}subset{\isacharparenright}\isanewline
\ \ \isacommand{have}\isamarkupfalse%
\ {\isachardoublequoteopen}{\isasymforall}S\ {\isasymin}\ C{\isachardot}\ {\isasymforall}s{\isasymsubseteq}S{\isachardot}\ s\ {\isasymin}\ C{\isachardoublequoteclose}\isanewline
\ \ \ \ \isacommand{using}\isamarkupfalse%
\ assms{\isacharparenleft}{\isadigit{2}}{\isacharparenright}\ \isacommand{by}\isamarkupfalse%
\ {\isacharparenleft}simp\ only{\isacharcolon}\ subset{\isacharunderscore}closed{\isacharunderscore}def{\isacharparenright}\isanewline
\ \ \isacommand{then}\isamarkupfalse%
\ \isacommand{have}\isamarkupfalse%
\ {\isachardoublequoteopen}{\isasymforall}s\ {\isasymsubseteq}\ K{\isachardot}\ s\ {\isasymin}\ C{\isachardoublequoteclose}\isanewline
\ \ \ \ \isacommand{using}\isamarkupfalse%
\ assms{\isacharparenleft}{\isadigit{3}}{\isacharparenright}\ \isacommand{by}\isamarkupfalse%
\ {\isacharparenleft}rule\ bspec{\isacharparenright}\isanewline
\ \ \isacommand{then}\isamarkupfalse%
\ \isacommand{have}\isamarkupfalse%
\ {\isadigit{3}}{\isacharcolon}{\isachardoublequoteopen}insert\ F\ {\isacharparenleft}pcp{\isacharunderscore}seq\ C\ S\ {\isacharparenleft}to{\isacharunderscore}nat\ F{\isacharparenright}{\isacharparenright}\ {\isasymin}\ C{\isachardoublequoteclose}\ \isanewline
\ \ \ \ \isacommand{using}\isamarkupfalse%
\ {\isacartoucheopen}insert\ F\ {\isacharparenleft}pcp{\isacharunderscore}seq\ C\ S\ {\isacharparenleft}to{\isacharunderscore}nat\ F{\isacharparenright}{\isacharparenright}\ {\isasymsubseteq}\ K{\isacartoucheclose}\ \isacommand{by}\isamarkupfalse%
\ {\isacharparenleft}rule\ sspec{\isacharparenright}\isanewline
\ \ \isacommand{show}\isamarkupfalse%
\ {\isachardoublequoteopen}False{\isachardoublequoteclose}\isanewline
\ \ \ \ \isacommand{using}\isamarkupfalse%
\ {\isadigit{2}}\ {\isadigit{3}}\ \isacommand{by}\isamarkupfalse%
\ {\isacharparenleft}rule\ notE{\isacharparenright}\isanewline
\isacommand{qed}\isamarkupfalse%
%
\endisatagproof
{\isafoldproof}%
%
\isadelimproof
%
\endisadelimproof
%
\begin{isamarkuptext}%
Análogamente a resultados anteriores, veamos su prueba automática.%
\end{isamarkuptext}\isamarkuptrue%
\isacommand{lemma}\isamarkupfalse%
\ cl{\isacharunderscore}max{\isacharcolon}\isanewline
\ \ \isakeyword{assumes}\ c{\isacharcolon}\ {\isachardoublequoteopen}pcp\ C{\isachardoublequoteclose}\isanewline
\ \ \isakeyword{assumes}\ sc{\isacharcolon}\ {\isachardoublequoteopen}subset{\isacharunderscore}closed\ C{\isachardoublequoteclose}\isanewline
\ \ \isakeyword{assumes}\ el{\isacharcolon}\ {\isachardoublequoteopen}K\ {\isasymin}\ C{\isachardoublequoteclose}\isanewline
\ \ \isakeyword{assumes}\ su{\isacharcolon}\ {\isachardoublequoteopen}pcp{\isacharunderscore}lim\ C\ S\ {\isasymsubseteq}\ K{\isachardoublequoteclose}\isanewline
\ \ \isakeyword{shows}\ {\isachardoublequoteopen}pcp{\isacharunderscore}lim\ C\ S\ {\isacharequal}\ K{\isachardoublequoteclose}\ {\isacharparenleft}\isakeyword{is}\ {\isacharquery}e{\isacharparenright}\isanewline
%
\isadelimproof
%
\endisadelimproof
%
\isatagproof
\isacommand{proof}\isamarkupfalse%
\ {\isacharparenleft}rule\ ccontr{\isacharparenright}\isanewline
\ \ \isacommand{assume}\isamarkupfalse%
\ {\isacartoucheopen}{\isasymnot}{\isacharquery}e{\isacartoucheclose}\isanewline
\ \ \isacommand{with}\isamarkupfalse%
\ su\ \isacommand{have}\isamarkupfalse%
\ {\isachardoublequoteopen}pcp{\isacharunderscore}lim\ C\ S\ {\isasymsubset}\ K{\isachardoublequoteclose}\ \isacommand{by}\isamarkupfalse%
\ simp\isanewline
\ \ \isacommand{then}\isamarkupfalse%
\ \isacommand{obtain}\isamarkupfalse%
\ F\ \isakeyword{where}\ e{\isacharcolon}\ {\isachardoublequoteopen}F\ {\isasymin}\ K{\isachardoublequoteclose}\ \isakeyword{and}\ ne{\isacharcolon}\ {\isachardoublequoteopen}F\ {\isasymnotin}\ pcp{\isacharunderscore}lim\ C\ S{\isachardoublequoteclose}\ \isacommand{by}\isamarkupfalse%
\ blast\isanewline
\ \ \isacommand{from}\isamarkupfalse%
\ ne\ \isacommand{have}\isamarkupfalse%
\ {\isachardoublequoteopen}F\ {\isasymnotin}\ pcp{\isacharunderscore}seq\ C\ S\ {\isacharparenleft}Suc\ {\isacharparenleft}to{\isacharunderscore}nat\ F{\isacharparenright}{\isacharparenright}{\isachardoublequoteclose}\ \isacommand{using}\isamarkupfalse%
\ pcp{\isacharunderscore}seq{\isacharunderscore}sub\ \isacommand{by}\isamarkupfalse%
\ fast\isanewline
\ \ \isacommand{hence}\isamarkupfalse%
\ {\isadigit{1}}{\isacharcolon}\ {\isachardoublequoteopen}insert\ F\ {\isacharparenleft}pcp{\isacharunderscore}seq\ C\ S\ {\isacharparenleft}to{\isacharunderscore}nat\ F{\isacharparenright}{\isacharparenright}\ {\isasymnotin}\ C{\isachardoublequoteclose}\ \isacommand{by}\isamarkupfalse%
\ {\isacharparenleft}simp\ add{\isacharcolon}\ Let{\isacharunderscore}def\ split{\isacharcolon}\ if{\isacharunderscore}splits{\isacharparenright}\isanewline
\ \ \isacommand{have}\isamarkupfalse%
\ {\isachardoublequoteopen}insert\ F\ {\isacharparenleft}pcp{\isacharunderscore}seq\ C\ S\ {\isacharparenleft}to{\isacharunderscore}nat\ F{\isacharparenright}{\isacharparenright}\ {\isasymsubseteq}\ K{\isachardoublequoteclose}\ \isacommand{using}\isamarkupfalse%
\ pcp{\isacharunderscore}seq{\isacharunderscore}sub\ e\ su\ \isacommand{by}\isamarkupfalse%
\ blast\isanewline
\ \ \isacommand{hence}\isamarkupfalse%
\ {\isachardoublequoteopen}insert\ F\ {\isacharparenleft}pcp{\isacharunderscore}seq\ C\ S\ {\isacharparenleft}to{\isacharunderscore}nat\ F{\isacharparenright}{\isacharparenright}\ {\isasymin}\ C{\isachardoublequoteclose}\ \isacommand{using}\isamarkupfalse%
\ sc\ \isanewline
\ \ \ \ \isacommand{unfolding}\isamarkupfalse%
\ subset{\isacharunderscore}closed{\isacharunderscore}def\ \isacommand{using}\isamarkupfalse%
\ el\ \isacommand{by}\isamarkupfalse%
\ blast\isanewline
\ \ \isacommand{with}\isamarkupfalse%
\ {\isadigit{1}}\ \isacommand{show}\isamarkupfalse%
\ False\ \isacommand{{\isachardot}{\isachardot}}\isamarkupfalse%
\isanewline
\isacommand{qed}\isamarkupfalse%
%
\endisatagproof
{\isafoldproof}%
%
\isadelimproof
%
\endisadelimproof
%
\begin{isamarkuptext}%
A continuación mostremos un resultado sobre el límite de la sucesión de \isa{{\isadigit{4}}{\isachardot}{\isadigit{1}}{\isachardot}{\isadigit{1}}} que es 
  consecuencia de que dicho límite sea un elemento maximal de la colección que lo define si esta
  verifica la propiedad de consistencia proposicional y es cerrada bajo subconjuntos.
  
  \begin{corolario}
    Sea \isa{C} una colección de conjuntos que verifica la propiedad de consistencia proposicional y
    es cerrada bajo subconjuntos, \isa{S} un conjunto, \isa{{\isacharbraceleft}S\isactrlsub n{\isacharbraceright}} la sucesión de conjuntos de \isa{C} a partir 
    de \isa{S} según la definición \isa{{\isadigit{4}}{\isachardot}{\isadigit{1}}{\isachardot}{\isadigit{1}}} y \isa{F} una fórmula proposicional. Entonces, si\\
    $\{F\} \cup \bigcup_{n = 0}^{\infty} S_{n} \in C$, se verifica que 
    $F \in \bigcup_{n = 0}^{\infty} S_{n}$. 
  \end{corolario}

  \begin{demostracion}
    Como \isa{C} es una colección que verifica la propiedad de consistencia proposicional y es cerrada 
    bajo subconjuntos, se tiene que el límite $\bigcup_{n = 0}^{\infty} S_{n}$ es maximal en \isa{C}. Por 
    lo tanto, si suponemos que $\{F\} \cup \bigcup_{n = 0}^{\infty} S_{n} \in C$, como el límite 
    está contenido en dicho conjunto, se cumple que 
    $\{F\} \cup \bigcup_{n = 0}^{\infty} S_{n} = \bigcup_{n = 0}^{\infty} S_{n}$, luego \isa{F} 
    pertenece al límite, como queríamos demostrar.
  \end{demostracion}

  Veamos su formalización y prueba detallada en Isabelle/HOL.%
\end{isamarkuptext}\isamarkuptrue%
\isacommand{lemma}\isamarkupfalse%
\isanewline
\ \ \isakeyword{assumes}\ {\isachardoublequoteopen}pcp\ C{\isachardoublequoteclose}\isanewline
\ \ \isakeyword{assumes}\ {\isachardoublequoteopen}subset{\isacharunderscore}closed\ C{\isachardoublequoteclose}\isanewline
\ \ \isakeyword{shows}\ {\isachardoublequoteopen}insert\ F\ {\isacharparenleft}pcp{\isacharunderscore}lim\ C\ S{\isacharparenright}\ {\isasymin}\ C\ {\isasymLongrightarrow}\ F\ {\isasymin}\ pcp{\isacharunderscore}lim\ C\ S{\isachardoublequoteclose}\isanewline
%
\isadelimproof
%
\endisadelimproof
%
\isatagproof
\isacommand{proof}\isamarkupfalse%
\ {\isacharminus}\isanewline
\ \ \isacommand{assume}\isamarkupfalse%
\ {\isachardoublequoteopen}insert\ F\ {\isacharparenleft}pcp{\isacharunderscore}lim\ C\ S{\isacharparenright}\ {\isasymin}\ C{\isachardoublequoteclose}\isanewline
\ \ \isacommand{have}\isamarkupfalse%
\ {\isachardoublequoteopen}pcp{\isacharunderscore}lim\ C\ S\ {\isasymsubseteq}\ insert\ F\ {\isacharparenleft}pcp{\isacharunderscore}lim\ C\ S{\isacharparenright}{\isachardoublequoteclose}\isanewline
\ \ \ \ \isacommand{by}\isamarkupfalse%
\ {\isacharparenleft}rule\ subset{\isacharunderscore}insertI{\isacharparenright}\ \isanewline
\ \ \isacommand{have}\isamarkupfalse%
\ {\isachardoublequoteopen}pcp{\isacharunderscore}lim\ C\ S\ {\isacharequal}\ insert\ F\ {\isacharparenleft}pcp{\isacharunderscore}lim\ C\ S{\isacharparenright}{\isachardoublequoteclose}\isanewline
\ \ \ \ \isacommand{using}\isamarkupfalse%
\ assms{\isacharparenleft}{\isadigit{1}}{\isacharparenright}\ assms{\isacharparenleft}{\isadigit{2}}{\isacharparenright}\ {\isacartoucheopen}insert\ F\ {\isacharparenleft}pcp{\isacharunderscore}lim\ C\ S{\isacharparenright}\ {\isasymin}\ C{\isacartoucheclose}\ {\isacartoucheopen}pcp{\isacharunderscore}lim\ C\ S\ {\isasymsubseteq}\ insert\ F\ {\isacharparenleft}pcp{\isacharunderscore}lim\ C\ S{\isacharparenright}{\isacartoucheclose}\ \isacommand{by}\isamarkupfalse%
\ {\isacharparenleft}rule\ cl{\isacharunderscore}max{\isacharparenright}\isanewline
\ \ \isacommand{then}\isamarkupfalse%
\ \isacommand{have}\isamarkupfalse%
\ {\isachardoublequoteopen}insert\ F\ {\isacharparenleft}pcp{\isacharunderscore}lim\ C\ S{\isacharparenright}\ {\isasymsubseteq}\ pcp{\isacharunderscore}lim\ C\ S{\isachardoublequoteclose}\isanewline
\ \ \ \ \isacommand{by}\isamarkupfalse%
\ {\isacharparenleft}rule\ equalityD{\isadigit{2}}{\isacharparenright}\isanewline
\ \ \isacommand{then}\isamarkupfalse%
\ \isacommand{have}\isamarkupfalse%
\ {\isachardoublequoteopen}F\ {\isasymin}\ pcp{\isacharunderscore}lim\ C\ S\ {\isasymand}\ pcp{\isacharunderscore}lim\ C\ S\ {\isasymsubseteq}\ pcp{\isacharunderscore}lim\ C\ S{\isachardoublequoteclose}\isanewline
\ \ \ \ \isacommand{by}\isamarkupfalse%
\ {\isacharparenleft}simp\ only{\isacharcolon}\ insert{\isacharunderscore}subset{\isacharparenright}\isanewline
\ \ \isacommand{thus}\isamarkupfalse%
\ {\isachardoublequoteopen}F\ {\isasymin}\ pcp{\isacharunderscore}lim\ C\ S{\isachardoublequoteclose}\isanewline
\ \ \ \ \isacommand{by}\isamarkupfalse%
\ {\isacharparenleft}rule\ conjunct{\isadigit{1}}{\isacharparenright}\isanewline
\isacommand{qed}\isamarkupfalse%
%
\endisatagproof
{\isafoldproof}%
%
\isadelimproof
%
\endisadelimproof
%
\begin{isamarkuptext}%
Mostremos su demostración automática.%
\end{isamarkuptext}\isamarkuptrue%
\isacommand{lemma}\isamarkupfalse%
\ cl{\isacharunderscore}max{\isacharprime}{\isacharcolon}\isanewline
\ \ \isakeyword{assumes}\ c{\isacharcolon}\ {\isachardoublequoteopen}pcp\ C{\isachardoublequoteclose}\isanewline
\ \ \isakeyword{assumes}\ sc{\isacharcolon}\ {\isachardoublequoteopen}subset{\isacharunderscore}closed\ C{\isachardoublequoteclose}\isanewline
\ \ \isakeyword{shows}\ {\isachardoublequoteopen}insert\ F\ {\isacharparenleft}pcp{\isacharunderscore}lim\ C\ S{\isacharparenright}\ {\isasymin}\ C\ {\isasymLongrightarrow}\ F\ {\isasymin}\ pcp{\isacharunderscore}lim\ C\ S{\isachardoublequoteclose}\isanewline
%
\isadelimproof
\ \ %
\endisadelimproof
%
\isatagproof
\isacommand{using}\isamarkupfalse%
\ cl{\isacharunderscore}max{\isacharbrackleft}OF\ assms{\isacharbrackright}\ \isacommand{by}\isamarkupfalse%
\ blast{\isacharplus}%
\endisatagproof
{\isafoldproof}%
%
\isadelimproof
%
\endisadelimproof
%
\begin{isamarkuptext}%
El siguiente resultado prueba que el límite de la sucesión definida en \isa{{\isadigit{4}}{\isachardot}{\isadigit{1}}{\isachardot}{\isadigit{1}}} es un conjunto
  de Hintikka si la colección que lo define verifica la propiedad de consistencia proposicional, es
  es cerrada bajo subconjuntos y es de carácter finito. Como consecuencia del \isa{teorema\ de\ Hintikka},
  se trata en particular de un conjunto satisfacible. 

  \begin{lema}
    Sea \isa{C} una colección de conjuntos que verifica la propiedad de consistencia proposicional, es
    es cerrada bajo subconjuntos y es de carácter finito. Sea \isa{S\ {\isasymin}\ C} y \isa{{\isacharbraceleft}S\isactrlsub n{\isacharbraceright}} la sucesión de
    conjuntos de \isa{C} a partir de \isa{S} según la definición \isa{{\isadigit{4}}{\isachardot}{\isadigit{1}}{\isachardot}{\isadigit{1}}}. Entonces, el límite de la sucesión
    \isa{{\isacharbraceleft}S\isactrlsub n{\isacharbraceright}} es un conjunto de Hintikka.
  \end{lema}

  \begin{demostracion}
    Para facilitar la lectura, vamos a notar por \isa{L\isactrlsub S\isactrlsub C} al límite de la sucesión \isa{{\isacharbraceleft}S\isactrlsub n{\isacharbraceright}} descrita 
    en el enunciado.

    Por resultados anteriores, como \isa{C} verifica la propiedad de consistencia proposicional, es
    es cerrada bajo subconjuntos y es de carácter finito, se tiene que \isa{L\isactrlsub S\isactrlsub C\ {\isasymin}\ C}. En particular, por 
    verificar la propiedad de consistencia proposicional, por el lema de\\ caracterización de dicha
    propiedad mediante notación uniforme, se cumplen las siguientes condiciones para \isa{L\isactrlsub S\isactrlsub C}:

    \begin{itemize}
      \item \isa{{\isasymbottom}\ {\isasymnotin}\ L\isactrlsub S\isactrlsub C}.
      \item Dada \isa{p} una fórmula atómica cualquiera, no se tiene 
      simultáneamente que\\ \isa{p\ {\isasymin}\ L\isactrlsub S\isactrlsub C} y \isa{{\isasymnot}\ p\ {\isasymin}\ L\isactrlsub S\isactrlsub C}.
      \item Para toda fórmula de tipo \isa{{\isasymalpha}} con componentes \isa{{\isasymalpha}\isactrlsub {\isadigit{1}}} y \isa{{\isasymalpha}\isactrlsub {\isadigit{2}}} tal que \isa{{\isasymalpha}}
      pertenece a \isa{L\isactrlsub S\isactrlsub C}, se tiene que \isa{{\isacharbraceleft}{\isasymalpha}\isactrlsub {\isadigit{1}}{\isacharcomma}{\isasymalpha}\isactrlsub {\isadigit{2}}{\isacharbraceright}\ {\isasymunion}\ L\isactrlsub S\isactrlsub C} pertenece a \isa{C}.
      \item Para toda fórmula de tipo \isa{{\isasymbeta}} con componentes \isa{{\isasymbeta}\isactrlsub {\isadigit{1}}} y \isa{{\isasymbeta}\isactrlsub {\isadigit{2}}} tal que \isa{{\isasymbeta}}
      pertenece a \isa{L\isactrlsub S\isactrlsub C}, se tiene que o bien \isa{{\isacharbraceleft}{\isasymbeta}\isactrlsub {\isadigit{1}}{\isacharbraceright}\ {\isasymunion}\ L\isactrlsub S\isactrlsub C} pertenece a \isa{C} o 
      bien \isa{{\isacharbraceleft}{\isasymbeta}\isactrlsub {\isadigit{2}}{\isacharbraceright}\ {\isasymunion}\ L\isactrlsub S\isactrlsub C} pertenece a \isa{C}.
    \end{itemize}

    Veamos que \isa{L\isactrlsub S\isactrlsub C} es un conjunto de Hintikka probando que cumple las condiciones del
    lema de caracterización de los conjuntos de Hintikka mediante notación uniforme, es decir,
    probaremos que \isa{L\isactrlsub S\isactrlsub C} verifica:

    \begin{itemize}
      \item \isa{{\isasymbottom}\ {\isasymnotin}\ L\isactrlsub S\isactrlsub C}.
      \item Dada \isa{p} una fórmula atómica cualquiera, no se tiene 
      simultáneamente que\\ \isa{p\ {\isasymin}\ L\isactrlsub S\isactrlsub C} y \isa{{\isasymnot}\ p\ {\isasymin}\ L\isactrlsub S\isactrlsub C}.
      \item Para toda fórmula de tipo \isa{{\isasymalpha}} con componentes \isa{{\isasymalpha}\isactrlsub {\isadigit{1}}} y \isa{{\isasymalpha}\isactrlsub {\isadigit{2}}} se verifica 
      que si la fórmula pertenece a \isa{L\isactrlsub S\isactrlsub C}, entonces \isa{{\isasymalpha}\isactrlsub {\isadigit{1}}} y \isa{{\isasymalpha}\isactrlsub {\isadigit{2}}} también.
      \item Para toda fórmula de tipo \isa{{\isasymbeta}} con componentes \isa{{\isasymbeta}\isactrlsub {\isadigit{1}}} y \isa{{\isasymbeta}\isactrlsub {\isadigit{2}}} se verifica 
      que si la fórmula pertenece a \isa{L\isactrlsub S\isactrlsub C}, entonces o bien \isa{{\isasymbeta}\isactrlsub {\isadigit{1}}} pertenece
      a \isa{L\isactrlsub S\isactrlsub C} o bien \isa{{\isasymbeta}\isactrlsub {\isadigit{2}}} pertenece a \isa{L\isactrlsub S\isactrlsub C}.
    \end{itemize} 

    Observemos que las dos primeras condiciones coinciden con las obtenidas anteriormente para \isa{L\isactrlsub S\isactrlsub C} 
    por el lema de caracterización de la propiedad de consistencia proposicional mediante notación
    uniforme. Veamos que, en efecto, se cumplen el resto de condiciones.

    En primer lugar, probemos que para una fórmula \isa{F} de tipo \isa{{\isasymalpha}} y componentes \isa{{\isasymalpha}\isactrlsub {\isadigit{1}}} y \isa{{\isasymalpha}\isactrlsub {\isadigit{2}}} tal que 
    \isa{F\ {\isasymin}\ L\isactrlsub S\isactrlsub C} se verifica que tanto \isa{{\isasymalpha}\isactrlsub {\isadigit{1}}} como \isa{{\isasymalpha}\isactrlsub {\isadigit{2}}} pertenecen a \isa{L\isactrlsub S\isactrlsub C}. Por la tercera condición 
    obtenida anteriormente para \isa{L\isactrlsub S\isactrlsub C} por el lema de caracterización de la propiedad de consistencia 
    proposicional mediante notación uniforme, se cumple que\\ \isa{{\isacharbraceleft}{\isasymalpha}\isactrlsub {\isadigit{1}}{\isacharcomma}{\isasymalpha}\isactrlsub {\isadigit{2}}{\isacharbraceright}\ {\isasymunion}\ L\isactrlsub S\isactrlsub C\ {\isasymin}\ C}. Observemos que
    se verifica \isa{L\isactrlsub S\isactrlsub C\ {\isasymsubseteq}\ {\isacharbraceleft}{\isasymalpha}\isactrlsub {\isadigit{1}}{\isacharcomma}{\isasymalpha}\isactrlsub {\isadigit{2}}{\isacharbraceright}\ {\isasymunion}\ L\isactrlsub S\isactrlsub C}. De este modo, como \isa{C} es una colección con la propiedad de 
    consistencia proposicional y cerrada bajo subconjuntos, por el lema \isa{{\isadigit{4}}{\isachardot}{\isadigit{2}}{\isachardot}{\isadigit{2}}} se tiene que 
    los conjuntos \isa{L\isactrlsub S\isactrlsub C} y \isa{{\isacharbraceleft}{\isasymalpha}\isactrlsub {\isadigit{1}}{\isacharcomma}{\isasymalpha}\isactrlsub {\isadigit{2}}{\isacharbraceright}\ {\isasymunion}\ L\isactrlsub S\isactrlsub C} coinciden. Por tanto, es claro que \isa{{\isasymalpha}\isactrlsub {\isadigit{1}}\ {\isasymin}\ L\isactrlsub S\isactrlsub C} y \isa{{\isasymalpha}\isactrlsub {\isadigit{2}}\ {\isasymin}\ L\isactrlsub S\isactrlsub C}, 
    como queríamos demostrar.

    Por último, demostremos que para una fórmula \isa{F} de tipo \isa{{\isasymbeta}} y componentes \isa{{\isasymbeta}\isactrlsub {\isadigit{1}}} y \isa{{\isasymbeta}\isactrlsub {\isadigit{2}}} tal que
    \isa{F\ {\isasymin}\ L\isactrlsub S\isactrlsub C} se verifica que o bien \isa{{\isasymbeta}\isactrlsub {\isadigit{1}}\ {\isasymin}\ L\isactrlsub S\isactrlsub C} o bien \isa{{\isasymbeta}\isactrlsub {\isadigit{2}}\ {\isasymin}\ L\isactrlsub S\isactrlsub C}. Por la cuarta condición obtenida 
    anteriormente para \isa{L\isactrlsub S\isactrlsub C} por el lema de caracterización de la propiedad de consistencia 
    proposicional mediante notación uniforme, se cumple que o bien\\ \isa{{\isacharbraceleft}{\isasymbeta}\isactrlsub {\isadigit{1}}{\isacharbraceright}\ {\isasymunion}\ L\isactrlsub S\isactrlsub C\ {\isasymin}\ C} o bien 
    \isa{{\isacharbraceleft}{\isasymbeta}\isactrlsub {\isadigit{2}}{\isacharbraceright}\ {\isasymunion}\ L\isactrlsub S\isactrlsub C\ {\isasymin}\ C}. De este modo, si suponemos que \isa{{\isacharbraceleft}{\isasymbeta}\isactrlsub {\isadigit{1}}{\isacharbraceright}\ {\isasymunion}\ L\isactrlsub S\isactrlsub C\ {\isasymin}\ C}, como \isa{C} tiene la propiedad de 
    consistencia proposicional y es cerrada bajo subconjuntos, por el corolario \isa{{\isadigit{4}}{\isachardot}{\isadigit{2}}{\isachardot}{\isadigit{3}}} se tiene 
    que \isa{{\isasymbeta}\isactrlsub {\isadigit{1}}\ {\isasymin}\ L\isactrlsub S\isactrlsub C}. Por tanto, se cumple que o bien \isa{{\isasymbeta}\isactrlsub {\isadigit{1}}\ {\isasymin}\ L\isactrlsub S\isactrlsub C} o bien \isa{{\isasymbeta}\isactrlsub {\isadigit{2}}\ {\isasymin}\ L\isactrlsub S\isactrlsub C}. Si suponemos que 
    \isa{{\isacharbraceleft}{\isasymbeta}\isactrlsub {\isadigit{2}}{\isacharbraceright}\ {\isasymunion}\ L\isactrlsub S\isactrlsub C\ {\isasymin}\ C}, se observa fácilmente que llegamos a la misma conclusión de manera análoga. 
    Por lo tanto, queda probado el resultado.
  \end{demostracion}

  Veamos su formalización y prueba detallada en Isabelle.%
\end{isamarkuptext}\isamarkuptrue%
\isacommand{lemma}\isamarkupfalse%
\isanewline
\ \ \isakeyword{assumes}\ {\isachardoublequoteopen}pcp\ C{\isachardoublequoteclose}\isanewline
\ \ \isakeyword{assumes}\ {\isachardoublequoteopen}subset{\isacharunderscore}closed\ C{\isachardoublequoteclose}\isanewline
\ \ \isakeyword{assumes}\ {\isachardoublequoteopen}finite{\isacharunderscore}character\ C{\isachardoublequoteclose}\isanewline
\ \ \isakeyword{assumes}\ {\isachardoublequoteopen}S\ {\isasymin}\ C{\isachardoublequoteclose}\isanewline
\ \ \isakeyword{shows}\ {\isachardoublequoteopen}Hintikka\ {\isacharparenleft}pcp{\isacharunderscore}lim\ C\ S{\isacharparenright}{\isachardoublequoteclose}\isanewline
%
\isadelimproof
%
\endisadelimproof
%
\isatagproof
\isacommand{proof}\isamarkupfalse%
\ {\isacharparenleft}rule\ Hintikka{\isacharunderscore}alt{\isadigit{2}}{\isacharparenright}\isanewline
\ \ \isacommand{let}\isamarkupfalse%
\ {\isacharquery}cl\ {\isacharequal}\ {\isachardoublequoteopen}pcp{\isacharunderscore}lim\ C\ S{\isachardoublequoteclose}\isanewline
\ \ \isacommand{have}\isamarkupfalse%
\ {\isachardoublequoteopen}{\isacharquery}cl\ {\isasymin}\ C{\isachardoublequoteclose}\isanewline
\ \ \ \ \isacommand{using}\isamarkupfalse%
\ assms{\isacharparenleft}{\isadigit{1}}{\isacharparenright}\ assms{\isacharparenleft}{\isadigit{4}}{\isacharparenright}\ assms{\isacharparenleft}{\isadigit{2}}{\isacharparenright}\ assms{\isacharparenleft}{\isadigit{3}}{\isacharparenright}\ \isacommand{by}\isamarkupfalse%
\ {\isacharparenleft}rule\ pcp{\isacharunderscore}lim{\isacharunderscore}in{\isacharparenright}\isanewline
\ \ \isacommand{have}\isamarkupfalse%
\ {\isachardoublequoteopen}{\isacharparenleft}{\isasymforall}S\ {\isasymin}\ C{\isachardot}\isanewline
\ \ {\isasymbottom}\ {\isasymnotin}\ S\isanewline
{\isasymand}\ {\isacharparenleft}{\isasymforall}k{\isachardot}\ Atom\ k\ {\isasymin}\ S\ {\isasymlongrightarrow}\ \isactrlbold {\isasymnot}\ {\isacharparenleft}Atom\ k{\isacharparenright}\ {\isasymin}\ S\ {\isasymlongrightarrow}\ False{\isacharparenright}\isanewline
{\isasymand}\ {\isacharparenleft}{\isasymforall}F\ G\ H{\isachardot}\ Con\ F\ G\ H\ {\isasymlongrightarrow}\ F\ {\isasymin}\ S\ {\isasymlongrightarrow}\ {\isacharbraceleft}G{\isacharcomma}H{\isacharbraceright}\ {\isasymunion}\ S\ {\isasymin}\ C{\isacharparenright}\isanewline
{\isasymand}\ {\isacharparenleft}{\isasymforall}F\ G\ H{\isachardot}\ Dis\ F\ G\ H\ {\isasymlongrightarrow}\ F\ {\isasymin}\ S\ {\isasymlongrightarrow}\ {\isacharbraceleft}G{\isacharbraceright}\ {\isasymunion}\ S\ {\isasymin}\ C\ {\isasymor}\ {\isacharbraceleft}H{\isacharbraceright}\ {\isasymunion}\ S\ {\isasymin}\ C{\isacharparenright}{\isacharparenright}{\isachardoublequoteclose}\isanewline
\ \ \ \ \isacommand{using}\isamarkupfalse%
\ assms{\isacharparenleft}{\isadigit{1}}{\isacharparenright}\ \isacommand{by}\isamarkupfalse%
\ {\isacharparenleft}rule\ pcp{\isacharunderscore}alt{\isadigit{1}}{\isacharparenright}\isanewline
\ \ \isacommand{then}\isamarkupfalse%
\ \isacommand{have}\isamarkupfalse%
\ d{\isacharcolon}{\isachardoublequoteopen}{\isasymbottom}\ {\isasymnotin}\ {\isacharquery}cl\isanewline
{\isasymand}\ {\isacharparenleft}{\isasymforall}k{\isachardot}\ Atom\ k\ {\isasymin}\ {\isacharquery}cl\ {\isasymlongrightarrow}\ \isactrlbold {\isasymnot}\ {\isacharparenleft}Atom\ k{\isacharparenright}\ {\isasymin}\ {\isacharquery}cl\ {\isasymlongrightarrow}\ False{\isacharparenright}\isanewline
{\isasymand}\ {\isacharparenleft}{\isasymforall}F\ G\ H{\isachardot}\ Con\ F\ G\ H\ {\isasymlongrightarrow}\ F\ {\isasymin}\ {\isacharquery}cl\ {\isasymlongrightarrow}\ {\isacharbraceleft}G{\isacharcomma}H{\isacharbraceright}\ {\isasymunion}\ {\isacharquery}cl\ {\isasymin}\ C{\isacharparenright}\isanewline
{\isasymand}\ {\isacharparenleft}{\isasymforall}F\ G\ H{\isachardot}\ Dis\ F\ G\ H\ {\isasymlongrightarrow}\ F\ {\isasymin}\ {\isacharquery}cl\ {\isasymlongrightarrow}\ {\isacharbraceleft}G{\isacharbraceright}\ {\isasymunion}\ {\isacharquery}cl\ {\isasymin}\ C\ {\isasymor}\ {\isacharbraceleft}H{\isacharbraceright}\ {\isasymunion}\ {\isacharquery}cl\ {\isasymin}\ C{\isacharparenright}{\isachardoublequoteclose}\isanewline
\ \ \ \ \isacommand{using}\isamarkupfalse%
\ {\isacartoucheopen}{\isacharquery}cl\ {\isasymin}\ C{\isacartoucheclose}\ \isacommand{by}\isamarkupfalse%
\ {\isacharparenleft}rule\ bspec{\isacharparenright}\isanewline
\ \ \isacommand{then}\isamarkupfalse%
\ \isacommand{have}\isamarkupfalse%
\ H{\isadigit{1}}{\isacharcolon}{\isachardoublequoteopen}{\isasymbottom}\ {\isasymnotin}\ {\isacharquery}cl{\isachardoublequoteclose}\isanewline
\ \ \ \ \isacommand{by}\isamarkupfalse%
\ {\isacharparenleft}rule\ conjunct{\isadigit{1}}{\isacharparenright}\isanewline
\ \ \isacommand{have}\isamarkupfalse%
\ H{\isadigit{2}}{\isacharcolon}{\isachardoublequoteopen}{\isasymforall}k{\isachardot}\ Atom\ k\ {\isasymin}\ {\isacharquery}cl\ {\isasymlongrightarrow}\ \isactrlbold {\isasymnot}\ {\isacharparenleft}Atom\ k{\isacharparenright}\ {\isasymin}\ {\isacharquery}cl\ {\isasymlongrightarrow}\ False{\isachardoublequoteclose}\isanewline
\ \ \ \ \isacommand{using}\isamarkupfalse%
\ d\ \isacommand{by}\isamarkupfalse%
\ {\isacharparenleft}iprover\ elim{\isacharcolon}\ conjunct{\isadigit{2}}\ conjunct{\isadigit{1}}{\isacharparenright}\isanewline
\ \ \isacommand{have}\isamarkupfalse%
\ Con{\isacharcolon}{\isachardoublequoteopen}{\isasymforall}F\ G\ H{\isachardot}\ Con\ F\ G\ H\ {\isasymlongrightarrow}\ F\ {\isasymin}\ {\isacharquery}cl\ {\isasymlongrightarrow}\ {\isacharbraceleft}G{\isacharcomma}H{\isacharbraceright}\ {\isasymunion}\ {\isacharquery}cl\ {\isasymin}\ C{\isachardoublequoteclose}\isanewline
\ \ \ \ \isacommand{using}\isamarkupfalse%
\ d\ \isacommand{by}\isamarkupfalse%
\ {\isacharparenleft}iprover\ elim{\isacharcolon}\ conjunct{\isadigit{2}}\ conjunct{\isadigit{1}}{\isacharparenright}\isanewline
\ \ \isacommand{have}\isamarkupfalse%
\ H{\isadigit{3}}{\isacharcolon}{\isachardoublequoteopen}{\isasymforall}F\ G\ H{\isachardot}\ Con\ F\ G\ H\ {\isasymlongrightarrow}\ F\ {\isasymin}\ {\isacharquery}cl\ {\isasymlongrightarrow}\ G\ {\isasymin}\ {\isacharquery}cl\ {\isasymand}\ H\ {\isasymin}\ {\isacharquery}cl{\isachardoublequoteclose}\isanewline
\ \ \isacommand{proof}\isamarkupfalse%
\ {\isacharparenleft}rule\ allI{\isacharparenright}{\isacharplus}\isanewline
\ \ \ \ \isacommand{fix}\isamarkupfalse%
\ F\ G\ H\isanewline
\ \ \ \ \isacommand{show}\isamarkupfalse%
\ {\isachardoublequoteopen}Con\ F\ G\ H\ {\isasymlongrightarrow}\ F\ {\isasymin}\ {\isacharquery}cl\ {\isasymlongrightarrow}\ G\ {\isasymin}\ {\isacharquery}cl\ {\isasymand}\ H\ {\isasymin}\ {\isacharquery}cl{\isachardoublequoteclose}\isanewline
\ \ \ \ \isacommand{proof}\isamarkupfalse%
\ {\isacharparenleft}rule\ impI{\isacharparenright}{\isacharplus}\isanewline
\ \ \ \ \ \ \isacommand{assume}\isamarkupfalse%
\ {\isachardoublequoteopen}Con\ F\ G\ H{\isachardoublequoteclose}\isanewline
\ \ \ \ \ \ \isacommand{assume}\isamarkupfalse%
\ {\isachardoublequoteopen}F\ {\isasymin}\ {\isacharquery}cl{\isachardoublequoteclose}\isanewline
\ \ \ \ \ \ \isacommand{have}\isamarkupfalse%
\ {\isachardoublequoteopen}Con\ F\ G\ H\ {\isasymlongrightarrow}\ F\ {\isasymin}\ {\isacharquery}cl\ {\isasymlongrightarrow}\ {\isacharbraceleft}G{\isacharcomma}H{\isacharbraceright}\ {\isasymunion}\ {\isacharquery}cl\ {\isasymin}\ C{\isachardoublequoteclose}\isanewline
\ \ \ \ \ \ \ \ \isacommand{using}\isamarkupfalse%
\ Con\ \isacommand{by}\isamarkupfalse%
\ {\isacharparenleft}iprover\ elim{\isacharcolon}\ allE{\isacharparenright}\isanewline
\ \ \ \ \ \ \isacommand{then}\isamarkupfalse%
\ \isacommand{have}\isamarkupfalse%
\ {\isachardoublequoteopen}F\ {\isasymin}\ {\isacharquery}cl\ {\isasymlongrightarrow}\ {\isacharbraceleft}G{\isacharcomma}H{\isacharbraceright}\ {\isasymunion}\ {\isacharquery}cl\ {\isasymin}\ C{\isachardoublequoteclose}\isanewline
\ \ \ \ \ \ \ \ \isacommand{using}\isamarkupfalse%
\ {\isacartoucheopen}Con\ F\ G\ H{\isacartoucheclose}\ \isacommand{by}\isamarkupfalse%
\ {\isacharparenleft}rule\ mp{\isacharparenright}\isanewline
\ \ \ \ \ \ \isacommand{then}\isamarkupfalse%
\ \isacommand{have}\isamarkupfalse%
\ {\isachardoublequoteopen}{\isacharbraceleft}G{\isacharcomma}H{\isacharbraceright}\ {\isasymunion}\ {\isacharquery}cl\ {\isasymin}\ C{\isachardoublequoteclose}\isanewline
\ \ \ \ \ \ \ \ \isacommand{using}\isamarkupfalse%
\ {\isacartoucheopen}F\ {\isasymin}\ {\isacharquery}cl{\isacartoucheclose}\ \isacommand{by}\isamarkupfalse%
\ {\isacharparenleft}rule\ mp{\isacharparenright}\isanewline
\ \ \ \ \ \ \isacommand{have}\isamarkupfalse%
\ {\isachardoublequoteopen}{\isacharparenleft}insert\ G\ {\isacharparenleft}insert\ H\ {\isacharquery}cl{\isacharparenright}{\isacharparenright}\ {\isacharequal}\ {\isacharbraceleft}G{\isacharcomma}H{\isacharbraceright}\ {\isasymunion}\ {\isacharquery}cl{\isachardoublequoteclose}\isanewline
\ \ \ \ \ \ \ \ \isacommand{by}\isamarkupfalse%
\ {\isacharparenleft}rule\ insertSetElem{\isacharparenright}\isanewline
\ \ \ \ \ \ \isacommand{then}\isamarkupfalse%
\ \isacommand{have}\isamarkupfalse%
\ {\isachardoublequoteopen}{\isacharparenleft}insert\ G\ {\isacharparenleft}insert\ H\ {\isacharquery}cl{\isacharparenright}{\isacharparenright}\ {\isasymin}\ C{\isachardoublequoteclose}\isanewline
\ \ \ \ \ \ \ \ \isacommand{using}\isamarkupfalse%
\ {\isacartoucheopen}{\isacharbraceleft}G{\isacharcomma}H{\isacharbraceright}\ {\isasymunion}\ {\isacharquery}cl\ {\isasymin}\ C{\isacartoucheclose}\ \isacommand{by}\isamarkupfalse%
\ {\isacharparenleft}simp\ only{\isacharcolon}\ {\isacartoucheopen}{\isacharparenleft}insert\ G\ {\isacharparenleft}insert\ H\ {\isacharquery}cl{\isacharparenright}{\isacharparenright}\ {\isacharequal}\ {\isacharbraceleft}G{\isacharcomma}H{\isacharbraceright}\ {\isasymunion}\ {\isacharquery}cl{\isacartoucheclose}{\isacharparenright}\isanewline
\ \ \ \ \ \ \isacommand{have}\isamarkupfalse%
\ {\isachardoublequoteopen}{\isacharquery}cl\ {\isasymsubseteq}\ insert\ H\ {\isacharquery}cl{\isachardoublequoteclose}\isanewline
\ \ \ \ \ \ \ \ \isacommand{by}\isamarkupfalse%
\ {\isacharparenleft}rule\ subset{\isacharunderscore}insertI{\isacharparenright}\isanewline
\ \ \ \ \ \ \isacommand{then}\isamarkupfalse%
\ \isacommand{have}\isamarkupfalse%
\ {\isachardoublequoteopen}{\isacharquery}cl\ {\isasymsubseteq}\ insert\ G\ {\isacharparenleft}insert\ H\ {\isacharquery}cl{\isacharparenright}{\isachardoublequoteclose}\isanewline
\ \ \ \ \ \ \ \ \isacommand{by}\isamarkupfalse%
\ {\isacharparenleft}rule\ subset{\isacharunderscore}insertI{\isadigit{2}}{\isacharparenright}\isanewline
\ \ \ \ \ \ \isacommand{have}\isamarkupfalse%
\ {\isachardoublequoteopen}{\isacharquery}cl\ {\isacharequal}\ insert\ G\ {\isacharparenleft}insert\ H\ {\isacharquery}cl{\isacharparenright}{\isachardoublequoteclose}\ \isanewline
\ \ \ \ \ \ \ \ \isacommand{using}\isamarkupfalse%
\ assms{\isacharparenleft}{\isadigit{1}}{\isacharparenright}\ assms{\isacharparenleft}{\isadigit{2}}{\isacharparenright}\ {\isacartoucheopen}insert\ G\ {\isacharparenleft}insert\ H\ {\isacharquery}cl{\isacharparenright}\ {\isasymin}\ C{\isacartoucheclose}\ {\isacartoucheopen}{\isacharquery}cl\ {\isasymsubseteq}\ insert\ G\ {\isacharparenleft}insert\ H\ {\isacharquery}cl{\isacharparenright}{\isacartoucheclose}\ \isacommand{by}\isamarkupfalse%
\ {\isacharparenleft}rule\ cl{\isacharunderscore}max{\isacharparenright}\isanewline
\ \ \ \ \ \ \isacommand{then}\isamarkupfalse%
\ \isacommand{have}\isamarkupfalse%
\ {\isachardoublequoteopen}insert\ G\ {\isacharparenleft}insert\ H\ {\isacharquery}cl{\isacharparenright}\ {\isasymsubseteq}\ {\isacharquery}cl{\isachardoublequoteclose}\isanewline
\ \ \ \ \ \ \ \ \isacommand{by}\isamarkupfalse%
\ {\isacharparenleft}simp\ only{\isacharcolon}\ equalityD{\isadigit{2}}{\isacharparenright}\isanewline
\ \ \ \ \ \ \isacommand{then}\isamarkupfalse%
\ \isacommand{have}\isamarkupfalse%
\ {\isachardoublequoteopen}G\ {\isasymin}\ {\isacharquery}cl\ {\isasymand}\ insert\ H\ {\isacharquery}cl\ {\isasymsubseteq}\ {\isacharquery}cl{\isachardoublequoteclose}\isanewline
\ \ \ \ \ \ \ \ \isacommand{by}\isamarkupfalse%
\ {\isacharparenleft}simp\ only{\isacharcolon}\ insert{\isacharunderscore}subset{\isacharparenright}\isanewline
\ \ \ \ \ \ \isacommand{then}\isamarkupfalse%
\ \isacommand{have}\isamarkupfalse%
\ {\isachardoublequoteopen}G\ {\isasymin}\ {\isacharquery}cl{\isachardoublequoteclose}\isanewline
\ \ \ \ \ \ \ \ \isacommand{by}\isamarkupfalse%
\ {\isacharparenleft}rule\ conjunct{\isadigit{1}}{\isacharparenright}\isanewline
\ \ \ \ \ \ \isacommand{have}\isamarkupfalse%
\ {\isachardoublequoteopen}insert\ H\ {\isacharquery}cl\ {\isasymsubseteq}\ {\isacharquery}cl{\isachardoublequoteclose}\isanewline
\ \ \ \ \ \ \ \ \isacommand{using}\isamarkupfalse%
\ {\isacartoucheopen}G\ {\isasymin}\ {\isacharquery}cl\ {\isasymand}\ insert\ H\ {\isacharquery}cl\ {\isasymsubseteq}\ {\isacharquery}cl{\isacartoucheclose}\ \isacommand{by}\isamarkupfalse%
\ {\isacharparenleft}rule\ conjunct{\isadigit{2}}{\isacharparenright}\isanewline
\ \ \ \ \ \ \isacommand{then}\isamarkupfalse%
\ \isacommand{have}\isamarkupfalse%
\ {\isachardoublequoteopen}H\ {\isasymin}\ {\isacharquery}cl\ {\isasymand}\ {\isacharquery}cl\ {\isasymsubseteq}\ {\isacharquery}cl{\isachardoublequoteclose}\isanewline
\ \ \ \ \ \ \ \ \isacommand{by}\isamarkupfalse%
\ {\isacharparenleft}simp\ only{\isacharcolon}\ insert{\isacharunderscore}subset{\isacharparenright}\isanewline
\ \ \ \ \ \ \isacommand{then}\isamarkupfalse%
\ \isacommand{have}\isamarkupfalse%
\ {\isachardoublequoteopen}H\ {\isasymin}\ {\isacharquery}cl{\isachardoublequoteclose}\isanewline
\ \ \ \ \ \ \ \ \isacommand{by}\isamarkupfalse%
\ {\isacharparenleft}rule\ conjunct{\isadigit{1}}{\isacharparenright}\isanewline
\ \ \ \ \ \ \isacommand{show}\isamarkupfalse%
\ {\isachardoublequoteopen}G\ {\isasymin}\ {\isacharquery}cl\ {\isasymand}\ H\ {\isasymin}\ {\isacharquery}cl{\isachardoublequoteclose}\isanewline
\ \ \ \ \ \ \ \ \isacommand{using}\isamarkupfalse%
\ {\isacartoucheopen}G\ {\isasymin}\ {\isacharquery}cl{\isacartoucheclose}\ {\isacartoucheopen}H\ {\isasymin}\ {\isacharquery}cl{\isacartoucheclose}\ \isacommand{by}\isamarkupfalse%
\ {\isacharparenleft}rule\ conjI{\isacharparenright}\isanewline
\ \ \ \ \isacommand{qed}\isamarkupfalse%
\isanewline
\ \ \isacommand{qed}\isamarkupfalse%
\isanewline
\ \ \isacommand{have}\isamarkupfalse%
\ Dis{\isacharcolon}{\isachardoublequoteopen}{\isasymforall}F\ G\ H{\isachardot}\ Dis\ F\ G\ H\ {\isasymlongrightarrow}\ F\ {\isasymin}\ {\isacharquery}cl\ {\isasymlongrightarrow}\ {\isacharbraceleft}G{\isacharbraceright}\ {\isasymunion}\ {\isacharquery}cl\ {\isasymin}\ C\ {\isasymor}\ {\isacharbraceleft}H{\isacharbraceright}\ {\isasymunion}\ {\isacharquery}cl\ {\isasymin}\ C{\isachardoublequoteclose}\isanewline
\ \ \ \ \isacommand{using}\isamarkupfalse%
\ d\ \isacommand{by}\isamarkupfalse%
\ {\isacharparenleft}iprover\ elim{\isacharcolon}\ conjunct{\isadigit{2}}\ conjunct{\isadigit{1}}{\isacharparenright}\isanewline
\ \ \isacommand{have}\isamarkupfalse%
\ H{\isadigit{4}}{\isacharcolon}{\isachardoublequoteopen}{\isasymforall}F\ G\ H{\isachardot}\ Dis\ F\ G\ H\ {\isasymlongrightarrow}\ F\ {\isasymin}\ {\isacharquery}cl\ {\isasymlongrightarrow}\ G\ {\isasymin}\ {\isacharquery}cl\ {\isasymor}\ H\ {\isasymin}\ {\isacharquery}cl{\isachardoublequoteclose}\isanewline
\ \ \isacommand{proof}\isamarkupfalse%
\ {\isacharparenleft}rule\ allI{\isacharparenright}{\isacharplus}\isanewline
\ \ \ \ \isacommand{fix}\isamarkupfalse%
\ F\ G\ H\isanewline
\ \ \ \ \isacommand{show}\isamarkupfalse%
\ {\isachardoublequoteopen}Dis\ F\ G\ H\ {\isasymlongrightarrow}\ F\ {\isasymin}\ {\isacharquery}cl\ {\isasymlongrightarrow}\ G\ {\isasymin}\ {\isacharquery}cl\ {\isasymor}\ H\ {\isasymin}\ {\isacharquery}cl{\isachardoublequoteclose}\isanewline
\ \ \ \ \isacommand{proof}\isamarkupfalse%
\ {\isacharparenleft}rule\ impI{\isacharparenright}{\isacharplus}\isanewline
\ \ \ \ \ \ \isacommand{assume}\isamarkupfalse%
\ {\isachardoublequoteopen}Dis\ F\ G\ H{\isachardoublequoteclose}\isanewline
\ \ \ \ \ \ \isacommand{assume}\isamarkupfalse%
\ {\isachardoublequoteopen}F\ {\isasymin}\ {\isacharquery}cl{\isachardoublequoteclose}\isanewline
\ \ \ \ \ \ \isacommand{have}\isamarkupfalse%
\ {\isachardoublequoteopen}Dis\ F\ G\ H\ {\isasymlongrightarrow}\ F\ {\isasymin}\ {\isacharquery}cl\ {\isasymlongrightarrow}\ {\isacharbraceleft}G{\isacharbraceright}\ {\isasymunion}\ {\isacharquery}cl\ {\isasymin}\ C\ {\isasymor}\ {\isacharbraceleft}H{\isacharbraceright}\ {\isasymunion}\ {\isacharquery}cl\ {\isasymin}\ C{\isachardoublequoteclose}\isanewline
\ \ \ \ \ \ \ \ \isacommand{using}\isamarkupfalse%
\ Dis\ \isacommand{by}\isamarkupfalse%
\ {\isacharparenleft}iprover\ elim{\isacharcolon}\ allE{\isacharparenright}\isanewline
\ \ \ \ \ \ \isacommand{then}\isamarkupfalse%
\ \isacommand{have}\isamarkupfalse%
\ {\isachardoublequoteopen}F\ {\isasymin}\ {\isacharquery}cl\ {\isasymlongrightarrow}\ {\isacharbraceleft}G{\isacharbraceright}\ {\isasymunion}\ {\isacharquery}cl\ {\isasymin}\ C\ {\isasymor}\ {\isacharbraceleft}H{\isacharbraceright}\ {\isasymunion}\ {\isacharquery}cl\ {\isasymin}\ C{\isachardoublequoteclose}\isanewline
\ \ \ \ \ \ \ \ \isacommand{using}\isamarkupfalse%
\ {\isacartoucheopen}Dis\ F\ G\ H{\isacartoucheclose}\ \isacommand{by}\isamarkupfalse%
\ {\isacharparenleft}rule\ mp{\isacharparenright}\isanewline
\ \ \ \ \ \ \isacommand{then}\isamarkupfalse%
\ \isacommand{have}\isamarkupfalse%
\ {\isachardoublequoteopen}{\isacharbraceleft}G{\isacharbraceright}\ {\isasymunion}\ {\isacharquery}cl\ {\isasymin}\ C\ {\isasymor}\ {\isacharbraceleft}H{\isacharbraceright}\ {\isasymunion}\ {\isacharquery}cl\ {\isasymin}\ C{\isachardoublequoteclose}\isanewline
\ \ \ \ \ \ \ \ \isacommand{using}\isamarkupfalse%
\ {\isacartoucheopen}F\ {\isasymin}\ {\isacharquery}cl{\isacartoucheclose}\ \isacommand{by}\isamarkupfalse%
\ {\isacharparenleft}rule\ mp{\isacharparenright}\isanewline
\ \ \ \ \ \ \isacommand{thus}\isamarkupfalse%
\ {\isachardoublequoteopen}G\ {\isasymin}\ {\isacharquery}cl\ {\isasymor}\ H\ {\isasymin}\ {\isacharquery}cl{\isachardoublequoteclose}\isanewline
\ \ \ \ \ \ \isacommand{proof}\isamarkupfalse%
\ {\isacharparenleft}rule\ disjE{\isacharparenright}\isanewline
\ \ \ \ \ \ \ \ \isacommand{assume}\isamarkupfalse%
\ {\isachardoublequoteopen}{\isacharbraceleft}G{\isacharbraceright}\ {\isasymunion}\ {\isacharquery}cl\ {\isasymin}\ C{\isachardoublequoteclose}\isanewline
\ \ \ \ \ \ \ \ \isacommand{have}\isamarkupfalse%
\ {\isachardoublequoteopen}insert\ G\ {\isacharquery}cl\ {\isacharequal}\ {\isacharbraceleft}G{\isacharbraceright}\ {\isasymunion}\ {\isacharquery}cl{\isachardoublequoteclose}\isanewline
\ \ \ \ \ \ \ \ \ \ \isacommand{by}\isamarkupfalse%
\ {\isacharparenleft}rule\ insert{\isacharunderscore}is{\isacharunderscore}Un{\isacharparenright}\isanewline
\ \ \ \ \ \ \ \ \isacommand{have}\isamarkupfalse%
\ {\isachardoublequoteopen}insert\ G\ {\isacharquery}cl\ {\isasymin}\ C{\isachardoublequoteclose}\isanewline
\ \ \ \ \ \ \ \ \ \ \isacommand{using}\isamarkupfalse%
\ {\isacartoucheopen}{\isacharbraceleft}G{\isacharbraceright}\ {\isasymunion}\ {\isacharquery}cl\ {\isasymin}\ C{\isacartoucheclose}\ \isacommand{by}\isamarkupfalse%
\ {\isacharparenleft}simp\ only{\isacharcolon}\ {\isacartoucheopen}insert\ G\ {\isacharquery}cl\ {\isacharequal}\ {\isacharbraceleft}G{\isacharbraceright}\ {\isasymunion}\ {\isacharquery}cl{\isacartoucheclose}{\isacharparenright}\isanewline
\ \ \ \ \ \ \ \ \isacommand{have}\isamarkupfalse%
\ {\isachardoublequoteopen}insert\ G\ {\isacharquery}cl\ {\isasymin}\ C\ {\isasymLongrightarrow}\ G\ {\isasymin}\ {\isacharquery}cl{\isachardoublequoteclose}\isanewline
\ \ \ \ \ \ \ \ \ \ \isacommand{using}\isamarkupfalse%
\ assms{\isacharparenleft}{\isadigit{1}}{\isacharparenright}\ assms{\isacharparenleft}{\isadigit{2}}{\isacharparenright}\ \isacommand{by}\isamarkupfalse%
\ {\isacharparenleft}rule\ cl{\isacharunderscore}max{\isacharprime}{\isacharparenright}\isanewline
\ \ \ \ \ \ \ \ \isacommand{then}\isamarkupfalse%
\ \isacommand{have}\isamarkupfalse%
\ {\isachardoublequoteopen}G\ {\isasymin}\ {\isacharquery}cl{\isachardoublequoteclose}\isanewline
\ \ \ \ \ \ \ \ \ \ \isacommand{by}\isamarkupfalse%
\ {\isacharparenleft}simp\ only{\isacharcolon}\ {\isacartoucheopen}insert\ G\ {\isacharquery}cl\ {\isasymin}\ C{\isacartoucheclose}{\isacharparenright}\isanewline
\ \ \ \ \ \ \ \ \isacommand{thus}\isamarkupfalse%
\ {\isachardoublequoteopen}G\ {\isasymin}\ {\isacharquery}cl\ {\isasymor}\ H\ {\isasymin}\ {\isacharquery}cl{\isachardoublequoteclose}\isanewline
\ \ \ \ \ \ \ \ \ \ \isacommand{by}\isamarkupfalse%
\ {\isacharparenleft}rule\ disjI{\isadigit{1}}{\isacharparenright}\isanewline
\ \ \ \ \ \ \isacommand{next}\isamarkupfalse%
\isanewline
\ \ \ \ \ \ \ \ \isacommand{assume}\isamarkupfalse%
\ {\isachardoublequoteopen}{\isacharbraceleft}H{\isacharbraceright}\ {\isasymunion}\ {\isacharquery}cl\ {\isasymin}\ C{\isachardoublequoteclose}\isanewline
\ \ \ \ \ \ \ \ \isacommand{have}\isamarkupfalse%
\ {\isachardoublequoteopen}insert\ H\ {\isacharquery}cl\ {\isacharequal}\ {\isacharbraceleft}H{\isacharbraceright}\ {\isasymunion}\ {\isacharquery}cl{\isachardoublequoteclose}\isanewline
\ \ \ \ \ \ \ \ \ \ \isacommand{by}\isamarkupfalse%
\ {\isacharparenleft}rule\ insert{\isacharunderscore}is{\isacharunderscore}Un{\isacharparenright}\isanewline
\ \ \ \ \ \ \ \ \isacommand{have}\isamarkupfalse%
\ {\isachardoublequoteopen}insert\ H\ {\isacharquery}cl\ {\isasymin}\ C{\isachardoublequoteclose}\isanewline
\ \ \ \ \ \ \ \ \ \ \isacommand{using}\isamarkupfalse%
\ {\isacartoucheopen}{\isacharbraceleft}H{\isacharbraceright}\ {\isasymunion}\ {\isacharquery}cl\ {\isasymin}\ C{\isacartoucheclose}\ \isacommand{by}\isamarkupfalse%
\ {\isacharparenleft}simp\ only{\isacharcolon}\ {\isacartoucheopen}insert\ H\ {\isacharquery}cl\ {\isacharequal}\ {\isacharbraceleft}H{\isacharbraceright}\ {\isasymunion}\ {\isacharquery}cl{\isacartoucheclose}{\isacharparenright}\isanewline
\ \ \ \ \ \ \ \ \isacommand{have}\isamarkupfalse%
\ {\isachardoublequoteopen}insert\ H\ {\isacharquery}cl\ {\isasymin}\ C\ {\isasymLongrightarrow}\ H\ {\isasymin}\ {\isacharquery}cl{\isachardoublequoteclose}\isanewline
\ \ \ \ \ \ \ \ \ \ \isacommand{using}\isamarkupfalse%
\ assms{\isacharparenleft}{\isadigit{1}}{\isacharparenright}\ assms{\isacharparenleft}{\isadigit{2}}{\isacharparenright}\ \isacommand{by}\isamarkupfalse%
\ {\isacharparenleft}rule\ cl{\isacharunderscore}max{\isacharprime}{\isacharparenright}\isanewline
\ \ \ \ \ \ \ \ \isacommand{then}\isamarkupfalse%
\ \isacommand{have}\isamarkupfalse%
\ {\isachardoublequoteopen}H\ {\isasymin}\ {\isacharquery}cl{\isachardoublequoteclose}\isanewline
\ \ \ \ \ \ \ \ \ \ \isacommand{by}\isamarkupfalse%
\ {\isacharparenleft}simp\ only{\isacharcolon}\ {\isacartoucheopen}insert\ H\ {\isacharquery}cl\ {\isasymin}\ C{\isacartoucheclose}{\isacharparenright}\isanewline
\ \ \ \ \ \ \ \ \isacommand{thus}\isamarkupfalse%
\ {\isachardoublequoteopen}G\ {\isasymin}\ {\isacharquery}cl\ {\isasymor}\ H\ {\isasymin}\ {\isacharquery}cl{\isachardoublequoteclose}\isanewline
\ \ \ \ \ \ \ \ \ \ \isacommand{by}\isamarkupfalse%
\ {\isacharparenleft}rule\ disjI{\isadigit{2}}{\isacharparenright}\isanewline
\ \ \ \ \ \ \isacommand{qed}\isamarkupfalse%
\isanewline
\ \ \ \ \isacommand{qed}\isamarkupfalse%
\isanewline
\ \ \isacommand{qed}\isamarkupfalse%
\isanewline
\ \ \isacommand{show}\isamarkupfalse%
\ {\isachardoublequoteopen}{\isasymbottom}\ {\isasymnotin}\ {\isacharquery}cl\ {\isasymand}\isanewline
\ \ \ \ {\isacharparenleft}{\isasymforall}k{\isachardot}\ Atom\ k\ {\isasymin}\ {\isacharquery}cl\ {\isasymlongrightarrow}\ \isactrlbold {\isasymnot}\ {\isacharparenleft}Atom\ k{\isacharparenright}\ {\isasymin}\ {\isacharquery}cl\ {\isasymlongrightarrow}\ False{\isacharparenright}\ {\isasymand}\isanewline
\ \ \ \ {\isacharparenleft}{\isasymforall}F\ G\ H{\isachardot}\ Con\ F\ G\ H\ {\isasymlongrightarrow}\ F\ {\isasymin}\ {\isacharquery}cl\ {\isasymlongrightarrow}\ G\ {\isasymin}\ {\isacharquery}cl\ {\isasymand}\ H\ {\isasymin}\ {\isacharquery}cl{\isacharparenright}\ {\isasymand}\isanewline
\ \ \ \ {\isacharparenleft}{\isasymforall}F\ G\ H{\isachardot}\ Dis\ F\ G\ H\ {\isasymlongrightarrow}\ F\ {\isasymin}\ {\isacharquery}cl\ {\isasymlongrightarrow}\ G\ {\isasymin}\ {\isacharquery}cl\ {\isasymor}\ H\ {\isasymin}\ {\isacharquery}cl{\isacharparenright}{\isachardoublequoteclose}\isanewline
\ \ \ \ \isacommand{using}\isamarkupfalse%
\ H{\isadigit{1}}\ H{\isadigit{2}}\ H{\isadigit{3}}\ H{\isadigit{4}}\ \isacommand{by}\isamarkupfalse%
\ {\isacharparenleft}iprover\ intro{\isacharcolon}\ conjI{\isacharparenright}\isanewline
\isacommand{qed}\isamarkupfalse%
%
\endisatagproof
{\isafoldproof}%
%
\isadelimproof
%
\endisadelimproof
%
\begin{isamarkuptext}%
Del mismo modo, podemos probar el resultado de manera automática como sigue.%
\end{isamarkuptext}\isamarkuptrue%
\isacommand{lemma}\isamarkupfalse%
\ pcp{\isacharunderscore}lim{\isacharunderscore}Hintikka{\isacharcolon}\isanewline
\ \ \isakeyword{assumes}\ c{\isacharcolon}\ {\isachardoublequoteopen}pcp\ C{\isachardoublequoteclose}\isanewline
\ \ \isakeyword{assumes}\ sc{\isacharcolon}\ {\isachardoublequoteopen}subset{\isacharunderscore}closed\ C{\isachardoublequoteclose}\isanewline
\ \ \isakeyword{assumes}\ fc{\isacharcolon}\ {\isachardoublequoteopen}finite{\isacharunderscore}character\ C{\isachardoublequoteclose}\isanewline
\ \ \isakeyword{assumes}\ el{\isacharcolon}\ {\isachardoublequoteopen}S\ {\isasymin}\ C{\isachardoublequoteclose}\isanewline
\ \ \isakeyword{shows}\ {\isachardoublequoteopen}Hintikka\ {\isacharparenleft}pcp{\isacharunderscore}lim\ C\ S{\isacharparenright}{\isachardoublequoteclose}\isanewline
%
\isadelimproof
%
\endisadelimproof
%
\isatagproof
\isacommand{proof}\isamarkupfalse%
\ {\isacharminus}\isanewline
\ \ \isacommand{let}\isamarkupfalse%
\ {\isacharquery}cl\ {\isacharequal}\ {\isachardoublequoteopen}pcp{\isacharunderscore}lim\ C\ S{\isachardoublequoteclose}\isanewline
\ \ \isacommand{have}\isamarkupfalse%
\ {\isachardoublequoteopen}{\isacharquery}cl\ {\isasymin}\ C{\isachardoublequoteclose}\ \isacommand{using}\isamarkupfalse%
\ pcp{\isacharunderscore}lim{\isacharunderscore}in{\isacharbrackleft}OF\ c\ el\ sc\ fc{\isacharbrackright}\ \isacommand{{\isachardot}}\isamarkupfalse%
\isanewline
\ \ \isacommand{from}\isamarkupfalse%
\ c{\isacharbrackleft}unfolded\ pcp{\isacharunderscore}alt{\isacharcomma}\ THEN\ bspec{\isacharcomma}\ OF\ this{\isacharbrackright}\isanewline
\ \ \isacommand{have}\isamarkupfalse%
\ d{\isacharcolon}\ {\isachardoublequoteopen}{\isasymbottom}\ {\isasymnotin}\ {\isacharquery}cl{\isachardoublequoteclose}\isanewline
\ \ \ \ {\isachardoublequoteopen}Atom\ k\ {\isasymin}\ {\isacharquery}cl\ {\isasymLongrightarrow}\ \isactrlbold {\isasymnot}\ {\isacharparenleft}Atom\ k{\isacharparenright}\ {\isasymin}\ {\isacharquery}cl\ {\isasymLongrightarrow}\ False{\isachardoublequoteclose}\isanewline
\ \ \ \ {\isachardoublequoteopen}Con\ F\ G\ H\ {\isasymLongrightarrow}\ F\ {\isasymin}\ {\isacharquery}cl\ {\isasymLongrightarrow}\ insert\ G\ {\isacharparenleft}insert\ H\ {\isacharquery}cl{\isacharparenright}\ {\isasymin}\ C{\isachardoublequoteclose}\isanewline
\ \ \ \ {\isachardoublequoteopen}Dis\ F\ G\ H\ {\isasymLongrightarrow}\ F\ {\isasymin}\ {\isacharquery}cl\ {\isasymLongrightarrow}\ insert\ G\ {\isacharquery}cl\ {\isasymin}\ C\ {\isasymor}\ insert\ H\ {\isacharquery}cl\ {\isasymin}\ C{\isachardoublequoteclose}\isanewline
\ \ \ \ \isakeyword{for}\ k\ F\ G\ H\ \isacommand{by}\isamarkupfalse%
\ force{\isacharplus}\isanewline
\ \ \isacommand{with}\isamarkupfalse%
\ d{\isacharparenleft}{\isadigit{1}}{\isacharcomma}{\isadigit{2}}{\isacharparenright}\ \isacommand{show}\isamarkupfalse%
\ {\isachardoublequoteopen}Hintikka\ {\isacharquery}cl{\isachardoublequoteclose}\ \isacommand{unfolding}\isamarkupfalse%
\ Hintikka{\isacharunderscore}alt\ \isanewline
\ \ \ \ \isacommand{using}\isamarkupfalse%
\ c\ cl{\isacharunderscore}max\ cl{\isacharunderscore}max{\isacharprime}\ d{\isacharparenleft}{\isadigit{4}}{\isacharparenright}\ sc\ \isacommand{by}\isamarkupfalse%
\ blast\isanewline
\isacommand{qed}\isamarkupfalse%
%
\endisatagproof
{\isafoldproof}%
%
\isadelimproof
%
\endisadelimproof
%
\begin{isamarkuptext}%
Finalmente, vamos a demostrar el \isa{teorema\ de\ existencia\ de\ modelo}. Para ello precisaremos de
  un resultado que indica que la satisfacibilidad de conjuntos de fórmulas se hereda por la 
  contención.

  \begin{lema}
    Todo subconjunto de un conjunto de fórmulas satisfacible es satisfacible.
  \end{lema}

  \begin{demostracion}
    Sea \isa{B} un conjunto de fórmulas satisfacible y \isa{A\ {\isasymsubseteq}\ B}. Veamos que \isa{A} es satisfacible.
    Por definición, como \isa{B} es satisfacible, existe una interpretación \isa{{\isasymA}} que es modelo de cada 
    fórmula de \isa{B}. Como \isa{A\ {\isasymsubseteq}\ B}, en particular \isa{{\isasymA}} es modelo de toda fórmula de \isa{A}. Por tanto, 
    \isa{A} es satisfacible, ya que existe una interpretación que es modelo de todas sus fórmulas.
  \end{demostracion}

  Su prueba detallada en Isabelle/HOL es la siguiente.%
\end{isamarkuptext}\isamarkuptrue%
\isacommand{lemma}\isamarkupfalse%
\ sat{\isacharunderscore}mono{\isacharcolon}\isanewline
\ \ \isakeyword{assumes}\ {\isachardoublequoteopen}A\ {\isasymsubseteq}\ B{\isachardoublequoteclose}\isanewline
\ \ \ \ \ \ \ \ \ \ {\isachardoublequoteopen}sat\ B{\isachardoublequoteclose}\isanewline
\ \ \ \ \ \ \ \ \isakeyword{shows}\ {\isachardoublequoteopen}sat\ A{\isachardoublequoteclose}\isanewline
%
\isadelimproof
\ \ %
\endisadelimproof
%
\isatagproof
\isacommand{unfolding}\isamarkupfalse%
\ sat{\isacharunderscore}def\isanewline
\isacommand{proof}\isamarkupfalse%
\ {\isacharminus}\isanewline
\ \isacommand{have}\isamarkupfalse%
\ satB{\isacharcolon}{\isachardoublequoteopen}{\isasymexists}{\isasymA}{\isachardot}\ {\isasymforall}F\ {\isasymin}\ B{\isachardot}\ {\isasymA}\ {\isasymTurnstile}\ F{\isachardoublequoteclose}\isanewline
\ \ \ \isacommand{using}\isamarkupfalse%
\ assms{\isacharparenleft}{\isadigit{2}}{\isacharparenright}\ \isacommand{by}\isamarkupfalse%
\ {\isacharparenleft}simp\ only{\isacharcolon}\ sat{\isacharunderscore}def{\isacharparenright}\isanewline
\ \isacommand{obtain}\isamarkupfalse%
\ {\isasymA}\ \isakeyword{where}\ {\isachardoublequoteopen}{\isasymforall}F\ {\isasymin}\ B{\isachardot}\ {\isasymA}\ {\isasymTurnstile}\ F{\isachardoublequoteclose}\isanewline
\ \ \ \ \isacommand{using}\isamarkupfalse%
\ satB\ \isacommand{by}\isamarkupfalse%
\ {\isacharparenleft}rule\ exE{\isacharparenright}\isanewline
\ \isacommand{have}\isamarkupfalse%
\ {\isachardoublequoteopen}{\isasymforall}F\ {\isasymin}\ A{\isachardot}\ {\isasymA}\ {\isasymTurnstile}\ F{\isachardoublequoteclose}\isanewline
\ \ \isacommand{proof}\isamarkupfalse%
\ {\isacharparenleft}rule\ ballI{\isacharparenright}\isanewline
\ \ \ \ \isacommand{fix}\isamarkupfalse%
\ F\isanewline
\ \ \ \ \isacommand{assume}\isamarkupfalse%
\ {\isachardoublequoteopen}F\ {\isasymin}\ A{\isachardoublequoteclose}\isanewline
\ \ \ \ \isacommand{have}\isamarkupfalse%
\ {\isachardoublequoteopen}F\ {\isasymin}\ A\ {\isasymlongrightarrow}\ F\ {\isasymin}\ B{\isachardoublequoteclose}\isanewline
\ \ \ \ \ \ \isacommand{using}\isamarkupfalse%
\ assms{\isacharparenleft}{\isadigit{1}}{\isacharparenright}\ \isacommand{by}\isamarkupfalse%
\ {\isacharparenleft}rule\ in{\isacharunderscore}mono{\isacharparenright}\isanewline
\ \ \ \ \isacommand{then}\isamarkupfalse%
\ \isacommand{have}\isamarkupfalse%
\ {\isachardoublequoteopen}F\ {\isasymin}\ B{\isachardoublequoteclose}\isanewline
\ \ \ \ \ \ \isacommand{using}\isamarkupfalse%
\ {\isacartoucheopen}F\ {\isasymin}\ A{\isacartoucheclose}\ \isacommand{by}\isamarkupfalse%
\ {\isacharparenleft}rule\ mp{\isacharparenright}\isanewline
\ \ \ \ \isacommand{show}\isamarkupfalse%
\ {\isachardoublequoteopen}{\isasymA}\ {\isasymTurnstile}\ F{\isachardoublequoteclose}\isanewline
\ \ \ \ \ \ \isacommand{using}\isamarkupfalse%
\ {\isacartoucheopen}{\isasymforall}F\ {\isasymin}\ B{\isachardot}\ {\isasymA}\ {\isasymTurnstile}\ F{\isacartoucheclose}\ {\isacartoucheopen}F\ {\isasymin}\ B{\isacartoucheclose}\ \isacommand{by}\isamarkupfalse%
\ {\isacharparenleft}rule\ bspec{\isacharparenright}\isanewline
\ \ \isacommand{qed}\isamarkupfalse%
\isanewline
\ \ \isacommand{thus}\isamarkupfalse%
\ {\isachardoublequoteopen}{\isasymexists}{\isasymA}{\isachardot}\ {\isasymforall}F\ {\isasymin}\ A{\isachardot}\ {\isasymA}\ {\isasymTurnstile}\ F{\isachardoublequoteclose}\isanewline
\ \ \ \ \isacommand{by}\isamarkupfalse%
\ {\isacharparenleft}simp\ only{\isacharcolon}\ exI{\isacharparenright}\isanewline
\isacommand{qed}\isamarkupfalse%
%
\endisatagproof
{\isafoldproof}%
%
\isadelimproof
%
\endisadelimproof
%
\begin{isamarkuptext}%
De este modo, procedamos finalmente con la demostración del teorema.

  \begin{teorema}[Teorema de Existencia de Modelo]
    Todo conjunto de fórmulas perteneciente a una colección que verifique la propiedad de
    consistencia proposicional es satisfacible. 
  \end{teorema}

  \begin{demostracion}
    Sea \isa{C} una colección de conjuntos de fórmulas proposicionales que verifique la propiedad de 
    consistencia proposicional y \isa{S\ {\isasymin}\ C}. Vamos a probar que \isa{S} es satisfacible.

    En primer lugar, como \isa{C} verifica la propiedad de consistencia proposicional, por el lema 
    \isa{{\isadigit{3}}{\isachardot}{\isadigit{0}}{\isachardot}{\isadigit{3}}} podemos extenderla a una colección \isa{C{\isacharprime}} que también verifique la propiedad y
    sea cerrada bajo subconjuntos. A su vez, por el lema \isa{{\isadigit{3}}{\isachardot}{\isadigit{0}}{\isachardot}{\isadigit{5}}}, como la extensión 
    \isa{C{\isacharprime}} es una colección con la propiedad de consistencia proposicional y cerrada bajo 
    subconjuntos, podemos extenderla a otra colección \isa{C{\isacharprime}{\isacharprime}} que también verifica la propiedad de
    consistencia proposicional y sea de carácter finito. De este modo, por la transitividad de la 
    contención, es claro que \isa{C{\isacharprime}{\isacharprime}} es una extensión de \isa{C}, luego \isa{S\ {\isasymin}\ C{\isacharprime}{\isacharprime}} por hipótesis. 
    Por otro lado, por el lema \isa{{\isadigit{3}}{\isachardot}{\isadigit{0}}{\isachardot}{\isadigit{4}}}, como \isa{C{\isacharprime}{\isacharprime}} es de carácter finito, se tiene que es 
    cerrada bajo subconjuntos. 

    En suma, \isa{C{\isacharprime}{\isacharprime}} es una extensión de \isa{C} que verifica la propiedad de consistencia proposicional, 
    es cerrada bajo subconjuntos y es de carácter finito. Luego, por el lema \isa{{\isadigit{4}}{\isachardot}{\isadigit{2}}{\isachardot}{\isadigit{4}}}, el límite de 
    la sucesión \isa{{\isacharbraceleft}S\isactrlsub n{\isacharbraceright}} de conjuntos de \isa{C{\isacharprime}{\isacharprime}} a partir de \isa{S} según la definición \isa{{\isadigit{4}}{\isachardot}{\isadigit{1}}{\isachardot}{\isadigit{1}}} es un 
    conjunto de Hintikka. Por tanto, por el \isa{teorema\ de\ Hintikka}, se trata de un conjunto 
    satisfacible. 

    Finalmente, puesto que para todo \isa{n\ {\isasymin}\ {\isasymnat}} se tiene que \isa{S\isactrlsub n} está contenido en el límite, en 
    particular el conjunto \isa{S\isactrlsub {\isadigit{0}}} está contenido en él. Por definición de la sucesión, dicho conjunto 
    coincide con \isa{S}. Por tanto, como \isa{S} está contenido en el límite que es un conjunto 
    satisfacible, queda demostrada la satisfacibilidad de \isa{S}.
  \end{demostracion}

  \comentario{Tal vez sería buena idea hacer un grafo similar al de ex3.}

  Mostremos su formalización y demostración detallada en Isabelle.%
\end{isamarkuptext}\isamarkuptrue%
\isacommand{theorem}\isamarkupfalse%
\isanewline
\ \ \isakeyword{fixes}\ S\ {\isacharcolon}{\isacharcolon}\ {\isachardoublequoteopen}{\isacharprime}a\ {\isacharcolon}{\isacharcolon}\ countable\ formula\ set{\isachardoublequoteclose}\isanewline
\ \ \isakeyword{assumes}\ {\isachardoublequoteopen}pcp\ C{\isachardoublequoteclose}\isanewline
\ \ \isakeyword{assumes}\ {\isachardoublequoteopen}S\ {\isasymin}\ C{\isachardoublequoteclose}\isanewline
\ \ \isakeyword{shows}\ {\isachardoublequoteopen}sat\ S{\isachardoublequoteclose}\isanewline
%
\isadelimproof
%
\endisadelimproof
%
\isatagproof
\isacommand{proof}\isamarkupfalse%
\ {\isacharminus}\isanewline
\ \ \isacommand{have}\isamarkupfalse%
\ {\isachardoublequoteopen}pcp\ C\ {\isasymLongrightarrow}\ {\isasymexists}C{\isacharprime}{\isachardot}\ C\ {\isasymsubseteq}\ C{\isacharprime}\ {\isasymand}\ pcp\ C{\isacharprime}\ {\isasymand}\ subset{\isacharunderscore}closed\ C{\isacharprime}{\isachardoublequoteclose}\isanewline
\ \ \ \ \isacommand{by}\isamarkupfalse%
\ {\isacharparenleft}rule\ ex{\isadigit{1}}{\isacharparenright}\isanewline
\ \ \isacommand{then}\isamarkupfalse%
\ \isacommand{have}\isamarkupfalse%
\ E{\isadigit{1}}{\isacharcolon}{\isachardoublequoteopen}{\isasymexists}C{\isacharprime}{\isachardot}\ C\ {\isasymsubseteq}\ C{\isacharprime}\ {\isasymand}\ pcp\ C{\isacharprime}\ {\isasymand}\ subset{\isacharunderscore}closed\ C{\isacharprime}{\isachardoublequoteclose}\isanewline
\ \ \ \ \isacommand{by}\isamarkupfalse%
\ {\isacharparenleft}simp\ only{\isacharcolon}\ assms{\isacharparenleft}{\isadigit{1}}{\isacharparenright}{\isacharparenright}\isanewline
\ \ \isacommand{obtain}\isamarkupfalse%
\ Ce{\isacharprime}\ \isakeyword{where}\ H{\isadigit{1}}{\isacharcolon}{\isachardoublequoteopen}C\ {\isasymsubseteq}\ Ce{\isacharprime}\ {\isasymand}\ pcp\ Ce{\isacharprime}\ {\isasymand}\ subset{\isacharunderscore}closed\ Ce{\isacharprime}{\isachardoublequoteclose}\isanewline
\ \ \ \ \isacommand{using}\isamarkupfalse%
\ E{\isadigit{1}}\ \isacommand{by}\isamarkupfalse%
\ {\isacharparenleft}rule\ exE{\isacharparenright}\isanewline
\ \ \isacommand{have}\isamarkupfalse%
\ {\isachardoublequoteopen}C\ {\isasymsubseteq}\ Ce{\isacharprime}{\isachardoublequoteclose}\isanewline
\ \ \ \ \isacommand{using}\isamarkupfalse%
\ H{\isadigit{1}}\ \isacommand{by}\isamarkupfalse%
\ {\isacharparenleft}rule\ conjunct{\isadigit{1}}{\isacharparenright}\isanewline
\ \ \isacommand{have}\isamarkupfalse%
\ {\isachardoublequoteopen}pcp\ Ce{\isacharprime}{\isachardoublequoteclose}\isanewline
\ \ \ \ \isacommand{using}\isamarkupfalse%
\ H{\isadigit{1}}\ \isacommand{by}\isamarkupfalse%
\ {\isacharparenleft}iprover\ elim{\isacharcolon}\ conjunct{\isadigit{2}}\ conjunct{\isadigit{1}}{\isacharparenright}\isanewline
\ \ \isacommand{have}\isamarkupfalse%
\ {\isachardoublequoteopen}subset{\isacharunderscore}closed\ Ce{\isacharprime}{\isachardoublequoteclose}\isanewline
\ \ \ \ \isacommand{using}\isamarkupfalse%
\ H{\isadigit{1}}\ \isacommand{by}\isamarkupfalse%
\ {\isacharparenleft}iprover\ elim{\isacharcolon}\ conjunct{\isadigit{2}}\ conjunct{\isadigit{1}}{\isacharparenright}\isanewline
\ \ \isacommand{have}\isamarkupfalse%
\ E{\isadigit{2}}{\isacharcolon}{\isachardoublequoteopen}{\isasymexists}Ce{\isachardot}\ Ce{\isacharprime}\ {\isasymsubseteq}\ Ce\ {\isasymand}\ pcp\ Ce\ {\isasymand}\ finite{\isacharunderscore}character\ Ce{\isachardoublequoteclose}\isanewline
\ \ \ \ \isacommand{using}\isamarkupfalse%
\ {\isacartoucheopen}pcp\ Ce{\isacharprime}{\isacartoucheclose}\ {\isacartoucheopen}subset{\isacharunderscore}closed\ Ce{\isacharprime}{\isacartoucheclose}\ \isacommand{by}\isamarkupfalse%
\ {\isacharparenleft}rule\ ex{\isadigit{3}}{\isacharparenright}\isanewline
\ \ \isacommand{obtain}\isamarkupfalse%
\ Ce\ \isakeyword{where}\ H{\isadigit{2}}{\isacharcolon}{\isachardoublequoteopen}Ce{\isacharprime}\ {\isasymsubseteq}\ Ce\ {\isasymand}\ pcp\ Ce\ {\isasymand}\ finite{\isacharunderscore}character\ Ce{\isachardoublequoteclose}\isanewline
\ \ \ \ \isacommand{using}\isamarkupfalse%
\ E{\isadigit{2}}\ \isacommand{by}\isamarkupfalse%
\ {\isacharparenleft}rule\ exE{\isacharparenright}\isanewline
\ \ \isacommand{have}\isamarkupfalse%
\ {\isachardoublequoteopen}Ce{\isacharprime}\ {\isasymsubseteq}\ Ce{\isachardoublequoteclose}\isanewline
\ \ \ \ \isacommand{using}\isamarkupfalse%
\ H{\isadigit{2}}\ \isacommand{by}\isamarkupfalse%
\ {\isacharparenleft}rule\ conjunct{\isadigit{1}}{\isacharparenright}\isanewline
\ \ \isacommand{then}\isamarkupfalse%
\ \isacommand{have}\isamarkupfalse%
\ Subset{\isacharcolon}{\isachardoublequoteopen}C\ {\isasymsubseteq}\ Ce{\isachardoublequoteclose}\isanewline
\ \ \ \ \isacommand{using}\isamarkupfalse%
\ {\isacartoucheopen}C\ {\isasymsubseteq}\ Ce{\isacharprime}{\isacartoucheclose}\ \isacommand{by}\isamarkupfalse%
\ {\isacharparenleft}simp\ only{\isacharcolon}\ subset{\isacharunderscore}trans{\isacharparenright}\isanewline
\ \ \isacommand{have}\isamarkupfalse%
\ Pcp{\isacharcolon}{\isachardoublequoteopen}pcp\ Ce{\isachardoublequoteclose}\isanewline
\ \ \ \ \isacommand{using}\isamarkupfalse%
\ H{\isadigit{2}}\ \isacommand{by}\isamarkupfalse%
\ {\isacharparenleft}iprover\ elim{\isacharcolon}\ conjunct{\isadigit{2}}\ conjunct{\isadigit{1}}{\isacharparenright}\isanewline
\ \ \isacommand{have}\isamarkupfalse%
\ FC{\isacharcolon}{\isachardoublequoteopen}finite{\isacharunderscore}character\ Ce{\isachardoublequoteclose}\isanewline
\ \ \ \ \isacommand{using}\isamarkupfalse%
\ H{\isadigit{2}}\ \isacommand{by}\isamarkupfalse%
\ {\isacharparenleft}iprover\ elim{\isacharcolon}\ conjunct{\isadigit{2}}\ conjunct{\isadigit{1}}{\isacharparenright}\isanewline
\ \ \isacommand{then}\isamarkupfalse%
\ \isacommand{have}\isamarkupfalse%
\ SC{\isacharcolon}{\isachardoublequoteopen}subset{\isacharunderscore}closed\ Ce{\isachardoublequoteclose}\isanewline
\ \ \ \ \isacommand{by}\isamarkupfalse%
\ {\isacharparenleft}rule\ ex{\isadigit{2}}{\isacharparenright}\isanewline
\ \ \isacommand{have}\isamarkupfalse%
\ {\isachardoublequoteopen}S\ {\isasymin}\ C\ {\isasymlongrightarrow}\ S\ {\isasymin}\ Ce{\isachardoublequoteclose}\isanewline
\ \ \ \ \isacommand{using}\isamarkupfalse%
\ {\isacartoucheopen}C\ {\isasymsubseteq}\ Ce{\isacartoucheclose}\ \isacommand{by}\isamarkupfalse%
\ {\isacharparenleft}rule\ in{\isacharunderscore}mono{\isacharparenright}\isanewline
\ \ \isacommand{then}\isamarkupfalse%
\ \isacommand{have}\isamarkupfalse%
\ {\isachardoublequoteopen}S\ {\isasymin}\ Ce{\isachardoublequoteclose}\ \isanewline
\ \ \ \ \isacommand{using}\isamarkupfalse%
\ assms{\isacharparenleft}{\isadigit{2}}{\isacharparenright}\ \isacommand{by}\isamarkupfalse%
\ {\isacharparenleft}rule\ mp{\isacharparenright}\isanewline
\ \ \isacommand{have}\isamarkupfalse%
\ {\isachardoublequoteopen}Hintikka\ {\isacharparenleft}pcp{\isacharunderscore}lim\ Ce\ S{\isacharparenright}{\isachardoublequoteclose}\isanewline
\ \ \ \ \isacommand{using}\isamarkupfalse%
\ Pcp\ SC\ FC\ {\isacartoucheopen}S\ {\isasymin}\ Ce{\isacartoucheclose}\ \isacommand{by}\isamarkupfalse%
\ {\isacharparenleft}rule\ pcp{\isacharunderscore}lim{\isacharunderscore}Hintikka{\isacharparenright}\isanewline
\ \ \isacommand{then}\isamarkupfalse%
\ \isacommand{have}\isamarkupfalse%
\ {\isachardoublequoteopen}sat\ {\isacharparenleft}pcp{\isacharunderscore}lim\ Ce\ S{\isacharparenright}{\isachardoublequoteclose}\isanewline
\ \ \ \ \isacommand{by}\isamarkupfalse%
\ {\isacharparenleft}rule\ Hintikkaslemma{\isacharparenright}\isanewline
\ \ \isacommand{have}\isamarkupfalse%
\ {\isachardoublequoteopen}pcp{\isacharunderscore}seq\ Ce\ S\ {\isadigit{0}}\ {\isacharequal}\ S{\isachardoublequoteclose}\isanewline
\ \ \ \ \isacommand{by}\isamarkupfalse%
\ {\isacharparenleft}simp\ only{\isacharcolon}\ pcp{\isacharunderscore}seq{\isachardot}simps{\isacharparenleft}{\isadigit{1}}{\isacharparenright}{\isacharparenright}\isanewline
\ \ \isacommand{have}\isamarkupfalse%
\ {\isachardoublequoteopen}pcp{\isacharunderscore}seq\ Ce\ S\ {\isadigit{0}}\ {\isasymsubseteq}\ pcp{\isacharunderscore}lim\ Ce\ S{\isachardoublequoteclose}\isanewline
\ \ \ \ \isacommand{by}\isamarkupfalse%
\ {\isacharparenleft}rule\ pcp{\isacharunderscore}seq{\isacharunderscore}sub{\isacharparenright}\isanewline
\ \ \isacommand{then}\isamarkupfalse%
\ \isacommand{have}\isamarkupfalse%
\ {\isachardoublequoteopen}S\ {\isasymsubseteq}\ pcp{\isacharunderscore}lim\ Ce\ S{\isachardoublequoteclose}\isanewline
\ \ \ \ \isacommand{by}\isamarkupfalse%
\ {\isacharparenleft}simp\ only{\isacharcolon}\ {\isacartoucheopen}pcp{\isacharunderscore}seq\ Ce\ S\ {\isadigit{0}}\ {\isacharequal}\ S{\isacartoucheclose}{\isacharparenright}\isanewline
\ \ \isacommand{thus}\isamarkupfalse%
\ {\isachardoublequoteopen}sat\ S{\isachardoublequoteclose}\isanewline
\ \ \ \ \isacommand{using}\isamarkupfalse%
\ {\isacartoucheopen}sat\ {\isacharparenleft}pcp{\isacharunderscore}lim\ Ce\ S{\isacharparenright}{\isacartoucheclose}\ \isacommand{by}\isamarkupfalse%
\ {\isacharparenleft}rule\ sat{\isacharunderscore}mono{\isacharparenright}\isanewline
\isacommand{qed}\isamarkupfalse%
%
\endisatagproof
{\isafoldproof}%
%
\isadelimproof
%
\endisadelimproof
%
\begin{isamarkuptext}%
Finalmente, demostremos el teorema de manera automática.%
\end{isamarkuptext}\isamarkuptrue%
\isacommand{theorem}\isamarkupfalse%
\ pcp{\isacharunderscore}sat{\isacharcolon}\isanewline
\ \ \isakeyword{fixes}\ S\ {\isacharcolon}{\isacharcolon}\ {\isachardoublequoteopen}{\isacharprime}a\ {\isacharcolon}{\isacharcolon}\ countable\ formula\ set{\isachardoublequoteclose}\isanewline
\ \ \isakeyword{assumes}\ c{\isacharcolon}\ {\isachardoublequoteopen}pcp\ C{\isachardoublequoteclose}\isanewline
\ \ \isakeyword{assumes}\ el{\isacharcolon}\ {\isachardoublequoteopen}S\ {\isasymin}\ C{\isachardoublequoteclose}\isanewline
\ \ \isakeyword{shows}\ {\isachardoublequoteopen}sat\ S{\isachardoublequoteclose}\isanewline
%
\isadelimproof
%
\endisadelimproof
%
\isatagproof
\isacommand{proof}\isamarkupfalse%
\ {\isacharminus}\isanewline
\ \ \isacommand{from}\isamarkupfalse%
\ c\ \isacommand{obtain}\isamarkupfalse%
\ Ce\ \isakeyword{where}\ \isanewline
\ \ \ \ \ \ {\isachardoublequoteopen}C\ {\isasymsubseteq}\ Ce{\isachardoublequoteclose}\ {\isachardoublequoteopen}pcp\ Ce{\isachardoublequoteclose}\ {\isachardoublequoteopen}subset{\isacharunderscore}closed\ Ce{\isachardoublequoteclose}\ {\isachardoublequoteopen}finite{\isacharunderscore}character\ Ce{\isachardoublequoteclose}\ \isanewline
\ \ \ \ \ \ \isacommand{using}\isamarkupfalse%
\ ex{\isadigit{1}}{\isacharbrackleft}\isakeyword{where}\ {\isacharprime}a{\isacharequal}{\isacharprime}a{\isacharbrackright}\ ex{\isadigit{2}}{\isacharbrackleft}\isakeyword{where}\ {\isacharprime}a{\isacharequal}{\isacharprime}a{\isacharbrackright}\ ex{\isadigit{3}}{\isacharbrackleft}\isakeyword{where}\ {\isacharprime}a{\isacharequal}{\isacharprime}a{\isacharbrackright}\isanewline
\ \ \ \ \isacommand{by}\isamarkupfalse%
\ {\isacharparenleft}meson\ dual{\isacharunderscore}order{\isachardot}trans\ ex{\isadigit{2}}{\isacharparenright}\isanewline
\ \ \isacommand{have}\isamarkupfalse%
\ {\isachardoublequoteopen}S\ {\isasymin}\ Ce{\isachardoublequoteclose}\ \isacommand{using}\isamarkupfalse%
\ {\isacartoucheopen}C\ {\isasymsubseteq}\ Ce{\isacartoucheclose}\ el\ \isacommand{{\isachardot}{\isachardot}}\isamarkupfalse%
\isanewline
\ \ \isacommand{with}\isamarkupfalse%
\ pcp{\isacharunderscore}lim{\isacharunderscore}Hintikka\ {\isacartoucheopen}pcp\ Ce{\isacartoucheclose}\ {\isacartoucheopen}subset{\isacharunderscore}closed\ Ce{\isacartoucheclose}\ {\isacartoucheopen}finite{\isacharunderscore}character\ Ce{\isacartoucheclose}\isanewline
\ \ \isacommand{have}\isamarkupfalse%
\ \ {\isachardoublequoteopen}Hintikka\ {\isacharparenleft}pcp{\isacharunderscore}lim\ Ce\ S{\isacharparenright}{\isachardoublequoteclose}\ \isacommand{{\isachardot}}\isamarkupfalse%
\isanewline
\ \ \isacommand{with}\isamarkupfalse%
\ Hintikkaslemma\ \isacommand{have}\isamarkupfalse%
\ {\isachardoublequoteopen}sat\ {\isacharparenleft}pcp{\isacharunderscore}lim\ Ce\ S{\isacharparenright}{\isachardoublequoteclose}\ \isacommand{{\isachardot}}\isamarkupfalse%
\isanewline
\ \ \isacommand{moreover}\isamarkupfalse%
\ \isacommand{have}\isamarkupfalse%
\ {\isachardoublequoteopen}S\ {\isasymsubseteq}\ pcp{\isacharunderscore}lim\ Ce\ S{\isachardoublequoteclose}\ \isanewline
\ \ \ \ \isacommand{using}\isamarkupfalse%
\ pcp{\isacharunderscore}seq{\isachardot}simps{\isacharparenleft}{\isadigit{1}}{\isacharparenright}\ pcp{\isacharunderscore}seq{\isacharunderscore}sub\ \isacommand{by}\isamarkupfalse%
\ fast\isanewline
\ \ \isacommand{ultimately}\isamarkupfalse%
\ \isacommand{show}\isamarkupfalse%
\ {\isacharquery}thesis\ \isacommand{unfolding}\isamarkupfalse%
\ sat{\isacharunderscore}def\ \isacommand{by}\isamarkupfalse%
\ fast\isanewline
\isacommand{qed}\isamarkupfalse%
%
\endisatagproof
{\isafoldproof}%
%
\isadelimproof
%
\endisadelimproof
%
\isadelimdocument
%
\endisadelimdocument
%
\isatagdocument
%
\isamarkupsection{Teorema de Compacidad%
}
\isamarkuptrue%
%
\endisatagdocument
{\isafolddocument}%
%
\isadelimdocument
%
\endisadelimdocument
%
\begin{isamarkuptext}%
En esta sección vamos demostrar el \isa{Teorema\ de\ Compacidad} para la lógica proposicional
  como consecuencia del \isa{Teorema\ de\ Existencia\ de\ Modelo}.

  \begin{teorema}[Teorema de Compacidad]
    Todo conjunto de fórmulas finitamente satisfacible es satisfacible.
  \end{teorema}

  Para su demostración consideraremos la colección formada por los conjuntos de fórmulas finitamente 
  satisfacibles. Probaremos que dicha colección verifica la propiedad de consistencia proposicional
  y, por el \isa{Teorema\ de\ Existencia\ de\ Modelo}, todo conjunto perteneciente a ella será
  satisfacible.

  Mostremos, previamente, dos resultados sobre subconjuntos finitos que emplearemos en la 
  demostración del teorema.

  \begin{lema}
    Sea un conjunto de la forma \isa{{\isacharbraceleft}a{\isacharbraceright}\ {\isasymunion}\ B} y \isa{S} un subconjunto finito suyo. Entonces,
    existe un subconjunto finito \isa{S{\isacharprime}} de \isa{B} tal que o bien \isa{S\ {\isacharequal}\ {\isacharbraceleft}a{\isacharbraceright}\ {\isasymunion}\ S{\isacharprime}} o bien \isa{S\ {\isacharequal}\ S{\isacharprime}}.
  \end{lema}

  \begin{demostracion}
    La prueba del resultado se realiza por inducción en la estructura recursiva de los conjuntos 
    finitos.

    En primer lugar, probemos el caso base correspondiente al conjunto vacío. Si consideramos 
    \isa{S\ {\isacharequal}\ {\isacharbraceleft}{\isacharbraceright}}, tomando también \isa{S{\isacharprime}\ {\isacharequal}\ {\isacharbraceleft}{\isacharbraceright}} es claro que verifica que es subconjunto finito de \isa{B}
    y o bien \isa{S\ {\isacharequal}\ {\isacharbraceleft}a{\isacharbraceright}\ {\isasymunion}\ S{\isacharprime}} o bien \isa{S\ {\isacharequal}\ S{\isacharprime}}.

    Veamos el caso de inducción. Sea \isa{S} un conjunto finito verificando la hipótesis de inducción:
    
    \isa{{\isacharparenleft}HI{\isacharparenright}{\isacharcolon}\ Si\ S\ {\isasymsubseteq}\ {\isacharbraceleft}a{\isacharbraceright}\ {\isasymunion}\ B{\isacharcomma}\ entonces\ existe\ un\ subconjunto\ finito\ S{\isacharprime}\ de\ B\ tal\ que\ o\ bien}

    \hspace{1cm}\isa{S\ {\isacharequal}\ {\isacharbraceleft}a{\isacharbraceright}\ {\isasymunion}\ S{\isacharprime}\ o\ bien\ S\ {\isacharequal}\ S{\isacharprime}{\isachardot}}

    Sea un elemento cualquiera \isa{x\ {\isasymnotin}\ S}. Vamos a probar que se verifica el resultado para el conjunto
    \isa{{\isacharbraceleft}x{\isacharbraceright}\ {\isasymunion}\ S}. Es decir, si \isa{{\isacharbraceleft}x{\isacharbraceright}\ {\isasymunion}\ S\ {\isasymsubseteq}\ {\isacharbraceleft}a{\isacharbraceright}\ {\isasymunion}\ B}, vamos a encontrar un subconjunto finito \isa{S{\isacharprime}{\isacharprime}} de 
    \isa{B} tal que o bien \isa{{\isacharbraceleft}x{\isacharbraceright}\ {\isasymunion}\ S\ {\isacharequal}\ {\isacharbraceleft}a{\isacharbraceright}\ {\isasymunion}\ S{\isacharprime}{\isacharprime}} o bien \isa{{\isacharbraceleft}x{\isacharbraceright}\ {\isasymunion}\ S\ {\isacharequal}\ S{\isacharprime}{\isacharprime}}.

    Supongamos, pues, que \isa{{\isacharbraceleft}x{\isacharbraceright}\ {\isasymunion}\ S\ {\isasymsubseteq}\ {\isacharbraceleft}a{\isacharbraceright}\ {\isasymunion}\ B}. En este caso, es claro que se verifica que
    \isa{x\ {\isasymin}\ {\isacharbraceleft}a{\isacharbraceright}\ {\isasymunion}\ B} y \isa{S\ {\isasymsubseteq}\ {\isacharbraceleft}a{\isacharbraceright}\ {\isasymunion}\ B}. Por lo último, aplicando hipótesis de inducción, podemos hallar 
    un subconjunto finito \isa{S{\isacharprime}} de \isa{B} tal que o bien \isa{S\ {\isacharequal}\ {\isacharbraceleft}a{\isacharbraceright}\ {\isasymunion}\ S{\isacharprime}} o bien \isa{S\ {\isacharequal}\ S{\isacharprime}}. Por otro lado,
    como \isa{x\ {\isasymin}\ {\isacharbraceleft}a{\isacharbraceright}\ {\isasymunion}\ B}, deducimos que o bien \isa{x\ {\isacharequal}\ a} o bien \isa{x\ {\isasymin}\ B}. En efecto, probemos que se 
    verifica el resultado para ambos casos de la última disyunción.

    En primer lugar, supongamos que \isa{x\ {\isacharequal}\ a}. En este caso, veremos que el conjunto \isa{S{\isacharprime}{\isacharprime}} que 
    verifica el resultado es el propio \isa{S{\isacharprime}}. Se observa fácilmente ya que, si \isa{S\ {\isacharequal}\ {\isacharbraceleft}a{\isacharbraceright}\ {\isasymunion}\ S{\isacharprime}}, como 
    \isa{x\ {\isacharequal}\ a} se obtiene que \isa{{\isacharbraceleft}x{\isacharbraceright}\ {\isasymunion}\ S\ {\isacharequal}\ {\isacharbraceleft}a{\isacharbraceright}\ {\isasymunion}\ S{\isacharprime}}, de modo que \isa{S{\isacharprime}{\isacharprime}\ {\isacharequal}\ S{\isacharprime}} es un subconjunto finito de 
    \isa{B} tal que o bien \isa{{\isacharbraceleft}x{\isacharbraceright}\ {\isasymunion}\ S\ {\isacharequal}\ {\isacharbraceleft}a{\isacharbraceright}\ {\isasymunion}\ S{\isacharprime}{\isacharprime}} o bien \isa{{\isacharbraceleft}x{\isacharbraceright}\ {\isasymunion}\ S\ {\isacharequal}\ S{\isacharprime}{\isacharprime}}. Por otro lado, suponiendo que 
    \isa{S\ {\isacharequal}\ S{\isacharprime}}, se deduce análogamente que \isa{{\isacharbraceleft}x{\isacharbraceright}\ {\isasymunion}\ S\ {\isacharequal}\ {\isacharbraceleft}a{\isacharbraceright}\ {\isasymunion}\ S{\isacharprime}} pues se tiene que \isa{x\ {\isacharequal}\ a}, llegando
    a la misma conclusión.

    Por otra parte, supongamos que \isa{x\ {\isasymin}\ B}. En este caso, el conjunto \isa{S{\isacharprime}{\isacharprime}} que verifica el
    resultado es el conjunto \isa{{\isacharbraceleft}x{\isacharbraceright}\ {\isasymunion}\ S{\isacharprime}}. Observemos que se trata de un subconjunto finito de \isa{B} ya 
    que \isa{S{\isacharprime}} es un subconjunto finito de \isa{B} y \isa{x\ {\isasymin}\ B}. Además, en efecto si \isa{S\ {\isacharequal}\ {\isacharbraceleft}a{\isacharbraceright}\ {\isasymunion}\ S{\isacharprime}}, se 
    deduce que \isa{{\isacharbraceleft}x{\isacharbraceright}\ {\isasymunion}\ S\ {\isacharequal}\ {\isacharbraceleft}x{\isacharcomma}a{\isacharbraceright}\ {\isasymunion}\ S{\isacharprime}\ {\isacharequal}\ {\isacharbraceleft}a{\isacharbraceright}\ {\isasymunion}\ S{\isacharprime}{\isacharprime}}, luego cumple que o bien\\ \isa{{\isacharbraceleft}x{\isacharbraceright}\ {\isasymunion}\ S\ {\isacharequal}\ {\isacharbraceleft}a{\isacharbraceright}\ {\isasymunion}\ S{\isacharprime}{\isacharprime}} o 
    bien \isa{{\isacharbraceleft}x{\isacharbraceright}\ {\isasymunion}\ S\ {\isacharequal}\ S{\isacharprime}{\isacharprime}}. Por otro lado, en el caso en que \isa{S\ {\isacharequal}\ S{\isacharprime}}, es claro que \isa{{\isacharbraceleft}x{\isacharbraceright}\ {\isasymunion}\ S\ {\isacharequal}\ S{\isacharprime}{\isacharprime}} 
    por la elección de \isa{S{\isacharprime}{\isacharprime}}, llegando la misma conclusión.
  \end{demostracion}

  Procedamos con la prueba detallada y formalización en Isabelle. Para ello, hemos utilizado el
  siguiente lema auxiliar.%
\end{isamarkuptext}\isamarkuptrue%
\isacommand{lemma}\isamarkupfalse%
\ subexI\ {\isacharbrackleft}intro{\isacharbrackright}{\isacharcolon}\ {\isachardoublequoteopen}P\ A\ {\isasymLongrightarrow}\ A\ {\isasymsubseteq}\ B\ {\isasymLongrightarrow}\ {\isasymexists}A{\isasymsubseteq}B{\isachardot}\ P\ A{\isachardoublequoteclose}\isanewline
%
\isadelimproof
\ \ %
\endisadelimproof
%
\isatagproof
\isacommand{by}\isamarkupfalse%
\ blast%
\endisatagproof
{\isafoldproof}%
%
\isadelimproof
%
\endisadelimproof
%
\begin{isamarkuptext}%
De este modo, probemos detalladamente el resultado.%
\end{isamarkuptext}\isamarkuptrue%
\isacommand{lemma}\isamarkupfalse%
\ finite{\isacharunderscore}subset{\isacharunderscore}insert{\isadigit{1}}{\isacharcolon}\isanewline
\ \ {\isachardoublequoteopen}{\isasymlbrakk}finite\ S{\isacharsemicolon}\ S\ {\isasymsubseteq}\ {\isacharbraceleft}a{\isacharbraceright}\ {\isasymunion}\ B\ {\isasymrbrakk}\ {\isasymLongrightarrow}\isanewline
\ \ \ \ \ {\isasymexists}S{\isacharprime}\ {\isasymsubseteq}\ B{\isachardot}\ finite\ S{\isacharprime}\ {\isasymand}\ {\isacharparenleft}S\ {\isacharequal}\ {\isacharbraceleft}a{\isacharbraceright}\ {\isasymunion}\ S{\isacharprime}\ {\isasymor}\ S\ {\isacharequal}\ S{\isacharprime}{\isacharparenright}{\isachardoublequoteclose}\isanewline
%
\isadelimproof
%
\endisadelimproof
%
\isatagproof
\isacommand{proof}\isamarkupfalse%
\ {\isacharparenleft}induct\ rule{\isacharcolon}\ finite{\isacharunderscore}induct{\isacharparenright}\isanewline
\ \ \isacommand{case}\isamarkupfalse%
\ empty\isanewline
\ \ \isacommand{have}\isamarkupfalse%
\ {\isachardoublequoteopen}{\isacharbraceleft}{\isacharbraceright}\ {\isasymsubseteq}\ B{\isachardoublequoteclose}\isanewline
\ \ \ \ \isacommand{by}\isamarkupfalse%
\ {\isacharparenleft}rule\ empty{\isacharunderscore}subsetI{\isacharparenright}\isanewline
\ \ \isacommand{have}\isamarkupfalse%
\ {\isadigit{1}}{\isacharcolon}{\isachardoublequoteopen}finite\ {\isacharbraceleft}{\isacharbraceright}{\isachardoublequoteclose}\isanewline
\ \ \ \ \isacommand{by}\isamarkupfalse%
\ {\isacharparenleft}simp\ only{\isacharcolon}\ finite{\isachardot}emptyI{\isacharparenright}\isanewline
\ \ \isacommand{have}\isamarkupfalse%
\ {\isachardoublequoteopen}{\isacharbraceleft}{\isacharbraceright}\ {\isacharequal}\ {\isacharbraceleft}{\isacharbraceright}{\isachardoublequoteclose}\isanewline
\ \ \ \ \isacommand{by}\isamarkupfalse%
\ {\isacharparenleft}simp\ only{\isacharcolon}\ simp{\isacharunderscore}thms{\isacharparenleft}{\isadigit{6}}{\isacharparenright}{\isacharparenright}\isanewline
\ \ \isacommand{then}\isamarkupfalse%
\ \isacommand{have}\isamarkupfalse%
\ {\isadigit{2}}{\isacharcolon}{\isachardoublequoteopen}{\isacharbraceleft}{\isacharbraceright}\ {\isacharequal}\ {\isacharbraceleft}a{\isacharbraceright}\ {\isasymunion}\ {\isacharbraceleft}{\isacharbraceright}\ {\isasymor}\ {\isacharbraceleft}{\isacharbraceright}\ {\isacharequal}\ {\isacharbraceleft}{\isacharbraceright}{\isachardoublequoteclose}\isanewline
\ \ \ \ \isacommand{by}\isamarkupfalse%
\ {\isacharparenleft}rule\ disjI{\isadigit{2}}{\isacharparenright}\isanewline
\ \ \isacommand{have}\isamarkupfalse%
\ {\isachardoublequoteopen}finite\ {\isacharbraceleft}{\isacharbraceright}\ {\isasymand}\ {\isacharparenleft}{\isacharbraceleft}{\isacharbraceright}\ {\isacharequal}\ {\isacharbraceleft}a{\isacharbraceright}\ {\isasymunion}\ {\isacharbraceleft}{\isacharbraceright}\ {\isasymor}\ {\isacharbraceleft}{\isacharbraceright}\ {\isacharequal}\ {\isacharbraceleft}{\isacharbraceright}{\isacharparenright}{\isachardoublequoteclose}\isanewline
\ \ \ \ \isacommand{using}\isamarkupfalse%
\ {\isadigit{1}}\ {\isadigit{2}}\ \isacommand{by}\isamarkupfalse%
\ {\isacharparenleft}rule\ conjI{\isacharparenright}\isanewline
\ \ \isacommand{thus}\isamarkupfalse%
\ {\isachardoublequoteopen}{\isasymexists}S{\isacharprime}\ {\isasymsubseteq}\ B{\isachardot}\ finite\ S{\isacharprime}\ {\isasymand}\ {\isacharparenleft}{\isacharbraceleft}{\isacharbraceright}\ {\isacharequal}\ {\isacharbraceleft}a{\isacharbraceright}\ {\isasymunion}\ S{\isacharprime}\ {\isasymor}\ {\isacharbraceleft}{\isacharbraceright}\ {\isacharequal}\ S{\isacharprime}{\isacharparenright}{\isachardoublequoteclose}\isanewline
\ \ \ \ \isacommand{using}\isamarkupfalse%
\ {\isacartoucheopen}{\isacharbraceleft}{\isacharbraceright}\ {\isasymsubseteq}\ B{\isacartoucheclose}\ \isacommand{by}\isamarkupfalse%
\ {\isacharparenleft}rule\ subexI{\isacharparenright}\isanewline
\isacommand{next}\isamarkupfalse%
\isanewline
\ \ \isacommand{case}\isamarkupfalse%
\ {\isacharparenleft}insert\ x\ S{\isacharparenright}\isanewline
\ \ \isacommand{assume}\isamarkupfalse%
\ {\isachardoublequoteopen}finite\ S{\isachardoublequoteclose}\isanewline
\ \ \isacommand{assume}\isamarkupfalse%
\ {\isachardoublequoteopen}x\ {\isasymnotin}\ S{\isachardoublequoteclose}\isanewline
\ \ \isacommand{assume}\isamarkupfalse%
\ HI{\isacharcolon}{\isachardoublequoteopen}S\ {\isasymsubseteq}\ {\isacharbraceleft}a{\isacharbraceright}\ {\isasymunion}\ B\ {\isasymLongrightarrow}\ {\isasymexists}S{\isacharprime}{\isasymsubseteq}B{\isachardot}\ finite\ S{\isacharprime}\ {\isasymand}\ {\isacharparenleft}S\ {\isacharequal}\ {\isacharbraceleft}a{\isacharbraceright}\ {\isasymunion}\ S{\isacharprime}\ {\isasymor}\ S\ {\isacharequal}\ S{\isacharprime}{\isacharparenright}{\isachardoublequoteclose}\isanewline
\ \ \isacommand{show}\isamarkupfalse%
\ {\isachardoublequoteopen}insert\ x\ S\ {\isasymsubseteq}\ {\isacharbraceleft}a{\isacharbraceright}\ {\isasymunion}\ B\ {\isasymLongrightarrow}\ {\isasymexists}S{\isacharprime}{\isacharprime}{\isasymsubseteq}B{\isachardot}\ finite\ S{\isacharprime}{\isacharprime}\ {\isasymand}\ {\isacharparenleft}insert\ x\ S\ {\isacharequal}\ {\isacharbraceleft}a{\isacharbraceright}\ {\isasymunion}\ S{\isacharprime}{\isacharprime}\ {\isasymor}\ insert\ x\ S\ {\isacharequal}\ S{\isacharprime}{\isacharprime}{\isacharparenright}{\isachardoublequoteclose}\isanewline
\ \ \isacommand{proof}\isamarkupfalse%
\ {\isacharminus}\isanewline
\ \ \ \ \isacommand{assume}\isamarkupfalse%
\ {\isachardoublequoteopen}insert\ x\ S\ {\isasymsubseteq}\ {\isacharbraceleft}a{\isacharbraceright}\ {\isasymunion}\ B{\isachardoublequoteclose}\ \isanewline
\ \ \ \ \isacommand{then}\isamarkupfalse%
\ \isacommand{have}\isamarkupfalse%
\ C{\isacharcolon}{\isachardoublequoteopen}x\ {\isasymin}\ {\isacharbraceleft}a{\isacharbraceright}\ {\isasymunion}\ B\ {\isasymand}\ S\ {\isasymsubseteq}\ {\isacharbraceleft}a{\isacharbraceright}\ {\isasymunion}\ B{\isachardoublequoteclose}\isanewline
\ \ \ \ \ \ \isacommand{by}\isamarkupfalse%
\ {\isacharparenleft}simp\ only{\isacharcolon}\ insert{\isacharunderscore}subset{\isacharparenright}\isanewline
\ \ \ \ \isacommand{then}\isamarkupfalse%
\ \isacommand{have}\isamarkupfalse%
\ {\isachardoublequoteopen}S\ {\isasymsubseteq}\ {\isacharbraceleft}a{\isacharbraceright}\ {\isasymunion}\ B{\isachardoublequoteclose}\isanewline
\ \ \ \ \ \ \isacommand{by}\isamarkupfalse%
\ {\isacharparenleft}rule\ conjunct{\isadigit{2}}{\isacharparenright}\isanewline
\ \ \ \ \isacommand{have}\isamarkupfalse%
\ Ex{\isadigit{1}}{\isacharcolon}{\isachardoublequoteopen}{\isasymexists}S{\isacharprime}{\isasymsubseteq}B{\isachardot}\ finite\ S{\isacharprime}\ {\isasymand}\ {\isacharparenleft}S\ {\isacharequal}\ {\isacharbraceleft}a{\isacharbraceright}\ {\isasymunion}\ S{\isacharprime}\ {\isasymor}\ S\ {\isacharequal}\ S{\isacharprime}{\isacharparenright}{\isachardoublequoteclose}\isanewline
\ \ \ \ \ \ \isacommand{using}\isamarkupfalse%
\ {\isacartoucheopen}S\ {\isasymsubseteq}\ {\isacharbraceleft}a{\isacharbraceright}\ {\isasymunion}\ B{\isacartoucheclose}\ \isacommand{by}\isamarkupfalse%
\ {\isacharparenleft}rule\ HI{\isacharparenright}\isanewline
\ \ \ \ \isacommand{obtain}\isamarkupfalse%
\ S{\isacharprime}\ \isakeyword{where}\ {\isachardoublequoteopen}S{\isacharprime}\ {\isasymsubseteq}\ B{\isachardoublequoteclose}\ \isakeyword{and}\ C{\isadigit{1}}{\isacharcolon}{\isachardoublequoteopen}finite\ S{\isacharprime}\ {\isasymand}\ {\isacharparenleft}S\ {\isacharequal}\ {\isacharbraceleft}a{\isacharbraceright}\ {\isasymunion}\ S{\isacharprime}\ {\isasymor}\ S\ {\isacharequal}\ S{\isacharprime}{\isacharparenright}{\isachardoublequoteclose}\isanewline
\ \ \ \ \ \ \isacommand{using}\isamarkupfalse%
\ Ex{\isadigit{1}}\ \isacommand{by}\isamarkupfalse%
\ {\isacharparenleft}rule\ subexE{\isacharparenright}\isanewline
\ \ \ \ \isacommand{have}\isamarkupfalse%
\ {\isachardoublequoteopen}finite\ S{\isacharprime}{\isachardoublequoteclose}\isanewline
\ \ \ \ \ \ \isacommand{using}\isamarkupfalse%
\ C{\isadigit{1}}\ \isacommand{by}\isamarkupfalse%
\ {\isacharparenleft}rule\ conjunct{\isadigit{1}}{\isacharparenright}\isanewline
\ \ \ \ \isacommand{then}\isamarkupfalse%
\ \isacommand{have}\isamarkupfalse%
\ {\isachardoublequoteopen}finite\ {\isacharparenleft}insert\ x\ S{\isacharprime}{\isacharparenright}{\isachardoublequoteclose}\isanewline
\ \ \ \ \ \ \isacommand{by}\isamarkupfalse%
\ {\isacharparenleft}simp\ only{\isacharcolon}\ finite{\isachardot}insertI{\isacharparenright}\isanewline
\ \ \ \ \isacommand{have}\isamarkupfalse%
\ {\isachardoublequoteopen}x\ {\isasymin}\ {\isacharbraceleft}a{\isacharbraceright}\ {\isasymunion}\ B{\isachardoublequoteclose}\isanewline
\ \ \ \ \ \ \isacommand{using}\isamarkupfalse%
\ C\ \isacommand{by}\isamarkupfalse%
\ {\isacharparenleft}rule\ conjunct{\isadigit{1}}{\isacharparenright}\isanewline
\ \ \ \ \isacommand{then}\isamarkupfalse%
\ \isacommand{have}\isamarkupfalse%
\ {\isachardoublequoteopen}x\ {\isasymin}\ {\isacharbraceleft}a{\isacharbraceright}\ {\isasymor}\ x\ {\isasymin}\ B{\isachardoublequoteclose}\isanewline
\ \ \ \ \ \ \isacommand{by}\isamarkupfalse%
\ {\isacharparenleft}simp\ only{\isacharcolon}\ Un{\isacharunderscore}iff{\isacharparenright}\isanewline
\ \ \ \ \isacommand{then}\isamarkupfalse%
\ \isacommand{have}\isamarkupfalse%
\ {\isachardoublequoteopen}x\ {\isacharequal}\ a\ {\isasymor}\ x\ {\isasymin}\ B{\isachardoublequoteclose}\isanewline
\ \ \ \ \ \ \isacommand{by}\isamarkupfalse%
\ {\isacharparenleft}simp\ only{\isacharcolon}\ singleton{\isacharunderscore}iff{\isacharparenright}\isanewline
\ \ \ \ \isacommand{thus}\isamarkupfalse%
\ {\isachardoublequoteopen}{\isasymexists}S{\isacharprime}{\isacharprime}{\isasymsubseteq}B{\isachardot}\ finite\ S{\isacharprime}{\isacharprime}\ {\isasymand}\ {\isacharparenleft}insert\ x\ S\ {\isacharequal}\ {\isacharbraceleft}a{\isacharbraceright}\ {\isasymunion}\ S{\isacharprime}{\isacharprime}\ {\isasymor}\ insert\ x\ S\ {\isacharequal}\ S{\isacharprime}{\isacharprime}{\isacharparenright}{\isachardoublequoteclose}\isanewline
\ \ \ \ \isacommand{proof}\isamarkupfalse%
\ {\isacharparenleft}rule\ disjE{\isacharparenright}\isanewline
\ \ \ \ \ \ \isacommand{assume}\isamarkupfalse%
\ {\isachardoublequoteopen}x\ {\isacharequal}\ a{\isachardoublequoteclose}\isanewline
\ \ \ \ \ \ \isacommand{have}\isamarkupfalse%
\ {\isachardoublequoteopen}S\ {\isacharequal}\ {\isacharbraceleft}a{\isacharbraceright}\ {\isasymunion}\ S{\isacharprime}\ {\isasymor}\ S\ {\isacharequal}\ S{\isacharprime}{\isachardoublequoteclose}\isanewline
\ \ \ \ \ \ \ \ \isacommand{using}\isamarkupfalse%
\ C{\isadigit{1}}\ \isacommand{by}\isamarkupfalse%
\ {\isacharparenleft}rule\ conjunct{\isadigit{2}}{\isacharparenright}\isanewline
\ \ \ \ \ \ \isacommand{thus}\isamarkupfalse%
\ {\isacharquery}thesis\isanewline
\ \ \ \ \ \ \isacommand{proof}\isamarkupfalse%
\ {\isacharparenleft}rule\ disjE{\isacharparenright}\isanewline
\ \ \ \ \ \ \ \ \isacommand{assume}\isamarkupfalse%
\ {\isachardoublequoteopen}S\ {\isacharequal}\ {\isacharbraceleft}a{\isacharbraceright}\ {\isasymunion}\ S{\isacharprime}{\isachardoublequoteclose}\isanewline
\ \ \ \ \ \ \ \ \isacommand{have}\isamarkupfalse%
\ {\isachardoublequoteopen}x\ {\isasymin}\ {\isacharbraceleft}a{\isacharbraceright}{\isachardoublequoteclose}\isanewline
\ \ \ \ \ \ \ \ \ \ \isacommand{using}\isamarkupfalse%
\ {\isacartoucheopen}x\ {\isacharequal}\ a{\isacartoucheclose}\ \isacommand{by}\isamarkupfalse%
\ {\isacharparenleft}simp\ only{\isacharcolon}\ singleton{\isacharunderscore}iff{\isacharparenright}\isanewline
\ \ \ \ \ \ \ \ \isacommand{then}\isamarkupfalse%
\ \isacommand{have}\isamarkupfalse%
\ {\isachardoublequoteopen}x\ {\isasymin}\ {\isacharbraceleft}a{\isacharbraceright}\ {\isasymunion}\ S{\isacharprime}{\isachardoublequoteclose}\ \isanewline
\ \ \ \ \ \ \ \ \ \ \isacommand{by}\isamarkupfalse%
\ {\isacharparenleft}simp\ only{\isacharcolon}\ UnI{\isadigit{1}}{\isacharparenright}\isanewline
\ \ \ \ \ \ \ \ \isacommand{then}\isamarkupfalse%
\ \isacommand{have}\isamarkupfalse%
\ {\isachardoublequoteopen}insert\ x\ {\isacharparenleft}{\isacharbraceleft}a{\isacharbraceright}\ {\isasymunion}\ S{\isacharprime}{\isacharparenright}\ {\isacharequal}\ {\isacharbraceleft}a{\isacharbraceright}\ {\isasymunion}\ S{\isacharprime}{\isachardoublequoteclose}\isanewline
\ \ \ \ \ \ \ \ \ \ \isacommand{by}\isamarkupfalse%
\ {\isacharparenleft}rule\ insert{\isacharunderscore}absorb{\isacharparenright}\isanewline
\ \ \ \ \ \ \ \ \isacommand{have}\isamarkupfalse%
\ {\isachardoublequoteopen}insert\ x\ S\ {\isacharequal}\ insert\ x\ {\isacharparenleft}{\isacharbraceleft}a{\isacharbraceright}\ {\isasymunion}\ S{\isacharprime}{\isacharparenright}{\isachardoublequoteclose}\isanewline
\ \ \ \ \ \ \ \ \ \ \isacommand{by}\isamarkupfalse%
\ {\isacharparenleft}simp\ only{\isacharcolon}\ {\isacartoucheopen}S\ {\isacharequal}\ {\isacharbraceleft}a{\isacharbraceright}\ {\isasymunion}\ S{\isacharprime}{\isacartoucheclose}{\isacharparenright}\isanewline
\ \ \ \ \ \ \ \ \isacommand{then}\isamarkupfalse%
\ \isacommand{have}\isamarkupfalse%
\ {\isachardoublequoteopen}insert\ x\ S\ {\isacharequal}\ {\isacharbraceleft}a{\isacharbraceright}\ {\isasymunion}\ S{\isacharprime}{\isachardoublequoteclose}\isanewline
\ \ \ \ \ \ \ \ \ \ \isacommand{by}\isamarkupfalse%
\ {\isacharparenleft}simp\ only{\isacharcolon}\ {\isacartoucheopen}insert\ x\ {\isacharparenleft}{\isacharbraceleft}a{\isacharbraceright}\ {\isasymunion}\ S{\isacharprime}{\isacharparenright}\ {\isacharequal}\ {\isacharbraceleft}a{\isacharbraceright}\ {\isasymunion}\ S{\isacharprime}{\isacartoucheclose}{\isacharparenright}\isanewline
\ \ \ \ \ \ \ \ \isacommand{then}\isamarkupfalse%
\ \isacommand{have}\isamarkupfalse%
\ {\isadigit{1}}{\isacharcolon}{\isachardoublequoteopen}insert\ x\ S\ {\isacharequal}\ {\isacharbraceleft}a{\isacharbraceright}\ {\isasymunion}\ S{\isacharprime}\ {\isasymor}\ insert\ x\ S\ {\isacharequal}\ S{\isacharprime}{\isachardoublequoteclose}\isanewline
\ \ \ \ \ \ \ \ \ \ \isacommand{by}\isamarkupfalse%
\ {\isacharparenleft}rule\ disjI{\isadigit{1}}{\isacharparenright}\isanewline
\ \ \ \ \ \ \ \ \isacommand{have}\isamarkupfalse%
\ {\isachardoublequoteopen}finite\ S{\isacharprime}\ {\isasymand}\ {\isacharparenleft}insert\ x\ S\ {\isacharequal}\ {\isacharbraceleft}a{\isacharbraceright}\ {\isasymunion}\ S{\isacharprime}\ {\isasymor}\ insert\ x\ S\ {\isacharequal}\ S{\isacharprime}{\isacharparenright}{\isachardoublequoteclose}\isanewline
\ \ \ \ \ \ \ \ \ \ \isacommand{using}\isamarkupfalse%
\ {\isacartoucheopen}finite\ S{\isacharprime}{\isacartoucheclose}\ {\isadigit{1}}\ \isacommand{by}\isamarkupfalse%
\ {\isacharparenleft}rule\ conjI{\isacharparenright}\isanewline
\ \ \ \ \ \ \ \ \isacommand{thus}\isamarkupfalse%
\ {\isacharquery}thesis\isanewline
\ \ \ \ \ \ \ \ \ \ \isacommand{using}\isamarkupfalse%
\ {\isacartoucheopen}S{\isacharprime}\ {\isasymsubseteq}\ B{\isacartoucheclose}\ \isacommand{by}\isamarkupfalse%
\ {\isacharparenleft}rule\ subexI{\isacharparenright}\isanewline
\ \ \ \ \ \ \isacommand{next}\isamarkupfalse%
\isanewline
\ \ \ \ \ \ \ \ \isacommand{assume}\isamarkupfalse%
\ {\isachardoublequoteopen}S\ {\isacharequal}\ S{\isacharprime}{\isachardoublequoteclose}\isanewline
\ \ \ \ \ \ \ \ \isacommand{have}\isamarkupfalse%
\ {\isachardoublequoteopen}insert\ x\ S\ {\isacharequal}\ {\isacharbraceleft}x{\isacharbraceright}\ {\isasymunion}\ S{\isachardoublequoteclose}\isanewline
\ \ \ \ \ \ \ \ \ \ \isacommand{by}\isamarkupfalse%
\ {\isacharparenleft}rule\ insert{\isacharunderscore}is{\isacharunderscore}Un{\isacharparenright}\isanewline
\ \ \ \ \ \ \ \ \isacommand{then}\isamarkupfalse%
\ \isacommand{have}\isamarkupfalse%
\ {\isachardoublequoteopen}insert\ x\ S\ {\isacharequal}\ {\isacharbraceleft}a{\isacharbraceright}\ {\isasymunion}\ S{\isacharprime}{\isachardoublequoteclose}\isanewline
\ \ \ \ \ \ \ \ \ \ \isacommand{by}\isamarkupfalse%
\ {\isacharparenleft}simp\ only{\isacharcolon}\ {\isacartoucheopen}x\ {\isacharequal}\ a{\isacartoucheclose}\ {\isacartoucheopen}S\ {\isacharequal}\ S{\isacharprime}{\isacartoucheclose}{\isacharparenright}\isanewline
\ \ \ \ \ \ \ \ \isacommand{then}\isamarkupfalse%
\ \isacommand{have}\isamarkupfalse%
\ {\isadigit{1}}{\isacharcolon}{\isachardoublequoteopen}insert\ x\ S\ {\isacharequal}\ {\isacharbraceleft}a{\isacharbraceright}\ {\isasymunion}\ S{\isacharprime}\ {\isasymor}\ insert\ x\ S\ {\isacharequal}\ S{\isacharprime}{\isachardoublequoteclose}\isanewline
\ \ \ \ \ \ \ \ \ \ \isacommand{by}\isamarkupfalse%
\ {\isacharparenleft}rule\ disjI{\isadigit{1}}{\isacharparenright}\isanewline
\ \ \ \ \ \ \ \ \isacommand{have}\isamarkupfalse%
\ {\isachardoublequoteopen}finite\ S{\isacharprime}\ {\isasymand}\ {\isacharparenleft}insert\ x\ S\ {\isacharequal}\ {\isacharbraceleft}a{\isacharbraceright}\ {\isasymunion}\ S{\isacharprime}\ {\isasymor}\ insert\ x\ S\ {\isacharequal}\ S{\isacharprime}{\isacharparenright}{\isachardoublequoteclose}\isanewline
\ \ \ \ \ \ \ \ \ \ \isacommand{using}\isamarkupfalse%
\ {\isacartoucheopen}finite\ S{\isacharprime}{\isacartoucheclose}\ {\isadigit{1}}\ \isacommand{by}\isamarkupfalse%
\ {\isacharparenleft}rule\ conjI{\isacharparenright}\isanewline
\ \ \ \ \ \ \ \ \isacommand{thus}\isamarkupfalse%
\ {\isacharquery}thesis\isanewline
\ \ \ \ \ \ \ \ \ \ \isacommand{using}\isamarkupfalse%
\ {\isacartoucheopen}S{\isacharprime}\ {\isasymsubseteq}\ B{\isacartoucheclose}\ \isacommand{by}\isamarkupfalse%
\ {\isacharparenleft}rule\ subexI{\isacharparenright}\isanewline
\ \ \ \ \ \ \isacommand{qed}\isamarkupfalse%
\isanewline
\ \ \ \ \isacommand{next}\isamarkupfalse%
\isanewline
\ \ \ \ \ \ \isacommand{assume}\isamarkupfalse%
\ {\isachardoublequoteopen}x\ {\isasymin}\ B{\isachardoublequoteclose}\isanewline
\ \ \ \ \ \ \isacommand{have}\isamarkupfalse%
\ {\isachardoublequoteopen}x\ {\isasymin}\ B\ {\isasymand}\ S{\isacharprime}\ {\isasymsubseteq}\ B{\isachardoublequoteclose}\isanewline
\ \ \ \ \ \ \ \ \isacommand{using}\isamarkupfalse%
\ {\isacartoucheopen}x\ {\isasymin}\ B{\isacartoucheclose}\ {\isacartoucheopen}S{\isacharprime}\ {\isasymsubseteq}\ B{\isacartoucheclose}\ \isacommand{by}\isamarkupfalse%
\ {\isacharparenleft}rule\ conjI{\isacharparenright}\isanewline
\ \ \ \ \ \ \isacommand{then}\isamarkupfalse%
\ \isacommand{have}\isamarkupfalse%
\ {\isachardoublequoteopen}insert\ x\ S{\isacharprime}\ {\isasymsubseteq}\ B{\isachardoublequoteclose}\isanewline
\ \ \ \ \ \ \ \ \isacommand{by}\isamarkupfalse%
\ {\isacharparenleft}simp\ only{\isacharcolon}\ insert{\isacharunderscore}subset{\isacharparenright}\isanewline
\ \ \ \ \ \ \isacommand{have}\isamarkupfalse%
\ {\isachardoublequoteopen}finite\ {\isacharparenleft}insert\ x\ S{\isacharprime}{\isacharparenright}{\isachardoublequoteclose}\isanewline
\ \ \ \ \ \ \ \ \isacommand{using}\isamarkupfalse%
\ {\isacartoucheopen}finite\ S{\isacharprime}{\isacartoucheclose}\ \isacommand{by}\isamarkupfalse%
\ {\isacharparenleft}simp\ only{\isacharcolon}\ finite{\isachardot}insertI{\isacharparenright}\isanewline
\ \ \ \ \ \ \isacommand{have}\isamarkupfalse%
\ {\isachardoublequoteopen}S\ {\isacharequal}\ {\isacharbraceleft}a{\isacharbraceright}\ {\isasymunion}\ S{\isacharprime}\ {\isasymor}\ S\ {\isacharequal}\ S{\isacharprime}{\isachardoublequoteclose}\isanewline
\ \ \ \ \ \ \ \ \isacommand{using}\isamarkupfalse%
\ C{\isadigit{1}}\ \isacommand{by}\isamarkupfalse%
\ {\isacharparenleft}rule\ conjunct{\isadigit{2}}{\isacharparenright}\isanewline
\ \ \ \ \ \ \isacommand{thus}\isamarkupfalse%
\ {\isacharquery}thesis\isanewline
\ \ \ \ \ \ \isacommand{proof}\isamarkupfalse%
\ {\isacharparenleft}rule\ disjE{\isacharparenright}\isanewline
\ \ \ \ \ \ \ \ \isacommand{assume}\isamarkupfalse%
\ {\isachardoublequoteopen}S\ {\isacharequal}\ {\isacharbraceleft}a{\isacharbraceright}\ {\isasymunion}\ S{\isacharprime}{\isachardoublequoteclose}\isanewline
\ \ \ \ \ \ \ \ \isacommand{have}\isamarkupfalse%
\ {\isachardoublequoteopen}insert\ x\ S\ {\isacharequal}\ insert\ x\ {\isacharparenleft}{\isacharbraceleft}a{\isacharbraceright}\ {\isasymunion}\ S{\isacharprime}{\isacharparenright}{\isachardoublequoteclose}\isanewline
\ \ \ \ \ \ \ \ \ \ \isacommand{by}\isamarkupfalse%
\ {\isacharparenleft}simp\ only{\isacharcolon}\ {\isacartoucheopen}S\ {\isacharequal}\ {\isacharbraceleft}a{\isacharbraceright}\ {\isasymunion}\ S{\isacharprime}{\isacartoucheclose}{\isacharparenright}\isanewline
\ \ \ \ \ \ \ \ \isacommand{then}\isamarkupfalse%
\ \isacommand{have}\isamarkupfalse%
\ {\isachardoublequoteopen}insert\ x\ S\ {\isacharequal}\ {\isacharbraceleft}a{\isacharbraceright}\ {\isasymunion}\ {\isacharparenleft}insert\ x\ S{\isacharprime}{\isacharparenright}{\isachardoublequoteclose}\isanewline
\ \ \ \ \ \ \ \ \ \ \isacommand{by}\isamarkupfalse%
\ blast\isanewline
\ \ \ \ \ \ \ \ \isacommand{then}\isamarkupfalse%
\ \isacommand{have}\isamarkupfalse%
\ {\isadigit{1}}{\isacharcolon}{\isachardoublequoteopen}insert\ x\ S\ {\isacharequal}\ {\isacharbraceleft}a{\isacharbraceright}\ {\isasymunion}\ {\isacharparenleft}insert\ x\ S{\isacharprime}{\isacharparenright}\ {\isasymor}\ insert\ x\ S\ {\isacharequal}\ insert\ x\ S{\isacharprime}{\isachardoublequoteclose}\isanewline
\ \ \ \ \ \ \ \ \ \ \isacommand{by}\isamarkupfalse%
\ {\isacharparenleft}rule\ disjI{\isadigit{1}}{\isacharparenright}\isanewline
\ \ \ \ \ \ \ \ \isacommand{have}\isamarkupfalse%
\ {\isachardoublequoteopen}finite\ {\isacharparenleft}insert\ x\ S{\isacharprime}{\isacharparenright}\ {\isasymand}\ {\isacharparenleft}insert\ x\ S\ {\isacharequal}\ {\isacharbraceleft}a{\isacharbraceright}\ {\isasymunion}\ {\isacharparenleft}insert\ x\ S{\isacharprime}{\isacharparenright}\ {\isasymor}\ insert\ x\ S\ {\isacharequal}\ insert\ x\ S{\isacharprime}{\isacharparenright}{\isachardoublequoteclose}\isanewline
\ \ \ \ \ \ \ \ \ \ \isacommand{using}\isamarkupfalse%
\ {\isacartoucheopen}finite\ {\isacharparenleft}insert\ x\ S{\isacharprime}{\isacharparenright}{\isacartoucheclose}\ {\isadigit{1}}\ \isacommand{by}\isamarkupfalse%
\ {\isacharparenleft}rule\ conjI{\isacharparenright}\isanewline
\ \ \ \ \ \ \ \ \isacommand{thus}\isamarkupfalse%
\ {\isacharquery}thesis\isanewline
\ \ \ \ \ \ \ \ \ \ \isacommand{using}\isamarkupfalse%
\ {\isacartoucheopen}insert\ x\ S{\isacharprime}\ {\isasymsubseteq}\ B{\isacartoucheclose}\ \isacommand{by}\isamarkupfalse%
\ {\isacharparenleft}rule\ subexI{\isacharparenright}\isanewline
\ \ \ \ \ \ \isacommand{next}\isamarkupfalse%
\isanewline
\ \ \ \ \ \ \ \ \isacommand{assume}\isamarkupfalse%
\ {\isachardoublequoteopen}S\ {\isacharequal}\ S{\isacharprime}{\isachardoublequoteclose}\isanewline
\ \ \ \ \ \ \ \ \isacommand{have}\isamarkupfalse%
\ {\isachardoublequoteopen}insert\ x\ S\ {\isacharequal}\ insert\ x\ S{\isacharprime}{\isachardoublequoteclose}\isanewline
\ \ \ \ \ \ \ \ \ \ \isacommand{by}\isamarkupfalse%
\ {\isacharparenleft}simp\ only{\isacharcolon}\ {\isacartoucheopen}S\ {\isacharequal}\ S{\isacharprime}{\isacartoucheclose}{\isacharparenright}\isanewline
\ \ \ \ \ \ \ \ \isacommand{then}\isamarkupfalse%
\ \isacommand{have}\isamarkupfalse%
\ {\isadigit{1}}{\isacharcolon}{\isachardoublequoteopen}insert\ x\ S\ {\isacharequal}\ {\isacharbraceleft}a{\isacharbraceright}\ {\isasymunion}\ {\isacharparenleft}insert\ x\ S{\isacharprime}{\isacharparenright}\ {\isasymor}\ insert\ x\ S\ {\isacharequal}\ insert\ x\ S{\isacharprime}{\isachardoublequoteclose}\isanewline
\ \ \ \ \ \ \ \ \ \ \isacommand{by}\isamarkupfalse%
\ {\isacharparenleft}rule\ disjI{\isadigit{2}}{\isacharparenright}\isanewline
\ \ \ \ \ \ \ \ \isacommand{have}\isamarkupfalse%
\ {\isachardoublequoteopen}finite\ {\isacharparenleft}insert\ x\ S{\isacharprime}{\isacharparenright}\ {\isasymand}\ {\isacharparenleft}insert\ x\ S\ {\isacharequal}\ {\isacharbraceleft}a{\isacharbraceright}\ {\isasymunion}\ {\isacharparenleft}insert\ x\ S{\isacharprime}{\isacharparenright}\ {\isasymor}\ insert\ x\ S\ {\isacharequal}\ insert\ x\ S{\isacharprime}{\isacharparenright}{\isachardoublequoteclose}\isanewline
\ \ \ \ \ \ \ \ \ \ \isacommand{using}\isamarkupfalse%
\ {\isacartoucheopen}finite\ {\isacharparenleft}insert\ x\ S{\isacharprime}{\isacharparenright}{\isacartoucheclose}\ {\isadigit{1}}\ \isacommand{by}\isamarkupfalse%
\ {\isacharparenleft}rule\ conjI{\isacharparenright}\isanewline
\ \ \ \ \ \ \ \ \isacommand{thus}\isamarkupfalse%
\ {\isacharquery}thesis\isanewline
\ \ \ \ \ \ \ \ \ \ \isacommand{using}\isamarkupfalse%
\ {\isacartoucheopen}insert\ x\ S{\isacharprime}\ {\isasymsubseteq}\ B{\isacartoucheclose}\ \isacommand{by}\isamarkupfalse%
\ {\isacharparenleft}rule\ subexI{\isacharparenright}\isanewline
\ \ \ \ \ \ \isacommand{qed}\isamarkupfalse%
\isanewline
\ \ \ \ \isacommand{qed}\isamarkupfalse%
\isanewline
\ \ \isacommand{qed}\isamarkupfalse%
\isanewline
\isacommand{qed}\isamarkupfalse%
%
\endisatagproof
{\isafoldproof}%
%
\isadelimproof
%
\endisadelimproof
%
\begin{isamarkuptext}%
El segundo resultado sobre subconjuntos finitos es consecuencia del anterior.

\begin{lema}
  Sea un conjunto de la forma \isa{{\isacharbraceleft}a{\isacharcomma}b{\isacharbraceright}\ {\isasymunion}\ B} y \isa{S} un subconjunto finito suyo. Entonces, existe un
  subconjunto finito \isa{S{\isacharprime}} de \isa{B} tal que se cumple \isa{S\ {\isacharequal}\ {\isacharbraceleft}a{\isacharcomma}b{\isacharbraceright}\ {\isasymunion}\ S{\isacharprime}}, \isa{S\ {\isacharequal}\ {\isacharbraceleft}a{\isacharbraceright}\ {\isasymunion}\ S{\isacharprime}},\\ \isa{S\ {\isacharequal}\ {\isacharbraceleft}b{\isacharbraceright}\ {\isasymunion}\ S{\isacharprime}} 
  o \isa{S\ {\isacharequal}\ S{\isacharprime}}.
\end{lema}

\begin{demostracion}
  En particular, \isa{S} es un subconjunto finito de \isa{{\isacharbraceleft}a{\isacharbraceright}\ {\isasymunion}\ {\isacharparenleft}{\isacharbraceleft}b{\isacharbraceright}\ {\isasymunion}\ B{\isacharparenright}} luego, aplicando el lema \isa{{\isadigit{4}}{\isachardot}{\isadigit{3}}{\isachardot}{\isadigit{2}}}
  anterior, podemos hallar un subconjunto finito \isa{S\isactrlsub {\isadigit{1}}} de \isa{{\isacharbraceleft}b{\isacharbraceright}\ {\isasymunion}\ B} tal que o bien \isa{S\ {\isacharequal}\ {\isacharbraceleft}a{\isacharbraceright}\ {\isasymunion}\ S\isactrlsub {\isadigit{1}}} o 
  bien \isa{S\ {\isacharequal}\ S\isactrlsub {\isadigit{1}}}. A su vez, podemos aplicar dicho resultado para el subconjunto finito \isa{S\isactrlsub {\isadigit{1}}} de 
  \isa{{\isacharbraceleft}b{\isacharbraceright}\ {\isasymunion}\ B}, obteniendo un subconjunto finito \isa{S\isactrlsub {\isadigit{2}}} de \isa{B} tal que o bien \isa{S\isactrlsub {\isadigit{1}}\ {\isacharequal}\ {\isacharbraceleft}b{\isacharbraceright}\ {\isasymunion}\ S\isactrlsub {\isadigit{2}}} o bien 
  \isa{S\isactrlsub {\isadigit{1}}\ {\isacharequal}\ S\isactrlsub {\isadigit{2}}}. Veamos que el lema se verifica en ambas opciones posibles de \isa{S\isactrlsub {\isadigit{1}}} para el conjunto 
  \isa{S{\isacharprime}\ {\isacharequal}\ S\isactrlsub {\isadigit{2}}}.

  En primer lugar, supongamos que \isa{S\isactrlsub {\isadigit{1}}\ {\isacharequal}\ {\isacharbraceleft}b{\isacharbraceright}\ {\isasymunion}\ S\isactrlsub {\isadigit{2}}}. De este modo, se verifica el resultado tanto para
  \isa{S\ {\isacharequal}\ {\isacharbraceleft}a{\isacharbraceright}\ {\isasymunion}\ S\isactrlsub {\isadigit{1}}} como para \isa{S\ {\isacharequal}\ S\isactrlsub {\isadigit{1}}}. En efecto, en la primera opción, por elección de \isa{S\isactrlsub {\isadigit{1}}}, es claro
  que \isa{S\ {\isacharequal}\ {\isacharbraceleft}a{\isacharcomma}b{\isacharbraceright}\ {\isasymunion}\ S\isactrlsub {\isadigit{2}}}. Finalmente, para \isa{S\ {\isacharequal}\ S\isactrlsub {\isadigit{1}}}, obtenemos que \isa{S\ {\isacharequal}\ {\isacharbraceleft}b{\isacharbraceright}\ {\isasymunion}\ S\isactrlsub {\isadigit{2}}}, lo que prueba
  igualmente el lema para \isa{S{\isacharprime}\ {\isacharequal}\ S\isactrlsub {\isadigit{2}}}.

  Por último, supongamos que \isa{S\isactrlsub {\isadigit{1}}\ {\isacharequal}\ S\isactrlsub {\isadigit{2}}}. Análogamente, el resultado es inmediato pues si 
  \isa{S\ {\isacharequal}\ {\isacharbraceleft}a{\isacharbraceright}\ {\isasymunion}\ S\isactrlsub {\isadigit{1}}} obtenemos que \isa{S\ {\isacharequal}\ {\isacharbraceleft}a{\isacharbraceright}\ {\isasymunion}\ S\isactrlsub {\isadigit{2}}}, y si suponemos \isa{S\ {\isacharequal}\ S\isactrlsub {\isadigit{1}}} obtenemos\\ \isa{S\ {\isacharequal}\ S\isactrlsub {\isadigit{2}}}, probando
  así el lema.
\end{demostracion}

  Su formalización y prueba detallada en Isabelle/HOL son las siguientes.%
\end{isamarkuptext}\isamarkuptrue%
\isacommand{lemma}\isamarkupfalse%
\ finite{\isacharunderscore}subset{\isacharunderscore}insert{\isadigit{2}}{\isacharcolon}\isanewline
\ \ \isakeyword{assumes}\ {\isachardoublequoteopen}finite\ S{\isachardoublequoteclose}\isanewline
\ \ \ \ \ \ \ \ \ \ {\isachardoublequoteopen}S\ {\isasymsubseteq}\ {\isacharbraceleft}a{\isacharcomma}b{\isacharbraceright}\ {\isasymunion}\ B{\isachardoublequoteclose}\isanewline
\ \ \ \ \ \ \ \ \isakeyword{shows}\ {\isachardoublequoteopen}{\isasymexists}S{\isacharprime}\ {\isasymsubseteq}\ B{\isachardot}\ finite\ S{\isacharprime}\ {\isasymand}\ {\isacharparenleft}S\ {\isacharequal}\ {\isacharbraceleft}a{\isacharcomma}b{\isacharbraceright}\ {\isasymunion}\ S{\isacharprime}\ {\isasymor}\ S\ {\isacharequal}\ {\isacharbraceleft}a{\isacharbraceright}\ {\isasymunion}\ S{\isacharprime}\ {\isasymor}\ S\ {\isacharequal}\ {\isacharbraceleft}b{\isacharbraceright}\ {\isasymunion}\ S{\isacharprime}\ {\isasymor}\ S\ {\isacharequal}\ S{\isacharprime}{\isacharparenright}{\isachardoublequoteclose}\isanewline
%
\isadelimproof
%
\endisadelimproof
%
\isatagproof
\isacommand{proof}\isamarkupfalse%
\ {\isacharminus}\isanewline
\ \ \isacommand{have}\isamarkupfalse%
\ {\isachardoublequoteopen}S\ {\isasymsubseteq}\ {\isacharbraceleft}a{\isacharbraceright}\ {\isasymunion}\ {\isacharparenleft}{\isacharbraceleft}b{\isacharbraceright}\ {\isasymunion}\ B{\isacharparenright}{\isachardoublequoteclose}\isanewline
\ \ \ \ \isacommand{using}\isamarkupfalse%
\ assms{\isacharparenleft}{\isadigit{2}}{\isacharparenright}\ \isacommand{by}\isamarkupfalse%
\ blast\isanewline
\ \ \isacommand{then}\isamarkupfalse%
\ \isacommand{have}\isamarkupfalse%
\ Ex{\isadigit{1}}{\isacharcolon}{\isachardoublequoteopen}{\isasymexists}S{\isadigit{1}}\ {\isasymsubseteq}\ {\isacharparenleft}{\isacharbraceleft}b{\isacharbraceright}\ {\isasymunion}\ B{\isacharparenright}{\isachardot}\ finite\ S{\isadigit{1}}\ {\isasymand}\ {\isacharparenleft}S\ {\isacharequal}\ {\isacharbraceleft}a{\isacharbraceright}\ {\isasymunion}\ S{\isadigit{1}}\ {\isasymor}\ S\ {\isacharequal}\ S{\isadigit{1}}{\isacharparenright}{\isachardoublequoteclose}\isanewline
\ \ \ \ \isacommand{using}\isamarkupfalse%
\ assms{\isacharparenleft}{\isadigit{1}}{\isacharparenright}\ \isacommand{by}\isamarkupfalse%
\ {\isacharparenleft}simp\ only{\isacharcolon}\ finite{\isacharunderscore}subset{\isacharunderscore}insert{\isadigit{1}}{\isacharparenright}\isanewline
\ \ \isacommand{obtain}\isamarkupfalse%
\ S{\isadigit{1}}\ \isakeyword{where}\ {\isachardoublequoteopen}S{\isadigit{1}}\ {\isasymsubseteq}\ {\isacharbraceleft}b{\isacharbraceright}\ {\isasymunion}\ B{\isachardoublequoteclose}\ \isakeyword{and}\ {\isadigit{1}}{\isacharcolon}{\isachardoublequoteopen}finite\ S{\isadigit{1}}\ {\isasymand}\ {\isacharparenleft}S\ {\isacharequal}\ {\isacharbraceleft}a{\isacharbraceright}\ {\isasymunion}\ S{\isadigit{1}}\ {\isasymor}\ S\ {\isacharequal}\ S{\isadigit{1}}{\isacharparenright}{\isachardoublequoteclose}\isanewline
\ \ \ \ \isacommand{using}\isamarkupfalse%
\ Ex{\isadigit{1}}\ \isacommand{by}\isamarkupfalse%
\ {\isacharparenleft}rule\ subexE{\isacharparenright}\isanewline
\ \ \isacommand{have}\isamarkupfalse%
\ {\isachardoublequoteopen}finite\ S{\isadigit{1}}{\isachardoublequoteclose}\isanewline
\ \ \ \ \isacommand{using}\isamarkupfalse%
\ {\isadigit{1}}\ \isacommand{by}\isamarkupfalse%
\ {\isacharparenleft}rule\ conjunct{\isadigit{1}}{\isacharparenright}\isanewline
\ \ \isacommand{have}\isamarkupfalse%
\ Ex{\isadigit{2}}{\isacharcolon}{\isachardoublequoteopen}{\isasymexists}S{\isadigit{2}}\ {\isasymsubseteq}\ B{\isachardot}\ finite\ S{\isadigit{2}}\ {\isasymand}\ {\isacharparenleft}S{\isadigit{1}}\ {\isacharequal}\ {\isacharbraceleft}b{\isacharbraceright}\ {\isasymunion}\ S{\isadigit{2}}\ {\isasymor}\ S{\isadigit{1}}\ {\isacharequal}\ S{\isadigit{2}}{\isacharparenright}{\isachardoublequoteclose}\isanewline
\ \ \ \ \isacommand{using}\isamarkupfalse%
\ {\isacartoucheopen}finite\ S{\isadigit{1}}{\isacartoucheclose}\ {\isacartoucheopen}S{\isadigit{1}}\ {\isasymsubseteq}\ {\isacharbraceleft}b{\isacharbraceright}\ {\isasymunion}\ B{\isacartoucheclose}\ \isacommand{by}\isamarkupfalse%
\ {\isacharparenleft}rule\ finite{\isacharunderscore}subset{\isacharunderscore}insert{\isadigit{1}}{\isacharparenright}\isanewline
\ \ \isacommand{obtain}\isamarkupfalse%
\ S{\isadigit{2}}\ \isakeyword{where}\ {\isachardoublequoteopen}S{\isadigit{2}}\ {\isasymsubseteq}\ B{\isachardoublequoteclose}\ \isakeyword{and}\ {\isadigit{2}}{\isacharcolon}{\isachardoublequoteopen}finite\ S{\isadigit{2}}\ {\isasymand}\ {\isacharparenleft}S{\isadigit{1}}\ {\isacharequal}\ {\isacharbraceleft}b{\isacharbraceright}\ {\isasymunion}\ S{\isadigit{2}}\ {\isasymor}\ S{\isadigit{1}}\ {\isacharequal}\ S{\isadigit{2}}{\isacharparenright}{\isachardoublequoteclose}\isanewline
\ \ \ \ \isacommand{using}\isamarkupfalse%
\ Ex{\isadigit{2}}\ \isacommand{by}\isamarkupfalse%
\ {\isacharparenleft}rule\ subexE{\isacharparenright}\isanewline
\ \ \isacommand{have}\isamarkupfalse%
\ {\isachardoublequoteopen}finite\ S{\isadigit{2}}{\isachardoublequoteclose}\isanewline
\ \ \ \ \isacommand{using}\isamarkupfalse%
\ {\isadigit{2}}\ \isacommand{by}\isamarkupfalse%
\ {\isacharparenleft}rule\ conjunct{\isadigit{1}}{\isacharparenright}\isanewline
\ \ \isacommand{have}\isamarkupfalse%
\ {\isachardoublequoteopen}S{\isadigit{1}}\ {\isacharequal}\ {\isacharbraceleft}b{\isacharbraceright}\ {\isasymunion}\ S{\isadigit{2}}\ {\isasymor}\ S{\isadigit{1}}\ {\isacharequal}\ S{\isadigit{2}}{\isachardoublequoteclose}\isanewline
\ \ \ \ \isacommand{using}\isamarkupfalse%
\ {\isadigit{2}}\ \isacommand{by}\isamarkupfalse%
\ {\isacharparenleft}rule\ conjunct{\isadigit{2}}{\isacharparenright}\isanewline
\ \ \isacommand{thus}\isamarkupfalse%
\ {\isacharquery}thesis\isanewline
\ \ \isacommand{proof}\isamarkupfalse%
\ {\isacharparenleft}rule\ disjE{\isacharparenright}\isanewline
\ \ \ \ \isacommand{assume}\isamarkupfalse%
\ {\isachardoublequoteopen}S{\isadigit{1}}\ {\isacharequal}\ {\isacharbraceleft}b{\isacharbraceright}\ {\isasymunion}\ S{\isadigit{2}}{\isachardoublequoteclose}\isanewline
\ \ \ \ \isacommand{have}\isamarkupfalse%
\ {\isachardoublequoteopen}S\ {\isacharequal}\ {\isacharbraceleft}a{\isacharbraceright}\ {\isasymunion}\ S{\isadigit{1}}\ {\isasymor}\ S\ {\isacharequal}\ S{\isadigit{1}}{\isachardoublequoteclose}\isanewline
\ \ \ \ \ \ \isacommand{using}\isamarkupfalse%
\ {\isadigit{1}}\ \isacommand{by}\isamarkupfalse%
\ {\isacharparenleft}rule\ conjunct{\isadigit{2}}{\isacharparenright}\isanewline
\ \ \ \ \isacommand{thus}\isamarkupfalse%
\ {\isacharquery}thesis\isanewline
\ \ \ \ \isacommand{proof}\isamarkupfalse%
\ {\isacharparenleft}rule\ disjE{\isacharparenright}\isanewline
\ \ \ \ \ \ \isacommand{assume}\isamarkupfalse%
\ {\isachardoublequoteopen}S\ {\isacharequal}\ {\isacharbraceleft}a{\isacharbraceright}\ {\isasymunion}\ S{\isadigit{1}}{\isachardoublequoteclose}\isanewline
\ \ \ \ \ \ \isacommand{then}\isamarkupfalse%
\ \isacommand{have}\isamarkupfalse%
\ {\isachardoublequoteopen}S\ {\isacharequal}\ {\isacharbraceleft}a{\isacharbraceright}\ {\isasymunion}\ {\isacharbraceleft}b{\isacharbraceright}\ {\isasymunion}\ S{\isadigit{2}}{\isachardoublequoteclose}\isanewline
\ \ \ \ \ \ \ \ \isacommand{by}\isamarkupfalse%
\ {\isacharparenleft}simp\ add{\isacharcolon}\ {\isacartoucheopen}S{\isadigit{1}}\ {\isacharequal}\ {\isacharbraceleft}b{\isacharbraceright}\ {\isasymunion}\ S{\isadigit{2}}{\isacartoucheclose}{\isacharparenright}\isanewline
\ \ \ \ \ \ \isacommand{then}\isamarkupfalse%
\ \isacommand{have}\isamarkupfalse%
\ {\isachardoublequoteopen}S\ {\isacharequal}\ {\isacharbraceleft}a{\isacharcomma}b{\isacharbraceright}\ {\isasymunion}\ S{\isadigit{2}}{\isachardoublequoteclose}\isanewline
\ \ \ \ \ \ \ \ \isacommand{by}\isamarkupfalse%
\ blast\isanewline
\ \ \ \ \ \ \isacommand{then}\isamarkupfalse%
\ \isacommand{have}\isamarkupfalse%
\ {\isachardoublequoteopen}S\ {\isacharequal}\ {\isacharbraceleft}a{\isacharcomma}b{\isacharbraceright}\ {\isasymunion}\ S{\isadigit{2}}\ {\isasymor}\ S\ {\isacharequal}\ {\isacharbraceleft}a{\isacharbraceright}\ {\isasymunion}\ S{\isadigit{2}}\ {\isasymor}\ S\ {\isacharequal}\ {\isacharbraceleft}b{\isacharbraceright}\ {\isasymunion}\ S{\isadigit{2}}\ {\isasymor}\ S\ {\isacharequal}\ S{\isadigit{2}}{\isachardoublequoteclose}\isanewline
\ \ \ \ \ \ \ \ \isacommand{by}\isamarkupfalse%
\ {\isacharparenleft}iprover\ intro{\isacharcolon}\ disjI{\isadigit{1}}{\isacharparenright}\isanewline
\ \ \ \ \ \ \isacommand{then}\isamarkupfalse%
\ \isacommand{have}\isamarkupfalse%
\ {\isachardoublequoteopen}finite\ S{\isadigit{2}}\ {\isasymand}\ {\isacharparenleft}S\ {\isacharequal}\ {\isacharbraceleft}a{\isacharcomma}b{\isacharbraceright}\ {\isasymunion}\ S{\isadigit{2}}\ {\isasymor}\ S\ {\isacharequal}\ {\isacharbraceleft}a{\isacharbraceright}\ {\isasymunion}\ S{\isadigit{2}}\ {\isasymor}\ S\ {\isacharequal}\ {\isacharbraceleft}b{\isacharbraceright}\ {\isasymunion}\ S{\isadigit{2}}\ {\isasymor}\ S\ {\isacharequal}\ S{\isadigit{2}}{\isacharparenright}{\isachardoublequoteclose}\isanewline
\ \ \ \ \ \ \ \ \isacommand{using}\isamarkupfalse%
\ {\isacartoucheopen}finite\ S{\isadigit{2}}{\isacartoucheclose}\ \isacommand{by}\isamarkupfalse%
\ {\isacharparenleft}iprover\ intro{\isacharcolon}\ conjI{\isacharparenright}\isanewline
\ \ \ \ \ \ \isacommand{thus}\isamarkupfalse%
\ {\isacharquery}thesis\isanewline
\ \ \ \ \ \ \ \ \isacommand{using}\isamarkupfalse%
\ {\isacartoucheopen}S{\isadigit{2}}\ {\isasymsubseteq}\ B{\isacartoucheclose}\ \isacommand{by}\isamarkupfalse%
\ {\isacharparenleft}rule\ subexI{\isacharparenright}\isanewline
\ \ \ \ \isacommand{next}\isamarkupfalse%
\isanewline
\ \ \ \ \ \ \isacommand{assume}\isamarkupfalse%
\ {\isachardoublequoteopen}S\ {\isacharequal}\ S{\isadigit{1}}{\isachardoublequoteclose}\isanewline
\ \ \ \ \ \ \isacommand{then}\isamarkupfalse%
\ \isacommand{have}\isamarkupfalse%
\ {\isachardoublequoteopen}S\ {\isacharequal}\ {\isacharbraceleft}b{\isacharbraceright}\ {\isasymunion}\ S{\isadigit{2}}{\isachardoublequoteclose}\isanewline
\ \ \ \ \ \ \ \ \isacommand{by}\isamarkupfalse%
\ {\isacharparenleft}simp\ add{\isacharcolon}\ {\isacartoucheopen}S{\isadigit{1}}\ {\isacharequal}\ {\isacharbraceleft}b{\isacharbraceright}\ {\isasymunion}\ S{\isadigit{2}}{\isacartoucheclose}{\isacharparenright}\isanewline
\ \ \ \ \ \ \isacommand{then}\isamarkupfalse%
\ \isacommand{have}\isamarkupfalse%
\ {\isachardoublequoteopen}S\ {\isacharequal}\ {\isacharbraceleft}a{\isacharcomma}b{\isacharbraceright}\ {\isasymunion}\ S{\isadigit{2}}\ {\isasymor}\ S\ {\isacharequal}\ {\isacharbraceleft}a{\isacharbraceright}\ {\isasymunion}\ S{\isadigit{2}}\ {\isasymor}\ S\ {\isacharequal}\ {\isacharbraceleft}b{\isacharbraceright}\ {\isasymunion}\ S{\isadigit{2}}\ {\isasymor}\ S\ {\isacharequal}\ S{\isadigit{2}}{\isachardoublequoteclose}\isanewline
\ \ \ \ \ \ \ \ \isacommand{by}\isamarkupfalse%
\ {\isacharparenleft}iprover\ intro{\isacharcolon}\ disjI{\isadigit{1}}{\isacharparenright}\isanewline
\ \ \ \ \ \ \isacommand{then}\isamarkupfalse%
\ \isacommand{have}\isamarkupfalse%
\ {\isachardoublequoteopen}finite\ S{\isadigit{2}}\ {\isasymand}\ {\isacharparenleft}S\ {\isacharequal}\ {\isacharbraceleft}a{\isacharcomma}b{\isacharbraceright}\ {\isasymunion}\ S{\isadigit{2}}\ {\isasymor}\ S\ {\isacharequal}\ {\isacharbraceleft}a{\isacharbraceright}\ {\isasymunion}\ S{\isadigit{2}}\ {\isasymor}\ S\ {\isacharequal}\ {\isacharbraceleft}b{\isacharbraceright}\ {\isasymunion}\ S{\isadigit{2}}\ {\isasymor}\ S\ {\isacharequal}\ S{\isadigit{2}}{\isacharparenright}{\isachardoublequoteclose}\isanewline
\ \ \ \ \ \ \ \ \isacommand{using}\isamarkupfalse%
\ {\isacartoucheopen}finite\ S{\isadigit{2}}{\isacartoucheclose}\ \isacommand{by}\isamarkupfalse%
\ {\isacharparenleft}iprover\ intro{\isacharcolon}\ conjI{\isacharparenright}\isanewline
\ \ \ \ \ \ \isacommand{thus}\isamarkupfalse%
\ {\isacharquery}thesis\isanewline
\ \ \ \ \ \ \ \ \isacommand{using}\isamarkupfalse%
\ {\isacartoucheopen}S{\isadigit{2}}\ {\isasymsubseteq}\ B{\isacartoucheclose}\ \isacommand{by}\isamarkupfalse%
\ {\isacharparenleft}rule\ subexI{\isacharparenright}\isanewline
\ \ \ \ \isacommand{qed}\isamarkupfalse%
\isanewline
\ \ \isacommand{next}\isamarkupfalse%
\isanewline
\ \ \ \ \isacommand{assume}\isamarkupfalse%
\ {\isachardoublequoteopen}S{\isadigit{1}}\ {\isacharequal}\ S{\isadigit{2}}{\isachardoublequoteclose}\isanewline
\ \ \ \ \isacommand{have}\isamarkupfalse%
\ {\isachardoublequoteopen}S\ {\isacharequal}\ {\isacharbraceleft}a{\isacharbraceright}\ {\isasymunion}\ S{\isadigit{1}}\ {\isasymor}\ S\ {\isacharequal}\ S{\isadigit{1}}{\isachardoublequoteclose}\isanewline
\ \ \ \ \ \ \isacommand{using}\isamarkupfalse%
\ {\isadigit{1}}\ \isacommand{by}\isamarkupfalse%
\ {\isacharparenleft}rule\ conjunct{\isadigit{2}}{\isacharparenright}\isanewline
\ \ \ \ \isacommand{thus}\isamarkupfalse%
\ {\isacharquery}thesis\isanewline
\ \ \ \ \isacommand{proof}\isamarkupfalse%
\ {\isacharparenleft}rule\ disjE{\isacharparenright}\isanewline
\ \ \ \ \ \ \isacommand{assume}\isamarkupfalse%
\ {\isachardoublequoteopen}S\ {\isacharequal}\ {\isacharbraceleft}a{\isacharbraceright}\ {\isasymunion}\ S{\isadigit{1}}{\isachardoublequoteclose}\isanewline
\ \ \ \ \ \ \isacommand{then}\isamarkupfalse%
\ \isacommand{have}\isamarkupfalse%
\ {\isachardoublequoteopen}S\ {\isacharequal}\ {\isacharbraceleft}a{\isacharbraceright}\ {\isasymunion}\ S{\isadigit{2}}{\isachardoublequoteclose}\isanewline
\ \ \ \ \ \ \ \ \isacommand{by}\isamarkupfalse%
\ {\isacharparenleft}simp\ only{\isacharcolon}\ {\isacartoucheopen}S{\isadigit{1}}\ {\isacharequal}\ S{\isadigit{2}}{\isacartoucheclose}{\isacharparenright}\isanewline
\ \ \ \ \ \ \isacommand{then}\isamarkupfalse%
\ \isacommand{have}\isamarkupfalse%
\ {\isachardoublequoteopen}S\ {\isacharequal}\ {\isacharbraceleft}a{\isacharcomma}b{\isacharbraceright}\ {\isasymunion}\ S{\isadigit{2}}\ {\isasymor}\ S\ {\isacharequal}\ {\isacharbraceleft}a{\isacharbraceright}\ {\isasymunion}\ S{\isadigit{2}}\ {\isasymor}\ S\ {\isacharequal}\ {\isacharbraceleft}b{\isacharbraceright}\ {\isasymunion}\ S{\isadigit{2}}\ {\isasymor}\ S\ {\isacharequal}\ S{\isadigit{2}}{\isachardoublequoteclose}\isanewline
\ \ \ \ \ \ \ \ \isacommand{by}\isamarkupfalse%
\ {\isacharparenleft}iprover\ intro{\isacharcolon}\ disjI{\isadigit{1}}{\isacharparenright}\isanewline
\ \ \ \ \ \ \isacommand{then}\isamarkupfalse%
\ \isacommand{have}\isamarkupfalse%
\ {\isachardoublequoteopen}finite\ S{\isadigit{2}}\ {\isasymand}\ {\isacharparenleft}S\ {\isacharequal}\ {\isacharbraceleft}a{\isacharcomma}b{\isacharbraceright}\ {\isasymunion}\ S{\isadigit{2}}\ {\isasymor}\ S\ {\isacharequal}\ {\isacharbraceleft}a{\isacharbraceright}\ {\isasymunion}\ S{\isadigit{2}}\ {\isasymor}\ S\ {\isacharequal}\ {\isacharbraceleft}b{\isacharbraceright}\ {\isasymunion}\ S{\isadigit{2}}\ {\isasymor}\ S\ {\isacharequal}\ S{\isadigit{2}}{\isacharparenright}{\isachardoublequoteclose}\isanewline
\ \ \ \ \ \ \ \ \isacommand{using}\isamarkupfalse%
\ {\isacartoucheopen}finite\ S{\isadigit{2}}{\isacartoucheclose}\ \isacommand{by}\isamarkupfalse%
\ {\isacharparenleft}iprover\ intro{\isacharcolon}\ conjI{\isacharparenright}\isanewline
\ \ \ \ \ \ \isacommand{thus}\isamarkupfalse%
\ {\isacharquery}thesis\isanewline
\ \ \ \ \ \ \ \ \isacommand{using}\isamarkupfalse%
\ {\isacartoucheopen}S{\isadigit{2}}\ {\isasymsubseteq}\ B{\isacartoucheclose}\ \isacommand{by}\isamarkupfalse%
\ {\isacharparenleft}rule\ subexI{\isacharparenright}\isanewline
\ \ \ \ \isacommand{next}\isamarkupfalse%
\isanewline
\ \ \ \ \ \ \isacommand{assume}\isamarkupfalse%
\ {\isachardoublequoteopen}S\ {\isacharequal}\ S{\isadigit{1}}{\isachardoublequoteclose}\isanewline
\ \ \ \ \ \ \isacommand{then}\isamarkupfalse%
\ \isacommand{have}\isamarkupfalse%
\ {\isachardoublequoteopen}S\ {\isacharequal}\ S{\isadigit{2}}{\isachardoublequoteclose}\isanewline
\ \ \ \ \ \ \ \ \isacommand{by}\isamarkupfalse%
\ {\isacharparenleft}simp\ only{\isacharcolon}\ {\isacartoucheopen}S{\isadigit{1}}\ {\isacharequal}\ S{\isadigit{2}}{\isacartoucheclose}{\isacharparenright}\isanewline
\ \ \ \ \ \ \isacommand{then}\isamarkupfalse%
\ \isacommand{have}\isamarkupfalse%
\ {\isachardoublequoteopen}S\ {\isacharequal}\ {\isacharbraceleft}a{\isacharcomma}b{\isacharbraceright}\ {\isasymunion}\ S{\isadigit{2}}\ {\isasymor}\ S\ {\isacharequal}\ {\isacharbraceleft}a{\isacharbraceright}\ {\isasymunion}\ S{\isadigit{2}}\ {\isasymor}\ S\ {\isacharequal}\ {\isacharbraceleft}b{\isacharbraceright}\ {\isasymunion}\ S{\isadigit{2}}\ {\isasymor}\ S\ {\isacharequal}\ S{\isadigit{2}}{\isachardoublequoteclose}\isanewline
\ \ \ \ \ \ \ \ \isacommand{by}\isamarkupfalse%
\ {\isacharparenleft}iprover\ intro{\isacharcolon}\ disjI{\isadigit{1}}{\isacharparenright}\isanewline
\ \ \ \ \ \ \isacommand{then}\isamarkupfalse%
\ \isacommand{have}\isamarkupfalse%
\ {\isachardoublequoteopen}finite\ S{\isadigit{2}}\ {\isasymand}\ {\isacharparenleft}S\ {\isacharequal}\ {\isacharbraceleft}a{\isacharcomma}b{\isacharbraceright}\ {\isasymunion}\ S{\isadigit{2}}\ {\isasymor}\ S\ {\isacharequal}\ {\isacharbraceleft}a{\isacharbraceright}\ {\isasymunion}\ S{\isadigit{2}}\ {\isasymor}\ S\ {\isacharequal}\ {\isacharbraceleft}b{\isacharbraceright}\ {\isasymunion}\ S{\isadigit{2}}\ {\isasymor}\ S\ {\isacharequal}\ S{\isadigit{2}}{\isacharparenright}{\isachardoublequoteclose}\isanewline
\ \ \ \ \ \ \ \ \isacommand{using}\isamarkupfalse%
\ {\isacartoucheopen}finite\ S{\isadigit{2}}{\isacartoucheclose}\ \isacommand{by}\isamarkupfalse%
\ {\isacharparenleft}iprover\ intro{\isacharcolon}\ conjI{\isacharparenright}\isanewline
\ \ \ \ \ \ \isacommand{thus}\isamarkupfalse%
\ {\isacharquery}thesis\isanewline
\ \ \ \ \ \ \ \ \isacommand{using}\isamarkupfalse%
\ {\isacartoucheopen}S{\isadigit{2}}\ {\isasymsubseteq}\ B{\isacartoucheclose}\ \isacommand{by}\isamarkupfalse%
\ {\isacharparenleft}rule\ subexI{\isacharparenright}\isanewline
\ \ \ \ \isacommand{qed}\isamarkupfalse%
\isanewline
\ \ \isacommand{qed}\isamarkupfalse%
\isanewline
\isacommand{qed}\isamarkupfalse%
%
\endisatagproof
{\isafoldproof}%
%
\isadelimproof
%
\endisadelimproof
%
\begin{isamarkuptext}%
Una vez introducidos los resultados anteriores, procedamos con la prueba del \isa{Teorema\ de\ Compacidad}.

  \begin{demostracion}
    Consideremos la colección \isa{C} formada por los conjuntos de fórmulas finitamente satisfacibles.
    Recordemos que un conjunto de fórmulas es finitamente satisfacible si todo subconjunto finito 
    suyo es satisfacible. Vamos a probar que dicho conjunto verifica la propiedad de consistencia 
    proposicional y, por el \isa{Teorema\ de\ Existencia\ de\ Modelo}, quedará probado que todo conjunto de 
    \isa{C} es satisfacible, lo que demuestra el teorema.

    Para probar que \isa{C} verifica la propiedad de consistencia proposicional, por el lema \isa{{\isadigit{2}}{\isachardot}{\isadigit{0}}{\isachardot}{\isadigit{2}}} de 
    caracterización mediante notación uniforme, basta demostrar que se verifican las siguientes 
    condiciones para todo conjunto \isa{W\ {\isasymin}\ C}:
    \begin{itemize}
     \item \isa{{\isasymbottom}\ {\isasymnotin}\ W}.
     \item Dada \isa{p} una fórmula atómica cualquiera, no se tiene 
      simultáneamente que\\ \isa{p\ {\isasymin}\ W} y \isa{{\isasymnot}\ p\ {\isasymin}\ W}.
     \item Para toda fórmula de tipo \isa{{\isasymalpha}} con componentes \isa{{\isasymalpha}\isactrlsub {\isadigit{1}}} y \isa{{\isasymalpha}\isactrlsub {\isadigit{2}}} tal que \isa{{\isasymalpha}}
      pertenece a \isa{W}, se tiene que \isa{{\isacharbraceleft}{\isasymalpha}\isactrlsub {\isadigit{1}}{\isacharcomma}{\isasymalpha}\isactrlsub {\isadigit{2}}{\isacharbraceright}\ {\isasymunion}\ W} pertenece a \isa{C}.
     \item Para toda fórmula de tipo \isa{{\isasymbeta}} con componentes \isa{{\isasymbeta}\isactrlsub {\isadigit{1}}} y \isa{{\isasymbeta}\isactrlsub {\isadigit{2}}} tal que \isa{{\isasymbeta}}
      pertenece a \isa{W}, se tiene que o bien \isa{{\isacharbraceleft}{\isasymbeta}\isactrlsub {\isadigit{1}}{\isacharbraceright}\ {\isasymunion}\ W} pertenece a \isa{C} o 
      bien \isa{{\isacharbraceleft}{\isasymbeta}\isactrlsub {\isadigit{2}}{\isacharbraceright}\ {\isasymunion}\ W} pertenece a \isa{C}.
    \end{itemize}

    De este modo, consideremos un conjunto cualquiera \isa{W\ {\isasymin}\ C} y procedamos a probar cada una de las
    condiciones anteriores.

    La primera condición se demuestra por reducción al absurdo. En efecto, si suponemos que 
    \isa{{\isasymbottom}\ {\isasymin}\ W}, es claro que \isa{{\isacharbraceleft}{\isasymbottom}{\isacharbraceright}} es un subconjunto finito de \isa{W}. Como \isa{W} es un conjunto
    finitamente satisfacible por pertenecer a \isa{C}, se tiene por lo anterior que \isa{{\isacharbraceleft}{\isasymbottom}{\isacharbraceright}} es 
    satisfacible. De este modo, llegamos a una contradicción pues, por definición, no existe ninguna 
    interpretación que sea modelo de \isa{{\isasymbottom}}.

    A continuación probaremos que, si \isa{W\ {\isasymin}\ C}, entonces dada \isa{p} una fórmula atómica cualquiera, no 
    se tiene simultáneamente que \isa{p\ {\isasymin}\ W} y \isa{{\isasymnot}\ p\ {\isasymin}\ W}. Veamos dicho resultado por reducción al 
    absurdo, suponiendo que tanto \isa{p} como \isa{{\isasymnot}\ p} están en \isa{W}. En este caso, \isa{{\isacharbraceleft}p{\isacharcomma}{\isasymnot}\ p{\isacharbraceright}} sería un
    subconjunto finito de \isa{W} y, por ser \isa{W} finitamente satisfacible ya que \isa{W\ {\isasymin}\ C}, obtendríamos 
    que \isa{{\isacharbraceleft}p{\isacharcomma}{\isasymnot}\ p{\isacharbraceright}} es satisfacible. Sin embargo, esto no es cierto ya que, en ese caso, existiría
    una interpretación que sería modelo tanto de \isa{p} como de \isa{{\isasymnot}\ p}, llegando así a una 
    contradicción.
  \end{demostracion}%
\end{isamarkuptext}\isamarkuptrue%
\isacommand{definition}\isamarkupfalse%
\ colecComp\ {\isacharcolon}{\isacharcolon}\ {\isachardoublequoteopen}{\isacharparenleft}{\isacharprime}a\ formula\ set{\isacharparenright}\ set{\isachardoublequoteclose}\isanewline
\ \ \isakeyword{where}\ colecComp{\isacharcolon}\ {\isachardoublequoteopen}colecComp\ {\isacharequal}\ {\isacharbraceleft}W{\isachardot}\ fin{\isacharunderscore}sat\ W{\isacharbraceright}{\isachardoublequoteclose}\isanewline
\isanewline
\isacommand{lemma}\isamarkupfalse%
\ set{\isacharunderscore}in{\isacharunderscore}colecComp{\isacharcolon}\ \isanewline
\ \ \isakeyword{assumes}\ {\isachardoublequoteopen}fin{\isacharunderscore}sat\ S{\isachardoublequoteclose}\isanewline
\ \ \isakeyword{shows}\ {\isachardoublequoteopen}S\ {\isasymin}\ colecComp{\isachardoublequoteclose}\isanewline
%
\isadelimproof
\ \ %
\endisadelimproof
%
\isatagproof
\isacommand{unfolding}\isamarkupfalse%
\ colecComp\ \isacommand{using}\isamarkupfalse%
\ assms\ \isacommand{unfolding}\isamarkupfalse%
\ fin{\isacharunderscore}sat{\isacharunderscore}def\ \isacommand{by}\isamarkupfalse%
\ {\isacharparenleft}rule\ CollectI{\isacharparenright}%
\endisatagproof
{\isafoldproof}%
%
\isadelimproof
\isanewline
%
\endisadelimproof
\isanewline
\isacommand{lemma}\isamarkupfalse%
\ colecComp{\isacharunderscore}subset{\isacharunderscore}finite{\isacharcolon}\ \isanewline
\ \ \isakeyword{assumes}\ {\isachardoublequoteopen}W\ {\isasymin}\ colecComp{\isachardoublequoteclose}\isanewline
\ \ \ \ \ \ \ \ \ \ {\isachardoublequoteopen}Wo\ {\isasymsubseteq}\ W{\isachardoublequoteclose}\isanewline
\ \ \ \ \ \ \ \ \ \ {\isachardoublequoteopen}finite\ Wo{\isachardoublequoteclose}\isanewline
\ \ \isakeyword{shows}\ {\isachardoublequoteopen}sat\ Wo{\isachardoublequoteclose}\ \isanewline
%
\isadelimproof
%
\endisadelimproof
%
\isatagproof
\isacommand{proof}\isamarkupfalse%
\ {\isacharminus}\isanewline
\ \ \isacommand{have}\isamarkupfalse%
\ {\isachardoublequoteopen}{\isasymforall}Wo\ {\isasymsubseteq}\ W{\isachardot}\ finite\ Wo\ {\isasymlongrightarrow}\ sat\ Wo{\isachardoublequoteclose}\isanewline
\ \ \ \ \isacommand{using}\isamarkupfalse%
\ assms{\isacharparenleft}{\isadigit{1}}{\isacharparenright}\ \isacommand{unfolding}\isamarkupfalse%
\ colecComp\ fin{\isacharunderscore}sat{\isacharunderscore}def\ \isacommand{by}\isamarkupfalse%
\ {\isacharparenleft}rule\ CollectD{\isacharparenright}\isanewline
\ \ \isacommand{then}\isamarkupfalse%
\ \isacommand{have}\isamarkupfalse%
\ {\isachardoublequoteopen}finite\ Wo\ {\isasymlongrightarrow}\ sat\ Wo{\isachardoublequoteclose}\isanewline
\ \ \ \ \isacommand{using}\isamarkupfalse%
\ {\isacartoucheopen}Wo\ {\isasymsubseteq}\ W{\isacartoucheclose}\ \isacommand{by}\isamarkupfalse%
\ {\isacharparenleft}rule\ sspec{\isacharparenright}\isanewline
\ \ \isacommand{thus}\isamarkupfalse%
\ {\isachardoublequoteopen}sat\ Wo{\isachardoublequoteclose}\isanewline
\ \ \ \ \isacommand{using}\isamarkupfalse%
\ {\isacartoucheopen}finite\ Wo{\isacartoucheclose}\ \isacommand{by}\isamarkupfalse%
\ {\isacharparenleft}rule\ mp{\isacharparenright}\isanewline
\isacommand{qed}\isamarkupfalse%
%
\endisatagproof
{\isafoldproof}%
%
\isadelimproof
\isanewline
%
\endisadelimproof
\isanewline
\isacommand{lemma}\isamarkupfalse%
\ not{\isacharunderscore}sat{\isacharunderscore}bot{\isacharcolon}\ {\isachardoublequoteopen}{\isasymnot}\ sat\ {\isacharbraceleft}{\isasymbottom}{\isacharbraceright}{\isachardoublequoteclose}\isanewline
%
\isadelimproof
%
\endisadelimproof
%
\isatagproof
\isacommand{proof}\isamarkupfalse%
\ {\isacharparenleft}rule\ ccontr{\isacharparenright}\isanewline
\ \ \isacommand{assume}\isamarkupfalse%
\ {\isachardoublequoteopen}{\isasymnot}{\isacharparenleft}{\isasymnot}sat{\isacharbraceleft}{\isasymbottom}\ {\isacharcolon}{\isacharcolon}\ {\isacharprime}a\ formula{\isacharbraceright}{\isacharparenright}{\isachardoublequoteclose}\isanewline
\ \ \isacommand{then}\isamarkupfalse%
\ \isacommand{have}\isamarkupfalse%
\ {\isachardoublequoteopen}sat\ {\isacharbraceleft}{\isasymbottom}\ {\isacharcolon}{\isacharcolon}\ {\isacharprime}a\ formula{\isacharbraceright}{\isachardoublequoteclose}\isanewline
\ \ \ \ \isacommand{by}\isamarkupfalse%
\ {\isacharparenleft}rule\ notnotD{\isacharparenright}\isanewline
\ \ \isacommand{then}\isamarkupfalse%
\ \isacommand{have}\isamarkupfalse%
\ Ex{\isacharcolon}{\isachardoublequoteopen}{\isasymexists}{\isasymA}{\isachardot}\ {\isasymforall}F\ {\isasymin}\ {\isacharbraceleft}{\isasymbottom}\ {\isacharcolon}{\isacharcolon}\ {\isacharprime}a\ formula{\isacharbraceright}{\isachardot}\ {\isasymA}\ {\isasymTurnstile}\ F{\isachardoublequoteclose}\isanewline
\ \ \ \ \isacommand{by}\isamarkupfalse%
\ {\isacharparenleft}simp\ only{\isacharcolon}\ sat{\isacharunderscore}def{\isacharparenright}\isanewline
\ \ \isacommand{obtain}\isamarkupfalse%
\ {\isasymA}\ \isakeyword{where}\ {\isadigit{1}}{\isacharcolon}{\isachardoublequoteopen}{\isasymforall}F\ {\isasymin}\ {\isacharbraceleft}{\isasymbottom}\ {\isacharcolon}{\isacharcolon}\ {\isacharprime}a\ formula{\isacharbraceright}{\isachardot}\ {\isasymA}\ {\isasymTurnstile}\ F{\isachardoublequoteclose}\isanewline
\ \ \ \ \isacommand{using}\isamarkupfalse%
\ Ex\ \isacommand{by}\isamarkupfalse%
\ {\isacharparenleft}rule\ exE{\isacharparenright}\isanewline
\ \ \isacommand{have}\isamarkupfalse%
\ {\isadigit{2}}{\isacharcolon}{\isachardoublequoteopen}{\isasymbottom}\ {\isasymin}\ {\isacharbraceleft}{\isasymbottom}{\isacharcolon}{\isacharcolon}\ {\isacharprime}a\ formula{\isacharbraceright}{\isachardoublequoteclose}\isanewline
\ \ \ \ \isacommand{by}\isamarkupfalse%
\ {\isacharparenleft}simp\ only{\isacharcolon}\ singletonI{\isacharparenright}\isanewline
\ \ \isacommand{have}\isamarkupfalse%
\ {\isachardoublequoteopen}{\isasymA}\ {\isasymTurnstile}\ {\isasymbottom}{\isachardoublequoteclose}\isanewline
\ \ \ \ \isacommand{using}\isamarkupfalse%
\ {\isadigit{1}}\ {\isadigit{2}}\ \isacommand{by}\isamarkupfalse%
\ {\isacharparenleft}rule\ bspec{\isacharparenright}\isanewline
\ \ \isacommand{thus}\isamarkupfalse%
\ {\isachardoublequoteopen}False{\isachardoublequoteclose}\isanewline
\ \ \ \ \isacommand{by}\isamarkupfalse%
\ {\isacharparenleft}simp\ only{\isacharcolon}\ formula{\isacharunderscore}semantics{\isachardot}simps{\isacharparenleft}{\isadigit{2}}{\isacharparenright}{\isacharparenright}\isanewline
\isacommand{qed}\isamarkupfalse%
%
\endisatagproof
{\isafoldproof}%
%
\isadelimproof
\isanewline
%
\endisadelimproof
\isanewline
\isacommand{lemma}\isamarkupfalse%
\ pcp{\isacharunderscore}colecComp{\isacharunderscore}bot{\isacharcolon}\isanewline
\ \ \isakeyword{assumes}\ {\isachardoublequoteopen}W\ {\isasymin}\ colecComp{\isachardoublequoteclose}\isanewline
\ \ \isakeyword{shows}\ {\isachardoublequoteopen}{\isasymbottom}\ {\isasymnotin}\ W{\isachardoublequoteclose}\isanewline
%
\isadelimproof
%
\endisadelimproof
%
\isatagproof
\isacommand{proof}\isamarkupfalse%
\ {\isacharparenleft}rule\ ccontr{\isacharparenright}\isanewline
\ \ \isacommand{assume}\isamarkupfalse%
\ {\isachardoublequoteopen}{\isasymnot}{\isacharparenleft}{\isasymbottom}\ {\isasymnotin}\ W{\isacharparenright}{\isachardoublequoteclose}\isanewline
\ \ \isacommand{then}\isamarkupfalse%
\ \isacommand{have}\isamarkupfalse%
\ {\isachardoublequoteopen}{\isasymbottom}\ {\isasymin}\ W{\isachardoublequoteclose}\isanewline
\ \ \ \ \isacommand{by}\isamarkupfalse%
\ {\isacharparenleft}rule\ notnotD{\isacharparenright}\isanewline
\ \ \isacommand{have}\isamarkupfalse%
\ {\isachardoublequoteopen}{\isacharbraceleft}{\isacharbraceright}\ {\isasymsubseteq}\ W{\isachardoublequoteclose}\ \isanewline
\ \ \ \ \isacommand{by}\isamarkupfalse%
\ {\isacharparenleft}simp\ only{\isacharcolon}\ empty{\isacharunderscore}subsetI{\isacharparenright}\ \isanewline
\ \ \isacommand{have}\isamarkupfalse%
\ {\isachardoublequoteopen}{\isasymbottom}\ {\isasymin}\ W\ {\isasymand}\ {\isacharbraceleft}{\isacharbraceright}\ {\isasymsubseteq}\ W{\isachardoublequoteclose}\isanewline
\ \ \ \ \isacommand{using}\isamarkupfalse%
\ {\isacartoucheopen}{\isasymbottom}\ {\isasymin}\ W{\isacartoucheclose}\ {\isacartoucheopen}{\isacharbraceleft}{\isacharbraceright}\ {\isasymsubseteq}\ W{\isacartoucheclose}\ \isacommand{by}\isamarkupfalse%
\ {\isacharparenleft}rule\ conjI{\isacharparenright}\isanewline
\ \ \isacommand{then}\isamarkupfalse%
\ \isacommand{have}\isamarkupfalse%
\ {\isachardoublequoteopen}{\isacharbraceleft}{\isasymbottom}{\isacharbraceright}\ {\isasymsubseteq}\ W{\isachardoublequoteclose}\isanewline
\ \ \ \ \isacommand{by}\isamarkupfalse%
\ {\isacharparenleft}simp\ only{\isacharcolon}\ insert{\isacharunderscore}subset{\isacharparenright}\isanewline
\ \ \isacommand{have}\isamarkupfalse%
\ {\isachardoublequoteopen}finite\ {\isacharbraceleft}{\isasymbottom}{\isacharbraceright}{\isachardoublequoteclose}\ \isanewline
\ \ \ \ \isacommand{by}\isamarkupfalse%
\ {\isacharparenleft}simp\ only{\isacharcolon}\ finite{\isachardot}emptyI\ finite{\isacharunderscore}insert{\isacharparenright}\isanewline
\ \ \isacommand{have}\isamarkupfalse%
\ {\isachardoublequoteopen}sat\ {\isacharbraceleft}{\isasymbottom}\ {\isacharcolon}{\isacharcolon}\ {\isacharprime}a\ formula{\isacharbraceright}{\isachardoublequoteclose}\ \isanewline
\ \ \ \ \isacommand{using}\isamarkupfalse%
\ assms\ {\isacartoucheopen}{\isacharbraceleft}{\isasymbottom}{\isacharbraceright}\ {\isasymsubseteq}\ W{\isacartoucheclose}\ {\isacartoucheopen}finite\ {\isacharbraceleft}{\isasymbottom}{\isacharbraceright}{\isacartoucheclose}\ \isacommand{by}\isamarkupfalse%
\ {\isacharparenleft}rule\ colecComp{\isacharunderscore}subset{\isacharunderscore}finite{\isacharparenright}\isanewline
\ \ \isacommand{have}\isamarkupfalse%
\ {\isachardoublequoteopen}{\isasymnot}\ sat\ {\isacharbraceleft}{\isasymbottom}\ {\isacharcolon}{\isacharcolon}\ {\isacharprime}a\ formula{\isacharbraceright}{\isachardoublequoteclose}\ \isanewline
\ \ \ \ \isacommand{by}\isamarkupfalse%
\ {\isacharparenleft}rule\ not{\isacharunderscore}sat{\isacharunderscore}bot{\isacharparenright}\isanewline
\ \ \isacommand{then}\isamarkupfalse%
\ \isacommand{show}\isamarkupfalse%
\ False\ \isanewline
\ \ \ \ \isacommand{using}\isamarkupfalse%
\ {\isacartoucheopen}sat\ {\isacharbraceleft}{\isasymbottom}\ {\isacharcolon}{\isacharcolon}\ {\isacharprime}a\ formula{\isacharbraceright}{\isacartoucheclose}\ \isacommand{by}\isamarkupfalse%
\ {\isacharparenleft}rule\ notE{\isacharparenright}\isanewline
\isacommand{qed}\isamarkupfalse%
%
\endisatagproof
{\isafoldproof}%
%
\isadelimproof
\isanewline
%
\endisadelimproof
\isanewline
\isacommand{lemma}\isamarkupfalse%
\ not{\isacharunderscore}sat{\isacharunderscore}atoms{\isacharcolon}\ {\isachardoublequoteopen}{\isasymnot}\ sat{\isacharparenleft}{\isacharbraceleft}Atom\ k{\isacharcomma}\ \isactrlbold {\isasymnot}\ {\isacharparenleft}Atom\ k{\isacharparenright}{\isacharbraceright}{\isacharparenright}{\isachardoublequoteclose}\isanewline
%
\isadelimproof
%
\endisadelimproof
%
\isatagproof
\isacommand{proof}\isamarkupfalse%
\ {\isacharparenleft}rule\ ccontr{\isacharparenright}\isanewline
\ \ \isacommand{assume}\isamarkupfalse%
\ {\isachardoublequoteopen}{\isasymnot}\ {\isasymnot}\ sat{\isacharparenleft}{\isacharbraceleft}Atom\ k{\isacharcomma}\ \isactrlbold {\isasymnot}\ {\isacharparenleft}Atom\ k{\isacharparenright}{\isacharbraceright}{\isacharparenright}{\isachardoublequoteclose}\isanewline
\ \ \isacommand{then}\isamarkupfalse%
\ \isacommand{have}\isamarkupfalse%
\ {\isachardoublequoteopen}sat{\isacharparenleft}{\isacharbraceleft}Atom\ k{\isacharcomma}\ \isactrlbold {\isasymnot}\ {\isacharparenleft}Atom\ k{\isacharparenright}{\isacharbraceright}{\isacharparenright}{\isachardoublequoteclose}\isanewline
\ \ \ \ \isacommand{by}\isamarkupfalse%
\ {\isacharparenleft}rule\ notnotD{\isacharparenright}\isanewline
\ \ \isacommand{then}\isamarkupfalse%
\ \isacommand{have}\isamarkupfalse%
\ Sat{\isacharcolon}{\isachardoublequoteopen}{\isasymexists}{\isasymA}{\isachardot}\ {\isasymforall}F\ {\isasymin}\ {\isacharbraceleft}Atom\ k{\isacharcomma}\ \isactrlbold {\isasymnot}{\isacharparenleft}Atom\ k{\isacharparenright}{\isacharbraceright}{\isachardot}\ {\isasymA}\ {\isasymTurnstile}\ F{\isachardoublequoteclose}\isanewline
\ \ \ \ \isacommand{by}\isamarkupfalse%
\ {\isacharparenleft}simp\ only{\isacharcolon}\ sat{\isacharunderscore}def{\isacharparenright}\isanewline
\ \ \isacommand{obtain}\isamarkupfalse%
\ {\isasymA}\ \isakeyword{where}\ H{\isacharcolon}{\isachardoublequoteopen}{\isasymforall}F\ {\isasymin}\ {\isacharbraceleft}Atom\ k{\isacharcomma}\ \isactrlbold {\isasymnot}{\isacharparenleft}Atom\ k{\isacharparenright}{\isacharbraceright}{\isachardot}\ {\isasymA}\ {\isasymTurnstile}\ F{\isachardoublequoteclose}\isanewline
\ \ \ \ \isacommand{using}\isamarkupfalse%
\ Sat\ \isacommand{by}\isamarkupfalse%
\ {\isacharparenleft}rule\ exE{\isacharparenright}\isanewline
\ \ \isacommand{have}\isamarkupfalse%
\ {\isachardoublequoteopen}Atom\ k\ {\isasymin}\ {\isacharbraceleft}Atom\ k{\isacharcomma}\ \isactrlbold {\isasymnot}{\isacharparenleft}Atom\ k{\isacharparenright}{\isacharbraceright}{\isachardoublequoteclose}\isanewline
\ \ \ \ \isacommand{by}\isamarkupfalse%
\ simp\isanewline
\ \ \isacommand{have}\isamarkupfalse%
\ {\isachardoublequoteopen}{\isasymA}\ {\isasymTurnstile}\ Atom\ k{\isachardoublequoteclose}\isanewline
\ \ \ \ \isacommand{using}\isamarkupfalse%
\ H\ {\isacartoucheopen}Atom\ k\ {\isasymin}\ {\isacharbraceleft}Atom\ k{\isacharcomma}\ \isactrlbold {\isasymnot}{\isacharparenleft}Atom\ k{\isacharparenright}{\isacharbraceright}{\isacartoucheclose}\ \isacommand{by}\isamarkupfalse%
\ {\isacharparenleft}rule\ bspec{\isacharparenright}\isanewline
\ \ \isacommand{have}\isamarkupfalse%
\ {\isachardoublequoteopen}\isactrlbold {\isasymnot}{\isacharparenleft}Atom\ k{\isacharparenright}\ {\isasymin}\ {\isacharbraceleft}Atom\ k{\isacharcomma}\ \isactrlbold {\isasymnot}{\isacharparenleft}Atom\ k{\isacharparenright}{\isacharbraceright}{\isachardoublequoteclose}\isanewline
\ \ \ \ \isacommand{by}\isamarkupfalse%
\ simp\isanewline
\ \ \isacommand{have}\isamarkupfalse%
\ {\isachardoublequoteopen}{\isasymA}\ {\isasymTurnstile}\ \isactrlbold {\isasymnot}{\isacharparenleft}Atom\ k{\isacharparenright}{\isachardoublequoteclose}\isanewline
\ \ \ \ \isacommand{using}\isamarkupfalse%
\ H\ {\isacartoucheopen}\isactrlbold {\isasymnot}{\isacharparenleft}Atom\ k{\isacharparenright}\ {\isasymin}\ {\isacharbraceleft}Atom\ k{\isacharcomma}\ \isactrlbold {\isasymnot}{\isacharparenleft}Atom\ k{\isacharparenright}{\isacharbraceright}{\isacartoucheclose}\ \isacommand{by}\isamarkupfalse%
\ {\isacharparenleft}rule\ bspec{\isacharparenright}\isanewline
\ \ \isacommand{then}\isamarkupfalse%
\ \isacommand{have}\isamarkupfalse%
\ {\isachardoublequoteopen}{\isasymnot}\ {\isasymA}\ {\isasymTurnstile}\ Atom\ k{\isachardoublequoteclose}\ \isanewline
\ \ \ \ \isacommand{by}\isamarkupfalse%
\ {\isacharparenleft}simp\ only{\isacharcolon}\ simp{\isacharunderscore}thms{\isacharparenleft}{\isadigit{8}}{\isacharparenright}\ formula{\isacharunderscore}semantics{\isachardot}simps{\isacharparenleft}{\isadigit{3}}{\isacharparenright}{\isacharparenright}\isanewline
\ \ \isacommand{thus}\isamarkupfalse%
\ {\isachardoublequoteopen}False{\isachardoublequoteclose}\isanewline
\ \ \ \ \isacommand{using}\isamarkupfalse%
\ {\isacartoucheopen}{\isasymA}\ {\isasymTurnstile}\ Atom\ k{\isacartoucheclose}\ \isacommand{by}\isamarkupfalse%
\ {\isacharparenleft}rule\ notE{\isacharparenright}\isanewline
\isacommand{qed}\isamarkupfalse%
%
\endisatagproof
{\isafoldproof}%
%
\isadelimproof
\isanewline
%
\endisadelimproof
\isanewline
\isacommand{lemma}\isamarkupfalse%
\ pcp{\isacharunderscore}colecComp{\isacharunderscore}atoms{\isacharcolon}\isanewline
\ \ \isakeyword{assumes}\ {\isachardoublequoteopen}W\ {\isasymin}\ colecComp{\isachardoublequoteclose}\isanewline
\ \ \isakeyword{shows}\ {\isachardoublequoteopen}{\isasymforall}k{\isachardot}\ Atom\ k\ {\isasymin}\ W\ {\isasymlongrightarrow}\ \isactrlbold {\isasymnot}\ {\isacharparenleft}Atom\ k{\isacharparenright}\ {\isasymin}\ W\ {\isasymlongrightarrow}\ False{\isachardoublequoteclose}\isanewline
%
\isadelimproof
%
\endisadelimproof
%
\isatagproof
\isacommand{proof}\isamarkupfalse%
\ {\isacharparenleft}rule\ allI{\isacharparenright}\isanewline
\ \ \isacommand{fix}\isamarkupfalse%
\ k\isanewline
\ \ \isacommand{show}\isamarkupfalse%
\ {\isachardoublequoteopen}Atom\ k\ {\isasymin}\ W\ {\isasymlongrightarrow}\ \isactrlbold {\isasymnot}\ {\isacharparenleft}Atom\ k{\isacharparenright}\ {\isasymin}\ W\ {\isasymlongrightarrow}\ False{\isachardoublequoteclose}\isanewline
\ \ \isacommand{proof}\isamarkupfalse%
\ {\isacharparenleft}rule\ impI{\isacharparenright}{\isacharplus}\isanewline
\ \ \ \ \isacommand{assume}\isamarkupfalse%
\ {\isadigit{1}}{\isacharcolon}{\isachardoublequoteopen}Atom\ k\ {\isasymin}\ W{\isachardoublequoteclose}\isanewline
\ \ \ \ \isacommand{assume}\isamarkupfalse%
\ {\isadigit{2}}{\isacharcolon}{\isachardoublequoteopen}\isactrlbold {\isasymnot}\ {\isacharparenleft}Atom\ k{\isacharparenright}\ {\isasymin}\ W{\isachardoublequoteclose}\isanewline
\ \ \ \ \isacommand{have}\isamarkupfalse%
\ {\isachardoublequoteopen}{\isacharbraceleft}{\isacharbraceright}\ {\isasymsubseteq}\ W{\isachardoublequoteclose}\isanewline
\ \ \ \ \ \ \isacommand{by}\isamarkupfalse%
\ {\isacharparenleft}simp\ only{\isacharcolon}\ empty{\isacharunderscore}subsetI{\isacharparenright}\ \isanewline
\ \ \ \ \isacommand{have}\isamarkupfalse%
\ {\isachardoublequoteopen}Atom\ k\ {\isasymin}\ W\ {\isasymand}\ {\isacharbraceleft}{\isacharbraceright}\ {\isasymsubseteq}\ W{\isachardoublequoteclose}\isanewline
\ \ \ \ \ \ \isacommand{using}\isamarkupfalse%
\ {\isadigit{1}}\ {\isacartoucheopen}{\isacharbraceleft}{\isacharbraceright}\ {\isasymsubseteq}\ W{\isacartoucheclose}\ \isacommand{by}\isamarkupfalse%
\ {\isacharparenleft}rule\ conjI{\isacharparenright}\isanewline
\ \ \ \ \isacommand{then}\isamarkupfalse%
\ \isacommand{have}\isamarkupfalse%
\ {\isachardoublequoteopen}{\isacharbraceleft}Atom\ k{\isacharbraceright}\ {\isasymsubseteq}\ W{\isachardoublequoteclose}\isanewline
\ \ \ \ \ \ \isacommand{by}\isamarkupfalse%
\ {\isacharparenleft}simp\ only{\isacharcolon}\ insert{\isacharunderscore}subset{\isacharparenright}\isanewline
\ \ \ \ \isacommand{have}\isamarkupfalse%
\ {\isachardoublequoteopen}\isactrlbold {\isasymnot}\ {\isacharparenleft}Atom\ k{\isacharparenright}\ {\isasymin}\ W\ {\isasymand}\ {\isacharbraceleft}{\isacharbraceright}\ {\isasymsubseteq}\ W{\isachardoublequoteclose}\isanewline
\ \ \ \ \ \ \isacommand{using}\isamarkupfalse%
\ {\isadigit{2}}\ {\isacartoucheopen}{\isacharbraceleft}{\isacharbraceright}\ {\isasymsubseteq}\ W{\isacartoucheclose}\ \isacommand{by}\isamarkupfalse%
\ {\isacharparenleft}rule\ conjI{\isacharparenright}\isanewline
\ \ \ \ \isacommand{then}\isamarkupfalse%
\ \isacommand{have}\isamarkupfalse%
\ {\isachardoublequoteopen}{\isacharbraceleft}\isactrlbold {\isasymnot}{\isacharparenleft}Atom\ k{\isacharparenright}{\isacharbraceright}\ {\isasymsubseteq}\ W{\isachardoublequoteclose}\isanewline
\ \ \ \ \ \ \isacommand{by}\isamarkupfalse%
\ {\isacharparenleft}simp\ only{\isacharcolon}\ insert{\isacharunderscore}subset{\isacharparenright}\isanewline
\ \ \ \ \isacommand{have}\isamarkupfalse%
\ {\isachardoublequoteopen}{\isacharbraceleft}Atom\ k{\isacharbraceright}\ {\isasymunion}\ {\isacharbraceleft}\isactrlbold {\isasymnot}{\isacharparenleft}Atom\ k{\isacharparenright}{\isacharbraceright}\ {\isasymsubseteq}\ W{\isachardoublequoteclose}\isanewline
\ \ \ \ \ \ \isacommand{using}\isamarkupfalse%
\ {\isacartoucheopen}{\isacharbraceleft}Atom\ k{\isacharbraceright}\ {\isasymsubseteq}\ W{\isacartoucheclose}\ {\isacartoucheopen}{\isacharbraceleft}\isactrlbold {\isasymnot}{\isacharparenleft}Atom\ k{\isacharparenright}{\isacharbraceright}\ {\isasymsubseteq}\ W{\isacartoucheclose}\ \isacommand{by}\isamarkupfalse%
\ {\isacharparenleft}simp\ only{\isacharcolon}\ Un{\isacharunderscore}least{\isacharparenright}\isanewline
\ \ \ \ \isacommand{then}\isamarkupfalse%
\ \isacommand{have}\isamarkupfalse%
\ {\isachardoublequoteopen}{\isacharbraceleft}Atom\ k{\isacharcomma}\ \isactrlbold {\isasymnot}{\isacharparenleft}Atom\ k{\isacharparenright}{\isacharbraceright}\ {\isasymsubseteq}\ W{\isachardoublequoteclose}\isanewline
\ \ \ \ \ \ \isacommand{by}\isamarkupfalse%
\ simp\ \isanewline
\ \ \ \ \isacommand{have}\isamarkupfalse%
\ {\isachardoublequoteopen}finite\ {\isacharbraceleft}Atom\ k{\isacharcomma}\ \isactrlbold {\isasymnot}{\isacharparenleft}Atom\ k{\isacharparenright}{\isacharbraceright}{\isachardoublequoteclose}\isanewline
\ \ \ \ \ \ \isacommand{by}\isamarkupfalse%
\ blast\isanewline
\ \ \ \ \isacommand{have}\isamarkupfalse%
\ {\isachardoublequoteopen}sat\ {\isacharparenleft}{\isacharbraceleft}Atom\ k{\isacharcomma}\ \isactrlbold {\isasymnot}{\isacharparenleft}Atom\ k{\isacharparenright}{\isacharbraceright}{\isacharparenright}{\isachardoublequoteclose}\isanewline
\ \ \ \ \ \ \isacommand{using}\isamarkupfalse%
\ assms\ {\isacartoucheopen}{\isacharbraceleft}Atom\ k{\isacharcomma}\ \isactrlbold {\isasymnot}{\isacharparenleft}Atom\ k{\isacharparenright}{\isacharbraceright}\ {\isasymsubseteq}\ W{\isacartoucheclose}\ {\isacartoucheopen}finite\ {\isacharbraceleft}Atom\ k{\isacharcomma}\ \isactrlbold {\isasymnot}{\isacharparenleft}Atom\ k{\isacharparenright}{\isacharbraceright}{\isacartoucheclose}\ \isacommand{by}\isamarkupfalse%
\ {\isacharparenleft}rule\ colecComp{\isacharunderscore}subset{\isacharunderscore}finite{\isacharparenright}\isanewline
\ \ \ \ \isacommand{have}\isamarkupfalse%
\ {\isachardoublequoteopen}{\isasymnot}\ sat\ {\isacharparenleft}{\isacharbraceleft}Atom\ k{\isacharcomma}\ \isactrlbold {\isasymnot}{\isacharparenleft}Atom\ k{\isacharparenright}{\isacharbraceright}{\isacharparenright}{\isachardoublequoteclose}\isanewline
\ \ \ \ \ \ \isacommand{by}\isamarkupfalse%
\ {\isacharparenleft}rule\ not{\isacharunderscore}sat{\isacharunderscore}atoms{\isacharparenright}\isanewline
\ \ \ \ \isacommand{thus}\isamarkupfalse%
\ {\isachardoublequoteopen}False{\isachardoublequoteclose}\isanewline
\ \ \ \ \ \ \isacommand{using}\isamarkupfalse%
\ {\isacartoucheopen}sat\ {\isacharparenleft}{\isacharbraceleft}Atom\ k{\isacharcomma}\ \isactrlbold {\isasymnot}{\isacharparenleft}Atom\ k{\isacharparenright}{\isacharbraceright}{\isacharparenright}{\isacartoucheclose}\ \isacommand{by}\isamarkupfalse%
\ {\isacharparenleft}rule\ notE{\isacharparenright}\isanewline
\ \ \isacommand{qed}\isamarkupfalse%
\isanewline
\isacommand{qed}\isamarkupfalse%
%
\endisatagproof
{\isafoldproof}%
%
\isadelimproof
\isanewline
%
\endisadelimproof
\isanewline
\isacommand{lemma}\isamarkupfalse%
\ pcp{\isacharunderscore}colecComp{\isacharunderscore}elem{\isacharunderscore}sat{\isacharcolon}\isanewline
\ \ \isakeyword{assumes}\ {\isachardoublequoteopen}W\ {\isasymin}\ colecComp{\isachardoublequoteclose}\isanewline
\ \ \ \ \ \ \ \ \ \ {\isachardoublequoteopen}F\ {\isasymin}\ W{\isachardoublequoteclose}\isanewline
\ \ \ \ \ \ \ \ \ \ {\isachardoublequoteopen}finite\ Wo{\isachardoublequoteclose}\isanewline
\ \ \ \ \ \ \ \ \ \ {\isachardoublequoteopen}Wo\ {\isasymsubseteq}\ W{\isachardoublequoteclose}\isanewline
\ \ \ \ \ \ \ \ \isakeyword{shows}\ {\isachardoublequoteopen}sat\ {\isacharparenleft}{\isacharbraceleft}F{\isacharbraceright}\ {\isasymunion}\ Wo{\isacharparenright}{\isachardoublequoteclose}\isanewline
%
\isadelimproof
%
\endisadelimproof
%
\isatagproof
\isacommand{proof}\isamarkupfalse%
\ {\isacharminus}\isanewline
\ \ \isacommand{have}\isamarkupfalse%
\ {\isadigit{1}}{\isacharcolon}{\isachardoublequoteopen}insert\ F\ Wo\ {\isacharequal}\ {\isacharbraceleft}F{\isacharbraceright}\ {\isasymunion}\ Wo{\isachardoublequoteclose}\isanewline
\ \ \ \ \isacommand{by}\isamarkupfalse%
\ {\isacharparenleft}rule\ insert{\isacharunderscore}is{\isacharunderscore}Un{\isacharparenright}\isanewline
\ \ \isacommand{have}\isamarkupfalse%
\ {\isachardoublequoteopen}finite\ {\isacharparenleft}insert\ F\ Wo{\isacharparenright}{\isachardoublequoteclose}\isanewline
\ \ \ \ \isacommand{using}\isamarkupfalse%
\ assms{\isacharparenleft}{\isadigit{3}}{\isacharparenright}\ \isacommand{by}\isamarkupfalse%
\ {\isacharparenleft}simp\ only{\isacharcolon}\ finite{\isacharunderscore}insert{\isacharparenright}\isanewline
\ \ \isacommand{then}\isamarkupfalse%
\ \isacommand{have}\isamarkupfalse%
\ {\isachardoublequoteopen}finite\ {\isacharparenleft}{\isacharbraceleft}F{\isacharbraceright}\ {\isasymunion}\ Wo{\isacharparenright}{\isachardoublequoteclose}\isanewline
\ \ \ \ \isacommand{by}\isamarkupfalse%
\ {\isacharparenleft}simp\ only{\isacharcolon}\ {\isadigit{1}}{\isacharparenright}\ \isanewline
\ \ \isacommand{have}\isamarkupfalse%
\ {\isachardoublequoteopen}F\ {\isasymin}\ W\ {\isasymand}\ Wo\ {\isasymsubseteq}\ W{\isachardoublequoteclose}\isanewline
\ \ \ \ \isacommand{using}\isamarkupfalse%
\ assms{\isacharparenleft}{\isadigit{2}}{\isacharparenright}\ assms{\isacharparenleft}{\isadigit{4}}{\isacharparenright}\ \isacommand{by}\isamarkupfalse%
\ {\isacharparenleft}rule\ conjI{\isacharparenright}\isanewline
\ \ \isacommand{then}\isamarkupfalse%
\ \isacommand{have}\isamarkupfalse%
\ {\isachardoublequoteopen}insert\ F\ Wo\ {\isasymsubseteq}\ W{\isachardoublequoteclose}\isanewline
\ \ \ \ \isacommand{by}\isamarkupfalse%
\ {\isacharparenleft}simp\ only{\isacharcolon}\ insert{\isacharunderscore}subset{\isacharparenright}\isanewline
\ \ \isacommand{then}\isamarkupfalse%
\ \isacommand{have}\isamarkupfalse%
\ {\isachardoublequoteopen}{\isacharbraceleft}F{\isacharbraceright}\ {\isasymunion}\ Wo\ {\isasymsubseteq}\ W{\isachardoublequoteclose}\isanewline
\ \ \ \ \isacommand{by}\isamarkupfalse%
\ {\isacharparenleft}simp\ only{\isacharcolon}\ {\isadigit{1}}{\isacharparenright}\isanewline
\ \ \isacommand{show}\isamarkupfalse%
\ {\isachardoublequoteopen}sat\ {\isacharparenleft}{\isacharbraceleft}F{\isacharbraceright}\ {\isasymunion}\ Wo{\isacharparenright}{\isachardoublequoteclose}\isanewline
\ \ \ \ \isacommand{using}\isamarkupfalse%
\ assms{\isacharparenleft}{\isadigit{1}}{\isacharparenright}\ {\isacartoucheopen}{\isacharbraceleft}F{\isacharbraceright}\ {\isasymunion}\ Wo\ {\isasymsubseteq}\ W{\isacartoucheclose}\ {\isacartoucheopen}finite\ {\isacharparenleft}{\isacharbraceleft}F{\isacharbraceright}\ {\isasymunion}\ Wo{\isacharparenright}{\isacartoucheclose}\ \isacommand{by}\isamarkupfalse%
\ {\isacharparenleft}rule\ colecComp{\isacharunderscore}subset{\isacharunderscore}finite{\isacharparenright}\isanewline
\isacommand{qed}\isamarkupfalse%
%
\endisatagproof
{\isafoldproof}%
%
\isadelimproof
\isanewline
%
\endisadelimproof
\isanewline
\isacommand{lemma}\isamarkupfalse%
\ ball{\isacharunderscore}Un{\isacharcolon}\ \isanewline
\ \ \isakeyword{assumes}\ {\isachardoublequoteopen}{\isasymforall}x\ {\isasymin}\ A{\isachardot}\ P\ x{\isachardoublequoteclose}\isanewline
\ \ \ \ \ \ \ \ \ \ {\isachardoublequoteopen}{\isasymforall}x\ {\isasymin}\ B{\isachardot}\ P\ x{\isachardoublequoteclose}\isanewline
\ \ \ \ \ \ \ \ \isakeyword{shows}\ {\isachardoublequoteopen}{\isasymforall}x\ {\isasymin}\ {\isacharparenleft}A\ {\isasymunion}\ B{\isacharparenright}{\isachardot}\ P\ x{\isachardoublequoteclose}\ \isanewline
%
\isadelimproof
\ \ %
\endisadelimproof
%
\isatagproof
\isacommand{using}\isamarkupfalse%
\ assms\ \isacommand{by}\isamarkupfalse%
\ blast%
\endisatagproof
{\isafoldproof}%
%
\isadelimproof
\isanewline
%
\endisadelimproof
\isanewline
\isacommand{lemma}\isamarkupfalse%
\ pcp{\isacharunderscore}colecComp{\isacharunderscore}CON{\isacharunderscore}sat{\isadigit{1}}{\isacharcolon}\isanewline
\ \ \isakeyword{assumes}\ {\isachardoublequoteopen}W\ {\isasymin}\ colecComp{\isachardoublequoteclose}\isanewline
\ \ \ \ \ \ \ \ \ \ {\isachardoublequoteopen}F\ {\isacharequal}\ G\ \isactrlbold {\isasymand}\ H{\isachardoublequoteclose}\isanewline
\ \ \ \ \ \ \ \ \ \ {\isachardoublequoteopen}F\ {\isasymin}\ W{\isachardoublequoteclose}\isanewline
\ \ \ \ \ \ \ \ \ \ {\isachardoublequoteopen}finite\ Wo{\isachardoublequoteclose}\isanewline
\ \ \ \ \ \ \ \ \ \ {\isachardoublequoteopen}Wo\ {\isasymsubseteq}\ W{\isachardoublequoteclose}\isanewline
\ \ \ \ \ \ \ \ \isakeyword{shows}\ {\isachardoublequoteopen}sat\ {\isacharparenleft}{\isacharbraceleft}G{\isacharcomma}H{\isacharcomma}F{\isacharbraceright}\ {\isasymunion}\ Wo{\isacharparenright}{\isachardoublequoteclose}\isanewline
%
\isadelimproof
%
\endisadelimproof
%
\isatagproof
\isacommand{proof}\isamarkupfalse%
\ {\isacharminus}\isanewline
\ \ \isacommand{have}\isamarkupfalse%
\ {\isachardoublequoteopen}sat\ {\isacharparenleft}{\isacharbraceleft}F{\isacharbraceright}\ {\isasymunion}\ Wo{\isacharparenright}{\isachardoublequoteclose}\isanewline
\ \ \ \ \isacommand{using}\isamarkupfalse%
\ assms{\isacharparenleft}{\isadigit{1}}{\isacharcomma}{\isadigit{3}}{\isacharcomma}{\isadigit{4}}{\isacharcomma}{\isadigit{5}}{\isacharparenright}\ \isacommand{by}\isamarkupfalse%
\ {\isacharparenleft}rule\ pcp{\isacharunderscore}colecComp{\isacharunderscore}elem{\isacharunderscore}sat{\isacharparenright}\isanewline
\ \ \isacommand{have}\isamarkupfalse%
\ {\isachardoublequoteopen}F\ {\isasymin}\ {\isacharbraceleft}F{\isacharbraceright}\ {\isasymunion}\ Wo{\isachardoublequoteclose}\isanewline
\ \ \ \ \isacommand{by}\isamarkupfalse%
\ {\isacharparenleft}simp\ add{\isacharcolon}\ insertI{\isadigit{1}}{\isacharparenright}\isanewline
\ \ \isacommand{have}\isamarkupfalse%
\ Ex{\isadigit{1}}{\isacharcolon}{\isachardoublequoteopen}{\isasymexists}{\isasymA}{\isachardot}\ {\isasymforall}F\ {\isasymin}\ {\isacharparenleft}{\isacharbraceleft}F{\isacharbraceright}\ {\isasymunion}\ Wo{\isacharparenright}{\isachardot}\ {\isasymA}\ {\isasymTurnstile}\ F{\isachardoublequoteclose}\isanewline
\ \ \ \ \isacommand{using}\isamarkupfalse%
\ {\isacartoucheopen}sat\ {\isacharparenleft}{\isacharbraceleft}F{\isacharbraceright}\ {\isasymunion}\ Wo{\isacharparenright}{\isacartoucheclose}\ \isacommand{by}\isamarkupfalse%
\ {\isacharparenleft}simp\ only{\isacharcolon}\ sat{\isacharunderscore}def{\isacharparenright}\isanewline
\ \ \isacommand{obtain}\isamarkupfalse%
\ {\isasymA}\ \isakeyword{where}\ Forall{\isadigit{1}}{\isacharcolon}{\isachardoublequoteopen}{\isasymforall}F\ {\isasymin}\ {\isacharparenleft}{\isacharbraceleft}F{\isacharbraceright}\ {\isasymunion}\ Wo{\isacharparenright}{\isachardot}\ {\isasymA}\ {\isasymTurnstile}\ F{\isachardoublequoteclose}\isanewline
\ \ \ \ \isacommand{using}\isamarkupfalse%
\ Ex{\isadigit{1}}\ \isacommand{by}\isamarkupfalse%
\ {\isacharparenleft}rule\ exE{\isacharparenright}\isanewline
\ \ \isacommand{have}\isamarkupfalse%
\ {\isachardoublequoteopen}{\isasymA}\ {\isasymTurnstile}\ F{\isachardoublequoteclose}\isanewline
\ \ \ \ \isacommand{using}\isamarkupfalse%
\ Forall{\isadigit{1}}\ {\isacartoucheopen}F\ {\isasymin}\ {\isacharbraceleft}F{\isacharbraceright}\ {\isasymunion}\ Wo{\isacartoucheclose}\ \isacommand{by}\isamarkupfalse%
\ {\isacharparenleft}rule\ bspec{\isacharparenright}\isanewline
\ \ \isacommand{then}\isamarkupfalse%
\ \isacommand{have}\isamarkupfalse%
\ {\isachardoublequoteopen}{\isasymA}\ {\isasymTurnstile}\ {\isacharparenleft}G\ \isactrlbold {\isasymand}\ H{\isacharparenright}{\isachardoublequoteclose}\isanewline
\ \ \ \ \isacommand{using}\isamarkupfalse%
\ assms{\isacharparenleft}{\isadigit{2}}{\isacharparenright}\ \isacommand{by}\isamarkupfalse%
\ {\isacharparenleft}simp\ only{\isacharcolon}\ {\isacartoucheopen}{\isasymA}\ {\isasymTurnstile}\ F{\isacartoucheclose}{\isacharparenright}\isanewline
\ \ \isacommand{then}\isamarkupfalse%
\ \isacommand{have}\isamarkupfalse%
\ {\isachardoublequoteopen}{\isasymA}\ {\isasymTurnstile}\ G\ {\isasymand}\ {\isasymA}\ {\isasymTurnstile}\ H{\isachardoublequoteclose}\isanewline
\ \ \ \ \isacommand{by}\isamarkupfalse%
\ {\isacharparenleft}simp\ only{\isacharcolon}\ formula{\isacharunderscore}semantics{\isachardot}simps{\isacharparenleft}{\isadigit{4}}{\isacharparenright}{\isacharparenright}\isanewline
\ \ \isacommand{then}\isamarkupfalse%
\ \isacommand{have}\isamarkupfalse%
\ {\isachardoublequoteopen}{\isasymA}\ {\isasymTurnstile}\ G{\isachardoublequoteclose}\isanewline
\ \ \ \ \isacommand{by}\isamarkupfalse%
\ {\isacharparenleft}rule\ conjunct{\isadigit{1}}{\isacharparenright}\isanewline
\ \ \isacommand{then}\isamarkupfalse%
\ \isacommand{have}\isamarkupfalse%
\ {\isadigit{1}}{\isacharcolon}{\isachardoublequoteopen}{\isasymforall}F\ {\isasymin}\ {\isacharbraceleft}G{\isacharbraceright}{\isachardot}\ {\isasymA}\ {\isasymTurnstile}\ F{\isachardoublequoteclose}\isanewline
\ \ \ \ \isacommand{by}\isamarkupfalse%
\ simp\isanewline
\ \ \isacommand{have}\isamarkupfalse%
\ {\isachardoublequoteopen}{\isasymA}\ {\isasymTurnstile}\ H{\isachardoublequoteclose}\isanewline
\ \ \ \ \isacommand{using}\isamarkupfalse%
\ {\isacartoucheopen}{\isasymA}\ {\isasymTurnstile}\ G\ {\isasymand}\ {\isasymA}\ {\isasymTurnstile}\ H{\isacartoucheclose}\ \isacommand{by}\isamarkupfalse%
\ {\isacharparenleft}rule\ conjunct{\isadigit{2}}{\isacharparenright}\isanewline
\ \ \isacommand{then}\isamarkupfalse%
\ \isacommand{have}\isamarkupfalse%
\ {\isadigit{2}}{\isacharcolon}{\isachardoublequoteopen}{\isasymforall}F\ {\isasymin}\ {\isacharbraceleft}H{\isacharbraceright}{\isachardot}\ {\isasymA}\ {\isasymTurnstile}\ F{\isachardoublequoteclose}\isanewline
\ \ \ \ \isacommand{by}\isamarkupfalse%
\ simp\isanewline
\ \ \isacommand{have}\isamarkupfalse%
\ {\isachardoublequoteopen}{\isasymforall}F\ {\isasymin}\ {\isacharparenleft}{\isacharbraceleft}G{\isacharbraceright}\ {\isasymunion}\ {\isacharbraceleft}H{\isacharbraceright}{\isacharparenright}\ {\isasymunion}\ {\isacharparenleft}{\isacharbraceleft}F{\isacharbraceright}\ {\isasymunion}\ Wo{\isacharparenright}{\isachardot}\ {\isasymA}\ {\isasymTurnstile}\ F{\isachardoublequoteclose}\isanewline
\ \ \ \ \isacommand{using}\isamarkupfalse%
\ Forall{\isadigit{1}}\ {\isadigit{1}}\ {\isadigit{2}}\ \isacommand{by}\isamarkupfalse%
\ {\isacharparenleft}iprover\ intro{\isacharcolon}\ ball{\isacharunderscore}Un{\isacharparenright}\isanewline
\ \ \isacommand{then}\isamarkupfalse%
\ \isacommand{have}\isamarkupfalse%
\ {\isachardoublequoteopen}{\isasymforall}F\ {\isasymin}\ {\isacharparenleft}{\isacharbraceleft}G{\isacharcomma}H{\isacharcomma}F{\isacharbraceright}\ {\isasymunion}\ Wo{\isacharparenright}{\isachardot}\ {\isasymA}\ {\isasymTurnstile}\ F{\isachardoublequoteclose}\isanewline
\ \ \ \ \isacommand{by}\isamarkupfalse%
\ simp\isanewline
\ \ \isacommand{then}\isamarkupfalse%
\ \isacommand{have}\isamarkupfalse%
\ {\isachardoublequoteopen}{\isasymexists}{\isasymA}{\isachardot}\ {\isasymforall}F\ {\isasymin}\ {\isacharparenleft}{\isacharbraceleft}G{\isacharcomma}H{\isacharcomma}F{\isacharbraceright}\ {\isasymunion}\ Wo{\isacharparenright}{\isachardot}\ {\isasymA}\ {\isasymTurnstile}\ F{\isachardoublequoteclose}\isanewline
\ \ \ \ \isacommand{by}\isamarkupfalse%
\ {\isacharparenleft}iprover\ intro{\isacharcolon}\ exI{\isacharparenright}\isanewline
\ \ \isacommand{thus}\isamarkupfalse%
\ {\isachardoublequoteopen}sat\ {\isacharparenleft}{\isacharbraceleft}G{\isacharcomma}H{\isacharcomma}F{\isacharbraceright}\ {\isasymunion}\ Wo{\isacharparenright}{\isachardoublequoteclose}\isanewline
\ \ \ \ \isacommand{by}\isamarkupfalse%
\ {\isacharparenleft}simp\ only{\isacharcolon}\ sat{\isacharunderscore}def{\isacharparenright}\isanewline
\isacommand{qed}\isamarkupfalse%
%
\endisatagproof
{\isafoldproof}%
%
\isadelimproof
\isanewline
%
\endisadelimproof
\isanewline
\isacommand{lemma}\isamarkupfalse%
\ pcp{\isacharunderscore}colecComp{\isacharunderscore}CON{\isacharunderscore}sat{\isadigit{2}}{\isacharcolon}\isanewline
\ \ \isakeyword{assumes}\ {\isachardoublequoteopen}W\ {\isasymin}\ colecComp{\isachardoublequoteclose}\isanewline
\ \ \ \ \ \ \ \ \ \ {\isachardoublequoteopen}F\ {\isacharequal}\ \isactrlbold {\isasymnot}{\isacharparenleft}G\ \isactrlbold {\isasymor}\ H{\isacharparenright}{\isachardoublequoteclose}\isanewline
\ \ \ \ \ \ \ \ \ \ {\isachardoublequoteopen}F\ {\isasymin}\ W{\isachardoublequoteclose}\isanewline
\ \ \ \ \ \ \ \ \ \ {\isachardoublequoteopen}finite\ Wo{\isachardoublequoteclose}\isanewline
\ \ \ \ \ \ \ \ \ \ {\isachardoublequoteopen}Wo\ {\isasymsubseteq}\ W{\isachardoublequoteclose}\isanewline
\ \ \ \ \ \ \ \ \isakeyword{shows}\ {\isachardoublequoteopen}sat\ {\isacharparenleft}{\isacharbraceleft}\isactrlbold {\isasymnot}\ G{\isacharcomma}\isactrlbold {\isasymnot}\ H{\isacharcomma}F{\isacharbraceright}\ {\isasymunion}\ Wo{\isacharparenright}{\isachardoublequoteclose}\isanewline
%
\isadelimproof
%
\endisadelimproof
%
\isatagproof
\isacommand{proof}\isamarkupfalse%
\ {\isacharminus}\isanewline
\ \ \isacommand{have}\isamarkupfalse%
\ {\isachardoublequoteopen}sat\ {\isacharparenleft}{\isacharbraceleft}F{\isacharbraceright}\ {\isasymunion}\ Wo{\isacharparenright}{\isachardoublequoteclose}\isanewline
\ \ \ \ \isacommand{using}\isamarkupfalse%
\ assms{\isacharparenleft}{\isadigit{1}}{\isacharcomma}{\isadigit{3}}{\isacharcomma}{\isadigit{4}}{\isacharcomma}{\isadigit{5}}{\isacharparenright}\ \isacommand{by}\isamarkupfalse%
\ {\isacharparenleft}rule\ pcp{\isacharunderscore}colecComp{\isacharunderscore}elem{\isacharunderscore}sat{\isacharparenright}\isanewline
\ \ \isacommand{have}\isamarkupfalse%
\ {\isachardoublequoteopen}F\ {\isasymin}\ {\isacharbraceleft}F{\isacharbraceright}\ {\isasymunion}\ Wo{\isachardoublequoteclose}\isanewline
\ \ \ \ \isacommand{by}\isamarkupfalse%
\ {\isacharparenleft}simp\ add{\isacharcolon}\ insertI{\isadigit{1}}{\isacharparenright}\isanewline
\ \ \isacommand{have}\isamarkupfalse%
\ Ex{\isadigit{1}}{\isacharcolon}{\isachardoublequoteopen}{\isasymexists}{\isasymA}{\isachardot}\ {\isasymforall}F\ {\isasymin}\ {\isacharparenleft}{\isacharbraceleft}F{\isacharbraceright}\ {\isasymunion}\ Wo{\isacharparenright}{\isachardot}\ {\isasymA}\ {\isasymTurnstile}\ F{\isachardoublequoteclose}\isanewline
\ \ \ \ \isacommand{using}\isamarkupfalse%
\ {\isacartoucheopen}sat\ {\isacharparenleft}{\isacharbraceleft}F{\isacharbraceright}\ {\isasymunion}\ Wo{\isacharparenright}{\isacartoucheclose}\ \isacommand{by}\isamarkupfalse%
\ {\isacharparenleft}simp\ only{\isacharcolon}\ sat{\isacharunderscore}def{\isacharparenright}\isanewline
\ \ \isacommand{obtain}\isamarkupfalse%
\ {\isasymA}\ \isakeyword{where}\ Forall{\isadigit{1}}{\isacharcolon}{\isachardoublequoteopen}{\isasymforall}F\ {\isasymin}\ {\isacharparenleft}{\isacharbraceleft}F{\isacharbraceright}\ {\isasymunion}\ Wo{\isacharparenright}{\isachardot}\ {\isasymA}\ {\isasymTurnstile}\ F{\isachardoublequoteclose}\isanewline
\ \ \ \ \isacommand{using}\isamarkupfalse%
\ Ex{\isadigit{1}}\ \isacommand{by}\isamarkupfalse%
\ {\isacharparenleft}rule\ exE{\isacharparenright}\isanewline
\ \ \isacommand{have}\isamarkupfalse%
\ {\isachardoublequoteopen}{\isasymA}\ {\isasymTurnstile}\ F{\isachardoublequoteclose}\isanewline
\ \ \ \ \isacommand{using}\isamarkupfalse%
\ Forall{\isadigit{1}}\ {\isacartoucheopen}F\ {\isasymin}\ {\isacharbraceleft}F{\isacharbraceright}\ {\isasymunion}\ Wo{\isacartoucheclose}\ \isacommand{by}\isamarkupfalse%
\ {\isacharparenleft}rule\ bspec{\isacharparenright}\isanewline
\ \ \isacommand{then}\isamarkupfalse%
\ \isacommand{have}\isamarkupfalse%
\ {\isachardoublequoteopen}{\isasymA}\ {\isasymTurnstile}\ \isactrlbold {\isasymnot}{\isacharparenleft}G\ \isactrlbold {\isasymor}\ H{\isacharparenright}{\isachardoublequoteclose}\isanewline
\ \ \ \ \isacommand{using}\isamarkupfalse%
\ assms{\isacharparenleft}{\isadigit{2}}{\isacharparenright}\ \isacommand{by}\isamarkupfalse%
\ {\isacharparenleft}simp\ only{\isacharcolon}\ {\isacartoucheopen}{\isasymA}\ {\isasymTurnstile}\ F{\isacartoucheclose}{\isacharparenright}\isanewline
\ \ \isacommand{then}\isamarkupfalse%
\ \isacommand{have}\isamarkupfalse%
\ {\isachardoublequoteopen}{\isasymnot}{\isacharparenleft}{\isasymA}\ {\isasymTurnstile}\ {\isacharparenleft}G\ \isactrlbold {\isasymor}\ H{\isacharparenright}{\isacharparenright}{\isachardoublequoteclose}\isanewline
\ \ \ \ \isacommand{by}\isamarkupfalse%
\ {\isacharparenleft}simp\ only{\isacharcolon}\ formula{\isacharunderscore}semantics{\isachardot}simps{\isacharparenleft}{\isadigit{3}}{\isacharparenright}\ simp{\isacharunderscore}thms{\isacharparenleft}{\isadigit{8}}{\isacharparenright}{\isacharparenright}\isanewline
\ \ \isacommand{then}\isamarkupfalse%
\ \isacommand{have}\isamarkupfalse%
\ {\isachardoublequoteopen}{\isasymnot}{\isacharparenleft}{\isasymA}\ {\isasymTurnstile}\ G\ {\isasymor}\ {\isasymA}\ {\isasymTurnstile}\ H{\isacharparenright}{\isachardoublequoteclose}\isanewline
\ \ \ \ \isacommand{by}\isamarkupfalse%
\ {\isacharparenleft}simp\ only{\isacharcolon}\ formula{\isacharunderscore}semantics{\isachardot}simps{\isacharparenleft}{\isadigit{5}}{\isacharparenright}\ simp{\isacharunderscore}thms{\isacharparenleft}{\isadigit{8}}{\isacharparenright}{\isacharparenright}\isanewline
\ \ \isacommand{then}\isamarkupfalse%
\ \isacommand{have}\isamarkupfalse%
\ {\isachardoublequoteopen}{\isasymnot}\ {\isasymA}\ {\isasymTurnstile}\ G\ {\isasymand}\ {\isasymnot}\ {\isasymA}\ {\isasymTurnstile}\ H{\isachardoublequoteclose}\ \isanewline
\ \ \ \ \isacommand{by}\isamarkupfalse%
\ {\isacharparenleft}simp\ only{\isacharcolon}\ de{\isacharunderscore}Morgan{\isacharunderscore}disj\ simp{\isacharunderscore}thms{\isacharparenleft}{\isadigit{8}}{\isacharparenright}{\isacharparenright}\isanewline
\ \ \isacommand{then}\isamarkupfalse%
\ \isacommand{have}\isamarkupfalse%
\ {\isachardoublequoteopen}{\isasymA}\ {\isasymTurnstile}\ \isactrlbold {\isasymnot}\ G\ {\isasymand}\ {\isasymA}\ {\isasymTurnstile}\ \isactrlbold {\isasymnot}\ H{\isachardoublequoteclose}\isanewline
\ \ \ \ \isacommand{by}\isamarkupfalse%
\ {\isacharparenleft}simp\ only{\isacharcolon}\ formula{\isacharunderscore}semantics{\isachardot}simps{\isacharparenleft}{\isadigit{3}}{\isacharparenright}\ simp{\isacharunderscore}thms{\isacharparenleft}{\isadigit{8}}{\isacharparenright}{\isacharparenright}\ \isanewline
\ \ \isacommand{then}\isamarkupfalse%
\ \isacommand{have}\isamarkupfalse%
\ {\isachardoublequoteopen}{\isasymA}\ {\isasymTurnstile}\ \isactrlbold {\isasymnot}\ G{\isachardoublequoteclose}\isanewline
\ \ \ \ \isacommand{by}\isamarkupfalse%
\ {\isacharparenleft}rule\ conjunct{\isadigit{1}}{\isacharparenright}\isanewline
\ \ \isacommand{then}\isamarkupfalse%
\ \isacommand{have}\isamarkupfalse%
\ {\isadigit{1}}{\isacharcolon}{\isachardoublequoteopen}{\isasymforall}F\ {\isasymin}\ {\isacharbraceleft}\isactrlbold {\isasymnot}\ G{\isacharbraceright}{\isachardot}\ {\isasymA}\ {\isasymTurnstile}\ F{\isachardoublequoteclose}\isanewline
\ \ \ \ \isacommand{by}\isamarkupfalse%
\ simp\isanewline
\ \ \isacommand{have}\isamarkupfalse%
\ {\isachardoublequoteopen}{\isasymA}\ {\isasymTurnstile}\ \isactrlbold {\isasymnot}\ H{\isachardoublequoteclose}\isanewline
\ \ \ \ \isacommand{using}\isamarkupfalse%
\ {\isacartoucheopen}{\isasymA}\ {\isasymTurnstile}\ \isactrlbold {\isasymnot}\ G\ {\isasymand}\ {\isasymA}\ {\isasymTurnstile}\ \isactrlbold {\isasymnot}\ H{\isacartoucheclose}\ \isacommand{by}\isamarkupfalse%
\ {\isacharparenleft}rule\ conjunct{\isadigit{2}}{\isacharparenright}\isanewline
\ \ \isacommand{then}\isamarkupfalse%
\ \isacommand{have}\isamarkupfalse%
\ {\isadigit{2}}{\isacharcolon}{\isachardoublequoteopen}{\isasymforall}F\ {\isasymin}\ {\isacharbraceleft}\isactrlbold {\isasymnot}\ H{\isacharbraceright}{\isachardot}\ {\isasymA}\ {\isasymTurnstile}\ F{\isachardoublequoteclose}\isanewline
\ \ \ \ \isacommand{by}\isamarkupfalse%
\ simp\isanewline
\ \ \isacommand{have}\isamarkupfalse%
\ {\isachardoublequoteopen}{\isasymforall}F\ {\isasymin}\ {\isacharparenleft}{\isacharbraceleft}\isactrlbold {\isasymnot}\ G{\isacharbraceright}\ {\isasymunion}\ {\isacharbraceleft}\isactrlbold {\isasymnot}\ H{\isacharbraceright}{\isacharparenright}\ {\isasymunion}\ {\isacharparenleft}{\isacharbraceleft}F{\isacharbraceright}\ {\isasymunion}\ Wo{\isacharparenright}{\isachardot}\ {\isasymA}\ {\isasymTurnstile}\ F{\isachardoublequoteclose}\isanewline
\ \ \ \ \isacommand{using}\isamarkupfalse%
\ Forall{\isadigit{1}}\ {\isadigit{1}}\ {\isadigit{2}}\ \isacommand{by}\isamarkupfalse%
\ {\isacharparenleft}iprover\ intro{\isacharcolon}\ ball{\isacharunderscore}Un{\isacharparenright}\isanewline
\ \ \isacommand{then}\isamarkupfalse%
\ \isacommand{have}\isamarkupfalse%
\ {\isachardoublequoteopen}{\isasymforall}F\ {\isasymin}\ {\isacharparenleft}{\isacharbraceleft}\isactrlbold {\isasymnot}\ G{\isacharcomma}\isactrlbold {\isasymnot}\ H{\isacharcomma}\ F{\isacharbraceright}\ {\isasymunion}\ Wo{\isacharparenright}{\isachardot}\ {\isasymA}\ {\isasymTurnstile}\ F{\isachardoublequoteclose}\isanewline
\ \ \ \ \isacommand{by}\isamarkupfalse%
\ simp\isanewline
\ \ \isacommand{then}\isamarkupfalse%
\ \isacommand{have}\isamarkupfalse%
\ {\isachardoublequoteopen}{\isasymexists}{\isasymA}{\isachardot}\ {\isasymforall}F\ {\isasymin}\ {\isacharparenleft}{\isacharbraceleft}\isactrlbold {\isasymnot}\ G{\isacharcomma}\isactrlbold {\isasymnot}\ H{\isacharcomma}F{\isacharbraceright}\ {\isasymunion}\ Wo{\isacharparenright}{\isachardot}\ {\isasymA}\ {\isasymTurnstile}\ F{\isachardoublequoteclose}\isanewline
\ \ \ \ \isacommand{by}\isamarkupfalse%
\ {\isacharparenleft}iprover\ intro{\isacharcolon}\ exI{\isacharparenright}\isanewline
\ \ \isacommand{thus}\isamarkupfalse%
\ {\isachardoublequoteopen}sat\ {\isacharparenleft}{\isacharbraceleft}\isactrlbold {\isasymnot}\ G{\isacharcomma}\isactrlbold {\isasymnot}\ H{\isacharcomma}F{\isacharbraceright}\ {\isasymunion}\ Wo{\isacharparenright}{\isachardoublequoteclose}\isanewline
\ \ \ \ \isacommand{by}\isamarkupfalse%
\ {\isacharparenleft}simp\ only{\isacharcolon}\ sat{\isacharunderscore}def{\isacharparenright}\isanewline
\isacommand{qed}\isamarkupfalse%
%
\endisatagproof
{\isafoldproof}%
%
\isadelimproof
\isanewline
%
\endisadelimproof
\isanewline
\isacommand{lemma}\isamarkupfalse%
\ pcp{\isacharunderscore}colecComp{\isacharunderscore}CON{\isacharunderscore}sat{\isadigit{3}}{\isacharcolon}\isanewline
\ \ \isakeyword{assumes}\ {\isachardoublequoteopen}W\ {\isasymin}\ colecComp{\isachardoublequoteclose}\isanewline
\ \ \ \ \ \ \ \ \ \ {\isachardoublequoteopen}F\ {\isacharequal}\ \isactrlbold {\isasymnot}\ {\isacharparenleft}G\ \isactrlbold {\isasymrightarrow}\ H{\isacharparenright}{\isachardoublequoteclose}\isanewline
\ \ \ \ \ \ \ \ \ \ {\isachardoublequoteopen}F\ {\isasymin}\ W{\isachardoublequoteclose}\isanewline
\ \ \ \ \ \ \ \ \ \ {\isachardoublequoteopen}finite\ Wo{\isachardoublequoteclose}\isanewline
\ \ \ \ \ \ \ \ \ \ {\isachardoublequoteopen}Wo\ {\isasymsubseteq}\ W{\isachardoublequoteclose}\isanewline
\ \ \ \ \ \ \ \ \isakeyword{shows}\ {\isachardoublequoteopen}sat\ {\isacharparenleft}{\isacharbraceleft}G{\isacharcomma}\isactrlbold {\isasymnot}\ H{\isacharcomma}F{\isacharbraceright}\ {\isasymunion}\ Wo{\isacharparenright}{\isachardoublequoteclose}\isanewline
%
\isadelimproof
%
\endisadelimproof
%
\isatagproof
\isacommand{proof}\isamarkupfalse%
\ {\isacharminus}\isanewline
\ \ \isacommand{have}\isamarkupfalse%
\ {\isachardoublequoteopen}sat\ {\isacharparenleft}{\isacharbraceleft}F{\isacharbraceright}\ {\isasymunion}\ Wo{\isacharparenright}{\isachardoublequoteclose}\isanewline
\ \ \ \ \isacommand{using}\isamarkupfalse%
\ assms{\isacharparenleft}{\isadigit{1}}{\isacharcomma}{\isadigit{3}}{\isacharcomma}{\isadigit{4}}{\isacharcomma}{\isadigit{5}}{\isacharparenright}\ \isacommand{by}\isamarkupfalse%
\ {\isacharparenleft}rule\ pcp{\isacharunderscore}colecComp{\isacharunderscore}elem{\isacharunderscore}sat{\isacharparenright}\isanewline
\ \ \isacommand{have}\isamarkupfalse%
\ {\isachardoublequoteopen}F\ {\isasymin}\ {\isacharbraceleft}F{\isacharbraceright}\ {\isasymunion}\ Wo{\isachardoublequoteclose}\isanewline
\ \ \ \ \isacommand{by}\isamarkupfalse%
\ {\isacharparenleft}simp\ add{\isacharcolon}\ insertI{\isadigit{1}}{\isacharparenright}\isanewline
\ \ \isacommand{have}\isamarkupfalse%
\ Ex{\isadigit{1}}{\isacharcolon}{\isachardoublequoteopen}{\isasymexists}{\isasymA}{\isachardot}\ {\isasymforall}F\ {\isasymin}\ {\isacharparenleft}{\isacharbraceleft}F{\isacharbraceright}\ {\isasymunion}\ Wo{\isacharparenright}{\isachardot}\ {\isasymA}\ {\isasymTurnstile}\ F{\isachardoublequoteclose}\isanewline
\ \ \ \ \isacommand{using}\isamarkupfalse%
\ {\isacartoucheopen}sat\ {\isacharparenleft}{\isacharbraceleft}F{\isacharbraceright}\ {\isasymunion}\ Wo{\isacharparenright}{\isacartoucheclose}\ \isacommand{by}\isamarkupfalse%
\ {\isacharparenleft}simp\ only{\isacharcolon}\ sat{\isacharunderscore}def{\isacharparenright}\isanewline
\ \ \isacommand{obtain}\isamarkupfalse%
\ {\isasymA}\ \isakeyword{where}\ Forall{\isadigit{1}}{\isacharcolon}{\isachardoublequoteopen}{\isasymforall}F\ {\isasymin}\ {\isacharparenleft}{\isacharbraceleft}F{\isacharbraceright}\ {\isasymunion}\ Wo{\isacharparenright}{\isachardot}\ {\isasymA}\ {\isasymTurnstile}\ F{\isachardoublequoteclose}\isanewline
\ \ \ \ \isacommand{using}\isamarkupfalse%
\ Ex{\isadigit{1}}\ \isacommand{by}\isamarkupfalse%
\ {\isacharparenleft}rule\ exE{\isacharparenright}\isanewline
\ \ \isacommand{have}\isamarkupfalse%
\ {\isachardoublequoteopen}{\isasymA}\ {\isasymTurnstile}\ F{\isachardoublequoteclose}\isanewline
\ \ \ \ \isacommand{using}\isamarkupfalse%
\ Forall{\isadigit{1}}\ {\isacartoucheopen}F\ {\isasymin}\ {\isacharbraceleft}F{\isacharbraceright}\ {\isasymunion}\ Wo{\isacartoucheclose}\ \isacommand{by}\isamarkupfalse%
\ {\isacharparenleft}rule\ bspec{\isacharparenright}\isanewline
\ \ \isacommand{then}\isamarkupfalse%
\ \isacommand{have}\isamarkupfalse%
\ {\isachardoublequoteopen}{\isasymA}\ {\isasymTurnstile}\ \isactrlbold {\isasymnot}{\isacharparenleft}G\ \isactrlbold {\isasymrightarrow}\ H{\isacharparenright}{\isachardoublequoteclose}\isanewline
\ \ \ \ \isacommand{using}\isamarkupfalse%
\ assms{\isacharparenleft}{\isadigit{2}}{\isacharparenright}\ \isacommand{by}\isamarkupfalse%
\ {\isacharparenleft}simp\ only{\isacharcolon}\ {\isacartoucheopen}{\isasymA}\ {\isasymTurnstile}\ F{\isacartoucheclose}{\isacharparenright}\isanewline
\ \ \isacommand{then}\isamarkupfalse%
\ \isacommand{have}\isamarkupfalse%
\ {\isachardoublequoteopen}{\isasymnot}{\isacharparenleft}{\isasymA}\ {\isasymTurnstile}\ {\isacharparenleft}G\ \isactrlbold {\isasymrightarrow}\ H{\isacharparenright}{\isacharparenright}{\isachardoublequoteclose}\isanewline
\ \ \ \ \isacommand{by}\isamarkupfalse%
\ {\isacharparenleft}simp\ only{\isacharcolon}\ formula{\isacharunderscore}semantics{\isachardot}simps{\isacharparenleft}{\isadigit{3}}{\isacharparenright}\ simp{\isacharunderscore}thms{\isacharparenleft}{\isadigit{8}}{\isacharparenright}{\isacharparenright}\isanewline
\ \ \isacommand{then}\isamarkupfalse%
\ \isacommand{have}\isamarkupfalse%
\ {\isachardoublequoteopen}{\isasymnot}{\isacharparenleft}{\isasymA}\ {\isasymTurnstile}\ G\ {\isasymlongrightarrow}\ {\isasymA}\ {\isasymTurnstile}\ H{\isacharparenright}{\isachardoublequoteclose}\isanewline
\ \ \ \ \isacommand{by}\isamarkupfalse%
\ {\isacharparenleft}simp\ only{\isacharcolon}\ formula{\isacharunderscore}semantics{\isachardot}simps{\isacharparenleft}{\isadigit{6}}{\isacharparenright}\ simp{\isacharunderscore}thms{\isacharparenleft}{\isadigit{8}}{\isacharparenright}{\isacharparenright}\isanewline
\ \ \isacommand{then}\isamarkupfalse%
\ \isacommand{have}\isamarkupfalse%
\ {\isachardoublequoteopen}{\isasymA}\ {\isasymTurnstile}\ G\ {\isasymand}\ {\isasymnot}\ {\isasymA}\ {\isasymTurnstile}\ H{\isachardoublequoteclose}\isanewline
\ \ \ \ \isacommand{by}\isamarkupfalse%
\ {\isacharparenleft}simp\ only{\isacharcolon}\ not{\isacharunderscore}imp\ simp{\isacharunderscore}thms{\isacharparenleft}{\isadigit{8}}{\isacharparenright}{\isacharparenright}\isanewline
\ \ \isacommand{then}\isamarkupfalse%
\ \isacommand{have}\isamarkupfalse%
\ {\isachardoublequoteopen}{\isasymA}\ {\isasymTurnstile}\ G\ {\isasymand}\ {\isasymA}\ {\isasymTurnstile}\ \isactrlbold {\isasymnot}\ H{\isachardoublequoteclose}\isanewline
\ \ \ \ \isacommand{by}\isamarkupfalse%
\ {\isacharparenleft}simp\ only{\isacharcolon}\ formula{\isacharunderscore}semantics{\isachardot}simps{\isacharparenleft}{\isadigit{3}}{\isacharparenright}\ simp{\isacharunderscore}thms{\isacharparenleft}{\isadigit{8}}{\isacharparenright}{\isacharparenright}\ \isanewline
\ \ \isacommand{then}\isamarkupfalse%
\ \isacommand{have}\isamarkupfalse%
\ {\isachardoublequoteopen}{\isasymA}\ {\isasymTurnstile}\ G{\isachardoublequoteclose}\isanewline
\ \ \ \ \isacommand{by}\isamarkupfalse%
\ {\isacharparenleft}rule\ conjunct{\isadigit{1}}{\isacharparenright}\isanewline
\ \ \isacommand{then}\isamarkupfalse%
\ \isacommand{have}\isamarkupfalse%
\ {\isadigit{1}}{\isacharcolon}{\isachardoublequoteopen}{\isasymforall}F\ {\isasymin}\ {\isacharbraceleft}G{\isacharbraceright}{\isachardot}\ {\isasymA}\ {\isasymTurnstile}\ F{\isachardoublequoteclose}\isanewline
\ \ \ \ \isacommand{by}\isamarkupfalse%
\ simp\isanewline
\ \ \isacommand{have}\isamarkupfalse%
\ {\isachardoublequoteopen}{\isasymA}\ {\isasymTurnstile}\ \isactrlbold {\isasymnot}\ H{\isachardoublequoteclose}\isanewline
\ \ \ \ \isacommand{using}\isamarkupfalse%
\ {\isacartoucheopen}{\isasymA}\ {\isasymTurnstile}\ G\ {\isasymand}\ {\isasymA}\ {\isasymTurnstile}\ \isactrlbold {\isasymnot}\ H{\isacartoucheclose}\ \isacommand{by}\isamarkupfalse%
\ {\isacharparenleft}rule\ conjunct{\isadigit{2}}{\isacharparenright}\isanewline
\ \ \isacommand{then}\isamarkupfalse%
\ \isacommand{have}\isamarkupfalse%
\ {\isadigit{2}}{\isacharcolon}{\isachardoublequoteopen}{\isasymforall}F\ {\isasymin}\ {\isacharbraceleft}\isactrlbold {\isasymnot}\ H{\isacharbraceright}{\isachardot}\ {\isasymA}\ {\isasymTurnstile}\ F{\isachardoublequoteclose}\isanewline
\ \ \ \ \isacommand{by}\isamarkupfalse%
\ simp\isanewline
\ \ \isacommand{have}\isamarkupfalse%
\ {\isachardoublequoteopen}{\isasymforall}F\ {\isasymin}\ {\isacharparenleft}{\isacharbraceleft}G{\isacharbraceright}\ {\isasymunion}\ {\isacharbraceleft}\isactrlbold {\isasymnot}\ H{\isacharbraceright}{\isacharparenright}\ {\isasymunion}\ {\isacharparenleft}{\isacharbraceleft}F{\isacharbraceright}\ {\isasymunion}\ Wo{\isacharparenright}{\isachardot}\ {\isasymA}\ {\isasymTurnstile}\ F{\isachardoublequoteclose}\isanewline
\ \ \ \ \isacommand{using}\isamarkupfalse%
\ Forall{\isadigit{1}}\ {\isadigit{1}}\ {\isadigit{2}}\ \isacommand{by}\isamarkupfalse%
\ {\isacharparenleft}iprover\ intro{\isacharcolon}\ ball{\isacharunderscore}Un{\isacharparenright}\isanewline
\ \ \isacommand{then}\isamarkupfalse%
\ \isacommand{have}\isamarkupfalse%
\ {\isachardoublequoteopen}{\isasymforall}F\ {\isasymin}\ {\isacharbraceleft}G{\isacharcomma}\isactrlbold {\isasymnot}\ H{\isacharcomma}F{\isacharbraceright}\ {\isasymunion}\ Wo{\isachardot}\ {\isasymA}\ {\isasymTurnstile}\ F{\isachardoublequoteclose}\isanewline
\ \ \ \ \isacommand{by}\isamarkupfalse%
\ simp\isanewline
\ \ \isacommand{then}\isamarkupfalse%
\ \isacommand{have}\isamarkupfalse%
\ {\isachardoublequoteopen}{\isasymexists}{\isasymA}{\isachardot}\ {\isasymforall}F\ {\isasymin}\ {\isacharparenleft}{\isacharbraceleft}G{\isacharcomma}\isactrlbold {\isasymnot}\ H{\isacharcomma}F{\isacharbraceright}\ {\isasymunion}\ Wo{\isacharparenright}{\isachardot}\ {\isasymA}\ {\isasymTurnstile}\ F{\isachardoublequoteclose}\isanewline
\ \ \ \ \isacommand{by}\isamarkupfalse%
\ {\isacharparenleft}iprover\ intro{\isacharcolon}\ exI{\isacharparenright}\isanewline
\ \ \isacommand{thus}\isamarkupfalse%
\ {\isachardoublequoteopen}sat\ {\isacharparenleft}{\isacharbraceleft}G{\isacharcomma}\isactrlbold {\isasymnot}\ H{\isacharcomma}F{\isacharbraceright}\ {\isasymunion}\ Wo{\isacharparenright}{\isachardoublequoteclose}\isanewline
\ \ \ \ \isacommand{by}\isamarkupfalse%
\ {\isacharparenleft}simp\ only{\isacharcolon}\ sat{\isacharunderscore}def{\isacharparenright}\isanewline
\isacommand{qed}\isamarkupfalse%
%
\endisatagproof
{\isafoldproof}%
%
\isadelimproof
\isanewline
%
\endisadelimproof
\isanewline
\isacommand{lemma}\isamarkupfalse%
\ pcp{\isacharunderscore}colecComp{\isacharunderscore}CON{\isacharunderscore}sat{\isadigit{4}}{\isacharcolon}\isanewline
\ \ \isakeyword{assumes}\ {\isachardoublequoteopen}W\ {\isasymin}\ colecComp{\isachardoublequoteclose}\isanewline
\ \ \ \ \ \ \ \ \ \ {\isachardoublequoteopen}F\ {\isacharequal}\ \isactrlbold {\isasymnot}\ {\isacharparenleft}\isactrlbold {\isasymnot}\ G{\isacharparenright}{\isachardoublequoteclose}\isanewline
\ \ \ \ \ \ \ \ \ \ {\isachardoublequoteopen}H\ {\isacharequal}\ G{\isachardoublequoteclose}\isanewline
\ \ \ \ \ \ \ \ \ \ {\isachardoublequoteopen}F\ {\isasymin}\ W{\isachardoublequoteclose}\isanewline
\ \ \ \ \ \ \ \ \ \ {\isachardoublequoteopen}finite\ Wo{\isachardoublequoteclose}\isanewline
\ \ \ \ \ \ \ \ \ \ {\isachardoublequoteopen}Wo\ {\isasymsubseteq}\ W{\isachardoublequoteclose}\isanewline
\ \ \ \ \ \ \ \ \isakeyword{shows}\ {\isachardoublequoteopen}sat\ {\isacharparenleft}{\isacharbraceleft}G{\isacharcomma}H{\isacharcomma}F{\isacharbraceright}\ {\isasymunion}\ Wo{\isacharparenright}{\isachardoublequoteclose}\isanewline
%
\isadelimproof
%
\endisadelimproof
%
\isatagproof
\isacommand{proof}\isamarkupfalse%
\ {\isacharminus}\isanewline
\ \ \isacommand{have}\isamarkupfalse%
\ {\isachardoublequoteopen}sat\ {\isacharparenleft}{\isacharbraceleft}F{\isacharbraceright}\ {\isasymunion}\ Wo{\isacharparenright}{\isachardoublequoteclose}\isanewline
\ \ \ \ \isacommand{using}\isamarkupfalse%
\ assms{\isacharparenleft}{\isadigit{1}}{\isacharcomma}{\isadigit{4}}{\isacharcomma}{\isadigit{5}}{\isacharcomma}{\isadigit{6}}{\isacharparenright}\ \isacommand{by}\isamarkupfalse%
\ {\isacharparenleft}rule\ pcp{\isacharunderscore}colecComp{\isacharunderscore}elem{\isacharunderscore}sat{\isacharparenright}\isanewline
\ \ \isacommand{have}\isamarkupfalse%
\ {\isachardoublequoteopen}F\ {\isasymin}\ {\isacharbraceleft}F{\isacharbraceright}\ {\isasymunion}\ Wo{\isachardoublequoteclose}\isanewline
\ \ \ \ \isacommand{by}\isamarkupfalse%
\ {\isacharparenleft}simp\ add{\isacharcolon}\ insertI{\isadigit{1}}{\isacharparenright}\isanewline
\ \ \isacommand{have}\isamarkupfalse%
\ Ex{\isadigit{1}}{\isacharcolon}{\isachardoublequoteopen}{\isasymexists}{\isasymA}{\isachardot}\ {\isasymforall}F\ {\isasymin}\ {\isacharparenleft}{\isacharbraceleft}F{\isacharbraceright}\ {\isasymunion}\ Wo{\isacharparenright}{\isachardot}\ {\isasymA}\ {\isasymTurnstile}\ F{\isachardoublequoteclose}\isanewline
\ \ \ \ \isacommand{using}\isamarkupfalse%
\ {\isacartoucheopen}sat\ {\isacharparenleft}{\isacharbraceleft}F{\isacharbraceright}\ {\isasymunion}\ Wo{\isacharparenright}{\isacartoucheclose}\ \isacommand{by}\isamarkupfalse%
\ {\isacharparenleft}simp\ only{\isacharcolon}\ sat{\isacharunderscore}def{\isacharparenright}\isanewline
\ \ \isacommand{obtain}\isamarkupfalse%
\ {\isasymA}\ \isakeyword{where}\ Forall{\isadigit{1}}{\isacharcolon}{\isachardoublequoteopen}{\isasymforall}F\ {\isasymin}\ {\isacharparenleft}{\isacharbraceleft}F{\isacharbraceright}\ {\isasymunion}\ Wo{\isacharparenright}{\isachardot}\ {\isasymA}\ {\isasymTurnstile}\ F{\isachardoublequoteclose}\isanewline
\ \ \ \ \isacommand{using}\isamarkupfalse%
\ Ex{\isadigit{1}}\ \isacommand{by}\isamarkupfalse%
\ {\isacharparenleft}rule\ exE{\isacharparenright}\isanewline
\ \ \isacommand{have}\isamarkupfalse%
\ {\isachardoublequoteopen}{\isasymA}\ {\isasymTurnstile}\ F{\isachardoublequoteclose}\isanewline
\ \ \ \ \isacommand{using}\isamarkupfalse%
\ Forall{\isadigit{1}}\ {\isacartoucheopen}F\ {\isasymin}\ {\isacharbraceleft}F{\isacharbraceright}\ {\isasymunion}\ Wo{\isacartoucheclose}\ \isacommand{by}\isamarkupfalse%
\ {\isacharparenleft}rule\ bspec{\isacharparenright}\isanewline
\ \ \isacommand{then}\isamarkupfalse%
\ \isacommand{have}\isamarkupfalse%
\ {\isachardoublequoteopen}{\isasymA}\ {\isasymTurnstile}\ \isactrlbold {\isasymnot}{\isacharparenleft}\isactrlbold {\isasymnot}\ G{\isacharparenright}{\isachardoublequoteclose}\isanewline
\ \ \ \ \isacommand{using}\isamarkupfalse%
\ assms{\isacharparenleft}{\isadigit{2}}{\isacharparenright}\ \isacommand{by}\isamarkupfalse%
\ {\isacharparenleft}simp\ only{\isacharcolon}\ {\isacartoucheopen}{\isasymA}\ {\isasymTurnstile}\ F{\isacartoucheclose}{\isacharparenright}\isanewline
\ \ \isacommand{then}\isamarkupfalse%
\ \isacommand{have}\isamarkupfalse%
\ {\isachardoublequoteopen}{\isasymnot}\ {\isasymA}\ {\isasymTurnstile}\ \isactrlbold {\isasymnot}\ G{\isachardoublequoteclose}\isanewline
\ \ \ \ \isacommand{by}\isamarkupfalse%
\ {\isacharparenleft}simp\ only{\isacharcolon}\ formula{\isacharunderscore}semantics{\isachardot}simps{\isacharparenleft}{\isadigit{3}}{\isacharparenright}\ simp{\isacharunderscore}thms{\isacharparenleft}{\isadigit{8}}{\isacharparenright}{\isacharparenright}\isanewline
\ \ \isacommand{then}\isamarkupfalse%
\ \isacommand{have}\isamarkupfalse%
\ {\isachardoublequoteopen}{\isasymnot}\ {\isasymnot}{\isasymA}\ {\isasymTurnstile}\ G{\isachardoublequoteclose}\isanewline
\ \ \ \ \isacommand{by}\isamarkupfalse%
\ {\isacharparenleft}simp\ only{\isacharcolon}\ formula{\isacharunderscore}semantics{\isachardot}simps{\isacharparenleft}{\isadigit{3}}{\isacharparenright}\ simp{\isacharunderscore}thms{\isacharparenleft}{\isadigit{8}}{\isacharparenright}{\isacharparenright}\isanewline
\ \ \isacommand{then}\isamarkupfalse%
\ \isacommand{have}\isamarkupfalse%
\ {\isachardoublequoteopen}{\isasymA}\ {\isasymTurnstile}\ G{\isachardoublequoteclose}\isanewline
\ \ \ \ \isacommand{by}\isamarkupfalse%
\ {\isacharparenleft}rule\ notnotD{\isacharparenright}\isanewline
\ \ \isacommand{then}\isamarkupfalse%
\ \isacommand{have}\isamarkupfalse%
\ {\isadigit{1}}{\isacharcolon}{\isachardoublequoteopen}{\isasymforall}F\ {\isasymin}\ {\isacharbraceleft}G{\isacharbraceright}{\isachardot}\ {\isasymA}\ {\isasymTurnstile}\ F{\isachardoublequoteclose}\isanewline
\ \ \ \ \isacommand{by}\isamarkupfalse%
\ simp\isanewline
\ \ \isacommand{have}\isamarkupfalse%
\ {\isachardoublequoteopen}{\isasymA}\ {\isasymTurnstile}\ H{\isachardoublequoteclose}\isanewline
\ \ \ \ \isacommand{using}\isamarkupfalse%
\ {\isacartoucheopen}{\isasymA}\ {\isasymTurnstile}\ G{\isacartoucheclose}\ \isacommand{by}\isamarkupfalse%
\ {\isacharparenleft}simp\ only{\isacharcolon}\ {\isacartoucheopen}H\ {\isacharequal}\ G{\isacartoucheclose}{\isacharparenright}\isanewline
\ \ \isacommand{then}\isamarkupfalse%
\ \isacommand{have}\isamarkupfalse%
\ {\isadigit{2}}{\isacharcolon}{\isachardoublequoteopen}{\isasymforall}F\ {\isasymin}\ {\isacharbraceleft}H{\isacharbraceright}{\isachardot}\ {\isasymA}\ {\isasymTurnstile}\ F{\isachardoublequoteclose}\isanewline
\ \ \ \ \isacommand{by}\isamarkupfalse%
\ simp\isanewline
\ \ \isacommand{have}\isamarkupfalse%
\ {\isachardoublequoteopen}{\isasymforall}F\ {\isasymin}\ {\isacharparenleft}{\isacharbraceleft}G{\isacharbraceright}\ {\isasymunion}\ {\isacharbraceleft}H{\isacharbraceright}{\isacharparenright}\ {\isasymunion}\ {\isacharparenleft}{\isacharbraceleft}F{\isacharbraceright}\ {\isasymunion}\ Wo{\isacharparenright}{\isachardot}\ {\isasymA}\ {\isasymTurnstile}\ F{\isachardoublequoteclose}\isanewline
\ \ \ \ \isacommand{using}\isamarkupfalse%
\ Forall{\isadigit{1}}\ {\isadigit{1}}\ {\isadigit{2}}\ \isacommand{by}\isamarkupfalse%
\ {\isacharparenleft}iprover\ intro{\isacharcolon}\ ball{\isacharunderscore}Un{\isacharparenright}\isanewline
\ \ \isacommand{then}\isamarkupfalse%
\ \isacommand{have}\isamarkupfalse%
\ {\isachardoublequoteopen}{\isasymforall}F\ {\isasymin}\ {\isacharbraceleft}G{\isacharcomma}H{\isacharcomma}F{\isacharbraceright}\ {\isasymunion}\ Wo{\isachardot}\ {\isasymA}\ {\isasymTurnstile}\ F{\isachardoublequoteclose}\isanewline
\ \ \ \ \isacommand{by}\isamarkupfalse%
\ simp\isanewline
\ \ \isacommand{then}\isamarkupfalse%
\ \isacommand{have}\isamarkupfalse%
\ {\isachardoublequoteopen}{\isasymexists}{\isasymA}{\isachardot}\ {\isasymforall}F\ {\isasymin}\ {\isacharparenleft}{\isacharbraceleft}G{\isacharcomma}H{\isacharcomma}F{\isacharbraceright}\ {\isasymunion}\ Wo{\isacharparenright}{\isachardot}\ {\isasymA}\ {\isasymTurnstile}\ F{\isachardoublequoteclose}\isanewline
\ \ \ \ \isacommand{by}\isamarkupfalse%
\ {\isacharparenleft}iprover\ intro{\isacharcolon}\ exI{\isacharparenright}\isanewline
\ \ \isacommand{thus}\isamarkupfalse%
\ {\isachardoublequoteopen}sat\ {\isacharparenleft}{\isacharbraceleft}G{\isacharcomma}H{\isacharcomma}F{\isacharbraceright}\ {\isasymunion}\ Wo{\isacharparenright}{\isachardoublequoteclose}\isanewline
\ \ \ \ \isacommand{by}\isamarkupfalse%
\ {\isacharparenleft}simp\ only{\isacharcolon}\ sat{\isacharunderscore}def{\isacharparenright}\isanewline
\isacommand{qed}\isamarkupfalse%
%
\endisatagproof
{\isafoldproof}%
%
\isadelimproof
\isanewline
%
\endisadelimproof
\isanewline
\isacommand{lemma}\isamarkupfalse%
\ pcp{\isacharunderscore}colecComp{\isacharunderscore}CON{\isacharunderscore}sat{\isacharcolon}\isanewline
\ \ \isakeyword{assumes}\ {\isachardoublequoteopen}W\ {\isasymin}\ colecComp{\isachardoublequoteclose}\isanewline
\ \ \ \ \ \ \ \ \ \ {\isachardoublequoteopen}Con\ F\ G\ H{\isachardoublequoteclose}\isanewline
\ \ \ \ \ \ \ \ \ \ {\isachardoublequoteopen}F\ {\isasymin}\ W{\isachardoublequoteclose}\isanewline
\ \ \ \ \ \ \ \ \ \ {\isachardoublequoteopen}finite\ Wo{\isachardoublequoteclose}\isanewline
\ \ \ \ \ \ \ \ \ \ {\isachardoublequoteopen}Wo\ {\isasymsubseteq}\ W{\isachardoublequoteclose}\isanewline
\ \ \ \ \ \ \ \ \isakeyword{shows}\ {\isachardoublequoteopen}sat\ {\isacharparenleft}{\isacharbraceleft}G{\isacharcomma}H{\isacharbraceright}\ {\isasymunion}\ Wo{\isacharparenright}{\isachardoublequoteclose}\isanewline
%
\isadelimproof
%
\endisadelimproof
%
\isatagproof
\isacommand{proof}\isamarkupfalse%
\ {\isacharminus}\isanewline
\ \ \isacommand{have}\isamarkupfalse%
\ {\isachardoublequoteopen}{\isacharbraceleft}G{\isacharcomma}H{\isacharbraceright}\ {\isasymunion}\ Wo\ {\isasymsubseteq}\ {\isacharbraceleft}G{\isacharcomma}H{\isacharcomma}F{\isacharbraceright}\ {\isasymunion}\ Wo{\isachardoublequoteclose}\isanewline
\ \ \ \ \isacommand{by}\isamarkupfalse%
\ blast\isanewline
\ \ \isacommand{have}\isamarkupfalse%
\ {\isachardoublequoteopen}F\ {\isacharequal}\ G\ \isactrlbold {\isasymand}\ H\ {\isasymor}\ \isanewline
\ \ \ \ {\isacharparenleft}{\isasymexists}F{\isadigit{1}}\ G{\isadigit{1}}{\isachardot}\ F\ {\isacharequal}\ \isactrlbold {\isasymnot}\ {\isacharparenleft}F{\isadigit{1}}\ \isactrlbold {\isasymor}\ G{\isadigit{1}}{\isacharparenright}\ {\isasymand}\ G\ {\isacharequal}\ \isactrlbold {\isasymnot}\ F{\isadigit{1}}\ {\isasymand}\ H\ {\isacharequal}\ \isactrlbold {\isasymnot}\ G{\isadigit{1}}{\isacharparenright}\ {\isasymor}\ \isanewline
\ \ \ \ {\isacharparenleft}{\isasymexists}H{\isadigit{1}}{\isachardot}\ F\ {\isacharequal}\ \isactrlbold {\isasymnot}\ {\isacharparenleft}G\ \isactrlbold {\isasymrightarrow}\ H{\isadigit{1}}{\isacharparenright}\ {\isasymand}\ H\ {\isacharequal}\ \isactrlbold {\isasymnot}\ H{\isadigit{1}}{\isacharparenright}\ {\isasymor}\ \isanewline
\ \ \ \ F\ {\isacharequal}\ \isactrlbold {\isasymnot}\ {\isacharparenleft}\isactrlbold {\isasymnot}\ G{\isacharparenright}\ {\isasymand}\ H\ {\isacharequal}\ G{\isachardoublequoteclose}\isanewline
\ \ \ \ \isacommand{using}\isamarkupfalse%
\ assms{\isacharparenleft}{\isadigit{2}}{\isacharparenright}\ \isacommand{by}\isamarkupfalse%
\ {\isacharparenleft}simp\ only{\isacharcolon}\ con{\isacharunderscore}dis{\isacharunderscore}simps{\isacharparenleft}{\isadigit{1}}{\isacharparenright}{\isacharparenright}\isanewline
\ \ \isacommand{then}\isamarkupfalse%
\ \isacommand{have}\isamarkupfalse%
\ {\isachardoublequoteopen}sat\ {\isacharparenleft}{\isacharbraceleft}G{\isacharcomma}H{\isacharcomma}F{\isacharbraceright}\ {\isasymunion}\ Wo{\isacharparenright}{\isachardoublequoteclose}\isanewline
\ \ \isacommand{proof}\isamarkupfalse%
\ {\isacharparenleft}rule\ disjE{\isacharparenright}\isanewline
\ \ \ \ \isacommand{assume}\isamarkupfalse%
\ {\isachardoublequoteopen}F\ {\isacharequal}\ G\ \isactrlbold {\isasymand}\ H{\isachardoublequoteclose}\isanewline
\ \ \ \ \isacommand{show}\isamarkupfalse%
\ {\isachardoublequoteopen}sat\ {\isacharparenleft}{\isacharbraceleft}G{\isacharcomma}H{\isacharcomma}F{\isacharbraceright}\ {\isasymunion}\ Wo{\isacharparenright}{\isachardoublequoteclose}\isanewline
\ \ \ \ \ \ \isacommand{using}\isamarkupfalse%
\ assms{\isacharparenleft}{\isadigit{1}}{\isacharparenright}\ {\isacartoucheopen}F\ {\isacharequal}\ G\ \isactrlbold {\isasymand}\ H{\isacartoucheclose}\ assms{\isacharparenleft}{\isadigit{3}}{\isacharcomma}{\isadigit{4}}{\isacharcomma}{\isadigit{5}}{\isacharparenright}\ \isacommand{by}\isamarkupfalse%
\ {\isacharparenleft}rule\ pcp{\isacharunderscore}colecComp{\isacharunderscore}CON{\isacharunderscore}sat{\isadigit{1}}{\isacharparenright}\isanewline
\ \ \isacommand{next}\isamarkupfalse%
\isanewline
\ \ \ \ \isacommand{assume}\isamarkupfalse%
\ {\isachardoublequoteopen}{\isacharparenleft}{\isasymexists}F{\isadigit{1}}\ G{\isadigit{1}}{\isachardot}\ F\ {\isacharequal}\ \isactrlbold {\isasymnot}\ {\isacharparenleft}F{\isadigit{1}}\ \isactrlbold {\isasymor}\ G{\isadigit{1}}{\isacharparenright}\ {\isasymand}\ G\ {\isacharequal}\ \isactrlbold {\isasymnot}\ F{\isadigit{1}}\ {\isasymand}\ H\ {\isacharequal}\ \isactrlbold {\isasymnot}\ G{\isadigit{1}}{\isacharparenright}\ {\isasymor}\ \isanewline
\ \ \ \ {\isacharparenleft}{\isasymexists}H{\isadigit{1}}{\isachardot}\ F\ {\isacharequal}\ \isactrlbold {\isasymnot}\ {\isacharparenleft}G\ \isactrlbold {\isasymrightarrow}\ H{\isadigit{1}}{\isacharparenright}\ {\isasymand}\ H\ {\isacharequal}\ \isactrlbold {\isasymnot}\ H{\isadigit{1}}{\isacharparenright}\ {\isasymor}\ \isanewline
\ \ \ \ F\ {\isacharequal}\ \isactrlbold {\isasymnot}\ {\isacharparenleft}\isactrlbold {\isasymnot}\ G{\isacharparenright}\ {\isasymand}\ H\ {\isacharequal}\ G{\isachardoublequoteclose}\isanewline
\ \ \ \ \isacommand{thus}\isamarkupfalse%
\ {\isachardoublequoteopen}sat\ {\isacharparenleft}{\isacharbraceleft}G{\isacharcomma}H{\isacharcomma}F{\isacharbraceright}\ {\isasymunion}\ Wo{\isacharparenright}{\isachardoublequoteclose}\isanewline
\ \ \ \ \isacommand{proof}\isamarkupfalse%
\ {\isacharparenleft}rule\ disjE{\isacharparenright}\isanewline
\ \ \ \ \ \ \isacommand{assume}\isamarkupfalse%
\ Ex{\isadigit{2}}{\isacharcolon}{\isachardoublequoteopen}{\isasymexists}F{\isadigit{1}}\ G{\isadigit{1}}{\isachardot}\ F\ {\isacharequal}\ \isactrlbold {\isasymnot}\ {\isacharparenleft}F{\isadigit{1}}\ \isactrlbold {\isasymor}\ G{\isadigit{1}}{\isacharparenright}\ {\isasymand}\ G\ {\isacharequal}\ \isactrlbold {\isasymnot}\ F{\isadigit{1}}\ {\isasymand}\ H\ {\isacharequal}\ \isactrlbold {\isasymnot}\ G{\isadigit{1}}{\isachardoublequoteclose}\ \isanewline
\ \ \ \ \ \ \isacommand{obtain}\isamarkupfalse%
\ F{\isadigit{1}}\ G{\isadigit{1}}\ \isakeyword{where}\ {\isadigit{2}}{\isacharcolon}{\isachardoublequoteopen}F\ {\isacharequal}\ \isactrlbold {\isasymnot}{\isacharparenleft}F{\isadigit{1}}\ \isactrlbold {\isasymor}\ G{\isadigit{1}}{\isacharparenright}\ {\isasymand}\ G\ {\isacharequal}\ \isactrlbold {\isasymnot}\ F{\isadigit{1}}\ {\isasymand}\ H\ {\isacharequal}\ \isactrlbold {\isasymnot}\ G{\isadigit{1}}{\isachardoublequoteclose}\isanewline
\ \ \ \ \ \ \ \ \isacommand{using}\isamarkupfalse%
\ Ex{\isadigit{2}}\ \isacommand{by}\isamarkupfalse%
\ {\isacharparenleft}iprover\ elim{\isacharcolon}\ exE{\isacharparenright}\isanewline
\ \ \ \ \ \ \isacommand{have}\isamarkupfalse%
\ {\isachardoublequoteopen}G\ {\isacharequal}\ \isactrlbold {\isasymnot}\ F{\isadigit{1}}{\isachardoublequoteclose}\isanewline
\ \ \ \ \ \ \ \ \isacommand{using}\isamarkupfalse%
\ {\isadigit{2}}\ \isacommand{by}\isamarkupfalse%
\ {\isacharparenleft}iprover\ elim{\isacharcolon}\ conjunct{\isadigit{1}}{\isacharparenright}\isanewline
\ \ \ \ \ \ \isacommand{have}\isamarkupfalse%
\ {\isachardoublequoteopen}H\ {\isacharequal}\ \isactrlbold {\isasymnot}\ G{\isadigit{1}}{\isachardoublequoteclose}\isanewline
\ \ \ \ \ \ \ \ \isacommand{using}\isamarkupfalse%
\ {\isadigit{2}}\ \isacommand{by}\isamarkupfalse%
\ {\isacharparenleft}iprover\ elim{\isacharcolon}\ conjunct{\isadigit{2}}{\isacharparenright}\isanewline
\ \ \ \ \ \ \isacommand{have}\isamarkupfalse%
\ {\isachardoublequoteopen}F\ {\isacharequal}\ \isactrlbold {\isasymnot}{\isacharparenleft}F{\isadigit{1}}\ \isactrlbold {\isasymor}\ G{\isadigit{1}}{\isacharparenright}{\isachardoublequoteclose}\isanewline
\ \ \ \ \ \ \ \ \isacommand{using}\isamarkupfalse%
\ {\isadigit{2}}\ \isacommand{by}\isamarkupfalse%
\ {\isacharparenleft}rule\ conjunct{\isadigit{1}}{\isacharparenright}\isanewline
\ \ \ \ \ \ \isacommand{have}\isamarkupfalse%
\ {\isachardoublequoteopen}sat\ {\isacharparenleft}{\isacharbraceleft}\isactrlbold {\isasymnot}\ F{\isadigit{1}}{\isacharcomma}\ \isactrlbold {\isasymnot}\ G{\isadigit{1}}{\isacharcomma}\ F{\isacharbraceright}\ {\isasymunion}\ Wo{\isacharparenright}{\isachardoublequoteclose}\isanewline
\ \ \ \ \ \ \ \ \isacommand{using}\isamarkupfalse%
\ assms{\isacharparenleft}{\isadigit{1}}{\isacharparenright}\ {\isacartoucheopen}F\ {\isacharequal}\ \isactrlbold {\isasymnot}{\isacharparenleft}F{\isadigit{1}}\ \isactrlbold {\isasymor}\ G{\isadigit{1}}{\isacharparenright}{\isacartoucheclose}\ assms{\isacharparenleft}{\isadigit{3}}{\isacharcomma}{\isadigit{4}}{\isacharcomma}{\isadigit{5}}{\isacharparenright}\ \isacommand{by}\isamarkupfalse%
\ {\isacharparenleft}rule\ pcp{\isacharunderscore}colecComp{\isacharunderscore}CON{\isacharunderscore}sat{\isadigit{2}}{\isacharparenright}\isanewline
\ \ \ \ \ \ \isacommand{thus}\isamarkupfalse%
\ {\isachardoublequoteopen}sat\ {\isacharparenleft}{\isacharbraceleft}G{\isacharcomma}H{\isacharcomma}F{\isacharbraceright}\ {\isasymunion}\ Wo{\isacharparenright}{\isachardoublequoteclose}\isanewline
\ \ \ \ \ \ \ \ \isacommand{by}\isamarkupfalse%
\ {\isacharparenleft}simp\ only{\isacharcolon}\ {\isacartoucheopen}G\ {\isacharequal}\ \isactrlbold {\isasymnot}\ F{\isadigit{1}}{\isacartoucheclose}\ {\isacartoucheopen}H\ {\isacharequal}\ \isactrlbold {\isasymnot}\ G{\isadigit{1}}{\isacartoucheclose}{\isacharparenright}\isanewline
\ \ \ \ \isacommand{next}\isamarkupfalse%
\isanewline
\ \ \ \ \ \ \isacommand{assume}\isamarkupfalse%
\ {\isachardoublequoteopen}{\isacharparenleft}{\isasymexists}H{\isadigit{1}}{\isachardot}\ F\ {\isacharequal}\ \isactrlbold {\isasymnot}\ {\isacharparenleft}G\ \isactrlbold {\isasymrightarrow}\ H{\isadigit{1}}{\isacharparenright}\ {\isasymand}\ H\ {\isacharequal}\ \isactrlbold {\isasymnot}\ H{\isadigit{1}}{\isacharparenright}\ {\isasymor}\ \isanewline
\ \ \ \ \ \ \ \ \ \ \ \ \ \ F\ {\isacharequal}\ \isactrlbold {\isasymnot}\ {\isacharparenleft}\isactrlbold {\isasymnot}\ G{\isacharparenright}\ {\isasymand}\ H\ {\isacharequal}\ G{\isachardoublequoteclose}\isanewline
\ \ \ \ \ \ \isacommand{thus}\isamarkupfalse%
\ {\isachardoublequoteopen}sat\ {\isacharparenleft}{\isacharbraceleft}G{\isacharcomma}H{\isacharcomma}F{\isacharbraceright}\ {\isasymunion}\ Wo{\isacharparenright}{\isachardoublequoteclose}\isanewline
\ \ \ \ \ \ \isacommand{proof}\isamarkupfalse%
\ {\isacharparenleft}rule\ disjE{\isacharparenright}\isanewline
\ \ \ \ \ \ \ \ \isacommand{assume}\isamarkupfalse%
\ Ex{\isadigit{3}}{\isacharcolon}{\isachardoublequoteopen}{\isasymexists}H{\isadigit{1}}{\isachardot}\ F\ {\isacharequal}\ \isactrlbold {\isasymnot}\ {\isacharparenleft}G\ \isactrlbold {\isasymrightarrow}\ H{\isadigit{1}}{\isacharparenright}\ {\isasymand}\ H\ {\isacharequal}\ \isactrlbold {\isasymnot}\ H{\isadigit{1}}{\isachardoublequoteclose}\isanewline
\ \ \ \ \ \ \ \ \isacommand{obtain}\isamarkupfalse%
\ H{\isadigit{1}}\ \isakeyword{where}\ {\isadigit{3}}{\isacharcolon}{\isachardoublequoteopen}F\ {\isacharequal}\ \isactrlbold {\isasymnot}{\isacharparenleft}G\ \isactrlbold {\isasymrightarrow}\ H{\isadigit{1}}{\isacharparenright}\ {\isasymand}\ H\ {\isacharequal}\ \isactrlbold {\isasymnot}\ H{\isadigit{1}}{\isachardoublequoteclose}\isanewline
\ \ \ \ \ \ \ \ \ \ \isacommand{using}\isamarkupfalse%
\ Ex{\isadigit{3}}\ \isacommand{by}\isamarkupfalse%
\ {\isacharparenleft}rule\ exE{\isacharparenright}\isanewline
\ \ \ \ \ \ \ \ \isacommand{have}\isamarkupfalse%
\ {\isachardoublequoteopen}H\ {\isacharequal}\ \isactrlbold {\isasymnot}\ H{\isadigit{1}}{\isachardoublequoteclose}\isanewline
\ \ \ \ \ \ \ \ \ \ \isacommand{using}\isamarkupfalse%
\ {\isadigit{3}}\ \isacommand{by}\isamarkupfalse%
\ {\isacharparenleft}rule\ conjunct{\isadigit{2}}{\isacharparenright}\isanewline
\ \ \ \ \ \ \ \ \isacommand{have}\isamarkupfalse%
\ {\isachardoublequoteopen}F\ {\isacharequal}\ \isactrlbold {\isasymnot}{\isacharparenleft}G\ \isactrlbold {\isasymrightarrow}\ H{\isadigit{1}}{\isacharparenright}{\isachardoublequoteclose}\isanewline
\ \ \ \ \ \ \ \ \ \ \isacommand{using}\isamarkupfalse%
\ {\isadigit{3}}\ \isacommand{by}\isamarkupfalse%
\ {\isacharparenleft}rule\ conjunct{\isadigit{1}}{\isacharparenright}\isanewline
\ \ \ \ \ \ \ \ \isacommand{have}\isamarkupfalse%
\ {\isachardoublequoteopen}sat\ {\isacharparenleft}{\isacharbraceleft}G{\isacharcomma}\ \isactrlbold {\isasymnot}\ H{\isadigit{1}}{\isacharcomma}\ F{\isacharbraceright}\ {\isasymunion}\ Wo{\isacharparenright}{\isachardoublequoteclose}\isanewline
\ \ \ \ \ \ \ \ \ \ \isacommand{using}\isamarkupfalse%
\ assms{\isacharparenleft}{\isadigit{1}}{\isacharparenright}\ {\isacartoucheopen}F\ {\isacharequal}\ \isactrlbold {\isasymnot}{\isacharparenleft}G\ \isactrlbold {\isasymrightarrow}\ H{\isadigit{1}}{\isacharparenright}{\isacartoucheclose}\ assms{\isacharparenleft}{\isadigit{3}}{\isacharcomma}{\isadigit{4}}{\isacharcomma}{\isadigit{5}}{\isacharparenright}\ \isacommand{by}\isamarkupfalse%
\ {\isacharparenleft}rule\ pcp{\isacharunderscore}colecComp{\isacharunderscore}CON{\isacharunderscore}sat{\isadigit{3}}{\isacharparenright}\isanewline
\ \ \ \ \ \ \ \ \isacommand{thus}\isamarkupfalse%
\ {\isachardoublequoteopen}sat\ {\isacharparenleft}{\isacharbraceleft}G{\isacharcomma}H{\isacharcomma}F{\isacharbraceright}\ {\isasymunion}\ Wo{\isacharparenright}{\isachardoublequoteclose}\isanewline
\ \ \ \ \ \ \ \ \ \ \isacommand{by}\isamarkupfalse%
\ {\isacharparenleft}simp\ only{\isacharcolon}\ {\isacartoucheopen}H\ {\isacharequal}\ \isactrlbold {\isasymnot}\ H{\isadigit{1}}{\isacartoucheclose}{\isacharparenright}\isanewline
\ \ \ \ \ \ \isacommand{next}\isamarkupfalse%
\isanewline
\ \ \ \ \ \ \ \ \isacommand{assume}\isamarkupfalse%
\ {\isachardoublequoteopen}F\ {\isacharequal}\ \isactrlbold {\isasymnot}\ {\isacharparenleft}\isactrlbold {\isasymnot}\ G{\isacharparenright}\ {\isasymand}\ H\ {\isacharequal}\ G{\isachardoublequoteclose}\isanewline
\ \ \ \ \ \ \ \ \isacommand{then}\isamarkupfalse%
\ \isacommand{have}\isamarkupfalse%
\ {\isachardoublequoteopen}H\ {\isacharequal}\ G{\isachardoublequoteclose}\isanewline
\ \ \ \ \ \ \ \ \ \ \isacommand{by}\isamarkupfalse%
\ {\isacharparenleft}rule\ conjunct{\isadigit{2}}{\isacharparenright}\isanewline
\ \ \ \ \ \ \ \ \isacommand{have}\isamarkupfalse%
\ {\isachardoublequoteopen}F\ {\isacharequal}\ \isactrlbold {\isasymnot}\ {\isacharparenleft}\isactrlbold {\isasymnot}\ G{\isacharparenright}{\isachardoublequoteclose}\isanewline
\ \ \ \ \ \ \ \ \ \ \isacommand{using}\isamarkupfalse%
\ {\isacartoucheopen}F\ {\isacharequal}\ \isactrlbold {\isasymnot}\ {\isacharparenleft}\isactrlbold {\isasymnot}\ G{\isacharparenright}\ {\isasymand}\ H\ {\isacharequal}\ G{\isacartoucheclose}\ \isacommand{by}\isamarkupfalse%
\ {\isacharparenleft}rule\ conjunct{\isadigit{1}}{\isacharparenright}\isanewline
\ \ \ \ \ \ \ \ \isacommand{show}\isamarkupfalse%
\ {\isachardoublequoteopen}sat\ {\isacharparenleft}{\isacharbraceleft}G{\isacharcomma}\ H{\isacharcomma}\ F{\isacharbraceright}\ {\isasymunion}\ Wo{\isacharparenright}{\isachardoublequoteclose}\isanewline
\ \ \ \ \ \ \ \ \ \ \isacommand{using}\isamarkupfalse%
\ assms{\isacharparenleft}{\isadigit{1}}{\isacharparenright}\ {\isacartoucheopen}F\ {\isacharequal}\ \isactrlbold {\isasymnot}{\isacharparenleft}\isactrlbold {\isasymnot}\ G{\isacharparenright}{\isacartoucheclose}\ {\isacartoucheopen}H\ {\isacharequal}\ G{\isacartoucheclose}\ assms{\isacharparenleft}{\isadigit{3}}{\isacharcomma}{\isadigit{4}}{\isacharcomma}{\isadigit{5}}{\isacharparenright}\ \isacommand{by}\isamarkupfalse%
\ {\isacharparenleft}rule\ pcp{\isacharunderscore}colecComp{\isacharunderscore}CON{\isacharunderscore}sat{\isadigit{4}}{\isacharparenright}\isanewline
\ \ \ \ \ \ \isacommand{qed}\isamarkupfalse%
\isanewline
\ \ \ \ \isacommand{qed}\isamarkupfalse%
\isanewline
\ \ \isacommand{qed}\isamarkupfalse%
\isanewline
\ \ \isacommand{thus}\isamarkupfalse%
\ {\isachardoublequoteopen}sat\ {\isacharparenleft}{\isacharbraceleft}G{\isacharcomma}H{\isacharbraceright}\ {\isasymunion}\ Wo{\isacharparenright}{\isachardoublequoteclose}\isanewline
\ \ \ \ \isacommand{using}\isamarkupfalse%
\ {\isacartoucheopen}{\isacharbraceleft}G{\isacharcomma}H{\isacharbraceright}\ {\isasymunion}\ Wo\ {\isasymsubseteq}\ {\isacharbraceleft}G{\isacharcomma}H{\isacharcomma}F{\isacharbraceright}\ {\isasymunion}\ Wo{\isacartoucheclose}\ \isacommand{by}\isamarkupfalse%
\ {\isacharparenleft}simp\ only{\isacharcolon}\ sat{\isacharunderscore}mono{\isacharparenright}\isanewline
\isacommand{qed}\isamarkupfalse%
%
\endisatagproof
{\isafoldproof}%
%
\isadelimproof
\isanewline
%
\endisadelimproof
\ \ \ \ \ \ \isanewline
\isacommand{lemma}\isamarkupfalse%
\ pcp{\isacharunderscore}colecComp{\isacharunderscore}CON{\isacharcolon}\isanewline
\ \ \isakeyword{assumes}\ {\isachardoublequoteopen}W\ {\isasymin}\ colecComp{\isachardoublequoteclose}\isanewline
\ \ \isakeyword{shows}\ {\isachardoublequoteopen}{\isasymforall}F\ G\ H{\isachardot}\ Con\ F\ G\ H\ {\isasymlongrightarrow}\ F\ {\isasymin}\ W\ {\isasymlongrightarrow}\ {\isacharbraceleft}G{\isacharcomma}H{\isacharbraceright}\ {\isasymunion}\ W\ {\isasymin}\ colecComp{\isachardoublequoteclose}\isanewline
%
\isadelimproof
%
\endisadelimproof
%
\isatagproof
\isacommand{proof}\isamarkupfalse%
\ {\isacharparenleft}rule\ allI{\isacharparenright}{\isacharplus}\isanewline
\ \ \isacommand{fix}\isamarkupfalse%
\ F\ G\ H\isanewline
\ \ \isacommand{show}\isamarkupfalse%
\ {\isachardoublequoteopen}Con\ F\ G\ H\ {\isasymlongrightarrow}\ F\ {\isasymin}\ W\ {\isasymlongrightarrow}\ {\isacharbraceleft}G{\isacharcomma}H{\isacharbraceright}\ {\isasymunion}\ W\ {\isasymin}\ colecComp{\isachardoublequoteclose}\isanewline
\ \ \isacommand{proof}\isamarkupfalse%
\ {\isacharparenleft}rule\ impI{\isacharparenright}{\isacharplus}\isanewline
\ \ \ \ \isacommand{assume}\isamarkupfalse%
\ {\isachardoublequoteopen}Con\ F\ G\ H{\isachardoublequoteclose}\isanewline
\ \ \ \ \isacommand{assume}\isamarkupfalse%
\ {\isachardoublequoteopen}F\ {\isasymin}\ W{\isachardoublequoteclose}\isanewline
\ \ \ \ \isacommand{show}\isamarkupfalse%
\ {\isachardoublequoteopen}{\isacharbraceleft}G{\isacharcomma}H{\isacharbraceright}\ {\isasymunion}\ W\ {\isasymin}\ colecComp{\isachardoublequoteclose}\isanewline
\ \ \ \ \ \ \isacommand{unfolding}\isamarkupfalse%
\ colecComp\ fin{\isacharunderscore}sat{\isacharunderscore}def\isanewline
\ \ \ \ \isacommand{proof}\isamarkupfalse%
\ {\isacharparenleft}rule\ CollectI{\isacharparenright}\isanewline
\ \ \ \ \ \ \isacommand{show}\isamarkupfalse%
\ {\isachardoublequoteopen}{\isasymforall}S{\isacharprime}\ {\isasymsubseteq}\ {\isacharbraceleft}G{\isacharcomma}H{\isacharbraceright}\ {\isasymunion}\ W{\isachardot}\ finite\ S{\isacharprime}\ {\isasymlongrightarrow}\ sat\ S{\isacharprime}{\isachardoublequoteclose}\isanewline
\ \ \ \ \ \ \isacommand{proof}\isamarkupfalse%
\ {\isacharparenleft}rule\ sallI{\isacharparenright}\isanewline
\ \ \ \ \ \ \ \ \isacommand{fix}\isamarkupfalse%
\ S{\isacharprime}\isanewline
\ \ \ \ \ \ \ \ \isacommand{assume}\isamarkupfalse%
\ {\isachardoublequoteopen}S{\isacharprime}\ {\isasymsubseteq}\ {\isacharbraceleft}G{\isacharcomma}H{\isacharbraceright}\ {\isasymunion}\ W{\isachardoublequoteclose}\isanewline
\ \ \ \ \ \ \ \ \isacommand{then}\isamarkupfalse%
\ \isacommand{have}\isamarkupfalse%
\ {\isachardoublequoteopen}S{\isacharprime}\ {\isasymsubseteq}\ {\isacharbraceleft}G{\isacharbraceright}\ {\isasymunion}\ {\isacharparenleft}{\isacharbraceleft}H{\isacharbraceright}\ {\isasymunion}\ W{\isacharparenright}{\isachardoublequoteclose}\isanewline
\ \ \ \ \ \ \ \ \ \ \isacommand{by}\isamarkupfalse%
\ blast\ \isanewline
\ \ \ \ \ \ \ \ \isacommand{show}\isamarkupfalse%
\ {\isachardoublequoteopen}finite\ S{\isacharprime}\ {\isasymlongrightarrow}\ sat\ S{\isacharprime}{\isachardoublequoteclose}\isanewline
\ \ \ \ \ \ \ \ \isacommand{proof}\isamarkupfalse%
\ {\isacharparenleft}rule\ impI{\isacharparenright}\isanewline
\ \ \ \ \ \ \ \ \ \ \isacommand{assume}\isamarkupfalse%
\ {\isachardoublequoteopen}finite\ S{\isacharprime}{\isachardoublequoteclose}\ \isanewline
\ \ \ \ \ \ \ \ \ \ \isacommand{have}\isamarkupfalse%
\ Ex{\isacharcolon}{\isachardoublequoteopen}{\isasymexists}Wo\ {\isasymsubseteq}\ W{\isachardot}\ finite\ Wo\ {\isasymand}\ {\isacharparenleft}S{\isacharprime}\ {\isacharequal}\ {\isacharbraceleft}G{\isacharcomma}H{\isacharbraceright}\ {\isasymunion}\ Wo\ {\isasymor}\ S{\isacharprime}\ {\isacharequal}\ {\isacharbraceleft}G{\isacharbraceright}\ {\isasymunion}\ Wo\ {\isasymor}\ S{\isacharprime}\ {\isacharequal}\ {\isacharbraceleft}H{\isacharbraceright}\ {\isasymunion}\ Wo\ {\isasymor}\ S{\isacharprime}\ {\isacharequal}\ Wo{\isacharparenright}{\isachardoublequoteclose}\isanewline
\ \ \ \ \ \ \ \ \ \ \ \ \isacommand{using}\isamarkupfalse%
\ {\isacartoucheopen}finite\ S{\isacharprime}{\isacartoucheclose}\ {\isacartoucheopen}S{\isacharprime}\ {\isasymsubseteq}\ {\isacharbraceleft}G{\isacharcomma}H{\isacharbraceright}\ {\isasymunion}\ W{\isacartoucheclose}\ \isacommand{by}\isamarkupfalse%
\ {\isacharparenleft}rule\ finite{\isacharunderscore}subset{\isacharunderscore}insert{\isadigit{2}}{\isacharparenright}\isanewline
\ \ \ \ \ \ \ \ \ \ \isacommand{obtain}\isamarkupfalse%
\ Wo{\isacharprime}\ \isakeyword{where}\ {\isachardoublequoteopen}Wo{\isacharprime}\ {\isasymsubseteq}\ W{\isachardoublequoteclose}\ \isakeyword{and}\ {\isadigit{1}}{\isacharcolon}{\isachardoublequoteopen}finite\ Wo{\isacharprime}\ {\isasymand}\ {\isacharparenleft}S{\isacharprime}\ {\isacharequal}\ {\isacharbraceleft}G{\isacharcomma}H{\isacharbraceright}\ {\isasymunion}\ Wo{\isacharprime}\ {\isasymor}\ S{\isacharprime}\ {\isacharequal}\ {\isacharbraceleft}G{\isacharbraceright}\ {\isasymunion}\ Wo{\isacharprime}\ {\isasymor}\ S{\isacharprime}\ {\isacharequal}\ {\isacharbraceleft}H{\isacharbraceright}\ {\isasymunion}\ Wo{\isacharprime}\ {\isasymor}\ S{\isacharprime}\ {\isacharequal}\ Wo{\isacharprime}{\isacharparenright}{\isachardoublequoteclose}\isanewline
\ \ \ \ \ \ \ \ \ \ \ \ \isacommand{using}\isamarkupfalse%
\ Ex\ \isacommand{by}\isamarkupfalse%
\ {\isacharparenleft}rule\ subexE{\isacharparenright}\isanewline
\ \ \ \ \ \ \ \ \ \ \isacommand{have}\isamarkupfalse%
\ {\isachardoublequoteopen}finite\ Wo{\isacharprime}{\isachardoublequoteclose}\isanewline
\ \ \ \ \ \ \ \ \ \ \ \ \isacommand{using}\isamarkupfalse%
\ {\isadigit{1}}\ \isacommand{by}\isamarkupfalse%
\ {\isacharparenleft}rule\ conjunct{\isadigit{1}}{\isacharparenright}\isanewline
\ \ \ \ \ \ \ \ \ \ \ \ \isacommand{have}\isamarkupfalse%
\ {\isachardoublequoteopen}sat\ {\isacharparenleft}{\isacharbraceleft}G{\isacharcomma}H{\isacharbraceright}\ {\isasymunion}\ Wo{\isacharprime}{\isacharparenright}{\isachardoublequoteclose}\ \isanewline
\ \ \ \ \ \ \ \ \ \ \ \ \ \ \isacommand{using}\isamarkupfalse%
\ {\isacartoucheopen}W\ {\isasymin}\ colecComp{\isacartoucheclose}\ {\isacartoucheopen}Con\ F\ G\ H{\isacartoucheclose}\ {\isacartoucheopen}F\ {\isasymin}\ W{\isacartoucheclose}\ {\isacartoucheopen}finite\ Wo{\isacharprime}{\isacartoucheclose}\ {\isacartoucheopen}Wo{\isacharprime}\ {\isasymsubseteq}\ W{\isacartoucheclose}\ \isacommand{by}\isamarkupfalse%
\ {\isacharparenleft}rule\ pcp{\isacharunderscore}colecComp{\isacharunderscore}CON{\isacharunderscore}sat{\isacharparenright}\isanewline
\ \ \ \ \ \ \ \ \ \ \isacommand{have}\isamarkupfalse%
\ {\isachardoublequoteopen}S{\isacharprime}\ {\isacharequal}\ {\isacharbraceleft}G{\isacharcomma}H{\isacharbraceright}\ {\isasymunion}\ Wo{\isacharprime}\ {\isasymor}\ S{\isacharprime}\ {\isacharequal}\ {\isacharbraceleft}G{\isacharbraceright}\ {\isasymunion}\ Wo{\isacharprime}\ {\isasymor}\ S{\isacharprime}\ {\isacharequal}\ {\isacharbraceleft}H{\isacharbraceright}\ {\isasymunion}\ Wo{\isacharprime}\ {\isasymor}\ S{\isacharprime}\ {\isacharequal}\ Wo{\isacharprime}{\isachardoublequoteclose}\isanewline
\ \ \ \ \ \ \ \ \ \ \ \ \isacommand{using}\isamarkupfalse%
\ {\isadigit{1}}\ \isacommand{by}\isamarkupfalse%
\ {\isacharparenleft}rule\ conjunct{\isadigit{2}}{\isacharparenright}\isanewline
\ \ \ \ \ \ \ \ \ \ \isacommand{thus}\isamarkupfalse%
\ {\isachardoublequoteopen}sat\ S{\isacharprime}{\isachardoublequoteclose}\isanewline
\ \ \ \ \ \ \ \ \ \ \isacommand{proof}\isamarkupfalse%
\ {\isacharparenleft}rule\ disjE{\isacharparenright}\isanewline
\ \ \ \ \ \ \ \ \ \ \ \ \isacommand{assume}\isamarkupfalse%
\ {\isachardoublequoteopen}S{\isacharprime}\ {\isacharequal}\ {\isacharbraceleft}G{\isacharcomma}H{\isacharbraceright}\ {\isasymunion}\ Wo{\isacharprime}{\isachardoublequoteclose}\isanewline
\ \ \ \ \ \ \ \ \ \ \ \ \isacommand{show}\isamarkupfalse%
\ {\isachardoublequoteopen}sat\ S{\isacharprime}{\isachardoublequoteclose}\isanewline
\ \ \ \ \ \ \ \ \ \ \ \ \ \ \isacommand{using}\isamarkupfalse%
\ {\isacartoucheopen}sat{\isacharparenleft}{\isacharbraceleft}G{\isacharcomma}H{\isacharbraceright}\ {\isasymunion}\ Wo{\isacharprime}{\isacharparenright}{\isacartoucheclose}\ \isacommand{by}\isamarkupfalse%
\ {\isacharparenleft}simp\ only{\isacharcolon}\ {\isacartoucheopen}S{\isacharprime}\ {\isacharequal}\ {\isacharbraceleft}G{\isacharcomma}H{\isacharbraceright}\ {\isasymunion}\ Wo{\isacharprime}{\isacartoucheclose}{\isacharparenright}\isanewline
\ \ \ \ \ \ \ \ \ \ \isacommand{next}\isamarkupfalse%
\isanewline
\ \ \ \ \ \ \ \ \ \ \ \ \isacommand{assume}\isamarkupfalse%
\ {\isachardoublequoteopen}S{\isacharprime}\ {\isacharequal}\ {\isacharbraceleft}G{\isacharbraceright}\ {\isasymunion}\ Wo{\isacharprime}\ {\isasymor}\ S{\isacharprime}\ {\isacharequal}\ {\isacharbraceleft}H{\isacharbraceright}\ {\isasymunion}\ Wo{\isacharprime}\ {\isasymor}\ S{\isacharprime}\ {\isacharequal}\ Wo{\isacharprime}{\isachardoublequoteclose}\isanewline
\ \ \ \ \ \ \ \ \ \ \ \ \isacommand{thus}\isamarkupfalse%
\ {\isachardoublequoteopen}sat\ S{\isacharprime}{\isachardoublequoteclose}\isanewline
\ \ \ \ \ \ \ \ \ \ \ \ \isacommand{proof}\isamarkupfalse%
\ {\isacharparenleft}rule\ disjE{\isacharparenright}\isanewline
\ \ \ \ \ \ \ \ \ \ \ \ \ \ \isacommand{assume}\isamarkupfalse%
\ {\isachardoublequoteopen}S{\isacharprime}\ {\isacharequal}\ {\isacharbraceleft}G{\isacharbraceright}\ {\isasymunion}\ Wo{\isacharprime}{\isachardoublequoteclose}\isanewline
\ \ \ \ \ \ \ \ \ \ \ \ \ \ \isacommand{then}\isamarkupfalse%
\ \isacommand{have}\isamarkupfalse%
\ {\isachardoublequoteopen}S{\isacharprime}\ {\isasymsubseteq}\ {\isacharbraceleft}G{\isacharcomma}H{\isacharbraceright}\ {\isasymunion}\ Wo{\isacharprime}{\isachardoublequoteclose}\isanewline
\ \ \ \ \ \ \ \ \ \ \ \ \ \ \ \ \isacommand{by}\isamarkupfalse%
\ blast\ \isanewline
\ \ \ \ \ \ \ \ \ \ \ \ \ \ \isacommand{thus}\isamarkupfalse%
\ {\isachardoublequoteopen}sat\ S{\isacharprime}{\isachardoublequoteclose}\isanewline
\ \ \ \ \ \ \ \ \ \ \ \ \ \ \ \ \isacommand{using}\isamarkupfalse%
\ {\isacartoucheopen}sat{\isacharparenleft}{\isacharbraceleft}G{\isacharcomma}H{\isacharbraceright}\ {\isasymunion}\ Wo{\isacharprime}{\isacharparenright}{\isacartoucheclose}\ \isacommand{by}\isamarkupfalse%
\ {\isacharparenleft}rule\ sat{\isacharunderscore}mono{\isacharparenright}\isanewline
\ \ \ \ \ \ \ \ \ \ \ \ \isacommand{next}\isamarkupfalse%
\isanewline
\ \ \ \ \ \ \ \ \ \ \ \ \ \ \isacommand{assume}\isamarkupfalse%
\ {\isachardoublequoteopen}S{\isacharprime}\ {\isacharequal}\ {\isacharbraceleft}H{\isacharbraceright}\ {\isasymunion}\ Wo{\isacharprime}\ {\isasymor}\ S{\isacharprime}\ {\isacharequal}\ Wo{\isacharprime}{\isachardoublequoteclose}\isanewline
\ \ \ \ \ \ \ \ \ \ \ \ \ \ \isacommand{thus}\isamarkupfalse%
\ {\isachardoublequoteopen}sat\ S{\isacharprime}{\isachardoublequoteclose}\isanewline
\ \ \ \ \ \ \ \ \ \ \ \ \ \ \isacommand{proof}\isamarkupfalse%
\ {\isacharparenleft}rule\ disjE{\isacharparenright}\isanewline
\ \ \ \ \ \ \ \ \ \ \ \ \ \ \ \ \isacommand{assume}\isamarkupfalse%
\ {\isachardoublequoteopen}S{\isacharprime}\ {\isacharequal}\ {\isacharbraceleft}H{\isacharbraceright}\ {\isasymunion}\ Wo{\isacharprime}{\isachardoublequoteclose}\isanewline
\ \ \ \ \ \ \ \ \ \ \ \ \ \ \ \ \isacommand{then}\isamarkupfalse%
\ \isacommand{have}\isamarkupfalse%
\ {\isachardoublequoteopen}S{\isacharprime}\ {\isasymsubseteq}\ {\isacharbraceleft}G{\isacharcomma}H{\isacharbraceright}\ {\isasymunion}\ Wo{\isacharprime}{\isachardoublequoteclose}\isanewline
\ \ \ \ \ \ \ \ \ \ \ \ \ \ \ \ \ \ \isacommand{by}\isamarkupfalse%
\ blast\ \isanewline
\ \ \ \ \ \ \ \ \ \ \ \ \ \ \ \ \isacommand{thus}\isamarkupfalse%
\ {\isachardoublequoteopen}sat\ S{\isacharprime}{\isachardoublequoteclose}\isanewline
\ \ \ \ \ \ \ \ \ \ \ \ \ \ \ \ \ \ \isacommand{using}\isamarkupfalse%
\ {\isacartoucheopen}sat{\isacharparenleft}{\isacharbraceleft}G{\isacharcomma}H{\isacharbraceright}\ {\isasymunion}\ Wo{\isacharprime}{\isacharparenright}{\isacartoucheclose}\ \isacommand{by}\isamarkupfalse%
\ {\isacharparenleft}rule\ sat{\isacharunderscore}mono{\isacharparenright}\isanewline
\ \ \ \ \ \ \ \ \ \ \ \ \ \ \isacommand{next}\isamarkupfalse%
\isanewline
\ \ \ \ \ \ \ \ \ \ \ \ \ \ \ \ \isacommand{assume}\isamarkupfalse%
\ {\isachardoublequoteopen}S{\isacharprime}\ {\isacharequal}\ Wo{\isacharprime}{\isachardoublequoteclose}\isanewline
\ \ \ \ \ \ \ \ \ \ \ \ \ \ \ \ \isacommand{then}\isamarkupfalse%
\ \isacommand{have}\isamarkupfalse%
\ {\isachardoublequoteopen}S{\isacharprime}\ {\isasymsubseteq}\ {\isacharbraceleft}G{\isacharcomma}H{\isacharbraceright}\ {\isasymunion}\ Wo{\isacharprime}{\isachardoublequoteclose}\isanewline
\ \ \ \ \ \ \ \ \ \ \ \ \ \ \ \ \ \ \isacommand{by}\isamarkupfalse%
\ {\isacharparenleft}simp\ only{\isacharcolon}\ Un{\isacharunderscore}upper{\isadigit{2}}{\isacharparenright}\isanewline
\ \ \ \ \ \ \ \ \ \ \ \ \ \ \ \ \isacommand{thus}\isamarkupfalse%
\ {\isachardoublequoteopen}sat\ S{\isacharprime}{\isachardoublequoteclose}\isanewline
\ \ \ \ \ \ \ \ \ \ \ \ \ \ \ \ \ \ \isacommand{using}\isamarkupfalse%
\ {\isacartoucheopen}sat{\isacharparenleft}{\isacharbraceleft}G{\isacharcomma}H{\isacharbraceright}\ {\isasymunion}\ Wo{\isacharprime}{\isacharparenright}{\isacartoucheclose}\ \isacommand{by}\isamarkupfalse%
\ {\isacharparenleft}rule\ sat{\isacharunderscore}mono{\isacharparenright}\isanewline
\ \ \ \ \ \ \ \ \ \ \ \ \ \ \isacommand{qed}\isamarkupfalse%
\isanewline
\ \ \ \ \ \ \ \ \ \ \ \ \isacommand{qed}\isamarkupfalse%
\isanewline
\ \ \ \ \ \ \ \ \ \ \isacommand{qed}\isamarkupfalse%
\isanewline
\ \ \ \ \ \ \ \ \isacommand{qed}\isamarkupfalse%
\isanewline
\ \ \ \ \ \ \isacommand{qed}\isamarkupfalse%
\isanewline
\ \ \ \ \isacommand{qed}\isamarkupfalse%
\isanewline
\ \ \isacommand{qed}\isamarkupfalse%
\isanewline
\isacommand{qed}\isamarkupfalse%
%
\endisatagproof
{\isafoldproof}%
%
\isadelimproof
\isanewline
%
\endisadelimproof
\isanewline
\isacommand{lemma}\isamarkupfalse%
\ pcp{\isacharunderscore}colecComp{\isacharunderscore}DIS{\isacharunderscore}sat{\isadigit{1}}{\isacharcolon}\isanewline
\ \ \isakeyword{assumes}\ {\isachardoublequoteopen}W\ {\isasymin}\ colecComp{\isachardoublequoteclose}\isanewline
\ \ \ \ \ \ \ \ \ \ {\isachardoublequoteopen}F\ {\isacharequal}\ G\ \isactrlbold {\isasymor}\ H{\isachardoublequoteclose}\isanewline
\ \ \ \ \ \ \ \ \ \ {\isachardoublequoteopen}F\ {\isasymin}\ W{\isachardoublequoteclose}\isanewline
\ \ \ \ \ \ \ \ \ \ {\isachardoublequoteopen}finite\ Wo{\isachardoublequoteclose}\isanewline
\ \ \ \ \ \ \ \ \ \ {\isachardoublequoteopen}Wo\ {\isasymsubseteq}\ W{\isachardoublequoteclose}\isanewline
\ \ \ \ \ \ \ \ \isakeyword{shows}\ {\isachardoublequoteopen}sat\ {\isacharparenleft}{\isacharbraceleft}G{\isacharcomma}F{\isacharbraceright}\ {\isasymunion}\ Wo{\isacharparenright}\ {\isasymor}\ sat\ {\isacharparenleft}{\isacharbraceleft}H{\isacharcomma}F{\isacharbraceright}\ {\isasymunion}\ Wo{\isacharparenright}{\isachardoublequoteclose}\isanewline
%
\isadelimproof
%
\endisadelimproof
%
\isatagproof
\isacommand{proof}\isamarkupfalse%
\ {\isacharminus}\isanewline
\ \ \isacommand{have}\isamarkupfalse%
\ {\isachardoublequoteopen}sat\ {\isacharparenleft}{\isacharbraceleft}F{\isacharbraceright}\ {\isasymunion}\ Wo{\isacharparenright}{\isachardoublequoteclose}\isanewline
\ \ \ \ \isacommand{using}\isamarkupfalse%
\ assms{\isacharparenleft}{\isadigit{1}}{\isacharcomma}{\isadigit{3}}{\isacharcomma}{\isadigit{4}}{\isacharcomma}{\isadigit{5}}{\isacharparenright}\ \isacommand{by}\isamarkupfalse%
\ {\isacharparenleft}rule\ pcp{\isacharunderscore}colecComp{\isacharunderscore}elem{\isacharunderscore}sat{\isacharparenright}\isanewline
\ \ \isacommand{have}\isamarkupfalse%
\ {\isachardoublequoteopen}F\ {\isasymin}\ {\isacharbraceleft}F{\isacharbraceright}\ {\isasymunion}\ Wo{\isachardoublequoteclose}\isanewline
\ \ \ \ \isacommand{by}\isamarkupfalse%
\ simp\ \isanewline
\ \ \isacommand{have}\isamarkupfalse%
\ Ex{\isadigit{1}}{\isacharcolon}{\isachardoublequoteopen}{\isasymexists}{\isasymA}{\isachardot}\ {\isasymforall}F\ {\isasymin}\ {\isacharparenleft}{\isacharbraceleft}F{\isacharbraceright}\ {\isasymunion}\ Wo{\isacharparenright}{\isachardot}\ {\isasymA}\ {\isasymTurnstile}\ F{\isachardoublequoteclose}\isanewline
\ \ \ \ \isacommand{using}\isamarkupfalse%
\ {\isacartoucheopen}sat\ {\isacharparenleft}{\isacharbraceleft}F{\isacharbraceright}\ {\isasymunion}\ Wo{\isacharparenright}{\isacartoucheclose}\ \isacommand{by}\isamarkupfalse%
\ {\isacharparenleft}simp\ only{\isacharcolon}\ sat{\isacharunderscore}def{\isacharparenright}\isanewline
\ \ \isacommand{obtain}\isamarkupfalse%
\ {\isasymA}\ \isakeyword{where}\ Forall{\isadigit{1}}{\isacharcolon}{\isachardoublequoteopen}{\isasymforall}F\ {\isasymin}\ {\isacharparenleft}{\isacharbraceleft}F{\isacharbraceright}\ {\isasymunion}\ Wo{\isacharparenright}{\isachardot}\ {\isasymA}\ {\isasymTurnstile}\ F{\isachardoublequoteclose}\isanewline
\ \ \ \ \isacommand{using}\isamarkupfalse%
\ Ex{\isadigit{1}}\ \isacommand{by}\isamarkupfalse%
\ {\isacharparenleft}rule\ exE{\isacharparenright}\isanewline
\ \ \isacommand{have}\isamarkupfalse%
\ {\isachardoublequoteopen}{\isasymA}\ {\isasymTurnstile}\ F{\isachardoublequoteclose}\isanewline
\ \ \ \ \isacommand{using}\isamarkupfalse%
\ Forall{\isadigit{1}}\ {\isacartoucheopen}F\ {\isasymin}\ {\isacharbraceleft}F{\isacharbraceright}\ {\isasymunion}\ Wo{\isacartoucheclose}\ \isacommand{by}\isamarkupfalse%
\ {\isacharparenleft}rule\ bspec{\isacharparenright}\isanewline
\ \ \isacommand{then}\isamarkupfalse%
\ \isacommand{have}\isamarkupfalse%
\ {\isachardoublequoteopen}{\isasymA}\ {\isasymTurnstile}\ {\isacharparenleft}G\ \isactrlbold {\isasymor}\ H{\isacharparenright}{\isachardoublequoteclose}\isanewline
\ \ \ \ \isacommand{using}\isamarkupfalse%
\ assms{\isacharparenleft}{\isadigit{2}}{\isacharparenright}\ \isacommand{by}\isamarkupfalse%
\ {\isacharparenleft}simp\ only{\isacharcolon}\ {\isacartoucheopen}{\isasymA}\ {\isasymTurnstile}\ F{\isacartoucheclose}{\isacharparenright}\isanewline
\ \ \isacommand{then}\isamarkupfalse%
\ \isacommand{have}\isamarkupfalse%
\ {\isachardoublequoteopen}{\isasymA}\ {\isasymTurnstile}\ G\ {\isasymor}\ {\isasymA}\ {\isasymTurnstile}\ H{\isachardoublequoteclose}\isanewline
\ \ \ \ \isacommand{by}\isamarkupfalse%
\ {\isacharparenleft}simp\ only{\isacharcolon}\ formula{\isacharunderscore}semantics{\isachardot}simps{\isacharparenleft}{\isadigit{5}}{\isacharparenright}{\isacharparenright}\isanewline
\ \ \isacommand{thus}\isamarkupfalse%
\ {\isacharquery}thesis\isanewline
\ \ \isacommand{proof}\isamarkupfalse%
\ {\isacharparenleft}rule\ disjE{\isacharparenright}\isanewline
\ \ \ \ \isacommand{assume}\isamarkupfalse%
\ {\isachardoublequoteopen}{\isasymA}\ {\isasymTurnstile}\ G{\isachardoublequoteclose}\isanewline
\ \ \ \ \isacommand{then}\isamarkupfalse%
\ \isacommand{have}\isamarkupfalse%
\ {\isachardoublequoteopen}{\isasymforall}F\ {\isasymin}\ {\isacharbraceleft}G{\isacharbraceright}{\isachardot}\ {\isasymA}\ {\isasymTurnstile}\ F{\isachardoublequoteclose}\isanewline
\ \ \ \ \ \ \isacommand{by}\isamarkupfalse%
\ simp\isanewline
\ \ \ \ \isacommand{then}\isamarkupfalse%
\ \isacommand{have}\isamarkupfalse%
\ {\isachardoublequoteopen}{\isasymforall}F\ {\isasymin}\ {\isacharparenleft}{\isacharbraceleft}G{\isacharbraceright}\ {\isasymunion}\ {\isacharparenleft}{\isacharbraceleft}F{\isacharbraceright}\ {\isasymunion}\ Wo{\isacharparenright}{\isacharparenright}{\isachardot}\ {\isasymA}\ {\isasymTurnstile}\ F{\isachardoublequoteclose}\isanewline
\ \ \ \ \ \ \isacommand{using}\isamarkupfalse%
\ Forall{\isadigit{1}}\ \isacommand{by}\isamarkupfalse%
\ {\isacharparenleft}rule\ ball{\isacharunderscore}Un{\isacharparenright}\isanewline
\ \ \ \ \isacommand{then}\isamarkupfalse%
\ \isacommand{have}\isamarkupfalse%
\ {\isachardoublequoteopen}{\isasymforall}F\ {\isasymin}\ {\isacharbraceleft}G{\isacharcomma}F{\isacharbraceright}\ {\isasymunion}\ Wo{\isachardot}\ {\isasymA}\ {\isasymTurnstile}\ F{\isachardoublequoteclose}\isanewline
\ \ \ \ \ \ \isacommand{by}\isamarkupfalse%
\ simp\ \isanewline
\ \ \ \ \isacommand{then}\isamarkupfalse%
\ \isacommand{have}\isamarkupfalse%
\ {\isachardoublequoteopen}{\isasymexists}{\isasymA}{\isachardot}\ {\isasymforall}F\ {\isasymin}\ {\isacharparenleft}{\isacharbraceleft}G{\isacharcomma}F{\isacharbraceright}\ {\isasymunion}\ Wo{\isacharparenright}{\isachardot}\ {\isasymA}\ {\isasymTurnstile}\ F{\isachardoublequoteclose}\isanewline
\ \ \ \ \ \ \isacommand{by}\isamarkupfalse%
\ {\isacharparenleft}iprover\ intro{\isacharcolon}\ exI{\isacharparenright}\isanewline
\ \ \ \ \isacommand{then}\isamarkupfalse%
\ \isacommand{have}\isamarkupfalse%
\ {\isachardoublequoteopen}sat\ {\isacharparenleft}{\isacharbraceleft}G{\isacharcomma}F{\isacharbraceright}\ {\isasymunion}\ Wo{\isacharparenright}{\isachardoublequoteclose}\isanewline
\ \ \ \ \ \ \isacommand{by}\isamarkupfalse%
\ {\isacharparenleft}simp\ only{\isacharcolon}\ sat{\isacharunderscore}def{\isacharparenright}\isanewline
\ \ \ \ \isacommand{thus}\isamarkupfalse%
\ {\isacharquery}thesis\isanewline
\ \ \ \ \ \ \isacommand{by}\isamarkupfalse%
\ {\isacharparenleft}rule\ disjI{\isadigit{1}}{\isacharparenright}\isanewline
\ \ \isacommand{next}\isamarkupfalse%
\isanewline
\ \ \ \ \isacommand{assume}\isamarkupfalse%
\ {\isachardoublequoteopen}{\isasymA}\ {\isasymTurnstile}\ H{\isachardoublequoteclose}\isanewline
\ \ \ \ \isacommand{then}\isamarkupfalse%
\ \isacommand{have}\isamarkupfalse%
\ {\isachardoublequoteopen}{\isasymforall}F\ {\isasymin}\ {\isacharbraceleft}H{\isacharbraceright}{\isachardot}\ {\isasymA}\ {\isasymTurnstile}\ F{\isachardoublequoteclose}\isanewline
\ \ \ \ \ \ \isacommand{by}\isamarkupfalse%
\ simp\isanewline
\ \ \ \ \isacommand{then}\isamarkupfalse%
\ \isacommand{have}\isamarkupfalse%
\ {\isachardoublequoteopen}{\isasymforall}F\ {\isasymin}\ {\isacharparenleft}{\isacharbraceleft}H{\isacharbraceright}\ {\isasymunion}\ {\isacharparenleft}{\isacharbraceleft}F{\isacharbraceright}\ {\isasymunion}\ Wo{\isacharparenright}{\isacharparenright}{\isachardot}\ {\isasymA}\ {\isasymTurnstile}\ F{\isachardoublequoteclose}\isanewline
\ \ \ \ \ \ \isacommand{using}\isamarkupfalse%
\ Forall{\isadigit{1}}\ \isacommand{by}\isamarkupfalse%
\ {\isacharparenleft}rule\ ball{\isacharunderscore}Un{\isacharparenright}\isanewline
\ \ \ \ \isacommand{then}\isamarkupfalse%
\ \isacommand{have}\isamarkupfalse%
\ {\isachardoublequoteopen}{\isasymforall}F\ {\isasymin}\ {\isacharbraceleft}H{\isacharcomma}F{\isacharbraceright}\ {\isasymunion}\ Wo{\isachardot}\ {\isasymA}\ {\isasymTurnstile}\ F{\isachardoublequoteclose}\isanewline
\ \ \ \ \ \ \isacommand{by}\isamarkupfalse%
\ simp\isanewline
\ \ \ \ \isacommand{then}\isamarkupfalse%
\ \isacommand{have}\isamarkupfalse%
\ {\isachardoublequoteopen}{\isasymexists}{\isasymA}{\isachardot}\ {\isasymforall}F\ {\isasymin}\ {\isacharparenleft}{\isacharbraceleft}H{\isacharcomma}F{\isacharbraceright}\ {\isasymunion}\ Wo{\isacharparenright}{\isachardot}\ {\isasymA}\ {\isasymTurnstile}\ F{\isachardoublequoteclose}\isanewline
\ \ \ \ \ \ \isacommand{by}\isamarkupfalse%
\ {\isacharparenleft}iprover\ intro{\isacharcolon}\ exI{\isacharparenright}\isanewline
\ \ \ \ \isacommand{then}\isamarkupfalse%
\ \isacommand{have}\isamarkupfalse%
\ {\isachardoublequoteopen}sat\ {\isacharparenleft}{\isacharbraceleft}H{\isacharcomma}F{\isacharbraceright}\ {\isasymunion}\ Wo{\isacharparenright}{\isachardoublequoteclose}\isanewline
\ \ \ \ \ \ \isacommand{by}\isamarkupfalse%
\ {\isacharparenleft}simp\ only{\isacharcolon}\ sat{\isacharunderscore}def{\isacharparenright}\isanewline
\ \ \ \ \isacommand{thus}\isamarkupfalse%
\ {\isacharquery}thesis\isanewline
\ \ \ \ \ \ \isacommand{by}\isamarkupfalse%
\ {\isacharparenleft}rule\ disjI{\isadigit{2}}{\isacharparenright}\isanewline
\ \ \isacommand{qed}\isamarkupfalse%
\isanewline
\isacommand{qed}\isamarkupfalse%
%
\endisatagproof
{\isafoldproof}%
%
\isadelimproof
\isanewline
%
\endisadelimproof
\isanewline
\isacommand{lemma}\isamarkupfalse%
\ pcp{\isacharunderscore}colecComp{\isacharunderscore}DIS{\isacharunderscore}sat{\isadigit{2}}{\isacharcolon}\isanewline
\ \ \isakeyword{assumes}\ {\isachardoublequoteopen}W\ {\isasymin}\ colecComp{\isachardoublequoteclose}\isanewline
\ \ \ \ \ \ \ \ \ \ {\isachardoublequoteopen}F\ {\isacharequal}\ G\ \isactrlbold {\isasymrightarrow}\ H{\isachardoublequoteclose}\isanewline
\ \ \ \ \ \ \ \ \ \ {\isachardoublequoteopen}F\ {\isasymin}\ W{\isachardoublequoteclose}\isanewline
\ \ \ \ \ \ \ \ \ \ {\isachardoublequoteopen}finite\ Wo{\isachardoublequoteclose}\isanewline
\ \ \ \ \ \ \ \ \ \ {\isachardoublequoteopen}Wo\ {\isasymsubseteq}\ W{\isachardoublequoteclose}\isanewline
\ \ \ \ \ \ \ \ \isakeyword{shows}\ {\isachardoublequoteopen}sat\ {\isacharparenleft}{\isacharbraceleft}\isactrlbold {\isasymnot}\ G{\isacharcomma}F{\isacharbraceright}\ {\isasymunion}\ Wo{\isacharparenright}\ {\isasymor}\ sat\ {\isacharparenleft}{\isacharbraceleft}H{\isacharcomma}F{\isacharbraceright}\ {\isasymunion}\ Wo{\isacharparenright}{\isachardoublequoteclose}\isanewline
%
\isadelimproof
%
\endisadelimproof
%
\isatagproof
\isacommand{proof}\isamarkupfalse%
\ {\isacharminus}\isanewline
\ \ \isacommand{have}\isamarkupfalse%
\ {\isachardoublequoteopen}sat\ {\isacharparenleft}{\isacharbraceleft}F{\isacharbraceright}\ {\isasymunion}\ Wo{\isacharparenright}{\isachardoublequoteclose}\isanewline
\ \ \ \ \isacommand{using}\isamarkupfalse%
\ assms{\isacharparenleft}{\isadigit{1}}{\isacharcomma}{\isadigit{3}}{\isacharcomma}{\isadigit{4}}{\isacharcomma}{\isadigit{5}}{\isacharparenright}\ \isacommand{by}\isamarkupfalse%
\ {\isacharparenleft}rule\ pcp{\isacharunderscore}colecComp{\isacharunderscore}elem{\isacharunderscore}sat{\isacharparenright}\isanewline
\ \ \isacommand{have}\isamarkupfalse%
\ {\isachardoublequoteopen}F\ {\isasymin}\ {\isacharbraceleft}F{\isacharbraceright}\ {\isasymunion}\ Wo{\isachardoublequoteclose}\isanewline
\ \ \ \ \isacommand{by}\isamarkupfalse%
\ simp\isanewline
\ \ \isacommand{have}\isamarkupfalse%
\ Ex{\isadigit{1}}{\isacharcolon}{\isachardoublequoteopen}{\isasymexists}{\isasymA}{\isachardot}\ {\isasymforall}F\ {\isasymin}\ {\isacharparenleft}{\isacharbraceleft}F{\isacharbraceright}\ {\isasymunion}\ Wo{\isacharparenright}{\isachardot}\ {\isasymA}\ {\isasymTurnstile}\ F{\isachardoublequoteclose}\isanewline
\ \ \ \ \isacommand{using}\isamarkupfalse%
\ {\isacartoucheopen}sat\ {\isacharparenleft}{\isacharbraceleft}F{\isacharbraceright}\ {\isasymunion}\ Wo{\isacharparenright}{\isacartoucheclose}\ \isacommand{by}\isamarkupfalse%
\ {\isacharparenleft}simp\ only{\isacharcolon}\ sat{\isacharunderscore}def{\isacharparenright}\isanewline
\ \ \isacommand{obtain}\isamarkupfalse%
\ {\isasymA}\ \isakeyword{where}\ Forall{\isadigit{1}}{\isacharcolon}{\isachardoublequoteopen}{\isasymforall}F\ {\isasymin}\ {\isacharparenleft}{\isacharbraceleft}F{\isacharbraceright}\ {\isasymunion}\ Wo{\isacharparenright}{\isachardot}\ {\isasymA}\ {\isasymTurnstile}\ F{\isachardoublequoteclose}\isanewline
\ \ \ \ \isacommand{using}\isamarkupfalse%
\ Ex{\isadigit{1}}\ \isacommand{by}\isamarkupfalse%
\ {\isacharparenleft}rule\ exE{\isacharparenright}\isanewline
\ \ \isacommand{have}\isamarkupfalse%
\ {\isachardoublequoteopen}{\isasymA}\ {\isasymTurnstile}\ F{\isachardoublequoteclose}\isanewline
\ \ \ \ \isacommand{using}\isamarkupfalse%
\ Forall{\isadigit{1}}\ {\isacartoucheopen}F\ {\isasymin}\ {\isacharbraceleft}F{\isacharbraceright}\ {\isasymunion}\ Wo{\isacartoucheclose}\ \isacommand{by}\isamarkupfalse%
\ {\isacharparenleft}rule\ bspec{\isacharparenright}\isanewline
\ \ \isacommand{then}\isamarkupfalse%
\ \isacommand{have}\isamarkupfalse%
\ {\isachardoublequoteopen}{\isasymA}\ {\isasymTurnstile}\ {\isacharparenleft}G\ \isactrlbold {\isasymrightarrow}\ H{\isacharparenright}{\isachardoublequoteclose}\isanewline
\ \ \ \ \isacommand{using}\isamarkupfalse%
\ assms{\isacharparenleft}{\isadigit{2}}{\isacharparenright}\ \isacommand{by}\isamarkupfalse%
\ {\isacharparenleft}simp\ only{\isacharcolon}\ {\isacartoucheopen}{\isasymA}\ {\isasymTurnstile}\ F{\isacartoucheclose}{\isacharparenright}\isanewline
\ \ \isacommand{then}\isamarkupfalse%
\ \isacommand{have}\isamarkupfalse%
\ {\isachardoublequoteopen}{\isasymA}\ {\isasymTurnstile}\ G\ {\isasymlongrightarrow}\ {\isasymA}\ {\isasymTurnstile}\ H{\isachardoublequoteclose}\isanewline
\ \ \ \ \isacommand{by}\isamarkupfalse%
\ {\isacharparenleft}simp\ only{\isacharcolon}\ formula{\isacharunderscore}semantics{\isachardot}simps{\isacharparenleft}{\isadigit{6}}{\isacharparenright}{\isacharparenright}\isanewline
\ \ \isacommand{then}\isamarkupfalse%
\ \isacommand{have}\isamarkupfalse%
\ {\isachardoublequoteopen}{\isacharparenleft}{\isasymnot}{\isacharparenleft}{\isasymnot}\ {\isasymA}\ {\isasymTurnstile}\ G{\isacharparenright}{\isacharparenright}\ {\isasymlongrightarrow}\ {\isasymA}\ {\isasymTurnstile}\ H{\isachardoublequoteclose}\isanewline
\ \ \ \ \isacommand{by}\isamarkupfalse%
\ {\isacharparenleft}simp\ only{\isacharcolon}\ not{\isacharunderscore}not{\isacharparenright}\isanewline
\ \ \isacommand{then}\isamarkupfalse%
\ \isacommand{have}\isamarkupfalse%
\ {\isachardoublequoteopen}{\isacharparenleft}{\isasymnot}\ {\isasymA}\ {\isasymTurnstile}\ G{\isacharparenright}\ {\isasymor}\ {\isasymA}\ {\isasymTurnstile}\ H{\isachardoublequoteclose}\isanewline
\ \ \ \ \isacommand{by}\isamarkupfalse%
\ {\isacharparenleft}simp\ only{\isacharcolon}\ disj{\isacharunderscore}imp{\isacharparenright}\isanewline
\ \ \isacommand{thus}\isamarkupfalse%
\ {\isacharquery}thesis\isanewline
\ \ \isacommand{proof}\isamarkupfalse%
\ {\isacharparenleft}rule\ disjE{\isacharparenright}\isanewline
\ \ \ \ \isacommand{assume}\isamarkupfalse%
\ {\isachardoublequoteopen}{\isasymnot}\ {\isasymA}\ {\isasymTurnstile}\ G{\isachardoublequoteclose}\isanewline
\ \ \ \ \isacommand{then}\isamarkupfalse%
\ \isacommand{have}\isamarkupfalse%
\ {\isachardoublequoteopen}{\isasymA}\ {\isasymTurnstile}\ {\isacharparenleft}\isactrlbold {\isasymnot}\ G{\isacharparenright}{\isachardoublequoteclose}\isanewline
\ \ \ \ \ \ \isacommand{by}\isamarkupfalse%
\ {\isacharparenleft}simp\ only{\isacharcolon}\ formula{\isacharunderscore}semantics{\isachardot}simps{\isacharparenleft}{\isadigit{3}}{\isacharparenright}\ simp{\isacharunderscore}thms{\isacharparenleft}{\isadigit{8}}{\isacharparenright}{\isacharparenright}\isanewline
\ \ \ \ \isacommand{then}\isamarkupfalse%
\ \isacommand{have}\isamarkupfalse%
\ {\isachardoublequoteopen}{\isasymforall}F\ {\isasymin}\ {\isacharbraceleft}\isactrlbold {\isasymnot}\ G{\isacharbraceright}{\isachardot}\ {\isasymA}\ {\isasymTurnstile}\ F{\isachardoublequoteclose}\isanewline
\ \ \ \ \ \ \isacommand{by}\isamarkupfalse%
\ simp\isanewline
\ \ \ \ \isacommand{then}\isamarkupfalse%
\ \isacommand{have}\isamarkupfalse%
\ {\isachardoublequoteopen}{\isasymforall}F\ {\isasymin}\ {\isacharparenleft}{\isacharbraceleft}\isactrlbold {\isasymnot}\ G{\isacharbraceright}\ {\isasymunion}\ {\isacharparenleft}{\isacharbraceleft}F{\isacharbraceright}\ {\isasymunion}\ Wo{\isacharparenright}{\isacharparenright}{\isachardot}\ {\isasymA}\ {\isasymTurnstile}\ F{\isachardoublequoteclose}\isanewline
\ \ \ \ \ \ \isacommand{using}\isamarkupfalse%
\ Forall{\isadigit{1}}\ \isacommand{by}\isamarkupfalse%
\ {\isacharparenleft}rule\ ball{\isacharunderscore}Un{\isacharparenright}\isanewline
\ \ \ \ \isacommand{then}\isamarkupfalse%
\ \isacommand{have}\isamarkupfalse%
\ {\isachardoublequoteopen}{\isasymforall}F\ {\isasymin}\ {\isacharbraceleft}\isactrlbold {\isasymnot}\ G{\isacharcomma}F{\isacharbraceright}\ {\isasymunion}\ Wo{\isachardot}\ {\isasymA}\ {\isasymTurnstile}\ F{\isachardoublequoteclose}\isanewline
\ \ \ \ \ \ \isacommand{by}\isamarkupfalse%
\ simp\isanewline
\ \ \ \ \isacommand{then}\isamarkupfalse%
\ \isacommand{have}\isamarkupfalse%
\ {\isachardoublequoteopen}{\isasymexists}{\isasymA}{\isachardot}\ {\isasymforall}F\ {\isasymin}\ {\isacharparenleft}{\isacharbraceleft}\isactrlbold {\isasymnot}\ G{\isacharcomma}F{\isacharbraceright}\ {\isasymunion}\ Wo{\isacharparenright}{\isachardot}\ {\isasymA}\ {\isasymTurnstile}\ F{\isachardoublequoteclose}\isanewline
\ \ \ \ \ \ \isacommand{by}\isamarkupfalse%
\ {\isacharparenleft}iprover\ intro{\isacharcolon}\ exI{\isacharparenright}\isanewline
\ \ \ \ \isacommand{then}\isamarkupfalse%
\ \isacommand{have}\isamarkupfalse%
\ {\isachardoublequoteopen}sat\ {\isacharparenleft}{\isacharbraceleft}\isactrlbold {\isasymnot}\ G{\isacharcomma}F{\isacharbraceright}\ {\isasymunion}\ Wo{\isacharparenright}{\isachardoublequoteclose}\isanewline
\ \ \ \ \ \ \isacommand{by}\isamarkupfalse%
\ {\isacharparenleft}simp\ only{\isacharcolon}\ sat{\isacharunderscore}def{\isacharparenright}\isanewline
\ \ \ \ \isacommand{thus}\isamarkupfalse%
\ {\isacharquery}thesis\isanewline
\ \ \ \ \ \ \isacommand{by}\isamarkupfalse%
\ {\isacharparenleft}rule\ disjI{\isadigit{1}}{\isacharparenright}\isanewline
\ \ \isacommand{next}\isamarkupfalse%
\isanewline
\ \ \ \ \isacommand{assume}\isamarkupfalse%
\ {\isachardoublequoteopen}{\isasymA}\ {\isasymTurnstile}\ H{\isachardoublequoteclose}\isanewline
\ \ \ \ \isacommand{then}\isamarkupfalse%
\ \isacommand{have}\isamarkupfalse%
\ {\isachardoublequoteopen}{\isasymforall}F\ {\isasymin}\ {\isacharbraceleft}H{\isacharbraceright}{\isachardot}\ {\isasymA}\ {\isasymTurnstile}\ F{\isachardoublequoteclose}\isanewline
\ \ \ \ \ \ \isacommand{by}\isamarkupfalse%
\ simp\isanewline
\ \ \ \ \isacommand{then}\isamarkupfalse%
\ \isacommand{have}\isamarkupfalse%
\ {\isachardoublequoteopen}{\isasymforall}F\ {\isasymin}\ {\isacharparenleft}{\isacharbraceleft}H{\isacharbraceright}\ {\isasymunion}\ {\isacharparenleft}{\isacharbraceleft}F{\isacharbraceright}\ {\isasymunion}\ Wo{\isacharparenright}{\isacharparenright}{\isachardot}\ {\isasymA}\ {\isasymTurnstile}\ F{\isachardoublequoteclose}\isanewline
\ \ \ \ \ \ \isacommand{using}\isamarkupfalse%
\ Forall{\isadigit{1}}\ \isacommand{by}\isamarkupfalse%
\ {\isacharparenleft}rule\ ball{\isacharunderscore}Un{\isacharparenright}\isanewline
\ \ \ \ \isacommand{then}\isamarkupfalse%
\ \isacommand{have}\isamarkupfalse%
\ {\isachardoublequoteopen}{\isasymforall}F\ {\isasymin}\ {\isacharbraceleft}H{\isacharcomma}F{\isacharbraceright}\ {\isasymunion}\ Wo{\isachardot}\ {\isasymA}\ {\isasymTurnstile}\ F{\isachardoublequoteclose}\isanewline
\ \ \ \ \ \ \isacommand{by}\isamarkupfalse%
\ simp\isanewline
\ \ \ \ \isacommand{then}\isamarkupfalse%
\ \isacommand{have}\isamarkupfalse%
\ {\isachardoublequoteopen}{\isasymexists}{\isasymA}{\isachardot}\ {\isasymforall}F\ {\isasymin}\ {\isacharparenleft}{\isacharbraceleft}H{\isacharcomma}F{\isacharbraceright}\ {\isasymunion}\ Wo{\isacharparenright}{\isachardot}\ {\isasymA}\ {\isasymTurnstile}\ F{\isachardoublequoteclose}\isanewline
\ \ \ \ \ \ \isacommand{by}\isamarkupfalse%
\ {\isacharparenleft}iprover\ intro{\isacharcolon}\ exI{\isacharparenright}\isanewline
\ \ \ \ \isacommand{then}\isamarkupfalse%
\ \isacommand{have}\isamarkupfalse%
\ {\isachardoublequoteopen}sat\ {\isacharparenleft}{\isacharbraceleft}H{\isacharcomma}F{\isacharbraceright}\ {\isasymunion}\ Wo{\isacharparenright}{\isachardoublequoteclose}\isanewline
\ \ \ \ \ \ \isacommand{by}\isamarkupfalse%
\ {\isacharparenleft}simp\ only{\isacharcolon}\ sat{\isacharunderscore}def{\isacharparenright}\isanewline
\ \ \ \ \isacommand{thus}\isamarkupfalse%
\ {\isacharquery}thesis\isanewline
\ \ \ \ \ \ \isacommand{by}\isamarkupfalse%
\ {\isacharparenleft}rule\ disjI{\isadigit{2}}{\isacharparenright}\isanewline
\ \ \isacommand{qed}\isamarkupfalse%
\isanewline
\isacommand{qed}\isamarkupfalse%
%
\endisatagproof
{\isafoldproof}%
%
\isadelimproof
\isanewline
%
\endisadelimproof
\isanewline
\isacommand{lemma}\isamarkupfalse%
\ pcp{\isacharunderscore}colecComp{\isacharunderscore}DIS{\isacharunderscore}sat{\isadigit{3}}{\isacharcolon}\isanewline
\ \ \isakeyword{assumes}\ {\isachardoublequoteopen}W\ {\isasymin}\ colecComp{\isachardoublequoteclose}\isanewline
\ \ \ \ \ \ \ \ \ \ {\isachardoublequoteopen}F\ {\isacharequal}\ \isactrlbold {\isasymnot}\ {\isacharparenleft}G\ \isactrlbold {\isasymand}\ H{\isacharparenright}{\isachardoublequoteclose}\isanewline
\ \ \ \ \ \ \ \ \ \ {\isachardoublequoteopen}F\ {\isasymin}\ W{\isachardoublequoteclose}\isanewline
\ \ \ \ \ \ \ \ \ \ {\isachardoublequoteopen}finite\ Wo{\isachardoublequoteclose}\isanewline
\ \ \ \ \ \ \ \ \ \ {\isachardoublequoteopen}Wo\ {\isasymsubseteq}\ W{\isachardoublequoteclose}\isanewline
\ \ \ \ \ \ \ \ \isakeyword{shows}\ {\isachardoublequoteopen}sat\ {\isacharparenleft}{\isacharbraceleft}\isactrlbold {\isasymnot}\ G{\isacharcomma}F{\isacharbraceright}\ {\isasymunion}\ Wo{\isacharparenright}\ {\isasymor}\ sat\ {\isacharparenleft}{\isacharbraceleft}\isactrlbold {\isasymnot}\ H{\isacharcomma}F{\isacharbraceright}\ {\isasymunion}\ Wo{\isacharparenright}{\isachardoublequoteclose}\isanewline
%
\isadelimproof
%
\endisadelimproof
%
\isatagproof
\isacommand{proof}\isamarkupfalse%
\ {\isacharminus}\isanewline
\ \ \isacommand{have}\isamarkupfalse%
\ {\isachardoublequoteopen}sat\ {\isacharparenleft}{\isacharbraceleft}F{\isacharbraceright}\ {\isasymunion}\ Wo{\isacharparenright}{\isachardoublequoteclose}\isanewline
\ \ \ \ \isacommand{using}\isamarkupfalse%
\ assms{\isacharparenleft}{\isadigit{1}}{\isacharcomma}{\isadigit{3}}{\isacharcomma}{\isadigit{4}}{\isacharcomma}{\isadigit{5}}{\isacharparenright}\ \isacommand{by}\isamarkupfalse%
\ {\isacharparenleft}rule\ pcp{\isacharunderscore}colecComp{\isacharunderscore}elem{\isacharunderscore}sat{\isacharparenright}\isanewline
\ \ \isacommand{have}\isamarkupfalse%
\ {\isachardoublequoteopen}F\ {\isasymin}\ {\isacharbraceleft}F{\isacharbraceright}\ {\isasymunion}\ Wo{\isachardoublequoteclose}\isanewline
\ \ \ \ \isacommand{by}\isamarkupfalse%
\ simp\isanewline
\ \ \isacommand{have}\isamarkupfalse%
\ Ex{\isadigit{1}}{\isacharcolon}{\isachardoublequoteopen}{\isasymexists}{\isasymA}{\isachardot}\ {\isasymforall}F\ {\isasymin}\ {\isacharparenleft}{\isacharbraceleft}F{\isacharbraceright}\ {\isasymunion}\ Wo{\isacharparenright}{\isachardot}\ {\isasymA}\ {\isasymTurnstile}\ F{\isachardoublequoteclose}\isanewline
\ \ \ \ \isacommand{using}\isamarkupfalse%
\ {\isacartoucheopen}sat\ {\isacharparenleft}{\isacharbraceleft}F{\isacharbraceright}\ {\isasymunion}\ Wo{\isacharparenright}{\isacartoucheclose}\ \isacommand{by}\isamarkupfalse%
\ {\isacharparenleft}simp\ only{\isacharcolon}\ sat{\isacharunderscore}def{\isacharparenright}\isanewline
\ \ \isacommand{obtain}\isamarkupfalse%
\ {\isasymA}\ \isakeyword{where}\ Forall{\isadigit{1}}{\isacharcolon}{\isachardoublequoteopen}{\isasymforall}F\ {\isasymin}\ {\isacharparenleft}{\isacharbraceleft}F{\isacharbraceright}\ {\isasymunion}\ Wo{\isacharparenright}{\isachardot}\ {\isasymA}\ {\isasymTurnstile}\ F{\isachardoublequoteclose}\isanewline
\ \ \ \ \isacommand{using}\isamarkupfalse%
\ Ex{\isadigit{1}}\ \isacommand{by}\isamarkupfalse%
\ {\isacharparenleft}rule\ exE{\isacharparenright}\isanewline
\ \ \isacommand{have}\isamarkupfalse%
\ {\isachardoublequoteopen}{\isasymA}\ {\isasymTurnstile}\ F{\isachardoublequoteclose}\isanewline
\ \ \ \ \isacommand{using}\isamarkupfalse%
\ Forall{\isadigit{1}}\ {\isacartoucheopen}F\ {\isasymin}\ {\isacharbraceleft}F{\isacharbraceright}\ {\isasymunion}\ Wo{\isacartoucheclose}\ \isacommand{by}\isamarkupfalse%
\ {\isacharparenleft}rule\ bspec{\isacharparenright}\isanewline
\ \ \isacommand{then}\isamarkupfalse%
\ \isacommand{have}\isamarkupfalse%
\ {\isachardoublequoteopen}{\isasymA}\ {\isasymTurnstile}\ \isactrlbold {\isasymnot}\ {\isacharparenleft}G\ \isactrlbold {\isasymand}\ H{\isacharparenright}{\isachardoublequoteclose}\isanewline
\ \ \ \ \isacommand{using}\isamarkupfalse%
\ assms{\isacharparenleft}{\isadigit{2}}{\isacharparenright}\ \isacommand{by}\isamarkupfalse%
\ {\isacharparenleft}simp\ only{\isacharcolon}\ {\isacartoucheopen}{\isasymA}\ {\isasymTurnstile}\ F{\isacartoucheclose}{\isacharparenright}\isanewline
\ \ \isacommand{then}\isamarkupfalse%
\ \isacommand{have}\isamarkupfalse%
\ {\isachardoublequoteopen}{\isasymnot}\ {\isacharparenleft}{\isasymA}\ {\isasymTurnstile}\ {\isacharparenleft}G\ \isactrlbold {\isasymand}\ H{\isacharparenright}{\isacharparenright}{\isachardoublequoteclose}\isanewline
\ \ \ \ \isacommand{by}\isamarkupfalse%
\ {\isacharparenleft}simp\ only{\isacharcolon}\ formula{\isacharunderscore}semantics{\isachardot}simps{\isacharparenleft}{\isadigit{3}}{\isacharparenright}\ simp{\isacharunderscore}thms{\isacharparenleft}{\isadigit{8}}{\isacharparenright}{\isacharparenright}\isanewline
\ \ \isacommand{then}\isamarkupfalse%
\ \isacommand{have}\isamarkupfalse%
\ {\isachardoublequoteopen}{\isasymnot}{\isacharparenleft}{\isasymA}\ {\isasymTurnstile}\ G\ {\isasymand}\ {\isasymA}\ {\isasymTurnstile}\ H{\isacharparenright}{\isachardoublequoteclose}\isanewline
\ \ \ \ \isacommand{by}\isamarkupfalse%
\ {\isacharparenleft}simp\ only{\isacharcolon}\ formula{\isacharunderscore}semantics{\isachardot}simps{\isacharparenleft}{\isadigit{4}}{\isacharparenright}\ simp{\isacharunderscore}thms{\isacharparenleft}{\isadigit{8}}{\isacharparenright}{\isacharparenright}\isanewline
\ \ \isacommand{then}\isamarkupfalse%
\ \isacommand{have}\isamarkupfalse%
\ {\isachardoublequoteopen}{\isasymnot}\ {\isacharparenleft}{\isasymA}\ {\isasymTurnstile}\ G{\isacharparenright}\ {\isasymor}\ {\isasymnot}\ {\isacharparenleft}{\isasymA}\ {\isasymTurnstile}\ H{\isacharparenright}{\isachardoublequoteclose}\isanewline
\ \ \ \ \isacommand{by}\isamarkupfalse%
\ {\isacharparenleft}simp\ only{\isacharcolon}\ de{\isacharunderscore}Morgan{\isacharunderscore}conj{\isacharparenright}\isanewline
\ \ \isacommand{thus}\isamarkupfalse%
\ {\isacharquery}thesis\isanewline
\ \ \isacommand{proof}\isamarkupfalse%
\ {\isacharparenleft}rule\ disjE{\isacharparenright}\isanewline
\ \ \ \ \isacommand{assume}\isamarkupfalse%
\ {\isachardoublequoteopen}{\isasymnot}\ {\isacharparenleft}{\isasymA}\ {\isasymTurnstile}\ G{\isacharparenright}{\isachardoublequoteclose}\isanewline
\ \ \ \ \isacommand{then}\isamarkupfalse%
\ \isacommand{have}\isamarkupfalse%
\ {\isachardoublequoteopen}{\isasymA}\ {\isasymTurnstile}\ \isactrlbold {\isasymnot}\ G{\isachardoublequoteclose}\isanewline
\ \ \ \ \ \ \isacommand{by}\isamarkupfalse%
\ {\isacharparenleft}simp\ only{\isacharcolon}\ formula{\isacharunderscore}semantics{\isachardot}simps{\isacharparenleft}{\isadigit{3}}{\isacharparenright}\ simp{\isacharunderscore}thms{\isacharparenleft}{\isadigit{8}}{\isacharparenright}{\isacharparenright}\isanewline
\ \ \ \ \isacommand{then}\isamarkupfalse%
\ \isacommand{have}\isamarkupfalse%
\ {\isachardoublequoteopen}{\isasymforall}F\ {\isasymin}\ {\isacharbraceleft}\isactrlbold {\isasymnot}\ G{\isacharbraceright}{\isachardot}\ {\isasymA}\ {\isasymTurnstile}\ F{\isachardoublequoteclose}\isanewline
\ \ \ \ \ \ \isacommand{by}\isamarkupfalse%
\ simp\isanewline
\ \ \ \ \isacommand{then}\isamarkupfalse%
\ \isacommand{have}\isamarkupfalse%
\ {\isachardoublequoteopen}{\isasymforall}F\ {\isasymin}\ {\isacharparenleft}{\isacharbraceleft}\isactrlbold {\isasymnot}\ G{\isacharbraceright}\ {\isasymunion}\ {\isacharparenleft}{\isacharbraceleft}F{\isacharbraceright}\ {\isasymunion}\ Wo{\isacharparenright}{\isacharparenright}{\isachardot}\ {\isasymA}\ {\isasymTurnstile}\ F{\isachardoublequoteclose}\isanewline
\ \ \ \ \ \ \isacommand{using}\isamarkupfalse%
\ Forall{\isadigit{1}}\ \isacommand{by}\isamarkupfalse%
\ {\isacharparenleft}rule\ ball{\isacharunderscore}Un{\isacharparenright}\isanewline
\ \ \ \ \isacommand{then}\isamarkupfalse%
\ \isacommand{have}\isamarkupfalse%
\ {\isachardoublequoteopen}{\isasymforall}F\ {\isasymin}\ {\isacharbraceleft}\isactrlbold {\isasymnot}\ G{\isacharcomma}F{\isacharbraceright}\ {\isasymunion}\ Wo{\isachardot}\ {\isasymA}\ {\isasymTurnstile}\ F{\isachardoublequoteclose}\isanewline
\ \ \ \ \ \ \isacommand{by}\isamarkupfalse%
\ simp\isanewline
\ \ \ \ \isacommand{then}\isamarkupfalse%
\ \isacommand{have}\isamarkupfalse%
\ {\isachardoublequoteopen}{\isasymexists}{\isasymA}{\isachardot}\ {\isasymforall}F\ {\isasymin}\ {\isacharparenleft}{\isacharbraceleft}\isactrlbold {\isasymnot}\ G{\isacharcomma}F{\isacharbraceright}\ {\isasymunion}\ Wo{\isacharparenright}{\isachardot}\ {\isasymA}\ {\isasymTurnstile}\ F{\isachardoublequoteclose}\isanewline
\ \ \ \ \ \ \isacommand{by}\isamarkupfalse%
\ {\isacharparenleft}iprover\ intro{\isacharcolon}\ exI{\isacharparenright}\isanewline
\ \ \ \ \isacommand{then}\isamarkupfalse%
\ \isacommand{have}\isamarkupfalse%
\ {\isachardoublequoteopen}sat\ {\isacharparenleft}{\isacharbraceleft}\isactrlbold {\isasymnot}\ G{\isacharcomma}F{\isacharbraceright}\ {\isasymunion}\ Wo{\isacharparenright}{\isachardoublequoteclose}\isanewline
\ \ \ \ \ \ \isacommand{by}\isamarkupfalse%
\ {\isacharparenleft}simp\ only{\isacharcolon}\ sat{\isacharunderscore}def{\isacharparenright}\isanewline
\ \ \ \ \isacommand{thus}\isamarkupfalse%
\ {\isacharquery}thesis\isanewline
\ \ \ \ \ \ \isacommand{by}\isamarkupfalse%
\ {\isacharparenleft}rule\ disjI{\isadigit{1}}{\isacharparenright}\isanewline
\ \ \isacommand{next}\isamarkupfalse%
\isanewline
\ \ \ \ \isacommand{assume}\isamarkupfalse%
\ {\isachardoublequoteopen}{\isasymnot}\ {\isacharparenleft}{\isasymA}\ {\isasymTurnstile}\ H{\isacharparenright}{\isachardoublequoteclose}\isanewline
\ \ \ \ \isacommand{then}\isamarkupfalse%
\ \isacommand{have}\isamarkupfalse%
\ {\isachardoublequoteopen}{\isasymA}\ {\isasymTurnstile}\ \isactrlbold {\isasymnot}\ H{\isachardoublequoteclose}\isanewline
\ \ \ \ \ \ \isacommand{by}\isamarkupfalse%
\ {\isacharparenleft}simp\ only{\isacharcolon}\ formula{\isacharunderscore}semantics{\isachardot}simps{\isacharparenleft}{\isadigit{3}}{\isacharparenright}\ simp{\isacharunderscore}thms{\isacharparenleft}{\isadigit{8}}{\isacharparenright}{\isacharparenright}\isanewline
\ \ \ \ \isacommand{then}\isamarkupfalse%
\ \isacommand{have}\isamarkupfalse%
\ {\isachardoublequoteopen}{\isasymforall}F\ {\isasymin}\ {\isacharbraceleft}\isactrlbold {\isasymnot}\ H{\isacharbraceright}{\isachardot}\ {\isasymA}\ {\isasymTurnstile}\ F{\isachardoublequoteclose}\isanewline
\ \ \ \ \ \ \isacommand{by}\isamarkupfalse%
\ simp\isanewline
\ \ \ \ \isacommand{then}\isamarkupfalse%
\ \isacommand{have}\isamarkupfalse%
\ {\isachardoublequoteopen}{\isasymforall}F\ {\isasymin}\ {\isacharparenleft}{\isacharbraceleft}\isactrlbold {\isasymnot}\ H{\isacharbraceright}\ {\isasymunion}\ {\isacharparenleft}{\isacharbraceleft}F{\isacharbraceright}\ {\isasymunion}\ Wo{\isacharparenright}{\isacharparenright}{\isachardot}\ {\isasymA}\ {\isasymTurnstile}\ F{\isachardoublequoteclose}\isanewline
\ \ \ \ \ \ \isacommand{using}\isamarkupfalse%
\ Forall{\isadigit{1}}\ \isacommand{by}\isamarkupfalse%
\ {\isacharparenleft}rule\ ball{\isacharunderscore}Un{\isacharparenright}\isanewline
\ \ \ \ \isacommand{then}\isamarkupfalse%
\ \isacommand{have}\isamarkupfalse%
\ {\isachardoublequoteopen}{\isasymforall}F\ {\isasymin}\ {\isacharbraceleft}\isactrlbold {\isasymnot}\ H{\isacharcomma}F{\isacharbraceright}\ {\isasymunion}\ Wo{\isachardot}\ {\isasymA}\ {\isasymTurnstile}\ F{\isachardoublequoteclose}\isanewline
\ \ \ \ \ \ \isacommand{by}\isamarkupfalse%
\ simp\isanewline
\ \ \ \ \isacommand{then}\isamarkupfalse%
\ \isacommand{have}\isamarkupfalse%
\ {\isachardoublequoteopen}{\isasymexists}{\isasymA}{\isachardot}\ {\isasymforall}F\ {\isasymin}\ {\isacharparenleft}{\isacharbraceleft}\isactrlbold {\isasymnot}\ H{\isacharcomma}F{\isacharbraceright}\ {\isasymunion}\ Wo{\isacharparenright}{\isachardot}\ {\isasymA}\ {\isasymTurnstile}\ F{\isachardoublequoteclose}\isanewline
\ \ \ \ \ \ \isacommand{by}\isamarkupfalse%
\ {\isacharparenleft}iprover\ intro{\isacharcolon}\ exI{\isacharparenright}\isanewline
\ \ \ \ \isacommand{then}\isamarkupfalse%
\ \isacommand{have}\isamarkupfalse%
\ {\isachardoublequoteopen}sat\ {\isacharparenleft}{\isacharbraceleft}\isactrlbold {\isasymnot}\ H{\isacharcomma}F{\isacharbraceright}\ {\isasymunion}\ Wo{\isacharparenright}{\isachardoublequoteclose}\isanewline
\ \ \ \ \ \ \isacommand{by}\isamarkupfalse%
\ {\isacharparenleft}simp\ only{\isacharcolon}\ sat{\isacharunderscore}def{\isacharparenright}\isanewline
\ \ \ \ \isacommand{thus}\isamarkupfalse%
\ {\isacharquery}thesis\isanewline
\ \ \ \ \ \ \isacommand{by}\isamarkupfalse%
\ {\isacharparenleft}rule\ disjI{\isadigit{2}}{\isacharparenright}\isanewline
\ \ \isacommand{qed}\isamarkupfalse%
\isanewline
\isacommand{qed}\isamarkupfalse%
%
\endisatagproof
{\isafoldproof}%
%
\isadelimproof
\isanewline
%
\endisadelimproof
\isanewline
\isacommand{lemma}\isamarkupfalse%
\ pcp{\isacharunderscore}colecComp{\isacharunderscore}DIS{\isacharunderscore}sat{\isadigit{4}}{\isacharcolon}\isanewline
\ \ \isakeyword{assumes}\ {\isachardoublequoteopen}W\ {\isasymin}\ colecComp{\isachardoublequoteclose}\isanewline
\ \ \ \ \ \ \ \ \ \ {\isachardoublequoteopen}F\ {\isacharequal}\ \isactrlbold {\isasymnot}\ {\isacharparenleft}\isactrlbold {\isasymnot}\ G{\isacharparenright}{\isachardoublequoteclose}\isanewline
\ \ \ \ \ \ \ \ \ \ {\isachardoublequoteopen}H\ {\isacharequal}\ G{\isachardoublequoteclose}\isanewline
\ \ \ \ \ \ \ \ \ \ {\isachardoublequoteopen}F\ {\isasymin}\ W{\isachardoublequoteclose}\isanewline
\ \ \ \ \ \ \ \ \ \ {\isachardoublequoteopen}finite\ Wo{\isachardoublequoteclose}\isanewline
\ \ \ \ \ \ \ \ \ \ {\isachardoublequoteopen}Wo\ {\isasymsubseteq}\ W{\isachardoublequoteclose}\isanewline
\ \ \ \ \ \ \ \ \isakeyword{shows}\ {\isachardoublequoteopen}sat\ {\isacharparenleft}{\isacharbraceleft}G{\isacharcomma}F{\isacharbraceright}\ {\isasymunion}\ Wo{\isacharparenright}\ {\isasymor}\ sat\ {\isacharparenleft}{\isacharbraceleft}H{\isacharcomma}F{\isacharbraceright}\ {\isasymunion}\ Wo{\isacharparenright}{\isachardoublequoteclose}\isanewline
%
\isadelimproof
%
\endisadelimproof
%
\isatagproof
\isacommand{proof}\isamarkupfalse%
\ {\isacharminus}\isanewline
\ \ \isacommand{have}\isamarkupfalse%
\ {\isachardoublequoteopen}sat\ {\isacharparenleft}{\isacharbraceleft}F{\isacharbraceright}\ {\isasymunion}\ Wo{\isacharparenright}{\isachardoublequoteclose}\isanewline
\ \ \ \ \isacommand{using}\isamarkupfalse%
\ assms{\isacharparenleft}{\isadigit{1}}{\isacharcomma}{\isadigit{4}}{\isacharcomma}{\isadigit{5}}{\isacharcomma}{\isadigit{6}}{\isacharparenright}\ \isacommand{by}\isamarkupfalse%
\ {\isacharparenleft}rule\ pcp{\isacharunderscore}colecComp{\isacharunderscore}elem{\isacharunderscore}sat{\isacharparenright}\isanewline
\ \ \isacommand{have}\isamarkupfalse%
\ {\isachardoublequoteopen}F\ {\isasymin}\ {\isacharbraceleft}F{\isacharbraceright}\ {\isasymunion}\ Wo{\isachardoublequoteclose}\isanewline
\ \ \ \ \isacommand{by}\isamarkupfalse%
\ simp\ \isanewline
\ \ \isacommand{have}\isamarkupfalse%
\ Ex{\isadigit{1}}{\isacharcolon}{\isachardoublequoteopen}{\isasymexists}{\isasymA}{\isachardot}\ {\isasymforall}F\ {\isasymin}\ {\isacharparenleft}{\isacharbraceleft}F{\isacharbraceright}\ {\isasymunion}\ Wo{\isacharparenright}{\isachardot}\ {\isasymA}\ {\isasymTurnstile}\ F{\isachardoublequoteclose}\isanewline
\ \ \ \ \isacommand{using}\isamarkupfalse%
\ {\isacartoucheopen}sat\ {\isacharparenleft}{\isacharbraceleft}F{\isacharbraceright}\ {\isasymunion}\ Wo{\isacharparenright}{\isacartoucheclose}\ \isacommand{by}\isamarkupfalse%
\ {\isacharparenleft}simp\ only{\isacharcolon}\ sat{\isacharunderscore}def{\isacharparenright}\isanewline
\ \ \isacommand{obtain}\isamarkupfalse%
\ {\isasymA}\ \isakeyword{where}\ Forall{\isadigit{1}}{\isacharcolon}{\isachardoublequoteopen}{\isasymforall}F\ {\isasymin}\ {\isacharparenleft}{\isacharbraceleft}F{\isacharbraceright}\ {\isasymunion}\ Wo{\isacharparenright}{\isachardot}\ {\isasymA}\ {\isasymTurnstile}\ F{\isachardoublequoteclose}\isanewline
\ \ \ \ \isacommand{using}\isamarkupfalse%
\ Ex{\isadigit{1}}\ \isacommand{by}\isamarkupfalse%
\ {\isacharparenleft}rule\ exE{\isacharparenright}\isanewline
\ \ \isacommand{have}\isamarkupfalse%
\ {\isachardoublequoteopen}{\isasymA}\ {\isasymTurnstile}\ F{\isachardoublequoteclose}\isanewline
\ \ \ \ \isacommand{using}\isamarkupfalse%
\ Forall{\isadigit{1}}\ {\isacartoucheopen}F\ {\isasymin}\ {\isacharbraceleft}F{\isacharbraceright}\ {\isasymunion}\ Wo{\isacartoucheclose}\ \isacommand{by}\isamarkupfalse%
\ {\isacharparenleft}rule\ bspec{\isacharparenright}\isanewline
\ \ \isacommand{then}\isamarkupfalse%
\ \isacommand{have}\isamarkupfalse%
\ {\isachardoublequoteopen}{\isasymA}\ {\isasymTurnstile}\ \isactrlbold {\isasymnot}{\isacharparenleft}\isactrlbold {\isasymnot}\ G{\isacharparenright}{\isachardoublequoteclose}\isanewline
\ \ \ \ \isacommand{using}\isamarkupfalse%
\ assms{\isacharparenleft}{\isadigit{2}}{\isacharparenright}\ \isacommand{by}\isamarkupfalse%
\ {\isacharparenleft}simp\ only{\isacharcolon}\ {\isacartoucheopen}{\isasymA}\ {\isasymTurnstile}\ F{\isacartoucheclose}{\isacharparenright}\isanewline
\ \ \isacommand{then}\isamarkupfalse%
\ \isacommand{have}\isamarkupfalse%
\ {\isachardoublequoteopen}{\isasymnot}\ {\isasymA}\ {\isasymTurnstile}\ \isactrlbold {\isasymnot}\ G{\isachardoublequoteclose}\isanewline
\ \ \ \ \isacommand{by}\isamarkupfalse%
\ {\isacharparenleft}simp\ only{\isacharcolon}\ formula{\isacharunderscore}semantics{\isachardot}simps{\isacharparenleft}{\isadigit{3}}{\isacharparenright}\ simp{\isacharunderscore}thms{\isacharparenleft}{\isadigit{8}}{\isacharparenright}{\isacharparenright}\isanewline
\ \ \isacommand{then}\isamarkupfalse%
\ \isacommand{have}\isamarkupfalse%
\ {\isachardoublequoteopen}{\isasymnot}\ {\isasymnot}{\isasymA}\ {\isasymTurnstile}\ G{\isachardoublequoteclose}\isanewline
\ \ \ \ \isacommand{by}\isamarkupfalse%
\ {\isacharparenleft}simp\ only{\isacharcolon}\ formula{\isacharunderscore}semantics{\isachardot}simps{\isacharparenleft}{\isadigit{3}}{\isacharparenright}\ simp{\isacharunderscore}thms{\isacharparenleft}{\isadigit{8}}{\isacharparenright}{\isacharparenright}\isanewline
\ \ \isacommand{then}\isamarkupfalse%
\ \isacommand{have}\isamarkupfalse%
\ {\isachardoublequoteopen}{\isasymA}\ {\isasymTurnstile}\ G{\isachardoublequoteclose}\isanewline
\ \ \ \ \isacommand{by}\isamarkupfalse%
\ {\isacharparenleft}rule\ notnotD{\isacharparenright}\isanewline
\ \ \isacommand{then}\isamarkupfalse%
\ \isacommand{have}\isamarkupfalse%
\ {\isachardoublequoteopen}{\isasymforall}F\ {\isasymin}\ {\isacharbraceleft}G{\isacharbraceright}{\isachardot}\ {\isasymA}\ {\isasymTurnstile}\ F{\isachardoublequoteclose}\isanewline
\ \ \ \ \isacommand{by}\isamarkupfalse%
\ simp\isanewline
\ \ \isacommand{then}\isamarkupfalse%
\ \isacommand{have}\isamarkupfalse%
\ {\isachardoublequoteopen}{\isasymforall}F\ {\isasymin}\ {\isacharparenleft}{\isacharbraceleft}G{\isacharbraceright}\ {\isasymunion}\ {\isacharparenleft}{\isacharbraceleft}F{\isacharbraceright}\ {\isasymunion}\ Wo{\isacharparenright}{\isacharparenright}{\isachardot}\ {\isasymA}\ {\isasymTurnstile}\ F{\isachardoublequoteclose}\isanewline
\ \ \ \ \isacommand{using}\isamarkupfalse%
\ Forall{\isadigit{1}}\ \isacommand{by}\isamarkupfalse%
\ {\isacharparenleft}rule\ ball{\isacharunderscore}Un{\isacharparenright}\isanewline
\ \ \isacommand{then}\isamarkupfalse%
\ \isacommand{have}\isamarkupfalse%
\ {\isachardoublequoteopen}{\isasymforall}F\ {\isasymin}\ {\isacharbraceleft}G{\isacharcomma}F{\isacharbraceright}\ {\isasymunion}\ Wo{\isachardot}\ {\isasymA}\ {\isasymTurnstile}\ F{\isachardoublequoteclose}\isanewline
\ \ \ \ \isacommand{by}\isamarkupfalse%
\ simp\isanewline
\ \ \isacommand{then}\isamarkupfalse%
\ \isacommand{have}\isamarkupfalse%
\ {\isachardoublequoteopen}{\isasymexists}{\isasymA}{\isachardot}\ {\isasymforall}F\ {\isasymin}\ {\isacharparenleft}{\isacharbraceleft}G{\isacharcomma}F{\isacharbraceright}\ {\isasymunion}\ Wo{\isacharparenright}{\isachardot}\ {\isasymA}\ {\isasymTurnstile}\ F{\isachardoublequoteclose}\isanewline
\ \ \ \ \isacommand{by}\isamarkupfalse%
\ {\isacharparenleft}iprover\ intro{\isacharcolon}\ exI{\isacharparenright}\isanewline
\ \ \isacommand{then}\isamarkupfalse%
\ \isacommand{have}\isamarkupfalse%
\ {\isachardoublequoteopen}sat\ {\isacharparenleft}{\isacharbraceleft}G{\isacharcomma}F{\isacharbraceright}\ {\isasymunion}\ Wo{\isacharparenright}{\isachardoublequoteclose}\isanewline
\ \ \ \ \isacommand{by}\isamarkupfalse%
\ {\isacharparenleft}simp\ only{\isacharcolon}\ sat{\isacharunderscore}def{\isacharparenright}\isanewline
\ \ \isacommand{thus}\isamarkupfalse%
\ {\isacharquery}thesis\isanewline
\ \ \ \ \isacommand{by}\isamarkupfalse%
\ {\isacharparenleft}rule\ disjI{\isadigit{1}}{\isacharparenright}\isanewline
\isacommand{qed}\isamarkupfalse%
%
\endisatagproof
{\isafoldproof}%
%
\isadelimproof
\isanewline
%
\endisadelimproof
\isanewline
\isacommand{lemma}\isamarkupfalse%
\ pcp{\isacharunderscore}colecComp{\isacharunderscore}DIS{\isacharunderscore}sat{\isacharcolon}\isanewline
\ \ \isakeyword{assumes}\ {\isachardoublequoteopen}W\ {\isasymin}\ colecComp{\isachardoublequoteclose}\isanewline
\ \ \ \ \ \ \ \ \ \ {\isachardoublequoteopen}Dis\ F\ G\ H{\isachardoublequoteclose}\isanewline
\ \ \ \ \ \ \ \ \ \ {\isachardoublequoteopen}F\ {\isasymin}\ W{\isachardoublequoteclose}\isanewline
\ \ \ \ \ \ \ \ \ \ {\isachardoublequoteopen}finite\ Wo{\isachardoublequoteclose}\isanewline
\ \ \ \ \ \ \ \ \ \ {\isachardoublequoteopen}Wo\ {\isasymsubseteq}\ W{\isachardoublequoteclose}\isanewline
\ \ \ \ \ \ \ \ \isakeyword{shows}\ {\isachardoublequoteopen}sat\ {\isacharparenleft}{\isacharbraceleft}G{\isacharbraceright}\ {\isasymunion}\ Wo{\isacharparenright}\ {\isasymor}\ sat\ {\isacharparenleft}{\isacharbraceleft}H{\isacharbraceright}\ {\isasymunion}\ Wo{\isacharparenright}{\isachardoublequoteclose}\isanewline
%
\isadelimproof
%
\endisadelimproof
%
\isatagproof
\isacommand{proof}\isamarkupfalse%
\ {\isacharminus}\isanewline
\ \ \isacommand{have}\isamarkupfalse%
\ {\isachardoublequoteopen}{\isacharparenleft}F\ {\isacharequal}\ G\ \isactrlbold {\isasymor}\ H\ {\isasymor}\ \isanewline
\ \ \ \ \ \ \ \ {\isacharparenleft}{\isasymexists}G{\isadigit{1}}\ H{\isadigit{1}}{\isachardot}\ F\ {\isacharequal}\ G{\isadigit{1}}\ \isactrlbold {\isasymrightarrow}\ H{\isadigit{1}}\ {\isasymand}\ G\ {\isacharequal}\ \isactrlbold {\isasymnot}\ G{\isadigit{1}}\ {\isasymand}\ H\ {\isacharequal}\ H{\isadigit{1}}{\isacharparenright}\ {\isasymor}\ \isanewline
\ \ \ \ \ \ \ \ {\isacharparenleft}{\isasymexists}G{\isadigit{1}}\ H{\isadigit{1}}{\isachardot}\ F\ {\isacharequal}\ \isactrlbold {\isasymnot}\ {\isacharparenleft}G{\isadigit{1}}\ \isactrlbold {\isasymand}\ H{\isadigit{1}}{\isacharparenright}\ {\isasymand}\ G\ {\isacharequal}\ \isactrlbold {\isasymnot}\ G{\isadigit{1}}\ {\isasymand}\ H\ {\isacharequal}\ \isactrlbold {\isasymnot}\ H{\isadigit{1}}{\isacharparenright}\ {\isasymor}\ \isanewline
\ \ \ \ \ \ \ \ F\ {\isacharequal}\ \isactrlbold {\isasymnot}\ {\isacharparenleft}\isactrlbold {\isasymnot}\ G{\isacharparenright}\ {\isasymand}\ H\ {\isacharequal}\ G{\isacharparenright}{\isachardoublequoteclose}\isanewline
\ \ \ \ \isacommand{using}\isamarkupfalse%
\ assms{\isacharparenleft}{\isadigit{2}}{\isacharparenright}\ \isacommand{by}\isamarkupfalse%
\ {\isacharparenleft}simp\ only{\isacharcolon}\ con{\isacharunderscore}dis{\isacharunderscore}simps{\isacharparenleft}{\isadigit{2}}{\isacharparenright}{\isacharparenright}\isanewline
\ \ \isacommand{thus}\isamarkupfalse%
\ {\isacharquery}thesis\isanewline
\ \ \isacommand{proof}\isamarkupfalse%
\ {\isacharparenleft}rule\ disjE{\isacharparenright}\isanewline
\ \ \ \ \isacommand{assume}\isamarkupfalse%
\ {\isachardoublequoteopen}F\ {\isacharequal}\ G\ \isactrlbold {\isasymor}\ H{\isachardoublequoteclose}\isanewline
\ \ \ \ \isacommand{have}\isamarkupfalse%
\ {\isachardoublequoteopen}sat\ {\isacharparenleft}{\isacharbraceleft}G{\isacharcomma}F{\isacharbraceright}\ {\isasymunion}\ Wo{\isacharparenright}\ {\isasymor}\ sat\ {\isacharparenleft}{\isacharbraceleft}H{\isacharcomma}F{\isacharbraceright}\ {\isasymunion}\ Wo{\isacharparenright}{\isachardoublequoteclose}\isanewline
\ \ \ \ \ \ \isacommand{using}\isamarkupfalse%
\ assms{\isacharparenleft}{\isadigit{1}}{\isacharparenright}\ {\isacartoucheopen}F\ {\isacharequal}\ G\ \isactrlbold {\isasymor}\ H{\isacartoucheclose}\ assms{\isacharparenleft}{\isadigit{3}}{\isacharcomma}{\isadigit{4}}{\isacharcomma}{\isadigit{5}}{\isacharparenright}\ \isacommand{by}\isamarkupfalse%
\ {\isacharparenleft}rule\ pcp{\isacharunderscore}colecComp{\isacharunderscore}DIS{\isacharunderscore}sat{\isadigit{1}}{\isacharparenright}\isanewline
\ \ \ \ \isacommand{thus}\isamarkupfalse%
\ {\isacharquery}thesis\isanewline
\ \ \ \ \isacommand{proof}\isamarkupfalse%
\ {\isacharparenleft}rule\ disjE{\isacharparenright}\isanewline
\ \ \ \ \ \ \isacommand{assume}\isamarkupfalse%
\ {\isachardoublequoteopen}sat\ {\isacharparenleft}{\isacharbraceleft}G{\isacharcomma}F{\isacharbraceright}\ {\isasymunion}\ Wo{\isacharparenright}{\isachardoublequoteclose}\isanewline
\ \ \ \ \ \ \isacommand{have}\isamarkupfalse%
\ {\isachardoublequoteopen}{\isacharbraceleft}G{\isacharbraceright}\ {\isasymunion}\ Wo\ {\isasymsubseteq}\ {\isacharbraceleft}G{\isacharcomma}F{\isacharbraceright}\ {\isasymunion}\ Wo{\isachardoublequoteclose}\isanewline
\ \ \ \ \ \ \ \ \isacommand{by}\isamarkupfalse%
\ blast\isanewline
\ \ \ \ \ \ \isacommand{then}\isamarkupfalse%
\ \isacommand{have}\isamarkupfalse%
\ {\isachardoublequoteopen}sat{\isacharparenleft}{\isacharbraceleft}G{\isacharbraceright}\ {\isasymunion}\ Wo{\isacharparenright}{\isachardoublequoteclose}\isanewline
\ \ \ \ \ \ \ \ \isacommand{using}\isamarkupfalse%
\ {\isacartoucheopen}sat{\isacharparenleft}{\isacharbraceleft}G{\isacharcomma}F{\isacharbraceright}\ {\isasymunion}\ Wo{\isacharparenright}{\isacartoucheclose}\ \isacommand{by}\isamarkupfalse%
\ {\isacharparenleft}rule\ sat{\isacharunderscore}mono{\isacharparenright}\isanewline
\ \ \ \ \ \ \isacommand{thus}\isamarkupfalse%
\ {\isacharquery}thesis\isanewline
\ \ \ \ \ \ \ \ \isacommand{by}\isamarkupfalse%
\ {\isacharparenleft}rule\ disjI{\isadigit{1}}{\isacharparenright}\isanewline
\ \ \ \ \isacommand{next}\isamarkupfalse%
\isanewline
\ \ \ \ \ \ \isacommand{assume}\isamarkupfalse%
\ {\isachardoublequoteopen}sat\ {\isacharparenleft}{\isacharbraceleft}H{\isacharcomma}F{\isacharbraceright}\ {\isasymunion}\ Wo{\isacharparenright}{\isachardoublequoteclose}\isanewline
\ \ \ \ \ \ \isacommand{have}\isamarkupfalse%
\ {\isachardoublequoteopen}{\isacharbraceleft}H{\isacharbraceright}\ {\isasymunion}\ Wo\ {\isasymsubseteq}\ {\isacharbraceleft}H{\isacharcomma}F{\isacharbraceright}\ {\isasymunion}\ Wo{\isachardoublequoteclose}\isanewline
\ \ \ \ \ \ \ \ \isacommand{by}\isamarkupfalse%
\ blast\isanewline
\ \ \ \ \ \ \isacommand{then}\isamarkupfalse%
\ \isacommand{have}\isamarkupfalse%
\ {\isachardoublequoteopen}sat{\isacharparenleft}{\isacharbraceleft}H{\isacharbraceright}\ {\isasymunion}\ Wo{\isacharparenright}{\isachardoublequoteclose}\isanewline
\ \ \ \ \ \ \ \ \isacommand{using}\isamarkupfalse%
\ {\isacartoucheopen}sat{\isacharparenleft}{\isacharbraceleft}H{\isacharcomma}F{\isacharbraceright}\ {\isasymunion}\ Wo{\isacharparenright}{\isacartoucheclose}\ \isacommand{by}\isamarkupfalse%
\ {\isacharparenleft}rule\ sat{\isacharunderscore}mono{\isacharparenright}\isanewline
\ \ \ \ \ \ \isacommand{thus}\isamarkupfalse%
\ {\isacharquery}thesis\isanewline
\ \ \ \ \ \ \ \ \isacommand{by}\isamarkupfalse%
\ {\isacharparenleft}rule\ disjI{\isadigit{2}}{\isacharparenright}\isanewline
\ \ \ \ \isacommand{qed}\isamarkupfalse%
\isanewline
\ \ \isacommand{next}\isamarkupfalse%
\isanewline
\ \ \ \ \isacommand{assume}\isamarkupfalse%
\ {\isachardoublequoteopen}{\isacharparenleft}{\isasymexists}G{\isadigit{1}}\ H{\isadigit{1}}{\isachardot}\ F\ {\isacharequal}\ G{\isadigit{1}}\ \isactrlbold {\isasymrightarrow}\ H{\isadigit{1}}\ {\isasymand}\ G\ {\isacharequal}\ \isactrlbold {\isasymnot}\ G{\isadigit{1}}\ {\isasymand}\ H\ {\isacharequal}\ H{\isadigit{1}}{\isacharparenright}\ {\isasymor}\ \isanewline
\ \ \ \ \ \ \ \ {\isacharparenleft}{\isasymexists}G{\isadigit{1}}\ H{\isadigit{1}}{\isachardot}\ F\ {\isacharequal}\ \isactrlbold {\isasymnot}\ {\isacharparenleft}G{\isadigit{1}}\ \isactrlbold {\isasymand}\ H{\isadigit{1}}{\isacharparenright}\ {\isasymand}\ G\ {\isacharequal}\ \isactrlbold {\isasymnot}\ G{\isadigit{1}}\ {\isasymand}\ H\ {\isacharequal}\ \isactrlbold {\isasymnot}\ H{\isadigit{1}}{\isacharparenright}\ {\isasymor}\ \isanewline
\ \ \ \ \ \ \ \ F\ {\isacharequal}\ \isactrlbold {\isasymnot}\ {\isacharparenleft}\isactrlbold {\isasymnot}\ G{\isacharparenright}\ {\isasymand}\ H\ {\isacharequal}\ G{\isachardoublequoteclose}\isanewline
\ \ \ \ \isacommand{thus}\isamarkupfalse%
\ {\isacharquery}thesis\isanewline
\ \ \ \ \isacommand{proof}\isamarkupfalse%
\ {\isacharparenleft}rule\ disjE{\isacharparenright}\isanewline
\ \ \ \ \ \ \isacommand{assume}\isamarkupfalse%
\ Ex{\isadigit{1}}{\isacharcolon}{\isachardoublequoteopen}{\isasymexists}G{\isadigit{1}}\ H{\isadigit{1}}{\isachardot}\ F\ {\isacharequal}\ G{\isadigit{1}}\ \isactrlbold {\isasymrightarrow}\ H{\isadigit{1}}\ {\isasymand}\ G\ {\isacharequal}\ \isactrlbold {\isasymnot}\ G{\isadigit{1}}\ {\isasymand}\ H\ {\isacharequal}\ H{\isadigit{1}}{\isachardoublequoteclose}\isanewline
\ \ \ \ \ \ \isacommand{obtain}\isamarkupfalse%
\ G{\isadigit{1}}\ H{\isadigit{1}}\ \isakeyword{where}\ C{\isadigit{1}}{\isacharcolon}{\isachardoublequoteopen}F\ {\isacharequal}\ G{\isadigit{1}}\ \isactrlbold {\isasymrightarrow}\ H{\isadigit{1}}\ {\isasymand}\ G\ {\isacharequal}\ \isactrlbold {\isasymnot}\ G{\isadigit{1}}\ {\isasymand}\ H\ {\isacharequal}\ H{\isadigit{1}}{\isachardoublequoteclose}\isanewline
\ \ \ \ \ \ \ \ \isacommand{using}\isamarkupfalse%
\ Ex{\isadigit{1}}\ \isacommand{by}\isamarkupfalse%
\ {\isacharparenleft}iprover\ elim{\isacharcolon}\ exE{\isacharparenright}\isanewline
\ \ \ \ \ \ \isacommand{have}\isamarkupfalse%
\ {\isachardoublequoteopen}F\ {\isacharequal}\ G{\isadigit{1}}\ \isactrlbold {\isasymrightarrow}\ H{\isadigit{1}}{\isachardoublequoteclose}\isanewline
\ \ \ \ \ \ \ \ \isacommand{using}\isamarkupfalse%
\ C{\isadigit{1}}\ \isacommand{by}\isamarkupfalse%
\ {\isacharparenleft}rule\ conjunct{\isadigit{1}}{\isacharparenright}\isanewline
\ \ \ \ \ \ \isacommand{have}\isamarkupfalse%
\ {\isachardoublequoteopen}G\ {\isacharequal}\ \isactrlbold {\isasymnot}\ G{\isadigit{1}}{\isachardoublequoteclose}\isanewline
\ \ \ \ \ \ \ \ \isacommand{using}\isamarkupfalse%
\ C{\isadigit{1}}\ \isacommand{by}\isamarkupfalse%
\ {\isacharparenleft}iprover\ elim{\isacharcolon}\ conjunct{\isadigit{1}}{\isacharparenright}\isanewline
\ \ \ \ \ \ \isacommand{have}\isamarkupfalse%
\ {\isachardoublequoteopen}H\ {\isacharequal}\ H{\isadigit{1}}{\isachardoublequoteclose}\isanewline
\ \ \ \ \ \ \ \ \isacommand{using}\isamarkupfalse%
\ C{\isadigit{1}}\ \isacommand{by}\isamarkupfalse%
\ {\isacharparenleft}iprover\ elim{\isacharcolon}\ conjunct{\isadigit{2}}{\isacharparenright}\isanewline
\ \ \ \ \ \ \isacommand{have}\isamarkupfalse%
\ {\isachardoublequoteopen}sat\ {\isacharparenleft}{\isacharbraceleft}\isactrlbold {\isasymnot}\ G{\isadigit{1}}{\isacharcomma}F{\isacharbraceright}\ {\isasymunion}\ Wo{\isacharparenright}\ {\isasymor}\ sat\ {\isacharparenleft}{\isacharbraceleft}H{\isadigit{1}}{\isacharcomma}F{\isacharbraceright}\ {\isasymunion}\ Wo{\isacharparenright}{\isachardoublequoteclose}\isanewline
\ \ \ \ \ \ \ \ \isacommand{using}\isamarkupfalse%
\ assms{\isacharparenleft}{\isadigit{1}}{\isacharparenright}\ {\isacartoucheopen}F\ {\isacharequal}\ G{\isadigit{1}}\ \isactrlbold {\isasymrightarrow}\ H{\isadigit{1}}{\isacartoucheclose}\ assms{\isacharparenleft}{\isadigit{3}}{\isacharcomma}{\isadigit{4}}{\isacharcomma}{\isadigit{5}}{\isacharparenright}\ \isacommand{by}\isamarkupfalse%
\ {\isacharparenleft}rule\ pcp{\isacharunderscore}colecComp{\isacharunderscore}DIS{\isacharunderscore}sat{\isadigit{2}}{\isacharparenright}\isanewline
\ \ \ \ \ \ \isacommand{thus}\isamarkupfalse%
\ {\isacharquery}thesis\isanewline
\ \ \ \ \ \ \isacommand{proof}\isamarkupfalse%
\ {\isacharparenleft}rule\ disjE{\isacharparenright}\isanewline
\ \ \ \ \ \ \ \ \isacommand{assume}\isamarkupfalse%
\ {\isachardoublequoteopen}sat\ {\isacharparenleft}{\isacharbraceleft}\isactrlbold {\isasymnot}\ G{\isadigit{1}}{\isacharcomma}F{\isacharbraceright}\ {\isasymunion}\ Wo{\isacharparenright}{\isachardoublequoteclose}\isanewline
\ \ \ \ \ \ \ \ \isacommand{have}\isamarkupfalse%
\ {\isachardoublequoteopen}{\isacharbraceleft}\isactrlbold {\isasymnot}\ G{\isadigit{1}}{\isacharbraceright}\ {\isasymunion}\ Wo\ {\isasymsubseteq}\ {\isacharbraceleft}\isactrlbold {\isasymnot}\ G{\isadigit{1}}{\isacharcomma}F{\isacharbraceright}\ {\isasymunion}\ Wo{\isachardoublequoteclose}\isanewline
\ \ \ \ \ \ \ \ \ \ \isacommand{by}\isamarkupfalse%
\ blast\isanewline
\ \ \ \ \ \ \ \ \isacommand{then}\isamarkupfalse%
\ \isacommand{have}\isamarkupfalse%
\ {\isachardoublequoteopen}sat{\isacharparenleft}{\isacharbraceleft}\isactrlbold {\isasymnot}\ G{\isadigit{1}}{\isacharbraceright}\ {\isasymunion}\ Wo{\isacharparenright}{\isachardoublequoteclose}\isanewline
\ \ \ \ \ \ \ \ \ \ \isacommand{using}\isamarkupfalse%
\ {\isacartoucheopen}sat{\isacharparenleft}{\isacharbraceleft}\isactrlbold {\isasymnot}\ G{\isadigit{1}}{\isacharcomma}F{\isacharbraceright}\ {\isasymunion}\ Wo{\isacharparenright}{\isacartoucheclose}\ \isacommand{by}\isamarkupfalse%
\ {\isacharparenleft}rule\ sat{\isacharunderscore}mono{\isacharparenright}\isanewline
\ \ \ \ \ \ \ \ \isacommand{then}\isamarkupfalse%
\ \isacommand{have}\isamarkupfalse%
\ {\isachardoublequoteopen}sat{\isacharparenleft}{\isacharbraceleft}G{\isacharbraceright}\ {\isasymunion}\ Wo{\isacharparenright}{\isachardoublequoteclose}\isanewline
\ \ \ \ \ \ \ \ \ \ \isacommand{by}\isamarkupfalse%
\ {\isacharparenleft}simp\ only{\isacharcolon}\ {\isacartoucheopen}G\ {\isacharequal}\ \isactrlbold {\isasymnot}\ G{\isadigit{1}}{\isacartoucheclose}{\isacharparenright}\isanewline
\ \ \ \ \ \ \ \ \isacommand{thus}\isamarkupfalse%
\ {\isacharquery}thesis\isanewline
\ \ \ \ \ \ \ \ \ \ \isacommand{by}\isamarkupfalse%
\ {\isacharparenleft}rule\ disjI{\isadigit{1}}{\isacharparenright}\isanewline
\ \ \ \ \isacommand{next}\isamarkupfalse%
\isanewline
\ \ \ \ \ \ \ \ \isacommand{assume}\isamarkupfalse%
\ {\isachardoublequoteopen}sat\ {\isacharparenleft}{\isacharbraceleft}H{\isadigit{1}}{\isacharcomma}F{\isacharbraceright}\ {\isasymunion}\ Wo{\isacharparenright}{\isachardoublequoteclose}\isanewline
\ \ \ \ \ \ \ \ \isacommand{have}\isamarkupfalse%
\ {\isachardoublequoteopen}{\isacharbraceleft}H{\isadigit{1}}{\isacharbraceright}\ {\isasymunion}\ Wo\ {\isasymsubseteq}\ {\isacharbraceleft}H{\isadigit{1}}{\isacharcomma}F{\isacharbraceright}\ {\isasymunion}\ Wo{\isachardoublequoteclose}\isanewline
\ \ \ \ \ \ \ \ \ \ \isacommand{by}\isamarkupfalse%
\ blast\isanewline
\ \ \ \ \ \ \ \ \isacommand{then}\isamarkupfalse%
\ \isacommand{have}\isamarkupfalse%
\ {\isachardoublequoteopen}sat{\isacharparenleft}{\isacharbraceleft}H{\isadigit{1}}{\isacharbraceright}\ {\isasymunion}\ Wo{\isacharparenright}{\isachardoublequoteclose}\isanewline
\ \ \ \ \ \ \ \ \ \ \isacommand{using}\isamarkupfalse%
\ {\isacartoucheopen}sat{\isacharparenleft}{\isacharbraceleft}H{\isadigit{1}}{\isacharcomma}F{\isacharbraceright}\ {\isasymunion}\ Wo{\isacharparenright}{\isacartoucheclose}\ \isacommand{by}\isamarkupfalse%
\ {\isacharparenleft}rule\ sat{\isacharunderscore}mono{\isacharparenright}\isanewline
\ \ \ \ \ \ \ \ \isacommand{then}\isamarkupfalse%
\ \isacommand{have}\isamarkupfalse%
\ {\isachardoublequoteopen}sat{\isacharparenleft}{\isacharbraceleft}H{\isacharbraceright}\ {\isasymunion}\ Wo{\isacharparenright}{\isachardoublequoteclose}\isanewline
\ \ \ \ \ \ \ \ \ \ \isacommand{by}\isamarkupfalse%
\ {\isacharparenleft}simp\ only{\isacharcolon}\ {\isacartoucheopen}H\ {\isacharequal}\ H{\isadigit{1}}{\isacartoucheclose}{\isacharparenright}\isanewline
\ \ \ \ \ \ \ \ \isacommand{thus}\isamarkupfalse%
\ {\isacharquery}thesis\isanewline
\ \ \ \ \ \ \ \ \ \ \isacommand{by}\isamarkupfalse%
\ {\isacharparenleft}rule\ disjI{\isadigit{2}}{\isacharparenright}\isanewline
\ \ \ \ \ \ \isacommand{qed}\isamarkupfalse%
\isanewline
\ \ \ \ \isacommand{next}\isamarkupfalse%
\isanewline
\ \ \ \ \ \ \isacommand{assume}\isamarkupfalse%
\ {\isachardoublequoteopen}{\isacharparenleft}{\isasymexists}G{\isadigit{1}}\ H{\isadigit{1}}{\isachardot}\ F\ {\isacharequal}\ \isactrlbold {\isasymnot}\ {\isacharparenleft}G{\isadigit{1}}\ \isactrlbold {\isasymand}\ H{\isadigit{1}}{\isacharparenright}\ {\isasymand}\ G\ {\isacharequal}\ \isactrlbold {\isasymnot}\ G{\isadigit{1}}\ {\isasymand}\ H\ {\isacharequal}\ \isactrlbold {\isasymnot}\ H{\isadigit{1}}{\isacharparenright}\ {\isasymor}\ \isanewline
\ \ \ \ \ \ \ \ F\ {\isacharequal}\ \isactrlbold {\isasymnot}\ {\isacharparenleft}\isactrlbold {\isasymnot}\ G{\isacharparenright}\ {\isasymand}\ H\ {\isacharequal}\ G{\isachardoublequoteclose}\isanewline
\ \ \ \ \ \ \isacommand{thus}\isamarkupfalse%
\ {\isacharquery}thesis\isanewline
\ \ \ \ \ \ \isacommand{proof}\isamarkupfalse%
\ {\isacharparenleft}rule\ disjE{\isacharparenright}\isanewline
\ \ \ \ \ \ \ \ \isacommand{assume}\isamarkupfalse%
\ Ex{\isadigit{2}}{\isacharcolon}{\isachardoublequoteopen}{\isasymexists}G{\isadigit{1}}\ H{\isadigit{1}}{\isachardot}\ F\ {\isacharequal}\ \isactrlbold {\isasymnot}\ {\isacharparenleft}G{\isadigit{1}}\ \isactrlbold {\isasymand}\ H{\isadigit{1}}{\isacharparenright}\ {\isasymand}\ G\ {\isacharequal}\ \isactrlbold {\isasymnot}\ G{\isadigit{1}}\ {\isasymand}\ H\ {\isacharequal}\ \isactrlbold {\isasymnot}\ H{\isadigit{1}}{\isachardoublequoteclose}\isanewline
\ \ \ \ \ \ \ \ \isacommand{obtain}\isamarkupfalse%
\ G{\isadigit{1}}\ H{\isadigit{1}}\ \isakeyword{where}\ C{\isadigit{2}}{\isacharcolon}{\isachardoublequoteopen}F\ {\isacharequal}\ \isactrlbold {\isasymnot}\ {\isacharparenleft}G{\isadigit{1}}\ \isactrlbold {\isasymand}\ H{\isadigit{1}}{\isacharparenright}\ {\isasymand}\ G\ {\isacharequal}\ \isactrlbold {\isasymnot}\ G{\isadigit{1}}\ {\isasymand}\ H\ {\isacharequal}\ \isactrlbold {\isasymnot}\ H{\isadigit{1}}{\isachardoublequoteclose}\isanewline
\ \ \ \ \ \ \ \ \ \ \isacommand{using}\isamarkupfalse%
\ Ex{\isadigit{2}}\ \isacommand{by}\isamarkupfalse%
\ {\isacharparenleft}iprover\ elim{\isacharcolon}\ exE{\isacharparenright}\isanewline
\ \ \ \ \ \ \ \ \isacommand{have}\isamarkupfalse%
\ {\isachardoublequoteopen}F\ {\isacharequal}\ \isactrlbold {\isasymnot}\ {\isacharparenleft}G{\isadigit{1}}\ \isactrlbold {\isasymand}\ H{\isadigit{1}}{\isacharparenright}{\isachardoublequoteclose}\isanewline
\ \ \ \ \ \ \ \ \ \ \isacommand{using}\isamarkupfalse%
\ C{\isadigit{2}}\ \isacommand{by}\isamarkupfalse%
\ {\isacharparenleft}rule\ conjunct{\isadigit{1}}{\isacharparenright}\isanewline
\ \ \ \ \ \ \ \ \isacommand{have}\isamarkupfalse%
\ {\isachardoublequoteopen}G\ {\isacharequal}\ \isactrlbold {\isasymnot}\ G{\isadigit{1}}{\isachardoublequoteclose}\isanewline
\ \ \ \ \ \ \ \ \ \ \isacommand{using}\isamarkupfalse%
\ C{\isadigit{2}}\ \isacommand{by}\isamarkupfalse%
\ {\isacharparenleft}iprover\ elim{\isacharcolon}\ conjunct{\isadigit{1}}{\isacharparenright}\isanewline
\ \ \ \ \ \ \ \ \isacommand{have}\isamarkupfalse%
\ {\isachardoublequoteopen}H\ {\isacharequal}\ \isactrlbold {\isasymnot}\ H{\isadigit{1}}{\isachardoublequoteclose}\isanewline
\ \ \ \ \ \ \ \ \ \ \isacommand{using}\isamarkupfalse%
\ C{\isadigit{2}}\ \isacommand{by}\isamarkupfalse%
\ {\isacharparenleft}iprover\ elim{\isacharcolon}\ conjunct{\isadigit{2}}{\isacharparenright}\isanewline
\ \ \ \ \ \ \ \ \isacommand{have}\isamarkupfalse%
\ {\isachardoublequoteopen}sat\ {\isacharparenleft}{\isacharbraceleft}\isactrlbold {\isasymnot}\ G{\isadigit{1}}{\isacharcomma}F{\isacharbraceright}\ {\isasymunion}\ Wo{\isacharparenright}\ {\isasymor}\ sat\ {\isacharparenleft}{\isacharbraceleft}\isactrlbold {\isasymnot}\ H{\isadigit{1}}{\isacharcomma}F{\isacharbraceright}\ {\isasymunion}\ Wo{\isacharparenright}{\isachardoublequoteclose}\isanewline
\ \ \ \ \ \ \ \ \ \ \isacommand{using}\isamarkupfalse%
\ assms{\isacharparenleft}{\isadigit{1}}{\isacharparenright}\ {\isacartoucheopen}F\ {\isacharequal}\ \isactrlbold {\isasymnot}\ {\isacharparenleft}G{\isadigit{1}}\ \isactrlbold {\isasymand}\ H{\isadigit{1}}{\isacharparenright}{\isacartoucheclose}\ assms{\isacharparenleft}{\isadigit{3}}{\isacharcomma}{\isadigit{4}}{\isacharcomma}{\isadigit{5}}{\isacharparenright}\ \isacommand{by}\isamarkupfalse%
\ {\isacharparenleft}rule\ pcp{\isacharunderscore}colecComp{\isacharunderscore}DIS{\isacharunderscore}sat{\isadigit{3}}{\isacharparenright}\isanewline
\ \ \ \ \ \ \ \ \isacommand{thus}\isamarkupfalse%
\ {\isacharquery}thesis\isanewline
\ \ \ \ \ \ \ \ \isacommand{proof}\isamarkupfalse%
\ {\isacharparenleft}rule\ disjE{\isacharparenright}\isanewline
\ \ \ \ \ \ \ \ \ \ \isacommand{assume}\isamarkupfalse%
\ {\isachardoublequoteopen}sat\ {\isacharparenleft}{\isacharbraceleft}\isactrlbold {\isasymnot}\ G{\isadigit{1}}{\isacharcomma}F{\isacharbraceright}\ {\isasymunion}\ Wo{\isacharparenright}{\isachardoublequoteclose}\isanewline
\ \ \ \ \ \ \ \ \ \ \isacommand{have}\isamarkupfalse%
\ {\isachardoublequoteopen}{\isacharbraceleft}\isactrlbold {\isasymnot}\ G{\isadigit{1}}{\isacharbraceright}\ {\isasymunion}\ Wo\ {\isasymsubseteq}\ {\isacharbraceleft}\isactrlbold {\isasymnot}\ G{\isadigit{1}}{\isacharcomma}F{\isacharbraceright}\ {\isasymunion}\ Wo{\isachardoublequoteclose}\isanewline
\ \ \ \ \ \ \ \ \ \ \ \ \isacommand{by}\isamarkupfalse%
\ blast\isanewline
\ \ \ \ \ \ \ \ \ \ \isacommand{then}\isamarkupfalse%
\ \isacommand{have}\isamarkupfalse%
\ {\isachardoublequoteopen}sat{\isacharparenleft}{\isacharbraceleft}\isactrlbold {\isasymnot}\ G{\isadigit{1}}{\isacharbraceright}\ {\isasymunion}\ Wo{\isacharparenright}{\isachardoublequoteclose}\isanewline
\ \ \ \ \ \ \ \ \ \ \ \ \isacommand{using}\isamarkupfalse%
\ {\isacartoucheopen}sat{\isacharparenleft}{\isacharbraceleft}\isactrlbold {\isasymnot}\ G{\isadigit{1}}{\isacharcomma}F{\isacharbraceright}\ {\isasymunion}\ Wo{\isacharparenright}{\isacartoucheclose}\ \isacommand{by}\isamarkupfalse%
\ {\isacharparenleft}rule\ sat{\isacharunderscore}mono{\isacharparenright}\isanewline
\ \ \ \ \ \ \ \ \ \ \isacommand{then}\isamarkupfalse%
\ \isacommand{have}\isamarkupfalse%
\ {\isachardoublequoteopen}sat{\isacharparenleft}{\isacharbraceleft}G{\isacharbraceright}\ {\isasymunion}\ Wo{\isacharparenright}{\isachardoublequoteclose}\isanewline
\ \ \ \ \ \ \ \ \ \ \ \ \isacommand{by}\isamarkupfalse%
\ {\isacharparenleft}simp\ only{\isacharcolon}\ {\isacartoucheopen}G\ {\isacharequal}\ \isactrlbold {\isasymnot}\ G{\isadigit{1}}{\isacartoucheclose}{\isacharparenright}\isanewline
\ \ \ \ \ \ \ \ \ \ \isacommand{thus}\isamarkupfalse%
\ {\isacharquery}thesis\isanewline
\ \ \ \ \ \ \ \ \ \ \ \ \isacommand{by}\isamarkupfalse%
\ {\isacharparenleft}rule\ disjI{\isadigit{1}}{\isacharparenright}\isanewline
\ \ \ \ \ \ \ \ \isacommand{next}\isamarkupfalse%
\isanewline
\ \ \ \ \ \ \ \ \ \ \isacommand{assume}\isamarkupfalse%
\ {\isachardoublequoteopen}sat\ {\isacharparenleft}{\isacharbraceleft}\isactrlbold {\isasymnot}\ H{\isadigit{1}}{\isacharcomma}F{\isacharbraceright}\ {\isasymunion}\ Wo{\isacharparenright}{\isachardoublequoteclose}\isanewline
\ \ \ \ \ \ \ \ \ \ \isacommand{have}\isamarkupfalse%
\ {\isachardoublequoteopen}{\isacharbraceleft}\isactrlbold {\isasymnot}\ H{\isadigit{1}}{\isacharbraceright}\ {\isasymunion}\ Wo\ {\isasymsubseteq}\ {\isacharbraceleft}\isactrlbold {\isasymnot}\ H{\isadigit{1}}{\isacharcomma}F{\isacharbraceright}\ {\isasymunion}\ Wo{\isachardoublequoteclose}\isanewline
\ \ \ \ \ \ \ \ \ \ \ \ \isacommand{by}\isamarkupfalse%
\ blast\isanewline
\ \ \ \ \ \ \ \ \ \ \isacommand{then}\isamarkupfalse%
\ \isacommand{have}\isamarkupfalse%
\ {\isachardoublequoteopen}sat{\isacharparenleft}{\isacharbraceleft}\isactrlbold {\isasymnot}\ H{\isadigit{1}}{\isacharbraceright}\ {\isasymunion}\ Wo{\isacharparenright}{\isachardoublequoteclose}\isanewline
\ \ \ \ \ \ \ \ \ \ \ \ \isacommand{using}\isamarkupfalse%
\ {\isacartoucheopen}sat{\isacharparenleft}{\isacharbraceleft}\isactrlbold {\isasymnot}\ H{\isadigit{1}}{\isacharcomma}F{\isacharbraceright}\ {\isasymunion}\ Wo{\isacharparenright}{\isacartoucheclose}\ \isacommand{by}\isamarkupfalse%
\ {\isacharparenleft}rule\ sat{\isacharunderscore}mono{\isacharparenright}\isanewline
\ \ \ \ \ \ \ \ \ \ \isacommand{then}\isamarkupfalse%
\ \isacommand{have}\isamarkupfalse%
\ {\isachardoublequoteopen}sat{\isacharparenleft}{\isacharbraceleft}H{\isacharbraceright}\ {\isasymunion}\ Wo{\isacharparenright}{\isachardoublequoteclose}\isanewline
\ \ \ \ \ \ \ \ \ \ \ \ \isacommand{by}\isamarkupfalse%
\ {\isacharparenleft}simp\ only{\isacharcolon}\ {\isacartoucheopen}H\ {\isacharequal}\ \isactrlbold {\isasymnot}\ H{\isadigit{1}}{\isacartoucheclose}{\isacharparenright}\isanewline
\ \ \ \ \ \ \ \ \ \ \isacommand{thus}\isamarkupfalse%
\ {\isacharquery}thesis\isanewline
\ \ \ \ \ \ \ \ \ \ \ \ \isacommand{by}\isamarkupfalse%
\ {\isacharparenleft}rule\ disjI{\isadigit{2}}{\isacharparenright}\isanewline
\ \ \ \ \ \ \ \ \isacommand{qed}\isamarkupfalse%
\isanewline
\ \ \ \ \ \ \isacommand{next}\isamarkupfalse%
\isanewline
\ \ \ \ \ \ \ \ \isacommand{assume}\isamarkupfalse%
\ {\isachardoublequoteopen}F\ {\isacharequal}\ \isactrlbold {\isasymnot}\ {\isacharparenleft}\isactrlbold {\isasymnot}\ G{\isacharparenright}\ {\isasymand}\ H\ {\isacharequal}\ G{\isachardoublequoteclose}\isanewline
\ \ \ \ \ \ \ \ \isacommand{then}\isamarkupfalse%
\ \isacommand{have}\isamarkupfalse%
\ {\isachardoublequoteopen}F\ {\isacharequal}\ \isactrlbold {\isasymnot}\ {\isacharparenleft}\isactrlbold {\isasymnot}\ G{\isacharparenright}{\isachardoublequoteclose}\isanewline
\ \ \ \ \ \ \ \ \ \ \isacommand{by}\isamarkupfalse%
\ {\isacharparenleft}rule\ conjunct{\isadigit{1}}{\isacharparenright}\isanewline
\ \ \ \ \ \ \ \ \isacommand{have}\isamarkupfalse%
\ {\isachardoublequoteopen}H\ {\isacharequal}\ G{\isachardoublequoteclose}\isanewline
\ \ \ \ \ \ \ \ \ \ \isacommand{using}\isamarkupfalse%
\ {\isacartoucheopen}F\ {\isacharequal}\ \isactrlbold {\isasymnot}\ {\isacharparenleft}\isactrlbold {\isasymnot}\ G{\isacharparenright}\ {\isasymand}\ H\ {\isacharequal}\ G{\isacartoucheclose}\ \isacommand{by}\isamarkupfalse%
\ {\isacharparenleft}rule\ conjunct{\isadigit{2}}{\isacharparenright}\isanewline
\ \ \ \ \ \ \ \ \isacommand{have}\isamarkupfalse%
\ {\isachardoublequoteopen}sat\ {\isacharparenleft}{\isacharbraceleft}G{\isacharcomma}F{\isacharbraceright}\ {\isasymunion}\ Wo{\isacharparenright}\ {\isasymor}\ sat\ {\isacharparenleft}{\isacharbraceleft}H{\isacharcomma}F{\isacharbraceright}\ {\isasymunion}\ Wo{\isacharparenright}{\isachardoublequoteclose}\isanewline
\ \ \ \ \ \ \ \ \ \ \isacommand{using}\isamarkupfalse%
\ assms{\isacharparenleft}{\isadigit{1}}{\isacharparenright}\ {\isacartoucheopen}F\ {\isacharequal}\ \isactrlbold {\isasymnot}\ {\isacharparenleft}\isactrlbold {\isasymnot}\ G{\isacharparenright}{\isacartoucheclose}\ {\isacartoucheopen}H\ {\isacharequal}\ G{\isacartoucheclose}\ assms{\isacharparenleft}{\isadigit{3}}{\isacharcomma}{\isadigit{4}}{\isacharcomma}{\isadigit{5}}{\isacharparenright}\ \isacommand{by}\isamarkupfalse%
\ {\isacharparenleft}rule\ pcp{\isacharunderscore}colecComp{\isacharunderscore}DIS{\isacharunderscore}sat{\isadigit{4}}{\isacharparenright}\isanewline
\ \ \ \ \ \ \ \ \isacommand{thus}\isamarkupfalse%
\ {\isacharquery}thesis\isanewline
\ \ \ \ \ \ \ \ \isacommand{proof}\isamarkupfalse%
\ {\isacharparenleft}rule\ disjE{\isacharparenright}\isanewline
\ \ \ \ \ \ \ \ \ \ \isacommand{assume}\isamarkupfalse%
\ {\isachardoublequoteopen}sat\ {\isacharparenleft}{\isacharbraceleft}G{\isacharcomma}F{\isacharbraceright}\ {\isasymunion}\ Wo{\isacharparenright}{\isachardoublequoteclose}\isanewline
\ \ \ \ \ \ \ \ \ \ \isacommand{have}\isamarkupfalse%
\ {\isachardoublequoteopen}{\isacharbraceleft}G{\isacharbraceright}\ {\isasymunion}\ Wo\ {\isasymsubseteq}\ {\isacharbraceleft}G{\isacharcomma}F{\isacharbraceright}\ {\isasymunion}\ Wo{\isachardoublequoteclose}\isanewline
\ \ \ \ \ \ \ \ \ \ \ \ \isacommand{by}\isamarkupfalse%
\ blast\isanewline
\ \ \ \ \ \ \ \ \ \ \isacommand{then}\isamarkupfalse%
\ \isacommand{have}\isamarkupfalse%
\ {\isachardoublequoteopen}sat{\isacharparenleft}{\isacharbraceleft}G{\isacharbraceright}\ {\isasymunion}\ Wo{\isacharparenright}{\isachardoublequoteclose}\isanewline
\ \ \ \ \ \ \ \ \ \ \ \ \isacommand{using}\isamarkupfalse%
\ {\isacartoucheopen}sat{\isacharparenleft}{\isacharbraceleft}G{\isacharcomma}F{\isacharbraceright}\ {\isasymunion}\ Wo{\isacharparenright}{\isacartoucheclose}\ \isacommand{by}\isamarkupfalse%
\ {\isacharparenleft}rule\ sat{\isacharunderscore}mono{\isacharparenright}\isanewline
\ \ \ \ \ \ \ \ \ \ \isacommand{thus}\isamarkupfalse%
\ {\isacharquery}thesis\isanewline
\ \ \ \ \ \ \ \ \ \ \ \ \isacommand{by}\isamarkupfalse%
\ {\isacharparenleft}rule\ disjI{\isadigit{1}}{\isacharparenright}\isanewline
\ \ \ \ \ \ \ \ \isacommand{next}\isamarkupfalse%
\isanewline
\ \ \ \ \ \ \ \ \ \ \isacommand{assume}\isamarkupfalse%
\ {\isachardoublequoteopen}sat\ {\isacharparenleft}{\isacharbraceleft}H{\isacharcomma}F{\isacharbraceright}\ {\isasymunion}\ Wo{\isacharparenright}{\isachardoublequoteclose}\isanewline
\ \ \ \ \ \ \ \ \ \ \isacommand{have}\isamarkupfalse%
\ {\isachardoublequoteopen}{\isacharbraceleft}H{\isacharbraceright}\ {\isasymunion}\ Wo\ {\isasymsubseteq}\ {\isacharbraceleft}H{\isacharcomma}F{\isacharbraceright}\ {\isasymunion}\ Wo{\isachardoublequoteclose}\isanewline
\ \ \ \ \ \ \ \ \ \ \ \ \isacommand{by}\isamarkupfalse%
\ blast\isanewline
\ \ \ \ \ \ \ \ \ \ \isacommand{then}\isamarkupfalse%
\ \isacommand{have}\isamarkupfalse%
\ {\isachardoublequoteopen}sat{\isacharparenleft}{\isacharbraceleft}H{\isacharbraceright}\ {\isasymunion}\ Wo{\isacharparenright}{\isachardoublequoteclose}\isanewline
\ \ \ \ \ \ \ \ \ \ \ \ \isacommand{using}\isamarkupfalse%
\ {\isacartoucheopen}sat{\isacharparenleft}{\isacharbraceleft}H{\isacharcomma}F{\isacharbraceright}\ {\isasymunion}\ Wo{\isacharparenright}{\isacartoucheclose}\ \isacommand{by}\isamarkupfalse%
\ {\isacharparenleft}rule\ sat{\isacharunderscore}mono{\isacharparenright}\isanewline
\ \ \ \ \ \ \ \ \ \ \isacommand{thus}\isamarkupfalse%
\ {\isacharquery}thesis\isanewline
\ \ \ \ \ \ \ \ \ \ \ \ \isacommand{by}\isamarkupfalse%
\ {\isacharparenleft}rule\ disjI{\isadigit{2}}{\isacharparenright}\isanewline
\ \ \ \ \ \ \ \ \isacommand{qed}\isamarkupfalse%
\isanewline
\ \ \ \ \ \ \isacommand{qed}\isamarkupfalse%
\isanewline
\ \ \ \ \isacommand{qed}\isamarkupfalse%
\isanewline
\ \ \isacommand{qed}\isamarkupfalse%
\isanewline
\isacommand{qed}\isamarkupfalse%
%
\endisatagproof
{\isafoldproof}%
%
\isadelimproof
\isanewline
%
\endisadelimproof
\isanewline
\isacommand{lemma}\isamarkupfalse%
\ sat{\isacharunderscore}subset{\isacharunderscore}ccontr{\isacharcolon}\isanewline
\ \ \isakeyword{assumes}\ {\isachardoublequoteopen}A\ {\isasymsubseteq}\ B{\isachardoublequoteclose}\isanewline
\ \ \ \ \ \ \ \ \ \ {\isachardoublequoteopen}{\isasymnot}\ sat\ A{\isachardoublequoteclose}\isanewline
\ \ \ \ \ \ \ \ \isakeyword{shows}\ {\isachardoublequoteopen}{\isasymnot}\ sat\ B{\isachardoublequoteclose}\isanewline
%
\isadelimproof
%
\endisadelimproof
%
\isatagproof
\isacommand{proof}\isamarkupfalse%
\ {\isacharminus}\isanewline
\ \ \isacommand{have}\isamarkupfalse%
\ {\isachardoublequoteopen}A\ {\isasymsubseteq}\ B\ {\isasymand}\ sat\ B\ {\isasymlongrightarrow}\ sat\ A{\isachardoublequoteclose}\isanewline
\ \ \ \ \isacommand{using}\isamarkupfalse%
\ sat{\isacharunderscore}mono\ \isacommand{by}\isamarkupfalse%
\ blast\isanewline
\ \ \isacommand{then}\isamarkupfalse%
\ \isacommand{have}\isamarkupfalse%
\ {\isachardoublequoteopen}{\isasymnot}{\isacharparenleft}A\ {\isasymsubseteq}\ B\ {\isasymand}\ sat\ B{\isacharparenright}{\isachardoublequoteclose}\isanewline
\ \ \ \ \isacommand{using}\isamarkupfalse%
\ assms{\isacharparenleft}{\isadigit{2}}{\isacharparenright}\ \isacommand{by}\isamarkupfalse%
\ {\isacharparenleft}rule\ mt{\isacharparenright}\isanewline
\ \ \isacommand{then}\isamarkupfalse%
\ \isacommand{have}\isamarkupfalse%
\ {\isachardoublequoteopen}{\isasymnot}{\isacharparenleft}A\ {\isasymsubseteq}\ B{\isacharparenright}\ {\isasymor}\ {\isasymnot}{\isacharparenleft}sat\ B{\isacharparenright}{\isachardoublequoteclose}\isanewline
\ \ \ \ \isacommand{by}\isamarkupfalse%
\ {\isacharparenleft}simp\ only{\isacharcolon}\ de{\isacharunderscore}Morgan{\isacharunderscore}conj{\isacharparenright}\isanewline
\ \ \isacommand{thus}\isamarkupfalse%
\ {\isacharquery}thesis\isanewline
\ \ \isacommand{proof}\isamarkupfalse%
\ {\isacharparenleft}rule\ disjE{\isacharparenright}\isanewline
\ \ \ \ \isacommand{assume}\isamarkupfalse%
\ {\isachardoublequoteopen}{\isasymnot}{\isacharparenleft}A\ {\isasymsubseteq}\ B{\isacharparenright}{\isachardoublequoteclose}\isanewline
\ \ \ \ \isacommand{thus}\isamarkupfalse%
\ {\isacharquery}thesis\isanewline
\ \ \ \ \ \ \isacommand{using}\isamarkupfalse%
\ assms{\isacharparenleft}{\isadigit{1}}{\isacharparenright}\ \isacommand{by}\isamarkupfalse%
\ {\isacharparenleft}rule\ notE{\isacharparenright}\isanewline
\ \ \isacommand{next}\isamarkupfalse%
\isanewline
\ \ \ \ \isacommand{assume}\isamarkupfalse%
\ {\isachardoublequoteopen}{\isasymnot}{\isacharparenleft}sat\ B{\isacharparenright}{\isachardoublequoteclose}\isanewline
\ \ \ \ \isacommand{thus}\isamarkupfalse%
\ {\isacharquery}thesis\isanewline
\ \ \ \ \ \ \isacommand{by}\isamarkupfalse%
\ this\isanewline
\ \ \isacommand{qed}\isamarkupfalse%
\isanewline
\isacommand{qed}\isamarkupfalse%
%
\endisatagproof
{\isafoldproof}%
%
\isadelimproof
\isanewline
%
\endisadelimproof
\isanewline
\isacommand{lemma}\isamarkupfalse%
\ not{\isacharunderscore}colecComp{\isacharcolon}\isanewline
\ \ \isakeyword{assumes}\ {\isachardoublequoteopen}W\ {\isasymin}\ colecComp{\isachardoublequoteclose}\isanewline
\ \ \ \ \ \ \ \ \ \ {\isachardoublequoteopen}{\isacharbraceleft}x{\isacharbraceright}\ {\isasymunion}\ W\ {\isasymnotin}\ colecComp{\isachardoublequoteclose}\isanewline
\ \ \ \ \ \ \ \ \isakeyword{shows}\ {\isachardoublequoteopen}{\isasymexists}Wo\ {\isasymsubseteq}\ W{\isachardot}\ finite\ Wo\ {\isasymand}\ {\isasymnot}{\isacharparenleft}sat\ {\isacharparenleft}{\isacharbraceleft}x{\isacharbraceright}\ {\isasymunion}\ Wo{\isacharparenright}{\isacharparenright}{\isachardoublequoteclose}\isanewline
%
\isadelimproof
%
\endisadelimproof
%
\isatagproof
\isacommand{proof}\isamarkupfalse%
\ {\isacharminus}\isanewline
\ \ \isacommand{have}\isamarkupfalse%
\ WCol{\isacharcolon}{\isachardoublequoteopen}{\isasymforall}S{\isacharprime}\ {\isasymsubseteq}\ W{\isachardot}\ finite\ S{\isacharprime}\ {\isasymlongrightarrow}\ sat\ S{\isacharprime}{\isachardoublequoteclose}\isanewline
\ \ \ \ \isacommand{using}\isamarkupfalse%
\ assms{\isacharparenleft}{\isadigit{1}}{\isacharparenright}\ \isacommand{unfolding}\isamarkupfalse%
\ colecComp\ fin{\isacharunderscore}sat{\isacharunderscore}def\ \isacommand{by}\isamarkupfalse%
\ {\isacharparenleft}rule\ CollectD{\isacharparenright}\ \isanewline
\ \ \isacommand{have}\isamarkupfalse%
\ {\isachardoublequoteopen}{\isasymnot}{\isacharparenleft}{\isasymforall}Wo\ {\isasymsubseteq}\ {\isacharbraceleft}x{\isacharbraceright}\ {\isasymunion}\ W{\isachardot}\ finite\ Wo\ {\isasymlongrightarrow}\ sat\ Wo{\isacharparenright}{\isachardoublequoteclose}\isanewline
\ \ \ \ \isacommand{using}\isamarkupfalse%
\ assms{\isacharparenleft}{\isadigit{2}}{\isacharparenright}\ \isacommand{unfolding}\isamarkupfalse%
\ colecComp\ fin{\isacharunderscore}sat{\isacharunderscore}def\ \isacommand{by}\isamarkupfalse%
\ {\isacharparenleft}simp\ only{\isacharcolon}\ mem{\isacharunderscore}Collect{\isacharunderscore}eq\ simp{\isacharunderscore}thms{\isacharparenleft}{\isadigit{8}}{\isacharparenright}{\isacharparenright}\isanewline
\ \ \isacommand{then}\isamarkupfalse%
\ \isacommand{have}\isamarkupfalse%
\ {\isachardoublequoteopen}{\isasymexists}Wo\ {\isasymsubseteq}\ {\isacharbraceleft}x{\isacharbraceright}\ {\isasymunion}\ W{\isachardot}\ {\isasymnot}{\isacharparenleft}finite\ Wo\ {\isasymlongrightarrow}\ sat\ Wo{\isacharparenright}{\isachardoublequoteclose}\isanewline
\ \ \ \ \isacommand{by}\isamarkupfalse%
\ {\isacharparenleft}rule\ sall{\isacharunderscore}simps{\isacharunderscore}not{\isacharunderscore}all{\isacharparenright}\isanewline
\ \ \isacommand{then}\isamarkupfalse%
\ \isacommand{have}\isamarkupfalse%
\ Ex{\isadigit{1}}{\isacharcolon}{\isachardoublequoteopen}{\isasymexists}Wo\ {\isasymsubseteq}\ {\isacharbraceleft}x{\isacharbraceright}\ {\isasymunion}\ W{\isachardot}\ finite\ Wo\ {\isasymand}\ {\isasymnot}{\isacharparenleft}sat\ Wo{\isacharparenright}{\isachardoublequoteclose}\isanewline
\ \ \ \ \isacommand{by}\isamarkupfalse%
\ {\isacharparenleft}simp\ only{\isacharcolon}\ not{\isacharunderscore}imp{\isacharparenright}\isanewline
\ \ \isacommand{obtain}\isamarkupfalse%
\ Wo{\isacharprime}\ \isakeyword{where}\ {\isachardoublequoteopen}Wo{\isacharprime}\ {\isasymsubseteq}\ {\isacharbraceleft}x{\isacharbraceright}\ {\isasymunion}\ W{\isachardoublequoteclose}\ \isakeyword{and}\ C{\isadigit{1}}{\isacharcolon}{\isachardoublequoteopen}finite\ Wo{\isacharprime}\ {\isasymand}\ {\isasymnot}{\isacharparenleft}sat\ Wo{\isacharprime}{\isacharparenright}{\isachardoublequoteclose}\isanewline
\ \ \ \ \isacommand{using}\isamarkupfalse%
\ Ex{\isadigit{1}}\ \isacommand{by}\isamarkupfalse%
\ {\isacharparenleft}rule\ subexE{\isacharparenright}\isanewline
\ \ \isacommand{have}\isamarkupfalse%
\ {\isachardoublequoteopen}finite\ Wo{\isacharprime}{\isachardoublequoteclose}\isanewline
\ \ \ \ \isacommand{using}\isamarkupfalse%
\ C{\isadigit{1}}\ \isacommand{by}\isamarkupfalse%
\ {\isacharparenleft}rule\ conjunct{\isadigit{1}}{\isacharparenright}\isanewline
\ \ \isacommand{have}\isamarkupfalse%
\ {\isachardoublequoteopen}{\isasymnot}{\isacharparenleft}sat\ Wo{\isacharprime}{\isacharparenright}{\isachardoublequoteclose}\isanewline
\ \ \ \ \isacommand{using}\isamarkupfalse%
\ C{\isadigit{1}}\ \isacommand{by}\isamarkupfalse%
\ {\isacharparenleft}rule\ conjunct{\isadigit{2}}{\isacharparenright}\isanewline
\ \ \isacommand{have}\isamarkupfalse%
\ Ex{\isadigit{2}}{\isacharcolon}{\isachardoublequoteopen}{\isasymexists}Wo\ {\isasymsubseteq}\ W{\isachardot}\ finite\ Wo\ {\isasymand}\ {\isacharparenleft}Wo{\isacharprime}\ {\isacharequal}\ {\isacharbraceleft}x{\isacharbraceright}\ {\isasymunion}\ Wo\ {\isasymor}\ Wo{\isacharprime}\ {\isacharequal}\ Wo{\isacharparenright}{\isachardoublequoteclose}\isanewline
\ \ \ \ \isacommand{using}\isamarkupfalse%
\ {\isacartoucheopen}finite\ Wo{\isacharprime}{\isacartoucheclose}\ {\isacartoucheopen}Wo{\isacharprime}\ {\isasymsubseteq}\ {\isacharbraceleft}x{\isacharbraceright}\ {\isasymunion}\ W{\isacartoucheclose}\ \isacommand{by}\isamarkupfalse%
\ {\isacharparenleft}rule\ finite{\isacharunderscore}subset{\isacharunderscore}insert{\isadigit{1}}{\isacharparenright}\isanewline
\ \ \isacommand{obtain}\isamarkupfalse%
\ Wo\ \isakeyword{where}\ {\isachardoublequoteopen}Wo\ {\isasymsubseteq}\ W{\isachardoublequoteclose}\ \isakeyword{and}\ C{\isadigit{2}}{\isacharcolon}{\isachardoublequoteopen}finite\ Wo\ {\isasymand}\ {\isacharparenleft}Wo{\isacharprime}\ {\isacharequal}\ {\isacharbraceleft}x{\isacharbraceright}\ {\isasymunion}\ Wo\ {\isasymor}\ Wo{\isacharprime}\ {\isacharequal}\ Wo{\isacharparenright}{\isachardoublequoteclose}\isanewline
\ \ \ \ \isacommand{using}\isamarkupfalse%
\ Ex{\isadigit{2}}\ \isacommand{by}\isamarkupfalse%
\ blast\isanewline
\ \ \isacommand{have}\isamarkupfalse%
\ {\isachardoublequoteopen}finite\ Wo{\isachardoublequoteclose}\isanewline
\ \ \ \ \isacommand{using}\isamarkupfalse%
\ C{\isadigit{2}}\ \isacommand{by}\isamarkupfalse%
\ {\isacharparenleft}rule\ conjunct{\isadigit{1}}{\isacharparenright}\isanewline
\ \ \isacommand{have}\isamarkupfalse%
\ {\isachardoublequoteopen}Wo{\isacharprime}\ {\isacharequal}\ {\isacharbraceleft}x{\isacharbraceright}\ {\isasymunion}\ Wo\ {\isasymor}\ Wo{\isacharprime}\ {\isacharequal}\ Wo{\isachardoublequoteclose}\isanewline
\ \ \ \ \isacommand{using}\isamarkupfalse%
\ C{\isadigit{2}}\ \isacommand{by}\isamarkupfalse%
\ {\isacharparenleft}rule\ conjunct{\isadigit{2}}{\isacharparenright}\isanewline
\ \ \isacommand{thus}\isamarkupfalse%
\ {\isacharquery}thesis\isanewline
\ \ \isacommand{proof}\isamarkupfalse%
\ {\isacharparenleft}rule\ disjE{\isacharparenright}\isanewline
\ \ \ \ \isacommand{assume}\isamarkupfalse%
\ {\isachardoublequoteopen}Wo{\isacharprime}\ {\isacharequal}\ {\isacharbraceleft}x{\isacharbraceright}\ {\isasymunion}\ Wo{\isachardoublequoteclose}\isanewline
\ \ \ \ \isacommand{then}\isamarkupfalse%
\ \isacommand{have}\isamarkupfalse%
\ {\isachardoublequoteopen}{\isasymnot}{\isacharparenleft}sat\ {\isacharparenleft}{\isacharbraceleft}x{\isacharbraceright}\ {\isasymunion}\ Wo{\isacharparenright}{\isacharparenright}{\isachardoublequoteclose}\ \isanewline
\ \ \ \ \ \ \isacommand{using}\isamarkupfalse%
\ {\isacartoucheopen}{\isasymnot}\ sat\ Wo{\isacharprime}{\isacartoucheclose}\ \isacommand{by}\isamarkupfalse%
\ {\isacharparenleft}simp\ only{\isacharcolon}\ {\isacartoucheopen}Wo{\isacharprime}\ {\isacharequal}\ {\isacharbraceleft}x{\isacharbraceright}\ {\isasymunion}\ Wo{\isacartoucheclose}\ simp{\isacharunderscore}thms{\isacharparenleft}{\isadigit{8}}{\isacharparenright}{\isacharparenright}\isanewline
\ \ \ \ \isacommand{have}\isamarkupfalse%
\ {\isachardoublequoteopen}finite\ Wo\ {\isasymand}\ {\isasymnot}{\isacharparenleft}sat\ {\isacharparenleft}{\isacharbraceleft}x{\isacharbraceright}\ {\isasymunion}\ Wo{\isacharparenright}{\isacharparenright}{\isachardoublequoteclose}\isanewline
\ \ \ \ \ \ \isacommand{using}\isamarkupfalse%
\ {\isacartoucheopen}finite\ Wo{\isacartoucheclose}\ {\isacartoucheopen}{\isasymnot}{\isacharparenleft}sat\ {\isacharparenleft}{\isacharbraceleft}x{\isacharbraceright}\ {\isasymunion}\ Wo{\isacharparenright}{\isacharparenright}{\isacartoucheclose}\ \isacommand{by}\isamarkupfalse%
\ {\isacharparenleft}rule\ conjI{\isacharparenright}\isanewline
\ \ \ \ \isacommand{thus}\isamarkupfalse%
\ {\isachardoublequoteopen}{\isasymexists}Wo\ {\isasymsubseteq}\ W{\isachardot}\ finite\ Wo\ {\isasymand}\ {\isasymnot}{\isacharparenleft}sat\ {\isacharparenleft}{\isacharbraceleft}x{\isacharbraceright}\ {\isasymunion}\ Wo{\isacharparenright}{\isacharparenright}{\isachardoublequoteclose}\isanewline
\ \ \ \ \ \ \isacommand{using}\isamarkupfalse%
\ {\isacartoucheopen}Wo\ {\isasymsubseteq}\ W{\isacartoucheclose}\ \isacommand{by}\isamarkupfalse%
\ {\isacharparenleft}rule\ subexI{\isacharparenright}\isanewline
\ \ \isacommand{next}\isamarkupfalse%
\isanewline
\ \ \ \ \isacommand{assume}\isamarkupfalse%
\ {\isachardoublequoteopen}Wo{\isacharprime}\ {\isacharequal}\ Wo{\isachardoublequoteclose}\isanewline
\ \ \ \ \isacommand{then}\isamarkupfalse%
\ \isacommand{have}\isamarkupfalse%
\ {\isachardoublequoteopen}Wo{\isacharprime}\ {\isasymsubseteq}\ W{\isachardoublequoteclose}\isanewline
\ \ \ \ \ \ \isacommand{using}\isamarkupfalse%
\ {\isacartoucheopen}Wo\ {\isasymsubseteq}\ W{\isacartoucheclose}\ \isacommand{by}\isamarkupfalse%
\ {\isacharparenleft}simp\ only{\isacharcolon}\ {\isacartoucheopen}Wo{\isacharprime}\ {\isacharequal}\ Wo{\isacartoucheclose}{\isacharparenright}\isanewline
\ \ \ \ \isacommand{have}\isamarkupfalse%
\ {\isachardoublequoteopen}finite\ Wo{\isacharprime}\ {\isasymlongrightarrow}\ sat\ Wo{\isacharprime}{\isachardoublequoteclose}\isanewline
\ \ \ \ \ \ \isacommand{using}\isamarkupfalse%
\ WCol\ {\isacartoucheopen}Wo{\isacharprime}\ {\isasymsubseteq}\ W{\isacartoucheclose}\ \isacommand{by}\isamarkupfalse%
\ {\isacharparenleft}rule\ sspec{\isacharparenright}\isanewline
\ \ \ \ \isacommand{then}\isamarkupfalse%
\ \isacommand{have}\isamarkupfalse%
\ {\isachardoublequoteopen}sat\ Wo{\isacharprime}{\isachardoublequoteclose}\isanewline
\ \ \ \ \ \ \isacommand{using}\isamarkupfalse%
\ {\isacartoucheopen}finite\ Wo{\isacharprime}{\isacartoucheclose}\ \isacommand{by}\isamarkupfalse%
\ {\isacharparenleft}rule\ mp{\isacharparenright}\isanewline
\ \ \ \ \isacommand{show}\isamarkupfalse%
\ {\isacharquery}thesis\isanewline
\ \ \ \ \ \ \isacommand{using}\isamarkupfalse%
\ {\isacartoucheopen}{\isasymnot}\ sat\ Wo{\isacharprime}{\isacartoucheclose}\ {\isacartoucheopen}sat\ Wo{\isacharprime}{\isacartoucheclose}\ \isacommand{by}\isamarkupfalse%
\ {\isacharparenleft}rule\ notE{\isacharparenright}\isanewline
\ \ \isacommand{qed}\isamarkupfalse%
\isanewline
\isacommand{qed}\isamarkupfalse%
%
\endisatagproof
{\isafoldproof}%
%
\isadelimproof
\isanewline
%
\endisadelimproof
\isanewline
\isacommand{lemma}\isamarkupfalse%
\ pcp{\isacharunderscore}colecComp{\isacharunderscore}DIS{\isacharcolon}\isanewline
\ \ \isakeyword{assumes}\ {\isachardoublequoteopen}W\ {\isasymin}\ colecComp{\isachardoublequoteclose}\isanewline
\ \ \isakeyword{shows}\ {\isachardoublequoteopen}{\isasymforall}F\ G\ H{\isachardot}\ Dis\ F\ G\ H\ {\isasymlongrightarrow}\ F\ {\isasymin}\ W\ {\isasymlongrightarrow}\ {\isacharbraceleft}G{\isacharbraceright}\ {\isasymunion}\ W\ {\isasymin}\ colecComp\ {\isasymor}\ {\isacharbraceleft}H{\isacharbraceright}\ {\isasymunion}\ W\ {\isasymin}\ colecComp{\isachardoublequoteclose}\isanewline
%
\isadelimproof
%
\endisadelimproof
%
\isatagproof
\isacommand{proof}\isamarkupfalse%
\ {\isacharparenleft}rule\ allI{\isacharparenright}{\isacharplus}\isanewline
\ \ \isacommand{fix}\isamarkupfalse%
\ F\ G\ H\isanewline
\ \ \isacommand{show}\isamarkupfalse%
\ {\isachardoublequoteopen}Dis\ F\ G\ H\ {\isasymlongrightarrow}\ F\ {\isasymin}\ W\ {\isasymlongrightarrow}\ {\isacharbraceleft}G{\isacharbraceright}\ {\isasymunion}\ W\ {\isasymin}\ colecComp\ {\isasymor}\ {\isacharbraceleft}H{\isacharbraceright}\ {\isasymunion}\ W\ {\isasymin}\ colecComp{\isachardoublequoteclose}\isanewline
\ \ \isacommand{proof}\isamarkupfalse%
\ {\isacharparenleft}rule\ impI{\isacharparenright}{\isacharplus}\isanewline
\ \ \ \ \isacommand{assume}\isamarkupfalse%
\ {\isachardoublequoteopen}Dis\ F\ G\ H{\isachardoublequoteclose}\isanewline
\ \ \ \ \isacommand{assume}\isamarkupfalse%
\ {\isachardoublequoteopen}F\ {\isasymin}\ W{\isachardoublequoteclose}\isanewline
\ \ \ \ \isacommand{show}\isamarkupfalse%
\ {\isachardoublequoteopen}{\isacharbraceleft}G{\isacharbraceright}\ {\isasymunion}\ W\ {\isasymin}\ colecComp\ {\isasymor}\ {\isacharbraceleft}H{\isacharbraceright}\ {\isasymunion}\ W\ {\isasymin}\ colecComp{\isachardoublequoteclose}\isanewline
\ \ \ \ \isacommand{proof}\isamarkupfalse%
\ {\isacharparenleft}rule\ ccontr{\isacharparenright}\isanewline
\ \ \ \ \ \ \isacommand{assume}\isamarkupfalse%
\ {\isachardoublequoteopen}{\isasymnot}{\isacharparenleft}{\isacharbraceleft}G{\isacharbraceright}\ {\isasymunion}\ W\ {\isasymin}\ colecComp\ {\isasymor}\ {\isacharbraceleft}H{\isacharbraceright}\ {\isasymunion}\ W\ {\isasymin}\ colecComp{\isacharparenright}{\isachardoublequoteclose}\isanewline
\ \ \ \ \ \ \isacommand{then}\isamarkupfalse%
\ \isacommand{have}\isamarkupfalse%
\ C{\isacharcolon}{\isachardoublequoteopen}{\isacharbraceleft}G{\isacharbraceright}\ {\isasymunion}\ W\ {\isasymnotin}\ colecComp\ {\isasymand}\ {\isacharbraceleft}H{\isacharbraceright}\ {\isasymunion}\ W\ {\isasymnotin}\ colecComp{\isachardoublequoteclose}\isanewline
\ \ \ \ \ \ \ \ \isacommand{by}\isamarkupfalse%
\ {\isacharparenleft}simp\ only{\isacharcolon}\ de{\isacharunderscore}Morgan{\isacharunderscore}disj\ simp{\isacharunderscore}thms{\isacharparenleft}{\isadigit{8}}{\isacharparenright}{\isacharparenright}\isanewline
\ \ \ \ \ \ \isacommand{then}\isamarkupfalse%
\ \isacommand{have}\isamarkupfalse%
\ {\isachardoublequoteopen}{\isacharbraceleft}G{\isacharbraceright}\ {\isasymunion}\ W\ {\isasymnotin}\ colecComp{\isachardoublequoteclose}\isanewline
\ \ \ \ \ \ \ \ \isacommand{by}\isamarkupfalse%
\ {\isacharparenleft}rule\ conjunct{\isadigit{1}}{\isacharparenright}\isanewline
\ \ \ \ \ \ \isacommand{have}\isamarkupfalse%
\ Ex{\isadigit{1}}{\isacharcolon}{\isachardoublequoteopen}{\isasymexists}Wo\ {\isasymsubseteq}\ W{\isachardot}\ finite\ Wo\ {\isasymand}\ {\isasymnot}{\isacharparenleft}sat\ {\isacharparenleft}{\isacharbraceleft}G{\isacharbraceright}\ {\isasymunion}\ Wo{\isacharparenright}{\isacharparenright}{\isachardoublequoteclose}\isanewline
\ \ \ \ \ \ \ \ \isacommand{using}\isamarkupfalse%
\ assms\ {\isacartoucheopen}{\isacharbraceleft}G{\isacharbraceright}\ {\isasymunion}\ W\ {\isasymnotin}\ colecComp{\isacartoucheclose}\ \isacommand{by}\isamarkupfalse%
\ {\isacharparenleft}rule\ not{\isacharunderscore}colecComp{\isacharparenright}\isanewline
\ \ \ \ \ \ \isacommand{obtain}\isamarkupfalse%
\ W{\isadigit{1}}\ \isakeyword{where}\ {\isachardoublequoteopen}W{\isadigit{1}}\ {\isasymsubseteq}\ W{\isachardoublequoteclose}\ \isakeyword{and}\ C{\isadigit{1}}{\isacharcolon}{\isachardoublequoteopen}finite\ W{\isadigit{1}}\ {\isasymand}\ {\isasymnot}{\isacharparenleft}sat\ {\isacharparenleft}{\isacharbraceleft}G{\isacharbraceright}\ {\isasymunion}\ W{\isadigit{1}}{\isacharparenright}{\isacharparenright}{\isachardoublequoteclose}\isanewline
\ \ \ \ \ \ \ \ \isacommand{using}\isamarkupfalse%
\ Ex{\isadigit{1}}\ \isacommand{by}\isamarkupfalse%
\ {\isacharparenleft}rule\ subexE{\isacharparenright}\isanewline
\ \ \ \ \ \ \isacommand{have}\isamarkupfalse%
\ {\isachardoublequoteopen}finite\ W{\isadigit{1}}{\isachardoublequoteclose}\isanewline
\ \ \ \ \ \ \ \ \isacommand{using}\isamarkupfalse%
\ C{\isadigit{1}}\ \isacommand{by}\isamarkupfalse%
\ {\isacharparenleft}rule\ conjunct{\isadigit{1}}{\isacharparenright}\isanewline
\ \ \ \ \ \ \isacommand{have}\isamarkupfalse%
\ {\isachardoublequoteopen}{\isasymnot}{\isacharparenleft}sat\ {\isacharparenleft}{\isacharbraceleft}G{\isacharbraceright}\ {\isasymunion}\ W{\isadigit{1}}{\isacharparenright}{\isacharparenright}{\isachardoublequoteclose}\isanewline
\ \ \ \ \ \ \ \ \isacommand{using}\isamarkupfalse%
\ C{\isadigit{1}}\ \isacommand{by}\isamarkupfalse%
\ {\isacharparenleft}rule\ conjunct{\isadigit{2}}{\isacharparenright}\isanewline
\ \ \ \ \ \ \isacommand{have}\isamarkupfalse%
\ {\isachardoublequoteopen}{\isacharbraceleft}H{\isacharbraceright}\ {\isasymunion}\ W\ {\isasymnotin}\ colecComp{\isachardoublequoteclose}\isanewline
\ \ \ \ \ \ \ \ \isacommand{using}\isamarkupfalse%
\ C\ \isacommand{by}\isamarkupfalse%
\ {\isacharparenleft}rule\ conjunct{\isadigit{2}}{\isacharparenright}\ \isanewline
\ \ \ \ \ \ \isacommand{have}\isamarkupfalse%
\ Ex{\isadigit{2}}{\isacharcolon}{\isachardoublequoteopen}{\isasymexists}Wo\ {\isasymsubseteq}\ W{\isachardot}\ finite\ Wo\ {\isasymand}\ {\isasymnot}{\isacharparenleft}sat\ {\isacharparenleft}{\isacharbraceleft}H{\isacharbraceright}\ {\isasymunion}\ Wo{\isacharparenright}{\isacharparenright}{\isachardoublequoteclose}\isanewline
\ \ \ \ \ \ \ \ \isacommand{using}\isamarkupfalse%
\ assms\ {\isacartoucheopen}{\isacharbraceleft}H{\isacharbraceright}\ {\isasymunion}\ W\ {\isasymnotin}\ colecComp{\isacartoucheclose}\ \isacommand{by}\isamarkupfalse%
\ {\isacharparenleft}rule\ not{\isacharunderscore}colecComp{\isacharparenright}\isanewline
\ \ \ \ \ \ \isacommand{obtain}\isamarkupfalse%
\ W{\isadigit{2}}\ \isakeyword{where}\ {\isachardoublequoteopen}W{\isadigit{2}}\ {\isasymsubseteq}\ W{\isachardoublequoteclose}\ \isakeyword{and}\ C{\isadigit{2}}{\isacharcolon}{\isachardoublequoteopen}finite\ W{\isadigit{2}}\ {\isasymand}\ {\isasymnot}{\isacharparenleft}sat\ {\isacharparenleft}{\isacharbraceleft}H{\isacharbraceright}\ {\isasymunion}\ W{\isadigit{2}}{\isacharparenright}{\isacharparenright}{\isachardoublequoteclose}\isanewline
\ \ \ \ \ \ \ \ \isacommand{using}\isamarkupfalse%
\ Ex{\isadigit{2}}\ \isacommand{by}\isamarkupfalse%
\ {\isacharparenleft}rule\ subexE{\isacharparenright}\isanewline
\ \ \ \ \ \ \isacommand{have}\isamarkupfalse%
\ {\isachardoublequoteopen}finite\ W{\isadigit{2}}{\isachardoublequoteclose}\isanewline
\ \ \ \ \ \ \ \ \isacommand{using}\isamarkupfalse%
\ C{\isadigit{2}}\ \isacommand{by}\isamarkupfalse%
\ {\isacharparenleft}rule\ conjunct{\isadigit{1}}{\isacharparenright}\isanewline
\ \ \ \ \ \ \isacommand{have}\isamarkupfalse%
\ {\isachardoublequoteopen}{\isasymnot}{\isacharparenleft}sat\ {\isacharparenleft}{\isacharbraceleft}H{\isacharbraceright}\ {\isasymunion}\ W{\isadigit{2}}{\isacharparenright}{\isacharparenright}{\isachardoublequoteclose}\isanewline
\ \ \ \ \ \ \ \ \isacommand{using}\isamarkupfalse%
\ C{\isadigit{2}}\ \isacommand{by}\isamarkupfalse%
\ {\isacharparenleft}rule\ conjunct{\isadigit{2}}{\isacharparenright}\isanewline
\ \ \ \ \ \ \isacommand{let}\isamarkupfalse%
\ {\isacharquery}Wo\ {\isacharequal}\ {\isachardoublequoteopen}W{\isadigit{1}}\ {\isasymunion}\ W{\isadigit{2}}{\isachardoublequoteclose}\isanewline
\ \ \ \ \ \ \isacommand{have}\isamarkupfalse%
\ {\isachardoublequoteopen}{\isacharquery}Wo\ {\isasymsubseteq}\ W{\isachardoublequoteclose}\isanewline
\ \ \ \ \ \ \ \ \isacommand{using}\isamarkupfalse%
\ {\isacartoucheopen}W{\isadigit{1}}\ {\isasymsubseteq}\ W{\isacartoucheclose}\ {\isacartoucheopen}W{\isadigit{2}}\ {\isasymsubseteq}\ W{\isacartoucheclose}\ \isacommand{by}\isamarkupfalse%
\ {\isacharparenleft}simp\ only{\isacharcolon}\ Un{\isacharunderscore}least{\isacharparenright}\isanewline
\ \ \ \ \ \ \isacommand{have}\isamarkupfalse%
\ {\isachardoublequoteopen}finite\ {\isacharquery}Wo{\isachardoublequoteclose}\isanewline
\ \ \ \ \ \ \ \ \isacommand{using}\isamarkupfalse%
\ {\isacartoucheopen}finite\ W{\isadigit{1}}{\isacartoucheclose}\ {\isacartoucheopen}finite\ W{\isadigit{2}}{\isacartoucheclose}\ \isacommand{by}\isamarkupfalse%
\ {\isacharparenleft}simp\ only{\isacharcolon}\ finite{\isacharunderscore}Un{\isacharparenright}\isanewline
\ \ \ \ \ \ \isacommand{have}\isamarkupfalse%
\ {\isachardoublequoteopen}{\isacharbraceleft}G{\isacharbraceright}\ {\isasymunion}\ W{\isadigit{1}}\ {\isasymsubseteq}\ {\isacharparenleft}{\isacharbraceleft}G{\isacharbraceright}\ {\isasymunion}\ W{\isadigit{1}}{\isacharparenright}\ {\isasymunion}\ W{\isadigit{2}}{\isachardoublequoteclose}\isanewline
\ \ \ \ \ \ \ \ \isacommand{by}\isamarkupfalse%
\ {\isacharparenleft}simp\ only{\isacharcolon}\ Un{\isacharunderscore}upper{\isadigit{1}}{\isacharparenright}\isanewline
\ \ \ \ \ \ \isacommand{then}\isamarkupfalse%
\ \isacommand{have}\isamarkupfalse%
\ {\isachardoublequoteopen}{\isacharbraceleft}G{\isacharbraceright}\ {\isasymunion}\ W{\isadigit{1}}\ {\isasymsubseteq}\ {\isacharbraceleft}G{\isacharbraceright}\ {\isasymunion}\ {\isacharquery}Wo{\isachardoublequoteclose}\isanewline
\ \ \ \ \ \ \ \ \isacommand{by}\isamarkupfalse%
\ {\isacharparenleft}simp\ only{\isacharcolon}\ Un{\isacharunderscore}assoc{\isacharparenright}\isanewline
\ \ \ \ \ \ \isacommand{have}\isamarkupfalse%
\ {\isachardoublequoteopen}{\isasymnot}\ sat\ {\isacharparenleft}{\isacharbraceleft}G{\isacharbraceright}\ {\isasymunion}\ {\isacharquery}Wo{\isacharparenright}{\isachardoublequoteclose}\isanewline
\ \ \ \ \ \ \ \ \isacommand{using}\isamarkupfalse%
\ {\isacartoucheopen}{\isacharbraceleft}G{\isacharbraceright}\ {\isasymunion}\ W{\isadigit{1}}\ {\isasymsubseteq}\ {\isacharbraceleft}G{\isacharbraceright}\ {\isasymunion}\ {\isacharquery}Wo{\isacartoucheclose}\ {\isacartoucheopen}{\isasymnot}\ sat\ {\isacharparenleft}{\isacharbraceleft}G{\isacharbraceright}\ {\isasymunion}\ W{\isadigit{1}}{\isacharparenright}{\isacartoucheclose}\ \isacommand{by}\isamarkupfalse%
\ {\isacharparenleft}rule\ sat{\isacharunderscore}subset{\isacharunderscore}ccontr{\isacharparenright}\isanewline
\ \ \ \ \ \ \isacommand{have}\isamarkupfalse%
\ {\isachardoublequoteopen}{\isacharbraceleft}H{\isacharbraceright}\ {\isasymunion}\ W{\isadigit{2}}\ {\isasymsubseteq}\ {\isacharparenleft}{\isacharbraceleft}H{\isacharbraceright}\ {\isasymunion}\ W{\isadigit{2}}{\isacharparenright}\ {\isasymunion}\ W{\isadigit{1}}{\isachardoublequoteclose}\isanewline
\ \ \ \ \ \ \ \ \isacommand{by}\isamarkupfalse%
\ {\isacharparenleft}simp\ only{\isacharcolon}\ Un{\isacharunderscore}upper{\isadigit{1}}{\isacharparenright}\isanewline
\ \ \ \ \ \ \isacommand{then}\isamarkupfalse%
\ \isacommand{have}\isamarkupfalse%
\ {\isachardoublequoteopen}{\isacharbraceleft}H{\isacharbraceright}\ {\isasymunion}\ W{\isadigit{2}}\ {\isasymsubseteq}\ {\isacharbraceleft}H{\isacharbraceright}\ {\isasymunion}\ {\isacharparenleft}W{\isadigit{2}}\ {\isasymunion}\ W{\isadigit{1}}{\isacharparenright}{\isachardoublequoteclose}\isanewline
\ \ \ \ \ \ \ \ \isacommand{by}\isamarkupfalse%
\ {\isacharparenleft}simp\ only{\isacharcolon}\ Un{\isacharunderscore}assoc{\isacharparenright}\ \isanewline
\ \ \ \ \ \ \isacommand{then}\isamarkupfalse%
\ \isacommand{have}\isamarkupfalse%
\ {\isachardoublequoteopen}{\isacharbraceleft}H{\isacharbraceright}\ {\isasymunion}\ W{\isadigit{2}}\ {\isasymsubseteq}\ {\isacharbraceleft}H{\isacharbraceright}\ {\isasymunion}\ {\isacharquery}Wo{\isachardoublequoteclose}\isanewline
\ \ \ \ \ \ \ \ \isacommand{by}\isamarkupfalse%
\ {\isacharparenleft}simp\ only{\isacharcolon}\ Un{\isacharunderscore}commute{\isacharparenright}\isanewline
\ \ \ \ \ \ \isacommand{have}\isamarkupfalse%
\ {\isachardoublequoteopen}{\isasymnot}\ sat\ {\isacharparenleft}{\isacharbraceleft}H{\isacharbraceright}\ {\isasymunion}\ {\isacharquery}Wo{\isacharparenright}{\isachardoublequoteclose}\isanewline
\ \ \ \ \ \ \ \ \isacommand{using}\isamarkupfalse%
\ {\isacartoucheopen}{\isacharbraceleft}H{\isacharbraceright}\ {\isasymunion}\ W{\isadigit{2}}\ {\isasymsubseteq}\ {\isacharbraceleft}H{\isacharbraceright}\ {\isasymunion}\ {\isacharquery}Wo{\isacartoucheclose}\ {\isacartoucheopen}{\isasymnot}\ sat\ {\isacharparenleft}{\isacharbraceleft}H{\isacharbraceright}\ {\isasymunion}\ W{\isadigit{2}}{\isacharparenright}{\isacartoucheclose}\ \isacommand{by}\isamarkupfalse%
\ {\isacharparenleft}rule\ sat{\isacharunderscore}subset{\isacharunderscore}ccontr{\isacharparenright}\isanewline
\ \ \ \ \ \ \isacommand{have}\isamarkupfalse%
\ {\isachardoublequoteopen}{\isasymnot}\ sat\ {\isacharparenleft}{\isacharbraceleft}G{\isacharbraceright}\ {\isasymunion}\ {\isacharquery}Wo{\isacharparenright}\ {\isasymand}\ {\isasymnot}\ sat\ {\isacharparenleft}{\isacharbraceleft}H{\isacharbraceright}\ {\isasymunion}\ {\isacharquery}Wo{\isacharparenright}{\isachardoublequoteclose}\isanewline
\ \ \ \ \ \ \ \ \isacommand{using}\isamarkupfalse%
\ {\isacartoucheopen}{\isasymnot}\ sat\ {\isacharparenleft}{\isacharbraceleft}G{\isacharbraceright}\ {\isasymunion}\ {\isacharquery}Wo{\isacharparenright}{\isacartoucheclose}\ {\isacartoucheopen}{\isasymnot}\ sat\ {\isacharparenleft}{\isacharbraceleft}H{\isacharbraceright}\ {\isasymunion}\ {\isacharquery}Wo{\isacharparenright}{\isacartoucheclose}\ \isacommand{by}\isamarkupfalse%
\ {\isacharparenleft}rule\ conjI{\isacharparenright}\isanewline
\ \ \ \ \ \ \isacommand{have}\isamarkupfalse%
\ {\isachardoublequoteopen}sat\ {\isacharparenleft}{\isacharbraceleft}G{\isacharbraceright}\ {\isasymunion}\ {\isacharquery}Wo{\isacharparenright}\ {\isasymor}\ sat\ {\isacharparenleft}{\isacharbraceleft}H{\isacharbraceright}\ {\isasymunion}\ {\isacharquery}Wo{\isacharparenright}{\isachardoublequoteclose}\isanewline
\ \ \ \ \ \ \ \ \isacommand{using}\isamarkupfalse%
\ assms{\isacharparenleft}{\isadigit{1}}{\isacharparenright}\ {\isacartoucheopen}Dis\ F\ G\ H{\isacartoucheclose}\ {\isacartoucheopen}F\ {\isasymin}\ W{\isacartoucheclose}\ {\isacartoucheopen}finite\ {\isacharquery}Wo{\isacartoucheclose}\ {\isacartoucheopen}{\isacharquery}Wo\ {\isasymsubseteq}\ W{\isacartoucheclose}\ \isacommand{by}\isamarkupfalse%
\ {\isacharparenleft}rule\ pcp{\isacharunderscore}colecComp{\isacharunderscore}DIS{\isacharunderscore}sat{\isacharparenright}\isanewline
\ \ \ \ \ \ \isacommand{then}\isamarkupfalse%
\ \isacommand{have}\isamarkupfalse%
\ {\isachardoublequoteopen}{\isasymnot}{\isasymnot}{\isacharparenleft}sat\ {\isacharparenleft}{\isacharbraceleft}G{\isacharbraceright}\ {\isasymunion}\ {\isacharquery}Wo{\isacharparenright}\ {\isasymor}\ sat\ {\isacharparenleft}{\isacharbraceleft}H{\isacharbraceright}\ {\isasymunion}\ {\isacharquery}Wo{\isacharparenright}{\isacharparenright}{\isachardoublequoteclose}\isanewline
\ \ \ \ \ \ \ \ \isacommand{by}\isamarkupfalse%
\ {\isacharparenleft}simp\ only{\isacharcolon}\ not{\isacharunderscore}not{\isacharparenright}\isanewline
\ \ \ \ \ \ \isacommand{then}\isamarkupfalse%
\ \isacommand{have}\isamarkupfalse%
\ {\isachardoublequoteopen}{\isasymnot}{\isacharparenleft}{\isasymnot}{\isacharparenleft}sat\ {\isacharparenleft}{\isacharbraceleft}G{\isacharbraceright}\ {\isasymunion}\ {\isacharquery}Wo{\isacharparenright}{\isacharparenright}\ {\isasymand}\ {\isasymnot}{\isacharparenleft}sat\ {\isacharparenleft}{\isacharbraceleft}H{\isacharbraceright}\ {\isasymunion}\ {\isacharquery}Wo{\isacharparenright}{\isacharparenright}{\isacharparenright}{\isachardoublequoteclose}\isanewline
\ \ \ \ \ \ \ \ \isacommand{by}\isamarkupfalse%
\ {\isacharparenleft}simp\ only{\isacharcolon}\ de{\isacharunderscore}Morgan{\isacharunderscore}disj\ simp{\isacharunderscore}thms{\isacharparenleft}{\isadigit{8}}{\isacharparenright}{\isacharparenright}\isanewline
\ \ \ \ \ \ \isacommand{thus}\isamarkupfalse%
\ {\isachardoublequoteopen}False{\isachardoublequoteclose}\isanewline
\ \ \ \ \ \ \ \ \isacommand{using}\isamarkupfalse%
\ {\isacartoucheopen}{\isasymnot}{\isacharparenleft}sat\ {\isacharparenleft}{\isacharbraceleft}G{\isacharbraceright}\ {\isasymunion}\ {\isacharquery}Wo{\isacharparenright}{\isacharparenright}\ {\isasymand}\ {\isasymnot}{\isacharparenleft}sat\ {\isacharparenleft}{\isacharbraceleft}H{\isacharbraceright}\ {\isasymunion}\ {\isacharquery}Wo{\isacharparenright}{\isacharparenright}{\isacartoucheclose}\ \isacommand{by}\isamarkupfalse%
\ {\isacharparenleft}rule\ notE{\isacharparenright}\isanewline
\ \ \ \ \isacommand{qed}\isamarkupfalse%
\isanewline
\ \ \isacommand{qed}\isamarkupfalse%
\isanewline
\isacommand{qed}\isamarkupfalse%
%
\endisatagproof
{\isafoldproof}%
%
\isadelimproof
\isanewline
%
\endisadelimproof
\isanewline
\isacommand{lemma}\isamarkupfalse%
\ pcp{\isacharunderscore}colecComp{\isacharcolon}\ {\isachardoublequoteopen}pcp\ colecComp{\isachardoublequoteclose}\isanewline
%
\isadelimproof
%
\endisadelimproof
%
\isatagproof
\isacommand{proof}\isamarkupfalse%
\ {\isacharparenleft}rule\ pcp{\isacharunderscore}alt{\isadigit{2}}{\isacharparenright}\isanewline
\ \ \isacommand{show}\isamarkupfalse%
\ {\isachardoublequoteopen}{\isasymforall}W\ {\isasymin}\ colecComp{\isachardot}\ {\isasymbottom}\ {\isasymnotin}\ W\isanewline
\ \ \ \ \ \ \ \ {\isasymand}\ {\isacharparenleft}{\isasymforall}k{\isachardot}\ Atom\ k\ {\isasymin}\ W\ {\isasymlongrightarrow}\ \isactrlbold {\isasymnot}\ {\isacharparenleft}Atom\ k{\isacharparenright}\ {\isasymin}\ W\ {\isasymlongrightarrow}\ False{\isacharparenright}\isanewline
\ \ \ \ \ \ \ \ {\isasymand}\ {\isacharparenleft}{\isasymforall}F\ G\ H{\isachardot}\ Con\ F\ G\ H\ {\isasymlongrightarrow}\ F\ {\isasymin}\ W\ {\isasymlongrightarrow}\ {\isacharbraceleft}G{\isacharcomma}H{\isacharbraceright}\ {\isasymunion}\ W\ {\isasymin}\ colecComp{\isacharparenright}\isanewline
\ \ \ \ \ \ \ \ {\isasymand}\ {\isacharparenleft}{\isasymforall}F\ G\ H{\isachardot}\ Dis\ F\ G\ H\ {\isasymlongrightarrow}\ F\ {\isasymin}\ W\ {\isasymlongrightarrow}\ {\isacharbraceleft}G{\isacharbraceright}\ {\isasymunion}\ W\ {\isasymin}\ colecComp\ {\isasymor}\ {\isacharbraceleft}H{\isacharbraceright}\ {\isasymunion}\ W\ {\isasymin}\ colecComp{\isacharparenright}{\isachardoublequoteclose}\isanewline
\ \ \isacommand{proof}\isamarkupfalse%
\ {\isacharparenleft}rule\ ballI{\isacharparenright}\isanewline
\ \ \ \ \isacommand{fix}\isamarkupfalse%
\ W\isanewline
\ \ \ \ \isacommand{assume}\isamarkupfalse%
\ H{\isacharcolon}{\isachardoublequoteopen}W\ {\isasymin}\ colecComp{\isachardoublequoteclose}\isanewline
\ \ \ \ \isacommand{have}\isamarkupfalse%
\ C{\isadigit{1}}{\isacharcolon}{\isachardoublequoteopen}{\isasymbottom}\ {\isasymnotin}\ W{\isachardoublequoteclose}\isanewline
\ \ \ \ \ \ \isacommand{using}\isamarkupfalse%
\ H\ \isacommand{by}\isamarkupfalse%
\ {\isacharparenleft}rule\ pcp{\isacharunderscore}colecComp{\isacharunderscore}bot{\isacharparenright}\isanewline
\ \ \ \ \isacommand{have}\isamarkupfalse%
\ C{\isadigit{2}}{\isacharcolon}{\isachardoublequoteopen}{\isasymforall}k{\isachardot}\ Atom\ k\ {\isasymin}\ W\ {\isasymlongrightarrow}\ \isactrlbold {\isasymnot}\ {\isacharparenleft}Atom\ k{\isacharparenright}\ {\isasymin}\ W\ {\isasymlongrightarrow}\ False{\isachardoublequoteclose}\isanewline
\ \ \ \ \ \ \isacommand{using}\isamarkupfalse%
\ H\ \isacommand{by}\isamarkupfalse%
\ {\isacharparenleft}rule\ pcp{\isacharunderscore}colecComp{\isacharunderscore}atoms{\isacharparenright}\isanewline
\ \ \ \ \isacommand{have}\isamarkupfalse%
\ C{\isadigit{3}}{\isacharcolon}{\isachardoublequoteopen}{\isasymforall}F\ G\ H{\isachardot}\ Con\ F\ G\ H\ {\isasymlongrightarrow}\ F\ {\isasymin}\ W\ {\isasymlongrightarrow}\ {\isacharbraceleft}G{\isacharcomma}H{\isacharbraceright}\ {\isasymunion}\ W\ {\isasymin}\ colecComp{\isachardoublequoteclose}\isanewline
\ \ \ \ \ \ \isacommand{using}\isamarkupfalse%
\ H\ \isacommand{by}\isamarkupfalse%
\ {\isacharparenleft}rule\ pcp{\isacharunderscore}colecComp{\isacharunderscore}CON{\isacharparenright}\isanewline
\ \ \ \ \isacommand{have}\isamarkupfalse%
\ C{\isadigit{4}}{\isacharcolon}{\isachardoublequoteopen}{\isasymforall}F\ G\ H{\isachardot}\ Dis\ F\ G\ H\ {\isasymlongrightarrow}\ F\ {\isasymin}\ W\ {\isasymlongrightarrow}\ {\isacharbraceleft}G{\isacharbraceright}\ {\isasymunion}\ W\ {\isasymin}\ colecComp\ {\isasymor}\ {\isacharbraceleft}H{\isacharbraceright}\ {\isasymunion}\ W\ {\isasymin}\ colecComp{\isachardoublequoteclose}\isanewline
\ \ \ \ \ \ \isacommand{using}\isamarkupfalse%
\ H\ \isacommand{by}\isamarkupfalse%
\ {\isacharparenleft}rule\ pcp{\isacharunderscore}colecComp{\isacharunderscore}DIS{\isacharparenright}\isanewline
\ \ \ \ \isacommand{show}\isamarkupfalse%
\ {\isachardoublequoteopen}{\isasymbottom}\ {\isasymnotin}\ W\isanewline
\ \ \ \ \ \ \ \ \ \ {\isasymand}\ {\isacharparenleft}{\isasymforall}k{\isachardot}\ Atom\ k\ {\isasymin}\ W\ {\isasymlongrightarrow}\ \isactrlbold {\isasymnot}\ {\isacharparenleft}Atom\ k{\isacharparenright}\ {\isasymin}\ W\ {\isasymlongrightarrow}\ False{\isacharparenright}\isanewline
\ \ \ \ \ \ \ \ \ \ {\isasymand}\ {\isacharparenleft}{\isasymforall}F\ G\ H{\isachardot}\ Con\ F\ G\ H\ {\isasymlongrightarrow}\ F\ {\isasymin}\ W\ {\isasymlongrightarrow}\ {\isacharbraceleft}G{\isacharcomma}H{\isacharbraceright}\ {\isasymunion}\ W\ {\isasymin}\ colecComp{\isacharparenright}\isanewline
\ \ \ \ \ \ \ \ \ \ {\isasymand}\ {\isacharparenleft}{\isasymforall}F\ G\ H{\isachardot}\ Dis\ F\ G\ H\ {\isasymlongrightarrow}\ F\ {\isasymin}\ W\ {\isasymlongrightarrow}\ {\isacharbraceleft}G{\isacharbraceright}\ {\isasymunion}\ W\ {\isasymin}\ colecComp\ {\isasymor}\ {\isacharbraceleft}H{\isacharbraceright}\ {\isasymunion}\ W\ {\isasymin}\ colecComp{\isacharparenright}{\isachardoublequoteclose}\isanewline
\ \ \ \ \ \ \isacommand{using}\isamarkupfalse%
\ C{\isadigit{1}}\ C{\isadigit{2}}\ C{\isadigit{3}}\ C{\isadigit{4}}\ \isacommand{by}\isamarkupfalse%
\ {\isacharparenleft}iprover\ intro{\isacharcolon}\ conjI{\isacharparenright}\isanewline
\ \ \isacommand{qed}\isamarkupfalse%
\isanewline
\isacommand{qed}\isamarkupfalse%
%
\endisatagproof
{\isafoldproof}%
%
\isadelimproof
\isanewline
%
\endisadelimproof
\isanewline
\isacommand{lemma}\isamarkupfalse%
\ prop{\isacharunderscore}Compactness{\isacharcolon}\isanewline
\ \ \isakeyword{fixes}\ S\ {\isacharcolon}{\isacharcolon}\ {\isachardoublequoteopen}{\isacharprime}a\ {\isacharcolon}{\isacharcolon}\ countable\ formula\ set{\isachardoublequoteclose}\isanewline
\ \ \isakeyword{assumes}\ {\isachardoublequoteopen}fin{\isacharunderscore}sat\ S{\isachardoublequoteclose}\isanewline
\ \ \isakeyword{shows}\ {\isachardoublequoteopen}sat\ S{\isachardoublequoteclose}\isanewline
%
\isadelimproof
%
\endisadelimproof
%
\isatagproof
\isacommand{proof}\isamarkupfalse%
\ {\isacharminus}\isanewline
\ \ \isacommand{have}\isamarkupfalse%
\ {\isachardoublequoteopen}pcp\ colecComp{\isachardoublequoteclose}\isanewline
\ \ \ \ \isacommand{by}\isamarkupfalse%
\ {\isacharparenleft}rule\ pcp{\isacharunderscore}colecComp{\isacharparenright}\isanewline
\ \ \isacommand{have}\isamarkupfalse%
\ {\isachardoublequoteopen}S\ {\isasymin}\ colecComp{\isachardoublequoteclose}\isanewline
\ \ \ \ \isacommand{using}\isamarkupfalse%
\ assms\ \isacommand{by}\isamarkupfalse%
\ {\isacharparenleft}rule\ set{\isacharunderscore}in{\isacharunderscore}colecComp{\isacharparenright}\isanewline
\ \ \isacommand{thus}\isamarkupfalse%
\ {\isachardoublequoteopen}sat\ S{\isachardoublequoteclose}\ \isanewline
\ \ \ \ \isacommand{using}\isamarkupfalse%
\ {\isacartoucheopen}pcp\ colecComp{\isacartoucheclose}\ pcp{\isacharunderscore}sat\ \isacommand{by}\isamarkupfalse%
\ blast\ \isanewline
\isacommand{qed}\isamarkupfalse%
\isanewline
%
\endisatagproof
{\isafoldproof}%
%
\isadelimproof
%
\endisadelimproof
%
\isadelimtheory
%
\endisadelimtheory
%
\isatagtheory
%
\endisatagtheory
{\isafoldtheory}%
%
\isadelimtheory
%
\endisadelimtheory
%
\end{isabellebody}%
\endinput
%:%file=~/TFM/TFM/TeoremaEx.thy%:%
%:%19=13%:%
%:%23=15%:%
%:%32=17%:%
%:%44=19%:%
%:%45=20%:%
%:%46=21%:%
%:%47=22%:%
%:%48=23%:%
%:%49=24%:%
%:%50=25%:%
%:%51=26%:%
%:%52=27%:%
%:%53=28%:%
%:%54=29%:%
%:%55=30%:%
%:%56=31%:%
%:%57=32%:%
%:%58=33%:%
%:%59=34%:%
%:%59=35%:%
%:%60=36%:%
%:%61=37%:%
%:%62=38%:%
%:%63=39%:%
%:%64=40%:%
%:%65=41%:%
%:%66=42%:%
%:%67=43%:%
%:%68=44%:%
%:%69=45%:%
%:%70=46%:%
%:%71=47%:%
%:%72=48%:%
%:%73=49%:%
%:%74=50%:%
%:%75=51%:%
%:%76=52%:%
%:%77=53%:%
%:%78=54%:%
%:%79=55%:%
%:%80=56%:%
%:%82=58%:%
%:%83=58%:%
%:%84=59%:%
%:%85=60%:%
%:%88=63%:%
%:%89=64%:%
%:%90=65%:%
%:%91=66%:%
%:%92=67%:%
%:%93=68%:%
%:%94=69%:%
%:%95=70%:%
%:%96=71%:%
%:%97=72%:%
%:%98=73%:%
%:%99=74%:%
%:%100=75%:%
%:%101=76%:%
%:%102=77%:%
%:%103=78%:%
%:%104=79%:%
%:%105=80%:%
%:%106=81%:%
%:%107=82%:%
%:%108=83%:%
%:%109=84%:%
%:%110=85%:%
%:%112=87%:%
%:%113=87%:%
%:%114=88%:%
%:%115=89%:%
%:%116=90%:%
%:%123=91%:%
%:%124=91%:%
%:%125=92%:%
%:%126=92%:%
%:%127=93%:%
%:%128=93%:%
%:%129=94%:%
%:%130=94%:%
%:%131=95%:%
%:%132=95%:%
%:%133=96%:%
%:%134=96%:%
%:%135=97%:%
%:%136=97%:%
%:%137=98%:%
%:%138=99%:%
%:%139=99%:%
%:%140=100%:%
%:%141=100%:%
%:%142=100%:%
%:%143=101%:%
%:%144=102%:%
%:%145=102%:%
%:%146=103%:%
%:%147=103%:%
%:%148=104%:%
%:%149=104%:%
%:%150=105%:%
%:%151=105%:%
%:%152=106%:%
%:%153=106%:%
%:%154=107%:%
%:%155=107%:%
%:%156=107%:%
%:%157=108%:%
%:%158=108%:%
%:%159=109%:%
%:%160=109%:%
%:%161=110%:%
%:%162=110%:%
%:%163=111%:%
%:%164=111%:%
%:%165=112%:%
%:%166=112%:%
%:%167=113%:%
%:%168=113%:%
%:%169=113%:%
%:%170=114%:%
%:%171=114%:%
%:%172=115%:%
%:%173=115%:%
%:%174=116%:%
%:%175=116%:%
%:%176=117%:%
%:%186=119%:%
%:%188=121%:%
%:%189=121%:%
%:%196=122%:%
%:%197=122%:%
%:%198=123%:%
%:%199=123%:%
%:%200=124%:%
%:%201=124%:%
%:%202=124%:%
%:%203=125%:%
%:%204=125%:%
%:%205=125%:%
%:%206=126%:%
%:%207=126%:%
%:%216=128%:%
%:%217=129%:%
%:%218=130%:%
%:%219=131%:%
%:%220=132%:%
%:%221=133%:%
%:%222=134%:%
%:%223=135%:%
%:%225=137%:%
%:%226=137%:%
%:%229=138%:%
%:%233=138%:%
%:%243=140%:%
%:%244=141%:%
%:%245=142%:%
%:%246=143%:%
%:%247=144%:%
%:%248=145%:%
%:%249=146%:%
%:%250=147%:%
%:%251=148%:%
%:%252=149%:%
%:%253=150%:%
%:%254=151%:%
%:%255=152%:%
%:%257=154%:%
%:%258=154%:%
%:%265=155%:%
%:%266=155%:%
%:%267=156%:%
%:%268=156%:%
%:%269=157%:%
%:%270=158%:%
%:%271=158%:%
%:%272=159%:%
%:%273=159%:%
%:%274=159%:%
%:%275=160%:%
%:%276=161%:%
%:%277=161%:%
%:%278=162%:%
%:%279=162%:%
%:%280=163%:%
%:%281=163%:%
%:%282=164%:%
%:%283=164%:%
%:%284=165%:%
%:%285=165%:%
%:%286=166%:%
%:%287=166%:%
%:%288=166%:%
%:%289=167%:%
%:%290=167%:%
%:%291=168%:%
%:%292=168%:%
%:%293=169%:%
%:%294=169%:%
%:%295=170%:%
%:%296=170%:%
%:%297=171%:%
%:%298=171%:%
%:%299=172%:%
%:%300=172%:%
%:%301=172%:%
%:%302=173%:%
%:%303=173%:%
%:%304=174%:%
%:%305=174%:%
%:%306=175%:%
%:%307=175%:%
%:%308=176%:%
%:%318=178%:%
%:%320=180%:%
%:%321=180%:%
%:%324=181%:%
%:%328=181%:%
%:%329=181%:%
%:%338=183%:%
%:%339=184%:%
%:%341=186%:%
%:%342=186%:%
%:%343=187%:%
%:%344=188%:%
%:%347=189%:%
%:%351=189%:%
%:%352=189%:%
%:%353=189%:%
%:%362=191%:%
%:%363=192%:%
%:%364=193%:%
%:%365=194%:%
%:%366=195%:%
%:%367=196%:%
%:%368=197%:%
%:%369=198%:%
%:%370=199%:%
%:%371=200%:%
%:%372=201%:%
%:%373=202%:%
%:%374=203%:%
%:%375=204%:%
%:%376=205%:%
%:%377=206%:%
%:%378=207%:%
%:%379=208%:%
%:%380=209%:%
%:%381=210%:%
%:%382=211%:%
%:%383=212%:%
%:%384=213%:%
%:%385=214%:%
%:%386=215%:%
%:%387=216%:%
%:%388=217%:%
%:%389=218%:%
%:%390=219%:%
%:%391=220%:%
%:%392=221%:%
%:%393=222%:%
%:%394=223%:%
%:%395=224%:%
%:%396=225%:%
%:%397=226%:%
%:%398=227%:%
%:%399=228%:%
%:%400=229%:%
%:%401=230%:%
%:%402=231%:%
%:%403=232%:%
%:%404=233%:%
%:%406=235%:%
%:%407=235%:%
%:%414=236%:%
%:%415=236%:%
%:%416=237%:%
%:%417=237%:%
%:%418=238%:%
%:%419=238%:%
%:%420=239%:%
%:%421=239%:%
%:%422=239%:%
%:%423=240%:%
%:%424=240%:%
%:%425=241%:%
%:%426=241%:%
%:%427=241%:%
%:%428=242%:%
%:%429=242%:%
%:%430=243%:%
%:%431=243%:%
%:%432=243%:%
%:%433=244%:%
%:%434=244%:%
%:%435=245%:%
%:%436=245%:%
%:%437=245%:%
%:%438=246%:%
%:%439=246%:%
%:%440=247%:%
%:%441=247%:%
%:%442=248%:%
%:%443=248%:%
%:%444=249%:%
%:%445=249%:%
%:%446=250%:%
%:%447=250%:%
%:%448=251%:%
%:%449=251%:%
%:%450=252%:%
%:%451=252%:%
%:%452=252%:%
%:%453=253%:%
%:%454=253%:%
%:%455=254%:%
%:%456=254%:%
%:%457=255%:%
%:%458=255%:%
%:%459=256%:%
%:%460=256%:%
%:%461=256%:%
%:%462=257%:%
%:%463=257%:%
%:%464=258%:%
%:%465=258%:%
%:%466=258%:%
%:%467=259%:%
%:%468=259%:%
%:%469=260%:%
%:%470=260%:%
%:%471=260%:%
%:%472=261%:%
%:%473=261%:%
%:%474=262%:%
%:%475=262%:%
%:%476=262%:%
%:%477=263%:%
%:%478=263%:%
%:%479=264%:%
%:%480=264%:%
%:%481=264%:%
%:%482=265%:%
%:%483=265%:%
%:%484=266%:%
%:%485=266%:%
%:%486=266%:%
%:%487=267%:%
%:%488=267%:%
%:%489=267%:%
%:%490=268%:%
%:%491=268%:%
%:%492=268%:%
%:%493=269%:%
%:%494=269%:%
%:%495=270%:%
%:%505=272%:%
%:%507=274%:%
%:%508=274%:%
%:%515=275%:%
%:%516=275%:%
%:%517=276%:%
%:%518=276%:%
%:%519=277%:%
%:%520=277%:%
%:%521=278%:%
%:%522=278%:%
%:%523=278%:%
%:%524=279%:%
%:%525=279%:%
%:%526=280%:%
%:%527=280%:%
%:%528=281%:%
%:%529=281%:%
%:%530=282%:%
%:%531=282%:%
%:%532=283%:%
%:%533=283%:%
%:%534=283%:%
%:%535=283%:%
%:%536=284%:%
%:%537=284%:%
%:%546=286%:%
%:%547=287%:%
%:%548=288%:%
%:%549=289%:%
%:%550=290%:%
%:%551=291%:%
%:%552=292%:%
%:%553=293%:%
%:%554=294%:%
%:%556=296%:%
%:%557=296%:%
%:%559=298%:%
%:%560=299%:%
%:%561=300%:%
%:%562=301%:%
%:%563=302%:%
%:%564=303%:%
%:%565=304%:%
%:%566=305%:%
%:%567=306%:%
%:%568=307%:%
%:%569=308%:%
%:%570=309%:%
%:%571=310%:%
%:%572=311%:%
%:%573=312%:%
%:%574=313%:%
%:%576=315%:%
%:%577=315%:%
%:%580=316%:%
%:%584=316%:%
%:%585=316%:%
%:%586=317%:%
%:%587=317%:%
%:%588=318%:%
%:%589=318%:%
%:%590=319%:%
%:%591=319%:%
%:%592=320%:%
%:%593=320%:%
%:%594=320%:%
%:%595=321%:%
%:%596=321%:%
%:%597=322%:%
%:%598=322%:%
%:%599=322%:%
%:%600=323%:%
%:%601=323%:%
%:%602=324%:%
%:%603=324%:%
%:%604=325%:%
%:%605=325%:%
%:%606=326%:%
%:%616=328%:%
%:%618=330%:%
%:%619=330%:%
%:%622=331%:%
%:%626=331%:%
%:%627=331%:%
%:%628=331%:%
%:%637=333%:%
%:%638=334%:%
%:%639=335%:%
%:%640=336%:%
%:%641=337%:%
%:%642=338%:%
%:%643=339%:%
%:%644=340%:%
%:%645=341%:%
%:%646=342%:%
%:%647=343%:%
%:%648=344%:%
%:%649=345%:%
%:%650=346%:%
%:%651=347%:%
%:%652=348%:%
%:%654=350%:%
%:%655=350%:%
%:%656=351%:%
%:%657=352%:%
%:%664=353%:%
%:%665=353%:%
%:%666=354%:%
%:%667=354%:%
%:%668=355%:%
%:%669=355%:%
%:%670=355%:%
%:%671=356%:%
%:%672=356%:%
%:%673=356%:%
%:%674=357%:%
%:%675=357%:%
%:%676=358%:%
%:%677=358%:%
%:%678=358%:%
%:%679=359%:%
%:%680=359%:%
%:%681=360%:%
%:%682=360%:%
%:%683=361%:%
%:%684=361%:%
%:%685=362%:%
%:%695=364%:%
%:%697=366%:%
%:%698=366%:%
%:%699=367%:%
%:%702=368%:%
%:%706=368%:%
%:%707=368%:%
%:%708=368%:%
%:%717=370%:%
%:%718=371%:%
%:%719=372%:%
%:%720=373%:%
%:%721=374%:%
%:%722=375%:%
%:%723=376%:%
%:%724=377%:%
%:%725=378%:%
%:%726=379%:%
%:%727=380%:%
%:%728=381%:%
%:%729=382%:%
%:%730=383%:%
%:%731=384%:%
%:%732=385%:%
%:%733=386%:%
%:%734=387%:%
%:%735=388%:%
%:%736=389%:%
%:%737=390%:%
%:%738=391%:%
%:%739=392%:%
%:%740=393%:%
%:%741=394%:%
%:%742=395%:%
%:%743=396%:%
%:%744=397%:%
%:%745=398%:%
%:%746=399%:%
%:%747=400%:%
%:%748=401%:%
%:%749=402%:%
%:%750=403%:%
%:%751=404%:%
%:%753=406%:%
%:%754=406%:%
%:%755=407%:%
%:%756=408%:%
%:%757=409%:%
%:%760=410%:%
%:%764=410%:%
%:%765=410%:%
%:%766=411%:%
%:%767=411%:%
%:%768=412%:%
%:%769=412%:%
%:%770=413%:%
%:%771=413%:%
%:%772=414%:%
%:%773=414%:%
%:%774=415%:%
%:%775=415%:%
%:%776=416%:%
%:%777=416%:%
%:%778=417%:%
%:%779=417%:%
%:%780=417%:%
%:%781=418%:%
%:%782=418%:%
%:%783=419%:%
%:%784=419%:%
%:%785=419%:%
%:%786=420%:%
%:%787=420%:%
%:%788=421%:%
%:%789=421%:%
%:%790=422%:%
%:%791=422%:%
%:%792=423%:%
%:%793=423%:%
%:%794=423%:%
%:%795=424%:%
%:%796=424%:%
%:%797=425%:%
%:%798=425%:%
%:%799=425%:%
%:%800=426%:%
%:%801=426%:%
%:%802=427%:%
%:%803=427%:%
%:%804=427%:%
%:%805=428%:%
%:%806=428%:%
%:%807=429%:%
%:%808=429%:%
%:%809=429%:%
%:%810=430%:%
%:%811=430%:%
%:%812=431%:%
%:%813=431%:%
%:%814=432%:%
%:%815=432%:%
%:%816=432%:%
%:%817=433%:%
%:%818=433%:%
%:%819=434%:%
%:%820=434%:%
%:%821=434%:%
%:%822=435%:%
%:%823=435%:%
%:%824=435%:%
%:%825=436%:%
%:%826=436%:%
%:%827=437%:%
%:%828=437%:%
%:%829=438%:%
%:%830=438%:%
%:%831=438%:%
%:%832=439%:%
%:%833=439%:%
%:%834=440%:%
%:%835=440%:%
%:%836=441%:%
%:%837=441%:%
%:%838=441%:%
%:%839=442%:%
%:%840=442%:%
%:%841=443%:%
%:%842=443%:%
%:%843=444%:%
%:%844=444%:%
%:%845=445%:%
%:%846=445%:%
%:%847=445%:%
%:%848=446%:%
%:%849=446%:%
%:%850=447%:%
%:%851=447%:%
%:%852=448%:%
%:%853=448%:%
%:%854=448%:%
%:%855=449%:%
%:%856=449%:%
%:%857=450%:%
%:%858=450%:%
%:%859=450%:%
%:%860=451%:%
%:%861=451%:%
%:%862=451%:%
%:%863=452%:%
%:%864=452%:%
%:%865=452%:%
%:%866=453%:%
%:%867=453%:%
%:%868=454%:%
%:%869=454%:%
%:%870=455%:%
%:%880=457%:%
%:%882=459%:%
%:%883=459%:%
%:%884=460%:%
%:%885=461%:%
%:%886=462%:%
%:%889=463%:%
%:%893=463%:%
%:%894=463%:%
%:%895=464%:%
%:%896=464%:%
%:%897=465%:%
%:%898=465%:%
%:%899=466%:%
%:%900=466%:%
%:%901=466%:%
%:%902=467%:%
%:%903=467%:%
%:%904=467%:%
%:%905=467%:%
%:%906=468%:%
%:%907=468%:%
%:%908=468%:%
%:%909=469%:%
%:%910=469%:%
%:%911=470%:%
%:%912=470%:%
%:%913=470%:%
%:%914=471%:%
%:%915=471%:%
%:%916=472%:%
%:%917=472%:%
%:%918=472%:%
%:%919=473%:%
%:%920=473%:%
%:%934=475%:%
%:%946=477%:%
%:%947=478%:%
%:%948=479%:%
%:%949=480%:%
%:%950=481%:%
%:%951=482%:%
%:%952=483%:%
%:%953=484%:%
%:%954=485%:%
%:%954=486%:%
%:%955=487%:%
%:%956=488%:%
%:%957=489%:%
%:%961=491%:%
%:%962=492%:%
%:%963=493%:%
%:%964=494%:%
%:%965=495%:%
%:%966=496%:%
%:%967=497%:%
%:%968=498%:%
%:%969=499%:%
%:%970=500%:%
%:%971=501%:%
%:%972=502%:%
%:%973=503%:%
%:%974=504%:%
%:%975=505%:%
%:%976=506%:%
%:%977=507%:%
%:%978=508%:%
%:%979=509%:%
%:%980=510%:%
%:%981=511%:%
%:%982=512%:%
%:%983=513%:%
%:%984=514%:%
%:%985=515%:%
%:%986=516%:%
%:%987=517%:%
%:%989=519%:%
%:%990=519%:%
%:%991=520%:%
%:%992=521%:%
%:%993=522%:%
%:%994=523%:%
%:%995=524%:%
%:%1002=525%:%
%:%1003=525%:%
%:%1004=526%:%
%:%1005=526%:%
%:%1006=527%:%
%:%1007=527%:%
%:%1008=527%:%
%:%1009=527%:%
%:%1010=528%:%
%:%1011=528%:%
%:%1012=528%:%
%:%1013=529%:%
%:%1014=529%:%
%:%1015=530%:%
%:%1016=530%:%
%:%1017=531%:%
%:%1018=531%:%
%:%1019=531%:%
%:%1020=531%:%
%:%1021=532%:%
%:%1022=532%:%
%:%1023=533%:%
%:%1024=533%:%
%:%1025=534%:%
%:%1026=534%:%
%:%1027=535%:%
%:%1028=535%:%
%:%1029=536%:%
%:%1030=536%:%
%:%1031=537%:%
%:%1032=537%:%
%:%1033=538%:%
%:%1034=538%:%
%:%1035=539%:%
%:%1036=539%:%
%:%1037=539%:%
%:%1038=540%:%
%:%1039=540%:%
%:%1040=540%:%
%:%1041=541%:%
%:%1042=541%:%
%:%1043=542%:%
%:%1044=542%:%
%:%1045=542%:%
%:%1046=543%:%
%:%1047=543%:%
%:%1048=544%:%
%:%1049=544%:%
%:%1050=544%:%
%:%1051=545%:%
%:%1052=545%:%
%:%1053=546%:%
%:%1054=546%:%
%:%1055=546%:%
%:%1056=547%:%
%:%1057=547%:%
%:%1058=548%:%
%:%1059=548%:%
%:%1060=548%:%
%:%1061=549%:%
%:%1062=549%:%
%:%1063=550%:%
%:%1064=550%:%
%:%1065=551%:%
%:%1066=551%:%
%:%1067=552%:%
%:%1068=552%:%
%:%1069=552%:%
%:%1070=553%:%
%:%1080=555%:%
%:%1082=557%:%
%:%1083=557%:%
%:%1084=558%:%
%:%1085=559%:%
%:%1086=560%:%
%:%1087=561%:%
%:%1088=562%:%
%:%1095=563%:%
%:%1096=563%:%
%:%1097=564%:%
%:%1098=564%:%
%:%1099=564%:%
%:%1100=564%:%
%:%1101=565%:%
%:%1102=565%:%
%:%1103=565%:%
%:%1104=565%:%
%:%1105=566%:%
%:%1106=566%:%
%:%1107=567%:%
%:%1108=567%:%
%:%1109=568%:%
%:%1110=568%:%
%:%1111=569%:%
%:%1112=569%:%
%:%1113=570%:%
%:%1114=570%:%
%:%1115=570%:%
%:%1116=571%:%
%:%1117=571%:%
%:%1118=572%:%
%:%1119=572%:%
%:%1120=573%:%
%:%1121=573%:%
%:%1122=574%:%
%:%1123=574%:%
%:%1124=575%:%
%:%1125=575%:%
%:%1126=575%:%
%:%1127=576%:%
%:%1128=576%:%
%:%1129=576%:%
%:%1130=576%:%
%:%1131=577%:%
%:%1132=577%:%
%:%1133=577%:%
%:%1134=578%:%
%:%1135=578%:%
%:%1136=579%:%
%:%1137=579%:%
%:%1138=579%:%
%:%1139=580%:%
%:%1140=580%:%
%:%1141=581%:%
%:%1142=581%:%
%:%1143=581%:%
%:%1144=582%:%
%:%1145=582%:%
%:%1146=583%:%
%:%1147=583%:%
%:%1148=584%:%
%:%1149=584%:%
%:%1150=584%:%
%:%1151=584%:%
%:%1152=585%:%
%:%1153=585%:%
%:%1154=586%:%
%:%1155=586%:%
%:%1156=586%:%
%:%1157=586%:%
%:%1158=586%:%
%:%1159=587%:%
%:%1169=589%:%
%:%1170=590%:%
%:%1171=591%:%
%:%1172=592%:%
%:%1173=593%:%
%:%1174=594%:%
%:%1175=595%:%
%:%1176=596%:%
%:%1177=597%:%
%:%1178=598%:%
%:%1179=599%:%
%:%1180=600%:%
%:%1181=601%:%
%:%1182=602%:%
%:%1183=603%:%
%:%1184=604%:%
%:%1185=605%:%
%:%1186=606%:%
%:%1187=607%:%
%:%1188=608%:%
%:%1189=609%:%
%:%1190=610%:%
%:%1191=611%:%
%:%1192=612%:%
%:%1193=613%:%
%:%1194=614%:%
%:%1195=615%:%
%:%1196=616%:%
%:%1197=617%:%
%:%1198=618%:%
%:%1199=619%:%
%:%1200=620%:%
%:%1201=621%:%
%:%1202=622%:%
%:%1204=624%:%
%:%1205=624%:%
%:%1206=625%:%
%:%1207=626%:%
%:%1208=627%:%
%:%1209=628%:%
%:%1210=629%:%
%:%1217=630%:%
%:%1218=630%:%
%:%1219=631%:%
%:%1220=631%:%
%:%1221=632%:%
%:%1222=632%:%
%:%1223=633%:%
%:%1224=633%:%
%:%1225=633%:%
%:%1226=634%:%
%:%1227=634%:%
%:%1228=635%:%
%:%1229=635%:%
%:%1230=636%:%
%:%1231=636%:%
%:%1232=636%:%
%:%1233=637%:%
%:%1234=637%:%
%:%1235=637%:%
%:%1236=638%:%
%:%1237=638%:%
%:%1238=638%:%
%:%1239=639%:%
%:%1240=639%:%
%:%1241=640%:%
%:%1242=640%:%
%:%1243=640%:%
%:%1244=641%:%
%:%1245=641%:%
%:%1246=642%:%
%:%1247=642%:%
%:%1248=643%:%
%:%1249=643%:%
%:%1250=643%:%
%:%1251=644%:%
%:%1252=644%:%
%:%1253=645%:%
%:%1254=645%:%
%:%1255=645%:%
%:%1256=646%:%
%:%1257=646%:%
%:%1258=647%:%
%:%1259=647%:%
%:%1260=647%:%
%:%1261=648%:%
%:%1262=648%:%
%:%1263=649%:%
%:%1264=649%:%
%:%1265=650%:%
%:%1266=650%:%
%:%1267=650%:%
%:%1268=651%:%
%:%1269=651%:%
%:%1270=652%:%
%:%1271=652%:%
%:%1272=652%:%
%:%1273=653%:%
%:%1274=653%:%
%:%1275=653%:%
%:%1276=654%:%
%:%1277=654%:%
%:%1278=655%:%
%:%1279=655%:%
%:%1280=656%:%
%:%1281=656%:%
%:%1282=657%:%
%:%1283=657%:%
%:%1284=657%:%
%:%1285=658%:%
%:%1286=658%:%
%:%1287=659%:%
%:%1288=659%:%
%:%1289=659%:%
%:%1290=660%:%
%:%1291=660%:%
%:%1292=661%:%
%:%1293=661%:%
%:%1294=662%:%
%:%1295=663%:%
%:%1296=663%:%
%:%1297=664%:%
%:%1298=664%:%
%:%1299=664%:%
%:%1300=665%:%
%:%1301=666%:%
%:%1302=666%:%
%:%1303=667%:%
%:%1304=667%:%
%:%1305=667%:%
%:%1306=668%:%
%:%1307=668%:%
%:%1308=668%:%
%:%1309=669%:%
%:%1310=669%:%
%:%1311=669%:%
%:%1312=670%:%
%:%1313=670%:%
%:%1314=671%:%
%:%1315=671%:%
%:%1316=671%:%
%:%1317=672%:%
%:%1318=672%:%
%:%1319=673%:%
%:%1320=673%:%
%:%1321=674%:%
%:%1322=674%:%
%:%1323=674%:%
%:%1324=675%:%
%:%1325=675%:%
%:%1326=676%:%
%:%1327=676%:%
%:%1328=677%:%
%:%1329=677%:%
%:%1330=678%:%
%:%1331=678%:%
%:%1332=678%:%
%:%1333=679%:%
%:%1334=679%:%
%:%1335=679%:%
%:%1336=680%:%
%:%1337=680%:%
%:%1338=680%:%
%:%1339=681%:%
%:%1340=681%:%
%:%1341=681%:%
%:%1342=682%:%
%:%1343=682%:%
%:%1344=683%:%
%:%1345=683%:%
%:%1346=683%:%
%:%1347=684%:%
%:%1348=684%:%
%:%1349=684%:%
%:%1350=685%:%
%:%1351=685%:%
%:%1352=685%:%
%:%1353=686%:%
%:%1354=686%:%
%:%1355=686%:%
%:%1356=687%:%
%:%1357=687%:%
%:%1358=687%:%
%:%1359=688%:%
%:%1360=688%:%
%:%1361=689%:%
%:%1362=689%:%
%:%1363=689%:%
%:%1364=690%:%
%:%1374=692%:%
%:%1376=694%:%
%:%1377=694%:%
%:%1378=695%:%
%:%1379=696%:%
%:%1380=697%:%
%:%1381=698%:%
%:%1382=699%:%
%:%1389=700%:%
%:%1390=700%:%
%:%1391=701%:%
%:%1392=701%:%
%:%1393=702%:%
%:%1394=702%:%
%:%1395=702%:%
%:%1396=702%:%
%:%1397=703%:%
%:%1398=703%:%
%:%1399=703%:%
%:%1400=703%:%
%:%1401=704%:%
%:%1402=704%:%
%:%1403=704%:%
%:%1404=704%:%
%:%1405=704%:%
%:%1406=705%:%
%:%1407=705%:%
%:%1408=705%:%
%:%1409=706%:%
%:%1410=706%:%
%:%1411=706%:%
%:%1412=706%:%
%:%1413=707%:%
%:%1414=707%:%
%:%1415=707%:%
%:%1416=708%:%
%:%1417=708%:%
%:%1418=708%:%
%:%1419=708%:%
%:%1420=709%:%
%:%1421=709%:%
%:%1422=709%:%
%:%1423=709%:%
%:%1424=710%:%
%:%1434=712%:%
%:%1435=713%:%
%:%1436=714%:%
%:%1437=715%:%
%:%1438=716%:%
%:%1439=717%:%
%:%1440=718%:%
%:%1441=719%:%
%:%1442=720%:%
%:%1443=721%:%
%:%1444=722%:%
%:%1445=723%:%
%:%1446=724%:%
%:%1447=725%:%
%:%1448=726%:%
%:%1449=727%:%
%:%1450=728%:%
%:%1451=729%:%
%:%1452=730%:%
%:%1453=731%:%
%:%1454=732%:%
%:%1455=733%:%
%:%1457=735%:%
%:%1458=735%:%
%:%1459=736%:%
%:%1460=737%:%
%:%1461=738%:%
%:%1468=739%:%
%:%1469=739%:%
%:%1470=740%:%
%:%1471=740%:%
%:%1472=741%:%
%:%1473=741%:%
%:%1474=742%:%
%:%1475=742%:%
%:%1476=743%:%
%:%1477=743%:%
%:%1478=744%:%
%:%1479=744%:%
%:%1480=744%:%
%:%1481=745%:%
%:%1482=745%:%
%:%1483=745%:%
%:%1484=746%:%
%:%1485=746%:%
%:%1486=747%:%
%:%1487=747%:%
%:%1488=747%:%
%:%1489=748%:%
%:%1490=748%:%
%:%1491=749%:%
%:%1492=749%:%
%:%1493=750%:%
%:%1494=750%:%
%:%1495=751%:%
%:%1505=753%:%
%:%1507=755%:%
%:%1508=755%:%
%:%1509=756%:%
%:%1510=757%:%
%:%1511=758%:%
%:%1514=759%:%
%:%1518=759%:%
%:%1519=759%:%
%:%1520=759%:%
%:%1529=761%:%
%:%1530=762%:%
%:%1531=763%:%
%:%1532=764%:%
%:%1533=765%:%
%:%1534=766%:%
%:%1535=767%:%
%:%1536=768%:%
%:%1537=769%:%
%:%1538=770%:%
%:%1539=771%:%
%:%1540=772%:%
%:%1541=773%:%
%:%1542=774%:%
%:%1543=775%:%
%:%1544=776%:%
%:%1545=777%:%
%:%1546=778%:%
%:%1547=779%:%
%:%1548=780%:%
%:%1549=781%:%
%:%1550=782%:%
%:%1551=783%:%
%:%1552=784%:%
%:%1553=785%:%
%:%1554=786%:%
%:%1555=787%:%
%:%1556=788%:%
%:%1557=789%:%
%:%1558=790%:%
%:%1559=791%:%
%:%1560=792%:%
%:%1561=793%:%
%:%1562=794%:%
%:%1563=795%:%
%:%1564=796%:%
%:%1565=797%:%
%:%1566=798%:%
%:%1567=799%:%
%:%1568=800%:%
%:%1569=801%:%
%:%1570=802%:%
%:%1571=803%:%
%:%1572=804%:%
%:%1573=805%:%
%:%1574=806%:%
%:%1575=807%:%
%:%1576=808%:%
%:%1577=809%:%
%:%1578=810%:%
%:%1579=811%:%
%:%1580=812%:%
%:%1581=813%:%
%:%1582=814%:%
%:%1583=815%:%
%:%1584=816%:%
%:%1585=817%:%
%:%1586=818%:%
%:%1587=819%:%
%:%1588=820%:%
%:%1589=821%:%
%:%1590=822%:%
%:%1591=823%:%
%:%1592=824%:%
%:%1593=825%:%
%:%1594=826%:%
%:%1595=827%:%
%:%1596=828%:%
%:%1597=829%:%
%:%1598=830%:%
%:%1599=831%:%
%:%1600=832%:%
%:%1602=834%:%
%:%1603=834%:%
%:%1604=835%:%
%:%1605=836%:%
%:%1606=837%:%
%:%1607=838%:%
%:%1608=839%:%
%:%1615=840%:%
%:%1616=840%:%
%:%1617=841%:%
%:%1618=841%:%
%:%1619=842%:%
%:%1620=842%:%
%:%1621=843%:%
%:%1622=843%:%
%:%1623=843%:%
%:%1624=844%:%
%:%1625=844%:%
%:%1629=848%:%
%:%1630=849%:%
%:%1631=849%:%
%:%1632=849%:%
%:%1633=850%:%
%:%1634=850%:%
%:%1635=850%:%
%:%1638=853%:%
%:%1639=854%:%
%:%1640=854%:%
%:%1641=854%:%
%:%1642=855%:%
%:%1643=855%:%
%:%1644=855%:%
%:%1645=856%:%
%:%1646=856%:%
%:%1647=857%:%
%:%1648=857%:%
%:%1649=858%:%
%:%1650=858%:%
%:%1651=858%:%
%:%1652=859%:%
%:%1653=859%:%
%:%1654=860%:%
%:%1655=860%:%
%:%1656=860%:%
%:%1657=861%:%
%:%1658=861%:%
%:%1659=862%:%
%:%1660=862%:%
%:%1661=863%:%
%:%1662=863%:%
%:%1663=864%:%
%:%1664=864%:%
%:%1665=865%:%
%:%1666=865%:%
%:%1667=866%:%
%:%1668=866%:%
%:%1669=867%:%
%:%1670=867%:%
%:%1671=868%:%
%:%1672=868%:%
%:%1673=869%:%
%:%1674=869%:%
%:%1675=869%:%
%:%1676=870%:%
%:%1677=870%:%
%:%1678=870%:%
%:%1679=871%:%
%:%1680=871%:%
%:%1681=871%:%
%:%1682=872%:%
%:%1683=872%:%
%:%1684=872%:%
%:%1685=873%:%
%:%1686=873%:%
%:%1687=873%:%
%:%1688=874%:%
%:%1689=874%:%
%:%1690=875%:%
%:%1691=875%:%
%:%1692=876%:%
%:%1693=876%:%
%:%1694=876%:%
%:%1695=877%:%
%:%1696=877%:%
%:%1697=877%:%
%:%1698=878%:%
%:%1699=878%:%
%:%1700=879%:%
%:%1701=879%:%
%:%1702=880%:%
%:%1703=880%:%
%:%1704=880%:%
%:%1705=881%:%
%:%1706=881%:%
%:%1707=882%:%
%:%1708=882%:%
%:%1709=883%:%
%:%1710=883%:%
%:%1711=883%:%
%:%1712=884%:%
%:%1713=884%:%
%:%1714=884%:%
%:%1715=885%:%
%:%1716=885%:%
%:%1717=886%:%
%:%1718=886%:%
%:%1719=886%:%
%:%1720=887%:%
%:%1721=887%:%
%:%1722=888%:%
%:%1723=888%:%
%:%1724=888%:%
%:%1725=889%:%
%:%1726=889%:%
%:%1727=890%:%
%:%1728=890%:%
%:%1729=891%:%
%:%1730=891%:%
%:%1731=891%:%
%:%1732=892%:%
%:%1733=892%:%
%:%1734=892%:%
%:%1735=893%:%
%:%1736=893%:%
%:%1737=894%:%
%:%1738=894%:%
%:%1739=894%:%
%:%1740=895%:%
%:%1741=895%:%
%:%1742=896%:%
%:%1743=896%:%
%:%1744=897%:%
%:%1745=897%:%
%:%1746=897%:%
%:%1747=898%:%
%:%1748=898%:%
%:%1749=899%:%
%:%1750=899%:%
%:%1751=900%:%
%:%1752=900%:%
%:%1753=901%:%
%:%1754=901%:%
%:%1755=901%:%
%:%1756=902%:%
%:%1757=902%:%
%:%1758=903%:%
%:%1759=903%:%
%:%1760=904%:%
%:%1761=904%:%
%:%1762=905%:%
%:%1763=905%:%
%:%1764=906%:%
%:%1765=906%:%
%:%1766=907%:%
%:%1767=907%:%
%:%1768=908%:%
%:%1769=908%:%
%:%1770=909%:%
%:%1771=909%:%
%:%1772=910%:%
%:%1773=910%:%
%:%1774=910%:%
%:%1775=911%:%
%:%1776=911%:%
%:%1777=911%:%
%:%1778=912%:%
%:%1779=912%:%
%:%1780=912%:%
%:%1781=913%:%
%:%1782=913%:%
%:%1783=913%:%
%:%1784=914%:%
%:%1785=914%:%
%:%1786=914%:%
%:%1787=915%:%
%:%1788=915%:%
%:%1789=916%:%
%:%1790=916%:%
%:%1791=917%:%
%:%1792=917%:%
%:%1793=918%:%
%:%1794=918%:%
%:%1795=919%:%
%:%1796=919%:%
%:%1797=920%:%
%:%1798=920%:%
%:%1799=921%:%
%:%1800=921%:%
%:%1801=921%:%
%:%1802=922%:%
%:%1803=922%:%
%:%1804=923%:%
%:%1805=923%:%
%:%1806=923%:%
%:%1807=924%:%
%:%1808=924%:%
%:%1809=924%:%
%:%1810=925%:%
%:%1811=925%:%
%:%1812=926%:%
%:%1813=926%:%
%:%1814=927%:%
%:%1815=927%:%
%:%1816=928%:%
%:%1817=928%:%
%:%1818=929%:%
%:%1819=929%:%
%:%1820=930%:%
%:%1821=930%:%
%:%1822=931%:%
%:%1823=931%:%
%:%1824=932%:%
%:%1825=932%:%
%:%1826=933%:%
%:%1827=933%:%
%:%1828=933%:%
%:%1829=934%:%
%:%1830=934%:%
%:%1831=935%:%
%:%1832=935%:%
%:%1833=935%:%
%:%1834=936%:%
%:%1835=936%:%
%:%1836=936%:%
%:%1837=937%:%
%:%1838=937%:%
%:%1839=938%:%
%:%1840=938%:%
%:%1841=939%:%
%:%1842=939%:%
%:%1843=940%:%
%:%1844=940%:%
%:%1845=941%:%
%:%1846=941%:%
%:%1847=942%:%
%:%1848=942%:%
%:%1849=943%:%
%:%1850=943%:%
%:%1853=946%:%
%:%1854=947%:%
%:%1855=947%:%
%:%1856=947%:%
%:%1857=948%:%
%:%1867=950%:%
%:%1869=952%:%
%:%1870=952%:%
%:%1871=953%:%
%:%1872=954%:%
%:%1873=955%:%
%:%1874=956%:%
%:%1875=957%:%
%:%1882=958%:%
%:%1883=958%:%
%:%1884=959%:%
%:%1885=959%:%
%:%1886=960%:%
%:%1887=960%:%
%:%1888=960%:%
%:%1889=960%:%
%:%1890=961%:%
%:%1891=961%:%
%:%1892=962%:%
%:%1893=962%:%
%:%1894=963%:%
%:%1895=964%:%
%:%1896=965%:%
%:%1897=966%:%
%:%1898=966%:%
%:%1899=967%:%
%:%1900=967%:%
%:%1901=967%:%
%:%1902=967%:%
%:%1903=968%:%
%:%1904=968%:%
%:%1905=968%:%
%:%1906=969%:%
%:%1916=971%:%
%:%1917=972%:%
%:%1918=973%:%
%:%1919=974%:%
%:%1920=975%:%
%:%1921=976%:%
%:%1922=977%:%
%:%1923=978%:%
%:%1924=979%:%
%:%1925=980%:%
%:%1926=981%:%
%:%1927=982%:%
%:%1928=983%:%
%:%1929=984%:%
%:%1930=985%:%
%:%1931=986%:%
%:%1933=988%:%
%:%1934=988%:%
%:%1935=989%:%
%:%1936=990%:%
%:%1937=991%:%
%:%1940=992%:%
%:%1944=992%:%
%:%1945=992%:%
%:%1946=993%:%
%:%1947=993%:%
%:%1948=994%:%
%:%1949=994%:%
%:%1950=995%:%
%:%1951=995%:%
%:%1952=995%:%
%:%1953=996%:%
%:%1954=996%:%
%:%1955=997%:%
%:%1956=997%:%
%:%1957=997%:%
%:%1958=998%:%
%:%1959=998%:%
%:%1960=999%:%
%:%1961=999%:%
%:%1962=1000%:%
%:%1963=1000%:%
%:%1964=1001%:%
%:%1965=1001%:%
%:%1966=1002%:%
%:%1967=1002%:%
%:%1968=1003%:%
%:%1969=1003%:%
%:%1970=1003%:%
%:%1971=1004%:%
%:%1972=1004%:%
%:%1973=1004%:%
%:%1974=1005%:%
%:%1975=1005%:%
%:%1976=1005%:%
%:%1977=1006%:%
%:%1978=1006%:%
%:%1979=1007%:%
%:%1980=1007%:%
%:%1981=1007%:%
%:%1982=1008%:%
%:%1983=1008%:%
%:%1984=1009%:%
%:%1985=1009%:%
%:%1986=1010%:%
%:%1987=1010%:%
%:%1988=1011%:%
%:%1998=1013%:%
%:%1999=1014%:%
%:%2000=1015%:%
%:%2001=1016%:%
%:%2002=1017%:%
%:%2003=1018%:%
%:%2004=1019%:%
%:%2005=1020%:%
%:%2006=1021%:%
%:%2007=1022%:%
%:%2008=1023%:%
%:%2009=1024%:%
%:%2010=1025%:%
%:%2011=1026%:%
%:%2012=1027%:%
%:%2013=1028%:%
%:%2014=1029%:%
%:%2015=1030%:%
%:%2016=1031%:%
%:%2017=1032%:%
%:%2018=1033%:%
%:%2019=1034%:%
%:%2020=1035%:%
%:%2021=1036%:%
%:%2022=1037%:%
%:%2023=1038%:%
%:%2024=1039%:%
%:%2025=1040%:%
%:%2026=1041%:%
%:%2027=1042%:%
%:%2028=1043%:%
%:%2029=1044%:%
%:%2030=1045%:%
%:%2031=1046%:%
%:%2032=1047%:%
%:%2033=1048%:%
%:%2035=1050%:%
%:%2036=1050%:%
%:%2037=1051%:%
%:%2038=1052%:%
%:%2039=1053%:%
%:%2040=1054%:%
%:%2047=1055%:%
%:%2048=1055%:%
%:%2049=1056%:%
%:%2050=1056%:%
%:%2051=1057%:%
%:%2052=1057%:%
%:%2053=1058%:%
%:%2054=1058%:%
%:%2055=1058%:%
%:%2056=1059%:%
%:%2057=1059%:%
%:%2058=1060%:%
%:%2059=1060%:%
%:%2060=1061%:%
%:%2061=1061%:%
%:%2062=1061%:%
%:%2063=1062%:%
%:%2064=1062%:%
%:%2065=1063%:%
%:%2066=1063%:%
%:%2067=1063%:%
%:%2068=1064%:%
%:%2069=1064%:%
%:%2070=1065%:%
%:%2071=1065%:%
%:%2072=1065%:%
%:%2073=1066%:%
%:%2074=1066%:%
%:%2075=1067%:%
%:%2076=1067%:%
%:%2077=1067%:%
%:%2078=1068%:%
%:%2079=1068%:%
%:%2080=1069%:%
%:%2081=1069%:%
%:%2082=1069%:%
%:%2083=1070%:%
%:%2084=1070%:%
%:%2085=1071%:%
%:%2086=1071%:%
%:%2087=1071%:%
%:%2088=1072%:%
%:%2089=1072%:%
%:%2090=1073%:%
%:%2091=1073%:%
%:%2092=1073%:%
%:%2093=1074%:%
%:%2094=1074%:%
%:%2095=1074%:%
%:%2096=1075%:%
%:%2097=1075%:%
%:%2098=1075%:%
%:%2099=1076%:%
%:%2100=1076%:%
%:%2101=1077%:%
%:%2102=1077%:%
%:%2103=1077%:%
%:%2104=1078%:%
%:%2105=1078%:%
%:%2106=1079%:%
%:%2107=1079%:%
%:%2108=1079%:%
%:%2109=1080%:%
%:%2110=1080%:%
%:%2111=1080%:%
%:%2112=1081%:%
%:%2113=1081%:%
%:%2114=1082%:%
%:%2115=1082%:%
%:%2116=1083%:%
%:%2117=1083%:%
%:%2118=1083%:%
%:%2119=1084%:%
%:%2120=1084%:%
%:%2121=1084%:%
%:%2122=1085%:%
%:%2123=1085%:%
%:%2124=1085%:%
%:%2125=1086%:%
%:%2126=1086%:%
%:%2127=1087%:%
%:%2128=1087%:%
%:%2129=1087%:%
%:%2130=1088%:%
%:%2131=1088%:%
%:%2132=1088%:%
%:%2133=1089%:%
%:%2134=1089%:%
%:%2135=1090%:%
%:%2136=1090%:%
%:%2137=1091%:%
%:%2138=1091%:%
%:%2139=1092%:%
%:%2140=1092%:%
%:%2141=1093%:%
%:%2142=1093%:%
%:%2143=1094%:%
%:%2144=1094%:%
%:%2145=1094%:%
%:%2146=1095%:%
%:%2147=1095%:%
%:%2148=1096%:%
%:%2149=1096%:%
%:%2150=1097%:%
%:%2151=1097%:%
%:%2152=1097%:%
%:%2153=1098%:%
%:%2163=1100%:%
%:%2165=1102%:%
%:%2166=1102%:%
%:%2167=1103%:%
%:%2168=1104%:%
%:%2169=1105%:%
%:%2170=1106%:%
%:%2177=1107%:%
%:%2178=1107%:%
%:%2179=1108%:%
%:%2180=1108%:%
%:%2181=1108%:%
%:%2182=1109%:%
%:%2183=1110%:%
%:%2184=1110%:%
%:%2185=1111%:%
%:%2186=1111%:%
%:%2187=1112%:%
%:%2188=1112%:%
%:%2189=1112%:%
%:%2190=1112%:%
%:%2191=1113%:%
%:%2192=1113%:%
%:%2193=1114%:%
%:%2194=1114%:%
%:%2195=1114%:%
%:%2196=1115%:%
%:%2197=1115%:%
%:%2198=1115%:%
%:%2199=1115%:%
%:%2200=1116%:%
%:%2201=1116%:%
%:%2202=1116%:%
%:%2203=1117%:%
%:%2204=1117%:%
%:%2205=1117%:%
%:%2206=1118%:%
%:%2207=1118%:%
%:%2208=1118%:%
%:%2209=1118%:%
%:%2210=1118%:%
%:%2211=1119%:%
%:%2226=1121%:%
%:%2238=1123%:%
%:%2239=1124%:%
%:%2240=1125%:%
%:%2241=1126%:%
%:%2242=1127%:%
%:%2243=1128%:%
%:%2244=1129%:%
%:%2245=1130%:%
%:%2246=1131%:%
%:%2247=1132%:%
%:%2248=1133%:%
%:%2249=1134%:%
%:%2250=1135%:%
%:%2251=1136%:%
%:%2252=1137%:%
%:%2253=1138%:%
%:%2254=1139%:%
%:%2255=1140%:%
%:%2256=1141%:%
%:%2257=1142%:%
%:%2258=1143%:%
%:%2259=1144%:%
%:%2260=1145%:%
%:%2261=1146%:%
%:%2262=1147%:%
%:%2263=1148%:%
%:%2264=1149%:%
%:%2265=1150%:%
%:%2266=1151%:%
%:%2267=1152%:%
%:%2268=1153%:%
%:%2269=1154%:%
%:%2270=1155%:%
%:%2271=1156%:%
%:%2272=1157%:%
%:%2273=1158%:%
%:%2274=1159%:%
%:%2275=1160%:%
%:%2276=1161%:%
%:%2277=1162%:%
%:%2278=1163%:%
%:%2279=1164%:%
%:%2280=1165%:%
%:%2281=1166%:%
%:%2282=1167%:%
%:%2283=1168%:%
%:%2284=1169%:%
%:%2285=1170%:%
%:%2286=1171%:%
%:%2287=1172%:%
%:%2288=1173%:%
%:%2289=1174%:%
%:%2290=1175%:%
%:%2291=1176%:%
%:%2292=1177%:%
%:%2293=1178%:%
%:%2294=1179%:%
%:%2295=1180%:%
%:%2296=1181%:%
%:%2297=1182%:%
%:%2298=1183%:%
%:%2300=1185%:%
%:%2301=1185%:%
%:%2304=1186%:%
%:%2308=1186%:%
%:%2309=1186%:%
%:%2318=1188%:%
%:%2320=1190%:%
%:%2321=1190%:%
%:%2322=1191%:%
%:%2323=1192%:%
%:%2330=1193%:%
%:%2331=1193%:%
%:%2332=1194%:%
%:%2333=1194%:%
%:%2334=1195%:%
%:%2335=1195%:%
%:%2336=1196%:%
%:%2337=1196%:%
%:%2338=1197%:%
%:%2339=1197%:%
%:%2340=1198%:%
%:%2341=1198%:%
%:%2342=1199%:%
%:%2343=1199%:%
%:%2344=1200%:%
%:%2345=1200%:%
%:%2346=1201%:%
%:%2347=1201%:%
%:%2348=1201%:%
%:%2349=1202%:%
%:%2350=1202%:%
%:%2351=1203%:%
%:%2352=1203%:%
%:%2353=1204%:%
%:%2354=1204%:%
%:%2355=1204%:%
%:%2356=1205%:%
%:%2357=1205%:%
%:%2358=1206%:%
%:%2359=1206%:%
%:%2360=1206%:%
%:%2361=1207%:%
%:%2362=1207%:%
%:%2363=1208%:%
%:%2364=1208%:%
%:%2365=1209%:%
%:%2366=1209%:%
%:%2367=1210%:%
%:%2368=1210%:%
%:%2369=1211%:%
%:%2370=1211%:%
%:%2371=1212%:%
%:%2372=1212%:%
%:%2373=1213%:%
%:%2374=1213%:%
%:%2375=1214%:%
%:%2376=1214%:%
%:%2377=1215%:%
%:%2378=1215%:%
%:%2379=1215%:%
%:%2380=1216%:%
%:%2381=1216%:%
%:%2382=1217%:%
%:%2383=1217%:%
%:%2384=1217%:%
%:%2385=1218%:%
%:%2386=1218%:%
%:%2387=1219%:%
%:%2388=1219%:%
%:%2389=1220%:%
%:%2390=1220%:%
%:%2391=1220%:%
%:%2392=1221%:%
%:%2393=1221%:%
%:%2394=1222%:%
%:%2395=1222%:%
%:%2396=1222%:%
%:%2397=1223%:%
%:%2398=1223%:%
%:%2399=1224%:%
%:%2400=1224%:%
%:%2401=1224%:%
%:%2402=1225%:%
%:%2403=1225%:%
%:%2404=1225%:%
%:%2405=1226%:%
%:%2406=1226%:%
%:%2407=1227%:%
%:%2408=1227%:%
%:%2409=1228%:%
%:%2410=1228%:%
%:%2411=1228%:%
%:%2412=1229%:%
%:%2413=1229%:%
%:%2414=1229%:%
%:%2415=1230%:%
%:%2416=1230%:%
%:%2417=1231%:%
%:%2418=1231%:%
%:%2419=1231%:%
%:%2420=1232%:%
%:%2421=1232%:%
%:%2422=1233%:%
%:%2423=1233%:%
%:%2424=1234%:%
%:%2425=1234%:%
%:%2426=1235%:%
%:%2427=1235%:%
%:%2428=1236%:%
%:%2429=1236%:%
%:%2430=1237%:%
%:%2431=1237%:%
%:%2432=1237%:%
%:%2433=1238%:%
%:%2434=1238%:%
%:%2435=1239%:%
%:%2436=1239%:%
%:%2437=1240%:%
%:%2438=1240%:%
%:%2439=1241%:%
%:%2440=1241%:%
%:%2441=1242%:%
%:%2442=1242%:%
%:%2443=1242%:%
%:%2444=1243%:%
%:%2445=1243%:%
%:%2446=1243%:%
%:%2447=1244%:%
%:%2448=1244%:%
%:%2449=1245%:%
%:%2450=1245%:%
%:%2451=1245%:%
%:%2452=1246%:%
%:%2453=1246%:%
%:%2454=1247%:%
%:%2455=1247%:%
%:%2456=1248%:%
%:%2457=1248%:%
%:%2458=1249%:%
%:%2459=1249%:%
%:%2460=1249%:%
%:%2461=1250%:%
%:%2462=1250%:%
%:%2463=1251%:%
%:%2464=1251%:%
%:%2465=1251%:%
%:%2466=1252%:%
%:%2467=1252%:%
%:%2468=1253%:%
%:%2469=1253%:%
%:%2470=1254%:%
%:%2471=1254%:%
%:%2472=1254%:%
%:%2473=1255%:%
%:%2474=1255%:%
%:%2475=1256%:%
%:%2476=1256%:%
%:%2477=1256%:%
%:%2478=1257%:%
%:%2479=1257%:%
%:%2480=1258%:%
%:%2481=1258%:%
%:%2482=1259%:%
%:%2483=1259%:%
%:%2484=1260%:%
%:%2485=1260%:%
%:%2486=1261%:%
%:%2487=1261%:%
%:%2488=1261%:%
%:%2489=1262%:%
%:%2490=1262%:%
%:%2491=1263%:%
%:%2492=1263%:%
%:%2493=1263%:%
%:%2494=1264%:%
%:%2495=1264%:%
%:%2496=1265%:%
%:%2497=1265%:%
%:%2498=1266%:%
%:%2499=1266%:%
%:%2500=1266%:%
%:%2501=1267%:%
%:%2502=1267%:%
%:%2503=1268%:%
%:%2504=1268%:%
%:%2505=1268%:%
%:%2506=1269%:%
%:%2507=1269%:%
%:%2508=1270%:%
%:%2509=1270%:%
%:%2510=1271%:%
%:%2511=1271%:%
%:%2512=1272%:%
%:%2513=1272%:%
%:%2514=1273%:%
%:%2515=1273%:%
%:%2516=1273%:%
%:%2517=1274%:%
%:%2518=1274%:%
%:%2519=1274%:%
%:%2520=1275%:%
%:%2521=1275%:%
%:%2522=1276%:%
%:%2523=1276%:%
%:%2524=1277%:%
%:%2525=1277%:%
%:%2526=1277%:%
%:%2527=1278%:%
%:%2528=1278%:%
%:%2529=1279%:%
%:%2530=1279%:%
%:%2531=1279%:%
%:%2532=1280%:%
%:%2533=1280%:%
%:%2534=1281%:%
%:%2535=1281%:%
%:%2536=1282%:%
%:%2537=1282%:%
%:%2538=1283%:%
%:%2539=1283%:%
%:%2540=1284%:%
%:%2541=1284%:%
%:%2542=1285%:%
%:%2543=1285%:%
%:%2544=1285%:%
%:%2545=1286%:%
%:%2546=1286%:%
%:%2547=1287%:%
%:%2548=1287%:%
%:%2549=1287%:%
%:%2550=1288%:%
%:%2551=1288%:%
%:%2552=1289%:%
%:%2553=1289%:%
%:%2554=1290%:%
%:%2555=1290%:%
%:%2556=1290%:%
%:%2557=1291%:%
%:%2558=1291%:%
%:%2559=1292%:%
%:%2560=1292%:%
%:%2561=1292%:%
%:%2562=1293%:%
%:%2563=1293%:%
%:%2564=1294%:%
%:%2565=1294%:%
%:%2566=1295%:%
%:%2567=1295%:%
%:%2568=1296%:%
%:%2569=1296%:%
%:%2570=1297%:%
%:%2571=1297%:%
%:%2572=1297%:%
%:%2573=1298%:%
%:%2574=1298%:%
%:%2575=1299%:%
%:%2576=1299%:%
%:%2577=1300%:%
%:%2578=1300%:%
%:%2579=1300%:%
%:%2580=1301%:%
%:%2581=1301%:%
%:%2582=1302%:%
%:%2583=1302%:%
%:%2584=1302%:%
%:%2585=1303%:%
%:%2586=1303%:%
%:%2587=1304%:%
%:%2588=1304%:%
%:%2589=1305%:%
%:%2590=1305%:%
%:%2591=1306%:%
%:%2601=1308%:%
%:%2602=1309%:%
%:%2603=1310%:%
%:%2604=1311%:%
%:%2605=1312%:%
%:%2606=1313%:%
%:%2607=1314%:%
%:%2608=1315%:%
%:%2609=1316%:%
%:%2610=1317%:%
%:%2611=1318%:%
%:%2612=1319%:%
%:%2613=1320%:%
%:%2614=1321%:%
%:%2615=1322%:%
%:%2616=1323%:%
%:%2617=1324%:%
%:%2618=1325%:%
%:%2619=1326%:%
%:%2620=1327%:%
%:%2621=1328%:%
%:%2622=1329%:%
%:%2623=1330%:%
%:%2624=1331%:%
%:%2625=1332%:%
%:%2626=1333%:%
%:%2627=1334%:%
%:%2629=1336%:%
%:%2630=1336%:%
%:%2631=1337%:%
%:%2632=1338%:%
%:%2633=1339%:%
%:%2640=1340%:%
%:%2641=1340%:%
%:%2642=1341%:%
%:%2643=1341%:%
%:%2644=1342%:%
%:%2645=1342%:%
%:%2646=1342%:%
%:%2647=1343%:%
%:%2648=1343%:%
%:%2649=1343%:%
%:%2650=1344%:%
%:%2651=1344%:%
%:%2652=1344%:%
%:%2653=1345%:%
%:%2654=1345%:%
%:%2655=1346%:%
%:%2656=1346%:%
%:%2657=1346%:%
%:%2658=1347%:%
%:%2659=1347%:%
%:%2660=1348%:%
%:%2661=1348%:%
%:%2662=1348%:%
%:%2663=1349%:%
%:%2664=1349%:%
%:%2665=1350%:%
%:%2666=1350%:%
%:%2667=1350%:%
%:%2668=1351%:%
%:%2669=1351%:%
%:%2670=1352%:%
%:%2671=1352%:%
%:%2672=1352%:%
%:%2673=1353%:%
%:%2674=1353%:%
%:%2675=1354%:%
%:%2676=1354%:%
%:%2677=1354%:%
%:%2678=1355%:%
%:%2679=1355%:%
%:%2680=1356%:%
%:%2681=1356%:%
%:%2682=1356%:%
%:%2683=1357%:%
%:%2684=1357%:%
%:%2685=1358%:%
%:%2686=1358%:%
%:%2687=1359%:%
%:%2688=1359%:%
%:%2689=1360%:%
%:%2690=1360%:%
%:%2691=1361%:%
%:%2692=1361%:%
%:%2693=1361%:%
%:%2694=1362%:%
%:%2695=1362%:%
%:%2696=1363%:%
%:%2697=1363%:%
%:%2698=1364%:%
%:%2699=1364%:%
%:%2700=1365%:%
%:%2701=1365%:%
%:%2702=1365%:%
%:%2703=1366%:%
%:%2704=1366%:%
%:%2705=1367%:%
%:%2706=1367%:%
%:%2707=1367%:%
%:%2708=1368%:%
%:%2709=1368%:%
%:%2710=1369%:%
%:%2711=1369%:%
%:%2712=1369%:%
%:%2713=1370%:%
%:%2714=1370%:%
%:%2715=1371%:%
%:%2716=1371%:%
%:%2717=1371%:%
%:%2718=1372%:%
%:%2719=1372%:%
%:%2720=1372%:%
%:%2721=1373%:%
%:%2722=1373%:%
%:%2723=1374%:%
%:%2724=1374%:%
%:%2725=1374%:%
%:%2726=1375%:%
%:%2727=1375%:%
%:%2728=1376%:%
%:%2729=1376%:%
%:%2730=1377%:%
%:%2731=1377%:%
%:%2732=1377%:%
%:%2733=1378%:%
%:%2734=1378%:%
%:%2735=1379%:%
%:%2736=1379%:%
%:%2737=1379%:%
%:%2738=1380%:%
%:%2739=1380%:%
%:%2740=1381%:%
%:%2741=1381%:%
%:%2742=1381%:%
%:%2743=1382%:%
%:%2744=1382%:%
%:%2745=1382%:%
%:%2746=1383%:%
%:%2747=1383%:%
%:%2748=1384%:%
%:%2749=1384%:%
%:%2750=1384%:%
%:%2751=1385%:%
%:%2752=1385%:%
%:%2753=1386%:%
%:%2754=1386%:%
%:%2755=1387%:%
%:%2756=1387%:%
%:%2757=1388%:%
%:%2758=1388%:%
%:%2759=1389%:%
%:%2760=1389%:%
%:%2761=1389%:%
%:%2762=1390%:%
%:%2763=1390%:%
%:%2764=1391%:%
%:%2765=1391%:%
%:%2766=1392%:%
%:%2767=1392%:%
%:%2768=1393%:%
%:%2769=1393%:%
%:%2770=1393%:%
%:%2771=1394%:%
%:%2772=1394%:%
%:%2773=1395%:%
%:%2774=1395%:%
%:%2775=1395%:%
%:%2776=1396%:%
%:%2777=1396%:%
%:%2778=1397%:%
%:%2779=1397%:%
%:%2780=1397%:%
%:%2781=1398%:%
%:%2782=1398%:%
%:%2783=1398%:%
%:%2784=1399%:%
%:%2785=1399%:%
%:%2786=1400%:%
%:%2787=1400%:%
%:%2788=1400%:%
%:%2789=1401%:%
%:%2790=1401%:%
%:%2791=1402%:%
%:%2792=1402%:%
%:%2793=1403%:%
%:%2794=1403%:%
%:%2795=1403%:%
%:%2796=1404%:%
%:%2797=1404%:%
%:%2798=1405%:%
%:%2799=1405%:%
%:%2800=1405%:%
%:%2801=1406%:%
%:%2802=1406%:%
%:%2803=1407%:%
%:%2804=1407%:%
%:%2805=1407%:%
%:%2806=1408%:%
%:%2807=1408%:%
%:%2808=1408%:%
%:%2809=1409%:%
%:%2810=1409%:%
%:%2811=1410%:%
%:%2812=1410%:%
%:%2813=1410%:%
%:%2814=1411%:%
%:%2815=1411%:%
%:%2816=1412%:%
%:%2817=1412%:%
%:%2818=1413%:%
%:%2828=1415%:%
%:%2828=1416%:%
%:%2829=1417%:%
%:%2830=1418%:%
%:%2831=1419%:%
%:%2832=1420%:%
%:%2833=1421%:%
%:%2834=1422%:%
%:%2835=1423%:%
%:%2836=1424%:%
%:%2837=1425%:%
%:%2838=1426%:%
%:%2839=1427%:%
%:%2840=1428%:%
%:%2841=1429%:%
%:%2842=1430%:%
%:%2843=1431%:%
%:%2844=1432%:%
%:%2845=1433%:%
%:%2846=1434%:%
%:%2847=1435%:%
%:%2848=1436%:%
%:%2849=1437%:%
%:%2850=1438%:%
%:%2851=1439%:%
%:%2852=1440%:%
%:%2853=1441%:%
%:%2854=1442%:%
%:%2855=1443%:%
%:%2856=1444%:%
%:%2857=1445%:%
%:%2858=1446%:%
%:%2859=1447%:%
%:%2860=1448%:%
%:%2861=1449%:%
%:%2862=1450%:%
%:%2863=1451%:%
%:%2864=1452%:%
%:%2865=1453%:%
%:%2866=1454%:%
%:%2867=1455%:%
%:%2869=1457%:%
%:%2870=1457%:%
%:%2871=1458%:%
%:%2872=1459%:%
%:%2873=1460%:%
%:%2874=1460%:%
%:%2875=1461%:%
%:%2876=1462%:%
%:%2879=1463%:%
%:%2883=1463%:%
%:%2884=1463%:%
%:%2885=1463%:%
%:%2886=1463%:%
%:%2887=1463%:%
%:%2892=1463%:%
%:%2895=1464%:%
%:%2896=1465%:%
%:%2897=1465%:%
%:%2898=1466%:%
%:%2899=1467%:%
%:%2900=1468%:%
%:%2901=1469%:%
%:%2908=1470%:%
%:%2909=1470%:%
%:%2910=1471%:%
%:%2911=1471%:%
%:%2912=1472%:%
%:%2913=1472%:%
%:%2914=1472%:%
%:%2915=1472%:%
%:%2916=1473%:%
%:%2917=1473%:%
%:%2918=1473%:%
%:%2919=1474%:%
%:%2920=1474%:%
%:%2921=1474%:%
%:%2922=1475%:%
%:%2923=1475%:%
%:%2924=1476%:%
%:%2925=1476%:%
%:%2926=1476%:%
%:%2927=1477%:%
%:%2933=1477%:%
%:%2936=1478%:%
%:%2937=1479%:%
%:%2938=1479%:%
%:%2945=1480%:%
%:%2946=1480%:%
%:%2947=1481%:%
%:%2948=1481%:%
%:%2949=1482%:%
%:%2950=1482%:%
%:%2951=1482%:%
%:%2952=1483%:%
%:%2953=1483%:%
%:%2954=1484%:%
%:%2955=1484%:%
%:%2956=1484%:%
%:%2957=1485%:%
%:%2958=1485%:%
%:%2959=1486%:%
%:%2960=1486%:%
%:%2961=1487%:%
%:%2962=1487%:%
%:%2963=1487%:%
%:%2964=1488%:%
%:%2965=1488%:%
%:%2966=1489%:%
%:%2967=1489%:%
%:%2968=1490%:%
%:%2969=1490%:%
%:%2970=1491%:%
%:%2971=1491%:%
%:%2972=1491%:%
%:%2973=1492%:%
%:%2974=1492%:%
%:%2975=1493%:%
%:%2976=1493%:%
%:%2977=1494%:%
%:%2983=1494%:%
%:%2986=1495%:%
%:%2987=1496%:%
%:%2988=1496%:%
%:%2989=1497%:%
%:%2990=1498%:%
%:%2997=1499%:%
%:%2998=1499%:%
%:%2999=1500%:%
%:%3000=1500%:%
%:%3001=1501%:%
%:%3002=1501%:%
%:%3003=1501%:%
%:%3004=1502%:%
%:%3005=1502%:%
%:%3006=1503%:%
%:%3007=1503%:%
%:%3008=1504%:%
%:%3009=1504%:%
%:%3010=1505%:%
%:%3011=1505%:%
%:%3012=1506%:%
%:%3013=1506%:%
%:%3014=1506%:%
%:%3015=1507%:%
%:%3016=1507%:%
%:%3017=1507%:%
%:%3018=1508%:%
%:%3019=1508%:%
%:%3020=1509%:%
%:%3021=1509%:%
%:%3022=1510%:%
%:%3023=1510%:%
%:%3024=1511%:%
%:%3025=1511%:%
%:%3026=1512%:%
%:%3027=1512%:%
%:%3028=1512%:%
%:%3029=1513%:%
%:%3030=1513%:%
%:%3031=1514%:%
%:%3032=1514%:%
%:%3033=1515%:%
%:%3034=1515%:%
%:%3035=1515%:%
%:%3036=1516%:%
%:%3037=1516%:%
%:%3038=1516%:%
%:%3039=1517%:%
%:%3045=1517%:%
%:%3048=1518%:%
%:%3049=1519%:%
%:%3050=1519%:%
%:%3057=1520%:%
%:%3058=1520%:%
%:%3059=1521%:%
%:%3060=1521%:%
%:%3061=1522%:%
%:%3062=1522%:%
%:%3063=1522%:%
%:%3064=1523%:%
%:%3065=1523%:%
%:%3066=1524%:%
%:%3067=1524%:%
%:%3068=1524%:%
%:%3069=1525%:%
%:%3070=1525%:%
%:%3071=1526%:%
%:%3072=1526%:%
%:%3073=1527%:%
%:%3074=1527%:%
%:%3075=1527%:%
%:%3076=1528%:%
%:%3077=1528%:%
%:%3078=1529%:%
%:%3079=1529%:%
%:%3080=1530%:%
%:%3081=1530%:%
%:%3082=1531%:%
%:%3083=1531%:%
%:%3084=1531%:%
%:%3085=1532%:%
%:%3086=1532%:%
%:%3087=1533%:%
%:%3088=1533%:%
%:%3089=1534%:%
%:%3090=1534%:%
%:%3091=1535%:%
%:%3092=1535%:%
%:%3093=1535%:%
%:%3094=1536%:%
%:%3095=1536%:%
%:%3096=1536%:%
%:%3097=1537%:%
%:%3098=1537%:%
%:%3099=1538%:%
%:%3100=1538%:%
%:%3101=1539%:%
%:%3102=1539%:%
%:%3103=1539%:%
%:%3104=1540%:%
%:%3110=1540%:%
%:%3113=1541%:%
%:%3114=1542%:%
%:%3115=1542%:%
%:%3116=1543%:%
%:%3117=1544%:%
%:%3124=1545%:%
%:%3125=1545%:%
%:%3126=1546%:%
%:%3127=1546%:%
%:%3128=1547%:%
%:%3129=1547%:%
%:%3130=1548%:%
%:%3131=1548%:%
%:%3132=1549%:%
%:%3133=1549%:%
%:%3134=1550%:%
%:%3135=1550%:%
%:%3136=1551%:%
%:%3137=1551%:%
%:%3138=1552%:%
%:%3139=1552%:%
%:%3140=1553%:%
%:%3141=1553%:%
%:%3142=1554%:%
%:%3143=1554%:%
%:%3144=1554%:%
%:%3145=1555%:%
%:%3146=1555%:%
%:%3147=1555%:%
%:%3148=1556%:%
%:%3149=1556%:%
%:%3150=1557%:%
%:%3151=1557%:%
%:%3152=1558%:%
%:%3153=1558%:%
%:%3154=1558%:%
%:%3155=1559%:%
%:%3156=1559%:%
%:%3157=1559%:%
%:%3158=1560%:%
%:%3159=1560%:%
%:%3160=1561%:%
%:%3161=1561%:%
%:%3162=1562%:%
%:%3163=1562%:%
%:%3164=1562%:%
%:%3165=1563%:%
%:%3166=1563%:%
%:%3167=1563%:%
%:%3168=1564%:%
%:%3169=1564%:%
%:%3170=1565%:%
%:%3171=1565%:%
%:%3172=1566%:%
%:%3173=1566%:%
%:%3174=1567%:%
%:%3175=1567%:%
%:%3176=1568%:%
%:%3177=1568%:%
%:%3178=1568%:%
%:%3179=1569%:%
%:%3180=1569%:%
%:%3181=1570%:%
%:%3182=1570%:%
%:%3183=1571%:%
%:%3184=1571%:%
%:%3185=1572%:%
%:%3186=1572%:%
%:%3187=1572%:%
%:%3188=1573%:%
%:%3189=1573%:%
%:%3190=1574%:%
%:%3196=1574%:%
%:%3199=1575%:%
%:%3200=1576%:%
%:%3201=1576%:%
%:%3202=1577%:%
%:%3203=1578%:%
%:%3204=1579%:%
%:%3205=1580%:%
%:%3206=1581%:%
%:%3213=1582%:%
%:%3214=1582%:%
%:%3215=1583%:%
%:%3216=1583%:%
%:%3217=1584%:%
%:%3218=1584%:%
%:%3219=1585%:%
%:%3220=1585%:%
%:%3221=1586%:%
%:%3222=1586%:%
%:%3223=1586%:%
%:%3224=1587%:%
%:%3225=1587%:%
%:%3226=1587%:%
%:%3227=1588%:%
%:%3228=1588%:%
%:%3229=1589%:%
%:%3230=1589%:%
%:%3231=1590%:%
%:%3232=1590%:%
%:%3233=1590%:%
%:%3234=1591%:%
%:%3235=1591%:%
%:%3236=1591%:%
%:%3237=1592%:%
%:%3238=1592%:%
%:%3239=1593%:%
%:%3240=1593%:%
%:%3241=1593%:%
%:%3242=1594%:%
%:%3243=1594%:%
%:%3244=1595%:%
%:%3245=1595%:%
%:%3246=1596%:%
%:%3247=1596%:%
%:%3248=1596%:%
%:%3249=1597%:%
%:%3255=1597%:%
%:%3258=1598%:%
%:%3259=1599%:%
%:%3260=1599%:%
%:%3261=1600%:%
%:%3262=1601%:%
%:%3263=1602%:%
%:%3266=1603%:%
%:%3270=1603%:%
%:%3271=1603%:%
%:%3272=1603%:%
%:%3277=1603%:%
%:%3280=1604%:%
%:%3281=1605%:%
%:%3282=1605%:%
%:%3283=1606%:%
%:%3284=1607%:%
%:%3285=1608%:%
%:%3286=1609%:%
%:%3287=1610%:%
%:%3288=1611%:%
%:%3295=1612%:%
%:%3296=1612%:%
%:%3297=1613%:%
%:%3298=1613%:%
%:%3299=1614%:%
%:%3300=1614%:%
%:%3301=1614%:%
%:%3302=1615%:%
%:%3303=1615%:%
%:%3304=1616%:%
%:%3305=1616%:%
%:%3306=1617%:%
%:%3307=1617%:%
%:%3308=1618%:%
%:%3309=1618%:%
%:%3310=1618%:%
%:%3311=1619%:%
%:%3312=1619%:%
%:%3313=1620%:%
%:%3314=1620%:%
%:%3315=1620%:%
%:%3316=1621%:%
%:%3317=1621%:%
%:%3318=1622%:%
%:%3319=1622%:%
%:%3320=1622%:%
%:%3321=1623%:%
%:%3322=1623%:%
%:%3323=1623%:%
%:%3324=1624%:%
%:%3325=1624%:%
%:%3326=1624%:%
%:%3327=1625%:%
%:%3328=1625%:%
%:%3329=1625%:%
%:%3330=1626%:%
%:%3331=1626%:%
%:%3332=1627%:%
%:%3333=1627%:%
%:%3334=1627%:%
%:%3335=1628%:%
%:%3336=1628%:%
%:%3337=1629%:%
%:%3338=1629%:%
%:%3339=1629%:%
%:%3340=1630%:%
%:%3341=1630%:%
%:%3342=1631%:%
%:%3343=1631%:%
%:%3344=1632%:%
%:%3345=1632%:%
%:%3346=1632%:%
%:%3347=1633%:%
%:%3348=1633%:%
%:%3349=1633%:%
%:%3350=1634%:%
%:%3351=1634%:%
%:%3352=1635%:%
%:%3353=1635%:%
%:%3354=1636%:%
%:%3355=1636%:%
%:%3356=1636%:%
%:%3357=1637%:%
%:%3358=1637%:%
%:%3359=1637%:%
%:%3360=1638%:%
%:%3361=1638%:%
%:%3362=1639%:%
%:%3363=1639%:%
%:%3364=1639%:%
%:%3365=1640%:%
%:%3366=1640%:%
%:%3367=1641%:%
%:%3368=1641%:%
%:%3369=1642%:%
%:%3370=1642%:%
%:%3371=1643%:%
%:%3377=1643%:%
%:%3380=1644%:%
%:%3381=1645%:%
%:%3382=1645%:%
%:%3383=1646%:%
%:%3384=1647%:%
%:%3385=1648%:%
%:%3386=1649%:%
%:%3387=1650%:%
%:%3388=1651%:%
%:%3395=1652%:%
%:%3396=1652%:%
%:%3397=1653%:%
%:%3398=1653%:%
%:%3399=1654%:%
%:%3400=1654%:%
%:%3401=1654%:%
%:%3402=1655%:%
%:%3403=1655%:%
%:%3404=1656%:%
%:%3405=1656%:%
%:%3406=1657%:%
%:%3407=1657%:%
%:%3408=1658%:%
%:%3409=1658%:%
%:%3410=1658%:%
%:%3411=1659%:%
%:%3412=1659%:%
%:%3413=1660%:%
%:%3414=1660%:%
%:%3415=1660%:%
%:%3416=1661%:%
%:%3417=1661%:%
%:%3418=1662%:%
%:%3419=1662%:%
%:%3420=1662%:%
%:%3421=1663%:%
%:%3422=1663%:%
%:%3423=1663%:%
%:%3424=1664%:%
%:%3425=1664%:%
%:%3426=1664%:%
%:%3427=1665%:%
%:%3428=1665%:%
%:%3429=1665%:%
%:%3430=1666%:%
%:%3431=1666%:%
%:%3432=1667%:%
%:%3433=1667%:%
%:%3434=1667%:%
%:%3435=1668%:%
%:%3436=1668%:%
%:%3437=1669%:%
%:%3438=1669%:%
%:%3439=1669%:%
%:%3440=1670%:%
%:%3441=1670%:%
%:%3442=1671%:%
%:%3443=1671%:%
%:%3444=1671%:%
%:%3445=1672%:%
%:%3446=1672%:%
%:%3447=1673%:%
%:%3448=1673%:%
%:%3449=1673%:%
%:%3450=1674%:%
%:%3451=1674%:%
%:%3452=1675%:%
%:%3453=1675%:%
%:%3454=1675%:%
%:%3455=1676%:%
%:%3456=1676%:%
%:%3457=1677%:%
%:%3458=1677%:%
%:%3459=1678%:%
%:%3460=1678%:%
%:%3461=1678%:%
%:%3462=1679%:%
%:%3463=1679%:%
%:%3464=1679%:%
%:%3465=1680%:%
%:%3466=1680%:%
%:%3467=1681%:%
%:%3468=1681%:%
%:%3469=1682%:%
%:%3470=1682%:%
%:%3471=1682%:%
%:%3472=1683%:%
%:%3473=1683%:%
%:%3474=1683%:%
%:%3475=1684%:%
%:%3476=1684%:%
%:%3477=1685%:%
%:%3478=1685%:%
%:%3479=1685%:%
%:%3480=1686%:%
%:%3481=1686%:%
%:%3482=1687%:%
%:%3483=1687%:%
%:%3484=1688%:%
%:%3485=1688%:%
%:%3486=1689%:%
%:%3492=1689%:%
%:%3495=1690%:%
%:%3496=1691%:%
%:%3497=1691%:%
%:%3498=1692%:%
%:%3499=1693%:%
%:%3500=1694%:%
%:%3501=1695%:%
%:%3502=1696%:%
%:%3503=1697%:%
%:%3510=1698%:%
%:%3511=1698%:%
%:%3512=1699%:%
%:%3513=1699%:%
%:%3514=1700%:%
%:%3515=1700%:%
%:%3516=1700%:%
%:%3517=1701%:%
%:%3518=1701%:%
%:%3519=1702%:%
%:%3520=1702%:%
%:%3521=1703%:%
%:%3522=1703%:%
%:%3523=1704%:%
%:%3524=1704%:%
%:%3525=1704%:%
%:%3526=1705%:%
%:%3527=1705%:%
%:%3528=1706%:%
%:%3529=1706%:%
%:%3530=1706%:%
%:%3531=1707%:%
%:%3532=1707%:%
%:%3533=1708%:%
%:%3534=1708%:%
%:%3535=1708%:%
%:%3536=1709%:%
%:%3537=1709%:%
%:%3538=1709%:%
%:%3539=1710%:%
%:%3540=1710%:%
%:%3541=1710%:%
%:%3542=1711%:%
%:%3543=1711%:%
%:%3544=1711%:%
%:%3545=1712%:%
%:%3546=1712%:%
%:%3547=1713%:%
%:%3548=1713%:%
%:%3549=1713%:%
%:%3550=1714%:%
%:%3551=1714%:%
%:%3552=1715%:%
%:%3553=1715%:%
%:%3554=1715%:%
%:%3555=1716%:%
%:%3556=1716%:%
%:%3557=1717%:%
%:%3558=1717%:%
%:%3559=1717%:%
%:%3560=1718%:%
%:%3561=1718%:%
%:%3562=1719%:%
%:%3563=1719%:%
%:%3564=1719%:%
%:%3565=1720%:%
%:%3566=1720%:%
%:%3567=1721%:%
%:%3568=1721%:%
%:%3569=1721%:%
%:%3570=1722%:%
%:%3571=1722%:%
%:%3572=1723%:%
%:%3573=1723%:%
%:%3574=1724%:%
%:%3575=1724%:%
%:%3576=1724%:%
%:%3577=1725%:%
%:%3578=1725%:%
%:%3579=1725%:%
%:%3580=1726%:%
%:%3581=1726%:%
%:%3582=1727%:%
%:%3583=1727%:%
%:%3584=1728%:%
%:%3585=1728%:%
%:%3586=1728%:%
%:%3587=1729%:%
%:%3588=1729%:%
%:%3589=1729%:%
%:%3590=1730%:%
%:%3591=1730%:%
%:%3592=1731%:%
%:%3593=1731%:%
%:%3594=1731%:%
%:%3595=1732%:%
%:%3596=1732%:%
%:%3597=1733%:%
%:%3598=1733%:%
%:%3599=1734%:%
%:%3600=1734%:%
%:%3601=1735%:%
%:%3607=1735%:%
%:%3610=1736%:%
%:%3611=1737%:%
%:%3612=1737%:%
%:%3613=1738%:%
%:%3614=1739%:%
%:%3615=1740%:%
%:%3616=1741%:%
%:%3617=1742%:%
%:%3618=1743%:%
%:%3619=1744%:%
%:%3626=1745%:%
%:%3627=1745%:%
%:%3628=1746%:%
%:%3629=1746%:%
%:%3630=1747%:%
%:%3631=1747%:%
%:%3632=1747%:%
%:%3633=1748%:%
%:%3634=1748%:%
%:%3635=1749%:%
%:%3636=1749%:%
%:%3637=1750%:%
%:%3638=1750%:%
%:%3639=1751%:%
%:%3640=1751%:%
%:%3641=1751%:%
%:%3642=1752%:%
%:%3643=1752%:%
%:%3644=1753%:%
%:%3645=1753%:%
%:%3646=1753%:%
%:%3647=1754%:%
%:%3648=1754%:%
%:%3649=1755%:%
%:%3650=1755%:%
%:%3651=1755%:%
%:%3652=1756%:%
%:%3653=1756%:%
%:%3654=1756%:%
%:%3655=1757%:%
%:%3656=1757%:%
%:%3657=1757%:%
%:%3658=1758%:%
%:%3659=1758%:%
%:%3660=1758%:%
%:%3661=1759%:%
%:%3662=1759%:%
%:%3663=1760%:%
%:%3664=1760%:%
%:%3665=1760%:%
%:%3666=1761%:%
%:%3667=1761%:%
%:%3668=1762%:%
%:%3669=1762%:%
%:%3670=1762%:%
%:%3671=1763%:%
%:%3672=1763%:%
%:%3673=1764%:%
%:%3674=1764%:%
%:%3675=1764%:%
%:%3676=1765%:%
%:%3677=1765%:%
%:%3678=1766%:%
%:%3679=1766%:%
%:%3680=1767%:%
%:%3681=1767%:%
%:%3682=1767%:%
%:%3683=1768%:%
%:%3684=1768%:%
%:%3685=1768%:%
%:%3686=1769%:%
%:%3687=1769%:%
%:%3688=1770%:%
%:%3689=1770%:%
%:%3690=1771%:%
%:%3691=1771%:%
%:%3692=1771%:%
%:%3693=1772%:%
%:%3694=1772%:%
%:%3695=1772%:%
%:%3696=1773%:%
%:%3697=1773%:%
%:%3698=1774%:%
%:%3699=1774%:%
%:%3700=1774%:%
%:%3701=1775%:%
%:%3702=1775%:%
%:%3703=1776%:%
%:%3704=1776%:%
%:%3705=1777%:%
%:%3706=1777%:%
%:%3707=1778%:%
%:%3713=1778%:%
%:%3716=1779%:%
%:%3717=1780%:%
%:%3718=1780%:%
%:%3719=1781%:%
%:%3720=1782%:%
%:%3721=1783%:%
%:%3722=1784%:%
%:%3723=1785%:%
%:%3724=1786%:%
%:%3731=1787%:%
%:%3732=1787%:%
%:%3733=1788%:%
%:%3734=1788%:%
%:%3735=1789%:%
%:%3736=1789%:%
%:%3737=1790%:%
%:%3738=1790%:%
%:%3741=1793%:%
%:%3742=1794%:%
%:%3743=1794%:%
%:%3744=1794%:%
%:%3745=1795%:%
%:%3746=1795%:%
%:%3747=1795%:%
%:%3748=1796%:%
%:%3749=1796%:%
%:%3750=1797%:%
%:%3751=1797%:%
%:%3752=1798%:%
%:%3753=1798%:%
%:%3754=1799%:%
%:%3755=1799%:%
%:%3756=1799%:%
%:%3757=1800%:%
%:%3758=1800%:%
%:%3759=1801%:%
%:%3760=1801%:%
%:%3762=1803%:%
%:%3763=1804%:%
%:%3764=1804%:%
%:%3765=1805%:%
%:%3766=1805%:%
%:%3767=1806%:%
%:%3768=1806%:%
%:%3769=1807%:%
%:%3770=1807%:%
%:%3771=1808%:%
%:%3772=1808%:%
%:%3773=1808%:%
%:%3774=1809%:%
%:%3775=1809%:%
%:%3776=1810%:%
%:%3777=1810%:%
%:%3778=1810%:%
%:%3779=1811%:%
%:%3780=1811%:%
%:%3781=1812%:%
%:%3782=1812%:%
%:%3783=1812%:%
%:%3784=1813%:%
%:%3785=1813%:%
%:%3786=1814%:%
%:%3787=1814%:%
%:%3788=1814%:%
%:%3789=1815%:%
%:%3790=1815%:%
%:%3791=1816%:%
%:%3792=1816%:%
%:%3793=1816%:%
%:%3794=1817%:%
%:%3795=1817%:%
%:%3796=1818%:%
%:%3797=1818%:%
%:%3798=1819%:%
%:%3799=1819%:%
%:%3800=1820%:%
%:%3801=1820%:%
%:%3802=1821%:%
%:%3803=1822%:%
%:%3804=1822%:%
%:%3805=1823%:%
%:%3806=1823%:%
%:%3807=1824%:%
%:%3808=1824%:%
%:%3809=1825%:%
%:%3810=1825%:%
%:%3811=1826%:%
%:%3812=1826%:%
%:%3813=1826%:%
%:%3814=1827%:%
%:%3815=1827%:%
%:%3816=1828%:%
%:%3817=1828%:%
%:%3818=1828%:%
%:%3819=1829%:%
%:%3820=1829%:%
%:%3821=1830%:%
%:%3822=1830%:%
%:%3823=1830%:%
%:%3824=1831%:%
%:%3825=1831%:%
%:%3826=1832%:%
%:%3827=1832%:%
%:%3828=1832%:%
%:%3829=1833%:%
%:%3830=1833%:%
%:%3831=1834%:%
%:%3832=1834%:%
%:%3833=1835%:%
%:%3834=1835%:%
%:%3835=1836%:%
%:%3836=1836%:%
%:%3837=1837%:%
%:%3838=1837%:%
%:%3839=1837%:%
%:%3840=1838%:%
%:%3841=1838%:%
%:%3842=1839%:%
%:%3843=1839%:%
%:%3844=1840%:%
%:%3845=1840%:%
%:%3846=1840%:%
%:%3847=1841%:%
%:%3848=1841%:%
%:%3849=1842%:%
%:%3850=1842%:%
%:%3851=1842%:%
%:%3852=1843%:%
%:%3853=1843%:%
%:%3854=1844%:%
%:%3855=1844%:%
%:%3856=1845%:%
%:%3857=1845%:%
%:%3858=1846%:%
%:%3859=1846%:%
%:%3860=1847%:%
%:%3861=1847%:%
%:%3862=1847%:%
%:%3863=1848%:%
%:%3869=1848%:%
%:%3872=1849%:%
%:%3873=1850%:%
%:%3874=1850%:%
%:%3875=1851%:%
%:%3876=1852%:%
%:%3883=1853%:%
%:%3884=1853%:%
%:%3885=1854%:%
%:%3886=1854%:%
%:%3887=1855%:%
%:%3888=1855%:%
%:%3889=1856%:%
%:%3890=1856%:%
%:%3891=1857%:%
%:%3892=1857%:%
%:%3893=1858%:%
%:%3894=1858%:%
%:%3895=1859%:%
%:%3896=1859%:%
%:%3897=1860%:%
%:%3898=1860%:%
%:%3899=1861%:%
%:%3900=1861%:%
%:%3901=1862%:%
%:%3902=1862%:%
%:%3903=1863%:%
%:%3904=1863%:%
%:%3905=1864%:%
%:%3906=1864%:%
%:%3907=1865%:%
%:%3908=1865%:%
%:%3909=1866%:%
%:%3910=1866%:%
%:%3911=1866%:%
%:%3912=1867%:%
%:%3913=1867%:%
%:%3914=1868%:%
%:%3915=1868%:%
%:%3916=1869%:%
%:%3917=1869%:%
%:%3918=1870%:%
%:%3919=1870%:%
%:%3920=1871%:%
%:%3921=1871%:%
%:%3922=1872%:%
%:%3923=1872%:%
%:%3924=1872%:%
%:%3925=1873%:%
%:%3926=1873%:%
%:%3927=1874%:%
%:%3928=1874%:%
%:%3929=1874%:%
%:%3930=1875%:%
%:%3931=1875%:%
%:%3932=1876%:%
%:%3933=1876%:%
%:%3934=1876%:%
%:%3935=1877%:%
%:%3936=1877%:%
%:%3937=1878%:%
%:%3938=1878%:%
%:%3939=1878%:%
%:%3940=1879%:%
%:%3941=1879%:%
%:%3942=1880%:%
%:%3943=1880%:%
%:%3944=1880%:%
%:%3945=1881%:%
%:%3946=1881%:%
%:%3947=1882%:%
%:%3948=1882%:%
%:%3949=1883%:%
%:%3950=1883%:%
%:%3951=1884%:%
%:%3952=1884%:%
%:%3953=1885%:%
%:%3954=1885%:%
%:%3955=1885%:%
%:%3956=1886%:%
%:%3957=1886%:%
%:%3958=1887%:%
%:%3959=1887%:%
%:%3960=1888%:%
%:%3961=1888%:%
%:%3962=1889%:%
%:%3963=1889%:%
%:%3964=1890%:%
%:%3965=1890%:%
%:%3966=1891%:%
%:%3967=1891%:%
%:%3968=1891%:%
%:%3969=1892%:%
%:%3970=1892%:%
%:%3971=1893%:%
%:%3972=1893%:%
%:%3973=1894%:%
%:%3974=1894%:%
%:%3975=1894%:%
%:%3976=1895%:%
%:%3977=1895%:%
%:%3978=1896%:%
%:%3979=1896%:%
%:%3980=1897%:%
%:%3981=1897%:%
%:%3982=1898%:%
%:%3983=1898%:%
%:%3984=1899%:%
%:%3985=1899%:%
%:%3986=1900%:%
%:%3987=1900%:%
%:%3988=1900%:%
%:%3989=1901%:%
%:%3990=1901%:%
%:%3991=1902%:%
%:%3992=1902%:%
%:%3993=1903%:%
%:%3994=1903%:%
%:%3995=1903%:%
%:%3996=1904%:%
%:%3997=1904%:%
%:%3998=1905%:%
%:%3999=1905%:%
%:%4000=1906%:%
%:%4001=1906%:%
%:%4002=1906%:%
%:%4003=1907%:%
%:%4004=1907%:%
%:%4005=1908%:%
%:%4006=1908%:%
%:%4007=1909%:%
%:%4008=1909%:%
%:%4009=1909%:%
%:%4010=1910%:%
%:%4011=1910%:%
%:%4012=1911%:%
%:%4013=1911%:%
%:%4014=1912%:%
%:%4015=1912%:%
%:%4016=1913%:%
%:%4017=1913%:%
%:%4018=1914%:%
%:%4019=1914%:%
%:%4020=1915%:%
%:%4021=1915%:%
%:%4022=1916%:%
%:%4023=1916%:%
%:%4024=1917%:%
%:%4030=1917%:%
%:%4033=1918%:%
%:%4034=1919%:%
%:%4035=1919%:%
%:%4036=1920%:%
%:%4037=1921%:%
%:%4038=1922%:%
%:%4039=1923%:%
%:%4040=1924%:%
%:%4041=1925%:%
%:%4048=1926%:%
%:%4049=1926%:%
%:%4050=1927%:%
%:%4051=1927%:%
%:%4052=1928%:%
%:%4053=1928%:%
%:%4054=1928%:%
%:%4055=1929%:%
%:%4056=1929%:%
%:%4057=1930%:%
%:%4058=1930%:%
%:%4059=1931%:%
%:%4060=1931%:%
%:%4061=1932%:%
%:%4062=1932%:%
%:%4063=1932%:%
%:%4064=1933%:%
%:%4065=1933%:%
%:%4066=1934%:%
%:%4067=1934%:%
%:%4068=1934%:%
%:%4069=1935%:%
%:%4070=1935%:%
%:%4071=1936%:%
%:%4072=1936%:%
%:%4073=1936%:%
%:%4074=1937%:%
%:%4075=1937%:%
%:%4076=1937%:%
%:%4077=1938%:%
%:%4078=1938%:%
%:%4079=1938%:%
%:%4080=1939%:%
%:%4081=1939%:%
%:%4082=1939%:%
%:%4083=1940%:%
%:%4084=1940%:%
%:%4085=1941%:%
%:%4086=1941%:%
%:%4087=1942%:%
%:%4088=1942%:%
%:%4089=1943%:%
%:%4090=1943%:%
%:%4091=1944%:%
%:%4092=1944%:%
%:%4093=1944%:%
%:%4094=1945%:%
%:%4095=1945%:%
%:%4096=1946%:%
%:%4097=1946%:%
%:%4098=1946%:%
%:%4099=1947%:%
%:%4100=1947%:%
%:%4101=1947%:%
%:%4102=1948%:%
%:%4103=1948%:%
%:%4104=1948%:%
%:%4105=1949%:%
%:%4106=1949%:%
%:%4107=1950%:%
%:%4108=1950%:%
%:%4109=1950%:%
%:%4110=1951%:%
%:%4111=1951%:%
%:%4112=1952%:%
%:%4113=1952%:%
%:%4114=1952%:%
%:%4115=1953%:%
%:%4116=1953%:%
%:%4117=1954%:%
%:%4118=1954%:%
%:%4119=1955%:%
%:%4120=1955%:%
%:%4121=1956%:%
%:%4122=1956%:%
%:%4123=1957%:%
%:%4124=1957%:%
%:%4125=1958%:%
%:%4126=1958%:%
%:%4127=1958%:%
%:%4128=1959%:%
%:%4129=1959%:%
%:%4130=1960%:%
%:%4131=1960%:%
%:%4132=1960%:%
%:%4133=1961%:%
%:%4134=1961%:%
%:%4135=1961%:%
%:%4136=1962%:%
%:%4137=1962%:%
%:%4138=1962%:%
%:%4139=1963%:%
%:%4140=1963%:%
%:%4141=1964%:%
%:%4142=1964%:%
%:%4143=1964%:%
%:%4144=1965%:%
%:%4145=1965%:%
%:%4146=1966%:%
%:%4147=1966%:%
%:%4148=1966%:%
%:%4149=1967%:%
%:%4150=1967%:%
%:%4151=1968%:%
%:%4152=1968%:%
%:%4153=1969%:%
%:%4154=1969%:%
%:%4155=1970%:%
%:%4156=1970%:%
%:%4157=1971%:%
%:%4163=1971%:%
%:%4166=1972%:%
%:%4167=1973%:%
%:%4168=1973%:%
%:%4169=1974%:%
%:%4170=1975%:%
%:%4171=1976%:%
%:%4172=1977%:%
%:%4173=1978%:%
%:%4174=1979%:%
%:%4181=1980%:%
%:%4182=1980%:%
%:%4183=1981%:%
%:%4184=1981%:%
%:%4185=1982%:%
%:%4186=1982%:%
%:%4187=1982%:%
%:%4188=1983%:%
%:%4189=1983%:%
%:%4190=1984%:%
%:%4191=1984%:%
%:%4192=1985%:%
%:%4193=1985%:%
%:%4194=1986%:%
%:%4195=1986%:%
%:%4196=1986%:%
%:%4197=1987%:%
%:%4198=1987%:%
%:%4199=1988%:%
%:%4200=1988%:%
%:%4201=1988%:%
%:%4202=1989%:%
%:%4203=1989%:%
%:%4204=1990%:%
%:%4205=1990%:%
%:%4206=1990%:%
%:%4207=1991%:%
%:%4208=1991%:%
%:%4209=1991%:%
%:%4210=1992%:%
%:%4211=1992%:%
%:%4212=1992%:%
%:%4213=1993%:%
%:%4214=1993%:%
%:%4215=1993%:%
%:%4216=1994%:%
%:%4217=1994%:%
%:%4218=1995%:%
%:%4219=1995%:%
%:%4220=1995%:%
%:%4221=1996%:%
%:%4222=1996%:%
%:%4223=1997%:%
%:%4224=1997%:%
%:%4225=1997%:%
%:%4226=1998%:%
%:%4227=1998%:%
%:%4228=1999%:%
%:%4229=1999%:%
%:%4230=2000%:%
%:%4231=2000%:%
%:%4232=2001%:%
%:%4233=2001%:%
%:%4234=2002%:%
%:%4235=2002%:%
%:%4236=2002%:%
%:%4237=2003%:%
%:%4238=2003%:%
%:%4239=2004%:%
%:%4240=2004%:%
%:%4241=2004%:%
%:%4242=2005%:%
%:%4243=2005%:%
%:%4244=2006%:%
%:%4245=2006%:%
%:%4246=2006%:%
%:%4247=2007%:%
%:%4248=2007%:%
%:%4249=2007%:%
%:%4250=2008%:%
%:%4251=2008%:%
%:%4252=2008%:%
%:%4253=2009%:%
%:%4254=2009%:%
%:%4255=2010%:%
%:%4256=2010%:%
%:%4257=2010%:%
%:%4258=2011%:%
%:%4259=2011%:%
%:%4260=2012%:%
%:%4261=2012%:%
%:%4262=2012%:%
%:%4263=2013%:%
%:%4264=2013%:%
%:%4265=2014%:%
%:%4266=2014%:%
%:%4267=2015%:%
%:%4268=2015%:%
%:%4269=2016%:%
%:%4270=2016%:%
%:%4271=2017%:%
%:%4272=2017%:%
%:%4273=2018%:%
%:%4274=2018%:%
%:%4275=2018%:%
%:%4276=2019%:%
%:%4277=2019%:%
%:%4278=2020%:%
%:%4279=2020%:%
%:%4280=2020%:%
%:%4281=2021%:%
%:%4282=2021%:%
%:%4283=2021%:%
%:%4284=2022%:%
%:%4285=2022%:%
%:%4286=2022%:%
%:%4287=2023%:%
%:%4288=2023%:%
%:%4289=2024%:%
%:%4290=2024%:%
%:%4291=2024%:%
%:%4292=2025%:%
%:%4293=2025%:%
%:%4294=2026%:%
%:%4295=2026%:%
%:%4296=2026%:%
%:%4297=2027%:%
%:%4298=2027%:%
%:%4299=2028%:%
%:%4300=2028%:%
%:%4301=2029%:%
%:%4302=2029%:%
%:%4303=2030%:%
%:%4304=2030%:%
%:%4305=2031%:%
%:%4311=2031%:%
%:%4314=2032%:%
%:%4315=2033%:%
%:%4316=2033%:%
%:%4317=2034%:%
%:%4318=2035%:%
%:%4319=2036%:%
%:%4320=2037%:%
%:%4321=2038%:%
%:%4322=2039%:%
%:%4329=2040%:%
%:%4330=2040%:%
%:%4331=2041%:%
%:%4332=2041%:%
%:%4333=2042%:%
%:%4334=2042%:%
%:%4335=2042%:%
%:%4336=2043%:%
%:%4337=2043%:%
%:%4338=2044%:%
%:%4339=2044%:%
%:%4340=2045%:%
%:%4341=2045%:%
%:%4342=2046%:%
%:%4343=2046%:%
%:%4344=2046%:%
%:%4345=2047%:%
%:%4346=2047%:%
%:%4347=2048%:%
%:%4348=2048%:%
%:%4349=2048%:%
%:%4350=2049%:%
%:%4351=2049%:%
%:%4352=2050%:%
%:%4353=2050%:%
%:%4354=2050%:%
%:%4355=2051%:%
%:%4356=2051%:%
%:%4357=2051%:%
%:%4358=2052%:%
%:%4359=2052%:%
%:%4360=2052%:%
%:%4361=2053%:%
%:%4362=2053%:%
%:%4363=2053%:%
%:%4364=2054%:%
%:%4365=2054%:%
%:%4366=2055%:%
%:%4367=2055%:%
%:%4368=2055%:%
%:%4369=2056%:%
%:%4370=2056%:%
%:%4371=2057%:%
%:%4372=2057%:%
%:%4373=2057%:%
%:%4374=2058%:%
%:%4375=2058%:%
%:%4376=2059%:%
%:%4377=2059%:%
%:%4378=2060%:%
%:%4379=2060%:%
%:%4380=2061%:%
%:%4381=2061%:%
%:%4382=2062%:%
%:%4383=2062%:%
%:%4384=2062%:%
%:%4385=2063%:%
%:%4386=2063%:%
%:%4387=2064%:%
%:%4388=2064%:%
%:%4389=2064%:%
%:%4390=2065%:%
%:%4391=2065%:%
%:%4392=2066%:%
%:%4393=2066%:%
%:%4394=2066%:%
%:%4395=2067%:%
%:%4396=2067%:%
%:%4397=2067%:%
%:%4398=2068%:%
%:%4399=2068%:%
%:%4400=2068%:%
%:%4401=2069%:%
%:%4402=2069%:%
%:%4403=2070%:%
%:%4404=2070%:%
%:%4405=2070%:%
%:%4406=2071%:%
%:%4407=2071%:%
%:%4408=2072%:%
%:%4409=2072%:%
%:%4410=2072%:%
%:%4411=2073%:%
%:%4412=2073%:%
%:%4413=2074%:%
%:%4414=2074%:%
%:%4415=2075%:%
%:%4416=2075%:%
%:%4417=2076%:%
%:%4418=2076%:%
%:%4419=2077%:%
%:%4420=2077%:%
%:%4421=2078%:%
%:%4422=2078%:%
%:%4423=2078%:%
%:%4424=2079%:%
%:%4425=2079%:%
%:%4426=2080%:%
%:%4427=2080%:%
%:%4428=2080%:%
%:%4429=2081%:%
%:%4430=2081%:%
%:%4431=2082%:%
%:%4432=2082%:%
%:%4433=2082%:%
%:%4434=2083%:%
%:%4435=2083%:%
%:%4436=2083%:%
%:%4437=2084%:%
%:%4438=2084%:%
%:%4439=2084%:%
%:%4440=2085%:%
%:%4441=2085%:%
%:%4442=2086%:%
%:%4443=2086%:%
%:%4444=2086%:%
%:%4445=2087%:%
%:%4446=2087%:%
%:%4447=2088%:%
%:%4448=2088%:%
%:%4449=2088%:%
%:%4450=2089%:%
%:%4451=2089%:%
%:%4452=2090%:%
%:%4453=2090%:%
%:%4454=2091%:%
%:%4455=2091%:%
%:%4456=2092%:%
%:%4457=2092%:%
%:%4458=2093%:%
%:%4464=2093%:%
%:%4467=2094%:%
%:%4468=2095%:%
%:%4469=2095%:%
%:%4470=2096%:%
%:%4471=2097%:%
%:%4472=2098%:%
%:%4473=2099%:%
%:%4474=2100%:%
%:%4475=2101%:%
%:%4476=2102%:%
%:%4483=2103%:%
%:%4484=2103%:%
%:%4485=2104%:%
%:%4486=2104%:%
%:%4487=2105%:%
%:%4488=2105%:%
%:%4489=2105%:%
%:%4490=2106%:%
%:%4491=2106%:%
%:%4492=2107%:%
%:%4493=2107%:%
%:%4494=2108%:%
%:%4495=2108%:%
%:%4496=2109%:%
%:%4497=2109%:%
%:%4498=2109%:%
%:%4499=2110%:%
%:%4500=2110%:%
%:%4501=2111%:%
%:%4502=2111%:%
%:%4503=2111%:%
%:%4504=2112%:%
%:%4505=2112%:%
%:%4506=2113%:%
%:%4507=2113%:%
%:%4508=2113%:%
%:%4509=2114%:%
%:%4510=2114%:%
%:%4511=2114%:%
%:%4512=2115%:%
%:%4513=2115%:%
%:%4514=2115%:%
%:%4515=2116%:%
%:%4516=2116%:%
%:%4517=2116%:%
%:%4518=2117%:%
%:%4519=2117%:%
%:%4520=2118%:%
%:%4521=2118%:%
%:%4522=2118%:%
%:%4523=2119%:%
%:%4524=2119%:%
%:%4525=2120%:%
%:%4526=2120%:%
%:%4527=2120%:%
%:%4528=2121%:%
%:%4529=2121%:%
%:%4530=2122%:%
%:%4531=2122%:%
%:%4532=2122%:%
%:%4533=2123%:%
%:%4534=2123%:%
%:%4535=2124%:%
%:%4536=2124%:%
%:%4537=2124%:%
%:%4538=2125%:%
%:%4539=2125%:%
%:%4540=2125%:%
%:%4541=2126%:%
%:%4542=2126%:%
%:%4543=2126%:%
%:%4544=2127%:%
%:%4545=2127%:%
%:%4546=2128%:%
%:%4547=2128%:%
%:%4548=2128%:%
%:%4549=2129%:%
%:%4550=2129%:%
%:%4551=2130%:%
%:%4552=2130%:%
%:%4553=2130%:%
%:%4554=2131%:%
%:%4555=2131%:%
%:%4556=2132%:%
%:%4557=2132%:%
%:%4558=2133%:%
%:%4559=2133%:%
%:%4560=2134%:%
%:%4566=2134%:%
%:%4569=2135%:%
%:%4570=2136%:%
%:%4571=2136%:%
%:%4572=2137%:%
%:%4573=2138%:%
%:%4574=2139%:%
%:%4575=2140%:%
%:%4576=2141%:%
%:%4577=2142%:%
%:%4584=2143%:%
%:%4585=2143%:%
%:%4586=2144%:%
%:%4587=2144%:%
%:%4590=2147%:%
%:%4591=2148%:%
%:%4592=2148%:%
%:%4593=2148%:%
%:%4594=2149%:%
%:%4595=2149%:%
%:%4596=2150%:%
%:%4597=2150%:%
%:%4598=2151%:%
%:%4599=2151%:%
%:%4600=2152%:%
%:%4601=2152%:%
%:%4602=2153%:%
%:%4603=2153%:%
%:%4604=2153%:%
%:%4605=2154%:%
%:%4606=2154%:%
%:%4607=2155%:%
%:%4608=2155%:%
%:%4609=2156%:%
%:%4610=2156%:%
%:%4611=2157%:%
%:%4612=2157%:%
%:%4613=2158%:%
%:%4614=2158%:%
%:%4615=2159%:%
%:%4616=2159%:%
%:%4617=2159%:%
%:%4618=2160%:%
%:%4619=2160%:%
%:%4620=2160%:%
%:%4621=2161%:%
%:%4622=2161%:%
%:%4623=2162%:%
%:%4624=2162%:%
%:%4625=2163%:%
%:%4626=2163%:%
%:%4627=2164%:%
%:%4628=2164%:%
%:%4629=2165%:%
%:%4630=2165%:%
%:%4631=2166%:%
%:%4632=2166%:%
%:%4633=2167%:%
%:%4634=2167%:%
%:%4635=2167%:%
%:%4636=2168%:%
%:%4637=2168%:%
%:%4638=2168%:%
%:%4639=2169%:%
%:%4640=2169%:%
%:%4641=2170%:%
%:%4642=2170%:%
%:%4643=2171%:%
%:%4644=2171%:%
%:%4645=2172%:%
%:%4646=2172%:%
%:%4647=2173%:%
%:%4648=2173%:%
%:%4650=2175%:%
%:%4651=2176%:%
%:%4652=2176%:%
%:%4653=2177%:%
%:%4654=2177%:%
%:%4655=2178%:%
%:%4656=2178%:%
%:%4657=2179%:%
%:%4658=2179%:%
%:%4659=2180%:%
%:%4660=2180%:%
%:%4661=2180%:%
%:%4662=2181%:%
%:%4663=2181%:%
%:%4664=2182%:%
%:%4665=2182%:%
%:%4666=2182%:%
%:%4667=2183%:%
%:%4668=2183%:%
%:%4669=2184%:%
%:%4670=2184%:%
%:%4671=2184%:%
%:%4672=2185%:%
%:%4673=2185%:%
%:%4674=2186%:%
%:%4675=2186%:%
%:%4676=2186%:%
%:%4677=2187%:%
%:%4678=2187%:%
%:%4679=2188%:%
%:%4680=2188%:%
%:%4681=2188%:%
%:%4682=2189%:%
%:%4683=2189%:%
%:%4684=2190%:%
%:%4685=2190%:%
%:%4686=2191%:%
%:%4687=2191%:%
%:%4688=2192%:%
%:%4689=2192%:%
%:%4690=2193%:%
%:%4691=2193%:%
%:%4692=2194%:%
%:%4693=2194%:%
%:%4694=2194%:%
%:%4695=2195%:%
%:%4696=2195%:%
%:%4697=2195%:%
%:%4698=2196%:%
%:%4699=2196%:%
%:%4700=2196%:%
%:%4701=2197%:%
%:%4702=2197%:%
%:%4703=2198%:%
%:%4704=2198%:%
%:%4705=2199%:%
%:%4706=2199%:%
%:%4707=2200%:%
%:%4708=2200%:%
%:%4709=2201%:%
%:%4710=2201%:%
%:%4711=2202%:%
%:%4712=2202%:%
%:%4713=2203%:%
%:%4714=2203%:%
%:%4715=2204%:%
%:%4716=2204%:%
%:%4717=2204%:%
%:%4718=2205%:%
%:%4719=2205%:%
%:%4720=2205%:%
%:%4721=2206%:%
%:%4722=2206%:%
%:%4723=2206%:%
%:%4724=2207%:%
%:%4725=2207%:%
%:%4726=2208%:%
%:%4727=2208%:%
%:%4728=2209%:%
%:%4729=2209%:%
%:%4730=2210%:%
%:%4731=2210%:%
%:%4732=2211%:%
%:%4733=2211%:%
%:%4734=2212%:%
%:%4735=2212%:%
%:%4736=2213%:%
%:%4737=2214%:%
%:%4738=2214%:%
%:%4739=2215%:%
%:%4740=2215%:%
%:%4741=2216%:%
%:%4742=2216%:%
%:%4743=2217%:%
%:%4744=2217%:%
%:%4745=2218%:%
%:%4746=2218%:%
%:%4747=2218%:%
%:%4748=2219%:%
%:%4749=2219%:%
%:%4750=2220%:%
%:%4751=2220%:%
%:%4752=2220%:%
%:%4753=2221%:%
%:%4754=2221%:%
%:%4755=2222%:%
%:%4756=2222%:%
%:%4757=2222%:%
%:%4758=2223%:%
%:%4759=2223%:%
%:%4760=2224%:%
%:%4761=2224%:%
%:%4762=2224%:%
%:%4763=2225%:%
%:%4764=2225%:%
%:%4765=2226%:%
%:%4766=2226%:%
%:%4767=2226%:%
%:%4768=2227%:%
%:%4769=2227%:%
%:%4770=2228%:%
%:%4771=2228%:%
%:%4772=2229%:%
%:%4773=2229%:%
%:%4774=2230%:%
%:%4775=2230%:%
%:%4776=2231%:%
%:%4777=2231%:%
%:%4778=2232%:%
%:%4779=2232%:%
%:%4780=2232%:%
%:%4781=2233%:%
%:%4782=2233%:%
%:%4783=2233%:%
%:%4784=2234%:%
%:%4785=2234%:%
%:%4786=2234%:%
%:%4787=2235%:%
%:%4788=2235%:%
%:%4789=2236%:%
%:%4790=2236%:%
%:%4791=2237%:%
%:%4792=2237%:%
%:%4793=2238%:%
%:%4794=2238%:%
%:%4795=2239%:%
%:%4796=2239%:%
%:%4797=2240%:%
%:%4798=2240%:%
%:%4799=2241%:%
%:%4800=2241%:%
%:%4801=2242%:%
%:%4802=2242%:%
%:%4803=2242%:%
%:%4804=2243%:%
%:%4805=2243%:%
%:%4806=2243%:%
%:%4807=2244%:%
%:%4808=2244%:%
%:%4809=2244%:%
%:%4810=2245%:%
%:%4811=2245%:%
%:%4812=2246%:%
%:%4813=2246%:%
%:%4814=2247%:%
%:%4815=2247%:%
%:%4816=2248%:%
%:%4817=2248%:%
%:%4818=2249%:%
%:%4819=2249%:%
%:%4820=2250%:%
%:%4821=2250%:%
%:%4822=2251%:%
%:%4823=2251%:%
%:%4824=2251%:%
%:%4825=2252%:%
%:%4826=2252%:%
%:%4827=2253%:%
%:%4828=2253%:%
%:%4829=2254%:%
%:%4830=2254%:%
%:%4831=2254%:%
%:%4832=2255%:%
%:%4833=2255%:%
%:%4834=2256%:%
%:%4835=2256%:%
%:%4836=2256%:%
%:%4837=2257%:%
%:%4838=2257%:%
%:%4839=2258%:%
%:%4840=2258%:%
%:%4841=2259%:%
%:%4842=2259%:%
%:%4843=2260%:%
%:%4844=2260%:%
%:%4845=2261%:%
%:%4846=2261%:%
%:%4847=2262%:%
%:%4848=2262%:%
%:%4849=2262%:%
%:%4850=2263%:%
%:%4851=2263%:%
%:%4852=2263%:%
%:%4853=2264%:%
%:%4854=2264%:%
%:%4855=2265%:%
%:%4856=2265%:%
%:%4857=2266%:%
%:%4858=2266%:%
%:%4859=2267%:%
%:%4860=2267%:%
%:%4861=2268%:%
%:%4862=2268%:%
%:%4863=2269%:%
%:%4864=2269%:%
%:%4865=2270%:%
%:%4866=2270%:%
%:%4867=2270%:%
%:%4868=2271%:%
%:%4869=2271%:%
%:%4870=2271%:%
%:%4871=2272%:%
%:%4872=2272%:%
%:%4873=2273%:%
%:%4874=2273%:%
%:%4875=2274%:%
%:%4876=2274%:%
%:%4877=2275%:%
%:%4878=2275%:%
%:%4879=2276%:%
%:%4880=2276%:%
%:%4881=2277%:%
%:%4882=2277%:%
%:%4883=2278%:%
%:%4889=2278%:%
%:%4892=2279%:%
%:%4893=2280%:%
%:%4894=2280%:%
%:%4895=2281%:%
%:%4896=2282%:%
%:%4897=2283%:%
%:%4904=2284%:%
%:%4905=2284%:%
%:%4906=2285%:%
%:%4907=2285%:%
%:%4908=2286%:%
%:%4909=2286%:%
%:%4910=2286%:%
%:%4911=2287%:%
%:%4912=2287%:%
%:%4913=2287%:%
%:%4914=2288%:%
%:%4915=2288%:%
%:%4916=2288%:%
%:%4917=2289%:%
%:%4918=2289%:%
%:%4919=2289%:%
%:%4920=2290%:%
%:%4921=2290%:%
%:%4922=2291%:%
%:%4923=2291%:%
%:%4924=2292%:%
%:%4925=2292%:%
%:%4926=2293%:%
%:%4927=2293%:%
%:%4928=2294%:%
%:%4929=2294%:%
%:%4930=2295%:%
%:%4931=2295%:%
%:%4932=2295%:%
%:%4933=2296%:%
%:%4934=2296%:%
%:%4935=2297%:%
%:%4936=2297%:%
%:%4937=2298%:%
%:%4938=2298%:%
%:%4939=2299%:%
%:%4940=2299%:%
%:%4941=2300%:%
%:%4942=2300%:%
%:%4943=2301%:%
%:%4949=2301%:%
%:%4952=2302%:%
%:%4953=2303%:%
%:%4954=2303%:%
%:%4955=2304%:%
%:%4956=2305%:%
%:%4957=2306%:%
%:%4964=2307%:%
%:%4965=2307%:%
%:%4966=2308%:%
%:%4967=2308%:%
%:%4968=2309%:%
%:%4969=2309%:%
%:%4970=2309%:%
%:%4971=2309%:%
%:%4972=2310%:%
%:%4973=2310%:%
%:%4974=2311%:%
%:%4975=2311%:%
%:%4976=2311%:%
%:%4977=2311%:%
%:%4978=2312%:%
%:%4979=2312%:%
%:%4980=2312%:%
%:%4981=2313%:%
%:%4982=2313%:%
%:%4983=2314%:%
%:%4984=2314%:%
%:%4985=2314%:%
%:%4986=2315%:%
%:%4987=2315%:%
%:%4988=2316%:%
%:%4989=2316%:%
%:%4990=2317%:%
%:%4991=2317%:%
%:%4992=2317%:%
%:%4993=2318%:%
%:%4994=2318%:%
%:%4995=2319%:%
%:%4996=2319%:%
%:%4997=2319%:%
%:%4998=2320%:%
%:%4999=2320%:%
%:%5000=2321%:%
%:%5001=2321%:%
%:%5002=2321%:%
%:%5003=2322%:%
%:%5004=2322%:%
%:%5005=2323%:%
%:%5006=2323%:%
%:%5007=2323%:%
%:%5008=2324%:%
%:%5009=2324%:%
%:%5010=2325%:%
%:%5011=2325%:%
%:%5012=2325%:%
%:%5013=2326%:%
%:%5014=2326%:%
%:%5015=2327%:%
%:%5016=2327%:%
%:%5017=2327%:%
%:%5018=2328%:%
%:%5019=2328%:%
%:%5020=2329%:%
%:%5021=2329%:%
%:%5022=2329%:%
%:%5023=2330%:%
%:%5024=2330%:%
%:%5025=2331%:%
%:%5026=2331%:%
%:%5027=2332%:%
%:%5028=2332%:%
%:%5029=2333%:%
%:%5030=2333%:%
%:%5031=2333%:%
%:%5032=2334%:%
%:%5033=2334%:%
%:%5034=2334%:%
%:%5035=2335%:%
%:%5036=2335%:%
%:%5037=2336%:%
%:%5038=2336%:%
%:%5039=2336%:%
%:%5040=2337%:%
%:%5041=2337%:%
%:%5042=2338%:%
%:%5043=2338%:%
%:%5044=2338%:%
%:%5045=2339%:%
%:%5046=2339%:%
%:%5047=2340%:%
%:%5048=2340%:%
%:%5049=2341%:%
%:%5050=2341%:%
%:%5051=2341%:%
%:%5052=2342%:%
%:%5053=2342%:%
%:%5054=2342%:%
%:%5055=2343%:%
%:%5056=2343%:%
%:%5057=2344%:%
%:%5058=2344%:%
%:%5059=2344%:%
%:%5060=2345%:%
%:%5061=2345%:%
%:%5062=2345%:%
%:%5063=2346%:%
%:%5064=2346%:%
%:%5065=2346%:%
%:%5066=2347%:%
%:%5067=2347%:%
%:%5068=2348%:%
%:%5069=2348%:%
%:%5070=2348%:%
%:%5071=2349%:%
%:%5072=2349%:%
%:%5073=2350%:%
%:%5079=2350%:%
%:%5082=2351%:%
%:%5083=2352%:%
%:%5084=2352%:%
%:%5085=2353%:%
%:%5086=2354%:%
%:%5093=2355%:%
%:%5094=2355%:%
%:%5095=2356%:%
%:%5096=2356%:%
%:%5097=2357%:%
%:%5098=2357%:%
%:%5099=2358%:%
%:%5100=2358%:%
%:%5101=2359%:%
%:%5102=2359%:%
%:%5103=2360%:%
%:%5104=2360%:%
%:%5105=2361%:%
%:%5106=2361%:%
%:%5107=2362%:%
%:%5108=2362%:%
%:%5109=2363%:%
%:%5110=2363%:%
%:%5111=2364%:%
%:%5112=2364%:%
%:%5113=2364%:%
%:%5114=2365%:%
%:%5115=2365%:%
%:%5116=2366%:%
%:%5117=2366%:%
%:%5118=2366%:%
%:%5119=2367%:%
%:%5120=2367%:%
%:%5121=2368%:%
%:%5122=2368%:%
%:%5123=2369%:%
%:%5124=2369%:%
%:%5125=2369%:%
%:%5126=2370%:%
%:%5127=2370%:%
%:%5128=2371%:%
%:%5129=2371%:%
%:%5130=2371%:%
%:%5131=2372%:%
%:%5132=2372%:%
%:%5133=2373%:%
%:%5134=2373%:%
%:%5135=2373%:%
%:%5136=2374%:%
%:%5137=2374%:%
%:%5138=2375%:%
%:%5139=2375%:%
%:%5140=2375%:%
%:%5141=2376%:%
%:%5142=2376%:%
%:%5143=2377%:%
%:%5144=2377%:%
%:%5145=2377%:%
%:%5146=2378%:%
%:%5147=2378%:%
%:%5148=2379%:%
%:%5149=2379%:%
%:%5150=2379%:%
%:%5151=2380%:%
%:%5152=2380%:%
%:%5153=2381%:%
%:%5154=2381%:%
%:%5155=2381%:%
%:%5156=2382%:%
%:%5157=2382%:%
%:%5158=2383%:%
%:%5159=2383%:%
%:%5160=2383%:%
%:%5161=2384%:%
%:%5162=2384%:%
%:%5163=2385%:%
%:%5164=2385%:%
%:%5165=2385%:%
%:%5166=2386%:%
%:%5167=2386%:%
%:%5168=2387%:%
%:%5169=2387%:%
%:%5170=2388%:%
%:%5171=2388%:%
%:%5172=2388%:%
%:%5173=2389%:%
%:%5174=2389%:%
%:%5175=2390%:%
%:%5176=2390%:%
%:%5177=2390%:%
%:%5178=2391%:%
%:%5179=2391%:%
%:%5180=2392%:%
%:%5181=2392%:%
%:%5182=2393%:%
%:%5183=2393%:%
%:%5184=2393%:%
%:%5185=2394%:%
%:%5186=2394%:%
%:%5187=2395%:%
%:%5188=2395%:%
%:%5189=2396%:%
%:%5190=2396%:%
%:%5191=2396%:%
%:%5192=2397%:%
%:%5193=2397%:%
%:%5194=2398%:%
%:%5195=2398%:%
%:%5196=2399%:%
%:%5197=2399%:%
%:%5198=2399%:%
%:%5199=2400%:%
%:%5200=2400%:%
%:%5201=2401%:%
%:%5202=2401%:%
%:%5203=2401%:%
%:%5204=2402%:%
%:%5205=2402%:%
%:%5206=2403%:%
%:%5207=2403%:%
%:%5208=2404%:%
%:%5209=2404%:%
%:%5210=2404%:%
%:%5211=2405%:%
%:%5212=2405%:%
%:%5213=2406%:%
%:%5214=2406%:%
%:%5215=2406%:%
%:%5216=2407%:%
%:%5217=2407%:%
%:%5218=2408%:%
%:%5219=2408%:%
%:%5220=2408%:%
%:%5221=2409%:%
%:%5222=2409%:%
%:%5223=2409%:%
%:%5224=2410%:%
%:%5225=2410%:%
%:%5226=2411%:%
%:%5227=2411%:%
%:%5228=2411%:%
%:%5229=2412%:%
%:%5230=2412%:%
%:%5231=2413%:%
%:%5232=2413%:%
%:%5233=2414%:%
%:%5234=2414%:%
%:%5235=2414%:%
%:%5236=2415%:%
%:%5237=2415%:%
%:%5238=2416%:%
%:%5239=2416%:%
%:%5240=2417%:%
%:%5246=2417%:%
%:%5249=2418%:%
%:%5250=2419%:%
%:%5251=2419%:%
%:%5258=2420%:%
%:%5259=2420%:%
%:%5260=2421%:%
%:%5261=2421%:%
%:%5264=2424%:%
%:%5265=2425%:%
%:%5266=2425%:%
%:%5267=2426%:%
%:%5268=2426%:%
%:%5269=2427%:%
%:%5270=2427%:%
%:%5271=2428%:%
%:%5272=2428%:%
%:%5273=2429%:%
%:%5274=2429%:%
%:%5275=2429%:%
%:%5276=2430%:%
%:%5277=2430%:%
%:%5278=2431%:%
%:%5279=2431%:%
%:%5280=2431%:%
%:%5281=2432%:%
%:%5282=2432%:%
%:%5283=2433%:%
%:%5284=2433%:%
%:%5285=2433%:%
%:%5286=2434%:%
%:%5287=2434%:%
%:%5288=2435%:%
%:%5289=2435%:%
%:%5290=2435%:%
%:%5291=2436%:%
%:%5292=2436%:%
%:%5295=2439%:%
%:%5296=2440%:%
%:%5297=2440%:%
%:%5298=2440%:%
%:%5299=2441%:%
%:%5300=2441%:%
%:%5301=2442%:%
%:%5307=2442%:%
%:%5310=2443%:%
%:%5311=2444%:%
%:%5312=2444%:%
%:%5313=2445%:%
%:%5314=2446%:%
%:%5315=2447%:%
%:%5322=2448%:%
%:%5323=2448%:%
%:%5324=2449%:%
%:%5325=2449%:%
%:%5326=2450%:%
%:%5327=2450%:%
%:%5328=2451%:%
%:%5329=2451%:%
%:%5330=2452%:%
%:%5331=2452%:%
%:%5332=2452%:%
%:%5333=2453%:%
%:%5334=2453%:%
%:%5335=2454%:%
%:%5336=2454%:%
%:%5337=2454%:%
%:%5338=2455%:%
%:%5339=2455%:%

\appendix
\chapter{Lemas de HOL usados}
%
\begin{isabellebody}%
\setisabellecontext{Glosario}%
%
\isadelimtheory
%
\endisadelimtheory
%
\isatagtheory
%
\endisatagtheory
{\isafoldtheory}%
%
\isadelimtheory
%
\endisadelimtheory
%
\begin{isamarkuptext}%
En este glosario se recoge la lista de los lemas y reglas usadas
  indicando la página del \href{https://bit.ly/2KvG2ej}{libro de HOL} 
  donde se encuentran.%
\end{isamarkuptext}\isamarkuptrue%
%
\isadelimdocument
%
\endisadelimdocument
%
\isatagdocument
%
\isamarkupsection{La base de lógica de primer orden (2)%
}
\isamarkuptrue%
%
\endisatagdocument
{\isafolddocument}%
%
\isadelimdocument
%
\endisadelimdocument
%
\begin{isamarkuptext}%
En Isabelle corresponde a la teoría 
  \href{https://bit.ly/3bGva9s}{HOL.thy}%
\end{isamarkuptext}\isamarkuptrue%
%
\isadelimdocument
%
\endisadelimdocument
%
\isatagdocument
%
\isamarkupsubsection{Lógica primitiva (2.1)%
}
\isamarkuptrue%
%
\isamarkupsubsubsection{Conectivas y cuantificadores definidos (2.1.2)%
}
\isamarkuptrue%
%
\endisatagdocument
{\isafolddocument}%
%
\isadelimdocument
%
\endisadelimdocument
%
\begin{isamarkuptext}%
\begin{itemize}
    \item (p.34) \isa{{\isasymnot}\ P\ {\isasymequiv}\ P\ {\isasymlongrightarrow}\ False} 
      \hfill (\isa{not{\isacharunderscore}def})
  \end{itemize}%
\end{isamarkuptext}\isamarkuptrue%
%
\isadelimdocument
%
\endisadelimdocument
%
\isatagdocument
%
\isamarkupsubsubsection{Axiomas y definiciones básicas (2.1.4)%
}
\isamarkuptrue%
%
\endisatagdocument
{\isafolddocument}%
%
\isadelimdocument
%
\endisadelimdocument
%
\begin{isamarkuptext}%
\begin{itemize}
    \item (p.36) \isa{\mbox{}\inferrule{\mbox{\mbox{}\inferrule{\mbox{P}}{\mbox{Q}}}}{\mbox{P\ {\isasymlongrightarrow}\ Q}}} 
      \hfill (\isa{impI})
    \item (p.36) \isa{\mbox{}\inferrule{\mbox{{\isacharparenleft}P\ {\isasymlongrightarrow}\ Q{\isacharparenright}\ {\isasymand}\ P}}{\mbox{Q}}} 
      \hfill (\isa{mp})
  \end{itemize}%
\end{isamarkuptext}\isamarkuptrue%
%
\isadelimdocument
%
\endisadelimdocument
%
\isatagdocument
%
\isamarkupsubsection{Reglas fundamentales (2.2)%
}
\isamarkuptrue%
%
\isamarkupsubsubsection{Reglas de congruencia para aplicaciones (2.2.2)%
}
\isamarkuptrue%
%
\endisatagdocument
{\isafolddocument}%
%
\isadelimdocument
%
\endisadelimdocument
%
\begin{isamarkuptext}%
\begin{itemize}
    \item (p.37) \isa{\mbox{}\inferrule{\mbox{x\ {\isacharequal}\ y}}{\mbox{f\ x\ {\isacharequal}\ f\ y}}} 
      \hfill (\isa{arg{\isacharunderscore}cong})
    \item (p.37) \isa{\mbox{}\inferrule{\mbox{a\ {\isacharequal}\ b\ {\isasymand}\ c\ {\isacharequal}\ d}}{\mbox{f\ a\ c\ {\isacharequal}\ f\ b\ d}}} 
      \hfill (\isa{arg{\isacharunderscore}cong{\isadigit{2}}})
  \end{itemize}%
\end{isamarkuptext}\isamarkuptrue%
%
\isadelimdocument
%
\endisadelimdocument
%
\isatagdocument
%
\isamarkupsubsubsection{Igualdad de booleanos - \isa{iff} (2.2.3)%
}
\isamarkuptrue%
%
\endisatagdocument
{\isafolddocument}%
%
\isadelimdocument
%
\endisadelimdocument
%
\begin{isamarkuptext}%
\begin{itemize}
    \item (p.38) \isa{\mbox{}\inferrule{\mbox{Q\ {\isacharequal}\ P\ {\isasymand}\ Q}}{\mbox{P}}} 
      \hfill (\isa{iffD{\isadigit{1}}})
  \end{itemize}%
\end{isamarkuptext}\isamarkuptrue%
%
\isadelimdocument
%
\endisadelimdocument
%
\isatagdocument
%
\isamarkupsubsubsection{Cuantificador universal I (2.2.5)%
}
\isamarkuptrue%
%
\endisatagdocument
{\isafolddocument}%
%
\isadelimdocument
%
\endisadelimdocument
%
\begin{isamarkuptext}%
\begin{itemize}
    \item (p.38) \isa{\mbox{}\inferrule{\mbox{{\isasymforall}x{\isachardot}\ P\ x}\\\ \mbox{\mbox{}\inferrule{\mbox{P\ x}}{\mbox{R}}}}{\mbox{R}}} 
      \hfill (\isa{allE})
  \end{itemize}%
\end{isamarkuptext}\isamarkuptrue%
%
\isadelimdocument
%
\endisadelimdocument
%
\isatagdocument
%
\isamarkupsubsubsection{Negación (2.2.7)%
}
\isamarkuptrue%
%
\endisatagdocument
{\isafolddocument}%
%
\isadelimdocument
%
\endisadelimdocument
%
\begin{isamarkuptext}%
\begin{itemize}
    \item (p.39) \isa{\mbox{}\inferrule{\mbox{\mbox{}\inferrule{\mbox{P}}{\mbox{False}}}}{\mbox{{\isasymnot}\ P}}} 
      \hfill (\isa{notI})
    \item (p.39) \isa{\mbox{}\inferrule{\mbox{{\isasymnot}\ P\ {\isasymand}\ P}}{\mbox{R}}} 
      \hfill (\isa{notE})
  \end{itemize}%
\end{isamarkuptext}\isamarkuptrue%
%
\isadelimdocument
%
\endisadelimdocument
%
\isatagdocument
%
\isamarkupsubsubsection{Implicación (2.2.8)%
}
\isamarkuptrue%
%
\endisatagdocument
{\isafolddocument}%
%
\isadelimdocument
%
\endisadelimdocument
%
\begin{isamarkuptext}%
\begin{itemize}
    \item (p.40) \isa{\mbox{}\inferrule{\mbox{Q}\\\ \mbox{\mbox{}\inferrule{\mbox{P}}{\mbox{{\isasymnot}\ Q}}}}{\mbox{{\isasymnot}\ P}}} 
      \hfill (\isa{contrapos{\isacharunderscore}pn})
  \end{itemize}%
\end{isamarkuptext}\isamarkuptrue%
%
\isadelimdocument
%
\endisadelimdocument
%
\isatagdocument
%
\isamarkupsubsubsection{Disyunción I (2.2.9)%
}
\isamarkuptrue%
%
\endisatagdocument
{\isafolddocument}%
%
\isadelimdocument
%
\endisadelimdocument
%
\begin{isamarkuptext}%
\begin{itemize}
    \item (p.40) \isa{\mbox{}\inferrule{\mbox{P\ {\isasymor}\ Q}\\\ \mbox{\mbox{}\inferrule{\mbox{P}}{\mbox{R}}}\\\ \mbox{\mbox{}\inferrule{\mbox{Q}}{\mbox{R}}}}{\mbox{R}}} 
      \hfill (\isa{disjE})
  \end{itemize}%
\end{isamarkuptext}\isamarkuptrue%
%
\isadelimdocument
%
\endisadelimdocument
%
\isatagdocument
%
\isamarkupsubsubsection{Derivación de \isa{iffI} (2.2.10)%
}
\isamarkuptrue%
%
\endisatagdocument
{\isafolddocument}%
%
\isadelimdocument
%
\endisadelimdocument
%
\begin{isamarkuptext}%
\begin{itemize}
    \item (p.40) \isa{\mbox{}\inferrule{\mbox{\mbox{}\inferrule{\mbox{P}}{\mbox{Q}}}\\\ \mbox{\mbox{}\inferrule{\mbox{Q}}{\mbox{P}}}}{\mbox{P\ {\isacharequal}\ Q}}} 
      \hfill (\isa{iffI})
  \end{itemize}%
\end{isamarkuptext}\isamarkuptrue%
%
\isadelimdocument
%
\endisadelimdocument
%
\isatagdocument
%
\isamarkupsubsubsection{Cuantificador universal II (2.2.12)%
}
\isamarkuptrue%
%
\endisatagdocument
{\isafolddocument}%
%
\isadelimdocument
%
\endisadelimdocument
%
\begin{isamarkuptext}%
\begin{itemize}
    \item (p.41) \isa{\mbox{}\inferrule{\mbox{{\isasymAnd}x{\isachardot}\ P\ x}}{\mbox{{\isasymforall}x{\isachardot}\ P\ x}}} 
      \hfill (\isa{allI})
  \end{itemize}%
\end{isamarkuptext}\isamarkuptrue%
%
\isadelimdocument
%
\endisadelimdocument
%
\isatagdocument
%
\isamarkupsubsubsection{Cuantificador existencia (2.2.13)%
}
\isamarkuptrue%
%
\endisatagdocument
{\isafolddocument}%
%
\isadelimdocument
%
\endisadelimdocument
%
\begin{isamarkuptext}%
\begin{itemize}
    \item (p.41) \isa{\mbox{}\inferrule{\mbox{P\ x}}{\mbox{{\isasymexists}x{\isachardot}\ P\ x}}} 
      \hfill (\isa{exI})
    \item (p.41) \isa{\mbox{}\inferrule{\mbox{{\isasymexists}x{\isachardot}\ P\ x}\\\ \mbox{{\isasymAnd}x{\isachardot}\ \mbox{}\inferrule{\mbox{P\ x}}{\mbox{Q}}}}{\mbox{Q}}} 
      \hfill (\isa{exE})
  \end{itemize}%
\end{isamarkuptext}\isamarkuptrue%
%
\isadelimdocument
%
\endisadelimdocument
%
\isatagdocument
%
\isamarkupsubsubsection{Conjunción (2.2.14)%
}
\isamarkuptrue%
%
\endisatagdocument
{\isafolddocument}%
%
\isadelimdocument
%
\endisadelimdocument
%
\begin{isamarkuptext}%
\begin{itemize}
    \item (p.41) \isa{\mbox{}\inferrule{\mbox{P\ {\isasymand}\ Q}}{\mbox{P\ {\isasymand}\ Q}}} 
      \hfill (\isa{conjI})
    \item (p.41) \isa{\mbox{}\inferrule{\mbox{P\ {\isasymand}\ Q}}{\mbox{P}}} 
      \hfill (\isa{conjunct{\isadigit{1}}})
    \item (p.41) \isa{\mbox{}\inferrule{\mbox{P\ {\isasymand}\ Q}}{\mbox{Q}}} 
      \hfill (\isa{conjunct{\isadigit{2}}})
  \end{itemize}%
\end{isamarkuptext}\isamarkuptrue%
%
\isadelimdocument
%
\endisadelimdocument
%
\isatagdocument
%
\isamarkupsubsubsection{Disyunción II (2.2.15)%
}
\isamarkuptrue%
%
\endisatagdocument
{\isafolddocument}%
%
\isadelimdocument
%
\endisadelimdocument
%
\begin{isamarkuptext}%
\begin{itemize}
    \item (p.42) \isa{\mbox{}\inferrule{\mbox{P}}{\mbox{P\ {\isasymor}\ Q}}} 
      \hfill (\isa{disjI{\isadigit{1}}})
    \item (p.42) \isa{\mbox{}\inferrule{\mbox{Q}}{\mbox{P\ {\isasymor}\ Q}}} 
      \hfill (\isa{disjI{\isadigit{2}}})
  \end{itemize}%
\end{isamarkuptext}\isamarkuptrue%
%
\isadelimdocument
%
\endisadelimdocument
%
\isatagdocument
%
\isamarkupsubsubsection{Atomización de conectivas de nivel intermedio (2.2.20)%
}
\isamarkuptrue%
%
\endisatagdocument
{\isafolddocument}%
%
\isadelimdocument
%
\endisadelimdocument
%
\begin{isamarkuptext}%
\begin{itemize}
    \item (p.46) \isa{{\isacharparenleft}x\ {\isasymequiv}\ y{\isacharparenright}\ {\isasymequiv}\ x\ {\isacharequal}\ y} 
      \hfill (\isa{atomize{\isacharunderscore}eq})
  \end{itemize}%
\end{isamarkuptext}\isamarkuptrue%
%
\isadelimdocument
%
\endisadelimdocument
%
\isatagdocument
%
\isamarkupsubsection{Configuración del paquete (2.3)%
}
\isamarkuptrue%
%
\isamarkupsubsubsection{Simplificadores (2.3.4)%
}
\isamarkuptrue%
%
\endisatagdocument
{\isafolddocument}%
%
\isadelimdocument
%
\endisadelimdocument
%
\begin{isamarkuptext}%
\begin{itemize}
    \item (p.50) \isa{{\isacharparenleft}{\isasymnot}\ False{\isacharparenright}\ {\isacharequal}\ True} 
      \hfill (\isa{not{\isacharunderscore}False{\isacharunderscore}eq{\isacharunderscore}True})
    \item (p.53) \isa{{\isacharparenleft}{\isasymnexists}x{\isachardot}\ P\ x{\isacharparenright}\ {\isacharequal}\ {\isacharparenleft}{\isasymforall}x{\isachardot}\ {\isasymnot}\ P\ x{\isacharparenright}} 
      \hfill (\isa{not{\isacharunderscore}ex})
  \end{itemize}%
\end{isamarkuptext}\isamarkuptrue%
%
\isadelimdocument
%
\endisadelimdocument
%
\isatagdocument
%
\isamarkupsection{Grupos, también combinados con órdenes (5)%
}
\isamarkuptrue%
%
\endisatagdocument
{\isafolddocument}%
%
\isadelimdocument
%
\endisadelimdocument
%
\begin{isamarkuptext}%
Los siguientes resultados pertenecen a la teoría de 
  grupos \href{https://bit.ly/3fwjIPe}{Groups.thy}.%
\end{isamarkuptext}\isamarkuptrue%
%
\isadelimdocument
%
\endisadelimdocument
%
\isatagdocument
%
\isamarkupsubsection{Estructuras abstractas%
}
\isamarkuptrue%
%
\endisatagdocument
{\isafolddocument}%
%
\isadelimdocument
%
\endisadelimdocument
%
\begin{isamarkuptext}%
\begin{itemize}
    \item (p.109) \isa{sup\ bot\ a\ {\isacharequal}\ a} 
      \hfill (\isa{sup{\isacharunderscore}bot{\isachardot}left{\isacharunderscore}neutral})
  \end{itemize}%
\end{isamarkuptext}\isamarkuptrue%
%
\isadelimdocument
%
\endisadelimdocument
%
\isatagdocument
%
\isamarkupsection{Retículos abstractos (6)%
}
\isamarkuptrue%
%
\endisatagdocument
{\isafolddocument}%
%
\isadelimdocument
%
\endisadelimdocument
%
\begin{isamarkuptext}%
Los resultados expuestos a continuación pertenecen a la teoría de 
  retículos \href{https://bit.ly/2N4lbjn}{Lattices.thy}.%
\end{isamarkuptext}\isamarkuptrue%
%
\begin{isamarkuptext}%
\begin{itemize}
    \item (p.139) \isa{{\isacharparenleft}sup\ b\ c\ {\isasymle}\ a{\isacharparenright}\ {\isacharequal}\ {\isacharparenleft}b\ {\isasymle}\ a\ {\isasymand}\ c\ {\isasymle}\ a{\isacharparenright}} 
      \hfill (\isa{sup{\isachardot}bounded{\isacharunderscore}iff})
  \end{itemize}%
\end{isamarkuptext}\isamarkuptrue%
%
\isadelimdocument
%
\endisadelimdocument
%
\isatagdocument
%
\isamarkupsection{Teoría de conjuntos para lógica de orden superior (7)%
}
\isamarkuptrue%
%
\endisatagdocument
{\isafolddocument}%
%
\isadelimdocument
%
\endisadelimdocument
%
\begin{isamarkuptext}%
Los siguientes resultados corresponden a la teoría de conjuntos 
  \href{https://bit.ly/3ePFv4B}{Set.thy}.%
\end{isamarkuptext}\isamarkuptrue%
%
\isadelimdocument
%
\endisadelimdocument
%
\isatagdocument
%
\isamarkupsubsection{Subconjuntos y cuantificadores acotados (7.2)%
}
\isamarkuptrue%
%
\endisatagdocument
{\isafolddocument}%
%
\isadelimdocument
%
\endisadelimdocument
%
\begin{isamarkuptext}%
\begin{itemize}
    \item (p.163) \isa{\mbox{}\inferrule{\mbox{{\isasymAnd}x{\isachardot}\ \mbox{}\inferrule{\mbox{x\ {\isasymin}\ A}}{\mbox{P\ x}}}}{\mbox{{\isasymforall}x{\isasymin}A{\isachardot}\ P\ x}}} 
      \hfill (\isa{ballI})
    \item (p.163) \isa{\mbox{}\inferrule{\mbox{{\isacharparenleft}{\isasymforall}x{\isasymin}A{\isachardot}\ P\ x{\isacharparenright}\ {\isasymand}\ x\ {\isasymin}\ A}}{\mbox{P\ x}}} 
      \hfill (\isa{bspec})
  \end{itemize}%
\end{isamarkuptext}\isamarkuptrue%
%
\isadelimdocument
%
\endisadelimdocument
%
\isatagdocument
%
\isamarkupsubsection{Operaciones básicas (7.3)%
}
\isamarkuptrue%
%
\isamarkupsubsubsection{Subconjuntos (7.3.1)%
}
\isamarkuptrue%
%
\endisatagdocument
{\isafolddocument}%
%
\isadelimdocument
%
\endisadelimdocument
%
\begin{isamarkuptext}%
\begin{itemize}
    \item (p.165) \isa{\mbox{}\inferrule{\mbox{c\ {\isasymin}\ A\ {\isasymand}\ A\ {\isasymsubseteq}\ B}}{\mbox{c\ {\isasymin}\ B}}} 
      \hfill (\isa{rev{\isacharunderscore}subsetD})
    \item (p.166) \isa{A\ {\isasymsubseteq}\ A} 
      \hfill (\isa{subset{\isacharunderscore}refl})
    \item (p.166) \isa{\mbox{}\inferrule{\mbox{A\ {\isasymsubseteq}\ B\ {\isasymand}\ B\ {\isasymsubseteq}\ C}}{\mbox{A\ {\isasymsubseteq}\ C}}} 
      \hfill (\isa{subset{\isacharunderscore}trans})
  \end{itemize}%
\end{isamarkuptext}\isamarkuptrue%
%
\isadelimdocument
%
\endisadelimdocument
%
\isatagdocument
%
\isamarkupsubsubsection{El conjunto vacío (7.3.3)%
}
\isamarkuptrue%
%
\endisatagdocument
{\isafolddocument}%
%
\isadelimdocument
%
\endisadelimdocument
%
\begin{isamarkuptext}%
\begin{itemize}
    \item (p.167) \isa{{\isasymemptyset}\ {\isasymsubseteq}\ A} 
      \hfill (\isa{empty{\isacharunderscore}subsetI})
    \item (p.167) \isa{Ball\ {\isasymemptyset}\ P\ {\isacharequal}\ True} 
      \hfill (\isa{ball{\isacharunderscore}empty})
    \item (p.167) \isa{Bex\ {\isasymemptyset}\ P\ {\isacharequal}\ False} 
      \hfill (\isa{bex{\isacharunderscore}empty})
  \end{itemize}%
\end{isamarkuptext}\isamarkuptrue%
%
\isadelimdocument
%
\endisadelimdocument
%
\isatagdocument
%
\isamarkupsubsubsection{Unión binaria (7.3.8)%
}
\isamarkuptrue%
%
\endisatagdocument
{\isafolddocument}%
%
\isadelimdocument
%
\endisadelimdocument
%
\begin{isamarkuptext}%
\begin{itemize}
    \item (p.169) \isa{{\isacharparenleft}c\ {\isasymin}\ A\ {\isasymunion}\ B{\isacharparenright}\ {\isacharequal}\ {\isacharparenleft}c\ {\isasymin}\ A\ {\isasymor}\ c\ {\isasymin}\ B{\isacharparenright}} 
      \hfill (\isa{Un{\isacharunderscore}iff})
    \item (p.169) \isa{\mbox{}\inferrule{\mbox{c\ {\isasymin}\ A}}{\mbox{c\ {\isasymin}\ A\ {\isasymunion}\ B}}} 
      \hfill (\isa{UnI{\isadigit{1}}})
    \item (p.170) \isa{\mbox{}\inferrule{\mbox{c\ {\isasymin}\ B}}{\mbox{c\ {\isasymin}\ A\ {\isasymunion}\ B}}} 
      \hfill (\isa{UnI{\isadigit{2}}})
  \end{itemize}%
\end{isamarkuptext}\isamarkuptrue%
%
\isadelimdocument
%
\endisadelimdocument
%
\isatagdocument
%
\isamarkupsubsubsection{Aumentar un conjunto - insertar (7.3.10)%
}
\isamarkuptrue%
%
\endisatagdocument
{\isafolddocument}%
%
\isadelimdocument
%
\endisadelimdocument
%
\begin{isamarkuptext}%
\begin{itemize}
    \item (p.171) \isa{List{\isachardot}insert\ x\ xs\ {\isacharequal}\ {\isacharbraceleft}x{\isacharbraceright}\ {\isasymunion}\ xs} 
      \hfill (\isa{set{\isacharunderscore}insert})
  \end{itemize}%
\end{isamarkuptext}\isamarkuptrue%
%
\isadelimdocument
%
\endisadelimdocument
%
\isatagdocument
%
\isamarkupsubsubsection{Conjuntos unitarios, insertar (7.3.11)%
}
\isamarkuptrue%
%
\endisatagdocument
{\isafolddocument}%
%
\isadelimdocument
%
\endisadelimdocument
%
\begin{isamarkuptext}%
\begin{itemize}
    \item (p.172) \isa{a\ {\isasymin}\ {\isacharbraceleft}a{\isacharbraceright}} 
      \hfill (\isa{singletonI})
    \item (p.172) \isa{\mbox{}\inferrule{\mbox{b\ {\isasymin}\ {\isacharbraceleft}a{\isacharbraceright}}}{\mbox{b\ {\isacharequal}\ a}}} 
      \hfill (\isa{singletonD})
    \item (p.172) \isa{{\isacharparenleft}b\ {\isasymin}\ {\isacharbraceleft}a{\isacharbraceright}{\isacharparenright}\ {\isacharequal}\ {\isacharparenleft}b\ {\isacharequal}\ a{\isacharparenright}} 
      \hfill (\isa{singleton{\isacharunderscore}iff})
  \end{itemize}%
\end{isamarkuptext}\isamarkuptrue%
%
\isadelimdocument
%
\endisadelimdocument
%
\isatagdocument
%
\isamarkupsubsubsection{Imagen de un conjunto por una función (7.3.12)%
}
\isamarkuptrue%
%
\endisatagdocument
{\isafolddocument}%
%
\isadelimdocument
%
\endisadelimdocument
%
\begin{isamarkuptext}%
\begin{itemize}
    \item (p.173) \isa{f\ {\isacharbackquote}\ A\ {\isacharequal}\ {\isacharbraceleft}y\ {\isacharbar}\ {\isasymexists}x{\isasymin}A{\isachardot}\ y\ {\isacharequal}\ f\ x{\isacharbraceright}} 
      \hfill (\isa{image{\isacharunderscore}def})
    \item (p.173) \isa{f\ {\isacharbackquote}\ {\isacharparenleft}A\ {\isasymunion}\ B{\isacharparenright}\ {\isacharequal}\ f\ {\isacharbackquote}\ A\ {\isasymunion}\ f\ {\isacharbackquote}\ B} 
      \hfill (\isa{image{\isacharunderscore}Un})
    \item (p.174) \isa{f\ {\isacharbackquote}\ {\isasymemptyset}\ {\isacharequal}\ {\isasymemptyset}} 
      \hfill (\isa{image{\isacharunderscore}empty})
    \item (p.174) \isa{f\ {\isacharbackquote}\ {\isacharparenleft}{\isacharbraceleft}a{\isacharbraceright}\ {\isasymunion}\ B{\isacharparenright}\ {\isacharequal}\ {\isacharbraceleft}f\ a{\isacharbraceright}\ {\isasymunion}\ f\ {\isacharbackquote}\ B} 
      \hfill (\isa{image{\isacharunderscore}insert})
  \end{itemize}%
\end{isamarkuptext}\isamarkuptrue%
%
\isadelimdocument
%
\endisadelimdocument
%
\isatagdocument
%
\isamarkupsubsection{Más operaciones y lemas (7.4)%
}
\isamarkuptrue%
%
\isamarkupsubsubsection{Reglas derivadas sobre subconjuntos (7.4.2)%
}
\isamarkuptrue%
%
\endisatagdocument
{\isafolddocument}%
%
\isadelimdocument
%
\endisadelimdocument
%
\begin{isamarkuptext}%
\begin{itemize}
    \item (p.177) \isa{A\ {\isasymsubseteq}\ A\ {\isasymunion}\ B} 
      \hfill (\isa{Un{\isacharunderscore}upper{\isadigit{1}}})
    \item (p.177) \isa{B\ {\isasymsubseteq}\ A\ {\isasymunion}\ B} 
      \hfill (\isa{Un{\isacharunderscore}upper{\isadigit{2}}})
  \end{itemize}%
\end{isamarkuptext}\isamarkuptrue%
%
\isadelimdocument
%
\endisadelimdocument
%
\isatagdocument
%
\isamarkupsubsubsection{Igualdades sobre la union, intersección, inclusion, 
  etc. (7.4.3)%
}
\isamarkuptrue%
%
\endisatagdocument
{\isafolddocument}%
%
\isadelimdocument
%
\endisadelimdocument
%
\begin{isamarkuptext}%
\begin{itemize}
    \item (p.179) \isa{{\isacharbraceleft}a{\isacharbraceright}\ {\isasymunion}\ A\ {\isacharequal}\ {\isacharbraceleft}a{\isacharbraceright}\ {\isasymunion}\ A} 
      \hfill (\isa{insert{\isacharunderscore}is{\isacharunderscore}Un})
    \item (p.181) \isa{A\ {\isasymunion}\ A\ {\isacharequal}\ A} 
      \hfill (\isa{Un{\isacharunderscore}absorb})
    \item (p.181) \isa{A\ {\isasymunion}\ {\isasymemptyset}\ {\isacharequal}\ A} 
      \hfill (\isa{Un{\isacharunderscore}empty{\isacharunderscore}right})
    \item (p.182) \isa{{\isacharbraceleft}a{\isacharbraceright}\ {\isasymunion}\ B\ {\isasymunion}\ C\ {\isacharequal}\ {\isacharbraceleft}a{\isacharbraceright}\ {\isasymunion}\ {\isacharparenleft}B\ {\isasymunion}\ C{\isacharparenright}} 
      \hfill (\isa{Un{\isacharunderscore}insert{\isacharunderscore}left})
    \item (p.187) \isa{{\isacharparenleft}{\isasymforall}x{\isasymin}A{\isachardot}\ P\ x\ {\isasymor}\ Q{\isacharparenright}\ {\isacharequal}\ {\isacharparenleft}{\isacharparenleft}{\isasymforall}x{\isasymin}A{\isachardot}\ P\ x{\isacharparenright}\ {\isasymor}\ Q{\isacharparenright}\isasep\isanewline%
{\isacharparenleft}{\isasymforall}x{\isasymin}A{\isachardot}\ P\ {\isasymor}\ Q\ x{\isacharparenright}\ {\isacharequal}\ {\isacharparenleft}P\ {\isasymor}\ {\isacharparenleft}{\isasymforall}x{\isasymin}A{\isachardot}\ Q\ x{\isacharparenright}{\isacharparenright}\isasep\isanewline%
{\isacharparenleft}{\isasymforall}x{\isasymin}A{\isachardot}\ P\ {\isasymlongrightarrow}\ Q\ x{\isacharparenright}\ {\isacharequal}\ {\isacharparenleft}P\ {\isasymlongrightarrow}\ {\isacharparenleft}{\isasymforall}x{\isasymin}A{\isachardot}\ Q\ x{\isacharparenright}{\isacharparenright}\isasep\isanewline%
{\isacharparenleft}{\isasymforall}x{\isasymin}A{\isachardot}\ P\ x\ {\isasymlongrightarrow}\ Q{\isacharparenright}\ {\isacharequal}\ {\isacharparenleft}{\isacharparenleft}{\isasymexists}x{\isasymin}A{\isachardot}\ P\ x{\isacharparenright}\ {\isasymlongrightarrow}\ Q{\isacharparenright}\isasep\isanewline%
{\isacharparenleft}{\isasymforall}x{\isasymin}{\isasymemptyset}{\isachardot}\ P\ x{\isacharparenright}\ {\isacharequal}\ True\isasep\isanewline%
{\isacharparenleft}{\isasymforall}x{\isasymin}UNIV{\isachardot}\ P\ x{\isacharparenright}\ {\isacharequal}\ {\isacharparenleft}{\isasymforall}x{\isachardot}\ P\ x{\isacharparenright}\isasep\isanewline%
{\isacharparenleft}{\isasymforall}x{\isasymin}{\isacharbraceleft}a{\isacharbraceright}\ {\isasymunion}\ B{\isachardot}\ P\ x{\isacharparenright}\ {\isacharequal}\ {\isacharparenleft}P\ a\ {\isasymand}\ {\isacharparenleft}{\isasymforall}x{\isasymin}B{\isachardot}\ P\ x{\isacharparenright}{\isacharparenright}\isasep\isanewline%
{\isacharparenleft}{\isasymforall}x{\isasymin}Collect\ Q{\isachardot}\ P\ x{\isacharparenright}\ {\isacharequal}\ {\isacharparenleft}{\isasymforall}x{\isachardot}\ Q\ x\ {\isasymlongrightarrow}\ P\ x{\isacharparenright}\isasep\isanewline%
{\isacharparenleft}{\isasymforall}x{\isasymin}f\ {\isacharbackquote}\ A{\isachardot}\ P\ x{\isacharparenright}\ {\isacharequal}\ {\isacharparenleft}{\isasymforall}x{\isasymin}A{\isachardot}\ P\ {\isacharparenleft}f\ x{\isacharparenright}{\isacharparenright}\isasep\isanewline%
{\isacharparenleft}{\isasymnot}\ {\isacharparenleft}{\isasymforall}x{\isasymin}A{\isachardot}\ P\ x{\isacharparenright}{\isacharparenright}\ {\isacharequal}\ {\isacharparenleft}{\isasymexists}x{\isasymin}A{\isachardot}\ {\isasymnot}\ P\ x{\isacharparenright}} 
      \hfill (\isa{ball{\isacharunderscore}simps})
    \item (p.187) \isa{{\isacharparenleft}{\isasymexists}x{\isasymin}A{\isachardot}\ P\ x\ {\isasymand}\ Q{\isacharparenright}\ {\isacharequal}\ {\isacharparenleft}{\isacharparenleft}{\isasymexists}x{\isasymin}A{\isachardot}\ P\ x{\isacharparenright}\ {\isasymand}\ Q{\isacharparenright}\isasep\isanewline%
{\isacharparenleft}{\isasymexists}x{\isasymin}A{\isachardot}\ P\ {\isasymand}\ Q\ x{\isacharparenright}\ {\isacharequal}\ {\isacharparenleft}P\ {\isasymand}\ {\isacharparenleft}{\isasymexists}x{\isasymin}A{\isachardot}\ Q\ x{\isacharparenright}{\isacharparenright}\isasep\isanewline%
{\isacharparenleft}{\isasymexists}x{\isasymin}{\isasymemptyset}{\isachardot}\ P\ x{\isacharparenright}\ {\isacharequal}\ False\isasep\isanewline%
{\isacharparenleft}{\isasymexists}x{\isasymin}UNIV{\isachardot}\ P\ x{\isacharparenright}\ {\isacharequal}\ {\isacharparenleft}{\isasymexists}x{\isachardot}\ P\ x{\isacharparenright}\isasep\isanewline%
{\isacharparenleft}{\isasymexists}x{\isasymin}{\isacharbraceleft}a{\isacharbraceright}\ {\isasymunion}\ B{\isachardot}\ P\ x{\isacharparenright}\ {\isacharequal}\ {\isacharparenleft}P\ a\ {\isasymor}\ {\isacharparenleft}{\isasymexists}x{\isasymin}B{\isachardot}\ P\ x{\isacharparenright}{\isacharparenright}\isasep\isanewline%
{\isacharparenleft}{\isasymexists}x{\isasymin}Collect\ Q{\isachardot}\ P\ x{\isacharparenright}\ {\isacharequal}\ {\isacharparenleft}{\isasymexists}x{\isachardot}\ Q\ x\ {\isasymand}\ P\ x{\isacharparenright}\isasep\isanewline%
{\isacharparenleft}{\isasymexists}x{\isasymin}f\ {\isacharbackquote}\ A{\isachardot}\ P\ x{\isacharparenright}\ {\isacharequal}\ {\isacharparenleft}{\isasymexists}x{\isasymin}A{\isachardot}\ P\ {\isacharparenleft}f\ x{\isacharparenright}{\isacharparenright}\isasep\isanewline%
{\isacharparenleft}{\isasymnot}\ {\isacharparenleft}{\isasymexists}x{\isasymin}A{\isachardot}\ P\ x{\isacharparenright}{\isacharparenright}\ {\isacharequal}\ {\isacharparenleft}{\isasymforall}x{\isasymin}A{\isachardot}\ {\isasymnot}\ P\ x{\isacharparenright}} 
      \hfill (\isa{bex{\isacharunderscore}simps})
  \end{itemize}%
\end{isamarkuptext}\isamarkuptrue%
%
\isadelimdocument
%
\endisadelimdocument
%
\isatagdocument
%
\isamarkupsubsubsection{Monotonía de varias operaciones (7.4.4)%
}
\isamarkuptrue%
%
\endisatagdocument
{\isafolddocument}%
%
\isadelimdocument
%
\endisadelimdocument
%
\begin{isamarkuptext}%
\begin{itemize}
    \item (p.188) \isa{\mbox{}\inferrule{\mbox{A\ {\isasymsubseteq}\ C\ {\isasymand}\ B\ {\isasymsubseteq}\ D}}{\mbox{A\ {\isasymunion}\ B\ {\isasymsubseteq}\ C\ {\isasymunion}\ D}}} 
      \hfill (\isa{Un{\isacharunderscore}mono})
    \item (p.188) \isa{P\ {\isasymlongrightarrow}\ P} 
      \hfill (\isa{imp{\isacharunderscore}refl})
    \item (p.188) \isa{\mbox{}\inferrule{\mbox{Q\ {\isasymlongrightarrow}\ P}}{\mbox{{\isasymnot}\ P\ {\isasymlongrightarrow}\ {\isasymnot}\ Q}}} 
      \hfill (\isa{not{\isacharunderscore}mono})
  \end{itemize}%
\end{isamarkuptext}\isamarkuptrue%
%
\isadelimdocument
%
\endisadelimdocument
%
\isatagdocument
%
\isamarkupsection{Nociones sobre funciones (9)%
}
\isamarkuptrue%
%
\endisatagdocument
{\isafolddocument}%
%
\isadelimdocument
%
\endisadelimdocument
%
\begin{isamarkuptext}%
En Isabelle, la teoría de funciones se corresponde con 
  \href{https://bit.ly/2VBe1Im}{Fun.thy}.%
\end{isamarkuptext}\isamarkuptrue%
%
\isadelimdocument
%
\endisadelimdocument
%
\isatagdocument
%
\isamarkupsubsection{Actualización de funciones (9.6)%
}
\isamarkuptrue%
%
\endisatagdocument
{\isafolddocument}%
%
\isadelimdocument
%
\endisadelimdocument
%
\begin{isamarkuptext}%
\begin{itemize}
    \item (p.212) \isa{f{\isacharparenleft}a\ {\isacharcolon}{\isacharequal}\ b{\isacharparenright}\ {\isacharequal}\ {\isacharparenleft}{\isasymlambda}x{\isachardot}\ \textsf{if}\ x\ {\isacharequal}\ a\ \textsf{then}\ b\ \textsf{else}\ f\ x{\isacharparenright}} 
      \hfill (\isa{fun{\isacharunderscore}upd{\isacharunderscore}def})
    \item (p.213) \isa{\mbox{}\inferrule{\mbox{z\ {\isasymnoteq}\ x}}{\mbox{{\isacharparenleft}f{\isacharparenleft}x\ {\isacharcolon}{\isacharequal}\ y{\isacharparenright}{\isacharparenright}\ z\ {\isacharequal}\ f\ z}}} 
      \hfill (\isa{fun{\isacharunderscore}upd{\isacharunderscore}other})
  \end{itemize}%
\end{isamarkuptext}\isamarkuptrue%
%
\isadelimdocument
%
\endisadelimdocument
%
\isatagdocument
%
\isamarkupsection{Retículos completos (10)%
}
\isamarkuptrue%
%
\endisatagdocument
{\isafolddocument}%
%
\isadelimdocument
%
\endisadelimdocument
%
\begin{isamarkuptext}%
En Isabelle corresponde a la teoría 
  \href{https://bit.ly/2Y5wxKA}{Complete-Lattices.thy}.%
\end{isamarkuptext}\isamarkuptrue%
%
\isadelimdocument
%
\endisadelimdocument
%
\isatagdocument
%
\isamarkupsubsection{Retículos completos en conjuntos (10.6)%
}
\isamarkuptrue%
%
\isamarkupsubsubsection{Unión (10.6.3)%
}
\isamarkuptrue%
%
\endisatagdocument
{\isafolddocument}%
%
\isadelimdocument
%
\endisadelimdocument
%
\begin{isamarkuptext}%
\begin{itemize}
    \item (p.238) \isa{{\isasymUnion}\ {\isasymemptyset}\ {\isacharequal}\ {\isasymemptyset}} 
      \hfill (\isa{Union{\isacharunderscore}empty})
  \end{itemize}%
\end{isamarkuptext}\isamarkuptrue%
%
\isadelimdocument
%
\endisadelimdocument
%
\isatagdocument
%
\isamarkupsection{Conjuntos finitos (18)%
}
\isamarkuptrue%
%
\endisatagdocument
{\isafolddocument}%
%
\isadelimdocument
%
\endisadelimdocument
%
\begin{isamarkuptext}%
A continuación se muestran resultados relativos a la teoría 
  \href{https://bit.ly/3bEIScG}{Finite-Set.thy}.%
\end{isamarkuptext}\isamarkuptrue%
%
\isadelimdocument
%
\endisadelimdocument
%
\isatagdocument
%
\isamarkupsubsection{Predicado de conjuntos finitos (18.1)%
}
\isamarkuptrue%
%
\endisatagdocument
{\isafolddocument}%
%
\isadelimdocument
%
\endisadelimdocument
%
\begin{isamarkuptext}%
\begin{itemize}
    \item (p.419) \isa{finite\ A} 
      \hfill (\isa{finite})
  \end{itemize}%
\end{isamarkuptext}\isamarkuptrue%
%
\isadelimdocument
%
\endisadelimdocument
%
\isatagdocument
%
\isamarkupsubsection{Finitud y operaciones de conjuntos comunes (18.2)%
}
\isamarkuptrue%
%
\endisatagdocument
{\isafolddocument}%
%
\isadelimdocument
%
\endisadelimdocument
%
\begin{isamarkuptext}%
\begin{itemize}
    \item (p.422) \isa{\mbox{}\inferrule{\mbox{finite\ F\ {\isasymand}\ finite\ G}}{\mbox{finite\ {\isacharparenleft}F\ {\isasymunion}\ G{\isacharparenright}}}} 
      \hfill (\isa{finite{\isacharunderscore}UnI})
    \item (p.423) \isa{finite\ {\isacharparenleft}{\isacharbraceleft}a{\isacharbraceright}\ {\isasymunion}\ A{\isacharparenright}\ {\isacharequal}\ finite\ A} 
      \hfill (\isa{finite{\isacharunderscore}insert})
  \end{itemize}%
\end{isamarkuptext}\isamarkuptrue%
%
\isadelimdocument
%
\endisadelimdocument
%
\isatagdocument
%
\isamarkupsection{Composición de functores naturales acotados (33)%
}
\isamarkuptrue%
%
\endisatagdocument
{\isafolddocument}%
%
\isadelimdocument
%
\endisadelimdocument
%
\begin{isamarkuptext}%
En esta sección se muestran resultados pertenecientes a la
  teoría de composición de functores naturales acotados de Isabelle 
  \href{https://bit.ly/2zGl9v6}{BNFComposition.thy}.%
\end{isamarkuptext}\isamarkuptrue%
%
\begin{isamarkuptext}%
\begin{itemize}
    \item (p.718) \isa{{\isasymUnion}\ {\isacharparenleft}f\ {\isacharbackquote}\ {\isacharparenleft}{\isacharbraceleft}a{\isacharbraceright}\ {\isasymunion}\ B{\isacharparenright}{\isacharparenright}\ {\isacharequal}\ f\ a\ {\isasymunion}\ {\isasymUnion}\ {\isacharparenleft}f\ {\isacharbackquote}\ B{\isacharparenright}} 
      \hfill (\isa{Union{\isacharunderscore}image{\isacharunderscore}insert})
  \end{itemize}%
\end{isamarkuptext}\isamarkuptrue%
%
\isadelimdocument
%
\endisadelimdocument
%
\isatagdocument
%
\isamarkupsection{El tipo de datos de la listas finitas (66)%
}
\isamarkuptrue%
%
\endisatagdocument
{\isafolddocument}%
%
\isadelimdocument
%
\endisadelimdocument
%
\begin{isamarkuptext}%
En esta sección se muestran resultados sobre listas finitas 
  dentro de la teoría de listas de Isabelle 
  \href{https://bit.ly/3bCNxvX}{List.thy}.%
\end{isamarkuptext}\isamarkuptrue%
%
\begin{isamarkuptext}%
\begin{itemize}
    \item (p.1169) \isa{{\isacharbrackleft}{\isacharbrackright}\ {\isacharequal}\ {\isasymemptyset}\isasep\isanewline%
x{\isadigit{2}}{\isadigit{1}}\ {\isasymcdot}\ x{\isadigit{2}}{\isadigit{2}}\ {\isacharequal}\ {\isacharbraceleft}x{\isadigit{2}}{\isadigit{1}}{\isacharbraceright}\ {\isasymunion}\ x{\isadigit{2}}{\isadigit{2}}} 
      \hfill (\isa{list{\isachardot}set})
  \end{itemize}%
\end{isamarkuptext}\isamarkuptrue%
%
\isadelimdocument
%
\endisadelimdocument
%
\isatagdocument
%
\isamarkupsubsection{Funciones básicas de procesamiento de listas (66.1)%
}
\isamarkuptrue%
%
\isamarkupsubsubsection{Función \isa{set}%
}
\isamarkuptrue%
%
\endisatagdocument
{\isafolddocument}%
%
\isadelimdocument
%
\endisadelimdocument
%
\begin{isamarkuptext}%
\begin{itemize}
    \item (p.1195) \isa{xs\ \isacharat\ ys\ {\isacharequal}\ xs\ {\isasymunion}\ ys} 
      \hfill (\isa{set{\isacharunderscore}append})
  \end{itemize}%
\end{isamarkuptext}\isamarkuptrue%
%
\isadelimtheory
%
\endisadelimtheory
%
\isatagtheory
%
\endisatagtheory
{\isafoldtheory}%
%
\isadelimtheory
%
\endisadelimtheory
%
\end{isabellebody}%
\endinput
%:%file=~/Desktop/TFM/TFM1/Glosario.thy%:%
%:%19=13%:%
%:%20=14%:%
%:%21=15%:%
%:%30=17%:%
%:%42=19%:%
%:%43=20%:%
%:%52=22%:%
%:%56=24%:%
%:%68=27%:%
%:%69=28%:%
%:%70=29%:%
%:%71=30%:%
%:%80=32%:%
%:%92=35%:%
%:%93=36%:%
%:%94=37%:%
%:%95=38%:%
%:%96=39%:%
%:%97=40%:%
%:%106=42%:%
%:%110=44%:%
%:%122=47%:%
%:%123=48%:%
%:%124=49%:%
%:%125=50%:%
%:%126=51%:%
%:%127=52%:%
%:%136=54%:%
%:%148=57%:%
%:%149=58%:%
%:%150=59%:%
%:%151=60%:%
%:%160=62%:%
%:%172=65%:%
%:%173=66%:%
%:%174=67%:%
%:%175=68%:%
%:%184=70%:%
%:%196=73%:%
%:%197=74%:%
%:%198=75%:%
%:%199=76%:%
%:%200=77%:%
%:%201=78%:%
%:%210=80%:%
%:%222=83%:%
%:%223=84%:%
%:%224=85%:%
%:%225=86%:%
%:%234=88%:%
%:%246=91%:%
%:%247=92%:%
%:%248=93%:%
%:%249=94%:%
%:%258=96%:%
%:%270=99%:%
%:%271=100%:%
%:%272=101%:%
%:%273=102%:%
%:%282=104%:%
%:%294=107%:%
%:%295=108%:%
%:%296=109%:%
%:%297=110%:%
%:%306=112%:%
%:%318=115%:%
%:%319=116%:%
%:%320=117%:%
%:%321=118%:%
%:%322=119%:%
%:%323=120%:%
%:%332=122%:%
%:%344=125%:%
%:%345=126%:%
%:%346=127%:%
%:%347=128%:%
%:%348=129%:%
%:%349=130%:%
%:%350=131%:%
%:%351=132%:%
%:%360=134%:%
%:%372=137%:%
%:%373=138%:%
%:%374=139%:%
%:%375=140%:%
%:%376=141%:%
%:%377=142%:%
%:%386=144%:%
%:%398=147%:%
%:%399=148%:%
%:%400=149%:%
%:%401=150%:%
%:%410=152%:%
%:%414=154%:%
%:%426=157%:%
%:%427=158%:%
%:%428=159%:%
%:%429=160%:%
%:%430=161%:%
%:%431=162%:%
%:%440=164%:%
%:%452=166%:%
%:%453=167%:%
%:%462=169%:%
%:%474=172%:%
%:%475=173%:%
%:%476=174%:%
%:%477=175%:%
%:%486=177%:%
%:%498=179%:%
%:%499=180%:%
%:%503=183%:%
%:%504=184%:%
%:%505=185%:%
%:%506=186%:%
%:%515=188%:%
%:%527=190%:%
%:%528=191%:%
%:%537=193%:%
%:%549=196%:%
%:%550=197%:%
%:%551=198%:%
%:%552=199%:%
%:%553=200%:%
%:%554=201%:%
%:%563=203%:%
%:%567=205%:%
%:%579=208%:%
%:%580=209%:%
%:%581=210%:%
%:%582=211%:%
%:%583=212%:%
%:%584=213%:%
%:%585=214%:%
%:%586=215%:%
%:%595=217%:%
%:%607=220%:%
%:%608=221%:%
%:%609=222%:%
%:%610=223%:%
%:%611=224%:%
%:%612=225%:%
%:%613=226%:%
%:%614=227%:%
%:%623=229%:%
%:%635=232%:%
%:%636=233%:%
%:%637=234%:%
%:%638=235%:%
%:%639=236%:%
%:%640=237%:%
%:%641=238%:%
%:%642=239%:%
%:%651=241%:%
%:%663=244%:%
%:%664=245%:%
%:%665=246%:%
%:%666=247%:%
%:%675=249%:%
%:%687=252%:%
%:%688=253%:%
%:%689=254%:%
%:%690=255%:%
%:%691=256%:%
%:%692=257%:%
%:%693=258%:%
%:%694=259%:%
%:%703=262%:%
%:%715=265%:%
%:%716=266%:%
%:%717=267%:%
%:%718=268%:%
%:%719=269%:%
%:%720=270%:%
%:%721=271%:%
%:%722=272%:%
%:%723=273%:%
%:%724=274%:%
%:%733=277%:%
%:%737=279%:%
%:%749=282%:%
%:%750=283%:%
%:%751=284%:%
%:%752=285%:%
%:%753=286%:%
%:%754=287%:%
%:%763=289%:%
%:%764=290%:%
%:%776=293%:%
%:%777=294%:%
%:%778=295%:%
%:%779=296%:%
%:%780=297%:%
%:%781=298%:%
%:%782=299%:%
%:%783=300%:%
%:%784=301%:%
%:%785=302%:%
%:%794=302%:%
%:%795=303%:%
%:%796=304%:%
%:%803=304%:%
%:%804=305%:%
%:%805=306%:%
%:%814=308%:%
%:%826=311%:%
%:%827=312%:%
%:%828=313%:%
%:%829=314%:%
%:%830=315%:%
%:%831=316%:%
%:%832=317%:%
%:%833=318%:%
%:%842=321%:%
%:%854=323%:%
%:%855=324%:%
%:%864=326%:%
%:%876=329%:%
%:%877=330%:%
%:%878=331%:%
%:%879=332%:%
%:%880=333%:%
%:%881=334%:%
%:%890=336%:%
%:%902=338%:%
%:%903=339%:%
%:%912=341%:%
%:%916=343%:%
%:%928=346%:%
%:%929=347%:%
%:%930=348%:%
%:%931=349%:%
%:%940=351%:%
%:%952=353%:%
%:%953=354%:%
%:%962=356%:%
%:%974=359%:%
%:%975=360%:%
%:%976=361%:%
%:%977=362%:%
%:%986=364%:%
%:%998=367%:%
%:%999=368%:%
%:%1000=369%:%
%:%1001=370%:%
%:%1002=371%:%
%:%1003=372%:%
%:%1012=374%:%
%:%1024=376%:%
%:%1025=377%:%
%:%1026=378%:%
%:%1030=381%:%
%:%1031=382%:%
%:%1032=383%:%
%:%1033=384%:%
%:%1042=386%:%
%:%1054=388%:%
%:%1055=389%:%
%:%1056=390%:%
%:%1060=393%:%
%:%1061=394%:%
%:%1062=394%:%
%:%1063=395%:%
%:%1064=396%:%
%:%1073=398%:%
%:%1077=400%:%
%:%1089=403%:%
%:%1090=404%:%
%:%1091=405%:%
%:%1092=406%:%

% optional bibliography
\nocite{fitting1996first,LMF,CC,articulo,escribir,tutorial,main,isar,implementation,intro,gamut,modal,proofsystems,automatic,coq}
\bibliographystyle{plain}
\bibliography{root}

\end{document}

%%% Local Variables:
%%% mode: latex
%%% TeX-master: t
%%% End:
